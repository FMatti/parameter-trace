
\section{Numerical experiments}
\label{sec:results}


In this section, we verify the performance of the Chebyshev-Nyström++ algorithm proposed in this paper in application scenarios from electronic structure interaction and neural network optimization.

While theoretically an important tool, we have observed that the preservation of non-negativity discussed in~\cref{sec:preservnonneg} is not needed in practice. The (slight) indefiniteness of the standard Chebyshev approximation $g_{\sigma}^{(m)}(t \mtx{I}_n - \mtx{A})$ from~\cref{equ:matrix-approximation} does not seem to impede the reliability of the Nyström++ estimator. Unless otherwise stated, we transform the input matrices such that their eigenvalues are contained in $[-1, 1]$, by approximating their spectrum with NumPy's Hermitian eigenvalue solver before we apply the methods to them. We compute spectral densities at $n_t = 100$ uniformly distributed values of the parameter $t \in [-1, 1]$. The integral involved in computing the $L^1$-errors is approximated with the composite midpoint quadrature rule using the $n_t$ values of $t$ as nodes. Both parameters introduced at the end of \cref{subsubsec:chebyshev-nystrom-implementation}, the eigenvalue truncation threshold and the parameter for detecting a vanishing spectral density, are fixed to $10^{-5}$.

Our implementations are written in Python 3.12.3 using the packages NumPy 2.0.0 and SciPy 1.14.1. They are executed on a single thread of a GitHub-hosted Ubuntu runner with a 64-bit processor and 16 GB of RAM. 

\subsection{Spectral density for Hamiltonian of electronic structure}
\label{subsec:hamiltonian}

We use the electronic structure interaction example from~\cite[Section 6]{lin-2017-randomized-estimation}, which involves the second order finite difference discretization of the Hamiltonian
\begin{equation}
    \mathcal{H} = - \Delta + V
    \label{equ:electronic-hamiltonian}
\end{equation}
in three dimensions. The potential $V$ interacting with the electrons is generated by Gaussian wells
    $v(r) = v_0 e^{-\lambda r^2}$, 
with $v_0 = -4$ and $\lambda = 8$, centered in cells of side length $L=6$ which are stacked $n_c \in \mathbb{N}$ times in each spatial dimension; see \cref{fig:gaussian-well}. The grid width of the finite difference discretization is fixed to $h=0.6$. For $n_c = 1$, this leads to a sparse matrix of size $1\,000 \times 1\,000$ and for $n_c = 3$ of size $27\,000 \times 27\,000$. This example represents an idealized model for the interaction of nuclei on a regular grid with electrons for a $k$-vector in the center of the first Brillouin zone. The distribution of the eigenvalues of the Hamiltonian --- its spectral density --- allows one to interpret the system's energy levels.

\begin{figure}[ht]
    \begin{subfigure}[b]{0.32\columnwidth}
        \scalebox{0.7}{%% Creator: Matplotlib, PGF backend
%%
%% To include the figure in your LaTeX document, write
%%   \input{<filename>.pgf}
%%
%% Make sure the required packages are loaded in your preamble
%%   \usepackage{pgf}
%%
%% Also ensure that all the required font packages are loaded; for instance,
%% the lmodern package is sometimes necessary when using math font.
%%   \usepackage{lmodern}
%%
%% Figures using additional raster images can only be included by \input if
%% they are in the same directory as the main LaTeX file. For loading figures
%% from other directories you can use the `import` package
%%   \usepackage{import}
%%
%% and then include the figures with
%%   \import{<path to file>}{<filename>.pgf}
%%
%% Matplotlib used the following preamble
%%   \def\mathdefault#1{#1}
%%   \everymath=\expandafter{\the\everymath\displaystyle}
%%   
%%   \makeatletter\@ifpackageloaded{underscore}{}{\usepackage[strings]{underscore}}\makeatother
%%
\begingroup%
\makeatletter%
\begin{pgfpicture}%
\pgfpathrectangle{\pgfpointorigin}{\pgfqpoint{1.969617in}{1.850740in}}%
\pgfusepath{use as bounding box, clip}%
\begin{pgfscope}%
\pgfsetbuttcap%
\pgfsetmiterjoin%
\definecolor{currentfill}{rgb}{1.000000,1.000000,1.000000}%
\pgfsetfillcolor{currentfill}%
\pgfsetlinewidth{0.000000pt}%
\definecolor{currentstroke}{rgb}{1.000000,1.000000,1.000000}%
\pgfsetstrokecolor{currentstroke}%
\pgfsetdash{}{0pt}%
\pgfpathmoveto{\pgfqpoint{0.000000in}{0.000000in}}%
\pgfpathlineto{\pgfqpoint{1.969617in}{0.000000in}}%
\pgfpathlineto{\pgfqpoint{1.969617in}{1.850740in}}%
\pgfpathlineto{\pgfqpoint{0.000000in}{1.850740in}}%
\pgfpathlineto{\pgfqpoint{0.000000in}{0.000000in}}%
\pgfpathclose%
\pgfusepath{fill}%
\end{pgfscope}%
\begin{pgfscope}%
\pgfsetbuttcap%
\pgfsetmiterjoin%
\definecolor{currentfill}{rgb}{1.000000,1.000000,1.000000}%
\pgfsetfillcolor{currentfill}%
\pgfsetlinewidth{0.000000pt}%
\definecolor{currentstroke}{rgb}{0.000000,0.000000,0.000000}%
\pgfsetstrokecolor{currentstroke}%
\pgfsetstrokeopacity{0.000000}%
\pgfsetdash{}{0pt}%
\pgfpathmoveto{\pgfqpoint{0.278819in}{0.345370in}}%
\pgfpathlineto{\pgfqpoint{1.828819in}{0.345370in}}%
\pgfpathlineto{\pgfqpoint{1.828819in}{1.692870in}}%
\pgfpathlineto{\pgfqpoint{0.278819in}{1.692870in}}%
\pgfpathlineto{\pgfqpoint{0.278819in}{0.345370in}}%
\pgfpathclose%
\pgfusepath{fill}%
\end{pgfscope}%
\begin{pgfscope}%
\pgfpathrectangle{\pgfqpoint{0.278819in}{0.345370in}}{\pgfqpoint{1.550000in}{1.347500in}}%
\pgfusepath{clip}%
\pgfsetbuttcap%
\pgfsetroundjoin%
\definecolor{currentfill}{rgb}{0.993545,0.862859,0.619299}%
\pgfsetfillcolor{currentfill}%
\pgfsetlinewidth{0.000000pt}%
\definecolor{currentstroke}{rgb}{0.000000,0.000000,0.000000}%
\pgfsetstrokecolor{currentstroke}%
\pgfsetdash{}{0pt}%
\pgfpathmoveto{\pgfqpoint{1.014677in}{0.873327in}}%
\pgfpathlineto{\pgfqpoint{1.030334in}{0.870750in}}%
\pgfpathlineto{\pgfqpoint{1.045990in}{0.869462in}}%
\pgfpathlineto{\pgfqpoint{1.061647in}{0.869462in}}%
\pgfpathlineto{\pgfqpoint{1.077303in}{0.870750in}}%
\pgfpathlineto{\pgfqpoint{1.092960in}{0.873327in}}%
\pgfpathlineto{\pgfqpoint{1.104615in}{0.876203in}}%
\pgfpathlineto{\pgfqpoint{1.108617in}{0.877276in}}%
\pgfpathlineto{\pgfqpoint{1.124273in}{0.882865in}}%
\pgfpathlineto{\pgfqpoint{1.139841in}{0.889814in}}%
\pgfpathlineto{\pgfqpoint{1.139930in}{0.889858in}}%
\pgfpathlineto{\pgfqpoint{1.155586in}{0.899059in}}%
\pgfpathlineto{\pgfqpoint{1.161972in}{0.903425in}}%
\pgfpathlineto{\pgfqpoint{1.171243in}{0.910510in}}%
\pgfpathlineto{\pgfqpoint{1.178750in}{0.917036in}}%
\pgfpathlineto{\pgfqpoint{1.186899in}{0.925096in}}%
\pgfpathlineto{\pgfqpoint{1.191922in}{0.930648in}}%
\pgfpathlineto{\pgfqpoint{1.202506in}{0.944259in}}%
\pgfpathlineto{\pgfqpoint{1.202556in}{0.944336in}}%
\pgfpathlineto{\pgfqpoint{1.210550in}{0.957870in}}%
\pgfpathlineto{\pgfqpoint{1.216979in}{0.971481in}}%
\pgfpathlineto{\pgfqpoint{1.218213in}{0.974959in}}%
\pgfpathlineto{\pgfqpoint{1.221521in}{0.985092in}}%
\pgfpathlineto{\pgfqpoint{1.224485in}{0.998703in}}%
\pgfpathlineto{\pgfqpoint{1.225967in}{1.012314in}}%
\pgfpathlineto{\pgfqpoint{1.225967in}{1.025925in}}%
\pgfpathlineto{\pgfqpoint{1.224485in}{1.039536in}}%
\pgfpathlineto{\pgfqpoint{1.221521in}{1.053148in}}%
\pgfpathlineto{\pgfqpoint{1.218213in}{1.063280in}}%
\pgfpathlineto{\pgfqpoint{1.216979in}{1.066759in}}%
\pgfpathlineto{\pgfqpoint{1.210550in}{1.080370in}}%
\pgfpathlineto{\pgfqpoint{1.202556in}{1.093903in}}%
\pgfpathlineto{\pgfqpoint{1.202506in}{1.093981in}}%
\pgfpathlineto{\pgfqpoint{1.191922in}{1.107592in}}%
\pgfpathlineto{\pgfqpoint{1.186899in}{1.113143in}}%
\pgfpathlineto{\pgfqpoint{1.178750in}{1.121203in}}%
\pgfpathlineto{\pgfqpoint{1.171243in}{1.127729in}}%
\pgfpathlineto{\pgfqpoint{1.161972in}{1.134814in}}%
\pgfpathlineto{\pgfqpoint{1.155586in}{1.139181in}}%
\pgfpathlineto{\pgfqpoint{1.139930in}{1.148381in}}%
\pgfpathlineto{\pgfqpoint{1.139841in}{1.148425in}}%
\pgfpathlineto{\pgfqpoint{1.124273in}{1.155375in}}%
\pgfpathlineto{\pgfqpoint{1.108617in}{1.160964in}}%
\pgfpathlineto{\pgfqpoint{1.104615in}{1.162036in}}%
\pgfpathlineto{\pgfqpoint{1.092960in}{1.164913in}}%
\pgfpathlineto{\pgfqpoint{1.077303in}{1.167489in}}%
\pgfpathlineto{\pgfqpoint{1.061647in}{1.168777in}}%
\pgfpathlineto{\pgfqpoint{1.045990in}{1.168777in}}%
\pgfpathlineto{\pgfqpoint{1.030334in}{1.167489in}}%
\pgfpathlineto{\pgfqpoint{1.014677in}{1.164913in}}%
\pgfpathlineto{\pgfqpoint{1.003022in}{1.162036in}}%
\pgfpathlineto{\pgfqpoint{0.999021in}{1.160964in}}%
\pgfpathlineto{\pgfqpoint{0.983364in}{1.155375in}}%
\pgfpathlineto{\pgfqpoint{0.967797in}{1.148425in}}%
\pgfpathlineto{\pgfqpoint{0.967708in}{1.148381in}}%
\pgfpathlineto{\pgfqpoint{0.952051in}{1.139181in}}%
\pgfpathlineto{\pgfqpoint{0.945666in}{1.134814in}}%
\pgfpathlineto{\pgfqpoint{0.936394in}{1.127729in}}%
\pgfpathlineto{\pgfqpoint{0.928888in}{1.121203in}}%
\pgfpathlineto{\pgfqpoint{0.920738in}{1.113143in}}%
\pgfpathlineto{\pgfqpoint{0.915715in}{1.107592in}}%
\pgfpathlineto{\pgfqpoint{0.905132in}{1.093981in}}%
\pgfpathlineto{\pgfqpoint{0.905081in}{1.093903in}}%
\pgfpathlineto{\pgfqpoint{0.897088in}{1.080370in}}%
\pgfpathlineto{\pgfqpoint{0.890659in}{1.066759in}}%
\pgfpathlineto{\pgfqpoint{0.889425in}{1.063280in}}%
\pgfpathlineto{\pgfqpoint{0.886116in}{1.053148in}}%
\pgfpathlineto{\pgfqpoint{0.883152in}{1.039536in}}%
\pgfpathlineto{\pgfqpoint{0.881671in}{1.025925in}}%
\pgfpathlineto{\pgfqpoint{0.881671in}{1.012314in}}%
\pgfpathlineto{\pgfqpoint{0.883152in}{0.998703in}}%
\pgfpathlineto{\pgfqpoint{0.886116in}{0.985092in}}%
\pgfpathlineto{\pgfqpoint{0.889425in}{0.974959in}}%
\pgfpathlineto{\pgfqpoint{0.890659in}{0.971481in}}%
\pgfpathlineto{\pgfqpoint{0.897088in}{0.957870in}}%
\pgfpathlineto{\pgfqpoint{0.905081in}{0.944336in}}%
\pgfpathlineto{\pgfqpoint{0.905132in}{0.944259in}}%
\pgfpathlineto{\pgfqpoint{0.915715in}{0.930648in}}%
\pgfpathlineto{\pgfqpoint{0.920738in}{0.925096in}}%
\pgfpathlineto{\pgfqpoint{0.928888in}{0.917036in}}%
\pgfpathlineto{\pgfqpoint{0.936394in}{0.910510in}}%
\pgfpathlineto{\pgfqpoint{0.945666in}{0.903425in}}%
\pgfpathlineto{\pgfqpoint{0.952051in}{0.899059in}}%
\pgfpathlineto{\pgfqpoint{0.967708in}{0.889858in}}%
\pgfpathlineto{\pgfqpoint{0.967797in}{0.889814in}}%
\pgfpathlineto{\pgfqpoint{0.983364in}{0.882865in}}%
\pgfpathlineto{\pgfqpoint{0.999021in}{0.877276in}}%
\pgfpathlineto{\pgfqpoint{1.003022in}{0.876203in}}%
\pgfpathlineto{\pgfqpoint{1.014677in}{0.873327in}}%
\pgfpathclose%
\pgfusepath{fill}%
\end{pgfscope}%
\begin{pgfscope}%
\pgfpathrectangle{\pgfqpoint{0.278819in}{0.345370in}}{\pgfqpoint{1.550000in}{1.347500in}}%
\pgfusepath{clip}%
\pgfsetbuttcap%
\pgfsetroundjoin%
\definecolor{currentfill}{rgb}{0.993326,0.602275,0.414390}%
\pgfsetfillcolor{currentfill}%
\pgfsetlinewidth{0.000000pt}%
\definecolor{currentstroke}{rgb}{0.000000,0.000000,0.000000}%
\pgfsetstrokecolor{currentstroke}%
\pgfsetdash{}{0pt}%
\pgfpathmoveto{\pgfqpoint{0.999021in}{0.739415in}}%
\pgfpathlineto{\pgfqpoint{1.014677in}{0.737106in}}%
\pgfpathlineto{\pgfqpoint{1.030334in}{0.735568in}}%
\pgfpathlineto{\pgfqpoint{1.045990in}{0.734799in}}%
\pgfpathlineto{\pgfqpoint{1.061647in}{0.734799in}}%
\pgfpathlineto{\pgfqpoint{1.077303in}{0.735568in}}%
\pgfpathlineto{\pgfqpoint{1.092960in}{0.737106in}}%
\pgfpathlineto{\pgfqpoint{1.108617in}{0.739415in}}%
\pgfpathlineto{\pgfqpoint{1.112069in}{0.740092in}}%
\pgfpathlineto{\pgfqpoint{1.124273in}{0.742535in}}%
\pgfpathlineto{\pgfqpoint{1.139930in}{0.746452in}}%
\pgfpathlineto{\pgfqpoint{1.155586in}{0.751154in}}%
\pgfpathlineto{\pgfqpoint{1.162877in}{0.753703in}}%
\pgfpathlineto{\pgfqpoint{1.171243in}{0.756707in}}%
\pgfpathlineto{\pgfqpoint{1.186899in}{0.763121in}}%
\pgfpathlineto{\pgfqpoint{1.196020in}{0.767314in}}%
\pgfpathlineto{\pgfqpoint{1.202556in}{0.770419in}}%
\pgfpathlineto{\pgfqpoint{1.218213in}{0.778656in}}%
\pgfpathlineto{\pgfqpoint{1.222157in}{0.780925in}}%
\pgfpathlineto{\pgfqpoint{1.233869in}{0.787933in}}%
\pgfpathlineto{\pgfqpoint{1.244037in}{0.794536in}}%
\pgfpathlineto{\pgfqpoint{1.249526in}{0.798265in}}%
\pgfpathlineto{\pgfqpoint{1.263051in}{0.808148in}}%
\pgfpathlineto{\pgfqpoint{1.265182in}{0.809786in}}%
\pgfpathlineto{\pgfqpoint{1.279772in}{0.821759in}}%
\pgfpathlineto{\pgfqpoint{1.280839in}{0.822686in}}%
\pgfpathlineto{\pgfqpoint{1.294610in}{0.835370in}}%
\pgfpathlineto{\pgfqpoint{1.296495in}{0.837223in}}%
\pgfpathlineto{\pgfqpoint{1.307864in}{0.848981in}}%
\pgfpathlineto{\pgfqpoint{1.312152in}{0.853753in}}%
\pgfpathlineto{\pgfqpoint{1.319748in}{0.862592in}}%
\pgfpathlineto{\pgfqpoint{1.327809in}{0.872774in}}%
\pgfpathlineto{\pgfqpoint{1.330419in}{0.876203in}}%
\pgfpathlineto{\pgfqpoint{1.339893in}{0.889814in}}%
\pgfpathlineto{\pgfqpoint{1.343465in}{0.895496in}}%
\pgfpathlineto{\pgfqpoint{1.348289in}{0.903425in}}%
\pgfpathlineto{\pgfqpoint{1.355666in}{0.917036in}}%
\pgfpathlineto{\pgfqpoint{1.359122in}{0.924309in}}%
\pgfpathlineto{\pgfqpoint{1.362054in}{0.930648in}}%
\pgfpathlineto{\pgfqpoint{1.367463in}{0.944259in}}%
\pgfpathlineto{\pgfqpoint{1.371968in}{0.957870in}}%
\pgfpathlineto{\pgfqpoint{1.374778in}{0.968479in}}%
\pgfpathlineto{\pgfqpoint{1.375557in}{0.971481in}}%
\pgfpathlineto{\pgfqpoint{1.378212in}{0.985092in}}%
\pgfpathlineto{\pgfqpoint{1.379982in}{0.998703in}}%
\pgfpathlineto{\pgfqpoint{1.380866in}{1.012314in}}%
\pgfpathlineto{\pgfqpoint{1.380866in}{1.025925in}}%
\pgfpathlineto{\pgfqpoint{1.379982in}{1.039536in}}%
\pgfpathlineto{\pgfqpoint{1.378212in}{1.053148in}}%
\pgfpathlineto{\pgfqpoint{1.375557in}{1.066759in}}%
\pgfpathlineto{\pgfqpoint{1.374778in}{1.069760in}}%
\pgfpathlineto{\pgfqpoint{1.371968in}{1.080370in}}%
\pgfpathlineto{\pgfqpoint{1.367463in}{1.093981in}}%
\pgfpathlineto{\pgfqpoint{1.362054in}{1.107592in}}%
\pgfpathlineto{\pgfqpoint{1.359122in}{1.113930in}}%
\pgfpathlineto{\pgfqpoint{1.355666in}{1.121203in}}%
\pgfpathlineto{\pgfqpoint{1.348289in}{1.134814in}}%
\pgfpathlineto{\pgfqpoint{1.343465in}{1.142743in}}%
\pgfpathlineto{\pgfqpoint{1.339893in}{1.148425in}}%
\pgfpathlineto{\pgfqpoint{1.330419in}{1.162036in}}%
\pgfpathlineto{\pgfqpoint{1.327809in}{1.165465in}}%
\pgfpathlineto{\pgfqpoint{1.319748in}{1.175647in}}%
\pgfpathlineto{\pgfqpoint{1.312152in}{1.184487in}}%
\pgfpathlineto{\pgfqpoint{1.307864in}{1.189259in}}%
\pgfpathlineto{\pgfqpoint{1.296495in}{1.201017in}}%
\pgfpathlineto{\pgfqpoint{1.294610in}{1.202870in}}%
\pgfpathlineto{\pgfqpoint{1.280839in}{1.215553in}}%
\pgfpathlineto{\pgfqpoint{1.279772in}{1.216481in}}%
\pgfpathlineto{\pgfqpoint{1.265182in}{1.228453in}}%
\pgfpathlineto{\pgfqpoint{1.263051in}{1.230092in}}%
\pgfpathlineto{\pgfqpoint{1.249526in}{1.239975in}}%
\pgfpathlineto{\pgfqpoint{1.244037in}{1.243703in}}%
\pgfpathlineto{\pgfqpoint{1.233869in}{1.250307in}}%
\pgfpathlineto{\pgfqpoint{1.222157in}{1.257314in}}%
\pgfpathlineto{\pgfqpoint{1.218213in}{1.259584in}}%
\pgfpathlineto{\pgfqpoint{1.202556in}{1.267820in}}%
\pgfpathlineto{\pgfqpoint{1.196020in}{1.270925in}}%
\pgfpathlineto{\pgfqpoint{1.186899in}{1.275119in}}%
\pgfpathlineto{\pgfqpoint{1.171243in}{1.281532in}}%
\pgfpathlineto{\pgfqpoint{1.162877in}{1.284536in}}%
\pgfpathlineto{\pgfqpoint{1.155586in}{1.287085in}}%
\pgfpathlineto{\pgfqpoint{1.139930in}{1.291788in}}%
\pgfpathlineto{\pgfqpoint{1.124273in}{1.295705in}}%
\pgfpathlineto{\pgfqpoint{1.112069in}{1.298148in}}%
\pgfpathlineto{\pgfqpoint{1.108617in}{1.298825in}}%
\pgfpathlineto{\pgfqpoint{1.092960in}{1.301133in}}%
\pgfpathlineto{\pgfqpoint{1.077303in}{1.302671in}}%
\pgfpathlineto{\pgfqpoint{1.061647in}{1.303440in}}%
\pgfpathlineto{\pgfqpoint{1.045990in}{1.303440in}}%
\pgfpathlineto{\pgfqpoint{1.030334in}{1.302671in}}%
\pgfpathlineto{\pgfqpoint{1.014677in}{1.301133in}}%
\pgfpathlineto{\pgfqpoint{0.999021in}{1.298825in}}%
\pgfpathlineto{\pgfqpoint{0.995568in}{1.298148in}}%
\pgfpathlineto{\pgfqpoint{0.983364in}{1.295705in}}%
\pgfpathlineto{\pgfqpoint{0.967708in}{1.291788in}}%
\pgfpathlineto{\pgfqpoint{0.952051in}{1.287085in}}%
\pgfpathlineto{\pgfqpoint{0.944760in}{1.284536in}}%
\pgfpathlineto{\pgfqpoint{0.936394in}{1.281532in}}%
\pgfpathlineto{\pgfqpoint{0.920738in}{1.275119in}}%
\pgfpathlineto{\pgfqpoint{0.911617in}{1.270925in}}%
\pgfpathlineto{\pgfqpoint{0.905081in}{1.267820in}}%
\pgfpathlineto{\pgfqpoint{0.889425in}{1.259584in}}%
\pgfpathlineto{\pgfqpoint{0.885481in}{1.257314in}}%
\pgfpathlineto{\pgfqpoint{0.873768in}{1.250307in}}%
\pgfpathlineto{\pgfqpoint{0.863600in}{1.243703in}}%
\pgfpathlineto{\pgfqpoint{0.858112in}{1.239975in}}%
\pgfpathlineto{\pgfqpoint{0.844586in}{1.230092in}}%
\pgfpathlineto{\pgfqpoint{0.842455in}{1.228453in}}%
\pgfpathlineto{\pgfqpoint{0.827865in}{1.216481in}}%
\pgfpathlineto{\pgfqpoint{0.826798in}{1.215553in}}%
\pgfpathlineto{\pgfqpoint{0.813027in}{1.202870in}}%
\pgfpathlineto{\pgfqpoint{0.811142in}{1.201017in}}%
\pgfpathlineto{\pgfqpoint{0.799774in}{1.189259in}}%
\pgfpathlineto{\pgfqpoint{0.795485in}{1.184487in}}%
\pgfpathlineto{\pgfqpoint{0.787889in}{1.175647in}}%
\pgfpathlineto{\pgfqpoint{0.779829in}{1.165465in}}%
\pgfpathlineto{\pgfqpoint{0.777218in}{1.162036in}}%
\pgfpathlineto{\pgfqpoint{0.767744in}{1.148425in}}%
\pgfpathlineto{\pgfqpoint{0.764172in}{1.142743in}}%
\pgfpathlineto{\pgfqpoint{0.759348in}{1.134814in}}%
\pgfpathlineto{\pgfqpoint{0.751971in}{1.121203in}}%
\pgfpathlineto{\pgfqpoint{0.748516in}{1.113930in}}%
\pgfpathlineto{\pgfqpoint{0.745583in}{1.107592in}}%
\pgfpathlineto{\pgfqpoint{0.740174in}{1.093981in}}%
\pgfpathlineto{\pgfqpoint{0.735669in}{1.080370in}}%
\pgfpathlineto{\pgfqpoint{0.732859in}{1.069760in}}%
\pgfpathlineto{\pgfqpoint{0.732080in}{1.066759in}}%
\pgfpathlineto{\pgfqpoint{0.729425in}{1.053148in}}%
\pgfpathlineto{\pgfqpoint{0.727655in}{1.039536in}}%
\pgfpathlineto{\pgfqpoint{0.726771in}{1.025925in}}%
\pgfpathlineto{\pgfqpoint{0.726771in}{1.012314in}}%
\pgfpathlineto{\pgfqpoint{0.727655in}{0.998703in}}%
\pgfpathlineto{\pgfqpoint{0.729425in}{0.985092in}}%
\pgfpathlineto{\pgfqpoint{0.732080in}{0.971481in}}%
\pgfpathlineto{\pgfqpoint{0.732859in}{0.968479in}}%
\pgfpathlineto{\pgfqpoint{0.735669in}{0.957870in}}%
\pgfpathlineto{\pgfqpoint{0.740174in}{0.944259in}}%
\pgfpathlineto{\pgfqpoint{0.745583in}{0.930647in}}%
\pgfpathlineto{\pgfqpoint{0.748516in}{0.924309in}}%
\pgfpathlineto{\pgfqpoint{0.751971in}{0.917036in}}%
\pgfpathlineto{\pgfqpoint{0.759348in}{0.903425in}}%
\pgfpathlineto{\pgfqpoint{0.764172in}{0.895496in}}%
\pgfpathlineto{\pgfqpoint{0.767744in}{0.889814in}}%
\pgfpathlineto{\pgfqpoint{0.777218in}{0.876203in}}%
\pgfpathlineto{\pgfqpoint{0.779829in}{0.872774in}}%
\pgfpathlineto{\pgfqpoint{0.787889in}{0.862592in}}%
\pgfpathlineto{\pgfqpoint{0.795485in}{0.853753in}}%
\pgfpathlineto{\pgfqpoint{0.799774in}{0.848981in}}%
\pgfpathlineto{\pgfqpoint{0.811142in}{0.837223in}}%
\pgfpathlineto{\pgfqpoint{0.813027in}{0.835370in}}%
\pgfpathlineto{\pgfqpoint{0.826798in}{0.822686in}}%
\pgfpathlineto{\pgfqpoint{0.827865in}{0.821759in}}%
\pgfpathlineto{\pgfqpoint{0.842455in}{0.809786in}}%
\pgfpathlineto{\pgfqpoint{0.844586in}{0.808148in}}%
\pgfpathlineto{\pgfqpoint{0.858112in}{0.798265in}}%
\pgfpathlineto{\pgfqpoint{0.863600in}{0.794536in}}%
\pgfpathlineto{\pgfqpoint{0.873768in}{0.787933in}}%
\pgfpathlineto{\pgfqpoint{0.885481in}{0.780925in}}%
\pgfpathlineto{\pgfqpoint{0.889425in}{0.778656in}}%
\pgfpathlineto{\pgfqpoint{0.905081in}{0.770419in}}%
\pgfpathlineto{\pgfqpoint{0.911617in}{0.767314in}}%
\pgfpathlineto{\pgfqpoint{0.920738in}{0.763121in}}%
\pgfpathlineto{\pgfqpoint{0.936394in}{0.756707in}}%
\pgfpathlineto{\pgfqpoint{0.944760in}{0.753703in}}%
\pgfpathlineto{\pgfqpoint{0.952051in}{0.751154in}}%
\pgfpathlineto{\pgfqpoint{0.967708in}{0.746452in}}%
\pgfpathlineto{\pgfqpoint{0.983364in}{0.742535in}}%
\pgfpathlineto{\pgfqpoint{0.995568in}{0.740092in}}%
\pgfpathlineto{\pgfqpoint{0.999021in}{0.739415in}}%
\pgfpathclose%
\pgfpathmoveto{\pgfqpoint{1.003022in}{0.876203in}}%
\pgfpathlineto{\pgfqpoint{0.999021in}{0.877276in}}%
\pgfpathlineto{\pgfqpoint{0.983364in}{0.882865in}}%
\pgfpathlineto{\pgfqpoint{0.967797in}{0.889814in}}%
\pgfpathlineto{\pgfqpoint{0.967708in}{0.889858in}}%
\pgfpathlineto{\pgfqpoint{0.952051in}{0.899059in}}%
\pgfpathlineto{\pgfqpoint{0.945666in}{0.903425in}}%
\pgfpathlineto{\pgfqpoint{0.936394in}{0.910510in}}%
\pgfpathlineto{\pgfqpoint{0.928888in}{0.917036in}}%
\pgfpathlineto{\pgfqpoint{0.920738in}{0.925096in}}%
\pgfpathlineto{\pgfqpoint{0.915715in}{0.930648in}}%
\pgfpathlineto{\pgfqpoint{0.905132in}{0.944259in}}%
\pgfpathlineto{\pgfqpoint{0.905081in}{0.944336in}}%
\pgfpathlineto{\pgfqpoint{0.897088in}{0.957870in}}%
\pgfpathlineto{\pgfqpoint{0.890659in}{0.971481in}}%
\pgfpathlineto{\pgfqpoint{0.889425in}{0.974959in}}%
\pgfpathlineto{\pgfqpoint{0.886116in}{0.985092in}}%
\pgfpathlineto{\pgfqpoint{0.883152in}{0.998703in}}%
\pgfpathlineto{\pgfqpoint{0.881671in}{1.012314in}}%
\pgfpathlineto{\pgfqpoint{0.881671in}{1.025925in}}%
\pgfpathlineto{\pgfqpoint{0.883152in}{1.039536in}}%
\pgfpathlineto{\pgfqpoint{0.886116in}{1.053148in}}%
\pgfpathlineto{\pgfqpoint{0.889425in}{1.063280in}}%
\pgfpathlineto{\pgfqpoint{0.890659in}{1.066759in}}%
\pgfpathlineto{\pgfqpoint{0.897088in}{1.080370in}}%
\pgfpathlineto{\pgfqpoint{0.905081in}{1.093903in}}%
\pgfpathlineto{\pgfqpoint{0.905132in}{1.093981in}}%
\pgfpathlineto{\pgfqpoint{0.915715in}{1.107592in}}%
\pgfpathlineto{\pgfqpoint{0.920738in}{1.113143in}}%
\pgfpathlineto{\pgfqpoint{0.928888in}{1.121203in}}%
\pgfpathlineto{\pgfqpoint{0.936394in}{1.127729in}}%
\pgfpathlineto{\pgfqpoint{0.945666in}{1.134814in}}%
\pgfpathlineto{\pgfqpoint{0.952051in}{1.139181in}}%
\pgfpathlineto{\pgfqpoint{0.967708in}{1.148381in}}%
\pgfpathlineto{\pgfqpoint{0.967797in}{1.148425in}}%
\pgfpathlineto{\pgfqpoint{0.983364in}{1.155375in}}%
\pgfpathlineto{\pgfqpoint{0.999021in}{1.160964in}}%
\pgfpathlineto{\pgfqpoint{1.003022in}{1.162036in}}%
\pgfpathlineto{\pgfqpoint{1.014677in}{1.164913in}}%
\pgfpathlineto{\pgfqpoint{1.030334in}{1.167489in}}%
\pgfpathlineto{\pgfqpoint{1.045990in}{1.168777in}}%
\pgfpathlineto{\pgfqpoint{1.061647in}{1.168777in}}%
\pgfpathlineto{\pgfqpoint{1.077303in}{1.167489in}}%
\pgfpathlineto{\pgfqpoint{1.092960in}{1.164913in}}%
\pgfpathlineto{\pgfqpoint{1.104615in}{1.162036in}}%
\pgfpathlineto{\pgfqpoint{1.108617in}{1.160964in}}%
\pgfpathlineto{\pgfqpoint{1.124273in}{1.155375in}}%
\pgfpathlineto{\pgfqpoint{1.139841in}{1.148425in}}%
\pgfpathlineto{\pgfqpoint{1.139930in}{1.148381in}}%
\pgfpathlineto{\pgfqpoint{1.155586in}{1.139181in}}%
\pgfpathlineto{\pgfqpoint{1.161972in}{1.134814in}}%
\pgfpathlineto{\pgfqpoint{1.171243in}{1.127729in}}%
\pgfpathlineto{\pgfqpoint{1.178750in}{1.121203in}}%
\pgfpathlineto{\pgfqpoint{1.186899in}{1.113143in}}%
\pgfpathlineto{\pgfqpoint{1.191922in}{1.107592in}}%
\pgfpathlineto{\pgfqpoint{1.202506in}{1.093981in}}%
\pgfpathlineto{\pgfqpoint{1.202556in}{1.093903in}}%
\pgfpathlineto{\pgfqpoint{1.210550in}{1.080370in}}%
\pgfpathlineto{\pgfqpoint{1.216979in}{1.066759in}}%
\pgfpathlineto{\pgfqpoint{1.218213in}{1.063280in}}%
\pgfpathlineto{\pgfqpoint{1.221521in}{1.053148in}}%
\pgfpathlineto{\pgfqpoint{1.224485in}{1.039536in}}%
\pgfpathlineto{\pgfqpoint{1.225967in}{1.025925in}}%
\pgfpathlineto{\pgfqpoint{1.225967in}{1.012314in}}%
\pgfpathlineto{\pgfqpoint{1.224485in}{0.998703in}}%
\pgfpathlineto{\pgfqpoint{1.221521in}{0.985092in}}%
\pgfpathlineto{\pgfqpoint{1.218213in}{0.974959in}}%
\pgfpathlineto{\pgfqpoint{1.216979in}{0.971481in}}%
\pgfpathlineto{\pgfqpoint{1.210550in}{0.957870in}}%
\pgfpathlineto{\pgfqpoint{1.202556in}{0.944336in}}%
\pgfpathlineto{\pgfqpoint{1.202506in}{0.944259in}}%
\pgfpathlineto{\pgfqpoint{1.191922in}{0.930648in}}%
\pgfpathlineto{\pgfqpoint{1.186899in}{0.925096in}}%
\pgfpathlineto{\pgfqpoint{1.178750in}{0.917036in}}%
\pgfpathlineto{\pgfqpoint{1.171243in}{0.910510in}}%
\pgfpathlineto{\pgfqpoint{1.161972in}{0.903425in}}%
\pgfpathlineto{\pgfqpoint{1.155586in}{0.899059in}}%
\pgfpathlineto{\pgfqpoint{1.139930in}{0.889858in}}%
\pgfpathlineto{\pgfqpoint{1.139841in}{0.889814in}}%
\pgfpathlineto{\pgfqpoint{1.124273in}{0.882865in}}%
\pgfpathlineto{\pgfqpoint{1.108617in}{0.877276in}}%
\pgfpathlineto{\pgfqpoint{1.104615in}{0.876203in}}%
\pgfpathlineto{\pgfqpoint{1.092960in}{0.873327in}}%
\pgfpathlineto{\pgfqpoint{1.077303in}{0.870750in}}%
\pgfpathlineto{\pgfqpoint{1.061647in}{0.869462in}}%
\pgfpathlineto{\pgfqpoint{1.045990in}{0.869462in}}%
\pgfpathlineto{\pgfqpoint{1.030334in}{0.870750in}}%
\pgfpathlineto{\pgfqpoint{1.014677in}{0.873327in}}%
\pgfpathlineto{\pgfqpoint{1.003022in}{0.876203in}}%
\pgfpathclose%
\pgfusepath{fill}%
\end{pgfscope}%
\begin{pgfscope}%
\pgfpathrectangle{\pgfqpoint{0.278819in}{0.345370in}}{\pgfqpoint{1.550000in}{1.347500in}}%
\pgfusepath{clip}%
\pgfsetbuttcap%
\pgfsetroundjoin%
\definecolor{currentfill}{rgb}{0.921884,0.341098,0.377376}%
\pgfsetfillcolor{currentfill}%
\pgfsetlinewidth{0.000000pt}%
\definecolor{currentstroke}{rgb}{0.000000,0.000000,0.000000}%
\pgfsetstrokecolor{currentstroke}%
\pgfsetdash{}{0pt}%
\pgfpathmoveto{\pgfqpoint{0.967708in}{0.631202in}}%
\pgfpathlineto{\pgfqpoint{0.983364in}{0.627741in}}%
\pgfpathlineto{\pgfqpoint{0.999021in}{0.624975in}}%
\pgfpathlineto{\pgfqpoint{1.014677in}{0.622900in}}%
\pgfpathlineto{\pgfqpoint{1.030334in}{0.621518in}}%
\pgfpathlineto{\pgfqpoint{1.045990in}{0.620826in}}%
\pgfpathlineto{\pgfqpoint{1.061647in}{0.620826in}}%
\pgfpathlineto{\pgfqpoint{1.077303in}{0.621518in}}%
\pgfpathlineto{\pgfqpoint{1.092960in}{0.622900in}}%
\pgfpathlineto{\pgfqpoint{1.108617in}{0.624975in}}%
\pgfpathlineto{\pgfqpoint{1.124273in}{0.627741in}}%
\pgfpathlineto{\pgfqpoint{1.139930in}{0.631202in}}%
\pgfpathlineto{\pgfqpoint{1.139935in}{0.631203in}}%
\pgfpathlineto{\pgfqpoint{1.155586in}{0.635291in}}%
\pgfpathlineto{\pgfqpoint{1.171243in}{0.640064in}}%
\pgfpathlineto{\pgfqpoint{1.184880in}{0.644814in}}%
\pgfpathlineto{\pgfqpoint{1.186899in}{0.645513in}}%
\pgfpathlineto{\pgfqpoint{1.202556in}{0.651585in}}%
\pgfpathlineto{\pgfqpoint{1.218213in}{0.658335in}}%
\pgfpathlineto{\pgfqpoint{1.218405in}{0.658425in}}%
\pgfpathlineto{\pgfqpoint{1.233869in}{0.665706in}}%
\pgfpathlineto{\pgfqpoint{1.246218in}{0.672036in}}%
\pgfpathlineto{\pgfqpoint{1.249526in}{0.673741in}}%
\pgfpathlineto{\pgfqpoint{1.265182in}{0.682419in}}%
\pgfpathlineto{\pgfqpoint{1.270627in}{0.685648in}}%
\pgfpathlineto{\pgfqpoint{1.280839in}{0.691764in}}%
\pgfpathlineto{\pgfqpoint{1.292598in}{0.699259in}}%
\pgfpathlineto{\pgfqpoint{1.296495in}{0.701780in}}%
\pgfpathlineto{\pgfqpoint{1.312152in}{0.712485in}}%
\pgfpathlineto{\pgfqpoint{1.312688in}{0.712870in}}%
\pgfpathlineto{\pgfqpoint{1.327809in}{0.723931in}}%
\pgfpathlineto{\pgfqpoint{1.331141in}{0.726481in}}%
\pgfpathlineto{\pgfqpoint{1.343465in}{0.736144in}}%
\pgfpathlineto{\pgfqpoint{1.348306in}{0.740092in}}%
\pgfpathlineto{\pgfqpoint{1.359122in}{0.749177in}}%
\pgfpathlineto{\pgfqpoint{1.364327in}{0.753703in}}%
\pgfpathlineto{\pgfqpoint{1.374778in}{0.763106in}}%
\pgfpathlineto{\pgfqpoint{1.379320in}{0.767314in}}%
\pgfpathlineto{\pgfqpoint{1.390435in}{0.778028in}}%
\pgfpathlineto{\pgfqpoint{1.393367in}{0.780925in}}%
\pgfpathlineto{\pgfqpoint{1.406091in}{0.794070in}}%
\pgfpathlineto{\pgfqpoint{1.406534in}{0.794536in}}%
\pgfpathlineto{\pgfqpoint{1.418847in}{0.808148in}}%
\pgfpathlineto{\pgfqpoint{1.421748in}{0.811536in}}%
\pgfpathlineto{\pgfqpoint{1.430369in}{0.821759in}}%
\pgfpathlineto{\pgfqpoint{1.437404in}{0.830636in}}%
\pgfpathlineto{\pgfqpoint{1.441118in}{0.835370in}}%
\pgfpathlineto{\pgfqpoint{1.451101in}{0.848981in}}%
\pgfpathlineto{\pgfqpoint{1.453061in}{0.851856in}}%
\pgfpathlineto{\pgfqpoint{1.460343in}{0.862592in}}%
\pgfpathlineto{\pgfqpoint{1.468718in}{0.876036in}}%
\pgfpathlineto{\pgfqpoint{1.468822in}{0.876203in}}%
\pgfpathlineto{\pgfqpoint{1.476585in}{0.889814in}}%
\pgfpathlineto{\pgfqpoint{1.483570in}{0.903425in}}%
\pgfpathlineto{\pgfqpoint{1.484374in}{0.905180in}}%
\pgfpathlineto{\pgfqpoint{1.489838in}{0.917036in}}%
\pgfpathlineto{\pgfqpoint{1.495329in}{0.930648in}}%
\pgfpathlineto{\pgfqpoint{1.500031in}{0.944254in}}%
\pgfpathlineto{\pgfqpoint{1.500032in}{0.944259in}}%
\pgfpathlineto{\pgfqpoint{1.504013in}{0.957870in}}%
\pgfpathlineto{\pgfqpoint{1.507195in}{0.971481in}}%
\pgfpathlineto{\pgfqpoint{1.509581in}{0.985092in}}%
\pgfpathlineto{\pgfqpoint{1.511172in}{0.998703in}}%
\pgfpathlineto{\pgfqpoint{1.511967in}{1.012314in}}%
\pgfpathlineto{\pgfqpoint{1.511967in}{1.025925in}}%
\pgfpathlineto{\pgfqpoint{1.511172in}{1.039536in}}%
\pgfpathlineto{\pgfqpoint{1.509581in}{1.053148in}}%
\pgfpathlineto{\pgfqpoint{1.507195in}{1.066759in}}%
\pgfpathlineto{\pgfqpoint{1.504013in}{1.080370in}}%
\pgfpathlineto{\pgfqpoint{1.500032in}{1.093981in}}%
\pgfpathlineto{\pgfqpoint{1.500031in}{1.093986in}}%
\pgfpathlineto{\pgfqpoint{1.495329in}{1.107592in}}%
\pgfpathlineto{\pgfqpoint{1.489838in}{1.121203in}}%
\pgfpathlineto{\pgfqpoint{1.484374in}{1.133059in}}%
\pgfpathlineto{\pgfqpoint{1.483570in}{1.134814in}}%
\pgfpathlineto{\pgfqpoint{1.476585in}{1.148425in}}%
\pgfpathlineto{\pgfqpoint{1.468822in}{1.162036in}}%
\pgfpathlineto{\pgfqpoint{1.468718in}{1.162204in}}%
\pgfpathlineto{\pgfqpoint{1.460343in}{1.175647in}}%
\pgfpathlineto{\pgfqpoint{1.453061in}{1.186383in}}%
\pgfpathlineto{\pgfqpoint{1.451101in}{1.189259in}}%
\pgfpathlineto{\pgfqpoint{1.441118in}{1.202870in}}%
\pgfpathlineto{\pgfqpoint{1.437404in}{1.207603in}}%
\pgfpathlineto{\pgfqpoint{1.430369in}{1.216481in}}%
\pgfpathlineto{\pgfqpoint{1.421748in}{1.226703in}}%
\pgfpathlineto{\pgfqpoint{1.418847in}{1.230092in}}%
\pgfpathlineto{\pgfqpoint{1.406534in}{1.243703in}}%
\pgfpathlineto{\pgfqpoint{1.406091in}{1.244169in}}%
\pgfpathlineto{\pgfqpoint{1.393367in}{1.257314in}}%
\pgfpathlineto{\pgfqpoint{1.390435in}{1.260212in}}%
\pgfpathlineto{\pgfqpoint{1.379320in}{1.270925in}}%
\pgfpathlineto{\pgfqpoint{1.374778in}{1.275134in}}%
\pgfpathlineto{\pgfqpoint{1.364327in}{1.284536in}}%
\pgfpathlineto{\pgfqpoint{1.359122in}{1.289062in}}%
\pgfpathlineto{\pgfqpoint{1.348306in}{1.298148in}}%
\pgfpathlineto{\pgfqpoint{1.343465in}{1.302096in}}%
\pgfpathlineto{\pgfqpoint{1.331141in}{1.311759in}}%
\pgfpathlineto{\pgfqpoint{1.327809in}{1.314308in}}%
\pgfpathlineto{\pgfqpoint{1.312688in}{1.325370in}}%
\pgfpathlineto{\pgfqpoint{1.312152in}{1.325754in}}%
\pgfpathlineto{\pgfqpoint{1.296495in}{1.336459in}}%
\pgfpathlineto{\pgfqpoint{1.292598in}{1.338981in}}%
\pgfpathlineto{\pgfqpoint{1.280839in}{1.346475in}}%
\pgfpathlineto{\pgfqpoint{1.270627in}{1.352592in}}%
\pgfpathlineto{\pgfqpoint{1.265182in}{1.355820in}}%
\pgfpathlineto{\pgfqpoint{1.249526in}{1.364499in}}%
\pgfpathlineto{\pgfqpoint{1.246218in}{1.366203in}}%
\pgfpathlineto{\pgfqpoint{1.233869in}{1.372534in}}%
\pgfpathlineto{\pgfqpoint{1.218405in}{1.379814in}}%
\pgfpathlineto{\pgfqpoint{1.218213in}{1.379905in}}%
\pgfpathlineto{\pgfqpoint{1.202556in}{1.386654in}}%
\pgfpathlineto{\pgfqpoint{1.186899in}{1.392726in}}%
\pgfpathlineto{\pgfqpoint{1.184880in}{1.393425in}}%
\pgfpathlineto{\pgfqpoint{1.171243in}{1.398176in}}%
\pgfpathlineto{\pgfqpoint{1.155586in}{1.402949in}}%
\pgfpathlineto{\pgfqpoint{1.139935in}{1.407036in}}%
\pgfpathlineto{\pgfqpoint{1.139930in}{1.407038in}}%
\pgfpathlineto{\pgfqpoint{1.124273in}{1.410498in}}%
\pgfpathlineto{\pgfqpoint{1.108617in}{1.413265in}}%
\pgfpathlineto{\pgfqpoint{1.092960in}{1.415339in}}%
\pgfpathlineto{\pgfqpoint{1.077303in}{1.416722in}}%
\pgfpathlineto{\pgfqpoint{1.061647in}{1.417413in}}%
\pgfpathlineto{\pgfqpoint{1.045990in}{1.417413in}}%
\pgfpathlineto{\pgfqpoint{1.030334in}{1.416722in}}%
\pgfpathlineto{\pgfqpoint{1.014677in}{1.415339in}}%
\pgfpathlineto{\pgfqpoint{0.999021in}{1.413265in}}%
\pgfpathlineto{\pgfqpoint{0.983364in}{1.410498in}}%
\pgfpathlineto{\pgfqpoint{0.967707in}{1.407038in}}%
\pgfpathlineto{\pgfqpoint{0.967702in}{1.407036in}}%
\pgfpathlineto{\pgfqpoint{0.952051in}{1.402949in}}%
\pgfpathlineto{\pgfqpoint{0.936394in}{1.398176in}}%
\pgfpathlineto{\pgfqpoint{0.922757in}{1.393425in}}%
\pgfpathlineto{\pgfqpoint{0.920738in}{1.392726in}}%
\pgfpathlineto{\pgfqpoint{0.905081in}{1.386654in}}%
\pgfpathlineto{\pgfqpoint{0.889425in}{1.379905in}}%
\pgfpathlineto{\pgfqpoint{0.889232in}{1.379814in}}%
\pgfpathlineto{\pgfqpoint{0.873768in}{1.372534in}}%
\pgfpathlineto{\pgfqpoint{0.861419in}{1.366203in}}%
\pgfpathlineto{\pgfqpoint{0.858112in}{1.364499in}}%
\pgfpathlineto{\pgfqpoint{0.842455in}{1.355820in}}%
\pgfpathlineto{\pgfqpoint{0.837010in}{1.352592in}}%
\pgfpathlineto{\pgfqpoint{0.826798in}{1.346475in}}%
\pgfpathlineto{\pgfqpoint{0.815040in}{1.338981in}}%
\pgfpathlineto{\pgfqpoint{0.811142in}{1.336459in}}%
\pgfpathlineto{\pgfqpoint{0.795485in}{1.325754in}}%
\pgfpathlineto{\pgfqpoint{0.794949in}{1.325370in}}%
\pgfpathlineto{\pgfqpoint{0.779829in}{1.314308in}}%
\pgfpathlineto{\pgfqpoint{0.776496in}{1.311759in}}%
\pgfpathlineto{\pgfqpoint{0.764172in}{1.302096in}}%
\pgfpathlineto{\pgfqpoint{0.759331in}{1.298148in}}%
\pgfpathlineto{\pgfqpoint{0.748516in}{1.289062in}}%
\pgfpathlineto{\pgfqpoint{0.743310in}{1.284536in}}%
\pgfpathlineto{\pgfqpoint{0.732859in}{1.275134in}}%
\pgfpathlineto{\pgfqpoint{0.728318in}{1.270925in}}%
\pgfpathlineto{\pgfqpoint{0.717202in}{1.260212in}}%
\pgfpathlineto{\pgfqpoint{0.714270in}{1.257314in}}%
\pgfpathlineto{\pgfqpoint{0.701546in}{1.244169in}}%
\pgfpathlineto{\pgfqpoint{0.701104in}{1.243703in}}%
\pgfpathlineto{\pgfqpoint{0.688790in}{1.230092in}}%
\pgfpathlineto{\pgfqpoint{0.685889in}{1.226703in}}%
\pgfpathlineto{\pgfqpoint{0.677269in}{1.216481in}}%
\pgfpathlineto{\pgfqpoint{0.670233in}{1.207603in}}%
\pgfpathlineto{\pgfqpoint{0.666520in}{1.202870in}}%
\pgfpathlineto{\pgfqpoint{0.656536in}{1.189259in}}%
\pgfpathlineto{\pgfqpoint{0.654576in}{1.186383in}}%
\pgfpathlineto{\pgfqpoint{0.647294in}{1.175647in}}%
\pgfpathlineto{\pgfqpoint{0.638920in}{1.162204in}}%
\pgfpathlineto{\pgfqpoint{0.638815in}{1.162036in}}%
\pgfpathlineto{\pgfqpoint{0.631052in}{1.148425in}}%
\pgfpathlineto{\pgfqpoint{0.624067in}{1.134814in}}%
\pgfpathlineto{\pgfqpoint{0.623263in}{1.133059in}}%
\pgfpathlineto{\pgfqpoint{0.617799in}{1.121203in}}%
\pgfpathlineto{\pgfqpoint{0.612308in}{1.107592in}}%
\pgfpathlineto{\pgfqpoint{0.607606in}{1.093986in}}%
\pgfpathlineto{\pgfqpoint{0.607605in}{1.093981in}}%
\pgfpathlineto{\pgfqpoint{0.603625in}{1.080370in}}%
\pgfpathlineto{\pgfqpoint{0.600442in}{1.066759in}}%
\pgfpathlineto{\pgfqpoint{0.598056in}{1.053148in}}%
\pgfpathlineto{\pgfqpoint{0.596466in}{1.039536in}}%
\pgfpathlineto{\pgfqpoint{0.595671in}{1.025925in}}%
\pgfpathlineto{\pgfqpoint{0.595671in}{1.012314in}}%
\pgfpathlineto{\pgfqpoint{0.596466in}{0.998703in}}%
\pgfpathlineto{\pgfqpoint{0.598056in}{0.985092in}}%
\pgfpathlineto{\pgfqpoint{0.600442in}{0.971481in}}%
\pgfpathlineto{\pgfqpoint{0.603625in}{0.957870in}}%
\pgfpathlineto{\pgfqpoint{0.607605in}{0.944259in}}%
\pgfpathlineto{\pgfqpoint{0.607606in}{0.944254in}}%
\pgfpathlineto{\pgfqpoint{0.612308in}{0.930648in}}%
\pgfpathlineto{\pgfqpoint{0.617799in}{0.917036in}}%
\pgfpathlineto{\pgfqpoint{0.623263in}{0.905180in}}%
\pgfpathlineto{\pgfqpoint{0.624067in}{0.903425in}}%
\pgfpathlineto{\pgfqpoint{0.631052in}{0.889814in}}%
\pgfpathlineto{\pgfqpoint{0.638815in}{0.876203in}}%
\pgfpathlineto{\pgfqpoint{0.638920in}{0.876036in}}%
\pgfpathlineto{\pgfqpoint{0.647294in}{0.862592in}}%
\pgfpathlineto{\pgfqpoint{0.654576in}{0.851856in}}%
\pgfpathlineto{\pgfqpoint{0.656536in}{0.848981in}}%
\pgfpathlineto{\pgfqpoint{0.666520in}{0.835370in}}%
\pgfpathlineto{\pgfqpoint{0.670233in}{0.830636in}}%
\pgfpathlineto{\pgfqpoint{0.677269in}{0.821759in}}%
\pgfpathlineto{\pgfqpoint{0.685889in}{0.811536in}}%
\pgfpathlineto{\pgfqpoint{0.688790in}{0.808148in}}%
\pgfpathlineto{\pgfqpoint{0.701104in}{0.794536in}}%
\pgfpathlineto{\pgfqpoint{0.701546in}{0.794070in}}%
\pgfpathlineto{\pgfqpoint{0.714270in}{0.780925in}}%
\pgfpathlineto{\pgfqpoint{0.717202in}{0.778028in}}%
\pgfpathlineto{\pgfqpoint{0.728318in}{0.767314in}}%
\pgfpathlineto{\pgfqpoint{0.732859in}{0.763106in}}%
\pgfpathlineto{\pgfqpoint{0.743310in}{0.753703in}}%
\pgfpathlineto{\pgfqpoint{0.748516in}{0.749177in}}%
\pgfpathlineto{\pgfqpoint{0.759331in}{0.740092in}}%
\pgfpathlineto{\pgfqpoint{0.764172in}{0.736144in}}%
\pgfpathlineto{\pgfqpoint{0.776496in}{0.726481in}}%
\pgfpathlineto{\pgfqpoint{0.779829in}{0.723931in}}%
\pgfpathlineto{\pgfqpoint{0.794949in}{0.712870in}}%
\pgfpathlineto{\pgfqpoint{0.795485in}{0.712485in}}%
\pgfpathlineto{\pgfqpoint{0.811142in}{0.701780in}}%
\pgfpathlineto{\pgfqpoint{0.815040in}{0.699259in}}%
\pgfpathlineto{\pgfqpoint{0.826798in}{0.691764in}}%
\pgfpathlineto{\pgfqpoint{0.837010in}{0.685648in}}%
\pgfpathlineto{\pgfqpoint{0.842455in}{0.682419in}}%
\pgfpathlineto{\pgfqpoint{0.858112in}{0.673741in}}%
\pgfpathlineto{\pgfqpoint{0.861419in}{0.672036in}}%
\pgfpathlineto{\pgfqpoint{0.873768in}{0.665706in}}%
\pgfpathlineto{\pgfqpoint{0.889232in}{0.658425in}}%
\pgfpathlineto{\pgfqpoint{0.889425in}{0.658335in}}%
\pgfpathlineto{\pgfqpoint{0.905081in}{0.651585in}}%
\pgfpathlineto{\pgfqpoint{0.920738in}{0.645513in}}%
\pgfpathlineto{\pgfqpoint{0.922757in}{0.644814in}}%
\pgfpathlineto{\pgfqpoint{0.936394in}{0.640064in}}%
\pgfpathlineto{\pgfqpoint{0.952051in}{0.635291in}}%
\pgfpathlineto{\pgfqpoint{0.967702in}{0.631203in}}%
\pgfpathlineto{\pgfqpoint{0.967708in}{0.631202in}}%
\pgfpathclose%
\pgfpathmoveto{\pgfqpoint{0.995568in}{0.740092in}}%
\pgfpathlineto{\pgfqpoint{0.983364in}{0.742535in}}%
\pgfpathlineto{\pgfqpoint{0.967708in}{0.746452in}}%
\pgfpathlineto{\pgfqpoint{0.952051in}{0.751154in}}%
\pgfpathlineto{\pgfqpoint{0.944760in}{0.753703in}}%
\pgfpathlineto{\pgfqpoint{0.936394in}{0.756707in}}%
\pgfpathlineto{\pgfqpoint{0.920738in}{0.763121in}}%
\pgfpathlineto{\pgfqpoint{0.911617in}{0.767314in}}%
\pgfpathlineto{\pgfqpoint{0.905081in}{0.770419in}}%
\pgfpathlineto{\pgfqpoint{0.889425in}{0.778656in}}%
\pgfpathlineto{\pgfqpoint{0.885481in}{0.780925in}}%
\pgfpathlineto{\pgfqpoint{0.873768in}{0.787933in}}%
\pgfpathlineto{\pgfqpoint{0.863600in}{0.794536in}}%
\pgfpathlineto{\pgfqpoint{0.858112in}{0.798265in}}%
\pgfpathlineto{\pgfqpoint{0.844586in}{0.808148in}}%
\pgfpathlineto{\pgfqpoint{0.842455in}{0.809786in}}%
\pgfpathlineto{\pgfqpoint{0.827865in}{0.821759in}}%
\pgfpathlineto{\pgfqpoint{0.826798in}{0.822686in}}%
\pgfpathlineto{\pgfqpoint{0.813027in}{0.835370in}}%
\pgfpathlineto{\pgfqpoint{0.811142in}{0.837223in}}%
\pgfpathlineto{\pgfqpoint{0.799774in}{0.848981in}}%
\pgfpathlineto{\pgfqpoint{0.795485in}{0.853753in}}%
\pgfpathlineto{\pgfqpoint{0.787889in}{0.862592in}}%
\pgfpathlineto{\pgfqpoint{0.779829in}{0.872774in}}%
\pgfpathlineto{\pgfqpoint{0.777218in}{0.876203in}}%
\pgfpathlineto{\pgfqpoint{0.767744in}{0.889814in}}%
\pgfpathlineto{\pgfqpoint{0.764172in}{0.895496in}}%
\pgfpathlineto{\pgfqpoint{0.759348in}{0.903425in}}%
\pgfpathlineto{\pgfqpoint{0.751971in}{0.917036in}}%
\pgfpathlineto{\pgfqpoint{0.748516in}{0.924309in}}%
\pgfpathlineto{\pgfqpoint{0.745583in}{0.930648in}}%
\pgfpathlineto{\pgfqpoint{0.740174in}{0.944259in}}%
\pgfpathlineto{\pgfqpoint{0.735669in}{0.957870in}}%
\pgfpathlineto{\pgfqpoint{0.732859in}{0.968479in}}%
\pgfpathlineto{\pgfqpoint{0.732080in}{0.971481in}}%
\pgfpathlineto{\pgfqpoint{0.729425in}{0.985092in}}%
\pgfpathlineto{\pgfqpoint{0.727655in}{0.998703in}}%
\pgfpathlineto{\pgfqpoint{0.726771in}{1.012314in}}%
\pgfpathlineto{\pgfqpoint{0.726771in}{1.025925in}}%
\pgfpathlineto{\pgfqpoint{0.727655in}{1.039536in}}%
\pgfpathlineto{\pgfqpoint{0.729425in}{1.053148in}}%
\pgfpathlineto{\pgfqpoint{0.732080in}{1.066759in}}%
\pgfpathlineto{\pgfqpoint{0.732859in}{1.069760in}}%
\pgfpathlineto{\pgfqpoint{0.735669in}{1.080370in}}%
\pgfpathlineto{\pgfqpoint{0.740174in}{1.093981in}}%
\pgfpathlineto{\pgfqpoint{0.745583in}{1.107592in}}%
\pgfpathlineto{\pgfqpoint{0.748516in}{1.113930in}}%
\pgfpathlineto{\pgfqpoint{0.751971in}{1.121203in}}%
\pgfpathlineto{\pgfqpoint{0.759348in}{1.134814in}}%
\pgfpathlineto{\pgfqpoint{0.764172in}{1.142743in}}%
\pgfpathlineto{\pgfqpoint{0.767744in}{1.148425in}}%
\pgfpathlineto{\pgfqpoint{0.777218in}{1.162036in}}%
\pgfpathlineto{\pgfqpoint{0.779829in}{1.165465in}}%
\pgfpathlineto{\pgfqpoint{0.787889in}{1.175647in}}%
\pgfpathlineto{\pgfqpoint{0.795485in}{1.184487in}}%
\pgfpathlineto{\pgfqpoint{0.799774in}{1.189259in}}%
\pgfpathlineto{\pgfqpoint{0.811142in}{1.201017in}}%
\pgfpathlineto{\pgfqpoint{0.813027in}{1.202870in}}%
\pgfpathlineto{\pgfqpoint{0.826798in}{1.215553in}}%
\pgfpathlineto{\pgfqpoint{0.827865in}{1.216481in}}%
\pgfpathlineto{\pgfqpoint{0.842455in}{1.228453in}}%
\pgfpathlineto{\pgfqpoint{0.844586in}{1.230092in}}%
\pgfpathlineto{\pgfqpoint{0.858112in}{1.239975in}}%
\pgfpathlineto{\pgfqpoint{0.863600in}{1.243703in}}%
\pgfpathlineto{\pgfqpoint{0.873768in}{1.250307in}}%
\pgfpathlineto{\pgfqpoint{0.885481in}{1.257314in}}%
\pgfpathlineto{\pgfqpoint{0.889425in}{1.259584in}}%
\pgfpathlineto{\pgfqpoint{0.905081in}{1.267820in}}%
\pgfpathlineto{\pgfqpoint{0.911617in}{1.270925in}}%
\pgfpathlineto{\pgfqpoint{0.920738in}{1.275119in}}%
\pgfpathlineto{\pgfqpoint{0.936394in}{1.281532in}}%
\pgfpathlineto{\pgfqpoint{0.944760in}{1.284536in}}%
\pgfpathlineto{\pgfqpoint{0.952051in}{1.287085in}}%
\pgfpathlineto{\pgfqpoint{0.967708in}{1.291788in}}%
\pgfpathlineto{\pgfqpoint{0.983364in}{1.295705in}}%
\pgfpathlineto{\pgfqpoint{0.995568in}{1.298148in}}%
\pgfpathlineto{\pgfqpoint{0.999021in}{1.298825in}}%
\pgfpathlineto{\pgfqpoint{1.014677in}{1.301133in}}%
\pgfpathlineto{\pgfqpoint{1.030334in}{1.302671in}}%
\pgfpathlineto{\pgfqpoint{1.045990in}{1.303440in}}%
\pgfpathlineto{\pgfqpoint{1.061647in}{1.303440in}}%
\pgfpathlineto{\pgfqpoint{1.077303in}{1.302671in}}%
\pgfpathlineto{\pgfqpoint{1.092960in}{1.301133in}}%
\pgfpathlineto{\pgfqpoint{1.108617in}{1.298825in}}%
\pgfpathlineto{\pgfqpoint{1.112069in}{1.298148in}}%
\pgfpathlineto{\pgfqpoint{1.124273in}{1.295705in}}%
\pgfpathlineto{\pgfqpoint{1.139930in}{1.291788in}}%
\pgfpathlineto{\pgfqpoint{1.155586in}{1.287085in}}%
\pgfpathlineto{\pgfqpoint{1.162877in}{1.284536in}}%
\pgfpathlineto{\pgfqpoint{1.171243in}{1.281532in}}%
\pgfpathlineto{\pgfqpoint{1.186899in}{1.275119in}}%
\pgfpathlineto{\pgfqpoint{1.196020in}{1.270925in}}%
\pgfpathlineto{\pgfqpoint{1.202556in}{1.267820in}}%
\pgfpathlineto{\pgfqpoint{1.218213in}{1.259584in}}%
\pgfpathlineto{\pgfqpoint{1.222157in}{1.257314in}}%
\pgfpathlineto{\pgfqpoint{1.233869in}{1.250307in}}%
\pgfpathlineto{\pgfqpoint{1.244037in}{1.243703in}}%
\pgfpathlineto{\pgfqpoint{1.249526in}{1.239975in}}%
\pgfpathlineto{\pgfqpoint{1.263051in}{1.230092in}}%
\pgfpathlineto{\pgfqpoint{1.265182in}{1.228453in}}%
\pgfpathlineto{\pgfqpoint{1.279772in}{1.216481in}}%
\pgfpathlineto{\pgfqpoint{1.280839in}{1.215553in}}%
\pgfpathlineto{\pgfqpoint{1.294610in}{1.202870in}}%
\pgfpathlineto{\pgfqpoint{1.296495in}{1.201017in}}%
\pgfpathlineto{\pgfqpoint{1.307864in}{1.189259in}}%
\pgfpathlineto{\pgfqpoint{1.312152in}{1.184487in}}%
\pgfpathlineto{\pgfqpoint{1.319748in}{1.175647in}}%
\pgfpathlineto{\pgfqpoint{1.327809in}{1.165465in}}%
\pgfpathlineto{\pgfqpoint{1.330419in}{1.162036in}}%
\pgfpathlineto{\pgfqpoint{1.339893in}{1.148425in}}%
\pgfpathlineto{\pgfqpoint{1.343465in}{1.142743in}}%
\pgfpathlineto{\pgfqpoint{1.348289in}{1.134814in}}%
\pgfpathlineto{\pgfqpoint{1.355666in}{1.121203in}}%
\pgfpathlineto{\pgfqpoint{1.359122in}{1.113930in}}%
\pgfpathlineto{\pgfqpoint{1.362054in}{1.107592in}}%
\pgfpathlineto{\pgfqpoint{1.367463in}{1.093981in}}%
\pgfpathlineto{\pgfqpoint{1.371968in}{1.080370in}}%
\pgfpathlineto{\pgfqpoint{1.374778in}{1.069760in}}%
\pgfpathlineto{\pgfqpoint{1.375557in}{1.066759in}}%
\pgfpathlineto{\pgfqpoint{1.378212in}{1.053148in}}%
\pgfpathlineto{\pgfqpoint{1.379982in}{1.039536in}}%
\pgfpathlineto{\pgfqpoint{1.380866in}{1.025925in}}%
\pgfpathlineto{\pgfqpoint{1.380866in}{1.012314in}}%
\pgfpathlineto{\pgfqpoint{1.379982in}{0.998703in}}%
\pgfpathlineto{\pgfqpoint{1.378212in}{0.985092in}}%
\pgfpathlineto{\pgfqpoint{1.375557in}{0.971481in}}%
\pgfpathlineto{\pgfqpoint{1.374778in}{0.968479in}}%
\pgfpathlineto{\pgfqpoint{1.371968in}{0.957870in}}%
\pgfpathlineto{\pgfqpoint{1.367463in}{0.944259in}}%
\pgfpathlineto{\pgfqpoint{1.362054in}{0.930647in}}%
\pgfpathlineto{\pgfqpoint{1.359122in}{0.924309in}}%
\pgfpathlineto{\pgfqpoint{1.355666in}{0.917036in}}%
\pgfpathlineto{\pgfqpoint{1.348289in}{0.903425in}}%
\pgfpathlineto{\pgfqpoint{1.343465in}{0.895496in}}%
\pgfpathlineto{\pgfqpoint{1.339893in}{0.889814in}}%
\pgfpathlineto{\pgfqpoint{1.330419in}{0.876203in}}%
\pgfpathlineto{\pgfqpoint{1.327809in}{0.872774in}}%
\pgfpathlineto{\pgfqpoint{1.319748in}{0.862592in}}%
\pgfpathlineto{\pgfqpoint{1.312152in}{0.853753in}}%
\pgfpathlineto{\pgfqpoint{1.307864in}{0.848981in}}%
\pgfpathlineto{\pgfqpoint{1.296495in}{0.837223in}}%
\pgfpathlineto{\pgfqpoint{1.294610in}{0.835370in}}%
\pgfpathlineto{\pgfqpoint{1.280839in}{0.822686in}}%
\pgfpathlineto{\pgfqpoint{1.279772in}{0.821759in}}%
\pgfpathlineto{\pgfqpoint{1.265182in}{0.809786in}}%
\pgfpathlineto{\pgfqpoint{1.263051in}{0.808148in}}%
\pgfpathlineto{\pgfqpoint{1.249526in}{0.798265in}}%
\pgfpathlineto{\pgfqpoint{1.244037in}{0.794536in}}%
\pgfpathlineto{\pgfqpoint{1.233869in}{0.787933in}}%
\pgfpathlineto{\pgfqpoint{1.222157in}{0.780925in}}%
\pgfpathlineto{\pgfqpoint{1.218213in}{0.778656in}}%
\pgfpathlineto{\pgfqpoint{1.202556in}{0.770419in}}%
\pgfpathlineto{\pgfqpoint{1.196020in}{0.767314in}}%
\pgfpathlineto{\pgfqpoint{1.186899in}{0.763121in}}%
\pgfpathlineto{\pgfqpoint{1.171243in}{0.756707in}}%
\pgfpathlineto{\pgfqpoint{1.162877in}{0.753703in}}%
\pgfpathlineto{\pgfqpoint{1.155586in}{0.751154in}}%
\pgfpathlineto{\pgfqpoint{1.139930in}{0.746452in}}%
\pgfpathlineto{\pgfqpoint{1.124273in}{0.742535in}}%
\pgfpathlineto{\pgfqpoint{1.112069in}{0.740092in}}%
\pgfpathlineto{\pgfqpoint{1.108617in}{0.739415in}}%
\pgfpathlineto{\pgfqpoint{1.092960in}{0.737106in}}%
\pgfpathlineto{\pgfqpoint{1.077303in}{0.735568in}}%
\pgfpathlineto{\pgfqpoint{1.061647in}{0.734799in}}%
\pgfpathlineto{\pgfqpoint{1.045990in}{0.734799in}}%
\pgfpathlineto{\pgfqpoint{1.030334in}{0.735568in}}%
\pgfpathlineto{\pgfqpoint{1.014677in}{0.737106in}}%
\pgfpathlineto{\pgfqpoint{0.999021in}{0.739415in}}%
\pgfpathlineto{\pgfqpoint{0.995568in}{0.740092in}}%
\pgfpathclose%
\pgfusepath{fill}%
\end{pgfscope}%
\begin{pgfscope}%
\pgfpathrectangle{\pgfqpoint{0.278819in}{0.345370in}}{\pgfqpoint{1.550000in}{1.347500in}}%
\pgfusepath{clip}%
\pgfsetbuttcap%
\pgfsetroundjoin%
\definecolor{currentfill}{rgb}{0.709962,0.212797,0.477201}%
\pgfsetfillcolor{currentfill}%
\pgfsetlinewidth{0.000000pt}%
\definecolor{currentstroke}{rgb}{0.000000,0.000000,0.000000}%
\pgfsetstrokecolor{currentstroke}%
\pgfsetdash{}{0pt}%
\pgfpathmoveto{\pgfqpoint{1.030334in}{0.481176in}}%
\pgfpathlineto{\pgfqpoint{1.045990in}{0.480145in}}%
\pgfpathlineto{\pgfqpoint{1.061647in}{0.480145in}}%
\pgfpathlineto{\pgfqpoint{1.077303in}{0.481176in}}%
\pgfpathlineto{\pgfqpoint{1.079619in}{0.481481in}}%
\pgfpathlineto{\pgfqpoint{1.092960in}{0.483093in}}%
\pgfpathlineto{\pgfqpoint{1.108617in}{0.485931in}}%
\pgfpathlineto{\pgfqpoint{1.124273in}{0.489717in}}%
\pgfpathlineto{\pgfqpoint{1.139930in}{0.494452in}}%
\pgfpathlineto{\pgfqpoint{1.141701in}{0.495092in}}%
\pgfpathlineto{\pgfqpoint{1.155586in}{0.499767in}}%
\pgfpathlineto{\pgfqpoint{1.171243in}{0.505916in}}%
\pgfpathlineto{\pgfqpoint{1.177470in}{0.508703in}}%
\pgfpathlineto{\pgfqpoint{1.186899in}{0.512673in}}%
\pgfpathlineto{\pgfqpoint{1.202556in}{0.520075in}}%
\pgfpathlineto{\pgfqpoint{1.206839in}{0.522314in}}%
\pgfpathlineto{\pgfqpoint{1.218213in}{0.527960in}}%
\pgfpathlineto{\pgfqpoint{1.232834in}{0.535925in}}%
\pgfpathlineto{\pgfqpoint{1.233869in}{0.536465in}}%
\pgfpathlineto{\pgfqpoint{1.249526in}{0.545304in}}%
\pgfpathlineto{\pgfqpoint{1.256482in}{0.549536in}}%
\pgfpathlineto{\pgfqpoint{1.265182in}{0.554640in}}%
\pgfpathlineto{\pgfqpoint{1.278730in}{0.563148in}}%
\pgfpathlineto{\pgfqpoint{1.280839in}{0.564433in}}%
\pgfpathlineto{\pgfqpoint{1.296495in}{0.574560in}}%
\pgfpathlineto{\pgfqpoint{1.299714in}{0.576759in}}%
\pgfpathlineto{\pgfqpoint{1.312152in}{0.585056in}}%
\pgfpathlineto{\pgfqpoint{1.319726in}{0.590370in}}%
\pgfpathlineto{\pgfqpoint{1.327809in}{0.595939in}}%
\pgfpathlineto{\pgfqpoint{1.338960in}{0.603981in}}%
\pgfpathlineto{\pgfqpoint{1.343465in}{0.607188in}}%
\pgfpathlineto{\pgfqpoint{1.357495in}{0.617592in}}%
\pgfpathlineto{\pgfqpoint{1.359122in}{0.618789in}}%
\pgfpathlineto{\pgfqpoint{1.374778in}{0.630726in}}%
\pgfpathlineto{\pgfqpoint{1.375385in}{0.631203in}}%
\pgfpathlineto{\pgfqpoint{1.390435in}{0.643000in}}%
\pgfpathlineto{\pgfqpoint{1.392686in}{0.644814in}}%
\pgfpathlineto{\pgfqpoint{1.406091in}{0.655636in}}%
\pgfpathlineto{\pgfqpoint{1.409467in}{0.658425in}}%
\pgfpathlineto{\pgfqpoint{1.421748in}{0.668636in}}%
\pgfpathlineto{\pgfqpoint{1.425760in}{0.672036in}}%
\pgfpathlineto{\pgfqpoint{1.437404in}{0.682010in}}%
\pgfpathlineto{\pgfqpoint{1.441588in}{0.685648in}}%
\pgfpathlineto{\pgfqpoint{1.453061in}{0.695771in}}%
\pgfpathlineto{\pgfqpoint{1.456972in}{0.699259in}}%
\pgfpathlineto{\pgfqpoint{1.468718in}{0.709935in}}%
\pgfpathlineto{\pgfqpoint{1.471926in}{0.712870in}}%
\pgfpathlineto{\pgfqpoint{1.484374in}{0.724524in}}%
\pgfpathlineto{\pgfqpoint{1.486461in}{0.726481in}}%
\pgfpathlineto{\pgfqpoint{1.500031in}{0.739564in}}%
\pgfpathlineto{\pgfqpoint{1.500580in}{0.740092in}}%
\pgfpathlineto{\pgfqpoint{1.514311in}{0.753703in}}%
\pgfpathlineto{\pgfqpoint{1.515687in}{0.755117in}}%
\pgfpathlineto{\pgfqpoint{1.527654in}{0.767314in}}%
\pgfpathlineto{\pgfqpoint{1.531344in}{0.771231in}}%
\pgfpathlineto{\pgfqpoint{1.540594in}{0.780925in}}%
\pgfpathlineto{\pgfqpoint{1.547000in}{0.787952in}}%
\pgfpathlineto{\pgfqpoint{1.553113in}{0.794536in}}%
\pgfpathlineto{\pgfqpoint{1.562657in}{0.805349in}}%
\pgfpathlineto{\pgfqpoint{1.565186in}{0.808148in}}%
\pgfpathlineto{\pgfqpoint{1.576835in}{0.821759in}}%
\pgfpathlineto{\pgfqpoint{1.578314in}{0.823592in}}%
\pgfpathlineto{\pgfqpoint{1.588100in}{0.835370in}}%
\pgfpathlineto{\pgfqpoint{1.593970in}{0.842933in}}%
\pgfpathlineto{\pgfqpoint{1.598838in}{0.848981in}}%
\pgfpathlineto{\pgfqpoint{1.609006in}{0.862592in}}%
\pgfpathlineto{\pgfqpoint{1.609627in}{0.863492in}}%
\pgfpathlineto{\pgfqpoint{1.618789in}{0.876203in}}%
\pgfpathlineto{\pgfqpoint{1.625283in}{0.886090in}}%
\pgfpathlineto{\pgfqpoint{1.627859in}{0.889814in}}%
\pgfpathlineto{\pgfqpoint{1.636373in}{0.903425in}}%
\pgfpathlineto{\pgfqpoint{1.640940in}{0.911623in}}%
\pgfpathlineto{\pgfqpoint{1.644146in}{0.917036in}}%
\pgfpathlineto{\pgfqpoint{1.651219in}{0.930648in}}%
\pgfpathlineto{\pgfqpoint{1.656596in}{0.942719in}}%
\pgfpathlineto{\pgfqpoint{1.657333in}{0.944259in}}%
\pgfpathlineto{\pgfqpoint{1.662779in}{0.957870in}}%
\pgfpathlineto{\pgfqpoint{1.667133in}{0.971481in}}%
\pgfpathlineto{\pgfqpoint{1.670398in}{0.985092in}}%
\pgfpathlineto{\pgfqpoint{1.672253in}{0.996690in}}%
\pgfpathlineto{\pgfqpoint{1.672603in}{0.998703in}}%
\pgfpathlineto{\pgfqpoint{1.673790in}{1.012314in}}%
\pgfpathlineto{\pgfqpoint{1.673790in}{1.025925in}}%
\pgfpathlineto{\pgfqpoint{1.672603in}{1.039536in}}%
\pgfpathlineto{\pgfqpoint{1.672253in}{1.041549in}}%
\pgfpathlineto{\pgfqpoint{1.670398in}{1.053148in}}%
\pgfpathlineto{\pgfqpoint{1.667133in}{1.066759in}}%
\pgfpathlineto{\pgfqpoint{1.662779in}{1.080370in}}%
\pgfpathlineto{\pgfqpoint{1.657333in}{1.093981in}}%
\pgfpathlineto{\pgfqpoint{1.656596in}{1.095520in}}%
\pgfpathlineto{\pgfqpoint{1.651219in}{1.107592in}}%
\pgfpathlineto{\pgfqpoint{1.644146in}{1.121203in}}%
\pgfpathlineto{\pgfqpoint{1.640940in}{1.126617in}}%
\pgfpathlineto{\pgfqpoint{1.636373in}{1.134814in}}%
\pgfpathlineto{\pgfqpoint{1.627859in}{1.148425in}}%
\pgfpathlineto{\pgfqpoint{1.625283in}{1.152149in}}%
\pgfpathlineto{\pgfqpoint{1.618789in}{1.162036in}}%
\pgfpathlineto{\pgfqpoint{1.609627in}{1.174747in}}%
\pgfpathlineto{\pgfqpoint{1.609006in}{1.175647in}}%
\pgfpathlineto{\pgfqpoint{1.598838in}{1.189259in}}%
\pgfpathlineto{\pgfqpoint{1.593970in}{1.195306in}}%
\pgfpathlineto{\pgfqpoint{1.588100in}{1.202870in}}%
\pgfpathlineto{\pgfqpoint{1.578314in}{1.214647in}}%
\pgfpathlineto{\pgfqpoint{1.576835in}{1.216481in}}%
\pgfpathlineto{\pgfqpoint{1.565186in}{1.230092in}}%
\pgfpathlineto{\pgfqpoint{1.562657in}{1.232890in}}%
\pgfpathlineto{\pgfqpoint{1.553113in}{1.243703in}}%
\pgfpathlineto{\pgfqpoint{1.547000in}{1.250287in}}%
\pgfpathlineto{\pgfqpoint{1.540594in}{1.257314in}}%
\pgfpathlineto{\pgfqpoint{1.531344in}{1.267009in}}%
\pgfpathlineto{\pgfqpoint{1.527654in}{1.270925in}}%
\pgfpathlineto{\pgfqpoint{1.515687in}{1.283122in}}%
\pgfpathlineto{\pgfqpoint{1.514311in}{1.284536in}}%
\pgfpathlineto{\pgfqpoint{1.500580in}{1.298148in}}%
\pgfpathlineto{\pgfqpoint{1.500031in}{1.298675in}}%
\pgfpathlineto{\pgfqpoint{1.486461in}{1.311759in}}%
\pgfpathlineto{\pgfqpoint{1.484374in}{1.313716in}}%
\pgfpathlineto{\pgfqpoint{1.471926in}{1.325370in}}%
\pgfpathlineto{\pgfqpoint{1.468718in}{1.328305in}}%
\pgfpathlineto{\pgfqpoint{1.456972in}{1.338981in}}%
\pgfpathlineto{\pgfqpoint{1.453061in}{1.342469in}}%
\pgfpathlineto{\pgfqpoint{1.441588in}{1.352592in}}%
\pgfpathlineto{\pgfqpoint{1.437404in}{1.356229in}}%
\pgfpathlineto{\pgfqpoint{1.425760in}{1.366203in}}%
\pgfpathlineto{\pgfqpoint{1.421748in}{1.369603in}}%
\pgfpathlineto{\pgfqpoint{1.409467in}{1.379814in}}%
\pgfpathlineto{\pgfqpoint{1.406091in}{1.382604in}}%
\pgfpathlineto{\pgfqpoint{1.392686in}{1.393425in}}%
\pgfpathlineto{\pgfqpoint{1.390435in}{1.395239in}}%
\pgfpathlineto{\pgfqpoint{1.375385in}{1.407036in}}%
\pgfpathlineto{\pgfqpoint{1.374778in}{1.407514in}}%
\pgfpathlineto{\pgfqpoint{1.359122in}{1.419451in}}%
\pgfpathlineto{\pgfqpoint{1.357495in}{1.420648in}}%
\pgfpathlineto{\pgfqpoint{1.343465in}{1.431051in}}%
\pgfpathlineto{\pgfqpoint{1.338960in}{1.434259in}}%
\pgfpathlineto{\pgfqpoint{1.327809in}{1.442300in}}%
\pgfpathlineto{\pgfqpoint{1.319726in}{1.447870in}}%
\pgfpathlineto{\pgfqpoint{1.312152in}{1.453184in}}%
\pgfpathlineto{\pgfqpoint{1.299714in}{1.461481in}}%
\pgfpathlineto{\pgfqpoint{1.296495in}{1.463679in}}%
\pgfpathlineto{\pgfqpoint{1.280839in}{1.473807in}}%
\pgfpathlineto{\pgfqpoint{1.278730in}{1.475092in}}%
\pgfpathlineto{\pgfqpoint{1.265182in}{1.483600in}}%
\pgfpathlineto{\pgfqpoint{1.256482in}{1.488703in}}%
\pgfpathlineto{\pgfqpoint{1.249526in}{1.492935in}}%
\pgfpathlineto{\pgfqpoint{1.233869in}{1.501774in}}%
\pgfpathlineto{\pgfqpoint{1.232834in}{1.502314in}}%
\pgfpathlineto{\pgfqpoint{1.218213in}{1.510280in}}%
\pgfpathlineto{\pgfqpoint{1.206839in}{1.515925in}}%
\pgfpathlineto{\pgfqpoint{1.202556in}{1.518164in}}%
\pgfpathlineto{\pgfqpoint{1.186899in}{1.525566in}}%
\pgfpathlineto{\pgfqpoint{1.177470in}{1.529536in}}%
\pgfpathlineto{\pgfqpoint{1.171243in}{1.532323in}}%
\pgfpathlineto{\pgfqpoint{1.155586in}{1.538473in}}%
\pgfpathlineto{\pgfqpoint{1.141701in}{1.543148in}}%
\pgfpathlineto{\pgfqpoint{1.139930in}{1.543788in}}%
\pgfpathlineto{\pgfqpoint{1.124273in}{1.548522in}}%
\pgfpathlineto{\pgfqpoint{1.108617in}{1.552308in}}%
\pgfpathlineto{\pgfqpoint{1.092960in}{1.555146in}}%
\pgfpathlineto{\pgfqpoint{1.079619in}{1.556759in}}%
\pgfpathlineto{\pgfqpoint{1.077303in}{1.557063in}}%
\pgfpathlineto{\pgfqpoint{1.061647in}{1.558094in}}%
\pgfpathlineto{\pgfqpoint{1.045990in}{1.558094in}}%
\pgfpathlineto{\pgfqpoint{1.030334in}{1.557063in}}%
\pgfpathlineto{\pgfqpoint{1.028018in}{1.556759in}}%
\pgfpathlineto{\pgfqpoint{1.014677in}{1.555146in}}%
\pgfpathlineto{\pgfqpoint{0.999021in}{1.552308in}}%
\pgfpathlineto{\pgfqpoint{0.983364in}{1.548522in}}%
\pgfpathlineto{\pgfqpoint{0.967708in}{1.543788in}}%
\pgfpathlineto{\pgfqpoint{0.965937in}{1.543148in}}%
\pgfpathlineto{\pgfqpoint{0.952051in}{1.538473in}}%
\pgfpathlineto{\pgfqpoint{0.936394in}{1.532323in}}%
\pgfpathlineto{\pgfqpoint{0.930167in}{1.529536in}}%
\pgfpathlineto{\pgfqpoint{0.920738in}{1.525566in}}%
\pgfpathlineto{\pgfqpoint{0.905081in}{1.518164in}}%
\pgfpathlineto{\pgfqpoint{0.900798in}{1.515925in}}%
\pgfpathlineto{\pgfqpoint{0.889425in}{1.510280in}}%
\pgfpathlineto{\pgfqpoint{0.874803in}{1.502314in}}%
\pgfpathlineto{\pgfqpoint{0.873768in}{1.501774in}}%
\pgfpathlineto{\pgfqpoint{0.858112in}{1.492935in}}%
\pgfpathlineto{\pgfqpoint{0.851155in}{1.488703in}}%
\pgfpathlineto{\pgfqpoint{0.842455in}{1.483600in}}%
\pgfpathlineto{\pgfqpoint{0.828907in}{1.475092in}}%
\pgfpathlineto{\pgfqpoint{0.826798in}{1.473807in}}%
\pgfpathlineto{\pgfqpoint{0.811142in}{1.463679in}}%
\pgfpathlineto{\pgfqpoint{0.807923in}{1.461481in}}%
\pgfpathlineto{\pgfqpoint{0.795485in}{1.453184in}}%
\pgfpathlineto{\pgfqpoint{0.787911in}{1.447870in}}%
\pgfpathlineto{\pgfqpoint{0.779829in}{1.442300in}}%
\pgfpathlineto{\pgfqpoint{0.768677in}{1.434259in}}%
\pgfpathlineto{\pgfqpoint{0.764172in}{1.431051in}}%
\pgfpathlineto{\pgfqpoint{0.750142in}{1.420648in}}%
\pgfpathlineto{\pgfqpoint{0.748516in}{1.419451in}}%
\pgfpathlineto{\pgfqpoint{0.732859in}{1.407514in}}%
\pgfpathlineto{\pgfqpoint{0.732252in}{1.407036in}}%
\pgfpathlineto{\pgfqpoint{0.717202in}{1.395239in}}%
\pgfpathlineto{\pgfqpoint{0.714951in}{1.393425in}}%
\pgfpathlineto{\pgfqpoint{0.701546in}{1.382604in}}%
\pgfpathlineto{\pgfqpoint{0.698170in}{1.379814in}}%
\pgfpathlineto{\pgfqpoint{0.685889in}{1.369603in}}%
\pgfpathlineto{\pgfqpoint{0.681877in}{1.366203in}}%
\pgfpathlineto{\pgfqpoint{0.670233in}{1.356229in}}%
\pgfpathlineto{\pgfqpoint{0.666049in}{1.352592in}}%
\pgfpathlineto{\pgfqpoint{0.654576in}{1.342469in}}%
\pgfpathlineto{\pgfqpoint{0.650665in}{1.338981in}}%
\pgfpathlineto{\pgfqpoint{0.638920in}{1.328305in}}%
\pgfpathlineto{\pgfqpoint{0.635711in}{1.325370in}}%
\pgfpathlineto{\pgfqpoint{0.623263in}{1.313716in}}%
\pgfpathlineto{\pgfqpoint{0.621177in}{1.311759in}}%
\pgfpathlineto{\pgfqpoint{0.607606in}{1.298675in}}%
\pgfpathlineto{\pgfqpoint{0.607057in}{1.298148in}}%
\pgfpathlineto{\pgfqpoint{0.593326in}{1.284536in}}%
\pgfpathlineto{\pgfqpoint{0.591950in}{1.283122in}}%
\pgfpathlineto{\pgfqpoint{0.579983in}{1.270925in}}%
\pgfpathlineto{\pgfqpoint{0.576293in}{1.267009in}}%
\pgfpathlineto{\pgfqpoint{0.567044in}{1.257314in}}%
\pgfpathlineto{\pgfqpoint{0.560637in}{1.250287in}}%
\pgfpathlineto{\pgfqpoint{0.554524in}{1.243703in}}%
\pgfpathlineto{\pgfqpoint{0.544980in}{1.232890in}}%
\pgfpathlineto{\pgfqpoint{0.542451in}{1.230092in}}%
\pgfpathlineto{\pgfqpoint{0.530802in}{1.216481in}}%
\pgfpathlineto{\pgfqpoint{0.529324in}{1.214647in}}%
\pgfpathlineto{\pgfqpoint{0.519537in}{1.202870in}}%
\pgfpathlineto{\pgfqpoint{0.513667in}{1.195306in}}%
\pgfpathlineto{\pgfqpoint{0.508799in}{1.189259in}}%
\pgfpathlineto{\pgfqpoint{0.498631in}{1.175647in}}%
\pgfpathlineto{\pgfqpoint{0.498011in}{1.174747in}}%
\pgfpathlineto{\pgfqpoint{0.488848in}{1.162036in}}%
\pgfpathlineto{\pgfqpoint{0.482354in}{1.152149in}}%
\pgfpathlineto{\pgfqpoint{0.479778in}{1.148425in}}%
\pgfpathlineto{\pgfqpoint{0.471264in}{1.134814in}}%
\pgfpathlineto{\pgfqpoint{0.466697in}{1.126617in}}%
\pgfpathlineto{\pgfqpoint{0.463492in}{1.121203in}}%
\pgfpathlineto{\pgfqpoint{0.456418in}{1.107592in}}%
\pgfpathlineto{\pgfqpoint{0.451041in}{1.095520in}}%
\pgfpathlineto{\pgfqpoint{0.450304in}{1.093981in}}%
\pgfpathlineto{\pgfqpoint{0.444858in}{1.080370in}}%
\pgfpathlineto{\pgfqpoint{0.440504in}{1.066759in}}%
\pgfpathlineto{\pgfqpoint{0.437239in}{1.053148in}}%
\pgfpathlineto{\pgfqpoint{0.435384in}{1.041549in}}%
\pgfpathlineto{\pgfqpoint{0.435034in}{1.039536in}}%
\pgfpathlineto{\pgfqpoint{0.433848in}{1.025925in}}%
\pgfpathlineto{\pgfqpoint{0.433848in}{1.012314in}}%
\pgfpathlineto{\pgfqpoint{0.435034in}{0.998703in}}%
\pgfpathlineto{\pgfqpoint{0.435384in}{0.996690in}}%
\pgfpathlineto{\pgfqpoint{0.437239in}{0.985092in}}%
\pgfpathlineto{\pgfqpoint{0.440504in}{0.971481in}}%
\pgfpathlineto{\pgfqpoint{0.444858in}{0.957870in}}%
\pgfpathlineto{\pgfqpoint{0.450304in}{0.944259in}}%
\pgfpathlineto{\pgfqpoint{0.451041in}{0.942719in}}%
\pgfpathlineto{\pgfqpoint{0.456418in}{0.930648in}}%
\pgfpathlineto{\pgfqpoint{0.463492in}{0.917036in}}%
\pgfpathlineto{\pgfqpoint{0.466697in}{0.911623in}}%
\pgfpathlineto{\pgfqpoint{0.471264in}{0.903425in}}%
\pgfpathlineto{\pgfqpoint{0.479778in}{0.889814in}}%
\pgfpathlineto{\pgfqpoint{0.482354in}{0.886090in}}%
\pgfpathlineto{\pgfqpoint{0.488848in}{0.876203in}}%
\pgfpathlineto{\pgfqpoint{0.498011in}{0.863492in}}%
\pgfpathlineto{\pgfqpoint{0.498631in}{0.862592in}}%
\pgfpathlineto{\pgfqpoint{0.508799in}{0.848981in}}%
\pgfpathlineto{\pgfqpoint{0.513667in}{0.842933in}}%
\pgfpathlineto{\pgfqpoint{0.519537in}{0.835370in}}%
\pgfpathlineto{\pgfqpoint{0.529324in}{0.823592in}}%
\pgfpathlineto{\pgfqpoint{0.530802in}{0.821759in}}%
\pgfpathlineto{\pgfqpoint{0.542451in}{0.808148in}}%
\pgfpathlineto{\pgfqpoint{0.544980in}{0.805349in}}%
\pgfpathlineto{\pgfqpoint{0.554524in}{0.794536in}}%
\pgfpathlineto{\pgfqpoint{0.560637in}{0.787952in}}%
\pgfpathlineto{\pgfqpoint{0.567044in}{0.780925in}}%
\pgfpathlineto{\pgfqpoint{0.576293in}{0.771231in}}%
\pgfpathlineto{\pgfqpoint{0.579983in}{0.767314in}}%
\pgfpathlineto{\pgfqpoint{0.591950in}{0.755117in}}%
\pgfpathlineto{\pgfqpoint{0.593326in}{0.753703in}}%
\pgfpathlineto{\pgfqpoint{0.607057in}{0.740092in}}%
\pgfpathlineto{\pgfqpoint{0.607606in}{0.739564in}}%
\pgfpathlineto{\pgfqpoint{0.621177in}{0.726481in}}%
\pgfpathlineto{\pgfqpoint{0.623263in}{0.724524in}}%
\pgfpathlineto{\pgfqpoint{0.635711in}{0.712870in}}%
\pgfpathlineto{\pgfqpoint{0.638920in}{0.709935in}}%
\pgfpathlineto{\pgfqpoint{0.650665in}{0.699259in}}%
\pgfpathlineto{\pgfqpoint{0.654576in}{0.695771in}}%
\pgfpathlineto{\pgfqpoint{0.666049in}{0.685648in}}%
\pgfpathlineto{\pgfqpoint{0.670233in}{0.682010in}}%
\pgfpathlineto{\pgfqpoint{0.681877in}{0.672036in}}%
\pgfpathlineto{\pgfqpoint{0.685889in}{0.668636in}}%
\pgfpathlineto{\pgfqpoint{0.698170in}{0.658425in}}%
\pgfpathlineto{\pgfqpoint{0.701546in}{0.655636in}}%
\pgfpathlineto{\pgfqpoint{0.714951in}{0.644814in}}%
\pgfpathlineto{\pgfqpoint{0.717202in}{0.643000in}}%
\pgfpathlineto{\pgfqpoint{0.732252in}{0.631203in}}%
\pgfpathlineto{\pgfqpoint{0.732859in}{0.630726in}}%
\pgfpathlineto{\pgfqpoint{0.748516in}{0.618789in}}%
\pgfpathlineto{\pgfqpoint{0.750142in}{0.617592in}}%
\pgfpathlineto{\pgfqpoint{0.764172in}{0.607188in}}%
\pgfpathlineto{\pgfqpoint{0.768677in}{0.603981in}}%
\pgfpathlineto{\pgfqpoint{0.779829in}{0.595939in}}%
\pgfpathlineto{\pgfqpoint{0.787911in}{0.590370in}}%
\pgfpathlineto{\pgfqpoint{0.795485in}{0.585056in}}%
\pgfpathlineto{\pgfqpoint{0.807923in}{0.576759in}}%
\pgfpathlineto{\pgfqpoint{0.811142in}{0.574560in}}%
\pgfpathlineto{\pgfqpoint{0.826798in}{0.564433in}}%
\pgfpathlineto{\pgfqpoint{0.828907in}{0.563148in}}%
\pgfpathlineto{\pgfqpoint{0.842455in}{0.554640in}}%
\pgfpathlineto{\pgfqpoint{0.851155in}{0.549536in}}%
\pgfpathlineto{\pgfqpoint{0.858112in}{0.545304in}}%
\pgfpathlineto{\pgfqpoint{0.873768in}{0.536465in}}%
\pgfpathlineto{\pgfqpoint{0.874803in}{0.535925in}}%
\pgfpathlineto{\pgfqpoint{0.889425in}{0.527960in}}%
\pgfpathlineto{\pgfqpoint{0.900798in}{0.522314in}}%
\pgfpathlineto{\pgfqpoint{0.905081in}{0.520075in}}%
\pgfpathlineto{\pgfqpoint{0.920738in}{0.512673in}}%
\pgfpathlineto{\pgfqpoint{0.930167in}{0.508703in}}%
\pgfpathlineto{\pgfqpoint{0.936394in}{0.505916in}}%
\pgfpathlineto{\pgfqpoint{0.952051in}{0.499767in}}%
\pgfpathlineto{\pgfqpoint{0.965937in}{0.495092in}}%
\pgfpathlineto{\pgfqpoint{0.967708in}{0.494452in}}%
\pgfpathlineto{\pgfqpoint{0.983364in}{0.489717in}}%
\pgfpathlineto{\pgfqpoint{0.999021in}{0.485931in}}%
\pgfpathlineto{\pgfqpoint{1.014677in}{0.483093in}}%
\pgfpathlineto{\pgfqpoint{1.028018in}{0.481481in}}%
\pgfpathlineto{\pgfqpoint{1.030334in}{0.481176in}}%
\pgfpathclose%
\pgfpathmoveto{\pgfqpoint{0.967702in}{0.631203in}}%
\pgfpathlineto{\pgfqpoint{0.952051in}{0.635291in}}%
\pgfpathlineto{\pgfqpoint{0.936394in}{0.640064in}}%
\pgfpathlineto{\pgfqpoint{0.922757in}{0.644814in}}%
\pgfpathlineto{\pgfqpoint{0.920738in}{0.645513in}}%
\pgfpathlineto{\pgfqpoint{0.905081in}{0.651585in}}%
\pgfpathlineto{\pgfqpoint{0.889425in}{0.658335in}}%
\pgfpathlineto{\pgfqpoint{0.889232in}{0.658425in}}%
\pgfpathlineto{\pgfqpoint{0.873768in}{0.665706in}}%
\pgfpathlineto{\pgfqpoint{0.861419in}{0.672036in}}%
\pgfpathlineto{\pgfqpoint{0.858112in}{0.673741in}}%
\pgfpathlineto{\pgfqpoint{0.842455in}{0.682419in}}%
\pgfpathlineto{\pgfqpoint{0.837010in}{0.685648in}}%
\pgfpathlineto{\pgfqpoint{0.826798in}{0.691764in}}%
\pgfpathlineto{\pgfqpoint{0.815040in}{0.699259in}}%
\pgfpathlineto{\pgfqpoint{0.811142in}{0.701780in}}%
\pgfpathlineto{\pgfqpoint{0.795485in}{0.712485in}}%
\pgfpathlineto{\pgfqpoint{0.794949in}{0.712870in}}%
\pgfpathlineto{\pgfqpoint{0.779829in}{0.723931in}}%
\pgfpathlineto{\pgfqpoint{0.776496in}{0.726481in}}%
\pgfpathlineto{\pgfqpoint{0.764172in}{0.736144in}}%
\pgfpathlineto{\pgfqpoint{0.759331in}{0.740092in}}%
\pgfpathlineto{\pgfqpoint{0.748516in}{0.749177in}}%
\pgfpathlineto{\pgfqpoint{0.743310in}{0.753703in}}%
\pgfpathlineto{\pgfqpoint{0.732859in}{0.763106in}}%
\pgfpathlineto{\pgfqpoint{0.728318in}{0.767314in}}%
\pgfpathlineto{\pgfqpoint{0.717202in}{0.778028in}}%
\pgfpathlineto{\pgfqpoint{0.714270in}{0.780925in}}%
\pgfpathlineto{\pgfqpoint{0.701546in}{0.794070in}}%
\pgfpathlineto{\pgfqpoint{0.701104in}{0.794536in}}%
\pgfpathlineto{\pgfqpoint{0.688790in}{0.808148in}}%
\pgfpathlineto{\pgfqpoint{0.685889in}{0.811536in}}%
\pgfpathlineto{\pgfqpoint{0.677269in}{0.821759in}}%
\pgfpathlineto{\pgfqpoint{0.670233in}{0.830636in}}%
\pgfpathlineto{\pgfqpoint{0.666520in}{0.835370in}}%
\pgfpathlineto{\pgfqpoint{0.656536in}{0.848981in}}%
\pgfpathlineto{\pgfqpoint{0.654576in}{0.851856in}}%
\pgfpathlineto{\pgfqpoint{0.647294in}{0.862592in}}%
\pgfpathlineto{\pgfqpoint{0.638920in}{0.876036in}}%
\pgfpathlineto{\pgfqpoint{0.638815in}{0.876203in}}%
\pgfpathlineto{\pgfqpoint{0.631052in}{0.889814in}}%
\pgfpathlineto{\pgfqpoint{0.624067in}{0.903425in}}%
\pgfpathlineto{\pgfqpoint{0.623263in}{0.905180in}}%
\pgfpathlineto{\pgfqpoint{0.617799in}{0.917036in}}%
\pgfpathlineto{\pgfqpoint{0.612308in}{0.930648in}}%
\pgfpathlineto{\pgfqpoint{0.607606in}{0.944254in}}%
\pgfpathlineto{\pgfqpoint{0.607605in}{0.944259in}}%
\pgfpathlineto{\pgfqpoint{0.603625in}{0.957870in}}%
\pgfpathlineto{\pgfqpoint{0.600442in}{0.971481in}}%
\pgfpathlineto{\pgfqpoint{0.598056in}{0.985092in}}%
\pgfpathlineto{\pgfqpoint{0.596466in}{0.998703in}}%
\pgfpathlineto{\pgfqpoint{0.595671in}{1.012314in}}%
\pgfpathlineto{\pgfqpoint{0.595671in}{1.025925in}}%
\pgfpathlineto{\pgfqpoint{0.596466in}{1.039536in}}%
\pgfpathlineto{\pgfqpoint{0.598056in}{1.053148in}}%
\pgfpathlineto{\pgfqpoint{0.600442in}{1.066759in}}%
\pgfpathlineto{\pgfqpoint{0.603625in}{1.080370in}}%
\pgfpathlineto{\pgfqpoint{0.607605in}{1.093981in}}%
\pgfpathlineto{\pgfqpoint{0.607606in}{1.093986in}}%
\pgfpathlineto{\pgfqpoint{0.612308in}{1.107592in}}%
\pgfpathlineto{\pgfqpoint{0.617799in}{1.121203in}}%
\pgfpathlineto{\pgfqpoint{0.623263in}{1.133059in}}%
\pgfpathlineto{\pgfqpoint{0.624067in}{1.134814in}}%
\pgfpathlineto{\pgfqpoint{0.631052in}{1.148425in}}%
\pgfpathlineto{\pgfqpoint{0.638815in}{1.162036in}}%
\pgfpathlineto{\pgfqpoint{0.638920in}{1.162204in}}%
\pgfpathlineto{\pgfqpoint{0.647294in}{1.175647in}}%
\pgfpathlineto{\pgfqpoint{0.654576in}{1.186383in}}%
\pgfpathlineto{\pgfqpoint{0.656536in}{1.189259in}}%
\pgfpathlineto{\pgfqpoint{0.666520in}{1.202870in}}%
\pgfpathlineto{\pgfqpoint{0.670233in}{1.207603in}}%
\pgfpathlineto{\pgfqpoint{0.677269in}{1.216481in}}%
\pgfpathlineto{\pgfqpoint{0.685889in}{1.226703in}}%
\pgfpathlineto{\pgfqpoint{0.688790in}{1.230092in}}%
\pgfpathlineto{\pgfqpoint{0.701104in}{1.243703in}}%
\pgfpathlineto{\pgfqpoint{0.701546in}{1.244169in}}%
\pgfpathlineto{\pgfqpoint{0.714270in}{1.257314in}}%
\pgfpathlineto{\pgfqpoint{0.717202in}{1.260212in}}%
\pgfpathlineto{\pgfqpoint{0.728318in}{1.270925in}}%
\pgfpathlineto{\pgfqpoint{0.732859in}{1.275134in}}%
\pgfpathlineto{\pgfqpoint{0.743310in}{1.284536in}}%
\pgfpathlineto{\pgfqpoint{0.748516in}{1.289062in}}%
\pgfpathlineto{\pgfqpoint{0.759331in}{1.298148in}}%
\pgfpathlineto{\pgfqpoint{0.764172in}{1.302096in}}%
\pgfpathlineto{\pgfqpoint{0.776496in}{1.311759in}}%
\pgfpathlineto{\pgfqpoint{0.779829in}{1.314308in}}%
\pgfpathlineto{\pgfqpoint{0.794949in}{1.325370in}}%
\pgfpathlineto{\pgfqpoint{0.795485in}{1.325754in}}%
\pgfpathlineto{\pgfqpoint{0.811142in}{1.336459in}}%
\pgfpathlineto{\pgfqpoint{0.815040in}{1.338981in}}%
\pgfpathlineto{\pgfqpoint{0.826798in}{1.346475in}}%
\pgfpathlineto{\pgfqpoint{0.837010in}{1.352592in}}%
\pgfpathlineto{\pgfqpoint{0.842455in}{1.355820in}}%
\pgfpathlineto{\pgfqpoint{0.858112in}{1.364499in}}%
\pgfpathlineto{\pgfqpoint{0.861419in}{1.366203in}}%
\pgfpathlineto{\pgfqpoint{0.873768in}{1.372534in}}%
\pgfpathlineto{\pgfqpoint{0.889232in}{1.379814in}}%
\pgfpathlineto{\pgfqpoint{0.889425in}{1.379905in}}%
\pgfpathlineto{\pgfqpoint{0.905081in}{1.386654in}}%
\pgfpathlineto{\pgfqpoint{0.920738in}{1.392726in}}%
\pgfpathlineto{\pgfqpoint{0.922757in}{1.393425in}}%
\pgfpathlineto{\pgfqpoint{0.936394in}{1.398176in}}%
\pgfpathlineto{\pgfqpoint{0.952051in}{1.402949in}}%
\pgfpathlineto{\pgfqpoint{0.967702in}{1.407036in}}%
\pgfpathlineto{\pgfqpoint{0.967708in}{1.407038in}}%
\pgfpathlineto{\pgfqpoint{0.983364in}{1.410498in}}%
\pgfpathlineto{\pgfqpoint{0.999021in}{1.413265in}}%
\pgfpathlineto{\pgfqpoint{1.014677in}{1.415339in}}%
\pgfpathlineto{\pgfqpoint{1.030334in}{1.416722in}}%
\pgfpathlineto{\pgfqpoint{1.045990in}{1.417413in}}%
\pgfpathlineto{\pgfqpoint{1.061647in}{1.417413in}}%
\pgfpathlineto{\pgfqpoint{1.077303in}{1.416722in}}%
\pgfpathlineto{\pgfqpoint{1.092960in}{1.415339in}}%
\pgfpathlineto{\pgfqpoint{1.108617in}{1.413265in}}%
\pgfpathlineto{\pgfqpoint{1.124273in}{1.410498in}}%
\pgfpathlineto{\pgfqpoint{1.139930in}{1.407038in}}%
\pgfpathlineto{\pgfqpoint{1.139935in}{1.407036in}}%
\pgfpathlineto{\pgfqpoint{1.155586in}{1.402949in}}%
\pgfpathlineto{\pgfqpoint{1.171243in}{1.398176in}}%
\pgfpathlineto{\pgfqpoint{1.184880in}{1.393425in}}%
\pgfpathlineto{\pgfqpoint{1.186899in}{1.392726in}}%
\pgfpathlineto{\pgfqpoint{1.202556in}{1.386654in}}%
\pgfpathlineto{\pgfqpoint{1.218213in}{1.379905in}}%
\pgfpathlineto{\pgfqpoint{1.218405in}{1.379814in}}%
\pgfpathlineto{\pgfqpoint{1.233869in}{1.372534in}}%
\pgfpathlineto{\pgfqpoint{1.246218in}{1.366203in}}%
\pgfpathlineto{\pgfqpoint{1.249526in}{1.364499in}}%
\pgfpathlineto{\pgfqpoint{1.265182in}{1.355820in}}%
\pgfpathlineto{\pgfqpoint{1.270627in}{1.352592in}}%
\pgfpathlineto{\pgfqpoint{1.280839in}{1.346475in}}%
\pgfpathlineto{\pgfqpoint{1.292598in}{1.338981in}}%
\pgfpathlineto{\pgfqpoint{1.296495in}{1.336459in}}%
\pgfpathlineto{\pgfqpoint{1.312152in}{1.325754in}}%
\pgfpathlineto{\pgfqpoint{1.312688in}{1.325370in}}%
\pgfpathlineto{\pgfqpoint{1.327809in}{1.314308in}}%
\pgfpathlineto{\pgfqpoint{1.331141in}{1.311759in}}%
\pgfpathlineto{\pgfqpoint{1.343465in}{1.302096in}}%
\pgfpathlineto{\pgfqpoint{1.348306in}{1.298148in}}%
\pgfpathlineto{\pgfqpoint{1.359122in}{1.289062in}}%
\pgfpathlineto{\pgfqpoint{1.364327in}{1.284536in}}%
\pgfpathlineto{\pgfqpoint{1.374778in}{1.275134in}}%
\pgfpathlineto{\pgfqpoint{1.379320in}{1.270925in}}%
\pgfpathlineto{\pgfqpoint{1.390435in}{1.260212in}}%
\pgfpathlineto{\pgfqpoint{1.393367in}{1.257314in}}%
\pgfpathlineto{\pgfqpoint{1.406091in}{1.244169in}}%
\pgfpathlineto{\pgfqpoint{1.406534in}{1.243703in}}%
\pgfpathlineto{\pgfqpoint{1.418847in}{1.230092in}}%
\pgfpathlineto{\pgfqpoint{1.421748in}{1.226703in}}%
\pgfpathlineto{\pgfqpoint{1.430369in}{1.216481in}}%
\pgfpathlineto{\pgfqpoint{1.437404in}{1.207603in}}%
\pgfpathlineto{\pgfqpoint{1.441118in}{1.202870in}}%
\pgfpathlineto{\pgfqpoint{1.451101in}{1.189259in}}%
\pgfpathlineto{\pgfqpoint{1.453061in}{1.186383in}}%
\pgfpathlineto{\pgfqpoint{1.460343in}{1.175647in}}%
\pgfpathlineto{\pgfqpoint{1.468718in}{1.162204in}}%
\pgfpathlineto{\pgfqpoint{1.468822in}{1.162036in}}%
\pgfpathlineto{\pgfqpoint{1.476585in}{1.148425in}}%
\pgfpathlineto{\pgfqpoint{1.483570in}{1.134814in}}%
\pgfpathlineto{\pgfqpoint{1.484374in}{1.133059in}}%
\pgfpathlineto{\pgfqpoint{1.489838in}{1.121203in}}%
\pgfpathlineto{\pgfqpoint{1.495329in}{1.107592in}}%
\pgfpathlineto{\pgfqpoint{1.500031in}{1.093986in}}%
\pgfpathlineto{\pgfqpoint{1.500032in}{1.093981in}}%
\pgfpathlineto{\pgfqpoint{1.504013in}{1.080370in}}%
\pgfpathlineto{\pgfqpoint{1.507195in}{1.066759in}}%
\pgfpathlineto{\pgfqpoint{1.509581in}{1.053148in}}%
\pgfpathlineto{\pgfqpoint{1.511172in}{1.039536in}}%
\pgfpathlineto{\pgfqpoint{1.511967in}{1.025925in}}%
\pgfpathlineto{\pgfqpoint{1.511967in}{1.012314in}}%
\pgfpathlineto{\pgfqpoint{1.511172in}{0.998703in}}%
\pgfpathlineto{\pgfqpoint{1.509581in}{0.985092in}}%
\pgfpathlineto{\pgfqpoint{1.507195in}{0.971481in}}%
\pgfpathlineto{\pgfqpoint{1.504013in}{0.957870in}}%
\pgfpathlineto{\pgfqpoint{1.500032in}{0.944259in}}%
\pgfpathlineto{\pgfqpoint{1.500031in}{0.944254in}}%
\pgfpathlineto{\pgfqpoint{1.495329in}{0.930648in}}%
\pgfpathlineto{\pgfqpoint{1.489838in}{0.917036in}}%
\pgfpathlineto{\pgfqpoint{1.484374in}{0.905180in}}%
\pgfpathlineto{\pgfqpoint{1.483570in}{0.903425in}}%
\pgfpathlineto{\pgfqpoint{1.476585in}{0.889814in}}%
\pgfpathlineto{\pgfqpoint{1.468822in}{0.876203in}}%
\pgfpathlineto{\pgfqpoint{1.468718in}{0.876036in}}%
\pgfpathlineto{\pgfqpoint{1.460343in}{0.862592in}}%
\pgfpathlineto{\pgfqpoint{1.453061in}{0.851856in}}%
\pgfpathlineto{\pgfqpoint{1.451101in}{0.848981in}}%
\pgfpathlineto{\pgfqpoint{1.441118in}{0.835370in}}%
\pgfpathlineto{\pgfqpoint{1.437404in}{0.830636in}}%
\pgfpathlineto{\pgfqpoint{1.430369in}{0.821759in}}%
\pgfpathlineto{\pgfqpoint{1.421748in}{0.811536in}}%
\pgfpathlineto{\pgfqpoint{1.418847in}{0.808148in}}%
\pgfpathlineto{\pgfqpoint{1.406534in}{0.794536in}}%
\pgfpathlineto{\pgfqpoint{1.406091in}{0.794070in}}%
\pgfpathlineto{\pgfqpoint{1.393367in}{0.780925in}}%
\pgfpathlineto{\pgfqpoint{1.390435in}{0.778028in}}%
\pgfpathlineto{\pgfqpoint{1.379320in}{0.767314in}}%
\pgfpathlineto{\pgfqpoint{1.374778in}{0.763106in}}%
\pgfpathlineto{\pgfqpoint{1.364327in}{0.753703in}}%
\pgfpathlineto{\pgfqpoint{1.359122in}{0.749177in}}%
\pgfpathlineto{\pgfqpoint{1.348306in}{0.740092in}}%
\pgfpathlineto{\pgfqpoint{1.343465in}{0.736144in}}%
\pgfpathlineto{\pgfqpoint{1.331141in}{0.726481in}}%
\pgfpathlineto{\pgfqpoint{1.327809in}{0.723931in}}%
\pgfpathlineto{\pgfqpoint{1.312688in}{0.712870in}}%
\pgfpathlineto{\pgfqpoint{1.312152in}{0.712485in}}%
\pgfpathlineto{\pgfqpoint{1.296495in}{0.701780in}}%
\pgfpathlineto{\pgfqpoint{1.292598in}{0.699259in}}%
\pgfpathlineto{\pgfqpoint{1.280839in}{0.691764in}}%
\pgfpathlineto{\pgfqpoint{1.270627in}{0.685648in}}%
\pgfpathlineto{\pgfqpoint{1.265182in}{0.682419in}}%
\pgfpathlineto{\pgfqpoint{1.249526in}{0.673741in}}%
\pgfpathlineto{\pgfqpoint{1.246218in}{0.672036in}}%
\pgfpathlineto{\pgfqpoint{1.233869in}{0.665706in}}%
\pgfpathlineto{\pgfqpoint{1.218405in}{0.658425in}}%
\pgfpathlineto{\pgfqpoint{1.218213in}{0.658335in}}%
\pgfpathlineto{\pgfqpoint{1.202556in}{0.651585in}}%
\pgfpathlineto{\pgfqpoint{1.186899in}{0.645513in}}%
\pgfpathlineto{\pgfqpoint{1.184880in}{0.644814in}}%
\pgfpathlineto{\pgfqpoint{1.171243in}{0.640064in}}%
\pgfpathlineto{\pgfqpoint{1.155586in}{0.635291in}}%
\pgfpathlineto{\pgfqpoint{1.139935in}{0.631203in}}%
\pgfpathlineto{\pgfqpoint{1.139930in}{0.631202in}}%
\pgfpathlineto{\pgfqpoint{1.124273in}{0.627741in}}%
\pgfpathlineto{\pgfqpoint{1.108617in}{0.624975in}}%
\pgfpathlineto{\pgfqpoint{1.092960in}{0.622900in}}%
\pgfpathlineto{\pgfqpoint{1.077303in}{0.621518in}}%
\pgfpathlineto{\pgfqpoint{1.061647in}{0.620826in}}%
\pgfpathlineto{\pgfqpoint{1.045990in}{0.620826in}}%
\pgfpathlineto{\pgfqpoint{1.030334in}{0.621518in}}%
\pgfpathlineto{\pgfqpoint{1.014677in}{0.622900in}}%
\pgfpathlineto{\pgfqpoint{0.999021in}{0.624975in}}%
\pgfpathlineto{\pgfqpoint{0.983364in}{0.627741in}}%
\pgfpathlineto{\pgfqpoint{0.967708in}{0.631202in}}%
\pgfpathlineto{\pgfqpoint{0.967702in}{0.631203in}}%
\pgfpathclose%
\pgfusepath{fill}%
\end{pgfscope}%
\begin{pgfscope}%
\pgfpathrectangle{\pgfqpoint{0.278819in}{0.345370in}}{\pgfqpoint{1.550000in}{1.347500in}}%
\pgfusepath{clip}%
\pgfsetbuttcap%
\pgfsetroundjoin%
\definecolor{currentfill}{rgb}{0.481929,0.136891,0.507989}%
\pgfsetfillcolor{currentfill}%
\pgfsetlinewidth{0.000000pt}%
\definecolor{currentstroke}{rgb}{0.000000,0.000000,0.000000}%
\pgfsetstrokecolor{currentstroke}%
\pgfsetdash{}{0pt}%
\pgfpathmoveto{\pgfqpoint{0.795485in}{0.345370in}}%
\pgfpathlineto{\pgfqpoint{0.811142in}{0.345370in}}%
\pgfpathlineto{\pgfqpoint{0.826798in}{0.345370in}}%
\pgfpathlineto{\pgfqpoint{0.842455in}{0.345370in}}%
\pgfpathlineto{\pgfqpoint{0.858112in}{0.345370in}}%
\pgfpathlineto{\pgfqpoint{0.873768in}{0.345370in}}%
\pgfpathlineto{\pgfqpoint{0.889425in}{0.345370in}}%
\pgfpathlineto{\pgfqpoint{0.905081in}{0.345370in}}%
\pgfpathlineto{\pgfqpoint{0.920738in}{0.345370in}}%
\pgfpathlineto{\pgfqpoint{0.936394in}{0.345370in}}%
\pgfpathlineto{\pgfqpoint{0.952051in}{0.345370in}}%
\pgfpathlineto{\pgfqpoint{0.967708in}{0.345370in}}%
\pgfpathlineto{\pgfqpoint{0.983364in}{0.345370in}}%
\pgfpathlineto{\pgfqpoint{0.999021in}{0.345370in}}%
\pgfpathlineto{\pgfqpoint{1.014677in}{0.345370in}}%
\pgfpathlineto{\pgfqpoint{1.030334in}{0.345370in}}%
\pgfpathlineto{\pgfqpoint{1.045990in}{0.345370in}}%
\pgfpathlineto{\pgfqpoint{1.061647in}{0.345370in}}%
\pgfpathlineto{\pgfqpoint{1.077303in}{0.345370in}}%
\pgfpathlineto{\pgfqpoint{1.092960in}{0.345370in}}%
\pgfpathlineto{\pgfqpoint{1.108617in}{0.345370in}}%
\pgfpathlineto{\pgfqpoint{1.124273in}{0.345370in}}%
\pgfpathlineto{\pgfqpoint{1.139930in}{0.345370in}}%
\pgfpathlineto{\pgfqpoint{1.155586in}{0.345370in}}%
\pgfpathlineto{\pgfqpoint{1.171243in}{0.345370in}}%
\pgfpathlineto{\pgfqpoint{1.186899in}{0.345370in}}%
\pgfpathlineto{\pgfqpoint{1.202556in}{0.345370in}}%
\pgfpathlineto{\pgfqpoint{1.218213in}{0.345370in}}%
\pgfpathlineto{\pgfqpoint{1.233869in}{0.345370in}}%
\pgfpathlineto{\pgfqpoint{1.249526in}{0.345370in}}%
\pgfpathlineto{\pgfqpoint{1.265182in}{0.345370in}}%
\pgfpathlineto{\pgfqpoint{1.280839in}{0.345370in}}%
\pgfpathlineto{\pgfqpoint{1.296495in}{0.345370in}}%
\pgfpathlineto{\pgfqpoint{1.312152in}{0.345370in}}%
\pgfpathlineto{\pgfqpoint{1.319147in}{0.345370in}}%
\pgfpathlineto{\pgfqpoint{1.319942in}{0.358981in}}%
\pgfpathlineto{\pgfqpoint{1.322318in}{0.372592in}}%
\pgfpathlineto{\pgfqpoint{1.326249in}{0.386203in}}%
\pgfpathlineto{\pgfqpoint{1.327809in}{0.390091in}}%
\pgfpathlineto{\pgfqpoint{1.331575in}{0.399814in}}%
\pgfpathlineto{\pgfqpoint{1.338264in}{0.413425in}}%
\pgfpathlineto{\pgfqpoint{1.343465in}{0.422220in}}%
\pgfpathlineto{\pgfqpoint{1.346223in}{0.427036in}}%
\pgfpathlineto{\pgfqpoint{1.355275in}{0.440648in}}%
\pgfpathlineto{\pgfqpoint{1.359122in}{0.445759in}}%
\pgfpathlineto{\pgfqpoint{1.365331in}{0.454259in}}%
\pgfpathlineto{\pgfqpoint{1.374778in}{0.465940in}}%
\pgfpathlineto{\pgfqpoint{1.376299in}{0.467870in}}%
\pgfpathlineto{\pgfqpoint{1.388009in}{0.481481in}}%
\pgfpathlineto{\pgfqpoint{1.390435in}{0.484098in}}%
\pgfpathlineto{\pgfqpoint{1.400390in}{0.495092in}}%
\pgfpathlineto{\pgfqpoint{1.406091in}{0.501007in}}%
\pgfpathlineto{\pgfqpoint{1.413363in}{0.508703in}}%
\pgfpathlineto{\pgfqpoint{1.421748in}{0.517120in}}%
\pgfpathlineto{\pgfqpoint{1.426838in}{0.522314in}}%
\pgfpathlineto{\pgfqpoint{1.437404in}{0.532624in}}%
\pgfpathlineto{\pgfqpoint{1.440744in}{0.535925in}}%
\pgfpathlineto{\pgfqpoint{1.453061in}{0.547652in}}%
\pgfpathlineto{\pgfqpoint{1.455021in}{0.549536in}}%
\pgfpathlineto{\pgfqpoint{1.468718in}{0.562299in}}%
\pgfpathlineto{\pgfqpoint{1.469623in}{0.563148in}}%
\pgfpathlineto{\pgfqpoint{1.484374in}{0.576631in}}%
\pgfpathlineto{\pgfqpoint{1.484514in}{0.576759in}}%
\pgfpathlineto{\pgfqpoint{1.499674in}{0.590370in}}%
\pgfpathlineto{\pgfqpoint{1.500031in}{0.590685in}}%
\pgfpathlineto{\pgfqpoint{1.515086in}{0.603981in}}%
\pgfpathlineto{\pgfqpoint{1.515687in}{0.604507in}}%
\pgfpathlineto{\pgfqpoint{1.530739in}{0.617592in}}%
\pgfpathlineto{\pgfqpoint{1.531344in}{0.618115in}}%
\pgfpathlineto{\pgfqpoint{1.546638in}{0.631203in}}%
\pgfpathlineto{\pgfqpoint{1.547000in}{0.631513in}}%
\pgfpathlineto{\pgfqpoint{1.562657in}{0.644693in}}%
\pgfpathlineto{\pgfqpoint{1.562804in}{0.644814in}}%
\pgfpathlineto{\pgfqpoint{1.578314in}{0.657638in}}%
\pgfpathlineto{\pgfqpoint{1.579290in}{0.658425in}}%
\pgfpathlineto{\pgfqpoint{1.593970in}{0.670333in}}%
\pgfpathlineto{\pgfqpoint{1.596137in}{0.672036in}}%
\pgfpathlineto{\pgfqpoint{1.609627in}{0.682744in}}%
\pgfpathlineto{\pgfqpoint{1.613424in}{0.685648in}}%
\pgfpathlineto{\pgfqpoint{1.625283in}{0.694834in}}%
\pgfpathlineto{\pgfqpoint{1.631259in}{0.699259in}}%
\pgfpathlineto{\pgfqpoint{1.640940in}{0.706548in}}%
\pgfpathlineto{\pgfqpoint{1.649793in}{0.712870in}}%
\pgfpathlineto{\pgfqpoint{1.656596in}{0.717826in}}%
\pgfpathlineto{\pgfqpoint{1.669243in}{0.726481in}}%
\pgfpathlineto{\pgfqpoint{1.672253in}{0.728590in}}%
\pgfpathlineto{\pgfqpoint{1.687910in}{0.738770in}}%
\pgfpathlineto{\pgfqpoint{1.690129in}{0.740092in}}%
\pgfpathlineto{\pgfqpoint{1.703566in}{0.748305in}}%
\pgfpathlineto{\pgfqpoint{1.713343in}{0.753703in}}%
\pgfpathlineto{\pgfqpoint{1.719223in}{0.757047in}}%
\pgfpathlineto{\pgfqpoint{1.734879in}{0.764917in}}%
\pgfpathlineto{\pgfqpoint{1.740419in}{0.767314in}}%
\pgfpathlineto{\pgfqpoint{1.750536in}{0.771836in}}%
\pgfpathlineto{\pgfqpoint{1.766192in}{0.777651in}}%
\pgfpathlineto{\pgfqpoint{1.777377in}{0.780925in}}%
\pgfpathlineto{\pgfqpoint{1.781849in}{0.782281in}}%
\pgfpathlineto{\pgfqpoint{1.797505in}{0.785699in}}%
\pgfpathlineto{\pgfqpoint{1.813162in}{0.787764in}}%
\pgfpathlineto{\pgfqpoint{1.828819in}{0.788455in}}%
\pgfpathlineto{\pgfqpoint{1.828819in}{0.794536in}}%
\pgfpathlineto{\pgfqpoint{1.828819in}{0.808148in}}%
\pgfpathlineto{\pgfqpoint{1.828819in}{0.821759in}}%
\pgfpathlineto{\pgfqpoint{1.828819in}{0.835370in}}%
\pgfpathlineto{\pgfqpoint{1.828819in}{0.848981in}}%
\pgfpathlineto{\pgfqpoint{1.828819in}{0.862592in}}%
\pgfpathlineto{\pgfqpoint{1.828819in}{0.876203in}}%
\pgfpathlineto{\pgfqpoint{1.828819in}{0.889814in}}%
\pgfpathlineto{\pgfqpoint{1.828819in}{0.903425in}}%
\pgfpathlineto{\pgfqpoint{1.828819in}{0.917036in}}%
\pgfpathlineto{\pgfqpoint{1.828819in}{0.930648in}}%
\pgfpathlineto{\pgfqpoint{1.828819in}{0.944259in}}%
\pgfpathlineto{\pgfqpoint{1.828819in}{0.957870in}}%
\pgfpathlineto{\pgfqpoint{1.828819in}{0.971481in}}%
\pgfpathlineto{\pgfqpoint{1.828819in}{0.985092in}}%
\pgfpathlineto{\pgfqpoint{1.828819in}{0.998703in}}%
\pgfpathlineto{\pgfqpoint{1.828819in}{1.012314in}}%
\pgfpathlineto{\pgfqpoint{1.828819in}{1.025925in}}%
\pgfpathlineto{\pgfqpoint{1.828819in}{1.039536in}}%
\pgfpathlineto{\pgfqpoint{1.828819in}{1.053148in}}%
\pgfpathlineto{\pgfqpoint{1.828819in}{1.066759in}}%
\pgfpathlineto{\pgfqpoint{1.828819in}{1.080370in}}%
\pgfpathlineto{\pgfqpoint{1.828819in}{1.093981in}}%
\pgfpathlineto{\pgfqpoint{1.828819in}{1.107592in}}%
\pgfpathlineto{\pgfqpoint{1.828819in}{1.121203in}}%
\pgfpathlineto{\pgfqpoint{1.828819in}{1.134814in}}%
\pgfpathlineto{\pgfqpoint{1.828819in}{1.148425in}}%
\pgfpathlineto{\pgfqpoint{1.828819in}{1.162036in}}%
\pgfpathlineto{\pgfqpoint{1.828819in}{1.175647in}}%
\pgfpathlineto{\pgfqpoint{1.828819in}{1.189259in}}%
\pgfpathlineto{\pgfqpoint{1.828819in}{1.202870in}}%
\pgfpathlineto{\pgfqpoint{1.828819in}{1.216481in}}%
\pgfpathlineto{\pgfqpoint{1.828819in}{1.230092in}}%
\pgfpathlineto{\pgfqpoint{1.828819in}{1.243703in}}%
\pgfpathlineto{\pgfqpoint{1.828819in}{1.249784in}}%
\pgfpathlineto{\pgfqpoint{1.813162in}{1.250475in}}%
\pgfpathlineto{\pgfqpoint{1.797505in}{1.252541in}}%
\pgfpathlineto{\pgfqpoint{1.781849in}{1.255958in}}%
\pgfpathlineto{\pgfqpoint{1.777377in}{1.257314in}}%
\pgfpathlineto{\pgfqpoint{1.766192in}{1.260588in}}%
\pgfpathlineto{\pgfqpoint{1.750536in}{1.266404in}}%
\pgfpathlineto{\pgfqpoint{1.740419in}{1.270925in}}%
\pgfpathlineto{\pgfqpoint{1.734879in}{1.273323in}}%
\pgfpathlineto{\pgfqpoint{1.719223in}{1.281192in}}%
\pgfpathlineto{\pgfqpoint{1.713343in}{1.284536in}}%
\pgfpathlineto{\pgfqpoint{1.703566in}{1.289935in}}%
\pgfpathlineto{\pgfqpoint{1.690129in}{1.298148in}}%
\pgfpathlineto{\pgfqpoint{1.687910in}{1.299469in}}%
\pgfpathlineto{\pgfqpoint{1.672253in}{1.309650in}}%
\pgfpathlineto{\pgfqpoint{1.669243in}{1.311759in}}%
\pgfpathlineto{\pgfqpoint{1.656596in}{1.320413in}}%
\pgfpathlineto{\pgfqpoint{1.649793in}{1.325370in}}%
\pgfpathlineto{\pgfqpoint{1.640940in}{1.331691in}}%
\pgfpathlineto{\pgfqpoint{1.631259in}{1.338981in}}%
\pgfpathlineto{\pgfqpoint{1.625283in}{1.343406in}}%
\pgfpathlineto{\pgfqpoint{1.613424in}{1.352592in}}%
\pgfpathlineto{\pgfqpoint{1.609627in}{1.355495in}}%
\pgfpathlineto{\pgfqpoint{1.596137in}{1.366203in}}%
\pgfpathlineto{\pgfqpoint{1.593970in}{1.367907in}}%
\pgfpathlineto{\pgfqpoint{1.579290in}{1.379814in}}%
\pgfpathlineto{\pgfqpoint{1.578314in}{1.380601in}}%
\pgfpathlineto{\pgfqpoint{1.562804in}{1.393425in}}%
\pgfpathlineto{\pgfqpoint{1.562657in}{1.393547in}}%
\pgfpathlineto{\pgfqpoint{1.547000in}{1.406727in}}%
\pgfpathlineto{\pgfqpoint{1.546638in}{1.407036in}}%
\pgfpathlineto{\pgfqpoint{1.531344in}{1.420124in}}%
\pgfpathlineto{\pgfqpoint{1.530739in}{1.420648in}}%
\pgfpathlineto{\pgfqpoint{1.515687in}{1.433733in}}%
\pgfpathlineto{\pgfqpoint{1.515086in}{1.434259in}}%
\pgfpathlineto{\pgfqpoint{1.500031in}{1.447555in}}%
\pgfpathlineto{\pgfqpoint{1.499674in}{1.447870in}}%
\pgfpathlineto{\pgfqpoint{1.484514in}{1.461481in}}%
\pgfpathlineto{\pgfqpoint{1.484374in}{1.461609in}}%
\pgfpathlineto{\pgfqpoint{1.469623in}{1.475092in}}%
\pgfpathlineto{\pgfqpoint{1.468718in}{1.475940in}}%
\pgfpathlineto{\pgfqpoint{1.455021in}{1.488703in}}%
\pgfpathlineto{\pgfqpoint{1.453061in}{1.490587in}}%
\pgfpathlineto{\pgfqpoint{1.440744in}{1.502314in}}%
\pgfpathlineto{\pgfqpoint{1.437404in}{1.505615in}}%
\pgfpathlineto{\pgfqpoint{1.426838in}{1.515925in}}%
\pgfpathlineto{\pgfqpoint{1.421748in}{1.521120in}}%
\pgfpathlineto{\pgfqpoint{1.413363in}{1.529536in}}%
\pgfpathlineto{\pgfqpoint{1.406091in}{1.537233in}}%
\pgfpathlineto{\pgfqpoint{1.400390in}{1.543148in}}%
\pgfpathlineto{\pgfqpoint{1.390435in}{1.554142in}}%
\pgfpathlineto{\pgfqpoint{1.388009in}{1.556759in}}%
\pgfpathlineto{\pgfqpoint{1.376299in}{1.570370in}}%
\pgfpathlineto{\pgfqpoint{1.374778in}{1.572300in}}%
\pgfpathlineto{\pgfqpoint{1.365331in}{1.583981in}}%
\pgfpathlineto{\pgfqpoint{1.359122in}{1.592480in}}%
\pgfpathlineto{\pgfqpoint{1.355275in}{1.597592in}}%
\pgfpathlineto{\pgfqpoint{1.346223in}{1.611203in}}%
\pgfpathlineto{\pgfqpoint{1.343465in}{1.616020in}}%
\pgfpathlineto{\pgfqpoint{1.338264in}{1.624814in}}%
\pgfpathlineto{\pgfqpoint{1.331575in}{1.638425in}}%
\pgfpathlineto{\pgfqpoint{1.327809in}{1.648149in}}%
\pgfpathlineto{\pgfqpoint{1.326249in}{1.652036in}}%
\pgfpathlineto{\pgfqpoint{1.322318in}{1.665648in}}%
\pgfpathlineto{\pgfqpoint{1.319942in}{1.679259in}}%
\pgfpathlineto{\pgfqpoint{1.319147in}{1.692870in}}%
\pgfpathlineto{\pgfqpoint{1.312152in}{1.692870in}}%
\pgfpathlineto{\pgfqpoint{1.296495in}{1.692870in}}%
\pgfpathlineto{\pgfqpoint{1.280839in}{1.692870in}}%
\pgfpathlineto{\pgfqpoint{1.265182in}{1.692870in}}%
\pgfpathlineto{\pgfqpoint{1.249526in}{1.692870in}}%
\pgfpathlineto{\pgfqpoint{1.233869in}{1.692870in}}%
\pgfpathlineto{\pgfqpoint{1.218213in}{1.692870in}}%
\pgfpathlineto{\pgfqpoint{1.202556in}{1.692870in}}%
\pgfpathlineto{\pgfqpoint{1.186899in}{1.692870in}}%
\pgfpathlineto{\pgfqpoint{1.171243in}{1.692870in}}%
\pgfpathlineto{\pgfqpoint{1.155586in}{1.692870in}}%
\pgfpathlineto{\pgfqpoint{1.139930in}{1.692870in}}%
\pgfpathlineto{\pgfqpoint{1.124273in}{1.692870in}}%
\pgfpathlineto{\pgfqpoint{1.108617in}{1.692870in}}%
\pgfpathlineto{\pgfqpoint{1.092960in}{1.692870in}}%
\pgfpathlineto{\pgfqpoint{1.077303in}{1.692870in}}%
\pgfpathlineto{\pgfqpoint{1.061647in}{1.692870in}}%
\pgfpathlineto{\pgfqpoint{1.045990in}{1.692870in}}%
\pgfpathlineto{\pgfqpoint{1.030334in}{1.692870in}}%
\pgfpathlineto{\pgfqpoint{1.014677in}{1.692870in}}%
\pgfpathlineto{\pgfqpoint{0.999021in}{1.692870in}}%
\pgfpathlineto{\pgfqpoint{0.983364in}{1.692870in}}%
\pgfpathlineto{\pgfqpoint{0.967708in}{1.692870in}}%
\pgfpathlineto{\pgfqpoint{0.952051in}{1.692870in}}%
\pgfpathlineto{\pgfqpoint{0.936394in}{1.692870in}}%
\pgfpathlineto{\pgfqpoint{0.920738in}{1.692870in}}%
\pgfpathlineto{\pgfqpoint{0.905081in}{1.692870in}}%
\pgfpathlineto{\pgfqpoint{0.889425in}{1.692870in}}%
\pgfpathlineto{\pgfqpoint{0.873768in}{1.692870in}}%
\pgfpathlineto{\pgfqpoint{0.858112in}{1.692870in}}%
\pgfpathlineto{\pgfqpoint{0.842455in}{1.692870in}}%
\pgfpathlineto{\pgfqpoint{0.826798in}{1.692870in}}%
\pgfpathlineto{\pgfqpoint{0.811142in}{1.692870in}}%
\pgfpathlineto{\pgfqpoint{0.795485in}{1.692870in}}%
\pgfpathlineto{\pgfqpoint{0.788490in}{1.692870in}}%
\pgfpathlineto{\pgfqpoint{0.787696in}{1.679259in}}%
\pgfpathlineto{\pgfqpoint{0.785320in}{1.665648in}}%
\pgfpathlineto{\pgfqpoint{0.781388in}{1.652036in}}%
\pgfpathlineto{\pgfqpoint{0.779829in}{1.648149in}}%
\pgfpathlineto{\pgfqpoint{0.776063in}{1.638425in}}%
\pgfpathlineto{\pgfqpoint{0.769373in}{1.624814in}}%
\pgfpathlineto{\pgfqpoint{0.764172in}{1.616020in}}%
\pgfpathlineto{\pgfqpoint{0.761415in}{1.611203in}}%
\pgfpathlineto{\pgfqpoint{0.752362in}{1.597592in}}%
\pgfpathlineto{\pgfqpoint{0.748516in}{1.592480in}}%
\pgfpathlineto{\pgfqpoint{0.742306in}{1.583981in}}%
\pgfpathlineto{\pgfqpoint{0.732859in}{1.572300in}}%
\pgfpathlineto{\pgfqpoint{0.731339in}{1.570370in}}%
\pgfpathlineto{\pgfqpoint{0.719628in}{1.556759in}}%
\pgfpathlineto{\pgfqpoint{0.717202in}{1.554142in}}%
\pgfpathlineto{\pgfqpoint{0.707247in}{1.543148in}}%
\pgfpathlineto{\pgfqpoint{0.701546in}{1.537233in}}%
\pgfpathlineto{\pgfqpoint{0.694275in}{1.529536in}}%
\pgfpathlineto{\pgfqpoint{0.685889in}{1.521120in}}%
\pgfpathlineto{\pgfqpoint{0.680799in}{1.515925in}}%
\pgfpathlineto{\pgfqpoint{0.670233in}{1.505615in}}%
\pgfpathlineto{\pgfqpoint{0.666893in}{1.502314in}}%
\pgfpathlineto{\pgfqpoint{0.654576in}{1.490587in}}%
\pgfpathlineto{\pgfqpoint{0.652616in}{1.488703in}}%
\pgfpathlineto{\pgfqpoint{0.638920in}{1.475940in}}%
\pgfpathlineto{\pgfqpoint{0.638014in}{1.475092in}}%
\pgfpathlineto{\pgfqpoint{0.623263in}{1.461609in}}%
\pgfpathlineto{\pgfqpoint{0.623124in}{1.461481in}}%
\pgfpathlineto{\pgfqpoint{0.607963in}{1.447870in}}%
\pgfpathlineto{\pgfqpoint{0.607606in}{1.447555in}}%
\pgfpathlineto{\pgfqpoint{0.592552in}{1.434259in}}%
\pgfpathlineto{\pgfqpoint{0.591950in}{1.433733in}}%
\pgfpathlineto{\pgfqpoint{0.576899in}{1.420648in}}%
\pgfpathlineto{\pgfqpoint{0.576293in}{1.420124in}}%
\pgfpathlineto{\pgfqpoint{0.560999in}{1.407036in}}%
\pgfpathlineto{\pgfqpoint{0.560637in}{1.406727in}}%
\pgfpathlineto{\pgfqpoint{0.544980in}{1.393547in}}%
\pgfpathlineto{\pgfqpoint{0.544833in}{1.393425in}}%
\pgfpathlineto{\pgfqpoint{0.529324in}{1.380601in}}%
\pgfpathlineto{\pgfqpoint{0.528348in}{1.379814in}}%
\pgfpathlineto{\pgfqpoint{0.513667in}{1.367907in}}%
\pgfpathlineto{\pgfqpoint{0.511500in}{1.366203in}}%
\pgfpathlineto{\pgfqpoint{0.498011in}{1.355495in}}%
\pgfpathlineto{\pgfqpoint{0.494213in}{1.352592in}}%
\pgfpathlineto{\pgfqpoint{0.482354in}{1.343406in}}%
\pgfpathlineto{\pgfqpoint{0.476379in}{1.338981in}}%
\pgfpathlineto{\pgfqpoint{0.466697in}{1.331691in}}%
\pgfpathlineto{\pgfqpoint{0.457844in}{1.325370in}}%
\pgfpathlineto{\pgfqpoint{0.451041in}{1.320413in}}%
\pgfpathlineto{\pgfqpoint{0.438394in}{1.311759in}}%
\pgfpathlineto{\pgfqpoint{0.435384in}{1.309650in}}%
\pgfpathlineto{\pgfqpoint{0.419728in}{1.299469in}}%
\pgfpathlineto{\pgfqpoint{0.417508in}{1.298148in}}%
\pgfpathlineto{\pgfqpoint{0.404071in}{1.289935in}}%
\pgfpathlineto{\pgfqpoint{0.394294in}{1.284536in}}%
\pgfpathlineto{\pgfqpoint{0.388415in}{1.281192in}}%
\pgfpathlineto{\pgfqpoint{0.372758in}{1.273323in}}%
\pgfpathlineto{\pgfqpoint{0.367218in}{1.270925in}}%
\pgfpathlineto{\pgfqpoint{0.357101in}{1.266404in}}%
\pgfpathlineto{\pgfqpoint{0.341445in}{1.260588in}}%
\pgfpathlineto{\pgfqpoint{0.330260in}{1.257314in}}%
\pgfpathlineto{\pgfqpoint{0.325788in}{1.255958in}}%
\pgfpathlineto{\pgfqpoint{0.310132in}{1.252541in}}%
\pgfpathlineto{\pgfqpoint{0.294475in}{1.250475in}}%
\pgfpathlineto{\pgfqpoint{0.278819in}{1.249784in}}%
\pgfpathlineto{\pgfqpoint{0.278819in}{1.243703in}}%
\pgfpathlineto{\pgfqpoint{0.278819in}{1.230092in}}%
\pgfpathlineto{\pgfqpoint{0.278819in}{1.216481in}}%
\pgfpathlineto{\pgfqpoint{0.278819in}{1.202870in}}%
\pgfpathlineto{\pgfqpoint{0.278819in}{1.189259in}}%
\pgfpathlineto{\pgfqpoint{0.278819in}{1.175647in}}%
\pgfpathlineto{\pgfqpoint{0.278819in}{1.162036in}}%
\pgfpathlineto{\pgfqpoint{0.278819in}{1.148425in}}%
\pgfpathlineto{\pgfqpoint{0.278819in}{1.134814in}}%
\pgfpathlineto{\pgfqpoint{0.278819in}{1.121203in}}%
\pgfpathlineto{\pgfqpoint{0.278819in}{1.107592in}}%
\pgfpathlineto{\pgfqpoint{0.278819in}{1.093981in}}%
\pgfpathlineto{\pgfqpoint{0.278819in}{1.080370in}}%
\pgfpathlineto{\pgfqpoint{0.278819in}{1.066759in}}%
\pgfpathlineto{\pgfqpoint{0.278819in}{1.053148in}}%
\pgfpathlineto{\pgfqpoint{0.278819in}{1.039536in}}%
\pgfpathlineto{\pgfqpoint{0.278819in}{1.025925in}}%
\pgfpathlineto{\pgfqpoint{0.278819in}{1.012314in}}%
\pgfpathlineto{\pgfqpoint{0.278819in}{0.998703in}}%
\pgfpathlineto{\pgfqpoint{0.278819in}{0.985092in}}%
\pgfpathlineto{\pgfqpoint{0.278819in}{0.971481in}}%
\pgfpathlineto{\pgfqpoint{0.278819in}{0.957870in}}%
\pgfpathlineto{\pgfqpoint{0.278819in}{0.944259in}}%
\pgfpathlineto{\pgfqpoint{0.278819in}{0.930648in}}%
\pgfpathlineto{\pgfqpoint{0.278819in}{0.917036in}}%
\pgfpathlineto{\pgfqpoint{0.278819in}{0.903425in}}%
\pgfpathlineto{\pgfqpoint{0.278819in}{0.889814in}}%
\pgfpathlineto{\pgfqpoint{0.278819in}{0.876203in}}%
\pgfpathlineto{\pgfqpoint{0.278819in}{0.862592in}}%
\pgfpathlineto{\pgfqpoint{0.278819in}{0.848981in}}%
\pgfpathlineto{\pgfqpoint{0.278819in}{0.835370in}}%
\pgfpathlineto{\pgfqpoint{0.278819in}{0.821759in}}%
\pgfpathlineto{\pgfqpoint{0.278819in}{0.808148in}}%
\pgfpathlineto{\pgfqpoint{0.278819in}{0.794536in}}%
\pgfpathlineto{\pgfqpoint{0.278819in}{0.788455in}}%
\pgfpathlineto{\pgfqpoint{0.294475in}{0.787764in}}%
\pgfpathlineto{\pgfqpoint{0.310132in}{0.785699in}}%
\pgfpathlineto{\pgfqpoint{0.325788in}{0.782281in}}%
\pgfpathlineto{\pgfqpoint{0.330260in}{0.780925in}}%
\pgfpathlineto{\pgfqpoint{0.341445in}{0.777651in}}%
\pgfpathlineto{\pgfqpoint{0.357101in}{0.771836in}}%
\pgfpathlineto{\pgfqpoint{0.367218in}{0.767314in}}%
\pgfpathlineto{\pgfqpoint{0.372758in}{0.764917in}}%
\pgfpathlineto{\pgfqpoint{0.388415in}{0.757047in}}%
\pgfpathlineto{\pgfqpoint{0.394294in}{0.753703in}}%
\pgfpathlineto{\pgfqpoint{0.404071in}{0.748305in}}%
\pgfpathlineto{\pgfqpoint{0.417508in}{0.740092in}}%
\pgfpathlineto{\pgfqpoint{0.419728in}{0.738770in}}%
\pgfpathlineto{\pgfqpoint{0.435384in}{0.728590in}}%
\pgfpathlineto{\pgfqpoint{0.438394in}{0.726481in}}%
\pgfpathlineto{\pgfqpoint{0.451041in}{0.717826in}}%
\pgfpathlineto{\pgfqpoint{0.457844in}{0.712870in}}%
\pgfpathlineto{\pgfqpoint{0.466697in}{0.706548in}}%
\pgfpathlineto{\pgfqpoint{0.476379in}{0.699259in}}%
\pgfpathlineto{\pgfqpoint{0.482354in}{0.694834in}}%
\pgfpathlineto{\pgfqpoint{0.494213in}{0.685648in}}%
\pgfpathlineto{\pgfqpoint{0.498011in}{0.682744in}}%
\pgfpathlineto{\pgfqpoint{0.511500in}{0.672036in}}%
\pgfpathlineto{\pgfqpoint{0.513667in}{0.670333in}}%
\pgfpathlineto{\pgfqpoint{0.528348in}{0.658425in}}%
\pgfpathlineto{\pgfqpoint{0.529324in}{0.657638in}}%
\pgfpathlineto{\pgfqpoint{0.544833in}{0.644814in}}%
\pgfpathlineto{\pgfqpoint{0.544980in}{0.644693in}}%
\pgfpathlineto{\pgfqpoint{0.560637in}{0.631513in}}%
\pgfpathlineto{\pgfqpoint{0.560999in}{0.631203in}}%
\pgfpathlineto{\pgfqpoint{0.576293in}{0.618115in}}%
\pgfpathlineto{\pgfqpoint{0.576899in}{0.617592in}}%
\pgfpathlineto{\pgfqpoint{0.591950in}{0.604507in}}%
\pgfpathlineto{\pgfqpoint{0.592552in}{0.603981in}}%
\pgfpathlineto{\pgfqpoint{0.607606in}{0.590685in}}%
\pgfpathlineto{\pgfqpoint{0.607963in}{0.590370in}}%
\pgfpathlineto{\pgfqpoint{0.623124in}{0.576759in}}%
\pgfpathlineto{\pgfqpoint{0.623263in}{0.576631in}}%
\pgfpathlineto{\pgfqpoint{0.638014in}{0.563148in}}%
\pgfpathlineto{\pgfqpoint{0.638920in}{0.562299in}}%
\pgfpathlineto{\pgfqpoint{0.652616in}{0.549536in}}%
\pgfpathlineto{\pgfqpoint{0.654576in}{0.547652in}}%
\pgfpathlineto{\pgfqpoint{0.666893in}{0.535925in}}%
\pgfpathlineto{\pgfqpoint{0.670233in}{0.532624in}}%
\pgfpathlineto{\pgfqpoint{0.680799in}{0.522314in}}%
\pgfpathlineto{\pgfqpoint{0.685889in}{0.517120in}}%
\pgfpathlineto{\pgfqpoint{0.694275in}{0.508703in}}%
\pgfpathlineto{\pgfqpoint{0.701546in}{0.501007in}}%
\pgfpathlineto{\pgfqpoint{0.707247in}{0.495092in}}%
\pgfpathlineto{\pgfqpoint{0.717202in}{0.484098in}}%
\pgfpathlineto{\pgfqpoint{0.719628in}{0.481481in}}%
\pgfpathlineto{\pgfqpoint{0.731339in}{0.467870in}}%
\pgfpathlineto{\pgfqpoint{0.732859in}{0.465940in}}%
\pgfpathlineto{\pgfqpoint{0.742306in}{0.454259in}}%
\pgfpathlineto{\pgfqpoint{0.748516in}{0.445759in}}%
\pgfpathlineto{\pgfqpoint{0.752362in}{0.440648in}}%
\pgfpathlineto{\pgfqpoint{0.761415in}{0.427036in}}%
\pgfpathlineto{\pgfqpoint{0.764172in}{0.422220in}}%
\pgfpathlineto{\pgfqpoint{0.769373in}{0.413425in}}%
\pgfpathlineto{\pgfqpoint{0.776063in}{0.399814in}}%
\pgfpathlineto{\pgfqpoint{0.779829in}{0.390091in}}%
\pgfpathlineto{\pgfqpoint{0.781388in}{0.386203in}}%
\pgfpathlineto{\pgfqpoint{0.785320in}{0.372592in}}%
\pgfpathlineto{\pgfqpoint{0.787696in}{0.358981in}}%
\pgfpathlineto{\pgfqpoint{0.788490in}{0.345370in}}%
\pgfpathlineto{\pgfqpoint{0.795485in}{0.345370in}}%
\pgfpathclose%
\pgfpathmoveto{\pgfqpoint{1.028018in}{0.481481in}}%
\pgfpathlineto{\pgfqpoint{1.014677in}{0.483093in}}%
\pgfpathlineto{\pgfqpoint{0.999021in}{0.485931in}}%
\pgfpathlineto{\pgfqpoint{0.983364in}{0.489717in}}%
\pgfpathlineto{\pgfqpoint{0.967708in}{0.494452in}}%
\pgfpathlineto{\pgfqpoint{0.965937in}{0.495092in}}%
\pgfpathlineto{\pgfqpoint{0.952051in}{0.499767in}}%
\pgfpathlineto{\pgfqpoint{0.936394in}{0.505916in}}%
\pgfpathlineto{\pgfqpoint{0.930167in}{0.508703in}}%
\pgfpathlineto{\pgfqpoint{0.920738in}{0.512673in}}%
\pgfpathlineto{\pgfqpoint{0.905081in}{0.520075in}}%
\pgfpathlineto{\pgfqpoint{0.900798in}{0.522314in}}%
\pgfpathlineto{\pgfqpoint{0.889425in}{0.527960in}}%
\pgfpathlineto{\pgfqpoint{0.874803in}{0.535925in}}%
\pgfpathlineto{\pgfqpoint{0.873768in}{0.536465in}}%
\pgfpathlineto{\pgfqpoint{0.858112in}{0.545304in}}%
\pgfpathlineto{\pgfqpoint{0.851155in}{0.549536in}}%
\pgfpathlineto{\pgfqpoint{0.842455in}{0.554640in}}%
\pgfpathlineto{\pgfqpoint{0.828907in}{0.563148in}}%
\pgfpathlineto{\pgfqpoint{0.826798in}{0.564433in}}%
\pgfpathlineto{\pgfqpoint{0.811142in}{0.574560in}}%
\pgfpathlineto{\pgfqpoint{0.807923in}{0.576759in}}%
\pgfpathlineto{\pgfqpoint{0.795485in}{0.585056in}}%
\pgfpathlineto{\pgfqpoint{0.787911in}{0.590370in}}%
\pgfpathlineto{\pgfqpoint{0.779829in}{0.595939in}}%
\pgfpathlineto{\pgfqpoint{0.768677in}{0.603981in}}%
\pgfpathlineto{\pgfqpoint{0.764172in}{0.607188in}}%
\pgfpathlineto{\pgfqpoint{0.750142in}{0.617592in}}%
\pgfpathlineto{\pgfqpoint{0.748516in}{0.618789in}}%
\pgfpathlineto{\pgfqpoint{0.732859in}{0.630726in}}%
\pgfpathlineto{\pgfqpoint{0.732252in}{0.631203in}}%
\pgfpathlineto{\pgfqpoint{0.717202in}{0.643000in}}%
\pgfpathlineto{\pgfqpoint{0.714951in}{0.644814in}}%
\pgfpathlineto{\pgfqpoint{0.701546in}{0.655636in}}%
\pgfpathlineto{\pgfqpoint{0.698170in}{0.658425in}}%
\pgfpathlineto{\pgfqpoint{0.685889in}{0.668636in}}%
\pgfpathlineto{\pgfqpoint{0.681877in}{0.672036in}}%
\pgfpathlineto{\pgfqpoint{0.670233in}{0.682010in}}%
\pgfpathlineto{\pgfqpoint{0.666049in}{0.685648in}}%
\pgfpathlineto{\pgfqpoint{0.654576in}{0.695771in}}%
\pgfpathlineto{\pgfqpoint{0.650665in}{0.699259in}}%
\pgfpathlineto{\pgfqpoint{0.638920in}{0.709935in}}%
\pgfpathlineto{\pgfqpoint{0.635711in}{0.712870in}}%
\pgfpathlineto{\pgfqpoint{0.623263in}{0.724524in}}%
\pgfpathlineto{\pgfqpoint{0.621177in}{0.726481in}}%
\pgfpathlineto{\pgfqpoint{0.607606in}{0.739564in}}%
\pgfpathlineto{\pgfqpoint{0.607057in}{0.740092in}}%
\pgfpathlineto{\pgfqpoint{0.593326in}{0.753703in}}%
\pgfpathlineto{\pgfqpoint{0.591950in}{0.755117in}}%
\pgfpathlineto{\pgfqpoint{0.579983in}{0.767314in}}%
\pgfpathlineto{\pgfqpoint{0.576293in}{0.771231in}}%
\pgfpathlineto{\pgfqpoint{0.567044in}{0.780925in}}%
\pgfpathlineto{\pgfqpoint{0.560637in}{0.787952in}}%
\pgfpathlineto{\pgfqpoint{0.554524in}{0.794536in}}%
\pgfpathlineto{\pgfqpoint{0.544980in}{0.805349in}}%
\pgfpathlineto{\pgfqpoint{0.542451in}{0.808148in}}%
\pgfpathlineto{\pgfqpoint{0.530802in}{0.821759in}}%
\pgfpathlineto{\pgfqpoint{0.529324in}{0.823592in}}%
\pgfpathlineto{\pgfqpoint{0.519537in}{0.835370in}}%
\pgfpathlineto{\pgfqpoint{0.513667in}{0.842933in}}%
\pgfpathlineto{\pgfqpoint{0.508799in}{0.848981in}}%
\pgfpathlineto{\pgfqpoint{0.498631in}{0.862592in}}%
\pgfpathlineto{\pgfqpoint{0.498011in}{0.863492in}}%
\pgfpathlineto{\pgfqpoint{0.488848in}{0.876203in}}%
\pgfpathlineto{\pgfqpoint{0.482354in}{0.886090in}}%
\pgfpathlineto{\pgfqpoint{0.479778in}{0.889814in}}%
\pgfpathlineto{\pgfqpoint{0.471264in}{0.903425in}}%
\pgfpathlineto{\pgfqpoint{0.466697in}{0.911623in}}%
\pgfpathlineto{\pgfqpoint{0.463492in}{0.917036in}}%
\pgfpathlineto{\pgfqpoint{0.456418in}{0.930648in}}%
\pgfpathlineto{\pgfqpoint{0.451041in}{0.942719in}}%
\pgfpathlineto{\pgfqpoint{0.450304in}{0.944259in}}%
\pgfpathlineto{\pgfqpoint{0.444858in}{0.957870in}}%
\pgfpathlineto{\pgfqpoint{0.440504in}{0.971481in}}%
\pgfpathlineto{\pgfqpoint{0.437239in}{0.985092in}}%
\pgfpathlineto{\pgfqpoint{0.435384in}{0.996690in}}%
\pgfpathlineto{\pgfqpoint{0.435034in}{0.998703in}}%
\pgfpathlineto{\pgfqpoint{0.433848in}{1.012314in}}%
\pgfpathlineto{\pgfqpoint{0.433848in}{1.025925in}}%
\pgfpathlineto{\pgfqpoint{0.435034in}{1.039536in}}%
\pgfpathlineto{\pgfqpoint{0.435384in}{1.041549in}}%
\pgfpathlineto{\pgfqpoint{0.437239in}{1.053148in}}%
\pgfpathlineto{\pgfqpoint{0.440504in}{1.066759in}}%
\pgfpathlineto{\pgfqpoint{0.444858in}{1.080370in}}%
\pgfpathlineto{\pgfqpoint{0.450304in}{1.093981in}}%
\pgfpathlineto{\pgfqpoint{0.451041in}{1.095520in}}%
\pgfpathlineto{\pgfqpoint{0.456418in}{1.107592in}}%
\pgfpathlineto{\pgfqpoint{0.463492in}{1.121203in}}%
\pgfpathlineto{\pgfqpoint{0.466697in}{1.126617in}}%
\pgfpathlineto{\pgfqpoint{0.471264in}{1.134814in}}%
\pgfpathlineto{\pgfqpoint{0.479778in}{1.148425in}}%
\pgfpathlineto{\pgfqpoint{0.482354in}{1.152149in}}%
\pgfpathlineto{\pgfqpoint{0.488848in}{1.162036in}}%
\pgfpathlineto{\pgfqpoint{0.498011in}{1.174747in}}%
\pgfpathlineto{\pgfqpoint{0.498631in}{1.175647in}}%
\pgfpathlineto{\pgfqpoint{0.508799in}{1.189259in}}%
\pgfpathlineto{\pgfqpoint{0.513667in}{1.195306in}}%
\pgfpathlineto{\pgfqpoint{0.519537in}{1.202870in}}%
\pgfpathlineto{\pgfqpoint{0.529324in}{1.214647in}}%
\pgfpathlineto{\pgfqpoint{0.530802in}{1.216481in}}%
\pgfpathlineto{\pgfqpoint{0.542451in}{1.230092in}}%
\pgfpathlineto{\pgfqpoint{0.544980in}{1.232890in}}%
\pgfpathlineto{\pgfqpoint{0.554524in}{1.243703in}}%
\pgfpathlineto{\pgfqpoint{0.560637in}{1.250287in}}%
\pgfpathlineto{\pgfqpoint{0.567044in}{1.257314in}}%
\pgfpathlineto{\pgfqpoint{0.576293in}{1.267009in}}%
\pgfpathlineto{\pgfqpoint{0.579983in}{1.270925in}}%
\pgfpathlineto{\pgfqpoint{0.591950in}{1.283122in}}%
\pgfpathlineto{\pgfqpoint{0.593326in}{1.284536in}}%
\pgfpathlineto{\pgfqpoint{0.607057in}{1.298148in}}%
\pgfpathlineto{\pgfqpoint{0.607606in}{1.298675in}}%
\pgfpathlineto{\pgfqpoint{0.621177in}{1.311759in}}%
\pgfpathlineto{\pgfqpoint{0.623263in}{1.313716in}}%
\pgfpathlineto{\pgfqpoint{0.635711in}{1.325370in}}%
\pgfpathlineto{\pgfqpoint{0.638920in}{1.328305in}}%
\pgfpathlineto{\pgfqpoint{0.650665in}{1.338981in}}%
\pgfpathlineto{\pgfqpoint{0.654576in}{1.342469in}}%
\pgfpathlineto{\pgfqpoint{0.666049in}{1.352592in}}%
\pgfpathlineto{\pgfqpoint{0.670233in}{1.356229in}}%
\pgfpathlineto{\pgfqpoint{0.681877in}{1.366203in}}%
\pgfpathlineto{\pgfqpoint{0.685889in}{1.369603in}}%
\pgfpathlineto{\pgfqpoint{0.698170in}{1.379814in}}%
\pgfpathlineto{\pgfqpoint{0.701546in}{1.382604in}}%
\pgfpathlineto{\pgfqpoint{0.714951in}{1.393425in}}%
\pgfpathlineto{\pgfqpoint{0.717202in}{1.395239in}}%
\pgfpathlineto{\pgfqpoint{0.732252in}{1.407036in}}%
\pgfpathlineto{\pgfqpoint{0.732859in}{1.407514in}}%
\pgfpathlineto{\pgfqpoint{0.748516in}{1.419451in}}%
\pgfpathlineto{\pgfqpoint{0.750142in}{1.420648in}}%
\pgfpathlineto{\pgfqpoint{0.764172in}{1.431051in}}%
\pgfpathlineto{\pgfqpoint{0.768677in}{1.434259in}}%
\pgfpathlineto{\pgfqpoint{0.779829in}{1.442300in}}%
\pgfpathlineto{\pgfqpoint{0.787911in}{1.447870in}}%
\pgfpathlineto{\pgfqpoint{0.795485in}{1.453184in}}%
\pgfpathlineto{\pgfqpoint{0.807923in}{1.461481in}}%
\pgfpathlineto{\pgfqpoint{0.811142in}{1.463679in}}%
\pgfpathlineto{\pgfqpoint{0.826798in}{1.473807in}}%
\pgfpathlineto{\pgfqpoint{0.828907in}{1.475092in}}%
\pgfpathlineto{\pgfqpoint{0.842455in}{1.483600in}}%
\pgfpathlineto{\pgfqpoint{0.851155in}{1.488703in}}%
\pgfpathlineto{\pgfqpoint{0.858112in}{1.492935in}}%
\pgfpathlineto{\pgfqpoint{0.873768in}{1.501774in}}%
\pgfpathlineto{\pgfqpoint{0.874803in}{1.502314in}}%
\pgfpathlineto{\pgfqpoint{0.889425in}{1.510280in}}%
\pgfpathlineto{\pgfqpoint{0.900798in}{1.515925in}}%
\pgfpathlineto{\pgfqpoint{0.905081in}{1.518164in}}%
\pgfpathlineto{\pgfqpoint{0.920738in}{1.525566in}}%
\pgfpathlineto{\pgfqpoint{0.930167in}{1.529536in}}%
\pgfpathlineto{\pgfqpoint{0.936394in}{1.532323in}}%
\pgfpathlineto{\pgfqpoint{0.952051in}{1.538473in}}%
\pgfpathlineto{\pgfqpoint{0.965937in}{1.543148in}}%
\pgfpathlineto{\pgfqpoint{0.967708in}{1.543788in}}%
\pgfpathlineto{\pgfqpoint{0.983364in}{1.548522in}}%
\pgfpathlineto{\pgfqpoint{0.999021in}{1.552308in}}%
\pgfpathlineto{\pgfqpoint{1.014677in}{1.555146in}}%
\pgfpathlineto{\pgfqpoint{1.028018in}{1.556759in}}%
\pgfpathlineto{\pgfqpoint{1.030334in}{1.557063in}}%
\pgfpathlineto{\pgfqpoint{1.045990in}{1.558094in}}%
\pgfpathlineto{\pgfqpoint{1.061647in}{1.558094in}}%
\pgfpathlineto{\pgfqpoint{1.077303in}{1.557063in}}%
\pgfpathlineto{\pgfqpoint{1.079619in}{1.556759in}}%
\pgfpathlineto{\pgfqpoint{1.092960in}{1.555146in}}%
\pgfpathlineto{\pgfqpoint{1.108617in}{1.552308in}}%
\pgfpathlineto{\pgfqpoint{1.124273in}{1.548522in}}%
\pgfpathlineto{\pgfqpoint{1.139930in}{1.543788in}}%
\pgfpathlineto{\pgfqpoint{1.141701in}{1.543148in}}%
\pgfpathlineto{\pgfqpoint{1.155586in}{1.538473in}}%
\pgfpathlineto{\pgfqpoint{1.171243in}{1.532323in}}%
\pgfpathlineto{\pgfqpoint{1.177470in}{1.529536in}}%
\pgfpathlineto{\pgfqpoint{1.186899in}{1.525566in}}%
\pgfpathlineto{\pgfqpoint{1.202556in}{1.518164in}}%
\pgfpathlineto{\pgfqpoint{1.206839in}{1.515925in}}%
\pgfpathlineto{\pgfqpoint{1.218213in}{1.510280in}}%
\pgfpathlineto{\pgfqpoint{1.232834in}{1.502314in}}%
\pgfpathlineto{\pgfqpoint{1.233869in}{1.501774in}}%
\pgfpathlineto{\pgfqpoint{1.249526in}{1.492935in}}%
\pgfpathlineto{\pgfqpoint{1.256482in}{1.488703in}}%
\pgfpathlineto{\pgfqpoint{1.265182in}{1.483600in}}%
\pgfpathlineto{\pgfqpoint{1.278730in}{1.475092in}}%
\pgfpathlineto{\pgfqpoint{1.280839in}{1.473807in}}%
\pgfpathlineto{\pgfqpoint{1.296495in}{1.463679in}}%
\pgfpathlineto{\pgfqpoint{1.299714in}{1.461481in}}%
\pgfpathlineto{\pgfqpoint{1.312152in}{1.453184in}}%
\pgfpathlineto{\pgfqpoint{1.319726in}{1.447870in}}%
\pgfpathlineto{\pgfqpoint{1.327809in}{1.442300in}}%
\pgfpathlineto{\pgfqpoint{1.338960in}{1.434259in}}%
\pgfpathlineto{\pgfqpoint{1.343465in}{1.431051in}}%
\pgfpathlineto{\pgfqpoint{1.357495in}{1.420648in}}%
\pgfpathlineto{\pgfqpoint{1.359122in}{1.419451in}}%
\pgfpathlineto{\pgfqpoint{1.374778in}{1.407514in}}%
\pgfpathlineto{\pgfqpoint{1.375385in}{1.407036in}}%
\pgfpathlineto{\pgfqpoint{1.390435in}{1.395239in}}%
\pgfpathlineto{\pgfqpoint{1.392686in}{1.393425in}}%
\pgfpathlineto{\pgfqpoint{1.406091in}{1.382604in}}%
\pgfpathlineto{\pgfqpoint{1.409467in}{1.379814in}}%
\pgfpathlineto{\pgfqpoint{1.421748in}{1.369603in}}%
\pgfpathlineto{\pgfqpoint{1.425760in}{1.366203in}}%
\pgfpathlineto{\pgfqpoint{1.437404in}{1.356229in}}%
\pgfpathlineto{\pgfqpoint{1.441588in}{1.352592in}}%
\pgfpathlineto{\pgfqpoint{1.453061in}{1.342469in}}%
\pgfpathlineto{\pgfqpoint{1.456972in}{1.338981in}}%
\pgfpathlineto{\pgfqpoint{1.468718in}{1.328305in}}%
\pgfpathlineto{\pgfqpoint{1.471926in}{1.325370in}}%
\pgfpathlineto{\pgfqpoint{1.484374in}{1.313716in}}%
\pgfpathlineto{\pgfqpoint{1.486461in}{1.311759in}}%
\pgfpathlineto{\pgfqpoint{1.500031in}{1.298675in}}%
\pgfpathlineto{\pgfqpoint{1.500580in}{1.298148in}}%
\pgfpathlineto{\pgfqpoint{1.514311in}{1.284536in}}%
\pgfpathlineto{\pgfqpoint{1.515687in}{1.283122in}}%
\pgfpathlineto{\pgfqpoint{1.527654in}{1.270925in}}%
\pgfpathlineto{\pgfqpoint{1.531344in}{1.267009in}}%
\pgfpathlineto{\pgfqpoint{1.540594in}{1.257314in}}%
\pgfpathlineto{\pgfqpoint{1.547000in}{1.250287in}}%
\pgfpathlineto{\pgfqpoint{1.553113in}{1.243703in}}%
\pgfpathlineto{\pgfqpoint{1.562657in}{1.232890in}}%
\pgfpathlineto{\pgfqpoint{1.565186in}{1.230092in}}%
\pgfpathlineto{\pgfqpoint{1.576835in}{1.216481in}}%
\pgfpathlineto{\pgfqpoint{1.578314in}{1.214647in}}%
\pgfpathlineto{\pgfqpoint{1.588100in}{1.202870in}}%
\pgfpathlineto{\pgfqpoint{1.593970in}{1.195306in}}%
\pgfpathlineto{\pgfqpoint{1.598838in}{1.189259in}}%
\pgfpathlineto{\pgfqpoint{1.609006in}{1.175647in}}%
\pgfpathlineto{\pgfqpoint{1.609627in}{1.174747in}}%
\pgfpathlineto{\pgfqpoint{1.618789in}{1.162036in}}%
\pgfpathlineto{\pgfqpoint{1.625283in}{1.152149in}}%
\pgfpathlineto{\pgfqpoint{1.627859in}{1.148425in}}%
\pgfpathlineto{\pgfqpoint{1.636373in}{1.134814in}}%
\pgfpathlineto{\pgfqpoint{1.640940in}{1.126617in}}%
\pgfpathlineto{\pgfqpoint{1.644146in}{1.121203in}}%
\pgfpathlineto{\pgfqpoint{1.651219in}{1.107592in}}%
\pgfpathlineto{\pgfqpoint{1.656596in}{1.095520in}}%
\pgfpathlineto{\pgfqpoint{1.657333in}{1.093981in}}%
\pgfpathlineto{\pgfqpoint{1.662779in}{1.080370in}}%
\pgfpathlineto{\pgfqpoint{1.667133in}{1.066759in}}%
\pgfpathlineto{\pgfqpoint{1.670398in}{1.053148in}}%
\pgfpathlineto{\pgfqpoint{1.672253in}{1.041549in}}%
\pgfpathlineto{\pgfqpoint{1.672603in}{1.039536in}}%
\pgfpathlineto{\pgfqpoint{1.673790in}{1.025925in}}%
\pgfpathlineto{\pgfqpoint{1.673790in}{1.012314in}}%
\pgfpathlineto{\pgfqpoint{1.672603in}{0.998703in}}%
\pgfpathlineto{\pgfqpoint{1.672253in}{0.996690in}}%
\pgfpathlineto{\pgfqpoint{1.670398in}{0.985092in}}%
\pgfpathlineto{\pgfqpoint{1.667133in}{0.971481in}}%
\pgfpathlineto{\pgfqpoint{1.662779in}{0.957870in}}%
\pgfpathlineto{\pgfqpoint{1.657333in}{0.944259in}}%
\pgfpathlineto{\pgfqpoint{1.656596in}{0.942719in}}%
\pgfpathlineto{\pgfqpoint{1.651219in}{0.930648in}}%
\pgfpathlineto{\pgfqpoint{1.644146in}{0.917036in}}%
\pgfpathlineto{\pgfqpoint{1.640940in}{0.911623in}}%
\pgfpathlineto{\pgfqpoint{1.636373in}{0.903425in}}%
\pgfpathlineto{\pgfqpoint{1.627859in}{0.889814in}}%
\pgfpathlineto{\pgfqpoint{1.625283in}{0.886090in}}%
\pgfpathlineto{\pgfqpoint{1.618789in}{0.876203in}}%
\pgfpathlineto{\pgfqpoint{1.609627in}{0.863492in}}%
\pgfpathlineto{\pgfqpoint{1.609006in}{0.862592in}}%
\pgfpathlineto{\pgfqpoint{1.598838in}{0.848981in}}%
\pgfpathlineto{\pgfqpoint{1.593970in}{0.842933in}}%
\pgfpathlineto{\pgfqpoint{1.588100in}{0.835370in}}%
\pgfpathlineto{\pgfqpoint{1.578314in}{0.823592in}}%
\pgfpathlineto{\pgfqpoint{1.576835in}{0.821759in}}%
\pgfpathlineto{\pgfqpoint{1.565186in}{0.808148in}}%
\pgfpathlineto{\pgfqpoint{1.562657in}{0.805349in}}%
\pgfpathlineto{\pgfqpoint{1.553113in}{0.794536in}}%
\pgfpathlineto{\pgfqpoint{1.547000in}{0.787952in}}%
\pgfpathlineto{\pgfqpoint{1.540594in}{0.780925in}}%
\pgfpathlineto{\pgfqpoint{1.531344in}{0.771231in}}%
\pgfpathlineto{\pgfqpoint{1.527654in}{0.767314in}}%
\pgfpathlineto{\pgfqpoint{1.515687in}{0.755117in}}%
\pgfpathlineto{\pgfqpoint{1.514311in}{0.753703in}}%
\pgfpathlineto{\pgfqpoint{1.500580in}{0.740092in}}%
\pgfpathlineto{\pgfqpoint{1.500031in}{0.739564in}}%
\pgfpathlineto{\pgfqpoint{1.486461in}{0.726481in}}%
\pgfpathlineto{\pgfqpoint{1.484374in}{0.724524in}}%
\pgfpathlineto{\pgfqpoint{1.471926in}{0.712870in}}%
\pgfpathlineto{\pgfqpoint{1.468718in}{0.709935in}}%
\pgfpathlineto{\pgfqpoint{1.456972in}{0.699259in}}%
\pgfpathlineto{\pgfqpoint{1.453061in}{0.695771in}}%
\pgfpathlineto{\pgfqpoint{1.441588in}{0.685648in}}%
\pgfpathlineto{\pgfqpoint{1.437404in}{0.682010in}}%
\pgfpathlineto{\pgfqpoint{1.425760in}{0.672036in}}%
\pgfpathlineto{\pgfqpoint{1.421748in}{0.668636in}}%
\pgfpathlineto{\pgfqpoint{1.409467in}{0.658425in}}%
\pgfpathlineto{\pgfqpoint{1.406091in}{0.655636in}}%
\pgfpathlineto{\pgfqpoint{1.392686in}{0.644814in}}%
\pgfpathlineto{\pgfqpoint{1.390435in}{0.643000in}}%
\pgfpathlineto{\pgfqpoint{1.375385in}{0.631203in}}%
\pgfpathlineto{\pgfqpoint{1.374778in}{0.630726in}}%
\pgfpathlineto{\pgfqpoint{1.359122in}{0.618789in}}%
\pgfpathlineto{\pgfqpoint{1.357495in}{0.617592in}}%
\pgfpathlineto{\pgfqpoint{1.343465in}{0.607188in}}%
\pgfpathlineto{\pgfqpoint{1.338960in}{0.603981in}}%
\pgfpathlineto{\pgfqpoint{1.327809in}{0.595939in}}%
\pgfpathlineto{\pgfqpoint{1.319726in}{0.590370in}}%
\pgfpathlineto{\pgfqpoint{1.312152in}{0.585056in}}%
\pgfpathlineto{\pgfqpoint{1.299714in}{0.576759in}}%
\pgfpathlineto{\pgfqpoint{1.296495in}{0.574560in}}%
\pgfpathlineto{\pgfqpoint{1.280839in}{0.564433in}}%
\pgfpathlineto{\pgfqpoint{1.278730in}{0.563148in}}%
\pgfpathlineto{\pgfqpoint{1.265182in}{0.554640in}}%
\pgfpathlineto{\pgfqpoint{1.256482in}{0.549536in}}%
\pgfpathlineto{\pgfqpoint{1.249526in}{0.545304in}}%
\pgfpathlineto{\pgfqpoint{1.233869in}{0.536465in}}%
\pgfpathlineto{\pgfqpoint{1.232834in}{0.535925in}}%
\pgfpathlineto{\pgfqpoint{1.218213in}{0.527960in}}%
\pgfpathlineto{\pgfqpoint{1.206839in}{0.522314in}}%
\pgfpathlineto{\pgfqpoint{1.202556in}{0.520075in}}%
\pgfpathlineto{\pgfqpoint{1.186899in}{0.512673in}}%
\pgfpathlineto{\pgfqpoint{1.177470in}{0.508703in}}%
\pgfpathlineto{\pgfqpoint{1.171243in}{0.505916in}}%
\pgfpathlineto{\pgfqpoint{1.155586in}{0.499767in}}%
\pgfpathlineto{\pgfqpoint{1.141701in}{0.495092in}}%
\pgfpathlineto{\pgfqpoint{1.139930in}{0.494452in}}%
\pgfpathlineto{\pgfqpoint{1.124273in}{0.489717in}}%
\pgfpathlineto{\pgfqpoint{1.108617in}{0.485931in}}%
\pgfpathlineto{\pgfqpoint{1.092960in}{0.483093in}}%
\pgfpathlineto{\pgfqpoint{1.079619in}{0.481481in}}%
\pgfpathlineto{\pgfqpoint{1.077303in}{0.481176in}}%
\pgfpathlineto{\pgfqpoint{1.061647in}{0.480145in}}%
\pgfpathlineto{\pgfqpoint{1.045990in}{0.480145in}}%
\pgfpathlineto{\pgfqpoint{1.030334in}{0.481176in}}%
\pgfpathlineto{\pgfqpoint{1.028018in}{0.481481in}}%
\pgfpathclose%
\pgfusepath{fill}%
\end{pgfscope}%
\begin{pgfscope}%
\pgfpathrectangle{\pgfqpoint{0.278819in}{0.345370in}}{\pgfqpoint{1.550000in}{1.347500in}}%
\pgfusepath{clip}%
\pgfsetbuttcap%
\pgfsetroundjoin%
\definecolor{currentfill}{rgb}{0.252220,0.059415,0.453248}%
\pgfsetfillcolor{currentfill}%
\pgfsetlinewidth{0.000000pt}%
\definecolor{currentstroke}{rgb}{0.000000,0.000000,0.000000}%
\pgfsetstrokecolor{currentstroke}%
\pgfsetdash{}{0pt}%
\pgfpathmoveto{\pgfqpoint{0.591950in}{0.345370in}}%
\pgfpathlineto{\pgfqpoint{0.607606in}{0.345370in}}%
\pgfpathlineto{\pgfqpoint{0.623263in}{0.345370in}}%
\pgfpathlineto{\pgfqpoint{0.638920in}{0.345370in}}%
\pgfpathlineto{\pgfqpoint{0.654576in}{0.345370in}}%
\pgfpathlineto{\pgfqpoint{0.670233in}{0.345370in}}%
\pgfpathlineto{\pgfqpoint{0.685889in}{0.345370in}}%
\pgfpathlineto{\pgfqpoint{0.701546in}{0.345370in}}%
\pgfpathlineto{\pgfqpoint{0.717202in}{0.345370in}}%
\pgfpathlineto{\pgfqpoint{0.732859in}{0.345370in}}%
\pgfpathlineto{\pgfqpoint{0.748516in}{0.345370in}}%
\pgfpathlineto{\pgfqpoint{0.764172in}{0.345370in}}%
\pgfpathlineto{\pgfqpoint{0.779829in}{0.345370in}}%
\pgfpathlineto{\pgfqpoint{0.788490in}{0.345370in}}%
\pgfpathlineto{\pgfqpoint{0.787696in}{0.358981in}}%
\pgfpathlineto{\pgfqpoint{0.785320in}{0.372592in}}%
\pgfpathlineto{\pgfqpoint{0.781388in}{0.386203in}}%
\pgfpathlineto{\pgfqpoint{0.779829in}{0.390091in}}%
\pgfpathlineto{\pgfqpoint{0.776063in}{0.399814in}}%
\pgfpathlineto{\pgfqpoint{0.769373in}{0.413425in}}%
\pgfpathlineto{\pgfqpoint{0.764172in}{0.422220in}}%
\pgfpathlineto{\pgfqpoint{0.761415in}{0.427036in}}%
\pgfpathlineto{\pgfqpoint{0.752362in}{0.440648in}}%
\pgfpathlineto{\pgfqpoint{0.748516in}{0.445759in}}%
\pgfpathlineto{\pgfqpoint{0.742306in}{0.454259in}}%
\pgfpathlineto{\pgfqpoint{0.732859in}{0.465940in}}%
\pgfpathlineto{\pgfqpoint{0.731339in}{0.467870in}}%
\pgfpathlineto{\pgfqpoint{0.719628in}{0.481481in}}%
\pgfpathlineto{\pgfqpoint{0.717202in}{0.484098in}}%
\pgfpathlineto{\pgfqpoint{0.707247in}{0.495092in}}%
\pgfpathlineto{\pgfqpoint{0.701546in}{0.501007in}}%
\pgfpathlineto{\pgfqpoint{0.694275in}{0.508703in}}%
\pgfpathlineto{\pgfqpoint{0.685889in}{0.517120in}}%
\pgfpathlineto{\pgfqpoint{0.680799in}{0.522314in}}%
\pgfpathlineto{\pgfqpoint{0.670233in}{0.532624in}}%
\pgfpathlineto{\pgfqpoint{0.666893in}{0.535925in}}%
\pgfpathlineto{\pgfqpoint{0.654576in}{0.547652in}}%
\pgfpathlineto{\pgfqpoint{0.652616in}{0.549536in}}%
\pgfpathlineto{\pgfqpoint{0.638920in}{0.562299in}}%
\pgfpathlineto{\pgfqpoint{0.638014in}{0.563148in}}%
\pgfpathlineto{\pgfqpoint{0.623263in}{0.576631in}}%
\pgfpathlineto{\pgfqpoint{0.623124in}{0.576759in}}%
\pgfpathlineto{\pgfqpoint{0.607963in}{0.590370in}}%
\pgfpathlineto{\pgfqpoint{0.607606in}{0.590685in}}%
\pgfpathlineto{\pgfqpoint{0.592552in}{0.603981in}}%
\pgfpathlineto{\pgfqpoint{0.591950in}{0.604507in}}%
\pgfpathlineto{\pgfqpoint{0.576899in}{0.617592in}}%
\pgfpathlineto{\pgfqpoint{0.576293in}{0.618115in}}%
\pgfpathlineto{\pgfqpoint{0.560999in}{0.631203in}}%
\pgfpathlineto{\pgfqpoint{0.560637in}{0.631513in}}%
\pgfpathlineto{\pgfqpoint{0.544980in}{0.644693in}}%
\pgfpathlineto{\pgfqpoint{0.544833in}{0.644814in}}%
\pgfpathlineto{\pgfqpoint{0.529324in}{0.657638in}}%
\pgfpathlineto{\pgfqpoint{0.528348in}{0.658425in}}%
\pgfpathlineto{\pgfqpoint{0.513667in}{0.670333in}}%
\pgfpathlineto{\pgfqpoint{0.511500in}{0.672036in}}%
\pgfpathlineto{\pgfqpoint{0.498011in}{0.682744in}}%
\pgfpathlineto{\pgfqpoint{0.494213in}{0.685648in}}%
\pgfpathlineto{\pgfqpoint{0.482354in}{0.694834in}}%
\pgfpathlineto{\pgfqpoint{0.476379in}{0.699259in}}%
\pgfpathlineto{\pgfqpoint{0.466697in}{0.706548in}}%
\pgfpathlineto{\pgfqpoint{0.457844in}{0.712870in}}%
\pgfpathlineto{\pgfqpoint{0.451041in}{0.717826in}}%
\pgfpathlineto{\pgfqpoint{0.438394in}{0.726481in}}%
\pgfpathlineto{\pgfqpoint{0.435384in}{0.728590in}}%
\pgfpathlineto{\pgfqpoint{0.419728in}{0.738770in}}%
\pgfpathlineto{\pgfqpoint{0.417508in}{0.740092in}}%
\pgfpathlineto{\pgfqpoint{0.404071in}{0.748305in}}%
\pgfpathlineto{\pgfqpoint{0.394294in}{0.753703in}}%
\pgfpathlineto{\pgfqpoint{0.388415in}{0.757047in}}%
\pgfpathlineto{\pgfqpoint{0.372758in}{0.764917in}}%
\pgfpathlineto{\pgfqpoint{0.367218in}{0.767314in}}%
\pgfpathlineto{\pgfqpoint{0.357101in}{0.771836in}}%
\pgfpathlineto{\pgfqpoint{0.341445in}{0.777651in}}%
\pgfpathlineto{\pgfqpoint{0.330260in}{0.780925in}}%
\pgfpathlineto{\pgfqpoint{0.325788in}{0.782281in}}%
\pgfpathlineto{\pgfqpoint{0.310132in}{0.785699in}}%
\pgfpathlineto{\pgfqpoint{0.294475in}{0.787764in}}%
\pgfpathlineto{\pgfqpoint{0.278819in}{0.788455in}}%
\pgfpathlineto{\pgfqpoint{0.278819in}{0.780925in}}%
\pgfpathlineto{\pgfqpoint{0.278819in}{0.767314in}}%
\pgfpathlineto{\pgfqpoint{0.278819in}{0.753703in}}%
\pgfpathlineto{\pgfqpoint{0.278819in}{0.740092in}}%
\pgfpathlineto{\pgfqpoint{0.278819in}{0.726481in}}%
\pgfpathlineto{\pgfqpoint{0.278819in}{0.712870in}}%
\pgfpathlineto{\pgfqpoint{0.278819in}{0.699259in}}%
\pgfpathlineto{\pgfqpoint{0.278819in}{0.685648in}}%
\pgfpathlineto{\pgfqpoint{0.278819in}{0.672036in}}%
\pgfpathlineto{\pgfqpoint{0.278819in}{0.658425in}}%
\pgfpathlineto{\pgfqpoint{0.278819in}{0.644814in}}%
\pgfpathlineto{\pgfqpoint{0.278819in}{0.631203in}}%
\pgfpathlineto{\pgfqpoint{0.278819in}{0.617592in}}%
\pgfpathlineto{\pgfqpoint{0.278819in}{0.605370in}}%
\pgfpathlineto{\pgfqpoint{0.294475in}{0.604830in}}%
\pgfpathlineto{\pgfqpoint{0.302717in}{0.603981in}}%
\pgfpathlineto{\pgfqpoint{0.310132in}{0.603199in}}%
\pgfpathlineto{\pgfqpoint{0.325788in}{0.600464in}}%
\pgfpathlineto{\pgfqpoint{0.341445in}{0.596676in}}%
\pgfpathlineto{\pgfqpoint{0.357101in}{0.591875in}}%
\pgfpathlineto{\pgfqpoint{0.361173in}{0.590370in}}%
\pgfpathlineto{\pgfqpoint{0.372758in}{0.585985in}}%
\pgfpathlineto{\pgfqpoint{0.388415in}{0.579117in}}%
\pgfpathlineto{\pgfqpoint{0.393163in}{0.576759in}}%
\pgfpathlineto{\pgfqpoint{0.404071in}{0.571197in}}%
\pgfpathlineto{\pgfqpoint{0.418346in}{0.563148in}}%
\pgfpathlineto{\pgfqpoint{0.419728in}{0.562344in}}%
\pgfpathlineto{\pgfqpoint{0.435384in}{0.552398in}}%
\pgfpathlineto{\pgfqpoint{0.439569in}{0.549536in}}%
\pgfpathlineto{\pgfqpoint{0.451041in}{0.541411in}}%
\pgfpathlineto{\pgfqpoint{0.458313in}{0.535925in}}%
\pgfpathlineto{\pgfqpoint{0.466697in}{0.529339in}}%
\pgfpathlineto{\pgfqpoint{0.475167in}{0.522314in}}%
\pgfpathlineto{\pgfqpoint{0.482354in}{0.516066in}}%
\pgfpathlineto{\pgfqpoint{0.490434in}{0.508703in}}%
\pgfpathlineto{\pgfqpoint{0.498011in}{0.501414in}}%
\pgfpathlineto{\pgfqpoint{0.504320in}{0.495092in}}%
\pgfpathlineto{\pgfqpoint{0.513667in}{0.485118in}}%
\pgfpathlineto{\pgfqpoint{0.516959in}{0.481481in}}%
\pgfpathlineto{\pgfqpoint{0.528400in}{0.467870in}}%
\pgfpathlineto{\pgfqpoint{0.529324in}{0.466669in}}%
\pgfpathlineto{\pgfqpoint{0.538583in}{0.454259in}}%
\pgfpathlineto{\pgfqpoint{0.544980in}{0.444775in}}%
\pgfpathlineto{\pgfqpoint{0.547693in}{0.440648in}}%
\pgfpathlineto{\pgfqpoint{0.555593in}{0.427036in}}%
\pgfpathlineto{\pgfqpoint{0.560637in}{0.416965in}}%
\pgfpathlineto{\pgfqpoint{0.562369in}{0.413425in}}%
\pgfpathlineto{\pgfqpoint{0.567891in}{0.399814in}}%
\pgfpathlineto{\pgfqpoint{0.572248in}{0.386203in}}%
\pgfpathlineto{\pgfqpoint{0.575394in}{0.372592in}}%
\pgfpathlineto{\pgfqpoint{0.576293in}{0.366146in}}%
\pgfpathlineto{\pgfqpoint{0.577271in}{0.358981in}}%
\pgfpathlineto{\pgfqpoint{0.577891in}{0.345370in}}%
\pgfpathlineto{\pgfqpoint{0.591950in}{0.345370in}}%
\pgfpathclose%
\pgfpathmoveto{\pgfqpoint{1.327809in}{0.345370in}}%
\pgfpathlineto{\pgfqpoint{1.343465in}{0.345370in}}%
\pgfpathlineto{\pgfqpoint{1.359122in}{0.345370in}}%
\pgfpathlineto{\pgfqpoint{1.374778in}{0.345370in}}%
\pgfpathlineto{\pgfqpoint{1.390435in}{0.345370in}}%
\pgfpathlineto{\pgfqpoint{1.406091in}{0.345370in}}%
\pgfpathlineto{\pgfqpoint{1.421748in}{0.345370in}}%
\pgfpathlineto{\pgfqpoint{1.437404in}{0.345370in}}%
\pgfpathlineto{\pgfqpoint{1.453061in}{0.345370in}}%
\pgfpathlineto{\pgfqpoint{1.468718in}{0.345370in}}%
\pgfpathlineto{\pgfqpoint{1.484374in}{0.345370in}}%
\pgfpathlineto{\pgfqpoint{1.500031in}{0.345370in}}%
\pgfpathlineto{\pgfqpoint{1.515687in}{0.345370in}}%
\pgfpathlineto{\pgfqpoint{1.529746in}{0.345370in}}%
\pgfpathlineto{\pgfqpoint{1.530367in}{0.358981in}}%
\pgfpathlineto{\pgfqpoint{1.531344in}{0.366146in}}%
\pgfpathlineto{\pgfqpoint{1.532243in}{0.372592in}}%
\pgfpathlineto{\pgfqpoint{1.535389in}{0.386203in}}%
\pgfpathlineto{\pgfqpoint{1.539746in}{0.399814in}}%
\pgfpathlineto{\pgfqpoint{1.545268in}{0.413425in}}%
\pgfpathlineto{\pgfqpoint{1.547000in}{0.416965in}}%
\pgfpathlineto{\pgfqpoint{1.552045in}{0.427036in}}%
\pgfpathlineto{\pgfqpoint{1.559944in}{0.440648in}}%
\pgfpathlineto{\pgfqpoint{1.562657in}{0.444775in}}%
\pgfpathlineto{\pgfqpoint{1.569055in}{0.454259in}}%
\pgfpathlineto{\pgfqpoint{1.578314in}{0.466669in}}%
\pgfpathlineto{\pgfqpoint{1.579237in}{0.467870in}}%
\pgfpathlineto{\pgfqpoint{1.590678in}{0.481481in}}%
\pgfpathlineto{\pgfqpoint{1.593970in}{0.485118in}}%
\pgfpathlineto{\pgfqpoint{1.603317in}{0.495092in}}%
\pgfpathlineto{\pgfqpoint{1.609627in}{0.501414in}}%
\pgfpathlineto{\pgfqpoint{1.617203in}{0.508703in}}%
\pgfpathlineto{\pgfqpoint{1.625283in}{0.516066in}}%
\pgfpathlineto{\pgfqpoint{1.632470in}{0.522314in}}%
\pgfpathlineto{\pgfqpoint{1.640940in}{0.529339in}}%
\pgfpathlineto{\pgfqpoint{1.649324in}{0.535925in}}%
\pgfpathlineto{\pgfqpoint{1.656596in}{0.541411in}}%
\pgfpathlineto{\pgfqpoint{1.668069in}{0.549536in}}%
\pgfpathlineto{\pgfqpoint{1.672253in}{0.552398in}}%
\pgfpathlineto{\pgfqpoint{1.687910in}{0.562344in}}%
\pgfpathlineto{\pgfqpoint{1.689291in}{0.563148in}}%
\pgfpathlineto{\pgfqpoint{1.703566in}{0.571197in}}%
\pgfpathlineto{\pgfqpoint{1.714474in}{0.576759in}}%
\pgfpathlineto{\pgfqpoint{1.719223in}{0.579117in}}%
\pgfpathlineto{\pgfqpoint{1.734879in}{0.585985in}}%
\pgfpathlineto{\pgfqpoint{1.746464in}{0.590370in}}%
\pgfpathlineto{\pgfqpoint{1.750536in}{0.591875in}}%
\pgfpathlineto{\pgfqpoint{1.766192in}{0.596676in}}%
\pgfpathlineto{\pgfqpoint{1.781849in}{0.600464in}}%
\pgfpathlineto{\pgfqpoint{1.797505in}{0.603199in}}%
\pgfpathlineto{\pgfqpoint{1.804921in}{0.603981in}}%
\pgfpathlineto{\pgfqpoint{1.813162in}{0.604830in}}%
\pgfpathlineto{\pgfqpoint{1.828819in}{0.605370in}}%
\pgfpathlineto{\pgfqpoint{1.828819in}{0.617592in}}%
\pgfpathlineto{\pgfqpoint{1.828819in}{0.631203in}}%
\pgfpathlineto{\pgfqpoint{1.828819in}{0.644814in}}%
\pgfpathlineto{\pgfqpoint{1.828819in}{0.658425in}}%
\pgfpathlineto{\pgfqpoint{1.828819in}{0.672036in}}%
\pgfpathlineto{\pgfqpoint{1.828819in}{0.685648in}}%
\pgfpathlineto{\pgfqpoint{1.828819in}{0.699259in}}%
\pgfpathlineto{\pgfqpoint{1.828819in}{0.712870in}}%
\pgfpathlineto{\pgfqpoint{1.828819in}{0.726481in}}%
\pgfpathlineto{\pgfqpoint{1.828819in}{0.740092in}}%
\pgfpathlineto{\pgfqpoint{1.828819in}{0.753703in}}%
\pgfpathlineto{\pgfqpoint{1.828819in}{0.767314in}}%
\pgfpathlineto{\pgfqpoint{1.828819in}{0.780925in}}%
\pgfpathlineto{\pgfqpoint{1.828819in}{0.788455in}}%
\pgfpathlineto{\pgfqpoint{1.813162in}{0.787764in}}%
\pgfpathlineto{\pgfqpoint{1.797505in}{0.785699in}}%
\pgfpathlineto{\pgfqpoint{1.781849in}{0.782281in}}%
\pgfpathlineto{\pgfqpoint{1.777377in}{0.780925in}}%
\pgfpathlineto{\pgfqpoint{1.766192in}{0.777651in}}%
\pgfpathlineto{\pgfqpoint{1.750536in}{0.771836in}}%
\pgfpathlineto{\pgfqpoint{1.740419in}{0.767314in}}%
\pgfpathlineto{\pgfqpoint{1.734879in}{0.764917in}}%
\pgfpathlineto{\pgfqpoint{1.719223in}{0.757047in}}%
\pgfpathlineto{\pgfqpoint{1.713343in}{0.753703in}}%
\pgfpathlineto{\pgfqpoint{1.703566in}{0.748305in}}%
\pgfpathlineto{\pgfqpoint{1.690129in}{0.740092in}}%
\pgfpathlineto{\pgfqpoint{1.687910in}{0.738770in}}%
\pgfpathlineto{\pgfqpoint{1.672253in}{0.728590in}}%
\pgfpathlineto{\pgfqpoint{1.669243in}{0.726481in}}%
\pgfpathlineto{\pgfqpoint{1.656596in}{0.717826in}}%
\pgfpathlineto{\pgfqpoint{1.649793in}{0.712870in}}%
\pgfpathlineto{\pgfqpoint{1.640940in}{0.706548in}}%
\pgfpathlineto{\pgfqpoint{1.631259in}{0.699259in}}%
\pgfpathlineto{\pgfqpoint{1.625283in}{0.694834in}}%
\pgfpathlineto{\pgfqpoint{1.613424in}{0.685648in}}%
\pgfpathlineto{\pgfqpoint{1.609627in}{0.682744in}}%
\pgfpathlineto{\pgfqpoint{1.596137in}{0.672036in}}%
\pgfpathlineto{\pgfqpoint{1.593970in}{0.670333in}}%
\pgfpathlineto{\pgfqpoint{1.579290in}{0.658425in}}%
\pgfpathlineto{\pgfqpoint{1.578314in}{0.657638in}}%
\pgfpathlineto{\pgfqpoint{1.562804in}{0.644814in}}%
\pgfpathlineto{\pgfqpoint{1.562657in}{0.644693in}}%
\pgfpathlineto{\pgfqpoint{1.547000in}{0.631513in}}%
\pgfpathlineto{\pgfqpoint{1.546638in}{0.631203in}}%
\pgfpathlineto{\pgfqpoint{1.531344in}{0.618115in}}%
\pgfpathlineto{\pgfqpoint{1.530739in}{0.617592in}}%
\pgfpathlineto{\pgfqpoint{1.515687in}{0.604507in}}%
\pgfpathlineto{\pgfqpoint{1.515086in}{0.603981in}}%
\pgfpathlineto{\pgfqpoint{1.500031in}{0.590685in}}%
\pgfpathlineto{\pgfqpoint{1.499674in}{0.590370in}}%
\pgfpathlineto{\pgfqpoint{1.484514in}{0.576759in}}%
\pgfpathlineto{\pgfqpoint{1.484374in}{0.576631in}}%
\pgfpathlineto{\pgfqpoint{1.469623in}{0.563148in}}%
\pgfpathlineto{\pgfqpoint{1.468718in}{0.562299in}}%
\pgfpathlineto{\pgfqpoint{1.455021in}{0.549536in}}%
\pgfpathlineto{\pgfqpoint{1.453061in}{0.547652in}}%
\pgfpathlineto{\pgfqpoint{1.440744in}{0.535925in}}%
\pgfpathlineto{\pgfqpoint{1.437404in}{0.532624in}}%
\pgfpathlineto{\pgfqpoint{1.426838in}{0.522314in}}%
\pgfpathlineto{\pgfqpoint{1.421748in}{0.517120in}}%
\pgfpathlineto{\pgfqpoint{1.413363in}{0.508703in}}%
\pgfpathlineto{\pgfqpoint{1.406091in}{0.501007in}}%
\pgfpathlineto{\pgfqpoint{1.400390in}{0.495092in}}%
\pgfpathlineto{\pgfqpoint{1.390435in}{0.484098in}}%
\pgfpathlineto{\pgfqpoint{1.388009in}{0.481481in}}%
\pgfpathlineto{\pgfqpoint{1.376299in}{0.467870in}}%
\pgfpathlineto{\pgfqpoint{1.374778in}{0.465940in}}%
\pgfpathlineto{\pgfqpoint{1.365331in}{0.454259in}}%
\pgfpathlineto{\pgfqpoint{1.359122in}{0.445759in}}%
\pgfpathlineto{\pgfqpoint{1.355275in}{0.440648in}}%
\pgfpathlineto{\pgfqpoint{1.346223in}{0.427036in}}%
\pgfpathlineto{\pgfqpoint{1.343465in}{0.422220in}}%
\pgfpathlineto{\pgfqpoint{1.338264in}{0.413425in}}%
\pgfpathlineto{\pgfqpoint{1.331575in}{0.399814in}}%
\pgfpathlineto{\pgfqpoint{1.327809in}{0.390091in}}%
\pgfpathlineto{\pgfqpoint{1.326249in}{0.386203in}}%
\pgfpathlineto{\pgfqpoint{1.322318in}{0.372592in}}%
\pgfpathlineto{\pgfqpoint{1.319942in}{0.358981in}}%
\pgfpathlineto{\pgfqpoint{1.319147in}{0.345370in}}%
\pgfpathlineto{\pgfqpoint{1.327809in}{0.345370in}}%
\pgfpathclose%
\pgfpathmoveto{\pgfqpoint{0.294475in}{1.250475in}}%
\pgfpathlineto{\pgfqpoint{0.310132in}{1.252541in}}%
\pgfpathlineto{\pgfqpoint{0.325788in}{1.255958in}}%
\pgfpathlineto{\pgfqpoint{0.330260in}{1.257314in}}%
\pgfpathlineto{\pgfqpoint{0.341445in}{1.260588in}}%
\pgfpathlineto{\pgfqpoint{0.357101in}{1.266404in}}%
\pgfpathlineto{\pgfqpoint{0.367218in}{1.270925in}}%
\pgfpathlineto{\pgfqpoint{0.372758in}{1.273323in}}%
\pgfpathlineto{\pgfqpoint{0.388415in}{1.281192in}}%
\pgfpathlineto{\pgfqpoint{0.394294in}{1.284536in}}%
\pgfpathlineto{\pgfqpoint{0.404071in}{1.289935in}}%
\pgfpathlineto{\pgfqpoint{0.417508in}{1.298148in}}%
\pgfpathlineto{\pgfqpoint{0.419728in}{1.299469in}}%
\pgfpathlineto{\pgfqpoint{0.435384in}{1.309650in}}%
\pgfpathlineto{\pgfqpoint{0.438394in}{1.311759in}}%
\pgfpathlineto{\pgfqpoint{0.451041in}{1.320413in}}%
\pgfpathlineto{\pgfqpoint{0.457844in}{1.325370in}}%
\pgfpathlineto{\pgfqpoint{0.466697in}{1.331691in}}%
\pgfpathlineto{\pgfqpoint{0.476379in}{1.338981in}}%
\pgfpathlineto{\pgfqpoint{0.482354in}{1.343406in}}%
\pgfpathlineto{\pgfqpoint{0.494213in}{1.352592in}}%
\pgfpathlineto{\pgfqpoint{0.498011in}{1.355495in}}%
\pgfpathlineto{\pgfqpoint{0.511500in}{1.366203in}}%
\pgfpathlineto{\pgfqpoint{0.513667in}{1.367907in}}%
\pgfpathlineto{\pgfqpoint{0.528348in}{1.379814in}}%
\pgfpathlineto{\pgfqpoint{0.529324in}{1.380601in}}%
\pgfpathlineto{\pgfqpoint{0.544833in}{1.393425in}}%
\pgfpathlineto{\pgfqpoint{0.544980in}{1.393547in}}%
\pgfpathlineto{\pgfqpoint{0.560637in}{1.406727in}}%
\pgfpathlineto{\pgfqpoint{0.560999in}{1.407036in}}%
\pgfpathlineto{\pgfqpoint{0.576293in}{1.420124in}}%
\pgfpathlineto{\pgfqpoint{0.576899in}{1.420648in}}%
\pgfpathlineto{\pgfqpoint{0.591950in}{1.433733in}}%
\pgfpathlineto{\pgfqpoint{0.592552in}{1.434259in}}%
\pgfpathlineto{\pgfqpoint{0.607606in}{1.447555in}}%
\pgfpathlineto{\pgfqpoint{0.607963in}{1.447870in}}%
\pgfpathlineto{\pgfqpoint{0.623124in}{1.461481in}}%
\pgfpathlineto{\pgfqpoint{0.623263in}{1.461609in}}%
\pgfpathlineto{\pgfqpoint{0.638014in}{1.475092in}}%
\pgfpathlineto{\pgfqpoint{0.638920in}{1.475940in}}%
\pgfpathlineto{\pgfqpoint{0.652616in}{1.488703in}}%
\pgfpathlineto{\pgfqpoint{0.654576in}{1.490587in}}%
\pgfpathlineto{\pgfqpoint{0.666893in}{1.502314in}}%
\pgfpathlineto{\pgfqpoint{0.670233in}{1.505615in}}%
\pgfpathlineto{\pgfqpoint{0.680799in}{1.515925in}}%
\pgfpathlineto{\pgfqpoint{0.685889in}{1.521120in}}%
\pgfpathlineto{\pgfqpoint{0.694275in}{1.529536in}}%
\pgfpathlineto{\pgfqpoint{0.701546in}{1.537233in}}%
\pgfpathlineto{\pgfqpoint{0.707247in}{1.543148in}}%
\pgfpathlineto{\pgfqpoint{0.717202in}{1.554142in}}%
\pgfpathlineto{\pgfqpoint{0.719628in}{1.556759in}}%
\pgfpathlineto{\pgfqpoint{0.731339in}{1.570370in}}%
\pgfpathlineto{\pgfqpoint{0.732859in}{1.572300in}}%
\pgfpathlineto{\pgfqpoint{0.742306in}{1.583981in}}%
\pgfpathlineto{\pgfqpoint{0.748516in}{1.592480in}}%
\pgfpathlineto{\pgfqpoint{0.752362in}{1.597592in}}%
\pgfpathlineto{\pgfqpoint{0.761415in}{1.611203in}}%
\pgfpathlineto{\pgfqpoint{0.764172in}{1.616020in}}%
\pgfpathlineto{\pgfqpoint{0.769373in}{1.624814in}}%
\pgfpathlineto{\pgfqpoint{0.776063in}{1.638425in}}%
\pgfpathlineto{\pgfqpoint{0.779829in}{1.648149in}}%
\pgfpathlineto{\pgfqpoint{0.781388in}{1.652036in}}%
\pgfpathlineto{\pgfqpoint{0.785320in}{1.665648in}}%
\pgfpathlineto{\pgfqpoint{0.787696in}{1.679259in}}%
\pgfpathlineto{\pgfqpoint{0.788490in}{1.692870in}}%
\pgfpathlineto{\pgfqpoint{0.779829in}{1.692870in}}%
\pgfpathlineto{\pgfqpoint{0.764172in}{1.692870in}}%
\pgfpathlineto{\pgfqpoint{0.748516in}{1.692870in}}%
\pgfpathlineto{\pgfqpoint{0.732859in}{1.692870in}}%
\pgfpathlineto{\pgfqpoint{0.717202in}{1.692870in}}%
\pgfpathlineto{\pgfqpoint{0.701546in}{1.692870in}}%
\pgfpathlineto{\pgfqpoint{0.685889in}{1.692870in}}%
\pgfpathlineto{\pgfqpoint{0.670233in}{1.692870in}}%
\pgfpathlineto{\pgfqpoint{0.654576in}{1.692870in}}%
\pgfpathlineto{\pgfqpoint{0.638920in}{1.692870in}}%
\pgfpathlineto{\pgfqpoint{0.623263in}{1.692870in}}%
\pgfpathlineto{\pgfqpoint{0.607606in}{1.692870in}}%
\pgfpathlineto{\pgfqpoint{0.591950in}{1.692870in}}%
\pgfpathlineto{\pgfqpoint{0.577891in}{1.692870in}}%
\pgfpathlineto{\pgfqpoint{0.577271in}{1.679259in}}%
\pgfpathlineto{\pgfqpoint{0.576293in}{1.672094in}}%
\pgfpathlineto{\pgfqpoint{0.575394in}{1.665648in}}%
\pgfpathlineto{\pgfqpoint{0.572248in}{1.652036in}}%
\pgfpathlineto{\pgfqpoint{0.567891in}{1.638425in}}%
\pgfpathlineto{\pgfqpoint{0.562369in}{1.624814in}}%
\pgfpathlineto{\pgfqpoint{0.560637in}{1.621275in}}%
\pgfpathlineto{\pgfqpoint{0.555593in}{1.611203in}}%
\pgfpathlineto{\pgfqpoint{0.547693in}{1.597592in}}%
\pgfpathlineto{\pgfqpoint{0.544980in}{1.593464in}}%
\pgfpathlineto{\pgfqpoint{0.538583in}{1.583981in}}%
\pgfpathlineto{\pgfqpoint{0.529324in}{1.571571in}}%
\pgfpathlineto{\pgfqpoint{0.528400in}{1.570370in}}%
\pgfpathlineto{\pgfqpoint{0.516959in}{1.556759in}}%
\pgfpathlineto{\pgfqpoint{0.513667in}{1.553121in}}%
\pgfpathlineto{\pgfqpoint{0.504320in}{1.543148in}}%
\pgfpathlineto{\pgfqpoint{0.498011in}{1.536825in}}%
\pgfpathlineto{\pgfqpoint{0.490434in}{1.529536in}}%
\pgfpathlineto{\pgfqpoint{0.482354in}{1.522173in}}%
\pgfpathlineto{\pgfqpoint{0.475167in}{1.515925in}}%
\pgfpathlineto{\pgfqpoint{0.466697in}{1.508900in}}%
\pgfpathlineto{\pgfqpoint{0.458313in}{1.502314in}}%
\pgfpathlineto{\pgfqpoint{0.451041in}{1.496829in}}%
\pgfpathlineto{\pgfqpoint{0.439569in}{1.488703in}}%
\pgfpathlineto{\pgfqpoint{0.435384in}{1.485841in}}%
\pgfpathlineto{\pgfqpoint{0.419728in}{1.475895in}}%
\pgfpathlineto{\pgfqpoint{0.418346in}{1.475092in}}%
\pgfpathlineto{\pgfqpoint{0.404071in}{1.467043in}}%
\pgfpathlineto{\pgfqpoint{0.393163in}{1.461481in}}%
\pgfpathlineto{\pgfqpoint{0.388415in}{1.459123in}}%
\pgfpathlineto{\pgfqpoint{0.372758in}{1.452255in}}%
\pgfpathlineto{\pgfqpoint{0.361173in}{1.447870in}}%
\pgfpathlineto{\pgfqpoint{0.357101in}{1.446364in}}%
\pgfpathlineto{\pgfqpoint{0.341445in}{1.441563in}}%
\pgfpathlineto{\pgfqpoint{0.325788in}{1.437776in}}%
\pgfpathlineto{\pgfqpoint{0.310132in}{1.435041in}}%
\pgfpathlineto{\pgfqpoint{0.302717in}{1.434259in}}%
\pgfpathlineto{\pgfqpoint{0.294475in}{1.433409in}}%
\pgfpathlineto{\pgfqpoint{0.278819in}{1.432870in}}%
\pgfpathlineto{\pgfqpoint{0.278819in}{1.420648in}}%
\pgfpathlineto{\pgfqpoint{0.278819in}{1.407036in}}%
\pgfpathlineto{\pgfqpoint{0.278819in}{1.393425in}}%
\pgfpathlineto{\pgfqpoint{0.278819in}{1.379814in}}%
\pgfpathlineto{\pgfqpoint{0.278819in}{1.366203in}}%
\pgfpathlineto{\pgfqpoint{0.278819in}{1.352592in}}%
\pgfpathlineto{\pgfqpoint{0.278819in}{1.338981in}}%
\pgfpathlineto{\pgfqpoint{0.278819in}{1.325370in}}%
\pgfpathlineto{\pgfqpoint{0.278819in}{1.311759in}}%
\pgfpathlineto{\pgfqpoint{0.278819in}{1.298148in}}%
\pgfpathlineto{\pgfqpoint{0.278819in}{1.284536in}}%
\pgfpathlineto{\pgfqpoint{0.278819in}{1.270925in}}%
\pgfpathlineto{\pgfqpoint{0.278819in}{1.257314in}}%
\pgfpathlineto{\pgfqpoint{0.278819in}{1.249784in}}%
\pgfpathlineto{\pgfqpoint{0.294475in}{1.250475in}}%
\pgfpathclose%
\pgfpathmoveto{\pgfqpoint{1.781849in}{1.255958in}}%
\pgfpathlineto{\pgfqpoint{1.797505in}{1.252541in}}%
\pgfpathlineto{\pgfqpoint{1.813162in}{1.250475in}}%
\pgfpathlineto{\pgfqpoint{1.828819in}{1.249784in}}%
\pgfpathlineto{\pgfqpoint{1.828819in}{1.257314in}}%
\pgfpathlineto{\pgfqpoint{1.828819in}{1.270925in}}%
\pgfpathlineto{\pgfqpoint{1.828819in}{1.284536in}}%
\pgfpathlineto{\pgfqpoint{1.828819in}{1.298148in}}%
\pgfpathlineto{\pgfqpoint{1.828819in}{1.311759in}}%
\pgfpathlineto{\pgfqpoint{1.828819in}{1.325370in}}%
\pgfpathlineto{\pgfqpoint{1.828819in}{1.338981in}}%
\pgfpathlineto{\pgfqpoint{1.828819in}{1.352592in}}%
\pgfpathlineto{\pgfqpoint{1.828819in}{1.366203in}}%
\pgfpathlineto{\pgfqpoint{1.828819in}{1.379814in}}%
\pgfpathlineto{\pgfqpoint{1.828819in}{1.393425in}}%
\pgfpathlineto{\pgfqpoint{1.828819in}{1.407036in}}%
\pgfpathlineto{\pgfqpoint{1.828819in}{1.420648in}}%
\pgfpathlineto{\pgfqpoint{1.828819in}{1.432870in}}%
\pgfpathlineto{\pgfqpoint{1.813162in}{1.433409in}}%
\pgfpathlineto{\pgfqpoint{1.804921in}{1.434259in}}%
\pgfpathlineto{\pgfqpoint{1.797505in}{1.435041in}}%
\pgfpathlineto{\pgfqpoint{1.781849in}{1.437776in}}%
\pgfpathlineto{\pgfqpoint{1.766192in}{1.441563in}}%
\pgfpathlineto{\pgfqpoint{1.750536in}{1.446364in}}%
\pgfpathlineto{\pgfqpoint{1.746464in}{1.447870in}}%
\pgfpathlineto{\pgfqpoint{1.734879in}{1.452255in}}%
\pgfpathlineto{\pgfqpoint{1.719223in}{1.459123in}}%
\pgfpathlineto{\pgfqpoint{1.714474in}{1.461481in}}%
\pgfpathlineto{\pgfqpoint{1.703566in}{1.467043in}}%
\pgfpathlineto{\pgfqpoint{1.689291in}{1.475092in}}%
\pgfpathlineto{\pgfqpoint{1.687910in}{1.475895in}}%
\pgfpathlineto{\pgfqpoint{1.672253in}{1.485841in}}%
\pgfpathlineto{\pgfqpoint{1.668069in}{1.488703in}}%
\pgfpathlineto{\pgfqpoint{1.656596in}{1.496829in}}%
\pgfpathlineto{\pgfqpoint{1.649324in}{1.502314in}}%
\pgfpathlineto{\pgfqpoint{1.640940in}{1.508900in}}%
\pgfpathlineto{\pgfqpoint{1.632470in}{1.515925in}}%
\pgfpathlineto{\pgfqpoint{1.625283in}{1.522173in}}%
\pgfpathlineto{\pgfqpoint{1.617203in}{1.529536in}}%
\pgfpathlineto{\pgfqpoint{1.609627in}{1.536825in}}%
\pgfpathlineto{\pgfqpoint{1.603317in}{1.543148in}}%
\pgfpathlineto{\pgfqpoint{1.593970in}{1.553121in}}%
\pgfpathlineto{\pgfqpoint{1.590678in}{1.556759in}}%
\pgfpathlineto{\pgfqpoint{1.579237in}{1.570370in}}%
\pgfpathlineto{\pgfqpoint{1.578314in}{1.571571in}}%
\pgfpathlineto{\pgfqpoint{1.569055in}{1.583981in}}%
\pgfpathlineto{\pgfqpoint{1.562657in}{1.593464in}}%
\pgfpathlineto{\pgfqpoint{1.559944in}{1.597592in}}%
\pgfpathlineto{\pgfqpoint{1.552045in}{1.611203in}}%
\pgfpathlineto{\pgfqpoint{1.547000in}{1.621275in}}%
\pgfpathlineto{\pgfqpoint{1.545268in}{1.624814in}}%
\pgfpathlineto{\pgfqpoint{1.539746in}{1.638425in}}%
\pgfpathlineto{\pgfqpoint{1.535389in}{1.652036in}}%
\pgfpathlineto{\pgfqpoint{1.532243in}{1.665648in}}%
\pgfpathlineto{\pgfqpoint{1.531344in}{1.672094in}}%
\pgfpathlineto{\pgfqpoint{1.530367in}{1.679259in}}%
\pgfpathlineto{\pgfqpoint{1.529746in}{1.692870in}}%
\pgfpathlineto{\pgfqpoint{1.515687in}{1.692870in}}%
\pgfpathlineto{\pgfqpoint{1.500031in}{1.692870in}}%
\pgfpathlineto{\pgfqpoint{1.484374in}{1.692870in}}%
\pgfpathlineto{\pgfqpoint{1.468718in}{1.692870in}}%
\pgfpathlineto{\pgfqpoint{1.453061in}{1.692870in}}%
\pgfpathlineto{\pgfqpoint{1.437404in}{1.692870in}}%
\pgfpathlineto{\pgfqpoint{1.421748in}{1.692870in}}%
\pgfpathlineto{\pgfqpoint{1.406091in}{1.692870in}}%
\pgfpathlineto{\pgfqpoint{1.390435in}{1.692870in}}%
\pgfpathlineto{\pgfqpoint{1.374778in}{1.692870in}}%
\pgfpathlineto{\pgfqpoint{1.359122in}{1.692870in}}%
\pgfpathlineto{\pgfqpoint{1.343465in}{1.692870in}}%
\pgfpathlineto{\pgfqpoint{1.327809in}{1.692870in}}%
\pgfpathlineto{\pgfqpoint{1.319147in}{1.692870in}}%
\pgfpathlineto{\pgfqpoint{1.319942in}{1.679259in}}%
\pgfpathlineto{\pgfqpoint{1.322318in}{1.665648in}}%
\pgfpathlineto{\pgfqpoint{1.326249in}{1.652036in}}%
\pgfpathlineto{\pgfqpoint{1.327809in}{1.648149in}}%
\pgfpathlineto{\pgfqpoint{1.331575in}{1.638425in}}%
\pgfpathlineto{\pgfqpoint{1.338264in}{1.624814in}}%
\pgfpathlineto{\pgfqpoint{1.343465in}{1.616020in}}%
\pgfpathlineto{\pgfqpoint{1.346223in}{1.611203in}}%
\pgfpathlineto{\pgfqpoint{1.355275in}{1.597592in}}%
\pgfpathlineto{\pgfqpoint{1.359122in}{1.592480in}}%
\pgfpathlineto{\pgfqpoint{1.365331in}{1.583981in}}%
\pgfpathlineto{\pgfqpoint{1.374778in}{1.572300in}}%
\pgfpathlineto{\pgfqpoint{1.376299in}{1.570370in}}%
\pgfpathlineto{\pgfqpoint{1.388009in}{1.556759in}}%
\pgfpathlineto{\pgfqpoint{1.390435in}{1.554142in}}%
\pgfpathlineto{\pgfqpoint{1.400390in}{1.543148in}}%
\pgfpathlineto{\pgfqpoint{1.406091in}{1.537233in}}%
\pgfpathlineto{\pgfqpoint{1.413363in}{1.529536in}}%
\pgfpathlineto{\pgfqpoint{1.421748in}{1.521120in}}%
\pgfpathlineto{\pgfqpoint{1.426838in}{1.515925in}}%
\pgfpathlineto{\pgfqpoint{1.437404in}{1.505615in}}%
\pgfpathlineto{\pgfqpoint{1.440744in}{1.502314in}}%
\pgfpathlineto{\pgfqpoint{1.453061in}{1.490587in}}%
\pgfpathlineto{\pgfqpoint{1.455021in}{1.488703in}}%
\pgfpathlineto{\pgfqpoint{1.468718in}{1.475940in}}%
\pgfpathlineto{\pgfqpoint{1.469623in}{1.475092in}}%
\pgfpathlineto{\pgfqpoint{1.484374in}{1.461609in}}%
\pgfpathlineto{\pgfqpoint{1.484514in}{1.461481in}}%
\pgfpathlineto{\pgfqpoint{1.499674in}{1.447870in}}%
\pgfpathlineto{\pgfqpoint{1.500031in}{1.447555in}}%
\pgfpathlineto{\pgfqpoint{1.515086in}{1.434259in}}%
\pgfpathlineto{\pgfqpoint{1.515687in}{1.433733in}}%
\pgfpathlineto{\pgfqpoint{1.530739in}{1.420648in}}%
\pgfpathlineto{\pgfqpoint{1.531344in}{1.420124in}}%
\pgfpathlineto{\pgfqpoint{1.546638in}{1.407036in}}%
\pgfpathlineto{\pgfqpoint{1.547000in}{1.406727in}}%
\pgfpathlineto{\pgfqpoint{1.562657in}{1.393547in}}%
\pgfpathlineto{\pgfqpoint{1.562804in}{1.393425in}}%
\pgfpathlineto{\pgfqpoint{1.578314in}{1.380601in}}%
\pgfpathlineto{\pgfqpoint{1.579290in}{1.379814in}}%
\pgfpathlineto{\pgfqpoint{1.593970in}{1.367907in}}%
\pgfpathlineto{\pgfqpoint{1.596137in}{1.366203in}}%
\pgfpathlineto{\pgfqpoint{1.609627in}{1.355495in}}%
\pgfpathlineto{\pgfqpoint{1.613424in}{1.352592in}}%
\pgfpathlineto{\pgfqpoint{1.625283in}{1.343406in}}%
\pgfpathlineto{\pgfqpoint{1.631259in}{1.338981in}}%
\pgfpathlineto{\pgfqpoint{1.640940in}{1.331691in}}%
\pgfpathlineto{\pgfqpoint{1.649793in}{1.325370in}}%
\pgfpathlineto{\pgfqpoint{1.656596in}{1.320413in}}%
\pgfpathlineto{\pgfqpoint{1.669243in}{1.311759in}}%
\pgfpathlineto{\pgfqpoint{1.672253in}{1.309650in}}%
\pgfpathlineto{\pgfqpoint{1.687910in}{1.299469in}}%
\pgfpathlineto{\pgfqpoint{1.690129in}{1.298148in}}%
\pgfpathlineto{\pgfqpoint{1.703566in}{1.289935in}}%
\pgfpathlineto{\pgfqpoint{1.713343in}{1.284536in}}%
\pgfpathlineto{\pgfqpoint{1.719223in}{1.281192in}}%
\pgfpathlineto{\pgfqpoint{1.734879in}{1.273323in}}%
\pgfpathlineto{\pgfqpoint{1.740419in}{1.270925in}}%
\pgfpathlineto{\pgfqpoint{1.750536in}{1.266404in}}%
\pgfpathlineto{\pgfqpoint{1.766192in}{1.260588in}}%
\pgfpathlineto{\pgfqpoint{1.777377in}{1.257314in}}%
\pgfpathlineto{\pgfqpoint{1.781849in}{1.255958in}}%
\pgfpathclose%
\pgfusepath{fill}%
\end{pgfscope}%
\begin{pgfscope}%
\pgfpathrectangle{\pgfqpoint{0.278819in}{0.345370in}}{\pgfqpoint{1.550000in}{1.347500in}}%
\pgfusepath{clip}%
\pgfsetbuttcap%
\pgfsetroundjoin%
\definecolor{currentfill}{rgb}{0.048062,0.036607,0.150327}%
\pgfsetfillcolor{currentfill}%
\pgfsetlinewidth{0.000000pt}%
\definecolor{currentstroke}{rgb}{0.000000,0.000000,0.000000}%
\pgfsetstrokecolor{currentstroke}%
\pgfsetdash{}{0pt}%
\pgfpathmoveto{\pgfqpoint{0.294475in}{0.345370in}}%
\pgfpathlineto{\pgfqpoint{0.310132in}{0.345370in}}%
\pgfpathlineto{\pgfqpoint{0.325788in}{0.345370in}}%
\pgfpathlineto{\pgfqpoint{0.341445in}{0.345370in}}%
\pgfpathlineto{\pgfqpoint{0.357101in}{0.345370in}}%
\pgfpathlineto{\pgfqpoint{0.372758in}{0.345370in}}%
\pgfpathlineto{\pgfqpoint{0.388415in}{0.345370in}}%
\pgfpathlineto{\pgfqpoint{0.404071in}{0.345370in}}%
\pgfpathlineto{\pgfqpoint{0.419728in}{0.345370in}}%
\pgfpathlineto{\pgfqpoint{0.435384in}{0.345370in}}%
\pgfpathlineto{\pgfqpoint{0.451041in}{0.345370in}}%
\pgfpathlineto{\pgfqpoint{0.466697in}{0.345370in}}%
\pgfpathlineto{\pgfqpoint{0.482354in}{0.345370in}}%
\pgfpathlineto{\pgfqpoint{0.498011in}{0.345370in}}%
\pgfpathlineto{\pgfqpoint{0.513667in}{0.345370in}}%
\pgfpathlineto{\pgfqpoint{0.529324in}{0.345370in}}%
\pgfpathlineto{\pgfqpoint{0.544980in}{0.345370in}}%
\pgfpathlineto{\pgfqpoint{0.560637in}{0.345370in}}%
\pgfpathlineto{\pgfqpoint{0.576293in}{0.345370in}}%
\pgfpathlineto{\pgfqpoint{0.577891in}{0.345370in}}%
\pgfpathlineto{\pgfqpoint{0.577271in}{0.358981in}}%
\pgfpathlineto{\pgfqpoint{0.576293in}{0.366146in}}%
\pgfpathlineto{\pgfqpoint{0.575394in}{0.372592in}}%
\pgfpathlineto{\pgfqpoint{0.572248in}{0.386203in}}%
\pgfpathlineto{\pgfqpoint{0.567891in}{0.399814in}}%
\pgfpathlineto{\pgfqpoint{0.562369in}{0.413425in}}%
\pgfpathlineto{\pgfqpoint{0.560637in}{0.416965in}}%
\pgfpathlineto{\pgfqpoint{0.555593in}{0.427036in}}%
\pgfpathlineto{\pgfqpoint{0.547693in}{0.440648in}}%
\pgfpathlineto{\pgfqpoint{0.544980in}{0.444775in}}%
\pgfpathlineto{\pgfqpoint{0.538583in}{0.454259in}}%
\pgfpathlineto{\pgfqpoint{0.529324in}{0.466669in}}%
\pgfpathlineto{\pgfqpoint{0.528400in}{0.467870in}}%
\pgfpathlineto{\pgfqpoint{0.516959in}{0.481481in}}%
\pgfpathlineto{\pgfqpoint{0.513667in}{0.485118in}}%
\pgfpathlineto{\pgfqpoint{0.504320in}{0.495092in}}%
\pgfpathlineto{\pgfqpoint{0.498011in}{0.501414in}}%
\pgfpathlineto{\pgfqpoint{0.490434in}{0.508703in}}%
\pgfpathlineto{\pgfqpoint{0.482354in}{0.516066in}}%
\pgfpathlineto{\pgfqpoint{0.475167in}{0.522314in}}%
\pgfpathlineto{\pgfqpoint{0.466697in}{0.529339in}}%
\pgfpathlineto{\pgfqpoint{0.458313in}{0.535925in}}%
\pgfpathlineto{\pgfqpoint{0.451041in}{0.541411in}}%
\pgfpathlineto{\pgfqpoint{0.439569in}{0.549536in}}%
\pgfpathlineto{\pgfqpoint{0.435384in}{0.552398in}}%
\pgfpathlineto{\pgfqpoint{0.419728in}{0.562344in}}%
\pgfpathlineto{\pgfqpoint{0.418346in}{0.563148in}}%
\pgfpathlineto{\pgfqpoint{0.404071in}{0.571197in}}%
\pgfpathlineto{\pgfqpoint{0.393163in}{0.576759in}}%
\pgfpathlineto{\pgfqpoint{0.388415in}{0.579117in}}%
\pgfpathlineto{\pgfqpoint{0.372758in}{0.585985in}}%
\pgfpathlineto{\pgfqpoint{0.361173in}{0.590370in}}%
\pgfpathlineto{\pgfqpoint{0.357101in}{0.591875in}}%
\pgfpathlineto{\pgfqpoint{0.341445in}{0.596676in}}%
\pgfpathlineto{\pgfqpoint{0.325788in}{0.600464in}}%
\pgfpathlineto{\pgfqpoint{0.310132in}{0.603199in}}%
\pgfpathlineto{\pgfqpoint{0.302717in}{0.603981in}}%
\pgfpathlineto{\pgfqpoint{0.294475in}{0.604830in}}%
\pgfpathlineto{\pgfqpoint{0.278819in}{0.605370in}}%
\pgfpathlineto{\pgfqpoint{0.278819in}{0.603981in}}%
\pgfpathlineto{\pgfqpoint{0.278819in}{0.590370in}}%
\pgfpathlineto{\pgfqpoint{0.278819in}{0.576759in}}%
\pgfpathlineto{\pgfqpoint{0.278819in}{0.563148in}}%
\pgfpathlineto{\pgfqpoint{0.278819in}{0.549536in}}%
\pgfpathlineto{\pgfqpoint{0.278819in}{0.535925in}}%
\pgfpathlineto{\pgfqpoint{0.278819in}{0.522314in}}%
\pgfpathlineto{\pgfqpoint{0.278819in}{0.508703in}}%
\pgfpathlineto{\pgfqpoint{0.278819in}{0.495092in}}%
\pgfpathlineto{\pgfqpoint{0.278819in}{0.481481in}}%
\pgfpathlineto{\pgfqpoint{0.278819in}{0.467870in}}%
\pgfpathlineto{\pgfqpoint{0.278819in}{0.454259in}}%
\pgfpathlineto{\pgfqpoint{0.278819in}{0.440648in}}%
\pgfpathlineto{\pgfqpoint{0.278819in}{0.427036in}}%
\pgfpathlineto{\pgfqpoint{0.278819in}{0.413425in}}%
\pgfpathlineto{\pgfqpoint{0.278819in}{0.399814in}}%
\pgfpathlineto{\pgfqpoint{0.278819in}{0.386203in}}%
\pgfpathlineto{\pgfqpoint{0.278819in}{0.372592in}}%
\pgfpathlineto{\pgfqpoint{0.278819in}{0.358981in}}%
\pgfpathlineto{\pgfqpoint{0.278819in}{0.345370in}}%
\pgfpathlineto{\pgfqpoint{0.294475in}{0.345370in}}%
\pgfpathclose%
\pgfpathmoveto{\pgfqpoint{1.531344in}{0.345370in}}%
\pgfpathlineto{\pgfqpoint{1.547000in}{0.345370in}}%
\pgfpathlineto{\pgfqpoint{1.562657in}{0.345370in}}%
\pgfpathlineto{\pgfqpoint{1.578314in}{0.345370in}}%
\pgfpathlineto{\pgfqpoint{1.593970in}{0.345370in}}%
\pgfpathlineto{\pgfqpoint{1.609627in}{0.345370in}}%
\pgfpathlineto{\pgfqpoint{1.625283in}{0.345370in}}%
\pgfpathlineto{\pgfqpoint{1.640940in}{0.345370in}}%
\pgfpathlineto{\pgfqpoint{1.656596in}{0.345370in}}%
\pgfpathlineto{\pgfqpoint{1.672253in}{0.345370in}}%
\pgfpathlineto{\pgfqpoint{1.687910in}{0.345370in}}%
\pgfpathlineto{\pgfqpoint{1.703566in}{0.345370in}}%
\pgfpathlineto{\pgfqpoint{1.719223in}{0.345370in}}%
\pgfpathlineto{\pgfqpoint{1.734879in}{0.345370in}}%
\pgfpathlineto{\pgfqpoint{1.750536in}{0.345370in}}%
\pgfpathlineto{\pgfqpoint{1.766192in}{0.345370in}}%
\pgfpathlineto{\pgfqpoint{1.781849in}{0.345370in}}%
\pgfpathlineto{\pgfqpoint{1.797505in}{0.345370in}}%
\pgfpathlineto{\pgfqpoint{1.813162in}{0.345370in}}%
\pgfpathlineto{\pgfqpoint{1.828819in}{0.345370in}}%
\pgfpathlineto{\pgfqpoint{1.828819in}{0.358981in}}%
\pgfpathlineto{\pgfqpoint{1.828819in}{0.372592in}}%
\pgfpathlineto{\pgfqpoint{1.828819in}{0.386203in}}%
\pgfpathlineto{\pgfqpoint{1.828819in}{0.399814in}}%
\pgfpathlineto{\pgfqpoint{1.828819in}{0.413425in}}%
\pgfpathlineto{\pgfqpoint{1.828819in}{0.427036in}}%
\pgfpathlineto{\pgfqpoint{1.828819in}{0.440648in}}%
\pgfpathlineto{\pgfqpoint{1.828819in}{0.454259in}}%
\pgfpathlineto{\pgfqpoint{1.828819in}{0.467870in}}%
\pgfpathlineto{\pgfqpoint{1.828819in}{0.481481in}}%
\pgfpathlineto{\pgfqpoint{1.828819in}{0.495092in}}%
\pgfpathlineto{\pgfqpoint{1.828819in}{0.508703in}}%
\pgfpathlineto{\pgfqpoint{1.828819in}{0.522314in}}%
\pgfpathlineto{\pgfqpoint{1.828819in}{0.535925in}}%
\pgfpathlineto{\pgfqpoint{1.828819in}{0.549536in}}%
\pgfpathlineto{\pgfqpoint{1.828819in}{0.563148in}}%
\pgfpathlineto{\pgfqpoint{1.828819in}{0.576759in}}%
\pgfpathlineto{\pgfqpoint{1.828819in}{0.590370in}}%
\pgfpathlineto{\pgfqpoint{1.828819in}{0.603981in}}%
\pgfpathlineto{\pgfqpoint{1.828819in}{0.605370in}}%
\pgfpathlineto{\pgfqpoint{1.813162in}{0.604830in}}%
\pgfpathlineto{\pgfqpoint{1.804921in}{0.603981in}}%
\pgfpathlineto{\pgfqpoint{1.797505in}{0.603199in}}%
\pgfpathlineto{\pgfqpoint{1.781849in}{0.600464in}}%
\pgfpathlineto{\pgfqpoint{1.766192in}{0.596676in}}%
\pgfpathlineto{\pgfqpoint{1.750536in}{0.591875in}}%
\pgfpathlineto{\pgfqpoint{1.746464in}{0.590370in}}%
\pgfpathlineto{\pgfqpoint{1.734879in}{0.585985in}}%
\pgfpathlineto{\pgfqpoint{1.719223in}{0.579117in}}%
\pgfpathlineto{\pgfqpoint{1.714474in}{0.576759in}}%
\pgfpathlineto{\pgfqpoint{1.703566in}{0.571197in}}%
\pgfpathlineto{\pgfqpoint{1.689291in}{0.563148in}}%
\pgfpathlineto{\pgfqpoint{1.687910in}{0.562344in}}%
\pgfpathlineto{\pgfqpoint{1.672253in}{0.552398in}}%
\pgfpathlineto{\pgfqpoint{1.668069in}{0.549536in}}%
\pgfpathlineto{\pgfqpoint{1.656596in}{0.541411in}}%
\pgfpathlineto{\pgfqpoint{1.649324in}{0.535925in}}%
\pgfpathlineto{\pgfqpoint{1.640940in}{0.529339in}}%
\pgfpathlineto{\pgfqpoint{1.632470in}{0.522314in}}%
\pgfpathlineto{\pgfqpoint{1.625283in}{0.516066in}}%
\pgfpathlineto{\pgfqpoint{1.617203in}{0.508703in}}%
\pgfpathlineto{\pgfqpoint{1.609627in}{0.501414in}}%
\pgfpathlineto{\pgfqpoint{1.603317in}{0.495092in}}%
\pgfpathlineto{\pgfqpoint{1.593970in}{0.485118in}}%
\pgfpathlineto{\pgfqpoint{1.590678in}{0.481481in}}%
\pgfpathlineto{\pgfqpoint{1.579237in}{0.467870in}}%
\pgfpathlineto{\pgfqpoint{1.578314in}{0.466669in}}%
\pgfpathlineto{\pgfqpoint{1.569055in}{0.454259in}}%
\pgfpathlineto{\pgfqpoint{1.562657in}{0.444775in}}%
\pgfpathlineto{\pgfqpoint{1.559944in}{0.440648in}}%
\pgfpathlineto{\pgfqpoint{1.552045in}{0.427036in}}%
\pgfpathlineto{\pgfqpoint{1.547000in}{0.416965in}}%
\pgfpathlineto{\pgfqpoint{1.545268in}{0.413425in}}%
\pgfpathlineto{\pgfqpoint{1.539746in}{0.399814in}}%
\pgfpathlineto{\pgfqpoint{1.535389in}{0.386203in}}%
\pgfpathlineto{\pgfqpoint{1.532243in}{0.372592in}}%
\pgfpathlineto{\pgfqpoint{1.531344in}{0.366146in}}%
\pgfpathlineto{\pgfqpoint{1.530367in}{0.358981in}}%
\pgfpathlineto{\pgfqpoint{1.529746in}{0.345370in}}%
\pgfpathlineto{\pgfqpoint{1.531344in}{0.345370in}}%
\pgfpathclose%
\pgfpathmoveto{\pgfqpoint{0.294475in}{1.433409in}}%
\pgfpathlineto{\pgfqpoint{0.302717in}{1.434259in}}%
\pgfpathlineto{\pgfqpoint{0.310132in}{1.435041in}}%
\pgfpathlineto{\pgfqpoint{0.325788in}{1.437776in}}%
\pgfpathlineto{\pgfqpoint{0.341445in}{1.441563in}}%
\pgfpathlineto{\pgfqpoint{0.357101in}{1.446364in}}%
\pgfpathlineto{\pgfqpoint{0.361173in}{1.447870in}}%
\pgfpathlineto{\pgfqpoint{0.372758in}{1.452255in}}%
\pgfpathlineto{\pgfqpoint{0.388415in}{1.459123in}}%
\pgfpathlineto{\pgfqpoint{0.393163in}{1.461481in}}%
\pgfpathlineto{\pgfqpoint{0.404071in}{1.467043in}}%
\pgfpathlineto{\pgfqpoint{0.418346in}{1.475092in}}%
\pgfpathlineto{\pgfqpoint{0.419728in}{1.475895in}}%
\pgfpathlineto{\pgfqpoint{0.435384in}{1.485841in}}%
\pgfpathlineto{\pgfqpoint{0.439569in}{1.488703in}}%
\pgfpathlineto{\pgfqpoint{0.451041in}{1.496829in}}%
\pgfpathlineto{\pgfqpoint{0.458313in}{1.502314in}}%
\pgfpathlineto{\pgfqpoint{0.466697in}{1.508900in}}%
\pgfpathlineto{\pgfqpoint{0.475167in}{1.515925in}}%
\pgfpathlineto{\pgfqpoint{0.482354in}{1.522173in}}%
\pgfpathlineto{\pgfqpoint{0.490434in}{1.529536in}}%
\pgfpathlineto{\pgfqpoint{0.498011in}{1.536825in}}%
\pgfpathlineto{\pgfqpoint{0.504320in}{1.543148in}}%
\pgfpathlineto{\pgfqpoint{0.513667in}{1.553121in}}%
\pgfpathlineto{\pgfqpoint{0.516959in}{1.556759in}}%
\pgfpathlineto{\pgfqpoint{0.528400in}{1.570370in}}%
\pgfpathlineto{\pgfqpoint{0.529324in}{1.571571in}}%
\pgfpathlineto{\pgfqpoint{0.538583in}{1.583981in}}%
\pgfpathlineto{\pgfqpoint{0.544980in}{1.593464in}}%
\pgfpathlineto{\pgfqpoint{0.547693in}{1.597592in}}%
\pgfpathlineto{\pgfqpoint{0.555593in}{1.611203in}}%
\pgfpathlineto{\pgfqpoint{0.560637in}{1.621275in}}%
\pgfpathlineto{\pgfqpoint{0.562369in}{1.624814in}}%
\pgfpathlineto{\pgfqpoint{0.567891in}{1.638425in}}%
\pgfpathlineto{\pgfqpoint{0.572248in}{1.652036in}}%
\pgfpathlineto{\pgfqpoint{0.575394in}{1.665648in}}%
\pgfpathlineto{\pgfqpoint{0.576293in}{1.672094in}}%
\pgfpathlineto{\pgfqpoint{0.577271in}{1.679259in}}%
\pgfpathlineto{\pgfqpoint{0.577891in}{1.692870in}}%
\pgfpathlineto{\pgfqpoint{0.576293in}{1.692870in}}%
\pgfpathlineto{\pgfqpoint{0.560637in}{1.692870in}}%
\pgfpathlineto{\pgfqpoint{0.544980in}{1.692870in}}%
\pgfpathlineto{\pgfqpoint{0.529324in}{1.692870in}}%
\pgfpathlineto{\pgfqpoint{0.513667in}{1.692870in}}%
\pgfpathlineto{\pgfqpoint{0.498011in}{1.692870in}}%
\pgfpathlineto{\pgfqpoint{0.482354in}{1.692870in}}%
\pgfpathlineto{\pgfqpoint{0.466697in}{1.692870in}}%
\pgfpathlineto{\pgfqpoint{0.451041in}{1.692870in}}%
\pgfpathlineto{\pgfqpoint{0.435384in}{1.692870in}}%
\pgfpathlineto{\pgfqpoint{0.419728in}{1.692870in}}%
\pgfpathlineto{\pgfqpoint{0.404071in}{1.692870in}}%
\pgfpathlineto{\pgfqpoint{0.388415in}{1.692870in}}%
\pgfpathlineto{\pgfqpoint{0.372758in}{1.692870in}}%
\pgfpathlineto{\pgfqpoint{0.357101in}{1.692870in}}%
\pgfpathlineto{\pgfqpoint{0.341445in}{1.692870in}}%
\pgfpathlineto{\pgfqpoint{0.325788in}{1.692870in}}%
\pgfpathlineto{\pgfqpoint{0.310132in}{1.692870in}}%
\pgfpathlineto{\pgfqpoint{0.294475in}{1.692870in}}%
\pgfpathlineto{\pgfqpoint{0.278819in}{1.692870in}}%
\pgfpathlineto{\pgfqpoint{0.278819in}{1.679259in}}%
\pgfpathlineto{\pgfqpoint{0.278819in}{1.665648in}}%
\pgfpathlineto{\pgfqpoint{0.278819in}{1.652036in}}%
\pgfpathlineto{\pgfqpoint{0.278819in}{1.638425in}}%
\pgfpathlineto{\pgfqpoint{0.278819in}{1.624814in}}%
\pgfpathlineto{\pgfqpoint{0.278819in}{1.611203in}}%
\pgfpathlineto{\pgfqpoint{0.278819in}{1.597592in}}%
\pgfpathlineto{\pgfqpoint{0.278819in}{1.583981in}}%
\pgfpathlineto{\pgfqpoint{0.278819in}{1.570370in}}%
\pgfpathlineto{\pgfqpoint{0.278819in}{1.556759in}}%
\pgfpathlineto{\pgfqpoint{0.278819in}{1.543148in}}%
\pgfpathlineto{\pgfqpoint{0.278819in}{1.529536in}}%
\pgfpathlineto{\pgfqpoint{0.278819in}{1.515925in}}%
\pgfpathlineto{\pgfqpoint{0.278819in}{1.502314in}}%
\pgfpathlineto{\pgfqpoint{0.278819in}{1.488703in}}%
\pgfpathlineto{\pgfqpoint{0.278819in}{1.475092in}}%
\pgfpathlineto{\pgfqpoint{0.278819in}{1.461481in}}%
\pgfpathlineto{\pgfqpoint{0.278819in}{1.447870in}}%
\pgfpathlineto{\pgfqpoint{0.278819in}{1.434259in}}%
\pgfpathlineto{\pgfqpoint{0.278819in}{1.432870in}}%
\pgfpathlineto{\pgfqpoint{0.294475in}{1.433409in}}%
\pgfpathclose%
\pgfpathmoveto{\pgfqpoint{1.813162in}{1.433409in}}%
\pgfpathlineto{\pgfqpoint{1.828819in}{1.432870in}}%
\pgfpathlineto{\pgfqpoint{1.828819in}{1.434259in}}%
\pgfpathlineto{\pgfqpoint{1.828819in}{1.447870in}}%
\pgfpathlineto{\pgfqpoint{1.828819in}{1.461481in}}%
\pgfpathlineto{\pgfqpoint{1.828819in}{1.475092in}}%
\pgfpathlineto{\pgfqpoint{1.828819in}{1.488703in}}%
\pgfpathlineto{\pgfqpoint{1.828819in}{1.502314in}}%
\pgfpathlineto{\pgfqpoint{1.828819in}{1.515925in}}%
\pgfpathlineto{\pgfqpoint{1.828819in}{1.529536in}}%
\pgfpathlineto{\pgfqpoint{1.828819in}{1.543148in}}%
\pgfpathlineto{\pgfqpoint{1.828819in}{1.556759in}}%
\pgfpathlineto{\pgfqpoint{1.828819in}{1.570370in}}%
\pgfpathlineto{\pgfqpoint{1.828819in}{1.583981in}}%
\pgfpathlineto{\pgfqpoint{1.828819in}{1.597592in}}%
\pgfpathlineto{\pgfqpoint{1.828819in}{1.611203in}}%
\pgfpathlineto{\pgfqpoint{1.828819in}{1.624814in}}%
\pgfpathlineto{\pgfqpoint{1.828819in}{1.638425in}}%
\pgfpathlineto{\pgfqpoint{1.828819in}{1.652036in}}%
\pgfpathlineto{\pgfqpoint{1.828819in}{1.665648in}}%
\pgfpathlineto{\pgfqpoint{1.828819in}{1.679259in}}%
\pgfpathlineto{\pgfqpoint{1.828819in}{1.692870in}}%
\pgfpathlineto{\pgfqpoint{1.813162in}{1.692870in}}%
\pgfpathlineto{\pgfqpoint{1.797505in}{1.692870in}}%
\pgfpathlineto{\pgfqpoint{1.781849in}{1.692870in}}%
\pgfpathlineto{\pgfqpoint{1.766192in}{1.692870in}}%
\pgfpathlineto{\pgfqpoint{1.750536in}{1.692870in}}%
\pgfpathlineto{\pgfqpoint{1.734879in}{1.692870in}}%
\pgfpathlineto{\pgfqpoint{1.719223in}{1.692870in}}%
\pgfpathlineto{\pgfqpoint{1.703566in}{1.692870in}}%
\pgfpathlineto{\pgfqpoint{1.687910in}{1.692870in}}%
\pgfpathlineto{\pgfqpoint{1.672253in}{1.692870in}}%
\pgfpathlineto{\pgfqpoint{1.656596in}{1.692870in}}%
\pgfpathlineto{\pgfqpoint{1.640940in}{1.692870in}}%
\pgfpathlineto{\pgfqpoint{1.625283in}{1.692870in}}%
\pgfpathlineto{\pgfqpoint{1.609627in}{1.692870in}}%
\pgfpathlineto{\pgfqpoint{1.593970in}{1.692870in}}%
\pgfpathlineto{\pgfqpoint{1.578314in}{1.692870in}}%
\pgfpathlineto{\pgfqpoint{1.562657in}{1.692870in}}%
\pgfpathlineto{\pgfqpoint{1.547000in}{1.692870in}}%
\pgfpathlineto{\pgfqpoint{1.531344in}{1.692870in}}%
\pgfpathlineto{\pgfqpoint{1.529746in}{1.692870in}}%
\pgfpathlineto{\pgfqpoint{1.530367in}{1.679259in}}%
\pgfpathlineto{\pgfqpoint{1.531344in}{1.672094in}}%
\pgfpathlineto{\pgfqpoint{1.532243in}{1.665648in}}%
\pgfpathlineto{\pgfqpoint{1.535389in}{1.652036in}}%
\pgfpathlineto{\pgfqpoint{1.539746in}{1.638425in}}%
\pgfpathlineto{\pgfqpoint{1.545268in}{1.624814in}}%
\pgfpathlineto{\pgfqpoint{1.547000in}{1.621275in}}%
\pgfpathlineto{\pgfqpoint{1.552045in}{1.611203in}}%
\pgfpathlineto{\pgfqpoint{1.559944in}{1.597592in}}%
\pgfpathlineto{\pgfqpoint{1.562657in}{1.593464in}}%
\pgfpathlineto{\pgfqpoint{1.569055in}{1.583981in}}%
\pgfpathlineto{\pgfqpoint{1.578314in}{1.571571in}}%
\pgfpathlineto{\pgfqpoint{1.579237in}{1.570370in}}%
\pgfpathlineto{\pgfqpoint{1.590678in}{1.556759in}}%
\pgfpathlineto{\pgfqpoint{1.593970in}{1.553121in}}%
\pgfpathlineto{\pgfqpoint{1.603317in}{1.543148in}}%
\pgfpathlineto{\pgfqpoint{1.609627in}{1.536825in}}%
\pgfpathlineto{\pgfqpoint{1.617203in}{1.529536in}}%
\pgfpathlineto{\pgfqpoint{1.625283in}{1.522173in}}%
\pgfpathlineto{\pgfqpoint{1.632470in}{1.515925in}}%
\pgfpathlineto{\pgfqpoint{1.640940in}{1.508900in}}%
\pgfpathlineto{\pgfqpoint{1.649324in}{1.502314in}}%
\pgfpathlineto{\pgfqpoint{1.656596in}{1.496829in}}%
\pgfpathlineto{\pgfqpoint{1.668069in}{1.488703in}}%
\pgfpathlineto{\pgfqpoint{1.672253in}{1.485841in}}%
\pgfpathlineto{\pgfqpoint{1.687910in}{1.475895in}}%
\pgfpathlineto{\pgfqpoint{1.689291in}{1.475092in}}%
\pgfpathlineto{\pgfqpoint{1.703566in}{1.467043in}}%
\pgfpathlineto{\pgfqpoint{1.714474in}{1.461481in}}%
\pgfpathlineto{\pgfqpoint{1.719223in}{1.459123in}}%
\pgfpathlineto{\pgfqpoint{1.734879in}{1.452255in}}%
\pgfpathlineto{\pgfqpoint{1.746464in}{1.447870in}}%
\pgfpathlineto{\pgfqpoint{1.750536in}{1.446364in}}%
\pgfpathlineto{\pgfqpoint{1.766192in}{1.441563in}}%
\pgfpathlineto{\pgfqpoint{1.781849in}{1.437776in}}%
\pgfpathlineto{\pgfqpoint{1.797505in}{1.435041in}}%
\pgfpathlineto{\pgfqpoint{1.804921in}{1.434259in}}%
\pgfpathlineto{\pgfqpoint{1.813162in}{1.433409in}}%
\pgfpathclose%
\pgfusepath{fill}%
\end{pgfscope}%
\begin{pgfscope}%
\pgfsetbuttcap%
\pgfsetroundjoin%
\definecolor{currentfill}{rgb}{0.000000,0.000000,0.000000}%
\pgfsetfillcolor{currentfill}%
\pgfsetlinewidth{0.803000pt}%
\definecolor{currentstroke}{rgb}{0.000000,0.000000,0.000000}%
\pgfsetstrokecolor{currentstroke}%
\pgfsetdash{}{0pt}%
\pgfsys@defobject{currentmarker}{\pgfqpoint{0.000000in}{-0.048611in}}{\pgfqpoint{0.000000in}{0.000000in}}{%
\pgfpathmoveto{\pgfqpoint{0.000000in}{0.000000in}}%
\pgfpathlineto{\pgfqpoint{0.000000in}{-0.048611in}}%
\pgfusepath{stroke,fill}%
}%
\begin{pgfscope}%
\pgfsys@transformshift{0.278819in}{0.345370in}%
\pgfsys@useobject{currentmarker}{}%
\end{pgfscope}%
\end{pgfscope}%
\begin{pgfscope}%
\definecolor{textcolor}{rgb}{0.000000,0.000000,0.000000}%
\pgfsetstrokecolor{textcolor}%
\pgfsetfillcolor{textcolor}%
\pgftext[x=0.278819in,y=0.248148in,,top]{\color{textcolor}{\rmfamily\fontsize{12.000000}{14.400000}\selectfont\catcode`\^=\active\def^{\ifmmode\sp\else\^{}\fi}\catcode`\%=\active\def%{\%}$\mathdefault{0}$}}%
\end{pgfscope}%
\begin{pgfscope}%
\pgfsetbuttcap%
\pgfsetroundjoin%
\definecolor{currentfill}{rgb}{0.000000,0.000000,0.000000}%
\pgfsetfillcolor{currentfill}%
\pgfsetlinewidth{0.803000pt}%
\definecolor{currentstroke}{rgb}{0.000000,0.000000,0.000000}%
\pgfsetstrokecolor{currentstroke}%
\pgfsetdash{}{0pt}%
\pgfsys@defobject{currentmarker}{\pgfqpoint{0.000000in}{-0.048611in}}{\pgfqpoint{0.000000in}{0.000000in}}{%
\pgfpathmoveto{\pgfqpoint{0.000000in}{0.000000in}}%
\pgfpathlineto{\pgfqpoint{0.000000in}{-0.048611in}}%
\pgfusepath{stroke,fill}%
}%
\begin{pgfscope}%
\pgfsys@transformshift{0.795485in}{0.345370in}%
\pgfsys@useobject{currentmarker}{}%
\end{pgfscope}%
\end{pgfscope}%
\begin{pgfscope}%
\definecolor{textcolor}{rgb}{0.000000,0.000000,0.000000}%
\pgfsetstrokecolor{textcolor}%
\pgfsetfillcolor{textcolor}%
\pgftext[x=0.795485in,y=0.248148in,,top]{\color{textcolor}{\rmfamily\fontsize{12.000000}{14.400000}\selectfont\catcode`\^=\active\def^{\ifmmode\sp\else\^{}\fi}\catcode`\%=\active\def%{\%}$\mathdefault{2}$}}%
\end{pgfscope}%
\begin{pgfscope}%
\pgfsetbuttcap%
\pgfsetroundjoin%
\definecolor{currentfill}{rgb}{0.000000,0.000000,0.000000}%
\pgfsetfillcolor{currentfill}%
\pgfsetlinewidth{0.803000pt}%
\definecolor{currentstroke}{rgb}{0.000000,0.000000,0.000000}%
\pgfsetstrokecolor{currentstroke}%
\pgfsetdash{}{0pt}%
\pgfsys@defobject{currentmarker}{\pgfqpoint{0.000000in}{-0.048611in}}{\pgfqpoint{0.000000in}{0.000000in}}{%
\pgfpathmoveto{\pgfqpoint{0.000000in}{0.000000in}}%
\pgfpathlineto{\pgfqpoint{0.000000in}{-0.048611in}}%
\pgfusepath{stroke,fill}%
}%
\begin{pgfscope}%
\pgfsys@transformshift{1.312152in}{0.345370in}%
\pgfsys@useobject{currentmarker}{}%
\end{pgfscope}%
\end{pgfscope}%
\begin{pgfscope}%
\definecolor{textcolor}{rgb}{0.000000,0.000000,0.000000}%
\pgfsetstrokecolor{textcolor}%
\pgfsetfillcolor{textcolor}%
\pgftext[x=1.312152in,y=0.248148in,,top]{\color{textcolor}{\rmfamily\fontsize{12.000000}{14.400000}\selectfont\catcode`\^=\active\def^{\ifmmode\sp\else\^{}\fi}\catcode`\%=\active\def%{\%}$\mathdefault{4}$}}%
\end{pgfscope}%
\begin{pgfscope}%
\pgfsetbuttcap%
\pgfsetroundjoin%
\definecolor{currentfill}{rgb}{0.000000,0.000000,0.000000}%
\pgfsetfillcolor{currentfill}%
\pgfsetlinewidth{0.803000pt}%
\definecolor{currentstroke}{rgb}{0.000000,0.000000,0.000000}%
\pgfsetstrokecolor{currentstroke}%
\pgfsetdash{}{0pt}%
\pgfsys@defobject{currentmarker}{\pgfqpoint{0.000000in}{-0.048611in}}{\pgfqpoint{0.000000in}{0.000000in}}{%
\pgfpathmoveto{\pgfqpoint{0.000000in}{0.000000in}}%
\pgfpathlineto{\pgfqpoint{0.000000in}{-0.048611in}}%
\pgfusepath{stroke,fill}%
}%
\begin{pgfscope}%
\pgfsys@transformshift{1.828819in}{0.345370in}%
\pgfsys@useobject{currentmarker}{}%
\end{pgfscope}%
\end{pgfscope}%
\begin{pgfscope}%
\definecolor{textcolor}{rgb}{0.000000,0.000000,0.000000}%
\pgfsetstrokecolor{textcolor}%
\pgfsetfillcolor{textcolor}%
\pgftext[x=1.828819in,y=0.248148in,,top]{\color{textcolor}{\rmfamily\fontsize{12.000000}{14.400000}\selectfont\catcode`\^=\active\def^{\ifmmode\sp\else\^{}\fi}\catcode`\%=\active\def%{\%}$\mathdefault{6}$}}%
\end{pgfscope}%
\begin{pgfscope}%
\pgfsetbuttcap%
\pgfsetroundjoin%
\definecolor{currentfill}{rgb}{0.000000,0.000000,0.000000}%
\pgfsetfillcolor{currentfill}%
\pgfsetlinewidth{0.803000pt}%
\definecolor{currentstroke}{rgb}{0.000000,0.000000,0.000000}%
\pgfsetstrokecolor{currentstroke}%
\pgfsetdash{}{0pt}%
\pgfsys@defobject{currentmarker}{\pgfqpoint{-0.048611in}{0.000000in}}{\pgfqpoint{-0.000000in}{0.000000in}}{%
\pgfpathmoveto{\pgfqpoint{-0.000000in}{0.000000in}}%
\pgfpathlineto{\pgfqpoint{-0.048611in}{0.000000in}}%
\pgfusepath{stroke,fill}%
}%
\begin{pgfscope}%
\pgfsys@transformshift{0.278819in}{0.345370in}%
\pgfsys@useobject{currentmarker}{}%
\end{pgfscope}%
\end{pgfscope}%
\begin{pgfscope}%
\definecolor{textcolor}{rgb}{0.000000,0.000000,0.000000}%
\pgfsetstrokecolor{textcolor}%
\pgfsetfillcolor{textcolor}%
\pgftext[x=0.100000in, y=0.287500in, left, base]{\color{textcolor}{\rmfamily\fontsize{12.000000}{14.400000}\selectfont\catcode`\^=\active\def^{\ifmmode\sp\else\^{}\fi}\catcode`\%=\active\def%{\%}$\mathdefault{0}$}}%
\end{pgfscope}%
\begin{pgfscope}%
\pgfsetbuttcap%
\pgfsetroundjoin%
\definecolor{currentfill}{rgb}{0.000000,0.000000,0.000000}%
\pgfsetfillcolor{currentfill}%
\pgfsetlinewidth{0.803000pt}%
\definecolor{currentstroke}{rgb}{0.000000,0.000000,0.000000}%
\pgfsetstrokecolor{currentstroke}%
\pgfsetdash{}{0pt}%
\pgfsys@defobject{currentmarker}{\pgfqpoint{-0.048611in}{0.000000in}}{\pgfqpoint{-0.000000in}{0.000000in}}{%
\pgfpathmoveto{\pgfqpoint{-0.000000in}{0.000000in}}%
\pgfpathlineto{\pgfqpoint{-0.048611in}{0.000000in}}%
\pgfusepath{stroke,fill}%
}%
\begin{pgfscope}%
\pgfsys@transformshift{0.278819in}{0.794536in}%
\pgfsys@useobject{currentmarker}{}%
\end{pgfscope}%
\end{pgfscope}%
\begin{pgfscope}%
\definecolor{textcolor}{rgb}{0.000000,0.000000,0.000000}%
\pgfsetstrokecolor{textcolor}%
\pgfsetfillcolor{textcolor}%
\pgftext[x=0.100000in, y=0.736666in, left, base]{\color{textcolor}{\rmfamily\fontsize{12.000000}{14.400000}\selectfont\catcode`\^=\active\def^{\ifmmode\sp\else\^{}\fi}\catcode`\%=\active\def%{\%}$\mathdefault{2}$}}%
\end{pgfscope}%
\begin{pgfscope}%
\pgfsetbuttcap%
\pgfsetroundjoin%
\definecolor{currentfill}{rgb}{0.000000,0.000000,0.000000}%
\pgfsetfillcolor{currentfill}%
\pgfsetlinewidth{0.803000pt}%
\definecolor{currentstroke}{rgb}{0.000000,0.000000,0.000000}%
\pgfsetstrokecolor{currentstroke}%
\pgfsetdash{}{0pt}%
\pgfsys@defobject{currentmarker}{\pgfqpoint{-0.048611in}{0.000000in}}{\pgfqpoint{-0.000000in}{0.000000in}}{%
\pgfpathmoveto{\pgfqpoint{-0.000000in}{0.000000in}}%
\pgfpathlineto{\pgfqpoint{-0.048611in}{0.000000in}}%
\pgfusepath{stroke,fill}%
}%
\begin{pgfscope}%
\pgfsys@transformshift{0.278819in}{1.243703in}%
\pgfsys@useobject{currentmarker}{}%
\end{pgfscope}%
\end{pgfscope}%
\begin{pgfscope}%
\definecolor{textcolor}{rgb}{0.000000,0.000000,0.000000}%
\pgfsetstrokecolor{textcolor}%
\pgfsetfillcolor{textcolor}%
\pgftext[x=0.100000in, y=1.185833in, left, base]{\color{textcolor}{\rmfamily\fontsize{12.000000}{14.400000}\selectfont\catcode`\^=\active\def^{\ifmmode\sp\else\^{}\fi}\catcode`\%=\active\def%{\%}$\mathdefault{4}$}}%
\end{pgfscope}%
\begin{pgfscope}%
\pgfsetbuttcap%
\pgfsetroundjoin%
\definecolor{currentfill}{rgb}{0.000000,0.000000,0.000000}%
\pgfsetfillcolor{currentfill}%
\pgfsetlinewidth{0.803000pt}%
\definecolor{currentstroke}{rgb}{0.000000,0.000000,0.000000}%
\pgfsetstrokecolor{currentstroke}%
\pgfsetdash{}{0pt}%
\pgfsys@defobject{currentmarker}{\pgfqpoint{-0.048611in}{0.000000in}}{\pgfqpoint{-0.000000in}{0.000000in}}{%
\pgfpathmoveto{\pgfqpoint{-0.000000in}{0.000000in}}%
\pgfpathlineto{\pgfqpoint{-0.048611in}{0.000000in}}%
\pgfusepath{stroke,fill}%
}%
\begin{pgfscope}%
\pgfsys@transformshift{0.278819in}{1.692870in}%
\pgfsys@useobject{currentmarker}{}%
\end{pgfscope}%
\end{pgfscope}%
\begin{pgfscope}%
\definecolor{textcolor}{rgb}{0.000000,0.000000,0.000000}%
\pgfsetstrokecolor{textcolor}%
\pgfsetfillcolor{textcolor}%
\pgftext[x=0.100000in, y=1.635000in, left, base]{\color{textcolor}{\rmfamily\fontsize{12.000000}{14.400000}\selectfont\catcode`\^=\active\def^{\ifmmode\sp\else\^{}\fi}\catcode`\%=\active\def%{\%}$\mathdefault{6}$}}%
\end{pgfscope}%
\begin{pgfscope}%
\pgfsetrectcap%
\pgfsetmiterjoin%
\pgfsetlinewidth{0.803000pt}%
\definecolor{currentstroke}{rgb}{0.000000,0.000000,0.000000}%
\pgfsetstrokecolor{currentstroke}%
\pgfsetdash{}{0pt}%
\pgfpathmoveto{\pgfqpoint{0.278819in}{0.345370in}}%
\pgfpathlineto{\pgfqpoint{0.278819in}{1.692870in}}%
\pgfusepath{stroke}%
\end{pgfscope}%
\begin{pgfscope}%
\pgfsetrectcap%
\pgfsetmiterjoin%
\pgfsetlinewidth{0.803000pt}%
\definecolor{currentstroke}{rgb}{0.000000,0.000000,0.000000}%
\pgfsetstrokecolor{currentstroke}%
\pgfsetdash{}{0pt}%
\pgfpathmoveto{\pgfqpoint{1.828819in}{0.345370in}}%
\pgfpathlineto{\pgfqpoint{1.828819in}{1.692870in}}%
\pgfusepath{stroke}%
\end{pgfscope}%
\begin{pgfscope}%
\pgfsetrectcap%
\pgfsetmiterjoin%
\pgfsetlinewidth{0.803000pt}%
\definecolor{currentstroke}{rgb}{0.000000,0.000000,0.000000}%
\pgfsetstrokecolor{currentstroke}%
\pgfsetdash{}{0pt}%
\pgfpathmoveto{\pgfqpoint{0.278819in}{0.345370in}}%
\pgfpathlineto{\pgfqpoint{1.828819in}{0.345370in}}%
\pgfusepath{stroke}%
\end{pgfscope}%
\begin{pgfscope}%
\pgfsetrectcap%
\pgfsetmiterjoin%
\pgfsetlinewidth{0.803000pt}%
\definecolor{currentstroke}{rgb}{0.000000,0.000000,0.000000}%
\pgfsetstrokecolor{currentstroke}%
\pgfsetdash{}{0pt}%
\pgfpathmoveto{\pgfqpoint{0.278819in}{1.692870in}}%
\pgfpathlineto{\pgfqpoint{1.828819in}{1.692870in}}%
\pgfusepath{stroke}%
\end{pgfscope}%
\end{pgfpicture}%
\makeatother%
\endgroup%}
        \caption{$n_c=1.$}
        \label{fig:gaussian-well-1}
    \end{subfigure}
    \begin{subfigure}[b]{0.32\columnwidth}
        \scalebox{0.7}{%% Creator: Matplotlib, PGF backend
%%
%% To include the figure in your LaTeX document, write
%%   \input{<filename>.pgf}
%%
%% Make sure the required packages are loaded in your preamble
%%   \usepackage{pgf}
%%
%% Also ensure that all the required font packages are loaded; for instance,
%% the lmodern package is sometimes necessary when using math font.
%%   \usepackage{lmodern}
%%
%% Figures using additional raster images can only be included by \input if
%% they are in the same directory as the main LaTeX file. For loading figures
%% from other directories you can use the `import` package
%%   \usepackage{import}
%%
%% and then include the figures with
%%   \import{<path to file>}{<filename>.pgf}
%%
%% Matplotlib used the following preamble
%%   \def\mathdefault#1{#1}
%%   \everymath=\expandafter{\the\everymath\displaystyle}
%%   
%%   \makeatletter\@ifpackageloaded{underscore}{}{\usepackage[strings]{underscore}}\makeatother
%%
\begingroup%
\makeatletter%
\begin{pgfpicture}%
\pgfpathrectangle{\pgfpointorigin}{\pgfqpoint{2.010415in}{1.792870in}}%
\pgfusepath{use as bounding box, clip}%
\begin{pgfscope}%
\pgfsetbuttcap%
\pgfsetmiterjoin%
\definecolor{currentfill}{rgb}{1.000000,1.000000,1.000000}%
\pgfsetfillcolor{currentfill}%
\pgfsetlinewidth{0.000000pt}%
\definecolor{currentstroke}{rgb}{1.000000,1.000000,1.000000}%
\pgfsetstrokecolor{currentstroke}%
\pgfsetdash{}{0pt}%
\pgfpathmoveto{\pgfqpoint{0.000000in}{0.000000in}}%
\pgfpathlineto{\pgfqpoint{2.010415in}{0.000000in}}%
\pgfpathlineto{\pgfqpoint{2.010415in}{1.792870in}}%
\pgfpathlineto{\pgfqpoint{0.000000in}{1.792870in}}%
\pgfpathlineto{\pgfqpoint{0.000000in}{0.000000in}}%
\pgfpathclose%
\pgfusepath{fill}%
\end{pgfscope}%
\begin{pgfscope}%
\pgfsetbuttcap%
\pgfsetmiterjoin%
\definecolor{currentfill}{rgb}{1.000000,1.000000,1.000000}%
\pgfsetfillcolor{currentfill}%
\pgfsetlinewidth{0.000000pt}%
\definecolor{currentstroke}{rgb}{0.000000,0.000000,0.000000}%
\pgfsetstrokecolor{currentstroke}%
\pgfsetstrokeopacity{0.000000}%
\pgfsetdash{}{0pt}%
\pgfpathmoveto{\pgfqpoint{0.360415in}{0.345370in}}%
\pgfpathlineto{\pgfqpoint{1.910415in}{0.345370in}}%
\pgfpathlineto{\pgfqpoint{1.910415in}{1.692870in}}%
\pgfpathlineto{\pgfqpoint{0.360415in}{1.692870in}}%
\pgfpathlineto{\pgfqpoint{0.360415in}{0.345370in}}%
\pgfpathclose%
\pgfusepath{fill}%
\end{pgfscope}%
\begin{pgfscope}%
\pgfpathrectangle{\pgfqpoint{0.360415in}{0.345370in}}{\pgfqpoint{1.550000in}{1.347500in}}%
\pgfusepath{clip}%
\pgfsetbuttcap%
\pgfsetroundjoin%
\definecolor{currentfill}{rgb}{0.993545,0.862859,0.619299}%
\pgfsetfillcolor{currentfill}%
\pgfsetlinewidth{0.000000pt}%
\definecolor{currentstroke}{rgb}{0.000000,0.000000,0.000000}%
\pgfsetstrokecolor{currentstroke}%
\pgfsetdash{}{0pt}%
\pgfpathmoveto{\pgfqpoint{0.720516in}{0.611575in}}%
\pgfpathlineto{\pgfqpoint{0.736173in}{0.608224in}}%
\pgfpathlineto{\pgfqpoint{0.751829in}{0.607554in}}%
\pgfpathlineto{\pgfqpoint{0.767486in}{0.609564in}}%
\pgfpathlineto{\pgfqpoint{0.783142in}{0.614257in}}%
\pgfpathlineto{\pgfqpoint{0.790248in}{0.617592in}}%
\pgfpathlineto{\pgfqpoint{0.798799in}{0.622459in}}%
\pgfpathlineto{\pgfqpoint{0.810130in}{0.631203in}}%
\pgfpathlineto{\pgfqpoint{0.814455in}{0.635525in}}%
\pgfpathlineto{\pgfqpoint{0.821899in}{0.644814in}}%
\pgfpathlineto{\pgfqpoint{0.829467in}{0.658425in}}%
\pgfpathlineto{\pgfqpoint{0.830112in}{0.660510in}}%
\pgfpathlineto{\pgfqpoint{0.833137in}{0.672036in}}%
\pgfpathlineto{\pgfqpoint{0.833852in}{0.685648in}}%
\pgfpathlineto{\pgfqpoint{0.831708in}{0.699259in}}%
\pgfpathlineto{\pgfqpoint{0.830112in}{0.703613in}}%
\pgfpathlineto{\pgfqpoint{0.826104in}{0.712870in}}%
\pgfpathlineto{\pgfqpoint{0.816851in}{0.726481in}}%
\pgfpathlineto{\pgfqpoint{0.814455in}{0.729086in}}%
\pgfpathlineto{\pgfqpoint{0.801795in}{0.740092in}}%
\pgfpathlineto{\pgfqpoint{0.798799in}{0.742175in}}%
\pgfpathlineto{\pgfqpoint{0.783142in}{0.750219in}}%
\pgfpathlineto{\pgfqpoint{0.772494in}{0.753703in}}%
\pgfpathlineto{\pgfqpoint{0.767486in}{0.755090in}}%
\pgfpathlineto{\pgfqpoint{0.751829in}{0.756954in}}%
\pgfpathlineto{\pgfqpoint{0.736173in}{0.756333in}}%
\pgfpathlineto{\pgfqpoint{0.722914in}{0.753703in}}%
\pgfpathlineto{\pgfqpoint{0.720516in}{0.753142in}}%
\pgfpathlineto{\pgfqpoint{0.704859in}{0.746563in}}%
\pgfpathlineto{\pgfqpoint{0.694174in}{0.740092in}}%
\pgfpathlineto{\pgfqpoint{0.689203in}{0.736332in}}%
\pgfpathlineto{\pgfqpoint{0.679145in}{0.726481in}}%
\pgfpathlineto{\pgfqpoint{0.673546in}{0.719047in}}%
\pgfpathlineto{\pgfqpoint{0.669710in}{0.712870in}}%
\pgfpathlineto{\pgfqpoint{0.664312in}{0.699259in}}%
\pgfpathlineto{\pgfqpoint{0.662000in}{0.685648in}}%
\pgfpathlineto{\pgfqpoint{0.662771in}{0.672036in}}%
\pgfpathlineto{\pgfqpoint{0.666625in}{0.658425in}}%
\pgfpathlineto{\pgfqpoint{0.673546in}{0.644857in}}%
\pgfpathlineto{\pgfqpoint{0.673573in}{0.644814in}}%
\pgfpathlineto{\pgfqpoint{0.685648in}{0.631203in}}%
\pgfpathlineto{\pgfqpoint{0.689203in}{0.628113in}}%
\pgfpathlineto{\pgfqpoint{0.704859in}{0.617615in}}%
\pgfpathlineto{\pgfqpoint{0.704909in}{0.617592in}}%
\pgfpathlineto{\pgfqpoint{0.720516in}{0.611575in}}%
\pgfpathclose%
\pgfpathmoveto{\pgfqpoint{1.487688in}{0.614257in}}%
\pgfpathlineto{\pgfqpoint{1.503344in}{0.609564in}}%
\pgfpathlineto{\pgfqpoint{1.519001in}{0.607554in}}%
\pgfpathlineto{\pgfqpoint{1.534657in}{0.608224in}}%
\pgfpathlineto{\pgfqpoint{1.550314in}{0.611575in}}%
\pgfpathlineto{\pgfqpoint{1.565921in}{0.617592in}}%
\pgfpathlineto{\pgfqpoint{1.565971in}{0.617615in}}%
\pgfpathlineto{\pgfqpoint{1.581627in}{0.628113in}}%
\pgfpathlineto{\pgfqpoint{1.585182in}{0.631203in}}%
\pgfpathlineto{\pgfqpoint{1.597257in}{0.644814in}}%
\pgfpathlineto{\pgfqpoint{1.597284in}{0.644857in}}%
\pgfpathlineto{\pgfqpoint{1.604205in}{0.658425in}}%
\pgfpathlineto{\pgfqpoint{1.608059in}{0.672036in}}%
\pgfpathlineto{\pgfqpoint{1.608830in}{0.685648in}}%
\pgfpathlineto{\pgfqpoint{1.606518in}{0.699259in}}%
\pgfpathlineto{\pgfqpoint{1.601120in}{0.712870in}}%
\pgfpathlineto{\pgfqpoint{1.597284in}{0.719047in}}%
\pgfpathlineto{\pgfqpoint{1.591685in}{0.726481in}}%
\pgfpathlineto{\pgfqpoint{1.581627in}{0.736332in}}%
\pgfpathlineto{\pgfqpoint{1.576656in}{0.740092in}}%
\pgfpathlineto{\pgfqpoint{1.565971in}{0.746563in}}%
\pgfpathlineto{\pgfqpoint{1.550314in}{0.753142in}}%
\pgfpathlineto{\pgfqpoint{1.547916in}{0.753703in}}%
\pgfpathlineto{\pgfqpoint{1.534657in}{0.756333in}}%
\pgfpathlineto{\pgfqpoint{1.519001in}{0.756954in}}%
\pgfpathlineto{\pgfqpoint{1.503344in}{0.755090in}}%
\pgfpathlineto{\pgfqpoint{1.498336in}{0.753703in}}%
\pgfpathlineto{\pgfqpoint{1.487688in}{0.750219in}}%
\pgfpathlineto{\pgfqpoint{1.472031in}{0.742175in}}%
\pgfpathlineto{\pgfqpoint{1.469035in}{0.740092in}}%
\pgfpathlineto{\pgfqpoint{1.456375in}{0.729086in}}%
\pgfpathlineto{\pgfqpoint{1.453979in}{0.726481in}}%
\pgfpathlineto{\pgfqpoint{1.444726in}{0.712870in}}%
\pgfpathlineto{\pgfqpoint{1.440718in}{0.703613in}}%
\pgfpathlineto{\pgfqpoint{1.439122in}{0.699259in}}%
\pgfpathlineto{\pgfqpoint{1.436978in}{0.685648in}}%
\pgfpathlineto{\pgfqpoint{1.437693in}{0.672036in}}%
\pgfpathlineto{\pgfqpoint{1.440718in}{0.660510in}}%
\pgfpathlineto{\pgfqpoint{1.441363in}{0.658425in}}%
\pgfpathlineto{\pgfqpoint{1.448931in}{0.644814in}}%
\pgfpathlineto{\pgfqpoint{1.456375in}{0.635525in}}%
\pgfpathlineto{\pgfqpoint{1.460700in}{0.631203in}}%
\pgfpathlineto{\pgfqpoint{1.472031in}{0.622459in}}%
\pgfpathlineto{\pgfqpoint{1.480582in}{0.617592in}}%
\pgfpathlineto{\pgfqpoint{1.487688in}{0.614257in}}%
\pgfpathclose%
\pgfpathmoveto{\pgfqpoint{0.736173in}{1.281906in}}%
\pgfpathlineto{\pgfqpoint{0.751829in}{1.281285in}}%
\pgfpathlineto{\pgfqpoint{0.767486in}{1.283149in}}%
\pgfpathlineto{\pgfqpoint{0.772494in}{1.284536in}}%
\pgfpathlineto{\pgfqpoint{0.783142in}{1.288021in}}%
\pgfpathlineto{\pgfqpoint{0.798799in}{1.296065in}}%
\pgfpathlineto{\pgfqpoint{0.801795in}{1.298148in}}%
\pgfpathlineto{\pgfqpoint{0.814455in}{1.309154in}}%
\pgfpathlineto{\pgfqpoint{0.816851in}{1.311759in}}%
\pgfpathlineto{\pgfqpoint{0.826104in}{1.325370in}}%
\pgfpathlineto{\pgfqpoint{0.830112in}{1.334626in}}%
\pgfpathlineto{\pgfqpoint{0.831708in}{1.338981in}}%
\pgfpathlineto{\pgfqpoint{0.833852in}{1.352592in}}%
\pgfpathlineto{\pgfqpoint{0.833137in}{1.366203in}}%
\pgfpathlineto{\pgfqpoint{0.830112in}{1.377730in}}%
\pgfpathlineto{\pgfqpoint{0.829467in}{1.379814in}}%
\pgfpathlineto{\pgfqpoint{0.821899in}{1.393425in}}%
\pgfpathlineto{\pgfqpoint{0.814455in}{1.402714in}}%
\pgfpathlineto{\pgfqpoint{0.810130in}{1.407036in}}%
\pgfpathlineto{\pgfqpoint{0.798799in}{1.415781in}}%
\pgfpathlineto{\pgfqpoint{0.790248in}{1.420648in}}%
\pgfpathlineto{\pgfqpoint{0.783142in}{1.423982in}}%
\pgfpathlineto{\pgfqpoint{0.767486in}{1.428675in}}%
\pgfpathlineto{\pgfqpoint{0.751829in}{1.430685in}}%
\pgfpathlineto{\pgfqpoint{0.736173in}{1.430015in}}%
\pgfpathlineto{\pgfqpoint{0.720516in}{1.426664in}}%
\pgfpathlineto{\pgfqpoint{0.704909in}{1.420648in}}%
\pgfpathlineto{\pgfqpoint{0.704859in}{1.420624in}}%
\pgfpathlineto{\pgfqpoint{0.689203in}{1.410127in}}%
\pgfpathlineto{\pgfqpoint{0.685648in}{1.407036in}}%
\pgfpathlineto{\pgfqpoint{0.673573in}{1.393425in}}%
\pgfpathlineto{\pgfqpoint{0.673546in}{1.393382in}}%
\pgfpathlineto{\pgfqpoint{0.666625in}{1.379814in}}%
\pgfpathlineto{\pgfqpoint{0.662771in}{1.366203in}}%
\pgfpathlineto{\pgfqpoint{0.662000in}{1.352592in}}%
\pgfpathlineto{\pgfqpoint{0.664312in}{1.338981in}}%
\pgfpathlineto{\pgfqpoint{0.669710in}{1.325370in}}%
\pgfpathlineto{\pgfqpoint{0.673546in}{1.319193in}}%
\pgfpathlineto{\pgfqpoint{0.679145in}{1.311759in}}%
\pgfpathlineto{\pgfqpoint{0.689203in}{1.301908in}}%
\pgfpathlineto{\pgfqpoint{0.694174in}{1.298148in}}%
\pgfpathlineto{\pgfqpoint{0.704859in}{1.291676in}}%
\pgfpathlineto{\pgfqpoint{0.720516in}{1.285097in}}%
\pgfpathlineto{\pgfqpoint{0.722914in}{1.284536in}}%
\pgfpathlineto{\pgfqpoint{0.736173in}{1.281906in}}%
\pgfpathclose%
\pgfpathmoveto{\pgfqpoint{1.503344in}{1.283149in}}%
\pgfpathlineto{\pgfqpoint{1.519001in}{1.281285in}}%
\pgfpathlineto{\pgfqpoint{1.534657in}{1.281906in}}%
\pgfpathlineto{\pgfqpoint{1.547916in}{1.284536in}}%
\pgfpathlineto{\pgfqpoint{1.550314in}{1.285097in}}%
\pgfpathlineto{\pgfqpoint{1.565971in}{1.291676in}}%
\pgfpathlineto{\pgfqpoint{1.576656in}{1.298148in}}%
\pgfpathlineto{\pgfqpoint{1.581627in}{1.301908in}}%
\pgfpathlineto{\pgfqpoint{1.591685in}{1.311759in}}%
\pgfpathlineto{\pgfqpoint{1.597284in}{1.319193in}}%
\pgfpathlineto{\pgfqpoint{1.601120in}{1.325370in}}%
\pgfpathlineto{\pgfqpoint{1.606518in}{1.338981in}}%
\pgfpathlineto{\pgfqpoint{1.608830in}{1.352592in}}%
\pgfpathlineto{\pgfqpoint{1.608059in}{1.366203in}}%
\pgfpathlineto{\pgfqpoint{1.604205in}{1.379814in}}%
\pgfpathlineto{\pgfqpoint{1.597284in}{1.393382in}}%
\pgfpathlineto{\pgfqpoint{1.597257in}{1.393425in}}%
\pgfpathlineto{\pgfqpoint{1.585182in}{1.407036in}}%
\pgfpathlineto{\pgfqpoint{1.581627in}{1.410127in}}%
\pgfpathlineto{\pgfqpoint{1.565971in}{1.420624in}}%
\pgfpathlineto{\pgfqpoint{1.565921in}{1.420648in}}%
\pgfpathlineto{\pgfqpoint{1.550314in}{1.426664in}}%
\pgfpathlineto{\pgfqpoint{1.534657in}{1.430015in}}%
\pgfpathlineto{\pgfqpoint{1.519001in}{1.430685in}}%
\pgfpathlineto{\pgfqpoint{1.503344in}{1.428675in}}%
\pgfpathlineto{\pgfqpoint{1.487688in}{1.423982in}}%
\pgfpathlineto{\pgfqpoint{1.480582in}{1.420648in}}%
\pgfpathlineto{\pgfqpoint{1.472031in}{1.415781in}}%
\pgfpathlineto{\pgfqpoint{1.460700in}{1.407036in}}%
\pgfpathlineto{\pgfqpoint{1.456375in}{1.402714in}}%
\pgfpathlineto{\pgfqpoint{1.448931in}{1.393425in}}%
\pgfpathlineto{\pgfqpoint{1.441363in}{1.379814in}}%
\pgfpathlineto{\pgfqpoint{1.440718in}{1.377730in}}%
\pgfpathlineto{\pgfqpoint{1.437693in}{1.366203in}}%
\pgfpathlineto{\pgfqpoint{1.436978in}{1.352592in}}%
\pgfpathlineto{\pgfqpoint{1.439122in}{1.338981in}}%
\pgfpathlineto{\pgfqpoint{1.440718in}{1.334626in}}%
\pgfpathlineto{\pgfqpoint{1.444726in}{1.325370in}}%
\pgfpathlineto{\pgfqpoint{1.453979in}{1.311759in}}%
\pgfpathlineto{\pgfqpoint{1.456375in}{1.309154in}}%
\pgfpathlineto{\pgfqpoint{1.469035in}{1.298148in}}%
\pgfpathlineto{\pgfqpoint{1.472031in}{1.296065in}}%
\pgfpathlineto{\pgfqpoint{1.487688in}{1.288021in}}%
\pgfpathlineto{\pgfqpoint{1.498336in}{1.284536in}}%
\pgfpathlineto{\pgfqpoint{1.503344in}{1.283149in}}%
\pgfpathclose%
\pgfusepath{fill}%
\end{pgfscope}%
\begin{pgfscope}%
\pgfpathrectangle{\pgfqpoint{0.360415in}{0.345370in}}{\pgfqpoint{1.550000in}{1.347500in}}%
\pgfusepath{clip}%
\pgfsetbuttcap%
\pgfsetroundjoin%
\definecolor{currentfill}{rgb}{0.993326,0.602275,0.414390}%
\pgfsetfillcolor{currentfill}%
\pgfsetlinewidth{0.000000pt}%
\definecolor{currentstroke}{rgb}{0.000000,0.000000,0.000000}%
\pgfsetstrokecolor{currentstroke}%
\pgfsetdash{}{0pt}%
\pgfpathmoveto{\pgfqpoint{0.704859in}{0.545942in}}%
\pgfpathlineto{\pgfqpoint{0.720516in}{0.542448in}}%
\pgfpathlineto{\pgfqpoint{0.736173in}{0.540508in}}%
\pgfpathlineto{\pgfqpoint{0.751829in}{0.540120in}}%
\pgfpathlineto{\pgfqpoint{0.767486in}{0.541284in}}%
\pgfpathlineto{\pgfqpoint{0.783142in}{0.544000in}}%
\pgfpathlineto{\pgfqpoint{0.798799in}{0.548273in}}%
\pgfpathlineto{\pgfqpoint{0.802220in}{0.549536in}}%
\pgfpathlineto{\pgfqpoint{0.814455in}{0.554307in}}%
\pgfpathlineto{\pgfqpoint{0.830112in}{0.562028in}}%
\pgfpathlineto{\pgfqpoint{0.832014in}{0.563148in}}%
\pgfpathlineto{\pgfqpoint{0.845769in}{0.571907in}}%
\pgfpathlineto{\pgfqpoint{0.852346in}{0.576759in}}%
\pgfpathlineto{\pgfqpoint{0.861425in}{0.584180in}}%
\pgfpathlineto{\pgfqpoint{0.868166in}{0.590370in}}%
\pgfpathlineto{\pgfqpoint{0.877082in}{0.599699in}}%
\pgfpathlineto{\pgfqpoint{0.880821in}{0.603981in}}%
\pgfpathlineto{\pgfqpoint{0.890925in}{0.617592in}}%
\pgfpathlineto{\pgfqpoint{0.892738in}{0.620583in}}%
\pgfpathlineto{\pgfqpoint{0.898773in}{0.631203in}}%
\pgfpathlineto{\pgfqpoint{0.904697in}{0.644814in}}%
\pgfpathlineto{\pgfqpoint{0.908395in}{0.657087in}}%
\pgfpathlineto{\pgfqpoint{0.908782in}{0.658425in}}%
\pgfpathlineto{\pgfqpoint{0.910980in}{0.672036in}}%
\pgfpathlineto{\pgfqpoint{0.911419in}{0.685648in}}%
\pgfpathlineto{\pgfqpoint{0.910101in}{0.699259in}}%
\pgfpathlineto{\pgfqpoint{0.908395in}{0.706820in}}%
\pgfpathlineto{\pgfqpoint{0.906975in}{0.712870in}}%
\pgfpathlineto{\pgfqpoint{0.901964in}{0.726481in}}%
\pgfpathlineto{\pgfqpoint{0.895124in}{0.740092in}}%
\pgfpathlineto{\pgfqpoint{0.892738in}{0.743874in}}%
\pgfpathlineto{\pgfqpoint{0.886114in}{0.753703in}}%
\pgfpathlineto{\pgfqpoint{0.877082in}{0.764844in}}%
\pgfpathlineto{\pgfqpoint{0.874892in}{0.767314in}}%
\pgfpathlineto{\pgfqpoint{0.861425in}{0.780449in}}%
\pgfpathlineto{\pgfqpoint{0.860877in}{0.780925in}}%
\pgfpathlineto{\pgfqpoint{0.845769in}{0.792633in}}%
\pgfpathlineto{\pgfqpoint{0.842927in}{0.794536in}}%
\pgfpathlineto{\pgfqpoint{0.830112in}{0.802388in}}%
\pgfpathlineto{\pgfqpoint{0.818806in}{0.808148in}}%
\pgfpathlineto{\pgfqpoint{0.814455in}{0.810222in}}%
\pgfpathlineto{\pgfqpoint{0.798799in}{0.816168in}}%
\pgfpathlineto{\pgfqpoint{0.783142in}{0.820524in}}%
\pgfpathlineto{\pgfqpoint{0.776183in}{0.821759in}}%
\pgfpathlineto{\pgfqpoint{0.767486in}{0.823242in}}%
\pgfpathlineto{\pgfqpoint{0.751829in}{0.824388in}}%
\pgfpathlineto{\pgfqpoint{0.736173in}{0.824006in}}%
\pgfpathlineto{\pgfqpoint{0.720516in}{0.822095in}}%
\pgfpathlineto{\pgfqpoint{0.718976in}{0.821759in}}%
\pgfpathlineto{\pgfqpoint{0.704859in}{0.818544in}}%
\pgfpathlineto{\pgfqpoint{0.689203in}{0.813394in}}%
\pgfpathlineto{\pgfqpoint{0.676987in}{0.808148in}}%
\pgfpathlineto{\pgfqpoint{0.673546in}{0.806571in}}%
\pgfpathlineto{\pgfqpoint{0.657890in}{0.797787in}}%
\pgfpathlineto{\pgfqpoint{0.652964in}{0.794536in}}%
\pgfpathlineto{\pgfqpoint{0.642233in}{0.786786in}}%
\pgfpathlineto{\pgfqpoint{0.635114in}{0.780925in}}%
\pgfpathlineto{\pgfqpoint{0.626577in}{0.773032in}}%
\pgfpathlineto{\pgfqpoint{0.620996in}{0.767314in}}%
\pgfpathlineto{\pgfqpoint{0.610920in}{0.755357in}}%
\pgfpathlineto{\pgfqpoint{0.609632in}{0.753703in}}%
\pgfpathlineto{\pgfqpoint{0.600751in}{0.740092in}}%
\pgfpathlineto{\pgfqpoint{0.595263in}{0.729455in}}%
\pgfpathlineto{\pgfqpoint{0.593810in}{0.726481in}}%
\pgfpathlineto{\pgfqpoint{0.588896in}{0.712870in}}%
\pgfpathlineto{\pgfqpoint{0.585771in}{0.699259in}}%
\pgfpathlineto{\pgfqpoint{0.584432in}{0.685648in}}%
\pgfpathlineto{\pgfqpoint{0.584878in}{0.672036in}}%
\pgfpathlineto{\pgfqpoint{0.587110in}{0.658425in}}%
\pgfpathlineto{\pgfqpoint{0.591129in}{0.644814in}}%
\pgfpathlineto{\pgfqpoint{0.595263in}{0.635103in}}%
\pgfpathlineto{\pgfqpoint{0.597014in}{0.631203in}}%
\pgfpathlineto{\pgfqpoint{0.604957in}{0.617592in}}%
\pgfpathlineto{\pgfqpoint{0.610920in}{0.609277in}}%
\pgfpathlineto{\pgfqpoint{0.615024in}{0.603981in}}%
\pgfpathlineto{\pgfqpoint{0.626577in}{0.591325in}}%
\pgfpathlineto{\pgfqpoint{0.627542in}{0.590370in}}%
\pgfpathlineto{\pgfqpoint{0.642233in}{0.577598in}}%
\pgfpathlineto{\pgfqpoint{0.643332in}{0.576759in}}%
\pgfpathlineto{\pgfqpoint{0.657890in}{0.566715in}}%
\pgfpathlineto{\pgfqpoint{0.663982in}{0.563148in}}%
\pgfpathlineto{\pgfqpoint{0.673546in}{0.557963in}}%
\pgfpathlineto{\pgfqpoint{0.689203in}{0.551058in}}%
\pgfpathlineto{\pgfqpoint{0.693689in}{0.549536in}}%
\pgfpathlineto{\pgfqpoint{0.704859in}{0.545942in}}%
\pgfpathclose%
\pgfpathmoveto{\pgfqpoint{0.704909in}{0.617592in}}%
\pgfpathlineto{\pgfqpoint{0.704859in}{0.617615in}}%
\pgfpathlineto{\pgfqpoint{0.689203in}{0.628113in}}%
\pgfpathlineto{\pgfqpoint{0.685648in}{0.631203in}}%
\pgfpathlineto{\pgfqpoint{0.673573in}{0.644814in}}%
\pgfpathlineto{\pgfqpoint{0.673546in}{0.644857in}}%
\pgfpathlineto{\pgfqpoint{0.666625in}{0.658425in}}%
\pgfpathlineto{\pgfqpoint{0.662771in}{0.672036in}}%
\pgfpathlineto{\pgfqpoint{0.662000in}{0.685648in}}%
\pgfpathlineto{\pgfqpoint{0.664312in}{0.699259in}}%
\pgfpathlineto{\pgfqpoint{0.669710in}{0.712870in}}%
\pgfpathlineto{\pgfqpoint{0.673546in}{0.719047in}}%
\pgfpathlineto{\pgfqpoint{0.679145in}{0.726481in}}%
\pgfpathlineto{\pgfqpoint{0.689203in}{0.736332in}}%
\pgfpathlineto{\pgfqpoint{0.694174in}{0.740092in}}%
\pgfpathlineto{\pgfqpoint{0.704859in}{0.746563in}}%
\pgfpathlineto{\pgfqpoint{0.720516in}{0.753142in}}%
\pgfpathlineto{\pgfqpoint{0.722914in}{0.753703in}}%
\pgfpathlineto{\pgfqpoint{0.736173in}{0.756333in}}%
\pgfpathlineto{\pgfqpoint{0.751829in}{0.756954in}}%
\pgfpathlineto{\pgfqpoint{0.767486in}{0.755090in}}%
\pgfpathlineto{\pgfqpoint{0.772494in}{0.753703in}}%
\pgfpathlineto{\pgfqpoint{0.783142in}{0.750219in}}%
\pgfpathlineto{\pgfqpoint{0.798799in}{0.742175in}}%
\pgfpathlineto{\pgfqpoint{0.801795in}{0.740092in}}%
\pgfpathlineto{\pgfqpoint{0.814455in}{0.729086in}}%
\pgfpathlineto{\pgfqpoint{0.816851in}{0.726481in}}%
\pgfpathlineto{\pgfqpoint{0.826104in}{0.712870in}}%
\pgfpathlineto{\pgfqpoint{0.830112in}{0.703613in}}%
\pgfpathlineto{\pgfqpoint{0.831708in}{0.699259in}}%
\pgfpathlineto{\pgfqpoint{0.833852in}{0.685648in}}%
\pgfpathlineto{\pgfqpoint{0.833137in}{0.672036in}}%
\pgfpathlineto{\pgfqpoint{0.830112in}{0.660510in}}%
\pgfpathlineto{\pgfqpoint{0.829467in}{0.658425in}}%
\pgfpathlineto{\pgfqpoint{0.821899in}{0.644814in}}%
\pgfpathlineto{\pgfqpoint{0.814455in}{0.635525in}}%
\pgfpathlineto{\pgfqpoint{0.810130in}{0.631203in}}%
\pgfpathlineto{\pgfqpoint{0.798799in}{0.622459in}}%
\pgfpathlineto{\pgfqpoint{0.790248in}{0.617592in}}%
\pgfpathlineto{\pgfqpoint{0.783142in}{0.614257in}}%
\pgfpathlineto{\pgfqpoint{0.767486in}{0.609564in}}%
\pgfpathlineto{\pgfqpoint{0.751829in}{0.607554in}}%
\pgfpathlineto{\pgfqpoint{0.736173in}{0.608224in}}%
\pgfpathlineto{\pgfqpoint{0.720516in}{0.611575in}}%
\pgfpathlineto{\pgfqpoint{0.704909in}{0.617592in}}%
\pgfpathclose%
\pgfpathmoveto{\pgfqpoint{1.472031in}{0.548273in}}%
\pgfpathlineto{\pgfqpoint{1.487688in}{0.544000in}}%
\pgfpathlineto{\pgfqpoint{1.503344in}{0.541284in}}%
\pgfpathlineto{\pgfqpoint{1.519001in}{0.540120in}}%
\pgfpathlineto{\pgfqpoint{1.534657in}{0.540508in}}%
\pgfpathlineto{\pgfqpoint{1.550314in}{0.542448in}}%
\pgfpathlineto{\pgfqpoint{1.565971in}{0.545942in}}%
\pgfpathlineto{\pgfqpoint{1.577141in}{0.549536in}}%
\pgfpathlineto{\pgfqpoint{1.581627in}{0.551058in}}%
\pgfpathlineto{\pgfqpoint{1.597284in}{0.557963in}}%
\pgfpathlineto{\pgfqpoint{1.606848in}{0.563148in}}%
\pgfpathlineto{\pgfqpoint{1.612940in}{0.566715in}}%
\pgfpathlineto{\pgfqpoint{1.627498in}{0.576759in}}%
\pgfpathlineto{\pgfqpoint{1.628597in}{0.577598in}}%
\pgfpathlineto{\pgfqpoint{1.643288in}{0.590370in}}%
\pgfpathlineto{\pgfqpoint{1.644253in}{0.591325in}}%
\pgfpathlineto{\pgfqpoint{1.655806in}{0.603981in}}%
\pgfpathlineto{\pgfqpoint{1.659910in}{0.609277in}}%
\pgfpathlineto{\pgfqpoint{1.665873in}{0.617592in}}%
\pgfpathlineto{\pgfqpoint{1.673816in}{0.631203in}}%
\pgfpathlineto{\pgfqpoint{1.675567in}{0.635103in}}%
\pgfpathlineto{\pgfqpoint{1.679701in}{0.644814in}}%
\pgfpathlineto{\pgfqpoint{1.683720in}{0.658425in}}%
\pgfpathlineto{\pgfqpoint{1.685952in}{0.672036in}}%
\pgfpathlineto{\pgfqpoint{1.686398in}{0.685648in}}%
\pgfpathlineto{\pgfqpoint{1.685059in}{0.699259in}}%
\pgfpathlineto{\pgfqpoint{1.681934in}{0.712870in}}%
\pgfpathlineto{\pgfqpoint{1.677020in}{0.726481in}}%
\pgfpathlineto{\pgfqpoint{1.675567in}{0.729455in}}%
\pgfpathlineto{\pgfqpoint{1.670079in}{0.740092in}}%
\pgfpathlineto{\pgfqpoint{1.661198in}{0.753703in}}%
\pgfpathlineto{\pgfqpoint{1.659910in}{0.755357in}}%
\pgfpathlineto{\pgfqpoint{1.649834in}{0.767314in}}%
\pgfpathlineto{\pgfqpoint{1.644253in}{0.773032in}}%
\pgfpathlineto{\pgfqpoint{1.635716in}{0.780925in}}%
\pgfpathlineto{\pgfqpoint{1.628597in}{0.786786in}}%
\pgfpathlineto{\pgfqpoint{1.617866in}{0.794536in}}%
\pgfpathlineto{\pgfqpoint{1.612940in}{0.797787in}}%
\pgfpathlineto{\pgfqpoint{1.597284in}{0.806571in}}%
\pgfpathlineto{\pgfqpoint{1.593843in}{0.808148in}}%
\pgfpathlineto{\pgfqpoint{1.581627in}{0.813394in}}%
\pgfpathlineto{\pgfqpoint{1.565971in}{0.818544in}}%
\pgfpathlineto{\pgfqpoint{1.551854in}{0.821759in}}%
\pgfpathlineto{\pgfqpoint{1.550314in}{0.822095in}}%
\pgfpathlineto{\pgfqpoint{1.534657in}{0.824006in}}%
\pgfpathlineto{\pgfqpoint{1.519001in}{0.824388in}}%
\pgfpathlineto{\pgfqpoint{1.503344in}{0.823242in}}%
\pgfpathlineto{\pgfqpoint{1.494647in}{0.821759in}}%
\pgfpathlineto{\pgfqpoint{1.487688in}{0.820524in}}%
\pgfpathlineto{\pgfqpoint{1.472031in}{0.816168in}}%
\pgfpathlineto{\pgfqpoint{1.456375in}{0.810222in}}%
\pgfpathlineto{\pgfqpoint{1.452024in}{0.808148in}}%
\pgfpathlineto{\pgfqpoint{1.440718in}{0.802388in}}%
\pgfpathlineto{\pgfqpoint{1.427903in}{0.794536in}}%
\pgfpathlineto{\pgfqpoint{1.425061in}{0.792633in}}%
\pgfpathlineto{\pgfqpoint{1.409953in}{0.780925in}}%
\pgfpathlineto{\pgfqpoint{1.409405in}{0.780449in}}%
\pgfpathlineto{\pgfqpoint{1.395938in}{0.767314in}}%
\pgfpathlineto{\pgfqpoint{1.393748in}{0.764844in}}%
\pgfpathlineto{\pgfqpoint{1.384716in}{0.753703in}}%
\pgfpathlineto{\pgfqpoint{1.378092in}{0.743874in}}%
\pgfpathlineto{\pgfqpoint{1.375706in}{0.740092in}}%
\pgfpathlineto{\pgfqpoint{1.368866in}{0.726481in}}%
\pgfpathlineto{\pgfqpoint{1.363855in}{0.712870in}}%
\pgfpathlineto{\pgfqpoint{1.362435in}{0.706820in}}%
\pgfpathlineto{\pgfqpoint{1.360729in}{0.699259in}}%
\pgfpathlineto{\pgfqpoint{1.359411in}{0.685648in}}%
\pgfpathlineto{\pgfqpoint{1.359850in}{0.672036in}}%
\pgfpathlineto{\pgfqpoint{1.362048in}{0.658425in}}%
\pgfpathlineto{\pgfqpoint{1.362435in}{0.657087in}}%
\pgfpathlineto{\pgfqpoint{1.366133in}{0.644814in}}%
\pgfpathlineto{\pgfqpoint{1.372057in}{0.631203in}}%
\pgfpathlineto{\pgfqpoint{1.378092in}{0.620583in}}%
\pgfpathlineto{\pgfqpoint{1.379905in}{0.617592in}}%
\pgfpathlineto{\pgfqpoint{1.390009in}{0.603981in}}%
\pgfpathlineto{\pgfqpoint{1.393748in}{0.599699in}}%
\pgfpathlineto{\pgfqpoint{1.402664in}{0.590370in}}%
\pgfpathlineto{\pgfqpoint{1.409405in}{0.584180in}}%
\pgfpathlineto{\pgfqpoint{1.418484in}{0.576759in}}%
\pgfpathlineto{\pgfqpoint{1.425061in}{0.571907in}}%
\pgfpathlineto{\pgfqpoint{1.438816in}{0.563148in}}%
\pgfpathlineto{\pgfqpoint{1.440718in}{0.562028in}}%
\pgfpathlineto{\pgfqpoint{1.456375in}{0.554307in}}%
\pgfpathlineto{\pgfqpoint{1.468610in}{0.549536in}}%
\pgfpathlineto{\pgfqpoint{1.472031in}{0.548273in}}%
\pgfpathclose%
\pgfpathmoveto{\pgfqpoint{1.480582in}{0.617592in}}%
\pgfpathlineto{\pgfqpoint{1.472031in}{0.622459in}}%
\pgfpathlineto{\pgfqpoint{1.460700in}{0.631203in}}%
\pgfpathlineto{\pgfqpoint{1.456375in}{0.635525in}}%
\pgfpathlineto{\pgfqpoint{1.448931in}{0.644814in}}%
\pgfpathlineto{\pgfqpoint{1.441363in}{0.658425in}}%
\pgfpathlineto{\pgfqpoint{1.440718in}{0.660510in}}%
\pgfpathlineto{\pgfqpoint{1.437693in}{0.672036in}}%
\pgfpathlineto{\pgfqpoint{1.436978in}{0.685648in}}%
\pgfpathlineto{\pgfqpoint{1.439122in}{0.699259in}}%
\pgfpathlineto{\pgfqpoint{1.440718in}{0.703613in}}%
\pgfpathlineto{\pgfqpoint{1.444726in}{0.712870in}}%
\pgfpathlineto{\pgfqpoint{1.453979in}{0.726481in}}%
\pgfpathlineto{\pgfqpoint{1.456375in}{0.729086in}}%
\pgfpathlineto{\pgfqpoint{1.469035in}{0.740092in}}%
\pgfpathlineto{\pgfqpoint{1.472031in}{0.742175in}}%
\pgfpathlineto{\pgfqpoint{1.487688in}{0.750219in}}%
\pgfpathlineto{\pgfqpoint{1.498336in}{0.753703in}}%
\pgfpathlineto{\pgfqpoint{1.503344in}{0.755090in}}%
\pgfpathlineto{\pgfqpoint{1.519001in}{0.756954in}}%
\pgfpathlineto{\pgfqpoint{1.534657in}{0.756333in}}%
\pgfpathlineto{\pgfqpoint{1.547916in}{0.753703in}}%
\pgfpathlineto{\pgfqpoint{1.550314in}{0.753142in}}%
\pgfpathlineto{\pgfqpoint{1.565971in}{0.746563in}}%
\pgfpathlineto{\pgfqpoint{1.576656in}{0.740092in}}%
\pgfpathlineto{\pgfqpoint{1.581627in}{0.736332in}}%
\pgfpathlineto{\pgfqpoint{1.591685in}{0.726481in}}%
\pgfpathlineto{\pgfqpoint{1.597284in}{0.719047in}}%
\pgfpathlineto{\pgfqpoint{1.601120in}{0.712870in}}%
\pgfpathlineto{\pgfqpoint{1.606518in}{0.699259in}}%
\pgfpathlineto{\pgfqpoint{1.608830in}{0.685648in}}%
\pgfpathlineto{\pgfqpoint{1.608059in}{0.672036in}}%
\pgfpathlineto{\pgfqpoint{1.604205in}{0.658425in}}%
\pgfpathlineto{\pgfqpoint{1.597284in}{0.644857in}}%
\pgfpathlineto{\pgfqpoint{1.597257in}{0.644814in}}%
\pgfpathlineto{\pgfqpoint{1.585182in}{0.631203in}}%
\pgfpathlineto{\pgfqpoint{1.581627in}{0.628113in}}%
\pgfpathlineto{\pgfqpoint{1.565971in}{0.617615in}}%
\pgfpathlineto{\pgfqpoint{1.565921in}{0.617592in}}%
\pgfpathlineto{\pgfqpoint{1.550314in}{0.611575in}}%
\pgfpathlineto{\pgfqpoint{1.534657in}{0.608224in}}%
\pgfpathlineto{\pgfqpoint{1.519001in}{0.607554in}}%
\pgfpathlineto{\pgfqpoint{1.503344in}{0.609564in}}%
\pgfpathlineto{\pgfqpoint{1.487688in}{0.614257in}}%
\pgfpathlineto{\pgfqpoint{1.480582in}{0.617592in}}%
\pgfpathclose%
\pgfpathmoveto{\pgfqpoint{0.720516in}{1.216144in}}%
\pgfpathlineto{\pgfqpoint{0.736173in}{1.214234in}}%
\pgfpathlineto{\pgfqpoint{0.751829in}{1.213852in}}%
\pgfpathlineto{\pgfqpoint{0.767486in}{1.214998in}}%
\pgfpathlineto{\pgfqpoint{0.776183in}{1.216481in}}%
\pgfpathlineto{\pgfqpoint{0.783142in}{1.217716in}}%
\pgfpathlineto{\pgfqpoint{0.798799in}{1.222072in}}%
\pgfpathlineto{\pgfqpoint{0.814455in}{1.228018in}}%
\pgfpathlineto{\pgfqpoint{0.818806in}{1.230092in}}%
\pgfpathlineto{\pgfqpoint{0.830112in}{1.235851in}}%
\pgfpathlineto{\pgfqpoint{0.842927in}{1.243703in}}%
\pgfpathlineto{\pgfqpoint{0.845769in}{1.245606in}}%
\pgfpathlineto{\pgfqpoint{0.860877in}{1.257314in}}%
\pgfpathlineto{\pgfqpoint{0.861425in}{1.257791in}}%
\pgfpathlineto{\pgfqpoint{0.874892in}{1.270925in}}%
\pgfpathlineto{\pgfqpoint{0.877082in}{1.273396in}}%
\pgfpathlineto{\pgfqpoint{0.886114in}{1.284536in}}%
\pgfpathlineto{\pgfqpoint{0.892738in}{1.294365in}}%
\pgfpathlineto{\pgfqpoint{0.895124in}{1.298148in}}%
\pgfpathlineto{\pgfqpoint{0.901964in}{1.311759in}}%
\pgfpathlineto{\pgfqpoint{0.906975in}{1.325370in}}%
\pgfpathlineto{\pgfqpoint{0.908395in}{1.331420in}}%
\pgfpathlineto{\pgfqpoint{0.910101in}{1.338981in}}%
\pgfpathlineto{\pgfqpoint{0.911419in}{1.352592in}}%
\pgfpathlineto{\pgfqpoint{0.910980in}{1.366203in}}%
\pgfpathlineto{\pgfqpoint{0.908782in}{1.379814in}}%
\pgfpathlineto{\pgfqpoint{0.908395in}{1.381153in}}%
\pgfpathlineto{\pgfqpoint{0.904697in}{1.393425in}}%
\pgfpathlineto{\pgfqpoint{0.898773in}{1.407036in}}%
\pgfpathlineto{\pgfqpoint{0.892738in}{1.417656in}}%
\pgfpathlineto{\pgfqpoint{0.890925in}{1.420648in}}%
\pgfpathlineto{\pgfqpoint{0.880821in}{1.434259in}}%
\pgfpathlineto{\pgfqpoint{0.877082in}{1.438541in}}%
\pgfpathlineto{\pgfqpoint{0.868166in}{1.447870in}}%
\pgfpathlineto{\pgfqpoint{0.861425in}{1.454059in}}%
\pgfpathlineto{\pgfqpoint{0.852346in}{1.461481in}}%
\pgfpathlineto{\pgfqpoint{0.845769in}{1.466333in}}%
\pgfpathlineto{\pgfqpoint{0.832014in}{1.475092in}}%
\pgfpathlineto{\pgfqpoint{0.830112in}{1.476212in}}%
\pgfpathlineto{\pgfqpoint{0.814455in}{1.483932in}}%
\pgfpathlineto{\pgfqpoint{0.802220in}{1.488703in}}%
\pgfpathlineto{\pgfqpoint{0.798799in}{1.489967in}}%
\pgfpathlineto{\pgfqpoint{0.783142in}{1.494239in}}%
\pgfpathlineto{\pgfqpoint{0.767486in}{1.496956in}}%
\pgfpathlineto{\pgfqpoint{0.751829in}{1.498119in}}%
\pgfpathlineto{\pgfqpoint{0.736173in}{1.497731in}}%
\pgfpathlineto{\pgfqpoint{0.720516in}{1.495792in}}%
\pgfpathlineto{\pgfqpoint{0.704859in}{1.492298in}}%
\pgfpathlineto{\pgfqpoint{0.693689in}{1.488703in}}%
\pgfpathlineto{\pgfqpoint{0.689203in}{1.487181in}}%
\pgfpathlineto{\pgfqpoint{0.673546in}{1.480276in}}%
\pgfpathlineto{\pgfqpoint{0.663982in}{1.475092in}}%
\pgfpathlineto{\pgfqpoint{0.657890in}{1.471525in}}%
\pgfpathlineto{\pgfqpoint{0.643332in}{1.461481in}}%
\pgfpathlineto{\pgfqpoint{0.642233in}{1.460642in}}%
\pgfpathlineto{\pgfqpoint{0.627542in}{1.447870in}}%
\pgfpathlineto{\pgfqpoint{0.626577in}{1.446914in}}%
\pgfpathlineto{\pgfqpoint{0.615024in}{1.434259in}}%
\pgfpathlineto{\pgfqpoint{0.610920in}{1.428962in}}%
\pgfpathlineto{\pgfqpoint{0.604957in}{1.420648in}}%
\pgfpathlineto{\pgfqpoint{0.597014in}{1.407036in}}%
\pgfpathlineto{\pgfqpoint{0.595263in}{1.403136in}}%
\pgfpathlineto{\pgfqpoint{0.591129in}{1.393425in}}%
\pgfpathlineto{\pgfqpoint{0.587110in}{1.379814in}}%
\pgfpathlineto{\pgfqpoint{0.584878in}{1.366203in}}%
\pgfpathlineto{\pgfqpoint{0.584432in}{1.352592in}}%
\pgfpathlineto{\pgfqpoint{0.585771in}{1.338981in}}%
\pgfpathlineto{\pgfqpoint{0.588896in}{1.325370in}}%
\pgfpathlineto{\pgfqpoint{0.593810in}{1.311759in}}%
\pgfpathlineto{\pgfqpoint{0.595263in}{1.308784in}}%
\pgfpathlineto{\pgfqpoint{0.600751in}{1.298148in}}%
\pgfpathlineto{\pgfqpoint{0.609632in}{1.284536in}}%
\pgfpathlineto{\pgfqpoint{0.610920in}{1.282883in}}%
\pgfpathlineto{\pgfqpoint{0.620996in}{1.270925in}}%
\pgfpathlineto{\pgfqpoint{0.626577in}{1.265207in}}%
\pgfpathlineto{\pgfqpoint{0.635114in}{1.257314in}}%
\pgfpathlineto{\pgfqpoint{0.642233in}{1.251454in}}%
\pgfpathlineto{\pgfqpoint{0.652964in}{1.243703in}}%
\pgfpathlineto{\pgfqpoint{0.657890in}{1.240453in}}%
\pgfpathlineto{\pgfqpoint{0.673546in}{1.231669in}}%
\pgfpathlineto{\pgfqpoint{0.676987in}{1.230092in}}%
\pgfpathlineto{\pgfqpoint{0.689203in}{1.224846in}}%
\pgfpathlineto{\pgfqpoint{0.704859in}{1.219695in}}%
\pgfpathlineto{\pgfqpoint{0.718976in}{1.216481in}}%
\pgfpathlineto{\pgfqpoint{0.720516in}{1.216144in}}%
\pgfpathclose%
\pgfpathmoveto{\pgfqpoint{0.722914in}{1.284536in}}%
\pgfpathlineto{\pgfqpoint{0.720516in}{1.285097in}}%
\pgfpathlineto{\pgfqpoint{0.704859in}{1.291676in}}%
\pgfpathlineto{\pgfqpoint{0.694174in}{1.298148in}}%
\pgfpathlineto{\pgfqpoint{0.689203in}{1.301908in}}%
\pgfpathlineto{\pgfqpoint{0.679145in}{1.311759in}}%
\pgfpathlineto{\pgfqpoint{0.673546in}{1.319193in}}%
\pgfpathlineto{\pgfqpoint{0.669710in}{1.325370in}}%
\pgfpathlineto{\pgfqpoint{0.664312in}{1.338981in}}%
\pgfpathlineto{\pgfqpoint{0.662000in}{1.352592in}}%
\pgfpathlineto{\pgfqpoint{0.662771in}{1.366203in}}%
\pgfpathlineto{\pgfqpoint{0.666625in}{1.379814in}}%
\pgfpathlineto{\pgfqpoint{0.673546in}{1.393382in}}%
\pgfpathlineto{\pgfqpoint{0.673573in}{1.393425in}}%
\pgfpathlineto{\pgfqpoint{0.685648in}{1.407036in}}%
\pgfpathlineto{\pgfqpoint{0.689203in}{1.410127in}}%
\pgfpathlineto{\pgfqpoint{0.704859in}{1.420624in}}%
\pgfpathlineto{\pgfqpoint{0.704909in}{1.420648in}}%
\pgfpathlineto{\pgfqpoint{0.720516in}{1.426664in}}%
\pgfpathlineto{\pgfqpoint{0.736173in}{1.430015in}}%
\pgfpathlineto{\pgfqpoint{0.751829in}{1.430685in}}%
\pgfpathlineto{\pgfqpoint{0.767486in}{1.428675in}}%
\pgfpathlineto{\pgfqpoint{0.783142in}{1.423982in}}%
\pgfpathlineto{\pgfqpoint{0.790248in}{1.420648in}}%
\pgfpathlineto{\pgfqpoint{0.798799in}{1.415781in}}%
\pgfpathlineto{\pgfqpoint{0.810130in}{1.407036in}}%
\pgfpathlineto{\pgfqpoint{0.814455in}{1.402714in}}%
\pgfpathlineto{\pgfqpoint{0.821899in}{1.393425in}}%
\pgfpathlineto{\pgfqpoint{0.829467in}{1.379814in}}%
\pgfpathlineto{\pgfqpoint{0.830112in}{1.377730in}}%
\pgfpathlineto{\pgfqpoint{0.833137in}{1.366203in}}%
\pgfpathlineto{\pgfqpoint{0.833852in}{1.352592in}}%
\pgfpathlineto{\pgfqpoint{0.831708in}{1.338981in}}%
\pgfpathlineto{\pgfqpoint{0.830112in}{1.334626in}}%
\pgfpathlineto{\pgfqpoint{0.826104in}{1.325370in}}%
\pgfpathlineto{\pgfqpoint{0.816851in}{1.311759in}}%
\pgfpathlineto{\pgfqpoint{0.814455in}{1.309154in}}%
\pgfpathlineto{\pgfqpoint{0.801795in}{1.298148in}}%
\pgfpathlineto{\pgfqpoint{0.798799in}{1.296065in}}%
\pgfpathlineto{\pgfqpoint{0.783142in}{1.288021in}}%
\pgfpathlineto{\pgfqpoint{0.772494in}{1.284536in}}%
\pgfpathlineto{\pgfqpoint{0.767486in}{1.283149in}}%
\pgfpathlineto{\pgfqpoint{0.751829in}{1.281285in}}%
\pgfpathlineto{\pgfqpoint{0.736173in}{1.281906in}}%
\pgfpathlineto{\pgfqpoint{0.722914in}{1.284536in}}%
\pgfpathclose%
\pgfpathmoveto{\pgfqpoint{1.503344in}{1.214998in}}%
\pgfpathlineto{\pgfqpoint{1.519001in}{1.213852in}}%
\pgfpathlineto{\pgfqpoint{1.534657in}{1.214234in}}%
\pgfpathlineto{\pgfqpoint{1.550314in}{1.216144in}}%
\pgfpathlineto{\pgfqpoint{1.551854in}{1.216481in}}%
\pgfpathlineto{\pgfqpoint{1.565971in}{1.219695in}}%
\pgfpathlineto{\pgfqpoint{1.581627in}{1.224846in}}%
\pgfpathlineto{\pgfqpoint{1.593843in}{1.230092in}}%
\pgfpathlineto{\pgfqpoint{1.597284in}{1.231669in}}%
\pgfpathlineto{\pgfqpoint{1.612940in}{1.240453in}}%
\pgfpathlineto{\pgfqpoint{1.617866in}{1.243703in}}%
\pgfpathlineto{\pgfqpoint{1.628597in}{1.251454in}}%
\pgfpathlineto{\pgfqpoint{1.635716in}{1.257314in}}%
\pgfpathlineto{\pgfqpoint{1.644253in}{1.265207in}}%
\pgfpathlineto{\pgfqpoint{1.649834in}{1.270925in}}%
\pgfpathlineto{\pgfqpoint{1.659910in}{1.282883in}}%
\pgfpathlineto{\pgfqpoint{1.661198in}{1.284536in}}%
\pgfpathlineto{\pgfqpoint{1.670079in}{1.298148in}}%
\pgfpathlineto{\pgfqpoint{1.675567in}{1.308784in}}%
\pgfpathlineto{\pgfqpoint{1.677020in}{1.311759in}}%
\pgfpathlineto{\pgfqpoint{1.681934in}{1.325370in}}%
\pgfpathlineto{\pgfqpoint{1.685059in}{1.338981in}}%
\pgfpathlineto{\pgfqpoint{1.686398in}{1.352592in}}%
\pgfpathlineto{\pgfqpoint{1.685952in}{1.366203in}}%
\pgfpathlineto{\pgfqpoint{1.683720in}{1.379814in}}%
\pgfpathlineto{\pgfqpoint{1.679701in}{1.393425in}}%
\pgfpathlineto{\pgfqpoint{1.675567in}{1.403136in}}%
\pgfpathlineto{\pgfqpoint{1.673816in}{1.407036in}}%
\pgfpathlineto{\pgfqpoint{1.665873in}{1.420648in}}%
\pgfpathlineto{\pgfqpoint{1.659910in}{1.428962in}}%
\pgfpathlineto{\pgfqpoint{1.655806in}{1.434259in}}%
\pgfpathlineto{\pgfqpoint{1.644253in}{1.446914in}}%
\pgfpathlineto{\pgfqpoint{1.643288in}{1.447870in}}%
\pgfpathlineto{\pgfqpoint{1.628597in}{1.460642in}}%
\pgfpathlineto{\pgfqpoint{1.627498in}{1.461481in}}%
\pgfpathlineto{\pgfqpoint{1.612940in}{1.471525in}}%
\pgfpathlineto{\pgfqpoint{1.606848in}{1.475092in}}%
\pgfpathlineto{\pgfqpoint{1.597284in}{1.480276in}}%
\pgfpathlineto{\pgfqpoint{1.581627in}{1.487181in}}%
\pgfpathlineto{\pgfqpoint{1.577141in}{1.488703in}}%
\pgfpathlineto{\pgfqpoint{1.565971in}{1.492298in}}%
\pgfpathlineto{\pgfqpoint{1.550314in}{1.495792in}}%
\pgfpathlineto{\pgfqpoint{1.534657in}{1.497731in}}%
\pgfpathlineto{\pgfqpoint{1.519001in}{1.498119in}}%
\pgfpathlineto{\pgfqpoint{1.503344in}{1.496956in}}%
\pgfpathlineto{\pgfqpoint{1.487688in}{1.494239in}}%
\pgfpathlineto{\pgfqpoint{1.472031in}{1.489967in}}%
\pgfpathlineto{\pgfqpoint{1.468610in}{1.488703in}}%
\pgfpathlineto{\pgfqpoint{1.456375in}{1.483932in}}%
\pgfpathlineto{\pgfqpoint{1.440718in}{1.476212in}}%
\pgfpathlineto{\pgfqpoint{1.438816in}{1.475092in}}%
\pgfpathlineto{\pgfqpoint{1.425061in}{1.466333in}}%
\pgfpathlineto{\pgfqpoint{1.418484in}{1.461481in}}%
\pgfpathlineto{\pgfqpoint{1.409405in}{1.454059in}}%
\pgfpathlineto{\pgfqpoint{1.402664in}{1.447870in}}%
\pgfpathlineto{\pgfqpoint{1.393748in}{1.438541in}}%
\pgfpathlineto{\pgfqpoint{1.390009in}{1.434259in}}%
\pgfpathlineto{\pgfqpoint{1.379905in}{1.420648in}}%
\pgfpathlineto{\pgfqpoint{1.378092in}{1.417656in}}%
\pgfpathlineto{\pgfqpoint{1.372057in}{1.407036in}}%
\pgfpathlineto{\pgfqpoint{1.366133in}{1.393425in}}%
\pgfpathlineto{\pgfqpoint{1.362435in}{1.381153in}}%
\pgfpathlineto{\pgfqpoint{1.362048in}{1.379814in}}%
\pgfpathlineto{\pgfqpoint{1.359850in}{1.366203in}}%
\pgfpathlineto{\pgfqpoint{1.359411in}{1.352592in}}%
\pgfpathlineto{\pgfqpoint{1.360729in}{1.338981in}}%
\pgfpathlineto{\pgfqpoint{1.362435in}{1.331420in}}%
\pgfpathlineto{\pgfqpoint{1.363855in}{1.325370in}}%
\pgfpathlineto{\pgfqpoint{1.368866in}{1.311759in}}%
\pgfpathlineto{\pgfqpoint{1.375706in}{1.298148in}}%
\pgfpathlineto{\pgfqpoint{1.378092in}{1.294365in}}%
\pgfpathlineto{\pgfqpoint{1.384716in}{1.284536in}}%
\pgfpathlineto{\pgfqpoint{1.393748in}{1.273396in}}%
\pgfpathlineto{\pgfqpoint{1.395938in}{1.270925in}}%
\pgfpathlineto{\pgfqpoint{1.409405in}{1.257791in}}%
\pgfpathlineto{\pgfqpoint{1.409953in}{1.257314in}}%
\pgfpathlineto{\pgfqpoint{1.425061in}{1.245606in}}%
\pgfpathlineto{\pgfqpoint{1.427903in}{1.243703in}}%
\pgfpathlineto{\pgfqpoint{1.440718in}{1.235851in}}%
\pgfpathlineto{\pgfqpoint{1.452024in}{1.230092in}}%
\pgfpathlineto{\pgfqpoint{1.456375in}{1.228018in}}%
\pgfpathlineto{\pgfqpoint{1.472031in}{1.222072in}}%
\pgfpathlineto{\pgfqpoint{1.487688in}{1.217716in}}%
\pgfpathlineto{\pgfqpoint{1.494647in}{1.216481in}}%
\pgfpathlineto{\pgfqpoint{1.503344in}{1.214998in}}%
\pgfpathclose%
\pgfpathmoveto{\pgfqpoint{1.498336in}{1.284536in}}%
\pgfpathlineto{\pgfqpoint{1.487688in}{1.288021in}}%
\pgfpathlineto{\pgfqpoint{1.472031in}{1.296065in}}%
\pgfpathlineto{\pgfqpoint{1.469035in}{1.298148in}}%
\pgfpathlineto{\pgfqpoint{1.456375in}{1.309154in}}%
\pgfpathlineto{\pgfqpoint{1.453979in}{1.311759in}}%
\pgfpathlineto{\pgfqpoint{1.444726in}{1.325370in}}%
\pgfpathlineto{\pgfqpoint{1.440718in}{1.334626in}}%
\pgfpathlineto{\pgfqpoint{1.439122in}{1.338981in}}%
\pgfpathlineto{\pgfqpoint{1.436978in}{1.352592in}}%
\pgfpathlineto{\pgfqpoint{1.437693in}{1.366203in}}%
\pgfpathlineto{\pgfqpoint{1.440718in}{1.377730in}}%
\pgfpathlineto{\pgfqpoint{1.441363in}{1.379814in}}%
\pgfpathlineto{\pgfqpoint{1.448931in}{1.393425in}}%
\pgfpathlineto{\pgfqpoint{1.456375in}{1.402714in}}%
\pgfpathlineto{\pgfqpoint{1.460700in}{1.407036in}}%
\pgfpathlineto{\pgfqpoint{1.472031in}{1.415781in}}%
\pgfpathlineto{\pgfqpoint{1.480582in}{1.420648in}}%
\pgfpathlineto{\pgfqpoint{1.487688in}{1.423982in}}%
\pgfpathlineto{\pgfqpoint{1.503344in}{1.428675in}}%
\pgfpathlineto{\pgfqpoint{1.519001in}{1.430685in}}%
\pgfpathlineto{\pgfqpoint{1.534657in}{1.430015in}}%
\pgfpathlineto{\pgfqpoint{1.550314in}{1.426664in}}%
\pgfpathlineto{\pgfqpoint{1.565921in}{1.420648in}}%
\pgfpathlineto{\pgfqpoint{1.565971in}{1.420624in}}%
\pgfpathlineto{\pgfqpoint{1.581627in}{1.410127in}}%
\pgfpathlineto{\pgfqpoint{1.585182in}{1.407036in}}%
\pgfpathlineto{\pgfqpoint{1.597257in}{1.393425in}}%
\pgfpathlineto{\pgfqpoint{1.597284in}{1.393382in}}%
\pgfpathlineto{\pgfqpoint{1.604205in}{1.379814in}}%
\pgfpathlineto{\pgfqpoint{1.608059in}{1.366203in}}%
\pgfpathlineto{\pgfqpoint{1.608830in}{1.352592in}}%
\pgfpathlineto{\pgfqpoint{1.606518in}{1.338981in}}%
\pgfpathlineto{\pgfqpoint{1.601120in}{1.325370in}}%
\pgfpathlineto{\pgfqpoint{1.597284in}{1.319193in}}%
\pgfpathlineto{\pgfqpoint{1.591685in}{1.311759in}}%
\pgfpathlineto{\pgfqpoint{1.581627in}{1.301908in}}%
\pgfpathlineto{\pgfqpoint{1.576656in}{1.298148in}}%
\pgfpathlineto{\pgfqpoint{1.565971in}{1.291676in}}%
\pgfpathlineto{\pgfqpoint{1.550314in}{1.285097in}}%
\pgfpathlineto{\pgfqpoint{1.547916in}{1.284536in}}%
\pgfpathlineto{\pgfqpoint{1.534657in}{1.281906in}}%
\pgfpathlineto{\pgfqpoint{1.519001in}{1.281285in}}%
\pgfpathlineto{\pgfqpoint{1.503344in}{1.283149in}}%
\pgfpathlineto{\pgfqpoint{1.498336in}{1.284536in}}%
\pgfpathclose%
\pgfusepath{fill}%
\end{pgfscope}%
\begin{pgfscope}%
\pgfpathrectangle{\pgfqpoint{0.360415in}{0.345370in}}{\pgfqpoint{1.550000in}{1.347500in}}%
\pgfusepath{clip}%
\pgfsetbuttcap%
\pgfsetroundjoin%
\definecolor{currentfill}{rgb}{0.921884,0.341098,0.377376}%
\pgfsetfillcolor{currentfill}%
\pgfsetlinewidth{0.000000pt}%
\definecolor{currentstroke}{rgb}{0.000000,0.000000,0.000000}%
\pgfsetstrokecolor{currentstroke}%
\pgfsetdash{}{0pt}%
\pgfpathmoveto{\pgfqpoint{0.689203in}{0.492698in}}%
\pgfpathlineto{\pgfqpoint{0.704859in}{0.488232in}}%
\pgfpathlineto{\pgfqpoint{0.720516in}{0.485143in}}%
\pgfpathlineto{\pgfqpoint{0.736173in}{0.483428in}}%
\pgfpathlineto{\pgfqpoint{0.751829in}{0.483085in}}%
\pgfpathlineto{\pgfqpoint{0.767486in}{0.484114in}}%
\pgfpathlineto{\pgfqpoint{0.783142in}{0.486516in}}%
\pgfpathlineto{\pgfqpoint{0.798799in}{0.490293in}}%
\pgfpathlineto{\pgfqpoint{0.813384in}{0.495092in}}%
\pgfpathlineto{\pgfqpoint{0.814455in}{0.495440in}}%
\pgfpathlineto{\pgfqpoint{0.830112in}{0.501827in}}%
\pgfpathlineto{\pgfqpoint{0.844056in}{0.508703in}}%
\pgfpathlineto{\pgfqpoint{0.845769in}{0.509555in}}%
\pgfpathlineto{\pgfqpoint{0.861425in}{0.518568in}}%
\pgfpathlineto{\pgfqpoint{0.867181in}{0.522314in}}%
\pgfpathlineto{\pgfqpoint{0.877082in}{0.528956in}}%
\pgfpathlineto{\pgfqpoint{0.886480in}{0.535925in}}%
\pgfpathlineto{\pgfqpoint{0.892738in}{0.540798in}}%
\pgfpathlineto{\pgfqpoint{0.903113in}{0.549536in}}%
\pgfpathlineto{\pgfqpoint{0.908395in}{0.554300in}}%
\pgfpathlineto{\pgfqpoint{0.917648in}{0.563148in}}%
\pgfpathlineto{\pgfqpoint{0.924051in}{0.569839in}}%
\pgfpathlineto{\pgfqpoint{0.930414in}{0.576759in}}%
\pgfpathlineto{\pgfqpoint{0.939708in}{0.588060in}}%
\pgfpathlineto{\pgfqpoint{0.941568in}{0.590370in}}%
\pgfpathlineto{\pgfqpoint{0.951184in}{0.603981in}}%
\pgfpathlineto{\pgfqpoint{0.955364in}{0.610973in}}%
\pgfpathlineto{\pgfqpoint{0.959321in}{0.617592in}}%
\pgfpathlineto{\pgfqpoint{0.965954in}{0.631203in}}%
\pgfpathlineto{\pgfqpoint{0.971021in}{0.644812in}}%
\pgfpathlineto{\pgfqpoint{0.971022in}{0.644814in}}%
\pgfpathlineto{\pgfqpoint{0.974639in}{0.658425in}}%
\pgfpathlineto{\pgfqpoint{0.976647in}{0.672036in}}%
\pgfpathlineto{\pgfqpoint{0.977049in}{0.685648in}}%
\pgfpathlineto{\pgfqpoint{0.975844in}{0.699259in}}%
\pgfpathlineto{\pgfqpoint{0.973032in}{0.712870in}}%
\pgfpathlineto{\pgfqpoint{0.971021in}{0.719085in}}%
\pgfpathlineto{\pgfqpoint{0.968684in}{0.726481in}}%
\pgfpathlineto{\pgfqpoint{0.962833in}{0.740092in}}%
\pgfpathlineto{\pgfqpoint{0.955417in}{0.753703in}}%
\pgfpathlineto{\pgfqpoint{0.955364in}{0.753784in}}%
\pgfpathlineto{\pgfqpoint{0.946568in}{0.767314in}}%
\pgfpathlineto{\pgfqpoint{0.939708in}{0.776365in}}%
\pgfpathlineto{\pgfqpoint{0.936182in}{0.780925in}}%
\pgfpathlineto{\pgfqpoint{0.924272in}{0.794536in}}%
\pgfpathlineto{\pgfqpoint{0.924051in}{0.794766in}}%
\pgfpathlineto{\pgfqpoint{0.910651in}{0.808148in}}%
\pgfpathlineto{\pgfqpoint{0.908395in}{0.810229in}}%
\pgfpathlineto{\pgfqpoint{0.895132in}{0.821759in}}%
\pgfpathlineto{\pgfqpoint{0.892738in}{0.823720in}}%
\pgfpathlineto{\pgfqpoint{0.877346in}{0.835370in}}%
\pgfpathlineto{\pgfqpoint{0.877082in}{0.835562in}}%
\pgfpathlineto{\pgfqpoint{0.861425in}{0.845915in}}%
\pgfpathlineto{\pgfqpoint{0.856180in}{0.848981in}}%
\pgfpathlineto{\pgfqpoint{0.845769in}{0.854944in}}%
\pgfpathlineto{\pgfqpoint{0.830205in}{0.862592in}}%
\pgfpathlineto{\pgfqpoint{0.830112in}{0.862638in}}%
\pgfpathlineto{\pgfqpoint{0.814455in}{0.869085in}}%
\pgfpathlineto{\pgfqpoint{0.798799in}{0.874171in}}%
\pgfpathlineto{\pgfqpoint{0.790291in}{0.876203in}}%
\pgfpathlineto{\pgfqpoint{0.783142in}{0.877951in}}%
\pgfpathlineto{\pgfqpoint{0.767486in}{0.880396in}}%
\pgfpathlineto{\pgfqpoint{0.751829in}{0.881443in}}%
\pgfpathlineto{\pgfqpoint{0.736173in}{0.881094in}}%
\pgfpathlineto{\pgfqpoint{0.720516in}{0.879349in}}%
\pgfpathlineto{\pgfqpoint{0.704859in}{0.876204in}}%
\pgfpathlineto{\pgfqpoint{0.704857in}{0.876203in}}%
\pgfpathlineto{\pgfqpoint{0.689203in}{0.871798in}}%
\pgfpathlineto{\pgfqpoint{0.673546in}{0.866032in}}%
\pgfpathlineto{\pgfqpoint{0.665933in}{0.862592in}}%
\pgfpathlineto{\pgfqpoint{0.657890in}{0.858958in}}%
\pgfpathlineto{\pgfqpoint{0.642233in}{0.850598in}}%
\pgfpathlineto{\pgfqpoint{0.639576in}{0.848981in}}%
\pgfpathlineto{\pgfqpoint{0.626577in}{0.840901in}}%
\pgfpathlineto{\pgfqpoint{0.618618in}{0.835370in}}%
\pgfpathlineto{\pgfqpoint{0.610920in}{0.829803in}}%
\pgfpathlineto{\pgfqpoint{0.600742in}{0.821759in}}%
\pgfpathlineto{\pgfqpoint{0.595263in}{0.817167in}}%
\pgfpathlineto{\pgfqpoint{0.585212in}{0.808148in}}%
\pgfpathlineto{\pgfqpoint{0.579607in}{0.802707in}}%
\pgfpathlineto{\pgfqpoint{0.571591in}{0.794536in}}%
\pgfpathlineto{\pgfqpoint{0.563950in}{0.785929in}}%
\pgfpathlineto{\pgfqpoint{0.559641in}{0.780925in}}%
\pgfpathlineto{\pgfqpoint{0.549274in}{0.767314in}}%
\pgfpathlineto{\pgfqpoint{0.548294in}{0.765826in}}%
\pgfpathlineto{\pgfqpoint{0.540384in}{0.753703in}}%
\pgfpathlineto{\pgfqpoint{0.533038in}{0.740092in}}%
\pgfpathlineto{\pgfqpoint{0.532637in}{0.739160in}}%
\pgfpathlineto{\pgfqpoint{0.527117in}{0.726481in}}%
\pgfpathlineto{\pgfqpoint{0.522772in}{0.712870in}}%
\pgfpathlineto{\pgfqpoint{0.520010in}{0.699259in}}%
\pgfpathlineto{\pgfqpoint{0.518826in}{0.685648in}}%
\pgfpathlineto{\pgfqpoint{0.519221in}{0.672036in}}%
\pgfpathlineto{\pgfqpoint{0.521193in}{0.658425in}}%
\pgfpathlineto{\pgfqpoint{0.524746in}{0.644814in}}%
\pgfpathlineto{\pgfqpoint{0.529884in}{0.631203in}}%
\pgfpathlineto{\pgfqpoint{0.532637in}{0.625589in}}%
\pgfpathlineto{\pgfqpoint{0.536517in}{0.617592in}}%
\pgfpathlineto{\pgfqpoint{0.544639in}{0.603981in}}%
\pgfpathlineto{\pgfqpoint{0.548294in}{0.598780in}}%
\pgfpathlineto{\pgfqpoint{0.554267in}{0.590370in}}%
\pgfpathlineto{\pgfqpoint{0.563950in}{0.578494in}}%
\pgfpathlineto{\pgfqpoint{0.565407in}{0.576759in}}%
\pgfpathlineto{\pgfqpoint{0.578147in}{0.563148in}}%
\pgfpathlineto{\pgfqpoint{0.579607in}{0.561718in}}%
\pgfpathlineto{\pgfqpoint{0.592683in}{0.549536in}}%
\pgfpathlineto{\pgfqpoint{0.595263in}{0.547293in}}%
\pgfpathlineto{\pgfqpoint{0.609275in}{0.535925in}}%
\pgfpathlineto{\pgfqpoint{0.610920in}{0.534656in}}%
\pgfpathlineto{\pgfqpoint{0.626577in}{0.523581in}}%
\pgfpathlineto{\pgfqpoint{0.628572in}{0.522314in}}%
\pgfpathlineto{\pgfqpoint{0.642233in}{0.513896in}}%
\pgfpathlineto{\pgfqpoint{0.651907in}{0.508703in}}%
\pgfpathlineto{\pgfqpoint{0.657890in}{0.505526in}}%
\pgfpathlineto{\pgfqpoint{0.673546in}{0.498465in}}%
\pgfpathlineto{\pgfqpoint{0.682745in}{0.495092in}}%
\pgfpathlineto{\pgfqpoint{0.689203in}{0.492698in}}%
\pgfpathclose%
\pgfpathmoveto{\pgfqpoint{0.693689in}{0.549536in}}%
\pgfpathlineto{\pgfqpoint{0.689203in}{0.551058in}}%
\pgfpathlineto{\pgfqpoint{0.673546in}{0.557963in}}%
\pgfpathlineto{\pgfqpoint{0.663982in}{0.563148in}}%
\pgfpathlineto{\pgfqpoint{0.657890in}{0.566715in}}%
\pgfpathlineto{\pgfqpoint{0.643332in}{0.576759in}}%
\pgfpathlineto{\pgfqpoint{0.642233in}{0.577598in}}%
\pgfpathlineto{\pgfqpoint{0.627542in}{0.590370in}}%
\pgfpathlineto{\pgfqpoint{0.626577in}{0.591325in}}%
\pgfpathlineto{\pgfqpoint{0.615024in}{0.603981in}}%
\pgfpathlineto{\pgfqpoint{0.610920in}{0.609277in}}%
\pgfpathlineto{\pgfqpoint{0.604957in}{0.617592in}}%
\pgfpathlineto{\pgfqpoint{0.597014in}{0.631203in}}%
\pgfpathlineto{\pgfqpoint{0.595263in}{0.635103in}}%
\pgfpathlineto{\pgfqpoint{0.591129in}{0.644814in}}%
\pgfpathlineto{\pgfqpoint{0.587110in}{0.658425in}}%
\pgfpathlineto{\pgfqpoint{0.584878in}{0.672036in}}%
\pgfpathlineto{\pgfqpoint{0.584432in}{0.685648in}}%
\pgfpathlineto{\pgfqpoint{0.585771in}{0.699259in}}%
\pgfpathlineto{\pgfqpoint{0.588896in}{0.712870in}}%
\pgfpathlineto{\pgfqpoint{0.593810in}{0.726481in}}%
\pgfpathlineto{\pgfqpoint{0.595263in}{0.729455in}}%
\pgfpathlineto{\pgfqpoint{0.600751in}{0.740092in}}%
\pgfpathlineto{\pgfqpoint{0.609632in}{0.753703in}}%
\pgfpathlineto{\pgfqpoint{0.610920in}{0.755357in}}%
\pgfpathlineto{\pgfqpoint{0.620996in}{0.767314in}}%
\pgfpathlineto{\pgfqpoint{0.626577in}{0.773032in}}%
\pgfpathlineto{\pgfqpoint{0.635114in}{0.780925in}}%
\pgfpathlineto{\pgfqpoint{0.642233in}{0.786786in}}%
\pgfpathlineto{\pgfqpoint{0.652964in}{0.794536in}}%
\pgfpathlineto{\pgfqpoint{0.657890in}{0.797787in}}%
\pgfpathlineto{\pgfqpoint{0.673546in}{0.806571in}}%
\pgfpathlineto{\pgfqpoint{0.676987in}{0.808148in}}%
\pgfpathlineto{\pgfqpoint{0.689203in}{0.813394in}}%
\pgfpathlineto{\pgfqpoint{0.704859in}{0.818544in}}%
\pgfpathlineto{\pgfqpoint{0.718976in}{0.821759in}}%
\pgfpathlineto{\pgfqpoint{0.720516in}{0.822095in}}%
\pgfpathlineto{\pgfqpoint{0.736173in}{0.824006in}}%
\pgfpathlineto{\pgfqpoint{0.751829in}{0.824388in}}%
\pgfpathlineto{\pgfqpoint{0.767486in}{0.823242in}}%
\pgfpathlineto{\pgfqpoint{0.776183in}{0.821759in}}%
\pgfpathlineto{\pgfqpoint{0.783142in}{0.820524in}}%
\pgfpathlineto{\pgfqpoint{0.798799in}{0.816168in}}%
\pgfpathlineto{\pgfqpoint{0.814455in}{0.810222in}}%
\pgfpathlineto{\pgfqpoint{0.818806in}{0.808148in}}%
\pgfpathlineto{\pgfqpoint{0.830112in}{0.802388in}}%
\pgfpathlineto{\pgfqpoint{0.842927in}{0.794536in}}%
\pgfpathlineto{\pgfqpoint{0.845769in}{0.792633in}}%
\pgfpathlineto{\pgfqpoint{0.860877in}{0.780925in}}%
\pgfpathlineto{\pgfqpoint{0.861425in}{0.780449in}}%
\pgfpathlineto{\pgfqpoint{0.874892in}{0.767314in}}%
\pgfpathlineto{\pgfqpoint{0.877082in}{0.764844in}}%
\pgfpathlineto{\pgfqpoint{0.886114in}{0.753703in}}%
\pgfpathlineto{\pgfqpoint{0.892738in}{0.743874in}}%
\pgfpathlineto{\pgfqpoint{0.895124in}{0.740092in}}%
\pgfpathlineto{\pgfqpoint{0.901964in}{0.726481in}}%
\pgfpathlineto{\pgfqpoint{0.906975in}{0.712870in}}%
\pgfpathlineto{\pgfqpoint{0.908395in}{0.706820in}}%
\pgfpathlineto{\pgfqpoint{0.910101in}{0.699259in}}%
\pgfpathlineto{\pgfqpoint{0.911419in}{0.685648in}}%
\pgfpathlineto{\pgfqpoint{0.910980in}{0.672036in}}%
\pgfpathlineto{\pgfqpoint{0.908782in}{0.658425in}}%
\pgfpathlineto{\pgfqpoint{0.908395in}{0.657087in}}%
\pgfpathlineto{\pgfqpoint{0.904697in}{0.644814in}}%
\pgfpathlineto{\pgfqpoint{0.898773in}{0.631203in}}%
\pgfpathlineto{\pgfqpoint{0.892738in}{0.620583in}}%
\pgfpathlineto{\pgfqpoint{0.890925in}{0.617592in}}%
\pgfpathlineto{\pgfqpoint{0.880821in}{0.603981in}}%
\pgfpathlineto{\pgfqpoint{0.877082in}{0.599699in}}%
\pgfpathlineto{\pgfqpoint{0.868166in}{0.590370in}}%
\pgfpathlineto{\pgfqpoint{0.861425in}{0.584180in}}%
\pgfpathlineto{\pgfqpoint{0.852346in}{0.576759in}}%
\pgfpathlineto{\pgfqpoint{0.845769in}{0.571907in}}%
\pgfpathlineto{\pgfqpoint{0.832014in}{0.563148in}}%
\pgfpathlineto{\pgfqpoint{0.830112in}{0.562028in}}%
\pgfpathlineto{\pgfqpoint{0.814455in}{0.554307in}}%
\pgfpathlineto{\pgfqpoint{0.802220in}{0.549536in}}%
\pgfpathlineto{\pgfqpoint{0.798799in}{0.548273in}}%
\pgfpathlineto{\pgfqpoint{0.783142in}{0.544000in}}%
\pgfpathlineto{\pgfqpoint{0.767486in}{0.541284in}}%
\pgfpathlineto{\pgfqpoint{0.751829in}{0.540120in}}%
\pgfpathlineto{\pgfqpoint{0.736173in}{0.540508in}}%
\pgfpathlineto{\pgfqpoint{0.720516in}{0.542448in}}%
\pgfpathlineto{\pgfqpoint{0.704859in}{0.545942in}}%
\pgfpathlineto{\pgfqpoint{0.693689in}{0.549536in}}%
\pgfpathclose%
\pgfpathmoveto{\pgfqpoint{1.472031in}{0.490293in}}%
\pgfpathlineto{\pgfqpoint{1.487688in}{0.486516in}}%
\pgfpathlineto{\pgfqpoint{1.503344in}{0.484114in}}%
\pgfpathlineto{\pgfqpoint{1.519001in}{0.483085in}}%
\pgfpathlineto{\pgfqpoint{1.534657in}{0.483428in}}%
\pgfpathlineto{\pgfqpoint{1.550314in}{0.485143in}}%
\pgfpathlineto{\pgfqpoint{1.565971in}{0.488232in}}%
\pgfpathlineto{\pgfqpoint{1.581627in}{0.492698in}}%
\pgfpathlineto{\pgfqpoint{1.588085in}{0.495092in}}%
\pgfpathlineto{\pgfqpoint{1.597284in}{0.498465in}}%
\pgfpathlineto{\pgfqpoint{1.612940in}{0.505526in}}%
\pgfpathlineto{\pgfqpoint{1.618923in}{0.508703in}}%
\pgfpathlineto{\pgfqpoint{1.628597in}{0.513896in}}%
\pgfpathlineto{\pgfqpoint{1.642258in}{0.522314in}}%
\pgfpathlineto{\pgfqpoint{1.644253in}{0.523581in}}%
\pgfpathlineto{\pgfqpoint{1.659910in}{0.534656in}}%
\pgfpathlineto{\pgfqpoint{1.661555in}{0.535925in}}%
\pgfpathlineto{\pgfqpoint{1.675567in}{0.547293in}}%
\pgfpathlineto{\pgfqpoint{1.678147in}{0.549536in}}%
\pgfpathlineto{\pgfqpoint{1.691223in}{0.561718in}}%
\pgfpathlineto{\pgfqpoint{1.692683in}{0.563148in}}%
\pgfpathlineto{\pgfqpoint{1.705423in}{0.576759in}}%
\pgfpathlineto{\pgfqpoint{1.706880in}{0.578494in}}%
\pgfpathlineto{\pgfqpoint{1.716563in}{0.590370in}}%
\pgfpathlineto{\pgfqpoint{1.722536in}{0.598780in}}%
\pgfpathlineto{\pgfqpoint{1.726191in}{0.603981in}}%
\pgfpathlineto{\pgfqpoint{1.734313in}{0.617592in}}%
\pgfpathlineto{\pgfqpoint{1.738193in}{0.625589in}}%
\pgfpathlineto{\pgfqpoint{1.740946in}{0.631203in}}%
\pgfpathlineto{\pgfqpoint{1.746084in}{0.644814in}}%
\pgfpathlineto{\pgfqpoint{1.749637in}{0.658425in}}%
\pgfpathlineto{\pgfqpoint{1.751609in}{0.672036in}}%
\pgfpathlineto{\pgfqpoint{1.752004in}{0.685648in}}%
\pgfpathlineto{\pgfqpoint{1.750820in}{0.699259in}}%
\pgfpathlineto{\pgfqpoint{1.748058in}{0.712870in}}%
\pgfpathlineto{\pgfqpoint{1.743713in}{0.726481in}}%
\pgfpathlineto{\pgfqpoint{1.738193in}{0.739160in}}%
\pgfpathlineto{\pgfqpoint{1.737792in}{0.740092in}}%
\pgfpathlineto{\pgfqpoint{1.730446in}{0.753703in}}%
\pgfpathlineto{\pgfqpoint{1.722536in}{0.765826in}}%
\pgfpathlineto{\pgfqpoint{1.721556in}{0.767314in}}%
\pgfpathlineto{\pgfqpoint{1.711189in}{0.780925in}}%
\pgfpathlineto{\pgfqpoint{1.706880in}{0.785929in}}%
\pgfpathlineto{\pgfqpoint{1.699239in}{0.794536in}}%
\pgfpathlineto{\pgfqpoint{1.691223in}{0.802707in}}%
\pgfpathlineto{\pgfqpoint{1.685618in}{0.808148in}}%
\pgfpathlineto{\pgfqpoint{1.675567in}{0.817167in}}%
\pgfpathlineto{\pgfqpoint{1.670088in}{0.821759in}}%
\pgfpathlineto{\pgfqpoint{1.659910in}{0.829803in}}%
\pgfpathlineto{\pgfqpoint{1.652212in}{0.835370in}}%
\pgfpathlineto{\pgfqpoint{1.644253in}{0.840901in}}%
\pgfpathlineto{\pgfqpoint{1.631254in}{0.848981in}}%
\pgfpathlineto{\pgfqpoint{1.628597in}{0.850598in}}%
\pgfpathlineto{\pgfqpoint{1.612940in}{0.858958in}}%
\pgfpathlineto{\pgfqpoint{1.604897in}{0.862592in}}%
\pgfpathlineto{\pgfqpoint{1.597284in}{0.866032in}}%
\pgfpathlineto{\pgfqpoint{1.581627in}{0.871798in}}%
\pgfpathlineto{\pgfqpoint{1.565973in}{0.876203in}}%
\pgfpathlineto{\pgfqpoint{1.565971in}{0.876204in}}%
\pgfpathlineto{\pgfqpoint{1.550314in}{0.879349in}}%
\pgfpathlineto{\pgfqpoint{1.534657in}{0.881094in}}%
\pgfpathlineto{\pgfqpoint{1.519001in}{0.881443in}}%
\pgfpathlineto{\pgfqpoint{1.503344in}{0.880396in}}%
\pgfpathlineto{\pgfqpoint{1.487688in}{0.877951in}}%
\pgfpathlineto{\pgfqpoint{1.480539in}{0.876203in}}%
\pgfpathlineto{\pgfqpoint{1.472031in}{0.874171in}}%
\pgfpathlineto{\pgfqpoint{1.456375in}{0.869085in}}%
\pgfpathlineto{\pgfqpoint{1.440718in}{0.862638in}}%
\pgfpathlineto{\pgfqpoint{1.440625in}{0.862592in}}%
\pgfpathlineto{\pgfqpoint{1.425061in}{0.854944in}}%
\pgfpathlineto{\pgfqpoint{1.414650in}{0.848981in}}%
\pgfpathlineto{\pgfqpoint{1.409405in}{0.845915in}}%
\pgfpathlineto{\pgfqpoint{1.393748in}{0.835562in}}%
\pgfpathlineto{\pgfqpoint{1.393484in}{0.835370in}}%
\pgfpathlineto{\pgfqpoint{1.378092in}{0.823720in}}%
\pgfpathlineto{\pgfqpoint{1.375698in}{0.821759in}}%
\pgfpathlineto{\pgfqpoint{1.362435in}{0.810229in}}%
\pgfpathlineto{\pgfqpoint{1.360179in}{0.808148in}}%
\pgfpathlineto{\pgfqpoint{1.346779in}{0.794766in}}%
\pgfpathlineto{\pgfqpoint{1.346558in}{0.794536in}}%
\pgfpathlineto{\pgfqpoint{1.334648in}{0.780925in}}%
\pgfpathlineto{\pgfqpoint{1.331122in}{0.776365in}}%
\pgfpathlineto{\pgfqpoint{1.324262in}{0.767314in}}%
\pgfpathlineto{\pgfqpoint{1.315466in}{0.753784in}}%
\pgfpathlineto{\pgfqpoint{1.315413in}{0.753703in}}%
\pgfpathlineto{\pgfqpoint{1.307997in}{0.740092in}}%
\pgfpathlineto{\pgfqpoint{1.302146in}{0.726481in}}%
\pgfpathlineto{\pgfqpoint{1.299809in}{0.719085in}}%
\pgfpathlineto{\pgfqpoint{1.297798in}{0.712870in}}%
\pgfpathlineto{\pgfqpoint{1.294986in}{0.699259in}}%
\pgfpathlineto{\pgfqpoint{1.293781in}{0.685648in}}%
\pgfpathlineto{\pgfqpoint{1.294183in}{0.672036in}}%
\pgfpathlineto{\pgfqpoint{1.296191in}{0.658425in}}%
\pgfpathlineto{\pgfqpoint{1.299808in}{0.644814in}}%
\pgfpathlineto{\pgfqpoint{1.299809in}{0.644812in}}%
\pgfpathlineto{\pgfqpoint{1.304876in}{0.631203in}}%
\pgfpathlineto{\pgfqpoint{1.311509in}{0.617592in}}%
\pgfpathlineto{\pgfqpoint{1.315466in}{0.610973in}}%
\pgfpathlineto{\pgfqpoint{1.319646in}{0.603981in}}%
\pgfpathlineto{\pgfqpoint{1.329262in}{0.590370in}}%
\pgfpathlineto{\pgfqpoint{1.331122in}{0.588060in}}%
\pgfpathlineto{\pgfqpoint{1.340416in}{0.576759in}}%
\pgfpathlineto{\pgfqpoint{1.346779in}{0.569839in}}%
\pgfpathlineto{\pgfqpoint{1.353182in}{0.563148in}}%
\pgfpathlineto{\pgfqpoint{1.362435in}{0.554300in}}%
\pgfpathlineto{\pgfqpoint{1.367717in}{0.549536in}}%
\pgfpathlineto{\pgfqpoint{1.378092in}{0.540798in}}%
\pgfpathlineto{\pgfqpoint{1.384350in}{0.535925in}}%
\pgfpathlineto{\pgfqpoint{1.393748in}{0.528956in}}%
\pgfpathlineto{\pgfqpoint{1.403649in}{0.522314in}}%
\pgfpathlineto{\pgfqpoint{1.409405in}{0.518568in}}%
\pgfpathlineto{\pgfqpoint{1.425061in}{0.509555in}}%
\pgfpathlineto{\pgfqpoint{1.426774in}{0.508703in}}%
\pgfpathlineto{\pgfqpoint{1.440718in}{0.501827in}}%
\pgfpathlineto{\pgfqpoint{1.456375in}{0.495440in}}%
\pgfpathlineto{\pgfqpoint{1.457446in}{0.495092in}}%
\pgfpathlineto{\pgfqpoint{1.472031in}{0.490293in}}%
\pgfpathclose%
\pgfpathmoveto{\pgfqpoint{1.468610in}{0.549536in}}%
\pgfpathlineto{\pgfqpoint{1.456375in}{0.554307in}}%
\pgfpathlineto{\pgfqpoint{1.440718in}{0.562028in}}%
\pgfpathlineto{\pgfqpoint{1.438816in}{0.563148in}}%
\pgfpathlineto{\pgfqpoint{1.425061in}{0.571907in}}%
\pgfpathlineto{\pgfqpoint{1.418484in}{0.576759in}}%
\pgfpathlineto{\pgfqpoint{1.409405in}{0.584180in}}%
\pgfpathlineto{\pgfqpoint{1.402664in}{0.590370in}}%
\pgfpathlineto{\pgfqpoint{1.393748in}{0.599699in}}%
\pgfpathlineto{\pgfqpoint{1.390009in}{0.603981in}}%
\pgfpathlineto{\pgfqpoint{1.379905in}{0.617592in}}%
\pgfpathlineto{\pgfqpoint{1.378092in}{0.620583in}}%
\pgfpathlineto{\pgfqpoint{1.372057in}{0.631203in}}%
\pgfpathlineto{\pgfqpoint{1.366133in}{0.644814in}}%
\pgfpathlineto{\pgfqpoint{1.362435in}{0.657087in}}%
\pgfpathlineto{\pgfqpoint{1.362048in}{0.658425in}}%
\pgfpathlineto{\pgfqpoint{1.359850in}{0.672036in}}%
\pgfpathlineto{\pgfqpoint{1.359411in}{0.685648in}}%
\pgfpathlineto{\pgfqpoint{1.360729in}{0.699259in}}%
\pgfpathlineto{\pgfqpoint{1.362435in}{0.706820in}}%
\pgfpathlineto{\pgfqpoint{1.363855in}{0.712870in}}%
\pgfpathlineto{\pgfqpoint{1.368866in}{0.726481in}}%
\pgfpathlineto{\pgfqpoint{1.375706in}{0.740092in}}%
\pgfpathlineto{\pgfqpoint{1.378092in}{0.743874in}}%
\pgfpathlineto{\pgfqpoint{1.384716in}{0.753703in}}%
\pgfpathlineto{\pgfqpoint{1.393748in}{0.764844in}}%
\pgfpathlineto{\pgfqpoint{1.395938in}{0.767314in}}%
\pgfpathlineto{\pgfqpoint{1.409405in}{0.780449in}}%
\pgfpathlineto{\pgfqpoint{1.409953in}{0.780925in}}%
\pgfpathlineto{\pgfqpoint{1.425061in}{0.792633in}}%
\pgfpathlineto{\pgfqpoint{1.427903in}{0.794536in}}%
\pgfpathlineto{\pgfqpoint{1.440718in}{0.802388in}}%
\pgfpathlineto{\pgfqpoint{1.452024in}{0.808148in}}%
\pgfpathlineto{\pgfqpoint{1.456375in}{0.810222in}}%
\pgfpathlineto{\pgfqpoint{1.472031in}{0.816168in}}%
\pgfpathlineto{\pgfqpoint{1.487688in}{0.820524in}}%
\pgfpathlineto{\pgfqpoint{1.494647in}{0.821759in}}%
\pgfpathlineto{\pgfqpoint{1.503344in}{0.823242in}}%
\pgfpathlineto{\pgfqpoint{1.519001in}{0.824388in}}%
\pgfpathlineto{\pgfqpoint{1.534657in}{0.824006in}}%
\pgfpathlineto{\pgfqpoint{1.550314in}{0.822095in}}%
\pgfpathlineto{\pgfqpoint{1.551854in}{0.821759in}}%
\pgfpathlineto{\pgfqpoint{1.565971in}{0.818544in}}%
\pgfpathlineto{\pgfqpoint{1.581627in}{0.813394in}}%
\pgfpathlineto{\pgfqpoint{1.593843in}{0.808148in}}%
\pgfpathlineto{\pgfqpoint{1.597284in}{0.806571in}}%
\pgfpathlineto{\pgfqpoint{1.612940in}{0.797787in}}%
\pgfpathlineto{\pgfqpoint{1.617866in}{0.794536in}}%
\pgfpathlineto{\pgfqpoint{1.628597in}{0.786786in}}%
\pgfpathlineto{\pgfqpoint{1.635716in}{0.780925in}}%
\pgfpathlineto{\pgfqpoint{1.644253in}{0.773032in}}%
\pgfpathlineto{\pgfqpoint{1.649834in}{0.767314in}}%
\pgfpathlineto{\pgfqpoint{1.659910in}{0.755357in}}%
\pgfpathlineto{\pgfqpoint{1.661198in}{0.753703in}}%
\pgfpathlineto{\pgfqpoint{1.670079in}{0.740092in}}%
\pgfpathlineto{\pgfqpoint{1.675567in}{0.729455in}}%
\pgfpathlineto{\pgfqpoint{1.677020in}{0.726481in}}%
\pgfpathlineto{\pgfqpoint{1.681934in}{0.712870in}}%
\pgfpathlineto{\pgfqpoint{1.685059in}{0.699259in}}%
\pgfpathlineto{\pgfqpoint{1.686398in}{0.685648in}}%
\pgfpathlineto{\pgfqpoint{1.685952in}{0.672036in}}%
\pgfpathlineto{\pgfqpoint{1.683720in}{0.658425in}}%
\pgfpathlineto{\pgfqpoint{1.679701in}{0.644814in}}%
\pgfpathlineto{\pgfqpoint{1.675567in}{0.635103in}}%
\pgfpathlineto{\pgfqpoint{1.673816in}{0.631203in}}%
\pgfpathlineto{\pgfqpoint{1.665873in}{0.617592in}}%
\pgfpathlineto{\pgfqpoint{1.659910in}{0.609277in}}%
\pgfpathlineto{\pgfqpoint{1.655806in}{0.603981in}}%
\pgfpathlineto{\pgfqpoint{1.644253in}{0.591325in}}%
\pgfpathlineto{\pgfqpoint{1.643288in}{0.590370in}}%
\pgfpathlineto{\pgfqpoint{1.628597in}{0.577598in}}%
\pgfpathlineto{\pgfqpoint{1.627498in}{0.576759in}}%
\pgfpathlineto{\pgfqpoint{1.612940in}{0.566715in}}%
\pgfpathlineto{\pgfqpoint{1.606848in}{0.563148in}}%
\pgfpathlineto{\pgfqpoint{1.597284in}{0.557963in}}%
\pgfpathlineto{\pgfqpoint{1.581627in}{0.551058in}}%
\pgfpathlineto{\pgfqpoint{1.577141in}{0.549536in}}%
\pgfpathlineto{\pgfqpoint{1.565971in}{0.545942in}}%
\pgfpathlineto{\pgfqpoint{1.550314in}{0.542448in}}%
\pgfpathlineto{\pgfqpoint{1.534657in}{0.540508in}}%
\pgfpathlineto{\pgfqpoint{1.519001in}{0.540120in}}%
\pgfpathlineto{\pgfqpoint{1.503344in}{0.541284in}}%
\pgfpathlineto{\pgfqpoint{1.487688in}{0.544000in}}%
\pgfpathlineto{\pgfqpoint{1.472031in}{0.548273in}}%
\pgfpathlineto{\pgfqpoint{1.468610in}{0.549536in}}%
\pgfpathclose%
\pgfpathmoveto{\pgfqpoint{0.704859in}{1.162036in}}%
\pgfpathlineto{\pgfqpoint{0.720516in}{1.158891in}}%
\pgfpathlineto{\pgfqpoint{0.736173in}{1.157145in}}%
\pgfpathlineto{\pgfqpoint{0.751829in}{1.156796in}}%
\pgfpathlineto{\pgfqpoint{0.767486in}{1.157843in}}%
\pgfpathlineto{\pgfqpoint{0.783142in}{1.160288in}}%
\pgfpathlineto{\pgfqpoint{0.790291in}{1.162036in}}%
\pgfpathlineto{\pgfqpoint{0.798799in}{1.164068in}}%
\pgfpathlineto{\pgfqpoint{0.814455in}{1.169154in}}%
\pgfpathlineto{\pgfqpoint{0.830112in}{1.175602in}}%
\pgfpathlineto{\pgfqpoint{0.830205in}{1.175647in}}%
\pgfpathlineto{\pgfqpoint{0.845769in}{1.183295in}}%
\pgfpathlineto{\pgfqpoint{0.856180in}{1.189259in}}%
\pgfpathlineto{\pgfqpoint{0.861425in}{1.192324in}}%
\pgfpathlineto{\pgfqpoint{0.877082in}{1.202678in}}%
\pgfpathlineto{\pgfqpoint{0.877346in}{1.202870in}}%
\pgfpathlineto{\pgfqpoint{0.892738in}{1.214520in}}%
\pgfpathlineto{\pgfqpoint{0.895132in}{1.216481in}}%
\pgfpathlineto{\pgfqpoint{0.908395in}{1.228011in}}%
\pgfpathlineto{\pgfqpoint{0.910651in}{1.230092in}}%
\pgfpathlineto{\pgfqpoint{0.924051in}{1.243474in}}%
\pgfpathlineto{\pgfqpoint{0.924272in}{1.243703in}}%
\pgfpathlineto{\pgfqpoint{0.936182in}{1.257314in}}%
\pgfpathlineto{\pgfqpoint{0.939708in}{1.261874in}}%
\pgfpathlineto{\pgfqpoint{0.946568in}{1.270925in}}%
\pgfpathlineto{\pgfqpoint{0.955364in}{1.284456in}}%
\pgfpathlineto{\pgfqpoint{0.955417in}{1.284536in}}%
\pgfpathlineto{\pgfqpoint{0.962833in}{1.298148in}}%
\pgfpathlineto{\pgfqpoint{0.968684in}{1.311759in}}%
\pgfpathlineto{\pgfqpoint{0.971021in}{1.319155in}}%
\pgfpathlineto{\pgfqpoint{0.973032in}{1.325370in}}%
\pgfpathlineto{\pgfqpoint{0.975844in}{1.338981in}}%
\pgfpathlineto{\pgfqpoint{0.977049in}{1.352592in}}%
\pgfpathlineto{\pgfqpoint{0.976647in}{1.366203in}}%
\pgfpathlineto{\pgfqpoint{0.974639in}{1.379814in}}%
\pgfpathlineto{\pgfqpoint{0.971022in}{1.393425in}}%
\pgfpathlineto{\pgfqpoint{0.971021in}{1.393428in}}%
\pgfpathlineto{\pgfqpoint{0.965954in}{1.407036in}}%
\pgfpathlineto{\pgfqpoint{0.959321in}{1.420648in}}%
\pgfpathlineto{\pgfqpoint{0.955364in}{1.427266in}}%
\pgfpathlineto{\pgfqpoint{0.951184in}{1.434259in}}%
\pgfpathlineto{\pgfqpoint{0.941568in}{1.447870in}}%
\pgfpathlineto{\pgfqpoint{0.939708in}{1.450180in}}%
\pgfpathlineto{\pgfqpoint{0.930414in}{1.461481in}}%
\pgfpathlineto{\pgfqpoint{0.924051in}{1.468400in}}%
\pgfpathlineto{\pgfqpoint{0.917648in}{1.475092in}}%
\pgfpathlineto{\pgfqpoint{0.908395in}{1.483940in}}%
\pgfpathlineto{\pgfqpoint{0.903113in}{1.488703in}}%
\pgfpathlineto{\pgfqpoint{0.892738in}{1.497441in}}%
\pgfpathlineto{\pgfqpoint{0.886480in}{1.502314in}}%
\pgfpathlineto{\pgfqpoint{0.877082in}{1.509283in}}%
\pgfpathlineto{\pgfqpoint{0.867181in}{1.515925in}}%
\pgfpathlineto{\pgfqpoint{0.861425in}{1.519671in}}%
\pgfpathlineto{\pgfqpoint{0.845769in}{1.528684in}}%
\pgfpathlineto{\pgfqpoint{0.844056in}{1.529536in}}%
\pgfpathlineto{\pgfqpoint{0.830112in}{1.536412in}}%
\pgfpathlineto{\pgfqpoint{0.814455in}{1.542799in}}%
\pgfpathlineto{\pgfqpoint{0.813384in}{1.543148in}}%
\pgfpathlineto{\pgfqpoint{0.798799in}{1.547947in}}%
\pgfpathlineto{\pgfqpoint{0.783142in}{1.551724in}}%
\pgfpathlineto{\pgfqpoint{0.767486in}{1.554125in}}%
\pgfpathlineto{\pgfqpoint{0.751829in}{1.555154in}}%
\pgfpathlineto{\pgfqpoint{0.736173in}{1.554811in}}%
\pgfpathlineto{\pgfqpoint{0.720516in}{1.553096in}}%
\pgfpathlineto{\pgfqpoint{0.704859in}{1.550007in}}%
\pgfpathlineto{\pgfqpoint{0.689203in}{1.545541in}}%
\pgfpathlineto{\pgfqpoint{0.682745in}{1.543148in}}%
\pgfpathlineto{\pgfqpoint{0.673546in}{1.539774in}}%
\pgfpathlineto{\pgfqpoint{0.657890in}{1.532713in}}%
\pgfpathlineto{\pgfqpoint{0.651907in}{1.529536in}}%
\pgfpathlineto{\pgfqpoint{0.642233in}{1.524343in}}%
\pgfpathlineto{\pgfqpoint{0.628572in}{1.515925in}}%
\pgfpathlineto{\pgfqpoint{0.626577in}{1.514659in}}%
\pgfpathlineto{\pgfqpoint{0.610920in}{1.503583in}}%
\pgfpathlineto{\pgfqpoint{0.609275in}{1.502314in}}%
\pgfpathlineto{\pgfqpoint{0.595263in}{1.490946in}}%
\pgfpathlineto{\pgfqpoint{0.592683in}{1.488703in}}%
\pgfpathlineto{\pgfqpoint{0.579607in}{1.476522in}}%
\pgfpathlineto{\pgfqpoint{0.578147in}{1.475092in}}%
\pgfpathlineto{\pgfqpoint{0.565407in}{1.461481in}}%
\pgfpathlineto{\pgfqpoint{0.563950in}{1.459746in}}%
\pgfpathlineto{\pgfqpoint{0.554267in}{1.447870in}}%
\pgfpathlineto{\pgfqpoint{0.548294in}{1.439459in}}%
\pgfpathlineto{\pgfqpoint{0.544639in}{1.434259in}}%
\pgfpathlineto{\pgfqpoint{0.536517in}{1.420648in}}%
\pgfpathlineto{\pgfqpoint{0.532637in}{1.412651in}}%
\pgfpathlineto{\pgfqpoint{0.529884in}{1.407036in}}%
\pgfpathlineto{\pgfqpoint{0.524746in}{1.393425in}}%
\pgfpathlineto{\pgfqpoint{0.521193in}{1.379814in}}%
\pgfpathlineto{\pgfqpoint{0.519221in}{1.366203in}}%
\pgfpathlineto{\pgfqpoint{0.518826in}{1.352592in}}%
\pgfpathlineto{\pgfqpoint{0.520010in}{1.338981in}}%
\pgfpathlineto{\pgfqpoint{0.522772in}{1.325370in}}%
\pgfpathlineto{\pgfqpoint{0.527117in}{1.311759in}}%
\pgfpathlineto{\pgfqpoint{0.532637in}{1.299079in}}%
\pgfpathlineto{\pgfqpoint{0.533038in}{1.298148in}}%
\pgfpathlineto{\pgfqpoint{0.540384in}{1.284536in}}%
\pgfpathlineto{\pgfqpoint{0.548294in}{1.272414in}}%
\pgfpathlineto{\pgfqpoint{0.549274in}{1.270925in}}%
\pgfpathlineto{\pgfqpoint{0.559641in}{1.257314in}}%
\pgfpathlineto{\pgfqpoint{0.563950in}{1.252310in}}%
\pgfpathlineto{\pgfqpoint{0.571591in}{1.243703in}}%
\pgfpathlineto{\pgfqpoint{0.579607in}{1.235532in}}%
\pgfpathlineto{\pgfqpoint{0.585212in}{1.230092in}}%
\pgfpathlineto{\pgfqpoint{0.595263in}{1.221073in}}%
\pgfpathlineto{\pgfqpoint{0.600742in}{1.216481in}}%
\pgfpathlineto{\pgfqpoint{0.610920in}{1.208436in}}%
\pgfpathlineto{\pgfqpoint{0.618618in}{1.202870in}}%
\pgfpathlineto{\pgfqpoint{0.626577in}{1.197338in}}%
\pgfpathlineto{\pgfqpoint{0.639576in}{1.189259in}}%
\pgfpathlineto{\pgfqpoint{0.642233in}{1.187642in}}%
\pgfpathlineto{\pgfqpoint{0.657890in}{1.179282in}}%
\pgfpathlineto{\pgfqpoint{0.665933in}{1.175647in}}%
\pgfpathlineto{\pgfqpoint{0.673546in}{1.172208in}}%
\pgfpathlineto{\pgfqpoint{0.689203in}{1.166441in}}%
\pgfpathlineto{\pgfqpoint{0.704857in}{1.162036in}}%
\pgfpathlineto{\pgfqpoint{0.704859in}{1.162036in}}%
\pgfpathclose%
\pgfpathmoveto{\pgfqpoint{0.718976in}{1.216481in}}%
\pgfpathlineto{\pgfqpoint{0.704859in}{1.219695in}}%
\pgfpathlineto{\pgfqpoint{0.689203in}{1.224846in}}%
\pgfpathlineto{\pgfqpoint{0.676987in}{1.230092in}}%
\pgfpathlineto{\pgfqpoint{0.673546in}{1.231669in}}%
\pgfpathlineto{\pgfqpoint{0.657890in}{1.240453in}}%
\pgfpathlineto{\pgfqpoint{0.652964in}{1.243703in}}%
\pgfpathlineto{\pgfqpoint{0.642233in}{1.251454in}}%
\pgfpathlineto{\pgfqpoint{0.635114in}{1.257314in}}%
\pgfpathlineto{\pgfqpoint{0.626577in}{1.265207in}}%
\pgfpathlineto{\pgfqpoint{0.620996in}{1.270925in}}%
\pgfpathlineto{\pgfqpoint{0.610920in}{1.282883in}}%
\pgfpathlineto{\pgfqpoint{0.609632in}{1.284536in}}%
\pgfpathlineto{\pgfqpoint{0.600751in}{1.298148in}}%
\pgfpathlineto{\pgfqpoint{0.595263in}{1.308784in}}%
\pgfpathlineto{\pgfqpoint{0.593810in}{1.311759in}}%
\pgfpathlineto{\pgfqpoint{0.588896in}{1.325370in}}%
\pgfpathlineto{\pgfqpoint{0.585771in}{1.338981in}}%
\pgfpathlineto{\pgfqpoint{0.584432in}{1.352592in}}%
\pgfpathlineto{\pgfqpoint{0.584878in}{1.366203in}}%
\pgfpathlineto{\pgfqpoint{0.587110in}{1.379814in}}%
\pgfpathlineto{\pgfqpoint{0.591129in}{1.393425in}}%
\pgfpathlineto{\pgfqpoint{0.595263in}{1.403136in}}%
\pgfpathlineto{\pgfqpoint{0.597014in}{1.407036in}}%
\pgfpathlineto{\pgfqpoint{0.604957in}{1.420648in}}%
\pgfpathlineto{\pgfqpoint{0.610920in}{1.428962in}}%
\pgfpathlineto{\pgfqpoint{0.615024in}{1.434259in}}%
\pgfpathlineto{\pgfqpoint{0.626577in}{1.446914in}}%
\pgfpathlineto{\pgfqpoint{0.627542in}{1.447870in}}%
\pgfpathlineto{\pgfqpoint{0.642233in}{1.460642in}}%
\pgfpathlineto{\pgfqpoint{0.643332in}{1.461481in}}%
\pgfpathlineto{\pgfqpoint{0.657890in}{1.471525in}}%
\pgfpathlineto{\pgfqpoint{0.663982in}{1.475092in}}%
\pgfpathlineto{\pgfqpoint{0.673546in}{1.480276in}}%
\pgfpathlineto{\pgfqpoint{0.689203in}{1.487181in}}%
\pgfpathlineto{\pgfqpoint{0.693689in}{1.488703in}}%
\pgfpathlineto{\pgfqpoint{0.704859in}{1.492298in}}%
\pgfpathlineto{\pgfqpoint{0.720516in}{1.495792in}}%
\pgfpathlineto{\pgfqpoint{0.736173in}{1.497731in}}%
\pgfpathlineto{\pgfqpoint{0.751829in}{1.498119in}}%
\pgfpathlineto{\pgfqpoint{0.767486in}{1.496956in}}%
\pgfpathlineto{\pgfqpoint{0.783142in}{1.494239in}}%
\pgfpathlineto{\pgfqpoint{0.798799in}{1.489967in}}%
\pgfpathlineto{\pgfqpoint{0.802220in}{1.488703in}}%
\pgfpathlineto{\pgfqpoint{0.814455in}{1.483932in}}%
\pgfpathlineto{\pgfqpoint{0.830112in}{1.476212in}}%
\pgfpathlineto{\pgfqpoint{0.832014in}{1.475092in}}%
\pgfpathlineto{\pgfqpoint{0.845769in}{1.466333in}}%
\pgfpathlineto{\pgfqpoint{0.852346in}{1.461481in}}%
\pgfpathlineto{\pgfqpoint{0.861425in}{1.454059in}}%
\pgfpathlineto{\pgfqpoint{0.868166in}{1.447870in}}%
\pgfpathlineto{\pgfqpoint{0.877082in}{1.438541in}}%
\pgfpathlineto{\pgfqpoint{0.880821in}{1.434259in}}%
\pgfpathlineto{\pgfqpoint{0.890925in}{1.420648in}}%
\pgfpathlineto{\pgfqpoint{0.892738in}{1.417656in}}%
\pgfpathlineto{\pgfqpoint{0.898773in}{1.407036in}}%
\pgfpathlineto{\pgfqpoint{0.904697in}{1.393425in}}%
\pgfpathlineto{\pgfqpoint{0.908395in}{1.381153in}}%
\pgfpathlineto{\pgfqpoint{0.908782in}{1.379814in}}%
\pgfpathlineto{\pgfqpoint{0.910980in}{1.366203in}}%
\pgfpathlineto{\pgfqpoint{0.911419in}{1.352592in}}%
\pgfpathlineto{\pgfqpoint{0.910101in}{1.338981in}}%
\pgfpathlineto{\pgfqpoint{0.908395in}{1.331420in}}%
\pgfpathlineto{\pgfqpoint{0.906975in}{1.325370in}}%
\pgfpathlineto{\pgfqpoint{0.901964in}{1.311759in}}%
\pgfpathlineto{\pgfqpoint{0.895124in}{1.298148in}}%
\pgfpathlineto{\pgfqpoint{0.892738in}{1.294365in}}%
\pgfpathlineto{\pgfqpoint{0.886114in}{1.284536in}}%
\pgfpathlineto{\pgfqpoint{0.877082in}{1.273396in}}%
\pgfpathlineto{\pgfqpoint{0.874892in}{1.270925in}}%
\pgfpathlineto{\pgfqpoint{0.861425in}{1.257791in}}%
\pgfpathlineto{\pgfqpoint{0.860877in}{1.257314in}}%
\pgfpathlineto{\pgfqpoint{0.845769in}{1.245606in}}%
\pgfpathlineto{\pgfqpoint{0.842927in}{1.243703in}}%
\pgfpathlineto{\pgfqpoint{0.830112in}{1.235851in}}%
\pgfpathlineto{\pgfqpoint{0.818806in}{1.230092in}}%
\pgfpathlineto{\pgfqpoint{0.814455in}{1.228018in}}%
\pgfpathlineto{\pgfqpoint{0.798799in}{1.222072in}}%
\pgfpathlineto{\pgfqpoint{0.783142in}{1.217716in}}%
\pgfpathlineto{\pgfqpoint{0.776183in}{1.216481in}}%
\pgfpathlineto{\pgfqpoint{0.767486in}{1.214998in}}%
\pgfpathlineto{\pgfqpoint{0.751829in}{1.213852in}}%
\pgfpathlineto{\pgfqpoint{0.736173in}{1.214234in}}%
\pgfpathlineto{\pgfqpoint{0.720516in}{1.216144in}}%
\pgfpathlineto{\pgfqpoint{0.718976in}{1.216481in}}%
\pgfpathclose%
\pgfpathmoveto{\pgfqpoint{1.487688in}{1.160288in}}%
\pgfpathlineto{\pgfqpoint{1.503344in}{1.157843in}}%
\pgfpathlineto{\pgfqpoint{1.519001in}{1.156796in}}%
\pgfpathlineto{\pgfqpoint{1.534657in}{1.157145in}}%
\pgfpathlineto{\pgfqpoint{1.550314in}{1.158891in}}%
\pgfpathlineto{\pgfqpoint{1.565971in}{1.162036in}}%
\pgfpathlineto{\pgfqpoint{1.565973in}{1.162036in}}%
\pgfpathlineto{\pgfqpoint{1.581627in}{1.166441in}}%
\pgfpathlineto{\pgfqpoint{1.597284in}{1.172208in}}%
\pgfpathlineto{\pgfqpoint{1.604897in}{1.175647in}}%
\pgfpathlineto{\pgfqpoint{1.612940in}{1.179282in}}%
\pgfpathlineto{\pgfqpoint{1.628597in}{1.187642in}}%
\pgfpathlineto{\pgfqpoint{1.631254in}{1.189259in}}%
\pgfpathlineto{\pgfqpoint{1.644253in}{1.197338in}}%
\pgfpathlineto{\pgfqpoint{1.652212in}{1.202870in}}%
\pgfpathlineto{\pgfqpoint{1.659910in}{1.208436in}}%
\pgfpathlineto{\pgfqpoint{1.670088in}{1.216481in}}%
\pgfpathlineto{\pgfqpoint{1.675567in}{1.221073in}}%
\pgfpathlineto{\pgfqpoint{1.685618in}{1.230092in}}%
\pgfpathlineto{\pgfqpoint{1.691223in}{1.235532in}}%
\pgfpathlineto{\pgfqpoint{1.699239in}{1.243703in}}%
\pgfpathlineto{\pgfqpoint{1.706880in}{1.252310in}}%
\pgfpathlineto{\pgfqpoint{1.711189in}{1.257314in}}%
\pgfpathlineto{\pgfqpoint{1.721556in}{1.270925in}}%
\pgfpathlineto{\pgfqpoint{1.722536in}{1.272414in}}%
\pgfpathlineto{\pgfqpoint{1.730446in}{1.284536in}}%
\pgfpathlineto{\pgfqpoint{1.737792in}{1.298148in}}%
\pgfpathlineto{\pgfqpoint{1.738193in}{1.299079in}}%
\pgfpathlineto{\pgfqpoint{1.743713in}{1.311759in}}%
\pgfpathlineto{\pgfqpoint{1.748058in}{1.325370in}}%
\pgfpathlineto{\pgfqpoint{1.750820in}{1.338981in}}%
\pgfpathlineto{\pgfqpoint{1.752004in}{1.352592in}}%
\pgfpathlineto{\pgfqpoint{1.751609in}{1.366203in}}%
\pgfpathlineto{\pgfqpoint{1.749637in}{1.379814in}}%
\pgfpathlineto{\pgfqpoint{1.746084in}{1.393425in}}%
\pgfpathlineto{\pgfqpoint{1.740946in}{1.407036in}}%
\pgfpathlineto{\pgfqpoint{1.738193in}{1.412651in}}%
\pgfpathlineto{\pgfqpoint{1.734313in}{1.420648in}}%
\pgfpathlineto{\pgfqpoint{1.726191in}{1.434259in}}%
\pgfpathlineto{\pgfqpoint{1.722536in}{1.439459in}}%
\pgfpathlineto{\pgfqpoint{1.716563in}{1.447870in}}%
\pgfpathlineto{\pgfqpoint{1.706880in}{1.459746in}}%
\pgfpathlineto{\pgfqpoint{1.705423in}{1.461481in}}%
\pgfpathlineto{\pgfqpoint{1.692683in}{1.475092in}}%
\pgfpathlineto{\pgfqpoint{1.691223in}{1.476522in}}%
\pgfpathlineto{\pgfqpoint{1.678147in}{1.488703in}}%
\pgfpathlineto{\pgfqpoint{1.675567in}{1.490946in}}%
\pgfpathlineto{\pgfqpoint{1.661555in}{1.502314in}}%
\pgfpathlineto{\pgfqpoint{1.659910in}{1.503583in}}%
\pgfpathlineto{\pgfqpoint{1.644253in}{1.514659in}}%
\pgfpathlineto{\pgfqpoint{1.642258in}{1.515925in}}%
\pgfpathlineto{\pgfqpoint{1.628597in}{1.524343in}}%
\pgfpathlineto{\pgfqpoint{1.618923in}{1.529536in}}%
\pgfpathlineto{\pgfqpoint{1.612940in}{1.532713in}}%
\pgfpathlineto{\pgfqpoint{1.597284in}{1.539774in}}%
\pgfpathlineto{\pgfqpoint{1.588085in}{1.543148in}}%
\pgfpathlineto{\pgfqpoint{1.581627in}{1.545541in}}%
\pgfpathlineto{\pgfqpoint{1.565971in}{1.550007in}}%
\pgfpathlineto{\pgfqpoint{1.550314in}{1.553096in}}%
\pgfpathlineto{\pgfqpoint{1.534657in}{1.554811in}}%
\pgfpathlineto{\pgfqpoint{1.519001in}{1.555154in}}%
\pgfpathlineto{\pgfqpoint{1.503344in}{1.554125in}}%
\pgfpathlineto{\pgfqpoint{1.487688in}{1.551724in}}%
\pgfpathlineto{\pgfqpoint{1.472031in}{1.547947in}}%
\pgfpathlineto{\pgfqpoint{1.457446in}{1.543148in}}%
\pgfpathlineto{\pgfqpoint{1.456375in}{1.542799in}}%
\pgfpathlineto{\pgfqpoint{1.440718in}{1.536412in}}%
\pgfpathlineto{\pgfqpoint{1.426774in}{1.529536in}}%
\pgfpathlineto{\pgfqpoint{1.425061in}{1.528684in}}%
\pgfpathlineto{\pgfqpoint{1.409405in}{1.519671in}}%
\pgfpathlineto{\pgfqpoint{1.403649in}{1.515925in}}%
\pgfpathlineto{\pgfqpoint{1.393748in}{1.509283in}}%
\pgfpathlineto{\pgfqpoint{1.384350in}{1.502314in}}%
\pgfpathlineto{\pgfqpoint{1.378092in}{1.497441in}}%
\pgfpathlineto{\pgfqpoint{1.367717in}{1.488703in}}%
\pgfpathlineto{\pgfqpoint{1.362435in}{1.483940in}}%
\pgfpathlineto{\pgfqpoint{1.353182in}{1.475092in}}%
\pgfpathlineto{\pgfqpoint{1.346779in}{1.468400in}}%
\pgfpathlineto{\pgfqpoint{1.340416in}{1.461481in}}%
\pgfpathlineto{\pgfqpoint{1.331122in}{1.450180in}}%
\pgfpathlineto{\pgfqpoint{1.329262in}{1.447870in}}%
\pgfpathlineto{\pgfqpoint{1.319646in}{1.434259in}}%
\pgfpathlineto{\pgfqpoint{1.315466in}{1.427266in}}%
\pgfpathlineto{\pgfqpoint{1.311509in}{1.420647in}}%
\pgfpathlineto{\pgfqpoint{1.304876in}{1.407036in}}%
\pgfpathlineto{\pgfqpoint{1.299809in}{1.393428in}}%
\pgfpathlineto{\pgfqpoint{1.299808in}{1.393425in}}%
\pgfpathlineto{\pgfqpoint{1.296191in}{1.379814in}}%
\pgfpathlineto{\pgfqpoint{1.294183in}{1.366203in}}%
\pgfpathlineto{\pgfqpoint{1.293781in}{1.352592in}}%
\pgfpathlineto{\pgfqpoint{1.294986in}{1.338981in}}%
\pgfpathlineto{\pgfqpoint{1.297798in}{1.325370in}}%
\pgfpathlineto{\pgfqpoint{1.299809in}{1.319155in}}%
\pgfpathlineto{\pgfqpoint{1.302146in}{1.311759in}}%
\pgfpathlineto{\pgfqpoint{1.307997in}{1.298148in}}%
\pgfpathlineto{\pgfqpoint{1.315413in}{1.284536in}}%
\pgfpathlineto{\pgfqpoint{1.315466in}{1.284456in}}%
\pgfpathlineto{\pgfqpoint{1.324262in}{1.270925in}}%
\pgfpathlineto{\pgfqpoint{1.331122in}{1.261874in}}%
\pgfpathlineto{\pgfqpoint{1.334648in}{1.257314in}}%
\pgfpathlineto{\pgfqpoint{1.346558in}{1.243703in}}%
\pgfpathlineto{\pgfqpoint{1.346779in}{1.243474in}}%
\pgfpathlineto{\pgfqpoint{1.360179in}{1.230092in}}%
\pgfpathlineto{\pgfqpoint{1.362435in}{1.228011in}}%
\pgfpathlineto{\pgfqpoint{1.375698in}{1.216481in}}%
\pgfpathlineto{\pgfqpoint{1.378092in}{1.214520in}}%
\pgfpathlineto{\pgfqpoint{1.393484in}{1.202870in}}%
\pgfpathlineto{\pgfqpoint{1.393748in}{1.202678in}}%
\pgfpathlineto{\pgfqpoint{1.409405in}{1.192324in}}%
\pgfpathlineto{\pgfqpoint{1.414650in}{1.189259in}}%
\pgfpathlineto{\pgfqpoint{1.425061in}{1.183295in}}%
\pgfpathlineto{\pgfqpoint{1.440625in}{1.175647in}}%
\pgfpathlineto{\pgfqpoint{1.440718in}{1.175602in}}%
\pgfpathlineto{\pgfqpoint{1.456375in}{1.169154in}}%
\pgfpathlineto{\pgfqpoint{1.472031in}{1.164068in}}%
\pgfpathlineto{\pgfqpoint{1.480539in}{1.162036in}}%
\pgfpathlineto{\pgfqpoint{1.487688in}{1.160288in}}%
\pgfpathclose%
\pgfpathmoveto{\pgfqpoint{1.494647in}{1.216481in}}%
\pgfpathlineto{\pgfqpoint{1.487688in}{1.217716in}}%
\pgfpathlineto{\pgfqpoint{1.472031in}{1.222072in}}%
\pgfpathlineto{\pgfqpoint{1.456375in}{1.228018in}}%
\pgfpathlineto{\pgfqpoint{1.452024in}{1.230092in}}%
\pgfpathlineto{\pgfqpoint{1.440718in}{1.235851in}}%
\pgfpathlineto{\pgfqpoint{1.427903in}{1.243703in}}%
\pgfpathlineto{\pgfqpoint{1.425061in}{1.245606in}}%
\pgfpathlineto{\pgfqpoint{1.409953in}{1.257314in}}%
\pgfpathlineto{\pgfqpoint{1.409405in}{1.257791in}}%
\pgfpathlineto{\pgfqpoint{1.395938in}{1.270925in}}%
\pgfpathlineto{\pgfqpoint{1.393748in}{1.273396in}}%
\pgfpathlineto{\pgfqpoint{1.384716in}{1.284536in}}%
\pgfpathlineto{\pgfqpoint{1.378092in}{1.294365in}}%
\pgfpathlineto{\pgfqpoint{1.375706in}{1.298148in}}%
\pgfpathlineto{\pgfqpoint{1.368866in}{1.311759in}}%
\pgfpathlineto{\pgfqpoint{1.363855in}{1.325370in}}%
\pgfpathlineto{\pgfqpoint{1.362435in}{1.331420in}}%
\pgfpathlineto{\pgfqpoint{1.360729in}{1.338981in}}%
\pgfpathlineto{\pgfqpoint{1.359411in}{1.352592in}}%
\pgfpathlineto{\pgfqpoint{1.359850in}{1.366203in}}%
\pgfpathlineto{\pgfqpoint{1.362048in}{1.379814in}}%
\pgfpathlineto{\pgfqpoint{1.362435in}{1.381153in}}%
\pgfpathlineto{\pgfqpoint{1.366133in}{1.393425in}}%
\pgfpathlineto{\pgfqpoint{1.372057in}{1.407036in}}%
\pgfpathlineto{\pgfqpoint{1.378092in}{1.417656in}}%
\pgfpathlineto{\pgfqpoint{1.379905in}{1.420648in}}%
\pgfpathlineto{\pgfqpoint{1.390009in}{1.434259in}}%
\pgfpathlineto{\pgfqpoint{1.393748in}{1.438541in}}%
\pgfpathlineto{\pgfqpoint{1.402664in}{1.447870in}}%
\pgfpathlineto{\pgfqpoint{1.409405in}{1.454059in}}%
\pgfpathlineto{\pgfqpoint{1.418484in}{1.461481in}}%
\pgfpathlineto{\pgfqpoint{1.425061in}{1.466333in}}%
\pgfpathlineto{\pgfqpoint{1.438816in}{1.475092in}}%
\pgfpathlineto{\pgfqpoint{1.440718in}{1.476212in}}%
\pgfpathlineto{\pgfqpoint{1.456375in}{1.483932in}}%
\pgfpathlineto{\pgfqpoint{1.468610in}{1.488703in}}%
\pgfpathlineto{\pgfqpoint{1.472031in}{1.489967in}}%
\pgfpathlineto{\pgfqpoint{1.487688in}{1.494239in}}%
\pgfpathlineto{\pgfqpoint{1.503344in}{1.496956in}}%
\pgfpathlineto{\pgfqpoint{1.519001in}{1.498119in}}%
\pgfpathlineto{\pgfqpoint{1.534657in}{1.497731in}}%
\pgfpathlineto{\pgfqpoint{1.550314in}{1.495792in}}%
\pgfpathlineto{\pgfqpoint{1.565971in}{1.492298in}}%
\pgfpathlineto{\pgfqpoint{1.577141in}{1.488703in}}%
\pgfpathlineto{\pgfqpoint{1.581627in}{1.487181in}}%
\pgfpathlineto{\pgfqpoint{1.597284in}{1.480276in}}%
\pgfpathlineto{\pgfqpoint{1.606848in}{1.475092in}}%
\pgfpathlineto{\pgfqpoint{1.612940in}{1.471525in}}%
\pgfpathlineto{\pgfqpoint{1.627498in}{1.461481in}}%
\pgfpathlineto{\pgfqpoint{1.628597in}{1.460642in}}%
\pgfpathlineto{\pgfqpoint{1.643288in}{1.447870in}}%
\pgfpathlineto{\pgfqpoint{1.644253in}{1.446914in}}%
\pgfpathlineto{\pgfqpoint{1.655806in}{1.434259in}}%
\pgfpathlineto{\pgfqpoint{1.659910in}{1.428962in}}%
\pgfpathlineto{\pgfqpoint{1.665873in}{1.420648in}}%
\pgfpathlineto{\pgfqpoint{1.673816in}{1.407036in}}%
\pgfpathlineto{\pgfqpoint{1.675567in}{1.403136in}}%
\pgfpathlineto{\pgfqpoint{1.679701in}{1.393425in}}%
\pgfpathlineto{\pgfqpoint{1.683720in}{1.379814in}}%
\pgfpathlineto{\pgfqpoint{1.685952in}{1.366203in}}%
\pgfpathlineto{\pgfqpoint{1.686398in}{1.352592in}}%
\pgfpathlineto{\pgfqpoint{1.685059in}{1.338981in}}%
\pgfpathlineto{\pgfqpoint{1.681934in}{1.325370in}}%
\pgfpathlineto{\pgfqpoint{1.677020in}{1.311759in}}%
\pgfpathlineto{\pgfqpoint{1.675567in}{1.308784in}}%
\pgfpathlineto{\pgfqpoint{1.670079in}{1.298148in}}%
\pgfpathlineto{\pgfqpoint{1.661198in}{1.284536in}}%
\pgfpathlineto{\pgfqpoint{1.659910in}{1.282883in}}%
\pgfpathlineto{\pgfqpoint{1.649834in}{1.270925in}}%
\pgfpathlineto{\pgfqpoint{1.644253in}{1.265207in}}%
\pgfpathlineto{\pgfqpoint{1.635716in}{1.257314in}}%
\pgfpathlineto{\pgfqpoint{1.628597in}{1.251454in}}%
\pgfpathlineto{\pgfqpoint{1.617866in}{1.243703in}}%
\pgfpathlineto{\pgfqpoint{1.612940in}{1.240453in}}%
\pgfpathlineto{\pgfqpoint{1.597284in}{1.231669in}}%
\pgfpathlineto{\pgfqpoint{1.593843in}{1.230092in}}%
\pgfpathlineto{\pgfqpoint{1.581627in}{1.224846in}}%
\pgfpathlineto{\pgfqpoint{1.565971in}{1.219695in}}%
\pgfpathlineto{\pgfqpoint{1.551854in}{1.216481in}}%
\pgfpathlineto{\pgfqpoint{1.550314in}{1.216144in}}%
\pgfpathlineto{\pgfqpoint{1.534657in}{1.214234in}}%
\pgfpathlineto{\pgfqpoint{1.519001in}{1.213852in}}%
\pgfpathlineto{\pgfqpoint{1.503344in}{1.214998in}}%
\pgfpathlineto{\pgfqpoint{1.494647in}{1.216481in}}%
\pgfpathclose%
\pgfusepath{fill}%
\end{pgfscope}%
\begin{pgfscope}%
\pgfpathrectangle{\pgfqpoint{0.360415in}{0.345370in}}{\pgfqpoint{1.550000in}{1.347500in}}%
\pgfusepath{clip}%
\pgfsetbuttcap%
\pgfsetroundjoin%
\definecolor{currentfill}{rgb}{0.709962,0.212797,0.477201}%
\pgfsetfillcolor{currentfill}%
\pgfsetlinewidth{0.000000pt}%
\definecolor{currentstroke}{rgb}{0.000000,0.000000,0.000000}%
\pgfsetstrokecolor{currentstroke}%
\pgfsetdash{}{0pt}%
\pgfpathmoveto{\pgfqpoint{0.736173in}{0.413265in}}%
\pgfpathlineto{\pgfqpoint{0.751829in}{0.412724in}}%
\pgfpathlineto{\pgfqpoint{0.758598in}{0.413425in}}%
\pgfpathlineto{\pgfqpoint{0.767486in}{0.414201in}}%
\pgfpathlineto{\pgfqpoint{0.783142in}{0.417387in}}%
\pgfpathlineto{\pgfqpoint{0.798799in}{0.422399in}}%
\pgfpathlineto{\pgfqpoint{0.809451in}{0.427036in}}%
\pgfpathlineto{\pgfqpoint{0.814455in}{0.428963in}}%
\pgfpathlineto{\pgfqpoint{0.830112in}{0.436547in}}%
\pgfpathlineto{\pgfqpoint{0.837164in}{0.440648in}}%
\pgfpathlineto{\pgfqpoint{0.845769in}{0.445230in}}%
\pgfpathlineto{\pgfqpoint{0.860342in}{0.454259in}}%
\pgfpathlineto{\pgfqpoint{0.861425in}{0.454890in}}%
\pgfpathlineto{\pgfqpoint{0.877082in}{0.465167in}}%
\pgfpathlineto{\pgfqpoint{0.880810in}{0.467870in}}%
\pgfpathlineto{\pgfqpoint{0.892738in}{0.476215in}}%
\pgfpathlineto{\pgfqpoint{0.899677in}{0.481481in}}%
\pgfpathlineto{\pgfqpoint{0.908395in}{0.487996in}}%
\pgfpathlineto{\pgfqpoint{0.917305in}{0.495092in}}%
\pgfpathlineto{\pgfqpoint{0.924051in}{0.500483in}}%
\pgfpathlineto{\pgfqpoint{0.933872in}{0.508703in}}%
\pgfpathlineto{\pgfqpoint{0.939708in}{0.513692in}}%
\pgfpathlineto{\pgfqpoint{0.949502in}{0.522314in}}%
\pgfpathlineto{\pgfqpoint{0.955364in}{0.527676in}}%
\pgfpathlineto{\pgfqpoint{0.964275in}{0.535925in}}%
\pgfpathlineto{\pgfqpoint{0.971021in}{0.542523in}}%
\pgfpathlineto{\pgfqpoint{0.978233in}{0.549536in}}%
\pgfpathlineto{\pgfqpoint{0.986678in}{0.558364in}}%
\pgfpathlineto{\pgfqpoint{0.991370in}{0.563148in}}%
\pgfpathlineto{\pgfqpoint{1.002334in}{0.575385in}}%
\pgfpathlineto{\pgfqpoint{1.003624in}{0.576759in}}%
\pgfpathlineto{\pgfqpoint{1.015114in}{0.590370in}}%
\pgfpathlineto{\pgfqpoint{1.017991in}{0.594269in}}%
\pgfpathlineto{\pgfqpoint{1.025704in}{0.603981in}}%
\pgfpathlineto{\pgfqpoint{1.033647in}{0.615805in}}%
\pgfpathlineto{\pgfqpoint{1.034978in}{0.617592in}}%
\pgfpathlineto{\pgfqpoint{1.043286in}{0.631203in}}%
\pgfpathlineto{\pgfqpoint{1.049304in}{0.644100in}}%
\pgfpathlineto{\pgfqpoint{1.049688in}{0.644814in}}%
\pgfpathlineto{\pgfqpoint{1.054800in}{0.658425in}}%
\pgfpathlineto{\pgfqpoint{1.057638in}{0.672036in}}%
\pgfpathlineto{\pgfqpoint{1.058206in}{0.685648in}}%
\pgfpathlineto{\pgfqpoint{1.056503in}{0.699259in}}%
\pgfpathlineto{\pgfqpoint{1.052529in}{0.712870in}}%
\pgfpathlineto{\pgfqpoint{1.049304in}{0.719919in}}%
\pgfpathlineto{\pgfqpoint{1.046705in}{0.726481in}}%
\pgfpathlineto{\pgfqpoint{1.039378in}{0.740092in}}%
\pgfpathlineto{\pgfqpoint{1.033647in}{0.748532in}}%
\pgfpathlineto{\pgfqpoint{1.030485in}{0.753703in}}%
\pgfpathlineto{\pgfqpoint{1.020488in}{0.767314in}}%
\pgfpathlineto{\pgfqpoint{1.017991in}{0.770255in}}%
\pgfpathlineto{\pgfqpoint{1.009564in}{0.780925in}}%
\pgfpathlineto{\pgfqpoint{1.002334in}{0.789017in}}%
\pgfpathlineto{\pgfqpoint{0.997632in}{0.794536in}}%
\pgfpathlineto{\pgfqpoint{0.986678in}{0.806159in}}%
\pgfpathlineto{\pgfqpoint{0.984851in}{0.808148in}}%
\pgfpathlineto{\pgfqpoint{0.971298in}{0.821759in}}%
\pgfpathlineto{\pgfqpoint{0.971021in}{0.822021in}}%
\pgfpathlineto{\pgfqpoint{0.956978in}{0.835370in}}%
\pgfpathlineto{\pgfqpoint{0.955364in}{0.836834in}}%
\pgfpathlineto{\pgfqpoint{0.941803in}{0.848981in}}%
\pgfpathlineto{\pgfqpoint{0.939708in}{0.850803in}}%
\pgfpathlineto{\pgfqpoint{0.925735in}{0.862592in}}%
\pgfpathlineto{\pgfqpoint{0.924051in}{0.863995in}}%
\pgfpathlineto{\pgfqpoint{0.908696in}{0.876203in}}%
\pgfpathlineto{\pgfqpoint{0.908395in}{0.876444in}}%
\pgfpathlineto{\pgfqpoint{0.892738in}{0.888227in}}%
\pgfpathlineto{\pgfqpoint{0.890451in}{0.889814in}}%
\pgfpathlineto{\pgfqpoint{0.877082in}{0.899337in}}%
\pgfpathlineto{\pgfqpoint{0.870732in}{0.903425in}}%
\pgfpathlineto{\pgfqpoint{0.861425in}{0.909711in}}%
\pgfpathlineto{\pgfqpoint{0.849152in}{0.917036in}}%
\pgfpathlineto{\pgfqpoint{0.845769in}{0.919207in}}%
\pgfpathlineto{\pgfqpoint{0.830112in}{0.927898in}}%
\pgfpathlineto{\pgfqpoint{0.824164in}{0.930648in}}%
\pgfpathlineto{\pgfqpoint{0.814455in}{0.935629in}}%
\pgfpathlineto{\pgfqpoint{0.798799in}{0.941999in}}%
\pgfpathlineto{\pgfqpoint{0.791251in}{0.944259in}}%
\pgfpathlineto{\pgfqpoint{0.783142in}{0.947062in}}%
\pgfpathlineto{\pgfqpoint{0.767486in}{0.950517in}}%
\pgfpathlineto{\pgfqpoint{0.751829in}{0.951997in}}%
\pgfpathlineto{\pgfqpoint{0.736173in}{0.951504in}}%
\pgfpathlineto{\pgfqpoint{0.720516in}{0.949037in}}%
\pgfpathlineto{\pgfqpoint{0.704859in}{0.944593in}}%
\pgfpathlineto{\pgfqpoint{0.704038in}{0.944259in}}%
\pgfpathlineto{\pgfqpoint{0.689203in}{0.939027in}}%
\pgfpathlineto{\pgfqpoint{0.673546in}{0.931804in}}%
\pgfpathlineto{\pgfqpoint{0.671491in}{0.930648in}}%
\pgfpathlineto{\pgfqpoint{0.657890in}{0.923742in}}%
\pgfpathlineto{\pgfqpoint{0.646719in}{0.917036in}}%
\pgfpathlineto{\pgfqpoint{0.642233in}{0.914535in}}%
\pgfpathlineto{\pgfqpoint{0.626577in}{0.904546in}}%
\pgfpathlineto{\pgfqpoint{0.624996in}{0.903425in}}%
\pgfpathlineto{\pgfqpoint{0.610920in}{0.893894in}}%
\pgfpathlineto{\pgfqpoint{0.605418in}{0.889814in}}%
\pgfpathlineto{\pgfqpoint{0.595263in}{0.882473in}}%
\pgfpathlineto{\pgfqpoint{0.587196in}{0.876203in}}%
\pgfpathlineto{\pgfqpoint{0.579607in}{0.870338in}}%
\pgfpathlineto{\pgfqpoint{0.570118in}{0.862592in}}%
\pgfpathlineto{\pgfqpoint{0.563950in}{0.857495in}}%
\pgfpathlineto{\pgfqpoint{0.554032in}{0.848981in}}%
\pgfpathlineto{\pgfqpoint{0.548294in}{0.843907in}}%
\pgfpathlineto{\pgfqpoint{0.538838in}{0.835370in}}%
\pgfpathlineto{\pgfqpoint{0.532637in}{0.829505in}}%
\pgfpathlineto{\pgfqpoint{0.524475in}{0.821759in}}%
\pgfpathlineto{\pgfqpoint{0.516981in}{0.814180in}}%
\pgfpathlineto{\pgfqpoint{0.510924in}{0.808148in}}%
\pgfpathlineto{\pgfqpoint{0.501324in}{0.797777in}}%
\pgfpathlineto{\pgfqpoint{0.498215in}{0.794536in}}%
\pgfpathlineto{\pgfqpoint{0.486394in}{0.780925in}}%
\pgfpathlineto{\pgfqpoint{0.485668in}{0.779983in}}%
\pgfpathlineto{\pgfqpoint{0.475282in}{0.767314in}}%
\pgfpathlineto{\pgfqpoint{0.470011in}{0.759834in}}%
\pgfpathlineto{\pgfqpoint{0.465295in}{0.753703in}}%
\pgfpathlineto{\pgfqpoint{0.456570in}{0.740092in}}%
\pgfpathlineto{\pgfqpoint{0.454354in}{0.735741in}}%
\pgfpathlineto{\pgfqpoint{0.449020in}{0.726481in}}%
\pgfpathlineto{\pgfqpoint{0.443255in}{0.712870in}}%
\pgfpathlineto{\pgfqpoint{0.439590in}{0.699259in}}%
\pgfpathlineto{\pgfqpoint{0.438698in}{0.691532in}}%
\pgfpathlineto{\pgfqpoint{0.437892in}{0.685648in}}%
\pgfpathlineto{\pgfqpoint{0.438514in}{0.672036in}}%
\pgfpathlineto{\pgfqpoint{0.438698in}{0.671230in}}%
\pgfpathlineto{\pgfqpoint{0.441160in}{0.658425in}}%
\pgfpathlineto{\pgfqpoint{0.445875in}{0.644814in}}%
\pgfpathlineto{\pgfqpoint{0.452691in}{0.631203in}}%
\pgfpathlineto{\pgfqpoint{0.454354in}{0.628639in}}%
\pgfpathlineto{\pgfqpoint{0.460702in}{0.617592in}}%
\pgfpathlineto{\pgfqpoint{0.470011in}{0.604449in}}%
\pgfpathlineto{\pgfqpoint{0.470314in}{0.603981in}}%
\pgfpathlineto{\pgfqpoint{0.480663in}{0.590370in}}%
\pgfpathlineto{\pgfqpoint{0.485668in}{0.584622in}}%
\pgfpathlineto{\pgfqpoint{0.492119in}{0.576759in}}%
\pgfpathlineto{\pgfqpoint{0.501324in}{0.566725in}}%
\pgfpathlineto{\pgfqpoint{0.504488in}{0.563148in}}%
\pgfpathlineto{\pgfqpoint{0.516981in}{0.550253in}}%
\pgfpathlineto{\pgfqpoint{0.517663in}{0.549536in}}%
\pgfpathlineto{\pgfqpoint{0.531586in}{0.535925in}}%
\pgfpathlineto{\pgfqpoint{0.532637in}{0.534951in}}%
\pgfpathlineto{\pgfqpoint{0.546331in}{0.522314in}}%
\pgfpathlineto{\pgfqpoint{0.548294in}{0.520571in}}%
\pgfpathlineto{\pgfqpoint{0.561945in}{0.508703in}}%
\pgfpathlineto{\pgfqpoint{0.563950in}{0.506997in}}%
\pgfpathlineto{\pgfqpoint{0.578486in}{0.495092in}}%
\pgfpathlineto{\pgfqpoint{0.579607in}{0.494178in}}%
\pgfpathlineto{\pgfqpoint{0.595263in}{0.482074in}}%
\pgfpathlineto{\pgfqpoint{0.596087in}{0.481481in}}%
\pgfpathlineto{\pgfqpoint{0.610920in}{0.470620in}}%
\pgfpathlineto{\pgfqpoint{0.615035in}{0.467870in}}%
\pgfpathlineto{\pgfqpoint{0.626577in}{0.459867in}}%
\pgfpathlineto{\pgfqpoint{0.635622in}{0.454259in}}%
\pgfpathlineto{\pgfqpoint{0.642233in}{0.449908in}}%
\pgfpathlineto{\pgfqpoint{0.657890in}{0.440911in}}%
\pgfpathlineto{\pgfqpoint{0.658428in}{0.440648in}}%
\pgfpathlineto{\pgfqpoint{0.673546in}{0.432555in}}%
\pgfpathlineto{\pgfqpoint{0.686254in}{0.427036in}}%
\pgfpathlineto{\pgfqpoint{0.689203in}{0.425590in}}%
\pgfpathlineto{\pgfqpoint{0.704859in}{0.419665in}}%
\pgfpathlineto{\pgfqpoint{0.720516in}{0.415566in}}%
\pgfpathlineto{\pgfqpoint{0.735244in}{0.413425in}}%
\pgfpathlineto{\pgfqpoint{0.736173in}{0.413265in}}%
\pgfpathclose%
\pgfpathmoveto{\pgfqpoint{0.682745in}{0.495092in}}%
\pgfpathlineto{\pgfqpoint{0.673546in}{0.498465in}}%
\pgfpathlineto{\pgfqpoint{0.657890in}{0.505526in}}%
\pgfpathlineto{\pgfqpoint{0.651907in}{0.508703in}}%
\pgfpathlineto{\pgfqpoint{0.642233in}{0.513896in}}%
\pgfpathlineto{\pgfqpoint{0.628572in}{0.522314in}}%
\pgfpathlineto{\pgfqpoint{0.626577in}{0.523581in}}%
\pgfpathlineto{\pgfqpoint{0.610920in}{0.534656in}}%
\pgfpathlineto{\pgfqpoint{0.609275in}{0.535925in}}%
\pgfpathlineto{\pgfqpoint{0.595263in}{0.547293in}}%
\pgfpathlineto{\pgfqpoint{0.592683in}{0.549536in}}%
\pgfpathlineto{\pgfqpoint{0.579607in}{0.561718in}}%
\pgfpathlineto{\pgfqpoint{0.578147in}{0.563148in}}%
\pgfpathlineto{\pgfqpoint{0.565407in}{0.576759in}}%
\pgfpathlineto{\pgfqpoint{0.563950in}{0.578494in}}%
\pgfpathlineto{\pgfqpoint{0.554267in}{0.590370in}}%
\pgfpathlineto{\pgfqpoint{0.548294in}{0.598780in}}%
\pgfpathlineto{\pgfqpoint{0.544639in}{0.603981in}}%
\pgfpathlineto{\pgfqpoint{0.536517in}{0.617592in}}%
\pgfpathlineto{\pgfqpoint{0.532637in}{0.625589in}}%
\pgfpathlineto{\pgfqpoint{0.529884in}{0.631203in}}%
\pgfpathlineto{\pgfqpoint{0.524746in}{0.644814in}}%
\pgfpathlineto{\pgfqpoint{0.521193in}{0.658425in}}%
\pgfpathlineto{\pgfqpoint{0.519221in}{0.672036in}}%
\pgfpathlineto{\pgfqpoint{0.518826in}{0.685648in}}%
\pgfpathlineto{\pgfqpoint{0.520010in}{0.699259in}}%
\pgfpathlineto{\pgfqpoint{0.522772in}{0.712870in}}%
\pgfpathlineto{\pgfqpoint{0.527117in}{0.726481in}}%
\pgfpathlineto{\pgfqpoint{0.532637in}{0.739160in}}%
\pgfpathlineto{\pgfqpoint{0.533038in}{0.740092in}}%
\pgfpathlineto{\pgfqpoint{0.540384in}{0.753703in}}%
\pgfpathlineto{\pgfqpoint{0.548294in}{0.765826in}}%
\pgfpathlineto{\pgfqpoint{0.549274in}{0.767314in}}%
\pgfpathlineto{\pgfqpoint{0.559641in}{0.780925in}}%
\pgfpathlineto{\pgfqpoint{0.563950in}{0.785929in}}%
\pgfpathlineto{\pgfqpoint{0.571591in}{0.794536in}}%
\pgfpathlineto{\pgfqpoint{0.579607in}{0.802707in}}%
\pgfpathlineto{\pgfqpoint{0.585212in}{0.808148in}}%
\pgfpathlineto{\pgfqpoint{0.595263in}{0.817167in}}%
\pgfpathlineto{\pgfqpoint{0.600742in}{0.821759in}}%
\pgfpathlineto{\pgfqpoint{0.610920in}{0.829803in}}%
\pgfpathlineto{\pgfqpoint{0.618618in}{0.835370in}}%
\pgfpathlineto{\pgfqpoint{0.626577in}{0.840901in}}%
\pgfpathlineto{\pgfqpoint{0.639576in}{0.848981in}}%
\pgfpathlineto{\pgfqpoint{0.642233in}{0.850598in}}%
\pgfpathlineto{\pgfqpoint{0.657890in}{0.858958in}}%
\pgfpathlineto{\pgfqpoint{0.665933in}{0.862592in}}%
\pgfpathlineto{\pgfqpoint{0.673546in}{0.866032in}}%
\pgfpathlineto{\pgfqpoint{0.689203in}{0.871798in}}%
\pgfpathlineto{\pgfqpoint{0.704857in}{0.876203in}}%
\pgfpathlineto{\pgfqpoint{0.704859in}{0.876204in}}%
\pgfpathlineto{\pgfqpoint{0.720516in}{0.879349in}}%
\pgfpathlineto{\pgfqpoint{0.736173in}{0.881094in}}%
\pgfpathlineto{\pgfqpoint{0.751829in}{0.881443in}}%
\pgfpathlineto{\pgfqpoint{0.767486in}{0.880396in}}%
\pgfpathlineto{\pgfqpoint{0.783142in}{0.877951in}}%
\pgfpathlineto{\pgfqpoint{0.790291in}{0.876203in}}%
\pgfpathlineto{\pgfqpoint{0.798799in}{0.874171in}}%
\pgfpathlineto{\pgfqpoint{0.814455in}{0.869085in}}%
\pgfpathlineto{\pgfqpoint{0.830112in}{0.862638in}}%
\pgfpathlineto{\pgfqpoint{0.830205in}{0.862592in}}%
\pgfpathlineto{\pgfqpoint{0.845769in}{0.854944in}}%
\pgfpathlineto{\pgfqpoint{0.856180in}{0.848981in}}%
\pgfpathlineto{\pgfqpoint{0.861425in}{0.845915in}}%
\pgfpathlineto{\pgfqpoint{0.877082in}{0.835562in}}%
\pgfpathlineto{\pgfqpoint{0.877346in}{0.835370in}}%
\pgfpathlineto{\pgfqpoint{0.892738in}{0.823720in}}%
\pgfpathlineto{\pgfqpoint{0.895132in}{0.821759in}}%
\pgfpathlineto{\pgfqpoint{0.908395in}{0.810229in}}%
\pgfpathlineto{\pgfqpoint{0.910651in}{0.808148in}}%
\pgfpathlineto{\pgfqpoint{0.924051in}{0.794766in}}%
\pgfpathlineto{\pgfqpoint{0.924272in}{0.794536in}}%
\pgfpathlineto{\pgfqpoint{0.936182in}{0.780925in}}%
\pgfpathlineto{\pgfqpoint{0.939708in}{0.776365in}}%
\pgfpathlineto{\pgfqpoint{0.946568in}{0.767314in}}%
\pgfpathlineto{\pgfqpoint{0.955364in}{0.753784in}}%
\pgfpathlineto{\pgfqpoint{0.955417in}{0.753703in}}%
\pgfpathlineto{\pgfqpoint{0.962833in}{0.740092in}}%
\pgfpathlineto{\pgfqpoint{0.968684in}{0.726481in}}%
\pgfpathlineto{\pgfqpoint{0.971021in}{0.719085in}}%
\pgfpathlineto{\pgfqpoint{0.973032in}{0.712870in}}%
\pgfpathlineto{\pgfqpoint{0.975844in}{0.699259in}}%
\pgfpathlineto{\pgfqpoint{0.977049in}{0.685648in}}%
\pgfpathlineto{\pgfqpoint{0.976647in}{0.672036in}}%
\pgfpathlineto{\pgfqpoint{0.974639in}{0.658425in}}%
\pgfpathlineto{\pgfqpoint{0.971022in}{0.644814in}}%
\pgfpathlineto{\pgfqpoint{0.971021in}{0.644812in}}%
\pgfpathlineto{\pgfqpoint{0.965954in}{0.631203in}}%
\pgfpathlineto{\pgfqpoint{0.959321in}{0.617592in}}%
\pgfpathlineto{\pgfqpoint{0.955364in}{0.610973in}}%
\pgfpathlineto{\pgfqpoint{0.951184in}{0.603981in}}%
\pgfpathlineto{\pgfqpoint{0.941568in}{0.590370in}}%
\pgfpathlineto{\pgfqpoint{0.939708in}{0.588060in}}%
\pgfpathlineto{\pgfqpoint{0.930414in}{0.576759in}}%
\pgfpathlineto{\pgfqpoint{0.924051in}{0.569839in}}%
\pgfpathlineto{\pgfqpoint{0.917648in}{0.563148in}}%
\pgfpathlineto{\pgfqpoint{0.908395in}{0.554300in}}%
\pgfpathlineto{\pgfqpoint{0.903113in}{0.549536in}}%
\pgfpathlineto{\pgfqpoint{0.892738in}{0.540798in}}%
\pgfpathlineto{\pgfqpoint{0.886480in}{0.535925in}}%
\pgfpathlineto{\pgfqpoint{0.877082in}{0.528956in}}%
\pgfpathlineto{\pgfqpoint{0.867181in}{0.522314in}}%
\pgfpathlineto{\pgfqpoint{0.861425in}{0.518568in}}%
\pgfpathlineto{\pgfqpoint{0.845769in}{0.509555in}}%
\pgfpathlineto{\pgfqpoint{0.844056in}{0.508703in}}%
\pgfpathlineto{\pgfqpoint{0.830112in}{0.501827in}}%
\pgfpathlineto{\pgfqpoint{0.814455in}{0.495440in}}%
\pgfpathlineto{\pgfqpoint{0.813384in}{0.495092in}}%
\pgfpathlineto{\pgfqpoint{0.798799in}{0.490293in}}%
\pgfpathlineto{\pgfqpoint{0.783142in}{0.486516in}}%
\pgfpathlineto{\pgfqpoint{0.767486in}{0.484114in}}%
\pgfpathlineto{\pgfqpoint{0.751829in}{0.483085in}}%
\pgfpathlineto{\pgfqpoint{0.736173in}{0.483428in}}%
\pgfpathlineto{\pgfqpoint{0.720516in}{0.485143in}}%
\pgfpathlineto{\pgfqpoint{0.704859in}{0.488232in}}%
\pgfpathlineto{\pgfqpoint{0.689203in}{0.492698in}}%
\pgfpathlineto{\pgfqpoint{0.682745in}{0.495092in}}%
\pgfpathclose%
\pgfpathmoveto{\pgfqpoint{1.519001in}{0.412724in}}%
\pgfpathlineto{\pgfqpoint{1.534657in}{0.413265in}}%
\pgfpathlineto{\pgfqpoint{1.535586in}{0.413425in}}%
\pgfpathlineto{\pgfqpoint{1.550314in}{0.415566in}}%
\pgfpathlineto{\pgfqpoint{1.565971in}{0.419665in}}%
\pgfpathlineto{\pgfqpoint{1.581627in}{0.425590in}}%
\pgfpathlineto{\pgfqpoint{1.584576in}{0.427036in}}%
\pgfpathlineto{\pgfqpoint{1.597284in}{0.432555in}}%
\pgfpathlineto{\pgfqpoint{1.612402in}{0.440648in}}%
\pgfpathlineto{\pgfqpoint{1.612940in}{0.440911in}}%
\pgfpathlineto{\pgfqpoint{1.628597in}{0.449908in}}%
\pgfpathlineto{\pgfqpoint{1.635208in}{0.454259in}}%
\pgfpathlineto{\pgfqpoint{1.644253in}{0.459867in}}%
\pgfpathlineto{\pgfqpoint{1.655795in}{0.467870in}}%
\pgfpathlineto{\pgfqpoint{1.659910in}{0.470620in}}%
\pgfpathlineto{\pgfqpoint{1.674743in}{0.481481in}}%
\pgfpathlineto{\pgfqpoint{1.675567in}{0.482074in}}%
\pgfpathlineto{\pgfqpoint{1.691223in}{0.494178in}}%
\pgfpathlineto{\pgfqpoint{1.692344in}{0.495092in}}%
\pgfpathlineto{\pgfqpoint{1.706880in}{0.506997in}}%
\pgfpathlineto{\pgfqpoint{1.708885in}{0.508703in}}%
\pgfpathlineto{\pgfqpoint{1.722536in}{0.520571in}}%
\pgfpathlineto{\pgfqpoint{1.724499in}{0.522314in}}%
\pgfpathlineto{\pgfqpoint{1.738193in}{0.534951in}}%
\pgfpathlineto{\pgfqpoint{1.739244in}{0.535925in}}%
\pgfpathlineto{\pgfqpoint{1.753167in}{0.549536in}}%
\pgfpathlineto{\pgfqpoint{1.753849in}{0.550253in}}%
\pgfpathlineto{\pgfqpoint{1.766342in}{0.563148in}}%
\pgfpathlineto{\pgfqpoint{1.769506in}{0.566725in}}%
\pgfpathlineto{\pgfqpoint{1.778711in}{0.576759in}}%
\pgfpathlineto{\pgfqpoint{1.785162in}{0.584622in}}%
\pgfpathlineto{\pgfqpoint{1.790167in}{0.590370in}}%
\pgfpathlineto{\pgfqpoint{1.800516in}{0.603981in}}%
\pgfpathlineto{\pgfqpoint{1.800819in}{0.604449in}}%
\pgfpathlineto{\pgfqpoint{1.810128in}{0.617592in}}%
\pgfpathlineto{\pgfqpoint{1.816476in}{0.628639in}}%
\pgfpathlineto{\pgfqpoint{1.818139in}{0.631203in}}%
\pgfpathlineto{\pgfqpoint{1.824955in}{0.644814in}}%
\pgfpathlineto{\pgfqpoint{1.829670in}{0.658425in}}%
\pgfpathlineto{\pgfqpoint{1.832132in}{0.671230in}}%
\pgfpathlineto{\pgfqpoint{1.832316in}{0.672036in}}%
\pgfpathlineto{\pgfqpoint{1.832938in}{0.685648in}}%
\pgfpathlineto{\pgfqpoint{1.832132in}{0.691532in}}%
\pgfpathlineto{\pgfqpoint{1.831240in}{0.699259in}}%
\pgfpathlineto{\pgfqpoint{1.827575in}{0.712870in}}%
\pgfpathlineto{\pgfqpoint{1.821810in}{0.726481in}}%
\pgfpathlineto{\pgfqpoint{1.816476in}{0.735741in}}%
\pgfpathlineto{\pgfqpoint{1.814260in}{0.740092in}}%
\pgfpathlineto{\pgfqpoint{1.805535in}{0.753703in}}%
\pgfpathlineto{\pgfqpoint{1.800819in}{0.759834in}}%
\pgfpathlineto{\pgfqpoint{1.795548in}{0.767314in}}%
\pgfpathlineto{\pgfqpoint{1.785162in}{0.779983in}}%
\pgfpathlineto{\pgfqpoint{1.784436in}{0.780925in}}%
\pgfpathlineto{\pgfqpoint{1.772615in}{0.794536in}}%
\pgfpathlineto{\pgfqpoint{1.769506in}{0.797777in}}%
\pgfpathlineto{\pgfqpoint{1.759906in}{0.808148in}}%
\pgfpathlineto{\pgfqpoint{1.753849in}{0.814180in}}%
\pgfpathlineto{\pgfqpoint{1.746355in}{0.821759in}}%
\pgfpathlineto{\pgfqpoint{1.738193in}{0.829505in}}%
\pgfpathlineto{\pgfqpoint{1.731992in}{0.835370in}}%
\pgfpathlineto{\pgfqpoint{1.722536in}{0.843907in}}%
\pgfpathlineto{\pgfqpoint{1.716798in}{0.848981in}}%
\pgfpathlineto{\pgfqpoint{1.706880in}{0.857495in}}%
\pgfpathlineto{\pgfqpoint{1.700712in}{0.862592in}}%
\pgfpathlineto{\pgfqpoint{1.691223in}{0.870338in}}%
\pgfpathlineto{\pgfqpoint{1.683634in}{0.876203in}}%
\pgfpathlineto{\pgfqpoint{1.675567in}{0.882473in}}%
\pgfpathlineto{\pgfqpoint{1.665412in}{0.889814in}}%
\pgfpathlineto{\pgfqpoint{1.659910in}{0.893894in}}%
\pgfpathlineto{\pgfqpoint{1.645834in}{0.903425in}}%
\pgfpathlineto{\pgfqpoint{1.644253in}{0.904546in}}%
\pgfpathlineto{\pgfqpoint{1.628597in}{0.914535in}}%
\pgfpathlineto{\pgfqpoint{1.624111in}{0.917036in}}%
\pgfpathlineto{\pgfqpoint{1.612940in}{0.923742in}}%
\pgfpathlineto{\pgfqpoint{1.599339in}{0.930648in}}%
\pgfpathlineto{\pgfqpoint{1.597284in}{0.931804in}}%
\pgfpathlineto{\pgfqpoint{1.581627in}{0.939027in}}%
\pgfpathlineto{\pgfqpoint{1.566792in}{0.944259in}}%
\pgfpathlineto{\pgfqpoint{1.565971in}{0.944593in}}%
\pgfpathlineto{\pgfqpoint{1.550314in}{0.949037in}}%
\pgfpathlineto{\pgfqpoint{1.534657in}{0.951504in}}%
\pgfpathlineto{\pgfqpoint{1.519001in}{0.951997in}}%
\pgfpathlineto{\pgfqpoint{1.503344in}{0.950517in}}%
\pgfpathlineto{\pgfqpoint{1.487688in}{0.947062in}}%
\pgfpathlineto{\pgfqpoint{1.479579in}{0.944259in}}%
\pgfpathlineto{\pgfqpoint{1.472031in}{0.941999in}}%
\pgfpathlineto{\pgfqpoint{1.456375in}{0.935629in}}%
\pgfpathlineto{\pgfqpoint{1.446666in}{0.930648in}}%
\pgfpathlineto{\pgfqpoint{1.440718in}{0.927898in}}%
\pgfpathlineto{\pgfqpoint{1.425061in}{0.919207in}}%
\pgfpathlineto{\pgfqpoint{1.421678in}{0.917036in}}%
\pgfpathlineto{\pgfqpoint{1.409405in}{0.909711in}}%
\pgfpathlineto{\pgfqpoint{1.400098in}{0.903425in}}%
\pgfpathlineto{\pgfqpoint{1.393748in}{0.899337in}}%
\pgfpathlineto{\pgfqpoint{1.380379in}{0.889814in}}%
\pgfpathlineto{\pgfqpoint{1.378092in}{0.888227in}}%
\pgfpathlineto{\pgfqpoint{1.362435in}{0.876444in}}%
\pgfpathlineto{\pgfqpoint{1.362134in}{0.876203in}}%
\pgfpathlineto{\pgfqpoint{1.346779in}{0.863995in}}%
\pgfpathlineto{\pgfqpoint{1.345095in}{0.862592in}}%
\pgfpathlineto{\pgfqpoint{1.331122in}{0.850803in}}%
\pgfpathlineto{\pgfqpoint{1.329027in}{0.848981in}}%
\pgfpathlineto{\pgfqpoint{1.315466in}{0.836834in}}%
\pgfpathlineto{\pgfqpoint{1.313852in}{0.835370in}}%
\pgfpathlineto{\pgfqpoint{1.299809in}{0.822021in}}%
\pgfpathlineto{\pgfqpoint{1.299532in}{0.821759in}}%
\pgfpathlineto{\pgfqpoint{1.285979in}{0.808148in}}%
\pgfpathlineto{\pgfqpoint{1.284152in}{0.806159in}}%
\pgfpathlineto{\pgfqpoint{1.273198in}{0.794536in}}%
\pgfpathlineto{\pgfqpoint{1.268496in}{0.789017in}}%
\pgfpathlineto{\pgfqpoint{1.261266in}{0.780925in}}%
\pgfpathlineto{\pgfqpoint{1.252839in}{0.770255in}}%
\pgfpathlineto{\pgfqpoint{1.250342in}{0.767314in}}%
\pgfpathlineto{\pgfqpoint{1.240345in}{0.753703in}}%
\pgfpathlineto{\pgfqpoint{1.237183in}{0.748532in}}%
\pgfpathlineto{\pgfqpoint{1.231452in}{0.740092in}}%
\pgfpathlineto{\pgfqpoint{1.224125in}{0.726481in}}%
\pgfpathlineto{\pgfqpoint{1.221526in}{0.719919in}}%
\pgfpathlineto{\pgfqpoint{1.218301in}{0.712870in}}%
\pgfpathlineto{\pgfqpoint{1.214327in}{0.699259in}}%
\pgfpathlineto{\pgfqpoint{1.212624in}{0.685648in}}%
\pgfpathlineto{\pgfqpoint{1.213192in}{0.672036in}}%
\pgfpathlineto{\pgfqpoint{1.216030in}{0.658425in}}%
\pgfpathlineto{\pgfqpoint{1.221142in}{0.644814in}}%
\pgfpathlineto{\pgfqpoint{1.221526in}{0.644100in}}%
\pgfpathlineto{\pgfqpoint{1.227544in}{0.631203in}}%
\pgfpathlineto{\pgfqpoint{1.235852in}{0.617592in}}%
\pgfpathlineto{\pgfqpoint{1.237183in}{0.615805in}}%
\pgfpathlineto{\pgfqpoint{1.245126in}{0.603981in}}%
\pgfpathlineto{\pgfqpoint{1.252839in}{0.594269in}}%
\pgfpathlineto{\pgfqpoint{1.255716in}{0.590370in}}%
\pgfpathlineto{\pgfqpoint{1.267206in}{0.576759in}}%
\pgfpathlineto{\pgfqpoint{1.268496in}{0.575385in}}%
\pgfpathlineto{\pgfqpoint{1.279460in}{0.563148in}}%
\pgfpathlineto{\pgfqpoint{1.284152in}{0.558364in}}%
\pgfpathlineto{\pgfqpoint{1.292597in}{0.549536in}}%
\pgfpathlineto{\pgfqpoint{1.299809in}{0.542523in}}%
\pgfpathlineto{\pgfqpoint{1.306555in}{0.535925in}}%
\pgfpathlineto{\pgfqpoint{1.315466in}{0.527676in}}%
\pgfpathlineto{\pgfqpoint{1.321328in}{0.522314in}}%
\pgfpathlineto{\pgfqpoint{1.331122in}{0.513692in}}%
\pgfpathlineto{\pgfqpoint{1.336958in}{0.508703in}}%
\pgfpathlineto{\pgfqpoint{1.346779in}{0.500483in}}%
\pgfpathlineto{\pgfqpoint{1.353525in}{0.495092in}}%
\pgfpathlineto{\pgfqpoint{1.362435in}{0.487996in}}%
\pgfpathlineto{\pgfqpoint{1.371153in}{0.481481in}}%
\pgfpathlineto{\pgfqpoint{1.378092in}{0.476215in}}%
\pgfpathlineto{\pgfqpoint{1.390020in}{0.467870in}}%
\pgfpathlineto{\pgfqpoint{1.393748in}{0.465167in}}%
\pgfpathlineto{\pgfqpoint{1.409405in}{0.454890in}}%
\pgfpathlineto{\pgfqpoint{1.410488in}{0.454259in}}%
\pgfpathlineto{\pgfqpoint{1.425061in}{0.445230in}}%
\pgfpathlineto{\pgfqpoint{1.433666in}{0.440648in}}%
\pgfpathlineto{\pgfqpoint{1.440718in}{0.436547in}}%
\pgfpathlineto{\pgfqpoint{1.456375in}{0.428963in}}%
\pgfpathlineto{\pgfqpoint{1.461379in}{0.427036in}}%
\pgfpathlineto{\pgfqpoint{1.472031in}{0.422399in}}%
\pgfpathlineto{\pgfqpoint{1.487688in}{0.417387in}}%
\pgfpathlineto{\pgfqpoint{1.503344in}{0.414201in}}%
\pgfpathlineto{\pgfqpoint{1.512232in}{0.413425in}}%
\pgfpathlineto{\pgfqpoint{1.519001in}{0.412724in}}%
\pgfpathclose%
\pgfpathmoveto{\pgfqpoint{1.457446in}{0.495092in}}%
\pgfpathlineto{\pgfqpoint{1.456375in}{0.495440in}}%
\pgfpathlineto{\pgfqpoint{1.440718in}{0.501827in}}%
\pgfpathlineto{\pgfqpoint{1.426774in}{0.508703in}}%
\pgfpathlineto{\pgfqpoint{1.425061in}{0.509555in}}%
\pgfpathlineto{\pgfqpoint{1.409405in}{0.518568in}}%
\pgfpathlineto{\pgfqpoint{1.403649in}{0.522314in}}%
\pgfpathlineto{\pgfqpoint{1.393748in}{0.528956in}}%
\pgfpathlineto{\pgfqpoint{1.384350in}{0.535925in}}%
\pgfpathlineto{\pgfqpoint{1.378092in}{0.540798in}}%
\pgfpathlineto{\pgfqpoint{1.367717in}{0.549536in}}%
\pgfpathlineto{\pgfqpoint{1.362435in}{0.554300in}}%
\pgfpathlineto{\pgfqpoint{1.353182in}{0.563148in}}%
\pgfpathlineto{\pgfqpoint{1.346779in}{0.569839in}}%
\pgfpathlineto{\pgfqpoint{1.340416in}{0.576759in}}%
\pgfpathlineto{\pgfqpoint{1.331122in}{0.588060in}}%
\pgfpathlineto{\pgfqpoint{1.329262in}{0.590370in}}%
\pgfpathlineto{\pgfqpoint{1.319646in}{0.603981in}}%
\pgfpathlineto{\pgfqpoint{1.315466in}{0.610973in}}%
\pgfpathlineto{\pgfqpoint{1.311509in}{0.617592in}}%
\pgfpathlineto{\pgfqpoint{1.304876in}{0.631203in}}%
\pgfpathlineto{\pgfqpoint{1.299809in}{0.644812in}}%
\pgfpathlineto{\pgfqpoint{1.299808in}{0.644814in}}%
\pgfpathlineto{\pgfqpoint{1.296191in}{0.658425in}}%
\pgfpathlineto{\pgfqpoint{1.294183in}{0.672036in}}%
\pgfpathlineto{\pgfqpoint{1.293781in}{0.685648in}}%
\pgfpathlineto{\pgfqpoint{1.294986in}{0.699259in}}%
\pgfpathlineto{\pgfqpoint{1.297798in}{0.712870in}}%
\pgfpathlineto{\pgfqpoint{1.299809in}{0.719085in}}%
\pgfpathlineto{\pgfqpoint{1.302146in}{0.726481in}}%
\pgfpathlineto{\pgfqpoint{1.307997in}{0.740092in}}%
\pgfpathlineto{\pgfqpoint{1.315413in}{0.753703in}}%
\pgfpathlineto{\pgfqpoint{1.315466in}{0.753784in}}%
\pgfpathlineto{\pgfqpoint{1.324262in}{0.767314in}}%
\pgfpathlineto{\pgfqpoint{1.331122in}{0.776365in}}%
\pgfpathlineto{\pgfqpoint{1.334648in}{0.780925in}}%
\pgfpathlineto{\pgfqpoint{1.346558in}{0.794536in}}%
\pgfpathlineto{\pgfqpoint{1.346779in}{0.794766in}}%
\pgfpathlineto{\pgfqpoint{1.360179in}{0.808148in}}%
\pgfpathlineto{\pgfqpoint{1.362435in}{0.810229in}}%
\pgfpathlineto{\pgfqpoint{1.375698in}{0.821759in}}%
\pgfpathlineto{\pgfqpoint{1.378092in}{0.823720in}}%
\pgfpathlineto{\pgfqpoint{1.393484in}{0.835370in}}%
\pgfpathlineto{\pgfqpoint{1.393748in}{0.835562in}}%
\pgfpathlineto{\pgfqpoint{1.409405in}{0.845915in}}%
\pgfpathlineto{\pgfqpoint{1.414650in}{0.848981in}}%
\pgfpathlineto{\pgfqpoint{1.425061in}{0.854944in}}%
\pgfpathlineto{\pgfqpoint{1.440625in}{0.862592in}}%
\pgfpathlineto{\pgfqpoint{1.440718in}{0.862638in}}%
\pgfpathlineto{\pgfqpoint{1.456375in}{0.869085in}}%
\pgfpathlineto{\pgfqpoint{1.472031in}{0.874171in}}%
\pgfpathlineto{\pgfqpoint{1.480539in}{0.876203in}}%
\pgfpathlineto{\pgfqpoint{1.487688in}{0.877951in}}%
\pgfpathlineto{\pgfqpoint{1.503344in}{0.880396in}}%
\pgfpathlineto{\pgfqpoint{1.519001in}{0.881443in}}%
\pgfpathlineto{\pgfqpoint{1.534657in}{0.881094in}}%
\pgfpathlineto{\pgfqpoint{1.550314in}{0.879349in}}%
\pgfpathlineto{\pgfqpoint{1.565971in}{0.876204in}}%
\pgfpathlineto{\pgfqpoint{1.565973in}{0.876203in}}%
\pgfpathlineto{\pgfqpoint{1.581627in}{0.871798in}}%
\pgfpathlineto{\pgfqpoint{1.597284in}{0.866032in}}%
\pgfpathlineto{\pgfqpoint{1.604897in}{0.862592in}}%
\pgfpathlineto{\pgfqpoint{1.612940in}{0.858958in}}%
\pgfpathlineto{\pgfqpoint{1.628597in}{0.850598in}}%
\pgfpathlineto{\pgfqpoint{1.631254in}{0.848981in}}%
\pgfpathlineto{\pgfqpoint{1.644253in}{0.840901in}}%
\pgfpathlineto{\pgfqpoint{1.652212in}{0.835370in}}%
\pgfpathlineto{\pgfqpoint{1.659910in}{0.829803in}}%
\pgfpathlineto{\pgfqpoint{1.670088in}{0.821759in}}%
\pgfpathlineto{\pgfqpoint{1.675567in}{0.817167in}}%
\pgfpathlineto{\pgfqpoint{1.685618in}{0.808148in}}%
\pgfpathlineto{\pgfqpoint{1.691223in}{0.802707in}}%
\pgfpathlineto{\pgfqpoint{1.699239in}{0.794536in}}%
\pgfpathlineto{\pgfqpoint{1.706880in}{0.785929in}}%
\pgfpathlineto{\pgfqpoint{1.711189in}{0.780925in}}%
\pgfpathlineto{\pgfqpoint{1.721556in}{0.767314in}}%
\pgfpathlineto{\pgfqpoint{1.722536in}{0.765826in}}%
\pgfpathlineto{\pgfqpoint{1.730446in}{0.753703in}}%
\pgfpathlineto{\pgfqpoint{1.737792in}{0.740092in}}%
\pgfpathlineto{\pgfqpoint{1.738193in}{0.739160in}}%
\pgfpathlineto{\pgfqpoint{1.743713in}{0.726481in}}%
\pgfpathlineto{\pgfqpoint{1.748058in}{0.712870in}}%
\pgfpathlineto{\pgfqpoint{1.750820in}{0.699259in}}%
\pgfpathlineto{\pgfqpoint{1.752004in}{0.685648in}}%
\pgfpathlineto{\pgfqpoint{1.751609in}{0.672036in}}%
\pgfpathlineto{\pgfqpoint{1.749637in}{0.658425in}}%
\pgfpathlineto{\pgfqpoint{1.746084in}{0.644814in}}%
\pgfpathlineto{\pgfqpoint{1.740946in}{0.631203in}}%
\pgfpathlineto{\pgfqpoint{1.738193in}{0.625589in}}%
\pgfpathlineto{\pgfqpoint{1.734313in}{0.617592in}}%
\pgfpathlineto{\pgfqpoint{1.726191in}{0.603981in}}%
\pgfpathlineto{\pgfqpoint{1.722536in}{0.598780in}}%
\pgfpathlineto{\pgfqpoint{1.716563in}{0.590370in}}%
\pgfpathlineto{\pgfqpoint{1.706880in}{0.578494in}}%
\pgfpathlineto{\pgfqpoint{1.705423in}{0.576759in}}%
\pgfpathlineto{\pgfqpoint{1.692683in}{0.563148in}}%
\pgfpathlineto{\pgfqpoint{1.691223in}{0.561718in}}%
\pgfpathlineto{\pgfqpoint{1.678147in}{0.549536in}}%
\pgfpathlineto{\pgfqpoint{1.675567in}{0.547293in}}%
\pgfpathlineto{\pgfqpoint{1.661555in}{0.535925in}}%
\pgfpathlineto{\pgfqpoint{1.659910in}{0.534656in}}%
\pgfpathlineto{\pgfqpoint{1.644253in}{0.523581in}}%
\pgfpathlineto{\pgfqpoint{1.642258in}{0.522314in}}%
\pgfpathlineto{\pgfqpoint{1.628597in}{0.513896in}}%
\pgfpathlineto{\pgfqpoint{1.618923in}{0.508703in}}%
\pgfpathlineto{\pgfqpoint{1.612940in}{0.505526in}}%
\pgfpathlineto{\pgfqpoint{1.597284in}{0.498465in}}%
\pgfpathlineto{\pgfqpoint{1.588085in}{0.495092in}}%
\pgfpathlineto{\pgfqpoint{1.581627in}{0.492698in}}%
\pgfpathlineto{\pgfqpoint{1.565971in}{0.488232in}}%
\pgfpathlineto{\pgfqpoint{1.550314in}{0.485143in}}%
\pgfpathlineto{\pgfqpoint{1.534657in}{0.483428in}}%
\pgfpathlineto{\pgfqpoint{1.519001in}{0.483085in}}%
\pgfpathlineto{\pgfqpoint{1.503344in}{0.484114in}}%
\pgfpathlineto{\pgfqpoint{1.487688in}{0.486516in}}%
\pgfpathlineto{\pgfqpoint{1.472031in}{0.490293in}}%
\pgfpathlineto{\pgfqpoint{1.457446in}{0.495092in}}%
\pgfpathclose%
\pgfpathmoveto{\pgfqpoint{0.704859in}{1.093647in}}%
\pgfpathlineto{\pgfqpoint{0.720516in}{1.089203in}}%
\pgfpathlineto{\pgfqpoint{0.736173in}{1.086735in}}%
\pgfpathlineto{\pgfqpoint{0.751829in}{1.086242in}}%
\pgfpathlineto{\pgfqpoint{0.767486in}{1.087722in}}%
\pgfpathlineto{\pgfqpoint{0.783142in}{1.091177in}}%
\pgfpathlineto{\pgfqpoint{0.791251in}{1.093981in}}%
\pgfpathlineto{\pgfqpoint{0.798799in}{1.096240in}}%
\pgfpathlineto{\pgfqpoint{0.814455in}{1.102610in}}%
\pgfpathlineto{\pgfqpoint{0.824164in}{1.107592in}}%
\pgfpathlineto{\pgfqpoint{0.830112in}{1.110342in}}%
\pgfpathlineto{\pgfqpoint{0.845769in}{1.119032in}}%
\pgfpathlineto{\pgfqpoint{0.849152in}{1.121203in}}%
\pgfpathlineto{\pgfqpoint{0.861425in}{1.128529in}}%
\pgfpathlineto{\pgfqpoint{0.870732in}{1.134814in}}%
\pgfpathlineto{\pgfqpoint{0.877082in}{1.138902in}}%
\pgfpathlineto{\pgfqpoint{0.890451in}{1.148425in}}%
\pgfpathlineto{\pgfqpoint{0.892738in}{1.150013in}}%
\pgfpathlineto{\pgfqpoint{0.908395in}{1.161795in}}%
\pgfpathlineto{\pgfqpoint{0.908696in}{1.162036in}}%
\pgfpathlineto{\pgfqpoint{0.924051in}{1.174245in}}%
\pgfpathlineto{\pgfqpoint{0.925735in}{1.175647in}}%
\pgfpathlineto{\pgfqpoint{0.939708in}{1.187437in}}%
\pgfpathlineto{\pgfqpoint{0.941803in}{1.189259in}}%
\pgfpathlineto{\pgfqpoint{0.955364in}{1.201406in}}%
\pgfpathlineto{\pgfqpoint{0.956978in}{1.202870in}}%
\pgfpathlineto{\pgfqpoint{0.971021in}{1.216219in}}%
\pgfpathlineto{\pgfqpoint{0.971298in}{1.216481in}}%
\pgfpathlineto{\pgfqpoint{0.984851in}{1.230092in}}%
\pgfpathlineto{\pgfqpoint{0.986678in}{1.232081in}}%
\pgfpathlineto{\pgfqpoint{0.997632in}{1.243703in}}%
\pgfpathlineto{\pgfqpoint{1.002334in}{1.249223in}}%
\pgfpathlineto{\pgfqpoint{1.009564in}{1.257314in}}%
\pgfpathlineto{\pgfqpoint{1.017991in}{1.267984in}}%
\pgfpathlineto{\pgfqpoint{1.020488in}{1.270925in}}%
\pgfpathlineto{\pgfqpoint{1.030485in}{1.284536in}}%
\pgfpathlineto{\pgfqpoint{1.033647in}{1.289708in}}%
\pgfpathlineto{\pgfqpoint{1.039378in}{1.298148in}}%
\pgfpathlineto{\pgfqpoint{1.046705in}{1.311759in}}%
\pgfpathlineto{\pgfqpoint{1.049304in}{1.318320in}}%
\pgfpathlineto{\pgfqpoint{1.052529in}{1.325370in}}%
\pgfpathlineto{\pgfqpoint{1.056503in}{1.338981in}}%
\pgfpathlineto{\pgfqpoint{1.058206in}{1.352592in}}%
\pgfpathlineto{\pgfqpoint{1.057638in}{1.366203in}}%
\pgfpathlineto{\pgfqpoint{1.054800in}{1.379814in}}%
\pgfpathlineto{\pgfqpoint{1.049688in}{1.393425in}}%
\pgfpathlineto{\pgfqpoint{1.049304in}{1.394139in}}%
\pgfpathlineto{\pgfqpoint{1.043286in}{1.407036in}}%
\pgfpathlineto{\pgfqpoint{1.034978in}{1.420648in}}%
\pgfpathlineto{\pgfqpoint{1.033647in}{1.422434in}}%
\pgfpathlineto{\pgfqpoint{1.025704in}{1.434259in}}%
\pgfpathlineto{\pgfqpoint{1.017991in}{1.443970in}}%
\pgfpathlineto{\pgfqpoint{1.015114in}{1.447870in}}%
\pgfpathlineto{\pgfqpoint{1.003624in}{1.461481in}}%
\pgfpathlineto{\pgfqpoint{1.002334in}{1.462855in}}%
\pgfpathlineto{\pgfqpoint{0.991370in}{1.475092in}}%
\pgfpathlineto{\pgfqpoint{0.986678in}{1.479876in}}%
\pgfpathlineto{\pgfqpoint{0.978233in}{1.488703in}}%
\pgfpathlineto{\pgfqpoint{0.971021in}{1.495716in}}%
\pgfpathlineto{\pgfqpoint{0.964275in}{1.502314in}}%
\pgfpathlineto{\pgfqpoint{0.955364in}{1.510564in}}%
\pgfpathlineto{\pgfqpoint{0.949502in}{1.515925in}}%
\pgfpathlineto{\pgfqpoint{0.939708in}{1.524548in}}%
\pgfpathlineto{\pgfqpoint{0.933872in}{1.529536in}}%
\pgfpathlineto{\pgfqpoint{0.924051in}{1.537757in}}%
\pgfpathlineto{\pgfqpoint{0.917305in}{1.543148in}}%
\pgfpathlineto{\pgfqpoint{0.908395in}{1.550243in}}%
\pgfpathlineto{\pgfqpoint{0.899677in}{1.556759in}}%
\pgfpathlineto{\pgfqpoint{0.892738in}{1.562024in}}%
\pgfpathlineto{\pgfqpoint{0.880810in}{1.570370in}}%
\pgfpathlineto{\pgfqpoint{0.877082in}{1.573072in}}%
\pgfpathlineto{\pgfqpoint{0.861425in}{1.583349in}}%
\pgfpathlineto{\pgfqpoint{0.860342in}{1.583981in}}%
\pgfpathlineto{\pgfqpoint{0.845769in}{1.593009in}}%
\pgfpathlineto{\pgfqpoint{0.837164in}{1.597592in}}%
\pgfpathlineto{\pgfqpoint{0.830112in}{1.601692in}}%
\pgfpathlineto{\pgfqpoint{0.814455in}{1.609277in}}%
\pgfpathlineto{\pgfqpoint{0.809451in}{1.611203in}}%
\pgfpathlineto{\pgfqpoint{0.798799in}{1.615841in}}%
\pgfpathlineto{\pgfqpoint{0.783142in}{1.620852in}}%
\pgfpathlineto{\pgfqpoint{0.767486in}{1.624039in}}%
\pgfpathlineto{\pgfqpoint{0.758598in}{1.624814in}}%
\pgfpathlineto{\pgfqpoint{0.751829in}{1.625515in}}%
\pgfpathlineto{\pgfqpoint{0.736173in}{1.624974in}}%
\pgfpathlineto{\pgfqpoint{0.735244in}{1.624814in}}%
\pgfpathlineto{\pgfqpoint{0.720516in}{1.622673in}}%
\pgfpathlineto{\pgfqpoint{0.704859in}{1.618575in}}%
\pgfpathlineto{\pgfqpoint{0.689203in}{1.612649in}}%
\pgfpathlineto{\pgfqpoint{0.686254in}{1.611203in}}%
\pgfpathlineto{\pgfqpoint{0.673546in}{1.605685in}}%
\pgfpathlineto{\pgfqpoint{0.658428in}{1.597592in}}%
\pgfpathlineto{\pgfqpoint{0.657890in}{1.597328in}}%
\pgfpathlineto{\pgfqpoint{0.642233in}{1.588331in}}%
\pgfpathlineto{\pgfqpoint{0.635622in}{1.583981in}}%
\pgfpathlineto{\pgfqpoint{0.626577in}{1.578373in}}%
\pgfpathlineto{\pgfqpoint{0.615035in}{1.570370in}}%
\pgfpathlineto{\pgfqpoint{0.610920in}{1.567619in}}%
\pgfpathlineto{\pgfqpoint{0.596087in}{1.556759in}}%
\pgfpathlineto{\pgfqpoint{0.595263in}{1.556165in}}%
\pgfpathlineto{\pgfqpoint{0.579607in}{1.544062in}}%
\pgfpathlineto{\pgfqpoint{0.578486in}{1.543148in}}%
\pgfpathlineto{\pgfqpoint{0.563950in}{1.531243in}}%
\pgfpathlineto{\pgfqpoint{0.561945in}{1.529536in}}%
\pgfpathlineto{\pgfqpoint{0.548294in}{1.517669in}}%
\pgfpathlineto{\pgfqpoint{0.546331in}{1.515925in}}%
\pgfpathlineto{\pgfqpoint{0.532637in}{1.503288in}}%
\pgfpathlineto{\pgfqpoint{0.531586in}{1.502314in}}%
\pgfpathlineto{\pgfqpoint{0.517663in}{1.488703in}}%
\pgfpathlineto{\pgfqpoint{0.516981in}{1.487987in}}%
\pgfpathlineto{\pgfqpoint{0.504488in}{1.475092in}}%
\pgfpathlineto{\pgfqpoint{0.501324in}{1.471515in}}%
\pgfpathlineto{\pgfqpoint{0.492119in}{1.461481in}}%
\pgfpathlineto{\pgfqpoint{0.485668in}{1.453617in}}%
\pgfpathlineto{\pgfqpoint{0.480663in}{1.447870in}}%
\pgfpathlineto{\pgfqpoint{0.470314in}{1.434259in}}%
\pgfpathlineto{\pgfqpoint{0.470011in}{1.433790in}}%
\pgfpathlineto{\pgfqpoint{0.460702in}{1.420648in}}%
\pgfpathlineto{\pgfqpoint{0.454354in}{1.409600in}}%
\pgfpathlineto{\pgfqpoint{0.452691in}{1.407036in}}%
\pgfpathlineto{\pgfqpoint{0.445875in}{1.393425in}}%
\pgfpathlineto{\pgfqpoint{0.441160in}{1.379814in}}%
\pgfpathlineto{\pgfqpoint{0.438698in}{1.367010in}}%
\pgfpathlineto{\pgfqpoint{0.438514in}{1.366203in}}%
\pgfpathlineto{\pgfqpoint{0.437892in}{1.352592in}}%
\pgfpathlineto{\pgfqpoint{0.438698in}{1.346708in}}%
\pgfpathlineto{\pgfqpoint{0.439590in}{1.338981in}}%
\pgfpathlineto{\pgfqpoint{0.443255in}{1.325370in}}%
\pgfpathlineto{\pgfqpoint{0.449020in}{1.311759in}}%
\pgfpathlineto{\pgfqpoint{0.454354in}{1.302498in}}%
\pgfpathlineto{\pgfqpoint{0.456570in}{1.298148in}}%
\pgfpathlineto{\pgfqpoint{0.465295in}{1.284536in}}%
\pgfpathlineto{\pgfqpoint{0.470011in}{1.278406in}}%
\pgfpathlineto{\pgfqpoint{0.475282in}{1.270925in}}%
\pgfpathlineto{\pgfqpoint{0.485668in}{1.258256in}}%
\pgfpathlineto{\pgfqpoint{0.486394in}{1.257314in}}%
\pgfpathlineto{\pgfqpoint{0.498215in}{1.243703in}}%
\pgfpathlineto{\pgfqpoint{0.501324in}{1.240462in}}%
\pgfpathlineto{\pgfqpoint{0.510924in}{1.230092in}}%
\pgfpathlineto{\pgfqpoint{0.516981in}{1.224059in}}%
\pgfpathlineto{\pgfqpoint{0.524475in}{1.216481in}}%
\pgfpathlineto{\pgfqpoint{0.532637in}{1.208734in}}%
\pgfpathlineto{\pgfqpoint{0.538838in}{1.202870in}}%
\pgfpathlineto{\pgfqpoint{0.548294in}{1.194332in}}%
\pgfpathlineto{\pgfqpoint{0.554032in}{1.189259in}}%
\pgfpathlineto{\pgfqpoint{0.563950in}{1.180744in}}%
\pgfpathlineto{\pgfqpoint{0.570118in}{1.175647in}}%
\pgfpathlineto{\pgfqpoint{0.579607in}{1.167901in}}%
\pgfpathlineto{\pgfqpoint{0.587196in}{1.162036in}}%
\pgfpathlineto{\pgfqpoint{0.595263in}{1.155767in}}%
\pgfpathlineto{\pgfqpoint{0.605418in}{1.148425in}}%
\pgfpathlineto{\pgfqpoint{0.610920in}{1.144346in}}%
\pgfpathlineto{\pgfqpoint{0.624996in}{1.134814in}}%
\pgfpathlineto{\pgfqpoint{0.626577in}{1.133693in}}%
\pgfpathlineto{\pgfqpoint{0.642233in}{1.123704in}}%
\pgfpathlineto{\pgfqpoint{0.646719in}{1.121203in}}%
\pgfpathlineto{\pgfqpoint{0.657890in}{1.114498in}}%
\pgfpathlineto{\pgfqpoint{0.671491in}{1.107592in}}%
\pgfpathlineto{\pgfqpoint{0.673546in}{1.106435in}}%
\pgfpathlineto{\pgfqpoint{0.689203in}{1.099212in}}%
\pgfpathlineto{\pgfqpoint{0.704038in}{1.093981in}}%
\pgfpathlineto{\pgfqpoint{0.704859in}{1.093647in}}%
\pgfpathclose%
\pgfpathmoveto{\pgfqpoint{0.704857in}{1.162036in}}%
\pgfpathlineto{\pgfqpoint{0.689203in}{1.166441in}}%
\pgfpathlineto{\pgfqpoint{0.673546in}{1.172208in}}%
\pgfpathlineto{\pgfqpoint{0.665933in}{1.175647in}}%
\pgfpathlineto{\pgfqpoint{0.657890in}{1.179282in}}%
\pgfpathlineto{\pgfqpoint{0.642233in}{1.187642in}}%
\pgfpathlineto{\pgfqpoint{0.639576in}{1.189259in}}%
\pgfpathlineto{\pgfqpoint{0.626577in}{1.197338in}}%
\pgfpathlineto{\pgfqpoint{0.618618in}{1.202870in}}%
\pgfpathlineto{\pgfqpoint{0.610920in}{1.208436in}}%
\pgfpathlineto{\pgfqpoint{0.600742in}{1.216481in}}%
\pgfpathlineto{\pgfqpoint{0.595263in}{1.221073in}}%
\pgfpathlineto{\pgfqpoint{0.585212in}{1.230092in}}%
\pgfpathlineto{\pgfqpoint{0.579607in}{1.235532in}}%
\pgfpathlineto{\pgfqpoint{0.571591in}{1.243703in}}%
\pgfpathlineto{\pgfqpoint{0.563950in}{1.252310in}}%
\pgfpathlineto{\pgfqpoint{0.559641in}{1.257314in}}%
\pgfpathlineto{\pgfqpoint{0.549274in}{1.270925in}}%
\pgfpathlineto{\pgfqpoint{0.548294in}{1.272414in}}%
\pgfpathlineto{\pgfqpoint{0.540384in}{1.284536in}}%
\pgfpathlineto{\pgfqpoint{0.533038in}{1.298148in}}%
\pgfpathlineto{\pgfqpoint{0.532637in}{1.299079in}}%
\pgfpathlineto{\pgfqpoint{0.527117in}{1.311759in}}%
\pgfpathlineto{\pgfqpoint{0.522772in}{1.325370in}}%
\pgfpathlineto{\pgfqpoint{0.520010in}{1.338981in}}%
\pgfpathlineto{\pgfqpoint{0.518826in}{1.352592in}}%
\pgfpathlineto{\pgfqpoint{0.519221in}{1.366203in}}%
\pgfpathlineto{\pgfqpoint{0.521193in}{1.379814in}}%
\pgfpathlineto{\pgfqpoint{0.524746in}{1.393425in}}%
\pgfpathlineto{\pgfqpoint{0.529884in}{1.407036in}}%
\pgfpathlineto{\pgfqpoint{0.532637in}{1.412651in}}%
\pgfpathlineto{\pgfqpoint{0.536517in}{1.420648in}}%
\pgfpathlineto{\pgfqpoint{0.544639in}{1.434259in}}%
\pgfpathlineto{\pgfqpoint{0.548294in}{1.439459in}}%
\pgfpathlineto{\pgfqpoint{0.554267in}{1.447870in}}%
\pgfpathlineto{\pgfqpoint{0.563950in}{1.459746in}}%
\pgfpathlineto{\pgfqpoint{0.565407in}{1.461481in}}%
\pgfpathlineto{\pgfqpoint{0.578147in}{1.475092in}}%
\pgfpathlineto{\pgfqpoint{0.579607in}{1.476522in}}%
\pgfpathlineto{\pgfqpoint{0.592683in}{1.488703in}}%
\pgfpathlineto{\pgfqpoint{0.595263in}{1.490946in}}%
\pgfpathlineto{\pgfqpoint{0.609275in}{1.502314in}}%
\pgfpathlineto{\pgfqpoint{0.610920in}{1.503583in}}%
\pgfpathlineto{\pgfqpoint{0.626577in}{1.514659in}}%
\pgfpathlineto{\pgfqpoint{0.628572in}{1.515925in}}%
\pgfpathlineto{\pgfqpoint{0.642233in}{1.524343in}}%
\pgfpathlineto{\pgfqpoint{0.651907in}{1.529536in}}%
\pgfpathlineto{\pgfqpoint{0.657890in}{1.532713in}}%
\pgfpathlineto{\pgfqpoint{0.673546in}{1.539774in}}%
\pgfpathlineto{\pgfqpoint{0.682745in}{1.543148in}}%
\pgfpathlineto{\pgfqpoint{0.689203in}{1.545541in}}%
\pgfpathlineto{\pgfqpoint{0.704859in}{1.550007in}}%
\pgfpathlineto{\pgfqpoint{0.720516in}{1.553096in}}%
\pgfpathlineto{\pgfqpoint{0.736173in}{1.554811in}}%
\pgfpathlineto{\pgfqpoint{0.751829in}{1.555154in}}%
\pgfpathlineto{\pgfqpoint{0.767486in}{1.554125in}}%
\pgfpathlineto{\pgfqpoint{0.783142in}{1.551724in}}%
\pgfpathlineto{\pgfqpoint{0.798799in}{1.547947in}}%
\pgfpathlineto{\pgfqpoint{0.813384in}{1.543148in}}%
\pgfpathlineto{\pgfqpoint{0.814455in}{1.542799in}}%
\pgfpathlineto{\pgfqpoint{0.830112in}{1.536412in}}%
\pgfpathlineto{\pgfqpoint{0.844056in}{1.529536in}}%
\pgfpathlineto{\pgfqpoint{0.845769in}{1.528684in}}%
\pgfpathlineto{\pgfqpoint{0.861425in}{1.519671in}}%
\pgfpathlineto{\pgfqpoint{0.867181in}{1.515925in}}%
\pgfpathlineto{\pgfqpoint{0.877082in}{1.509283in}}%
\pgfpathlineto{\pgfqpoint{0.886480in}{1.502314in}}%
\pgfpathlineto{\pgfqpoint{0.892738in}{1.497441in}}%
\pgfpathlineto{\pgfqpoint{0.903113in}{1.488703in}}%
\pgfpathlineto{\pgfqpoint{0.908395in}{1.483940in}}%
\pgfpathlineto{\pgfqpoint{0.917648in}{1.475092in}}%
\pgfpathlineto{\pgfqpoint{0.924051in}{1.468400in}}%
\pgfpathlineto{\pgfqpoint{0.930414in}{1.461481in}}%
\pgfpathlineto{\pgfqpoint{0.939708in}{1.450180in}}%
\pgfpathlineto{\pgfqpoint{0.941568in}{1.447870in}}%
\pgfpathlineto{\pgfqpoint{0.951184in}{1.434259in}}%
\pgfpathlineto{\pgfqpoint{0.955364in}{1.427266in}}%
\pgfpathlineto{\pgfqpoint{0.959321in}{1.420648in}}%
\pgfpathlineto{\pgfqpoint{0.965954in}{1.407036in}}%
\pgfpathlineto{\pgfqpoint{0.971021in}{1.393428in}}%
\pgfpathlineto{\pgfqpoint{0.971022in}{1.393425in}}%
\pgfpathlineto{\pgfqpoint{0.974639in}{1.379814in}}%
\pgfpathlineto{\pgfqpoint{0.976647in}{1.366203in}}%
\pgfpathlineto{\pgfqpoint{0.977049in}{1.352592in}}%
\pgfpathlineto{\pgfqpoint{0.975844in}{1.338981in}}%
\pgfpathlineto{\pgfqpoint{0.973032in}{1.325370in}}%
\pgfpathlineto{\pgfqpoint{0.971021in}{1.319155in}}%
\pgfpathlineto{\pgfqpoint{0.968684in}{1.311759in}}%
\pgfpathlineto{\pgfqpoint{0.962833in}{1.298148in}}%
\pgfpathlineto{\pgfqpoint{0.955417in}{1.284536in}}%
\pgfpathlineto{\pgfqpoint{0.955364in}{1.284456in}}%
\pgfpathlineto{\pgfqpoint{0.946568in}{1.270925in}}%
\pgfpathlineto{\pgfqpoint{0.939708in}{1.261874in}}%
\pgfpathlineto{\pgfqpoint{0.936182in}{1.257314in}}%
\pgfpathlineto{\pgfqpoint{0.924272in}{1.243703in}}%
\pgfpathlineto{\pgfqpoint{0.924051in}{1.243474in}}%
\pgfpathlineto{\pgfqpoint{0.910651in}{1.230092in}}%
\pgfpathlineto{\pgfqpoint{0.908395in}{1.228011in}}%
\pgfpathlineto{\pgfqpoint{0.895132in}{1.216481in}}%
\pgfpathlineto{\pgfqpoint{0.892738in}{1.214520in}}%
\pgfpathlineto{\pgfqpoint{0.877346in}{1.202870in}}%
\pgfpathlineto{\pgfqpoint{0.877082in}{1.202678in}}%
\pgfpathlineto{\pgfqpoint{0.861425in}{1.192324in}}%
\pgfpathlineto{\pgfqpoint{0.856180in}{1.189259in}}%
\pgfpathlineto{\pgfqpoint{0.845769in}{1.183295in}}%
\pgfpathlineto{\pgfqpoint{0.830205in}{1.175647in}}%
\pgfpathlineto{\pgfqpoint{0.830112in}{1.175602in}}%
\pgfpathlineto{\pgfqpoint{0.814455in}{1.169154in}}%
\pgfpathlineto{\pgfqpoint{0.798799in}{1.164068in}}%
\pgfpathlineto{\pgfqpoint{0.790291in}{1.162036in}}%
\pgfpathlineto{\pgfqpoint{0.783142in}{1.160288in}}%
\pgfpathlineto{\pgfqpoint{0.767486in}{1.157843in}}%
\pgfpathlineto{\pgfqpoint{0.751829in}{1.156796in}}%
\pgfpathlineto{\pgfqpoint{0.736173in}{1.157145in}}%
\pgfpathlineto{\pgfqpoint{0.720516in}{1.158891in}}%
\pgfpathlineto{\pgfqpoint{0.704859in}{1.162036in}}%
\pgfpathlineto{\pgfqpoint{0.704857in}{1.162036in}}%
\pgfpathclose%
\pgfpathmoveto{\pgfqpoint{1.487688in}{1.091177in}}%
\pgfpathlineto{\pgfqpoint{1.503344in}{1.087722in}}%
\pgfpathlineto{\pgfqpoint{1.519001in}{1.086242in}}%
\pgfpathlineto{\pgfqpoint{1.534657in}{1.086735in}}%
\pgfpathlineto{\pgfqpoint{1.550314in}{1.089203in}}%
\pgfpathlineto{\pgfqpoint{1.565971in}{1.093647in}}%
\pgfpathlineto{\pgfqpoint{1.566792in}{1.093981in}}%
\pgfpathlineto{\pgfqpoint{1.581627in}{1.099212in}}%
\pgfpathlineto{\pgfqpoint{1.597284in}{1.106435in}}%
\pgfpathlineto{\pgfqpoint{1.599339in}{1.107592in}}%
\pgfpathlineto{\pgfqpoint{1.612940in}{1.114498in}}%
\pgfpathlineto{\pgfqpoint{1.624111in}{1.121203in}}%
\pgfpathlineto{\pgfqpoint{1.628597in}{1.123704in}}%
\pgfpathlineto{\pgfqpoint{1.644253in}{1.133693in}}%
\pgfpathlineto{\pgfqpoint{1.645834in}{1.134814in}}%
\pgfpathlineto{\pgfqpoint{1.659910in}{1.144346in}}%
\pgfpathlineto{\pgfqpoint{1.665412in}{1.148425in}}%
\pgfpathlineto{\pgfqpoint{1.675567in}{1.155767in}}%
\pgfpathlineto{\pgfqpoint{1.683634in}{1.162036in}}%
\pgfpathlineto{\pgfqpoint{1.691223in}{1.167901in}}%
\pgfpathlineto{\pgfqpoint{1.700712in}{1.175647in}}%
\pgfpathlineto{\pgfqpoint{1.706880in}{1.180744in}}%
\pgfpathlineto{\pgfqpoint{1.716798in}{1.189259in}}%
\pgfpathlineto{\pgfqpoint{1.722536in}{1.194332in}}%
\pgfpathlineto{\pgfqpoint{1.731992in}{1.202870in}}%
\pgfpathlineto{\pgfqpoint{1.738193in}{1.208734in}}%
\pgfpathlineto{\pgfqpoint{1.746355in}{1.216481in}}%
\pgfpathlineto{\pgfqpoint{1.753849in}{1.224059in}}%
\pgfpathlineto{\pgfqpoint{1.759906in}{1.230092in}}%
\pgfpathlineto{\pgfqpoint{1.769506in}{1.240462in}}%
\pgfpathlineto{\pgfqpoint{1.772615in}{1.243703in}}%
\pgfpathlineto{\pgfqpoint{1.784436in}{1.257314in}}%
\pgfpathlineto{\pgfqpoint{1.785162in}{1.258256in}}%
\pgfpathlineto{\pgfqpoint{1.795548in}{1.270925in}}%
\pgfpathlineto{\pgfqpoint{1.800819in}{1.278406in}}%
\pgfpathlineto{\pgfqpoint{1.805535in}{1.284536in}}%
\pgfpathlineto{\pgfqpoint{1.814260in}{1.298148in}}%
\pgfpathlineto{\pgfqpoint{1.816476in}{1.302498in}}%
\pgfpathlineto{\pgfqpoint{1.821810in}{1.311759in}}%
\pgfpathlineto{\pgfqpoint{1.827575in}{1.325370in}}%
\pgfpathlineto{\pgfqpoint{1.831240in}{1.338981in}}%
\pgfpathlineto{\pgfqpoint{1.832132in}{1.346708in}}%
\pgfpathlineto{\pgfqpoint{1.832938in}{1.352592in}}%
\pgfpathlineto{\pgfqpoint{1.832316in}{1.366203in}}%
\pgfpathlineto{\pgfqpoint{1.832132in}{1.367010in}}%
\pgfpathlineto{\pgfqpoint{1.829670in}{1.379814in}}%
\pgfpathlineto{\pgfqpoint{1.824955in}{1.393425in}}%
\pgfpathlineto{\pgfqpoint{1.818139in}{1.407036in}}%
\pgfpathlineto{\pgfqpoint{1.816476in}{1.409600in}}%
\pgfpathlineto{\pgfqpoint{1.810128in}{1.420648in}}%
\pgfpathlineto{\pgfqpoint{1.800819in}{1.433790in}}%
\pgfpathlineto{\pgfqpoint{1.800516in}{1.434259in}}%
\pgfpathlineto{\pgfqpoint{1.790167in}{1.447870in}}%
\pgfpathlineto{\pgfqpoint{1.785162in}{1.453617in}}%
\pgfpathlineto{\pgfqpoint{1.778711in}{1.461481in}}%
\pgfpathlineto{\pgfqpoint{1.769506in}{1.471515in}}%
\pgfpathlineto{\pgfqpoint{1.766342in}{1.475092in}}%
\pgfpathlineto{\pgfqpoint{1.753849in}{1.487987in}}%
\pgfpathlineto{\pgfqpoint{1.753167in}{1.488703in}}%
\pgfpathlineto{\pgfqpoint{1.739244in}{1.502314in}}%
\pgfpathlineto{\pgfqpoint{1.738193in}{1.503288in}}%
\pgfpathlineto{\pgfqpoint{1.724499in}{1.515925in}}%
\pgfpathlineto{\pgfqpoint{1.722536in}{1.517669in}}%
\pgfpathlineto{\pgfqpoint{1.708885in}{1.529536in}}%
\pgfpathlineto{\pgfqpoint{1.706880in}{1.531243in}}%
\pgfpathlineto{\pgfqpoint{1.692344in}{1.543148in}}%
\pgfpathlineto{\pgfqpoint{1.691223in}{1.544062in}}%
\pgfpathlineto{\pgfqpoint{1.675567in}{1.556165in}}%
\pgfpathlineto{\pgfqpoint{1.674743in}{1.556759in}}%
\pgfpathlineto{\pgfqpoint{1.659910in}{1.567619in}}%
\pgfpathlineto{\pgfqpoint{1.655795in}{1.570370in}}%
\pgfpathlineto{\pgfqpoint{1.644253in}{1.578373in}}%
\pgfpathlineto{\pgfqpoint{1.635208in}{1.583981in}}%
\pgfpathlineto{\pgfqpoint{1.628597in}{1.588331in}}%
\pgfpathlineto{\pgfqpoint{1.612940in}{1.597328in}}%
\pgfpathlineto{\pgfqpoint{1.612402in}{1.597592in}}%
\pgfpathlineto{\pgfqpoint{1.597284in}{1.605685in}}%
\pgfpathlineto{\pgfqpoint{1.584576in}{1.611203in}}%
\pgfpathlineto{\pgfqpoint{1.581627in}{1.612649in}}%
\pgfpathlineto{\pgfqpoint{1.565971in}{1.618575in}}%
\pgfpathlineto{\pgfqpoint{1.550314in}{1.622673in}}%
\pgfpathlineto{\pgfqpoint{1.535586in}{1.624814in}}%
\pgfpathlineto{\pgfqpoint{1.534657in}{1.624974in}}%
\pgfpathlineto{\pgfqpoint{1.519001in}{1.625515in}}%
\pgfpathlineto{\pgfqpoint{1.512232in}{1.624814in}}%
\pgfpathlineto{\pgfqpoint{1.503344in}{1.624039in}}%
\pgfpathlineto{\pgfqpoint{1.487688in}{1.620852in}}%
\pgfpathlineto{\pgfqpoint{1.472031in}{1.615841in}}%
\pgfpathlineto{\pgfqpoint{1.461379in}{1.611203in}}%
\pgfpathlineto{\pgfqpoint{1.456375in}{1.609277in}}%
\pgfpathlineto{\pgfqpoint{1.440718in}{1.601692in}}%
\pgfpathlineto{\pgfqpoint{1.433666in}{1.597592in}}%
\pgfpathlineto{\pgfqpoint{1.425061in}{1.593009in}}%
\pgfpathlineto{\pgfqpoint{1.410488in}{1.583981in}}%
\pgfpathlineto{\pgfqpoint{1.409405in}{1.583349in}}%
\pgfpathlineto{\pgfqpoint{1.393748in}{1.573072in}}%
\pgfpathlineto{\pgfqpoint{1.390020in}{1.570370in}}%
\pgfpathlineto{\pgfqpoint{1.378092in}{1.562024in}}%
\pgfpathlineto{\pgfqpoint{1.371153in}{1.556759in}}%
\pgfpathlineto{\pgfqpoint{1.362435in}{1.550243in}}%
\pgfpathlineto{\pgfqpoint{1.353525in}{1.543148in}}%
\pgfpathlineto{\pgfqpoint{1.346779in}{1.537757in}}%
\pgfpathlineto{\pgfqpoint{1.336958in}{1.529536in}}%
\pgfpathlineto{\pgfqpoint{1.331122in}{1.524548in}}%
\pgfpathlineto{\pgfqpoint{1.321328in}{1.515925in}}%
\pgfpathlineto{\pgfqpoint{1.315466in}{1.510564in}}%
\pgfpathlineto{\pgfqpoint{1.306555in}{1.502314in}}%
\pgfpathlineto{\pgfqpoint{1.299809in}{1.495716in}}%
\pgfpathlineto{\pgfqpoint{1.292597in}{1.488703in}}%
\pgfpathlineto{\pgfqpoint{1.284152in}{1.479876in}}%
\pgfpathlineto{\pgfqpoint{1.279460in}{1.475092in}}%
\pgfpathlineto{\pgfqpoint{1.268496in}{1.462855in}}%
\pgfpathlineto{\pgfqpoint{1.267206in}{1.461481in}}%
\pgfpathlineto{\pgfqpoint{1.255716in}{1.447870in}}%
\pgfpathlineto{\pgfqpoint{1.252839in}{1.443970in}}%
\pgfpathlineto{\pgfqpoint{1.245126in}{1.434259in}}%
\pgfpathlineto{\pgfqpoint{1.237183in}{1.422434in}}%
\pgfpathlineto{\pgfqpoint{1.235852in}{1.420647in}}%
\pgfpathlineto{\pgfqpoint{1.227544in}{1.407036in}}%
\pgfpathlineto{\pgfqpoint{1.221526in}{1.394139in}}%
\pgfpathlineto{\pgfqpoint{1.221142in}{1.393425in}}%
\pgfpathlineto{\pgfqpoint{1.216030in}{1.379814in}}%
\pgfpathlineto{\pgfqpoint{1.213192in}{1.366203in}}%
\pgfpathlineto{\pgfqpoint{1.212624in}{1.352592in}}%
\pgfpathlineto{\pgfqpoint{1.214327in}{1.338981in}}%
\pgfpathlineto{\pgfqpoint{1.218301in}{1.325370in}}%
\pgfpathlineto{\pgfqpoint{1.221526in}{1.318320in}}%
\pgfpathlineto{\pgfqpoint{1.224125in}{1.311759in}}%
\pgfpathlineto{\pgfqpoint{1.231452in}{1.298148in}}%
\pgfpathlineto{\pgfqpoint{1.237183in}{1.289708in}}%
\pgfpathlineto{\pgfqpoint{1.240345in}{1.284536in}}%
\pgfpathlineto{\pgfqpoint{1.250342in}{1.270925in}}%
\pgfpathlineto{\pgfqpoint{1.252839in}{1.267984in}}%
\pgfpathlineto{\pgfqpoint{1.261266in}{1.257314in}}%
\pgfpathlineto{\pgfqpoint{1.268496in}{1.249223in}}%
\pgfpathlineto{\pgfqpoint{1.273198in}{1.243703in}}%
\pgfpathlineto{\pgfqpoint{1.284152in}{1.232081in}}%
\pgfpathlineto{\pgfqpoint{1.285979in}{1.230092in}}%
\pgfpathlineto{\pgfqpoint{1.299532in}{1.216481in}}%
\pgfpathlineto{\pgfqpoint{1.299809in}{1.216219in}}%
\pgfpathlineto{\pgfqpoint{1.313852in}{1.202870in}}%
\pgfpathlineto{\pgfqpoint{1.315466in}{1.201406in}}%
\pgfpathlineto{\pgfqpoint{1.329027in}{1.189259in}}%
\pgfpathlineto{\pgfqpoint{1.331122in}{1.187437in}}%
\pgfpathlineto{\pgfqpoint{1.345095in}{1.175647in}}%
\pgfpathlineto{\pgfqpoint{1.346779in}{1.174245in}}%
\pgfpathlineto{\pgfqpoint{1.362134in}{1.162036in}}%
\pgfpathlineto{\pgfqpoint{1.362435in}{1.161795in}}%
\pgfpathlineto{\pgfqpoint{1.378092in}{1.150013in}}%
\pgfpathlineto{\pgfqpoint{1.380379in}{1.148425in}}%
\pgfpathlineto{\pgfqpoint{1.393748in}{1.138902in}}%
\pgfpathlineto{\pgfqpoint{1.400098in}{1.134814in}}%
\pgfpathlineto{\pgfqpoint{1.409405in}{1.128529in}}%
\pgfpathlineto{\pgfqpoint{1.421678in}{1.121203in}}%
\pgfpathlineto{\pgfqpoint{1.425061in}{1.119032in}}%
\pgfpathlineto{\pgfqpoint{1.440718in}{1.110342in}}%
\pgfpathlineto{\pgfqpoint{1.446666in}{1.107592in}}%
\pgfpathlineto{\pgfqpoint{1.456375in}{1.102610in}}%
\pgfpathlineto{\pgfqpoint{1.472031in}{1.096240in}}%
\pgfpathlineto{\pgfqpoint{1.479579in}{1.093981in}}%
\pgfpathlineto{\pgfqpoint{1.487688in}{1.091177in}}%
\pgfpathclose%
\pgfpathmoveto{\pgfqpoint{1.480539in}{1.162036in}}%
\pgfpathlineto{\pgfqpoint{1.472031in}{1.164068in}}%
\pgfpathlineto{\pgfqpoint{1.456375in}{1.169154in}}%
\pgfpathlineto{\pgfqpoint{1.440718in}{1.175602in}}%
\pgfpathlineto{\pgfqpoint{1.440625in}{1.175647in}}%
\pgfpathlineto{\pgfqpoint{1.425061in}{1.183295in}}%
\pgfpathlineto{\pgfqpoint{1.414650in}{1.189259in}}%
\pgfpathlineto{\pgfqpoint{1.409405in}{1.192324in}}%
\pgfpathlineto{\pgfqpoint{1.393748in}{1.202678in}}%
\pgfpathlineto{\pgfqpoint{1.393484in}{1.202870in}}%
\pgfpathlineto{\pgfqpoint{1.378092in}{1.214520in}}%
\pgfpathlineto{\pgfqpoint{1.375698in}{1.216481in}}%
\pgfpathlineto{\pgfqpoint{1.362435in}{1.228011in}}%
\pgfpathlineto{\pgfqpoint{1.360179in}{1.230092in}}%
\pgfpathlineto{\pgfqpoint{1.346779in}{1.243474in}}%
\pgfpathlineto{\pgfqpoint{1.346558in}{1.243703in}}%
\pgfpathlineto{\pgfqpoint{1.334648in}{1.257314in}}%
\pgfpathlineto{\pgfqpoint{1.331122in}{1.261874in}}%
\pgfpathlineto{\pgfqpoint{1.324262in}{1.270925in}}%
\pgfpathlineto{\pgfqpoint{1.315466in}{1.284456in}}%
\pgfpathlineto{\pgfqpoint{1.315413in}{1.284536in}}%
\pgfpathlineto{\pgfqpoint{1.307997in}{1.298148in}}%
\pgfpathlineto{\pgfqpoint{1.302146in}{1.311759in}}%
\pgfpathlineto{\pgfqpoint{1.299809in}{1.319155in}}%
\pgfpathlineto{\pgfqpoint{1.297798in}{1.325370in}}%
\pgfpathlineto{\pgfqpoint{1.294986in}{1.338981in}}%
\pgfpathlineto{\pgfqpoint{1.293781in}{1.352592in}}%
\pgfpathlineto{\pgfqpoint{1.294183in}{1.366203in}}%
\pgfpathlineto{\pgfqpoint{1.296191in}{1.379814in}}%
\pgfpathlineto{\pgfqpoint{1.299808in}{1.393425in}}%
\pgfpathlineto{\pgfqpoint{1.299809in}{1.393428in}}%
\pgfpathlineto{\pgfqpoint{1.304876in}{1.407036in}}%
\pgfpathlineto{\pgfqpoint{1.311509in}{1.420648in}}%
\pgfpathlineto{\pgfqpoint{1.315466in}{1.427266in}}%
\pgfpathlineto{\pgfqpoint{1.319646in}{1.434259in}}%
\pgfpathlineto{\pgfqpoint{1.329262in}{1.447870in}}%
\pgfpathlineto{\pgfqpoint{1.331122in}{1.450180in}}%
\pgfpathlineto{\pgfqpoint{1.340416in}{1.461481in}}%
\pgfpathlineto{\pgfqpoint{1.346779in}{1.468400in}}%
\pgfpathlineto{\pgfqpoint{1.353182in}{1.475092in}}%
\pgfpathlineto{\pgfqpoint{1.362435in}{1.483940in}}%
\pgfpathlineto{\pgfqpoint{1.367717in}{1.488703in}}%
\pgfpathlineto{\pgfqpoint{1.378092in}{1.497441in}}%
\pgfpathlineto{\pgfqpoint{1.384350in}{1.502314in}}%
\pgfpathlineto{\pgfqpoint{1.393748in}{1.509283in}}%
\pgfpathlineto{\pgfqpoint{1.403649in}{1.515925in}}%
\pgfpathlineto{\pgfqpoint{1.409405in}{1.519671in}}%
\pgfpathlineto{\pgfqpoint{1.425061in}{1.528684in}}%
\pgfpathlineto{\pgfqpoint{1.426774in}{1.529536in}}%
\pgfpathlineto{\pgfqpoint{1.440718in}{1.536412in}}%
\pgfpathlineto{\pgfqpoint{1.456375in}{1.542799in}}%
\pgfpathlineto{\pgfqpoint{1.457446in}{1.543148in}}%
\pgfpathlineto{\pgfqpoint{1.472031in}{1.547947in}}%
\pgfpathlineto{\pgfqpoint{1.487688in}{1.551724in}}%
\pgfpathlineto{\pgfqpoint{1.503344in}{1.554125in}}%
\pgfpathlineto{\pgfqpoint{1.519001in}{1.555154in}}%
\pgfpathlineto{\pgfqpoint{1.534657in}{1.554811in}}%
\pgfpathlineto{\pgfqpoint{1.550314in}{1.553096in}}%
\pgfpathlineto{\pgfqpoint{1.565971in}{1.550007in}}%
\pgfpathlineto{\pgfqpoint{1.581627in}{1.545541in}}%
\pgfpathlineto{\pgfqpoint{1.588085in}{1.543148in}}%
\pgfpathlineto{\pgfqpoint{1.597284in}{1.539774in}}%
\pgfpathlineto{\pgfqpoint{1.612940in}{1.532713in}}%
\pgfpathlineto{\pgfqpoint{1.618923in}{1.529536in}}%
\pgfpathlineto{\pgfqpoint{1.628597in}{1.524343in}}%
\pgfpathlineto{\pgfqpoint{1.642258in}{1.515925in}}%
\pgfpathlineto{\pgfqpoint{1.644253in}{1.514659in}}%
\pgfpathlineto{\pgfqpoint{1.659910in}{1.503583in}}%
\pgfpathlineto{\pgfqpoint{1.661555in}{1.502314in}}%
\pgfpathlineto{\pgfqpoint{1.675567in}{1.490946in}}%
\pgfpathlineto{\pgfqpoint{1.678147in}{1.488703in}}%
\pgfpathlineto{\pgfqpoint{1.691223in}{1.476522in}}%
\pgfpathlineto{\pgfqpoint{1.692683in}{1.475092in}}%
\pgfpathlineto{\pgfqpoint{1.705423in}{1.461481in}}%
\pgfpathlineto{\pgfqpoint{1.706880in}{1.459746in}}%
\pgfpathlineto{\pgfqpoint{1.716563in}{1.447870in}}%
\pgfpathlineto{\pgfqpoint{1.722536in}{1.439459in}}%
\pgfpathlineto{\pgfqpoint{1.726191in}{1.434259in}}%
\pgfpathlineto{\pgfqpoint{1.734313in}{1.420648in}}%
\pgfpathlineto{\pgfqpoint{1.738193in}{1.412651in}}%
\pgfpathlineto{\pgfqpoint{1.740946in}{1.407036in}}%
\pgfpathlineto{\pgfqpoint{1.746084in}{1.393425in}}%
\pgfpathlineto{\pgfqpoint{1.749637in}{1.379814in}}%
\pgfpathlineto{\pgfqpoint{1.751609in}{1.366203in}}%
\pgfpathlineto{\pgfqpoint{1.752004in}{1.352592in}}%
\pgfpathlineto{\pgfqpoint{1.750820in}{1.338981in}}%
\pgfpathlineto{\pgfqpoint{1.748058in}{1.325370in}}%
\pgfpathlineto{\pgfqpoint{1.743713in}{1.311759in}}%
\pgfpathlineto{\pgfqpoint{1.738193in}{1.299079in}}%
\pgfpathlineto{\pgfqpoint{1.737792in}{1.298148in}}%
\pgfpathlineto{\pgfqpoint{1.730446in}{1.284536in}}%
\pgfpathlineto{\pgfqpoint{1.722536in}{1.272414in}}%
\pgfpathlineto{\pgfqpoint{1.721556in}{1.270925in}}%
\pgfpathlineto{\pgfqpoint{1.711189in}{1.257314in}}%
\pgfpathlineto{\pgfqpoint{1.706880in}{1.252310in}}%
\pgfpathlineto{\pgfqpoint{1.699239in}{1.243703in}}%
\pgfpathlineto{\pgfqpoint{1.691223in}{1.235532in}}%
\pgfpathlineto{\pgfqpoint{1.685618in}{1.230092in}}%
\pgfpathlineto{\pgfqpoint{1.675567in}{1.221073in}}%
\pgfpathlineto{\pgfqpoint{1.670088in}{1.216481in}}%
\pgfpathlineto{\pgfqpoint{1.659910in}{1.208436in}}%
\pgfpathlineto{\pgfqpoint{1.652212in}{1.202870in}}%
\pgfpathlineto{\pgfqpoint{1.644253in}{1.197338in}}%
\pgfpathlineto{\pgfqpoint{1.631254in}{1.189259in}}%
\pgfpathlineto{\pgfqpoint{1.628597in}{1.187642in}}%
\pgfpathlineto{\pgfqpoint{1.612940in}{1.179282in}}%
\pgfpathlineto{\pgfqpoint{1.604897in}{1.175647in}}%
\pgfpathlineto{\pgfqpoint{1.597284in}{1.172208in}}%
\pgfpathlineto{\pgfqpoint{1.581627in}{1.166441in}}%
\pgfpathlineto{\pgfqpoint{1.565973in}{1.162036in}}%
\pgfpathlineto{\pgfqpoint{1.565971in}{1.162036in}}%
\pgfpathlineto{\pgfqpoint{1.550314in}{1.158891in}}%
\pgfpathlineto{\pgfqpoint{1.534657in}{1.157145in}}%
\pgfpathlineto{\pgfqpoint{1.519001in}{1.156796in}}%
\pgfpathlineto{\pgfqpoint{1.503344in}{1.157843in}}%
\pgfpathlineto{\pgfqpoint{1.487688in}{1.160288in}}%
\pgfpathlineto{\pgfqpoint{1.480539in}{1.162036in}}%
\pgfpathclose%
\pgfusepath{fill}%
\end{pgfscope}%
\begin{pgfscope}%
\pgfpathrectangle{\pgfqpoint{0.360415in}{0.345370in}}{\pgfqpoint{1.550000in}{1.347500in}}%
\pgfusepath{clip}%
\pgfsetbuttcap%
\pgfsetroundjoin%
\definecolor{currentfill}{rgb}{0.481929,0.136891,0.507989}%
\pgfsetfillcolor{currentfill}%
\pgfsetlinewidth{0.000000pt}%
\definecolor{currentstroke}{rgb}{0.000000,0.000000,0.000000}%
\pgfsetstrokecolor{currentstroke}%
\pgfsetdash{}{0pt}%
\pgfpathmoveto{\pgfqpoint{0.626577in}{0.345370in}}%
\pgfpathlineto{\pgfqpoint{0.642233in}{0.345370in}}%
\pgfpathlineto{\pgfqpoint{0.657890in}{0.345370in}}%
\pgfpathlineto{\pgfqpoint{0.673546in}{0.345370in}}%
\pgfpathlineto{\pgfqpoint{0.689203in}{0.345370in}}%
\pgfpathlineto{\pgfqpoint{0.704859in}{0.345370in}}%
\pgfpathlineto{\pgfqpoint{0.720516in}{0.345370in}}%
\pgfpathlineto{\pgfqpoint{0.736173in}{0.345370in}}%
\pgfpathlineto{\pgfqpoint{0.751829in}{0.345370in}}%
\pgfpathlineto{\pgfqpoint{0.767486in}{0.345370in}}%
\pgfpathlineto{\pgfqpoint{0.783142in}{0.345370in}}%
\pgfpathlineto{\pgfqpoint{0.798799in}{0.345370in}}%
\pgfpathlineto{\pgfqpoint{0.814455in}{0.345370in}}%
\pgfpathlineto{\pgfqpoint{0.830112in}{0.345370in}}%
\pgfpathlineto{\pgfqpoint{0.845769in}{0.345370in}}%
\pgfpathlineto{\pgfqpoint{0.861425in}{0.345370in}}%
\pgfpathlineto{\pgfqpoint{0.877082in}{0.345370in}}%
\pgfpathlineto{\pgfqpoint{0.880525in}{0.345370in}}%
\pgfpathlineto{\pgfqpoint{0.882085in}{0.358981in}}%
\pgfpathlineto{\pgfqpoint{0.886700in}{0.372592in}}%
\pgfpathlineto{\pgfqpoint{0.892738in}{0.383562in}}%
\pgfpathlineto{\pgfqpoint{0.894102in}{0.386203in}}%
\pgfpathlineto{\pgfqpoint{0.903620in}{0.399814in}}%
\pgfpathlineto{\pgfqpoint{0.908395in}{0.405366in}}%
\pgfpathlineto{\pgfqpoint{0.914967in}{0.413425in}}%
\pgfpathlineto{\pgfqpoint{0.924051in}{0.423042in}}%
\pgfpathlineto{\pgfqpoint{0.927678in}{0.427036in}}%
\pgfpathlineto{\pgfqpoint{0.939708in}{0.438948in}}%
\pgfpathlineto{\pgfqpoint{0.941380in}{0.440648in}}%
\pgfpathlineto{\pgfqpoint{0.955364in}{0.453825in}}%
\pgfpathlineto{\pgfqpoint{0.955820in}{0.454259in}}%
\pgfpathlineto{\pgfqpoint{0.970844in}{0.467870in}}%
\pgfpathlineto{\pgfqpoint{0.971021in}{0.468025in}}%
\pgfpathlineto{\pgfqpoint{0.986378in}{0.481481in}}%
\pgfpathlineto{\pgfqpoint{0.986678in}{0.481740in}}%
\pgfpathlineto{\pgfqpoint{1.002334in}{0.495031in}}%
\pgfpathlineto{\pgfqpoint{1.002409in}{0.495092in}}%
\pgfpathlineto{\pgfqpoint{1.017991in}{0.507848in}}%
\pgfpathlineto{\pgfqpoint{1.019102in}{0.508703in}}%
\pgfpathlineto{\pgfqpoint{1.033647in}{0.520102in}}%
\pgfpathlineto{\pgfqpoint{1.036735in}{0.522314in}}%
\pgfpathlineto{\pgfqpoint{1.049304in}{0.531617in}}%
\pgfpathlineto{\pgfqpoint{1.055900in}{0.535925in}}%
\pgfpathlineto{\pgfqpoint{1.064960in}{0.542123in}}%
\pgfpathlineto{\pgfqpoint{1.077845in}{0.549536in}}%
\pgfpathlineto{\pgfqpoint{1.080617in}{0.551230in}}%
\pgfpathlineto{\pgfqpoint{1.096274in}{0.558662in}}%
\pgfpathlineto{\pgfqpoint{1.109968in}{0.563148in}}%
\pgfpathlineto{\pgfqpoint{1.111930in}{0.563838in}}%
\pgfpathlineto{\pgfqpoint{1.127587in}{0.566629in}}%
\pgfpathlineto{\pgfqpoint{1.143243in}{0.566629in}}%
\pgfpathlineto{\pgfqpoint{1.158900in}{0.563838in}}%
\pgfpathlineto{\pgfqpoint{1.160862in}{0.563148in}}%
\pgfpathlineto{\pgfqpoint{1.174556in}{0.558662in}}%
\pgfpathlineto{\pgfqpoint{1.190213in}{0.551230in}}%
\pgfpathlineto{\pgfqpoint{1.192985in}{0.549536in}}%
\pgfpathlineto{\pgfqpoint{1.205870in}{0.542123in}}%
\pgfpathlineto{\pgfqpoint{1.214930in}{0.535925in}}%
\pgfpathlineto{\pgfqpoint{1.221526in}{0.531617in}}%
\pgfpathlineto{\pgfqpoint{1.234095in}{0.522314in}}%
\pgfpathlineto{\pgfqpoint{1.237183in}{0.520102in}}%
\pgfpathlineto{\pgfqpoint{1.251728in}{0.508703in}}%
\pgfpathlineto{\pgfqpoint{1.252839in}{0.507848in}}%
\pgfpathlineto{\pgfqpoint{1.268421in}{0.495092in}}%
\pgfpathlineto{\pgfqpoint{1.268496in}{0.495031in}}%
\pgfpathlineto{\pgfqpoint{1.284152in}{0.481740in}}%
\pgfpathlineto{\pgfqpoint{1.284452in}{0.481481in}}%
\pgfpathlineto{\pgfqpoint{1.299809in}{0.468025in}}%
\pgfpathlineto{\pgfqpoint{1.299986in}{0.467870in}}%
\pgfpathlineto{\pgfqpoint{1.315010in}{0.454259in}}%
\pgfpathlineto{\pgfqpoint{1.315466in}{0.453825in}}%
\pgfpathlineto{\pgfqpoint{1.329450in}{0.440648in}}%
\pgfpathlineto{\pgfqpoint{1.331122in}{0.438948in}}%
\pgfpathlineto{\pgfqpoint{1.343152in}{0.427036in}}%
\pgfpathlineto{\pgfqpoint{1.346779in}{0.423042in}}%
\pgfpathlineto{\pgfqpoint{1.355863in}{0.413425in}}%
\pgfpathlineto{\pgfqpoint{1.362435in}{0.405366in}}%
\pgfpathlineto{\pgfqpoint{1.367210in}{0.399814in}}%
\pgfpathlineto{\pgfqpoint{1.376728in}{0.386203in}}%
\pgfpathlineto{\pgfqpoint{1.378092in}{0.383562in}}%
\pgfpathlineto{\pgfqpoint{1.384130in}{0.372592in}}%
\pgfpathlineto{\pgfqpoint{1.388745in}{0.358981in}}%
\pgfpathlineto{\pgfqpoint{1.390305in}{0.345370in}}%
\pgfpathlineto{\pgfqpoint{1.393748in}{0.345370in}}%
\pgfpathlineto{\pgfqpoint{1.409405in}{0.345370in}}%
\pgfpathlineto{\pgfqpoint{1.425061in}{0.345370in}}%
\pgfpathlineto{\pgfqpoint{1.440718in}{0.345370in}}%
\pgfpathlineto{\pgfqpoint{1.456375in}{0.345370in}}%
\pgfpathlineto{\pgfqpoint{1.472031in}{0.345370in}}%
\pgfpathlineto{\pgfqpoint{1.487688in}{0.345370in}}%
\pgfpathlineto{\pgfqpoint{1.503344in}{0.345370in}}%
\pgfpathlineto{\pgfqpoint{1.519001in}{0.345370in}}%
\pgfpathlineto{\pgfqpoint{1.534657in}{0.345370in}}%
\pgfpathlineto{\pgfqpoint{1.550314in}{0.345370in}}%
\pgfpathlineto{\pgfqpoint{1.565971in}{0.345370in}}%
\pgfpathlineto{\pgfqpoint{1.581627in}{0.345370in}}%
\pgfpathlineto{\pgfqpoint{1.597284in}{0.345370in}}%
\pgfpathlineto{\pgfqpoint{1.612940in}{0.345370in}}%
\pgfpathlineto{\pgfqpoint{1.628597in}{0.345370in}}%
\pgfpathlineto{\pgfqpoint{1.644253in}{0.345370in}}%
\pgfpathlineto{\pgfqpoint{1.655500in}{0.345370in}}%
\pgfpathlineto{\pgfqpoint{1.657115in}{0.358981in}}%
\pgfpathlineto{\pgfqpoint{1.659910in}{0.366929in}}%
\pgfpathlineto{\pgfqpoint{1.661768in}{0.372592in}}%
\pgfpathlineto{\pgfqpoint{1.669032in}{0.386203in}}%
\pgfpathlineto{\pgfqpoint{1.675567in}{0.395281in}}%
\pgfpathlineto{\pgfqpoint{1.678644in}{0.399814in}}%
\pgfpathlineto{\pgfqpoint{1.690000in}{0.413425in}}%
\pgfpathlineto{\pgfqpoint{1.691223in}{0.414684in}}%
\pgfpathlineto{\pgfqpoint{1.702669in}{0.427036in}}%
\pgfpathlineto{\pgfqpoint{1.706880in}{0.431121in}}%
\pgfpathlineto{\pgfqpoint{1.716376in}{0.440648in}}%
\pgfpathlineto{\pgfqpoint{1.722536in}{0.446378in}}%
\pgfpathlineto{\pgfqpoint{1.730844in}{0.454259in}}%
\pgfpathlineto{\pgfqpoint{1.738193in}{0.460885in}}%
\pgfpathlineto{\pgfqpoint{1.745903in}{0.467870in}}%
\pgfpathlineto{\pgfqpoint{1.753849in}{0.474858in}}%
\pgfpathlineto{\pgfqpoint{1.761468in}{0.481481in}}%
\pgfpathlineto{\pgfqpoint{1.769506in}{0.488389in}}%
\pgfpathlineto{\pgfqpoint{1.777541in}{0.495092in}}%
\pgfpathlineto{\pgfqpoint{1.785162in}{0.501480in}}%
\pgfpathlineto{\pgfqpoint{1.794228in}{0.508703in}}%
\pgfpathlineto{\pgfqpoint{1.800819in}{0.514059in}}%
\pgfpathlineto{\pgfqpoint{1.811778in}{0.522314in}}%
\pgfpathlineto{\pgfqpoint{1.816476in}{0.525975in}}%
\pgfpathlineto{\pgfqpoint{1.830684in}{0.535925in}}%
\pgfpathlineto{\pgfqpoint{1.832132in}{0.536989in}}%
\pgfpathlineto{\pgfqpoint{1.847789in}{0.546861in}}%
\pgfpathlineto{\pgfqpoint{1.853004in}{0.549536in}}%
\pgfpathlineto{\pgfqpoint{1.863445in}{0.555217in}}%
\pgfpathlineto{\pgfqpoint{1.879102in}{0.561532in}}%
\pgfpathlineto{\pgfqpoint{1.885616in}{0.563148in}}%
\pgfpathlineto{\pgfqpoint{1.894758in}{0.565578in}}%
\pgfpathlineto{\pgfqpoint{1.910415in}{0.566981in}}%
\pgfpathlineto{\pgfqpoint{1.910415in}{0.576759in}}%
\pgfpathlineto{\pgfqpoint{1.910415in}{0.590370in}}%
\pgfpathlineto{\pgfqpoint{1.910415in}{0.603981in}}%
\pgfpathlineto{\pgfqpoint{1.910415in}{0.617592in}}%
\pgfpathlineto{\pgfqpoint{1.910415in}{0.631203in}}%
\pgfpathlineto{\pgfqpoint{1.910415in}{0.644814in}}%
\pgfpathlineto{\pgfqpoint{1.910415in}{0.658425in}}%
\pgfpathlineto{\pgfqpoint{1.910415in}{0.672036in}}%
\pgfpathlineto{\pgfqpoint{1.910415in}{0.685648in}}%
\pgfpathlineto{\pgfqpoint{1.910415in}{0.699259in}}%
\pgfpathlineto{\pgfqpoint{1.910415in}{0.712870in}}%
\pgfpathlineto{\pgfqpoint{1.910415in}{0.726481in}}%
\pgfpathlineto{\pgfqpoint{1.910415in}{0.740092in}}%
\pgfpathlineto{\pgfqpoint{1.910415in}{0.753703in}}%
\pgfpathlineto{\pgfqpoint{1.910415in}{0.767314in}}%
\pgfpathlineto{\pgfqpoint{1.910415in}{0.780925in}}%
\pgfpathlineto{\pgfqpoint{1.910415in}{0.794536in}}%
\pgfpathlineto{\pgfqpoint{1.910415in}{0.797530in}}%
\pgfpathlineto{\pgfqpoint{1.894758in}{0.798886in}}%
\pgfpathlineto{\pgfqpoint{1.879102in}{0.802899in}}%
\pgfpathlineto{\pgfqpoint{1.866483in}{0.808148in}}%
\pgfpathlineto{\pgfqpoint{1.863445in}{0.809333in}}%
\pgfpathlineto{\pgfqpoint{1.847789in}{0.817608in}}%
\pgfpathlineto{\pgfqpoint{1.841403in}{0.821759in}}%
\pgfpathlineto{\pgfqpoint{1.832132in}{0.827473in}}%
\pgfpathlineto{\pgfqpoint{1.821070in}{0.835370in}}%
\pgfpathlineto{\pgfqpoint{1.816476in}{0.838523in}}%
\pgfpathlineto{\pgfqpoint{1.802774in}{0.848981in}}%
\pgfpathlineto{\pgfqpoint{1.800819in}{0.850435in}}%
\pgfpathlineto{\pgfqpoint{1.785662in}{0.862592in}}%
\pgfpathlineto{\pgfqpoint{1.785162in}{0.862988in}}%
\pgfpathlineto{\pgfqpoint{1.769506in}{0.876049in}}%
\pgfpathlineto{\pgfqpoint{1.769327in}{0.876203in}}%
\pgfpathlineto{\pgfqpoint{1.753849in}{0.889554in}}%
\pgfpathlineto{\pgfqpoint{1.753551in}{0.889814in}}%
\pgfpathlineto{\pgfqpoint{1.738263in}{0.903425in}}%
\pgfpathlineto{\pgfqpoint{1.738193in}{0.903490in}}%
\pgfpathlineto{\pgfqpoint{1.723520in}{0.917036in}}%
\pgfpathlineto{\pgfqpoint{1.722536in}{0.918003in}}%
\pgfpathlineto{\pgfqpoint{1.709424in}{0.930648in}}%
\pgfpathlineto{\pgfqpoint{1.706880in}{0.933332in}}%
\pgfpathlineto{\pgfqpoint{1.696179in}{0.944259in}}%
\pgfpathlineto{\pgfqpoint{1.691223in}{0.949993in}}%
\pgfpathlineto{\pgfqpoint{1.684094in}{0.957870in}}%
\pgfpathlineto{\pgfqpoint{1.675567in}{0.969071in}}%
\pgfpathlineto{\pgfqpoint{1.673618in}{0.971481in}}%
\pgfpathlineto{\pgfqpoint{1.665069in}{0.985092in}}%
\pgfpathlineto{\pgfqpoint{1.659910in}{0.996997in}}%
\pgfpathlineto{\pgfqpoint{1.659116in}{0.998703in}}%
\pgfpathlineto{\pgfqpoint{1.655905in}{1.012314in}}%
\pgfpathlineto{\pgfqpoint{1.655905in}{1.025925in}}%
\pgfpathlineto{\pgfqpoint{1.659116in}{1.039536in}}%
\pgfpathlineto{\pgfqpoint{1.659910in}{1.041242in}}%
\pgfpathlineto{\pgfqpoint{1.665069in}{1.053148in}}%
\pgfpathlineto{\pgfqpoint{1.673618in}{1.066759in}}%
\pgfpathlineto{\pgfqpoint{1.675567in}{1.069168in}}%
\pgfpathlineto{\pgfqpoint{1.684094in}{1.080370in}}%
\pgfpathlineto{\pgfqpoint{1.691223in}{1.088246in}}%
\pgfpathlineto{\pgfqpoint{1.696179in}{1.093981in}}%
\pgfpathlineto{\pgfqpoint{1.706880in}{1.104908in}}%
\pgfpathlineto{\pgfqpoint{1.709424in}{1.107592in}}%
\pgfpathlineto{\pgfqpoint{1.722536in}{1.120237in}}%
\pgfpathlineto{\pgfqpoint{1.723520in}{1.121203in}}%
\pgfpathlineto{\pgfqpoint{1.738193in}{1.134749in}}%
\pgfpathlineto{\pgfqpoint{1.738263in}{1.134814in}}%
\pgfpathlineto{\pgfqpoint{1.753551in}{1.148425in}}%
\pgfpathlineto{\pgfqpoint{1.753849in}{1.148686in}}%
\pgfpathlineto{\pgfqpoint{1.769327in}{1.162036in}}%
\pgfpathlineto{\pgfqpoint{1.769506in}{1.162190in}}%
\pgfpathlineto{\pgfqpoint{1.785162in}{1.175252in}}%
\pgfpathlineto{\pgfqpoint{1.785662in}{1.175647in}}%
\pgfpathlineto{\pgfqpoint{1.800819in}{1.187805in}}%
\pgfpathlineto{\pgfqpoint{1.802774in}{1.189259in}}%
\pgfpathlineto{\pgfqpoint{1.816476in}{1.199716in}}%
\pgfpathlineto{\pgfqpoint{1.821070in}{1.202870in}}%
\pgfpathlineto{\pgfqpoint{1.832132in}{1.210767in}}%
\pgfpathlineto{\pgfqpoint{1.841403in}{1.216481in}}%
\pgfpathlineto{\pgfqpoint{1.847789in}{1.220632in}}%
\pgfpathlineto{\pgfqpoint{1.863445in}{1.228906in}}%
\pgfpathlineto{\pgfqpoint{1.866483in}{1.230092in}}%
\pgfpathlineto{\pgfqpoint{1.879102in}{1.235341in}}%
\pgfpathlineto{\pgfqpoint{1.894758in}{1.239353in}}%
\pgfpathlineto{\pgfqpoint{1.910415in}{1.240710in}}%
\pgfpathlineto{\pgfqpoint{1.910415in}{1.243703in}}%
\pgfpathlineto{\pgfqpoint{1.910415in}{1.257314in}}%
\pgfpathlineto{\pgfqpoint{1.910415in}{1.270925in}}%
\pgfpathlineto{\pgfqpoint{1.910415in}{1.284536in}}%
\pgfpathlineto{\pgfqpoint{1.910415in}{1.298148in}}%
\pgfpathlineto{\pgfqpoint{1.910415in}{1.311759in}}%
\pgfpathlineto{\pgfqpoint{1.910415in}{1.325370in}}%
\pgfpathlineto{\pgfqpoint{1.910415in}{1.338981in}}%
\pgfpathlineto{\pgfqpoint{1.910415in}{1.352592in}}%
\pgfpathlineto{\pgfqpoint{1.910415in}{1.366203in}}%
\pgfpathlineto{\pgfqpoint{1.910415in}{1.379814in}}%
\pgfpathlineto{\pgfqpoint{1.910415in}{1.393425in}}%
\pgfpathlineto{\pgfqpoint{1.910415in}{1.407036in}}%
\pgfpathlineto{\pgfqpoint{1.910415in}{1.420648in}}%
\pgfpathlineto{\pgfqpoint{1.910415in}{1.434259in}}%
\pgfpathlineto{\pgfqpoint{1.910415in}{1.447870in}}%
\pgfpathlineto{\pgfqpoint{1.910415in}{1.461481in}}%
\pgfpathlineto{\pgfqpoint{1.910415in}{1.471258in}}%
\pgfpathlineto{\pgfqpoint{1.894758in}{1.472662in}}%
\pgfpathlineto{\pgfqpoint{1.885616in}{1.475092in}}%
\pgfpathlineto{\pgfqpoint{1.879102in}{1.476708in}}%
\pgfpathlineto{\pgfqpoint{1.863445in}{1.483023in}}%
\pgfpathlineto{\pgfqpoint{1.853004in}{1.488703in}}%
\pgfpathlineto{\pgfqpoint{1.847789in}{1.491379in}}%
\pgfpathlineto{\pgfqpoint{1.832132in}{1.501251in}}%
\pgfpathlineto{\pgfqpoint{1.830684in}{1.502314in}}%
\pgfpathlineto{\pgfqpoint{1.816476in}{1.512264in}}%
\pgfpathlineto{\pgfqpoint{1.811778in}{1.515925in}}%
\pgfpathlineto{\pgfqpoint{1.800819in}{1.524181in}}%
\pgfpathlineto{\pgfqpoint{1.794228in}{1.529536in}}%
\pgfpathlineto{\pgfqpoint{1.785162in}{1.536759in}}%
\pgfpathlineto{\pgfqpoint{1.777541in}{1.543148in}}%
\pgfpathlineto{\pgfqpoint{1.769506in}{1.549851in}}%
\pgfpathlineto{\pgfqpoint{1.761468in}{1.556759in}}%
\pgfpathlineto{\pgfqpoint{1.753849in}{1.563382in}}%
\pgfpathlineto{\pgfqpoint{1.745903in}{1.570370in}}%
\pgfpathlineto{\pgfqpoint{1.738193in}{1.577355in}}%
\pgfpathlineto{\pgfqpoint{1.730844in}{1.583981in}}%
\pgfpathlineto{\pgfqpoint{1.722536in}{1.591862in}}%
\pgfpathlineto{\pgfqpoint{1.716376in}{1.597592in}}%
\pgfpathlineto{\pgfqpoint{1.706880in}{1.607119in}}%
\pgfpathlineto{\pgfqpoint{1.702669in}{1.611203in}}%
\pgfpathlineto{\pgfqpoint{1.691223in}{1.623555in}}%
\pgfpathlineto{\pgfqpoint{1.690000in}{1.624814in}}%
\pgfpathlineto{\pgfqpoint{1.678644in}{1.638425in}}%
\pgfpathlineto{\pgfqpoint{1.675567in}{1.642959in}}%
\pgfpathlineto{\pgfqpoint{1.669032in}{1.652036in}}%
\pgfpathlineto{\pgfqpoint{1.661768in}{1.665648in}}%
\pgfpathlineto{\pgfqpoint{1.659910in}{1.671310in}}%
\pgfpathlineto{\pgfqpoint{1.657115in}{1.679259in}}%
\pgfpathlineto{\pgfqpoint{1.655500in}{1.692870in}}%
\pgfpathlineto{\pgfqpoint{1.644253in}{1.692870in}}%
\pgfpathlineto{\pgfqpoint{1.628597in}{1.692870in}}%
\pgfpathlineto{\pgfqpoint{1.612940in}{1.692870in}}%
\pgfpathlineto{\pgfqpoint{1.597284in}{1.692870in}}%
\pgfpathlineto{\pgfqpoint{1.581627in}{1.692870in}}%
\pgfpathlineto{\pgfqpoint{1.565971in}{1.692870in}}%
\pgfpathlineto{\pgfqpoint{1.550314in}{1.692870in}}%
\pgfpathlineto{\pgfqpoint{1.534657in}{1.692870in}}%
\pgfpathlineto{\pgfqpoint{1.519001in}{1.692870in}}%
\pgfpathlineto{\pgfqpoint{1.503344in}{1.692870in}}%
\pgfpathlineto{\pgfqpoint{1.487688in}{1.692870in}}%
\pgfpathlineto{\pgfqpoint{1.472031in}{1.692870in}}%
\pgfpathlineto{\pgfqpoint{1.456375in}{1.692870in}}%
\pgfpathlineto{\pgfqpoint{1.440718in}{1.692870in}}%
\pgfpathlineto{\pgfqpoint{1.425061in}{1.692870in}}%
\pgfpathlineto{\pgfqpoint{1.409405in}{1.692870in}}%
\pgfpathlineto{\pgfqpoint{1.393748in}{1.692870in}}%
\pgfpathlineto{\pgfqpoint{1.390305in}{1.692870in}}%
\pgfpathlineto{\pgfqpoint{1.388745in}{1.679259in}}%
\pgfpathlineto{\pgfqpoint{1.384130in}{1.665648in}}%
\pgfpathlineto{\pgfqpoint{1.378092in}{1.654678in}}%
\pgfpathlineto{\pgfqpoint{1.376728in}{1.652036in}}%
\pgfpathlineto{\pgfqpoint{1.367210in}{1.638425in}}%
\pgfpathlineto{\pgfqpoint{1.362435in}{1.632874in}}%
\pgfpathlineto{\pgfqpoint{1.355863in}{1.624814in}}%
\pgfpathlineto{\pgfqpoint{1.346779in}{1.615197in}}%
\pgfpathlineto{\pgfqpoint{1.343152in}{1.611203in}}%
\pgfpathlineto{\pgfqpoint{1.331122in}{1.599291in}}%
\pgfpathlineto{\pgfqpoint{1.329450in}{1.597592in}}%
\pgfpathlineto{\pgfqpoint{1.315466in}{1.584415in}}%
\pgfpathlineto{\pgfqpoint{1.315010in}{1.583981in}}%
\pgfpathlineto{\pgfqpoint{1.299986in}{1.570370in}}%
\pgfpathlineto{\pgfqpoint{1.299809in}{1.570214in}}%
\pgfpathlineto{\pgfqpoint{1.284452in}{1.556759in}}%
\pgfpathlineto{\pgfqpoint{1.284152in}{1.556499in}}%
\pgfpathlineto{\pgfqpoint{1.268496in}{1.543209in}}%
\pgfpathlineto{\pgfqpoint{1.268421in}{1.543148in}}%
\pgfpathlineto{\pgfqpoint{1.252839in}{1.530391in}}%
\pgfpathlineto{\pgfqpoint{1.251728in}{1.529536in}}%
\pgfpathlineto{\pgfqpoint{1.237183in}{1.518137in}}%
\pgfpathlineto{\pgfqpoint{1.234095in}{1.515925in}}%
\pgfpathlineto{\pgfqpoint{1.221526in}{1.506623in}}%
\pgfpathlineto{\pgfqpoint{1.214930in}{1.502314in}}%
\pgfpathlineto{\pgfqpoint{1.205870in}{1.496117in}}%
\pgfpathlineto{\pgfqpoint{1.192985in}{1.488703in}}%
\pgfpathlineto{\pgfqpoint{1.190213in}{1.487009in}}%
\pgfpathlineto{\pgfqpoint{1.174556in}{1.479577in}}%
\pgfpathlineto{\pgfqpoint{1.160862in}{1.475092in}}%
\pgfpathlineto{\pgfqpoint{1.158900in}{1.474402in}}%
\pgfpathlineto{\pgfqpoint{1.143243in}{1.471610in}}%
\pgfpathlineto{\pgfqpoint{1.127587in}{1.471610in}}%
\pgfpathlineto{\pgfqpoint{1.111930in}{1.474402in}}%
\pgfpathlineto{\pgfqpoint{1.109968in}{1.475092in}}%
\pgfpathlineto{\pgfqpoint{1.096274in}{1.479577in}}%
\pgfpathlineto{\pgfqpoint{1.080617in}{1.487009in}}%
\pgfpathlineto{\pgfqpoint{1.077845in}{1.488703in}}%
\pgfpathlineto{\pgfqpoint{1.064960in}{1.496117in}}%
\pgfpathlineto{\pgfqpoint{1.055900in}{1.502314in}}%
\pgfpathlineto{\pgfqpoint{1.049304in}{1.506623in}}%
\pgfpathlineto{\pgfqpoint{1.036735in}{1.515925in}}%
\pgfpathlineto{\pgfqpoint{1.033647in}{1.518137in}}%
\pgfpathlineto{\pgfqpoint{1.019102in}{1.529536in}}%
\pgfpathlineto{\pgfqpoint{1.017991in}{1.530391in}}%
\pgfpathlineto{\pgfqpoint{1.002409in}{1.543148in}}%
\pgfpathlineto{\pgfqpoint{1.002334in}{1.543209in}}%
\pgfpathlineto{\pgfqpoint{0.986678in}{1.556499in}}%
\pgfpathlineto{\pgfqpoint{0.986378in}{1.556759in}}%
\pgfpathlineto{\pgfqpoint{0.971021in}{1.570214in}}%
\pgfpathlineto{\pgfqpoint{0.970844in}{1.570370in}}%
\pgfpathlineto{\pgfqpoint{0.955820in}{1.583981in}}%
\pgfpathlineto{\pgfqpoint{0.955364in}{1.584415in}}%
\pgfpathlineto{\pgfqpoint{0.941380in}{1.597592in}}%
\pgfpathlineto{\pgfqpoint{0.939708in}{1.599291in}}%
\pgfpathlineto{\pgfqpoint{0.927678in}{1.611203in}}%
\pgfpathlineto{\pgfqpoint{0.924051in}{1.615197in}}%
\pgfpathlineto{\pgfqpoint{0.914967in}{1.624814in}}%
\pgfpathlineto{\pgfqpoint{0.908395in}{1.632874in}}%
\pgfpathlineto{\pgfqpoint{0.903620in}{1.638425in}}%
\pgfpathlineto{\pgfqpoint{0.894102in}{1.652036in}}%
\pgfpathlineto{\pgfqpoint{0.892738in}{1.654678in}}%
\pgfpathlineto{\pgfqpoint{0.886700in}{1.665648in}}%
\pgfpathlineto{\pgfqpoint{0.882085in}{1.679259in}}%
\pgfpathlineto{\pgfqpoint{0.880525in}{1.692870in}}%
\pgfpathlineto{\pgfqpoint{0.877082in}{1.692870in}}%
\pgfpathlineto{\pgfqpoint{0.861425in}{1.692870in}}%
\pgfpathlineto{\pgfqpoint{0.845769in}{1.692870in}}%
\pgfpathlineto{\pgfqpoint{0.830112in}{1.692870in}}%
\pgfpathlineto{\pgfqpoint{0.814455in}{1.692870in}}%
\pgfpathlineto{\pgfqpoint{0.798799in}{1.692870in}}%
\pgfpathlineto{\pgfqpoint{0.783142in}{1.692870in}}%
\pgfpathlineto{\pgfqpoint{0.767486in}{1.692870in}}%
\pgfpathlineto{\pgfqpoint{0.751829in}{1.692870in}}%
\pgfpathlineto{\pgfqpoint{0.736173in}{1.692870in}}%
\pgfpathlineto{\pgfqpoint{0.720516in}{1.692870in}}%
\pgfpathlineto{\pgfqpoint{0.704859in}{1.692870in}}%
\pgfpathlineto{\pgfqpoint{0.689203in}{1.692870in}}%
\pgfpathlineto{\pgfqpoint{0.673546in}{1.692870in}}%
\pgfpathlineto{\pgfqpoint{0.657890in}{1.692870in}}%
\pgfpathlineto{\pgfqpoint{0.642233in}{1.692870in}}%
\pgfpathlineto{\pgfqpoint{0.626577in}{1.692870in}}%
\pgfpathlineto{\pgfqpoint{0.615330in}{1.692870in}}%
\pgfpathlineto{\pgfqpoint{0.613715in}{1.679259in}}%
\pgfpathlineto{\pgfqpoint{0.610920in}{1.671310in}}%
\pgfpathlineto{\pgfqpoint{0.609062in}{1.665648in}}%
\pgfpathlineto{\pgfqpoint{0.601798in}{1.652036in}}%
\pgfpathlineto{\pgfqpoint{0.595263in}{1.642959in}}%
\pgfpathlineto{\pgfqpoint{0.592186in}{1.638425in}}%
\pgfpathlineto{\pgfqpoint{0.580830in}{1.624814in}}%
\pgfpathlineto{\pgfqpoint{0.579607in}{1.623555in}}%
\pgfpathlineto{\pgfqpoint{0.568161in}{1.611203in}}%
\pgfpathlineto{\pgfqpoint{0.563950in}{1.607119in}}%
\pgfpathlineto{\pgfqpoint{0.554454in}{1.597592in}}%
\pgfpathlineto{\pgfqpoint{0.548294in}{1.591862in}}%
\pgfpathlineto{\pgfqpoint{0.539986in}{1.583981in}}%
\pgfpathlineto{\pgfqpoint{0.532637in}{1.577355in}}%
\pgfpathlineto{\pgfqpoint{0.524927in}{1.570370in}}%
\pgfpathlineto{\pgfqpoint{0.516981in}{1.563382in}}%
\pgfpathlineto{\pgfqpoint{0.509362in}{1.556759in}}%
\pgfpathlineto{\pgfqpoint{0.501324in}{1.549851in}}%
\pgfpathlineto{\pgfqpoint{0.493289in}{1.543148in}}%
\pgfpathlineto{\pgfqpoint{0.485668in}{1.536759in}}%
\pgfpathlineto{\pgfqpoint{0.476602in}{1.529536in}}%
\pgfpathlineto{\pgfqpoint{0.470011in}{1.524181in}}%
\pgfpathlineto{\pgfqpoint{0.459052in}{1.515925in}}%
\pgfpathlineto{\pgfqpoint{0.454354in}{1.512264in}}%
\pgfpathlineto{\pgfqpoint{0.440146in}{1.502314in}}%
\pgfpathlineto{\pgfqpoint{0.438698in}{1.501251in}}%
\pgfpathlineto{\pgfqpoint{0.423041in}{1.491379in}}%
\pgfpathlineto{\pgfqpoint{0.417826in}{1.488703in}}%
\pgfpathlineto{\pgfqpoint{0.407385in}{1.483023in}}%
\pgfpathlineto{\pgfqpoint{0.391728in}{1.476708in}}%
\pgfpathlineto{\pgfqpoint{0.385214in}{1.475092in}}%
\pgfpathlineto{\pgfqpoint{0.376072in}{1.472662in}}%
\pgfpathlineto{\pgfqpoint{0.360415in}{1.471258in}}%
\pgfpathlineto{\pgfqpoint{0.360415in}{1.461481in}}%
\pgfpathlineto{\pgfqpoint{0.360415in}{1.447870in}}%
\pgfpathlineto{\pgfqpoint{0.360415in}{1.434259in}}%
\pgfpathlineto{\pgfqpoint{0.360415in}{1.420648in}}%
\pgfpathlineto{\pgfqpoint{0.360415in}{1.407036in}}%
\pgfpathlineto{\pgfqpoint{0.360415in}{1.393425in}}%
\pgfpathlineto{\pgfqpoint{0.360415in}{1.379814in}}%
\pgfpathlineto{\pgfqpoint{0.360415in}{1.366203in}}%
\pgfpathlineto{\pgfqpoint{0.360415in}{1.352592in}}%
\pgfpathlineto{\pgfqpoint{0.360415in}{1.338981in}}%
\pgfpathlineto{\pgfqpoint{0.360415in}{1.325370in}}%
\pgfpathlineto{\pgfqpoint{0.360415in}{1.311759in}}%
\pgfpathlineto{\pgfqpoint{0.360415in}{1.298148in}}%
\pgfpathlineto{\pgfqpoint{0.360415in}{1.284536in}}%
\pgfpathlineto{\pgfqpoint{0.360415in}{1.270925in}}%
\pgfpathlineto{\pgfqpoint{0.360415in}{1.257314in}}%
\pgfpathlineto{\pgfqpoint{0.360415in}{1.243703in}}%
\pgfpathlineto{\pgfqpoint{0.360415in}{1.240710in}}%
\pgfpathlineto{\pgfqpoint{0.376072in}{1.239353in}}%
\pgfpathlineto{\pgfqpoint{0.391728in}{1.235341in}}%
\pgfpathlineto{\pgfqpoint{0.404347in}{1.230092in}}%
\pgfpathlineto{\pgfqpoint{0.407385in}{1.228906in}}%
\pgfpathlineto{\pgfqpoint{0.423041in}{1.220632in}}%
\pgfpathlineto{\pgfqpoint{0.429427in}{1.216481in}}%
\pgfpathlineto{\pgfqpoint{0.438698in}{1.210767in}}%
\pgfpathlineto{\pgfqpoint{0.449760in}{1.202870in}}%
\pgfpathlineto{\pgfqpoint{0.454354in}{1.199716in}}%
\pgfpathlineto{\pgfqpoint{0.468056in}{1.189259in}}%
\pgfpathlineto{\pgfqpoint{0.470011in}{1.187805in}}%
\pgfpathlineto{\pgfqpoint{0.485168in}{1.175647in}}%
\pgfpathlineto{\pgfqpoint{0.485668in}{1.175252in}}%
\pgfpathlineto{\pgfqpoint{0.501324in}{1.162190in}}%
\pgfpathlineto{\pgfqpoint{0.501503in}{1.162036in}}%
\pgfpathlineto{\pgfqpoint{0.516981in}{1.148686in}}%
\pgfpathlineto{\pgfqpoint{0.517279in}{1.148425in}}%
\pgfpathlineto{\pgfqpoint{0.532567in}{1.134814in}}%
\pgfpathlineto{\pgfqpoint{0.532637in}{1.134749in}}%
\pgfpathlineto{\pgfqpoint{0.547310in}{1.121203in}}%
\pgfpathlineto{\pgfqpoint{0.548294in}{1.120237in}}%
\pgfpathlineto{\pgfqpoint{0.561406in}{1.107592in}}%
\pgfpathlineto{\pgfqpoint{0.563950in}{1.104908in}}%
\pgfpathlineto{\pgfqpoint{0.574651in}{1.093981in}}%
\pgfpathlineto{\pgfqpoint{0.579607in}{1.088246in}}%
\pgfpathlineto{\pgfqpoint{0.586736in}{1.080370in}}%
\pgfpathlineto{\pgfqpoint{0.595263in}{1.069168in}}%
\pgfpathlineto{\pgfqpoint{0.597212in}{1.066759in}}%
\pgfpathlineto{\pgfqpoint{0.605761in}{1.053147in}}%
\pgfpathlineto{\pgfqpoint{0.610920in}{1.041242in}}%
\pgfpathlineto{\pgfqpoint{0.611714in}{1.039536in}}%
\pgfpathlineto{\pgfqpoint{0.614925in}{1.025925in}}%
\pgfpathlineto{\pgfqpoint{0.614925in}{1.012314in}}%
\pgfpathlineto{\pgfqpoint{0.611714in}{0.998703in}}%
\pgfpathlineto{\pgfqpoint{0.610920in}{0.996997in}}%
\pgfpathlineto{\pgfqpoint{0.605761in}{0.985092in}}%
\pgfpathlineto{\pgfqpoint{0.597212in}{0.971481in}}%
\pgfpathlineto{\pgfqpoint{0.595263in}{0.969071in}}%
\pgfpathlineto{\pgfqpoint{0.586736in}{0.957870in}}%
\pgfpathlineto{\pgfqpoint{0.579607in}{0.949993in}}%
\pgfpathlineto{\pgfqpoint{0.574651in}{0.944259in}}%
\pgfpathlineto{\pgfqpoint{0.563950in}{0.933332in}}%
\pgfpathlineto{\pgfqpoint{0.561406in}{0.930648in}}%
\pgfpathlineto{\pgfqpoint{0.548294in}{0.918003in}}%
\pgfpathlineto{\pgfqpoint{0.547310in}{0.917036in}}%
\pgfpathlineto{\pgfqpoint{0.532637in}{0.903490in}}%
\pgfpathlineto{\pgfqpoint{0.532567in}{0.903425in}}%
\pgfpathlineto{\pgfqpoint{0.517279in}{0.889814in}}%
\pgfpathlineto{\pgfqpoint{0.516981in}{0.889554in}}%
\pgfpathlineto{\pgfqpoint{0.501503in}{0.876203in}}%
\pgfpathlineto{\pgfqpoint{0.501324in}{0.876049in}}%
\pgfpathlineto{\pgfqpoint{0.485668in}{0.862988in}}%
\pgfpathlineto{\pgfqpoint{0.485168in}{0.862592in}}%
\pgfpathlineto{\pgfqpoint{0.470011in}{0.850435in}}%
\pgfpathlineto{\pgfqpoint{0.468056in}{0.848981in}}%
\pgfpathlineto{\pgfqpoint{0.454354in}{0.838523in}}%
\pgfpathlineto{\pgfqpoint{0.449760in}{0.835370in}}%
\pgfpathlineto{\pgfqpoint{0.438698in}{0.827473in}}%
\pgfpathlineto{\pgfqpoint{0.429427in}{0.821759in}}%
\pgfpathlineto{\pgfqpoint{0.423041in}{0.817608in}}%
\pgfpathlineto{\pgfqpoint{0.407385in}{0.809333in}}%
\pgfpathlineto{\pgfqpoint{0.404347in}{0.808148in}}%
\pgfpathlineto{\pgfqpoint{0.391728in}{0.802899in}}%
\pgfpathlineto{\pgfqpoint{0.376072in}{0.798886in}}%
\pgfpathlineto{\pgfqpoint{0.360415in}{0.797530in}}%
\pgfpathlineto{\pgfqpoint{0.360415in}{0.794536in}}%
\pgfpathlineto{\pgfqpoint{0.360415in}{0.780925in}}%
\pgfpathlineto{\pgfqpoint{0.360415in}{0.767314in}}%
\pgfpathlineto{\pgfqpoint{0.360415in}{0.753703in}}%
\pgfpathlineto{\pgfqpoint{0.360415in}{0.740092in}}%
\pgfpathlineto{\pgfqpoint{0.360415in}{0.726481in}}%
\pgfpathlineto{\pgfqpoint{0.360415in}{0.712870in}}%
\pgfpathlineto{\pgfqpoint{0.360415in}{0.699259in}}%
\pgfpathlineto{\pgfqpoint{0.360415in}{0.685648in}}%
\pgfpathlineto{\pgfqpoint{0.360415in}{0.672036in}}%
\pgfpathlineto{\pgfqpoint{0.360415in}{0.658425in}}%
\pgfpathlineto{\pgfqpoint{0.360415in}{0.644814in}}%
\pgfpathlineto{\pgfqpoint{0.360415in}{0.631203in}}%
\pgfpathlineto{\pgfqpoint{0.360415in}{0.617592in}}%
\pgfpathlineto{\pgfqpoint{0.360415in}{0.603981in}}%
\pgfpathlineto{\pgfqpoint{0.360415in}{0.590370in}}%
\pgfpathlineto{\pgfqpoint{0.360415in}{0.576759in}}%
\pgfpathlineto{\pgfqpoint{0.360415in}{0.566981in}}%
\pgfpathlineto{\pgfqpoint{0.376072in}{0.565578in}}%
\pgfpathlineto{\pgfqpoint{0.385214in}{0.563148in}}%
\pgfpathlineto{\pgfqpoint{0.391728in}{0.561532in}}%
\pgfpathlineto{\pgfqpoint{0.407385in}{0.555217in}}%
\pgfpathlineto{\pgfqpoint{0.417826in}{0.549536in}}%
\pgfpathlineto{\pgfqpoint{0.423041in}{0.546861in}}%
\pgfpathlineto{\pgfqpoint{0.438698in}{0.536989in}}%
\pgfpathlineto{\pgfqpoint{0.440146in}{0.535925in}}%
\pgfpathlineto{\pgfqpoint{0.454354in}{0.525975in}}%
\pgfpathlineto{\pgfqpoint{0.459052in}{0.522314in}}%
\pgfpathlineto{\pgfqpoint{0.470011in}{0.514059in}}%
\pgfpathlineto{\pgfqpoint{0.476602in}{0.508703in}}%
\pgfpathlineto{\pgfqpoint{0.485668in}{0.501480in}}%
\pgfpathlineto{\pgfqpoint{0.493289in}{0.495092in}}%
\pgfpathlineto{\pgfqpoint{0.501324in}{0.488389in}}%
\pgfpathlineto{\pgfqpoint{0.509362in}{0.481481in}}%
\pgfpathlineto{\pgfqpoint{0.516981in}{0.474858in}}%
\pgfpathlineto{\pgfqpoint{0.524927in}{0.467870in}}%
\pgfpathlineto{\pgfqpoint{0.532637in}{0.460885in}}%
\pgfpathlineto{\pgfqpoint{0.539986in}{0.454259in}}%
\pgfpathlineto{\pgfqpoint{0.548294in}{0.446378in}}%
\pgfpathlineto{\pgfqpoint{0.554454in}{0.440648in}}%
\pgfpathlineto{\pgfqpoint{0.563950in}{0.431121in}}%
\pgfpathlineto{\pgfqpoint{0.568161in}{0.427036in}}%
\pgfpathlineto{\pgfqpoint{0.579607in}{0.414684in}}%
\pgfpathlineto{\pgfqpoint{0.580830in}{0.413425in}}%
\pgfpathlineto{\pgfqpoint{0.592186in}{0.399814in}}%
\pgfpathlineto{\pgfqpoint{0.595263in}{0.395281in}}%
\pgfpathlineto{\pgfqpoint{0.601798in}{0.386203in}}%
\pgfpathlineto{\pgfqpoint{0.609062in}{0.372592in}}%
\pgfpathlineto{\pgfqpoint{0.610920in}{0.366929in}}%
\pgfpathlineto{\pgfqpoint{0.613715in}{0.358981in}}%
\pgfpathlineto{\pgfqpoint{0.615330in}{0.345370in}}%
\pgfpathlineto{\pgfqpoint{0.626577in}{0.345370in}}%
\pgfpathclose%
\pgfpathmoveto{\pgfqpoint{0.735244in}{0.413425in}}%
\pgfpathlineto{\pgfqpoint{0.720516in}{0.415566in}}%
\pgfpathlineto{\pgfqpoint{0.704859in}{0.419665in}}%
\pgfpathlineto{\pgfqpoint{0.689203in}{0.425590in}}%
\pgfpathlineto{\pgfqpoint{0.686254in}{0.427036in}}%
\pgfpathlineto{\pgfqpoint{0.673546in}{0.432555in}}%
\pgfpathlineto{\pgfqpoint{0.658428in}{0.440648in}}%
\pgfpathlineto{\pgfqpoint{0.657890in}{0.440911in}}%
\pgfpathlineto{\pgfqpoint{0.642233in}{0.449908in}}%
\pgfpathlineto{\pgfqpoint{0.635622in}{0.454259in}}%
\pgfpathlineto{\pgfqpoint{0.626577in}{0.459867in}}%
\pgfpathlineto{\pgfqpoint{0.615035in}{0.467870in}}%
\pgfpathlineto{\pgfqpoint{0.610920in}{0.470620in}}%
\pgfpathlineto{\pgfqpoint{0.596087in}{0.481481in}}%
\pgfpathlineto{\pgfqpoint{0.595263in}{0.482074in}}%
\pgfpathlineto{\pgfqpoint{0.579607in}{0.494178in}}%
\pgfpathlineto{\pgfqpoint{0.578486in}{0.495092in}}%
\pgfpathlineto{\pgfqpoint{0.563950in}{0.506997in}}%
\pgfpathlineto{\pgfqpoint{0.561945in}{0.508703in}}%
\pgfpathlineto{\pgfqpoint{0.548294in}{0.520571in}}%
\pgfpathlineto{\pgfqpoint{0.546331in}{0.522314in}}%
\pgfpathlineto{\pgfqpoint{0.532637in}{0.534951in}}%
\pgfpathlineto{\pgfqpoint{0.531586in}{0.535925in}}%
\pgfpathlineto{\pgfqpoint{0.517663in}{0.549536in}}%
\pgfpathlineto{\pgfqpoint{0.516981in}{0.550253in}}%
\pgfpathlineto{\pgfqpoint{0.504488in}{0.563148in}}%
\pgfpathlineto{\pgfqpoint{0.501324in}{0.566725in}}%
\pgfpathlineto{\pgfqpoint{0.492119in}{0.576759in}}%
\pgfpathlineto{\pgfqpoint{0.485668in}{0.584622in}}%
\pgfpathlineto{\pgfqpoint{0.480663in}{0.590370in}}%
\pgfpathlineto{\pgfqpoint{0.470314in}{0.603981in}}%
\pgfpathlineto{\pgfqpoint{0.470011in}{0.604449in}}%
\pgfpathlineto{\pgfqpoint{0.460702in}{0.617592in}}%
\pgfpathlineto{\pgfqpoint{0.454354in}{0.628639in}}%
\pgfpathlineto{\pgfqpoint{0.452691in}{0.631203in}}%
\pgfpathlineto{\pgfqpoint{0.445875in}{0.644814in}}%
\pgfpathlineto{\pgfqpoint{0.441160in}{0.658425in}}%
\pgfpathlineto{\pgfqpoint{0.438698in}{0.671230in}}%
\pgfpathlineto{\pgfqpoint{0.438514in}{0.672036in}}%
\pgfpathlineto{\pgfqpoint{0.437892in}{0.685648in}}%
\pgfpathlineto{\pgfqpoint{0.438698in}{0.691532in}}%
\pgfpathlineto{\pgfqpoint{0.439590in}{0.699259in}}%
\pgfpathlineto{\pgfqpoint{0.443255in}{0.712870in}}%
\pgfpathlineto{\pgfqpoint{0.449020in}{0.726481in}}%
\pgfpathlineto{\pgfqpoint{0.454354in}{0.735741in}}%
\pgfpathlineto{\pgfqpoint{0.456570in}{0.740092in}}%
\pgfpathlineto{\pgfqpoint{0.465295in}{0.753703in}}%
\pgfpathlineto{\pgfqpoint{0.470011in}{0.759834in}}%
\pgfpathlineto{\pgfqpoint{0.475282in}{0.767314in}}%
\pgfpathlineto{\pgfqpoint{0.485668in}{0.779983in}}%
\pgfpathlineto{\pgfqpoint{0.486394in}{0.780925in}}%
\pgfpathlineto{\pgfqpoint{0.498215in}{0.794536in}}%
\pgfpathlineto{\pgfqpoint{0.501324in}{0.797777in}}%
\pgfpathlineto{\pgfqpoint{0.510924in}{0.808148in}}%
\pgfpathlineto{\pgfqpoint{0.516981in}{0.814180in}}%
\pgfpathlineto{\pgfqpoint{0.524475in}{0.821759in}}%
\pgfpathlineto{\pgfqpoint{0.532637in}{0.829505in}}%
\pgfpathlineto{\pgfqpoint{0.538838in}{0.835370in}}%
\pgfpathlineto{\pgfqpoint{0.548294in}{0.843907in}}%
\pgfpathlineto{\pgfqpoint{0.554032in}{0.848981in}}%
\pgfpathlineto{\pgfqpoint{0.563950in}{0.857495in}}%
\pgfpathlineto{\pgfqpoint{0.570118in}{0.862592in}}%
\pgfpathlineto{\pgfqpoint{0.579607in}{0.870338in}}%
\pgfpathlineto{\pgfqpoint{0.587196in}{0.876203in}}%
\pgfpathlineto{\pgfqpoint{0.595263in}{0.882473in}}%
\pgfpathlineto{\pgfqpoint{0.605418in}{0.889814in}}%
\pgfpathlineto{\pgfqpoint{0.610920in}{0.893894in}}%
\pgfpathlineto{\pgfqpoint{0.624996in}{0.903425in}}%
\pgfpathlineto{\pgfqpoint{0.626577in}{0.904546in}}%
\pgfpathlineto{\pgfqpoint{0.642233in}{0.914535in}}%
\pgfpathlineto{\pgfqpoint{0.646719in}{0.917036in}}%
\pgfpathlineto{\pgfqpoint{0.657890in}{0.923742in}}%
\pgfpathlineto{\pgfqpoint{0.671491in}{0.930648in}}%
\pgfpathlineto{\pgfqpoint{0.673546in}{0.931804in}}%
\pgfpathlineto{\pgfqpoint{0.689203in}{0.939027in}}%
\pgfpathlineto{\pgfqpoint{0.704038in}{0.944259in}}%
\pgfpathlineto{\pgfqpoint{0.704859in}{0.944593in}}%
\pgfpathlineto{\pgfqpoint{0.720516in}{0.949037in}}%
\pgfpathlineto{\pgfqpoint{0.736173in}{0.951504in}}%
\pgfpathlineto{\pgfqpoint{0.751829in}{0.951997in}}%
\pgfpathlineto{\pgfqpoint{0.767486in}{0.950517in}}%
\pgfpathlineto{\pgfqpoint{0.783142in}{0.947062in}}%
\pgfpathlineto{\pgfqpoint{0.791251in}{0.944259in}}%
\pgfpathlineto{\pgfqpoint{0.798799in}{0.941999in}}%
\pgfpathlineto{\pgfqpoint{0.814455in}{0.935629in}}%
\pgfpathlineto{\pgfqpoint{0.824164in}{0.930648in}}%
\pgfpathlineto{\pgfqpoint{0.830112in}{0.927898in}}%
\pgfpathlineto{\pgfqpoint{0.845769in}{0.919207in}}%
\pgfpathlineto{\pgfqpoint{0.849152in}{0.917036in}}%
\pgfpathlineto{\pgfqpoint{0.861425in}{0.909711in}}%
\pgfpathlineto{\pgfqpoint{0.870732in}{0.903425in}}%
\pgfpathlineto{\pgfqpoint{0.877082in}{0.899337in}}%
\pgfpathlineto{\pgfqpoint{0.890451in}{0.889814in}}%
\pgfpathlineto{\pgfqpoint{0.892738in}{0.888227in}}%
\pgfpathlineto{\pgfqpoint{0.908395in}{0.876444in}}%
\pgfpathlineto{\pgfqpoint{0.908696in}{0.876203in}}%
\pgfpathlineto{\pgfqpoint{0.924051in}{0.863995in}}%
\pgfpathlineto{\pgfqpoint{0.925735in}{0.862592in}}%
\pgfpathlineto{\pgfqpoint{0.939708in}{0.850803in}}%
\pgfpathlineto{\pgfqpoint{0.941803in}{0.848981in}}%
\pgfpathlineto{\pgfqpoint{0.955364in}{0.836834in}}%
\pgfpathlineto{\pgfqpoint{0.956978in}{0.835370in}}%
\pgfpathlineto{\pgfqpoint{0.971021in}{0.822021in}}%
\pgfpathlineto{\pgfqpoint{0.971298in}{0.821759in}}%
\pgfpathlineto{\pgfqpoint{0.984851in}{0.808148in}}%
\pgfpathlineto{\pgfqpoint{0.986678in}{0.806159in}}%
\pgfpathlineto{\pgfqpoint{0.997632in}{0.794536in}}%
\pgfpathlineto{\pgfqpoint{1.002334in}{0.789017in}}%
\pgfpathlineto{\pgfqpoint{1.009564in}{0.780925in}}%
\pgfpathlineto{\pgfqpoint{1.017991in}{0.770255in}}%
\pgfpathlineto{\pgfqpoint{1.020488in}{0.767314in}}%
\pgfpathlineto{\pgfqpoint{1.030485in}{0.753703in}}%
\pgfpathlineto{\pgfqpoint{1.033647in}{0.748532in}}%
\pgfpathlineto{\pgfqpoint{1.039378in}{0.740092in}}%
\pgfpathlineto{\pgfqpoint{1.046705in}{0.726481in}}%
\pgfpathlineto{\pgfqpoint{1.049304in}{0.719919in}}%
\pgfpathlineto{\pgfqpoint{1.052529in}{0.712870in}}%
\pgfpathlineto{\pgfqpoint{1.056503in}{0.699259in}}%
\pgfpathlineto{\pgfqpoint{1.058206in}{0.685648in}}%
\pgfpathlineto{\pgfqpoint{1.057638in}{0.672036in}}%
\pgfpathlineto{\pgfqpoint{1.054800in}{0.658425in}}%
\pgfpathlineto{\pgfqpoint{1.049688in}{0.644814in}}%
\pgfpathlineto{\pgfqpoint{1.049304in}{0.644100in}}%
\pgfpathlineto{\pgfqpoint{1.043286in}{0.631203in}}%
\pgfpathlineto{\pgfqpoint{1.034978in}{0.617592in}}%
\pgfpathlineto{\pgfqpoint{1.033647in}{0.615805in}}%
\pgfpathlineto{\pgfqpoint{1.025704in}{0.603981in}}%
\pgfpathlineto{\pgfqpoint{1.017991in}{0.594269in}}%
\pgfpathlineto{\pgfqpoint{1.015114in}{0.590370in}}%
\pgfpathlineto{\pgfqpoint{1.003624in}{0.576759in}}%
\pgfpathlineto{\pgfqpoint{1.002334in}{0.575385in}}%
\pgfpathlineto{\pgfqpoint{0.991370in}{0.563148in}}%
\pgfpathlineto{\pgfqpoint{0.986678in}{0.558364in}}%
\pgfpathlineto{\pgfqpoint{0.978233in}{0.549536in}}%
\pgfpathlineto{\pgfqpoint{0.971021in}{0.542523in}}%
\pgfpathlineto{\pgfqpoint{0.964275in}{0.535925in}}%
\pgfpathlineto{\pgfqpoint{0.955364in}{0.527676in}}%
\pgfpathlineto{\pgfqpoint{0.949502in}{0.522314in}}%
\pgfpathlineto{\pgfqpoint{0.939708in}{0.513692in}}%
\pgfpathlineto{\pgfqpoint{0.933872in}{0.508703in}}%
\pgfpathlineto{\pgfqpoint{0.924051in}{0.500483in}}%
\pgfpathlineto{\pgfqpoint{0.917305in}{0.495092in}}%
\pgfpathlineto{\pgfqpoint{0.908395in}{0.487996in}}%
\pgfpathlineto{\pgfqpoint{0.899677in}{0.481481in}}%
\pgfpathlineto{\pgfqpoint{0.892738in}{0.476215in}}%
\pgfpathlineto{\pgfqpoint{0.880810in}{0.467870in}}%
\pgfpathlineto{\pgfqpoint{0.877082in}{0.465167in}}%
\pgfpathlineto{\pgfqpoint{0.861425in}{0.454890in}}%
\pgfpathlineto{\pgfqpoint{0.860342in}{0.454259in}}%
\pgfpathlineto{\pgfqpoint{0.845769in}{0.445230in}}%
\pgfpathlineto{\pgfqpoint{0.837164in}{0.440648in}}%
\pgfpathlineto{\pgfqpoint{0.830112in}{0.436547in}}%
\pgfpathlineto{\pgfqpoint{0.814455in}{0.428963in}}%
\pgfpathlineto{\pgfqpoint{0.809451in}{0.427036in}}%
\pgfpathlineto{\pgfqpoint{0.798799in}{0.422399in}}%
\pgfpathlineto{\pgfqpoint{0.783142in}{0.417387in}}%
\pgfpathlineto{\pgfqpoint{0.767486in}{0.414201in}}%
\pgfpathlineto{\pgfqpoint{0.758598in}{0.413425in}}%
\pgfpathlineto{\pgfqpoint{0.751829in}{0.412724in}}%
\pgfpathlineto{\pgfqpoint{0.736173in}{0.413265in}}%
\pgfpathlineto{\pgfqpoint{0.735244in}{0.413425in}}%
\pgfpathclose%
\pgfpathmoveto{\pgfqpoint{1.512232in}{0.413425in}}%
\pgfpathlineto{\pgfqpoint{1.503344in}{0.414201in}}%
\pgfpathlineto{\pgfqpoint{1.487688in}{0.417387in}}%
\pgfpathlineto{\pgfqpoint{1.472031in}{0.422399in}}%
\pgfpathlineto{\pgfqpoint{1.461379in}{0.427036in}}%
\pgfpathlineto{\pgfqpoint{1.456375in}{0.428963in}}%
\pgfpathlineto{\pgfqpoint{1.440718in}{0.436547in}}%
\pgfpathlineto{\pgfqpoint{1.433666in}{0.440648in}}%
\pgfpathlineto{\pgfqpoint{1.425061in}{0.445230in}}%
\pgfpathlineto{\pgfqpoint{1.410488in}{0.454259in}}%
\pgfpathlineto{\pgfqpoint{1.409405in}{0.454890in}}%
\pgfpathlineto{\pgfqpoint{1.393748in}{0.465167in}}%
\pgfpathlineto{\pgfqpoint{1.390020in}{0.467870in}}%
\pgfpathlineto{\pgfqpoint{1.378092in}{0.476215in}}%
\pgfpathlineto{\pgfqpoint{1.371153in}{0.481481in}}%
\pgfpathlineto{\pgfqpoint{1.362435in}{0.487996in}}%
\pgfpathlineto{\pgfqpoint{1.353525in}{0.495092in}}%
\pgfpathlineto{\pgfqpoint{1.346779in}{0.500483in}}%
\pgfpathlineto{\pgfqpoint{1.336958in}{0.508703in}}%
\pgfpathlineto{\pgfqpoint{1.331122in}{0.513692in}}%
\pgfpathlineto{\pgfqpoint{1.321328in}{0.522314in}}%
\pgfpathlineto{\pgfqpoint{1.315466in}{0.527676in}}%
\pgfpathlineto{\pgfqpoint{1.306555in}{0.535925in}}%
\pgfpathlineto{\pgfqpoint{1.299809in}{0.542523in}}%
\pgfpathlineto{\pgfqpoint{1.292597in}{0.549536in}}%
\pgfpathlineto{\pgfqpoint{1.284152in}{0.558364in}}%
\pgfpathlineto{\pgfqpoint{1.279460in}{0.563148in}}%
\pgfpathlineto{\pgfqpoint{1.268496in}{0.575385in}}%
\pgfpathlineto{\pgfqpoint{1.267206in}{0.576759in}}%
\pgfpathlineto{\pgfqpoint{1.255716in}{0.590370in}}%
\pgfpathlineto{\pgfqpoint{1.252839in}{0.594269in}}%
\pgfpathlineto{\pgfqpoint{1.245126in}{0.603981in}}%
\pgfpathlineto{\pgfqpoint{1.237183in}{0.615805in}}%
\pgfpathlineto{\pgfqpoint{1.235852in}{0.617592in}}%
\pgfpathlineto{\pgfqpoint{1.227544in}{0.631203in}}%
\pgfpathlineto{\pgfqpoint{1.221526in}{0.644100in}}%
\pgfpathlineto{\pgfqpoint{1.221142in}{0.644814in}}%
\pgfpathlineto{\pgfqpoint{1.216030in}{0.658425in}}%
\pgfpathlineto{\pgfqpoint{1.213192in}{0.672036in}}%
\pgfpathlineto{\pgfqpoint{1.212624in}{0.685648in}}%
\pgfpathlineto{\pgfqpoint{1.214327in}{0.699259in}}%
\pgfpathlineto{\pgfqpoint{1.218301in}{0.712870in}}%
\pgfpathlineto{\pgfqpoint{1.221526in}{0.719919in}}%
\pgfpathlineto{\pgfqpoint{1.224125in}{0.726481in}}%
\pgfpathlineto{\pgfqpoint{1.231452in}{0.740092in}}%
\pgfpathlineto{\pgfqpoint{1.237183in}{0.748532in}}%
\pgfpathlineto{\pgfqpoint{1.240345in}{0.753703in}}%
\pgfpathlineto{\pgfqpoint{1.250342in}{0.767314in}}%
\pgfpathlineto{\pgfqpoint{1.252839in}{0.770255in}}%
\pgfpathlineto{\pgfqpoint{1.261266in}{0.780925in}}%
\pgfpathlineto{\pgfqpoint{1.268496in}{0.789017in}}%
\pgfpathlineto{\pgfqpoint{1.273198in}{0.794536in}}%
\pgfpathlineto{\pgfqpoint{1.284152in}{0.806159in}}%
\pgfpathlineto{\pgfqpoint{1.285979in}{0.808148in}}%
\pgfpathlineto{\pgfqpoint{1.299532in}{0.821759in}}%
\pgfpathlineto{\pgfqpoint{1.299809in}{0.822021in}}%
\pgfpathlineto{\pgfqpoint{1.313852in}{0.835370in}}%
\pgfpathlineto{\pgfqpoint{1.315466in}{0.836834in}}%
\pgfpathlineto{\pgfqpoint{1.329027in}{0.848981in}}%
\pgfpathlineto{\pgfqpoint{1.331122in}{0.850803in}}%
\pgfpathlineto{\pgfqpoint{1.345095in}{0.862592in}}%
\pgfpathlineto{\pgfqpoint{1.346779in}{0.863995in}}%
\pgfpathlineto{\pgfqpoint{1.362134in}{0.876203in}}%
\pgfpathlineto{\pgfqpoint{1.362435in}{0.876444in}}%
\pgfpathlineto{\pgfqpoint{1.378092in}{0.888227in}}%
\pgfpathlineto{\pgfqpoint{1.380379in}{0.889814in}}%
\pgfpathlineto{\pgfqpoint{1.393748in}{0.899337in}}%
\pgfpathlineto{\pgfqpoint{1.400098in}{0.903425in}}%
\pgfpathlineto{\pgfqpoint{1.409405in}{0.909711in}}%
\pgfpathlineto{\pgfqpoint{1.421678in}{0.917036in}}%
\pgfpathlineto{\pgfqpoint{1.425061in}{0.919207in}}%
\pgfpathlineto{\pgfqpoint{1.440718in}{0.927898in}}%
\pgfpathlineto{\pgfqpoint{1.446666in}{0.930648in}}%
\pgfpathlineto{\pgfqpoint{1.456375in}{0.935629in}}%
\pgfpathlineto{\pgfqpoint{1.472031in}{0.941999in}}%
\pgfpathlineto{\pgfqpoint{1.479579in}{0.944259in}}%
\pgfpathlineto{\pgfqpoint{1.487688in}{0.947062in}}%
\pgfpathlineto{\pgfqpoint{1.503344in}{0.950517in}}%
\pgfpathlineto{\pgfqpoint{1.519001in}{0.951997in}}%
\pgfpathlineto{\pgfqpoint{1.534657in}{0.951504in}}%
\pgfpathlineto{\pgfqpoint{1.550314in}{0.949037in}}%
\pgfpathlineto{\pgfqpoint{1.565971in}{0.944593in}}%
\pgfpathlineto{\pgfqpoint{1.566792in}{0.944259in}}%
\pgfpathlineto{\pgfqpoint{1.581627in}{0.939027in}}%
\pgfpathlineto{\pgfqpoint{1.597284in}{0.931804in}}%
\pgfpathlineto{\pgfqpoint{1.599339in}{0.930648in}}%
\pgfpathlineto{\pgfqpoint{1.612940in}{0.923742in}}%
\pgfpathlineto{\pgfqpoint{1.624111in}{0.917036in}}%
\pgfpathlineto{\pgfqpoint{1.628597in}{0.914535in}}%
\pgfpathlineto{\pgfqpoint{1.644253in}{0.904546in}}%
\pgfpathlineto{\pgfqpoint{1.645834in}{0.903425in}}%
\pgfpathlineto{\pgfqpoint{1.659910in}{0.893894in}}%
\pgfpathlineto{\pgfqpoint{1.665412in}{0.889814in}}%
\pgfpathlineto{\pgfqpoint{1.675567in}{0.882473in}}%
\pgfpathlineto{\pgfqpoint{1.683634in}{0.876203in}}%
\pgfpathlineto{\pgfqpoint{1.691223in}{0.870338in}}%
\pgfpathlineto{\pgfqpoint{1.700712in}{0.862592in}}%
\pgfpathlineto{\pgfqpoint{1.706880in}{0.857495in}}%
\pgfpathlineto{\pgfqpoint{1.716798in}{0.848981in}}%
\pgfpathlineto{\pgfqpoint{1.722536in}{0.843907in}}%
\pgfpathlineto{\pgfqpoint{1.731992in}{0.835370in}}%
\pgfpathlineto{\pgfqpoint{1.738193in}{0.829505in}}%
\pgfpathlineto{\pgfqpoint{1.746355in}{0.821759in}}%
\pgfpathlineto{\pgfqpoint{1.753849in}{0.814180in}}%
\pgfpathlineto{\pgfqpoint{1.759906in}{0.808148in}}%
\pgfpathlineto{\pgfqpoint{1.769506in}{0.797777in}}%
\pgfpathlineto{\pgfqpoint{1.772615in}{0.794536in}}%
\pgfpathlineto{\pgfqpoint{1.784436in}{0.780925in}}%
\pgfpathlineto{\pgfqpoint{1.785162in}{0.779983in}}%
\pgfpathlineto{\pgfqpoint{1.795548in}{0.767314in}}%
\pgfpathlineto{\pgfqpoint{1.800819in}{0.759834in}}%
\pgfpathlineto{\pgfqpoint{1.805535in}{0.753703in}}%
\pgfpathlineto{\pgfqpoint{1.814260in}{0.740092in}}%
\pgfpathlineto{\pgfqpoint{1.816476in}{0.735741in}}%
\pgfpathlineto{\pgfqpoint{1.821810in}{0.726481in}}%
\pgfpathlineto{\pgfqpoint{1.827575in}{0.712870in}}%
\pgfpathlineto{\pgfqpoint{1.831240in}{0.699259in}}%
\pgfpathlineto{\pgfqpoint{1.832132in}{0.691532in}}%
\pgfpathlineto{\pgfqpoint{1.832938in}{0.685648in}}%
\pgfpathlineto{\pgfqpoint{1.832316in}{0.672036in}}%
\pgfpathlineto{\pgfqpoint{1.832132in}{0.671230in}}%
\pgfpathlineto{\pgfqpoint{1.829670in}{0.658425in}}%
\pgfpathlineto{\pgfqpoint{1.824955in}{0.644814in}}%
\pgfpathlineto{\pgfqpoint{1.818139in}{0.631203in}}%
\pgfpathlineto{\pgfqpoint{1.816476in}{0.628639in}}%
\pgfpathlineto{\pgfqpoint{1.810128in}{0.617592in}}%
\pgfpathlineto{\pgfqpoint{1.800819in}{0.604449in}}%
\pgfpathlineto{\pgfqpoint{1.800516in}{0.603981in}}%
\pgfpathlineto{\pgfqpoint{1.790167in}{0.590370in}}%
\pgfpathlineto{\pgfqpoint{1.785162in}{0.584622in}}%
\pgfpathlineto{\pgfqpoint{1.778711in}{0.576759in}}%
\pgfpathlineto{\pgfqpoint{1.769506in}{0.566725in}}%
\pgfpathlineto{\pgfqpoint{1.766342in}{0.563148in}}%
\pgfpathlineto{\pgfqpoint{1.753849in}{0.550253in}}%
\pgfpathlineto{\pgfqpoint{1.753167in}{0.549536in}}%
\pgfpathlineto{\pgfqpoint{1.739244in}{0.535925in}}%
\pgfpathlineto{\pgfqpoint{1.738193in}{0.534951in}}%
\pgfpathlineto{\pgfqpoint{1.724499in}{0.522314in}}%
\pgfpathlineto{\pgfqpoint{1.722536in}{0.520571in}}%
\pgfpathlineto{\pgfqpoint{1.708885in}{0.508703in}}%
\pgfpathlineto{\pgfqpoint{1.706880in}{0.506997in}}%
\pgfpathlineto{\pgfqpoint{1.692344in}{0.495092in}}%
\pgfpathlineto{\pgfqpoint{1.691223in}{0.494178in}}%
\pgfpathlineto{\pgfqpoint{1.675567in}{0.482074in}}%
\pgfpathlineto{\pgfqpoint{1.674743in}{0.481481in}}%
\pgfpathlineto{\pgfqpoint{1.659910in}{0.470620in}}%
\pgfpathlineto{\pgfqpoint{1.655795in}{0.467870in}}%
\pgfpathlineto{\pgfqpoint{1.644253in}{0.459867in}}%
\pgfpathlineto{\pgfqpoint{1.635208in}{0.454259in}}%
\pgfpathlineto{\pgfqpoint{1.628597in}{0.449908in}}%
\pgfpathlineto{\pgfqpoint{1.612940in}{0.440911in}}%
\pgfpathlineto{\pgfqpoint{1.612402in}{0.440648in}}%
\pgfpathlineto{\pgfqpoint{1.597284in}{0.432555in}}%
\pgfpathlineto{\pgfqpoint{1.584576in}{0.427036in}}%
\pgfpathlineto{\pgfqpoint{1.581627in}{0.425590in}}%
\pgfpathlineto{\pgfqpoint{1.565971in}{0.419665in}}%
\pgfpathlineto{\pgfqpoint{1.550314in}{0.415566in}}%
\pgfpathlineto{\pgfqpoint{1.535586in}{0.413425in}}%
\pgfpathlineto{\pgfqpoint{1.534657in}{0.413265in}}%
\pgfpathlineto{\pgfqpoint{1.519001in}{0.412724in}}%
\pgfpathlineto{\pgfqpoint{1.512232in}{0.413425in}}%
\pgfpathclose%
\pgfpathmoveto{\pgfqpoint{1.091617in}{0.808148in}}%
\pgfpathlineto{\pgfqpoint{1.080617in}{0.813225in}}%
\pgfpathlineto{\pgfqpoint{1.066134in}{0.821759in}}%
\pgfpathlineto{\pgfqpoint{1.064960in}{0.822415in}}%
\pgfpathlineto{\pgfqpoint{1.049304in}{0.832876in}}%
\pgfpathlineto{\pgfqpoint{1.046016in}{0.835370in}}%
\pgfpathlineto{\pgfqpoint{1.033647in}{0.844377in}}%
\pgfpathlineto{\pgfqpoint{1.027871in}{0.848981in}}%
\pgfpathlineto{\pgfqpoint{1.017991in}{0.856648in}}%
\pgfpathlineto{\pgfqpoint{1.010796in}{0.862592in}}%
\pgfpathlineto{\pgfqpoint{1.002334in}{0.869497in}}%
\pgfpathlineto{\pgfqpoint{0.994437in}{0.876203in}}%
\pgfpathlineto{\pgfqpoint{0.986678in}{0.882813in}}%
\pgfpathlineto{\pgfqpoint{0.978624in}{0.889814in}}%
\pgfpathlineto{\pgfqpoint{0.971021in}{0.896560in}}%
\pgfpathlineto{\pgfqpoint{0.963307in}{0.903425in}}%
\pgfpathlineto{\pgfqpoint{0.955364in}{0.910782in}}%
\pgfpathlineto{\pgfqpoint{0.948528in}{0.917036in}}%
\pgfpathlineto{\pgfqpoint{0.939708in}{0.925626in}}%
\pgfpathlineto{\pgfqpoint{0.934412in}{0.930648in}}%
\pgfpathlineto{\pgfqpoint{0.924051in}{0.941400in}}%
\pgfpathlineto{\pgfqpoint{0.921182in}{0.944259in}}%
\pgfpathlineto{\pgfqpoint{0.909150in}{0.957870in}}%
\pgfpathlineto{\pgfqpoint{0.908395in}{0.958890in}}%
\pgfpathlineto{\pgfqpoint{0.898579in}{0.971481in}}%
\pgfpathlineto{\pgfqpoint{0.892738in}{0.981044in}}%
\pgfpathlineto{\pgfqpoint{0.890097in}{0.985092in}}%
\pgfpathlineto{\pgfqpoint{0.884020in}{0.998703in}}%
\pgfpathlineto{\pgfqpoint{0.880916in}{1.012314in}}%
\pgfpathlineto{\pgfqpoint{0.880916in}{1.025925in}}%
\pgfpathlineto{\pgfqpoint{0.884020in}{1.039536in}}%
\pgfpathlineto{\pgfqpoint{0.890097in}{1.053148in}}%
\pgfpathlineto{\pgfqpoint{0.892738in}{1.057196in}}%
\pgfpathlineto{\pgfqpoint{0.898579in}{1.066759in}}%
\pgfpathlineto{\pgfqpoint{0.908395in}{1.079350in}}%
\pgfpathlineto{\pgfqpoint{0.909150in}{1.080370in}}%
\pgfpathlineto{\pgfqpoint{0.921182in}{1.093981in}}%
\pgfpathlineto{\pgfqpoint{0.924051in}{1.096839in}}%
\pgfpathlineto{\pgfqpoint{0.934412in}{1.107592in}}%
\pgfpathlineto{\pgfqpoint{0.939708in}{1.112613in}}%
\pgfpathlineto{\pgfqpoint{0.948528in}{1.121203in}}%
\pgfpathlineto{\pgfqpoint{0.955364in}{1.127458in}}%
\pgfpathlineto{\pgfqpoint{0.963307in}{1.134814in}}%
\pgfpathlineto{\pgfqpoint{0.971021in}{1.141680in}}%
\pgfpathlineto{\pgfqpoint{0.978624in}{1.148425in}}%
\pgfpathlineto{\pgfqpoint{0.986678in}{1.155427in}}%
\pgfpathlineto{\pgfqpoint{0.994437in}{1.162036in}}%
\pgfpathlineto{\pgfqpoint{1.002334in}{1.168742in}}%
\pgfpathlineto{\pgfqpoint{1.010796in}{1.175647in}}%
\pgfpathlineto{\pgfqpoint{1.017991in}{1.181591in}}%
\pgfpathlineto{\pgfqpoint{1.027871in}{1.189259in}}%
\pgfpathlineto{\pgfqpoint{1.033647in}{1.193862in}}%
\pgfpathlineto{\pgfqpoint{1.046016in}{1.202870in}}%
\pgfpathlineto{\pgfqpoint{1.049304in}{1.205364in}}%
\pgfpathlineto{\pgfqpoint{1.064960in}{1.215824in}}%
\pgfpathlineto{\pgfqpoint{1.066134in}{1.216481in}}%
\pgfpathlineto{\pgfqpoint{1.080617in}{1.225014in}}%
\pgfpathlineto{\pgfqpoint{1.091617in}{1.230092in}}%
\pgfpathlineto{\pgfqpoint{1.096274in}{1.232388in}}%
\pgfpathlineto{\pgfqpoint{1.111930in}{1.237671in}}%
\pgfpathlineto{\pgfqpoint{1.127587in}{1.240370in}}%
\pgfpathlineto{\pgfqpoint{1.143243in}{1.240370in}}%
\pgfpathlineto{\pgfqpoint{1.158900in}{1.237671in}}%
\pgfpathlineto{\pgfqpoint{1.174556in}{1.232388in}}%
\pgfpathlineto{\pgfqpoint{1.179213in}{1.230092in}}%
\pgfpathlineto{\pgfqpoint{1.190213in}{1.225014in}}%
\pgfpathlineto{\pgfqpoint{1.204696in}{1.216481in}}%
\pgfpathlineto{\pgfqpoint{1.205870in}{1.215824in}}%
\pgfpathlineto{\pgfqpoint{1.221526in}{1.205364in}}%
\pgfpathlineto{\pgfqpoint{1.224814in}{1.202870in}}%
\pgfpathlineto{\pgfqpoint{1.237183in}{1.193862in}}%
\pgfpathlineto{\pgfqpoint{1.242959in}{1.189259in}}%
\pgfpathlineto{\pgfqpoint{1.252839in}{1.181591in}}%
\pgfpathlineto{\pgfqpoint{1.260034in}{1.175647in}}%
\pgfpathlineto{\pgfqpoint{1.268496in}{1.168742in}}%
\pgfpathlineto{\pgfqpoint{1.276393in}{1.162036in}}%
\pgfpathlineto{\pgfqpoint{1.284152in}{1.155427in}}%
\pgfpathlineto{\pgfqpoint{1.292206in}{1.148425in}}%
\pgfpathlineto{\pgfqpoint{1.299809in}{1.141680in}}%
\pgfpathlineto{\pgfqpoint{1.307523in}{1.134814in}}%
\pgfpathlineto{\pgfqpoint{1.315466in}{1.127458in}}%
\pgfpathlineto{\pgfqpoint{1.322302in}{1.121203in}}%
\pgfpathlineto{\pgfqpoint{1.331122in}{1.112613in}}%
\pgfpathlineto{\pgfqpoint{1.336418in}{1.107592in}}%
\pgfpathlineto{\pgfqpoint{1.346779in}{1.096839in}}%
\pgfpathlineto{\pgfqpoint{1.349648in}{1.093981in}}%
\pgfpathlineto{\pgfqpoint{1.361680in}{1.080370in}}%
\pgfpathlineto{\pgfqpoint{1.362435in}{1.079350in}}%
\pgfpathlineto{\pgfqpoint{1.372251in}{1.066759in}}%
\pgfpathlineto{\pgfqpoint{1.378092in}{1.057196in}}%
\pgfpathlineto{\pgfqpoint{1.380733in}{1.053148in}}%
\pgfpathlineto{\pgfqpoint{1.386810in}{1.039536in}}%
\pgfpathlineto{\pgfqpoint{1.389914in}{1.025925in}}%
\pgfpathlineto{\pgfqpoint{1.389914in}{1.012314in}}%
\pgfpathlineto{\pgfqpoint{1.386810in}{0.998703in}}%
\pgfpathlineto{\pgfqpoint{1.380733in}{0.985092in}}%
\pgfpathlineto{\pgfqpoint{1.378092in}{0.981044in}}%
\pgfpathlineto{\pgfqpoint{1.372251in}{0.971481in}}%
\pgfpathlineto{\pgfqpoint{1.362435in}{0.958890in}}%
\pgfpathlineto{\pgfqpoint{1.361680in}{0.957870in}}%
\pgfpathlineto{\pgfqpoint{1.349648in}{0.944259in}}%
\pgfpathlineto{\pgfqpoint{1.346779in}{0.941400in}}%
\pgfpathlineto{\pgfqpoint{1.336418in}{0.930648in}}%
\pgfpathlineto{\pgfqpoint{1.331122in}{0.925626in}}%
\pgfpathlineto{\pgfqpoint{1.322302in}{0.917036in}}%
\pgfpathlineto{\pgfqpoint{1.315466in}{0.910782in}}%
\pgfpathlineto{\pgfqpoint{1.307523in}{0.903425in}}%
\pgfpathlineto{\pgfqpoint{1.299809in}{0.896560in}}%
\pgfpathlineto{\pgfqpoint{1.292206in}{0.889814in}}%
\pgfpathlineto{\pgfqpoint{1.284152in}{0.882813in}}%
\pgfpathlineto{\pgfqpoint{1.276393in}{0.876203in}}%
\pgfpathlineto{\pgfqpoint{1.268496in}{0.869497in}}%
\pgfpathlineto{\pgfqpoint{1.260034in}{0.862592in}}%
\pgfpathlineto{\pgfqpoint{1.252839in}{0.856648in}}%
\pgfpathlineto{\pgfqpoint{1.242959in}{0.848981in}}%
\pgfpathlineto{\pgfqpoint{1.237183in}{0.844377in}}%
\pgfpathlineto{\pgfqpoint{1.224814in}{0.835370in}}%
\pgfpathlineto{\pgfqpoint{1.221526in}{0.832876in}}%
\pgfpathlineto{\pgfqpoint{1.205870in}{0.822415in}}%
\pgfpathlineto{\pgfqpoint{1.204696in}{0.821759in}}%
\pgfpathlineto{\pgfqpoint{1.190213in}{0.813225in}}%
\pgfpathlineto{\pgfqpoint{1.179213in}{0.808148in}}%
\pgfpathlineto{\pgfqpoint{1.174556in}{0.805852in}}%
\pgfpathlineto{\pgfqpoint{1.158900in}{0.800569in}}%
\pgfpathlineto{\pgfqpoint{1.143243in}{0.797870in}}%
\pgfpathlineto{\pgfqpoint{1.127587in}{0.797870in}}%
\pgfpathlineto{\pgfqpoint{1.111930in}{0.800569in}}%
\pgfpathlineto{\pgfqpoint{1.096274in}{0.805852in}}%
\pgfpathlineto{\pgfqpoint{1.091617in}{0.808148in}}%
\pgfpathclose%
\pgfpathmoveto{\pgfqpoint{0.704038in}{1.093981in}}%
\pgfpathlineto{\pgfqpoint{0.689203in}{1.099212in}}%
\pgfpathlineto{\pgfqpoint{0.673546in}{1.106435in}}%
\pgfpathlineto{\pgfqpoint{0.671491in}{1.107592in}}%
\pgfpathlineto{\pgfqpoint{0.657890in}{1.114498in}}%
\pgfpathlineto{\pgfqpoint{0.646719in}{1.121203in}}%
\pgfpathlineto{\pgfqpoint{0.642233in}{1.123704in}}%
\pgfpathlineto{\pgfqpoint{0.626577in}{1.133693in}}%
\pgfpathlineto{\pgfqpoint{0.624996in}{1.134814in}}%
\pgfpathlineto{\pgfqpoint{0.610920in}{1.144346in}}%
\pgfpathlineto{\pgfqpoint{0.605418in}{1.148425in}}%
\pgfpathlineto{\pgfqpoint{0.595263in}{1.155767in}}%
\pgfpathlineto{\pgfqpoint{0.587196in}{1.162036in}}%
\pgfpathlineto{\pgfqpoint{0.579607in}{1.167901in}}%
\pgfpathlineto{\pgfqpoint{0.570118in}{1.175647in}}%
\pgfpathlineto{\pgfqpoint{0.563950in}{1.180744in}}%
\pgfpathlineto{\pgfqpoint{0.554032in}{1.189259in}}%
\pgfpathlineto{\pgfqpoint{0.548294in}{1.194332in}}%
\pgfpathlineto{\pgfqpoint{0.538838in}{1.202870in}}%
\pgfpathlineto{\pgfqpoint{0.532637in}{1.208734in}}%
\pgfpathlineto{\pgfqpoint{0.524475in}{1.216481in}}%
\pgfpathlineto{\pgfqpoint{0.516981in}{1.224059in}}%
\pgfpathlineto{\pgfqpoint{0.510924in}{1.230092in}}%
\pgfpathlineto{\pgfqpoint{0.501324in}{1.240462in}}%
\pgfpathlineto{\pgfqpoint{0.498215in}{1.243703in}}%
\pgfpathlineto{\pgfqpoint{0.486394in}{1.257314in}}%
\pgfpathlineto{\pgfqpoint{0.485668in}{1.258256in}}%
\pgfpathlineto{\pgfqpoint{0.475282in}{1.270925in}}%
\pgfpathlineto{\pgfqpoint{0.470011in}{1.278406in}}%
\pgfpathlineto{\pgfqpoint{0.465295in}{1.284536in}}%
\pgfpathlineto{\pgfqpoint{0.456570in}{1.298148in}}%
\pgfpathlineto{\pgfqpoint{0.454354in}{1.302498in}}%
\pgfpathlineto{\pgfqpoint{0.449020in}{1.311759in}}%
\pgfpathlineto{\pgfqpoint{0.443255in}{1.325370in}}%
\pgfpathlineto{\pgfqpoint{0.439590in}{1.338981in}}%
\pgfpathlineto{\pgfqpoint{0.438698in}{1.346708in}}%
\pgfpathlineto{\pgfqpoint{0.437892in}{1.352592in}}%
\pgfpathlineto{\pgfqpoint{0.438514in}{1.366203in}}%
\pgfpathlineto{\pgfqpoint{0.438698in}{1.367010in}}%
\pgfpathlineto{\pgfqpoint{0.441160in}{1.379814in}}%
\pgfpathlineto{\pgfqpoint{0.445875in}{1.393425in}}%
\pgfpathlineto{\pgfqpoint{0.452691in}{1.407036in}}%
\pgfpathlineto{\pgfqpoint{0.454354in}{1.409600in}}%
\pgfpathlineto{\pgfqpoint{0.460702in}{1.420648in}}%
\pgfpathlineto{\pgfqpoint{0.470011in}{1.433790in}}%
\pgfpathlineto{\pgfqpoint{0.470314in}{1.434259in}}%
\pgfpathlineto{\pgfqpoint{0.480663in}{1.447870in}}%
\pgfpathlineto{\pgfqpoint{0.485668in}{1.453617in}}%
\pgfpathlineto{\pgfqpoint{0.492119in}{1.461481in}}%
\pgfpathlineto{\pgfqpoint{0.501324in}{1.471515in}}%
\pgfpathlineto{\pgfqpoint{0.504488in}{1.475092in}}%
\pgfpathlineto{\pgfqpoint{0.516981in}{1.487987in}}%
\pgfpathlineto{\pgfqpoint{0.517663in}{1.488703in}}%
\pgfpathlineto{\pgfqpoint{0.531586in}{1.502314in}}%
\pgfpathlineto{\pgfqpoint{0.532637in}{1.503288in}}%
\pgfpathlineto{\pgfqpoint{0.546331in}{1.515925in}}%
\pgfpathlineto{\pgfqpoint{0.548294in}{1.517669in}}%
\pgfpathlineto{\pgfqpoint{0.561945in}{1.529536in}}%
\pgfpathlineto{\pgfqpoint{0.563950in}{1.531243in}}%
\pgfpathlineto{\pgfqpoint{0.578486in}{1.543148in}}%
\pgfpathlineto{\pgfqpoint{0.579607in}{1.544062in}}%
\pgfpathlineto{\pgfqpoint{0.595263in}{1.556165in}}%
\pgfpathlineto{\pgfqpoint{0.596087in}{1.556759in}}%
\pgfpathlineto{\pgfqpoint{0.610920in}{1.567619in}}%
\pgfpathlineto{\pgfqpoint{0.615035in}{1.570370in}}%
\pgfpathlineto{\pgfqpoint{0.626577in}{1.578373in}}%
\pgfpathlineto{\pgfqpoint{0.635622in}{1.583981in}}%
\pgfpathlineto{\pgfqpoint{0.642233in}{1.588331in}}%
\pgfpathlineto{\pgfqpoint{0.657890in}{1.597328in}}%
\pgfpathlineto{\pgfqpoint{0.658428in}{1.597592in}}%
\pgfpathlineto{\pgfqpoint{0.673546in}{1.605685in}}%
\pgfpathlineto{\pgfqpoint{0.686254in}{1.611203in}}%
\pgfpathlineto{\pgfqpoint{0.689203in}{1.612649in}}%
\pgfpathlineto{\pgfqpoint{0.704859in}{1.618575in}}%
\pgfpathlineto{\pgfqpoint{0.720516in}{1.622673in}}%
\pgfpathlineto{\pgfqpoint{0.735244in}{1.624814in}}%
\pgfpathlineto{\pgfqpoint{0.736173in}{1.624974in}}%
\pgfpathlineto{\pgfqpoint{0.751829in}{1.625515in}}%
\pgfpathlineto{\pgfqpoint{0.758598in}{1.624814in}}%
\pgfpathlineto{\pgfqpoint{0.767486in}{1.624039in}}%
\pgfpathlineto{\pgfqpoint{0.783142in}{1.620852in}}%
\pgfpathlineto{\pgfqpoint{0.798799in}{1.615841in}}%
\pgfpathlineto{\pgfqpoint{0.809451in}{1.611203in}}%
\pgfpathlineto{\pgfqpoint{0.814455in}{1.609277in}}%
\pgfpathlineto{\pgfqpoint{0.830112in}{1.601692in}}%
\pgfpathlineto{\pgfqpoint{0.837164in}{1.597592in}}%
\pgfpathlineto{\pgfqpoint{0.845769in}{1.593009in}}%
\pgfpathlineto{\pgfqpoint{0.860342in}{1.583981in}}%
\pgfpathlineto{\pgfqpoint{0.861425in}{1.583349in}}%
\pgfpathlineto{\pgfqpoint{0.877082in}{1.573072in}}%
\pgfpathlineto{\pgfqpoint{0.880810in}{1.570370in}}%
\pgfpathlineto{\pgfqpoint{0.892738in}{1.562024in}}%
\pgfpathlineto{\pgfqpoint{0.899677in}{1.556759in}}%
\pgfpathlineto{\pgfqpoint{0.908395in}{1.550243in}}%
\pgfpathlineto{\pgfqpoint{0.917305in}{1.543148in}}%
\pgfpathlineto{\pgfqpoint{0.924051in}{1.537757in}}%
\pgfpathlineto{\pgfqpoint{0.933872in}{1.529536in}}%
\pgfpathlineto{\pgfqpoint{0.939708in}{1.524548in}}%
\pgfpathlineto{\pgfqpoint{0.949502in}{1.515925in}}%
\pgfpathlineto{\pgfqpoint{0.955364in}{1.510564in}}%
\pgfpathlineto{\pgfqpoint{0.964275in}{1.502314in}}%
\pgfpathlineto{\pgfqpoint{0.971021in}{1.495716in}}%
\pgfpathlineto{\pgfqpoint{0.978233in}{1.488703in}}%
\pgfpathlineto{\pgfqpoint{0.986678in}{1.479876in}}%
\pgfpathlineto{\pgfqpoint{0.991370in}{1.475092in}}%
\pgfpathlineto{\pgfqpoint{1.002334in}{1.462855in}}%
\pgfpathlineto{\pgfqpoint{1.003624in}{1.461481in}}%
\pgfpathlineto{\pgfqpoint{1.015114in}{1.447870in}}%
\pgfpathlineto{\pgfqpoint{1.017991in}{1.443970in}}%
\pgfpathlineto{\pgfqpoint{1.025704in}{1.434259in}}%
\pgfpathlineto{\pgfqpoint{1.033647in}{1.422434in}}%
\pgfpathlineto{\pgfqpoint{1.034978in}{1.420648in}}%
\pgfpathlineto{\pgfqpoint{1.043286in}{1.407036in}}%
\pgfpathlineto{\pgfqpoint{1.049304in}{1.394139in}}%
\pgfpathlineto{\pgfqpoint{1.049688in}{1.393425in}}%
\pgfpathlineto{\pgfqpoint{1.054800in}{1.379814in}}%
\pgfpathlineto{\pgfqpoint{1.057638in}{1.366203in}}%
\pgfpathlineto{\pgfqpoint{1.058206in}{1.352592in}}%
\pgfpathlineto{\pgfqpoint{1.056503in}{1.338981in}}%
\pgfpathlineto{\pgfqpoint{1.052529in}{1.325370in}}%
\pgfpathlineto{\pgfqpoint{1.049304in}{1.318320in}}%
\pgfpathlineto{\pgfqpoint{1.046705in}{1.311759in}}%
\pgfpathlineto{\pgfqpoint{1.039378in}{1.298148in}}%
\pgfpathlineto{\pgfqpoint{1.033647in}{1.289708in}}%
\pgfpathlineto{\pgfqpoint{1.030485in}{1.284536in}}%
\pgfpathlineto{\pgfqpoint{1.020488in}{1.270925in}}%
\pgfpathlineto{\pgfqpoint{1.017991in}{1.267984in}}%
\pgfpathlineto{\pgfqpoint{1.009564in}{1.257314in}}%
\pgfpathlineto{\pgfqpoint{1.002334in}{1.249223in}}%
\pgfpathlineto{\pgfqpoint{0.997632in}{1.243703in}}%
\pgfpathlineto{\pgfqpoint{0.986678in}{1.232081in}}%
\pgfpathlineto{\pgfqpoint{0.984851in}{1.230092in}}%
\pgfpathlineto{\pgfqpoint{0.971298in}{1.216481in}}%
\pgfpathlineto{\pgfqpoint{0.971021in}{1.216219in}}%
\pgfpathlineto{\pgfqpoint{0.956978in}{1.202870in}}%
\pgfpathlineto{\pgfqpoint{0.955364in}{1.201406in}}%
\pgfpathlineto{\pgfqpoint{0.941803in}{1.189259in}}%
\pgfpathlineto{\pgfqpoint{0.939708in}{1.187437in}}%
\pgfpathlineto{\pgfqpoint{0.925735in}{1.175647in}}%
\pgfpathlineto{\pgfqpoint{0.924051in}{1.174245in}}%
\pgfpathlineto{\pgfqpoint{0.908696in}{1.162036in}}%
\pgfpathlineto{\pgfqpoint{0.908395in}{1.161795in}}%
\pgfpathlineto{\pgfqpoint{0.892738in}{1.150013in}}%
\pgfpathlineto{\pgfqpoint{0.890451in}{1.148425in}}%
\pgfpathlineto{\pgfqpoint{0.877082in}{1.138902in}}%
\pgfpathlineto{\pgfqpoint{0.870732in}{1.134814in}}%
\pgfpathlineto{\pgfqpoint{0.861425in}{1.128529in}}%
\pgfpathlineto{\pgfqpoint{0.849152in}{1.121203in}}%
\pgfpathlineto{\pgfqpoint{0.845769in}{1.119032in}}%
\pgfpathlineto{\pgfqpoint{0.830112in}{1.110342in}}%
\pgfpathlineto{\pgfqpoint{0.824164in}{1.107592in}}%
\pgfpathlineto{\pgfqpoint{0.814455in}{1.102610in}}%
\pgfpathlineto{\pgfqpoint{0.798799in}{1.096240in}}%
\pgfpathlineto{\pgfqpoint{0.791251in}{1.093981in}}%
\pgfpathlineto{\pgfqpoint{0.783142in}{1.091177in}}%
\pgfpathlineto{\pgfqpoint{0.767486in}{1.087722in}}%
\pgfpathlineto{\pgfqpoint{0.751829in}{1.086242in}}%
\pgfpathlineto{\pgfqpoint{0.736173in}{1.086735in}}%
\pgfpathlineto{\pgfqpoint{0.720516in}{1.089203in}}%
\pgfpathlineto{\pgfqpoint{0.704859in}{1.093647in}}%
\pgfpathlineto{\pgfqpoint{0.704038in}{1.093981in}}%
\pgfpathclose%
\pgfpathmoveto{\pgfqpoint{1.479579in}{1.093981in}}%
\pgfpathlineto{\pgfqpoint{1.472031in}{1.096240in}}%
\pgfpathlineto{\pgfqpoint{1.456375in}{1.102610in}}%
\pgfpathlineto{\pgfqpoint{1.446666in}{1.107592in}}%
\pgfpathlineto{\pgfqpoint{1.440718in}{1.110342in}}%
\pgfpathlineto{\pgfqpoint{1.425061in}{1.119032in}}%
\pgfpathlineto{\pgfqpoint{1.421678in}{1.121203in}}%
\pgfpathlineto{\pgfqpoint{1.409405in}{1.128529in}}%
\pgfpathlineto{\pgfqpoint{1.400098in}{1.134814in}}%
\pgfpathlineto{\pgfqpoint{1.393748in}{1.138902in}}%
\pgfpathlineto{\pgfqpoint{1.380379in}{1.148425in}}%
\pgfpathlineto{\pgfqpoint{1.378092in}{1.150013in}}%
\pgfpathlineto{\pgfqpoint{1.362435in}{1.161795in}}%
\pgfpathlineto{\pgfqpoint{1.362134in}{1.162036in}}%
\pgfpathlineto{\pgfqpoint{1.346779in}{1.174245in}}%
\pgfpathlineto{\pgfqpoint{1.345095in}{1.175647in}}%
\pgfpathlineto{\pgfqpoint{1.331122in}{1.187437in}}%
\pgfpathlineto{\pgfqpoint{1.329027in}{1.189259in}}%
\pgfpathlineto{\pgfqpoint{1.315466in}{1.201406in}}%
\pgfpathlineto{\pgfqpoint{1.313852in}{1.202870in}}%
\pgfpathlineto{\pgfqpoint{1.299809in}{1.216219in}}%
\pgfpathlineto{\pgfqpoint{1.299532in}{1.216481in}}%
\pgfpathlineto{\pgfqpoint{1.285979in}{1.230092in}}%
\pgfpathlineto{\pgfqpoint{1.284152in}{1.232081in}}%
\pgfpathlineto{\pgfqpoint{1.273198in}{1.243703in}}%
\pgfpathlineto{\pgfqpoint{1.268496in}{1.249223in}}%
\pgfpathlineto{\pgfqpoint{1.261266in}{1.257314in}}%
\pgfpathlineto{\pgfqpoint{1.252839in}{1.267984in}}%
\pgfpathlineto{\pgfqpoint{1.250342in}{1.270925in}}%
\pgfpathlineto{\pgfqpoint{1.240345in}{1.284536in}}%
\pgfpathlineto{\pgfqpoint{1.237183in}{1.289708in}}%
\pgfpathlineto{\pgfqpoint{1.231452in}{1.298148in}}%
\pgfpathlineto{\pgfqpoint{1.224125in}{1.311759in}}%
\pgfpathlineto{\pgfqpoint{1.221526in}{1.318320in}}%
\pgfpathlineto{\pgfqpoint{1.218301in}{1.325370in}}%
\pgfpathlineto{\pgfqpoint{1.214327in}{1.338981in}}%
\pgfpathlineto{\pgfqpoint{1.212624in}{1.352592in}}%
\pgfpathlineto{\pgfqpoint{1.213192in}{1.366203in}}%
\pgfpathlineto{\pgfqpoint{1.216030in}{1.379814in}}%
\pgfpathlineto{\pgfqpoint{1.221142in}{1.393425in}}%
\pgfpathlineto{\pgfqpoint{1.221526in}{1.394139in}}%
\pgfpathlineto{\pgfqpoint{1.227544in}{1.407036in}}%
\pgfpathlineto{\pgfqpoint{1.235852in}{1.420648in}}%
\pgfpathlineto{\pgfqpoint{1.237183in}{1.422434in}}%
\pgfpathlineto{\pgfqpoint{1.245126in}{1.434259in}}%
\pgfpathlineto{\pgfqpoint{1.252839in}{1.443970in}}%
\pgfpathlineto{\pgfqpoint{1.255716in}{1.447870in}}%
\pgfpathlineto{\pgfqpoint{1.267206in}{1.461481in}}%
\pgfpathlineto{\pgfqpoint{1.268496in}{1.462855in}}%
\pgfpathlineto{\pgfqpoint{1.279460in}{1.475092in}}%
\pgfpathlineto{\pgfqpoint{1.284152in}{1.479876in}}%
\pgfpathlineto{\pgfqpoint{1.292597in}{1.488703in}}%
\pgfpathlineto{\pgfqpoint{1.299809in}{1.495716in}}%
\pgfpathlineto{\pgfqpoint{1.306555in}{1.502314in}}%
\pgfpathlineto{\pgfqpoint{1.315466in}{1.510564in}}%
\pgfpathlineto{\pgfqpoint{1.321328in}{1.515925in}}%
\pgfpathlineto{\pgfqpoint{1.331122in}{1.524548in}}%
\pgfpathlineto{\pgfqpoint{1.336958in}{1.529536in}}%
\pgfpathlineto{\pgfqpoint{1.346779in}{1.537757in}}%
\pgfpathlineto{\pgfqpoint{1.353525in}{1.543148in}}%
\pgfpathlineto{\pgfqpoint{1.362435in}{1.550243in}}%
\pgfpathlineto{\pgfqpoint{1.371153in}{1.556759in}}%
\pgfpathlineto{\pgfqpoint{1.378092in}{1.562024in}}%
\pgfpathlineto{\pgfqpoint{1.390020in}{1.570370in}}%
\pgfpathlineto{\pgfqpoint{1.393748in}{1.573072in}}%
\pgfpathlineto{\pgfqpoint{1.409405in}{1.583349in}}%
\pgfpathlineto{\pgfqpoint{1.410488in}{1.583981in}}%
\pgfpathlineto{\pgfqpoint{1.425061in}{1.593009in}}%
\pgfpathlineto{\pgfqpoint{1.433666in}{1.597592in}}%
\pgfpathlineto{\pgfqpoint{1.440718in}{1.601692in}}%
\pgfpathlineto{\pgfqpoint{1.456375in}{1.609277in}}%
\pgfpathlineto{\pgfqpoint{1.461379in}{1.611203in}}%
\pgfpathlineto{\pgfqpoint{1.472031in}{1.615841in}}%
\pgfpathlineto{\pgfqpoint{1.487688in}{1.620852in}}%
\pgfpathlineto{\pgfqpoint{1.503344in}{1.624039in}}%
\pgfpathlineto{\pgfqpoint{1.512232in}{1.624814in}}%
\pgfpathlineto{\pgfqpoint{1.519001in}{1.625515in}}%
\pgfpathlineto{\pgfqpoint{1.534657in}{1.624974in}}%
\pgfpathlineto{\pgfqpoint{1.535586in}{1.624814in}}%
\pgfpathlineto{\pgfqpoint{1.550314in}{1.622673in}}%
\pgfpathlineto{\pgfqpoint{1.565971in}{1.618575in}}%
\pgfpathlineto{\pgfqpoint{1.581627in}{1.612649in}}%
\pgfpathlineto{\pgfqpoint{1.584576in}{1.611203in}}%
\pgfpathlineto{\pgfqpoint{1.597284in}{1.605685in}}%
\pgfpathlineto{\pgfqpoint{1.612402in}{1.597592in}}%
\pgfpathlineto{\pgfqpoint{1.612940in}{1.597328in}}%
\pgfpathlineto{\pgfqpoint{1.628597in}{1.588331in}}%
\pgfpathlineto{\pgfqpoint{1.635208in}{1.583981in}}%
\pgfpathlineto{\pgfqpoint{1.644253in}{1.578373in}}%
\pgfpathlineto{\pgfqpoint{1.655795in}{1.570370in}}%
\pgfpathlineto{\pgfqpoint{1.659910in}{1.567619in}}%
\pgfpathlineto{\pgfqpoint{1.674743in}{1.556759in}}%
\pgfpathlineto{\pgfqpoint{1.675567in}{1.556165in}}%
\pgfpathlineto{\pgfqpoint{1.691223in}{1.544062in}}%
\pgfpathlineto{\pgfqpoint{1.692344in}{1.543148in}}%
\pgfpathlineto{\pgfqpoint{1.706880in}{1.531243in}}%
\pgfpathlineto{\pgfqpoint{1.708885in}{1.529536in}}%
\pgfpathlineto{\pgfqpoint{1.722536in}{1.517669in}}%
\pgfpathlineto{\pgfqpoint{1.724499in}{1.515925in}}%
\pgfpathlineto{\pgfqpoint{1.738193in}{1.503288in}}%
\pgfpathlineto{\pgfqpoint{1.739244in}{1.502314in}}%
\pgfpathlineto{\pgfqpoint{1.753167in}{1.488703in}}%
\pgfpathlineto{\pgfqpoint{1.753849in}{1.487987in}}%
\pgfpathlineto{\pgfqpoint{1.766342in}{1.475092in}}%
\pgfpathlineto{\pgfqpoint{1.769506in}{1.471515in}}%
\pgfpathlineto{\pgfqpoint{1.778711in}{1.461481in}}%
\pgfpathlineto{\pgfqpoint{1.785162in}{1.453617in}}%
\pgfpathlineto{\pgfqpoint{1.790167in}{1.447870in}}%
\pgfpathlineto{\pgfqpoint{1.800516in}{1.434259in}}%
\pgfpathlineto{\pgfqpoint{1.800819in}{1.433790in}}%
\pgfpathlineto{\pgfqpoint{1.810128in}{1.420648in}}%
\pgfpathlineto{\pgfqpoint{1.816476in}{1.409600in}}%
\pgfpathlineto{\pgfqpoint{1.818139in}{1.407036in}}%
\pgfpathlineto{\pgfqpoint{1.824955in}{1.393425in}}%
\pgfpathlineto{\pgfqpoint{1.829670in}{1.379814in}}%
\pgfpathlineto{\pgfqpoint{1.832132in}{1.367010in}}%
\pgfpathlineto{\pgfqpoint{1.832316in}{1.366203in}}%
\pgfpathlineto{\pgfqpoint{1.832938in}{1.352592in}}%
\pgfpathlineto{\pgfqpoint{1.832132in}{1.346708in}}%
\pgfpathlineto{\pgfqpoint{1.831240in}{1.338981in}}%
\pgfpathlineto{\pgfqpoint{1.827575in}{1.325370in}}%
\pgfpathlineto{\pgfqpoint{1.821810in}{1.311759in}}%
\pgfpathlineto{\pgfqpoint{1.816476in}{1.302498in}}%
\pgfpathlineto{\pgfqpoint{1.814260in}{1.298148in}}%
\pgfpathlineto{\pgfqpoint{1.805535in}{1.284536in}}%
\pgfpathlineto{\pgfqpoint{1.800819in}{1.278406in}}%
\pgfpathlineto{\pgfqpoint{1.795548in}{1.270925in}}%
\pgfpathlineto{\pgfqpoint{1.785162in}{1.258256in}}%
\pgfpathlineto{\pgfqpoint{1.784436in}{1.257314in}}%
\pgfpathlineto{\pgfqpoint{1.772615in}{1.243703in}}%
\pgfpathlineto{\pgfqpoint{1.769506in}{1.240462in}}%
\pgfpathlineto{\pgfqpoint{1.759906in}{1.230092in}}%
\pgfpathlineto{\pgfqpoint{1.753849in}{1.224059in}}%
\pgfpathlineto{\pgfqpoint{1.746355in}{1.216481in}}%
\pgfpathlineto{\pgfqpoint{1.738193in}{1.208734in}}%
\pgfpathlineto{\pgfqpoint{1.731992in}{1.202870in}}%
\pgfpathlineto{\pgfqpoint{1.722536in}{1.194332in}}%
\pgfpathlineto{\pgfqpoint{1.716798in}{1.189259in}}%
\pgfpathlineto{\pgfqpoint{1.706880in}{1.180744in}}%
\pgfpathlineto{\pgfqpoint{1.700712in}{1.175647in}}%
\pgfpathlineto{\pgfqpoint{1.691223in}{1.167901in}}%
\pgfpathlineto{\pgfqpoint{1.683634in}{1.162036in}}%
\pgfpathlineto{\pgfqpoint{1.675567in}{1.155767in}}%
\pgfpathlineto{\pgfqpoint{1.665412in}{1.148425in}}%
\pgfpathlineto{\pgfqpoint{1.659910in}{1.144346in}}%
\pgfpathlineto{\pgfqpoint{1.645834in}{1.134814in}}%
\pgfpathlineto{\pgfqpoint{1.644253in}{1.133693in}}%
\pgfpathlineto{\pgfqpoint{1.628597in}{1.123704in}}%
\pgfpathlineto{\pgfqpoint{1.624111in}{1.121203in}}%
\pgfpathlineto{\pgfqpoint{1.612940in}{1.114498in}}%
\pgfpathlineto{\pgfqpoint{1.599339in}{1.107592in}}%
\pgfpathlineto{\pgfqpoint{1.597284in}{1.106435in}}%
\pgfpathlineto{\pgfqpoint{1.581627in}{1.099212in}}%
\pgfpathlineto{\pgfqpoint{1.566792in}{1.093981in}}%
\pgfpathlineto{\pgfqpoint{1.565971in}{1.093647in}}%
\pgfpathlineto{\pgfqpoint{1.550314in}{1.089203in}}%
\pgfpathlineto{\pgfqpoint{1.534657in}{1.086735in}}%
\pgfpathlineto{\pgfqpoint{1.519001in}{1.086242in}}%
\pgfpathlineto{\pgfqpoint{1.503344in}{1.087722in}}%
\pgfpathlineto{\pgfqpoint{1.487688in}{1.091177in}}%
\pgfpathlineto{\pgfqpoint{1.479579in}{1.093981in}}%
\pgfpathclose%
\pgfusepath{fill}%
\end{pgfscope}%
\begin{pgfscope}%
\pgfpathrectangle{\pgfqpoint{0.360415in}{0.345370in}}{\pgfqpoint{1.550000in}{1.347500in}}%
\pgfusepath{clip}%
\pgfsetbuttcap%
\pgfsetroundjoin%
\definecolor{currentfill}{rgb}{0.252220,0.059415,0.453248}%
\pgfsetfillcolor{currentfill}%
\pgfsetlinewidth{0.000000pt}%
\definecolor{currentstroke}{rgb}{0.000000,0.000000,0.000000}%
\pgfsetstrokecolor{currentstroke}%
\pgfsetdash{}{0pt}%
\pgfpathmoveto{\pgfqpoint{0.516981in}{0.345370in}}%
\pgfpathlineto{\pgfqpoint{0.532637in}{0.345370in}}%
\pgfpathlineto{\pgfqpoint{0.548294in}{0.345370in}}%
\pgfpathlineto{\pgfqpoint{0.563950in}{0.345370in}}%
\pgfpathlineto{\pgfqpoint{0.579607in}{0.345370in}}%
\pgfpathlineto{\pgfqpoint{0.595263in}{0.345370in}}%
\pgfpathlineto{\pgfqpoint{0.610920in}{0.345370in}}%
\pgfpathlineto{\pgfqpoint{0.615330in}{0.345370in}}%
\pgfpathlineto{\pgfqpoint{0.613715in}{0.358981in}}%
\pgfpathlineto{\pgfqpoint{0.610920in}{0.366929in}}%
\pgfpathlineto{\pgfqpoint{0.609062in}{0.372592in}}%
\pgfpathlineto{\pgfqpoint{0.601798in}{0.386203in}}%
\pgfpathlineto{\pgfqpoint{0.595263in}{0.395281in}}%
\pgfpathlineto{\pgfqpoint{0.592186in}{0.399814in}}%
\pgfpathlineto{\pgfqpoint{0.580830in}{0.413425in}}%
\pgfpathlineto{\pgfqpoint{0.579607in}{0.414684in}}%
\pgfpathlineto{\pgfqpoint{0.568161in}{0.427036in}}%
\pgfpathlineto{\pgfqpoint{0.563950in}{0.431121in}}%
\pgfpathlineto{\pgfqpoint{0.554454in}{0.440648in}}%
\pgfpathlineto{\pgfqpoint{0.548294in}{0.446378in}}%
\pgfpathlineto{\pgfqpoint{0.539986in}{0.454259in}}%
\pgfpathlineto{\pgfqpoint{0.532637in}{0.460885in}}%
\pgfpathlineto{\pgfqpoint{0.524927in}{0.467870in}}%
\pgfpathlineto{\pgfqpoint{0.516981in}{0.474858in}}%
\pgfpathlineto{\pgfqpoint{0.509362in}{0.481481in}}%
\pgfpathlineto{\pgfqpoint{0.501324in}{0.488389in}}%
\pgfpathlineto{\pgfqpoint{0.493289in}{0.495092in}}%
\pgfpathlineto{\pgfqpoint{0.485668in}{0.501480in}}%
\pgfpathlineto{\pgfqpoint{0.476602in}{0.508703in}}%
\pgfpathlineto{\pgfqpoint{0.470011in}{0.514059in}}%
\pgfpathlineto{\pgfqpoint{0.459052in}{0.522314in}}%
\pgfpathlineto{\pgfqpoint{0.454354in}{0.525975in}}%
\pgfpathlineto{\pgfqpoint{0.440146in}{0.535925in}}%
\pgfpathlineto{\pgfqpoint{0.438698in}{0.536989in}}%
\pgfpathlineto{\pgfqpoint{0.423041in}{0.546861in}}%
\pgfpathlineto{\pgfqpoint{0.417826in}{0.549536in}}%
\pgfpathlineto{\pgfqpoint{0.407385in}{0.555217in}}%
\pgfpathlineto{\pgfqpoint{0.391728in}{0.561532in}}%
\pgfpathlineto{\pgfqpoint{0.385214in}{0.563148in}}%
\pgfpathlineto{\pgfqpoint{0.376072in}{0.565578in}}%
\pgfpathlineto{\pgfqpoint{0.360415in}{0.566981in}}%
\pgfpathlineto{\pgfqpoint{0.360415in}{0.563148in}}%
\pgfpathlineto{\pgfqpoint{0.360415in}{0.549536in}}%
\pgfpathlineto{\pgfqpoint{0.360415in}{0.535925in}}%
\pgfpathlineto{\pgfqpoint{0.360415in}{0.522314in}}%
\pgfpathlineto{\pgfqpoint{0.360415in}{0.508703in}}%
\pgfpathlineto{\pgfqpoint{0.360415in}{0.495092in}}%
\pgfpathlineto{\pgfqpoint{0.360415in}{0.481481in}}%
\pgfpathlineto{\pgfqpoint{0.360415in}{0.475295in}}%
\pgfpathlineto{\pgfqpoint{0.376072in}{0.474205in}}%
\pgfpathlineto{\pgfqpoint{0.391728in}{0.470984in}}%
\pgfpathlineto{\pgfqpoint{0.401032in}{0.467870in}}%
\pgfpathlineto{\pgfqpoint{0.407385in}{0.465640in}}%
\pgfpathlineto{\pgfqpoint{0.423041in}{0.458214in}}%
\pgfpathlineto{\pgfqpoint{0.429826in}{0.454259in}}%
\pgfpathlineto{\pgfqpoint{0.438698in}{0.448762in}}%
\pgfpathlineto{\pgfqpoint{0.450003in}{0.440648in}}%
\pgfpathlineto{\pgfqpoint{0.454354in}{0.437257in}}%
\pgfpathlineto{\pgfqpoint{0.466111in}{0.427036in}}%
\pgfpathlineto{\pgfqpoint{0.470011in}{0.423254in}}%
\pgfpathlineto{\pgfqpoint{0.479345in}{0.413425in}}%
\pgfpathlineto{\pgfqpoint{0.485668in}{0.405712in}}%
\pgfpathlineto{\pgfqpoint{0.490218in}{0.399814in}}%
\pgfpathlineto{\pgfqpoint{0.498759in}{0.386203in}}%
\pgfpathlineto{\pgfqpoint{0.501324in}{0.380680in}}%
\pgfpathlineto{\pgfqpoint{0.504907in}{0.372592in}}%
\pgfpathlineto{\pgfqpoint{0.508612in}{0.358981in}}%
\pgfpathlineto{\pgfqpoint{0.509865in}{0.345370in}}%
\pgfpathlineto{\pgfqpoint{0.516981in}{0.345370in}}%
\pgfpathclose%
\pgfpathmoveto{\pgfqpoint{0.892738in}{0.345370in}}%
\pgfpathlineto{\pgfqpoint{0.908395in}{0.345370in}}%
\pgfpathlineto{\pgfqpoint{0.924051in}{0.345370in}}%
\pgfpathlineto{\pgfqpoint{0.939708in}{0.345370in}}%
\pgfpathlineto{\pgfqpoint{0.955364in}{0.345370in}}%
\pgfpathlineto{\pgfqpoint{0.971021in}{0.345370in}}%
\pgfpathlineto{\pgfqpoint{0.985887in}{0.345370in}}%
\pgfpathlineto{\pgfqpoint{0.986678in}{0.354148in}}%
\pgfpathlineto{\pgfqpoint{0.987134in}{0.358981in}}%
\pgfpathlineto{\pgfqpoint{0.990940in}{0.372592in}}%
\pgfpathlineto{\pgfqpoint{0.997106in}{0.386203in}}%
\pgfpathlineto{\pgfqpoint{1.002334in}{0.394756in}}%
\pgfpathlineto{\pgfqpoint{1.005597in}{0.399814in}}%
\pgfpathlineto{\pgfqpoint{1.016377in}{0.413425in}}%
\pgfpathlineto{\pgfqpoint{1.017991in}{0.415175in}}%
\pgfpathlineto{\pgfqpoint{1.029712in}{0.427036in}}%
\pgfpathlineto{\pgfqpoint{1.033647in}{0.430609in}}%
\pgfpathlineto{\pgfqpoint{1.045789in}{0.440648in}}%
\pgfpathlineto{\pgfqpoint{1.049304in}{0.443328in}}%
\pgfpathlineto{\pgfqpoint{1.064960in}{0.453848in}}%
\pgfpathlineto{\pgfqpoint{1.065691in}{0.454259in}}%
\pgfpathlineto{\pgfqpoint{1.080617in}{0.462147in}}%
\pgfpathlineto{\pgfqpoint{1.094399in}{0.467870in}}%
\pgfpathlineto{\pgfqpoint{1.096274in}{0.468613in}}%
\pgfpathlineto{\pgfqpoint{1.111930in}{0.472855in}}%
\pgfpathlineto{\pgfqpoint{1.127587in}{0.475022in}}%
\pgfpathlineto{\pgfqpoint{1.143243in}{0.475022in}}%
\pgfpathlineto{\pgfqpoint{1.158900in}{0.472855in}}%
\pgfpathlineto{\pgfqpoint{1.174556in}{0.468613in}}%
\pgfpathlineto{\pgfqpoint{1.176431in}{0.467870in}}%
\pgfpathlineto{\pgfqpoint{1.190213in}{0.462147in}}%
\pgfpathlineto{\pgfqpoint{1.205139in}{0.454259in}}%
\pgfpathlineto{\pgfqpoint{1.205870in}{0.453848in}}%
\pgfpathlineto{\pgfqpoint{1.221526in}{0.443328in}}%
\pgfpathlineto{\pgfqpoint{1.225041in}{0.440648in}}%
\pgfpathlineto{\pgfqpoint{1.237183in}{0.430609in}}%
\pgfpathlineto{\pgfqpoint{1.241118in}{0.427036in}}%
\pgfpathlineto{\pgfqpoint{1.252839in}{0.415175in}}%
\pgfpathlineto{\pgfqpoint{1.254453in}{0.413425in}}%
\pgfpathlineto{\pgfqpoint{1.265233in}{0.399814in}}%
\pgfpathlineto{\pgfqpoint{1.268496in}{0.394756in}}%
\pgfpathlineto{\pgfqpoint{1.273724in}{0.386203in}}%
\pgfpathlineto{\pgfqpoint{1.279890in}{0.372592in}}%
\pgfpathlineto{\pgfqpoint{1.283696in}{0.358981in}}%
\pgfpathlineto{\pgfqpoint{1.284152in}{0.354148in}}%
\pgfpathlineto{\pgfqpoint{1.284943in}{0.345370in}}%
\pgfpathlineto{\pgfqpoint{1.299809in}{0.345370in}}%
\pgfpathlineto{\pgfqpoint{1.315466in}{0.345370in}}%
\pgfpathlineto{\pgfqpoint{1.331122in}{0.345370in}}%
\pgfpathlineto{\pgfqpoint{1.346779in}{0.345370in}}%
\pgfpathlineto{\pgfqpoint{1.362435in}{0.345370in}}%
\pgfpathlineto{\pgfqpoint{1.378092in}{0.345370in}}%
\pgfpathlineto{\pgfqpoint{1.390305in}{0.345370in}}%
\pgfpathlineto{\pgfqpoint{1.388745in}{0.358981in}}%
\pgfpathlineto{\pgfqpoint{1.384130in}{0.372592in}}%
\pgfpathlineto{\pgfqpoint{1.378092in}{0.383562in}}%
\pgfpathlineto{\pgfqpoint{1.376728in}{0.386203in}}%
\pgfpathlineto{\pgfqpoint{1.367210in}{0.399814in}}%
\pgfpathlineto{\pgfqpoint{1.362435in}{0.405366in}}%
\pgfpathlineto{\pgfqpoint{1.355863in}{0.413425in}}%
\pgfpathlineto{\pgfqpoint{1.346779in}{0.423042in}}%
\pgfpathlineto{\pgfqpoint{1.343152in}{0.427036in}}%
\pgfpathlineto{\pgfqpoint{1.331122in}{0.438948in}}%
\pgfpathlineto{\pgfqpoint{1.329450in}{0.440648in}}%
\pgfpathlineto{\pgfqpoint{1.315466in}{0.453825in}}%
\pgfpathlineto{\pgfqpoint{1.315010in}{0.454259in}}%
\pgfpathlineto{\pgfqpoint{1.299986in}{0.467870in}}%
\pgfpathlineto{\pgfqpoint{1.299809in}{0.468025in}}%
\pgfpathlineto{\pgfqpoint{1.284452in}{0.481481in}}%
\pgfpathlineto{\pgfqpoint{1.284152in}{0.481740in}}%
\pgfpathlineto{\pgfqpoint{1.268496in}{0.495031in}}%
\pgfpathlineto{\pgfqpoint{1.268421in}{0.495092in}}%
\pgfpathlineto{\pgfqpoint{1.252839in}{0.507848in}}%
\pgfpathlineto{\pgfqpoint{1.251728in}{0.508703in}}%
\pgfpathlineto{\pgfqpoint{1.237183in}{0.520102in}}%
\pgfpathlineto{\pgfqpoint{1.234095in}{0.522314in}}%
\pgfpathlineto{\pgfqpoint{1.221526in}{0.531617in}}%
\pgfpathlineto{\pgfqpoint{1.214930in}{0.535925in}}%
\pgfpathlineto{\pgfqpoint{1.205870in}{0.542123in}}%
\pgfpathlineto{\pgfqpoint{1.192985in}{0.549536in}}%
\pgfpathlineto{\pgfqpoint{1.190213in}{0.551230in}}%
\pgfpathlineto{\pgfqpoint{1.174556in}{0.558662in}}%
\pgfpathlineto{\pgfqpoint{1.160862in}{0.563148in}}%
\pgfpathlineto{\pgfqpoint{1.158900in}{0.563838in}}%
\pgfpathlineto{\pgfqpoint{1.143243in}{0.566629in}}%
\pgfpathlineto{\pgfqpoint{1.127587in}{0.566629in}}%
\pgfpathlineto{\pgfqpoint{1.111930in}{0.563838in}}%
\pgfpathlineto{\pgfqpoint{1.109968in}{0.563148in}}%
\pgfpathlineto{\pgfqpoint{1.096274in}{0.558662in}}%
\pgfpathlineto{\pgfqpoint{1.080617in}{0.551230in}}%
\pgfpathlineto{\pgfqpoint{1.077845in}{0.549536in}}%
\pgfpathlineto{\pgfqpoint{1.064960in}{0.542123in}}%
\pgfpathlineto{\pgfqpoint{1.055900in}{0.535925in}}%
\pgfpathlineto{\pgfqpoint{1.049304in}{0.531617in}}%
\pgfpathlineto{\pgfqpoint{1.036735in}{0.522314in}}%
\pgfpathlineto{\pgfqpoint{1.033647in}{0.520102in}}%
\pgfpathlineto{\pgfqpoint{1.019102in}{0.508703in}}%
\pgfpathlineto{\pgfqpoint{1.017991in}{0.507848in}}%
\pgfpathlineto{\pgfqpoint{1.002409in}{0.495092in}}%
\pgfpathlineto{\pgfqpoint{1.002334in}{0.495031in}}%
\pgfpathlineto{\pgfqpoint{0.986678in}{0.481740in}}%
\pgfpathlineto{\pgfqpoint{0.986378in}{0.481481in}}%
\pgfpathlineto{\pgfqpoint{0.971021in}{0.468025in}}%
\pgfpathlineto{\pgfqpoint{0.970844in}{0.467870in}}%
\pgfpathlineto{\pgfqpoint{0.955820in}{0.454259in}}%
\pgfpathlineto{\pgfqpoint{0.955364in}{0.453825in}}%
\pgfpathlineto{\pgfqpoint{0.941380in}{0.440648in}}%
\pgfpathlineto{\pgfqpoint{0.939708in}{0.438948in}}%
\pgfpathlineto{\pgfqpoint{0.927678in}{0.427036in}}%
\pgfpathlineto{\pgfqpoint{0.924051in}{0.423042in}}%
\pgfpathlineto{\pgfqpoint{0.914967in}{0.413425in}}%
\pgfpathlineto{\pgfqpoint{0.908395in}{0.405366in}}%
\pgfpathlineto{\pgfqpoint{0.903620in}{0.399814in}}%
\pgfpathlineto{\pgfqpoint{0.894102in}{0.386203in}}%
\pgfpathlineto{\pgfqpoint{0.892738in}{0.383562in}}%
\pgfpathlineto{\pgfqpoint{0.886700in}{0.372592in}}%
\pgfpathlineto{\pgfqpoint{0.882085in}{0.358981in}}%
\pgfpathlineto{\pgfqpoint{0.880525in}{0.345370in}}%
\pgfpathlineto{\pgfqpoint{0.892738in}{0.345370in}}%
\pgfpathclose%
\pgfpathmoveto{\pgfqpoint{1.659910in}{0.345370in}}%
\pgfpathlineto{\pgfqpoint{1.675567in}{0.345370in}}%
\pgfpathlineto{\pgfqpoint{1.691223in}{0.345370in}}%
\pgfpathlineto{\pgfqpoint{1.706880in}{0.345370in}}%
\pgfpathlineto{\pgfqpoint{1.722536in}{0.345370in}}%
\pgfpathlineto{\pgfqpoint{1.738193in}{0.345370in}}%
\pgfpathlineto{\pgfqpoint{1.753849in}{0.345370in}}%
\pgfpathlineto{\pgfqpoint{1.760965in}{0.345370in}}%
\pgfpathlineto{\pgfqpoint{1.762218in}{0.358981in}}%
\pgfpathlineto{\pgfqpoint{1.765923in}{0.372592in}}%
\pgfpathlineto{\pgfqpoint{1.769506in}{0.380680in}}%
\pgfpathlineto{\pgfqpoint{1.772071in}{0.386203in}}%
\pgfpathlineto{\pgfqpoint{1.780612in}{0.399814in}}%
\pgfpathlineto{\pgfqpoint{1.785162in}{0.405712in}}%
\pgfpathlineto{\pgfqpoint{1.791485in}{0.413425in}}%
\pgfpathlineto{\pgfqpoint{1.800819in}{0.423254in}}%
\pgfpathlineto{\pgfqpoint{1.804719in}{0.427036in}}%
\pgfpathlineto{\pgfqpoint{1.816476in}{0.437257in}}%
\pgfpathlineto{\pgfqpoint{1.820827in}{0.440648in}}%
\pgfpathlineto{\pgfqpoint{1.832132in}{0.448762in}}%
\pgfpathlineto{\pgfqpoint{1.841004in}{0.454259in}}%
\pgfpathlineto{\pgfqpoint{1.847789in}{0.458214in}}%
\pgfpathlineto{\pgfqpoint{1.863445in}{0.465640in}}%
\pgfpathlineto{\pgfqpoint{1.869798in}{0.467870in}}%
\pgfpathlineto{\pgfqpoint{1.879102in}{0.470984in}}%
\pgfpathlineto{\pgfqpoint{1.894758in}{0.474205in}}%
\pgfpathlineto{\pgfqpoint{1.910415in}{0.475295in}}%
\pgfpathlineto{\pgfqpoint{1.910415in}{0.481481in}}%
\pgfpathlineto{\pgfqpoint{1.910415in}{0.495092in}}%
\pgfpathlineto{\pgfqpoint{1.910415in}{0.508703in}}%
\pgfpathlineto{\pgfqpoint{1.910415in}{0.522314in}}%
\pgfpathlineto{\pgfqpoint{1.910415in}{0.535925in}}%
\pgfpathlineto{\pgfqpoint{1.910415in}{0.549536in}}%
\pgfpathlineto{\pgfqpoint{1.910415in}{0.563148in}}%
\pgfpathlineto{\pgfqpoint{1.910415in}{0.566981in}}%
\pgfpathlineto{\pgfqpoint{1.894758in}{0.565578in}}%
\pgfpathlineto{\pgfqpoint{1.885616in}{0.563148in}}%
\pgfpathlineto{\pgfqpoint{1.879102in}{0.561532in}}%
\pgfpathlineto{\pgfqpoint{1.863445in}{0.555217in}}%
\pgfpathlineto{\pgfqpoint{1.853004in}{0.549536in}}%
\pgfpathlineto{\pgfqpoint{1.847789in}{0.546861in}}%
\pgfpathlineto{\pgfqpoint{1.832132in}{0.536989in}}%
\pgfpathlineto{\pgfqpoint{1.830684in}{0.535925in}}%
\pgfpathlineto{\pgfqpoint{1.816476in}{0.525975in}}%
\pgfpathlineto{\pgfqpoint{1.811778in}{0.522314in}}%
\pgfpathlineto{\pgfqpoint{1.800819in}{0.514059in}}%
\pgfpathlineto{\pgfqpoint{1.794228in}{0.508703in}}%
\pgfpathlineto{\pgfqpoint{1.785162in}{0.501480in}}%
\pgfpathlineto{\pgfqpoint{1.777541in}{0.495092in}}%
\pgfpathlineto{\pgfqpoint{1.769506in}{0.488389in}}%
\pgfpathlineto{\pgfqpoint{1.761468in}{0.481481in}}%
\pgfpathlineto{\pgfqpoint{1.753849in}{0.474858in}}%
\pgfpathlineto{\pgfqpoint{1.745903in}{0.467870in}}%
\pgfpathlineto{\pgfqpoint{1.738193in}{0.460885in}}%
\pgfpathlineto{\pgfqpoint{1.730844in}{0.454259in}}%
\pgfpathlineto{\pgfqpoint{1.722536in}{0.446378in}}%
\pgfpathlineto{\pgfqpoint{1.716376in}{0.440648in}}%
\pgfpathlineto{\pgfqpoint{1.706880in}{0.431121in}}%
\pgfpathlineto{\pgfqpoint{1.702669in}{0.427036in}}%
\pgfpathlineto{\pgfqpoint{1.691223in}{0.414684in}}%
\pgfpathlineto{\pgfqpoint{1.690000in}{0.413425in}}%
\pgfpathlineto{\pgfqpoint{1.678644in}{0.399814in}}%
\pgfpathlineto{\pgfqpoint{1.675567in}{0.395281in}}%
\pgfpathlineto{\pgfqpoint{1.669032in}{0.386203in}}%
\pgfpathlineto{\pgfqpoint{1.661768in}{0.372592in}}%
\pgfpathlineto{\pgfqpoint{1.659910in}{0.366929in}}%
\pgfpathlineto{\pgfqpoint{1.657115in}{0.358981in}}%
\pgfpathlineto{\pgfqpoint{1.655500in}{0.345370in}}%
\pgfpathlineto{\pgfqpoint{1.659910in}{0.345370in}}%
\pgfpathclose%
\pgfpathmoveto{\pgfqpoint{0.376072in}{0.798886in}}%
\pgfpathlineto{\pgfqpoint{0.391728in}{0.802899in}}%
\pgfpathlineto{\pgfqpoint{0.404347in}{0.808148in}}%
\pgfpathlineto{\pgfqpoint{0.407385in}{0.809333in}}%
\pgfpathlineto{\pgfqpoint{0.423041in}{0.817608in}}%
\pgfpathlineto{\pgfqpoint{0.429427in}{0.821759in}}%
\pgfpathlineto{\pgfqpoint{0.438698in}{0.827473in}}%
\pgfpathlineto{\pgfqpoint{0.449760in}{0.835370in}}%
\pgfpathlineto{\pgfqpoint{0.454354in}{0.838523in}}%
\pgfpathlineto{\pgfqpoint{0.468056in}{0.848981in}}%
\pgfpathlineto{\pgfqpoint{0.470011in}{0.850435in}}%
\pgfpathlineto{\pgfqpoint{0.485168in}{0.862592in}}%
\pgfpathlineto{\pgfqpoint{0.485668in}{0.862988in}}%
\pgfpathlineto{\pgfqpoint{0.501324in}{0.876049in}}%
\pgfpathlineto{\pgfqpoint{0.501503in}{0.876203in}}%
\pgfpathlineto{\pgfqpoint{0.516981in}{0.889554in}}%
\pgfpathlineto{\pgfqpoint{0.517279in}{0.889814in}}%
\pgfpathlineto{\pgfqpoint{0.532567in}{0.903425in}}%
\pgfpathlineto{\pgfqpoint{0.532637in}{0.903490in}}%
\pgfpathlineto{\pgfqpoint{0.547310in}{0.917036in}}%
\pgfpathlineto{\pgfqpoint{0.548294in}{0.918003in}}%
\pgfpathlineto{\pgfqpoint{0.561406in}{0.930648in}}%
\pgfpathlineto{\pgfqpoint{0.563950in}{0.933332in}}%
\pgfpathlineto{\pgfqpoint{0.574651in}{0.944259in}}%
\pgfpathlineto{\pgfqpoint{0.579607in}{0.949993in}}%
\pgfpathlineto{\pgfqpoint{0.586736in}{0.957870in}}%
\pgfpathlineto{\pgfqpoint{0.595263in}{0.969071in}}%
\pgfpathlineto{\pgfqpoint{0.597212in}{0.971481in}}%
\pgfpathlineto{\pgfqpoint{0.605761in}{0.985092in}}%
\pgfpathlineto{\pgfqpoint{0.610920in}{0.996997in}}%
\pgfpathlineto{\pgfqpoint{0.611714in}{0.998703in}}%
\pgfpathlineto{\pgfqpoint{0.614925in}{1.012314in}}%
\pgfpathlineto{\pgfqpoint{0.614925in}{1.025925in}}%
\pgfpathlineto{\pgfqpoint{0.611714in}{1.039536in}}%
\pgfpathlineto{\pgfqpoint{0.610920in}{1.041242in}}%
\pgfpathlineto{\pgfqpoint{0.605761in}{1.053148in}}%
\pgfpathlineto{\pgfqpoint{0.597212in}{1.066759in}}%
\pgfpathlineto{\pgfqpoint{0.595263in}{1.069168in}}%
\pgfpathlineto{\pgfqpoint{0.586736in}{1.080370in}}%
\pgfpathlineto{\pgfqpoint{0.579607in}{1.088246in}}%
\pgfpathlineto{\pgfqpoint{0.574651in}{1.093981in}}%
\pgfpathlineto{\pgfqpoint{0.563950in}{1.104908in}}%
\pgfpathlineto{\pgfqpoint{0.561406in}{1.107592in}}%
\pgfpathlineto{\pgfqpoint{0.548294in}{1.120237in}}%
\pgfpathlineto{\pgfqpoint{0.547310in}{1.121203in}}%
\pgfpathlineto{\pgfqpoint{0.532637in}{1.134749in}}%
\pgfpathlineto{\pgfqpoint{0.532567in}{1.134814in}}%
\pgfpathlineto{\pgfqpoint{0.517279in}{1.148425in}}%
\pgfpathlineto{\pgfqpoint{0.516981in}{1.148686in}}%
\pgfpathlineto{\pgfqpoint{0.501503in}{1.162036in}}%
\pgfpathlineto{\pgfqpoint{0.501324in}{1.162190in}}%
\pgfpathlineto{\pgfqpoint{0.485668in}{1.175252in}}%
\pgfpathlineto{\pgfqpoint{0.485168in}{1.175647in}}%
\pgfpathlineto{\pgfqpoint{0.470011in}{1.187805in}}%
\pgfpathlineto{\pgfqpoint{0.468056in}{1.189259in}}%
\pgfpathlineto{\pgfqpoint{0.454354in}{1.199716in}}%
\pgfpathlineto{\pgfqpoint{0.449760in}{1.202870in}}%
\pgfpathlineto{\pgfqpoint{0.438698in}{1.210767in}}%
\pgfpathlineto{\pgfqpoint{0.429427in}{1.216481in}}%
\pgfpathlineto{\pgfqpoint{0.423041in}{1.220632in}}%
\pgfpathlineto{\pgfqpoint{0.407385in}{1.228906in}}%
\pgfpathlineto{\pgfqpoint{0.404347in}{1.230092in}}%
\pgfpathlineto{\pgfqpoint{0.391728in}{1.235341in}}%
\pgfpathlineto{\pgfqpoint{0.376072in}{1.239353in}}%
\pgfpathlineto{\pgfqpoint{0.360415in}{1.240710in}}%
\pgfpathlineto{\pgfqpoint{0.360415in}{1.230092in}}%
\pgfpathlineto{\pgfqpoint{0.360415in}{1.216481in}}%
\pgfpathlineto{\pgfqpoint{0.360415in}{1.202870in}}%
\pgfpathlineto{\pgfqpoint{0.360415in}{1.189259in}}%
\pgfpathlineto{\pgfqpoint{0.360415in}{1.175647in}}%
\pgfpathlineto{\pgfqpoint{0.360415in}{1.162036in}}%
\pgfpathlineto{\pgfqpoint{0.360415in}{1.149113in}}%
\pgfpathlineto{\pgfqpoint{0.370512in}{1.148425in}}%
\pgfpathlineto{\pgfqpoint{0.376072in}{1.148029in}}%
\pgfpathlineto{\pgfqpoint{0.391728in}{1.144719in}}%
\pgfpathlineto{\pgfqpoint{0.407385in}{1.139360in}}%
\pgfpathlineto{\pgfqpoint{0.417223in}{1.134814in}}%
\pgfpathlineto{\pgfqpoint{0.423041in}{1.131978in}}%
\pgfpathlineto{\pgfqpoint{0.438698in}{1.122606in}}%
\pgfpathlineto{\pgfqpoint{0.440711in}{1.121203in}}%
\pgfpathlineto{\pgfqpoint{0.454354in}{1.111013in}}%
\pgfpathlineto{\pgfqpoint{0.458464in}{1.107592in}}%
\pgfpathlineto{\pgfqpoint{0.470011in}{1.097036in}}%
\pgfpathlineto{\pgfqpoint{0.473094in}{1.093981in}}%
\pgfpathlineto{\pgfqpoint{0.485195in}{1.080370in}}%
\pgfpathlineto{\pgfqpoint{0.485668in}{1.079735in}}%
\pgfpathlineto{\pgfqpoint{0.494741in}{1.066759in}}%
\pgfpathlineto{\pgfqpoint{0.501324in}{1.054777in}}%
\pgfpathlineto{\pgfqpoint{0.502179in}{1.053148in}}%
\pgfpathlineto{\pgfqpoint{0.507058in}{1.039536in}}%
\pgfpathlineto{\pgfqpoint{0.509551in}{1.025925in}}%
\pgfpathlineto{\pgfqpoint{0.509551in}{1.012314in}}%
\pgfpathlineto{\pgfqpoint{0.507058in}{0.998703in}}%
\pgfpathlineto{\pgfqpoint{0.502179in}{0.985092in}}%
\pgfpathlineto{\pgfqpoint{0.501324in}{0.983463in}}%
\pgfpathlineto{\pgfqpoint{0.494741in}{0.971481in}}%
\pgfpathlineto{\pgfqpoint{0.485668in}{0.958504in}}%
\pgfpathlineto{\pgfqpoint{0.485195in}{0.957870in}}%
\pgfpathlineto{\pgfqpoint{0.473094in}{0.944259in}}%
\pgfpathlineto{\pgfqpoint{0.470011in}{0.941203in}}%
\pgfpathlineto{\pgfqpoint{0.458464in}{0.930648in}}%
\pgfpathlineto{\pgfqpoint{0.454354in}{0.927226in}}%
\pgfpathlineto{\pgfqpoint{0.440711in}{0.917036in}}%
\pgfpathlineto{\pgfqpoint{0.438698in}{0.915634in}}%
\pgfpathlineto{\pgfqpoint{0.423041in}{0.906261in}}%
\pgfpathlineto{\pgfqpoint{0.417223in}{0.903425in}}%
\pgfpathlineto{\pgfqpoint{0.407385in}{0.898880in}}%
\pgfpathlineto{\pgfqpoint{0.391728in}{0.893520in}}%
\pgfpathlineto{\pgfqpoint{0.376072in}{0.890211in}}%
\pgfpathlineto{\pgfqpoint{0.370512in}{0.889814in}}%
\pgfpathlineto{\pgfqpoint{0.360415in}{0.889127in}}%
\pgfpathlineto{\pgfqpoint{0.360415in}{0.876203in}}%
\pgfpathlineto{\pgfqpoint{0.360415in}{0.862592in}}%
\pgfpathlineto{\pgfqpoint{0.360415in}{0.848981in}}%
\pgfpathlineto{\pgfqpoint{0.360415in}{0.835370in}}%
\pgfpathlineto{\pgfqpoint{0.360415in}{0.821759in}}%
\pgfpathlineto{\pgfqpoint{0.360415in}{0.808148in}}%
\pgfpathlineto{\pgfqpoint{0.360415in}{0.797530in}}%
\pgfpathlineto{\pgfqpoint{0.376072in}{0.798886in}}%
\pgfpathclose%
\pgfpathmoveto{\pgfqpoint{1.096274in}{0.805852in}}%
\pgfpathlineto{\pgfqpoint{1.111930in}{0.800569in}}%
\pgfpathlineto{\pgfqpoint{1.127587in}{0.797870in}}%
\pgfpathlineto{\pgfqpoint{1.143243in}{0.797870in}}%
\pgfpathlineto{\pgfqpoint{1.158900in}{0.800569in}}%
\pgfpathlineto{\pgfqpoint{1.174556in}{0.805852in}}%
\pgfpathlineto{\pgfqpoint{1.179213in}{0.808148in}}%
\pgfpathlineto{\pgfqpoint{1.190213in}{0.813225in}}%
\pgfpathlineto{\pgfqpoint{1.204696in}{0.821759in}}%
\pgfpathlineto{\pgfqpoint{1.205870in}{0.822415in}}%
\pgfpathlineto{\pgfqpoint{1.221526in}{0.832876in}}%
\pgfpathlineto{\pgfqpoint{1.224814in}{0.835370in}}%
\pgfpathlineto{\pgfqpoint{1.237183in}{0.844377in}}%
\pgfpathlineto{\pgfqpoint{1.242959in}{0.848981in}}%
\pgfpathlineto{\pgfqpoint{1.252839in}{0.856648in}}%
\pgfpathlineto{\pgfqpoint{1.260034in}{0.862592in}}%
\pgfpathlineto{\pgfqpoint{1.268496in}{0.869497in}}%
\pgfpathlineto{\pgfqpoint{1.276393in}{0.876203in}}%
\pgfpathlineto{\pgfqpoint{1.284152in}{0.882813in}}%
\pgfpathlineto{\pgfqpoint{1.292206in}{0.889814in}}%
\pgfpathlineto{\pgfqpoint{1.299809in}{0.896560in}}%
\pgfpathlineto{\pgfqpoint{1.307523in}{0.903425in}}%
\pgfpathlineto{\pgfqpoint{1.315466in}{0.910782in}}%
\pgfpathlineto{\pgfqpoint{1.322302in}{0.917036in}}%
\pgfpathlineto{\pgfqpoint{1.331122in}{0.925626in}}%
\pgfpathlineto{\pgfqpoint{1.336418in}{0.930648in}}%
\pgfpathlineto{\pgfqpoint{1.346779in}{0.941400in}}%
\pgfpathlineto{\pgfqpoint{1.349648in}{0.944259in}}%
\pgfpathlineto{\pgfqpoint{1.361680in}{0.957870in}}%
\pgfpathlineto{\pgfqpoint{1.362435in}{0.958890in}}%
\pgfpathlineto{\pgfqpoint{1.372251in}{0.971481in}}%
\pgfpathlineto{\pgfqpoint{1.378092in}{0.981044in}}%
\pgfpathlineto{\pgfqpoint{1.380733in}{0.985092in}}%
\pgfpathlineto{\pgfqpoint{1.386810in}{0.998703in}}%
\pgfpathlineto{\pgfqpoint{1.389914in}{1.012314in}}%
\pgfpathlineto{\pgfqpoint{1.389914in}{1.025925in}}%
\pgfpathlineto{\pgfqpoint{1.386810in}{1.039536in}}%
\pgfpathlineto{\pgfqpoint{1.380733in}{1.053148in}}%
\pgfpathlineto{\pgfqpoint{1.378092in}{1.057196in}}%
\pgfpathlineto{\pgfqpoint{1.372251in}{1.066759in}}%
\pgfpathlineto{\pgfqpoint{1.362435in}{1.079350in}}%
\pgfpathlineto{\pgfqpoint{1.361680in}{1.080370in}}%
\pgfpathlineto{\pgfqpoint{1.349648in}{1.093981in}}%
\pgfpathlineto{\pgfqpoint{1.346779in}{1.096839in}}%
\pgfpathlineto{\pgfqpoint{1.336418in}{1.107592in}}%
\pgfpathlineto{\pgfqpoint{1.331122in}{1.112613in}}%
\pgfpathlineto{\pgfqpoint{1.322302in}{1.121203in}}%
\pgfpathlineto{\pgfqpoint{1.315466in}{1.127458in}}%
\pgfpathlineto{\pgfqpoint{1.307523in}{1.134814in}}%
\pgfpathlineto{\pgfqpoint{1.299809in}{1.141680in}}%
\pgfpathlineto{\pgfqpoint{1.292206in}{1.148425in}}%
\pgfpathlineto{\pgfqpoint{1.284152in}{1.155427in}}%
\pgfpathlineto{\pgfqpoint{1.276393in}{1.162036in}}%
\pgfpathlineto{\pgfqpoint{1.268496in}{1.168742in}}%
\pgfpathlineto{\pgfqpoint{1.260034in}{1.175647in}}%
\pgfpathlineto{\pgfqpoint{1.252839in}{1.181591in}}%
\pgfpathlineto{\pgfqpoint{1.242959in}{1.189259in}}%
\pgfpathlineto{\pgfqpoint{1.237183in}{1.193862in}}%
\pgfpathlineto{\pgfqpoint{1.224814in}{1.202870in}}%
\pgfpathlineto{\pgfqpoint{1.221526in}{1.205364in}}%
\pgfpathlineto{\pgfqpoint{1.205870in}{1.215824in}}%
\pgfpathlineto{\pgfqpoint{1.204696in}{1.216481in}}%
\pgfpathlineto{\pgfqpoint{1.190213in}{1.225014in}}%
\pgfpathlineto{\pgfqpoint{1.179213in}{1.230092in}}%
\pgfpathlineto{\pgfqpoint{1.174556in}{1.232388in}}%
\pgfpathlineto{\pgfqpoint{1.158900in}{1.237671in}}%
\pgfpathlineto{\pgfqpoint{1.143243in}{1.240370in}}%
\pgfpathlineto{\pgfqpoint{1.127587in}{1.240370in}}%
\pgfpathlineto{\pgfqpoint{1.111930in}{1.237671in}}%
\pgfpathlineto{\pgfqpoint{1.096274in}{1.232388in}}%
\pgfpathlineto{\pgfqpoint{1.091617in}{1.230092in}}%
\pgfpathlineto{\pgfqpoint{1.080617in}{1.225014in}}%
\pgfpathlineto{\pgfqpoint{1.066134in}{1.216481in}}%
\pgfpathlineto{\pgfqpoint{1.064960in}{1.215824in}}%
\pgfpathlineto{\pgfqpoint{1.049304in}{1.205364in}}%
\pgfpathlineto{\pgfqpoint{1.046016in}{1.202870in}}%
\pgfpathlineto{\pgfqpoint{1.033647in}{1.193862in}}%
\pgfpathlineto{\pgfqpoint{1.027871in}{1.189259in}}%
\pgfpathlineto{\pgfqpoint{1.017991in}{1.181591in}}%
\pgfpathlineto{\pgfqpoint{1.010796in}{1.175647in}}%
\pgfpathlineto{\pgfqpoint{1.002334in}{1.168742in}}%
\pgfpathlineto{\pgfqpoint{0.994437in}{1.162036in}}%
\pgfpathlineto{\pgfqpoint{0.986678in}{1.155427in}}%
\pgfpathlineto{\pgfqpoint{0.978624in}{1.148425in}}%
\pgfpathlineto{\pgfqpoint{0.971021in}{1.141680in}}%
\pgfpathlineto{\pgfqpoint{0.963307in}{1.134814in}}%
\pgfpathlineto{\pgfqpoint{0.955364in}{1.127458in}}%
\pgfpathlineto{\pgfqpoint{0.948528in}{1.121203in}}%
\pgfpathlineto{\pgfqpoint{0.939708in}{1.112613in}}%
\pgfpathlineto{\pgfqpoint{0.934412in}{1.107592in}}%
\pgfpathlineto{\pgfqpoint{0.924051in}{1.096839in}}%
\pgfpathlineto{\pgfqpoint{0.921182in}{1.093981in}}%
\pgfpathlineto{\pgfqpoint{0.909150in}{1.080370in}}%
\pgfpathlineto{\pgfqpoint{0.908395in}{1.079350in}}%
\pgfpathlineto{\pgfqpoint{0.898579in}{1.066759in}}%
\pgfpathlineto{\pgfqpoint{0.892738in}{1.057196in}}%
\pgfpathlineto{\pgfqpoint{0.890097in}{1.053148in}}%
\pgfpathlineto{\pgfqpoint{0.884020in}{1.039536in}}%
\pgfpathlineto{\pgfqpoint{0.880916in}{1.025925in}}%
\pgfpathlineto{\pgfqpoint{0.880916in}{1.012314in}}%
\pgfpathlineto{\pgfqpoint{0.884020in}{0.998703in}}%
\pgfpathlineto{\pgfqpoint{0.890097in}{0.985092in}}%
\pgfpathlineto{\pgfqpoint{0.892738in}{0.981044in}}%
\pgfpathlineto{\pgfqpoint{0.898579in}{0.971481in}}%
\pgfpathlineto{\pgfqpoint{0.908395in}{0.958890in}}%
\pgfpathlineto{\pgfqpoint{0.909150in}{0.957870in}}%
\pgfpathlineto{\pgfqpoint{0.921182in}{0.944259in}}%
\pgfpathlineto{\pgfqpoint{0.924051in}{0.941400in}}%
\pgfpathlineto{\pgfqpoint{0.934412in}{0.930648in}}%
\pgfpathlineto{\pgfqpoint{0.939708in}{0.925626in}}%
\pgfpathlineto{\pgfqpoint{0.948528in}{0.917036in}}%
\pgfpathlineto{\pgfqpoint{0.955364in}{0.910782in}}%
\pgfpathlineto{\pgfqpoint{0.963307in}{0.903425in}}%
\pgfpathlineto{\pgfqpoint{0.971021in}{0.896560in}}%
\pgfpathlineto{\pgfqpoint{0.978624in}{0.889814in}}%
\pgfpathlineto{\pgfqpoint{0.986678in}{0.882813in}}%
\pgfpathlineto{\pgfqpoint{0.994437in}{0.876203in}}%
\pgfpathlineto{\pgfqpoint{1.002334in}{0.869497in}}%
\pgfpathlineto{\pgfqpoint{1.010796in}{0.862592in}}%
\pgfpathlineto{\pgfqpoint{1.017991in}{0.856648in}}%
\pgfpathlineto{\pgfqpoint{1.027871in}{0.848981in}}%
\pgfpathlineto{\pgfqpoint{1.033647in}{0.844377in}}%
\pgfpathlineto{\pgfqpoint{1.046016in}{0.835370in}}%
\pgfpathlineto{\pgfqpoint{1.049304in}{0.832876in}}%
\pgfpathlineto{\pgfqpoint{1.064960in}{0.822415in}}%
\pgfpathlineto{\pgfqpoint{1.066134in}{0.821759in}}%
\pgfpathlineto{\pgfqpoint{1.080617in}{0.813225in}}%
\pgfpathlineto{\pgfqpoint{1.091617in}{0.808148in}}%
\pgfpathlineto{\pgfqpoint{1.096274in}{0.805852in}}%
\pgfpathclose%
\pgfpathmoveto{\pgfqpoint{1.124491in}{0.889814in}}%
\pgfpathlineto{\pgfqpoint{1.111930in}{0.891598in}}%
\pgfpathlineto{\pgfqpoint{1.096274in}{0.895956in}}%
\pgfpathlineto{\pgfqpoint{1.080617in}{0.902263in}}%
\pgfpathlineto{\pgfqpoint{1.078379in}{0.903425in}}%
\pgfpathlineto{\pgfqpoint{1.064960in}{0.910760in}}%
\pgfpathlineto{\pgfqpoint{1.055288in}{0.917036in}}%
\pgfpathlineto{\pgfqpoint{1.049304in}{0.921205in}}%
\pgfpathlineto{\pgfqpoint{1.037361in}{0.930648in}}%
\pgfpathlineto{\pgfqpoint{1.033647in}{0.933876in}}%
\pgfpathlineto{\pgfqpoint{1.022785in}{0.944259in}}%
\pgfpathlineto{\pgfqpoint{1.017991in}{0.949461in}}%
\pgfpathlineto{\pgfqpoint{1.010771in}{0.957870in}}%
\pgfpathlineto{\pgfqpoint{1.002334in}{0.969535in}}%
\pgfpathlineto{\pgfqpoint{1.000998in}{0.971481in}}%
\pgfpathlineto{\pgfqpoint{0.993742in}{0.985092in}}%
\pgfpathlineto{\pgfqpoint{0.988730in}{0.998703in}}%
\pgfpathlineto{\pgfqpoint{0.986678in}{1.009623in}}%
\pgfpathlineto{\pgfqpoint{0.986194in}{1.012314in}}%
\pgfpathlineto{\pgfqpoint{0.986194in}{1.025925in}}%
\pgfpathlineto{\pgfqpoint{0.986678in}{1.028617in}}%
\pgfpathlineto{\pgfqpoint{0.988730in}{1.039536in}}%
\pgfpathlineto{\pgfqpoint{0.993742in}{1.053148in}}%
\pgfpathlineto{\pgfqpoint{1.000998in}{1.066759in}}%
\pgfpathlineto{\pgfqpoint{1.002334in}{1.068705in}}%
\pgfpathlineto{\pgfqpoint{1.010771in}{1.080370in}}%
\pgfpathlineto{\pgfqpoint{1.017991in}{1.088779in}}%
\pgfpathlineto{\pgfqpoint{1.022785in}{1.093981in}}%
\pgfpathlineto{\pgfqpoint{1.033647in}{1.104363in}}%
\pgfpathlineto{\pgfqpoint{1.037361in}{1.107592in}}%
\pgfpathlineto{\pgfqpoint{1.049304in}{1.117035in}}%
\pgfpathlineto{\pgfqpoint{1.055288in}{1.121203in}}%
\pgfpathlineto{\pgfqpoint{1.064960in}{1.127480in}}%
\pgfpathlineto{\pgfqpoint{1.078379in}{1.134814in}}%
\pgfpathlineto{\pgfqpoint{1.080617in}{1.135976in}}%
\pgfpathlineto{\pgfqpoint{1.096274in}{1.142284in}}%
\pgfpathlineto{\pgfqpoint{1.111930in}{1.146641in}}%
\pgfpathlineto{\pgfqpoint{1.124491in}{1.148425in}}%
\pgfpathlineto{\pgfqpoint{1.127587in}{1.148846in}}%
\pgfpathlineto{\pgfqpoint{1.143243in}{1.148846in}}%
\pgfpathlineto{\pgfqpoint{1.146339in}{1.148425in}}%
\pgfpathlineto{\pgfqpoint{1.158900in}{1.146641in}}%
\pgfpathlineto{\pgfqpoint{1.174556in}{1.142284in}}%
\pgfpathlineto{\pgfqpoint{1.190213in}{1.135976in}}%
\pgfpathlineto{\pgfqpoint{1.192451in}{1.134814in}}%
\pgfpathlineto{\pgfqpoint{1.205870in}{1.127480in}}%
\pgfpathlineto{\pgfqpoint{1.215542in}{1.121203in}}%
\pgfpathlineto{\pgfqpoint{1.221526in}{1.117035in}}%
\pgfpathlineto{\pgfqpoint{1.233469in}{1.107592in}}%
\pgfpathlineto{\pgfqpoint{1.237183in}{1.104363in}}%
\pgfpathlineto{\pgfqpoint{1.248045in}{1.093981in}}%
\pgfpathlineto{\pgfqpoint{1.252839in}{1.088779in}}%
\pgfpathlineto{\pgfqpoint{1.260059in}{1.080370in}}%
\pgfpathlineto{\pgfqpoint{1.268496in}{1.068705in}}%
\pgfpathlineto{\pgfqpoint{1.269832in}{1.066759in}}%
\pgfpathlineto{\pgfqpoint{1.277088in}{1.053148in}}%
\pgfpathlineto{\pgfqpoint{1.282100in}{1.039536in}}%
\pgfpathlineto{\pgfqpoint{1.284152in}{1.028617in}}%
\pgfpathlineto{\pgfqpoint{1.284636in}{1.025925in}}%
\pgfpathlineto{\pgfqpoint{1.284636in}{1.012314in}}%
\pgfpathlineto{\pgfqpoint{1.284152in}{1.009623in}}%
\pgfpathlineto{\pgfqpoint{1.282100in}{0.998703in}}%
\pgfpathlineto{\pgfqpoint{1.277088in}{0.985092in}}%
\pgfpathlineto{\pgfqpoint{1.269832in}{0.971481in}}%
\pgfpathlineto{\pgfqpoint{1.268496in}{0.969535in}}%
\pgfpathlineto{\pgfqpoint{1.260059in}{0.957870in}}%
\pgfpathlineto{\pgfqpoint{1.252839in}{0.949461in}}%
\pgfpathlineto{\pgfqpoint{1.248045in}{0.944259in}}%
\pgfpathlineto{\pgfqpoint{1.237183in}{0.933876in}}%
\pgfpathlineto{\pgfqpoint{1.233469in}{0.930648in}}%
\pgfpathlineto{\pgfqpoint{1.221526in}{0.921205in}}%
\pgfpathlineto{\pgfqpoint{1.215542in}{0.917036in}}%
\pgfpathlineto{\pgfqpoint{1.205870in}{0.910760in}}%
\pgfpathlineto{\pgfqpoint{1.192451in}{0.903425in}}%
\pgfpathlineto{\pgfqpoint{1.190213in}{0.902263in}}%
\pgfpathlineto{\pgfqpoint{1.174556in}{0.895956in}}%
\pgfpathlineto{\pgfqpoint{1.158900in}{0.891598in}}%
\pgfpathlineto{\pgfqpoint{1.146339in}{0.889814in}}%
\pgfpathlineto{\pgfqpoint{1.143243in}{0.889394in}}%
\pgfpathlineto{\pgfqpoint{1.127587in}{0.889394in}}%
\pgfpathlineto{\pgfqpoint{1.124491in}{0.889814in}}%
\pgfpathclose%
\pgfpathmoveto{\pgfqpoint{1.879102in}{0.802899in}}%
\pgfpathlineto{\pgfqpoint{1.894758in}{0.798886in}}%
\pgfpathlineto{\pgfqpoint{1.910415in}{0.797530in}}%
\pgfpathlineto{\pgfqpoint{1.910415in}{0.808148in}}%
\pgfpathlineto{\pgfqpoint{1.910415in}{0.821759in}}%
\pgfpathlineto{\pgfqpoint{1.910415in}{0.835370in}}%
\pgfpathlineto{\pgfqpoint{1.910415in}{0.848981in}}%
\pgfpathlineto{\pgfqpoint{1.910415in}{0.862592in}}%
\pgfpathlineto{\pgfqpoint{1.910415in}{0.876203in}}%
\pgfpathlineto{\pgfqpoint{1.910415in}{0.889127in}}%
\pgfpathlineto{\pgfqpoint{1.900318in}{0.889814in}}%
\pgfpathlineto{\pgfqpoint{1.894758in}{0.890211in}}%
\pgfpathlineto{\pgfqpoint{1.879102in}{0.893520in}}%
\pgfpathlineto{\pgfqpoint{1.863445in}{0.898880in}}%
\pgfpathlineto{\pgfqpoint{1.853607in}{0.903425in}}%
\pgfpathlineto{\pgfqpoint{1.847789in}{0.906261in}}%
\pgfpathlineto{\pgfqpoint{1.832132in}{0.915634in}}%
\pgfpathlineto{\pgfqpoint{1.830119in}{0.917036in}}%
\pgfpathlineto{\pgfqpoint{1.816476in}{0.927226in}}%
\pgfpathlineto{\pgfqpoint{1.812366in}{0.930648in}}%
\pgfpathlineto{\pgfqpoint{1.800819in}{0.941203in}}%
\pgfpathlineto{\pgfqpoint{1.797736in}{0.944259in}}%
\pgfpathlineto{\pgfqpoint{1.785635in}{0.957870in}}%
\pgfpathlineto{\pgfqpoint{1.785162in}{0.958504in}}%
\pgfpathlineto{\pgfqpoint{1.776089in}{0.971481in}}%
\pgfpathlineto{\pgfqpoint{1.769506in}{0.983463in}}%
\pgfpathlineto{\pgfqpoint{1.768651in}{0.985092in}}%
\pgfpathlineto{\pgfqpoint{1.763772in}{0.998703in}}%
\pgfpathlineto{\pgfqpoint{1.761279in}{1.012314in}}%
\pgfpathlineto{\pgfqpoint{1.761279in}{1.025925in}}%
\pgfpathlineto{\pgfqpoint{1.763772in}{1.039536in}}%
\pgfpathlineto{\pgfqpoint{1.768651in}{1.053148in}}%
\pgfpathlineto{\pgfqpoint{1.769506in}{1.054777in}}%
\pgfpathlineto{\pgfqpoint{1.776089in}{1.066759in}}%
\pgfpathlineto{\pgfqpoint{1.785162in}{1.079735in}}%
\pgfpathlineto{\pgfqpoint{1.785635in}{1.080370in}}%
\pgfpathlineto{\pgfqpoint{1.797736in}{1.093981in}}%
\pgfpathlineto{\pgfqpoint{1.800819in}{1.097036in}}%
\pgfpathlineto{\pgfqpoint{1.812366in}{1.107592in}}%
\pgfpathlineto{\pgfqpoint{1.816476in}{1.111013in}}%
\pgfpathlineto{\pgfqpoint{1.830119in}{1.121203in}}%
\pgfpathlineto{\pgfqpoint{1.832132in}{1.122606in}}%
\pgfpathlineto{\pgfqpoint{1.847789in}{1.131978in}}%
\pgfpathlineto{\pgfqpoint{1.853607in}{1.134814in}}%
\pgfpathlineto{\pgfqpoint{1.863445in}{1.139360in}}%
\pgfpathlineto{\pgfqpoint{1.879102in}{1.144719in}}%
\pgfpathlineto{\pgfqpoint{1.894758in}{1.148029in}}%
\pgfpathlineto{\pgfqpoint{1.900318in}{1.148425in}}%
\pgfpathlineto{\pgfqpoint{1.910415in}{1.149113in}}%
\pgfpathlineto{\pgfqpoint{1.910415in}{1.162036in}}%
\pgfpathlineto{\pgfqpoint{1.910415in}{1.175647in}}%
\pgfpathlineto{\pgfqpoint{1.910415in}{1.189259in}}%
\pgfpathlineto{\pgfqpoint{1.910415in}{1.202870in}}%
\pgfpathlineto{\pgfqpoint{1.910415in}{1.216481in}}%
\pgfpathlineto{\pgfqpoint{1.910415in}{1.230092in}}%
\pgfpathlineto{\pgfqpoint{1.910415in}{1.240710in}}%
\pgfpathlineto{\pgfqpoint{1.894758in}{1.239353in}}%
\pgfpathlineto{\pgfqpoint{1.879102in}{1.235341in}}%
\pgfpathlineto{\pgfqpoint{1.866483in}{1.230092in}}%
\pgfpathlineto{\pgfqpoint{1.863445in}{1.228906in}}%
\pgfpathlineto{\pgfqpoint{1.847789in}{1.220632in}}%
\pgfpathlineto{\pgfqpoint{1.841403in}{1.216481in}}%
\pgfpathlineto{\pgfqpoint{1.832132in}{1.210767in}}%
\pgfpathlineto{\pgfqpoint{1.821070in}{1.202870in}}%
\pgfpathlineto{\pgfqpoint{1.816476in}{1.199716in}}%
\pgfpathlineto{\pgfqpoint{1.802774in}{1.189259in}}%
\pgfpathlineto{\pgfqpoint{1.800819in}{1.187805in}}%
\pgfpathlineto{\pgfqpoint{1.785662in}{1.175647in}}%
\pgfpathlineto{\pgfqpoint{1.785162in}{1.175252in}}%
\pgfpathlineto{\pgfqpoint{1.769506in}{1.162190in}}%
\pgfpathlineto{\pgfqpoint{1.769327in}{1.162036in}}%
\pgfpathlineto{\pgfqpoint{1.753849in}{1.148686in}}%
\pgfpathlineto{\pgfqpoint{1.753551in}{1.148425in}}%
\pgfpathlineto{\pgfqpoint{1.738263in}{1.134814in}}%
\pgfpathlineto{\pgfqpoint{1.738193in}{1.134749in}}%
\pgfpathlineto{\pgfqpoint{1.723520in}{1.121203in}}%
\pgfpathlineto{\pgfqpoint{1.722536in}{1.120237in}}%
\pgfpathlineto{\pgfqpoint{1.709424in}{1.107592in}}%
\pgfpathlineto{\pgfqpoint{1.706880in}{1.104908in}}%
\pgfpathlineto{\pgfqpoint{1.696179in}{1.093981in}}%
\pgfpathlineto{\pgfqpoint{1.691223in}{1.088246in}}%
\pgfpathlineto{\pgfqpoint{1.684094in}{1.080370in}}%
\pgfpathlineto{\pgfqpoint{1.675567in}{1.069168in}}%
\pgfpathlineto{\pgfqpoint{1.673618in}{1.066759in}}%
\pgfpathlineto{\pgfqpoint{1.665069in}{1.053148in}}%
\pgfpathlineto{\pgfqpoint{1.659910in}{1.041242in}}%
\pgfpathlineto{\pgfqpoint{1.659116in}{1.039536in}}%
\pgfpathlineto{\pgfqpoint{1.655905in}{1.025925in}}%
\pgfpathlineto{\pgfqpoint{1.655905in}{1.012314in}}%
\pgfpathlineto{\pgfqpoint{1.659116in}{0.998703in}}%
\pgfpathlineto{\pgfqpoint{1.659910in}{0.996997in}}%
\pgfpathlineto{\pgfqpoint{1.665069in}{0.985092in}}%
\pgfpathlineto{\pgfqpoint{1.673618in}{0.971481in}}%
\pgfpathlineto{\pgfqpoint{1.675567in}{0.969071in}}%
\pgfpathlineto{\pgfqpoint{1.684094in}{0.957870in}}%
\pgfpathlineto{\pgfqpoint{1.691223in}{0.949993in}}%
\pgfpathlineto{\pgfqpoint{1.696179in}{0.944259in}}%
\pgfpathlineto{\pgfqpoint{1.706880in}{0.933332in}}%
\pgfpathlineto{\pgfqpoint{1.709424in}{0.930648in}}%
\pgfpathlineto{\pgfqpoint{1.722536in}{0.918003in}}%
\pgfpathlineto{\pgfqpoint{1.723520in}{0.917036in}}%
\pgfpathlineto{\pgfqpoint{1.738193in}{0.903490in}}%
\pgfpathlineto{\pgfqpoint{1.738263in}{0.903425in}}%
\pgfpathlineto{\pgfqpoint{1.753551in}{0.889814in}}%
\pgfpathlineto{\pgfqpoint{1.753849in}{0.889554in}}%
\pgfpathlineto{\pgfqpoint{1.769327in}{0.876203in}}%
\pgfpathlineto{\pgfqpoint{1.769506in}{0.876049in}}%
\pgfpathlineto{\pgfqpoint{1.785162in}{0.862988in}}%
\pgfpathlineto{\pgfqpoint{1.785662in}{0.862592in}}%
\pgfpathlineto{\pgfqpoint{1.800819in}{0.850435in}}%
\pgfpathlineto{\pgfqpoint{1.802774in}{0.848981in}}%
\pgfpathlineto{\pgfqpoint{1.816476in}{0.838523in}}%
\pgfpathlineto{\pgfqpoint{1.821070in}{0.835370in}}%
\pgfpathlineto{\pgfqpoint{1.832132in}{0.827473in}}%
\pgfpathlineto{\pgfqpoint{1.841403in}{0.821759in}}%
\pgfpathlineto{\pgfqpoint{1.847789in}{0.817608in}}%
\pgfpathlineto{\pgfqpoint{1.863445in}{0.809333in}}%
\pgfpathlineto{\pgfqpoint{1.866483in}{0.808148in}}%
\pgfpathlineto{\pgfqpoint{1.879102in}{0.802899in}}%
\pgfpathclose%
\pgfpathmoveto{\pgfqpoint{0.376072in}{1.472662in}}%
\pgfpathlineto{\pgfqpoint{0.385214in}{1.475092in}}%
\pgfpathlineto{\pgfqpoint{0.391728in}{1.476708in}}%
\pgfpathlineto{\pgfqpoint{0.407385in}{1.483023in}}%
\pgfpathlineto{\pgfqpoint{0.417826in}{1.488703in}}%
\pgfpathlineto{\pgfqpoint{0.423041in}{1.491379in}}%
\pgfpathlineto{\pgfqpoint{0.438698in}{1.501251in}}%
\pgfpathlineto{\pgfqpoint{0.440146in}{1.502314in}}%
\pgfpathlineto{\pgfqpoint{0.454354in}{1.512264in}}%
\pgfpathlineto{\pgfqpoint{0.459052in}{1.515925in}}%
\pgfpathlineto{\pgfqpoint{0.470011in}{1.524181in}}%
\pgfpathlineto{\pgfqpoint{0.476602in}{1.529536in}}%
\pgfpathlineto{\pgfqpoint{0.485668in}{1.536759in}}%
\pgfpathlineto{\pgfqpoint{0.493289in}{1.543148in}}%
\pgfpathlineto{\pgfqpoint{0.501324in}{1.549851in}}%
\pgfpathlineto{\pgfqpoint{0.509362in}{1.556759in}}%
\pgfpathlineto{\pgfqpoint{0.516981in}{1.563382in}}%
\pgfpathlineto{\pgfqpoint{0.524927in}{1.570370in}}%
\pgfpathlineto{\pgfqpoint{0.532637in}{1.577355in}}%
\pgfpathlineto{\pgfqpoint{0.539986in}{1.583981in}}%
\pgfpathlineto{\pgfqpoint{0.548294in}{1.591862in}}%
\pgfpathlineto{\pgfqpoint{0.554454in}{1.597592in}}%
\pgfpathlineto{\pgfqpoint{0.563950in}{1.607119in}}%
\pgfpathlineto{\pgfqpoint{0.568161in}{1.611203in}}%
\pgfpathlineto{\pgfqpoint{0.579607in}{1.623555in}}%
\pgfpathlineto{\pgfqpoint{0.580830in}{1.624814in}}%
\pgfpathlineto{\pgfqpoint{0.592186in}{1.638425in}}%
\pgfpathlineto{\pgfqpoint{0.595263in}{1.642959in}}%
\pgfpathlineto{\pgfqpoint{0.601798in}{1.652036in}}%
\pgfpathlineto{\pgfqpoint{0.609062in}{1.665648in}}%
\pgfpathlineto{\pgfqpoint{0.610920in}{1.671310in}}%
\pgfpathlineto{\pgfqpoint{0.613715in}{1.679259in}}%
\pgfpathlineto{\pgfqpoint{0.615330in}{1.692870in}}%
\pgfpathlineto{\pgfqpoint{0.610920in}{1.692870in}}%
\pgfpathlineto{\pgfqpoint{0.595263in}{1.692870in}}%
\pgfpathlineto{\pgfqpoint{0.579607in}{1.692870in}}%
\pgfpathlineto{\pgfqpoint{0.563950in}{1.692870in}}%
\pgfpathlineto{\pgfqpoint{0.548294in}{1.692870in}}%
\pgfpathlineto{\pgfqpoint{0.532637in}{1.692870in}}%
\pgfpathlineto{\pgfqpoint{0.516981in}{1.692870in}}%
\pgfpathlineto{\pgfqpoint{0.509865in}{1.692870in}}%
\pgfpathlineto{\pgfqpoint{0.508612in}{1.679259in}}%
\pgfpathlineto{\pgfqpoint{0.504907in}{1.665648in}}%
\pgfpathlineto{\pgfqpoint{0.501324in}{1.657559in}}%
\pgfpathlineto{\pgfqpoint{0.498759in}{1.652036in}}%
\pgfpathlineto{\pgfqpoint{0.490218in}{1.638425in}}%
\pgfpathlineto{\pgfqpoint{0.485668in}{1.632527in}}%
\pgfpathlineto{\pgfqpoint{0.479345in}{1.624814in}}%
\pgfpathlineto{\pgfqpoint{0.470011in}{1.614986in}}%
\pgfpathlineto{\pgfqpoint{0.466111in}{1.611203in}}%
\pgfpathlineto{\pgfqpoint{0.454354in}{1.600982in}}%
\pgfpathlineto{\pgfqpoint{0.450003in}{1.597592in}}%
\pgfpathlineto{\pgfqpoint{0.438698in}{1.589478in}}%
\pgfpathlineto{\pgfqpoint{0.429826in}{1.583981in}}%
\pgfpathlineto{\pgfqpoint{0.423041in}{1.580025in}}%
\pgfpathlineto{\pgfqpoint{0.407385in}{1.572600in}}%
\pgfpathlineto{\pgfqpoint{0.401032in}{1.570370in}}%
\pgfpathlineto{\pgfqpoint{0.391728in}{1.567255in}}%
\pgfpathlineto{\pgfqpoint{0.376072in}{1.564034in}}%
\pgfpathlineto{\pgfqpoint{0.360415in}{1.562945in}}%
\pgfpathlineto{\pgfqpoint{0.360415in}{1.556759in}}%
\pgfpathlineto{\pgfqpoint{0.360415in}{1.543148in}}%
\pgfpathlineto{\pgfqpoint{0.360415in}{1.529536in}}%
\pgfpathlineto{\pgfqpoint{0.360415in}{1.515925in}}%
\pgfpathlineto{\pgfqpoint{0.360415in}{1.502314in}}%
\pgfpathlineto{\pgfqpoint{0.360415in}{1.488703in}}%
\pgfpathlineto{\pgfqpoint{0.360415in}{1.475092in}}%
\pgfpathlineto{\pgfqpoint{0.360415in}{1.471258in}}%
\pgfpathlineto{\pgfqpoint{0.376072in}{1.472662in}}%
\pgfpathclose%
\pgfpathmoveto{\pgfqpoint{1.111930in}{1.474402in}}%
\pgfpathlineto{\pgfqpoint{1.127587in}{1.471610in}}%
\pgfpathlineto{\pgfqpoint{1.143243in}{1.471610in}}%
\pgfpathlineto{\pgfqpoint{1.158900in}{1.474402in}}%
\pgfpathlineto{\pgfqpoint{1.160862in}{1.475092in}}%
\pgfpathlineto{\pgfqpoint{1.174556in}{1.479577in}}%
\pgfpathlineto{\pgfqpoint{1.190213in}{1.487009in}}%
\pgfpathlineto{\pgfqpoint{1.192985in}{1.488703in}}%
\pgfpathlineto{\pgfqpoint{1.205870in}{1.496117in}}%
\pgfpathlineto{\pgfqpoint{1.214930in}{1.502314in}}%
\pgfpathlineto{\pgfqpoint{1.221526in}{1.506623in}}%
\pgfpathlineto{\pgfqpoint{1.234095in}{1.515925in}}%
\pgfpathlineto{\pgfqpoint{1.237183in}{1.518137in}}%
\pgfpathlineto{\pgfqpoint{1.251728in}{1.529536in}}%
\pgfpathlineto{\pgfqpoint{1.252839in}{1.530391in}}%
\pgfpathlineto{\pgfqpoint{1.268421in}{1.543148in}}%
\pgfpathlineto{\pgfqpoint{1.268496in}{1.543209in}}%
\pgfpathlineto{\pgfqpoint{1.284152in}{1.556499in}}%
\pgfpathlineto{\pgfqpoint{1.284452in}{1.556759in}}%
\pgfpathlineto{\pgfqpoint{1.299809in}{1.570214in}}%
\pgfpathlineto{\pgfqpoint{1.299986in}{1.570370in}}%
\pgfpathlineto{\pgfqpoint{1.315010in}{1.583981in}}%
\pgfpathlineto{\pgfqpoint{1.315466in}{1.584415in}}%
\pgfpathlineto{\pgfqpoint{1.329450in}{1.597592in}}%
\pgfpathlineto{\pgfqpoint{1.331122in}{1.599291in}}%
\pgfpathlineto{\pgfqpoint{1.343152in}{1.611203in}}%
\pgfpathlineto{\pgfqpoint{1.346779in}{1.615197in}}%
\pgfpathlineto{\pgfqpoint{1.355863in}{1.624814in}}%
\pgfpathlineto{\pgfqpoint{1.362435in}{1.632874in}}%
\pgfpathlineto{\pgfqpoint{1.367210in}{1.638425in}}%
\pgfpathlineto{\pgfqpoint{1.376728in}{1.652036in}}%
\pgfpathlineto{\pgfqpoint{1.378092in}{1.654678in}}%
\pgfpathlineto{\pgfqpoint{1.384130in}{1.665648in}}%
\pgfpathlineto{\pgfqpoint{1.388745in}{1.679259in}}%
\pgfpathlineto{\pgfqpoint{1.390305in}{1.692870in}}%
\pgfpathlineto{\pgfqpoint{1.378092in}{1.692870in}}%
\pgfpathlineto{\pgfqpoint{1.362435in}{1.692870in}}%
\pgfpathlineto{\pgfqpoint{1.346779in}{1.692870in}}%
\pgfpathlineto{\pgfqpoint{1.331122in}{1.692870in}}%
\pgfpathlineto{\pgfqpoint{1.315466in}{1.692870in}}%
\pgfpathlineto{\pgfqpoint{1.299809in}{1.692870in}}%
\pgfpathlineto{\pgfqpoint{1.284943in}{1.692870in}}%
\pgfpathlineto{\pgfqpoint{1.284152in}{1.684092in}}%
\pgfpathlineto{\pgfqpoint{1.283696in}{1.679259in}}%
\pgfpathlineto{\pgfqpoint{1.279890in}{1.665648in}}%
\pgfpathlineto{\pgfqpoint{1.273724in}{1.652036in}}%
\pgfpathlineto{\pgfqpoint{1.268496in}{1.643483in}}%
\pgfpathlineto{\pgfqpoint{1.265233in}{1.638425in}}%
\pgfpathlineto{\pgfqpoint{1.254453in}{1.624814in}}%
\pgfpathlineto{\pgfqpoint{1.252839in}{1.623064in}}%
\pgfpathlineto{\pgfqpoint{1.241118in}{1.611203in}}%
\pgfpathlineto{\pgfqpoint{1.237183in}{1.607630in}}%
\pgfpathlineto{\pgfqpoint{1.225041in}{1.597592in}}%
\pgfpathlineto{\pgfqpoint{1.221526in}{1.594912in}}%
\pgfpathlineto{\pgfqpoint{1.205870in}{1.584392in}}%
\pgfpathlineto{\pgfqpoint{1.205139in}{1.583981in}}%
\pgfpathlineto{\pgfqpoint{1.190213in}{1.576093in}}%
\pgfpathlineto{\pgfqpoint{1.176431in}{1.570370in}}%
\pgfpathlineto{\pgfqpoint{1.174556in}{1.569626in}}%
\pgfpathlineto{\pgfqpoint{1.158900in}{1.565385in}}%
\pgfpathlineto{\pgfqpoint{1.143243in}{1.563218in}}%
\pgfpathlineto{\pgfqpoint{1.127587in}{1.563218in}}%
\pgfpathlineto{\pgfqpoint{1.111930in}{1.565385in}}%
\pgfpathlineto{\pgfqpoint{1.096274in}{1.569626in}}%
\pgfpathlineto{\pgfqpoint{1.094399in}{1.570370in}}%
\pgfpathlineto{\pgfqpoint{1.080617in}{1.576093in}}%
\pgfpathlineto{\pgfqpoint{1.065691in}{1.583981in}}%
\pgfpathlineto{\pgfqpoint{1.064960in}{1.584392in}}%
\pgfpathlineto{\pgfqpoint{1.049304in}{1.594912in}}%
\pgfpathlineto{\pgfqpoint{1.045789in}{1.597592in}}%
\pgfpathlineto{\pgfqpoint{1.033647in}{1.607630in}}%
\pgfpathlineto{\pgfqpoint{1.029712in}{1.611203in}}%
\pgfpathlineto{\pgfqpoint{1.017991in}{1.623064in}}%
\pgfpathlineto{\pgfqpoint{1.016377in}{1.624814in}}%
\pgfpathlineto{\pgfqpoint{1.005597in}{1.638425in}}%
\pgfpathlineto{\pgfqpoint{1.002334in}{1.643483in}}%
\pgfpathlineto{\pgfqpoint{0.997106in}{1.652036in}}%
\pgfpathlineto{\pgfqpoint{0.990940in}{1.665648in}}%
\pgfpathlineto{\pgfqpoint{0.987134in}{1.679259in}}%
\pgfpathlineto{\pgfqpoint{0.986678in}{1.684092in}}%
\pgfpathlineto{\pgfqpoint{0.985887in}{1.692870in}}%
\pgfpathlineto{\pgfqpoint{0.971021in}{1.692870in}}%
\pgfpathlineto{\pgfqpoint{0.955364in}{1.692870in}}%
\pgfpathlineto{\pgfqpoint{0.939708in}{1.692870in}}%
\pgfpathlineto{\pgfqpoint{0.924051in}{1.692870in}}%
\pgfpathlineto{\pgfqpoint{0.908395in}{1.692870in}}%
\pgfpathlineto{\pgfqpoint{0.892738in}{1.692870in}}%
\pgfpathlineto{\pgfqpoint{0.880525in}{1.692870in}}%
\pgfpathlineto{\pgfqpoint{0.882085in}{1.679259in}}%
\pgfpathlineto{\pgfqpoint{0.886700in}{1.665648in}}%
\pgfpathlineto{\pgfqpoint{0.892738in}{1.654678in}}%
\pgfpathlineto{\pgfqpoint{0.894102in}{1.652036in}}%
\pgfpathlineto{\pgfqpoint{0.903620in}{1.638425in}}%
\pgfpathlineto{\pgfqpoint{0.908395in}{1.632874in}}%
\pgfpathlineto{\pgfqpoint{0.914967in}{1.624814in}}%
\pgfpathlineto{\pgfqpoint{0.924051in}{1.615197in}}%
\pgfpathlineto{\pgfqpoint{0.927678in}{1.611203in}}%
\pgfpathlineto{\pgfqpoint{0.939708in}{1.599291in}}%
\pgfpathlineto{\pgfqpoint{0.941380in}{1.597592in}}%
\pgfpathlineto{\pgfqpoint{0.955364in}{1.584415in}}%
\pgfpathlineto{\pgfqpoint{0.955820in}{1.583981in}}%
\pgfpathlineto{\pgfqpoint{0.970844in}{1.570370in}}%
\pgfpathlineto{\pgfqpoint{0.971021in}{1.570214in}}%
\pgfpathlineto{\pgfqpoint{0.986378in}{1.556759in}}%
\pgfpathlineto{\pgfqpoint{0.986678in}{1.556499in}}%
\pgfpathlineto{\pgfqpoint{1.002334in}{1.543209in}}%
\pgfpathlineto{\pgfqpoint{1.002409in}{1.543148in}}%
\pgfpathlineto{\pgfqpoint{1.017991in}{1.530391in}}%
\pgfpathlineto{\pgfqpoint{1.019102in}{1.529536in}}%
\pgfpathlineto{\pgfqpoint{1.033647in}{1.518137in}}%
\pgfpathlineto{\pgfqpoint{1.036735in}{1.515925in}}%
\pgfpathlineto{\pgfqpoint{1.049304in}{1.506623in}}%
\pgfpathlineto{\pgfqpoint{1.055900in}{1.502314in}}%
\pgfpathlineto{\pgfqpoint{1.064960in}{1.496117in}}%
\pgfpathlineto{\pgfqpoint{1.077845in}{1.488703in}}%
\pgfpathlineto{\pgfqpoint{1.080617in}{1.487009in}}%
\pgfpathlineto{\pgfqpoint{1.096274in}{1.479577in}}%
\pgfpathlineto{\pgfqpoint{1.109968in}{1.475092in}}%
\pgfpathlineto{\pgfqpoint{1.111930in}{1.474402in}}%
\pgfpathclose%
\pgfpathmoveto{\pgfqpoint{1.894758in}{1.472662in}}%
\pgfpathlineto{\pgfqpoint{1.910415in}{1.471258in}}%
\pgfpathlineto{\pgfqpoint{1.910415in}{1.475092in}}%
\pgfpathlineto{\pgfqpoint{1.910415in}{1.488703in}}%
\pgfpathlineto{\pgfqpoint{1.910415in}{1.502314in}}%
\pgfpathlineto{\pgfqpoint{1.910415in}{1.515925in}}%
\pgfpathlineto{\pgfqpoint{1.910415in}{1.529536in}}%
\pgfpathlineto{\pgfqpoint{1.910415in}{1.543148in}}%
\pgfpathlineto{\pgfqpoint{1.910415in}{1.556759in}}%
\pgfpathlineto{\pgfqpoint{1.910415in}{1.562945in}}%
\pgfpathlineto{\pgfqpoint{1.894758in}{1.564034in}}%
\pgfpathlineto{\pgfqpoint{1.879102in}{1.567255in}}%
\pgfpathlineto{\pgfqpoint{1.869798in}{1.570370in}}%
\pgfpathlineto{\pgfqpoint{1.863445in}{1.572600in}}%
\pgfpathlineto{\pgfqpoint{1.847789in}{1.580025in}}%
\pgfpathlineto{\pgfqpoint{1.841004in}{1.583981in}}%
\pgfpathlineto{\pgfqpoint{1.832132in}{1.589478in}}%
\pgfpathlineto{\pgfqpoint{1.820827in}{1.597592in}}%
\pgfpathlineto{\pgfqpoint{1.816476in}{1.600982in}}%
\pgfpathlineto{\pgfqpoint{1.804719in}{1.611203in}}%
\pgfpathlineto{\pgfqpoint{1.800819in}{1.614986in}}%
\pgfpathlineto{\pgfqpoint{1.791485in}{1.624814in}}%
\pgfpathlineto{\pgfqpoint{1.785162in}{1.632527in}}%
\pgfpathlineto{\pgfqpoint{1.780612in}{1.638425in}}%
\pgfpathlineto{\pgfqpoint{1.772071in}{1.652036in}}%
\pgfpathlineto{\pgfqpoint{1.769506in}{1.657559in}}%
\pgfpathlineto{\pgfqpoint{1.765923in}{1.665648in}}%
\pgfpathlineto{\pgfqpoint{1.762218in}{1.679259in}}%
\pgfpathlineto{\pgfqpoint{1.760965in}{1.692870in}}%
\pgfpathlineto{\pgfqpoint{1.753849in}{1.692870in}}%
\pgfpathlineto{\pgfqpoint{1.738193in}{1.692870in}}%
\pgfpathlineto{\pgfqpoint{1.722536in}{1.692870in}}%
\pgfpathlineto{\pgfqpoint{1.706880in}{1.692870in}}%
\pgfpathlineto{\pgfqpoint{1.691223in}{1.692870in}}%
\pgfpathlineto{\pgfqpoint{1.675567in}{1.692870in}}%
\pgfpathlineto{\pgfqpoint{1.659910in}{1.692870in}}%
\pgfpathlineto{\pgfqpoint{1.655500in}{1.692870in}}%
\pgfpathlineto{\pgfqpoint{1.657115in}{1.679259in}}%
\pgfpathlineto{\pgfqpoint{1.659910in}{1.671310in}}%
\pgfpathlineto{\pgfqpoint{1.661768in}{1.665648in}}%
\pgfpathlineto{\pgfqpoint{1.669032in}{1.652036in}}%
\pgfpathlineto{\pgfqpoint{1.675567in}{1.642959in}}%
\pgfpathlineto{\pgfqpoint{1.678644in}{1.638425in}}%
\pgfpathlineto{\pgfqpoint{1.690000in}{1.624814in}}%
\pgfpathlineto{\pgfqpoint{1.691223in}{1.623555in}}%
\pgfpathlineto{\pgfqpoint{1.702669in}{1.611203in}}%
\pgfpathlineto{\pgfqpoint{1.706880in}{1.607119in}}%
\pgfpathlineto{\pgfqpoint{1.716376in}{1.597592in}}%
\pgfpathlineto{\pgfqpoint{1.722536in}{1.591862in}}%
\pgfpathlineto{\pgfqpoint{1.730844in}{1.583981in}}%
\pgfpathlineto{\pgfqpoint{1.738193in}{1.577355in}}%
\pgfpathlineto{\pgfqpoint{1.745903in}{1.570370in}}%
\pgfpathlineto{\pgfqpoint{1.753849in}{1.563382in}}%
\pgfpathlineto{\pgfqpoint{1.761468in}{1.556759in}}%
\pgfpathlineto{\pgfqpoint{1.769506in}{1.549851in}}%
\pgfpathlineto{\pgfqpoint{1.777541in}{1.543148in}}%
\pgfpathlineto{\pgfqpoint{1.785162in}{1.536759in}}%
\pgfpathlineto{\pgfqpoint{1.794228in}{1.529536in}}%
\pgfpathlineto{\pgfqpoint{1.800819in}{1.524181in}}%
\pgfpathlineto{\pgfqpoint{1.811778in}{1.515925in}}%
\pgfpathlineto{\pgfqpoint{1.816476in}{1.512264in}}%
\pgfpathlineto{\pgfqpoint{1.830684in}{1.502314in}}%
\pgfpathlineto{\pgfqpoint{1.832132in}{1.501251in}}%
\pgfpathlineto{\pgfqpoint{1.847789in}{1.491379in}}%
\pgfpathlineto{\pgfqpoint{1.853004in}{1.488703in}}%
\pgfpathlineto{\pgfqpoint{1.863445in}{1.483023in}}%
\pgfpathlineto{\pgfqpoint{1.879102in}{1.476708in}}%
\pgfpathlineto{\pgfqpoint{1.885616in}{1.475092in}}%
\pgfpathlineto{\pgfqpoint{1.894758in}{1.472662in}}%
\pgfpathclose%
\pgfusepath{fill}%
\end{pgfscope}%
\begin{pgfscope}%
\pgfpathrectangle{\pgfqpoint{0.360415in}{0.345370in}}{\pgfqpoint{1.550000in}{1.347500in}}%
\pgfusepath{clip}%
\pgfsetbuttcap%
\pgfsetroundjoin%
\definecolor{currentfill}{rgb}{0.048062,0.036607,0.150327}%
\pgfsetfillcolor{currentfill}%
\pgfsetlinewidth{0.000000pt}%
\definecolor{currentstroke}{rgb}{0.000000,0.000000,0.000000}%
\pgfsetstrokecolor{currentstroke}%
\pgfsetdash{}{0pt}%
\pgfpathmoveto{\pgfqpoint{0.376072in}{0.345370in}}%
\pgfpathlineto{\pgfqpoint{0.391728in}{0.345370in}}%
\pgfpathlineto{\pgfqpoint{0.407385in}{0.345370in}}%
\pgfpathlineto{\pgfqpoint{0.423041in}{0.345370in}}%
\pgfpathlineto{\pgfqpoint{0.438698in}{0.345370in}}%
\pgfpathlineto{\pgfqpoint{0.454354in}{0.345370in}}%
\pgfpathlineto{\pgfqpoint{0.470011in}{0.345370in}}%
\pgfpathlineto{\pgfqpoint{0.485668in}{0.345370in}}%
\pgfpathlineto{\pgfqpoint{0.501324in}{0.345370in}}%
\pgfpathlineto{\pgfqpoint{0.509865in}{0.345370in}}%
\pgfpathlineto{\pgfqpoint{0.508612in}{0.358981in}}%
\pgfpathlineto{\pgfqpoint{0.504907in}{0.372592in}}%
\pgfpathlineto{\pgfqpoint{0.501324in}{0.380680in}}%
\pgfpathlineto{\pgfqpoint{0.498759in}{0.386203in}}%
\pgfpathlineto{\pgfqpoint{0.490218in}{0.399814in}}%
\pgfpathlineto{\pgfqpoint{0.485668in}{0.405712in}}%
\pgfpathlineto{\pgfqpoint{0.479345in}{0.413425in}}%
\pgfpathlineto{\pgfqpoint{0.470011in}{0.423254in}}%
\pgfpathlineto{\pgfqpoint{0.466111in}{0.427036in}}%
\pgfpathlineto{\pgfqpoint{0.454354in}{0.437257in}}%
\pgfpathlineto{\pgfqpoint{0.450003in}{0.440648in}}%
\pgfpathlineto{\pgfqpoint{0.438698in}{0.448762in}}%
\pgfpathlineto{\pgfqpoint{0.429826in}{0.454259in}}%
\pgfpathlineto{\pgfqpoint{0.423041in}{0.458214in}}%
\pgfpathlineto{\pgfqpoint{0.407385in}{0.465640in}}%
\pgfpathlineto{\pgfqpoint{0.401032in}{0.467870in}}%
\pgfpathlineto{\pgfqpoint{0.391728in}{0.470984in}}%
\pgfpathlineto{\pgfqpoint{0.376072in}{0.474205in}}%
\pgfpathlineto{\pgfqpoint{0.360415in}{0.475295in}}%
\pgfpathlineto{\pgfqpoint{0.360415in}{0.467870in}}%
\pgfpathlineto{\pgfqpoint{0.360415in}{0.454259in}}%
\pgfpathlineto{\pgfqpoint{0.360415in}{0.440648in}}%
\pgfpathlineto{\pgfqpoint{0.360415in}{0.427036in}}%
\pgfpathlineto{\pgfqpoint{0.360415in}{0.413425in}}%
\pgfpathlineto{\pgfqpoint{0.360415in}{0.399814in}}%
\pgfpathlineto{\pgfqpoint{0.360415in}{0.386203in}}%
\pgfpathlineto{\pgfqpoint{0.360415in}{0.372592in}}%
\pgfpathlineto{\pgfqpoint{0.360415in}{0.358981in}}%
\pgfpathlineto{\pgfqpoint{0.360415in}{0.345370in}}%
\pgfpathlineto{\pgfqpoint{0.376072in}{0.345370in}}%
\pgfpathclose%
\pgfpathmoveto{\pgfqpoint{0.986678in}{0.345370in}}%
\pgfpathlineto{\pgfqpoint{1.002334in}{0.345370in}}%
\pgfpathlineto{\pgfqpoint{1.017991in}{0.345370in}}%
\pgfpathlineto{\pgfqpoint{1.033647in}{0.345370in}}%
\pgfpathlineto{\pgfqpoint{1.049304in}{0.345370in}}%
\pgfpathlineto{\pgfqpoint{1.064960in}{0.345370in}}%
\pgfpathlineto{\pgfqpoint{1.080617in}{0.345370in}}%
\pgfpathlineto{\pgfqpoint{1.096274in}{0.345370in}}%
\pgfpathlineto{\pgfqpoint{1.111930in}{0.345370in}}%
\pgfpathlineto{\pgfqpoint{1.127587in}{0.345370in}}%
\pgfpathlineto{\pgfqpoint{1.143243in}{0.345370in}}%
\pgfpathlineto{\pgfqpoint{1.158900in}{0.345370in}}%
\pgfpathlineto{\pgfqpoint{1.174556in}{0.345370in}}%
\pgfpathlineto{\pgfqpoint{1.190213in}{0.345370in}}%
\pgfpathlineto{\pgfqpoint{1.205870in}{0.345370in}}%
\pgfpathlineto{\pgfqpoint{1.221526in}{0.345370in}}%
\pgfpathlineto{\pgfqpoint{1.237183in}{0.345370in}}%
\pgfpathlineto{\pgfqpoint{1.252839in}{0.345370in}}%
\pgfpathlineto{\pgfqpoint{1.268496in}{0.345370in}}%
\pgfpathlineto{\pgfqpoint{1.284152in}{0.345370in}}%
\pgfpathlineto{\pgfqpoint{1.284943in}{0.345370in}}%
\pgfpathlineto{\pgfqpoint{1.284152in}{0.354148in}}%
\pgfpathlineto{\pgfqpoint{1.283696in}{0.358981in}}%
\pgfpathlineto{\pgfqpoint{1.279890in}{0.372592in}}%
\pgfpathlineto{\pgfqpoint{1.273724in}{0.386203in}}%
\pgfpathlineto{\pgfqpoint{1.268496in}{0.394756in}}%
\pgfpathlineto{\pgfqpoint{1.265233in}{0.399814in}}%
\pgfpathlineto{\pgfqpoint{1.254453in}{0.413425in}}%
\pgfpathlineto{\pgfqpoint{1.252839in}{0.415175in}}%
\pgfpathlineto{\pgfqpoint{1.241118in}{0.427036in}}%
\pgfpathlineto{\pgfqpoint{1.237183in}{0.430609in}}%
\pgfpathlineto{\pgfqpoint{1.225041in}{0.440648in}}%
\pgfpathlineto{\pgfqpoint{1.221526in}{0.443328in}}%
\pgfpathlineto{\pgfqpoint{1.205870in}{0.453848in}}%
\pgfpathlineto{\pgfqpoint{1.205139in}{0.454259in}}%
\pgfpathlineto{\pgfqpoint{1.190213in}{0.462147in}}%
\pgfpathlineto{\pgfqpoint{1.176431in}{0.467870in}}%
\pgfpathlineto{\pgfqpoint{1.174556in}{0.468613in}}%
\pgfpathlineto{\pgfqpoint{1.158900in}{0.472855in}}%
\pgfpathlineto{\pgfqpoint{1.143243in}{0.475022in}}%
\pgfpathlineto{\pgfqpoint{1.127587in}{0.475022in}}%
\pgfpathlineto{\pgfqpoint{1.111930in}{0.472855in}}%
\pgfpathlineto{\pgfqpoint{1.096274in}{0.468613in}}%
\pgfpathlineto{\pgfqpoint{1.094399in}{0.467870in}}%
\pgfpathlineto{\pgfqpoint{1.080617in}{0.462147in}}%
\pgfpathlineto{\pgfqpoint{1.065691in}{0.454259in}}%
\pgfpathlineto{\pgfqpoint{1.064960in}{0.453848in}}%
\pgfpathlineto{\pgfqpoint{1.049304in}{0.443328in}}%
\pgfpathlineto{\pgfqpoint{1.045789in}{0.440648in}}%
\pgfpathlineto{\pgfqpoint{1.033647in}{0.430609in}}%
\pgfpathlineto{\pgfqpoint{1.029712in}{0.427036in}}%
\pgfpathlineto{\pgfqpoint{1.017991in}{0.415175in}}%
\pgfpathlineto{\pgfqpoint{1.016377in}{0.413425in}}%
\pgfpathlineto{\pgfqpoint{1.005597in}{0.399814in}}%
\pgfpathlineto{\pgfqpoint{1.002334in}{0.394756in}}%
\pgfpathlineto{\pgfqpoint{0.997106in}{0.386203in}}%
\pgfpathlineto{\pgfqpoint{0.990940in}{0.372592in}}%
\pgfpathlineto{\pgfqpoint{0.987134in}{0.358981in}}%
\pgfpathlineto{\pgfqpoint{0.986678in}{0.354148in}}%
\pgfpathlineto{\pgfqpoint{0.985887in}{0.345370in}}%
\pgfpathlineto{\pgfqpoint{0.986678in}{0.345370in}}%
\pgfpathclose%
\pgfpathmoveto{\pgfqpoint{1.769506in}{0.345370in}}%
\pgfpathlineto{\pgfqpoint{1.785162in}{0.345370in}}%
\pgfpathlineto{\pgfqpoint{1.800819in}{0.345370in}}%
\pgfpathlineto{\pgfqpoint{1.816476in}{0.345370in}}%
\pgfpathlineto{\pgfqpoint{1.832132in}{0.345370in}}%
\pgfpathlineto{\pgfqpoint{1.847789in}{0.345370in}}%
\pgfpathlineto{\pgfqpoint{1.863445in}{0.345370in}}%
\pgfpathlineto{\pgfqpoint{1.879102in}{0.345370in}}%
\pgfpathlineto{\pgfqpoint{1.894758in}{0.345370in}}%
\pgfpathlineto{\pgfqpoint{1.910415in}{0.345370in}}%
\pgfpathlineto{\pgfqpoint{1.910415in}{0.358981in}}%
\pgfpathlineto{\pgfqpoint{1.910415in}{0.372592in}}%
\pgfpathlineto{\pgfqpoint{1.910415in}{0.386203in}}%
\pgfpathlineto{\pgfqpoint{1.910415in}{0.399814in}}%
\pgfpathlineto{\pgfqpoint{1.910415in}{0.413425in}}%
\pgfpathlineto{\pgfqpoint{1.910415in}{0.427036in}}%
\pgfpathlineto{\pgfqpoint{1.910415in}{0.440648in}}%
\pgfpathlineto{\pgfqpoint{1.910415in}{0.454259in}}%
\pgfpathlineto{\pgfqpoint{1.910415in}{0.467870in}}%
\pgfpathlineto{\pgfqpoint{1.910415in}{0.475295in}}%
\pgfpathlineto{\pgfqpoint{1.894758in}{0.474205in}}%
\pgfpathlineto{\pgfqpoint{1.879102in}{0.470984in}}%
\pgfpathlineto{\pgfqpoint{1.869798in}{0.467870in}}%
\pgfpathlineto{\pgfqpoint{1.863445in}{0.465640in}}%
\pgfpathlineto{\pgfqpoint{1.847789in}{0.458214in}}%
\pgfpathlineto{\pgfqpoint{1.841004in}{0.454259in}}%
\pgfpathlineto{\pgfqpoint{1.832132in}{0.448762in}}%
\pgfpathlineto{\pgfqpoint{1.820827in}{0.440648in}}%
\pgfpathlineto{\pgfqpoint{1.816476in}{0.437257in}}%
\pgfpathlineto{\pgfqpoint{1.804719in}{0.427036in}}%
\pgfpathlineto{\pgfqpoint{1.800819in}{0.423254in}}%
\pgfpathlineto{\pgfqpoint{1.791485in}{0.413425in}}%
\pgfpathlineto{\pgfqpoint{1.785162in}{0.405712in}}%
\pgfpathlineto{\pgfqpoint{1.780612in}{0.399814in}}%
\pgfpathlineto{\pgfqpoint{1.772071in}{0.386203in}}%
\pgfpathlineto{\pgfqpoint{1.769506in}{0.380680in}}%
\pgfpathlineto{\pgfqpoint{1.765923in}{0.372592in}}%
\pgfpathlineto{\pgfqpoint{1.762218in}{0.358981in}}%
\pgfpathlineto{\pgfqpoint{1.760965in}{0.345370in}}%
\pgfpathlineto{\pgfqpoint{1.769506in}{0.345370in}}%
\pgfpathclose%
\pgfpathmoveto{\pgfqpoint{0.370512in}{0.889814in}}%
\pgfpathlineto{\pgfqpoint{0.376072in}{0.890211in}}%
\pgfpathlineto{\pgfqpoint{0.391728in}{0.893520in}}%
\pgfpathlineto{\pgfqpoint{0.407385in}{0.898880in}}%
\pgfpathlineto{\pgfqpoint{0.417223in}{0.903425in}}%
\pgfpathlineto{\pgfqpoint{0.423041in}{0.906261in}}%
\pgfpathlineto{\pgfqpoint{0.438698in}{0.915634in}}%
\pgfpathlineto{\pgfqpoint{0.440711in}{0.917036in}}%
\pgfpathlineto{\pgfqpoint{0.454354in}{0.927226in}}%
\pgfpathlineto{\pgfqpoint{0.458464in}{0.930648in}}%
\pgfpathlineto{\pgfqpoint{0.470011in}{0.941203in}}%
\pgfpathlineto{\pgfqpoint{0.473094in}{0.944259in}}%
\pgfpathlineto{\pgfqpoint{0.485195in}{0.957870in}}%
\pgfpathlineto{\pgfqpoint{0.485668in}{0.958504in}}%
\pgfpathlineto{\pgfqpoint{0.494741in}{0.971481in}}%
\pgfpathlineto{\pgfqpoint{0.501324in}{0.983463in}}%
\pgfpathlineto{\pgfqpoint{0.502179in}{0.985092in}}%
\pgfpathlineto{\pgfqpoint{0.507058in}{0.998703in}}%
\pgfpathlineto{\pgfqpoint{0.509551in}{1.012314in}}%
\pgfpathlineto{\pgfqpoint{0.509551in}{1.025925in}}%
\pgfpathlineto{\pgfqpoint{0.507058in}{1.039536in}}%
\pgfpathlineto{\pgfqpoint{0.502179in}{1.053148in}}%
\pgfpathlineto{\pgfqpoint{0.501324in}{1.054777in}}%
\pgfpathlineto{\pgfqpoint{0.494741in}{1.066759in}}%
\pgfpathlineto{\pgfqpoint{0.485668in}{1.079735in}}%
\pgfpathlineto{\pgfqpoint{0.485195in}{1.080370in}}%
\pgfpathlineto{\pgfqpoint{0.473094in}{1.093981in}}%
\pgfpathlineto{\pgfqpoint{0.470011in}{1.097036in}}%
\pgfpathlineto{\pgfqpoint{0.458464in}{1.107592in}}%
\pgfpathlineto{\pgfqpoint{0.454354in}{1.111013in}}%
\pgfpathlineto{\pgfqpoint{0.440711in}{1.121203in}}%
\pgfpathlineto{\pgfqpoint{0.438698in}{1.122606in}}%
\pgfpathlineto{\pgfqpoint{0.423041in}{1.131978in}}%
\pgfpathlineto{\pgfqpoint{0.417223in}{1.134814in}}%
\pgfpathlineto{\pgfqpoint{0.407385in}{1.139360in}}%
\pgfpathlineto{\pgfqpoint{0.391728in}{1.144719in}}%
\pgfpathlineto{\pgfqpoint{0.376072in}{1.148029in}}%
\pgfpathlineto{\pgfqpoint{0.370512in}{1.148425in}}%
\pgfpathlineto{\pgfqpoint{0.360415in}{1.149113in}}%
\pgfpathlineto{\pgfqpoint{0.360415in}{1.148425in}}%
\pgfpathlineto{\pgfqpoint{0.360415in}{1.134814in}}%
\pgfpathlineto{\pgfqpoint{0.360415in}{1.121203in}}%
\pgfpathlineto{\pgfqpoint{0.360415in}{1.107592in}}%
\pgfpathlineto{\pgfqpoint{0.360415in}{1.093981in}}%
\pgfpathlineto{\pgfqpoint{0.360415in}{1.080370in}}%
\pgfpathlineto{\pgfqpoint{0.360415in}{1.066759in}}%
\pgfpathlineto{\pgfqpoint{0.360415in}{1.053148in}}%
\pgfpathlineto{\pgfqpoint{0.360415in}{1.039536in}}%
\pgfpathlineto{\pgfqpoint{0.360415in}{1.025925in}}%
\pgfpathlineto{\pgfqpoint{0.360415in}{1.012314in}}%
\pgfpathlineto{\pgfqpoint{0.360415in}{0.998703in}}%
\pgfpathlineto{\pgfqpoint{0.360415in}{0.985092in}}%
\pgfpathlineto{\pgfqpoint{0.360415in}{0.971481in}}%
\pgfpathlineto{\pgfqpoint{0.360415in}{0.957870in}}%
\pgfpathlineto{\pgfqpoint{0.360415in}{0.944259in}}%
\pgfpathlineto{\pgfqpoint{0.360415in}{0.930648in}}%
\pgfpathlineto{\pgfqpoint{0.360415in}{0.917036in}}%
\pgfpathlineto{\pgfqpoint{0.360415in}{0.903425in}}%
\pgfpathlineto{\pgfqpoint{0.360415in}{0.889814in}}%
\pgfpathlineto{\pgfqpoint{0.360415in}{0.889127in}}%
\pgfpathlineto{\pgfqpoint{0.370512in}{0.889814in}}%
\pgfpathclose%
\pgfpathmoveto{\pgfqpoint{1.127587in}{0.889394in}}%
\pgfpathlineto{\pgfqpoint{1.143243in}{0.889394in}}%
\pgfpathlineto{\pgfqpoint{1.146339in}{0.889814in}}%
\pgfpathlineto{\pgfqpoint{1.158900in}{0.891598in}}%
\pgfpathlineto{\pgfqpoint{1.174556in}{0.895956in}}%
\pgfpathlineto{\pgfqpoint{1.190213in}{0.902263in}}%
\pgfpathlineto{\pgfqpoint{1.192451in}{0.903425in}}%
\pgfpathlineto{\pgfqpoint{1.205870in}{0.910760in}}%
\pgfpathlineto{\pgfqpoint{1.215542in}{0.917036in}}%
\pgfpathlineto{\pgfqpoint{1.221526in}{0.921205in}}%
\pgfpathlineto{\pgfqpoint{1.233469in}{0.930648in}}%
\pgfpathlineto{\pgfqpoint{1.237183in}{0.933876in}}%
\pgfpathlineto{\pgfqpoint{1.248045in}{0.944259in}}%
\pgfpathlineto{\pgfqpoint{1.252839in}{0.949461in}}%
\pgfpathlineto{\pgfqpoint{1.260059in}{0.957870in}}%
\pgfpathlineto{\pgfqpoint{1.268496in}{0.969535in}}%
\pgfpathlineto{\pgfqpoint{1.269832in}{0.971481in}}%
\pgfpathlineto{\pgfqpoint{1.277088in}{0.985092in}}%
\pgfpathlineto{\pgfqpoint{1.282100in}{0.998703in}}%
\pgfpathlineto{\pgfqpoint{1.284152in}{1.009623in}}%
\pgfpathlineto{\pgfqpoint{1.284636in}{1.012314in}}%
\pgfpathlineto{\pgfqpoint{1.284636in}{1.025925in}}%
\pgfpathlineto{\pgfqpoint{1.284152in}{1.028617in}}%
\pgfpathlineto{\pgfqpoint{1.282100in}{1.039536in}}%
\pgfpathlineto{\pgfqpoint{1.277088in}{1.053148in}}%
\pgfpathlineto{\pgfqpoint{1.269832in}{1.066759in}}%
\pgfpathlineto{\pgfqpoint{1.268496in}{1.068705in}}%
\pgfpathlineto{\pgfqpoint{1.260059in}{1.080370in}}%
\pgfpathlineto{\pgfqpoint{1.252839in}{1.088779in}}%
\pgfpathlineto{\pgfqpoint{1.248045in}{1.093981in}}%
\pgfpathlineto{\pgfqpoint{1.237183in}{1.104363in}}%
\pgfpathlineto{\pgfqpoint{1.233469in}{1.107592in}}%
\pgfpathlineto{\pgfqpoint{1.221526in}{1.117035in}}%
\pgfpathlineto{\pgfqpoint{1.215542in}{1.121203in}}%
\pgfpathlineto{\pgfqpoint{1.205870in}{1.127480in}}%
\pgfpathlineto{\pgfqpoint{1.192451in}{1.134814in}}%
\pgfpathlineto{\pgfqpoint{1.190213in}{1.135976in}}%
\pgfpathlineto{\pgfqpoint{1.174556in}{1.142284in}}%
\pgfpathlineto{\pgfqpoint{1.158900in}{1.146641in}}%
\pgfpathlineto{\pgfqpoint{1.146339in}{1.148425in}}%
\pgfpathlineto{\pgfqpoint{1.143243in}{1.148846in}}%
\pgfpathlineto{\pgfqpoint{1.127587in}{1.148846in}}%
\pgfpathlineto{\pgfqpoint{1.124491in}{1.148425in}}%
\pgfpathlineto{\pgfqpoint{1.111930in}{1.146641in}}%
\pgfpathlineto{\pgfqpoint{1.096274in}{1.142284in}}%
\pgfpathlineto{\pgfqpoint{1.080617in}{1.135976in}}%
\pgfpathlineto{\pgfqpoint{1.078379in}{1.134814in}}%
\pgfpathlineto{\pgfqpoint{1.064960in}{1.127480in}}%
\pgfpathlineto{\pgfqpoint{1.055288in}{1.121203in}}%
\pgfpathlineto{\pgfqpoint{1.049304in}{1.117035in}}%
\pgfpathlineto{\pgfqpoint{1.037361in}{1.107592in}}%
\pgfpathlineto{\pgfqpoint{1.033647in}{1.104363in}}%
\pgfpathlineto{\pgfqpoint{1.022785in}{1.093981in}}%
\pgfpathlineto{\pgfqpoint{1.017991in}{1.088779in}}%
\pgfpathlineto{\pgfqpoint{1.010771in}{1.080370in}}%
\pgfpathlineto{\pgfqpoint{1.002334in}{1.068705in}}%
\pgfpathlineto{\pgfqpoint{1.000998in}{1.066759in}}%
\pgfpathlineto{\pgfqpoint{0.993742in}{1.053148in}}%
\pgfpathlineto{\pgfqpoint{0.988730in}{1.039536in}}%
\pgfpathlineto{\pgfqpoint{0.986678in}{1.028617in}}%
\pgfpathlineto{\pgfqpoint{0.986194in}{1.025925in}}%
\pgfpathlineto{\pgfqpoint{0.986194in}{1.012314in}}%
\pgfpathlineto{\pgfqpoint{0.986678in}{1.009623in}}%
\pgfpathlineto{\pgfqpoint{0.988730in}{0.998703in}}%
\pgfpathlineto{\pgfqpoint{0.993742in}{0.985092in}}%
\pgfpathlineto{\pgfqpoint{1.000998in}{0.971481in}}%
\pgfpathlineto{\pgfqpoint{1.002334in}{0.969535in}}%
\pgfpathlineto{\pgfqpoint{1.010771in}{0.957870in}}%
\pgfpathlineto{\pgfqpoint{1.017991in}{0.949461in}}%
\pgfpathlineto{\pgfqpoint{1.022785in}{0.944259in}}%
\pgfpathlineto{\pgfqpoint{1.033647in}{0.933876in}}%
\pgfpathlineto{\pgfqpoint{1.037361in}{0.930648in}}%
\pgfpathlineto{\pgfqpoint{1.049304in}{0.921205in}}%
\pgfpathlineto{\pgfqpoint{1.055288in}{0.917036in}}%
\pgfpathlineto{\pgfqpoint{1.064960in}{0.910760in}}%
\pgfpathlineto{\pgfqpoint{1.078379in}{0.903425in}}%
\pgfpathlineto{\pgfqpoint{1.080617in}{0.902263in}}%
\pgfpathlineto{\pgfqpoint{1.096274in}{0.895956in}}%
\pgfpathlineto{\pgfqpoint{1.111930in}{0.891598in}}%
\pgfpathlineto{\pgfqpoint{1.124491in}{0.889814in}}%
\pgfpathlineto{\pgfqpoint{1.127587in}{0.889394in}}%
\pgfpathclose%
\pgfpathmoveto{\pgfqpoint{1.910415in}{0.889127in}}%
\pgfpathlineto{\pgfqpoint{1.910415in}{0.889814in}}%
\pgfpathlineto{\pgfqpoint{1.910415in}{0.903425in}}%
\pgfpathlineto{\pgfqpoint{1.910415in}{0.917036in}}%
\pgfpathlineto{\pgfqpoint{1.910415in}{0.930648in}}%
\pgfpathlineto{\pgfqpoint{1.910415in}{0.944259in}}%
\pgfpathlineto{\pgfqpoint{1.910415in}{0.957870in}}%
\pgfpathlineto{\pgfqpoint{1.910415in}{0.971481in}}%
\pgfpathlineto{\pgfqpoint{1.910415in}{0.985092in}}%
\pgfpathlineto{\pgfqpoint{1.910415in}{0.998703in}}%
\pgfpathlineto{\pgfqpoint{1.910415in}{1.012314in}}%
\pgfpathlineto{\pgfqpoint{1.910415in}{1.025925in}}%
\pgfpathlineto{\pgfqpoint{1.910415in}{1.039536in}}%
\pgfpathlineto{\pgfqpoint{1.910415in}{1.053148in}}%
\pgfpathlineto{\pgfqpoint{1.910415in}{1.066759in}}%
\pgfpathlineto{\pgfqpoint{1.910415in}{1.080370in}}%
\pgfpathlineto{\pgfqpoint{1.910415in}{1.093981in}}%
\pgfpathlineto{\pgfqpoint{1.910415in}{1.107592in}}%
\pgfpathlineto{\pgfqpoint{1.910415in}{1.121203in}}%
\pgfpathlineto{\pgfqpoint{1.910415in}{1.134814in}}%
\pgfpathlineto{\pgfqpoint{1.910415in}{1.148425in}}%
\pgfpathlineto{\pgfqpoint{1.910415in}{1.149113in}}%
\pgfpathlineto{\pgfqpoint{1.900318in}{1.148425in}}%
\pgfpathlineto{\pgfqpoint{1.894758in}{1.148029in}}%
\pgfpathlineto{\pgfqpoint{1.879102in}{1.144719in}}%
\pgfpathlineto{\pgfqpoint{1.863445in}{1.139360in}}%
\pgfpathlineto{\pgfqpoint{1.853607in}{1.134814in}}%
\pgfpathlineto{\pgfqpoint{1.847789in}{1.131978in}}%
\pgfpathlineto{\pgfqpoint{1.832132in}{1.122606in}}%
\pgfpathlineto{\pgfqpoint{1.830119in}{1.121203in}}%
\pgfpathlineto{\pgfqpoint{1.816476in}{1.111013in}}%
\pgfpathlineto{\pgfqpoint{1.812366in}{1.107592in}}%
\pgfpathlineto{\pgfqpoint{1.800819in}{1.097036in}}%
\pgfpathlineto{\pgfqpoint{1.797736in}{1.093981in}}%
\pgfpathlineto{\pgfqpoint{1.785635in}{1.080370in}}%
\pgfpathlineto{\pgfqpoint{1.785162in}{1.079735in}}%
\pgfpathlineto{\pgfqpoint{1.776089in}{1.066759in}}%
\pgfpathlineto{\pgfqpoint{1.769506in}{1.054777in}}%
\pgfpathlineto{\pgfqpoint{1.768651in}{1.053148in}}%
\pgfpathlineto{\pgfqpoint{1.763772in}{1.039536in}}%
\pgfpathlineto{\pgfqpoint{1.761279in}{1.025925in}}%
\pgfpathlineto{\pgfqpoint{1.761279in}{1.012314in}}%
\pgfpathlineto{\pgfqpoint{1.763772in}{0.998703in}}%
\pgfpathlineto{\pgfqpoint{1.768651in}{0.985092in}}%
\pgfpathlineto{\pgfqpoint{1.769506in}{0.983463in}}%
\pgfpathlineto{\pgfqpoint{1.776089in}{0.971481in}}%
\pgfpathlineto{\pgfqpoint{1.785162in}{0.958504in}}%
\pgfpathlineto{\pgfqpoint{1.785635in}{0.957870in}}%
\pgfpathlineto{\pgfqpoint{1.797736in}{0.944259in}}%
\pgfpathlineto{\pgfqpoint{1.800819in}{0.941203in}}%
\pgfpathlineto{\pgfqpoint{1.812366in}{0.930648in}}%
\pgfpathlineto{\pgfqpoint{1.816476in}{0.927226in}}%
\pgfpathlineto{\pgfqpoint{1.830119in}{0.917036in}}%
\pgfpathlineto{\pgfqpoint{1.832132in}{0.915634in}}%
\pgfpathlineto{\pgfqpoint{1.847789in}{0.906261in}}%
\pgfpathlineto{\pgfqpoint{1.853607in}{0.903425in}}%
\pgfpathlineto{\pgfqpoint{1.863445in}{0.898880in}}%
\pgfpathlineto{\pgfqpoint{1.879102in}{0.893520in}}%
\pgfpathlineto{\pgfqpoint{1.894758in}{0.890211in}}%
\pgfpathlineto{\pgfqpoint{1.900318in}{0.889814in}}%
\pgfpathlineto{\pgfqpoint{1.910415in}{0.889127in}}%
\pgfpathclose%
\pgfpathmoveto{\pgfqpoint{0.376072in}{1.564034in}}%
\pgfpathlineto{\pgfqpoint{0.391728in}{1.567255in}}%
\pgfpathlineto{\pgfqpoint{0.401032in}{1.570370in}}%
\pgfpathlineto{\pgfqpoint{0.407385in}{1.572600in}}%
\pgfpathlineto{\pgfqpoint{0.423041in}{1.580025in}}%
\pgfpathlineto{\pgfqpoint{0.429826in}{1.583981in}}%
\pgfpathlineto{\pgfqpoint{0.438698in}{1.589478in}}%
\pgfpathlineto{\pgfqpoint{0.450003in}{1.597592in}}%
\pgfpathlineto{\pgfqpoint{0.454354in}{1.600982in}}%
\pgfpathlineto{\pgfqpoint{0.466111in}{1.611203in}}%
\pgfpathlineto{\pgfqpoint{0.470011in}{1.614986in}}%
\pgfpathlineto{\pgfqpoint{0.479345in}{1.624814in}}%
\pgfpathlineto{\pgfqpoint{0.485668in}{1.632527in}}%
\pgfpathlineto{\pgfqpoint{0.490218in}{1.638425in}}%
\pgfpathlineto{\pgfqpoint{0.498759in}{1.652036in}}%
\pgfpathlineto{\pgfqpoint{0.501324in}{1.657559in}}%
\pgfpathlineto{\pgfqpoint{0.504907in}{1.665648in}}%
\pgfpathlineto{\pgfqpoint{0.508612in}{1.679259in}}%
\pgfpathlineto{\pgfqpoint{0.509865in}{1.692870in}}%
\pgfpathlineto{\pgfqpoint{0.501324in}{1.692870in}}%
\pgfpathlineto{\pgfqpoint{0.485668in}{1.692870in}}%
\pgfpathlineto{\pgfqpoint{0.470011in}{1.692870in}}%
\pgfpathlineto{\pgfqpoint{0.454354in}{1.692870in}}%
\pgfpathlineto{\pgfqpoint{0.438698in}{1.692870in}}%
\pgfpathlineto{\pgfqpoint{0.423041in}{1.692870in}}%
\pgfpathlineto{\pgfqpoint{0.407385in}{1.692870in}}%
\pgfpathlineto{\pgfqpoint{0.391728in}{1.692870in}}%
\pgfpathlineto{\pgfqpoint{0.376072in}{1.692870in}}%
\pgfpathlineto{\pgfqpoint{0.360415in}{1.692870in}}%
\pgfpathlineto{\pgfqpoint{0.360415in}{1.679259in}}%
\pgfpathlineto{\pgfqpoint{0.360415in}{1.665648in}}%
\pgfpathlineto{\pgfqpoint{0.360415in}{1.652036in}}%
\pgfpathlineto{\pgfqpoint{0.360415in}{1.638425in}}%
\pgfpathlineto{\pgfqpoint{0.360415in}{1.624814in}}%
\pgfpathlineto{\pgfqpoint{0.360415in}{1.611203in}}%
\pgfpathlineto{\pgfqpoint{0.360415in}{1.597592in}}%
\pgfpathlineto{\pgfqpoint{0.360415in}{1.583981in}}%
\pgfpathlineto{\pgfqpoint{0.360415in}{1.570370in}}%
\pgfpathlineto{\pgfqpoint{0.360415in}{1.562945in}}%
\pgfpathlineto{\pgfqpoint{0.376072in}{1.564034in}}%
\pgfpathclose%
\pgfpathmoveto{\pgfqpoint{1.096274in}{1.569626in}}%
\pgfpathlineto{\pgfqpoint{1.111930in}{1.565385in}}%
\pgfpathlineto{\pgfqpoint{1.127587in}{1.563218in}}%
\pgfpathlineto{\pgfqpoint{1.143243in}{1.563218in}}%
\pgfpathlineto{\pgfqpoint{1.158900in}{1.565385in}}%
\pgfpathlineto{\pgfqpoint{1.174556in}{1.569626in}}%
\pgfpathlineto{\pgfqpoint{1.176431in}{1.570370in}}%
\pgfpathlineto{\pgfqpoint{1.190213in}{1.576093in}}%
\pgfpathlineto{\pgfqpoint{1.205139in}{1.583981in}}%
\pgfpathlineto{\pgfqpoint{1.205870in}{1.584392in}}%
\pgfpathlineto{\pgfqpoint{1.221526in}{1.594912in}}%
\pgfpathlineto{\pgfqpoint{1.225041in}{1.597592in}}%
\pgfpathlineto{\pgfqpoint{1.237183in}{1.607630in}}%
\pgfpathlineto{\pgfqpoint{1.241118in}{1.611203in}}%
\pgfpathlineto{\pgfqpoint{1.252839in}{1.623064in}}%
\pgfpathlineto{\pgfqpoint{1.254453in}{1.624814in}}%
\pgfpathlineto{\pgfqpoint{1.265233in}{1.638425in}}%
\pgfpathlineto{\pgfqpoint{1.268496in}{1.643483in}}%
\pgfpathlineto{\pgfqpoint{1.273724in}{1.652036in}}%
\pgfpathlineto{\pgfqpoint{1.279890in}{1.665648in}}%
\pgfpathlineto{\pgfqpoint{1.283696in}{1.679259in}}%
\pgfpathlineto{\pgfqpoint{1.284152in}{1.684092in}}%
\pgfpathlineto{\pgfqpoint{1.284943in}{1.692870in}}%
\pgfpathlineto{\pgfqpoint{1.284152in}{1.692870in}}%
\pgfpathlineto{\pgfqpoint{1.268496in}{1.692870in}}%
\pgfpathlineto{\pgfqpoint{1.252839in}{1.692870in}}%
\pgfpathlineto{\pgfqpoint{1.237183in}{1.692870in}}%
\pgfpathlineto{\pgfqpoint{1.221526in}{1.692870in}}%
\pgfpathlineto{\pgfqpoint{1.205870in}{1.692870in}}%
\pgfpathlineto{\pgfqpoint{1.190213in}{1.692870in}}%
\pgfpathlineto{\pgfqpoint{1.174556in}{1.692870in}}%
\pgfpathlineto{\pgfqpoint{1.158900in}{1.692870in}}%
\pgfpathlineto{\pgfqpoint{1.143243in}{1.692870in}}%
\pgfpathlineto{\pgfqpoint{1.127587in}{1.692870in}}%
\pgfpathlineto{\pgfqpoint{1.111930in}{1.692870in}}%
\pgfpathlineto{\pgfqpoint{1.096274in}{1.692870in}}%
\pgfpathlineto{\pgfqpoint{1.080617in}{1.692870in}}%
\pgfpathlineto{\pgfqpoint{1.064960in}{1.692870in}}%
\pgfpathlineto{\pgfqpoint{1.049304in}{1.692870in}}%
\pgfpathlineto{\pgfqpoint{1.033647in}{1.692870in}}%
\pgfpathlineto{\pgfqpoint{1.017991in}{1.692870in}}%
\pgfpathlineto{\pgfqpoint{1.002334in}{1.692870in}}%
\pgfpathlineto{\pgfqpoint{0.986678in}{1.692870in}}%
\pgfpathlineto{\pgfqpoint{0.985887in}{1.692870in}}%
\pgfpathlineto{\pgfqpoint{0.986678in}{1.684092in}}%
\pgfpathlineto{\pgfqpoint{0.987134in}{1.679259in}}%
\pgfpathlineto{\pgfqpoint{0.990940in}{1.665648in}}%
\pgfpathlineto{\pgfqpoint{0.997106in}{1.652036in}}%
\pgfpathlineto{\pgfqpoint{1.002334in}{1.643483in}}%
\pgfpathlineto{\pgfqpoint{1.005597in}{1.638425in}}%
\pgfpathlineto{\pgfqpoint{1.016377in}{1.624814in}}%
\pgfpathlineto{\pgfqpoint{1.017991in}{1.623064in}}%
\pgfpathlineto{\pgfqpoint{1.029712in}{1.611203in}}%
\pgfpathlineto{\pgfqpoint{1.033647in}{1.607630in}}%
\pgfpathlineto{\pgfqpoint{1.045789in}{1.597592in}}%
\pgfpathlineto{\pgfqpoint{1.049304in}{1.594912in}}%
\pgfpathlineto{\pgfqpoint{1.064960in}{1.584392in}}%
\pgfpathlineto{\pgfqpoint{1.065691in}{1.583981in}}%
\pgfpathlineto{\pgfqpoint{1.080617in}{1.576093in}}%
\pgfpathlineto{\pgfqpoint{1.094399in}{1.570370in}}%
\pgfpathlineto{\pgfqpoint{1.096274in}{1.569626in}}%
\pgfpathclose%
\pgfpathmoveto{\pgfqpoint{1.879102in}{1.567255in}}%
\pgfpathlineto{\pgfqpoint{1.894758in}{1.564034in}}%
\pgfpathlineto{\pgfqpoint{1.910415in}{1.562945in}}%
\pgfpathlineto{\pgfqpoint{1.910415in}{1.570370in}}%
\pgfpathlineto{\pgfqpoint{1.910415in}{1.583981in}}%
\pgfpathlineto{\pgfqpoint{1.910415in}{1.597592in}}%
\pgfpathlineto{\pgfqpoint{1.910415in}{1.611203in}}%
\pgfpathlineto{\pgfqpoint{1.910415in}{1.624814in}}%
\pgfpathlineto{\pgfqpoint{1.910415in}{1.638425in}}%
\pgfpathlineto{\pgfqpoint{1.910415in}{1.652036in}}%
\pgfpathlineto{\pgfqpoint{1.910415in}{1.665648in}}%
\pgfpathlineto{\pgfqpoint{1.910415in}{1.679259in}}%
\pgfpathlineto{\pgfqpoint{1.910415in}{1.692870in}}%
\pgfpathlineto{\pgfqpoint{1.894758in}{1.692870in}}%
\pgfpathlineto{\pgfqpoint{1.879102in}{1.692870in}}%
\pgfpathlineto{\pgfqpoint{1.863445in}{1.692870in}}%
\pgfpathlineto{\pgfqpoint{1.847789in}{1.692870in}}%
\pgfpathlineto{\pgfqpoint{1.832132in}{1.692870in}}%
\pgfpathlineto{\pgfqpoint{1.816476in}{1.692870in}}%
\pgfpathlineto{\pgfqpoint{1.800819in}{1.692870in}}%
\pgfpathlineto{\pgfqpoint{1.785162in}{1.692870in}}%
\pgfpathlineto{\pgfqpoint{1.769506in}{1.692870in}}%
\pgfpathlineto{\pgfqpoint{1.760965in}{1.692870in}}%
\pgfpathlineto{\pgfqpoint{1.762218in}{1.679259in}}%
\pgfpathlineto{\pgfqpoint{1.765923in}{1.665648in}}%
\pgfpathlineto{\pgfqpoint{1.769506in}{1.657559in}}%
\pgfpathlineto{\pgfqpoint{1.772071in}{1.652036in}}%
\pgfpathlineto{\pgfqpoint{1.780612in}{1.638425in}}%
\pgfpathlineto{\pgfqpoint{1.785162in}{1.632527in}}%
\pgfpathlineto{\pgfqpoint{1.791485in}{1.624814in}}%
\pgfpathlineto{\pgfqpoint{1.800819in}{1.614986in}}%
\pgfpathlineto{\pgfqpoint{1.804719in}{1.611203in}}%
\pgfpathlineto{\pgfqpoint{1.816476in}{1.600982in}}%
\pgfpathlineto{\pgfqpoint{1.820827in}{1.597592in}}%
\pgfpathlineto{\pgfqpoint{1.832132in}{1.589478in}}%
\pgfpathlineto{\pgfqpoint{1.841004in}{1.583981in}}%
\pgfpathlineto{\pgfqpoint{1.847789in}{1.580025in}}%
\pgfpathlineto{\pgfqpoint{1.863445in}{1.572600in}}%
\pgfpathlineto{\pgfqpoint{1.869798in}{1.570370in}}%
\pgfpathlineto{\pgfqpoint{1.879102in}{1.567255in}}%
\pgfpathclose%
\pgfusepath{fill}%
\end{pgfscope}%
\begin{pgfscope}%
\pgfsetbuttcap%
\pgfsetroundjoin%
\definecolor{currentfill}{rgb}{0.000000,0.000000,0.000000}%
\pgfsetfillcolor{currentfill}%
\pgfsetlinewidth{0.803000pt}%
\definecolor{currentstroke}{rgb}{0.000000,0.000000,0.000000}%
\pgfsetstrokecolor{currentstroke}%
\pgfsetdash{}{0pt}%
\pgfsys@defobject{currentmarker}{\pgfqpoint{0.000000in}{-0.048611in}}{\pgfqpoint{0.000000in}{0.000000in}}{%
\pgfpathmoveto{\pgfqpoint{0.000000in}{0.000000in}}%
\pgfpathlineto{\pgfqpoint{0.000000in}{-0.048611in}}%
\pgfusepath{stroke,fill}%
}%
\begin{pgfscope}%
\pgfsys@transformshift{0.360415in}{0.345370in}%
\pgfsys@useobject{currentmarker}{}%
\end{pgfscope}%
\end{pgfscope}%
\begin{pgfscope}%
\definecolor{textcolor}{rgb}{0.000000,0.000000,0.000000}%
\pgfsetstrokecolor{textcolor}%
\pgfsetfillcolor{textcolor}%
\pgftext[x=0.360415in,y=0.248148in,,top]{\color{textcolor}{\rmfamily\fontsize{12.000000}{14.400000}\selectfont\catcode`\^=\active\def^{\ifmmode\sp\else\^{}\fi}\catcode`\%=\active\def%{\%}$\mathdefault{0}$}}%
\end{pgfscope}%
\begin{pgfscope}%
\pgfsetbuttcap%
\pgfsetroundjoin%
\definecolor{currentfill}{rgb}{0.000000,0.000000,0.000000}%
\pgfsetfillcolor{currentfill}%
\pgfsetlinewidth{0.803000pt}%
\definecolor{currentstroke}{rgb}{0.000000,0.000000,0.000000}%
\pgfsetstrokecolor{currentstroke}%
\pgfsetdash{}{0pt}%
\pgfsys@defobject{currentmarker}{\pgfqpoint{0.000000in}{-0.048611in}}{\pgfqpoint{0.000000in}{0.000000in}}{%
\pgfpathmoveto{\pgfqpoint{0.000000in}{0.000000in}}%
\pgfpathlineto{\pgfqpoint{0.000000in}{-0.048611in}}%
\pgfusepath{stroke,fill}%
}%
\begin{pgfscope}%
\pgfsys@transformshift{1.006248in}{0.345370in}%
\pgfsys@useobject{currentmarker}{}%
\end{pgfscope}%
\end{pgfscope}%
\begin{pgfscope}%
\definecolor{textcolor}{rgb}{0.000000,0.000000,0.000000}%
\pgfsetstrokecolor{textcolor}%
\pgfsetfillcolor{textcolor}%
\pgftext[x=1.006248in,y=0.248148in,,top]{\color{textcolor}{\rmfamily\fontsize{12.000000}{14.400000}\selectfont\catcode`\^=\active\def^{\ifmmode\sp\else\^{}\fi}\catcode`\%=\active\def%{\%}$\mathdefault{5}$}}%
\end{pgfscope}%
\begin{pgfscope}%
\pgfsetbuttcap%
\pgfsetroundjoin%
\definecolor{currentfill}{rgb}{0.000000,0.000000,0.000000}%
\pgfsetfillcolor{currentfill}%
\pgfsetlinewidth{0.803000pt}%
\definecolor{currentstroke}{rgb}{0.000000,0.000000,0.000000}%
\pgfsetstrokecolor{currentstroke}%
\pgfsetdash{}{0pt}%
\pgfsys@defobject{currentmarker}{\pgfqpoint{0.000000in}{-0.048611in}}{\pgfqpoint{0.000000in}{0.000000in}}{%
\pgfpathmoveto{\pgfqpoint{0.000000in}{0.000000in}}%
\pgfpathlineto{\pgfqpoint{0.000000in}{-0.048611in}}%
\pgfusepath{stroke,fill}%
}%
\begin{pgfscope}%
\pgfsys@transformshift{1.652082in}{0.345370in}%
\pgfsys@useobject{currentmarker}{}%
\end{pgfscope}%
\end{pgfscope}%
\begin{pgfscope}%
\definecolor{textcolor}{rgb}{0.000000,0.000000,0.000000}%
\pgfsetstrokecolor{textcolor}%
\pgfsetfillcolor{textcolor}%
\pgftext[x=1.652082in,y=0.248148in,,top]{\color{textcolor}{\rmfamily\fontsize{12.000000}{14.400000}\selectfont\catcode`\^=\active\def^{\ifmmode\sp\else\^{}\fi}\catcode`\%=\active\def%{\%}$\mathdefault{10}$}}%
\end{pgfscope}%
\begin{pgfscope}%
\pgfsetbuttcap%
\pgfsetroundjoin%
\definecolor{currentfill}{rgb}{0.000000,0.000000,0.000000}%
\pgfsetfillcolor{currentfill}%
\pgfsetlinewidth{0.803000pt}%
\definecolor{currentstroke}{rgb}{0.000000,0.000000,0.000000}%
\pgfsetstrokecolor{currentstroke}%
\pgfsetdash{}{0pt}%
\pgfsys@defobject{currentmarker}{\pgfqpoint{-0.048611in}{0.000000in}}{\pgfqpoint{-0.000000in}{0.000000in}}{%
\pgfpathmoveto{\pgfqpoint{-0.000000in}{0.000000in}}%
\pgfpathlineto{\pgfqpoint{-0.048611in}{0.000000in}}%
\pgfusepath{stroke,fill}%
}%
\begin{pgfscope}%
\pgfsys@transformshift{0.360415in}{0.345370in}%
\pgfsys@useobject{currentmarker}{}%
\end{pgfscope}%
\end{pgfscope}%
\begin{pgfscope}%
\definecolor{textcolor}{rgb}{0.000000,0.000000,0.000000}%
\pgfsetstrokecolor{textcolor}%
\pgfsetfillcolor{textcolor}%
\pgftext[x=0.181596in, y=0.287500in, left, base]{\color{textcolor}{\rmfamily\fontsize{12.000000}{14.400000}\selectfont\catcode`\^=\active\def^{\ifmmode\sp\else\^{}\fi}\catcode`\%=\active\def%{\%}$\mathdefault{0}$}}%
\end{pgfscope}%
\begin{pgfscope}%
\pgfsetbuttcap%
\pgfsetroundjoin%
\definecolor{currentfill}{rgb}{0.000000,0.000000,0.000000}%
\pgfsetfillcolor{currentfill}%
\pgfsetlinewidth{0.803000pt}%
\definecolor{currentstroke}{rgb}{0.000000,0.000000,0.000000}%
\pgfsetstrokecolor{currentstroke}%
\pgfsetdash{}{0pt}%
\pgfsys@defobject{currentmarker}{\pgfqpoint{-0.048611in}{0.000000in}}{\pgfqpoint{-0.000000in}{0.000000in}}{%
\pgfpathmoveto{\pgfqpoint{-0.000000in}{0.000000in}}%
\pgfpathlineto{\pgfqpoint{-0.048611in}{0.000000in}}%
\pgfusepath{stroke,fill}%
}%
\begin{pgfscope}%
\pgfsys@transformshift{0.360415in}{0.906828in}%
\pgfsys@useobject{currentmarker}{}%
\end{pgfscope}%
\end{pgfscope}%
\begin{pgfscope}%
\definecolor{textcolor}{rgb}{0.000000,0.000000,0.000000}%
\pgfsetstrokecolor{textcolor}%
\pgfsetfillcolor{textcolor}%
\pgftext[x=0.181596in, y=0.848958in, left, base]{\color{textcolor}{\rmfamily\fontsize{12.000000}{14.400000}\selectfont\catcode`\^=\active\def^{\ifmmode\sp\else\^{}\fi}\catcode`\%=\active\def%{\%}$\mathdefault{5}$}}%
\end{pgfscope}%
\begin{pgfscope}%
\pgfsetbuttcap%
\pgfsetroundjoin%
\definecolor{currentfill}{rgb}{0.000000,0.000000,0.000000}%
\pgfsetfillcolor{currentfill}%
\pgfsetlinewidth{0.803000pt}%
\definecolor{currentstroke}{rgb}{0.000000,0.000000,0.000000}%
\pgfsetstrokecolor{currentstroke}%
\pgfsetdash{}{0pt}%
\pgfsys@defobject{currentmarker}{\pgfqpoint{-0.048611in}{0.000000in}}{\pgfqpoint{-0.000000in}{0.000000in}}{%
\pgfpathmoveto{\pgfqpoint{-0.000000in}{0.000000in}}%
\pgfpathlineto{\pgfqpoint{-0.048611in}{0.000000in}}%
\pgfusepath{stroke,fill}%
}%
\begin{pgfscope}%
\pgfsys@transformshift{0.360415in}{1.468286in}%
\pgfsys@useobject{currentmarker}{}%
\end{pgfscope}%
\end{pgfscope}%
\begin{pgfscope}%
\definecolor{textcolor}{rgb}{0.000000,0.000000,0.000000}%
\pgfsetstrokecolor{textcolor}%
\pgfsetfillcolor{textcolor}%
\pgftext[x=0.100000in, y=1.410416in, left, base]{\color{textcolor}{\rmfamily\fontsize{12.000000}{14.400000}\selectfont\catcode`\^=\active\def^{\ifmmode\sp\else\^{}\fi}\catcode`\%=\active\def%{\%}$\mathdefault{10}$}}%
\end{pgfscope}%
\begin{pgfscope}%
\pgfsetrectcap%
\pgfsetmiterjoin%
\pgfsetlinewidth{0.803000pt}%
\definecolor{currentstroke}{rgb}{0.000000,0.000000,0.000000}%
\pgfsetstrokecolor{currentstroke}%
\pgfsetdash{}{0pt}%
\pgfpathmoveto{\pgfqpoint{0.360415in}{0.345370in}}%
\pgfpathlineto{\pgfqpoint{0.360415in}{1.692870in}}%
\pgfusepath{stroke}%
\end{pgfscope}%
\begin{pgfscope}%
\pgfsetrectcap%
\pgfsetmiterjoin%
\pgfsetlinewidth{0.803000pt}%
\definecolor{currentstroke}{rgb}{0.000000,0.000000,0.000000}%
\pgfsetstrokecolor{currentstroke}%
\pgfsetdash{}{0pt}%
\pgfpathmoveto{\pgfqpoint{1.910415in}{0.345370in}}%
\pgfpathlineto{\pgfqpoint{1.910415in}{1.692870in}}%
\pgfusepath{stroke}%
\end{pgfscope}%
\begin{pgfscope}%
\pgfsetrectcap%
\pgfsetmiterjoin%
\pgfsetlinewidth{0.803000pt}%
\definecolor{currentstroke}{rgb}{0.000000,0.000000,0.000000}%
\pgfsetstrokecolor{currentstroke}%
\pgfsetdash{}{0pt}%
\pgfpathmoveto{\pgfqpoint{0.360415in}{0.345370in}}%
\pgfpathlineto{\pgfqpoint{1.910415in}{0.345370in}}%
\pgfusepath{stroke}%
\end{pgfscope}%
\begin{pgfscope}%
\pgfsetrectcap%
\pgfsetmiterjoin%
\pgfsetlinewidth{0.803000pt}%
\definecolor{currentstroke}{rgb}{0.000000,0.000000,0.000000}%
\pgfsetstrokecolor{currentstroke}%
\pgfsetdash{}{0pt}%
\pgfpathmoveto{\pgfqpoint{0.360415in}{1.692870in}}%
\pgfpathlineto{\pgfqpoint{1.910415in}{1.692870in}}%
\pgfusepath{stroke}%
\end{pgfscope}%
\end{pgfpicture}%
\makeatother%
\endgroup%}
        \caption{$n_c=2.$}
        \label{fig:gaussian-well-2}
    \end{subfigure}
    \begin{subfigure}[b]{0.32\columnwidth}
        \scalebox{0.7}{%% Creator: Matplotlib, PGF backend
%%
%% To include the figure in your LaTeX document, write
%%   \input{<filename>.pgf}
%%
%% Make sure the required packages are loaded in your preamble
%%   \usepackage{pgf}
%%
%% Also ensure that all the required font packages are loaded; for instance,
%% the lmodern package is sometimes necessary when using math font.
%%   \usepackage{lmodern}
%%
%% Figures using additional raster images can only be included by \input if
%% they are in the same directory as the main LaTeX file. For loading figures
%% from other directories you can use the `import` package
%%   \usepackage{import}
%%
%% and then include the figures with
%%   \import{<path to file>}{<filename>.pgf}
%%
%% Matplotlib used the following preamble
%%   \def\mathdefault#1{#1}
%%   \everymath=\expandafter{\the\everymath\displaystyle}
%%   
%%   \ifdefined\pdftexversion\else  % non-pdftex case.
%%     \usepackage{fontspec}
%%     \setmainfont{DejaVuSerif.ttf}[Path=\detokenize{/opt/hostedtoolcache/Python/3.12.3/x64/lib/python3.12/site-packages/matplotlib/mpl-data/fonts/ttf/}]
%%     \setsansfont{DejaVuSans.ttf}[Path=\detokenize{/opt/hostedtoolcache/Python/3.12.3/x64/lib/python3.12/site-packages/matplotlib/mpl-data/fonts/ttf/}]
%%     \setmonofont{DejaVuSansMono.ttf}[Path=\detokenize{/opt/hostedtoolcache/Python/3.12.3/x64/lib/python3.12/site-packages/matplotlib/mpl-data/fonts/ttf/}]
%%   \fi
%%   \makeatletter\@ifpackageloaded{underscore}{}{\usepackage[strings]{underscore}}\makeatother
%%
\begingroup%
\makeatletter%
\begin{pgfpicture}%
\pgfpathrectangle{\pgfpointorigin}{\pgfqpoint{2.010415in}{1.792870in}}%
\pgfusepath{use as bounding box, clip}%
\begin{pgfscope}%
\pgfsetbuttcap%
\pgfsetmiterjoin%
\definecolor{currentfill}{rgb}{1.000000,1.000000,1.000000}%
\pgfsetfillcolor{currentfill}%
\pgfsetlinewidth{0.000000pt}%
\definecolor{currentstroke}{rgb}{1.000000,1.000000,1.000000}%
\pgfsetstrokecolor{currentstroke}%
\pgfsetdash{}{0pt}%
\pgfpathmoveto{\pgfqpoint{0.000000in}{0.000000in}}%
\pgfpathlineto{\pgfqpoint{2.010415in}{0.000000in}}%
\pgfpathlineto{\pgfqpoint{2.010415in}{1.792870in}}%
\pgfpathlineto{\pgfqpoint{0.000000in}{1.792870in}}%
\pgfpathlineto{\pgfqpoint{0.000000in}{0.000000in}}%
\pgfpathclose%
\pgfusepath{fill}%
\end{pgfscope}%
\begin{pgfscope}%
\pgfsetbuttcap%
\pgfsetmiterjoin%
\definecolor{currentfill}{rgb}{1.000000,1.000000,1.000000}%
\pgfsetfillcolor{currentfill}%
\pgfsetlinewidth{0.000000pt}%
\definecolor{currentstroke}{rgb}{0.000000,0.000000,0.000000}%
\pgfsetstrokecolor{currentstroke}%
\pgfsetstrokeopacity{0.000000}%
\pgfsetdash{}{0pt}%
\pgfpathmoveto{\pgfqpoint{0.360415in}{0.345370in}}%
\pgfpathlineto{\pgfqpoint{1.910415in}{0.345370in}}%
\pgfpathlineto{\pgfqpoint{1.910415in}{1.692870in}}%
\pgfpathlineto{\pgfqpoint{0.360415in}{1.692870in}}%
\pgfpathlineto{\pgfqpoint{0.360415in}{0.345370in}}%
\pgfpathclose%
\pgfusepath{fill}%
\end{pgfscope}%
\begin{pgfscope}%
\pgfpathrectangle{\pgfqpoint{0.360415in}{0.345370in}}{\pgfqpoint{1.550000in}{1.347500in}}%
\pgfusepath{clip}%
\pgfsetbuttcap%
\pgfsetroundjoin%
\definecolor{currentfill}{rgb}{0.972530,0.881250,0.144923}%
\pgfsetfillcolor{currentfill}%
\pgfsetlinewidth{0.000000pt}%
\definecolor{currentstroke}{rgb}{0.000000,0.000000,0.000000}%
\pgfsetstrokecolor{currentstroke}%
\pgfsetdash{}{0pt}%
\pgfpathmoveto{\pgfqpoint{0.610920in}{0.507302in}}%
\pgfpathlineto{\pgfqpoint{0.626577in}{0.507302in}}%
\pgfpathlineto{\pgfqpoint{0.633961in}{0.508703in}}%
\pgfpathlineto{\pgfqpoint{0.642233in}{0.510567in}}%
\pgfpathlineto{\pgfqpoint{0.657890in}{0.517610in}}%
\pgfpathlineto{\pgfqpoint{0.664928in}{0.522314in}}%
\pgfpathlineto{\pgfqpoint{0.673546in}{0.529807in}}%
\pgfpathlineto{\pgfqpoint{0.678958in}{0.535925in}}%
\pgfpathlineto{\pgfqpoint{0.687059in}{0.549536in}}%
\pgfpathlineto{\pgfqpoint{0.689203in}{0.556728in}}%
\pgfpathlineto{\pgfqpoint{0.690815in}{0.563148in}}%
\pgfpathlineto{\pgfqpoint{0.690815in}{0.576759in}}%
\pgfpathlineto{\pgfqpoint{0.689203in}{0.583178in}}%
\pgfpathlineto{\pgfqpoint{0.687059in}{0.590370in}}%
\pgfpathlineto{\pgfqpoint{0.678958in}{0.603981in}}%
\pgfpathlineto{\pgfqpoint{0.673546in}{0.610099in}}%
\pgfpathlineto{\pgfqpoint{0.664928in}{0.617592in}}%
\pgfpathlineto{\pgfqpoint{0.657890in}{0.622296in}}%
\pgfpathlineto{\pgfqpoint{0.642233in}{0.629339in}}%
\pgfpathlineto{\pgfqpoint{0.633961in}{0.631203in}}%
\pgfpathlineto{\pgfqpoint{0.626577in}{0.632604in}}%
\pgfpathlineto{\pgfqpoint{0.610920in}{0.632604in}}%
\pgfpathlineto{\pgfqpoint{0.603536in}{0.631203in}}%
\pgfpathlineto{\pgfqpoint{0.595263in}{0.629339in}}%
\pgfpathlineto{\pgfqpoint{0.579607in}{0.622296in}}%
\pgfpathlineto{\pgfqpoint{0.572569in}{0.617592in}}%
\pgfpathlineto{\pgfqpoint{0.563950in}{0.610099in}}%
\pgfpathlineto{\pgfqpoint{0.558539in}{0.603981in}}%
\pgfpathlineto{\pgfqpoint{0.550438in}{0.590370in}}%
\pgfpathlineto{\pgfqpoint{0.548294in}{0.583178in}}%
\pgfpathlineto{\pgfqpoint{0.546682in}{0.576759in}}%
\pgfpathlineto{\pgfqpoint{0.546682in}{0.563148in}}%
\pgfpathlineto{\pgfqpoint{0.548294in}{0.556728in}}%
\pgfpathlineto{\pgfqpoint{0.550438in}{0.549536in}}%
\pgfpathlineto{\pgfqpoint{0.558539in}{0.535925in}}%
\pgfpathlineto{\pgfqpoint{0.563950in}{0.529807in}}%
\pgfpathlineto{\pgfqpoint{0.572569in}{0.522314in}}%
\pgfpathlineto{\pgfqpoint{0.579607in}{0.517610in}}%
\pgfpathlineto{\pgfqpoint{0.595263in}{0.510567in}}%
\pgfpathlineto{\pgfqpoint{0.603536in}{0.508703in}}%
\pgfpathlineto{\pgfqpoint{0.610920in}{0.507302in}}%
\pgfpathclose%
\pgfpathmoveto{\pgfqpoint{1.127587in}{0.507302in}}%
\pgfpathlineto{\pgfqpoint{1.143243in}{0.507302in}}%
\pgfpathlineto{\pgfqpoint{1.150628in}{0.508703in}}%
\pgfpathlineto{\pgfqpoint{1.158900in}{0.510567in}}%
\pgfpathlineto{\pgfqpoint{1.174556in}{0.517610in}}%
\pgfpathlineto{\pgfqpoint{1.181595in}{0.522314in}}%
\pgfpathlineto{\pgfqpoint{1.190213in}{0.529807in}}%
\pgfpathlineto{\pgfqpoint{1.195625in}{0.535925in}}%
\pgfpathlineto{\pgfqpoint{1.203725in}{0.549536in}}%
\pgfpathlineto{\pgfqpoint{1.205870in}{0.556728in}}%
\pgfpathlineto{\pgfqpoint{1.207482in}{0.563148in}}%
\pgfpathlineto{\pgfqpoint{1.207482in}{0.576759in}}%
\pgfpathlineto{\pgfqpoint{1.205870in}{0.583178in}}%
\pgfpathlineto{\pgfqpoint{1.203725in}{0.590370in}}%
\pgfpathlineto{\pgfqpoint{1.195625in}{0.603981in}}%
\pgfpathlineto{\pgfqpoint{1.190213in}{0.610099in}}%
\pgfpathlineto{\pgfqpoint{1.181595in}{0.617592in}}%
\pgfpathlineto{\pgfqpoint{1.174556in}{0.622296in}}%
\pgfpathlineto{\pgfqpoint{1.158900in}{0.629339in}}%
\pgfpathlineto{\pgfqpoint{1.150628in}{0.631203in}}%
\pgfpathlineto{\pgfqpoint{1.143243in}{0.632604in}}%
\pgfpathlineto{\pgfqpoint{1.127587in}{0.632604in}}%
\pgfpathlineto{\pgfqpoint{1.120202in}{0.631203in}}%
\pgfpathlineto{\pgfqpoint{1.111930in}{0.629339in}}%
\pgfpathlineto{\pgfqpoint{1.096274in}{0.622296in}}%
\pgfpathlineto{\pgfqpoint{1.089235in}{0.617592in}}%
\pgfpathlineto{\pgfqpoint{1.080617in}{0.610099in}}%
\pgfpathlineto{\pgfqpoint{1.075205in}{0.603981in}}%
\pgfpathlineto{\pgfqpoint{1.067105in}{0.590370in}}%
\pgfpathlineto{\pgfqpoint{1.064960in}{0.583178in}}%
\pgfpathlineto{\pgfqpoint{1.063348in}{0.576759in}}%
\pgfpathlineto{\pgfqpoint{1.063348in}{0.563148in}}%
\pgfpathlineto{\pgfqpoint{1.064960in}{0.556728in}}%
\pgfpathlineto{\pgfqpoint{1.067105in}{0.549536in}}%
\pgfpathlineto{\pgfqpoint{1.075205in}{0.535925in}}%
\pgfpathlineto{\pgfqpoint{1.080617in}{0.529807in}}%
\pgfpathlineto{\pgfqpoint{1.089235in}{0.522314in}}%
\pgfpathlineto{\pgfqpoint{1.096274in}{0.517610in}}%
\pgfpathlineto{\pgfqpoint{1.111930in}{0.510567in}}%
\pgfpathlineto{\pgfqpoint{1.120202in}{0.508703in}}%
\pgfpathlineto{\pgfqpoint{1.127587in}{0.507302in}}%
\pgfpathclose%
\pgfpathmoveto{\pgfqpoint{1.644253in}{0.507302in}}%
\pgfpathlineto{\pgfqpoint{1.659910in}{0.507302in}}%
\pgfpathlineto{\pgfqpoint{1.667294in}{0.508703in}}%
\pgfpathlineto{\pgfqpoint{1.675567in}{0.510567in}}%
\pgfpathlineto{\pgfqpoint{1.691223in}{0.517610in}}%
\pgfpathlineto{\pgfqpoint{1.698261in}{0.522314in}}%
\pgfpathlineto{\pgfqpoint{1.706880in}{0.529807in}}%
\pgfpathlineto{\pgfqpoint{1.712291in}{0.535925in}}%
\pgfpathlineto{\pgfqpoint{1.720392in}{0.549536in}}%
\pgfpathlineto{\pgfqpoint{1.722536in}{0.556728in}}%
\pgfpathlineto{\pgfqpoint{1.724148in}{0.563148in}}%
\pgfpathlineto{\pgfqpoint{1.724148in}{0.576759in}}%
\pgfpathlineto{\pgfqpoint{1.722536in}{0.583178in}}%
\pgfpathlineto{\pgfqpoint{1.720392in}{0.590370in}}%
\pgfpathlineto{\pgfqpoint{1.712291in}{0.603981in}}%
\pgfpathlineto{\pgfqpoint{1.706880in}{0.610099in}}%
\pgfpathlineto{\pgfqpoint{1.698261in}{0.617592in}}%
\pgfpathlineto{\pgfqpoint{1.691223in}{0.622296in}}%
\pgfpathlineto{\pgfqpoint{1.675567in}{0.629339in}}%
\pgfpathlineto{\pgfqpoint{1.667294in}{0.631203in}}%
\pgfpathlineto{\pgfqpoint{1.659910in}{0.632604in}}%
\pgfpathlineto{\pgfqpoint{1.644253in}{0.632604in}}%
\pgfpathlineto{\pgfqpoint{1.636869in}{0.631203in}}%
\pgfpathlineto{\pgfqpoint{1.628597in}{0.629339in}}%
\pgfpathlineto{\pgfqpoint{1.612940in}{0.622296in}}%
\pgfpathlineto{\pgfqpoint{1.605902in}{0.617592in}}%
\pgfpathlineto{\pgfqpoint{1.597284in}{0.610099in}}%
\pgfpathlineto{\pgfqpoint{1.591872in}{0.603981in}}%
\pgfpathlineto{\pgfqpoint{1.583771in}{0.590370in}}%
\pgfpathlineto{\pgfqpoint{1.581627in}{0.583178in}}%
\pgfpathlineto{\pgfqpoint{1.580015in}{0.576759in}}%
\pgfpathlineto{\pgfqpoint{1.580015in}{0.563148in}}%
\pgfpathlineto{\pgfqpoint{1.581627in}{0.556728in}}%
\pgfpathlineto{\pgfqpoint{1.583771in}{0.549536in}}%
\pgfpathlineto{\pgfqpoint{1.591872in}{0.535925in}}%
\pgfpathlineto{\pgfqpoint{1.597284in}{0.529807in}}%
\pgfpathlineto{\pgfqpoint{1.605902in}{0.522314in}}%
\pgfpathlineto{\pgfqpoint{1.612940in}{0.517610in}}%
\pgfpathlineto{\pgfqpoint{1.628597in}{0.510567in}}%
\pgfpathlineto{\pgfqpoint{1.636869in}{0.508703in}}%
\pgfpathlineto{\pgfqpoint{1.644253in}{0.507302in}}%
\pgfpathclose%
\pgfpathmoveto{\pgfqpoint{0.610920in}{0.956468in}}%
\pgfpathlineto{\pgfqpoint{0.626577in}{0.956468in}}%
\pgfpathlineto{\pgfqpoint{0.633961in}{0.957870in}}%
\pgfpathlineto{\pgfqpoint{0.642233in}{0.959734in}}%
\pgfpathlineto{\pgfqpoint{0.657890in}{0.966776in}}%
\pgfpathlineto{\pgfqpoint{0.664928in}{0.971481in}}%
\pgfpathlineto{\pgfqpoint{0.673546in}{0.978973in}}%
\pgfpathlineto{\pgfqpoint{0.678958in}{0.985092in}}%
\pgfpathlineto{\pgfqpoint{0.687059in}{0.998703in}}%
\pgfpathlineto{\pgfqpoint{0.689203in}{1.005895in}}%
\pgfpathlineto{\pgfqpoint{0.690815in}{1.012314in}}%
\pgfpathlineto{\pgfqpoint{0.690815in}{1.025925in}}%
\pgfpathlineto{\pgfqpoint{0.689203in}{1.032345in}}%
\pgfpathlineto{\pgfqpoint{0.687059in}{1.039536in}}%
\pgfpathlineto{\pgfqpoint{0.678958in}{1.053148in}}%
\pgfpathlineto{\pgfqpoint{0.673546in}{1.059266in}}%
\pgfpathlineto{\pgfqpoint{0.664928in}{1.066759in}}%
\pgfpathlineto{\pgfqpoint{0.657890in}{1.071463in}}%
\pgfpathlineto{\pgfqpoint{0.642233in}{1.078506in}}%
\pgfpathlineto{\pgfqpoint{0.633961in}{1.080370in}}%
\pgfpathlineto{\pgfqpoint{0.626577in}{1.081771in}}%
\pgfpathlineto{\pgfqpoint{0.610920in}{1.081771in}}%
\pgfpathlineto{\pgfqpoint{0.603536in}{1.080370in}}%
\pgfpathlineto{\pgfqpoint{0.595263in}{1.078506in}}%
\pgfpathlineto{\pgfqpoint{0.579607in}{1.071463in}}%
\pgfpathlineto{\pgfqpoint{0.572569in}{1.066759in}}%
\pgfpathlineto{\pgfqpoint{0.563950in}{1.059266in}}%
\pgfpathlineto{\pgfqpoint{0.558539in}{1.053148in}}%
\pgfpathlineto{\pgfqpoint{0.550438in}{1.039536in}}%
\pgfpathlineto{\pgfqpoint{0.548294in}{1.032345in}}%
\pgfpathlineto{\pgfqpoint{0.546682in}{1.025925in}}%
\pgfpathlineto{\pgfqpoint{0.546682in}{1.012314in}}%
\pgfpathlineto{\pgfqpoint{0.548294in}{1.005895in}}%
\pgfpathlineto{\pgfqpoint{0.550438in}{0.998703in}}%
\pgfpathlineto{\pgfqpoint{0.558539in}{0.985092in}}%
\pgfpathlineto{\pgfqpoint{0.563950in}{0.978973in}}%
\pgfpathlineto{\pgfqpoint{0.572569in}{0.971481in}}%
\pgfpathlineto{\pgfqpoint{0.579607in}{0.966776in}}%
\pgfpathlineto{\pgfqpoint{0.595263in}{0.959734in}}%
\pgfpathlineto{\pgfqpoint{0.603536in}{0.957870in}}%
\pgfpathlineto{\pgfqpoint{0.610920in}{0.956468in}}%
\pgfpathclose%
\pgfpathmoveto{\pgfqpoint{1.127587in}{0.956468in}}%
\pgfpathlineto{\pgfqpoint{1.143243in}{0.956468in}}%
\pgfpathlineto{\pgfqpoint{1.150628in}{0.957870in}}%
\pgfpathlineto{\pgfqpoint{1.158900in}{0.959734in}}%
\pgfpathlineto{\pgfqpoint{1.174556in}{0.966776in}}%
\pgfpathlineto{\pgfqpoint{1.181595in}{0.971481in}}%
\pgfpathlineto{\pgfqpoint{1.190213in}{0.978973in}}%
\pgfpathlineto{\pgfqpoint{1.195625in}{0.985092in}}%
\pgfpathlineto{\pgfqpoint{1.203725in}{0.998703in}}%
\pgfpathlineto{\pgfqpoint{1.205870in}{1.005895in}}%
\pgfpathlineto{\pgfqpoint{1.207482in}{1.012314in}}%
\pgfpathlineto{\pgfqpoint{1.207482in}{1.025925in}}%
\pgfpathlineto{\pgfqpoint{1.205870in}{1.032345in}}%
\pgfpathlineto{\pgfqpoint{1.203725in}{1.039536in}}%
\pgfpathlineto{\pgfqpoint{1.195625in}{1.053148in}}%
\pgfpathlineto{\pgfqpoint{1.190213in}{1.059266in}}%
\pgfpathlineto{\pgfqpoint{1.181595in}{1.066759in}}%
\pgfpathlineto{\pgfqpoint{1.174556in}{1.071463in}}%
\pgfpathlineto{\pgfqpoint{1.158900in}{1.078506in}}%
\pgfpathlineto{\pgfqpoint{1.150628in}{1.080370in}}%
\pgfpathlineto{\pgfqpoint{1.143243in}{1.081771in}}%
\pgfpathlineto{\pgfqpoint{1.127587in}{1.081771in}}%
\pgfpathlineto{\pgfqpoint{1.120202in}{1.080370in}}%
\pgfpathlineto{\pgfqpoint{1.111930in}{1.078506in}}%
\pgfpathlineto{\pgfqpoint{1.096274in}{1.071463in}}%
\pgfpathlineto{\pgfqpoint{1.089235in}{1.066759in}}%
\pgfpathlineto{\pgfqpoint{1.080617in}{1.059266in}}%
\pgfpathlineto{\pgfqpoint{1.075205in}{1.053148in}}%
\pgfpathlineto{\pgfqpoint{1.067105in}{1.039536in}}%
\pgfpathlineto{\pgfqpoint{1.064960in}{1.032345in}}%
\pgfpathlineto{\pgfqpoint{1.063348in}{1.025925in}}%
\pgfpathlineto{\pgfqpoint{1.063348in}{1.012314in}}%
\pgfpathlineto{\pgfqpoint{1.064960in}{1.005895in}}%
\pgfpathlineto{\pgfqpoint{1.067105in}{0.998703in}}%
\pgfpathlineto{\pgfqpoint{1.075205in}{0.985092in}}%
\pgfpathlineto{\pgfqpoint{1.080617in}{0.978973in}}%
\pgfpathlineto{\pgfqpoint{1.089235in}{0.971481in}}%
\pgfpathlineto{\pgfqpoint{1.096274in}{0.966776in}}%
\pgfpathlineto{\pgfqpoint{1.111930in}{0.959734in}}%
\pgfpathlineto{\pgfqpoint{1.120202in}{0.957870in}}%
\pgfpathlineto{\pgfqpoint{1.127587in}{0.956468in}}%
\pgfpathclose%
\pgfpathmoveto{\pgfqpoint{1.644253in}{0.956468in}}%
\pgfpathlineto{\pgfqpoint{1.659910in}{0.956468in}}%
\pgfpathlineto{\pgfqpoint{1.667294in}{0.957870in}}%
\pgfpathlineto{\pgfqpoint{1.675567in}{0.959734in}}%
\pgfpathlineto{\pgfqpoint{1.691223in}{0.966776in}}%
\pgfpathlineto{\pgfqpoint{1.698261in}{0.971481in}}%
\pgfpathlineto{\pgfqpoint{1.706880in}{0.978973in}}%
\pgfpathlineto{\pgfqpoint{1.712291in}{0.985092in}}%
\pgfpathlineto{\pgfqpoint{1.720392in}{0.998703in}}%
\pgfpathlineto{\pgfqpoint{1.722536in}{1.005895in}}%
\pgfpathlineto{\pgfqpoint{1.724148in}{1.012314in}}%
\pgfpathlineto{\pgfqpoint{1.724148in}{1.025925in}}%
\pgfpathlineto{\pgfqpoint{1.722536in}{1.032345in}}%
\pgfpathlineto{\pgfqpoint{1.720392in}{1.039536in}}%
\pgfpathlineto{\pgfqpoint{1.712291in}{1.053148in}}%
\pgfpathlineto{\pgfqpoint{1.706880in}{1.059266in}}%
\pgfpathlineto{\pgfqpoint{1.698261in}{1.066759in}}%
\pgfpathlineto{\pgfqpoint{1.691223in}{1.071463in}}%
\pgfpathlineto{\pgfqpoint{1.675567in}{1.078506in}}%
\pgfpathlineto{\pgfqpoint{1.667294in}{1.080370in}}%
\pgfpathlineto{\pgfqpoint{1.659910in}{1.081771in}}%
\pgfpathlineto{\pgfqpoint{1.644253in}{1.081771in}}%
\pgfpathlineto{\pgfqpoint{1.636869in}{1.080370in}}%
\pgfpathlineto{\pgfqpoint{1.628597in}{1.078506in}}%
\pgfpathlineto{\pgfqpoint{1.612940in}{1.071463in}}%
\pgfpathlineto{\pgfqpoint{1.605902in}{1.066759in}}%
\pgfpathlineto{\pgfqpoint{1.597284in}{1.059266in}}%
\pgfpathlineto{\pgfqpoint{1.591872in}{1.053148in}}%
\pgfpathlineto{\pgfqpoint{1.583771in}{1.039536in}}%
\pgfpathlineto{\pgfqpoint{1.581627in}{1.032345in}}%
\pgfpathlineto{\pgfqpoint{1.580015in}{1.025925in}}%
\pgfpathlineto{\pgfqpoint{1.580015in}{1.012314in}}%
\pgfpathlineto{\pgfqpoint{1.581627in}{1.005895in}}%
\pgfpathlineto{\pgfqpoint{1.583771in}{0.998703in}}%
\pgfpathlineto{\pgfqpoint{1.591872in}{0.985092in}}%
\pgfpathlineto{\pgfqpoint{1.597284in}{0.978973in}}%
\pgfpathlineto{\pgfqpoint{1.605902in}{0.971481in}}%
\pgfpathlineto{\pgfqpoint{1.612940in}{0.966776in}}%
\pgfpathlineto{\pgfqpoint{1.628597in}{0.959734in}}%
\pgfpathlineto{\pgfqpoint{1.636869in}{0.957870in}}%
\pgfpathlineto{\pgfqpoint{1.644253in}{0.956468in}}%
\pgfpathclose%
\pgfpathmoveto{\pgfqpoint{0.610920in}{1.405635in}}%
\pgfpathlineto{\pgfqpoint{0.626577in}{1.405635in}}%
\pgfpathlineto{\pgfqpoint{0.633961in}{1.407036in}}%
\pgfpathlineto{\pgfqpoint{0.642233in}{1.408900in}}%
\pgfpathlineto{\pgfqpoint{0.657890in}{1.415943in}}%
\pgfpathlineto{\pgfqpoint{0.664928in}{1.420648in}}%
\pgfpathlineto{\pgfqpoint{0.673546in}{1.428140in}}%
\pgfpathlineto{\pgfqpoint{0.678958in}{1.434259in}}%
\pgfpathlineto{\pgfqpoint{0.687059in}{1.447870in}}%
\pgfpathlineto{\pgfqpoint{0.689203in}{1.455061in}}%
\pgfpathlineto{\pgfqpoint{0.690815in}{1.461481in}}%
\pgfpathlineto{\pgfqpoint{0.690815in}{1.475092in}}%
\pgfpathlineto{\pgfqpoint{0.689203in}{1.481511in}}%
\pgfpathlineto{\pgfqpoint{0.687059in}{1.488703in}}%
\pgfpathlineto{\pgfqpoint{0.678958in}{1.502314in}}%
\pgfpathlineto{\pgfqpoint{0.673546in}{1.508433in}}%
\pgfpathlineto{\pgfqpoint{0.664928in}{1.515925in}}%
\pgfpathlineto{\pgfqpoint{0.657890in}{1.520630in}}%
\pgfpathlineto{\pgfqpoint{0.642233in}{1.527672in}}%
\pgfpathlineto{\pgfqpoint{0.633961in}{1.529536in}}%
\pgfpathlineto{\pgfqpoint{0.626577in}{1.530938in}}%
\pgfpathlineto{\pgfqpoint{0.610920in}{1.530938in}}%
\pgfpathlineto{\pgfqpoint{0.603536in}{1.529536in}}%
\pgfpathlineto{\pgfqpoint{0.595263in}{1.527672in}}%
\pgfpathlineto{\pgfqpoint{0.579607in}{1.520630in}}%
\pgfpathlineto{\pgfqpoint{0.572569in}{1.515925in}}%
\pgfpathlineto{\pgfqpoint{0.563950in}{1.508433in}}%
\pgfpathlineto{\pgfqpoint{0.558539in}{1.502314in}}%
\pgfpathlineto{\pgfqpoint{0.550438in}{1.488703in}}%
\pgfpathlineto{\pgfqpoint{0.548294in}{1.481511in}}%
\pgfpathlineto{\pgfqpoint{0.546682in}{1.475092in}}%
\pgfpathlineto{\pgfqpoint{0.546682in}{1.461481in}}%
\pgfpathlineto{\pgfqpoint{0.548294in}{1.455061in}}%
\pgfpathlineto{\pgfqpoint{0.550438in}{1.447870in}}%
\pgfpathlineto{\pgfqpoint{0.558539in}{1.434259in}}%
\pgfpathlineto{\pgfqpoint{0.563950in}{1.428140in}}%
\pgfpathlineto{\pgfqpoint{0.572569in}{1.420648in}}%
\pgfpathlineto{\pgfqpoint{0.579607in}{1.415943in}}%
\pgfpathlineto{\pgfqpoint{0.595263in}{1.408900in}}%
\pgfpathlineto{\pgfqpoint{0.603536in}{1.407036in}}%
\pgfpathlineto{\pgfqpoint{0.610920in}{1.405635in}}%
\pgfpathclose%
\pgfpathmoveto{\pgfqpoint{1.127587in}{1.405635in}}%
\pgfpathlineto{\pgfqpoint{1.143243in}{1.405635in}}%
\pgfpathlineto{\pgfqpoint{1.150628in}{1.407036in}}%
\pgfpathlineto{\pgfqpoint{1.158900in}{1.408900in}}%
\pgfpathlineto{\pgfqpoint{1.174556in}{1.415943in}}%
\pgfpathlineto{\pgfqpoint{1.181595in}{1.420648in}}%
\pgfpathlineto{\pgfqpoint{1.190213in}{1.428140in}}%
\pgfpathlineto{\pgfqpoint{1.195625in}{1.434259in}}%
\pgfpathlineto{\pgfqpoint{1.203725in}{1.447870in}}%
\pgfpathlineto{\pgfqpoint{1.205870in}{1.455061in}}%
\pgfpathlineto{\pgfqpoint{1.207482in}{1.461481in}}%
\pgfpathlineto{\pgfqpoint{1.207482in}{1.475092in}}%
\pgfpathlineto{\pgfqpoint{1.205870in}{1.481511in}}%
\pgfpathlineto{\pgfqpoint{1.203725in}{1.488703in}}%
\pgfpathlineto{\pgfqpoint{1.195625in}{1.502314in}}%
\pgfpathlineto{\pgfqpoint{1.190213in}{1.508433in}}%
\pgfpathlineto{\pgfqpoint{1.181595in}{1.515925in}}%
\pgfpathlineto{\pgfqpoint{1.174556in}{1.520630in}}%
\pgfpathlineto{\pgfqpoint{1.158900in}{1.527672in}}%
\pgfpathlineto{\pgfqpoint{1.150628in}{1.529536in}}%
\pgfpathlineto{\pgfqpoint{1.143243in}{1.530938in}}%
\pgfpathlineto{\pgfqpoint{1.127587in}{1.530938in}}%
\pgfpathlineto{\pgfqpoint{1.120202in}{1.529536in}}%
\pgfpathlineto{\pgfqpoint{1.111930in}{1.527672in}}%
\pgfpathlineto{\pgfqpoint{1.096274in}{1.520630in}}%
\pgfpathlineto{\pgfqpoint{1.089235in}{1.515925in}}%
\pgfpathlineto{\pgfqpoint{1.080617in}{1.508433in}}%
\pgfpathlineto{\pgfqpoint{1.075205in}{1.502314in}}%
\pgfpathlineto{\pgfqpoint{1.067105in}{1.488703in}}%
\pgfpathlineto{\pgfqpoint{1.064960in}{1.481511in}}%
\pgfpathlineto{\pgfqpoint{1.063348in}{1.475092in}}%
\pgfpathlineto{\pgfqpoint{1.063348in}{1.461481in}}%
\pgfpathlineto{\pgfqpoint{1.064960in}{1.455061in}}%
\pgfpathlineto{\pgfqpoint{1.067105in}{1.447870in}}%
\pgfpathlineto{\pgfqpoint{1.075205in}{1.434259in}}%
\pgfpathlineto{\pgfqpoint{1.080617in}{1.428140in}}%
\pgfpathlineto{\pgfqpoint{1.089235in}{1.420648in}}%
\pgfpathlineto{\pgfqpoint{1.096274in}{1.415943in}}%
\pgfpathlineto{\pgfqpoint{1.111930in}{1.408900in}}%
\pgfpathlineto{\pgfqpoint{1.120202in}{1.407036in}}%
\pgfpathlineto{\pgfqpoint{1.127587in}{1.405635in}}%
\pgfpathclose%
\pgfpathmoveto{\pgfqpoint{1.644253in}{1.405635in}}%
\pgfpathlineto{\pgfqpoint{1.659910in}{1.405635in}}%
\pgfpathlineto{\pgfqpoint{1.667294in}{1.407036in}}%
\pgfpathlineto{\pgfqpoint{1.675567in}{1.408900in}}%
\pgfpathlineto{\pgfqpoint{1.691223in}{1.415943in}}%
\pgfpathlineto{\pgfqpoint{1.698261in}{1.420648in}}%
\pgfpathlineto{\pgfqpoint{1.706880in}{1.428140in}}%
\pgfpathlineto{\pgfqpoint{1.712291in}{1.434259in}}%
\pgfpathlineto{\pgfqpoint{1.720392in}{1.447870in}}%
\pgfpathlineto{\pgfqpoint{1.722536in}{1.455061in}}%
\pgfpathlineto{\pgfqpoint{1.724148in}{1.461481in}}%
\pgfpathlineto{\pgfqpoint{1.724148in}{1.475092in}}%
\pgfpathlineto{\pgfqpoint{1.722536in}{1.481511in}}%
\pgfpathlineto{\pgfqpoint{1.720392in}{1.488703in}}%
\pgfpathlineto{\pgfqpoint{1.712291in}{1.502314in}}%
\pgfpathlineto{\pgfqpoint{1.706880in}{1.508433in}}%
\pgfpathlineto{\pgfqpoint{1.698261in}{1.515925in}}%
\pgfpathlineto{\pgfqpoint{1.691223in}{1.520630in}}%
\pgfpathlineto{\pgfqpoint{1.675567in}{1.527672in}}%
\pgfpathlineto{\pgfqpoint{1.667294in}{1.529536in}}%
\pgfpathlineto{\pgfqpoint{1.659910in}{1.530938in}}%
\pgfpathlineto{\pgfqpoint{1.644253in}{1.530938in}}%
\pgfpathlineto{\pgfqpoint{1.636869in}{1.529536in}}%
\pgfpathlineto{\pgfqpoint{1.628597in}{1.527672in}}%
\pgfpathlineto{\pgfqpoint{1.612940in}{1.520630in}}%
\pgfpathlineto{\pgfqpoint{1.605902in}{1.515925in}}%
\pgfpathlineto{\pgfqpoint{1.597284in}{1.508433in}}%
\pgfpathlineto{\pgfqpoint{1.591872in}{1.502314in}}%
\pgfpathlineto{\pgfqpoint{1.583771in}{1.488703in}}%
\pgfpathlineto{\pgfqpoint{1.581627in}{1.481511in}}%
\pgfpathlineto{\pgfqpoint{1.580015in}{1.475092in}}%
\pgfpathlineto{\pgfqpoint{1.580015in}{1.461481in}}%
\pgfpathlineto{\pgfqpoint{1.581627in}{1.455061in}}%
\pgfpathlineto{\pgfqpoint{1.583771in}{1.447870in}}%
\pgfpathlineto{\pgfqpoint{1.591872in}{1.434259in}}%
\pgfpathlineto{\pgfqpoint{1.597284in}{1.428140in}}%
\pgfpathlineto{\pgfqpoint{1.605902in}{1.420648in}}%
\pgfpathlineto{\pgfqpoint{1.612940in}{1.415943in}}%
\pgfpathlineto{\pgfqpoint{1.628597in}{1.408900in}}%
\pgfpathlineto{\pgfqpoint{1.636869in}{1.407036in}}%
\pgfpathlineto{\pgfqpoint{1.644253in}{1.405635in}}%
\pgfpathclose%
\pgfusepath{fill}%
\end{pgfscope}%
\begin{pgfscope}%
\pgfpathrectangle{\pgfqpoint{0.360415in}{0.345370in}}{\pgfqpoint{1.550000in}{1.347500in}}%
\pgfusepath{clip}%
\pgfsetbuttcap%
\pgfsetroundjoin%
\definecolor{currentfill}{rgb}{0.993814,0.704741,0.183043}%
\pgfsetfillcolor{currentfill}%
\pgfsetlinewidth{0.000000pt}%
\definecolor{currentstroke}{rgb}{0.000000,0.000000,0.000000}%
\pgfsetstrokecolor{currentstroke}%
\pgfsetdash{}{0pt}%
\pgfpathmoveto{\pgfqpoint{0.595263in}{0.477817in}}%
\pgfpathlineto{\pgfqpoint{0.610920in}{0.475505in}}%
\pgfpathlineto{\pgfqpoint{0.626577in}{0.475505in}}%
\pgfpathlineto{\pgfqpoint{0.642233in}{0.477817in}}%
\pgfpathlineto{\pgfqpoint{0.654661in}{0.481481in}}%
\pgfpathlineto{\pgfqpoint{0.657890in}{0.482510in}}%
\pgfpathlineto{\pgfqpoint{0.673546in}{0.489931in}}%
\pgfpathlineto{\pgfqpoint{0.681809in}{0.495092in}}%
\pgfpathlineto{\pgfqpoint{0.689203in}{0.500377in}}%
\pgfpathlineto{\pgfqpoint{0.698780in}{0.508703in}}%
\pgfpathlineto{\pgfqpoint{0.704859in}{0.515131in}}%
\pgfpathlineto{\pgfqpoint{0.710796in}{0.522314in}}%
\pgfpathlineto{\pgfqpoint{0.719332in}{0.535925in}}%
\pgfpathlineto{\pgfqpoint{0.720516in}{0.538732in}}%
\pgfpathlineto{\pgfqpoint{0.724731in}{0.549536in}}%
\pgfpathlineto{\pgfqpoint{0.727390in}{0.563148in}}%
\pgfpathlineto{\pgfqpoint{0.727390in}{0.576759in}}%
\pgfpathlineto{\pgfqpoint{0.724731in}{0.590370in}}%
\pgfpathlineto{\pgfqpoint{0.720516in}{0.601174in}}%
\pgfpathlineto{\pgfqpoint{0.719332in}{0.603981in}}%
\pgfpathlineto{\pgfqpoint{0.710796in}{0.617592in}}%
\pgfpathlineto{\pgfqpoint{0.704859in}{0.624775in}}%
\pgfpathlineto{\pgfqpoint{0.698780in}{0.631203in}}%
\pgfpathlineto{\pgfqpoint{0.689203in}{0.639529in}}%
\pgfpathlineto{\pgfqpoint{0.681809in}{0.644814in}}%
\pgfpathlineto{\pgfqpoint{0.673546in}{0.649976in}}%
\pgfpathlineto{\pgfqpoint{0.657890in}{0.657396in}}%
\pgfpathlineto{\pgfqpoint{0.654661in}{0.658425in}}%
\pgfpathlineto{\pgfqpoint{0.642233in}{0.662089in}}%
\pgfpathlineto{\pgfqpoint{0.626577in}{0.664401in}}%
\pgfpathlineto{\pgfqpoint{0.610920in}{0.664401in}}%
\pgfpathlineto{\pgfqpoint{0.595263in}{0.662089in}}%
\pgfpathlineto{\pgfqpoint{0.582836in}{0.658425in}}%
\pgfpathlineto{\pgfqpoint{0.579607in}{0.657396in}}%
\pgfpathlineto{\pgfqpoint{0.563950in}{0.649976in}}%
\pgfpathlineto{\pgfqpoint{0.555688in}{0.644814in}}%
\pgfpathlineto{\pgfqpoint{0.548294in}{0.639529in}}%
\pgfpathlineto{\pgfqpoint{0.538716in}{0.631203in}}%
\pgfpathlineto{\pgfqpoint{0.532637in}{0.624775in}}%
\pgfpathlineto{\pgfqpoint{0.526700in}{0.617592in}}%
\pgfpathlineto{\pgfqpoint{0.518165in}{0.603981in}}%
\pgfpathlineto{\pgfqpoint{0.516981in}{0.601174in}}%
\pgfpathlineto{\pgfqpoint{0.512766in}{0.590370in}}%
\pgfpathlineto{\pgfqpoint{0.510107in}{0.576759in}}%
\pgfpathlineto{\pgfqpoint{0.510107in}{0.563148in}}%
\pgfpathlineto{\pgfqpoint{0.512766in}{0.549536in}}%
\pgfpathlineto{\pgfqpoint{0.516981in}{0.538732in}}%
\pgfpathlineto{\pgfqpoint{0.518165in}{0.535925in}}%
\pgfpathlineto{\pgfqpoint{0.526700in}{0.522314in}}%
\pgfpathlineto{\pgfqpoint{0.532637in}{0.515131in}}%
\pgfpathlineto{\pgfqpoint{0.538716in}{0.508703in}}%
\pgfpathlineto{\pgfqpoint{0.548294in}{0.500377in}}%
\pgfpathlineto{\pgfqpoint{0.555688in}{0.495092in}}%
\pgfpathlineto{\pgfqpoint{0.563950in}{0.489931in}}%
\pgfpathlineto{\pgfqpoint{0.579607in}{0.482510in}}%
\pgfpathlineto{\pgfqpoint{0.582836in}{0.481481in}}%
\pgfpathlineto{\pgfqpoint{0.595263in}{0.477817in}}%
\pgfpathclose%
\pgfpathmoveto{\pgfqpoint{0.603536in}{0.508703in}}%
\pgfpathlineto{\pgfqpoint{0.595263in}{0.510567in}}%
\pgfpathlineto{\pgfqpoint{0.579607in}{0.517610in}}%
\pgfpathlineto{\pgfqpoint{0.572569in}{0.522314in}}%
\pgfpathlineto{\pgfqpoint{0.563950in}{0.529807in}}%
\pgfpathlineto{\pgfqpoint{0.558539in}{0.535925in}}%
\pgfpathlineto{\pgfqpoint{0.550438in}{0.549536in}}%
\pgfpathlineto{\pgfqpoint{0.548294in}{0.556728in}}%
\pgfpathlineto{\pgfqpoint{0.546682in}{0.563148in}}%
\pgfpathlineto{\pgfqpoint{0.546682in}{0.576759in}}%
\pgfpathlineto{\pgfqpoint{0.548294in}{0.583178in}}%
\pgfpathlineto{\pgfqpoint{0.550438in}{0.590370in}}%
\pgfpathlineto{\pgfqpoint{0.558539in}{0.603981in}}%
\pgfpathlineto{\pgfqpoint{0.563950in}{0.610099in}}%
\pgfpathlineto{\pgfqpoint{0.572569in}{0.617592in}}%
\pgfpathlineto{\pgfqpoint{0.579607in}{0.622296in}}%
\pgfpathlineto{\pgfqpoint{0.595263in}{0.629339in}}%
\pgfpathlineto{\pgfqpoint{0.603536in}{0.631203in}}%
\pgfpathlineto{\pgfqpoint{0.610920in}{0.632604in}}%
\pgfpathlineto{\pgfqpoint{0.626577in}{0.632604in}}%
\pgfpathlineto{\pgfqpoint{0.633961in}{0.631203in}}%
\pgfpathlineto{\pgfqpoint{0.642233in}{0.629339in}}%
\pgfpathlineto{\pgfqpoint{0.657890in}{0.622296in}}%
\pgfpathlineto{\pgfqpoint{0.664928in}{0.617592in}}%
\pgfpathlineto{\pgfqpoint{0.673546in}{0.610099in}}%
\pgfpathlineto{\pgfqpoint{0.678958in}{0.603981in}}%
\pgfpathlineto{\pgfqpoint{0.687059in}{0.590370in}}%
\pgfpathlineto{\pgfqpoint{0.689203in}{0.583178in}}%
\pgfpathlineto{\pgfqpoint{0.690815in}{0.576759in}}%
\pgfpathlineto{\pgfqpoint{0.690815in}{0.563148in}}%
\pgfpathlineto{\pgfqpoint{0.689203in}{0.556728in}}%
\pgfpathlineto{\pgfqpoint{0.687059in}{0.549536in}}%
\pgfpathlineto{\pgfqpoint{0.678958in}{0.535925in}}%
\pgfpathlineto{\pgfqpoint{0.673546in}{0.529807in}}%
\pgfpathlineto{\pgfqpoint{0.664928in}{0.522314in}}%
\pgfpathlineto{\pgfqpoint{0.657890in}{0.517610in}}%
\pgfpathlineto{\pgfqpoint{0.642233in}{0.510567in}}%
\pgfpathlineto{\pgfqpoint{0.633961in}{0.508703in}}%
\pgfpathlineto{\pgfqpoint{0.626577in}{0.507302in}}%
\pgfpathlineto{\pgfqpoint{0.610920in}{0.507302in}}%
\pgfpathlineto{\pgfqpoint{0.603536in}{0.508703in}}%
\pgfpathclose%
\pgfpathmoveto{\pgfqpoint{1.111930in}{0.477817in}}%
\pgfpathlineto{\pgfqpoint{1.127587in}{0.475505in}}%
\pgfpathlineto{\pgfqpoint{1.143243in}{0.475505in}}%
\pgfpathlineto{\pgfqpoint{1.158900in}{0.477817in}}%
\pgfpathlineto{\pgfqpoint{1.171328in}{0.481481in}}%
\pgfpathlineto{\pgfqpoint{1.174556in}{0.482510in}}%
\pgfpathlineto{\pgfqpoint{1.190213in}{0.489931in}}%
\pgfpathlineto{\pgfqpoint{1.198476in}{0.495092in}}%
\pgfpathlineto{\pgfqpoint{1.205870in}{0.500377in}}%
\pgfpathlineto{\pgfqpoint{1.215447in}{0.508703in}}%
\pgfpathlineto{\pgfqpoint{1.221526in}{0.515131in}}%
\pgfpathlineto{\pgfqpoint{1.227463in}{0.522314in}}%
\pgfpathlineto{\pgfqpoint{1.235999in}{0.535925in}}%
\pgfpathlineto{\pgfqpoint{1.237183in}{0.538732in}}%
\pgfpathlineto{\pgfqpoint{1.241397in}{0.549536in}}%
\pgfpathlineto{\pgfqpoint{1.244056in}{0.563148in}}%
\pgfpathlineto{\pgfqpoint{1.244056in}{0.576759in}}%
\pgfpathlineto{\pgfqpoint{1.241397in}{0.590370in}}%
\pgfpathlineto{\pgfqpoint{1.237183in}{0.601174in}}%
\pgfpathlineto{\pgfqpoint{1.235999in}{0.603981in}}%
\pgfpathlineto{\pgfqpoint{1.227463in}{0.617592in}}%
\pgfpathlineto{\pgfqpoint{1.221526in}{0.624775in}}%
\pgfpathlineto{\pgfqpoint{1.215447in}{0.631203in}}%
\pgfpathlineto{\pgfqpoint{1.205870in}{0.639529in}}%
\pgfpathlineto{\pgfqpoint{1.198476in}{0.644814in}}%
\pgfpathlineto{\pgfqpoint{1.190213in}{0.649976in}}%
\pgfpathlineto{\pgfqpoint{1.174556in}{0.657396in}}%
\pgfpathlineto{\pgfqpoint{1.171328in}{0.658425in}}%
\pgfpathlineto{\pgfqpoint{1.158900in}{0.662089in}}%
\pgfpathlineto{\pgfqpoint{1.143243in}{0.664401in}}%
\pgfpathlineto{\pgfqpoint{1.127587in}{0.664401in}}%
\pgfpathlineto{\pgfqpoint{1.111930in}{0.662089in}}%
\pgfpathlineto{\pgfqpoint{1.099502in}{0.658425in}}%
\pgfpathlineto{\pgfqpoint{1.096274in}{0.657396in}}%
\pgfpathlineto{\pgfqpoint{1.080617in}{0.649976in}}%
\pgfpathlineto{\pgfqpoint{1.072354in}{0.644814in}}%
\pgfpathlineto{\pgfqpoint{1.064960in}{0.639529in}}%
\pgfpathlineto{\pgfqpoint{1.055383in}{0.631203in}}%
\pgfpathlineto{\pgfqpoint{1.049304in}{0.624775in}}%
\pgfpathlineto{\pgfqpoint{1.043367in}{0.617592in}}%
\pgfpathlineto{\pgfqpoint{1.034831in}{0.603981in}}%
\pgfpathlineto{\pgfqpoint{1.033647in}{0.601174in}}%
\pgfpathlineto{\pgfqpoint{1.029433in}{0.590370in}}%
\pgfpathlineto{\pgfqpoint{1.026774in}{0.576759in}}%
\pgfpathlineto{\pgfqpoint{1.026774in}{0.563148in}}%
\pgfpathlineto{\pgfqpoint{1.029433in}{0.549536in}}%
\pgfpathlineto{\pgfqpoint{1.033647in}{0.538732in}}%
\pgfpathlineto{\pgfqpoint{1.034831in}{0.535925in}}%
\pgfpathlineto{\pgfqpoint{1.043367in}{0.522314in}}%
\pgfpathlineto{\pgfqpoint{1.049304in}{0.515131in}}%
\pgfpathlineto{\pgfqpoint{1.055383in}{0.508703in}}%
\pgfpathlineto{\pgfqpoint{1.064960in}{0.500377in}}%
\pgfpathlineto{\pgfqpoint{1.072354in}{0.495092in}}%
\pgfpathlineto{\pgfqpoint{1.080617in}{0.489931in}}%
\pgfpathlineto{\pgfqpoint{1.096274in}{0.482510in}}%
\pgfpathlineto{\pgfqpoint{1.099502in}{0.481481in}}%
\pgfpathlineto{\pgfqpoint{1.111930in}{0.477817in}}%
\pgfpathclose%
\pgfpathmoveto{\pgfqpoint{1.120202in}{0.508703in}}%
\pgfpathlineto{\pgfqpoint{1.111930in}{0.510567in}}%
\pgfpathlineto{\pgfqpoint{1.096274in}{0.517610in}}%
\pgfpathlineto{\pgfqpoint{1.089235in}{0.522314in}}%
\pgfpathlineto{\pgfqpoint{1.080617in}{0.529807in}}%
\pgfpathlineto{\pgfqpoint{1.075205in}{0.535925in}}%
\pgfpathlineto{\pgfqpoint{1.067105in}{0.549536in}}%
\pgfpathlineto{\pgfqpoint{1.064960in}{0.556728in}}%
\pgfpathlineto{\pgfqpoint{1.063348in}{0.563148in}}%
\pgfpathlineto{\pgfqpoint{1.063348in}{0.576759in}}%
\pgfpathlineto{\pgfqpoint{1.064960in}{0.583178in}}%
\pgfpathlineto{\pgfqpoint{1.067105in}{0.590370in}}%
\pgfpathlineto{\pgfqpoint{1.075205in}{0.603981in}}%
\pgfpathlineto{\pgfqpoint{1.080617in}{0.610099in}}%
\pgfpathlineto{\pgfqpoint{1.089235in}{0.617592in}}%
\pgfpathlineto{\pgfqpoint{1.096274in}{0.622296in}}%
\pgfpathlineto{\pgfqpoint{1.111930in}{0.629339in}}%
\pgfpathlineto{\pgfqpoint{1.120202in}{0.631203in}}%
\pgfpathlineto{\pgfqpoint{1.127587in}{0.632604in}}%
\pgfpathlineto{\pgfqpoint{1.143243in}{0.632604in}}%
\pgfpathlineto{\pgfqpoint{1.150628in}{0.631203in}}%
\pgfpathlineto{\pgfqpoint{1.158900in}{0.629339in}}%
\pgfpathlineto{\pgfqpoint{1.174556in}{0.622296in}}%
\pgfpathlineto{\pgfqpoint{1.181595in}{0.617592in}}%
\pgfpathlineto{\pgfqpoint{1.190213in}{0.610099in}}%
\pgfpathlineto{\pgfqpoint{1.195625in}{0.603981in}}%
\pgfpathlineto{\pgfqpoint{1.203725in}{0.590370in}}%
\pgfpathlineto{\pgfqpoint{1.205870in}{0.583178in}}%
\pgfpathlineto{\pgfqpoint{1.207482in}{0.576759in}}%
\pgfpathlineto{\pgfqpoint{1.207482in}{0.563148in}}%
\pgfpathlineto{\pgfqpoint{1.205870in}{0.556728in}}%
\pgfpathlineto{\pgfqpoint{1.203725in}{0.549536in}}%
\pgfpathlineto{\pgfqpoint{1.195625in}{0.535925in}}%
\pgfpathlineto{\pgfqpoint{1.190213in}{0.529807in}}%
\pgfpathlineto{\pgfqpoint{1.181595in}{0.522314in}}%
\pgfpathlineto{\pgfqpoint{1.174556in}{0.517610in}}%
\pgfpathlineto{\pgfqpoint{1.158900in}{0.510567in}}%
\pgfpathlineto{\pgfqpoint{1.150628in}{0.508703in}}%
\pgfpathlineto{\pgfqpoint{1.143243in}{0.507302in}}%
\pgfpathlineto{\pgfqpoint{1.127587in}{0.507302in}}%
\pgfpathlineto{\pgfqpoint{1.120202in}{0.508703in}}%
\pgfpathclose%
\pgfpathmoveto{\pgfqpoint{1.628597in}{0.477817in}}%
\pgfpathlineto{\pgfqpoint{1.644253in}{0.475505in}}%
\pgfpathlineto{\pgfqpoint{1.659910in}{0.475505in}}%
\pgfpathlineto{\pgfqpoint{1.675567in}{0.477817in}}%
\pgfpathlineto{\pgfqpoint{1.687994in}{0.481481in}}%
\pgfpathlineto{\pgfqpoint{1.691223in}{0.482510in}}%
\pgfpathlineto{\pgfqpoint{1.706880in}{0.489931in}}%
\pgfpathlineto{\pgfqpoint{1.715142in}{0.495092in}}%
\pgfpathlineto{\pgfqpoint{1.722536in}{0.500377in}}%
\pgfpathlineto{\pgfqpoint{1.732114in}{0.508703in}}%
\pgfpathlineto{\pgfqpoint{1.738193in}{0.515131in}}%
\pgfpathlineto{\pgfqpoint{1.744130in}{0.522314in}}%
\pgfpathlineto{\pgfqpoint{1.752665in}{0.535925in}}%
\pgfpathlineto{\pgfqpoint{1.753849in}{0.538732in}}%
\pgfpathlineto{\pgfqpoint{1.758064in}{0.549536in}}%
\pgfpathlineto{\pgfqpoint{1.760723in}{0.563148in}}%
\pgfpathlineto{\pgfqpoint{1.760723in}{0.576759in}}%
\pgfpathlineto{\pgfqpoint{1.758064in}{0.590370in}}%
\pgfpathlineto{\pgfqpoint{1.753849in}{0.601174in}}%
\pgfpathlineto{\pgfqpoint{1.752665in}{0.603981in}}%
\pgfpathlineto{\pgfqpoint{1.744130in}{0.617592in}}%
\pgfpathlineto{\pgfqpoint{1.738193in}{0.624775in}}%
\pgfpathlineto{\pgfqpoint{1.732114in}{0.631203in}}%
\pgfpathlineto{\pgfqpoint{1.722536in}{0.639529in}}%
\pgfpathlineto{\pgfqpoint{1.715142in}{0.644814in}}%
\pgfpathlineto{\pgfqpoint{1.706880in}{0.649976in}}%
\pgfpathlineto{\pgfqpoint{1.691223in}{0.657396in}}%
\pgfpathlineto{\pgfqpoint{1.687994in}{0.658425in}}%
\pgfpathlineto{\pgfqpoint{1.675567in}{0.662089in}}%
\pgfpathlineto{\pgfqpoint{1.659910in}{0.664401in}}%
\pgfpathlineto{\pgfqpoint{1.644253in}{0.664401in}}%
\pgfpathlineto{\pgfqpoint{1.628597in}{0.662089in}}%
\pgfpathlineto{\pgfqpoint{1.616169in}{0.658425in}}%
\pgfpathlineto{\pgfqpoint{1.612940in}{0.657396in}}%
\pgfpathlineto{\pgfqpoint{1.597284in}{0.649976in}}%
\pgfpathlineto{\pgfqpoint{1.589021in}{0.644814in}}%
\pgfpathlineto{\pgfqpoint{1.581627in}{0.639529in}}%
\pgfpathlineto{\pgfqpoint{1.572050in}{0.631203in}}%
\pgfpathlineto{\pgfqpoint{1.565971in}{0.624775in}}%
\pgfpathlineto{\pgfqpoint{1.560034in}{0.617592in}}%
\pgfpathlineto{\pgfqpoint{1.551498in}{0.603981in}}%
\pgfpathlineto{\pgfqpoint{1.550314in}{0.601174in}}%
\pgfpathlineto{\pgfqpoint{1.546099in}{0.590370in}}%
\pgfpathlineto{\pgfqpoint{1.543440in}{0.576759in}}%
\pgfpathlineto{\pgfqpoint{1.543440in}{0.563148in}}%
\pgfpathlineto{\pgfqpoint{1.546099in}{0.549536in}}%
\pgfpathlineto{\pgfqpoint{1.550314in}{0.538732in}}%
\pgfpathlineto{\pgfqpoint{1.551498in}{0.535925in}}%
\pgfpathlineto{\pgfqpoint{1.560034in}{0.522314in}}%
\pgfpathlineto{\pgfqpoint{1.565971in}{0.515131in}}%
\pgfpathlineto{\pgfqpoint{1.572050in}{0.508703in}}%
\pgfpathlineto{\pgfqpoint{1.581627in}{0.500377in}}%
\pgfpathlineto{\pgfqpoint{1.589021in}{0.495092in}}%
\pgfpathlineto{\pgfqpoint{1.597284in}{0.489931in}}%
\pgfpathlineto{\pgfqpoint{1.612940in}{0.482510in}}%
\pgfpathlineto{\pgfqpoint{1.616169in}{0.481481in}}%
\pgfpathlineto{\pgfqpoint{1.628597in}{0.477817in}}%
\pgfpathclose%
\pgfpathmoveto{\pgfqpoint{1.636869in}{0.508703in}}%
\pgfpathlineto{\pgfqpoint{1.628597in}{0.510567in}}%
\pgfpathlineto{\pgfqpoint{1.612940in}{0.517610in}}%
\pgfpathlineto{\pgfqpoint{1.605902in}{0.522314in}}%
\pgfpathlineto{\pgfqpoint{1.597284in}{0.529807in}}%
\pgfpathlineto{\pgfqpoint{1.591872in}{0.535925in}}%
\pgfpathlineto{\pgfqpoint{1.583771in}{0.549536in}}%
\pgfpathlineto{\pgfqpoint{1.581627in}{0.556728in}}%
\pgfpathlineto{\pgfqpoint{1.580015in}{0.563148in}}%
\pgfpathlineto{\pgfqpoint{1.580015in}{0.576759in}}%
\pgfpathlineto{\pgfqpoint{1.581627in}{0.583178in}}%
\pgfpathlineto{\pgfqpoint{1.583771in}{0.590370in}}%
\pgfpathlineto{\pgfqpoint{1.591872in}{0.603981in}}%
\pgfpathlineto{\pgfqpoint{1.597284in}{0.610099in}}%
\pgfpathlineto{\pgfqpoint{1.605902in}{0.617592in}}%
\pgfpathlineto{\pgfqpoint{1.612940in}{0.622296in}}%
\pgfpathlineto{\pgfqpoint{1.628597in}{0.629339in}}%
\pgfpathlineto{\pgfqpoint{1.636869in}{0.631203in}}%
\pgfpathlineto{\pgfqpoint{1.644253in}{0.632604in}}%
\pgfpathlineto{\pgfqpoint{1.659910in}{0.632604in}}%
\pgfpathlineto{\pgfqpoint{1.667294in}{0.631203in}}%
\pgfpathlineto{\pgfqpoint{1.675567in}{0.629339in}}%
\pgfpathlineto{\pgfqpoint{1.691223in}{0.622296in}}%
\pgfpathlineto{\pgfqpoint{1.698261in}{0.617592in}}%
\pgfpathlineto{\pgfqpoint{1.706880in}{0.610099in}}%
\pgfpathlineto{\pgfqpoint{1.712291in}{0.603981in}}%
\pgfpathlineto{\pgfqpoint{1.720392in}{0.590370in}}%
\pgfpathlineto{\pgfqpoint{1.722536in}{0.583178in}}%
\pgfpathlineto{\pgfqpoint{1.724148in}{0.576759in}}%
\pgfpathlineto{\pgfqpoint{1.724148in}{0.563148in}}%
\pgfpathlineto{\pgfqpoint{1.722536in}{0.556728in}}%
\pgfpathlineto{\pgfqpoint{1.720392in}{0.549536in}}%
\pgfpathlineto{\pgfqpoint{1.712291in}{0.535925in}}%
\pgfpathlineto{\pgfqpoint{1.706880in}{0.529807in}}%
\pgfpathlineto{\pgfqpoint{1.698261in}{0.522314in}}%
\pgfpathlineto{\pgfqpoint{1.691223in}{0.517610in}}%
\pgfpathlineto{\pgfqpoint{1.675567in}{0.510567in}}%
\pgfpathlineto{\pgfqpoint{1.667294in}{0.508703in}}%
\pgfpathlineto{\pgfqpoint{1.659910in}{0.507302in}}%
\pgfpathlineto{\pgfqpoint{1.644253in}{0.507302in}}%
\pgfpathlineto{\pgfqpoint{1.636869in}{0.508703in}}%
\pgfpathclose%
\pgfpathmoveto{\pgfqpoint{0.595263in}{0.926983in}}%
\pgfpathlineto{\pgfqpoint{0.610920in}{0.924672in}}%
\pgfpathlineto{\pgfqpoint{0.626577in}{0.924672in}}%
\pgfpathlineto{\pgfqpoint{0.642233in}{0.926983in}}%
\pgfpathlineto{\pgfqpoint{0.654661in}{0.930648in}}%
\pgfpathlineto{\pgfqpoint{0.657890in}{0.931677in}}%
\pgfpathlineto{\pgfqpoint{0.673546in}{0.939097in}}%
\pgfpathlineto{\pgfqpoint{0.681809in}{0.944259in}}%
\pgfpathlineto{\pgfqpoint{0.689203in}{0.949543in}}%
\pgfpathlineto{\pgfqpoint{0.698780in}{0.957870in}}%
\pgfpathlineto{\pgfqpoint{0.704859in}{0.964298in}}%
\pgfpathlineto{\pgfqpoint{0.710796in}{0.971481in}}%
\pgfpathlineto{\pgfqpoint{0.719332in}{0.985092in}}%
\pgfpathlineto{\pgfqpoint{0.720516in}{0.987899in}}%
\pgfpathlineto{\pgfqpoint{0.724731in}{0.998703in}}%
\pgfpathlineto{\pgfqpoint{0.727390in}{1.012314in}}%
\pgfpathlineto{\pgfqpoint{0.727390in}{1.025925in}}%
\pgfpathlineto{\pgfqpoint{0.724731in}{1.039536in}}%
\pgfpathlineto{\pgfqpoint{0.720516in}{1.050341in}}%
\pgfpathlineto{\pgfqpoint{0.719332in}{1.053148in}}%
\pgfpathlineto{\pgfqpoint{0.710796in}{1.066759in}}%
\pgfpathlineto{\pgfqpoint{0.704859in}{1.073942in}}%
\pgfpathlineto{\pgfqpoint{0.698780in}{1.080370in}}%
\pgfpathlineto{\pgfqpoint{0.689203in}{1.088696in}}%
\pgfpathlineto{\pgfqpoint{0.681809in}{1.093981in}}%
\pgfpathlineto{\pgfqpoint{0.673546in}{1.099142in}}%
\pgfpathlineto{\pgfqpoint{0.657890in}{1.106563in}}%
\pgfpathlineto{\pgfqpoint{0.654661in}{1.107592in}}%
\pgfpathlineto{\pgfqpoint{0.642233in}{1.111256in}}%
\pgfpathlineto{\pgfqpoint{0.626577in}{1.113568in}}%
\pgfpathlineto{\pgfqpoint{0.610920in}{1.113568in}}%
\pgfpathlineto{\pgfqpoint{0.595263in}{1.111256in}}%
\pgfpathlineto{\pgfqpoint{0.582836in}{1.107592in}}%
\pgfpathlineto{\pgfqpoint{0.579607in}{1.106563in}}%
\pgfpathlineto{\pgfqpoint{0.563950in}{1.099142in}}%
\pgfpathlineto{\pgfqpoint{0.555688in}{1.093981in}}%
\pgfpathlineto{\pgfqpoint{0.548294in}{1.088696in}}%
\pgfpathlineto{\pgfqpoint{0.538716in}{1.080370in}}%
\pgfpathlineto{\pgfqpoint{0.532637in}{1.073942in}}%
\pgfpathlineto{\pgfqpoint{0.526700in}{1.066759in}}%
\pgfpathlineto{\pgfqpoint{0.518165in}{1.053148in}}%
\pgfpathlineto{\pgfqpoint{0.516981in}{1.050341in}}%
\pgfpathlineto{\pgfqpoint{0.512766in}{1.039536in}}%
\pgfpathlineto{\pgfqpoint{0.510107in}{1.025925in}}%
\pgfpathlineto{\pgfqpoint{0.510107in}{1.012314in}}%
\pgfpathlineto{\pgfqpoint{0.512766in}{0.998703in}}%
\pgfpathlineto{\pgfqpoint{0.516981in}{0.987899in}}%
\pgfpathlineto{\pgfqpoint{0.518165in}{0.985092in}}%
\pgfpathlineto{\pgfqpoint{0.526700in}{0.971481in}}%
\pgfpathlineto{\pgfqpoint{0.532637in}{0.964298in}}%
\pgfpathlineto{\pgfqpoint{0.538716in}{0.957870in}}%
\pgfpathlineto{\pgfqpoint{0.548294in}{0.949543in}}%
\pgfpathlineto{\pgfqpoint{0.555688in}{0.944259in}}%
\pgfpathlineto{\pgfqpoint{0.563950in}{0.939097in}}%
\pgfpathlineto{\pgfqpoint{0.579607in}{0.931677in}}%
\pgfpathlineto{\pgfqpoint{0.582836in}{0.930648in}}%
\pgfpathlineto{\pgfqpoint{0.595263in}{0.926983in}}%
\pgfpathclose%
\pgfpathmoveto{\pgfqpoint{0.603536in}{0.957870in}}%
\pgfpathlineto{\pgfqpoint{0.595263in}{0.959734in}}%
\pgfpathlineto{\pgfqpoint{0.579607in}{0.966776in}}%
\pgfpathlineto{\pgfqpoint{0.572569in}{0.971481in}}%
\pgfpathlineto{\pgfqpoint{0.563950in}{0.978973in}}%
\pgfpathlineto{\pgfqpoint{0.558539in}{0.985092in}}%
\pgfpathlineto{\pgfqpoint{0.550438in}{0.998703in}}%
\pgfpathlineto{\pgfqpoint{0.548294in}{1.005895in}}%
\pgfpathlineto{\pgfqpoint{0.546682in}{1.012314in}}%
\pgfpathlineto{\pgfqpoint{0.546682in}{1.025925in}}%
\pgfpathlineto{\pgfqpoint{0.548294in}{1.032345in}}%
\pgfpathlineto{\pgfqpoint{0.550438in}{1.039536in}}%
\pgfpathlineto{\pgfqpoint{0.558539in}{1.053148in}}%
\pgfpathlineto{\pgfqpoint{0.563950in}{1.059266in}}%
\pgfpathlineto{\pgfqpoint{0.572569in}{1.066759in}}%
\pgfpathlineto{\pgfqpoint{0.579607in}{1.071463in}}%
\pgfpathlineto{\pgfqpoint{0.595263in}{1.078506in}}%
\pgfpathlineto{\pgfqpoint{0.603536in}{1.080370in}}%
\pgfpathlineto{\pgfqpoint{0.610920in}{1.081771in}}%
\pgfpathlineto{\pgfqpoint{0.626577in}{1.081771in}}%
\pgfpathlineto{\pgfqpoint{0.633961in}{1.080370in}}%
\pgfpathlineto{\pgfqpoint{0.642233in}{1.078506in}}%
\pgfpathlineto{\pgfqpoint{0.657890in}{1.071463in}}%
\pgfpathlineto{\pgfqpoint{0.664928in}{1.066759in}}%
\pgfpathlineto{\pgfqpoint{0.673546in}{1.059266in}}%
\pgfpathlineto{\pgfqpoint{0.678958in}{1.053148in}}%
\pgfpathlineto{\pgfqpoint{0.687059in}{1.039536in}}%
\pgfpathlineto{\pgfqpoint{0.689203in}{1.032345in}}%
\pgfpathlineto{\pgfqpoint{0.690815in}{1.025925in}}%
\pgfpathlineto{\pgfqpoint{0.690815in}{1.012314in}}%
\pgfpathlineto{\pgfqpoint{0.689203in}{1.005895in}}%
\pgfpathlineto{\pgfqpoint{0.687059in}{0.998703in}}%
\pgfpathlineto{\pgfqpoint{0.678958in}{0.985092in}}%
\pgfpathlineto{\pgfqpoint{0.673546in}{0.978973in}}%
\pgfpathlineto{\pgfqpoint{0.664928in}{0.971481in}}%
\pgfpathlineto{\pgfqpoint{0.657890in}{0.966776in}}%
\pgfpathlineto{\pgfqpoint{0.642233in}{0.959734in}}%
\pgfpathlineto{\pgfqpoint{0.633961in}{0.957870in}}%
\pgfpathlineto{\pgfqpoint{0.626577in}{0.956468in}}%
\pgfpathlineto{\pgfqpoint{0.610920in}{0.956468in}}%
\pgfpathlineto{\pgfqpoint{0.603536in}{0.957870in}}%
\pgfpathclose%
\pgfpathmoveto{\pgfqpoint{1.111930in}{0.926983in}}%
\pgfpathlineto{\pgfqpoint{1.127587in}{0.924672in}}%
\pgfpathlineto{\pgfqpoint{1.143243in}{0.924672in}}%
\pgfpathlineto{\pgfqpoint{1.158900in}{0.926983in}}%
\pgfpathlineto{\pgfqpoint{1.171328in}{0.930648in}}%
\pgfpathlineto{\pgfqpoint{1.174556in}{0.931677in}}%
\pgfpathlineto{\pgfqpoint{1.190213in}{0.939097in}}%
\pgfpathlineto{\pgfqpoint{1.198476in}{0.944259in}}%
\pgfpathlineto{\pgfqpoint{1.205870in}{0.949543in}}%
\pgfpathlineto{\pgfqpoint{1.215447in}{0.957870in}}%
\pgfpathlineto{\pgfqpoint{1.221526in}{0.964298in}}%
\pgfpathlineto{\pgfqpoint{1.227463in}{0.971481in}}%
\pgfpathlineto{\pgfqpoint{1.235999in}{0.985092in}}%
\pgfpathlineto{\pgfqpoint{1.237183in}{0.987899in}}%
\pgfpathlineto{\pgfqpoint{1.241397in}{0.998703in}}%
\pgfpathlineto{\pgfqpoint{1.244056in}{1.012314in}}%
\pgfpathlineto{\pgfqpoint{1.244056in}{1.025925in}}%
\pgfpathlineto{\pgfqpoint{1.241397in}{1.039536in}}%
\pgfpathlineto{\pgfqpoint{1.237183in}{1.050341in}}%
\pgfpathlineto{\pgfqpoint{1.235999in}{1.053148in}}%
\pgfpathlineto{\pgfqpoint{1.227463in}{1.066759in}}%
\pgfpathlineto{\pgfqpoint{1.221526in}{1.073942in}}%
\pgfpathlineto{\pgfqpoint{1.215447in}{1.080370in}}%
\pgfpathlineto{\pgfqpoint{1.205870in}{1.088696in}}%
\pgfpathlineto{\pgfqpoint{1.198476in}{1.093981in}}%
\pgfpathlineto{\pgfqpoint{1.190213in}{1.099142in}}%
\pgfpathlineto{\pgfqpoint{1.174556in}{1.106563in}}%
\pgfpathlineto{\pgfqpoint{1.171328in}{1.107592in}}%
\pgfpathlineto{\pgfqpoint{1.158900in}{1.111256in}}%
\pgfpathlineto{\pgfqpoint{1.143243in}{1.113568in}}%
\pgfpathlineto{\pgfqpoint{1.127587in}{1.113568in}}%
\pgfpathlineto{\pgfqpoint{1.111930in}{1.111256in}}%
\pgfpathlineto{\pgfqpoint{1.099502in}{1.107592in}}%
\pgfpathlineto{\pgfqpoint{1.096274in}{1.106563in}}%
\pgfpathlineto{\pgfqpoint{1.080617in}{1.099142in}}%
\pgfpathlineto{\pgfqpoint{1.072354in}{1.093981in}}%
\pgfpathlineto{\pgfqpoint{1.064960in}{1.088696in}}%
\pgfpathlineto{\pgfqpoint{1.055383in}{1.080370in}}%
\pgfpathlineto{\pgfqpoint{1.049304in}{1.073942in}}%
\pgfpathlineto{\pgfqpoint{1.043367in}{1.066759in}}%
\pgfpathlineto{\pgfqpoint{1.034831in}{1.053148in}}%
\pgfpathlineto{\pgfqpoint{1.033647in}{1.050341in}}%
\pgfpathlineto{\pgfqpoint{1.029433in}{1.039536in}}%
\pgfpathlineto{\pgfqpoint{1.026774in}{1.025925in}}%
\pgfpathlineto{\pgfqpoint{1.026774in}{1.012314in}}%
\pgfpathlineto{\pgfqpoint{1.029433in}{0.998703in}}%
\pgfpathlineto{\pgfqpoint{1.033647in}{0.987899in}}%
\pgfpathlineto{\pgfqpoint{1.034831in}{0.985092in}}%
\pgfpathlineto{\pgfqpoint{1.043367in}{0.971481in}}%
\pgfpathlineto{\pgfqpoint{1.049304in}{0.964298in}}%
\pgfpathlineto{\pgfqpoint{1.055383in}{0.957870in}}%
\pgfpathlineto{\pgfqpoint{1.064960in}{0.949543in}}%
\pgfpathlineto{\pgfqpoint{1.072354in}{0.944259in}}%
\pgfpathlineto{\pgfqpoint{1.080617in}{0.939097in}}%
\pgfpathlineto{\pgfqpoint{1.096274in}{0.931677in}}%
\pgfpathlineto{\pgfqpoint{1.099502in}{0.930648in}}%
\pgfpathlineto{\pgfqpoint{1.111930in}{0.926983in}}%
\pgfpathclose%
\pgfpathmoveto{\pgfqpoint{1.120202in}{0.957870in}}%
\pgfpathlineto{\pgfqpoint{1.111930in}{0.959734in}}%
\pgfpathlineto{\pgfqpoint{1.096274in}{0.966776in}}%
\pgfpathlineto{\pgfqpoint{1.089235in}{0.971481in}}%
\pgfpathlineto{\pgfqpoint{1.080617in}{0.978973in}}%
\pgfpathlineto{\pgfqpoint{1.075205in}{0.985092in}}%
\pgfpathlineto{\pgfqpoint{1.067105in}{0.998703in}}%
\pgfpathlineto{\pgfqpoint{1.064960in}{1.005895in}}%
\pgfpathlineto{\pgfqpoint{1.063348in}{1.012314in}}%
\pgfpathlineto{\pgfqpoint{1.063348in}{1.025925in}}%
\pgfpathlineto{\pgfqpoint{1.064960in}{1.032345in}}%
\pgfpathlineto{\pgfqpoint{1.067105in}{1.039536in}}%
\pgfpathlineto{\pgfqpoint{1.075205in}{1.053148in}}%
\pgfpathlineto{\pgfqpoint{1.080617in}{1.059266in}}%
\pgfpathlineto{\pgfqpoint{1.089235in}{1.066759in}}%
\pgfpathlineto{\pgfqpoint{1.096274in}{1.071463in}}%
\pgfpathlineto{\pgfqpoint{1.111930in}{1.078506in}}%
\pgfpathlineto{\pgfqpoint{1.120202in}{1.080370in}}%
\pgfpathlineto{\pgfqpoint{1.127587in}{1.081771in}}%
\pgfpathlineto{\pgfqpoint{1.143243in}{1.081771in}}%
\pgfpathlineto{\pgfqpoint{1.150628in}{1.080370in}}%
\pgfpathlineto{\pgfqpoint{1.158900in}{1.078506in}}%
\pgfpathlineto{\pgfqpoint{1.174556in}{1.071463in}}%
\pgfpathlineto{\pgfqpoint{1.181595in}{1.066759in}}%
\pgfpathlineto{\pgfqpoint{1.190213in}{1.059266in}}%
\pgfpathlineto{\pgfqpoint{1.195625in}{1.053148in}}%
\pgfpathlineto{\pgfqpoint{1.203725in}{1.039536in}}%
\pgfpathlineto{\pgfqpoint{1.205870in}{1.032345in}}%
\pgfpathlineto{\pgfqpoint{1.207482in}{1.025925in}}%
\pgfpathlineto{\pgfqpoint{1.207482in}{1.012314in}}%
\pgfpathlineto{\pgfqpoint{1.205870in}{1.005895in}}%
\pgfpathlineto{\pgfqpoint{1.203725in}{0.998703in}}%
\pgfpathlineto{\pgfqpoint{1.195625in}{0.985092in}}%
\pgfpathlineto{\pgfqpoint{1.190213in}{0.978973in}}%
\pgfpathlineto{\pgfqpoint{1.181595in}{0.971481in}}%
\pgfpathlineto{\pgfqpoint{1.174556in}{0.966776in}}%
\pgfpathlineto{\pgfqpoint{1.158900in}{0.959734in}}%
\pgfpathlineto{\pgfqpoint{1.150628in}{0.957870in}}%
\pgfpathlineto{\pgfqpoint{1.143243in}{0.956468in}}%
\pgfpathlineto{\pgfqpoint{1.127587in}{0.956468in}}%
\pgfpathlineto{\pgfqpoint{1.120202in}{0.957870in}}%
\pgfpathclose%
\pgfpathmoveto{\pgfqpoint{1.628597in}{0.926983in}}%
\pgfpathlineto{\pgfqpoint{1.644253in}{0.924672in}}%
\pgfpathlineto{\pgfqpoint{1.659910in}{0.924672in}}%
\pgfpathlineto{\pgfqpoint{1.675567in}{0.926983in}}%
\pgfpathlineto{\pgfqpoint{1.687994in}{0.930648in}}%
\pgfpathlineto{\pgfqpoint{1.691223in}{0.931677in}}%
\pgfpathlineto{\pgfqpoint{1.706880in}{0.939097in}}%
\pgfpathlineto{\pgfqpoint{1.715142in}{0.944259in}}%
\pgfpathlineto{\pgfqpoint{1.722536in}{0.949543in}}%
\pgfpathlineto{\pgfqpoint{1.732114in}{0.957870in}}%
\pgfpathlineto{\pgfqpoint{1.738193in}{0.964298in}}%
\pgfpathlineto{\pgfqpoint{1.744130in}{0.971481in}}%
\pgfpathlineto{\pgfqpoint{1.752665in}{0.985092in}}%
\pgfpathlineto{\pgfqpoint{1.753849in}{0.987899in}}%
\pgfpathlineto{\pgfqpoint{1.758064in}{0.998703in}}%
\pgfpathlineto{\pgfqpoint{1.760723in}{1.012314in}}%
\pgfpathlineto{\pgfqpoint{1.760723in}{1.025925in}}%
\pgfpathlineto{\pgfqpoint{1.758064in}{1.039536in}}%
\pgfpathlineto{\pgfqpoint{1.753849in}{1.050341in}}%
\pgfpathlineto{\pgfqpoint{1.752665in}{1.053148in}}%
\pgfpathlineto{\pgfqpoint{1.744130in}{1.066759in}}%
\pgfpathlineto{\pgfqpoint{1.738193in}{1.073942in}}%
\pgfpathlineto{\pgfqpoint{1.732114in}{1.080370in}}%
\pgfpathlineto{\pgfqpoint{1.722536in}{1.088696in}}%
\pgfpathlineto{\pgfqpoint{1.715142in}{1.093981in}}%
\pgfpathlineto{\pgfqpoint{1.706880in}{1.099142in}}%
\pgfpathlineto{\pgfqpoint{1.691223in}{1.106563in}}%
\pgfpathlineto{\pgfqpoint{1.687994in}{1.107592in}}%
\pgfpathlineto{\pgfqpoint{1.675567in}{1.111256in}}%
\pgfpathlineto{\pgfqpoint{1.659910in}{1.113568in}}%
\pgfpathlineto{\pgfqpoint{1.644253in}{1.113568in}}%
\pgfpathlineto{\pgfqpoint{1.628597in}{1.111256in}}%
\pgfpathlineto{\pgfqpoint{1.616169in}{1.107592in}}%
\pgfpathlineto{\pgfqpoint{1.612940in}{1.106563in}}%
\pgfpathlineto{\pgfqpoint{1.597284in}{1.099142in}}%
\pgfpathlineto{\pgfqpoint{1.589021in}{1.093981in}}%
\pgfpathlineto{\pgfqpoint{1.581627in}{1.088696in}}%
\pgfpathlineto{\pgfqpoint{1.572050in}{1.080370in}}%
\pgfpathlineto{\pgfqpoint{1.565971in}{1.073942in}}%
\pgfpathlineto{\pgfqpoint{1.560034in}{1.066759in}}%
\pgfpathlineto{\pgfqpoint{1.551498in}{1.053148in}}%
\pgfpathlineto{\pgfqpoint{1.550314in}{1.050341in}}%
\pgfpathlineto{\pgfqpoint{1.546099in}{1.039536in}}%
\pgfpathlineto{\pgfqpoint{1.543440in}{1.025925in}}%
\pgfpathlineto{\pgfqpoint{1.543440in}{1.012314in}}%
\pgfpathlineto{\pgfqpoint{1.546099in}{0.998703in}}%
\pgfpathlineto{\pgfqpoint{1.550314in}{0.987899in}}%
\pgfpathlineto{\pgfqpoint{1.551498in}{0.985092in}}%
\pgfpathlineto{\pgfqpoint{1.560034in}{0.971481in}}%
\pgfpathlineto{\pgfqpoint{1.565971in}{0.964298in}}%
\pgfpathlineto{\pgfqpoint{1.572050in}{0.957870in}}%
\pgfpathlineto{\pgfqpoint{1.581627in}{0.949543in}}%
\pgfpathlineto{\pgfqpoint{1.589021in}{0.944259in}}%
\pgfpathlineto{\pgfqpoint{1.597284in}{0.939097in}}%
\pgfpathlineto{\pgfqpoint{1.612940in}{0.931677in}}%
\pgfpathlineto{\pgfqpoint{1.616169in}{0.930648in}}%
\pgfpathlineto{\pgfqpoint{1.628597in}{0.926983in}}%
\pgfpathclose%
\pgfpathmoveto{\pgfqpoint{1.636869in}{0.957870in}}%
\pgfpathlineto{\pgfqpoint{1.628597in}{0.959734in}}%
\pgfpathlineto{\pgfqpoint{1.612940in}{0.966776in}}%
\pgfpathlineto{\pgfqpoint{1.605902in}{0.971481in}}%
\pgfpathlineto{\pgfqpoint{1.597284in}{0.978973in}}%
\pgfpathlineto{\pgfqpoint{1.591872in}{0.985092in}}%
\pgfpathlineto{\pgfqpoint{1.583771in}{0.998703in}}%
\pgfpathlineto{\pgfqpoint{1.581627in}{1.005895in}}%
\pgfpathlineto{\pgfqpoint{1.580015in}{1.012314in}}%
\pgfpathlineto{\pgfqpoint{1.580015in}{1.025925in}}%
\pgfpathlineto{\pgfqpoint{1.581627in}{1.032345in}}%
\pgfpathlineto{\pgfqpoint{1.583771in}{1.039536in}}%
\pgfpathlineto{\pgfqpoint{1.591872in}{1.053148in}}%
\pgfpathlineto{\pgfqpoint{1.597284in}{1.059266in}}%
\pgfpathlineto{\pgfqpoint{1.605902in}{1.066759in}}%
\pgfpathlineto{\pgfqpoint{1.612940in}{1.071463in}}%
\pgfpathlineto{\pgfqpoint{1.628597in}{1.078506in}}%
\pgfpathlineto{\pgfqpoint{1.636869in}{1.080370in}}%
\pgfpathlineto{\pgfqpoint{1.644253in}{1.081771in}}%
\pgfpathlineto{\pgfqpoint{1.659910in}{1.081771in}}%
\pgfpathlineto{\pgfqpoint{1.667294in}{1.080370in}}%
\pgfpathlineto{\pgfqpoint{1.675567in}{1.078506in}}%
\pgfpathlineto{\pgfqpoint{1.691223in}{1.071463in}}%
\pgfpathlineto{\pgfqpoint{1.698261in}{1.066759in}}%
\pgfpathlineto{\pgfqpoint{1.706880in}{1.059266in}}%
\pgfpathlineto{\pgfqpoint{1.712291in}{1.053148in}}%
\pgfpathlineto{\pgfqpoint{1.720392in}{1.039536in}}%
\pgfpathlineto{\pgfqpoint{1.722536in}{1.032345in}}%
\pgfpathlineto{\pgfqpoint{1.724148in}{1.025925in}}%
\pgfpathlineto{\pgfqpoint{1.724148in}{1.012314in}}%
\pgfpathlineto{\pgfqpoint{1.722536in}{1.005895in}}%
\pgfpathlineto{\pgfqpoint{1.720392in}{0.998703in}}%
\pgfpathlineto{\pgfqpoint{1.712291in}{0.985092in}}%
\pgfpathlineto{\pgfqpoint{1.706880in}{0.978973in}}%
\pgfpathlineto{\pgfqpoint{1.698261in}{0.971481in}}%
\pgfpathlineto{\pgfqpoint{1.691223in}{0.966776in}}%
\pgfpathlineto{\pgfqpoint{1.675567in}{0.959734in}}%
\pgfpathlineto{\pgfqpoint{1.667294in}{0.957870in}}%
\pgfpathlineto{\pgfqpoint{1.659910in}{0.956468in}}%
\pgfpathlineto{\pgfqpoint{1.644253in}{0.956468in}}%
\pgfpathlineto{\pgfqpoint{1.636869in}{0.957870in}}%
\pgfpathclose%
\pgfpathmoveto{\pgfqpoint{0.595263in}{1.376150in}}%
\pgfpathlineto{\pgfqpoint{0.610920in}{1.373839in}}%
\pgfpathlineto{\pgfqpoint{0.626577in}{1.373839in}}%
\pgfpathlineto{\pgfqpoint{0.642233in}{1.376150in}}%
\pgfpathlineto{\pgfqpoint{0.654661in}{1.379814in}}%
\pgfpathlineto{\pgfqpoint{0.657890in}{1.380844in}}%
\pgfpathlineto{\pgfqpoint{0.673546in}{1.388264in}}%
\pgfpathlineto{\pgfqpoint{0.681809in}{1.393425in}}%
\pgfpathlineto{\pgfqpoint{0.689203in}{1.398710in}}%
\pgfpathlineto{\pgfqpoint{0.698780in}{1.407036in}}%
\pgfpathlineto{\pgfqpoint{0.704859in}{1.413464in}}%
\pgfpathlineto{\pgfqpoint{0.710796in}{1.420648in}}%
\pgfpathlineto{\pgfqpoint{0.719332in}{1.434259in}}%
\pgfpathlineto{\pgfqpoint{0.720516in}{1.437066in}}%
\pgfpathlineto{\pgfqpoint{0.724731in}{1.447870in}}%
\pgfpathlineto{\pgfqpoint{0.727390in}{1.461481in}}%
\pgfpathlineto{\pgfqpoint{0.727390in}{1.475092in}}%
\pgfpathlineto{\pgfqpoint{0.724731in}{1.488703in}}%
\pgfpathlineto{\pgfqpoint{0.720516in}{1.499507in}}%
\pgfpathlineto{\pgfqpoint{0.719332in}{1.502314in}}%
\pgfpathlineto{\pgfqpoint{0.710796in}{1.515925in}}%
\pgfpathlineto{\pgfqpoint{0.704859in}{1.523109in}}%
\pgfpathlineto{\pgfqpoint{0.698780in}{1.529536in}}%
\pgfpathlineto{\pgfqpoint{0.689203in}{1.537863in}}%
\pgfpathlineto{\pgfqpoint{0.681809in}{1.543148in}}%
\pgfpathlineto{\pgfqpoint{0.673546in}{1.548309in}}%
\pgfpathlineto{\pgfqpoint{0.657890in}{1.555729in}}%
\pgfpathlineto{\pgfqpoint{0.654661in}{1.556759in}}%
\pgfpathlineto{\pgfqpoint{0.642233in}{1.560423in}}%
\pgfpathlineto{\pgfqpoint{0.626577in}{1.562734in}}%
\pgfpathlineto{\pgfqpoint{0.610920in}{1.562734in}}%
\pgfpathlineto{\pgfqpoint{0.595263in}{1.560423in}}%
\pgfpathlineto{\pgfqpoint{0.582836in}{1.556759in}}%
\pgfpathlineto{\pgfqpoint{0.579607in}{1.555729in}}%
\pgfpathlineto{\pgfqpoint{0.563950in}{1.548309in}}%
\pgfpathlineto{\pgfqpoint{0.555688in}{1.543148in}}%
\pgfpathlineto{\pgfqpoint{0.548294in}{1.537863in}}%
\pgfpathlineto{\pgfqpoint{0.538716in}{1.529536in}}%
\pgfpathlineto{\pgfqpoint{0.532637in}{1.523109in}}%
\pgfpathlineto{\pgfqpoint{0.526700in}{1.515925in}}%
\pgfpathlineto{\pgfqpoint{0.518165in}{1.502314in}}%
\pgfpathlineto{\pgfqpoint{0.516981in}{1.499507in}}%
\pgfpathlineto{\pgfqpoint{0.512766in}{1.488703in}}%
\pgfpathlineto{\pgfqpoint{0.510107in}{1.475092in}}%
\pgfpathlineto{\pgfqpoint{0.510107in}{1.461481in}}%
\pgfpathlineto{\pgfqpoint{0.512766in}{1.447870in}}%
\pgfpathlineto{\pgfqpoint{0.516981in}{1.437066in}}%
\pgfpathlineto{\pgfqpoint{0.518165in}{1.434259in}}%
\pgfpathlineto{\pgfqpoint{0.526700in}{1.420648in}}%
\pgfpathlineto{\pgfqpoint{0.532637in}{1.413464in}}%
\pgfpathlineto{\pgfqpoint{0.538716in}{1.407036in}}%
\pgfpathlineto{\pgfqpoint{0.548294in}{1.398710in}}%
\pgfpathlineto{\pgfqpoint{0.555688in}{1.393425in}}%
\pgfpathlineto{\pgfqpoint{0.563950in}{1.388264in}}%
\pgfpathlineto{\pgfqpoint{0.579607in}{1.380844in}}%
\pgfpathlineto{\pgfqpoint{0.582836in}{1.379814in}}%
\pgfpathlineto{\pgfqpoint{0.595263in}{1.376150in}}%
\pgfpathclose%
\pgfpathmoveto{\pgfqpoint{0.603536in}{1.407036in}}%
\pgfpathlineto{\pgfqpoint{0.595263in}{1.408900in}}%
\pgfpathlineto{\pgfqpoint{0.579607in}{1.415943in}}%
\pgfpathlineto{\pgfqpoint{0.572569in}{1.420648in}}%
\pgfpathlineto{\pgfqpoint{0.563950in}{1.428140in}}%
\pgfpathlineto{\pgfqpoint{0.558539in}{1.434259in}}%
\pgfpathlineto{\pgfqpoint{0.550438in}{1.447870in}}%
\pgfpathlineto{\pgfqpoint{0.548294in}{1.455061in}}%
\pgfpathlineto{\pgfqpoint{0.546682in}{1.461481in}}%
\pgfpathlineto{\pgfqpoint{0.546682in}{1.475092in}}%
\pgfpathlineto{\pgfqpoint{0.548294in}{1.481511in}}%
\pgfpathlineto{\pgfqpoint{0.550438in}{1.488703in}}%
\pgfpathlineto{\pgfqpoint{0.558539in}{1.502314in}}%
\pgfpathlineto{\pgfqpoint{0.563950in}{1.508433in}}%
\pgfpathlineto{\pgfqpoint{0.572569in}{1.515925in}}%
\pgfpathlineto{\pgfqpoint{0.579607in}{1.520630in}}%
\pgfpathlineto{\pgfqpoint{0.595263in}{1.527672in}}%
\pgfpathlineto{\pgfqpoint{0.603536in}{1.529536in}}%
\pgfpathlineto{\pgfqpoint{0.610920in}{1.530938in}}%
\pgfpathlineto{\pgfqpoint{0.626577in}{1.530938in}}%
\pgfpathlineto{\pgfqpoint{0.633961in}{1.529536in}}%
\pgfpathlineto{\pgfqpoint{0.642233in}{1.527672in}}%
\pgfpathlineto{\pgfqpoint{0.657890in}{1.520630in}}%
\pgfpathlineto{\pgfqpoint{0.664928in}{1.515925in}}%
\pgfpathlineto{\pgfqpoint{0.673546in}{1.508433in}}%
\pgfpathlineto{\pgfqpoint{0.678958in}{1.502314in}}%
\pgfpathlineto{\pgfqpoint{0.687059in}{1.488703in}}%
\pgfpathlineto{\pgfqpoint{0.689203in}{1.481511in}}%
\pgfpathlineto{\pgfqpoint{0.690815in}{1.475092in}}%
\pgfpathlineto{\pgfqpoint{0.690815in}{1.461481in}}%
\pgfpathlineto{\pgfqpoint{0.689203in}{1.455061in}}%
\pgfpathlineto{\pgfqpoint{0.687059in}{1.447870in}}%
\pgfpathlineto{\pgfqpoint{0.678958in}{1.434259in}}%
\pgfpathlineto{\pgfqpoint{0.673546in}{1.428140in}}%
\pgfpathlineto{\pgfqpoint{0.664928in}{1.420648in}}%
\pgfpathlineto{\pgfqpoint{0.657890in}{1.415943in}}%
\pgfpathlineto{\pgfqpoint{0.642233in}{1.408900in}}%
\pgfpathlineto{\pgfqpoint{0.633961in}{1.407036in}}%
\pgfpathlineto{\pgfqpoint{0.626577in}{1.405635in}}%
\pgfpathlineto{\pgfqpoint{0.610920in}{1.405635in}}%
\pgfpathlineto{\pgfqpoint{0.603536in}{1.407036in}}%
\pgfpathclose%
\pgfpathmoveto{\pgfqpoint{1.111930in}{1.376150in}}%
\pgfpathlineto{\pgfqpoint{1.127587in}{1.373839in}}%
\pgfpathlineto{\pgfqpoint{1.143243in}{1.373839in}}%
\pgfpathlineto{\pgfqpoint{1.158900in}{1.376150in}}%
\pgfpathlineto{\pgfqpoint{1.171328in}{1.379814in}}%
\pgfpathlineto{\pgfqpoint{1.174556in}{1.380844in}}%
\pgfpathlineto{\pgfqpoint{1.190213in}{1.388264in}}%
\pgfpathlineto{\pgfqpoint{1.198476in}{1.393425in}}%
\pgfpathlineto{\pgfqpoint{1.205870in}{1.398710in}}%
\pgfpathlineto{\pgfqpoint{1.215447in}{1.407036in}}%
\pgfpathlineto{\pgfqpoint{1.221526in}{1.413464in}}%
\pgfpathlineto{\pgfqpoint{1.227463in}{1.420648in}}%
\pgfpathlineto{\pgfqpoint{1.235999in}{1.434259in}}%
\pgfpathlineto{\pgfqpoint{1.237183in}{1.437066in}}%
\pgfpathlineto{\pgfqpoint{1.241397in}{1.447870in}}%
\pgfpathlineto{\pgfqpoint{1.244056in}{1.461481in}}%
\pgfpathlineto{\pgfqpoint{1.244056in}{1.475092in}}%
\pgfpathlineto{\pgfqpoint{1.241397in}{1.488703in}}%
\pgfpathlineto{\pgfqpoint{1.237183in}{1.499507in}}%
\pgfpathlineto{\pgfqpoint{1.235999in}{1.502314in}}%
\pgfpathlineto{\pgfqpoint{1.227463in}{1.515925in}}%
\pgfpathlineto{\pgfqpoint{1.221526in}{1.523109in}}%
\pgfpathlineto{\pgfqpoint{1.215447in}{1.529536in}}%
\pgfpathlineto{\pgfqpoint{1.205870in}{1.537863in}}%
\pgfpathlineto{\pgfqpoint{1.198476in}{1.543148in}}%
\pgfpathlineto{\pgfqpoint{1.190213in}{1.548309in}}%
\pgfpathlineto{\pgfqpoint{1.174556in}{1.555729in}}%
\pgfpathlineto{\pgfqpoint{1.171328in}{1.556759in}}%
\pgfpathlineto{\pgfqpoint{1.158900in}{1.560423in}}%
\pgfpathlineto{\pgfqpoint{1.143243in}{1.562734in}}%
\pgfpathlineto{\pgfqpoint{1.127587in}{1.562734in}}%
\pgfpathlineto{\pgfqpoint{1.111930in}{1.560423in}}%
\pgfpathlineto{\pgfqpoint{1.099502in}{1.556759in}}%
\pgfpathlineto{\pgfqpoint{1.096274in}{1.555729in}}%
\pgfpathlineto{\pgfqpoint{1.080617in}{1.548309in}}%
\pgfpathlineto{\pgfqpoint{1.072354in}{1.543148in}}%
\pgfpathlineto{\pgfqpoint{1.064960in}{1.537863in}}%
\pgfpathlineto{\pgfqpoint{1.055383in}{1.529536in}}%
\pgfpathlineto{\pgfqpoint{1.049304in}{1.523109in}}%
\pgfpathlineto{\pgfqpoint{1.043367in}{1.515925in}}%
\pgfpathlineto{\pgfqpoint{1.034831in}{1.502314in}}%
\pgfpathlineto{\pgfqpoint{1.033647in}{1.499507in}}%
\pgfpathlineto{\pgfqpoint{1.029433in}{1.488703in}}%
\pgfpathlineto{\pgfqpoint{1.026774in}{1.475092in}}%
\pgfpathlineto{\pgfqpoint{1.026774in}{1.461481in}}%
\pgfpathlineto{\pgfqpoint{1.029433in}{1.447870in}}%
\pgfpathlineto{\pgfqpoint{1.033647in}{1.437066in}}%
\pgfpathlineto{\pgfqpoint{1.034831in}{1.434259in}}%
\pgfpathlineto{\pgfqpoint{1.043367in}{1.420648in}}%
\pgfpathlineto{\pgfqpoint{1.049304in}{1.413464in}}%
\pgfpathlineto{\pgfqpoint{1.055383in}{1.407036in}}%
\pgfpathlineto{\pgfqpoint{1.064960in}{1.398710in}}%
\pgfpathlineto{\pgfqpoint{1.072354in}{1.393425in}}%
\pgfpathlineto{\pgfqpoint{1.080617in}{1.388264in}}%
\pgfpathlineto{\pgfqpoint{1.096274in}{1.380844in}}%
\pgfpathlineto{\pgfqpoint{1.099502in}{1.379814in}}%
\pgfpathlineto{\pgfqpoint{1.111930in}{1.376150in}}%
\pgfpathclose%
\pgfpathmoveto{\pgfqpoint{1.120202in}{1.407036in}}%
\pgfpathlineto{\pgfqpoint{1.111930in}{1.408900in}}%
\pgfpathlineto{\pgfqpoint{1.096274in}{1.415943in}}%
\pgfpathlineto{\pgfqpoint{1.089235in}{1.420648in}}%
\pgfpathlineto{\pgfqpoint{1.080617in}{1.428140in}}%
\pgfpathlineto{\pgfqpoint{1.075205in}{1.434259in}}%
\pgfpathlineto{\pgfqpoint{1.067105in}{1.447870in}}%
\pgfpathlineto{\pgfqpoint{1.064960in}{1.455061in}}%
\pgfpathlineto{\pgfqpoint{1.063348in}{1.461481in}}%
\pgfpathlineto{\pgfqpoint{1.063348in}{1.475092in}}%
\pgfpathlineto{\pgfqpoint{1.064960in}{1.481511in}}%
\pgfpathlineto{\pgfqpoint{1.067105in}{1.488703in}}%
\pgfpathlineto{\pgfqpoint{1.075205in}{1.502314in}}%
\pgfpathlineto{\pgfqpoint{1.080617in}{1.508433in}}%
\pgfpathlineto{\pgfqpoint{1.089235in}{1.515925in}}%
\pgfpathlineto{\pgfqpoint{1.096274in}{1.520630in}}%
\pgfpathlineto{\pgfqpoint{1.111930in}{1.527672in}}%
\pgfpathlineto{\pgfqpoint{1.120202in}{1.529536in}}%
\pgfpathlineto{\pgfqpoint{1.127587in}{1.530938in}}%
\pgfpathlineto{\pgfqpoint{1.143243in}{1.530938in}}%
\pgfpathlineto{\pgfqpoint{1.150628in}{1.529536in}}%
\pgfpathlineto{\pgfqpoint{1.158900in}{1.527672in}}%
\pgfpathlineto{\pgfqpoint{1.174556in}{1.520630in}}%
\pgfpathlineto{\pgfqpoint{1.181595in}{1.515925in}}%
\pgfpathlineto{\pgfqpoint{1.190213in}{1.508433in}}%
\pgfpathlineto{\pgfqpoint{1.195625in}{1.502314in}}%
\pgfpathlineto{\pgfqpoint{1.203725in}{1.488703in}}%
\pgfpathlineto{\pgfqpoint{1.205870in}{1.481511in}}%
\pgfpathlineto{\pgfqpoint{1.207482in}{1.475092in}}%
\pgfpathlineto{\pgfqpoint{1.207482in}{1.461481in}}%
\pgfpathlineto{\pgfqpoint{1.205870in}{1.455061in}}%
\pgfpathlineto{\pgfqpoint{1.203725in}{1.447870in}}%
\pgfpathlineto{\pgfqpoint{1.195625in}{1.434259in}}%
\pgfpathlineto{\pgfqpoint{1.190213in}{1.428140in}}%
\pgfpathlineto{\pgfqpoint{1.181595in}{1.420648in}}%
\pgfpathlineto{\pgfqpoint{1.174556in}{1.415943in}}%
\pgfpathlineto{\pgfqpoint{1.158900in}{1.408900in}}%
\pgfpathlineto{\pgfqpoint{1.150628in}{1.407036in}}%
\pgfpathlineto{\pgfqpoint{1.143243in}{1.405635in}}%
\pgfpathlineto{\pgfqpoint{1.127587in}{1.405635in}}%
\pgfpathlineto{\pgfqpoint{1.120202in}{1.407036in}}%
\pgfpathclose%
\pgfpathmoveto{\pgfqpoint{1.628597in}{1.376150in}}%
\pgfpathlineto{\pgfqpoint{1.644253in}{1.373839in}}%
\pgfpathlineto{\pgfqpoint{1.659910in}{1.373839in}}%
\pgfpathlineto{\pgfqpoint{1.675567in}{1.376150in}}%
\pgfpathlineto{\pgfqpoint{1.687994in}{1.379814in}}%
\pgfpathlineto{\pgfqpoint{1.691223in}{1.380844in}}%
\pgfpathlineto{\pgfqpoint{1.706880in}{1.388264in}}%
\pgfpathlineto{\pgfqpoint{1.715142in}{1.393425in}}%
\pgfpathlineto{\pgfqpoint{1.722536in}{1.398710in}}%
\pgfpathlineto{\pgfqpoint{1.732114in}{1.407036in}}%
\pgfpathlineto{\pgfqpoint{1.738193in}{1.413464in}}%
\pgfpathlineto{\pgfqpoint{1.744130in}{1.420648in}}%
\pgfpathlineto{\pgfqpoint{1.752665in}{1.434259in}}%
\pgfpathlineto{\pgfqpoint{1.753849in}{1.437066in}}%
\pgfpathlineto{\pgfqpoint{1.758064in}{1.447870in}}%
\pgfpathlineto{\pgfqpoint{1.760723in}{1.461481in}}%
\pgfpathlineto{\pgfqpoint{1.760723in}{1.475092in}}%
\pgfpathlineto{\pgfqpoint{1.758064in}{1.488703in}}%
\pgfpathlineto{\pgfqpoint{1.753849in}{1.499507in}}%
\pgfpathlineto{\pgfqpoint{1.752665in}{1.502314in}}%
\pgfpathlineto{\pgfqpoint{1.744130in}{1.515925in}}%
\pgfpathlineto{\pgfqpoint{1.738193in}{1.523109in}}%
\pgfpathlineto{\pgfqpoint{1.732114in}{1.529536in}}%
\pgfpathlineto{\pgfqpoint{1.722536in}{1.537863in}}%
\pgfpathlineto{\pgfqpoint{1.715142in}{1.543148in}}%
\pgfpathlineto{\pgfqpoint{1.706880in}{1.548309in}}%
\pgfpathlineto{\pgfqpoint{1.691223in}{1.555729in}}%
\pgfpathlineto{\pgfqpoint{1.687994in}{1.556759in}}%
\pgfpathlineto{\pgfqpoint{1.675567in}{1.560423in}}%
\pgfpathlineto{\pgfqpoint{1.659910in}{1.562734in}}%
\pgfpathlineto{\pgfqpoint{1.644253in}{1.562734in}}%
\pgfpathlineto{\pgfqpoint{1.628597in}{1.560423in}}%
\pgfpathlineto{\pgfqpoint{1.616169in}{1.556759in}}%
\pgfpathlineto{\pgfqpoint{1.612940in}{1.555729in}}%
\pgfpathlineto{\pgfqpoint{1.597284in}{1.548309in}}%
\pgfpathlineto{\pgfqpoint{1.589021in}{1.543148in}}%
\pgfpathlineto{\pgfqpoint{1.581627in}{1.537863in}}%
\pgfpathlineto{\pgfqpoint{1.572050in}{1.529536in}}%
\pgfpathlineto{\pgfqpoint{1.565971in}{1.523109in}}%
\pgfpathlineto{\pgfqpoint{1.560034in}{1.515925in}}%
\pgfpathlineto{\pgfqpoint{1.551498in}{1.502314in}}%
\pgfpathlineto{\pgfqpoint{1.550314in}{1.499507in}}%
\pgfpathlineto{\pgfqpoint{1.546099in}{1.488703in}}%
\pgfpathlineto{\pgfqpoint{1.543440in}{1.475092in}}%
\pgfpathlineto{\pgfqpoint{1.543440in}{1.461481in}}%
\pgfpathlineto{\pgfqpoint{1.546099in}{1.447870in}}%
\pgfpathlineto{\pgfqpoint{1.550314in}{1.437066in}}%
\pgfpathlineto{\pgfqpoint{1.551498in}{1.434259in}}%
\pgfpathlineto{\pgfqpoint{1.560034in}{1.420648in}}%
\pgfpathlineto{\pgfqpoint{1.565971in}{1.413464in}}%
\pgfpathlineto{\pgfqpoint{1.572050in}{1.407036in}}%
\pgfpathlineto{\pgfqpoint{1.581627in}{1.398710in}}%
\pgfpathlineto{\pgfqpoint{1.589021in}{1.393425in}}%
\pgfpathlineto{\pgfqpoint{1.597284in}{1.388264in}}%
\pgfpathlineto{\pgfqpoint{1.612940in}{1.380844in}}%
\pgfpathlineto{\pgfqpoint{1.616169in}{1.379814in}}%
\pgfpathlineto{\pgfqpoint{1.628597in}{1.376150in}}%
\pgfpathclose%
\pgfpathmoveto{\pgfqpoint{1.636869in}{1.407036in}}%
\pgfpathlineto{\pgfqpoint{1.628597in}{1.408900in}}%
\pgfpathlineto{\pgfqpoint{1.612940in}{1.415943in}}%
\pgfpathlineto{\pgfqpoint{1.605902in}{1.420648in}}%
\pgfpathlineto{\pgfqpoint{1.597284in}{1.428140in}}%
\pgfpathlineto{\pgfqpoint{1.591872in}{1.434259in}}%
\pgfpathlineto{\pgfqpoint{1.583771in}{1.447870in}}%
\pgfpathlineto{\pgfqpoint{1.581627in}{1.455061in}}%
\pgfpathlineto{\pgfqpoint{1.580015in}{1.461481in}}%
\pgfpathlineto{\pgfqpoint{1.580015in}{1.475092in}}%
\pgfpathlineto{\pgfqpoint{1.581627in}{1.481511in}}%
\pgfpathlineto{\pgfqpoint{1.583771in}{1.488703in}}%
\pgfpathlineto{\pgfqpoint{1.591872in}{1.502314in}}%
\pgfpathlineto{\pgfqpoint{1.597284in}{1.508433in}}%
\pgfpathlineto{\pgfqpoint{1.605902in}{1.515925in}}%
\pgfpathlineto{\pgfqpoint{1.612940in}{1.520630in}}%
\pgfpathlineto{\pgfqpoint{1.628597in}{1.527672in}}%
\pgfpathlineto{\pgfqpoint{1.636869in}{1.529536in}}%
\pgfpathlineto{\pgfqpoint{1.644253in}{1.530938in}}%
\pgfpathlineto{\pgfqpoint{1.659910in}{1.530938in}}%
\pgfpathlineto{\pgfqpoint{1.667294in}{1.529536in}}%
\pgfpathlineto{\pgfqpoint{1.675567in}{1.527672in}}%
\pgfpathlineto{\pgfqpoint{1.691223in}{1.520630in}}%
\pgfpathlineto{\pgfqpoint{1.698261in}{1.515925in}}%
\pgfpathlineto{\pgfqpoint{1.706880in}{1.508433in}}%
\pgfpathlineto{\pgfqpoint{1.712291in}{1.502314in}}%
\pgfpathlineto{\pgfqpoint{1.720392in}{1.488703in}}%
\pgfpathlineto{\pgfqpoint{1.722536in}{1.481511in}}%
\pgfpathlineto{\pgfqpoint{1.724148in}{1.475092in}}%
\pgfpathlineto{\pgfqpoint{1.724148in}{1.461481in}}%
\pgfpathlineto{\pgfqpoint{1.722536in}{1.455061in}}%
\pgfpathlineto{\pgfqpoint{1.720392in}{1.447870in}}%
\pgfpathlineto{\pgfqpoint{1.712291in}{1.434259in}}%
\pgfpathlineto{\pgfqpoint{1.706880in}{1.428140in}}%
\pgfpathlineto{\pgfqpoint{1.698261in}{1.420648in}}%
\pgfpathlineto{\pgfqpoint{1.691223in}{1.415943in}}%
\pgfpathlineto{\pgfqpoint{1.675567in}{1.408900in}}%
\pgfpathlineto{\pgfqpoint{1.667294in}{1.407036in}}%
\pgfpathlineto{\pgfqpoint{1.659910in}{1.405635in}}%
\pgfpathlineto{\pgfqpoint{1.644253in}{1.405635in}}%
\pgfpathlineto{\pgfqpoint{1.636869in}{1.407036in}}%
\pgfpathclose%
\pgfusepath{fill}%
\end{pgfscope}%
\begin{pgfscope}%
\pgfpathrectangle{\pgfqpoint{0.360415in}{0.345370in}}{\pgfqpoint{1.550000in}{1.347500in}}%
\pgfusepath{clip}%
\pgfsetbuttcap%
\pgfsetroundjoin%
\definecolor{currentfill}{rgb}{0.959424,0.543431,0.278701}%
\pgfsetfillcolor{currentfill}%
\pgfsetlinewidth{0.000000pt}%
\definecolor{currentstroke}{rgb}{0.000000,0.000000,0.000000}%
\pgfsetstrokecolor{currentstroke}%
\pgfsetdash{}{0pt}%
\pgfpathmoveto{\pgfqpoint{0.579607in}{0.453266in}}%
\pgfpathlineto{\pgfqpoint{0.595263in}{0.449091in}}%
\pgfpathlineto{\pgfqpoint{0.610920in}{0.447006in}}%
\pgfpathlineto{\pgfqpoint{0.626577in}{0.447006in}}%
\pgfpathlineto{\pgfqpoint{0.642233in}{0.449091in}}%
\pgfpathlineto{\pgfqpoint{0.657890in}{0.453266in}}%
\pgfpathlineto{\pgfqpoint{0.660410in}{0.454259in}}%
\pgfpathlineto{\pgfqpoint{0.673546in}{0.459483in}}%
\pgfpathlineto{\pgfqpoint{0.689203in}{0.467762in}}%
\pgfpathlineto{\pgfqpoint{0.689371in}{0.467870in}}%
\pgfpathlineto{\pgfqpoint{0.704859in}{0.478364in}}%
\pgfpathlineto{\pgfqpoint{0.708824in}{0.481481in}}%
\pgfpathlineto{\pgfqpoint{0.720516in}{0.491646in}}%
\pgfpathlineto{\pgfqpoint{0.724101in}{0.495092in}}%
\pgfpathlineto{\pgfqpoint{0.736173in}{0.508557in}}%
\pgfpathlineto{\pgfqpoint{0.736296in}{0.508703in}}%
\pgfpathlineto{\pgfqpoint{0.745820in}{0.522314in}}%
\pgfpathlineto{\pgfqpoint{0.751829in}{0.533734in}}%
\pgfpathlineto{\pgfqpoint{0.752971in}{0.535925in}}%
\pgfpathlineto{\pgfqpoint{0.757773in}{0.549536in}}%
\pgfpathlineto{\pgfqpoint{0.760171in}{0.563148in}}%
\pgfpathlineto{\pgfqpoint{0.760171in}{0.576759in}}%
\pgfpathlineto{\pgfqpoint{0.757773in}{0.590370in}}%
\pgfpathlineto{\pgfqpoint{0.752971in}{0.603981in}}%
\pgfpathlineto{\pgfqpoint{0.751829in}{0.606172in}}%
\pgfpathlineto{\pgfqpoint{0.745820in}{0.617592in}}%
\pgfpathlineto{\pgfqpoint{0.736296in}{0.631203in}}%
\pgfpathlineto{\pgfqpoint{0.736173in}{0.631349in}}%
\pgfpathlineto{\pgfqpoint{0.724101in}{0.644814in}}%
\pgfpathlineto{\pgfqpoint{0.720516in}{0.648261in}}%
\pgfpathlineto{\pgfqpoint{0.708824in}{0.658425in}}%
\pgfpathlineto{\pgfqpoint{0.704859in}{0.661542in}}%
\pgfpathlineto{\pgfqpoint{0.689371in}{0.672036in}}%
\pgfpathlineto{\pgfqpoint{0.689203in}{0.672144in}}%
\pgfpathlineto{\pgfqpoint{0.673546in}{0.680424in}}%
\pgfpathlineto{\pgfqpoint{0.660410in}{0.685648in}}%
\pgfpathlineto{\pgfqpoint{0.657890in}{0.686641in}}%
\pgfpathlineto{\pgfqpoint{0.642233in}{0.690815in}}%
\pgfpathlineto{\pgfqpoint{0.626577in}{0.692900in}}%
\pgfpathlineto{\pgfqpoint{0.610920in}{0.692900in}}%
\pgfpathlineto{\pgfqpoint{0.595263in}{0.690815in}}%
\pgfpathlineto{\pgfqpoint{0.579607in}{0.686641in}}%
\pgfpathlineto{\pgfqpoint{0.577087in}{0.685648in}}%
\pgfpathlineto{\pgfqpoint{0.563950in}{0.680424in}}%
\pgfpathlineto{\pgfqpoint{0.548294in}{0.672144in}}%
\pgfpathlineto{\pgfqpoint{0.548126in}{0.672036in}}%
\pgfpathlineto{\pgfqpoint{0.532637in}{0.661542in}}%
\pgfpathlineto{\pgfqpoint{0.528673in}{0.658425in}}%
\pgfpathlineto{\pgfqpoint{0.516981in}{0.648261in}}%
\pgfpathlineto{\pgfqpoint{0.513396in}{0.644814in}}%
\pgfpathlineto{\pgfqpoint{0.501324in}{0.631349in}}%
\pgfpathlineto{\pgfqpoint{0.501200in}{0.631203in}}%
\pgfpathlineto{\pgfqpoint{0.491676in}{0.617592in}}%
\pgfpathlineto{\pgfqpoint{0.485668in}{0.606172in}}%
\pgfpathlineto{\pgfqpoint{0.484525in}{0.603981in}}%
\pgfpathlineto{\pgfqpoint{0.479723in}{0.590370in}}%
\pgfpathlineto{\pgfqpoint{0.477326in}{0.576759in}}%
\pgfpathlineto{\pgfqpoint{0.477326in}{0.563148in}}%
\pgfpathlineto{\pgfqpoint{0.479723in}{0.549536in}}%
\pgfpathlineto{\pgfqpoint{0.484525in}{0.535925in}}%
\pgfpathlineto{\pgfqpoint{0.485668in}{0.533734in}}%
\pgfpathlineto{\pgfqpoint{0.491676in}{0.522314in}}%
\pgfpathlineto{\pgfqpoint{0.501200in}{0.508703in}}%
\pgfpathlineto{\pgfqpoint{0.501324in}{0.508557in}}%
\pgfpathlineto{\pgfqpoint{0.513396in}{0.495092in}}%
\pgfpathlineto{\pgfqpoint{0.516981in}{0.491646in}}%
\pgfpathlineto{\pgfqpoint{0.528673in}{0.481481in}}%
\pgfpathlineto{\pgfqpoint{0.532637in}{0.478364in}}%
\pgfpathlineto{\pgfqpoint{0.548126in}{0.467870in}}%
\pgfpathlineto{\pgfqpoint{0.548294in}{0.467762in}}%
\pgfpathlineto{\pgfqpoint{0.563950in}{0.459483in}}%
\pgfpathlineto{\pgfqpoint{0.577087in}{0.454259in}}%
\pgfpathlineto{\pgfqpoint{0.579607in}{0.453266in}}%
\pgfpathclose%
\pgfpathmoveto{\pgfqpoint{0.582836in}{0.481481in}}%
\pgfpathlineto{\pgfqpoint{0.579607in}{0.482510in}}%
\pgfpathlineto{\pgfqpoint{0.563950in}{0.489931in}}%
\pgfpathlineto{\pgfqpoint{0.555688in}{0.495092in}}%
\pgfpathlineto{\pgfqpoint{0.548294in}{0.500377in}}%
\pgfpathlineto{\pgfqpoint{0.538716in}{0.508703in}}%
\pgfpathlineto{\pgfqpoint{0.532637in}{0.515131in}}%
\pgfpathlineto{\pgfqpoint{0.526700in}{0.522314in}}%
\pgfpathlineto{\pgfqpoint{0.518165in}{0.535925in}}%
\pgfpathlineto{\pgfqpoint{0.516981in}{0.538732in}}%
\pgfpathlineto{\pgfqpoint{0.512766in}{0.549536in}}%
\pgfpathlineto{\pgfqpoint{0.510107in}{0.563148in}}%
\pgfpathlineto{\pgfqpoint{0.510107in}{0.576759in}}%
\pgfpathlineto{\pgfqpoint{0.512766in}{0.590370in}}%
\pgfpathlineto{\pgfqpoint{0.516981in}{0.601174in}}%
\pgfpathlineto{\pgfqpoint{0.518165in}{0.603981in}}%
\pgfpathlineto{\pgfqpoint{0.526700in}{0.617592in}}%
\pgfpathlineto{\pgfqpoint{0.532637in}{0.624775in}}%
\pgfpathlineto{\pgfqpoint{0.538716in}{0.631203in}}%
\pgfpathlineto{\pgfqpoint{0.548294in}{0.639529in}}%
\pgfpathlineto{\pgfqpoint{0.555688in}{0.644814in}}%
\pgfpathlineto{\pgfqpoint{0.563950in}{0.649976in}}%
\pgfpathlineto{\pgfqpoint{0.579607in}{0.657396in}}%
\pgfpathlineto{\pgfqpoint{0.582836in}{0.658425in}}%
\pgfpathlineto{\pgfqpoint{0.595263in}{0.662089in}}%
\pgfpathlineto{\pgfqpoint{0.610920in}{0.664401in}}%
\pgfpathlineto{\pgfqpoint{0.626577in}{0.664401in}}%
\pgfpathlineto{\pgfqpoint{0.642233in}{0.662089in}}%
\pgfpathlineto{\pgfqpoint{0.654661in}{0.658425in}}%
\pgfpathlineto{\pgfqpoint{0.657890in}{0.657396in}}%
\pgfpathlineto{\pgfqpoint{0.673546in}{0.649976in}}%
\pgfpathlineto{\pgfqpoint{0.681809in}{0.644814in}}%
\pgfpathlineto{\pgfqpoint{0.689203in}{0.639529in}}%
\pgfpathlineto{\pgfqpoint{0.698780in}{0.631203in}}%
\pgfpathlineto{\pgfqpoint{0.704859in}{0.624775in}}%
\pgfpathlineto{\pgfqpoint{0.710796in}{0.617592in}}%
\pgfpathlineto{\pgfqpoint{0.719332in}{0.603981in}}%
\pgfpathlineto{\pgfqpoint{0.720516in}{0.601174in}}%
\pgfpathlineto{\pgfqpoint{0.724731in}{0.590370in}}%
\pgfpathlineto{\pgfqpoint{0.727390in}{0.576759in}}%
\pgfpathlineto{\pgfqpoint{0.727390in}{0.563148in}}%
\pgfpathlineto{\pgfqpoint{0.724731in}{0.549536in}}%
\pgfpathlineto{\pgfqpoint{0.720516in}{0.538732in}}%
\pgfpathlineto{\pgfqpoint{0.719332in}{0.535925in}}%
\pgfpathlineto{\pgfqpoint{0.710796in}{0.522314in}}%
\pgfpathlineto{\pgfqpoint{0.704859in}{0.515131in}}%
\pgfpathlineto{\pgfqpoint{0.698780in}{0.508703in}}%
\pgfpathlineto{\pgfqpoint{0.689203in}{0.500377in}}%
\pgfpathlineto{\pgfqpoint{0.681809in}{0.495092in}}%
\pgfpathlineto{\pgfqpoint{0.673546in}{0.489931in}}%
\pgfpathlineto{\pgfqpoint{0.657890in}{0.482510in}}%
\pgfpathlineto{\pgfqpoint{0.654661in}{0.481481in}}%
\pgfpathlineto{\pgfqpoint{0.642233in}{0.477817in}}%
\pgfpathlineto{\pgfqpoint{0.626577in}{0.475505in}}%
\pgfpathlineto{\pgfqpoint{0.610920in}{0.475505in}}%
\pgfpathlineto{\pgfqpoint{0.595263in}{0.477817in}}%
\pgfpathlineto{\pgfqpoint{0.582836in}{0.481481in}}%
\pgfpathclose%
\pgfpathmoveto{\pgfqpoint{1.096274in}{0.453266in}}%
\pgfpathlineto{\pgfqpoint{1.111930in}{0.449091in}}%
\pgfpathlineto{\pgfqpoint{1.127587in}{0.447006in}}%
\pgfpathlineto{\pgfqpoint{1.143243in}{0.447006in}}%
\pgfpathlineto{\pgfqpoint{1.158900in}{0.449091in}}%
\pgfpathlineto{\pgfqpoint{1.174556in}{0.453266in}}%
\pgfpathlineto{\pgfqpoint{1.177077in}{0.454259in}}%
\pgfpathlineto{\pgfqpoint{1.190213in}{0.459483in}}%
\pgfpathlineto{\pgfqpoint{1.205870in}{0.467762in}}%
\pgfpathlineto{\pgfqpoint{1.206038in}{0.467870in}}%
\pgfpathlineto{\pgfqpoint{1.221526in}{0.478364in}}%
\pgfpathlineto{\pgfqpoint{1.225490in}{0.481481in}}%
\pgfpathlineto{\pgfqpoint{1.237183in}{0.491646in}}%
\pgfpathlineto{\pgfqpoint{1.240768in}{0.495092in}}%
\pgfpathlineto{\pgfqpoint{1.252839in}{0.508557in}}%
\pgfpathlineto{\pgfqpoint{1.252963in}{0.508703in}}%
\pgfpathlineto{\pgfqpoint{1.262487in}{0.522314in}}%
\pgfpathlineto{\pgfqpoint{1.268496in}{0.533734in}}%
\pgfpathlineto{\pgfqpoint{1.269638in}{0.535925in}}%
\pgfpathlineto{\pgfqpoint{1.274440in}{0.549536in}}%
\pgfpathlineto{\pgfqpoint{1.276838in}{0.563148in}}%
\pgfpathlineto{\pgfqpoint{1.276838in}{0.576759in}}%
\pgfpathlineto{\pgfqpoint{1.274440in}{0.590370in}}%
\pgfpathlineto{\pgfqpoint{1.269638in}{0.603981in}}%
\pgfpathlineto{\pgfqpoint{1.268496in}{0.606172in}}%
\pgfpathlineto{\pgfqpoint{1.262487in}{0.617592in}}%
\pgfpathlineto{\pgfqpoint{1.252963in}{0.631203in}}%
\pgfpathlineto{\pgfqpoint{1.252839in}{0.631349in}}%
\pgfpathlineto{\pgfqpoint{1.240768in}{0.644814in}}%
\pgfpathlineto{\pgfqpoint{1.237183in}{0.648261in}}%
\pgfpathlineto{\pgfqpoint{1.225490in}{0.658425in}}%
\pgfpathlineto{\pgfqpoint{1.221526in}{0.661542in}}%
\pgfpathlineto{\pgfqpoint{1.206038in}{0.672036in}}%
\pgfpathlineto{\pgfqpoint{1.205870in}{0.672144in}}%
\pgfpathlineto{\pgfqpoint{1.190213in}{0.680424in}}%
\pgfpathlineto{\pgfqpoint{1.177077in}{0.685648in}}%
\pgfpathlineto{\pgfqpoint{1.174556in}{0.686641in}}%
\pgfpathlineto{\pgfqpoint{1.158900in}{0.690815in}}%
\pgfpathlineto{\pgfqpoint{1.143243in}{0.692900in}}%
\pgfpathlineto{\pgfqpoint{1.127587in}{0.692900in}}%
\pgfpathlineto{\pgfqpoint{1.111930in}{0.690815in}}%
\pgfpathlineto{\pgfqpoint{1.096274in}{0.686641in}}%
\pgfpathlineto{\pgfqpoint{1.093753in}{0.685648in}}%
\pgfpathlineto{\pgfqpoint{1.080617in}{0.680424in}}%
\pgfpathlineto{\pgfqpoint{1.064960in}{0.672144in}}%
\pgfpathlineto{\pgfqpoint{1.064792in}{0.672036in}}%
\pgfpathlineto{\pgfqpoint{1.049304in}{0.661542in}}%
\pgfpathlineto{\pgfqpoint{1.045340in}{0.658425in}}%
\pgfpathlineto{\pgfqpoint{1.033647in}{0.648261in}}%
\pgfpathlineto{\pgfqpoint{1.030062in}{0.644814in}}%
\pgfpathlineto{\pgfqpoint{1.017991in}{0.631349in}}%
\pgfpathlineto{\pgfqpoint{1.017867in}{0.631203in}}%
\pgfpathlineto{\pgfqpoint{1.008343in}{0.617592in}}%
\pgfpathlineto{\pgfqpoint{1.002334in}{0.606172in}}%
\pgfpathlineto{\pgfqpoint{1.001192in}{0.603981in}}%
\pgfpathlineto{\pgfqpoint{0.996390in}{0.590370in}}%
\pgfpathlineto{\pgfqpoint{0.993992in}{0.576759in}}%
\pgfpathlineto{\pgfqpoint{0.993992in}{0.563148in}}%
\pgfpathlineto{\pgfqpoint{0.996390in}{0.549536in}}%
\pgfpathlineto{\pgfqpoint{1.001192in}{0.535925in}}%
\pgfpathlineto{\pgfqpoint{1.002334in}{0.533734in}}%
\pgfpathlineto{\pgfqpoint{1.008343in}{0.522314in}}%
\pgfpathlineto{\pgfqpoint{1.017867in}{0.508703in}}%
\pgfpathlineto{\pgfqpoint{1.017991in}{0.508557in}}%
\pgfpathlineto{\pgfqpoint{1.030062in}{0.495092in}}%
\pgfpathlineto{\pgfqpoint{1.033647in}{0.491646in}}%
\pgfpathlineto{\pgfqpoint{1.045340in}{0.481481in}}%
\pgfpathlineto{\pgfqpoint{1.049304in}{0.478364in}}%
\pgfpathlineto{\pgfqpoint{1.064792in}{0.467870in}}%
\pgfpathlineto{\pgfqpoint{1.064960in}{0.467762in}}%
\pgfpathlineto{\pgfqpoint{1.080617in}{0.459483in}}%
\pgfpathlineto{\pgfqpoint{1.093753in}{0.454259in}}%
\pgfpathlineto{\pgfqpoint{1.096274in}{0.453266in}}%
\pgfpathclose%
\pgfpathmoveto{\pgfqpoint{1.099502in}{0.481481in}}%
\pgfpathlineto{\pgfqpoint{1.096274in}{0.482510in}}%
\pgfpathlineto{\pgfqpoint{1.080617in}{0.489931in}}%
\pgfpathlineto{\pgfqpoint{1.072354in}{0.495092in}}%
\pgfpathlineto{\pgfqpoint{1.064960in}{0.500377in}}%
\pgfpathlineto{\pgfqpoint{1.055383in}{0.508703in}}%
\pgfpathlineto{\pgfqpoint{1.049304in}{0.515131in}}%
\pgfpathlineto{\pgfqpoint{1.043367in}{0.522314in}}%
\pgfpathlineto{\pgfqpoint{1.034831in}{0.535925in}}%
\pgfpathlineto{\pgfqpoint{1.033647in}{0.538732in}}%
\pgfpathlineto{\pgfqpoint{1.029433in}{0.549536in}}%
\pgfpathlineto{\pgfqpoint{1.026774in}{0.563148in}}%
\pgfpathlineto{\pgfqpoint{1.026774in}{0.576759in}}%
\pgfpathlineto{\pgfqpoint{1.029433in}{0.590370in}}%
\pgfpathlineto{\pgfqpoint{1.033647in}{0.601174in}}%
\pgfpathlineto{\pgfqpoint{1.034831in}{0.603981in}}%
\pgfpathlineto{\pgfqpoint{1.043367in}{0.617592in}}%
\pgfpathlineto{\pgfqpoint{1.049304in}{0.624775in}}%
\pgfpathlineto{\pgfqpoint{1.055383in}{0.631203in}}%
\pgfpathlineto{\pgfqpoint{1.064960in}{0.639529in}}%
\pgfpathlineto{\pgfqpoint{1.072354in}{0.644814in}}%
\pgfpathlineto{\pgfqpoint{1.080617in}{0.649976in}}%
\pgfpathlineto{\pgfqpoint{1.096274in}{0.657396in}}%
\pgfpathlineto{\pgfqpoint{1.099502in}{0.658425in}}%
\pgfpathlineto{\pgfqpoint{1.111930in}{0.662089in}}%
\pgfpathlineto{\pgfqpoint{1.127587in}{0.664401in}}%
\pgfpathlineto{\pgfqpoint{1.143243in}{0.664401in}}%
\pgfpathlineto{\pgfqpoint{1.158900in}{0.662089in}}%
\pgfpathlineto{\pgfqpoint{1.171328in}{0.658425in}}%
\pgfpathlineto{\pgfqpoint{1.174556in}{0.657396in}}%
\pgfpathlineto{\pgfqpoint{1.190213in}{0.649976in}}%
\pgfpathlineto{\pgfqpoint{1.198476in}{0.644814in}}%
\pgfpathlineto{\pgfqpoint{1.205870in}{0.639529in}}%
\pgfpathlineto{\pgfqpoint{1.215447in}{0.631203in}}%
\pgfpathlineto{\pgfqpoint{1.221526in}{0.624775in}}%
\pgfpathlineto{\pgfqpoint{1.227463in}{0.617592in}}%
\pgfpathlineto{\pgfqpoint{1.235999in}{0.603981in}}%
\pgfpathlineto{\pgfqpoint{1.237183in}{0.601174in}}%
\pgfpathlineto{\pgfqpoint{1.241397in}{0.590370in}}%
\pgfpathlineto{\pgfqpoint{1.244056in}{0.576759in}}%
\pgfpathlineto{\pgfqpoint{1.244056in}{0.563148in}}%
\pgfpathlineto{\pgfqpoint{1.241397in}{0.549536in}}%
\pgfpathlineto{\pgfqpoint{1.237183in}{0.538732in}}%
\pgfpathlineto{\pgfqpoint{1.235999in}{0.535925in}}%
\pgfpathlineto{\pgfqpoint{1.227463in}{0.522314in}}%
\pgfpathlineto{\pgfqpoint{1.221526in}{0.515131in}}%
\pgfpathlineto{\pgfqpoint{1.215447in}{0.508703in}}%
\pgfpathlineto{\pgfqpoint{1.205870in}{0.500377in}}%
\pgfpathlineto{\pgfqpoint{1.198476in}{0.495092in}}%
\pgfpathlineto{\pgfqpoint{1.190213in}{0.489931in}}%
\pgfpathlineto{\pgfqpoint{1.174556in}{0.482510in}}%
\pgfpathlineto{\pgfqpoint{1.171328in}{0.481481in}}%
\pgfpathlineto{\pgfqpoint{1.158900in}{0.477817in}}%
\pgfpathlineto{\pgfqpoint{1.143243in}{0.475505in}}%
\pgfpathlineto{\pgfqpoint{1.127587in}{0.475505in}}%
\pgfpathlineto{\pgfqpoint{1.111930in}{0.477817in}}%
\pgfpathlineto{\pgfqpoint{1.099502in}{0.481481in}}%
\pgfpathclose%
\pgfpathmoveto{\pgfqpoint{1.612940in}{0.453266in}}%
\pgfpathlineto{\pgfqpoint{1.628597in}{0.449091in}}%
\pgfpathlineto{\pgfqpoint{1.644253in}{0.447006in}}%
\pgfpathlineto{\pgfqpoint{1.659910in}{0.447006in}}%
\pgfpathlineto{\pgfqpoint{1.675567in}{0.449091in}}%
\pgfpathlineto{\pgfqpoint{1.691223in}{0.453266in}}%
\pgfpathlineto{\pgfqpoint{1.693743in}{0.454259in}}%
\pgfpathlineto{\pgfqpoint{1.706880in}{0.459483in}}%
\pgfpathlineto{\pgfqpoint{1.722536in}{0.467762in}}%
\pgfpathlineto{\pgfqpoint{1.722704in}{0.467870in}}%
\pgfpathlineto{\pgfqpoint{1.738193in}{0.478364in}}%
\pgfpathlineto{\pgfqpoint{1.742157in}{0.481481in}}%
\pgfpathlineto{\pgfqpoint{1.753849in}{0.491646in}}%
\pgfpathlineto{\pgfqpoint{1.757434in}{0.495092in}}%
\pgfpathlineto{\pgfqpoint{1.769506in}{0.508557in}}%
\pgfpathlineto{\pgfqpoint{1.769630in}{0.508703in}}%
\pgfpathlineto{\pgfqpoint{1.779154in}{0.522314in}}%
\pgfpathlineto{\pgfqpoint{1.785162in}{0.533734in}}%
\pgfpathlineto{\pgfqpoint{1.786305in}{0.535925in}}%
\pgfpathlineto{\pgfqpoint{1.791107in}{0.549536in}}%
\pgfpathlineto{\pgfqpoint{1.793504in}{0.563148in}}%
\pgfpathlineto{\pgfqpoint{1.793504in}{0.576759in}}%
\pgfpathlineto{\pgfqpoint{1.791107in}{0.590370in}}%
\pgfpathlineto{\pgfqpoint{1.786305in}{0.603981in}}%
\pgfpathlineto{\pgfqpoint{1.785162in}{0.606172in}}%
\pgfpathlineto{\pgfqpoint{1.779154in}{0.617592in}}%
\pgfpathlineto{\pgfqpoint{1.769630in}{0.631203in}}%
\pgfpathlineto{\pgfqpoint{1.769506in}{0.631349in}}%
\pgfpathlineto{\pgfqpoint{1.757434in}{0.644814in}}%
\pgfpathlineto{\pgfqpoint{1.753849in}{0.648261in}}%
\pgfpathlineto{\pgfqpoint{1.742157in}{0.658425in}}%
\pgfpathlineto{\pgfqpoint{1.738193in}{0.661542in}}%
\pgfpathlineto{\pgfqpoint{1.722704in}{0.672036in}}%
\pgfpathlineto{\pgfqpoint{1.722536in}{0.672144in}}%
\pgfpathlineto{\pgfqpoint{1.706880in}{0.680424in}}%
\pgfpathlineto{\pgfqpoint{1.693743in}{0.685648in}}%
\pgfpathlineto{\pgfqpoint{1.691223in}{0.686641in}}%
\pgfpathlineto{\pgfqpoint{1.675567in}{0.690815in}}%
\pgfpathlineto{\pgfqpoint{1.659910in}{0.692900in}}%
\pgfpathlineto{\pgfqpoint{1.644253in}{0.692900in}}%
\pgfpathlineto{\pgfqpoint{1.628597in}{0.690815in}}%
\pgfpathlineto{\pgfqpoint{1.612940in}{0.686641in}}%
\pgfpathlineto{\pgfqpoint{1.610420in}{0.685648in}}%
\pgfpathlineto{\pgfqpoint{1.597284in}{0.680424in}}%
\pgfpathlineto{\pgfqpoint{1.581627in}{0.672144in}}%
\pgfpathlineto{\pgfqpoint{1.581459in}{0.672036in}}%
\pgfpathlineto{\pgfqpoint{1.565971in}{0.661542in}}%
\pgfpathlineto{\pgfqpoint{1.562006in}{0.658425in}}%
\pgfpathlineto{\pgfqpoint{1.550314in}{0.648261in}}%
\pgfpathlineto{\pgfqpoint{1.546729in}{0.644814in}}%
\pgfpathlineto{\pgfqpoint{1.534657in}{0.631349in}}%
\pgfpathlineto{\pgfqpoint{1.534534in}{0.631203in}}%
\pgfpathlineto{\pgfqpoint{1.525010in}{0.617592in}}%
\pgfpathlineto{\pgfqpoint{1.519001in}{0.606172in}}%
\pgfpathlineto{\pgfqpoint{1.517859in}{0.603981in}}%
\pgfpathlineto{\pgfqpoint{1.513057in}{0.590370in}}%
\pgfpathlineto{\pgfqpoint{1.510659in}{0.576759in}}%
\pgfpathlineto{\pgfqpoint{1.510659in}{0.563148in}}%
\pgfpathlineto{\pgfqpoint{1.513057in}{0.549536in}}%
\pgfpathlineto{\pgfqpoint{1.517859in}{0.535925in}}%
\pgfpathlineto{\pgfqpoint{1.519001in}{0.533734in}}%
\pgfpathlineto{\pgfqpoint{1.525010in}{0.522314in}}%
\pgfpathlineto{\pgfqpoint{1.534534in}{0.508703in}}%
\pgfpathlineto{\pgfqpoint{1.534657in}{0.508557in}}%
\pgfpathlineto{\pgfqpoint{1.546729in}{0.495092in}}%
\pgfpathlineto{\pgfqpoint{1.550314in}{0.491646in}}%
\pgfpathlineto{\pgfqpoint{1.562006in}{0.481481in}}%
\pgfpathlineto{\pgfqpoint{1.565971in}{0.478364in}}%
\pgfpathlineto{\pgfqpoint{1.581459in}{0.467870in}}%
\pgfpathlineto{\pgfqpoint{1.581627in}{0.467762in}}%
\pgfpathlineto{\pgfqpoint{1.597284in}{0.459483in}}%
\pgfpathlineto{\pgfqpoint{1.610420in}{0.454259in}}%
\pgfpathlineto{\pgfqpoint{1.612940in}{0.453266in}}%
\pgfpathclose%
\pgfpathmoveto{\pgfqpoint{1.616169in}{0.481481in}}%
\pgfpathlineto{\pgfqpoint{1.612940in}{0.482510in}}%
\pgfpathlineto{\pgfqpoint{1.597284in}{0.489931in}}%
\pgfpathlineto{\pgfqpoint{1.589021in}{0.495092in}}%
\pgfpathlineto{\pgfqpoint{1.581627in}{0.500377in}}%
\pgfpathlineto{\pgfqpoint{1.572050in}{0.508703in}}%
\pgfpathlineto{\pgfqpoint{1.565971in}{0.515131in}}%
\pgfpathlineto{\pgfqpoint{1.560034in}{0.522314in}}%
\pgfpathlineto{\pgfqpoint{1.551498in}{0.535925in}}%
\pgfpathlineto{\pgfqpoint{1.550314in}{0.538732in}}%
\pgfpathlineto{\pgfqpoint{1.546099in}{0.549536in}}%
\pgfpathlineto{\pgfqpoint{1.543440in}{0.563148in}}%
\pgfpathlineto{\pgfqpoint{1.543440in}{0.576759in}}%
\pgfpathlineto{\pgfqpoint{1.546099in}{0.590370in}}%
\pgfpathlineto{\pgfqpoint{1.550314in}{0.601174in}}%
\pgfpathlineto{\pgfqpoint{1.551498in}{0.603981in}}%
\pgfpathlineto{\pgfqpoint{1.560034in}{0.617592in}}%
\pgfpathlineto{\pgfqpoint{1.565971in}{0.624775in}}%
\pgfpathlineto{\pgfqpoint{1.572050in}{0.631203in}}%
\pgfpathlineto{\pgfqpoint{1.581627in}{0.639529in}}%
\pgfpathlineto{\pgfqpoint{1.589021in}{0.644814in}}%
\pgfpathlineto{\pgfqpoint{1.597284in}{0.649976in}}%
\pgfpathlineto{\pgfqpoint{1.612940in}{0.657396in}}%
\pgfpathlineto{\pgfqpoint{1.616169in}{0.658425in}}%
\pgfpathlineto{\pgfqpoint{1.628597in}{0.662089in}}%
\pgfpathlineto{\pgfqpoint{1.644253in}{0.664401in}}%
\pgfpathlineto{\pgfqpoint{1.659910in}{0.664401in}}%
\pgfpathlineto{\pgfqpoint{1.675567in}{0.662089in}}%
\pgfpathlineto{\pgfqpoint{1.687994in}{0.658425in}}%
\pgfpathlineto{\pgfqpoint{1.691223in}{0.657396in}}%
\pgfpathlineto{\pgfqpoint{1.706880in}{0.649976in}}%
\pgfpathlineto{\pgfqpoint{1.715142in}{0.644814in}}%
\pgfpathlineto{\pgfqpoint{1.722536in}{0.639529in}}%
\pgfpathlineto{\pgfqpoint{1.732114in}{0.631203in}}%
\pgfpathlineto{\pgfqpoint{1.738193in}{0.624775in}}%
\pgfpathlineto{\pgfqpoint{1.744130in}{0.617592in}}%
\pgfpathlineto{\pgfqpoint{1.752665in}{0.603981in}}%
\pgfpathlineto{\pgfqpoint{1.753849in}{0.601174in}}%
\pgfpathlineto{\pgfqpoint{1.758064in}{0.590370in}}%
\pgfpathlineto{\pgfqpoint{1.760723in}{0.576759in}}%
\pgfpathlineto{\pgfqpoint{1.760723in}{0.563148in}}%
\pgfpathlineto{\pgfqpoint{1.758064in}{0.549536in}}%
\pgfpathlineto{\pgfqpoint{1.753849in}{0.538732in}}%
\pgfpathlineto{\pgfqpoint{1.752665in}{0.535925in}}%
\pgfpathlineto{\pgfqpoint{1.744130in}{0.522314in}}%
\pgfpathlineto{\pgfqpoint{1.738193in}{0.515131in}}%
\pgfpathlineto{\pgfqpoint{1.732114in}{0.508703in}}%
\pgfpathlineto{\pgfqpoint{1.722536in}{0.500377in}}%
\pgfpathlineto{\pgfqpoint{1.715142in}{0.495092in}}%
\pgfpathlineto{\pgfqpoint{1.706880in}{0.489931in}}%
\pgfpathlineto{\pgfqpoint{1.691223in}{0.482510in}}%
\pgfpathlineto{\pgfqpoint{1.687994in}{0.481481in}}%
\pgfpathlineto{\pgfqpoint{1.675567in}{0.477817in}}%
\pgfpathlineto{\pgfqpoint{1.659910in}{0.475505in}}%
\pgfpathlineto{\pgfqpoint{1.644253in}{0.475505in}}%
\pgfpathlineto{\pgfqpoint{1.628597in}{0.477817in}}%
\pgfpathlineto{\pgfqpoint{1.616169in}{0.481481in}}%
\pgfpathclose%
\pgfpathmoveto{\pgfqpoint{0.579607in}{0.902432in}}%
\pgfpathlineto{\pgfqpoint{0.595263in}{0.898258in}}%
\pgfpathlineto{\pgfqpoint{0.610920in}{0.896173in}}%
\pgfpathlineto{\pgfqpoint{0.626577in}{0.896173in}}%
\pgfpathlineto{\pgfqpoint{0.642233in}{0.898258in}}%
\pgfpathlineto{\pgfqpoint{0.657890in}{0.902432in}}%
\pgfpathlineto{\pgfqpoint{0.660410in}{0.903425in}}%
\pgfpathlineto{\pgfqpoint{0.673546in}{0.908649in}}%
\pgfpathlineto{\pgfqpoint{0.689203in}{0.916929in}}%
\pgfpathlineto{\pgfqpoint{0.689371in}{0.917036in}}%
\pgfpathlineto{\pgfqpoint{0.704859in}{0.927531in}}%
\pgfpathlineto{\pgfqpoint{0.708824in}{0.930648in}}%
\pgfpathlineto{\pgfqpoint{0.720516in}{0.940812in}}%
\pgfpathlineto{\pgfqpoint{0.724101in}{0.944259in}}%
\pgfpathlineto{\pgfqpoint{0.736173in}{0.957724in}}%
\pgfpathlineto{\pgfqpoint{0.736296in}{0.957870in}}%
\pgfpathlineto{\pgfqpoint{0.745820in}{0.971481in}}%
\pgfpathlineto{\pgfqpoint{0.751829in}{0.982901in}}%
\pgfpathlineto{\pgfqpoint{0.752971in}{0.985092in}}%
\pgfpathlineto{\pgfqpoint{0.757773in}{0.998703in}}%
\pgfpathlineto{\pgfqpoint{0.760171in}{1.012314in}}%
\pgfpathlineto{\pgfqpoint{0.760171in}{1.025925in}}%
\pgfpathlineto{\pgfqpoint{0.757773in}{1.039536in}}%
\pgfpathlineto{\pgfqpoint{0.752971in}{1.053148in}}%
\pgfpathlineto{\pgfqpoint{0.751829in}{1.055339in}}%
\pgfpathlineto{\pgfqpoint{0.745820in}{1.066759in}}%
\pgfpathlineto{\pgfqpoint{0.736296in}{1.080370in}}%
\pgfpathlineto{\pgfqpoint{0.736173in}{1.080516in}}%
\pgfpathlineto{\pgfqpoint{0.724101in}{1.093981in}}%
\pgfpathlineto{\pgfqpoint{0.720516in}{1.097427in}}%
\pgfpathlineto{\pgfqpoint{0.708824in}{1.107592in}}%
\pgfpathlineto{\pgfqpoint{0.704859in}{1.110709in}}%
\pgfpathlineto{\pgfqpoint{0.689371in}{1.121203in}}%
\pgfpathlineto{\pgfqpoint{0.689203in}{1.121311in}}%
\pgfpathlineto{\pgfqpoint{0.673546in}{1.129590in}}%
\pgfpathlineto{\pgfqpoint{0.660410in}{1.134814in}}%
\pgfpathlineto{\pgfqpoint{0.657890in}{1.135807in}}%
\pgfpathlineto{\pgfqpoint{0.642233in}{1.139982in}}%
\pgfpathlineto{\pgfqpoint{0.626577in}{1.142066in}}%
\pgfpathlineto{\pgfqpoint{0.610920in}{1.142066in}}%
\pgfpathlineto{\pgfqpoint{0.595263in}{1.139982in}}%
\pgfpathlineto{\pgfqpoint{0.579607in}{1.135807in}}%
\pgfpathlineto{\pgfqpoint{0.577087in}{1.134814in}}%
\pgfpathlineto{\pgfqpoint{0.563950in}{1.129590in}}%
\pgfpathlineto{\pgfqpoint{0.548294in}{1.121311in}}%
\pgfpathlineto{\pgfqpoint{0.548126in}{1.121203in}}%
\pgfpathlineto{\pgfqpoint{0.532637in}{1.110709in}}%
\pgfpathlineto{\pgfqpoint{0.528673in}{1.107592in}}%
\pgfpathlineto{\pgfqpoint{0.516981in}{1.097427in}}%
\pgfpathlineto{\pgfqpoint{0.513396in}{1.093981in}}%
\pgfpathlineto{\pgfqpoint{0.501324in}{1.080516in}}%
\pgfpathlineto{\pgfqpoint{0.501200in}{1.080370in}}%
\pgfpathlineto{\pgfqpoint{0.491676in}{1.066759in}}%
\pgfpathlineto{\pgfqpoint{0.485668in}{1.055339in}}%
\pgfpathlineto{\pgfqpoint{0.484525in}{1.053148in}}%
\pgfpathlineto{\pgfqpoint{0.479723in}{1.039536in}}%
\pgfpathlineto{\pgfqpoint{0.477326in}{1.025925in}}%
\pgfpathlineto{\pgfqpoint{0.477326in}{1.012314in}}%
\pgfpathlineto{\pgfqpoint{0.479723in}{0.998703in}}%
\pgfpathlineto{\pgfqpoint{0.484525in}{0.985092in}}%
\pgfpathlineto{\pgfqpoint{0.485668in}{0.982901in}}%
\pgfpathlineto{\pgfqpoint{0.491676in}{0.971481in}}%
\pgfpathlineto{\pgfqpoint{0.501200in}{0.957870in}}%
\pgfpathlineto{\pgfqpoint{0.501324in}{0.957724in}}%
\pgfpathlineto{\pgfqpoint{0.513396in}{0.944259in}}%
\pgfpathlineto{\pgfqpoint{0.516981in}{0.940812in}}%
\pgfpathlineto{\pgfqpoint{0.528673in}{0.930648in}}%
\pgfpathlineto{\pgfqpoint{0.532637in}{0.927531in}}%
\pgfpathlineto{\pgfqpoint{0.548126in}{0.917036in}}%
\pgfpathlineto{\pgfqpoint{0.548294in}{0.916929in}}%
\pgfpathlineto{\pgfqpoint{0.563950in}{0.908649in}}%
\pgfpathlineto{\pgfqpoint{0.577087in}{0.903425in}}%
\pgfpathlineto{\pgfqpoint{0.579607in}{0.902432in}}%
\pgfpathclose%
\pgfpathmoveto{\pgfqpoint{0.582836in}{0.930648in}}%
\pgfpathlineto{\pgfqpoint{0.579607in}{0.931677in}}%
\pgfpathlineto{\pgfqpoint{0.563950in}{0.939097in}}%
\pgfpathlineto{\pgfqpoint{0.555688in}{0.944259in}}%
\pgfpathlineto{\pgfqpoint{0.548294in}{0.949543in}}%
\pgfpathlineto{\pgfqpoint{0.538716in}{0.957870in}}%
\pgfpathlineto{\pgfqpoint{0.532637in}{0.964298in}}%
\pgfpathlineto{\pgfqpoint{0.526700in}{0.971481in}}%
\pgfpathlineto{\pgfqpoint{0.518165in}{0.985092in}}%
\pgfpathlineto{\pgfqpoint{0.516981in}{0.987899in}}%
\pgfpathlineto{\pgfqpoint{0.512766in}{0.998703in}}%
\pgfpathlineto{\pgfqpoint{0.510107in}{1.012314in}}%
\pgfpathlineto{\pgfqpoint{0.510107in}{1.025925in}}%
\pgfpathlineto{\pgfqpoint{0.512766in}{1.039536in}}%
\pgfpathlineto{\pgfqpoint{0.516981in}{1.050341in}}%
\pgfpathlineto{\pgfqpoint{0.518165in}{1.053148in}}%
\pgfpathlineto{\pgfqpoint{0.526700in}{1.066759in}}%
\pgfpathlineto{\pgfqpoint{0.532637in}{1.073942in}}%
\pgfpathlineto{\pgfqpoint{0.538716in}{1.080370in}}%
\pgfpathlineto{\pgfqpoint{0.548294in}{1.088696in}}%
\pgfpathlineto{\pgfqpoint{0.555688in}{1.093981in}}%
\pgfpathlineto{\pgfqpoint{0.563950in}{1.099142in}}%
\pgfpathlineto{\pgfqpoint{0.579607in}{1.106563in}}%
\pgfpathlineto{\pgfqpoint{0.582836in}{1.107592in}}%
\pgfpathlineto{\pgfqpoint{0.595263in}{1.111256in}}%
\pgfpathlineto{\pgfqpoint{0.610920in}{1.113568in}}%
\pgfpathlineto{\pgfqpoint{0.626577in}{1.113568in}}%
\pgfpathlineto{\pgfqpoint{0.642233in}{1.111256in}}%
\pgfpathlineto{\pgfqpoint{0.654661in}{1.107592in}}%
\pgfpathlineto{\pgfqpoint{0.657890in}{1.106563in}}%
\pgfpathlineto{\pgfqpoint{0.673546in}{1.099142in}}%
\pgfpathlineto{\pgfqpoint{0.681809in}{1.093981in}}%
\pgfpathlineto{\pgfqpoint{0.689203in}{1.088696in}}%
\pgfpathlineto{\pgfqpoint{0.698780in}{1.080370in}}%
\pgfpathlineto{\pgfqpoint{0.704859in}{1.073942in}}%
\pgfpathlineto{\pgfqpoint{0.710796in}{1.066759in}}%
\pgfpathlineto{\pgfqpoint{0.719332in}{1.053148in}}%
\pgfpathlineto{\pgfqpoint{0.720516in}{1.050341in}}%
\pgfpathlineto{\pgfqpoint{0.724731in}{1.039536in}}%
\pgfpathlineto{\pgfqpoint{0.727390in}{1.025925in}}%
\pgfpathlineto{\pgfqpoint{0.727390in}{1.012314in}}%
\pgfpathlineto{\pgfqpoint{0.724731in}{0.998703in}}%
\pgfpathlineto{\pgfqpoint{0.720516in}{0.987899in}}%
\pgfpathlineto{\pgfqpoint{0.719332in}{0.985092in}}%
\pgfpathlineto{\pgfqpoint{0.710796in}{0.971481in}}%
\pgfpathlineto{\pgfqpoint{0.704859in}{0.964298in}}%
\pgfpathlineto{\pgfqpoint{0.698780in}{0.957870in}}%
\pgfpathlineto{\pgfqpoint{0.689203in}{0.949543in}}%
\pgfpathlineto{\pgfqpoint{0.681809in}{0.944259in}}%
\pgfpathlineto{\pgfqpoint{0.673546in}{0.939097in}}%
\pgfpathlineto{\pgfqpoint{0.657890in}{0.931677in}}%
\pgfpathlineto{\pgfqpoint{0.654661in}{0.930648in}}%
\pgfpathlineto{\pgfqpoint{0.642233in}{0.926983in}}%
\pgfpathlineto{\pgfqpoint{0.626577in}{0.924672in}}%
\pgfpathlineto{\pgfqpoint{0.610920in}{0.924672in}}%
\pgfpathlineto{\pgfqpoint{0.595263in}{0.926983in}}%
\pgfpathlineto{\pgfqpoint{0.582836in}{0.930648in}}%
\pgfpathclose%
\pgfpathmoveto{\pgfqpoint{1.096274in}{0.902432in}}%
\pgfpathlineto{\pgfqpoint{1.111930in}{0.898258in}}%
\pgfpathlineto{\pgfqpoint{1.127587in}{0.896173in}}%
\pgfpathlineto{\pgfqpoint{1.143243in}{0.896173in}}%
\pgfpathlineto{\pgfqpoint{1.158900in}{0.898258in}}%
\pgfpathlineto{\pgfqpoint{1.174556in}{0.902432in}}%
\pgfpathlineto{\pgfqpoint{1.177077in}{0.903425in}}%
\pgfpathlineto{\pgfqpoint{1.190213in}{0.908649in}}%
\pgfpathlineto{\pgfqpoint{1.205870in}{0.916929in}}%
\pgfpathlineto{\pgfqpoint{1.206038in}{0.917036in}}%
\pgfpathlineto{\pgfqpoint{1.221526in}{0.927531in}}%
\pgfpathlineto{\pgfqpoint{1.225490in}{0.930648in}}%
\pgfpathlineto{\pgfqpoint{1.237183in}{0.940812in}}%
\pgfpathlineto{\pgfqpoint{1.240768in}{0.944259in}}%
\pgfpathlineto{\pgfqpoint{1.252839in}{0.957724in}}%
\pgfpathlineto{\pgfqpoint{1.252963in}{0.957870in}}%
\pgfpathlineto{\pgfqpoint{1.262487in}{0.971481in}}%
\pgfpathlineto{\pgfqpoint{1.268496in}{0.982901in}}%
\pgfpathlineto{\pgfqpoint{1.269638in}{0.985092in}}%
\pgfpathlineto{\pgfqpoint{1.274440in}{0.998703in}}%
\pgfpathlineto{\pgfqpoint{1.276838in}{1.012314in}}%
\pgfpathlineto{\pgfqpoint{1.276838in}{1.025925in}}%
\pgfpathlineto{\pgfqpoint{1.274440in}{1.039536in}}%
\pgfpathlineto{\pgfqpoint{1.269638in}{1.053148in}}%
\pgfpathlineto{\pgfqpoint{1.268496in}{1.055339in}}%
\pgfpathlineto{\pgfqpoint{1.262487in}{1.066759in}}%
\pgfpathlineto{\pgfqpoint{1.252963in}{1.080370in}}%
\pgfpathlineto{\pgfqpoint{1.252839in}{1.080516in}}%
\pgfpathlineto{\pgfqpoint{1.240768in}{1.093981in}}%
\pgfpathlineto{\pgfqpoint{1.237183in}{1.097427in}}%
\pgfpathlineto{\pgfqpoint{1.225490in}{1.107592in}}%
\pgfpathlineto{\pgfqpoint{1.221526in}{1.110709in}}%
\pgfpathlineto{\pgfqpoint{1.206038in}{1.121203in}}%
\pgfpathlineto{\pgfqpoint{1.205870in}{1.121311in}}%
\pgfpathlineto{\pgfqpoint{1.190213in}{1.129590in}}%
\pgfpathlineto{\pgfqpoint{1.177077in}{1.134814in}}%
\pgfpathlineto{\pgfqpoint{1.174556in}{1.135807in}}%
\pgfpathlineto{\pgfqpoint{1.158900in}{1.139982in}}%
\pgfpathlineto{\pgfqpoint{1.143243in}{1.142066in}}%
\pgfpathlineto{\pgfqpoint{1.127587in}{1.142066in}}%
\pgfpathlineto{\pgfqpoint{1.111930in}{1.139982in}}%
\pgfpathlineto{\pgfqpoint{1.096274in}{1.135807in}}%
\pgfpathlineto{\pgfqpoint{1.093753in}{1.134814in}}%
\pgfpathlineto{\pgfqpoint{1.080617in}{1.129590in}}%
\pgfpathlineto{\pgfqpoint{1.064960in}{1.121311in}}%
\pgfpathlineto{\pgfqpoint{1.064792in}{1.121203in}}%
\pgfpathlineto{\pgfqpoint{1.049304in}{1.110709in}}%
\pgfpathlineto{\pgfqpoint{1.045340in}{1.107592in}}%
\pgfpathlineto{\pgfqpoint{1.033647in}{1.097427in}}%
\pgfpathlineto{\pgfqpoint{1.030062in}{1.093981in}}%
\pgfpathlineto{\pgfqpoint{1.017991in}{1.080516in}}%
\pgfpathlineto{\pgfqpoint{1.017867in}{1.080370in}}%
\pgfpathlineto{\pgfqpoint{1.008343in}{1.066759in}}%
\pgfpathlineto{\pgfqpoint{1.002334in}{1.055339in}}%
\pgfpathlineto{\pgfqpoint{1.001192in}{1.053148in}}%
\pgfpathlineto{\pgfqpoint{0.996390in}{1.039536in}}%
\pgfpathlineto{\pgfqpoint{0.993992in}{1.025925in}}%
\pgfpathlineto{\pgfqpoint{0.993992in}{1.012314in}}%
\pgfpathlineto{\pgfqpoint{0.996390in}{0.998703in}}%
\pgfpathlineto{\pgfqpoint{1.001192in}{0.985092in}}%
\pgfpathlineto{\pgfqpoint{1.002334in}{0.982901in}}%
\pgfpathlineto{\pgfqpoint{1.008343in}{0.971481in}}%
\pgfpathlineto{\pgfqpoint{1.017867in}{0.957870in}}%
\pgfpathlineto{\pgfqpoint{1.017991in}{0.957724in}}%
\pgfpathlineto{\pgfqpoint{1.030062in}{0.944259in}}%
\pgfpathlineto{\pgfqpoint{1.033647in}{0.940812in}}%
\pgfpathlineto{\pgfqpoint{1.045340in}{0.930648in}}%
\pgfpathlineto{\pgfqpoint{1.049304in}{0.927531in}}%
\pgfpathlineto{\pgfqpoint{1.064792in}{0.917036in}}%
\pgfpathlineto{\pgfqpoint{1.064960in}{0.916929in}}%
\pgfpathlineto{\pgfqpoint{1.080617in}{0.908649in}}%
\pgfpathlineto{\pgfqpoint{1.093753in}{0.903425in}}%
\pgfpathlineto{\pgfqpoint{1.096274in}{0.902432in}}%
\pgfpathclose%
\pgfpathmoveto{\pgfqpoint{1.099502in}{0.930648in}}%
\pgfpathlineto{\pgfqpoint{1.096274in}{0.931677in}}%
\pgfpathlineto{\pgfqpoint{1.080617in}{0.939097in}}%
\pgfpathlineto{\pgfqpoint{1.072354in}{0.944259in}}%
\pgfpathlineto{\pgfqpoint{1.064960in}{0.949543in}}%
\pgfpathlineto{\pgfqpoint{1.055383in}{0.957870in}}%
\pgfpathlineto{\pgfqpoint{1.049304in}{0.964298in}}%
\pgfpathlineto{\pgfqpoint{1.043367in}{0.971481in}}%
\pgfpathlineto{\pgfqpoint{1.034831in}{0.985092in}}%
\pgfpathlineto{\pgfqpoint{1.033647in}{0.987899in}}%
\pgfpathlineto{\pgfqpoint{1.029433in}{0.998703in}}%
\pgfpathlineto{\pgfqpoint{1.026774in}{1.012314in}}%
\pgfpathlineto{\pgfqpoint{1.026774in}{1.025925in}}%
\pgfpathlineto{\pgfqpoint{1.029433in}{1.039536in}}%
\pgfpathlineto{\pgfqpoint{1.033647in}{1.050341in}}%
\pgfpathlineto{\pgfqpoint{1.034831in}{1.053148in}}%
\pgfpathlineto{\pgfqpoint{1.043367in}{1.066759in}}%
\pgfpathlineto{\pgfqpoint{1.049304in}{1.073942in}}%
\pgfpathlineto{\pgfqpoint{1.055383in}{1.080370in}}%
\pgfpathlineto{\pgfqpoint{1.064960in}{1.088696in}}%
\pgfpathlineto{\pgfqpoint{1.072354in}{1.093981in}}%
\pgfpathlineto{\pgfqpoint{1.080617in}{1.099142in}}%
\pgfpathlineto{\pgfqpoint{1.096274in}{1.106563in}}%
\pgfpathlineto{\pgfqpoint{1.099502in}{1.107592in}}%
\pgfpathlineto{\pgfqpoint{1.111930in}{1.111256in}}%
\pgfpathlineto{\pgfqpoint{1.127587in}{1.113568in}}%
\pgfpathlineto{\pgfqpoint{1.143243in}{1.113568in}}%
\pgfpathlineto{\pgfqpoint{1.158900in}{1.111256in}}%
\pgfpathlineto{\pgfqpoint{1.171328in}{1.107592in}}%
\pgfpathlineto{\pgfqpoint{1.174556in}{1.106563in}}%
\pgfpathlineto{\pgfqpoint{1.190213in}{1.099142in}}%
\pgfpathlineto{\pgfqpoint{1.198476in}{1.093981in}}%
\pgfpathlineto{\pgfqpoint{1.205870in}{1.088696in}}%
\pgfpathlineto{\pgfqpoint{1.215447in}{1.080370in}}%
\pgfpathlineto{\pgfqpoint{1.221526in}{1.073942in}}%
\pgfpathlineto{\pgfqpoint{1.227463in}{1.066759in}}%
\pgfpathlineto{\pgfqpoint{1.235999in}{1.053148in}}%
\pgfpathlineto{\pgfqpoint{1.237183in}{1.050341in}}%
\pgfpathlineto{\pgfqpoint{1.241397in}{1.039536in}}%
\pgfpathlineto{\pgfqpoint{1.244056in}{1.025925in}}%
\pgfpathlineto{\pgfqpoint{1.244056in}{1.012314in}}%
\pgfpathlineto{\pgfqpoint{1.241397in}{0.998703in}}%
\pgfpathlineto{\pgfqpoint{1.237183in}{0.987899in}}%
\pgfpathlineto{\pgfqpoint{1.235999in}{0.985092in}}%
\pgfpathlineto{\pgfqpoint{1.227463in}{0.971481in}}%
\pgfpathlineto{\pgfqpoint{1.221526in}{0.964298in}}%
\pgfpathlineto{\pgfqpoint{1.215447in}{0.957870in}}%
\pgfpathlineto{\pgfqpoint{1.205870in}{0.949543in}}%
\pgfpathlineto{\pgfqpoint{1.198476in}{0.944259in}}%
\pgfpathlineto{\pgfqpoint{1.190213in}{0.939097in}}%
\pgfpathlineto{\pgfqpoint{1.174556in}{0.931677in}}%
\pgfpathlineto{\pgfqpoint{1.171328in}{0.930648in}}%
\pgfpathlineto{\pgfqpoint{1.158900in}{0.926983in}}%
\pgfpathlineto{\pgfqpoint{1.143243in}{0.924672in}}%
\pgfpathlineto{\pgfqpoint{1.127587in}{0.924672in}}%
\pgfpathlineto{\pgfqpoint{1.111930in}{0.926983in}}%
\pgfpathlineto{\pgfqpoint{1.099502in}{0.930648in}}%
\pgfpathclose%
\pgfpathmoveto{\pgfqpoint{1.612940in}{0.902432in}}%
\pgfpathlineto{\pgfqpoint{1.628597in}{0.898258in}}%
\pgfpathlineto{\pgfqpoint{1.644253in}{0.896173in}}%
\pgfpathlineto{\pgfqpoint{1.659910in}{0.896173in}}%
\pgfpathlineto{\pgfqpoint{1.675567in}{0.898258in}}%
\pgfpathlineto{\pgfqpoint{1.691223in}{0.902432in}}%
\pgfpathlineto{\pgfqpoint{1.693743in}{0.903425in}}%
\pgfpathlineto{\pgfqpoint{1.706880in}{0.908649in}}%
\pgfpathlineto{\pgfqpoint{1.722536in}{0.916929in}}%
\pgfpathlineto{\pgfqpoint{1.722704in}{0.917036in}}%
\pgfpathlineto{\pgfqpoint{1.738193in}{0.927531in}}%
\pgfpathlineto{\pgfqpoint{1.742157in}{0.930648in}}%
\pgfpathlineto{\pgfqpoint{1.753849in}{0.940812in}}%
\pgfpathlineto{\pgfqpoint{1.757434in}{0.944259in}}%
\pgfpathlineto{\pgfqpoint{1.769506in}{0.957724in}}%
\pgfpathlineto{\pgfqpoint{1.769630in}{0.957870in}}%
\pgfpathlineto{\pgfqpoint{1.779154in}{0.971481in}}%
\pgfpathlineto{\pgfqpoint{1.785162in}{0.982901in}}%
\pgfpathlineto{\pgfqpoint{1.786305in}{0.985092in}}%
\pgfpathlineto{\pgfqpoint{1.791107in}{0.998703in}}%
\pgfpathlineto{\pgfqpoint{1.793504in}{1.012314in}}%
\pgfpathlineto{\pgfqpoint{1.793504in}{1.025925in}}%
\pgfpathlineto{\pgfqpoint{1.791107in}{1.039536in}}%
\pgfpathlineto{\pgfqpoint{1.786305in}{1.053148in}}%
\pgfpathlineto{\pgfqpoint{1.785162in}{1.055339in}}%
\pgfpathlineto{\pgfqpoint{1.779154in}{1.066759in}}%
\pgfpathlineto{\pgfqpoint{1.769630in}{1.080370in}}%
\pgfpathlineto{\pgfqpoint{1.769506in}{1.080516in}}%
\pgfpathlineto{\pgfqpoint{1.757434in}{1.093981in}}%
\pgfpathlineto{\pgfqpoint{1.753849in}{1.097427in}}%
\pgfpathlineto{\pgfqpoint{1.742157in}{1.107592in}}%
\pgfpathlineto{\pgfqpoint{1.738193in}{1.110709in}}%
\pgfpathlineto{\pgfqpoint{1.722704in}{1.121203in}}%
\pgfpathlineto{\pgfqpoint{1.722536in}{1.121311in}}%
\pgfpathlineto{\pgfqpoint{1.706880in}{1.129590in}}%
\pgfpathlineto{\pgfqpoint{1.693743in}{1.134814in}}%
\pgfpathlineto{\pgfqpoint{1.691223in}{1.135807in}}%
\pgfpathlineto{\pgfqpoint{1.675567in}{1.139982in}}%
\pgfpathlineto{\pgfqpoint{1.659910in}{1.142066in}}%
\pgfpathlineto{\pgfqpoint{1.644253in}{1.142066in}}%
\pgfpathlineto{\pgfqpoint{1.628597in}{1.139982in}}%
\pgfpathlineto{\pgfqpoint{1.612940in}{1.135807in}}%
\pgfpathlineto{\pgfqpoint{1.610420in}{1.134814in}}%
\pgfpathlineto{\pgfqpoint{1.597284in}{1.129590in}}%
\pgfpathlineto{\pgfqpoint{1.581627in}{1.121311in}}%
\pgfpathlineto{\pgfqpoint{1.581459in}{1.121203in}}%
\pgfpathlineto{\pgfqpoint{1.565971in}{1.110709in}}%
\pgfpathlineto{\pgfqpoint{1.562006in}{1.107592in}}%
\pgfpathlineto{\pgfqpoint{1.550314in}{1.097427in}}%
\pgfpathlineto{\pgfqpoint{1.546729in}{1.093981in}}%
\pgfpathlineto{\pgfqpoint{1.534657in}{1.080516in}}%
\pgfpathlineto{\pgfqpoint{1.534534in}{1.080370in}}%
\pgfpathlineto{\pgfqpoint{1.525010in}{1.066759in}}%
\pgfpathlineto{\pgfqpoint{1.519001in}{1.055339in}}%
\pgfpathlineto{\pgfqpoint{1.517859in}{1.053148in}}%
\pgfpathlineto{\pgfqpoint{1.513057in}{1.039536in}}%
\pgfpathlineto{\pgfqpoint{1.510659in}{1.025925in}}%
\pgfpathlineto{\pgfqpoint{1.510659in}{1.012314in}}%
\pgfpathlineto{\pgfqpoint{1.513057in}{0.998703in}}%
\pgfpathlineto{\pgfqpoint{1.517859in}{0.985092in}}%
\pgfpathlineto{\pgfqpoint{1.519001in}{0.982901in}}%
\pgfpathlineto{\pgfqpoint{1.525010in}{0.971481in}}%
\pgfpathlineto{\pgfqpoint{1.534534in}{0.957870in}}%
\pgfpathlineto{\pgfqpoint{1.534657in}{0.957724in}}%
\pgfpathlineto{\pgfqpoint{1.546729in}{0.944259in}}%
\pgfpathlineto{\pgfqpoint{1.550314in}{0.940812in}}%
\pgfpathlineto{\pgfqpoint{1.562006in}{0.930648in}}%
\pgfpathlineto{\pgfqpoint{1.565971in}{0.927531in}}%
\pgfpathlineto{\pgfqpoint{1.581459in}{0.917036in}}%
\pgfpathlineto{\pgfqpoint{1.581627in}{0.916929in}}%
\pgfpathlineto{\pgfqpoint{1.597284in}{0.908649in}}%
\pgfpathlineto{\pgfqpoint{1.610420in}{0.903425in}}%
\pgfpathlineto{\pgfqpoint{1.612940in}{0.902432in}}%
\pgfpathclose%
\pgfpathmoveto{\pgfqpoint{1.616169in}{0.930648in}}%
\pgfpathlineto{\pgfqpoint{1.612940in}{0.931677in}}%
\pgfpathlineto{\pgfqpoint{1.597284in}{0.939097in}}%
\pgfpathlineto{\pgfqpoint{1.589021in}{0.944259in}}%
\pgfpathlineto{\pgfqpoint{1.581627in}{0.949543in}}%
\pgfpathlineto{\pgfqpoint{1.572050in}{0.957870in}}%
\pgfpathlineto{\pgfqpoint{1.565971in}{0.964298in}}%
\pgfpathlineto{\pgfqpoint{1.560034in}{0.971481in}}%
\pgfpathlineto{\pgfqpoint{1.551498in}{0.985092in}}%
\pgfpathlineto{\pgfqpoint{1.550314in}{0.987899in}}%
\pgfpathlineto{\pgfqpoint{1.546099in}{0.998703in}}%
\pgfpathlineto{\pgfqpoint{1.543440in}{1.012314in}}%
\pgfpathlineto{\pgfqpoint{1.543440in}{1.025925in}}%
\pgfpathlineto{\pgfqpoint{1.546099in}{1.039536in}}%
\pgfpathlineto{\pgfqpoint{1.550314in}{1.050341in}}%
\pgfpathlineto{\pgfqpoint{1.551498in}{1.053148in}}%
\pgfpathlineto{\pgfqpoint{1.560034in}{1.066759in}}%
\pgfpathlineto{\pgfqpoint{1.565971in}{1.073942in}}%
\pgfpathlineto{\pgfqpoint{1.572050in}{1.080370in}}%
\pgfpathlineto{\pgfqpoint{1.581627in}{1.088696in}}%
\pgfpathlineto{\pgfqpoint{1.589021in}{1.093981in}}%
\pgfpathlineto{\pgfqpoint{1.597284in}{1.099142in}}%
\pgfpathlineto{\pgfqpoint{1.612940in}{1.106563in}}%
\pgfpathlineto{\pgfqpoint{1.616169in}{1.107592in}}%
\pgfpathlineto{\pgfqpoint{1.628597in}{1.111256in}}%
\pgfpathlineto{\pgfqpoint{1.644253in}{1.113568in}}%
\pgfpathlineto{\pgfqpoint{1.659910in}{1.113568in}}%
\pgfpathlineto{\pgfqpoint{1.675567in}{1.111256in}}%
\pgfpathlineto{\pgfqpoint{1.687994in}{1.107592in}}%
\pgfpathlineto{\pgfqpoint{1.691223in}{1.106563in}}%
\pgfpathlineto{\pgfqpoint{1.706880in}{1.099142in}}%
\pgfpathlineto{\pgfqpoint{1.715142in}{1.093981in}}%
\pgfpathlineto{\pgfqpoint{1.722536in}{1.088696in}}%
\pgfpathlineto{\pgfqpoint{1.732114in}{1.080370in}}%
\pgfpathlineto{\pgfqpoint{1.738193in}{1.073942in}}%
\pgfpathlineto{\pgfqpoint{1.744130in}{1.066759in}}%
\pgfpathlineto{\pgfqpoint{1.752665in}{1.053148in}}%
\pgfpathlineto{\pgfqpoint{1.753849in}{1.050341in}}%
\pgfpathlineto{\pgfqpoint{1.758064in}{1.039536in}}%
\pgfpathlineto{\pgfqpoint{1.760723in}{1.025925in}}%
\pgfpathlineto{\pgfqpoint{1.760723in}{1.012314in}}%
\pgfpathlineto{\pgfqpoint{1.758064in}{0.998703in}}%
\pgfpathlineto{\pgfqpoint{1.753849in}{0.987899in}}%
\pgfpathlineto{\pgfqpoint{1.752665in}{0.985092in}}%
\pgfpathlineto{\pgfqpoint{1.744130in}{0.971481in}}%
\pgfpathlineto{\pgfqpoint{1.738193in}{0.964298in}}%
\pgfpathlineto{\pgfqpoint{1.732114in}{0.957870in}}%
\pgfpathlineto{\pgfqpoint{1.722536in}{0.949543in}}%
\pgfpathlineto{\pgfqpoint{1.715142in}{0.944259in}}%
\pgfpathlineto{\pgfqpoint{1.706880in}{0.939097in}}%
\pgfpathlineto{\pgfqpoint{1.691223in}{0.931677in}}%
\pgfpathlineto{\pgfqpoint{1.687994in}{0.930648in}}%
\pgfpathlineto{\pgfqpoint{1.675567in}{0.926983in}}%
\pgfpathlineto{\pgfqpoint{1.659910in}{0.924672in}}%
\pgfpathlineto{\pgfqpoint{1.644253in}{0.924672in}}%
\pgfpathlineto{\pgfqpoint{1.628597in}{0.926983in}}%
\pgfpathlineto{\pgfqpoint{1.616169in}{0.930648in}}%
\pgfpathclose%
\pgfpathmoveto{\pgfqpoint{0.579607in}{1.351599in}}%
\pgfpathlineto{\pgfqpoint{0.595263in}{1.347424in}}%
\pgfpathlineto{\pgfqpoint{0.610920in}{1.345340in}}%
\pgfpathlineto{\pgfqpoint{0.626577in}{1.345340in}}%
\pgfpathlineto{\pgfqpoint{0.642233in}{1.347424in}}%
\pgfpathlineto{\pgfqpoint{0.657890in}{1.351599in}}%
\pgfpathlineto{\pgfqpoint{0.660410in}{1.352592in}}%
\pgfpathlineto{\pgfqpoint{0.673546in}{1.357816in}}%
\pgfpathlineto{\pgfqpoint{0.689203in}{1.366095in}}%
\pgfpathlineto{\pgfqpoint{0.689371in}{1.366203in}}%
\pgfpathlineto{\pgfqpoint{0.704859in}{1.376697in}}%
\pgfpathlineto{\pgfqpoint{0.708824in}{1.379814in}}%
\pgfpathlineto{\pgfqpoint{0.720516in}{1.389979in}}%
\pgfpathlineto{\pgfqpoint{0.724101in}{1.393425in}}%
\pgfpathlineto{\pgfqpoint{0.736173in}{1.406890in}}%
\pgfpathlineto{\pgfqpoint{0.736296in}{1.407036in}}%
\pgfpathlineto{\pgfqpoint{0.745820in}{1.420648in}}%
\pgfpathlineto{\pgfqpoint{0.751829in}{1.432067in}}%
\pgfpathlineto{\pgfqpoint{0.752971in}{1.434259in}}%
\pgfpathlineto{\pgfqpoint{0.757773in}{1.447870in}}%
\pgfpathlineto{\pgfqpoint{0.760171in}{1.461481in}}%
\pgfpathlineto{\pgfqpoint{0.760171in}{1.475092in}}%
\pgfpathlineto{\pgfqpoint{0.757773in}{1.488703in}}%
\pgfpathlineto{\pgfqpoint{0.752971in}{1.502314in}}%
\pgfpathlineto{\pgfqpoint{0.751829in}{1.504505in}}%
\pgfpathlineto{\pgfqpoint{0.745820in}{1.515925in}}%
\pgfpathlineto{\pgfqpoint{0.736296in}{1.529536in}}%
\pgfpathlineto{\pgfqpoint{0.736173in}{1.529683in}}%
\pgfpathlineto{\pgfqpoint{0.724101in}{1.543148in}}%
\pgfpathlineto{\pgfqpoint{0.720516in}{1.546594in}}%
\pgfpathlineto{\pgfqpoint{0.708824in}{1.556759in}}%
\pgfpathlineto{\pgfqpoint{0.704859in}{1.559875in}}%
\pgfpathlineto{\pgfqpoint{0.689371in}{1.570370in}}%
\pgfpathlineto{\pgfqpoint{0.689203in}{1.570477in}}%
\pgfpathlineto{\pgfqpoint{0.673546in}{1.578757in}}%
\pgfpathlineto{\pgfqpoint{0.660410in}{1.583981in}}%
\pgfpathlineto{\pgfqpoint{0.657890in}{1.584974in}}%
\pgfpathlineto{\pgfqpoint{0.642233in}{1.589148in}}%
\pgfpathlineto{\pgfqpoint{0.626577in}{1.591233in}}%
\pgfpathlineto{\pgfqpoint{0.610920in}{1.591233in}}%
\pgfpathlineto{\pgfqpoint{0.595263in}{1.589148in}}%
\pgfpathlineto{\pgfqpoint{0.579607in}{1.584974in}}%
\pgfpathlineto{\pgfqpoint{0.577087in}{1.583981in}}%
\pgfpathlineto{\pgfqpoint{0.563950in}{1.578757in}}%
\pgfpathlineto{\pgfqpoint{0.548294in}{1.570477in}}%
\pgfpathlineto{\pgfqpoint{0.548126in}{1.570370in}}%
\pgfpathlineto{\pgfqpoint{0.532637in}{1.559875in}}%
\pgfpathlineto{\pgfqpoint{0.528673in}{1.556759in}}%
\pgfpathlineto{\pgfqpoint{0.516981in}{1.546594in}}%
\pgfpathlineto{\pgfqpoint{0.513396in}{1.543148in}}%
\pgfpathlineto{\pgfqpoint{0.501324in}{1.529683in}}%
\pgfpathlineto{\pgfqpoint{0.501200in}{1.529536in}}%
\pgfpathlineto{\pgfqpoint{0.491676in}{1.515925in}}%
\pgfpathlineto{\pgfqpoint{0.485668in}{1.504505in}}%
\pgfpathlineto{\pgfqpoint{0.484525in}{1.502314in}}%
\pgfpathlineto{\pgfqpoint{0.479723in}{1.488703in}}%
\pgfpathlineto{\pgfqpoint{0.477326in}{1.475092in}}%
\pgfpathlineto{\pgfqpoint{0.477326in}{1.461481in}}%
\pgfpathlineto{\pgfqpoint{0.479723in}{1.447870in}}%
\pgfpathlineto{\pgfqpoint{0.484525in}{1.434259in}}%
\pgfpathlineto{\pgfqpoint{0.485668in}{1.432067in}}%
\pgfpathlineto{\pgfqpoint{0.491676in}{1.420648in}}%
\pgfpathlineto{\pgfqpoint{0.501200in}{1.407036in}}%
\pgfpathlineto{\pgfqpoint{0.501324in}{1.406890in}}%
\pgfpathlineto{\pgfqpoint{0.513396in}{1.393425in}}%
\pgfpathlineto{\pgfqpoint{0.516981in}{1.389979in}}%
\pgfpathlineto{\pgfqpoint{0.528673in}{1.379814in}}%
\pgfpathlineto{\pgfqpoint{0.532637in}{1.376697in}}%
\pgfpathlineto{\pgfqpoint{0.548126in}{1.366203in}}%
\pgfpathlineto{\pgfqpoint{0.548294in}{1.366095in}}%
\pgfpathlineto{\pgfqpoint{0.563950in}{1.357816in}}%
\pgfpathlineto{\pgfqpoint{0.577087in}{1.352592in}}%
\pgfpathlineto{\pgfqpoint{0.579607in}{1.351599in}}%
\pgfpathclose%
\pgfpathmoveto{\pgfqpoint{0.582836in}{1.379814in}}%
\pgfpathlineto{\pgfqpoint{0.579607in}{1.380844in}}%
\pgfpathlineto{\pgfqpoint{0.563950in}{1.388264in}}%
\pgfpathlineto{\pgfqpoint{0.555688in}{1.393425in}}%
\pgfpathlineto{\pgfqpoint{0.548294in}{1.398710in}}%
\pgfpathlineto{\pgfqpoint{0.538716in}{1.407036in}}%
\pgfpathlineto{\pgfqpoint{0.532637in}{1.413464in}}%
\pgfpathlineto{\pgfqpoint{0.526700in}{1.420648in}}%
\pgfpathlineto{\pgfqpoint{0.518165in}{1.434259in}}%
\pgfpathlineto{\pgfqpoint{0.516981in}{1.437066in}}%
\pgfpathlineto{\pgfqpoint{0.512766in}{1.447870in}}%
\pgfpathlineto{\pgfqpoint{0.510107in}{1.461481in}}%
\pgfpathlineto{\pgfqpoint{0.510107in}{1.475092in}}%
\pgfpathlineto{\pgfqpoint{0.512766in}{1.488703in}}%
\pgfpathlineto{\pgfqpoint{0.516981in}{1.499507in}}%
\pgfpathlineto{\pgfqpoint{0.518165in}{1.502314in}}%
\pgfpathlineto{\pgfqpoint{0.526700in}{1.515925in}}%
\pgfpathlineto{\pgfqpoint{0.532637in}{1.523109in}}%
\pgfpathlineto{\pgfqpoint{0.538716in}{1.529536in}}%
\pgfpathlineto{\pgfqpoint{0.548294in}{1.537863in}}%
\pgfpathlineto{\pgfqpoint{0.555688in}{1.543148in}}%
\pgfpathlineto{\pgfqpoint{0.563950in}{1.548309in}}%
\pgfpathlineto{\pgfqpoint{0.579607in}{1.555729in}}%
\pgfpathlineto{\pgfqpoint{0.582836in}{1.556759in}}%
\pgfpathlineto{\pgfqpoint{0.595263in}{1.560423in}}%
\pgfpathlineto{\pgfqpoint{0.610920in}{1.562734in}}%
\pgfpathlineto{\pgfqpoint{0.626577in}{1.562734in}}%
\pgfpathlineto{\pgfqpoint{0.642233in}{1.560423in}}%
\pgfpathlineto{\pgfqpoint{0.654661in}{1.556759in}}%
\pgfpathlineto{\pgfqpoint{0.657890in}{1.555729in}}%
\pgfpathlineto{\pgfqpoint{0.673546in}{1.548309in}}%
\pgfpathlineto{\pgfqpoint{0.681809in}{1.543148in}}%
\pgfpathlineto{\pgfqpoint{0.689203in}{1.537863in}}%
\pgfpathlineto{\pgfqpoint{0.698780in}{1.529536in}}%
\pgfpathlineto{\pgfqpoint{0.704859in}{1.523109in}}%
\pgfpathlineto{\pgfqpoint{0.710796in}{1.515925in}}%
\pgfpathlineto{\pgfqpoint{0.719332in}{1.502314in}}%
\pgfpathlineto{\pgfqpoint{0.720516in}{1.499507in}}%
\pgfpathlineto{\pgfqpoint{0.724731in}{1.488703in}}%
\pgfpathlineto{\pgfqpoint{0.727390in}{1.475092in}}%
\pgfpathlineto{\pgfqpoint{0.727390in}{1.461481in}}%
\pgfpathlineto{\pgfqpoint{0.724731in}{1.447870in}}%
\pgfpathlineto{\pgfqpoint{0.720516in}{1.437066in}}%
\pgfpathlineto{\pgfqpoint{0.719332in}{1.434259in}}%
\pgfpathlineto{\pgfqpoint{0.710796in}{1.420648in}}%
\pgfpathlineto{\pgfqpoint{0.704859in}{1.413464in}}%
\pgfpathlineto{\pgfqpoint{0.698780in}{1.407036in}}%
\pgfpathlineto{\pgfqpoint{0.689203in}{1.398710in}}%
\pgfpathlineto{\pgfqpoint{0.681809in}{1.393425in}}%
\pgfpathlineto{\pgfqpoint{0.673546in}{1.388264in}}%
\pgfpathlineto{\pgfqpoint{0.657890in}{1.380844in}}%
\pgfpathlineto{\pgfqpoint{0.654661in}{1.379814in}}%
\pgfpathlineto{\pgfqpoint{0.642233in}{1.376150in}}%
\pgfpathlineto{\pgfqpoint{0.626577in}{1.373839in}}%
\pgfpathlineto{\pgfqpoint{0.610920in}{1.373839in}}%
\pgfpathlineto{\pgfqpoint{0.595263in}{1.376150in}}%
\pgfpathlineto{\pgfqpoint{0.582836in}{1.379814in}}%
\pgfpathclose%
\pgfpathmoveto{\pgfqpoint{1.096274in}{1.351599in}}%
\pgfpathlineto{\pgfqpoint{1.111930in}{1.347424in}}%
\pgfpathlineto{\pgfqpoint{1.127587in}{1.345340in}}%
\pgfpathlineto{\pgfqpoint{1.143243in}{1.345340in}}%
\pgfpathlineto{\pgfqpoint{1.158900in}{1.347424in}}%
\pgfpathlineto{\pgfqpoint{1.174556in}{1.351599in}}%
\pgfpathlineto{\pgfqpoint{1.177077in}{1.352592in}}%
\pgfpathlineto{\pgfqpoint{1.190213in}{1.357816in}}%
\pgfpathlineto{\pgfqpoint{1.205870in}{1.366095in}}%
\pgfpathlineto{\pgfqpoint{1.206038in}{1.366203in}}%
\pgfpathlineto{\pgfqpoint{1.221526in}{1.376697in}}%
\pgfpathlineto{\pgfqpoint{1.225490in}{1.379814in}}%
\pgfpathlineto{\pgfqpoint{1.237183in}{1.389979in}}%
\pgfpathlineto{\pgfqpoint{1.240768in}{1.393425in}}%
\pgfpathlineto{\pgfqpoint{1.252839in}{1.406890in}}%
\pgfpathlineto{\pgfqpoint{1.252963in}{1.407036in}}%
\pgfpathlineto{\pgfqpoint{1.262487in}{1.420648in}}%
\pgfpathlineto{\pgfqpoint{1.268496in}{1.432067in}}%
\pgfpathlineto{\pgfqpoint{1.269638in}{1.434259in}}%
\pgfpathlineto{\pgfqpoint{1.274440in}{1.447870in}}%
\pgfpathlineto{\pgfqpoint{1.276838in}{1.461481in}}%
\pgfpathlineto{\pgfqpoint{1.276838in}{1.475092in}}%
\pgfpathlineto{\pgfqpoint{1.274440in}{1.488703in}}%
\pgfpathlineto{\pgfqpoint{1.269638in}{1.502314in}}%
\pgfpathlineto{\pgfqpoint{1.268496in}{1.504505in}}%
\pgfpathlineto{\pgfqpoint{1.262487in}{1.515925in}}%
\pgfpathlineto{\pgfqpoint{1.252963in}{1.529536in}}%
\pgfpathlineto{\pgfqpoint{1.252839in}{1.529683in}}%
\pgfpathlineto{\pgfqpoint{1.240768in}{1.543148in}}%
\pgfpathlineto{\pgfqpoint{1.237183in}{1.546594in}}%
\pgfpathlineto{\pgfqpoint{1.225490in}{1.556759in}}%
\pgfpathlineto{\pgfqpoint{1.221526in}{1.559875in}}%
\pgfpathlineto{\pgfqpoint{1.206038in}{1.570370in}}%
\pgfpathlineto{\pgfqpoint{1.205870in}{1.570477in}}%
\pgfpathlineto{\pgfqpoint{1.190213in}{1.578757in}}%
\pgfpathlineto{\pgfqpoint{1.177077in}{1.583981in}}%
\pgfpathlineto{\pgfqpoint{1.174556in}{1.584974in}}%
\pgfpathlineto{\pgfqpoint{1.158900in}{1.589148in}}%
\pgfpathlineto{\pgfqpoint{1.143243in}{1.591233in}}%
\pgfpathlineto{\pgfqpoint{1.127587in}{1.591233in}}%
\pgfpathlineto{\pgfqpoint{1.111930in}{1.589148in}}%
\pgfpathlineto{\pgfqpoint{1.096274in}{1.584974in}}%
\pgfpathlineto{\pgfqpoint{1.093753in}{1.583981in}}%
\pgfpathlineto{\pgfqpoint{1.080617in}{1.578757in}}%
\pgfpathlineto{\pgfqpoint{1.064960in}{1.570477in}}%
\pgfpathlineto{\pgfqpoint{1.064792in}{1.570370in}}%
\pgfpathlineto{\pgfqpoint{1.049304in}{1.559875in}}%
\pgfpathlineto{\pgfqpoint{1.045340in}{1.556759in}}%
\pgfpathlineto{\pgfqpoint{1.033647in}{1.546594in}}%
\pgfpathlineto{\pgfqpoint{1.030062in}{1.543148in}}%
\pgfpathlineto{\pgfqpoint{1.017991in}{1.529683in}}%
\pgfpathlineto{\pgfqpoint{1.017867in}{1.529536in}}%
\pgfpathlineto{\pgfqpoint{1.008343in}{1.515925in}}%
\pgfpathlineto{\pgfqpoint{1.002334in}{1.504505in}}%
\pgfpathlineto{\pgfqpoint{1.001192in}{1.502314in}}%
\pgfpathlineto{\pgfqpoint{0.996390in}{1.488703in}}%
\pgfpathlineto{\pgfqpoint{0.993992in}{1.475092in}}%
\pgfpathlineto{\pgfqpoint{0.993992in}{1.461481in}}%
\pgfpathlineto{\pgfqpoint{0.996390in}{1.447870in}}%
\pgfpathlineto{\pgfqpoint{1.001192in}{1.434259in}}%
\pgfpathlineto{\pgfqpoint{1.002334in}{1.432067in}}%
\pgfpathlineto{\pgfqpoint{1.008343in}{1.420648in}}%
\pgfpathlineto{\pgfqpoint{1.017867in}{1.407036in}}%
\pgfpathlineto{\pgfqpoint{1.017991in}{1.406890in}}%
\pgfpathlineto{\pgfqpoint{1.030062in}{1.393425in}}%
\pgfpathlineto{\pgfqpoint{1.033647in}{1.389979in}}%
\pgfpathlineto{\pgfqpoint{1.045340in}{1.379814in}}%
\pgfpathlineto{\pgfqpoint{1.049304in}{1.376697in}}%
\pgfpathlineto{\pgfqpoint{1.064792in}{1.366203in}}%
\pgfpathlineto{\pgfqpoint{1.064960in}{1.366095in}}%
\pgfpathlineto{\pgfqpoint{1.080617in}{1.357816in}}%
\pgfpathlineto{\pgfqpoint{1.093753in}{1.352592in}}%
\pgfpathlineto{\pgfqpoint{1.096274in}{1.351599in}}%
\pgfpathclose%
\pgfpathmoveto{\pgfqpoint{1.099502in}{1.379814in}}%
\pgfpathlineto{\pgfqpoint{1.096274in}{1.380844in}}%
\pgfpathlineto{\pgfqpoint{1.080617in}{1.388264in}}%
\pgfpathlineto{\pgfqpoint{1.072354in}{1.393425in}}%
\pgfpathlineto{\pgfqpoint{1.064960in}{1.398710in}}%
\pgfpathlineto{\pgfqpoint{1.055383in}{1.407036in}}%
\pgfpathlineto{\pgfqpoint{1.049304in}{1.413464in}}%
\pgfpathlineto{\pgfqpoint{1.043367in}{1.420648in}}%
\pgfpathlineto{\pgfqpoint{1.034831in}{1.434259in}}%
\pgfpathlineto{\pgfqpoint{1.033647in}{1.437066in}}%
\pgfpathlineto{\pgfqpoint{1.029433in}{1.447870in}}%
\pgfpathlineto{\pgfqpoint{1.026774in}{1.461481in}}%
\pgfpathlineto{\pgfqpoint{1.026774in}{1.475092in}}%
\pgfpathlineto{\pgfqpoint{1.029433in}{1.488703in}}%
\pgfpathlineto{\pgfqpoint{1.033647in}{1.499507in}}%
\pgfpathlineto{\pgfqpoint{1.034831in}{1.502314in}}%
\pgfpathlineto{\pgfqpoint{1.043367in}{1.515925in}}%
\pgfpathlineto{\pgfqpoint{1.049304in}{1.523109in}}%
\pgfpathlineto{\pgfqpoint{1.055383in}{1.529536in}}%
\pgfpathlineto{\pgfqpoint{1.064960in}{1.537863in}}%
\pgfpathlineto{\pgfqpoint{1.072354in}{1.543148in}}%
\pgfpathlineto{\pgfqpoint{1.080617in}{1.548309in}}%
\pgfpathlineto{\pgfqpoint{1.096274in}{1.555729in}}%
\pgfpathlineto{\pgfqpoint{1.099502in}{1.556759in}}%
\pgfpathlineto{\pgfqpoint{1.111930in}{1.560423in}}%
\pgfpathlineto{\pgfqpoint{1.127587in}{1.562734in}}%
\pgfpathlineto{\pgfqpoint{1.143243in}{1.562734in}}%
\pgfpathlineto{\pgfqpoint{1.158900in}{1.560423in}}%
\pgfpathlineto{\pgfqpoint{1.171328in}{1.556759in}}%
\pgfpathlineto{\pgfqpoint{1.174556in}{1.555729in}}%
\pgfpathlineto{\pgfqpoint{1.190213in}{1.548309in}}%
\pgfpathlineto{\pgfqpoint{1.198476in}{1.543148in}}%
\pgfpathlineto{\pgfqpoint{1.205870in}{1.537863in}}%
\pgfpathlineto{\pgfqpoint{1.215447in}{1.529536in}}%
\pgfpathlineto{\pgfqpoint{1.221526in}{1.523109in}}%
\pgfpathlineto{\pgfqpoint{1.227463in}{1.515925in}}%
\pgfpathlineto{\pgfqpoint{1.235999in}{1.502314in}}%
\pgfpathlineto{\pgfqpoint{1.237183in}{1.499507in}}%
\pgfpathlineto{\pgfqpoint{1.241397in}{1.488703in}}%
\pgfpathlineto{\pgfqpoint{1.244056in}{1.475092in}}%
\pgfpathlineto{\pgfqpoint{1.244056in}{1.461481in}}%
\pgfpathlineto{\pgfqpoint{1.241397in}{1.447870in}}%
\pgfpathlineto{\pgfqpoint{1.237183in}{1.437066in}}%
\pgfpathlineto{\pgfqpoint{1.235999in}{1.434259in}}%
\pgfpathlineto{\pgfqpoint{1.227463in}{1.420648in}}%
\pgfpathlineto{\pgfqpoint{1.221526in}{1.413464in}}%
\pgfpathlineto{\pgfqpoint{1.215447in}{1.407036in}}%
\pgfpathlineto{\pgfqpoint{1.205870in}{1.398710in}}%
\pgfpathlineto{\pgfqpoint{1.198476in}{1.393425in}}%
\pgfpathlineto{\pgfqpoint{1.190213in}{1.388264in}}%
\pgfpathlineto{\pgfqpoint{1.174556in}{1.380844in}}%
\pgfpathlineto{\pgfqpoint{1.171328in}{1.379814in}}%
\pgfpathlineto{\pgfqpoint{1.158900in}{1.376150in}}%
\pgfpathlineto{\pgfqpoint{1.143243in}{1.373839in}}%
\pgfpathlineto{\pgfqpoint{1.127587in}{1.373839in}}%
\pgfpathlineto{\pgfqpoint{1.111930in}{1.376150in}}%
\pgfpathlineto{\pgfqpoint{1.099502in}{1.379814in}}%
\pgfpathclose%
\pgfpathmoveto{\pgfqpoint{1.612940in}{1.351599in}}%
\pgfpathlineto{\pgfqpoint{1.628597in}{1.347424in}}%
\pgfpathlineto{\pgfqpoint{1.644253in}{1.345340in}}%
\pgfpathlineto{\pgfqpoint{1.659910in}{1.345340in}}%
\pgfpathlineto{\pgfqpoint{1.675567in}{1.347424in}}%
\pgfpathlineto{\pgfqpoint{1.691223in}{1.351599in}}%
\pgfpathlineto{\pgfqpoint{1.693743in}{1.352592in}}%
\pgfpathlineto{\pgfqpoint{1.706880in}{1.357816in}}%
\pgfpathlineto{\pgfqpoint{1.722536in}{1.366095in}}%
\pgfpathlineto{\pgfqpoint{1.722704in}{1.366203in}}%
\pgfpathlineto{\pgfqpoint{1.738193in}{1.376697in}}%
\pgfpathlineto{\pgfqpoint{1.742157in}{1.379814in}}%
\pgfpathlineto{\pgfqpoint{1.753849in}{1.389979in}}%
\pgfpathlineto{\pgfqpoint{1.757434in}{1.393425in}}%
\pgfpathlineto{\pgfqpoint{1.769506in}{1.406890in}}%
\pgfpathlineto{\pgfqpoint{1.769630in}{1.407036in}}%
\pgfpathlineto{\pgfqpoint{1.779154in}{1.420648in}}%
\pgfpathlineto{\pgfqpoint{1.785162in}{1.432067in}}%
\pgfpathlineto{\pgfqpoint{1.786305in}{1.434259in}}%
\pgfpathlineto{\pgfqpoint{1.791107in}{1.447870in}}%
\pgfpathlineto{\pgfqpoint{1.793504in}{1.461481in}}%
\pgfpathlineto{\pgfqpoint{1.793504in}{1.475092in}}%
\pgfpathlineto{\pgfqpoint{1.791107in}{1.488703in}}%
\pgfpathlineto{\pgfqpoint{1.786305in}{1.502314in}}%
\pgfpathlineto{\pgfqpoint{1.785162in}{1.504505in}}%
\pgfpathlineto{\pgfqpoint{1.779154in}{1.515925in}}%
\pgfpathlineto{\pgfqpoint{1.769630in}{1.529536in}}%
\pgfpathlineto{\pgfqpoint{1.769506in}{1.529683in}}%
\pgfpathlineto{\pgfqpoint{1.757434in}{1.543148in}}%
\pgfpathlineto{\pgfqpoint{1.753849in}{1.546594in}}%
\pgfpathlineto{\pgfqpoint{1.742157in}{1.556759in}}%
\pgfpathlineto{\pgfqpoint{1.738193in}{1.559875in}}%
\pgfpathlineto{\pgfqpoint{1.722704in}{1.570370in}}%
\pgfpathlineto{\pgfqpoint{1.722536in}{1.570477in}}%
\pgfpathlineto{\pgfqpoint{1.706880in}{1.578757in}}%
\pgfpathlineto{\pgfqpoint{1.693743in}{1.583981in}}%
\pgfpathlineto{\pgfqpoint{1.691223in}{1.584974in}}%
\pgfpathlineto{\pgfqpoint{1.675567in}{1.589148in}}%
\pgfpathlineto{\pgfqpoint{1.659910in}{1.591233in}}%
\pgfpathlineto{\pgfqpoint{1.644253in}{1.591233in}}%
\pgfpathlineto{\pgfqpoint{1.628597in}{1.589148in}}%
\pgfpathlineto{\pgfqpoint{1.612940in}{1.584974in}}%
\pgfpathlineto{\pgfqpoint{1.610420in}{1.583981in}}%
\pgfpathlineto{\pgfqpoint{1.597284in}{1.578757in}}%
\pgfpathlineto{\pgfqpoint{1.581627in}{1.570477in}}%
\pgfpathlineto{\pgfqpoint{1.581459in}{1.570370in}}%
\pgfpathlineto{\pgfqpoint{1.565971in}{1.559875in}}%
\pgfpathlineto{\pgfqpoint{1.562006in}{1.556759in}}%
\pgfpathlineto{\pgfqpoint{1.550314in}{1.546594in}}%
\pgfpathlineto{\pgfqpoint{1.546729in}{1.543148in}}%
\pgfpathlineto{\pgfqpoint{1.534657in}{1.529683in}}%
\pgfpathlineto{\pgfqpoint{1.534534in}{1.529536in}}%
\pgfpathlineto{\pgfqpoint{1.525010in}{1.515925in}}%
\pgfpathlineto{\pgfqpoint{1.519001in}{1.504505in}}%
\pgfpathlineto{\pgfqpoint{1.517859in}{1.502314in}}%
\pgfpathlineto{\pgfqpoint{1.513057in}{1.488703in}}%
\pgfpathlineto{\pgfqpoint{1.510659in}{1.475092in}}%
\pgfpathlineto{\pgfqpoint{1.510659in}{1.461481in}}%
\pgfpathlineto{\pgfqpoint{1.513057in}{1.447870in}}%
\pgfpathlineto{\pgfqpoint{1.517859in}{1.434259in}}%
\pgfpathlineto{\pgfqpoint{1.519001in}{1.432067in}}%
\pgfpathlineto{\pgfqpoint{1.525010in}{1.420648in}}%
\pgfpathlineto{\pgfqpoint{1.534534in}{1.407036in}}%
\pgfpathlineto{\pgfqpoint{1.534657in}{1.406890in}}%
\pgfpathlineto{\pgfqpoint{1.546729in}{1.393425in}}%
\pgfpathlineto{\pgfqpoint{1.550314in}{1.389979in}}%
\pgfpathlineto{\pgfqpoint{1.562006in}{1.379814in}}%
\pgfpathlineto{\pgfqpoint{1.565971in}{1.376697in}}%
\pgfpathlineto{\pgfqpoint{1.581459in}{1.366203in}}%
\pgfpathlineto{\pgfqpoint{1.581627in}{1.366095in}}%
\pgfpathlineto{\pgfqpoint{1.597284in}{1.357816in}}%
\pgfpathlineto{\pgfqpoint{1.610420in}{1.352592in}}%
\pgfpathlineto{\pgfqpoint{1.612940in}{1.351599in}}%
\pgfpathclose%
\pgfpathmoveto{\pgfqpoint{1.616169in}{1.379814in}}%
\pgfpathlineto{\pgfqpoint{1.612940in}{1.380844in}}%
\pgfpathlineto{\pgfqpoint{1.597284in}{1.388264in}}%
\pgfpathlineto{\pgfqpoint{1.589021in}{1.393425in}}%
\pgfpathlineto{\pgfqpoint{1.581627in}{1.398710in}}%
\pgfpathlineto{\pgfqpoint{1.572050in}{1.407036in}}%
\pgfpathlineto{\pgfqpoint{1.565971in}{1.413464in}}%
\pgfpathlineto{\pgfqpoint{1.560034in}{1.420648in}}%
\pgfpathlineto{\pgfqpoint{1.551498in}{1.434259in}}%
\pgfpathlineto{\pgfqpoint{1.550314in}{1.437066in}}%
\pgfpathlineto{\pgfqpoint{1.546099in}{1.447870in}}%
\pgfpathlineto{\pgfqpoint{1.543440in}{1.461481in}}%
\pgfpathlineto{\pgfqpoint{1.543440in}{1.475092in}}%
\pgfpathlineto{\pgfqpoint{1.546099in}{1.488703in}}%
\pgfpathlineto{\pgfqpoint{1.550314in}{1.499507in}}%
\pgfpathlineto{\pgfqpoint{1.551498in}{1.502314in}}%
\pgfpathlineto{\pgfqpoint{1.560034in}{1.515925in}}%
\pgfpathlineto{\pgfqpoint{1.565971in}{1.523109in}}%
\pgfpathlineto{\pgfqpoint{1.572050in}{1.529536in}}%
\pgfpathlineto{\pgfqpoint{1.581627in}{1.537863in}}%
\pgfpathlineto{\pgfqpoint{1.589021in}{1.543148in}}%
\pgfpathlineto{\pgfqpoint{1.597284in}{1.548309in}}%
\pgfpathlineto{\pgfqpoint{1.612940in}{1.555729in}}%
\pgfpathlineto{\pgfqpoint{1.616169in}{1.556759in}}%
\pgfpathlineto{\pgfqpoint{1.628597in}{1.560423in}}%
\pgfpathlineto{\pgfqpoint{1.644253in}{1.562734in}}%
\pgfpathlineto{\pgfqpoint{1.659910in}{1.562734in}}%
\pgfpathlineto{\pgfqpoint{1.675567in}{1.560423in}}%
\pgfpathlineto{\pgfqpoint{1.687994in}{1.556759in}}%
\pgfpathlineto{\pgfqpoint{1.691223in}{1.555729in}}%
\pgfpathlineto{\pgfqpoint{1.706880in}{1.548309in}}%
\pgfpathlineto{\pgfqpoint{1.715142in}{1.543148in}}%
\pgfpathlineto{\pgfqpoint{1.722536in}{1.537863in}}%
\pgfpathlineto{\pgfqpoint{1.732114in}{1.529536in}}%
\pgfpathlineto{\pgfqpoint{1.738193in}{1.523109in}}%
\pgfpathlineto{\pgfqpoint{1.744130in}{1.515925in}}%
\pgfpathlineto{\pgfqpoint{1.752665in}{1.502314in}}%
\pgfpathlineto{\pgfqpoint{1.753849in}{1.499507in}}%
\pgfpathlineto{\pgfqpoint{1.758064in}{1.488703in}}%
\pgfpathlineto{\pgfqpoint{1.760723in}{1.475092in}}%
\pgfpathlineto{\pgfqpoint{1.760723in}{1.461481in}}%
\pgfpathlineto{\pgfqpoint{1.758064in}{1.447870in}}%
\pgfpathlineto{\pgfqpoint{1.753849in}{1.437066in}}%
\pgfpathlineto{\pgfqpoint{1.752665in}{1.434259in}}%
\pgfpathlineto{\pgfqpoint{1.744130in}{1.420648in}}%
\pgfpathlineto{\pgfqpoint{1.738193in}{1.413464in}}%
\pgfpathlineto{\pgfqpoint{1.732114in}{1.407036in}}%
\pgfpathlineto{\pgfqpoint{1.722536in}{1.398710in}}%
\pgfpathlineto{\pgfqpoint{1.715142in}{1.393425in}}%
\pgfpathlineto{\pgfqpoint{1.706880in}{1.388264in}}%
\pgfpathlineto{\pgfqpoint{1.691223in}{1.380844in}}%
\pgfpathlineto{\pgfqpoint{1.687994in}{1.379814in}}%
\pgfpathlineto{\pgfqpoint{1.675567in}{1.376150in}}%
\pgfpathlineto{\pgfqpoint{1.659910in}{1.373839in}}%
\pgfpathlineto{\pgfqpoint{1.644253in}{1.373839in}}%
\pgfpathlineto{\pgfqpoint{1.628597in}{1.376150in}}%
\pgfpathlineto{\pgfqpoint{1.616169in}{1.379814in}}%
\pgfpathclose%
\pgfusepath{fill}%
\end{pgfscope}%
\begin{pgfscope}%
\pgfpathrectangle{\pgfqpoint{0.360415in}{0.345370in}}{\pgfqpoint{1.550000in}{1.347500in}}%
\pgfusepath{clip}%
\pgfsetbuttcap%
\pgfsetroundjoin%
\definecolor{currentfill}{rgb}{0.890340,0.406398,0.373130}%
\pgfsetfillcolor{currentfill}%
\pgfsetlinewidth{0.000000pt}%
\definecolor{currentstroke}{rgb}{0.000000,0.000000,0.000000}%
\pgfsetstrokecolor{currentstroke}%
\pgfsetdash{}{0pt}%
\pgfpathmoveto{\pgfqpoint{0.579607in}{0.423047in}}%
\pgfpathlineto{\pgfqpoint{0.595263in}{0.418701in}}%
\pgfpathlineto{\pgfqpoint{0.610920in}{0.416530in}}%
\pgfpathlineto{\pgfqpoint{0.626577in}{0.416530in}}%
\pgfpathlineto{\pgfqpoint{0.642233in}{0.418701in}}%
\pgfpathlineto{\pgfqpoint{0.657890in}{0.423047in}}%
\pgfpathlineto{\pgfqpoint{0.667531in}{0.427036in}}%
\pgfpathlineto{\pgfqpoint{0.673546in}{0.429360in}}%
\pgfpathlineto{\pgfqpoint{0.689203in}{0.437333in}}%
\pgfpathlineto{\pgfqpoint{0.694529in}{0.440648in}}%
\pgfpathlineto{\pgfqpoint{0.704859in}{0.446971in}}%
\pgfpathlineto{\pgfqpoint{0.715077in}{0.454259in}}%
\pgfpathlineto{\pgfqpoint{0.720516in}{0.458237in}}%
\pgfpathlineto{\pgfqpoint{0.732349in}{0.467870in}}%
\pgfpathlineto{\pgfqpoint{0.736173in}{0.471194in}}%
\pgfpathlineto{\pgfqpoint{0.747253in}{0.481481in}}%
\pgfpathlineto{\pgfqpoint{0.751829in}{0.486209in}}%
\pgfpathlineto{\pgfqpoint{0.760212in}{0.495092in}}%
\pgfpathlineto{\pgfqpoint{0.767486in}{0.504073in}}%
\pgfpathlineto{\pgfqpoint{0.771299in}{0.508703in}}%
\pgfpathlineto{\pgfqpoint{0.780469in}{0.522314in}}%
\pgfpathlineto{\pgfqpoint{0.783142in}{0.527544in}}%
\pgfpathlineto{\pgfqpoint{0.787731in}{0.535925in}}%
\pgfpathlineto{\pgfqpoint{0.792731in}{0.549536in}}%
\pgfpathlineto{\pgfqpoint{0.795227in}{0.563148in}}%
\pgfpathlineto{\pgfqpoint{0.795227in}{0.576759in}}%
\pgfpathlineto{\pgfqpoint{0.792731in}{0.590370in}}%
\pgfpathlineto{\pgfqpoint{0.787731in}{0.603981in}}%
\pgfpathlineto{\pgfqpoint{0.783142in}{0.612362in}}%
\pgfpathlineto{\pgfqpoint{0.780469in}{0.617592in}}%
\pgfpathlineto{\pgfqpoint{0.771299in}{0.631203in}}%
\pgfpathlineto{\pgfqpoint{0.767486in}{0.635833in}}%
\pgfpathlineto{\pgfqpoint{0.760212in}{0.644814in}}%
\pgfpathlineto{\pgfqpoint{0.751829in}{0.653697in}}%
\pgfpathlineto{\pgfqpoint{0.747253in}{0.658425in}}%
\pgfpathlineto{\pgfqpoint{0.736173in}{0.668712in}}%
\pgfpathlineto{\pgfqpoint{0.732349in}{0.672036in}}%
\pgfpathlineto{\pgfqpoint{0.720516in}{0.681669in}}%
\pgfpathlineto{\pgfqpoint{0.715077in}{0.685648in}}%
\pgfpathlineto{\pgfqpoint{0.704859in}{0.692935in}}%
\pgfpathlineto{\pgfqpoint{0.694529in}{0.699259in}}%
\pgfpathlineto{\pgfqpoint{0.689203in}{0.702573in}}%
\pgfpathlineto{\pgfqpoint{0.673546in}{0.710546in}}%
\pgfpathlineto{\pgfqpoint{0.667531in}{0.712870in}}%
\pgfpathlineto{\pgfqpoint{0.657890in}{0.716859in}}%
\pgfpathlineto{\pgfqpoint{0.642233in}{0.721205in}}%
\pgfpathlineto{\pgfqpoint{0.626577in}{0.723376in}}%
\pgfpathlineto{\pgfqpoint{0.610920in}{0.723376in}}%
\pgfpathlineto{\pgfqpoint{0.595263in}{0.721205in}}%
\pgfpathlineto{\pgfqpoint{0.579607in}{0.716859in}}%
\pgfpathlineto{\pgfqpoint{0.569966in}{0.712870in}}%
\pgfpathlineto{\pgfqpoint{0.563950in}{0.710546in}}%
\pgfpathlineto{\pgfqpoint{0.548294in}{0.702573in}}%
\pgfpathlineto{\pgfqpoint{0.542968in}{0.699259in}}%
\pgfpathlineto{\pgfqpoint{0.532637in}{0.692935in}}%
\pgfpathlineto{\pgfqpoint{0.522419in}{0.685648in}}%
\pgfpathlineto{\pgfqpoint{0.516981in}{0.681669in}}%
\pgfpathlineto{\pgfqpoint{0.505148in}{0.672036in}}%
\pgfpathlineto{\pgfqpoint{0.501324in}{0.668712in}}%
\pgfpathlineto{\pgfqpoint{0.490243in}{0.658425in}}%
\pgfpathlineto{\pgfqpoint{0.485668in}{0.653697in}}%
\pgfpathlineto{\pgfqpoint{0.477285in}{0.644814in}}%
\pgfpathlineto{\pgfqpoint{0.470011in}{0.635833in}}%
\pgfpathlineto{\pgfqpoint{0.466198in}{0.631203in}}%
\pgfpathlineto{\pgfqpoint{0.457028in}{0.617592in}}%
\pgfpathlineto{\pgfqpoint{0.454354in}{0.612362in}}%
\pgfpathlineto{\pgfqpoint{0.449765in}{0.603981in}}%
\pgfpathlineto{\pgfqpoint{0.444766in}{0.590370in}}%
\pgfpathlineto{\pgfqpoint{0.442270in}{0.576759in}}%
\pgfpathlineto{\pgfqpoint{0.442270in}{0.563148in}}%
\pgfpathlineto{\pgfqpoint{0.444766in}{0.549536in}}%
\pgfpathlineto{\pgfqpoint{0.449765in}{0.535925in}}%
\pgfpathlineto{\pgfqpoint{0.454354in}{0.527544in}}%
\pgfpathlineto{\pgfqpoint{0.457028in}{0.522314in}}%
\pgfpathlineto{\pgfqpoint{0.466198in}{0.508703in}}%
\pgfpathlineto{\pgfqpoint{0.470011in}{0.504073in}}%
\pgfpathlineto{\pgfqpoint{0.477285in}{0.495092in}}%
\pgfpathlineto{\pgfqpoint{0.485668in}{0.486209in}}%
\pgfpathlineto{\pgfqpoint{0.490243in}{0.481481in}}%
\pgfpathlineto{\pgfqpoint{0.501324in}{0.471194in}}%
\pgfpathlineto{\pgfqpoint{0.505148in}{0.467870in}}%
\pgfpathlineto{\pgfqpoint{0.516981in}{0.458237in}}%
\pgfpathlineto{\pgfqpoint{0.522419in}{0.454259in}}%
\pgfpathlineto{\pgfqpoint{0.532637in}{0.446971in}}%
\pgfpathlineto{\pgfqpoint{0.542968in}{0.440648in}}%
\pgfpathlineto{\pgfqpoint{0.548294in}{0.437333in}}%
\pgfpathlineto{\pgfqpoint{0.563950in}{0.429360in}}%
\pgfpathlineto{\pgfqpoint{0.569966in}{0.427036in}}%
\pgfpathlineto{\pgfqpoint{0.579607in}{0.423047in}}%
\pgfpathclose%
\pgfpathmoveto{\pgfqpoint{0.577087in}{0.454259in}}%
\pgfpathlineto{\pgfqpoint{0.563950in}{0.459483in}}%
\pgfpathlineto{\pgfqpoint{0.548294in}{0.467762in}}%
\pgfpathlineto{\pgfqpoint{0.548126in}{0.467870in}}%
\pgfpathlineto{\pgfqpoint{0.532637in}{0.478364in}}%
\pgfpathlineto{\pgfqpoint{0.528673in}{0.481481in}}%
\pgfpathlineto{\pgfqpoint{0.516981in}{0.491646in}}%
\pgfpathlineto{\pgfqpoint{0.513396in}{0.495092in}}%
\pgfpathlineto{\pgfqpoint{0.501324in}{0.508557in}}%
\pgfpathlineto{\pgfqpoint{0.501200in}{0.508703in}}%
\pgfpathlineto{\pgfqpoint{0.491676in}{0.522314in}}%
\pgfpathlineto{\pgfqpoint{0.485668in}{0.533734in}}%
\pgfpathlineto{\pgfqpoint{0.484525in}{0.535925in}}%
\pgfpathlineto{\pgfqpoint{0.479723in}{0.549536in}}%
\pgfpathlineto{\pgfqpoint{0.477326in}{0.563148in}}%
\pgfpathlineto{\pgfqpoint{0.477326in}{0.576759in}}%
\pgfpathlineto{\pgfqpoint{0.479723in}{0.590370in}}%
\pgfpathlineto{\pgfqpoint{0.484525in}{0.603981in}}%
\pgfpathlineto{\pgfqpoint{0.485668in}{0.606172in}}%
\pgfpathlineto{\pgfqpoint{0.491676in}{0.617592in}}%
\pgfpathlineto{\pgfqpoint{0.501200in}{0.631203in}}%
\pgfpathlineto{\pgfqpoint{0.501324in}{0.631349in}}%
\pgfpathlineto{\pgfqpoint{0.513396in}{0.644814in}}%
\pgfpathlineto{\pgfqpoint{0.516981in}{0.648261in}}%
\pgfpathlineto{\pgfqpoint{0.528673in}{0.658425in}}%
\pgfpathlineto{\pgfqpoint{0.532637in}{0.661542in}}%
\pgfpathlineto{\pgfqpoint{0.548126in}{0.672036in}}%
\pgfpathlineto{\pgfqpoint{0.548294in}{0.672144in}}%
\pgfpathlineto{\pgfqpoint{0.563950in}{0.680424in}}%
\pgfpathlineto{\pgfqpoint{0.577087in}{0.685648in}}%
\pgfpathlineto{\pgfqpoint{0.579607in}{0.686641in}}%
\pgfpathlineto{\pgfqpoint{0.595263in}{0.690815in}}%
\pgfpathlineto{\pgfqpoint{0.610920in}{0.692900in}}%
\pgfpathlineto{\pgfqpoint{0.626577in}{0.692900in}}%
\pgfpathlineto{\pgfqpoint{0.642233in}{0.690815in}}%
\pgfpathlineto{\pgfqpoint{0.657890in}{0.686641in}}%
\pgfpathlineto{\pgfqpoint{0.660410in}{0.685648in}}%
\pgfpathlineto{\pgfqpoint{0.673546in}{0.680424in}}%
\pgfpathlineto{\pgfqpoint{0.689203in}{0.672144in}}%
\pgfpathlineto{\pgfqpoint{0.689371in}{0.672036in}}%
\pgfpathlineto{\pgfqpoint{0.704859in}{0.661542in}}%
\pgfpathlineto{\pgfqpoint{0.708824in}{0.658425in}}%
\pgfpathlineto{\pgfqpoint{0.720516in}{0.648261in}}%
\pgfpathlineto{\pgfqpoint{0.724101in}{0.644814in}}%
\pgfpathlineto{\pgfqpoint{0.736173in}{0.631349in}}%
\pgfpathlineto{\pgfqpoint{0.736296in}{0.631203in}}%
\pgfpathlineto{\pgfqpoint{0.745820in}{0.617592in}}%
\pgfpathlineto{\pgfqpoint{0.751829in}{0.606172in}}%
\pgfpathlineto{\pgfqpoint{0.752971in}{0.603981in}}%
\pgfpathlineto{\pgfqpoint{0.757773in}{0.590370in}}%
\pgfpathlineto{\pgfqpoint{0.760171in}{0.576759in}}%
\pgfpathlineto{\pgfqpoint{0.760171in}{0.563148in}}%
\pgfpathlineto{\pgfqpoint{0.757773in}{0.549536in}}%
\pgfpathlineto{\pgfqpoint{0.752971in}{0.535925in}}%
\pgfpathlineto{\pgfqpoint{0.751829in}{0.533734in}}%
\pgfpathlineto{\pgfqpoint{0.745820in}{0.522314in}}%
\pgfpathlineto{\pgfqpoint{0.736296in}{0.508703in}}%
\pgfpathlineto{\pgfqpoint{0.736173in}{0.508557in}}%
\pgfpathlineto{\pgfqpoint{0.724101in}{0.495092in}}%
\pgfpathlineto{\pgfqpoint{0.720516in}{0.491646in}}%
\pgfpathlineto{\pgfqpoint{0.708824in}{0.481481in}}%
\pgfpathlineto{\pgfqpoint{0.704859in}{0.478364in}}%
\pgfpathlineto{\pgfqpoint{0.689371in}{0.467870in}}%
\pgfpathlineto{\pgfqpoint{0.689203in}{0.467762in}}%
\pgfpathlineto{\pgfqpoint{0.673546in}{0.459483in}}%
\pgfpathlineto{\pgfqpoint{0.660410in}{0.454259in}}%
\pgfpathlineto{\pgfqpoint{0.657890in}{0.453266in}}%
\pgfpathlineto{\pgfqpoint{0.642233in}{0.449091in}}%
\pgfpathlineto{\pgfqpoint{0.626577in}{0.447006in}}%
\pgfpathlineto{\pgfqpoint{0.610920in}{0.447006in}}%
\pgfpathlineto{\pgfqpoint{0.595263in}{0.449091in}}%
\pgfpathlineto{\pgfqpoint{0.579607in}{0.453266in}}%
\pgfpathlineto{\pgfqpoint{0.577087in}{0.454259in}}%
\pgfpathclose%
\pgfpathmoveto{\pgfqpoint{1.096274in}{0.423047in}}%
\pgfpathlineto{\pgfqpoint{1.111930in}{0.418701in}}%
\pgfpathlineto{\pgfqpoint{1.127587in}{0.416530in}}%
\pgfpathlineto{\pgfqpoint{1.143243in}{0.416530in}}%
\pgfpathlineto{\pgfqpoint{1.158900in}{0.418701in}}%
\pgfpathlineto{\pgfqpoint{1.174556in}{0.423047in}}%
\pgfpathlineto{\pgfqpoint{1.184197in}{0.427036in}}%
\pgfpathlineto{\pgfqpoint{1.190213in}{0.429360in}}%
\pgfpathlineto{\pgfqpoint{1.205870in}{0.437333in}}%
\pgfpathlineto{\pgfqpoint{1.211195in}{0.440648in}}%
\pgfpathlineto{\pgfqpoint{1.221526in}{0.446971in}}%
\pgfpathlineto{\pgfqpoint{1.231744in}{0.454259in}}%
\pgfpathlineto{\pgfqpoint{1.237183in}{0.458237in}}%
\pgfpathlineto{\pgfqpoint{1.249015in}{0.467870in}}%
\pgfpathlineto{\pgfqpoint{1.252839in}{0.471194in}}%
\pgfpathlineto{\pgfqpoint{1.263920in}{0.481481in}}%
\pgfpathlineto{\pgfqpoint{1.268496in}{0.486209in}}%
\pgfpathlineto{\pgfqpoint{1.276878in}{0.495092in}}%
\pgfpathlineto{\pgfqpoint{1.284152in}{0.504073in}}%
\pgfpathlineto{\pgfqpoint{1.287965in}{0.508703in}}%
\pgfpathlineto{\pgfqpoint{1.297136in}{0.522314in}}%
\pgfpathlineto{\pgfqpoint{1.299809in}{0.527544in}}%
\pgfpathlineto{\pgfqpoint{1.304398in}{0.535925in}}%
\pgfpathlineto{\pgfqpoint{1.309397in}{0.549536in}}%
\pgfpathlineto{\pgfqpoint{1.311894in}{0.563148in}}%
\pgfpathlineto{\pgfqpoint{1.311894in}{0.576759in}}%
\pgfpathlineto{\pgfqpoint{1.309397in}{0.590370in}}%
\pgfpathlineto{\pgfqpoint{1.304398in}{0.603981in}}%
\pgfpathlineto{\pgfqpoint{1.299809in}{0.612362in}}%
\pgfpathlineto{\pgfqpoint{1.297136in}{0.617592in}}%
\pgfpathlineto{\pgfqpoint{1.287965in}{0.631203in}}%
\pgfpathlineto{\pgfqpoint{1.284152in}{0.635833in}}%
\pgfpathlineto{\pgfqpoint{1.276878in}{0.644814in}}%
\pgfpathlineto{\pgfqpoint{1.268496in}{0.653697in}}%
\pgfpathlineto{\pgfqpoint{1.263920in}{0.658425in}}%
\pgfpathlineto{\pgfqpoint{1.252839in}{0.668712in}}%
\pgfpathlineto{\pgfqpoint{1.249015in}{0.672036in}}%
\pgfpathlineto{\pgfqpoint{1.237183in}{0.681669in}}%
\pgfpathlineto{\pgfqpoint{1.231744in}{0.685648in}}%
\pgfpathlineto{\pgfqpoint{1.221526in}{0.692935in}}%
\pgfpathlineto{\pgfqpoint{1.211195in}{0.699259in}}%
\pgfpathlineto{\pgfqpoint{1.205870in}{0.702573in}}%
\pgfpathlineto{\pgfqpoint{1.190213in}{0.710546in}}%
\pgfpathlineto{\pgfqpoint{1.184197in}{0.712870in}}%
\pgfpathlineto{\pgfqpoint{1.174556in}{0.716859in}}%
\pgfpathlineto{\pgfqpoint{1.158900in}{0.721205in}}%
\pgfpathlineto{\pgfqpoint{1.143243in}{0.723376in}}%
\pgfpathlineto{\pgfqpoint{1.127587in}{0.723376in}}%
\pgfpathlineto{\pgfqpoint{1.111930in}{0.721205in}}%
\pgfpathlineto{\pgfqpoint{1.096274in}{0.716859in}}%
\pgfpathlineto{\pgfqpoint{1.086633in}{0.712870in}}%
\pgfpathlineto{\pgfqpoint{1.080617in}{0.710546in}}%
\pgfpathlineto{\pgfqpoint{1.064960in}{0.702573in}}%
\pgfpathlineto{\pgfqpoint{1.059635in}{0.699259in}}%
\pgfpathlineto{\pgfqpoint{1.049304in}{0.692935in}}%
\pgfpathlineto{\pgfqpoint{1.039086in}{0.685648in}}%
\pgfpathlineto{\pgfqpoint{1.033647in}{0.681669in}}%
\pgfpathlineto{\pgfqpoint{1.021815in}{0.672036in}}%
\pgfpathlineto{\pgfqpoint{1.017991in}{0.668712in}}%
\pgfpathlineto{\pgfqpoint{1.006910in}{0.658425in}}%
\pgfpathlineto{\pgfqpoint{1.002334in}{0.653697in}}%
\pgfpathlineto{\pgfqpoint{0.993952in}{0.644814in}}%
\pgfpathlineto{\pgfqpoint{0.986678in}{0.635833in}}%
\pgfpathlineto{\pgfqpoint{0.982865in}{0.631203in}}%
\pgfpathlineto{\pgfqpoint{0.973694in}{0.617592in}}%
\pgfpathlineto{\pgfqpoint{0.971021in}{0.612362in}}%
\pgfpathlineto{\pgfqpoint{0.966432in}{0.603981in}}%
\pgfpathlineto{\pgfqpoint{0.961433in}{0.590370in}}%
\pgfpathlineto{\pgfqpoint{0.958936in}{0.576759in}}%
\pgfpathlineto{\pgfqpoint{0.958936in}{0.563148in}}%
\pgfpathlineto{\pgfqpoint{0.961433in}{0.549536in}}%
\pgfpathlineto{\pgfqpoint{0.966432in}{0.535925in}}%
\pgfpathlineto{\pgfqpoint{0.971021in}{0.527544in}}%
\pgfpathlineto{\pgfqpoint{0.973694in}{0.522314in}}%
\pgfpathlineto{\pgfqpoint{0.982865in}{0.508703in}}%
\pgfpathlineto{\pgfqpoint{0.986678in}{0.504073in}}%
\pgfpathlineto{\pgfqpoint{0.993952in}{0.495092in}}%
\pgfpathlineto{\pgfqpoint{1.002334in}{0.486209in}}%
\pgfpathlineto{\pgfqpoint{1.006910in}{0.481481in}}%
\pgfpathlineto{\pgfqpoint{1.017991in}{0.471194in}}%
\pgfpathlineto{\pgfqpoint{1.021815in}{0.467870in}}%
\pgfpathlineto{\pgfqpoint{1.033647in}{0.458237in}}%
\pgfpathlineto{\pgfqpoint{1.039086in}{0.454259in}}%
\pgfpathlineto{\pgfqpoint{1.049304in}{0.446971in}}%
\pgfpathlineto{\pgfqpoint{1.059635in}{0.440648in}}%
\pgfpathlineto{\pgfqpoint{1.064960in}{0.437333in}}%
\pgfpathlineto{\pgfqpoint{1.080617in}{0.429360in}}%
\pgfpathlineto{\pgfqpoint{1.086633in}{0.427036in}}%
\pgfpathlineto{\pgfqpoint{1.096274in}{0.423047in}}%
\pgfpathclose%
\pgfpathmoveto{\pgfqpoint{1.093753in}{0.454259in}}%
\pgfpathlineto{\pgfqpoint{1.080617in}{0.459483in}}%
\pgfpathlineto{\pgfqpoint{1.064960in}{0.467762in}}%
\pgfpathlineto{\pgfqpoint{1.064792in}{0.467870in}}%
\pgfpathlineto{\pgfqpoint{1.049304in}{0.478364in}}%
\pgfpathlineto{\pgfqpoint{1.045340in}{0.481481in}}%
\pgfpathlineto{\pgfqpoint{1.033647in}{0.491646in}}%
\pgfpathlineto{\pgfqpoint{1.030062in}{0.495092in}}%
\pgfpathlineto{\pgfqpoint{1.017991in}{0.508557in}}%
\pgfpathlineto{\pgfqpoint{1.017867in}{0.508703in}}%
\pgfpathlineto{\pgfqpoint{1.008343in}{0.522314in}}%
\pgfpathlineto{\pgfqpoint{1.002334in}{0.533734in}}%
\pgfpathlineto{\pgfqpoint{1.001192in}{0.535925in}}%
\pgfpathlineto{\pgfqpoint{0.996390in}{0.549536in}}%
\pgfpathlineto{\pgfqpoint{0.993992in}{0.563148in}}%
\pgfpathlineto{\pgfqpoint{0.993992in}{0.576759in}}%
\pgfpathlineto{\pgfqpoint{0.996390in}{0.590370in}}%
\pgfpathlineto{\pgfqpoint{1.001192in}{0.603981in}}%
\pgfpathlineto{\pgfqpoint{1.002334in}{0.606172in}}%
\pgfpathlineto{\pgfqpoint{1.008343in}{0.617592in}}%
\pgfpathlineto{\pgfqpoint{1.017867in}{0.631203in}}%
\pgfpathlineto{\pgfqpoint{1.017991in}{0.631349in}}%
\pgfpathlineto{\pgfqpoint{1.030062in}{0.644814in}}%
\pgfpathlineto{\pgfqpoint{1.033647in}{0.648261in}}%
\pgfpathlineto{\pgfqpoint{1.045340in}{0.658425in}}%
\pgfpathlineto{\pgfqpoint{1.049304in}{0.661542in}}%
\pgfpathlineto{\pgfqpoint{1.064792in}{0.672036in}}%
\pgfpathlineto{\pgfqpoint{1.064960in}{0.672144in}}%
\pgfpathlineto{\pgfqpoint{1.080617in}{0.680424in}}%
\pgfpathlineto{\pgfqpoint{1.093753in}{0.685648in}}%
\pgfpathlineto{\pgfqpoint{1.096274in}{0.686641in}}%
\pgfpathlineto{\pgfqpoint{1.111930in}{0.690815in}}%
\pgfpathlineto{\pgfqpoint{1.127587in}{0.692900in}}%
\pgfpathlineto{\pgfqpoint{1.143243in}{0.692900in}}%
\pgfpathlineto{\pgfqpoint{1.158900in}{0.690815in}}%
\pgfpathlineto{\pgfqpoint{1.174556in}{0.686641in}}%
\pgfpathlineto{\pgfqpoint{1.177077in}{0.685648in}}%
\pgfpathlineto{\pgfqpoint{1.190213in}{0.680424in}}%
\pgfpathlineto{\pgfqpoint{1.205870in}{0.672144in}}%
\pgfpathlineto{\pgfqpoint{1.206038in}{0.672036in}}%
\pgfpathlineto{\pgfqpoint{1.221526in}{0.661542in}}%
\pgfpathlineto{\pgfqpoint{1.225490in}{0.658425in}}%
\pgfpathlineto{\pgfqpoint{1.237183in}{0.648261in}}%
\pgfpathlineto{\pgfqpoint{1.240768in}{0.644814in}}%
\pgfpathlineto{\pgfqpoint{1.252839in}{0.631349in}}%
\pgfpathlineto{\pgfqpoint{1.252963in}{0.631203in}}%
\pgfpathlineto{\pgfqpoint{1.262487in}{0.617592in}}%
\pgfpathlineto{\pgfqpoint{1.268496in}{0.606172in}}%
\pgfpathlineto{\pgfqpoint{1.269638in}{0.603981in}}%
\pgfpathlineto{\pgfqpoint{1.274440in}{0.590370in}}%
\pgfpathlineto{\pgfqpoint{1.276838in}{0.576759in}}%
\pgfpathlineto{\pgfqpoint{1.276838in}{0.563148in}}%
\pgfpathlineto{\pgfqpoint{1.274440in}{0.549536in}}%
\pgfpathlineto{\pgfqpoint{1.269638in}{0.535925in}}%
\pgfpathlineto{\pgfqpoint{1.268496in}{0.533734in}}%
\pgfpathlineto{\pgfqpoint{1.262487in}{0.522314in}}%
\pgfpathlineto{\pgfqpoint{1.252963in}{0.508703in}}%
\pgfpathlineto{\pgfqpoint{1.252839in}{0.508557in}}%
\pgfpathlineto{\pgfqpoint{1.240768in}{0.495092in}}%
\pgfpathlineto{\pgfqpoint{1.237183in}{0.491646in}}%
\pgfpathlineto{\pgfqpoint{1.225490in}{0.481481in}}%
\pgfpathlineto{\pgfqpoint{1.221526in}{0.478364in}}%
\pgfpathlineto{\pgfqpoint{1.206038in}{0.467870in}}%
\pgfpathlineto{\pgfqpoint{1.205870in}{0.467762in}}%
\pgfpathlineto{\pgfqpoint{1.190213in}{0.459483in}}%
\pgfpathlineto{\pgfqpoint{1.177077in}{0.454259in}}%
\pgfpathlineto{\pgfqpoint{1.174556in}{0.453266in}}%
\pgfpathlineto{\pgfqpoint{1.158900in}{0.449091in}}%
\pgfpathlineto{\pgfqpoint{1.143243in}{0.447006in}}%
\pgfpathlineto{\pgfqpoint{1.127587in}{0.447006in}}%
\pgfpathlineto{\pgfqpoint{1.111930in}{0.449091in}}%
\pgfpathlineto{\pgfqpoint{1.096274in}{0.453266in}}%
\pgfpathlineto{\pgfqpoint{1.093753in}{0.454259in}}%
\pgfpathclose%
\pgfpathmoveto{\pgfqpoint{1.612940in}{0.423047in}}%
\pgfpathlineto{\pgfqpoint{1.628597in}{0.418701in}}%
\pgfpathlineto{\pgfqpoint{1.644253in}{0.416530in}}%
\pgfpathlineto{\pgfqpoint{1.659910in}{0.416530in}}%
\pgfpathlineto{\pgfqpoint{1.675567in}{0.418701in}}%
\pgfpathlineto{\pgfqpoint{1.691223in}{0.423047in}}%
\pgfpathlineto{\pgfqpoint{1.700864in}{0.427036in}}%
\pgfpathlineto{\pgfqpoint{1.706880in}{0.429360in}}%
\pgfpathlineto{\pgfqpoint{1.722536in}{0.437333in}}%
\pgfpathlineto{\pgfqpoint{1.727862in}{0.440648in}}%
\pgfpathlineto{\pgfqpoint{1.738193in}{0.446971in}}%
\pgfpathlineto{\pgfqpoint{1.748411in}{0.454259in}}%
\pgfpathlineto{\pgfqpoint{1.753849in}{0.458237in}}%
\pgfpathlineto{\pgfqpoint{1.765682in}{0.467870in}}%
\pgfpathlineto{\pgfqpoint{1.769506in}{0.471194in}}%
\pgfpathlineto{\pgfqpoint{1.780587in}{0.481481in}}%
\pgfpathlineto{\pgfqpoint{1.785162in}{0.486209in}}%
\pgfpathlineto{\pgfqpoint{1.793545in}{0.495092in}}%
\pgfpathlineto{\pgfqpoint{1.800819in}{0.504073in}}%
\pgfpathlineto{\pgfqpoint{1.804632in}{0.508703in}}%
\pgfpathlineto{\pgfqpoint{1.813802in}{0.522314in}}%
\pgfpathlineto{\pgfqpoint{1.816476in}{0.527544in}}%
\pgfpathlineto{\pgfqpoint{1.821065in}{0.535925in}}%
\pgfpathlineto{\pgfqpoint{1.826064in}{0.549536in}}%
\pgfpathlineto{\pgfqpoint{1.828560in}{0.563148in}}%
\pgfpathlineto{\pgfqpoint{1.828560in}{0.576759in}}%
\pgfpathlineto{\pgfqpoint{1.826064in}{0.590370in}}%
\pgfpathlineto{\pgfqpoint{1.821065in}{0.603981in}}%
\pgfpathlineto{\pgfqpoint{1.816476in}{0.612362in}}%
\pgfpathlineto{\pgfqpoint{1.813802in}{0.617592in}}%
\pgfpathlineto{\pgfqpoint{1.804632in}{0.631203in}}%
\pgfpathlineto{\pgfqpoint{1.800819in}{0.635833in}}%
\pgfpathlineto{\pgfqpoint{1.793545in}{0.644814in}}%
\pgfpathlineto{\pgfqpoint{1.785162in}{0.653697in}}%
\pgfpathlineto{\pgfqpoint{1.780587in}{0.658425in}}%
\pgfpathlineto{\pgfqpoint{1.769506in}{0.668712in}}%
\pgfpathlineto{\pgfqpoint{1.765682in}{0.672036in}}%
\pgfpathlineto{\pgfqpoint{1.753849in}{0.681669in}}%
\pgfpathlineto{\pgfqpoint{1.748411in}{0.685648in}}%
\pgfpathlineto{\pgfqpoint{1.738193in}{0.692935in}}%
\pgfpathlineto{\pgfqpoint{1.727862in}{0.699259in}}%
\pgfpathlineto{\pgfqpoint{1.722536in}{0.702573in}}%
\pgfpathlineto{\pgfqpoint{1.706880in}{0.710546in}}%
\pgfpathlineto{\pgfqpoint{1.700864in}{0.712870in}}%
\pgfpathlineto{\pgfqpoint{1.691223in}{0.716859in}}%
\pgfpathlineto{\pgfqpoint{1.675567in}{0.721205in}}%
\pgfpathlineto{\pgfqpoint{1.659910in}{0.723376in}}%
\pgfpathlineto{\pgfqpoint{1.644253in}{0.723376in}}%
\pgfpathlineto{\pgfqpoint{1.628597in}{0.721205in}}%
\pgfpathlineto{\pgfqpoint{1.612940in}{0.716859in}}%
\pgfpathlineto{\pgfqpoint{1.603299in}{0.712870in}}%
\pgfpathlineto{\pgfqpoint{1.597284in}{0.710546in}}%
\pgfpathlineto{\pgfqpoint{1.581627in}{0.702573in}}%
\pgfpathlineto{\pgfqpoint{1.576301in}{0.699259in}}%
\pgfpathlineto{\pgfqpoint{1.565971in}{0.692935in}}%
\pgfpathlineto{\pgfqpoint{1.555753in}{0.685648in}}%
\pgfpathlineto{\pgfqpoint{1.550314in}{0.681669in}}%
\pgfpathlineto{\pgfqpoint{1.538481in}{0.672036in}}%
\pgfpathlineto{\pgfqpoint{1.534657in}{0.668712in}}%
\pgfpathlineto{\pgfqpoint{1.523577in}{0.658425in}}%
\pgfpathlineto{\pgfqpoint{1.519001in}{0.653697in}}%
\pgfpathlineto{\pgfqpoint{1.510618in}{0.644814in}}%
\pgfpathlineto{\pgfqpoint{1.503344in}{0.635833in}}%
\pgfpathlineto{\pgfqpoint{1.499531in}{0.631203in}}%
\pgfpathlineto{\pgfqpoint{1.490361in}{0.617592in}}%
\pgfpathlineto{\pgfqpoint{1.487688in}{0.612362in}}%
\pgfpathlineto{\pgfqpoint{1.483099in}{0.603981in}}%
\pgfpathlineto{\pgfqpoint{1.478099in}{0.590370in}}%
\pgfpathlineto{\pgfqpoint{1.475603in}{0.576759in}}%
\pgfpathlineto{\pgfqpoint{1.475603in}{0.563148in}}%
\pgfpathlineto{\pgfqpoint{1.478099in}{0.549536in}}%
\pgfpathlineto{\pgfqpoint{1.483099in}{0.535925in}}%
\pgfpathlineto{\pgfqpoint{1.487688in}{0.527544in}}%
\pgfpathlineto{\pgfqpoint{1.490361in}{0.522314in}}%
\pgfpathlineto{\pgfqpoint{1.499531in}{0.508703in}}%
\pgfpathlineto{\pgfqpoint{1.503344in}{0.504073in}}%
\pgfpathlineto{\pgfqpoint{1.510618in}{0.495092in}}%
\pgfpathlineto{\pgfqpoint{1.519001in}{0.486209in}}%
\pgfpathlineto{\pgfqpoint{1.523577in}{0.481481in}}%
\pgfpathlineto{\pgfqpoint{1.534657in}{0.471194in}}%
\pgfpathlineto{\pgfqpoint{1.538481in}{0.467870in}}%
\pgfpathlineto{\pgfqpoint{1.550314in}{0.458237in}}%
\pgfpathlineto{\pgfqpoint{1.555753in}{0.454259in}}%
\pgfpathlineto{\pgfqpoint{1.565971in}{0.446971in}}%
\pgfpathlineto{\pgfqpoint{1.576301in}{0.440648in}}%
\pgfpathlineto{\pgfqpoint{1.581627in}{0.437333in}}%
\pgfpathlineto{\pgfqpoint{1.597284in}{0.429360in}}%
\pgfpathlineto{\pgfqpoint{1.603299in}{0.427036in}}%
\pgfpathlineto{\pgfqpoint{1.612940in}{0.423047in}}%
\pgfpathclose%
\pgfpathmoveto{\pgfqpoint{1.610420in}{0.454259in}}%
\pgfpathlineto{\pgfqpoint{1.597284in}{0.459483in}}%
\pgfpathlineto{\pgfqpoint{1.581627in}{0.467762in}}%
\pgfpathlineto{\pgfqpoint{1.581459in}{0.467870in}}%
\pgfpathlineto{\pgfqpoint{1.565971in}{0.478364in}}%
\pgfpathlineto{\pgfqpoint{1.562006in}{0.481481in}}%
\pgfpathlineto{\pgfqpoint{1.550314in}{0.491646in}}%
\pgfpathlineto{\pgfqpoint{1.546729in}{0.495092in}}%
\pgfpathlineto{\pgfqpoint{1.534657in}{0.508557in}}%
\pgfpathlineto{\pgfqpoint{1.534534in}{0.508703in}}%
\pgfpathlineto{\pgfqpoint{1.525010in}{0.522314in}}%
\pgfpathlineto{\pgfqpoint{1.519001in}{0.533734in}}%
\pgfpathlineto{\pgfqpoint{1.517859in}{0.535925in}}%
\pgfpathlineto{\pgfqpoint{1.513057in}{0.549536in}}%
\pgfpathlineto{\pgfqpoint{1.510659in}{0.563148in}}%
\pgfpathlineto{\pgfqpoint{1.510659in}{0.576759in}}%
\pgfpathlineto{\pgfqpoint{1.513057in}{0.590370in}}%
\pgfpathlineto{\pgfqpoint{1.517859in}{0.603981in}}%
\pgfpathlineto{\pgfqpoint{1.519001in}{0.606172in}}%
\pgfpathlineto{\pgfqpoint{1.525010in}{0.617592in}}%
\pgfpathlineto{\pgfqpoint{1.534534in}{0.631203in}}%
\pgfpathlineto{\pgfqpoint{1.534657in}{0.631349in}}%
\pgfpathlineto{\pgfqpoint{1.546729in}{0.644814in}}%
\pgfpathlineto{\pgfqpoint{1.550314in}{0.648261in}}%
\pgfpathlineto{\pgfqpoint{1.562006in}{0.658425in}}%
\pgfpathlineto{\pgfqpoint{1.565971in}{0.661542in}}%
\pgfpathlineto{\pgfqpoint{1.581459in}{0.672036in}}%
\pgfpathlineto{\pgfqpoint{1.581627in}{0.672144in}}%
\pgfpathlineto{\pgfqpoint{1.597284in}{0.680424in}}%
\pgfpathlineto{\pgfqpoint{1.610420in}{0.685648in}}%
\pgfpathlineto{\pgfqpoint{1.612940in}{0.686641in}}%
\pgfpathlineto{\pgfqpoint{1.628597in}{0.690815in}}%
\pgfpathlineto{\pgfqpoint{1.644253in}{0.692900in}}%
\pgfpathlineto{\pgfqpoint{1.659910in}{0.692900in}}%
\pgfpathlineto{\pgfqpoint{1.675567in}{0.690815in}}%
\pgfpathlineto{\pgfqpoint{1.691223in}{0.686641in}}%
\pgfpathlineto{\pgfqpoint{1.693743in}{0.685648in}}%
\pgfpathlineto{\pgfqpoint{1.706880in}{0.680424in}}%
\pgfpathlineto{\pgfqpoint{1.722536in}{0.672144in}}%
\pgfpathlineto{\pgfqpoint{1.722704in}{0.672036in}}%
\pgfpathlineto{\pgfqpoint{1.738193in}{0.661542in}}%
\pgfpathlineto{\pgfqpoint{1.742157in}{0.658425in}}%
\pgfpathlineto{\pgfqpoint{1.753849in}{0.648261in}}%
\pgfpathlineto{\pgfqpoint{1.757434in}{0.644814in}}%
\pgfpathlineto{\pgfqpoint{1.769506in}{0.631349in}}%
\pgfpathlineto{\pgfqpoint{1.769630in}{0.631203in}}%
\pgfpathlineto{\pgfqpoint{1.779154in}{0.617592in}}%
\pgfpathlineto{\pgfqpoint{1.785162in}{0.606172in}}%
\pgfpathlineto{\pgfqpoint{1.786305in}{0.603981in}}%
\pgfpathlineto{\pgfqpoint{1.791107in}{0.590370in}}%
\pgfpathlineto{\pgfqpoint{1.793504in}{0.576759in}}%
\pgfpathlineto{\pgfqpoint{1.793504in}{0.563148in}}%
\pgfpathlineto{\pgfqpoint{1.791107in}{0.549536in}}%
\pgfpathlineto{\pgfqpoint{1.786305in}{0.535925in}}%
\pgfpathlineto{\pgfqpoint{1.785162in}{0.533734in}}%
\pgfpathlineto{\pgfqpoint{1.779154in}{0.522314in}}%
\pgfpathlineto{\pgfqpoint{1.769630in}{0.508703in}}%
\pgfpathlineto{\pgfqpoint{1.769506in}{0.508557in}}%
\pgfpathlineto{\pgfqpoint{1.757434in}{0.495092in}}%
\pgfpathlineto{\pgfqpoint{1.753849in}{0.491646in}}%
\pgfpathlineto{\pgfqpoint{1.742157in}{0.481481in}}%
\pgfpathlineto{\pgfqpoint{1.738193in}{0.478364in}}%
\pgfpathlineto{\pgfqpoint{1.722704in}{0.467870in}}%
\pgfpathlineto{\pgfqpoint{1.722536in}{0.467762in}}%
\pgfpathlineto{\pgfqpoint{1.706880in}{0.459483in}}%
\pgfpathlineto{\pgfqpoint{1.693743in}{0.454259in}}%
\pgfpathlineto{\pgfqpoint{1.691223in}{0.453266in}}%
\pgfpathlineto{\pgfqpoint{1.675567in}{0.449091in}}%
\pgfpathlineto{\pgfqpoint{1.659910in}{0.447006in}}%
\pgfpathlineto{\pgfqpoint{1.644253in}{0.447006in}}%
\pgfpathlineto{\pgfqpoint{1.628597in}{0.449091in}}%
\pgfpathlineto{\pgfqpoint{1.612940in}{0.453266in}}%
\pgfpathlineto{\pgfqpoint{1.610420in}{0.454259in}}%
\pgfpathclose%
\pgfpathmoveto{\pgfqpoint{0.579607in}{0.872213in}}%
\pgfpathlineto{\pgfqpoint{0.595263in}{0.867867in}}%
\pgfpathlineto{\pgfqpoint{0.610920in}{0.865697in}}%
\pgfpathlineto{\pgfqpoint{0.626577in}{0.865697in}}%
\pgfpathlineto{\pgfqpoint{0.642233in}{0.867867in}}%
\pgfpathlineto{\pgfqpoint{0.657890in}{0.872213in}}%
\pgfpathlineto{\pgfqpoint{0.667531in}{0.876203in}}%
\pgfpathlineto{\pgfqpoint{0.673546in}{0.878527in}}%
\pgfpathlineto{\pgfqpoint{0.689203in}{0.886499in}}%
\pgfpathlineto{\pgfqpoint{0.694529in}{0.889814in}}%
\pgfpathlineto{\pgfqpoint{0.704859in}{0.896138in}}%
\pgfpathlineto{\pgfqpoint{0.715077in}{0.903425in}}%
\pgfpathlineto{\pgfqpoint{0.720516in}{0.907403in}}%
\pgfpathlineto{\pgfqpoint{0.732349in}{0.917036in}}%
\pgfpathlineto{\pgfqpoint{0.736173in}{0.920361in}}%
\pgfpathlineto{\pgfqpoint{0.747253in}{0.930648in}}%
\pgfpathlineto{\pgfqpoint{0.751829in}{0.935376in}}%
\pgfpathlineto{\pgfqpoint{0.760212in}{0.944259in}}%
\pgfpathlineto{\pgfqpoint{0.767486in}{0.953240in}}%
\pgfpathlineto{\pgfqpoint{0.771299in}{0.957870in}}%
\pgfpathlineto{\pgfqpoint{0.780469in}{0.971481in}}%
\pgfpathlineto{\pgfqpoint{0.783142in}{0.976711in}}%
\pgfpathlineto{\pgfqpoint{0.787731in}{0.985092in}}%
\pgfpathlineto{\pgfqpoint{0.792731in}{0.998703in}}%
\pgfpathlineto{\pgfqpoint{0.795227in}{1.012314in}}%
\pgfpathlineto{\pgfqpoint{0.795227in}{1.025925in}}%
\pgfpathlineto{\pgfqpoint{0.792731in}{1.039536in}}%
\pgfpathlineto{\pgfqpoint{0.787731in}{1.053148in}}%
\pgfpathlineto{\pgfqpoint{0.783142in}{1.061529in}}%
\pgfpathlineto{\pgfqpoint{0.780469in}{1.066759in}}%
\pgfpathlineto{\pgfqpoint{0.771299in}{1.080370in}}%
\pgfpathlineto{\pgfqpoint{0.767486in}{1.085000in}}%
\pgfpathlineto{\pgfqpoint{0.760212in}{1.093981in}}%
\pgfpathlineto{\pgfqpoint{0.751829in}{1.102864in}}%
\pgfpathlineto{\pgfqpoint{0.747253in}{1.107592in}}%
\pgfpathlineto{\pgfqpoint{0.736173in}{1.117879in}}%
\pgfpathlineto{\pgfqpoint{0.732349in}{1.121203in}}%
\pgfpathlineto{\pgfqpoint{0.720516in}{1.130836in}}%
\pgfpathlineto{\pgfqpoint{0.715077in}{1.134814in}}%
\pgfpathlineto{\pgfqpoint{0.704859in}{1.142101in}}%
\pgfpathlineto{\pgfqpoint{0.694529in}{1.148425in}}%
\pgfpathlineto{\pgfqpoint{0.689203in}{1.151740in}}%
\pgfpathlineto{\pgfqpoint{0.673546in}{1.159712in}}%
\pgfpathlineto{\pgfqpoint{0.667531in}{1.162036in}}%
\pgfpathlineto{\pgfqpoint{0.657890in}{1.166026in}}%
\pgfpathlineto{\pgfqpoint{0.642233in}{1.170372in}}%
\pgfpathlineto{\pgfqpoint{0.626577in}{1.172542in}}%
\pgfpathlineto{\pgfqpoint{0.610920in}{1.172542in}}%
\pgfpathlineto{\pgfqpoint{0.595263in}{1.170372in}}%
\pgfpathlineto{\pgfqpoint{0.579607in}{1.166026in}}%
\pgfpathlineto{\pgfqpoint{0.569966in}{1.162036in}}%
\pgfpathlineto{\pgfqpoint{0.563950in}{1.159712in}}%
\pgfpathlineto{\pgfqpoint{0.548294in}{1.151740in}}%
\pgfpathlineto{\pgfqpoint{0.542968in}{1.148425in}}%
\pgfpathlineto{\pgfqpoint{0.532637in}{1.142101in}}%
\pgfpathlineto{\pgfqpoint{0.522419in}{1.134814in}}%
\pgfpathlineto{\pgfqpoint{0.516981in}{1.130836in}}%
\pgfpathlineto{\pgfqpoint{0.505148in}{1.121203in}}%
\pgfpathlineto{\pgfqpoint{0.501324in}{1.117879in}}%
\pgfpathlineto{\pgfqpoint{0.490243in}{1.107592in}}%
\pgfpathlineto{\pgfqpoint{0.485668in}{1.102864in}}%
\pgfpathlineto{\pgfqpoint{0.477285in}{1.093981in}}%
\pgfpathlineto{\pgfqpoint{0.470011in}{1.085000in}}%
\pgfpathlineto{\pgfqpoint{0.466198in}{1.080370in}}%
\pgfpathlineto{\pgfqpoint{0.457028in}{1.066759in}}%
\pgfpathlineto{\pgfqpoint{0.454354in}{1.061529in}}%
\pgfpathlineto{\pgfqpoint{0.449765in}{1.053148in}}%
\pgfpathlineto{\pgfqpoint{0.444766in}{1.039536in}}%
\pgfpathlineto{\pgfqpoint{0.442270in}{1.025925in}}%
\pgfpathlineto{\pgfqpoint{0.442270in}{1.012314in}}%
\pgfpathlineto{\pgfqpoint{0.444766in}{0.998703in}}%
\pgfpathlineto{\pgfqpoint{0.449765in}{0.985092in}}%
\pgfpathlineto{\pgfqpoint{0.454354in}{0.976711in}}%
\pgfpathlineto{\pgfqpoint{0.457028in}{0.971481in}}%
\pgfpathlineto{\pgfqpoint{0.466198in}{0.957870in}}%
\pgfpathlineto{\pgfqpoint{0.470011in}{0.953240in}}%
\pgfpathlineto{\pgfqpoint{0.477285in}{0.944259in}}%
\pgfpathlineto{\pgfqpoint{0.485668in}{0.935376in}}%
\pgfpathlineto{\pgfqpoint{0.490243in}{0.930648in}}%
\pgfpathlineto{\pgfqpoint{0.501324in}{0.920361in}}%
\pgfpathlineto{\pgfqpoint{0.505148in}{0.917036in}}%
\pgfpathlineto{\pgfqpoint{0.516981in}{0.907403in}}%
\pgfpathlineto{\pgfqpoint{0.522419in}{0.903425in}}%
\pgfpathlineto{\pgfqpoint{0.532637in}{0.896138in}}%
\pgfpathlineto{\pgfqpoint{0.542968in}{0.889814in}}%
\pgfpathlineto{\pgfqpoint{0.548294in}{0.886499in}}%
\pgfpathlineto{\pgfqpoint{0.563950in}{0.878527in}}%
\pgfpathlineto{\pgfqpoint{0.569966in}{0.876203in}}%
\pgfpathlineto{\pgfqpoint{0.579607in}{0.872213in}}%
\pgfpathclose%
\pgfpathmoveto{\pgfqpoint{0.577087in}{0.903425in}}%
\pgfpathlineto{\pgfqpoint{0.563950in}{0.908649in}}%
\pgfpathlineto{\pgfqpoint{0.548294in}{0.916929in}}%
\pgfpathlineto{\pgfqpoint{0.548126in}{0.917036in}}%
\pgfpathlineto{\pgfqpoint{0.532637in}{0.927531in}}%
\pgfpathlineto{\pgfqpoint{0.528673in}{0.930648in}}%
\pgfpathlineto{\pgfqpoint{0.516981in}{0.940812in}}%
\pgfpathlineto{\pgfqpoint{0.513396in}{0.944259in}}%
\pgfpathlineto{\pgfqpoint{0.501324in}{0.957724in}}%
\pgfpathlineto{\pgfqpoint{0.501200in}{0.957870in}}%
\pgfpathlineto{\pgfqpoint{0.491676in}{0.971481in}}%
\pgfpathlineto{\pgfqpoint{0.485668in}{0.982901in}}%
\pgfpathlineto{\pgfqpoint{0.484525in}{0.985092in}}%
\pgfpathlineto{\pgfqpoint{0.479723in}{0.998703in}}%
\pgfpathlineto{\pgfqpoint{0.477326in}{1.012314in}}%
\pgfpathlineto{\pgfqpoint{0.477326in}{1.025925in}}%
\pgfpathlineto{\pgfqpoint{0.479723in}{1.039536in}}%
\pgfpathlineto{\pgfqpoint{0.484525in}{1.053148in}}%
\pgfpathlineto{\pgfqpoint{0.485668in}{1.055339in}}%
\pgfpathlineto{\pgfqpoint{0.491676in}{1.066759in}}%
\pgfpathlineto{\pgfqpoint{0.501200in}{1.080370in}}%
\pgfpathlineto{\pgfqpoint{0.501324in}{1.080516in}}%
\pgfpathlineto{\pgfqpoint{0.513396in}{1.093981in}}%
\pgfpathlineto{\pgfqpoint{0.516981in}{1.097427in}}%
\pgfpathlineto{\pgfqpoint{0.528673in}{1.107592in}}%
\pgfpathlineto{\pgfqpoint{0.532637in}{1.110709in}}%
\pgfpathlineto{\pgfqpoint{0.548126in}{1.121203in}}%
\pgfpathlineto{\pgfqpoint{0.548294in}{1.121311in}}%
\pgfpathlineto{\pgfqpoint{0.563950in}{1.129590in}}%
\pgfpathlineto{\pgfqpoint{0.577087in}{1.134814in}}%
\pgfpathlineto{\pgfqpoint{0.579607in}{1.135807in}}%
\pgfpathlineto{\pgfqpoint{0.595263in}{1.139982in}}%
\pgfpathlineto{\pgfqpoint{0.610920in}{1.142066in}}%
\pgfpathlineto{\pgfqpoint{0.626577in}{1.142066in}}%
\pgfpathlineto{\pgfqpoint{0.642233in}{1.139982in}}%
\pgfpathlineto{\pgfqpoint{0.657890in}{1.135807in}}%
\pgfpathlineto{\pgfqpoint{0.660410in}{1.134814in}}%
\pgfpathlineto{\pgfqpoint{0.673546in}{1.129590in}}%
\pgfpathlineto{\pgfqpoint{0.689203in}{1.121311in}}%
\pgfpathlineto{\pgfqpoint{0.689371in}{1.121203in}}%
\pgfpathlineto{\pgfqpoint{0.704859in}{1.110709in}}%
\pgfpathlineto{\pgfqpoint{0.708824in}{1.107592in}}%
\pgfpathlineto{\pgfqpoint{0.720516in}{1.097427in}}%
\pgfpathlineto{\pgfqpoint{0.724101in}{1.093981in}}%
\pgfpathlineto{\pgfqpoint{0.736173in}{1.080516in}}%
\pgfpathlineto{\pgfqpoint{0.736296in}{1.080370in}}%
\pgfpathlineto{\pgfqpoint{0.745820in}{1.066759in}}%
\pgfpathlineto{\pgfqpoint{0.751829in}{1.055339in}}%
\pgfpathlineto{\pgfqpoint{0.752971in}{1.053148in}}%
\pgfpathlineto{\pgfqpoint{0.757773in}{1.039536in}}%
\pgfpathlineto{\pgfqpoint{0.760171in}{1.025925in}}%
\pgfpathlineto{\pgfqpoint{0.760171in}{1.012314in}}%
\pgfpathlineto{\pgfqpoint{0.757773in}{0.998703in}}%
\pgfpathlineto{\pgfqpoint{0.752971in}{0.985092in}}%
\pgfpathlineto{\pgfqpoint{0.751829in}{0.982901in}}%
\pgfpathlineto{\pgfqpoint{0.745820in}{0.971481in}}%
\pgfpathlineto{\pgfqpoint{0.736296in}{0.957870in}}%
\pgfpathlineto{\pgfqpoint{0.736173in}{0.957724in}}%
\pgfpathlineto{\pgfqpoint{0.724101in}{0.944259in}}%
\pgfpathlineto{\pgfqpoint{0.720516in}{0.940812in}}%
\pgfpathlineto{\pgfqpoint{0.708824in}{0.930648in}}%
\pgfpathlineto{\pgfqpoint{0.704859in}{0.927531in}}%
\pgfpathlineto{\pgfqpoint{0.689371in}{0.917036in}}%
\pgfpathlineto{\pgfqpoint{0.689203in}{0.916929in}}%
\pgfpathlineto{\pgfqpoint{0.673546in}{0.908649in}}%
\pgfpathlineto{\pgfqpoint{0.660410in}{0.903425in}}%
\pgfpathlineto{\pgfqpoint{0.657890in}{0.902432in}}%
\pgfpathlineto{\pgfqpoint{0.642233in}{0.898258in}}%
\pgfpathlineto{\pgfqpoint{0.626577in}{0.896173in}}%
\pgfpathlineto{\pgfqpoint{0.610920in}{0.896173in}}%
\pgfpathlineto{\pgfqpoint{0.595263in}{0.898258in}}%
\pgfpathlineto{\pgfqpoint{0.579607in}{0.902432in}}%
\pgfpathlineto{\pgfqpoint{0.577087in}{0.903425in}}%
\pgfpathclose%
\pgfpathmoveto{\pgfqpoint{1.096274in}{0.872213in}}%
\pgfpathlineto{\pgfqpoint{1.111930in}{0.867867in}}%
\pgfpathlineto{\pgfqpoint{1.127587in}{0.865697in}}%
\pgfpathlineto{\pgfqpoint{1.143243in}{0.865697in}}%
\pgfpathlineto{\pgfqpoint{1.158900in}{0.867867in}}%
\pgfpathlineto{\pgfqpoint{1.174556in}{0.872213in}}%
\pgfpathlineto{\pgfqpoint{1.184197in}{0.876203in}}%
\pgfpathlineto{\pgfqpoint{1.190213in}{0.878527in}}%
\pgfpathlineto{\pgfqpoint{1.205870in}{0.886499in}}%
\pgfpathlineto{\pgfqpoint{1.211195in}{0.889814in}}%
\pgfpathlineto{\pgfqpoint{1.221526in}{0.896138in}}%
\pgfpathlineto{\pgfqpoint{1.231744in}{0.903425in}}%
\pgfpathlineto{\pgfqpoint{1.237183in}{0.907403in}}%
\pgfpathlineto{\pgfqpoint{1.249015in}{0.917036in}}%
\pgfpathlineto{\pgfqpoint{1.252839in}{0.920361in}}%
\pgfpathlineto{\pgfqpoint{1.263920in}{0.930648in}}%
\pgfpathlineto{\pgfqpoint{1.268496in}{0.935376in}}%
\pgfpathlineto{\pgfqpoint{1.276878in}{0.944259in}}%
\pgfpathlineto{\pgfqpoint{1.284152in}{0.953240in}}%
\pgfpathlineto{\pgfqpoint{1.287965in}{0.957870in}}%
\pgfpathlineto{\pgfqpoint{1.297136in}{0.971481in}}%
\pgfpathlineto{\pgfqpoint{1.299809in}{0.976711in}}%
\pgfpathlineto{\pgfqpoint{1.304398in}{0.985092in}}%
\pgfpathlineto{\pgfqpoint{1.309397in}{0.998703in}}%
\pgfpathlineto{\pgfqpoint{1.311894in}{1.012314in}}%
\pgfpathlineto{\pgfqpoint{1.311894in}{1.025925in}}%
\pgfpathlineto{\pgfqpoint{1.309397in}{1.039536in}}%
\pgfpathlineto{\pgfqpoint{1.304398in}{1.053148in}}%
\pgfpathlineto{\pgfqpoint{1.299809in}{1.061529in}}%
\pgfpathlineto{\pgfqpoint{1.297136in}{1.066759in}}%
\pgfpathlineto{\pgfqpoint{1.287965in}{1.080370in}}%
\pgfpathlineto{\pgfqpoint{1.284152in}{1.085000in}}%
\pgfpathlineto{\pgfqpoint{1.276878in}{1.093981in}}%
\pgfpathlineto{\pgfqpoint{1.268496in}{1.102864in}}%
\pgfpathlineto{\pgfqpoint{1.263920in}{1.107592in}}%
\pgfpathlineto{\pgfqpoint{1.252839in}{1.117879in}}%
\pgfpathlineto{\pgfqpoint{1.249015in}{1.121203in}}%
\pgfpathlineto{\pgfqpoint{1.237183in}{1.130836in}}%
\pgfpathlineto{\pgfqpoint{1.231744in}{1.134814in}}%
\pgfpathlineto{\pgfqpoint{1.221526in}{1.142101in}}%
\pgfpathlineto{\pgfqpoint{1.211195in}{1.148425in}}%
\pgfpathlineto{\pgfqpoint{1.205870in}{1.151740in}}%
\pgfpathlineto{\pgfqpoint{1.190213in}{1.159712in}}%
\pgfpathlineto{\pgfqpoint{1.184197in}{1.162036in}}%
\pgfpathlineto{\pgfqpoint{1.174556in}{1.166026in}}%
\pgfpathlineto{\pgfqpoint{1.158900in}{1.170372in}}%
\pgfpathlineto{\pgfqpoint{1.143243in}{1.172542in}}%
\pgfpathlineto{\pgfqpoint{1.127587in}{1.172542in}}%
\pgfpathlineto{\pgfqpoint{1.111930in}{1.170372in}}%
\pgfpathlineto{\pgfqpoint{1.096274in}{1.166026in}}%
\pgfpathlineto{\pgfqpoint{1.086633in}{1.162036in}}%
\pgfpathlineto{\pgfqpoint{1.080617in}{1.159712in}}%
\pgfpathlineto{\pgfqpoint{1.064960in}{1.151740in}}%
\pgfpathlineto{\pgfqpoint{1.059635in}{1.148425in}}%
\pgfpathlineto{\pgfqpoint{1.049304in}{1.142101in}}%
\pgfpathlineto{\pgfqpoint{1.039086in}{1.134814in}}%
\pgfpathlineto{\pgfqpoint{1.033647in}{1.130836in}}%
\pgfpathlineto{\pgfqpoint{1.021815in}{1.121203in}}%
\pgfpathlineto{\pgfqpoint{1.017991in}{1.117879in}}%
\pgfpathlineto{\pgfqpoint{1.006910in}{1.107592in}}%
\pgfpathlineto{\pgfqpoint{1.002334in}{1.102864in}}%
\pgfpathlineto{\pgfqpoint{0.993952in}{1.093981in}}%
\pgfpathlineto{\pgfqpoint{0.986678in}{1.085000in}}%
\pgfpathlineto{\pgfqpoint{0.982865in}{1.080370in}}%
\pgfpathlineto{\pgfqpoint{0.973694in}{1.066759in}}%
\pgfpathlineto{\pgfqpoint{0.971021in}{1.061529in}}%
\pgfpathlineto{\pgfqpoint{0.966432in}{1.053148in}}%
\pgfpathlineto{\pgfqpoint{0.961433in}{1.039536in}}%
\pgfpathlineto{\pgfqpoint{0.958936in}{1.025925in}}%
\pgfpathlineto{\pgfqpoint{0.958936in}{1.012314in}}%
\pgfpathlineto{\pgfqpoint{0.961433in}{0.998703in}}%
\pgfpathlineto{\pgfqpoint{0.966432in}{0.985092in}}%
\pgfpathlineto{\pgfqpoint{0.971021in}{0.976711in}}%
\pgfpathlineto{\pgfqpoint{0.973694in}{0.971481in}}%
\pgfpathlineto{\pgfqpoint{0.982865in}{0.957870in}}%
\pgfpathlineto{\pgfqpoint{0.986678in}{0.953240in}}%
\pgfpathlineto{\pgfqpoint{0.993952in}{0.944259in}}%
\pgfpathlineto{\pgfqpoint{1.002334in}{0.935376in}}%
\pgfpathlineto{\pgfqpoint{1.006910in}{0.930648in}}%
\pgfpathlineto{\pgfqpoint{1.017991in}{0.920361in}}%
\pgfpathlineto{\pgfqpoint{1.021815in}{0.917036in}}%
\pgfpathlineto{\pgfqpoint{1.033647in}{0.907403in}}%
\pgfpathlineto{\pgfqpoint{1.039086in}{0.903425in}}%
\pgfpathlineto{\pgfqpoint{1.049304in}{0.896138in}}%
\pgfpathlineto{\pgfqpoint{1.059635in}{0.889814in}}%
\pgfpathlineto{\pgfqpoint{1.064960in}{0.886499in}}%
\pgfpathlineto{\pgfqpoint{1.080617in}{0.878527in}}%
\pgfpathlineto{\pgfqpoint{1.086633in}{0.876203in}}%
\pgfpathlineto{\pgfqpoint{1.096274in}{0.872213in}}%
\pgfpathclose%
\pgfpathmoveto{\pgfqpoint{1.093753in}{0.903425in}}%
\pgfpathlineto{\pgfqpoint{1.080617in}{0.908649in}}%
\pgfpathlineto{\pgfqpoint{1.064960in}{0.916929in}}%
\pgfpathlineto{\pgfqpoint{1.064792in}{0.917036in}}%
\pgfpathlineto{\pgfqpoint{1.049304in}{0.927531in}}%
\pgfpathlineto{\pgfqpoint{1.045340in}{0.930648in}}%
\pgfpathlineto{\pgfqpoint{1.033647in}{0.940812in}}%
\pgfpathlineto{\pgfqpoint{1.030062in}{0.944259in}}%
\pgfpathlineto{\pgfqpoint{1.017991in}{0.957724in}}%
\pgfpathlineto{\pgfqpoint{1.017867in}{0.957870in}}%
\pgfpathlineto{\pgfqpoint{1.008343in}{0.971481in}}%
\pgfpathlineto{\pgfqpoint{1.002334in}{0.982901in}}%
\pgfpathlineto{\pgfqpoint{1.001192in}{0.985092in}}%
\pgfpathlineto{\pgfqpoint{0.996390in}{0.998703in}}%
\pgfpathlineto{\pgfqpoint{0.993992in}{1.012314in}}%
\pgfpathlineto{\pgfqpoint{0.993992in}{1.025925in}}%
\pgfpathlineto{\pgfqpoint{0.996390in}{1.039536in}}%
\pgfpathlineto{\pgfqpoint{1.001192in}{1.053148in}}%
\pgfpathlineto{\pgfqpoint{1.002334in}{1.055339in}}%
\pgfpathlineto{\pgfqpoint{1.008343in}{1.066759in}}%
\pgfpathlineto{\pgfqpoint{1.017867in}{1.080370in}}%
\pgfpathlineto{\pgfqpoint{1.017991in}{1.080516in}}%
\pgfpathlineto{\pgfqpoint{1.030062in}{1.093981in}}%
\pgfpathlineto{\pgfqpoint{1.033647in}{1.097427in}}%
\pgfpathlineto{\pgfqpoint{1.045340in}{1.107592in}}%
\pgfpathlineto{\pgfqpoint{1.049304in}{1.110709in}}%
\pgfpathlineto{\pgfqpoint{1.064792in}{1.121203in}}%
\pgfpathlineto{\pgfqpoint{1.064960in}{1.121311in}}%
\pgfpathlineto{\pgfqpoint{1.080617in}{1.129590in}}%
\pgfpathlineto{\pgfqpoint{1.093753in}{1.134814in}}%
\pgfpathlineto{\pgfqpoint{1.096274in}{1.135807in}}%
\pgfpathlineto{\pgfqpoint{1.111930in}{1.139982in}}%
\pgfpathlineto{\pgfqpoint{1.127587in}{1.142066in}}%
\pgfpathlineto{\pgfqpoint{1.143243in}{1.142066in}}%
\pgfpathlineto{\pgfqpoint{1.158900in}{1.139982in}}%
\pgfpathlineto{\pgfqpoint{1.174556in}{1.135807in}}%
\pgfpathlineto{\pgfqpoint{1.177077in}{1.134814in}}%
\pgfpathlineto{\pgfqpoint{1.190213in}{1.129590in}}%
\pgfpathlineto{\pgfqpoint{1.205870in}{1.121311in}}%
\pgfpathlineto{\pgfqpoint{1.206038in}{1.121203in}}%
\pgfpathlineto{\pgfqpoint{1.221526in}{1.110709in}}%
\pgfpathlineto{\pgfqpoint{1.225490in}{1.107592in}}%
\pgfpathlineto{\pgfqpoint{1.237183in}{1.097427in}}%
\pgfpathlineto{\pgfqpoint{1.240768in}{1.093981in}}%
\pgfpathlineto{\pgfqpoint{1.252839in}{1.080516in}}%
\pgfpathlineto{\pgfqpoint{1.252963in}{1.080370in}}%
\pgfpathlineto{\pgfqpoint{1.262487in}{1.066759in}}%
\pgfpathlineto{\pgfqpoint{1.268496in}{1.055339in}}%
\pgfpathlineto{\pgfqpoint{1.269638in}{1.053148in}}%
\pgfpathlineto{\pgfqpoint{1.274440in}{1.039536in}}%
\pgfpathlineto{\pgfqpoint{1.276838in}{1.025925in}}%
\pgfpathlineto{\pgfqpoint{1.276838in}{1.012314in}}%
\pgfpathlineto{\pgfqpoint{1.274440in}{0.998703in}}%
\pgfpathlineto{\pgfqpoint{1.269638in}{0.985092in}}%
\pgfpathlineto{\pgfqpoint{1.268496in}{0.982901in}}%
\pgfpathlineto{\pgfqpoint{1.262487in}{0.971481in}}%
\pgfpathlineto{\pgfqpoint{1.252963in}{0.957870in}}%
\pgfpathlineto{\pgfqpoint{1.252839in}{0.957724in}}%
\pgfpathlineto{\pgfqpoint{1.240768in}{0.944259in}}%
\pgfpathlineto{\pgfqpoint{1.237183in}{0.940812in}}%
\pgfpathlineto{\pgfqpoint{1.225490in}{0.930648in}}%
\pgfpathlineto{\pgfqpoint{1.221526in}{0.927531in}}%
\pgfpathlineto{\pgfqpoint{1.206038in}{0.917036in}}%
\pgfpathlineto{\pgfqpoint{1.205870in}{0.916929in}}%
\pgfpathlineto{\pgfqpoint{1.190213in}{0.908649in}}%
\pgfpathlineto{\pgfqpoint{1.177077in}{0.903425in}}%
\pgfpathlineto{\pgfqpoint{1.174556in}{0.902432in}}%
\pgfpathlineto{\pgfqpoint{1.158900in}{0.898258in}}%
\pgfpathlineto{\pgfqpoint{1.143243in}{0.896173in}}%
\pgfpathlineto{\pgfqpoint{1.127587in}{0.896173in}}%
\pgfpathlineto{\pgfqpoint{1.111930in}{0.898258in}}%
\pgfpathlineto{\pgfqpoint{1.096274in}{0.902432in}}%
\pgfpathlineto{\pgfqpoint{1.093753in}{0.903425in}}%
\pgfpathclose%
\pgfpathmoveto{\pgfqpoint{1.612940in}{0.872213in}}%
\pgfpathlineto{\pgfqpoint{1.628597in}{0.867867in}}%
\pgfpathlineto{\pgfqpoint{1.644253in}{0.865697in}}%
\pgfpathlineto{\pgfqpoint{1.659910in}{0.865697in}}%
\pgfpathlineto{\pgfqpoint{1.675567in}{0.867867in}}%
\pgfpathlineto{\pgfqpoint{1.691223in}{0.872213in}}%
\pgfpathlineto{\pgfqpoint{1.700864in}{0.876203in}}%
\pgfpathlineto{\pgfqpoint{1.706880in}{0.878527in}}%
\pgfpathlineto{\pgfqpoint{1.722536in}{0.886499in}}%
\pgfpathlineto{\pgfqpoint{1.727862in}{0.889814in}}%
\pgfpathlineto{\pgfqpoint{1.738193in}{0.896138in}}%
\pgfpathlineto{\pgfqpoint{1.748411in}{0.903425in}}%
\pgfpathlineto{\pgfqpoint{1.753849in}{0.907403in}}%
\pgfpathlineto{\pgfqpoint{1.765682in}{0.917036in}}%
\pgfpathlineto{\pgfqpoint{1.769506in}{0.920361in}}%
\pgfpathlineto{\pgfqpoint{1.780587in}{0.930648in}}%
\pgfpathlineto{\pgfqpoint{1.785162in}{0.935376in}}%
\pgfpathlineto{\pgfqpoint{1.793545in}{0.944259in}}%
\pgfpathlineto{\pgfqpoint{1.800819in}{0.953240in}}%
\pgfpathlineto{\pgfqpoint{1.804632in}{0.957870in}}%
\pgfpathlineto{\pgfqpoint{1.813802in}{0.971481in}}%
\pgfpathlineto{\pgfqpoint{1.816476in}{0.976711in}}%
\pgfpathlineto{\pgfqpoint{1.821065in}{0.985092in}}%
\pgfpathlineto{\pgfqpoint{1.826064in}{0.998703in}}%
\pgfpathlineto{\pgfqpoint{1.828560in}{1.012314in}}%
\pgfpathlineto{\pgfqpoint{1.828560in}{1.025925in}}%
\pgfpathlineto{\pgfqpoint{1.826064in}{1.039536in}}%
\pgfpathlineto{\pgfqpoint{1.821065in}{1.053148in}}%
\pgfpathlineto{\pgfqpoint{1.816476in}{1.061529in}}%
\pgfpathlineto{\pgfqpoint{1.813802in}{1.066759in}}%
\pgfpathlineto{\pgfqpoint{1.804632in}{1.080370in}}%
\pgfpathlineto{\pgfqpoint{1.800819in}{1.085000in}}%
\pgfpathlineto{\pgfqpoint{1.793545in}{1.093981in}}%
\pgfpathlineto{\pgfqpoint{1.785162in}{1.102864in}}%
\pgfpathlineto{\pgfqpoint{1.780587in}{1.107592in}}%
\pgfpathlineto{\pgfqpoint{1.769506in}{1.117879in}}%
\pgfpathlineto{\pgfqpoint{1.765682in}{1.121203in}}%
\pgfpathlineto{\pgfqpoint{1.753849in}{1.130836in}}%
\pgfpathlineto{\pgfqpoint{1.748411in}{1.134814in}}%
\pgfpathlineto{\pgfqpoint{1.738193in}{1.142101in}}%
\pgfpathlineto{\pgfqpoint{1.727862in}{1.148425in}}%
\pgfpathlineto{\pgfqpoint{1.722536in}{1.151740in}}%
\pgfpathlineto{\pgfqpoint{1.706880in}{1.159712in}}%
\pgfpathlineto{\pgfqpoint{1.700864in}{1.162036in}}%
\pgfpathlineto{\pgfqpoint{1.691223in}{1.166026in}}%
\pgfpathlineto{\pgfqpoint{1.675567in}{1.170372in}}%
\pgfpathlineto{\pgfqpoint{1.659910in}{1.172542in}}%
\pgfpathlineto{\pgfqpoint{1.644253in}{1.172542in}}%
\pgfpathlineto{\pgfqpoint{1.628597in}{1.170372in}}%
\pgfpathlineto{\pgfqpoint{1.612940in}{1.166026in}}%
\pgfpathlineto{\pgfqpoint{1.603299in}{1.162036in}}%
\pgfpathlineto{\pgfqpoint{1.597284in}{1.159712in}}%
\pgfpathlineto{\pgfqpoint{1.581627in}{1.151740in}}%
\pgfpathlineto{\pgfqpoint{1.576301in}{1.148425in}}%
\pgfpathlineto{\pgfqpoint{1.565971in}{1.142101in}}%
\pgfpathlineto{\pgfqpoint{1.555753in}{1.134814in}}%
\pgfpathlineto{\pgfqpoint{1.550314in}{1.130836in}}%
\pgfpathlineto{\pgfqpoint{1.538481in}{1.121203in}}%
\pgfpathlineto{\pgfqpoint{1.534657in}{1.117879in}}%
\pgfpathlineto{\pgfqpoint{1.523577in}{1.107592in}}%
\pgfpathlineto{\pgfqpoint{1.519001in}{1.102864in}}%
\pgfpathlineto{\pgfqpoint{1.510618in}{1.093981in}}%
\pgfpathlineto{\pgfqpoint{1.503344in}{1.085000in}}%
\pgfpathlineto{\pgfqpoint{1.499531in}{1.080370in}}%
\pgfpathlineto{\pgfqpoint{1.490361in}{1.066759in}}%
\pgfpathlineto{\pgfqpoint{1.487688in}{1.061529in}}%
\pgfpathlineto{\pgfqpoint{1.483099in}{1.053148in}}%
\pgfpathlineto{\pgfqpoint{1.478099in}{1.039536in}}%
\pgfpathlineto{\pgfqpoint{1.475603in}{1.025925in}}%
\pgfpathlineto{\pgfqpoint{1.475603in}{1.012314in}}%
\pgfpathlineto{\pgfqpoint{1.478099in}{0.998703in}}%
\pgfpathlineto{\pgfqpoint{1.483099in}{0.985092in}}%
\pgfpathlineto{\pgfqpoint{1.487688in}{0.976711in}}%
\pgfpathlineto{\pgfqpoint{1.490361in}{0.971481in}}%
\pgfpathlineto{\pgfqpoint{1.499531in}{0.957870in}}%
\pgfpathlineto{\pgfqpoint{1.503344in}{0.953240in}}%
\pgfpathlineto{\pgfqpoint{1.510618in}{0.944259in}}%
\pgfpathlineto{\pgfqpoint{1.519001in}{0.935376in}}%
\pgfpathlineto{\pgfqpoint{1.523577in}{0.930648in}}%
\pgfpathlineto{\pgfqpoint{1.534657in}{0.920361in}}%
\pgfpathlineto{\pgfqpoint{1.538481in}{0.917036in}}%
\pgfpathlineto{\pgfqpoint{1.550314in}{0.907403in}}%
\pgfpathlineto{\pgfqpoint{1.555753in}{0.903425in}}%
\pgfpathlineto{\pgfqpoint{1.565971in}{0.896138in}}%
\pgfpathlineto{\pgfqpoint{1.576301in}{0.889814in}}%
\pgfpathlineto{\pgfqpoint{1.581627in}{0.886499in}}%
\pgfpathlineto{\pgfqpoint{1.597284in}{0.878527in}}%
\pgfpathlineto{\pgfqpoint{1.603299in}{0.876203in}}%
\pgfpathlineto{\pgfqpoint{1.612940in}{0.872213in}}%
\pgfpathclose%
\pgfpathmoveto{\pgfqpoint{1.610420in}{0.903425in}}%
\pgfpathlineto{\pgfqpoint{1.597284in}{0.908649in}}%
\pgfpathlineto{\pgfqpoint{1.581627in}{0.916929in}}%
\pgfpathlineto{\pgfqpoint{1.581459in}{0.917036in}}%
\pgfpathlineto{\pgfqpoint{1.565971in}{0.927531in}}%
\pgfpathlineto{\pgfqpoint{1.562006in}{0.930648in}}%
\pgfpathlineto{\pgfqpoint{1.550314in}{0.940812in}}%
\pgfpathlineto{\pgfqpoint{1.546729in}{0.944259in}}%
\pgfpathlineto{\pgfqpoint{1.534657in}{0.957724in}}%
\pgfpathlineto{\pgfqpoint{1.534534in}{0.957870in}}%
\pgfpathlineto{\pgfqpoint{1.525010in}{0.971481in}}%
\pgfpathlineto{\pgfqpoint{1.519001in}{0.982901in}}%
\pgfpathlineto{\pgfqpoint{1.517859in}{0.985092in}}%
\pgfpathlineto{\pgfqpoint{1.513057in}{0.998703in}}%
\pgfpathlineto{\pgfqpoint{1.510659in}{1.012314in}}%
\pgfpathlineto{\pgfqpoint{1.510659in}{1.025925in}}%
\pgfpathlineto{\pgfqpoint{1.513057in}{1.039536in}}%
\pgfpathlineto{\pgfqpoint{1.517859in}{1.053148in}}%
\pgfpathlineto{\pgfqpoint{1.519001in}{1.055339in}}%
\pgfpathlineto{\pgfqpoint{1.525010in}{1.066759in}}%
\pgfpathlineto{\pgfqpoint{1.534534in}{1.080370in}}%
\pgfpathlineto{\pgfqpoint{1.534657in}{1.080516in}}%
\pgfpathlineto{\pgfqpoint{1.546729in}{1.093981in}}%
\pgfpathlineto{\pgfqpoint{1.550314in}{1.097427in}}%
\pgfpathlineto{\pgfqpoint{1.562006in}{1.107592in}}%
\pgfpathlineto{\pgfqpoint{1.565971in}{1.110709in}}%
\pgfpathlineto{\pgfqpoint{1.581459in}{1.121203in}}%
\pgfpathlineto{\pgfqpoint{1.581627in}{1.121311in}}%
\pgfpathlineto{\pgfqpoint{1.597284in}{1.129590in}}%
\pgfpathlineto{\pgfqpoint{1.610420in}{1.134814in}}%
\pgfpathlineto{\pgfqpoint{1.612940in}{1.135807in}}%
\pgfpathlineto{\pgfqpoint{1.628597in}{1.139982in}}%
\pgfpathlineto{\pgfqpoint{1.644253in}{1.142066in}}%
\pgfpathlineto{\pgfqpoint{1.659910in}{1.142066in}}%
\pgfpathlineto{\pgfqpoint{1.675567in}{1.139982in}}%
\pgfpathlineto{\pgfqpoint{1.691223in}{1.135807in}}%
\pgfpathlineto{\pgfqpoint{1.693743in}{1.134814in}}%
\pgfpathlineto{\pgfqpoint{1.706880in}{1.129590in}}%
\pgfpathlineto{\pgfqpoint{1.722536in}{1.121311in}}%
\pgfpathlineto{\pgfqpoint{1.722704in}{1.121203in}}%
\pgfpathlineto{\pgfqpoint{1.738193in}{1.110709in}}%
\pgfpathlineto{\pgfqpoint{1.742157in}{1.107592in}}%
\pgfpathlineto{\pgfqpoint{1.753849in}{1.097427in}}%
\pgfpathlineto{\pgfqpoint{1.757434in}{1.093981in}}%
\pgfpathlineto{\pgfqpoint{1.769506in}{1.080516in}}%
\pgfpathlineto{\pgfqpoint{1.769630in}{1.080370in}}%
\pgfpathlineto{\pgfqpoint{1.779154in}{1.066759in}}%
\pgfpathlineto{\pgfqpoint{1.785162in}{1.055339in}}%
\pgfpathlineto{\pgfqpoint{1.786305in}{1.053148in}}%
\pgfpathlineto{\pgfqpoint{1.791107in}{1.039536in}}%
\pgfpathlineto{\pgfqpoint{1.793504in}{1.025925in}}%
\pgfpathlineto{\pgfqpoint{1.793504in}{1.012314in}}%
\pgfpathlineto{\pgfqpoint{1.791107in}{0.998703in}}%
\pgfpathlineto{\pgfqpoint{1.786305in}{0.985092in}}%
\pgfpathlineto{\pgfqpoint{1.785162in}{0.982901in}}%
\pgfpathlineto{\pgfqpoint{1.779154in}{0.971481in}}%
\pgfpathlineto{\pgfqpoint{1.769630in}{0.957870in}}%
\pgfpathlineto{\pgfqpoint{1.769506in}{0.957724in}}%
\pgfpathlineto{\pgfqpoint{1.757434in}{0.944259in}}%
\pgfpathlineto{\pgfqpoint{1.753849in}{0.940812in}}%
\pgfpathlineto{\pgfqpoint{1.742157in}{0.930648in}}%
\pgfpathlineto{\pgfqpoint{1.738193in}{0.927531in}}%
\pgfpathlineto{\pgfqpoint{1.722704in}{0.917036in}}%
\pgfpathlineto{\pgfqpoint{1.722536in}{0.916929in}}%
\pgfpathlineto{\pgfqpoint{1.706880in}{0.908649in}}%
\pgfpathlineto{\pgfqpoint{1.693743in}{0.903425in}}%
\pgfpathlineto{\pgfqpoint{1.691223in}{0.902432in}}%
\pgfpathlineto{\pgfqpoint{1.675567in}{0.898258in}}%
\pgfpathlineto{\pgfqpoint{1.659910in}{0.896173in}}%
\pgfpathlineto{\pgfqpoint{1.644253in}{0.896173in}}%
\pgfpathlineto{\pgfqpoint{1.628597in}{0.898258in}}%
\pgfpathlineto{\pgfqpoint{1.612940in}{0.902432in}}%
\pgfpathlineto{\pgfqpoint{1.610420in}{0.903425in}}%
\pgfpathclose%
\pgfpathmoveto{\pgfqpoint{0.579607in}{1.321380in}}%
\pgfpathlineto{\pgfqpoint{0.595263in}{1.317034in}}%
\pgfpathlineto{\pgfqpoint{0.610920in}{1.314864in}}%
\pgfpathlineto{\pgfqpoint{0.626577in}{1.314864in}}%
\pgfpathlineto{\pgfqpoint{0.642233in}{1.317034in}}%
\pgfpathlineto{\pgfqpoint{0.657890in}{1.321380in}}%
\pgfpathlineto{\pgfqpoint{0.667531in}{1.325370in}}%
\pgfpathlineto{\pgfqpoint{0.673546in}{1.327694in}}%
\pgfpathlineto{\pgfqpoint{0.689203in}{1.335666in}}%
\pgfpathlineto{\pgfqpoint{0.694529in}{1.338981in}}%
\pgfpathlineto{\pgfqpoint{0.704859in}{1.345305in}}%
\pgfpathlineto{\pgfqpoint{0.715077in}{1.352592in}}%
\pgfpathlineto{\pgfqpoint{0.720516in}{1.356570in}}%
\pgfpathlineto{\pgfqpoint{0.732349in}{1.366203in}}%
\pgfpathlineto{\pgfqpoint{0.736173in}{1.369527in}}%
\pgfpathlineto{\pgfqpoint{0.747253in}{1.379814in}}%
\pgfpathlineto{\pgfqpoint{0.751829in}{1.384542in}}%
\pgfpathlineto{\pgfqpoint{0.760212in}{1.393425in}}%
\pgfpathlineto{\pgfqpoint{0.767486in}{1.402407in}}%
\pgfpathlineto{\pgfqpoint{0.771299in}{1.407036in}}%
\pgfpathlineto{\pgfqpoint{0.780469in}{1.420648in}}%
\pgfpathlineto{\pgfqpoint{0.783142in}{1.425877in}}%
\pgfpathlineto{\pgfqpoint{0.787731in}{1.434259in}}%
\pgfpathlineto{\pgfqpoint{0.792731in}{1.447870in}}%
\pgfpathlineto{\pgfqpoint{0.795227in}{1.461481in}}%
\pgfpathlineto{\pgfqpoint{0.795227in}{1.475092in}}%
\pgfpathlineto{\pgfqpoint{0.792731in}{1.488703in}}%
\pgfpathlineto{\pgfqpoint{0.787731in}{1.502314in}}%
\pgfpathlineto{\pgfqpoint{0.783142in}{1.510695in}}%
\pgfpathlineto{\pgfqpoint{0.780469in}{1.515925in}}%
\pgfpathlineto{\pgfqpoint{0.771299in}{1.529536in}}%
\pgfpathlineto{\pgfqpoint{0.767486in}{1.534166in}}%
\pgfpathlineto{\pgfqpoint{0.760212in}{1.543148in}}%
\pgfpathlineto{\pgfqpoint{0.751829in}{1.552031in}}%
\pgfpathlineto{\pgfqpoint{0.747253in}{1.556759in}}%
\pgfpathlineto{\pgfqpoint{0.736173in}{1.567045in}}%
\pgfpathlineto{\pgfqpoint{0.732349in}{1.570370in}}%
\pgfpathlineto{\pgfqpoint{0.720516in}{1.580003in}}%
\pgfpathlineto{\pgfqpoint{0.715077in}{1.583981in}}%
\pgfpathlineto{\pgfqpoint{0.704859in}{1.591268in}}%
\pgfpathlineto{\pgfqpoint{0.694529in}{1.597592in}}%
\pgfpathlineto{\pgfqpoint{0.689203in}{1.600907in}}%
\pgfpathlineto{\pgfqpoint{0.673546in}{1.608879in}}%
\pgfpathlineto{\pgfqpoint{0.667531in}{1.611203in}}%
\pgfpathlineto{\pgfqpoint{0.657890in}{1.615193in}}%
\pgfpathlineto{\pgfqpoint{0.642233in}{1.619539in}}%
\pgfpathlineto{\pgfqpoint{0.626577in}{1.621709in}}%
\pgfpathlineto{\pgfqpoint{0.610920in}{1.621709in}}%
\pgfpathlineto{\pgfqpoint{0.595263in}{1.619539in}}%
\pgfpathlineto{\pgfqpoint{0.579607in}{1.615193in}}%
\pgfpathlineto{\pgfqpoint{0.569966in}{1.611203in}}%
\pgfpathlineto{\pgfqpoint{0.563950in}{1.608879in}}%
\pgfpathlineto{\pgfqpoint{0.548294in}{1.600907in}}%
\pgfpathlineto{\pgfqpoint{0.542968in}{1.597592in}}%
\pgfpathlineto{\pgfqpoint{0.532637in}{1.591268in}}%
\pgfpathlineto{\pgfqpoint{0.522419in}{1.583981in}}%
\pgfpathlineto{\pgfqpoint{0.516981in}{1.580003in}}%
\pgfpathlineto{\pgfqpoint{0.505148in}{1.570370in}}%
\pgfpathlineto{\pgfqpoint{0.501324in}{1.567045in}}%
\pgfpathlineto{\pgfqpoint{0.490243in}{1.556759in}}%
\pgfpathlineto{\pgfqpoint{0.485668in}{1.552031in}}%
\pgfpathlineto{\pgfqpoint{0.477285in}{1.543148in}}%
\pgfpathlineto{\pgfqpoint{0.470011in}{1.534166in}}%
\pgfpathlineto{\pgfqpoint{0.466198in}{1.529536in}}%
\pgfpathlineto{\pgfqpoint{0.457028in}{1.515925in}}%
\pgfpathlineto{\pgfqpoint{0.454354in}{1.510695in}}%
\pgfpathlineto{\pgfqpoint{0.449765in}{1.502314in}}%
\pgfpathlineto{\pgfqpoint{0.444766in}{1.488703in}}%
\pgfpathlineto{\pgfqpoint{0.442270in}{1.475092in}}%
\pgfpathlineto{\pgfqpoint{0.442270in}{1.461481in}}%
\pgfpathlineto{\pgfqpoint{0.444766in}{1.447870in}}%
\pgfpathlineto{\pgfqpoint{0.449765in}{1.434259in}}%
\pgfpathlineto{\pgfqpoint{0.454354in}{1.425877in}}%
\pgfpathlineto{\pgfqpoint{0.457028in}{1.420648in}}%
\pgfpathlineto{\pgfqpoint{0.466198in}{1.407036in}}%
\pgfpathlineto{\pgfqpoint{0.470011in}{1.402407in}}%
\pgfpathlineto{\pgfqpoint{0.477285in}{1.393425in}}%
\pgfpathlineto{\pgfqpoint{0.485668in}{1.384542in}}%
\pgfpathlineto{\pgfqpoint{0.490243in}{1.379814in}}%
\pgfpathlineto{\pgfqpoint{0.501324in}{1.369527in}}%
\pgfpathlineto{\pgfqpoint{0.505148in}{1.366203in}}%
\pgfpathlineto{\pgfqpoint{0.516981in}{1.356570in}}%
\pgfpathlineto{\pgfqpoint{0.522419in}{1.352592in}}%
\pgfpathlineto{\pgfqpoint{0.532637in}{1.345305in}}%
\pgfpathlineto{\pgfqpoint{0.542968in}{1.338981in}}%
\pgfpathlineto{\pgfqpoint{0.548294in}{1.335666in}}%
\pgfpathlineto{\pgfqpoint{0.563950in}{1.327694in}}%
\pgfpathlineto{\pgfqpoint{0.569966in}{1.325370in}}%
\pgfpathlineto{\pgfqpoint{0.579607in}{1.321380in}}%
\pgfpathclose%
\pgfpathmoveto{\pgfqpoint{0.577087in}{1.352592in}}%
\pgfpathlineto{\pgfqpoint{0.563950in}{1.357816in}}%
\pgfpathlineto{\pgfqpoint{0.548294in}{1.366095in}}%
\pgfpathlineto{\pgfqpoint{0.548126in}{1.366203in}}%
\pgfpathlineto{\pgfqpoint{0.532637in}{1.376697in}}%
\pgfpathlineto{\pgfqpoint{0.528673in}{1.379814in}}%
\pgfpathlineto{\pgfqpoint{0.516981in}{1.389979in}}%
\pgfpathlineto{\pgfqpoint{0.513396in}{1.393425in}}%
\pgfpathlineto{\pgfqpoint{0.501324in}{1.406890in}}%
\pgfpathlineto{\pgfqpoint{0.501200in}{1.407036in}}%
\pgfpathlineto{\pgfqpoint{0.491676in}{1.420648in}}%
\pgfpathlineto{\pgfqpoint{0.485668in}{1.432067in}}%
\pgfpathlineto{\pgfqpoint{0.484525in}{1.434259in}}%
\pgfpathlineto{\pgfqpoint{0.479723in}{1.447870in}}%
\pgfpathlineto{\pgfqpoint{0.477326in}{1.461481in}}%
\pgfpathlineto{\pgfqpoint{0.477326in}{1.475092in}}%
\pgfpathlineto{\pgfqpoint{0.479723in}{1.488703in}}%
\pgfpathlineto{\pgfqpoint{0.484525in}{1.502314in}}%
\pgfpathlineto{\pgfqpoint{0.485668in}{1.504505in}}%
\pgfpathlineto{\pgfqpoint{0.491676in}{1.515925in}}%
\pgfpathlineto{\pgfqpoint{0.501200in}{1.529536in}}%
\pgfpathlineto{\pgfqpoint{0.501324in}{1.529683in}}%
\pgfpathlineto{\pgfqpoint{0.513396in}{1.543148in}}%
\pgfpathlineto{\pgfqpoint{0.516981in}{1.546594in}}%
\pgfpathlineto{\pgfqpoint{0.528673in}{1.556759in}}%
\pgfpathlineto{\pgfqpoint{0.532637in}{1.559875in}}%
\pgfpathlineto{\pgfqpoint{0.548126in}{1.570370in}}%
\pgfpathlineto{\pgfqpoint{0.548294in}{1.570477in}}%
\pgfpathlineto{\pgfqpoint{0.563950in}{1.578757in}}%
\pgfpathlineto{\pgfqpoint{0.577087in}{1.583981in}}%
\pgfpathlineto{\pgfqpoint{0.579607in}{1.584974in}}%
\pgfpathlineto{\pgfqpoint{0.595263in}{1.589148in}}%
\pgfpathlineto{\pgfqpoint{0.610920in}{1.591233in}}%
\pgfpathlineto{\pgfqpoint{0.626577in}{1.591233in}}%
\pgfpathlineto{\pgfqpoint{0.642233in}{1.589148in}}%
\pgfpathlineto{\pgfqpoint{0.657890in}{1.584974in}}%
\pgfpathlineto{\pgfqpoint{0.660410in}{1.583981in}}%
\pgfpathlineto{\pgfqpoint{0.673546in}{1.578757in}}%
\pgfpathlineto{\pgfqpoint{0.689203in}{1.570477in}}%
\pgfpathlineto{\pgfqpoint{0.689371in}{1.570370in}}%
\pgfpathlineto{\pgfqpoint{0.704859in}{1.559875in}}%
\pgfpathlineto{\pgfqpoint{0.708824in}{1.556759in}}%
\pgfpathlineto{\pgfqpoint{0.720516in}{1.546594in}}%
\pgfpathlineto{\pgfqpoint{0.724101in}{1.543148in}}%
\pgfpathlineto{\pgfqpoint{0.736173in}{1.529683in}}%
\pgfpathlineto{\pgfqpoint{0.736296in}{1.529536in}}%
\pgfpathlineto{\pgfqpoint{0.745820in}{1.515925in}}%
\pgfpathlineto{\pgfqpoint{0.751829in}{1.504505in}}%
\pgfpathlineto{\pgfqpoint{0.752971in}{1.502314in}}%
\pgfpathlineto{\pgfqpoint{0.757773in}{1.488703in}}%
\pgfpathlineto{\pgfqpoint{0.760171in}{1.475092in}}%
\pgfpathlineto{\pgfqpoint{0.760171in}{1.461481in}}%
\pgfpathlineto{\pgfqpoint{0.757773in}{1.447870in}}%
\pgfpathlineto{\pgfqpoint{0.752971in}{1.434259in}}%
\pgfpathlineto{\pgfqpoint{0.751829in}{1.432067in}}%
\pgfpathlineto{\pgfqpoint{0.745820in}{1.420648in}}%
\pgfpathlineto{\pgfqpoint{0.736296in}{1.407036in}}%
\pgfpathlineto{\pgfqpoint{0.736173in}{1.406890in}}%
\pgfpathlineto{\pgfqpoint{0.724101in}{1.393425in}}%
\pgfpathlineto{\pgfqpoint{0.720516in}{1.389979in}}%
\pgfpathlineto{\pgfqpoint{0.708824in}{1.379814in}}%
\pgfpathlineto{\pgfqpoint{0.704859in}{1.376697in}}%
\pgfpathlineto{\pgfqpoint{0.689371in}{1.366203in}}%
\pgfpathlineto{\pgfqpoint{0.689203in}{1.366095in}}%
\pgfpathlineto{\pgfqpoint{0.673546in}{1.357816in}}%
\pgfpathlineto{\pgfqpoint{0.660410in}{1.352592in}}%
\pgfpathlineto{\pgfqpoint{0.657890in}{1.351599in}}%
\pgfpathlineto{\pgfqpoint{0.642233in}{1.347424in}}%
\pgfpathlineto{\pgfqpoint{0.626577in}{1.345340in}}%
\pgfpathlineto{\pgfqpoint{0.610920in}{1.345340in}}%
\pgfpathlineto{\pgfqpoint{0.595263in}{1.347424in}}%
\pgfpathlineto{\pgfqpoint{0.579607in}{1.351599in}}%
\pgfpathlineto{\pgfqpoint{0.577087in}{1.352592in}}%
\pgfpathclose%
\pgfpathmoveto{\pgfqpoint{1.096274in}{1.321380in}}%
\pgfpathlineto{\pgfqpoint{1.111930in}{1.317034in}}%
\pgfpathlineto{\pgfqpoint{1.127587in}{1.314864in}}%
\pgfpathlineto{\pgfqpoint{1.143243in}{1.314864in}}%
\pgfpathlineto{\pgfqpoint{1.158900in}{1.317034in}}%
\pgfpathlineto{\pgfqpoint{1.174556in}{1.321380in}}%
\pgfpathlineto{\pgfqpoint{1.184197in}{1.325370in}}%
\pgfpathlineto{\pgfqpoint{1.190213in}{1.327694in}}%
\pgfpathlineto{\pgfqpoint{1.205870in}{1.335666in}}%
\pgfpathlineto{\pgfqpoint{1.211195in}{1.338981in}}%
\pgfpathlineto{\pgfqpoint{1.221526in}{1.345305in}}%
\pgfpathlineto{\pgfqpoint{1.231744in}{1.352592in}}%
\pgfpathlineto{\pgfqpoint{1.237183in}{1.356570in}}%
\pgfpathlineto{\pgfqpoint{1.249015in}{1.366203in}}%
\pgfpathlineto{\pgfqpoint{1.252839in}{1.369527in}}%
\pgfpathlineto{\pgfqpoint{1.263920in}{1.379814in}}%
\pgfpathlineto{\pgfqpoint{1.268496in}{1.384542in}}%
\pgfpathlineto{\pgfqpoint{1.276878in}{1.393425in}}%
\pgfpathlineto{\pgfqpoint{1.284152in}{1.402407in}}%
\pgfpathlineto{\pgfqpoint{1.287965in}{1.407036in}}%
\pgfpathlineto{\pgfqpoint{1.297136in}{1.420648in}}%
\pgfpathlineto{\pgfqpoint{1.299809in}{1.425877in}}%
\pgfpathlineto{\pgfqpoint{1.304398in}{1.434259in}}%
\pgfpathlineto{\pgfqpoint{1.309397in}{1.447870in}}%
\pgfpathlineto{\pgfqpoint{1.311894in}{1.461481in}}%
\pgfpathlineto{\pgfqpoint{1.311894in}{1.475092in}}%
\pgfpathlineto{\pgfqpoint{1.309397in}{1.488703in}}%
\pgfpathlineto{\pgfqpoint{1.304398in}{1.502314in}}%
\pgfpathlineto{\pgfqpoint{1.299809in}{1.510695in}}%
\pgfpathlineto{\pgfqpoint{1.297136in}{1.515925in}}%
\pgfpathlineto{\pgfqpoint{1.287965in}{1.529536in}}%
\pgfpathlineto{\pgfqpoint{1.284152in}{1.534166in}}%
\pgfpathlineto{\pgfqpoint{1.276878in}{1.543148in}}%
\pgfpathlineto{\pgfqpoint{1.268496in}{1.552031in}}%
\pgfpathlineto{\pgfqpoint{1.263920in}{1.556759in}}%
\pgfpathlineto{\pgfqpoint{1.252839in}{1.567045in}}%
\pgfpathlineto{\pgfqpoint{1.249015in}{1.570370in}}%
\pgfpathlineto{\pgfqpoint{1.237183in}{1.580003in}}%
\pgfpathlineto{\pgfqpoint{1.231744in}{1.583981in}}%
\pgfpathlineto{\pgfqpoint{1.221526in}{1.591268in}}%
\pgfpathlineto{\pgfqpoint{1.211195in}{1.597592in}}%
\pgfpathlineto{\pgfqpoint{1.205870in}{1.600907in}}%
\pgfpathlineto{\pgfqpoint{1.190213in}{1.608879in}}%
\pgfpathlineto{\pgfqpoint{1.184197in}{1.611203in}}%
\pgfpathlineto{\pgfqpoint{1.174556in}{1.615193in}}%
\pgfpathlineto{\pgfqpoint{1.158900in}{1.619539in}}%
\pgfpathlineto{\pgfqpoint{1.143243in}{1.621709in}}%
\pgfpathlineto{\pgfqpoint{1.127587in}{1.621709in}}%
\pgfpathlineto{\pgfqpoint{1.111930in}{1.619539in}}%
\pgfpathlineto{\pgfqpoint{1.096274in}{1.615193in}}%
\pgfpathlineto{\pgfqpoint{1.086633in}{1.611203in}}%
\pgfpathlineto{\pgfqpoint{1.080617in}{1.608879in}}%
\pgfpathlineto{\pgfqpoint{1.064960in}{1.600907in}}%
\pgfpathlineto{\pgfqpoint{1.059635in}{1.597592in}}%
\pgfpathlineto{\pgfqpoint{1.049304in}{1.591268in}}%
\pgfpathlineto{\pgfqpoint{1.039086in}{1.583981in}}%
\pgfpathlineto{\pgfqpoint{1.033647in}{1.580003in}}%
\pgfpathlineto{\pgfqpoint{1.021815in}{1.570370in}}%
\pgfpathlineto{\pgfqpoint{1.017991in}{1.567045in}}%
\pgfpathlineto{\pgfqpoint{1.006910in}{1.556759in}}%
\pgfpathlineto{\pgfqpoint{1.002334in}{1.552031in}}%
\pgfpathlineto{\pgfqpoint{0.993952in}{1.543148in}}%
\pgfpathlineto{\pgfqpoint{0.986678in}{1.534166in}}%
\pgfpathlineto{\pgfqpoint{0.982865in}{1.529536in}}%
\pgfpathlineto{\pgfqpoint{0.973694in}{1.515925in}}%
\pgfpathlineto{\pgfqpoint{0.971021in}{1.510695in}}%
\pgfpathlineto{\pgfqpoint{0.966432in}{1.502314in}}%
\pgfpathlineto{\pgfqpoint{0.961433in}{1.488703in}}%
\pgfpathlineto{\pgfqpoint{0.958936in}{1.475092in}}%
\pgfpathlineto{\pgfqpoint{0.958936in}{1.461481in}}%
\pgfpathlineto{\pgfqpoint{0.961433in}{1.447870in}}%
\pgfpathlineto{\pgfqpoint{0.966432in}{1.434259in}}%
\pgfpathlineto{\pgfqpoint{0.971021in}{1.425877in}}%
\pgfpathlineto{\pgfqpoint{0.973694in}{1.420648in}}%
\pgfpathlineto{\pgfqpoint{0.982865in}{1.407036in}}%
\pgfpathlineto{\pgfqpoint{0.986678in}{1.402407in}}%
\pgfpathlineto{\pgfqpoint{0.993952in}{1.393425in}}%
\pgfpathlineto{\pgfqpoint{1.002334in}{1.384542in}}%
\pgfpathlineto{\pgfqpoint{1.006910in}{1.379814in}}%
\pgfpathlineto{\pgfqpoint{1.017991in}{1.369527in}}%
\pgfpathlineto{\pgfqpoint{1.021815in}{1.366203in}}%
\pgfpathlineto{\pgfqpoint{1.033647in}{1.356570in}}%
\pgfpathlineto{\pgfqpoint{1.039086in}{1.352592in}}%
\pgfpathlineto{\pgfqpoint{1.049304in}{1.345305in}}%
\pgfpathlineto{\pgfqpoint{1.059635in}{1.338981in}}%
\pgfpathlineto{\pgfqpoint{1.064960in}{1.335666in}}%
\pgfpathlineto{\pgfqpoint{1.080617in}{1.327694in}}%
\pgfpathlineto{\pgfqpoint{1.086633in}{1.325370in}}%
\pgfpathlineto{\pgfqpoint{1.096274in}{1.321380in}}%
\pgfpathclose%
\pgfpathmoveto{\pgfqpoint{1.093753in}{1.352592in}}%
\pgfpathlineto{\pgfqpoint{1.080617in}{1.357816in}}%
\pgfpathlineto{\pgfqpoint{1.064960in}{1.366095in}}%
\pgfpathlineto{\pgfqpoint{1.064792in}{1.366203in}}%
\pgfpathlineto{\pgfqpoint{1.049304in}{1.376697in}}%
\pgfpathlineto{\pgfqpoint{1.045340in}{1.379814in}}%
\pgfpathlineto{\pgfqpoint{1.033647in}{1.389979in}}%
\pgfpathlineto{\pgfqpoint{1.030062in}{1.393425in}}%
\pgfpathlineto{\pgfqpoint{1.017991in}{1.406890in}}%
\pgfpathlineto{\pgfqpoint{1.017867in}{1.407036in}}%
\pgfpathlineto{\pgfqpoint{1.008343in}{1.420648in}}%
\pgfpathlineto{\pgfqpoint{1.002334in}{1.432067in}}%
\pgfpathlineto{\pgfqpoint{1.001192in}{1.434259in}}%
\pgfpathlineto{\pgfqpoint{0.996390in}{1.447870in}}%
\pgfpathlineto{\pgfqpoint{0.993992in}{1.461481in}}%
\pgfpathlineto{\pgfqpoint{0.993992in}{1.475092in}}%
\pgfpathlineto{\pgfqpoint{0.996390in}{1.488703in}}%
\pgfpathlineto{\pgfqpoint{1.001192in}{1.502314in}}%
\pgfpathlineto{\pgfqpoint{1.002334in}{1.504505in}}%
\pgfpathlineto{\pgfqpoint{1.008343in}{1.515925in}}%
\pgfpathlineto{\pgfqpoint{1.017867in}{1.529536in}}%
\pgfpathlineto{\pgfqpoint{1.017991in}{1.529683in}}%
\pgfpathlineto{\pgfqpoint{1.030062in}{1.543148in}}%
\pgfpathlineto{\pgfqpoint{1.033647in}{1.546594in}}%
\pgfpathlineto{\pgfqpoint{1.045340in}{1.556759in}}%
\pgfpathlineto{\pgfqpoint{1.049304in}{1.559875in}}%
\pgfpathlineto{\pgfqpoint{1.064792in}{1.570370in}}%
\pgfpathlineto{\pgfqpoint{1.064960in}{1.570477in}}%
\pgfpathlineto{\pgfqpoint{1.080617in}{1.578757in}}%
\pgfpathlineto{\pgfqpoint{1.093753in}{1.583981in}}%
\pgfpathlineto{\pgfqpoint{1.096274in}{1.584974in}}%
\pgfpathlineto{\pgfqpoint{1.111930in}{1.589148in}}%
\pgfpathlineto{\pgfqpoint{1.127587in}{1.591233in}}%
\pgfpathlineto{\pgfqpoint{1.143243in}{1.591233in}}%
\pgfpathlineto{\pgfqpoint{1.158900in}{1.589148in}}%
\pgfpathlineto{\pgfqpoint{1.174556in}{1.584974in}}%
\pgfpathlineto{\pgfqpoint{1.177077in}{1.583981in}}%
\pgfpathlineto{\pgfqpoint{1.190213in}{1.578757in}}%
\pgfpathlineto{\pgfqpoint{1.205870in}{1.570477in}}%
\pgfpathlineto{\pgfqpoint{1.206038in}{1.570370in}}%
\pgfpathlineto{\pgfqpoint{1.221526in}{1.559875in}}%
\pgfpathlineto{\pgfqpoint{1.225490in}{1.556759in}}%
\pgfpathlineto{\pgfqpoint{1.237183in}{1.546594in}}%
\pgfpathlineto{\pgfqpoint{1.240768in}{1.543148in}}%
\pgfpathlineto{\pgfqpoint{1.252839in}{1.529683in}}%
\pgfpathlineto{\pgfqpoint{1.252963in}{1.529536in}}%
\pgfpathlineto{\pgfqpoint{1.262487in}{1.515925in}}%
\pgfpathlineto{\pgfqpoint{1.268496in}{1.504505in}}%
\pgfpathlineto{\pgfqpoint{1.269638in}{1.502314in}}%
\pgfpathlineto{\pgfqpoint{1.274440in}{1.488703in}}%
\pgfpathlineto{\pgfqpoint{1.276838in}{1.475092in}}%
\pgfpathlineto{\pgfqpoint{1.276838in}{1.461481in}}%
\pgfpathlineto{\pgfqpoint{1.274440in}{1.447870in}}%
\pgfpathlineto{\pgfqpoint{1.269638in}{1.434259in}}%
\pgfpathlineto{\pgfqpoint{1.268496in}{1.432067in}}%
\pgfpathlineto{\pgfqpoint{1.262487in}{1.420648in}}%
\pgfpathlineto{\pgfqpoint{1.252963in}{1.407036in}}%
\pgfpathlineto{\pgfqpoint{1.252839in}{1.406890in}}%
\pgfpathlineto{\pgfqpoint{1.240768in}{1.393425in}}%
\pgfpathlineto{\pgfqpoint{1.237183in}{1.389979in}}%
\pgfpathlineto{\pgfqpoint{1.225490in}{1.379814in}}%
\pgfpathlineto{\pgfqpoint{1.221526in}{1.376697in}}%
\pgfpathlineto{\pgfqpoint{1.206038in}{1.366203in}}%
\pgfpathlineto{\pgfqpoint{1.205870in}{1.366095in}}%
\pgfpathlineto{\pgfqpoint{1.190213in}{1.357816in}}%
\pgfpathlineto{\pgfqpoint{1.177077in}{1.352592in}}%
\pgfpathlineto{\pgfqpoint{1.174556in}{1.351599in}}%
\pgfpathlineto{\pgfqpoint{1.158900in}{1.347424in}}%
\pgfpathlineto{\pgfqpoint{1.143243in}{1.345340in}}%
\pgfpathlineto{\pgfqpoint{1.127587in}{1.345340in}}%
\pgfpathlineto{\pgfqpoint{1.111930in}{1.347424in}}%
\pgfpathlineto{\pgfqpoint{1.096274in}{1.351599in}}%
\pgfpathlineto{\pgfqpoint{1.093753in}{1.352592in}}%
\pgfpathclose%
\pgfpathmoveto{\pgfqpoint{1.612940in}{1.321380in}}%
\pgfpathlineto{\pgfqpoint{1.628597in}{1.317034in}}%
\pgfpathlineto{\pgfqpoint{1.644253in}{1.314864in}}%
\pgfpathlineto{\pgfqpoint{1.659910in}{1.314864in}}%
\pgfpathlineto{\pgfqpoint{1.675567in}{1.317034in}}%
\pgfpathlineto{\pgfqpoint{1.691223in}{1.321380in}}%
\pgfpathlineto{\pgfqpoint{1.700864in}{1.325370in}}%
\pgfpathlineto{\pgfqpoint{1.706880in}{1.327694in}}%
\pgfpathlineto{\pgfqpoint{1.722536in}{1.335666in}}%
\pgfpathlineto{\pgfqpoint{1.727862in}{1.338981in}}%
\pgfpathlineto{\pgfqpoint{1.738193in}{1.345305in}}%
\pgfpathlineto{\pgfqpoint{1.748411in}{1.352592in}}%
\pgfpathlineto{\pgfqpoint{1.753849in}{1.356570in}}%
\pgfpathlineto{\pgfqpoint{1.765682in}{1.366203in}}%
\pgfpathlineto{\pgfqpoint{1.769506in}{1.369527in}}%
\pgfpathlineto{\pgfqpoint{1.780587in}{1.379814in}}%
\pgfpathlineto{\pgfqpoint{1.785162in}{1.384542in}}%
\pgfpathlineto{\pgfqpoint{1.793545in}{1.393425in}}%
\pgfpathlineto{\pgfqpoint{1.800819in}{1.402407in}}%
\pgfpathlineto{\pgfqpoint{1.804632in}{1.407036in}}%
\pgfpathlineto{\pgfqpoint{1.813802in}{1.420648in}}%
\pgfpathlineto{\pgfqpoint{1.816476in}{1.425877in}}%
\pgfpathlineto{\pgfqpoint{1.821065in}{1.434259in}}%
\pgfpathlineto{\pgfqpoint{1.826064in}{1.447870in}}%
\pgfpathlineto{\pgfqpoint{1.828560in}{1.461481in}}%
\pgfpathlineto{\pgfqpoint{1.828560in}{1.475092in}}%
\pgfpathlineto{\pgfqpoint{1.826064in}{1.488703in}}%
\pgfpathlineto{\pgfqpoint{1.821065in}{1.502314in}}%
\pgfpathlineto{\pgfqpoint{1.816476in}{1.510695in}}%
\pgfpathlineto{\pgfqpoint{1.813802in}{1.515925in}}%
\pgfpathlineto{\pgfqpoint{1.804632in}{1.529536in}}%
\pgfpathlineto{\pgfqpoint{1.800819in}{1.534166in}}%
\pgfpathlineto{\pgfqpoint{1.793545in}{1.543148in}}%
\pgfpathlineto{\pgfqpoint{1.785162in}{1.552031in}}%
\pgfpathlineto{\pgfqpoint{1.780587in}{1.556759in}}%
\pgfpathlineto{\pgfqpoint{1.769506in}{1.567045in}}%
\pgfpathlineto{\pgfqpoint{1.765682in}{1.570370in}}%
\pgfpathlineto{\pgfqpoint{1.753849in}{1.580003in}}%
\pgfpathlineto{\pgfqpoint{1.748411in}{1.583981in}}%
\pgfpathlineto{\pgfqpoint{1.738193in}{1.591268in}}%
\pgfpathlineto{\pgfqpoint{1.727862in}{1.597592in}}%
\pgfpathlineto{\pgfqpoint{1.722536in}{1.600907in}}%
\pgfpathlineto{\pgfqpoint{1.706880in}{1.608879in}}%
\pgfpathlineto{\pgfqpoint{1.700864in}{1.611203in}}%
\pgfpathlineto{\pgfqpoint{1.691223in}{1.615193in}}%
\pgfpathlineto{\pgfqpoint{1.675567in}{1.619539in}}%
\pgfpathlineto{\pgfqpoint{1.659910in}{1.621709in}}%
\pgfpathlineto{\pgfqpoint{1.644253in}{1.621709in}}%
\pgfpathlineto{\pgfqpoint{1.628597in}{1.619539in}}%
\pgfpathlineto{\pgfqpoint{1.612940in}{1.615193in}}%
\pgfpathlineto{\pgfqpoint{1.603299in}{1.611203in}}%
\pgfpathlineto{\pgfqpoint{1.597284in}{1.608879in}}%
\pgfpathlineto{\pgfqpoint{1.581627in}{1.600907in}}%
\pgfpathlineto{\pgfqpoint{1.576301in}{1.597592in}}%
\pgfpathlineto{\pgfqpoint{1.565971in}{1.591268in}}%
\pgfpathlineto{\pgfqpoint{1.555753in}{1.583981in}}%
\pgfpathlineto{\pgfqpoint{1.550314in}{1.580003in}}%
\pgfpathlineto{\pgfqpoint{1.538481in}{1.570370in}}%
\pgfpathlineto{\pgfqpoint{1.534657in}{1.567045in}}%
\pgfpathlineto{\pgfqpoint{1.523577in}{1.556759in}}%
\pgfpathlineto{\pgfqpoint{1.519001in}{1.552031in}}%
\pgfpathlineto{\pgfqpoint{1.510618in}{1.543148in}}%
\pgfpathlineto{\pgfqpoint{1.503344in}{1.534166in}}%
\pgfpathlineto{\pgfqpoint{1.499531in}{1.529536in}}%
\pgfpathlineto{\pgfqpoint{1.490361in}{1.515925in}}%
\pgfpathlineto{\pgfqpoint{1.487688in}{1.510695in}}%
\pgfpathlineto{\pgfqpoint{1.483099in}{1.502314in}}%
\pgfpathlineto{\pgfqpoint{1.478099in}{1.488703in}}%
\pgfpathlineto{\pgfqpoint{1.475603in}{1.475092in}}%
\pgfpathlineto{\pgfqpoint{1.475603in}{1.461481in}}%
\pgfpathlineto{\pgfqpoint{1.478099in}{1.447870in}}%
\pgfpathlineto{\pgfqpoint{1.483099in}{1.434259in}}%
\pgfpathlineto{\pgfqpoint{1.487688in}{1.425877in}}%
\pgfpathlineto{\pgfqpoint{1.490361in}{1.420648in}}%
\pgfpathlineto{\pgfqpoint{1.499531in}{1.407036in}}%
\pgfpathlineto{\pgfqpoint{1.503344in}{1.402407in}}%
\pgfpathlineto{\pgfqpoint{1.510618in}{1.393425in}}%
\pgfpathlineto{\pgfqpoint{1.519001in}{1.384542in}}%
\pgfpathlineto{\pgfqpoint{1.523577in}{1.379814in}}%
\pgfpathlineto{\pgfqpoint{1.534657in}{1.369527in}}%
\pgfpathlineto{\pgfqpoint{1.538481in}{1.366203in}}%
\pgfpathlineto{\pgfqpoint{1.550314in}{1.356570in}}%
\pgfpathlineto{\pgfqpoint{1.555753in}{1.352592in}}%
\pgfpathlineto{\pgfqpoint{1.565971in}{1.345305in}}%
\pgfpathlineto{\pgfqpoint{1.576301in}{1.338981in}}%
\pgfpathlineto{\pgfqpoint{1.581627in}{1.335666in}}%
\pgfpathlineto{\pgfqpoint{1.597284in}{1.327694in}}%
\pgfpathlineto{\pgfqpoint{1.603299in}{1.325370in}}%
\pgfpathlineto{\pgfqpoint{1.612940in}{1.321380in}}%
\pgfpathclose%
\pgfpathmoveto{\pgfqpoint{1.610420in}{1.352592in}}%
\pgfpathlineto{\pgfqpoint{1.597284in}{1.357816in}}%
\pgfpathlineto{\pgfqpoint{1.581627in}{1.366095in}}%
\pgfpathlineto{\pgfqpoint{1.581459in}{1.366203in}}%
\pgfpathlineto{\pgfqpoint{1.565971in}{1.376697in}}%
\pgfpathlineto{\pgfqpoint{1.562006in}{1.379814in}}%
\pgfpathlineto{\pgfqpoint{1.550314in}{1.389979in}}%
\pgfpathlineto{\pgfqpoint{1.546729in}{1.393425in}}%
\pgfpathlineto{\pgfqpoint{1.534657in}{1.406890in}}%
\pgfpathlineto{\pgfqpoint{1.534534in}{1.407036in}}%
\pgfpathlineto{\pgfqpoint{1.525010in}{1.420648in}}%
\pgfpathlineto{\pgfqpoint{1.519001in}{1.432067in}}%
\pgfpathlineto{\pgfqpoint{1.517859in}{1.434259in}}%
\pgfpathlineto{\pgfqpoint{1.513057in}{1.447870in}}%
\pgfpathlineto{\pgfqpoint{1.510659in}{1.461481in}}%
\pgfpathlineto{\pgfqpoint{1.510659in}{1.475092in}}%
\pgfpathlineto{\pgfqpoint{1.513057in}{1.488703in}}%
\pgfpathlineto{\pgfqpoint{1.517859in}{1.502314in}}%
\pgfpathlineto{\pgfqpoint{1.519001in}{1.504505in}}%
\pgfpathlineto{\pgfqpoint{1.525010in}{1.515925in}}%
\pgfpathlineto{\pgfqpoint{1.534534in}{1.529536in}}%
\pgfpathlineto{\pgfqpoint{1.534657in}{1.529683in}}%
\pgfpathlineto{\pgfqpoint{1.546729in}{1.543148in}}%
\pgfpathlineto{\pgfqpoint{1.550314in}{1.546594in}}%
\pgfpathlineto{\pgfqpoint{1.562006in}{1.556759in}}%
\pgfpathlineto{\pgfqpoint{1.565971in}{1.559875in}}%
\pgfpathlineto{\pgfqpoint{1.581459in}{1.570370in}}%
\pgfpathlineto{\pgfqpoint{1.581627in}{1.570477in}}%
\pgfpathlineto{\pgfqpoint{1.597284in}{1.578757in}}%
\pgfpathlineto{\pgfqpoint{1.610420in}{1.583981in}}%
\pgfpathlineto{\pgfqpoint{1.612940in}{1.584974in}}%
\pgfpathlineto{\pgfqpoint{1.628597in}{1.589148in}}%
\pgfpathlineto{\pgfqpoint{1.644253in}{1.591233in}}%
\pgfpathlineto{\pgfqpoint{1.659910in}{1.591233in}}%
\pgfpathlineto{\pgfqpoint{1.675567in}{1.589148in}}%
\pgfpathlineto{\pgfqpoint{1.691223in}{1.584974in}}%
\pgfpathlineto{\pgfqpoint{1.693743in}{1.583981in}}%
\pgfpathlineto{\pgfqpoint{1.706880in}{1.578757in}}%
\pgfpathlineto{\pgfqpoint{1.722536in}{1.570477in}}%
\pgfpathlineto{\pgfqpoint{1.722704in}{1.570370in}}%
\pgfpathlineto{\pgfqpoint{1.738193in}{1.559875in}}%
\pgfpathlineto{\pgfqpoint{1.742157in}{1.556759in}}%
\pgfpathlineto{\pgfqpoint{1.753849in}{1.546594in}}%
\pgfpathlineto{\pgfqpoint{1.757434in}{1.543148in}}%
\pgfpathlineto{\pgfqpoint{1.769506in}{1.529683in}}%
\pgfpathlineto{\pgfqpoint{1.769630in}{1.529536in}}%
\pgfpathlineto{\pgfqpoint{1.779154in}{1.515925in}}%
\pgfpathlineto{\pgfqpoint{1.785162in}{1.504505in}}%
\pgfpathlineto{\pgfqpoint{1.786305in}{1.502314in}}%
\pgfpathlineto{\pgfqpoint{1.791107in}{1.488703in}}%
\pgfpathlineto{\pgfqpoint{1.793504in}{1.475092in}}%
\pgfpathlineto{\pgfqpoint{1.793504in}{1.461481in}}%
\pgfpathlineto{\pgfqpoint{1.791107in}{1.447870in}}%
\pgfpathlineto{\pgfqpoint{1.786305in}{1.434259in}}%
\pgfpathlineto{\pgfqpoint{1.785162in}{1.432067in}}%
\pgfpathlineto{\pgfqpoint{1.779154in}{1.420648in}}%
\pgfpathlineto{\pgfqpoint{1.769630in}{1.407036in}}%
\pgfpathlineto{\pgfqpoint{1.769506in}{1.406890in}}%
\pgfpathlineto{\pgfqpoint{1.757434in}{1.393425in}}%
\pgfpathlineto{\pgfqpoint{1.753849in}{1.389979in}}%
\pgfpathlineto{\pgfqpoint{1.742157in}{1.379814in}}%
\pgfpathlineto{\pgfqpoint{1.738193in}{1.376697in}}%
\pgfpathlineto{\pgfqpoint{1.722704in}{1.366203in}}%
\pgfpathlineto{\pgfqpoint{1.722536in}{1.366095in}}%
\pgfpathlineto{\pgfqpoint{1.706880in}{1.357816in}}%
\pgfpathlineto{\pgfqpoint{1.693743in}{1.352592in}}%
\pgfpathlineto{\pgfqpoint{1.691223in}{1.351599in}}%
\pgfpathlineto{\pgfqpoint{1.675567in}{1.347424in}}%
\pgfpathlineto{\pgfqpoint{1.659910in}{1.345340in}}%
\pgfpathlineto{\pgfqpoint{1.644253in}{1.345340in}}%
\pgfpathlineto{\pgfqpoint{1.628597in}{1.347424in}}%
\pgfpathlineto{\pgfqpoint{1.612940in}{1.351599in}}%
\pgfpathlineto{\pgfqpoint{1.610420in}{1.352592in}}%
\pgfpathclose%
\pgfusepath{fill}%
\end{pgfscope}%
\begin{pgfscope}%
\pgfpathrectangle{\pgfqpoint{0.360415in}{0.345370in}}{\pgfqpoint{1.550000in}{1.347500in}}%
\pgfusepath{clip}%
\pgfsetbuttcap%
\pgfsetroundjoin%
\definecolor{currentfill}{rgb}{0.794549,0.275770,0.473117}%
\pgfsetfillcolor{currentfill}%
\pgfsetlinewidth{0.000000pt}%
\definecolor{currentstroke}{rgb}{0.000000,0.000000,0.000000}%
\pgfsetstrokecolor{currentstroke}%
\pgfsetdash{}{0pt}%
\pgfpathmoveto{\pgfqpoint{0.610920in}{0.370140in}}%
\pgfpathlineto{\pgfqpoint{0.626577in}{0.370140in}}%
\pgfpathlineto{\pgfqpoint{0.632980in}{0.372592in}}%
\pgfpathlineto{\pgfqpoint{0.642233in}{0.374784in}}%
\pgfpathlineto{\pgfqpoint{0.657890in}{0.382193in}}%
\pgfpathlineto{\pgfqpoint{0.663603in}{0.386203in}}%
\pgfpathlineto{\pgfqpoint{0.673546in}{0.391476in}}%
\pgfpathlineto{\pgfqpoint{0.685490in}{0.399814in}}%
\pgfpathlineto{\pgfqpoint{0.689203in}{0.401987in}}%
\pgfpathlineto{\pgfqpoint{0.704859in}{0.413200in}}%
\pgfpathlineto{\pgfqpoint{0.705133in}{0.413425in}}%
\pgfpathlineto{\pgfqpoint{0.720516in}{0.424882in}}%
\pgfpathlineto{\pgfqpoint{0.723118in}{0.427036in}}%
\pgfpathlineto{\pgfqpoint{0.736173in}{0.437333in}}%
\pgfpathlineto{\pgfqpoint{0.740093in}{0.440648in}}%
\pgfpathlineto{\pgfqpoint{0.751829in}{0.450517in}}%
\pgfpathlineto{\pgfqpoint{0.756132in}{0.454259in}}%
\pgfpathlineto{\pgfqpoint{0.767486in}{0.464462in}}%
\pgfpathlineto{\pgfqpoint{0.771298in}{0.467870in}}%
\pgfpathlineto{\pgfqpoint{0.783142in}{0.479219in}}%
\pgfpathlineto{\pgfqpoint{0.785621in}{0.481481in}}%
\pgfpathlineto{\pgfqpoint{0.798799in}{0.494854in}}%
\pgfpathlineto{\pgfqpoint{0.799058in}{0.495092in}}%
\pgfpathlineto{\pgfqpoint{0.811956in}{0.508703in}}%
\pgfpathlineto{\pgfqpoint{0.814455in}{0.511931in}}%
\pgfpathlineto{\pgfqpoint{0.824047in}{0.522314in}}%
\pgfpathlineto{\pgfqpoint{0.830112in}{0.530959in}}%
\pgfpathlineto{\pgfqpoint{0.834725in}{0.535925in}}%
\pgfpathlineto{\pgfqpoint{0.843247in}{0.549536in}}%
\pgfpathlineto{\pgfqpoint{0.845769in}{0.557581in}}%
\pgfpathlineto{\pgfqpoint{0.848589in}{0.563148in}}%
\pgfpathlineto{\pgfqpoint{0.848589in}{0.576759in}}%
\pgfpathlineto{\pgfqpoint{0.845769in}{0.582325in}}%
\pgfpathlineto{\pgfqpoint{0.843247in}{0.590370in}}%
\pgfpathlineto{\pgfqpoint{0.834725in}{0.603981in}}%
\pgfpathlineto{\pgfqpoint{0.830112in}{0.608947in}}%
\pgfpathlineto{\pgfqpoint{0.824047in}{0.617592in}}%
\pgfpathlineto{\pgfqpoint{0.814455in}{0.627975in}}%
\pgfpathlineto{\pgfqpoint{0.811956in}{0.631203in}}%
\pgfpathlineto{\pgfqpoint{0.799058in}{0.644814in}}%
\pgfpathlineto{\pgfqpoint{0.798799in}{0.645052in}}%
\pgfpathlineto{\pgfqpoint{0.785621in}{0.658425in}}%
\pgfpathlineto{\pgfqpoint{0.783142in}{0.660687in}}%
\pgfpathlineto{\pgfqpoint{0.771298in}{0.672036in}}%
\pgfpathlineto{\pgfqpoint{0.767486in}{0.675444in}}%
\pgfpathlineto{\pgfqpoint{0.756132in}{0.685648in}}%
\pgfpathlineto{\pgfqpoint{0.751829in}{0.689389in}}%
\pgfpathlineto{\pgfqpoint{0.740093in}{0.699259in}}%
\pgfpathlineto{\pgfqpoint{0.736173in}{0.702573in}}%
\pgfpathlineto{\pgfqpoint{0.723118in}{0.712870in}}%
\pgfpathlineto{\pgfqpoint{0.720516in}{0.715025in}}%
\pgfpathlineto{\pgfqpoint{0.705133in}{0.726481in}}%
\pgfpathlineto{\pgfqpoint{0.704859in}{0.726706in}}%
\pgfpathlineto{\pgfqpoint{0.689203in}{0.737919in}}%
\pgfpathlineto{\pgfqpoint{0.685490in}{0.740092in}}%
\pgfpathlineto{\pgfqpoint{0.673546in}{0.748430in}}%
\pgfpathlineto{\pgfqpoint{0.663603in}{0.753703in}}%
\pgfpathlineto{\pgfqpoint{0.657890in}{0.757713in}}%
\pgfpathlineto{\pgfqpoint{0.642233in}{0.765122in}}%
\pgfpathlineto{\pgfqpoint{0.632980in}{0.767314in}}%
\pgfpathlineto{\pgfqpoint{0.626577in}{0.769767in}}%
\pgfpathlineto{\pgfqpoint{0.610920in}{0.769767in}}%
\pgfpathlineto{\pgfqpoint{0.604517in}{0.767314in}}%
\pgfpathlineto{\pgfqpoint{0.595263in}{0.765122in}}%
\pgfpathlineto{\pgfqpoint{0.579607in}{0.757713in}}%
\pgfpathlineto{\pgfqpoint{0.573894in}{0.753703in}}%
\pgfpathlineto{\pgfqpoint{0.563950in}{0.748430in}}%
\pgfpathlineto{\pgfqpoint{0.552007in}{0.740092in}}%
\pgfpathlineto{\pgfqpoint{0.548294in}{0.737919in}}%
\pgfpathlineto{\pgfqpoint{0.532637in}{0.726706in}}%
\pgfpathlineto{\pgfqpoint{0.532364in}{0.726481in}}%
\pgfpathlineto{\pgfqpoint{0.516981in}{0.715025in}}%
\pgfpathlineto{\pgfqpoint{0.514379in}{0.712870in}}%
\pgfpathlineto{\pgfqpoint{0.501324in}{0.702573in}}%
\pgfpathlineto{\pgfqpoint{0.497404in}{0.699259in}}%
\pgfpathlineto{\pgfqpoint{0.485668in}{0.689389in}}%
\pgfpathlineto{\pgfqpoint{0.481364in}{0.685648in}}%
\pgfpathlineto{\pgfqpoint{0.470011in}{0.675444in}}%
\pgfpathlineto{\pgfqpoint{0.466198in}{0.672036in}}%
\pgfpathlineto{\pgfqpoint{0.454354in}{0.660687in}}%
\pgfpathlineto{\pgfqpoint{0.451876in}{0.658425in}}%
\pgfpathlineto{\pgfqpoint{0.438698in}{0.645052in}}%
\pgfpathlineto{\pgfqpoint{0.438438in}{0.644814in}}%
\pgfpathlineto{\pgfqpoint{0.425541in}{0.631203in}}%
\pgfpathlineto{\pgfqpoint{0.423041in}{0.627975in}}%
\pgfpathlineto{\pgfqpoint{0.413450in}{0.617592in}}%
\pgfpathlineto{\pgfqpoint{0.407385in}{0.608947in}}%
\pgfpathlineto{\pgfqpoint{0.402772in}{0.603981in}}%
\pgfpathlineto{\pgfqpoint{0.394250in}{0.590370in}}%
\pgfpathlineto{\pgfqpoint{0.391728in}{0.582325in}}%
\pgfpathlineto{\pgfqpoint{0.388907in}{0.576759in}}%
\pgfpathlineto{\pgfqpoint{0.388907in}{0.563148in}}%
\pgfpathlineto{\pgfqpoint{0.391728in}{0.557581in}}%
\pgfpathlineto{\pgfqpoint{0.394250in}{0.549536in}}%
\pgfpathlineto{\pgfqpoint{0.402772in}{0.535925in}}%
\pgfpathlineto{\pgfqpoint{0.407385in}{0.530959in}}%
\pgfpathlineto{\pgfqpoint{0.413450in}{0.522314in}}%
\pgfpathlineto{\pgfqpoint{0.423041in}{0.511931in}}%
\pgfpathlineto{\pgfqpoint{0.425541in}{0.508703in}}%
\pgfpathlineto{\pgfqpoint{0.438438in}{0.495092in}}%
\pgfpathlineto{\pgfqpoint{0.438698in}{0.494854in}}%
\pgfpathlineto{\pgfqpoint{0.451876in}{0.481481in}}%
\pgfpathlineto{\pgfqpoint{0.454354in}{0.479219in}}%
\pgfpathlineto{\pgfqpoint{0.466198in}{0.467870in}}%
\pgfpathlineto{\pgfqpoint{0.470011in}{0.464462in}}%
\pgfpathlineto{\pgfqpoint{0.481364in}{0.454259in}}%
\pgfpathlineto{\pgfqpoint{0.485668in}{0.450517in}}%
\pgfpathlineto{\pgfqpoint{0.497404in}{0.440648in}}%
\pgfpathlineto{\pgfqpoint{0.501324in}{0.437333in}}%
\pgfpathlineto{\pgfqpoint{0.514379in}{0.427036in}}%
\pgfpathlineto{\pgfqpoint{0.516981in}{0.424882in}}%
\pgfpathlineto{\pgfqpoint{0.532364in}{0.413425in}}%
\pgfpathlineto{\pgfqpoint{0.532637in}{0.413200in}}%
\pgfpathlineto{\pgfqpoint{0.548294in}{0.401987in}}%
\pgfpathlineto{\pgfqpoint{0.552007in}{0.399814in}}%
\pgfpathlineto{\pgfqpoint{0.563950in}{0.391476in}}%
\pgfpathlineto{\pgfqpoint{0.573894in}{0.386203in}}%
\pgfpathlineto{\pgfqpoint{0.579607in}{0.382193in}}%
\pgfpathlineto{\pgfqpoint{0.595263in}{0.374784in}}%
\pgfpathlineto{\pgfqpoint{0.604517in}{0.372592in}}%
\pgfpathlineto{\pgfqpoint{0.610920in}{0.370140in}}%
\pgfpathclose%
\pgfpathmoveto{\pgfqpoint{0.569966in}{0.427036in}}%
\pgfpathlineto{\pgfqpoint{0.563950in}{0.429360in}}%
\pgfpathlineto{\pgfqpoint{0.548294in}{0.437333in}}%
\pgfpathlineto{\pgfqpoint{0.542968in}{0.440648in}}%
\pgfpathlineto{\pgfqpoint{0.532637in}{0.446971in}}%
\pgfpathlineto{\pgfqpoint{0.522419in}{0.454259in}}%
\pgfpathlineto{\pgfqpoint{0.516981in}{0.458237in}}%
\pgfpathlineto{\pgfqpoint{0.505148in}{0.467870in}}%
\pgfpathlineto{\pgfqpoint{0.501324in}{0.471194in}}%
\pgfpathlineto{\pgfqpoint{0.490243in}{0.481481in}}%
\pgfpathlineto{\pgfqpoint{0.485668in}{0.486209in}}%
\pgfpathlineto{\pgfqpoint{0.477285in}{0.495092in}}%
\pgfpathlineto{\pgfqpoint{0.470011in}{0.504073in}}%
\pgfpathlineto{\pgfqpoint{0.466198in}{0.508703in}}%
\pgfpathlineto{\pgfqpoint{0.457028in}{0.522314in}}%
\pgfpathlineto{\pgfqpoint{0.454354in}{0.527544in}}%
\pgfpathlineto{\pgfqpoint{0.449765in}{0.535925in}}%
\pgfpathlineto{\pgfqpoint{0.444766in}{0.549536in}}%
\pgfpathlineto{\pgfqpoint{0.442270in}{0.563148in}}%
\pgfpathlineto{\pgfqpoint{0.442270in}{0.576759in}}%
\pgfpathlineto{\pgfqpoint{0.444766in}{0.590370in}}%
\pgfpathlineto{\pgfqpoint{0.449765in}{0.603981in}}%
\pgfpathlineto{\pgfqpoint{0.454354in}{0.612362in}}%
\pgfpathlineto{\pgfqpoint{0.457028in}{0.617592in}}%
\pgfpathlineto{\pgfqpoint{0.466198in}{0.631203in}}%
\pgfpathlineto{\pgfqpoint{0.470011in}{0.635833in}}%
\pgfpathlineto{\pgfqpoint{0.477285in}{0.644814in}}%
\pgfpathlineto{\pgfqpoint{0.485668in}{0.653697in}}%
\pgfpathlineto{\pgfqpoint{0.490243in}{0.658425in}}%
\pgfpathlineto{\pgfqpoint{0.501324in}{0.668712in}}%
\pgfpathlineto{\pgfqpoint{0.505148in}{0.672036in}}%
\pgfpathlineto{\pgfqpoint{0.516981in}{0.681669in}}%
\pgfpathlineto{\pgfqpoint{0.522419in}{0.685648in}}%
\pgfpathlineto{\pgfqpoint{0.532637in}{0.692935in}}%
\pgfpathlineto{\pgfqpoint{0.542968in}{0.699259in}}%
\pgfpathlineto{\pgfqpoint{0.548294in}{0.702573in}}%
\pgfpathlineto{\pgfqpoint{0.563950in}{0.710546in}}%
\pgfpathlineto{\pgfqpoint{0.569966in}{0.712870in}}%
\pgfpathlineto{\pgfqpoint{0.579607in}{0.716859in}}%
\pgfpathlineto{\pgfqpoint{0.595263in}{0.721205in}}%
\pgfpathlineto{\pgfqpoint{0.610920in}{0.723376in}}%
\pgfpathlineto{\pgfqpoint{0.626577in}{0.723376in}}%
\pgfpathlineto{\pgfqpoint{0.642233in}{0.721205in}}%
\pgfpathlineto{\pgfqpoint{0.657890in}{0.716859in}}%
\pgfpathlineto{\pgfqpoint{0.667531in}{0.712870in}}%
\pgfpathlineto{\pgfqpoint{0.673546in}{0.710546in}}%
\pgfpathlineto{\pgfqpoint{0.689203in}{0.702573in}}%
\pgfpathlineto{\pgfqpoint{0.694529in}{0.699259in}}%
\pgfpathlineto{\pgfqpoint{0.704859in}{0.692935in}}%
\pgfpathlineto{\pgfqpoint{0.715077in}{0.685648in}}%
\pgfpathlineto{\pgfqpoint{0.720516in}{0.681669in}}%
\pgfpathlineto{\pgfqpoint{0.732349in}{0.672036in}}%
\pgfpathlineto{\pgfqpoint{0.736173in}{0.668712in}}%
\pgfpathlineto{\pgfqpoint{0.747253in}{0.658425in}}%
\pgfpathlineto{\pgfqpoint{0.751829in}{0.653697in}}%
\pgfpathlineto{\pgfqpoint{0.760212in}{0.644814in}}%
\pgfpathlineto{\pgfqpoint{0.767486in}{0.635833in}}%
\pgfpathlineto{\pgfqpoint{0.771299in}{0.631203in}}%
\pgfpathlineto{\pgfqpoint{0.780469in}{0.617592in}}%
\pgfpathlineto{\pgfqpoint{0.783142in}{0.612362in}}%
\pgfpathlineto{\pgfqpoint{0.787731in}{0.603981in}}%
\pgfpathlineto{\pgfqpoint{0.792731in}{0.590370in}}%
\pgfpathlineto{\pgfqpoint{0.795227in}{0.576759in}}%
\pgfpathlineto{\pgfqpoint{0.795227in}{0.563148in}}%
\pgfpathlineto{\pgfqpoint{0.792731in}{0.549536in}}%
\pgfpathlineto{\pgfqpoint{0.787731in}{0.535925in}}%
\pgfpathlineto{\pgfqpoint{0.783142in}{0.527544in}}%
\pgfpathlineto{\pgfqpoint{0.780469in}{0.522314in}}%
\pgfpathlineto{\pgfqpoint{0.771299in}{0.508703in}}%
\pgfpathlineto{\pgfqpoint{0.767486in}{0.504073in}}%
\pgfpathlineto{\pgfqpoint{0.760212in}{0.495092in}}%
\pgfpathlineto{\pgfqpoint{0.751829in}{0.486209in}}%
\pgfpathlineto{\pgfqpoint{0.747253in}{0.481481in}}%
\pgfpathlineto{\pgfqpoint{0.736173in}{0.471194in}}%
\pgfpathlineto{\pgfqpoint{0.732349in}{0.467870in}}%
\pgfpathlineto{\pgfqpoint{0.720516in}{0.458237in}}%
\pgfpathlineto{\pgfqpoint{0.715077in}{0.454259in}}%
\pgfpathlineto{\pgfqpoint{0.704859in}{0.446971in}}%
\pgfpathlineto{\pgfqpoint{0.694529in}{0.440648in}}%
\pgfpathlineto{\pgfqpoint{0.689203in}{0.437333in}}%
\pgfpathlineto{\pgfqpoint{0.673546in}{0.429360in}}%
\pgfpathlineto{\pgfqpoint{0.667531in}{0.427036in}}%
\pgfpathlineto{\pgfqpoint{0.657890in}{0.423047in}}%
\pgfpathlineto{\pgfqpoint{0.642233in}{0.418701in}}%
\pgfpathlineto{\pgfqpoint{0.626577in}{0.416530in}}%
\pgfpathlineto{\pgfqpoint{0.610920in}{0.416530in}}%
\pgfpathlineto{\pgfqpoint{0.595263in}{0.418701in}}%
\pgfpathlineto{\pgfqpoint{0.579607in}{0.423047in}}%
\pgfpathlineto{\pgfqpoint{0.569966in}{0.427036in}}%
\pgfpathclose%
\pgfpathmoveto{\pgfqpoint{1.127587in}{0.370140in}}%
\pgfpathlineto{\pgfqpoint{1.143243in}{0.370140in}}%
\pgfpathlineto{\pgfqpoint{1.149647in}{0.372592in}}%
\pgfpathlineto{\pgfqpoint{1.158900in}{0.374784in}}%
\pgfpathlineto{\pgfqpoint{1.174556in}{0.382193in}}%
\pgfpathlineto{\pgfqpoint{1.180269in}{0.386203in}}%
\pgfpathlineto{\pgfqpoint{1.190213in}{0.391476in}}%
\pgfpathlineto{\pgfqpoint{1.202156in}{0.399814in}}%
\pgfpathlineto{\pgfqpoint{1.205870in}{0.401987in}}%
\pgfpathlineto{\pgfqpoint{1.221526in}{0.413200in}}%
\pgfpathlineto{\pgfqpoint{1.221800in}{0.413425in}}%
\pgfpathlineto{\pgfqpoint{1.237183in}{0.424882in}}%
\pgfpathlineto{\pgfqpoint{1.239784in}{0.427036in}}%
\pgfpathlineto{\pgfqpoint{1.252839in}{0.437333in}}%
\pgfpathlineto{\pgfqpoint{1.256759in}{0.440648in}}%
\pgfpathlineto{\pgfqpoint{1.268496in}{0.450517in}}%
\pgfpathlineto{\pgfqpoint{1.272799in}{0.454259in}}%
\pgfpathlineto{\pgfqpoint{1.284152in}{0.464462in}}%
\pgfpathlineto{\pgfqpoint{1.287965in}{0.467870in}}%
\pgfpathlineto{\pgfqpoint{1.299809in}{0.479219in}}%
\pgfpathlineto{\pgfqpoint{1.302288in}{0.481481in}}%
\pgfpathlineto{\pgfqpoint{1.315466in}{0.494854in}}%
\pgfpathlineto{\pgfqpoint{1.315725in}{0.495092in}}%
\pgfpathlineto{\pgfqpoint{1.328622in}{0.508703in}}%
\pgfpathlineto{\pgfqpoint{1.331122in}{0.511931in}}%
\pgfpathlineto{\pgfqpoint{1.340713in}{0.522314in}}%
\pgfpathlineto{\pgfqpoint{1.346779in}{0.530959in}}%
\pgfpathlineto{\pgfqpoint{1.351392in}{0.535925in}}%
\pgfpathlineto{\pgfqpoint{1.359913in}{0.549536in}}%
\pgfpathlineto{\pgfqpoint{1.362435in}{0.557581in}}%
\pgfpathlineto{\pgfqpoint{1.365256in}{0.563148in}}%
\pgfpathlineto{\pgfqpoint{1.365256in}{0.576759in}}%
\pgfpathlineto{\pgfqpoint{1.362435in}{0.582325in}}%
\pgfpathlineto{\pgfqpoint{1.359913in}{0.590370in}}%
\pgfpathlineto{\pgfqpoint{1.351392in}{0.603981in}}%
\pgfpathlineto{\pgfqpoint{1.346779in}{0.608947in}}%
\pgfpathlineto{\pgfqpoint{1.340713in}{0.617592in}}%
\pgfpathlineto{\pgfqpoint{1.331122in}{0.627975in}}%
\pgfpathlineto{\pgfqpoint{1.328622in}{0.631203in}}%
\pgfpathlineto{\pgfqpoint{1.315725in}{0.644814in}}%
\pgfpathlineto{\pgfqpoint{1.315466in}{0.645052in}}%
\pgfpathlineto{\pgfqpoint{1.302288in}{0.658425in}}%
\pgfpathlineto{\pgfqpoint{1.299809in}{0.660687in}}%
\pgfpathlineto{\pgfqpoint{1.287965in}{0.672036in}}%
\pgfpathlineto{\pgfqpoint{1.284152in}{0.675444in}}%
\pgfpathlineto{\pgfqpoint{1.272799in}{0.685648in}}%
\pgfpathlineto{\pgfqpoint{1.268496in}{0.689389in}}%
\pgfpathlineto{\pgfqpoint{1.256759in}{0.699259in}}%
\pgfpathlineto{\pgfqpoint{1.252839in}{0.702573in}}%
\pgfpathlineto{\pgfqpoint{1.239784in}{0.712870in}}%
\pgfpathlineto{\pgfqpoint{1.237183in}{0.715025in}}%
\pgfpathlineto{\pgfqpoint{1.221800in}{0.726481in}}%
\pgfpathlineto{\pgfqpoint{1.221526in}{0.726706in}}%
\pgfpathlineto{\pgfqpoint{1.205870in}{0.737919in}}%
\pgfpathlineto{\pgfqpoint{1.202156in}{0.740092in}}%
\pgfpathlineto{\pgfqpoint{1.190213in}{0.748430in}}%
\pgfpathlineto{\pgfqpoint{1.180269in}{0.753703in}}%
\pgfpathlineto{\pgfqpoint{1.174556in}{0.757713in}}%
\pgfpathlineto{\pgfqpoint{1.158900in}{0.765122in}}%
\pgfpathlineto{\pgfqpoint{1.149647in}{0.767314in}}%
\pgfpathlineto{\pgfqpoint{1.143243in}{0.769767in}}%
\pgfpathlineto{\pgfqpoint{1.127587in}{0.769767in}}%
\pgfpathlineto{\pgfqpoint{1.121183in}{0.767314in}}%
\pgfpathlineto{\pgfqpoint{1.111930in}{0.765122in}}%
\pgfpathlineto{\pgfqpoint{1.096274in}{0.757713in}}%
\pgfpathlineto{\pgfqpoint{1.090561in}{0.753703in}}%
\pgfpathlineto{\pgfqpoint{1.080617in}{0.748430in}}%
\pgfpathlineto{\pgfqpoint{1.068674in}{0.740092in}}%
\pgfpathlineto{\pgfqpoint{1.064960in}{0.737919in}}%
\pgfpathlineto{\pgfqpoint{1.049304in}{0.726706in}}%
\pgfpathlineto{\pgfqpoint{1.049030in}{0.726481in}}%
\pgfpathlineto{\pgfqpoint{1.033647in}{0.715025in}}%
\pgfpathlineto{\pgfqpoint{1.031046in}{0.712870in}}%
\pgfpathlineto{\pgfqpoint{1.017991in}{0.702573in}}%
\pgfpathlineto{\pgfqpoint{1.014071in}{0.699259in}}%
\pgfpathlineto{\pgfqpoint{1.002334in}{0.689389in}}%
\pgfpathlineto{\pgfqpoint{0.998031in}{0.685648in}}%
\pgfpathlineto{\pgfqpoint{0.986678in}{0.675444in}}%
\pgfpathlineto{\pgfqpoint{0.982865in}{0.672036in}}%
\pgfpathlineto{\pgfqpoint{0.971021in}{0.660687in}}%
\pgfpathlineto{\pgfqpoint{0.968542in}{0.658425in}}%
\pgfpathlineto{\pgfqpoint{0.955364in}{0.645052in}}%
\pgfpathlineto{\pgfqpoint{0.955105in}{0.644814in}}%
\pgfpathlineto{\pgfqpoint{0.942208in}{0.631203in}}%
\pgfpathlineto{\pgfqpoint{0.939708in}{0.627975in}}%
\pgfpathlineto{\pgfqpoint{0.930117in}{0.617592in}}%
\pgfpathlineto{\pgfqpoint{0.924051in}{0.608947in}}%
\pgfpathlineto{\pgfqpoint{0.919438in}{0.603981in}}%
\pgfpathlineto{\pgfqpoint{0.910917in}{0.590370in}}%
\pgfpathlineto{\pgfqpoint{0.908395in}{0.582325in}}%
\pgfpathlineto{\pgfqpoint{0.905574in}{0.576759in}}%
\pgfpathlineto{\pgfqpoint{0.905574in}{0.563148in}}%
\pgfpathlineto{\pgfqpoint{0.908395in}{0.557581in}}%
\pgfpathlineto{\pgfqpoint{0.910917in}{0.549536in}}%
\pgfpathlineto{\pgfqpoint{0.919438in}{0.535925in}}%
\pgfpathlineto{\pgfqpoint{0.924051in}{0.530959in}}%
\pgfpathlineto{\pgfqpoint{0.930117in}{0.522314in}}%
\pgfpathlineto{\pgfqpoint{0.939708in}{0.511931in}}%
\pgfpathlineto{\pgfqpoint{0.942208in}{0.508703in}}%
\pgfpathlineto{\pgfqpoint{0.955105in}{0.495092in}}%
\pgfpathlineto{\pgfqpoint{0.955364in}{0.494854in}}%
\pgfpathlineto{\pgfqpoint{0.968542in}{0.481481in}}%
\pgfpathlineto{\pgfqpoint{0.971021in}{0.479219in}}%
\pgfpathlineto{\pgfqpoint{0.982865in}{0.467870in}}%
\pgfpathlineto{\pgfqpoint{0.986678in}{0.464462in}}%
\pgfpathlineto{\pgfqpoint{0.998031in}{0.454259in}}%
\pgfpathlineto{\pgfqpoint{1.002334in}{0.450517in}}%
\pgfpathlineto{\pgfqpoint{1.014071in}{0.440648in}}%
\pgfpathlineto{\pgfqpoint{1.017991in}{0.437333in}}%
\pgfpathlineto{\pgfqpoint{1.031046in}{0.427036in}}%
\pgfpathlineto{\pgfqpoint{1.033647in}{0.424882in}}%
\pgfpathlineto{\pgfqpoint{1.049030in}{0.413425in}}%
\pgfpathlineto{\pgfqpoint{1.049304in}{0.413200in}}%
\pgfpathlineto{\pgfqpoint{1.064960in}{0.401987in}}%
\pgfpathlineto{\pgfqpoint{1.068674in}{0.399814in}}%
\pgfpathlineto{\pgfqpoint{1.080617in}{0.391476in}}%
\pgfpathlineto{\pgfqpoint{1.090561in}{0.386203in}}%
\pgfpathlineto{\pgfqpoint{1.096274in}{0.382193in}}%
\pgfpathlineto{\pgfqpoint{1.111930in}{0.374784in}}%
\pgfpathlineto{\pgfqpoint{1.121183in}{0.372592in}}%
\pgfpathlineto{\pgfqpoint{1.127587in}{0.370140in}}%
\pgfpathclose%
\pgfpathmoveto{\pgfqpoint{1.086633in}{0.427036in}}%
\pgfpathlineto{\pgfqpoint{1.080617in}{0.429360in}}%
\pgfpathlineto{\pgfqpoint{1.064960in}{0.437333in}}%
\pgfpathlineto{\pgfqpoint{1.059635in}{0.440648in}}%
\pgfpathlineto{\pgfqpoint{1.049304in}{0.446971in}}%
\pgfpathlineto{\pgfqpoint{1.039086in}{0.454259in}}%
\pgfpathlineto{\pgfqpoint{1.033647in}{0.458237in}}%
\pgfpathlineto{\pgfqpoint{1.021815in}{0.467870in}}%
\pgfpathlineto{\pgfqpoint{1.017991in}{0.471194in}}%
\pgfpathlineto{\pgfqpoint{1.006910in}{0.481481in}}%
\pgfpathlineto{\pgfqpoint{1.002334in}{0.486209in}}%
\pgfpathlineto{\pgfqpoint{0.993952in}{0.495092in}}%
\pgfpathlineto{\pgfqpoint{0.986678in}{0.504073in}}%
\pgfpathlineto{\pgfqpoint{0.982865in}{0.508703in}}%
\pgfpathlineto{\pgfqpoint{0.973694in}{0.522314in}}%
\pgfpathlineto{\pgfqpoint{0.971021in}{0.527544in}}%
\pgfpathlineto{\pgfqpoint{0.966432in}{0.535925in}}%
\pgfpathlineto{\pgfqpoint{0.961433in}{0.549536in}}%
\pgfpathlineto{\pgfqpoint{0.958936in}{0.563148in}}%
\pgfpathlineto{\pgfqpoint{0.958936in}{0.576759in}}%
\pgfpathlineto{\pgfqpoint{0.961433in}{0.590370in}}%
\pgfpathlineto{\pgfqpoint{0.966432in}{0.603981in}}%
\pgfpathlineto{\pgfqpoint{0.971021in}{0.612362in}}%
\pgfpathlineto{\pgfqpoint{0.973694in}{0.617592in}}%
\pgfpathlineto{\pgfqpoint{0.982865in}{0.631203in}}%
\pgfpathlineto{\pgfqpoint{0.986678in}{0.635833in}}%
\pgfpathlineto{\pgfqpoint{0.993952in}{0.644814in}}%
\pgfpathlineto{\pgfqpoint{1.002334in}{0.653697in}}%
\pgfpathlineto{\pgfqpoint{1.006910in}{0.658425in}}%
\pgfpathlineto{\pgfqpoint{1.017991in}{0.668712in}}%
\pgfpathlineto{\pgfqpoint{1.021815in}{0.672036in}}%
\pgfpathlineto{\pgfqpoint{1.033647in}{0.681669in}}%
\pgfpathlineto{\pgfqpoint{1.039086in}{0.685648in}}%
\pgfpathlineto{\pgfqpoint{1.049304in}{0.692935in}}%
\pgfpathlineto{\pgfqpoint{1.059635in}{0.699259in}}%
\pgfpathlineto{\pgfqpoint{1.064960in}{0.702573in}}%
\pgfpathlineto{\pgfqpoint{1.080617in}{0.710546in}}%
\pgfpathlineto{\pgfqpoint{1.086633in}{0.712870in}}%
\pgfpathlineto{\pgfqpoint{1.096274in}{0.716859in}}%
\pgfpathlineto{\pgfqpoint{1.111930in}{0.721205in}}%
\pgfpathlineto{\pgfqpoint{1.127587in}{0.723376in}}%
\pgfpathlineto{\pgfqpoint{1.143243in}{0.723376in}}%
\pgfpathlineto{\pgfqpoint{1.158900in}{0.721205in}}%
\pgfpathlineto{\pgfqpoint{1.174556in}{0.716859in}}%
\pgfpathlineto{\pgfqpoint{1.184197in}{0.712870in}}%
\pgfpathlineto{\pgfqpoint{1.190213in}{0.710546in}}%
\pgfpathlineto{\pgfqpoint{1.205870in}{0.702573in}}%
\pgfpathlineto{\pgfqpoint{1.211195in}{0.699259in}}%
\pgfpathlineto{\pgfqpoint{1.221526in}{0.692935in}}%
\pgfpathlineto{\pgfqpoint{1.231744in}{0.685648in}}%
\pgfpathlineto{\pgfqpoint{1.237183in}{0.681669in}}%
\pgfpathlineto{\pgfqpoint{1.249015in}{0.672036in}}%
\pgfpathlineto{\pgfqpoint{1.252839in}{0.668712in}}%
\pgfpathlineto{\pgfqpoint{1.263920in}{0.658425in}}%
\pgfpathlineto{\pgfqpoint{1.268496in}{0.653697in}}%
\pgfpathlineto{\pgfqpoint{1.276878in}{0.644814in}}%
\pgfpathlineto{\pgfqpoint{1.284152in}{0.635833in}}%
\pgfpathlineto{\pgfqpoint{1.287965in}{0.631203in}}%
\pgfpathlineto{\pgfqpoint{1.297136in}{0.617592in}}%
\pgfpathlineto{\pgfqpoint{1.299809in}{0.612362in}}%
\pgfpathlineto{\pgfqpoint{1.304398in}{0.603981in}}%
\pgfpathlineto{\pgfqpoint{1.309397in}{0.590370in}}%
\pgfpathlineto{\pgfqpoint{1.311894in}{0.576759in}}%
\pgfpathlineto{\pgfqpoint{1.311894in}{0.563148in}}%
\pgfpathlineto{\pgfqpoint{1.309397in}{0.549536in}}%
\pgfpathlineto{\pgfqpoint{1.304398in}{0.535925in}}%
\pgfpathlineto{\pgfqpoint{1.299809in}{0.527544in}}%
\pgfpathlineto{\pgfqpoint{1.297136in}{0.522314in}}%
\pgfpathlineto{\pgfqpoint{1.287965in}{0.508703in}}%
\pgfpathlineto{\pgfqpoint{1.284152in}{0.504073in}}%
\pgfpathlineto{\pgfqpoint{1.276878in}{0.495092in}}%
\pgfpathlineto{\pgfqpoint{1.268496in}{0.486209in}}%
\pgfpathlineto{\pgfqpoint{1.263920in}{0.481481in}}%
\pgfpathlineto{\pgfqpoint{1.252839in}{0.471194in}}%
\pgfpathlineto{\pgfqpoint{1.249015in}{0.467870in}}%
\pgfpathlineto{\pgfqpoint{1.237183in}{0.458237in}}%
\pgfpathlineto{\pgfqpoint{1.231744in}{0.454259in}}%
\pgfpathlineto{\pgfqpoint{1.221526in}{0.446971in}}%
\pgfpathlineto{\pgfqpoint{1.211195in}{0.440648in}}%
\pgfpathlineto{\pgfqpoint{1.205870in}{0.437333in}}%
\pgfpathlineto{\pgfqpoint{1.190213in}{0.429360in}}%
\pgfpathlineto{\pgfqpoint{1.184197in}{0.427036in}}%
\pgfpathlineto{\pgfqpoint{1.174556in}{0.423047in}}%
\pgfpathlineto{\pgfqpoint{1.158900in}{0.418701in}}%
\pgfpathlineto{\pgfqpoint{1.143243in}{0.416530in}}%
\pgfpathlineto{\pgfqpoint{1.127587in}{0.416530in}}%
\pgfpathlineto{\pgfqpoint{1.111930in}{0.418701in}}%
\pgfpathlineto{\pgfqpoint{1.096274in}{0.423047in}}%
\pgfpathlineto{\pgfqpoint{1.086633in}{0.427036in}}%
\pgfpathclose%
\pgfpathmoveto{\pgfqpoint{1.644253in}{0.370140in}}%
\pgfpathlineto{\pgfqpoint{1.659910in}{0.370140in}}%
\pgfpathlineto{\pgfqpoint{1.666313in}{0.372592in}}%
\pgfpathlineto{\pgfqpoint{1.675567in}{0.374784in}}%
\pgfpathlineto{\pgfqpoint{1.691223in}{0.382193in}}%
\pgfpathlineto{\pgfqpoint{1.696936in}{0.386203in}}%
\pgfpathlineto{\pgfqpoint{1.706880in}{0.391476in}}%
\pgfpathlineto{\pgfqpoint{1.718823in}{0.399814in}}%
\pgfpathlineto{\pgfqpoint{1.722536in}{0.401987in}}%
\pgfpathlineto{\pgfqpoint{1.738193in}{0.413200in}}%
\pgfpathlineto{\pgfqpoint{1.738466in}{0.413425in}}%
\pgfpathlineto{\pgfqpoint{1.753849in}{0.424882in}}%
\pgfpathlineto{\pgfqpoint{1.756451in}{0.427036in}}%
\pgfpathlineto{\pgfqpoint{1.769506in}{0.437333in}}%
\pgfpathlineto{\pgfqpoint{1.773426in}{0.440648in}}%
\pgfpathlineto{\pgfqpoint{1.785162in}{0.450517in}}%
\pgfpathlineto{\pgfqpoint{1.789466in}{0.454259in}}%
\pgfpathlineto{\pgfqpoint{1.800819in}{0.464462in}}%
\pgfpathlineto{\pgfqpoint{1.804632in}{0.467870in}}%
\pgfpathlineto{\pgfqpoint{1.816476in}{0.479219in}}%
\pgfpathlineto{\pgfqpoint{1.818954in}{0.481481in}}%
\pgfpathlineto{\pgfqpoint{1.832132in}{0.494854in}}%
\pgfpathlineto{\pgfqpoint{1.832392in}{0.495092in}}%
\pgfpathlineto{\pgfqpoint{1.845289in}{0.508703in}}%
\pgfpathlineto{\pgfqpoint{1.847789in}{0.511931in}}%
\pgfpathlineto{\pgfqpoint{1.857380in}{0.522314in}}%
\pgfpathlineto{\pgfqpoint{1.863445in}{0.530959in}}%
\pgfpathlineto{\pgfqpoint{1.868058in}{0.535925in}}%
\pgfpathlineto{\pgfqpoint{1.876580in}{0.549536in}}%
\pgfpathlineto{\pgfqpoint{1.879102in}{0.557581in}}%
\pgfpathlineto{\pgfqpoint{1.881923in}{0.563148in}}%
\pgfpathlineto{\pgfqpoint{1.881923in}{0.576759in}}%
\pgfpathlineto{\pgfqpoint{1.879102in}{0.582325in}}%
\pgfpathlineto{\pgfqpoint{1.876580in}{0.590370in}}%
\pgfpathlineto{\pgfqpoint{1.868058in}{0.603981in}}%
\pgfpathlineto{\pgfqpoint{1.863445in}{0.608947in}}%
\pgfpathlineto{\pgfqpoint{1.857380in}{0.617592in}}%
\pgfpathlineto{\pgfqpoint{1.847789in}{0.627975in}}%
\pgfpathlineto{\pgfqpoint{1.845289in}{0.631203in}}%
\pgfpathlineto{\pgfqpoint{1.832392in}{0.644814in}}%
\pgfpathlineto{\pgfqpoint{1.832132in}{0.645052in}}%
\pgfpathlineto{\pgfqpoint{1.818954in}{0.658425in}}%
\pgfpathlineto{\pgfqpoint{1.816476in}{0.660687in}}%
\pgfpathlineto{\pgfqpoint{1.804632in}{0.672036in}}%
\pgfpathlineto{\pgfqpoint{1.800819in}{0.675444in}}%
\pgfpathlineto{\pgfqpoint{1.789466in}{0.685648in}}%
\pgfpathlineto{\pgfqpoint{1.785162in}{0.689389in}}%
\pgfpathlineto{\pgfqpoint{1.773426in}{0.699259in}}%
\pgfpathlineto{\pgfqpoint{1.769506in}{0.702573in}}%
\pgfpathlineto{\pgfqpoint{1.756451in}{0.712870in}}%
\pgfpathlineto{\pgfqpoint{1.753849in}{0.715025in}}%
\pgfpathlineto{\pgfqpoint{1.738466in}{0.726481in}}%
\pgfpathlineto{\pgfqpoint{1.738193in}{0.726706in}}%
\pgfpathlineto{\pgfqpoint{1.722536in}{0.737919in}}%
\pgfpathlineto{\pgfqpoint{1.718823in}{0.740092in}}%
\pgfpathlineto{\pgfqpoint{1.706880in}{0.748430in}}%
\pgfpathlineto{\pgfqpoint{1.696936in}{0.753703in}}%
\pgfpathlineto{\pgfqpoint{1.691223in}{0.757713in}}%
\pgfpathlineto{\pgfqpoint{1.675567in}{0.765122in}}%
\pgfpathlineto{\pgfqpoint{1.666313in}{0.767314in}}%
\pgfpathlineto{\pgfqpoint{1.659910in}{0.769767in}}%
\pgfpathlineto{\pgfqpoint{1.644253in}{0.769767in}}%
\pgfpathlineto{\pgfqpoint{1.637850in}{0.767314in}}%
\pgfpathlineto{\pgfqpoint{1.628597in}{0.765122in}}%
\pgfpathlineto{\pgfqpoint{1.612940in}{0.757713in}}%
\pgfpathlineto{\pgfqpoint{1.607227in}{0.753703in}}%
\pgfpathlineto{\pgfqpoint{1.597284in}{0.748430in}}%
\pgfpathlineto{\pgfqpoint{1.585340in}{0.740092in}}%
\pgfpathlineto{\pgfqpoint{1.581627in}{0.737919in}}%
\pgfpathlineto{\pgfqpoint{1.565971in}{0.726706in}}%
\pgfpathlineto{\pgfqpoint{1.565697in}{0.726481in}}%
\pgfpathlineto{\pgfqpoint{1.550314in}{0.715025in}}%
\pgfpathlineto{\pgfqpoint{1.547712in}{0.712870in}}%
\pgfpathlineto{\pgfqpoint{1.534657in}{0.702573in}}%
\pgfpathlineto{\pgfqpoint{1.530737in}{0.699259in}}%
\pgfpathlineto{\pgfqpoint{1.519001in}{0.689389in}}%
\pgfpathlineto{\pgfqpoint{1.514698in}{0.685648in}}%
\pgfpathlineto{\pgfqpoint{1.503344in}{0.675444in}}%
\pgfpathlineto{\pgfqpoint{1.499532in}{0.672036in}}%
\pgfpathlineto{\pgfqpoint{1.487688in}{0.660687in}}%
\pgfpathlineto{\pgfqpoint{1.485209in}{0.658425in}}%
\pgfpathlineto{\pgfqpoint{1.472031in}{0.645052in}}%
\pgfpathlineto{\pgfqpoint{1.471772in}{0.644814in}}%
\pgfpathlineto{\pgfqpoint{1.458874in}{0.631203in}}%
\pgfpathlineto{\pgfqpoint{1.456375in}{0.627975in}}%
\pgfpathlineto{\pgfqpoint{1.446783in}{0.617592in}}%
\pgfpathlineto{\pgfqpoint{1.440718in}{0.608947in}}%
\pgfpathlineto{\pgfqpoint{1.436105in}{0.603981in}}%
\pgfpathlineto{\pgfqpoint{1.427583in}{0.590370in}}%
\pgfpathlineto{\pgfqpoint{1.425061in}{0.582325in}}%
\pgfpathlineto{\pgfqpoint{1.422241in}{0.576759in}}%
\pgfpathlineto{\pgfqpoint{1.422241in}{0.563148in}}%
\pgfpathlineto{\pgfqpoint{1.425061in}{0.557581in}}%
\pgfpathlineto{\pgfqpoint{1.427583in}{0.549536in}}%
\pgfpathlineto{\pgfqpoint{1.436105in}{0.535925in}}%
\pgfpathlineto{\pgfqpoint{1.440718in}{0.530959in}}%
\pgfpathlineto{\pgfqpoint{1.446783in}{0.522314in}}%
\pgfpathlineto{\pgfqpoint{1.456375in}{0.511931in}}%
\pgfpathlineto{\pgfqpoint{1.458874in}{0.508703in}}%
\pgfpathlineto{\pgfqpoint{1.471772in}{0.495092in}}%
\pgfpathlineto{\pgfqpoint{1.472031in}{0.494854in}}%
\pgfpathlineto{\pgfqpoint{1.485209in}{0.481481in}}%
\pgfpathlineto{\pgfqpoint{1.487688in}{0.479219in}}%
\pgfpathlineto{\pgfqpoint{1.499532in}{0.467870in}}%
\pgfpathlineto{\pgfqpoint{1.503344in}{0.464462in}}%
\pgfpathlineto{\pgfqpoint{1.514698in}{0.454259in}}%
\pgfpathlineto{\pgfqpoint{1.519001in}{0.450517in}}%
\pgfpathlineto{\pgfqpoint{1.530737in}{0.440648in}}%
\pgfpathlineto{\pgfqpoint{1.534657in}{0.437333in}}%
\pgfpathlineto{\pgfqpoint{1.547712in}{0.427036in}}%
\pgfpathlineto{\pgfqpoint{1.550314in}{0.424882in}}%
\pgfpathlineto{\pgfqpoint{1.565697in}{0.413425in}}%
\pgfpathlineto{\pgfqpoint{1.565971in}{0.413200in}}%
\pgfpathlineto{\pgfqpoint{1.581627in}{0.401987in}}%
\pgfpathlineto{\pgfqpoint{1.585340in}{0.399814in}}%
\pgfpathlineto{\pgfqpoint{1.597284in}{0.391476in}}%
\pgfpathlineto{\pgfqpoint{1.607227in}{0.386203in}}%
\pgfpathlineto{\pgfqpoint{1.612940in}{0.382193in}}%
\pgfpathlineto{\pgfqpoint{1.628597in}{0.374784in}}%
\pgfpathlineto{\pgfqpoint{1.637850in}{0.372592in}}%
\pgfpathlineto{\pgfqpoint{1.644253in}{0.370140in}}%
\pgfpathclose%
\pgfpathmoveto{\pgfqpoint{1.603299in}{0.427036in}}%
\pgfpathlineto{\pgfqpoint{1.597284in}{0.429360in}}%
\pgfpathlineto{\pgfqpoint{1.581627in}{0.437333in}}%
\pgfpathlineto{\pgfqpoint{1.576301in}{0.440648in}}%
\pgfpathlineto{\pgfqpoint{1.565971in}{0.446971in}}%
\pgfpathlineto{\pgfqpoint{1.555753in}{0.454259in}}%
\pgfpathlineto{\pgfqpoint{1.550314in}{0.458237in}}%
\pgfpathlineto{\pgfqpoint{1.538481in}{0.467870in}}%
\pgfpathlineto{\pgfqpoint{1.534657in}{0.471194in}}%
\pgfpathlineto{\pgfqpoint{1.523577in}{0.481481in}}%
\pgfpathlineto{\pgfqpoint{1.519001in}{0.486209in}}%
\pgfpathlineto{\pgfqpoint{1.510618in}{0.495092in}}%
\pgfpathlineto{\pgfqpoint{1.503344in}{0.504073in}}%
\pgfpathlineto{\pgfqpoint{1.499531in}{0.508703in}}%
\pgfpathlineto{\pgfqpoint{1.490361in}{0.522314in}}%
\pgfpathlineto{\pgfqpoint{1.487688in}{0.527544in}}%
\pgfpathlineto{\pgfqpoint{1.483099in}{0.535925in}}%
\pgfpathlineto{\pgfqpoint{1.478099in}{0.549536in}}%
\pgfpathlineto{\pgfqpoint{1.475603in}{0.563148in}}%
\pgfpathlineto{\pgfqpoint{1.475603in}{0.576759in}}%
\pgfpathlineto{\pgfqpoint{1.478099in}{0.590370in}}%
\pgfpathlineto{\pgfqpoint{1.483099in}{0.603981in}}%
\pgfpathlineto{\pgfqpoint{1.487688in}{0.612362in}}%
\pgfpathlineto{\pgfqpoint{1.490361in}{0.617592in}}%
\pgfpathlineto{\pgfqpoint{1.499531in}{0.631203in}}%
\pgfpathlineto{\pgfqpoint{1.503344in}{0.635833in}}%
\pgfpathlineto{\pgfqpoint{1.510618in}{0.644814in}}%
\pgfpathlineto{\pgfqpoint{1.519001in}{0.653697in}}%
\pgfpathlineto{\pgfqpoint{1.523577in}{0.658425in}}%
\pgfpathlineto{\pgfqpoint{1.534657in}{0.668712in}}%
\pgfpathlineto{\pgfqpoint{1.538481in}{0.672036in}}%
\pgfpathlineto{\pgfqpoint{1.550314in}{0.681669in}}%
\pgfpathlineto{\pgfqpoint{1.555753in}{0.685648in}}%
\pgfpathlineto{\pgfqpoint{1.565971in}{0.692935in}}%
\pgfpathlineto{\pgfqpoint{1.576301in}{0.699259in}}%
\pgfpathlineto{\pgfqpoint{1.581627in}{0.702573in}}%
\pgfpathlineto{\pgfqpoint{1.597284in}{0.710546in}}%
\pgfpathlineto{\pgfqpoint{1.603299in}{0.712870in}}%
\pgfpathlineto{\pgfqpoint{1.612940in}{0.716859in}}%
\pgfpathlineto{\pgfqpoint{1.628597in}{0.721205in}}%
\pgfpathlineto{\pgfqpoint{1.644253in}{0.723376in}}%
\pgfpathlineto{\pgfqpoint{1.659910in}{0.723376in}}%
\pgfpathlineto{\pgfqpoint{1.675567in}{0.721205in}}%
\pgfpathlineto{\pgfqpoint{1.691223in}{0.716859in}}%
\pgfpathlineto{\pgfqpoint{1.700864in}{0.712870in}}%
\pgfpathlineto{\pgfqpoint{1.706880in}{0.710546in}}%
\pgfpathlineto{\pgfqpoint{1.722536in}{0.702573in}}%
\pgfpathlineto{\pgfqpoint{1.727862in}{0.699259in}}%
\pgfpathlineto{\pgfqpoint{1.738193in}{0.692935in}}%
\pgfpathlineto{\pgfqpoint{1.748411in}{0.685648in}}%
\pgfpathlineto{\pgfqpoint{1.753849in}{0.681669in}}%
\pgfpathlineto{\pgfqpoint{1.765682in}{0.672036in}}%
\pgfpathlineto{\pgfqpoint{1.769506in}{0.668712in}}%
\pgfpathlineto{\pgfqpoint{1.780587in}{0.658425in}}%
\pgfpathlineto{\pgfqpoint{1.785162in}{0.653697in}}%
\pgfpathlineto{\pgfqpoint{1.793545in}{0.644814in}}%
\pgfpathlineto{\pgfqpoint{1.800819in}{0.635833in}}%
\pgfpathlineto{\pgfqpoint{1.804632in}{0.631203in}}%
\pgfpathlineto{\pgfqpoint{1.813802in}{0.617592in}}%
\pgfpathlineto{\pgfqpoint{1.816476in}{0.612362in}}%
\pgfpathlineto{\pgfqpoint{1.821065in}{0.603981in}}%
\pgfpathlineto{\pgfqpoint{1.826064in}{0.590370in}}%
\pgfpathlineto{\pgfqpoint{1.828560in}{0.576759in}}%
\pgfpathlineto{\pgfqpoint{1.828560in}{0.563148in}}%
\pgfpathlineto{\pgfqpoint{1.826064in}{0.549536in}}%
\pgfpathlineto{\pgfqpoint{1.821065in}{0.535925in}}%
\pgfpathlineto{\pgfqpoint{1.816476in}{0.527544in}}%
\pgfpathlineto{\pgfqpoint{1.813802in}{0.522314in}}%
\pgfpathlineto{\pgfqpoint{1.804632in}{0.508703in}}%
\pgfpathlineto{\pgfqpoint{1.800819in}{0.504073in}}%
\pgfpathlineto{\pgfqpoint{1.793545in}{0.495092in}}%
\pgfpathlineto{\pgfqpoint{1.785162in}{0.486209in}}%
\pgfpathlineto{\pgfqpoint{1.780587in}{0.481481in}}%
\pgfpathlineto{\pgfqpoint{1.769506in}{0.471194in}}%
\pgfpathlineto{\pgfqpoint{1.765682in}{0.467870in}}%
\pgfpathlineto{\pgfqpoint{1.753849in}{0.458237in}}%
\pgfpathlineto{\pgfqpoint{1.748411in}{0.454259in}}%
\pgfpathlineto{\pgfqpoint{1.738193in}{0.446971in}}%
\pgfpathlineto{\pgfqpoint{1.727862in}{0.440648in}}%
\pgfpathlineto{\pgfqpoint{1.722536in}{0.437333in}}%
\pgfpathlineto{\pgfqpoint{1.706880in}{0.429360in}}%
\pgfpathlineto{\pgfqpoint{1.700864in}{0.427036in}}%
\pgfpathlineto{\pgfqpoint{1.691223in}{0.423047in}}%
\pgfpathlineto{\pgfqpoint{1.675567in}{0.418701in}}%
\pgfpathlineto{\pgfqpoint{1.659910in}{0.416530in}}%
\pgfpathlineto{\pgfqpoint{1.644253in}{0.416530in}}%
\pgfpathlineto{\pgfqpoint{1.628597in}{0.418701in}}%
\pgfpathlineto{\pgfqpoint{1.612940in}{0.423047in}}%
\pgfpathlineto{\pgfqpoint{1.603299in}{0.427036in}}%
\pgfpathclose%
\pgfpathmoveto{\pgfqpoint{0.610920in}{0.819306in}}%
\pgfpathlineto{\pgfqpoint{0.626577in}{0.819306in}}%
\pgfpathlineto{\pgfqpoint{0.632980in}{0.821759in}}%
\pgfpathlineto{\pgfqpoint{0.642233in}{0.823951in}}%
\pgfpathlineto{\pgfqpoint{0.657890in}{0.831359in}}%
\pgfpathlineto{\pgfqpoint{0.663603in}{0.835370in}}%
\pgfpathlineto{\pgfqpoint{0.673546in}{0.840643in}}%
\pgfpathlineto{\pgfqpoint{0.685490in}{0.848981in}}%
\pgfpathlineto{\pgfqpoint{0.689203in}{0.851154in}}%
\pgfpathlineto{\pgfqpoint{0.704859in}{0.862366in}}%
\pgfpathlineto{\pgfqpoint{0.705133in}{0.862592in}}%
\pgfpathlineto{\pgfqpoint{0.720516in}{0.874048in}}%
\pgfpathlineto{\pgfqpoint{0.723118in}{0.876203in}}%
\pgfpathlineto{\pgfqpoint{0.736173in}{0.886500in}}%
\pgfpathlineto{\pgfqpoint{0.740093in}{0.889814in}}%
\pgfpathlineto{\pgfqpoint{0.751829in}{0.899684in}}%
\pgfpathlineto{\pgfqpoint{0.756132in}{0.903425in}}%
\pgfpathlineto{\pgfqpoint{0.767486in}{0.913628in}}%
\pgfpathlineto{\pgfqpoint{0.771298in}{0.917036in}}%
\pgfpathlineto{\pgfqpoint{0.783142in}{0.928386in}}%
\pgfpathlineto{\pgfqpoint{0.785621in}{0.930648in}}%
\pgfpathlineto{\pgfqpoint{0.798799in}{0.944021in}}%
\pgfpathlineto{\pgfqpoint{0.799058in}{0.944259in}}%
\pgfpathlineto{\pgfqpoint{0.811956in}{0.957870in}}%
\pgfpathlineto{\pgfqpoint{0.814455in}{0.961098in}}%
\pgfpathlineto{\pgfqpoint{0.824047in}{0.971481in}}%
\pgfpathlineto{\pgfqpoint{0.830112in}{0.980126in}}%
\pgfpathlineto{\pgfqpoint{0.834725in}{0.985092in}}%
\pgfpathlineto{\pgfqpoint{0.843247in}{0.998703in}}%
\pgfpathlineto{\pgfqpoint{0.845769in}{1.006747in}}%
\pgfpathlineto{\pgfqpoint{0.848589in}{1.012314in}}%
\pgfpathlineto{\pgfqpoint{0.848589in}{1.025925in}}%
\pgfpathlineto{\pgfqpoint{0.845769in}{1.031492in}}%
\pgfpathlineto{\pgfqpoint{0.843247in}{1.039536in}}%
\pgfpathlineto{\pgfqpoint{0.834725in}{1.053148in}}%
\pgfpathlineto{\pgfqpoint{0.830112in}{1.058114in}}%
\pgfpathlineto{\pgfqpoint{0.824047in}{1.066759in}}%
\pgfpathlineto{\pgfqpoint{0.814455in}{1.077142in}}%
\pgfpathlineto{\pgfqpoint{0.811956in}{1.080370in}}%
\pgfpathlineto{\pgfqpoint{0.799058in}{1.093981in}}%
\pgfpathlineto{\pgfqpoint{0.798799in}{1.094219in}}%
\pgfpathlineto{\pgfqpoint{0.785621in}{1.107592in}}%
\pgfpathlineto{\pgfqpoint{0.783142in}{1.109854in}}%
\pgfpathlineto{\pgfqpoint{0.771298in}{1.121203in}}%
\pgfpathlineto{\pgfqpoint{0.767486in}{1.124611in}}%
\pgfpathlineto{\pgfqpoint{0.756132in}{1.134814in}}%
\pgfpathlineto{\pgfqpoint{0.751829in}{1.138555in}}%
\pgfpathlineto{\pgfqpoint{0.740093in}{1.148425in}}%
\pgfpathlineto{\pgfqpoint{0.736173in}{1.151740in}}%
\pgfpathlineto{\pgfqpoint{0.723118in}{1.162036in}}%
\pgfpathlineto{\pgfqpoint{0.720516in}{1.164191in}}%
\pgfpathlineto{\pgfqpoint{0.705133in}{1.175647in}}%
\pgfpathlineto{\pgfqpoint{0.704859in}{1.175873in}}%
\pgfpathlineto{\pgfqpoint{0.689203in}{1.187085in}}%
\pgfpathlineto{\pgfqpoint{0.685490in}{1.189259in}}%
\pgfpathlineto{\pgfqpoint{0.673546in}{1.197597in}}%
\pgfpathlineto{\pgfqpoint{0.663603in}{1.202870in}}%
\pgfpathlineto{\pgfqpoint{0.657890in}{1.206880in}}%
\pgfpathlineto{\pgfqpoint{0.642233in}{1.214288in}}%
\pgfpathlineto{\pgfqpoint{0.632980in}{1.216481in}}%
\pgfpathlineto{\pgfqpoint{0.626577in}{1.218933in}}%
\pgfpathlineto{\pgfqpoint{0.610920in}{1.218933in}}%
\pgfpathlineto{\pgfqpoint{0.604517in}{1.216481in}}%
\pgfpathlineto{\pgfqpoint{0.595263in}{1.214288in}}%
\pgfpathlineto{\pgfqpoint{0.579607in}{1.206880in}}%
\pgfpathlineto{\pgfqpoint{0.573894in}{1.202870in}}%
\pgfpathlineto{\pgfqpoint{0.563950in}{1.197597in}}%
\pgfpathlineto{\pgfqpoint{0.552007in}{1.189259in}}%
\pgfpathlineto{\pgfqpoint{0.548294in}{1.187085in}}%
\pgfpathlineto{\pgfqpoint{0.532637in}{1.175873in}}%
\pgfpathlineto{\pgfqpoint{0.532364in}{1.175647in}}%
\pgfpathlineto{\pgfqpoint{0.516981in}{1.164191in}}%
\pgfpathlineto{\pgfqpoint{0.514379in}{1.162036in}}%
\pgfpathlineto{\pgfqpoint{0.501324in}{1.151740in}}%
\pgfpathlineto{\pgfqpoint{0.497404in}{1.148425in}}%
\pgfpathlineto{\pgfqpoint{0.485668in}{1.138555in}}%
\pgfpathlineto{\pgfqpoint{0.481364in}{1.134814in}}%
\pgfpathlineto{\pgfqpoint{0.470011in}{1.124611in}}%
\pgfpathlineto{\pgfqpoint{0.466198in}{1.121203in}}%
\pgfpathlineto{\pgfqpoint{0.454354in}{1.109854in}}%
\pgfpathlineto{\pgfqpoint{0.451876in}{1.107592in}}%
\pgfpathlineto{\pgfqpoint{0.438698in}{1.094219in}}%
\pgfpathlineto{\pgfqpoint{0.438438in}{1.093981in}}%
\pgfpathlineto{\pgfqpoint{0.425541in}{1.080370in}}%
\pgfpathlineto{\pgfqpoint{0.423041in}{1.077142in}}%
\pgfpathlineto{\pgfqpoint{0.413450in}{1.066759in}}%
\pgfpathlineto{\pgfqpoint{0.407385in}{1.058114in}}%
\pgfpathlineto{\pgfqpoint{0.402772in}{1.053148in}}%
\pgfpathlineto{\pgfqpoint{0.394250in}{1.039536in}}%
\pgfpathlineto{\pgfqpoint{0.391728in}{1.031492in}}%
\pgfpathlineto{\pgfqpoint{0.388907in}{1.025925in}}%
\pgfpathlineto{\pgfqpoint{0.388907in}{1.012314in}}%
\pgfpathlineto{\pgfqpoint{0.391728in}{1.006747in}}%
\pgfpathlineto{\pgfqpoint{0.394250in}{0.998703in}}%
\pgfpathlineto{\pgfqpoint{0.402772in}{0.985092in}}%
\pgfpathlineto{\pgfqpoint{0.407385in}{0.980126in}}%
\pgfpathlineto{\pgfqpoint{0.413450in}{0.971481in}}%
\pgfpathlineto{\pgfqpoint{0.423041in}{0.961098in}}%
\pgfpathlineto{\pgfqpoint{0.425541in}{0.957870in}}%
\pgfpathlineto{\pgfqpoint{0.438438in}{0.944259in}}%
\pgfpathlineto{\pgfqpoint{0.438698in}{0.944021in}}%
\pgfpathlineto{\pgfqpoint{0.451876in}{0.930648in}}%
\pgfpathlineto{\pgfqpoint{0.454354in}{0.928386in}}%
\pgfpathlineto{\pgfqpoint{0.466198in}{0.917036in}}%
\pgfpathlineto{\pgfqpoint{0.470011in}{0.913628in}}%
\pgfpathlineto{\pgfqpoint{0.481364in}{0.903425in}}%
\pgfpathlineto{\pgfqpoint{0.485668in}{0.899684in}}%
\pgfpathlineto{\pgfqpoint{0.497404in}{0.889814in}}%
\pgfpathlineto{\pgfqpoint{0.501324in}{0.886500in}}%
\pgfpathlineto{\pgfqpoint{0.514379in}{0.876203in}}%
\pgfpathlineto{\pgfqpoint{0.516981in}{0.874048in}}%
\pgfpathlineto{\pgfqpoint{0.532364in}{0.862592in}}%
\pgfpathlineto{\pgfqpoint{0.532637in}{0.862366in}}%
\pgfpathlineto{\pgfqpoint{0.548294in}{0.851154in}}%
\pgfpathlineto{\pgfqpoint{0.552007in}{0.848981in}}%
\pgfpathlineto{\pgfqpoint{0.563950in}{0.840643in}}%
\pgfpathlineto{\pgfqpoint{0.573894in}{0.835370in}}%
\pgfpathlineto{\pgfqpoint{0.579607in}{0.831359in}}%
\pgfpathlineto{\pgfqpoint{0.595263in}{0.823951in}}%
\pgfpathlineto{\pgfqpoint{0.604517in}{0.821759in}}%
\pgfpathlineto{\pgfqpoint{0.610920in}{0.819306in}}%
\pgfpathclose%
\pgfpathmoveto{\pgfqpoint{0.569966in}{0.876203in}}%
\pgfpathlineto{\pgfqpoint{0.563950in}{0.878527in}}%
\pgfpathlineto{\pgfqpoint{0.548294in}{0.886499in}}%
\pgfpathlineto{\pgfqpoint{0.542968in}{0.889814in}}%
\pgfpathlineto{\pgfqpoint{0.532637in}{0.896138in}}%
\pgfpathlineto{\pgfqpoint{0.522419in}{0.903425in}}%
\pgfpathlineto{\pgfqpoint{0.516981in}{0.907403in}}%
\pgfpathlineto{\pgfqpoint{0.505148in}{0.917036in}}%
\pgfpathlineto{\pgfqpoint{0.501324in}{0.920361in}}%
\pgfpathlineto{\pgfqpoint{0.490243in}{0.930648in}}%
\pgfpathlineto{\pgfqpoint{0.485668in}{0.935376in}}%
\pgfpathlineto{\pgfqpoint{0.477285in}{0.944259in}}%
\pgfpathlineto{\pgfqpoint{0.470011in}{0.953240in}}%
\pgfpathlineto{\pgfqpoint{0.466198in}{0.957870in}}%
\pgfpathlineto{\pgfqpoint{0.457028in}{0.971481in}}%
\pgfpathlineto{\pgfqpoint{0.454354in}{0.976711in}}%
\pgfpathlineto{\pgfqpoint{0.449765in}{0.985092in}}%
\pgfpathlineto{\pgfqpoint{0.444766in}{0.998703in}}%
\pgfpathlineto{\pgfqpoint{0.442270in}{1.012314in}}%
\pgfpathlineto{\pgfqpoint{0.442270in}{1.025925in}}%
\pgfpathlineto{\pgfqpoint{0.444766in}{1.039536in}}%
\pgfpathlineto{\pgfqpoint{0.449765in}{1.053148in}}%
\pgfpathlineto{\pgfqpoint{0.454354in}{1.061529in}}%
\pgfpathlineto{\pgfqpoint{0.457028in}{1.066759in}}%
\pgfpathlineto{\pgfqpoint{0.466198in}{1.080370in}}%
\pgfpathlineto{\pgfqpoint{0.470011in}{1.085000in}}%
\pgfpathlineto{\pgfqpoint{0.477285in}{1.093981in}}%
\pgfpathlineto{\pgfqpoint{0.485668in}{1.102864in}}%
\pgfpathlineto{\pgfqpoint{0.490243in}{1.107592in}}%
\pgfpathlineto{\pgfqpoint{0.501324in}{1.117879in}}%
\pgfpathlineto{\pgfqpoint{0.505148in}{1.121203in}}%
\pgfpathlineto{\pgfqpoint{0.516981in}{1.130836in}}%
\pgfpathlineto{\pgfqpoint{0.522419in}{1.134814in}}%
\pgfpathlineto{\pgfqpoint{0.532637in}{1.142101in}}%
\pgfpathlineto{\pgfqpoint{0.542968in}{1.148425in}}%
\pgfpathlineto{\pgfqpoint{0.548294in}{1.151740in}}%
\pgfpathlineto{\pgfqpoint{0.563950in}{1.159712in}}%
\pgfpathlineto{\pgfqpoint{0.569966in}{1.162036in}}%
\pgfpathlineto{\pgfqpoint{0.579607in}{1.166026in}}%
\pgfpathlineto{\pgfqpoint{0.595263in}{1.170372in}}%
\pgfpathlineto{\pgfqpoint{0.610920in}{1.172542in}}%
\pgfpathlineto{\pgfqpoint{0.626577in}{1.172542in}}%
\pgfpathlineto{\pgfqpoint{0.642233in}{1.170372in}}%
\pgfpathlineto{\pgfqpoint{0.657890in}{1.166026in}}%
\pgfpathlineto{\pgfqpoint{0.667531in}{1.162036in}}%
\pgfpathlineto{\pgfqpoint{0.673546in}{1.159712in}}%
\pgfpathlineto{\pgfqpoint{0.689203in}{1.151740in}}%
\pgfpathlineto{\pgfqpoint{0.694529in}{1.148425in}}%
\pgfpathlineto{\pgfqpoint{0.704859in}{1.142101in}}%
\pgfpathlineto{\pgfqpoint{0.715077in}{1.134814in}}%
\pgfpathlineto{\pgfqpoint{0.720516in}{1.130836in}}%
\pgfpathlineto{\pgfqpoint{0.732349in}{1.121203in}}%
\pgfpathlineto{\pgfqpoint{0.736173in}{1.117879in}}%
\pgfpathlineto{\pgfqpoint{0.747253in}{1.107592in}}%
\pgfpathlineto{\pgfqpoint{0.751829in}{1.102864in}}%
\pgfpathlineto{\pgfqpoint{0.760212in}{1.093981in}}%
\pgfpathlineto{\pgfqpoint{0.767486in}{1.085000in}}%
\pgfpathlineto{\pgfqpoint{0.771299in}{1.080370in}}%
\pgfpathlineto{\pgfqpoint{0.780469in}{1.066759in}}%
\pgfpathlineto{\pgfqpoint{0.783142in}{1.061529in}}%
\pgfpathlineto{\pgfqpoint{0.787731in}{1.053148in}}%
\pgfpathlineto{\pgfqpoint{0.792731in}{1.039536in}}%
\pgfpathlineto{\pgfqpoint{0.795227in}{1.025925in}}%
\pgfpathlineto{\pgfqpoint{0.795227in}{1.012314in}}%
\pgfpathlineto{\pgfqpoint{0.792731in}{0.998703in}}%
\pgfpathlineto{\pgfqpoint{0.787731in}{0.985092in}}%
\pgfpathlineto{\pgfqpoint{0.783142in}{0.976711in}}%
\pgfpathlineto{\pgfqpoint{0.780469in}{0.971481in}}%
\pgfpathlineto{\pgfqpoint{0.771299in}{0.957870in}}%
\pgfpathlineto{\pgfqpoint{0.767486in}{0.953240in}}%
\pgfpathlineto{\pgfqpoint{0.760212in}{0.944259in}}%
\pgfpathlineto{\pgfqpoint{0.751829in}{0.935376in}}%
\pgfpathlineto{\pgfqpoint{0.747253in}{0.930648in}}%
\pgfpathlineto{\pgfqpoint{0.736173in}{0.920361in}}%
\pgfpathlineto{\pgfqpoint{0.732349in}{0.917036in}}%
\pgfpathlineto{\pgfqpoint{0.720516in}{0.907403in}}%
\pgfpathlineto{\pgfqpoint{0.715077in}{0.903425in}}%
\pgfpathlineto{\pgfqpoint{0.704859in}{0.896138in}}%
\pgfpathlineto{\pgfqpoint{0.694529in}{0.889814in}}%
\pgfpathlineto{\pgfqpoint{0.689203in}{0.886499in}}%
\pgfpathlineto{\pgfqpoint{0.673546in}{0.878527in}}%
\pgfpathlineto{\pgfqpoint{0.667531in}{0.876203in}}%
\pgfpathlineto{\pgfqpoint{0.657890in}{0.872213in}}%
\pgfpathlineto{\pgfqpoint{0.642233in}{0.867867in}}%
\pgfpathlineto{\pgfqpoint{0.626577in}{0.865697in}}%
\pgfpathlineto{\pgfqpoint{0.610920in}{0.865697in}}%
\pgfpathlineto{\pgfqpoint{0.595263in}{0.867867in}}%
\pgfpathlineto{\pgfqpoint{0.579607in}{0.872213in}}%
\pgfpathlineto{\pgfqpoint{0.569966in}{0.876203in}}%
\pgfpathclose%
\pgfpathmoveto{\pgfqpoint{1.127587in}{0.819306in}}%
\pgfpathlineto{\pgfqpoint{1.143243in}{0.819306in}}%
\pgfpathlineto{\pgfqpoint{1.149647in}{0.821759in}}%
\pgfpathlineto{\pgfqpoint{1.158900in}{0.823951in}}%
\pgfpathlineto{\pgfqpoint{1.174556in}{0.831359in}}%
\pgfpathlineto{\pgfqpoint{1.180269in}{0.835370in}}%
\pgfpathlineto{\pgfqpoint{1.190213in}{0.840643in}}%
\pgfpathlineto{\pgfqpoint{1.202156in}{0.848981in}}%
\pgfpathlineto{\pgfqpoint{1.205870in}{0.851154in}}%
\pgfpathlineto{\pgfqpoint{1.221526in}{0.862366in}}%
\pgfpathlineto{\pgfqpoint{1.221800in}{0.862592in}}%
\pgfpathlineto{\pgfqpoint{1.237183in}{0.874048in}}%
\pgfpathlineto{\pgfqpoint{1.239784in}{0.876203in}}%
\pgfpathlineto{\pgfqpoint{1.252839in}{0.886500in}}%
\pgfpathlineto{\pgfqpoint{1.256759in}{0.889814in}}%
\pgfpathlineto{\pgfqpoint{1.268496in}{0.899684in}}%
\pgfpathlineto{\pgfqpoint{1.272799in}{0.903425in}}%
\pgfpathlineto{\pgfqpoint{1.284152in}{0.913628in}}%
\pgfpathlineto{\pgfqpoint{1.287965in}{0.917036in}}%
\pgfpathlineto{\pgfqpoint{1.299809in}{0.928386in}}%
\pgfpathlineto{\pgfqpoint{1.302288in}{0.930648in}}%
\pgfpathlineto{\pgfqpoint{1.315466in}{0.944021in}}%
\pgfpathlineto{\pgfqpoint{1.315725in}{0.944259in}}%
\pgfpathlineto{\pgfqpoint{1.328622in}{0.957870in}}%
\pgfpathlineto{\pgfqpoint{1.331122in}{0.961098in}}%
\pgfpathlineto{\pgfqpoint{1.340713in}{0.971481in}}%
\pgfpathlineto{\pgfqpoint{1.346779in}{0.980126in}}%
\pgfpathlineto{\pgfqpoint{1.351392in}{0.985092in}}%
\pgfpathlineto{\pgfqpoint{1.359913in}{0.998703in}}%
\pgfpathlineto{\pgfqpoint{1.362435in}{1.006747in}}%
\pgfpathlineto{\pgfqpoint{1.365256in}{1.012314in}}%
\pgfpathlineto{\pgfqpoint{1.365256in}{1.025925in}}%
\pgfpathlineto{\pgfqpoint{1.362435in}{1.031492in}}%
\pgfpathlineto{\pgfqpoint{1.359913in}{1.039536in}}%
\pgfpathlineto{\pgfqpoint{1.351392in}{1.053148in}}%
\pgfpathlineto{\pgfqpoint{1.346779in}{1.058114in}}%
\pgfpathlineto{\pgfqpoint{1.340713in}{1.066759in}}%
\pgfpathlineto{\pgfqpoint{1.331122in}{1.077142in}}%
\pgfpathlineto{\pgfqpoint{1.328622in}{1.080370in}}%
\pgfpathlineto{\pgfqpoint{1.315725in}{1.093981in}}%
\pgfpathlineto{\pgfqpoint{1.315466in}{1.094219in}}%
\pgfpathlineto{\pgfqpoint{1.302288in}{1.107592in}}%
\pgfpathlineto{\pgfqpoint{1.299809in}{1.109854in}}%
\pgfpathlineto{\pgfqpoint{1.287965in}{1.121203in}}%
\pgfpathlineto{\pgfqpoint{1.284152in}{1.124611in}}%
\pgfpathlineto{\pgfqpoint{1.272799in}{1.134814in}}%
\pgfpathlineto{\pgfqpoint{1.268496in}{1.138555in}}%
\pgfpathlineto{\pgfqpoint{1.256759in}{1.148425in}}%
\pgfpathlineto{\pgfqpoint{1.252839in}{1.151740in}}%
\pgfpathlineto{\pgfqpoint{1.239784in}{1.162036in}}%
\pgfpathlineto{\pgfqpoint{1.237183in}{1.164191in}}%
\pgfpathlineto{\pgfqpoint{1.221800in}{1.175647in}}%
\pgfpathlineto{\pgfqpoint{1.221526in}{1.175873in}}%
\pgfpathlineto{\pgfqpoint{1.205870in}{1.187085in}}%
\pgfpathlineto{\pgfqpoint{1.202156in}{1.189259in}}%
\pgfpathlineto{\pgfqpoint{1.190213in}{1.197597in}}%
\pgfpathlineto{\pgfqpoint{1.180269in}{1.202870in}}%
\pgfpathlineto{\pgfqpoint{1.174556in}{1.206880in}}%
\pgfpathlineto{\pgfqpoint{1.158900in}{1.214288in}}%
\pgfpathlineto{\pgfqpoint{1.149647in}{1.216481in}}%
\pgfpathlineto{\pgfqpoint{1.143243in}{1.218933in}}%
\pgfpathlineto{\pgfqpoint{1.127587in}{1.218933in}}%
\pgfpathlineto{\pgfqpoint{1.121183in}{1.216481in}}%
\pgfpathlineto{\pgfqpoint{1.111930in}{1.214288in}}%
\pgfpathlineto{\pgfqpoint{1.096274in}{1.206880in}}%
\pgfpathlineto{\pgfqpoint{1.090561in}{1.202870in}}%
\pgfpathlineto{\pgfqpoint{1.080617in}{1.197597in}}%
\pgfpathlineto{\pgfqpoint{1.068674in}{1.189259in}}%
\pgfpathlineto{\pgfqpoint{1.064960in}{1.187085in}}%
\pgfpathlineto{\pgfqpoint{1.049304in}{1.175873in}}%
\pgfpathlineto{\pgfqpoint{1.049030in}{1.175647in}}%
\pgfpathlineto{\pgfqpoint{1.033647in}{1.164191in}}%
\pgfpathlineto{\pgfqpoint{1.031046in}{1.162036in}}%
\pgfpathlineto{\pgfqpoint{1.017991in}{1.151740in}}%
\pgfpathlineto{\pgfqpoint{1.014071in}{1.148425in}}%
\pgfpathlineto{\pgfqpoint{1.002334in}{1.138555in}}%
\pgfpathlineto{\pgfqpoint{0.998031in}{1.134814in}}%
\pgfpathlineto{\pgfqpoint{0.986678in}{1.124611in}}%
\pgfpathlineto{\pgfqpoint{0.982865in}{1.121203in}}%
\pgfpathlineto{\pgfqpoint{0.971021in}{1.109854in}}%
\pgfpathlineto{\pgfqpoint{0.968542in}{1.107592in}}%
\pgfpathlineto{\pgfqpoint{0.955364in}{1.094219in}}%
\pgfpathlineto{\pgfqpoint{0.955105in}{1.093981in}}%
\pgfpathlineto{\pgfqpoint{0.942208in}{1.080370in}}%
\pgfpathlineto{\pgfqpoint{0.939708in}{1.077142in}}%
\pgfpathlineto{\pgfqpoint{0.930117in}{1.066759in}}%
\pgfpathlineto{\pgfqpoint{0.924051in}{1.058114in}}%
\pgfpathlineto{\pgfqpoint{0.919438in}{1.053148in}}%
\pgfpathlineto{\pgfqpoint{0.910917in}{1.039536in}}%
\pgfpathlineto{\pgfqpoint{0.908395in}{1.031492in}}%
\pgfpathlineto{\pgfqpoint{0.905574in}{1.025925in}}%
\pgfpathlineto{\pgfqpoint{0.905574in}{1.012314in}}%
\pgfpathlineto{\pgfqpoint{0.908395in}{1.006747in}}%
\pgfpathlineto{\pgfqpoint{0.910917in}{0.998703in}}%
\pgfpathlineto{\pgfqpoint{0.919438in}{0.985092in}}%
\pgfpathlineto{\pgfqpoint{0.924051in}{0.980126in}}%
\pgfpathlineto{\pgfqpoint{0.930117in}{0.971481in}}%
\pgfpathlineto{\pgfqpoint{0.939708in}{0.961098in}}%
\pgfpathlineto{\pgfqpoint{0.942208in}{0.957870in}}%
\pgfpathlineto{\pgfqpoint{0.955105in}{0.944259in}}%
\pgfpathlineto{\pgfqpoint{0.955364in}{0.944021in}}%
\pgfpathlineto{\pgfqpoint{0.968542in}{0.930648in}}%
\pgfpathlineto{\pgfqpoint{0.971021in}{0.928386in}}%
\pgfpathlineto{\pgfqpoint{0.982865in}{0.917036in}}%
\pgfpathlineto{\pgfqpoint{0.986678in}{0.913628in}}%
\pgfpathlineto{\pgfqpoint{0.998031in}{0.903425in}}%
\pgfpathlineto{\pgfqpoint{1.002334in}{0.899684in}}%
\pgfpathlineto{\pgfqpoint{1.014071in}{0.889814in}}%
\pgfpathlineto{\pgfqpoint{1.017991in}{0.886500in}}%
\pgfpathlineto{\pgfqpoint{1.031046in}{0.876203in}}%
\pgfpathlineto{\pgfqpoint{1.033647in}{0.874048in}}%
\pgfpathlineto{\pgfqpoint{1.049030in}{0.862592in}}%
\pgfpathlineto{\pgfqpoint{1.049304in}{0.862366in}}%
\pgfpathlineto{\pgfqpoint{1.064960in}{0.851154in}}%
\pgfpathlineto{\pgfqpoint{1.068674in}{0.848981in}}%
\pgfpathlineto{\pgfqpoint{1.080617in}{0.840643in}}%
\pgfpathlineto{\pgfqpoint{1.090561in}{0.835370in}}%
\pgfpathlineto{\pgfqpoint{1.096274in}{0.831359in}}%
\pgfpathlineto{\pgfqpoint{1.111930in}{0.823951in}}%
\pgfpathlineto{\pgfqpoint{1.121183in}{0.821759in}}%
\pgfpathlineto{\pgfqpoint{1.127587in}{0.819306in}}%
\pgfpathclose%
\pgfpathmoveto{\pgfqpoint{1.086633in}{0.876203in}}%
\pgfpathlineto{\pgfqpoint{1.080617in}{0.878527in}}%
\pgfpathlineto{\pgfqpoint{1.064960in}{0.886499in}}%
\pgfpathlineto{\pgfqpoint{1.059635in}{0.889814in}}%
\pgfpathlineto{\pgfqpoint{1.049304in}{0.896138in}}%
\pgfpathlineto{\pgfqpoint{1.039086in}{0.903425in}}%
\pgfpathlineto{\pgfqpoint{1.033647in}{0.907403in}}%
\pgfpathlineto{\pgfqpoint{1.021815in}{0.917036in}}%
\pgfpathlineto{\pgfqpoint{1.017991in}{0.920361in}}%
\pgfpathlineto{\pgfqpoint{1.006910in}{0.930648in}}%
\pgfpathlineto{\pgfqpoint{1.002334in}{0.935376in}}%
\pgfpathlineto{\pgfqpoint{0.993952in}{0.944259in}}%
\pgfpathlineto{\pgfqpoint{0.986678in}{0.953240in}}%
\pgfpathlineto{\pgfqpoint{0.982865in}{0.957870in}}%
\pgfpathlineto{\pgfqpoint{0.973694in}{0.971481in}}%
\pgfpathlineto{\pgfqpoint{0.971021in}{0.976711in}}%
\pgfpathlineto{\pgfqpoint{0.966432in}{0.985092in}}%
\pgfpathlineto{\pgfqpoint{0.961433in}{0.998703in}}%
\pgfpathlineto{\pgfqpoint{0.958936in}{1.012314in}}%
\pgfpathlineto{\pgfqpoint{0.958936in}{1.025925in}}%
\pgfpathlineto{\pgfqpoint{0.961433in}{1.039536in}}%
\pgfpathlineto{\pgfqpoint{0.966432in}{1.053148in}}%
\pgfpathlineto{\pgfqpoint{0.971021in}{1.061529in}}%
\pgfpathlineto{\pgfqpoint{0.973694in}{1.066759in}}%
\pgfpathlineto{\pgfqpoint{0.982865in}{1.080370in}}%
\pgfpathlineto{\pgfqpoint{0.986678in}{1.085000in}}%
\pgfpathlineto{\pgfqpoint{0.993952in}{1.093981in}}%
\pgfpathlineto{\pgfqpoint{1.002334in}{1.102864in}}%
\pgfpathlineto{\pgfqpoint{1.006910in}{1.107592in}}%
\pgfpathlineto{\pgfqpoint{1.017991in}{1.117879in}}%
\pgfpathlineto{\pgfqpoint{1.021815in}{1.121203in}}%
\pgfpathlineto{\pgfqpoint{1.033647in}{1.130836in}}%
\pgfpathlineto{\pgfqpoint{1.039086in}{1.134814in}}%
\pgfpathlineto{\pgfqpoint{1.049304in}{1.142101in}}%
\pgfpathlineto{\pgfqpoint{1.059635in}{1.148425in}}%
\pgfpathlineto{\pgfqpoint{1.064960in}{1.151740in}}%
\pgfpathlineto{\pgfqpoint{1.080617in}{1.159712in}}%
\pgfpathlineto{\pgfqpoint{1.086633in}{1.162036in}}%
\pgfpathlineto{\pgfqpoint{1.096274in}{1.166026in}}%
\pgfpathlineto{\pgfqpoint{1.111930in}{1.170372in}}%
\pgfpathlineto{\pgfqpoint{1.127587in}{1.172542in}}%
\pgfpathlineto{\pgfqpoint{1.143243in}{1.172542in}}%
\pgfpathlineto{\pgfqpoint{1.158900in}{1.170372in}}%
\pgfpathlineto{\pgfqpoint{1.174556in}{1.166026in}}%
\pgfpathlineto{\pgfqpoint{1.184197in}{1.162036in}}%
\pgfpathlineto{\pgfqpoint{1.190213in}{1.159712in}}%
\pgfpathlineto{\pgfqpoint{1.205870in}{1.151740in}}%
\pgfpathlineto{\pgfqpoint{1.211195in}{1.148425in}}%
\pgfpathlineto{\pgfqpoint{1.221526in}{1.142101in}}%
\pgfpathlineto{\pgfqpoint{1.231744in}{1.134814in}}%
\pgfpathlineto{\pgfqpoint{1.237183in}{1.130836in}}%
\pgfpathlineto{\pgfqpoint{1.249015in}{1.121203in}}%
\pgfpathlineto{\pgfqpoint{1.252839in}{1.117879in}}%
\pgfpathlineto{\pgfqpoint{1.263920in}{1.107592in}}%
\pgfpathlineto{\pgfqpoint{1.268496in}{1.102864in}}%
\pgfpathlineto{\pgfqpoint{1.276878in}{1.093981in}}%
\pgfpathlineto{\pgfqpoint{1.284152in}{1.085000in}}%
\pgfpathlineto{\pgfqpoint{1.287965in}{1.080370in}}%
\pgfpathlineto{\pgfqpoint{1.297136in}{1.066759in}}%
\pgfpathlineto{\pgfqpoint{1.299809in}{1.061529in}}%
\pgfpathlineto{\pgfqpoint{1.304398in}{1.053148in}}%
\pgfpathlineto{\pgfqpoint{1.309397in}{1.039536in}}%
\pgfpathlineto{\pgfqpoint{1.311894in}{1.025925in}}%
\pgfpathlineto{\pgfqpoint{1.311894in}{1.012314in}}%
\pgfpathlineto{\pgfqpoint{1.309397in}{0.998703in}}%
\pgfpathlineto{\pgfqpoint{1.304398in}{0.985092in}}%
\pgfpathlineto{\pgfqpoint{1.299809in}{0.976711in}}%
\pgfpathlineto{\pgfqpoint{1.297136in}{0.971481in}}%
\pgfpathlineto{\pgfqpoint{1.287965in}{0.957870in}}%
\pgfpathlineto{\pgfqpoint{1.284152in}{0.953240in}}%
\pgfpathlineto{\pgfqpoint{1.276878in}{0.944259in}}%
\pgfpathlineto{\pgfqpoint{1.268496in}{0.935376in}}%
\pgfpathlineto{\pgfqpoint{1.263920in}{0.930648in}}%
\pgfpathlineto{\pgfqpoint{1.252839in}{0.920361in}}%
\pgfpathlineto{\pgfqpoint{1.249015in}{0.917036in}}%
\pgfpathlineto{\pgfqpoint{1.237183in}{0.907403in}}%
\pgfpathlineto{\pgfqpoint{1.231744in}{0.903425in}}%
\pgfpathlineto{\pgfqpoint{1.221526in}{0.896138in}}%
\pgfpathlineto{\pgfqpoint{1.211195in}{0.889814in}}%
\pgfpathlineto{\pgfqpoint{1.205870in}{0.886499in}}%
\pgfpathlineto{\pgfqpoint{1.190213in}{0.878527in}}%
\pgfpathlineto{\pgfqpoint{1.184197in}{0.876203in}}%
\pgfpathlineto{\pgfqpoint{1.174556in}{0.872213in}}%
\pgfpathlineto{\pgfqpoint{1.158900in}{0.867867in}}%
\pgfpathlineto{\pgfqpoint{1.143243in}{0.865697in}}%
\pgfpathlineto{\pgfqpoint{1.127587in}{0.865697in}}%
\pgfpathlineto{\pgfqpoint{1.111930in}{0.867867in}}%
\pgfpathlineto{\pgfqpoint{1.096274in}{0.872213in}}%
\pgfpathlineto{\pgfqpoint{1.086633in}{0.876203in}}%
\pgfpathclose%
\pgfpathmoveto{\pgfqpoint{1.644253in}{0.819306in}}%
\pgfpathlineto{\pgfqpoint{1.659910in}{0.819306in}}%
\pgfpathlineto{\pgfqpoint{1.666313in}{0.821759in}}%
\pgfpathlineto{\pgfqpoint{1.675567in}{0.823951in}}%
\pgfpathlineto{\pgfqpoint{1.691223in}{0.831359in}}%
\pgfpathlineto{\pgfqpoint{1.696936in}{0.835370in}}%
\pgfpathlineto{\pgfqpoint{1.706880in}{0.840643in}}%
\pgfpathlineto{\pgfqpoint{1.718823in}{0.848981in}}%
\pgfpathlineto{\pgfqpoint{1.722536in}{0.851154in}}%
\pgfpathlineto{\pgfqpoint{1.738193in}{0.862366in}}%
\pgfpathlineto{\pgfqpoint{1.738466in}{0.862592in}}%
\pgfpathlineto{\pgfqpoint{1.753849in}{0.874048in}}%
\pgfpathlineto{\pgfqpoint{1.756451in}{0.876203in}}%
\pgfpathlineto{\pgfqpoint{1.769506in}{0.886500in}}%
\pgfpathlineto{\pgfqpoint{1.773426in}{0.889814in}}%
\pgfpathlineto{\pgfqpoint{1.785162in}{0.899684in}}%
\pgfpathlineto{\pgfqpoint{1.789466in}{0.903425in}}%
\pgfpathlineto{\pgfqpoint{1.800819in}{0.913628in}}%
\pgfpathlineto{\pgfqpoint{1.804632in}{0.917036in}}%
\pgfpathlineto{\pgfqpoint{1.816476in}{0.928386in}}%
\pgfpathlineto{\pgfqpoint{1.818954in}{0.930648in}}%
\pgfpathlineto{\pgfqpoint{1.832132in}{0.944021in}}%
\pgfpathlineto{\pgfqpoint{1.832392in}{0.944259in}}%
\pgfpathlineto{\pgfqpoint{1.845289in}{0.957870in}}%
\pgfpathlineto{\pgfqpoint{1.847789in}{0.961098in}}%
\pgfpathlineto{\pgfqpoint{1.857380in}{0.971481in}}%
\pgfpathlineto{\pgfqpoint{1.863445in}{0.980126in}}%
\pgfpathlineto{\pgfqpoint{1.868058in}{0.985092in}}%
\pgfpathlineto{\pgfqpoint{1.876580in}{0.998703in}}%
\pgfpathlineto{\pgfqpoint{1.879102in}{1.006747in}}%
\pgfpathlineto{\pgfqpoint{1.881923in}{1.012314in}}%
\pgfpathlineto{\pgfqpoint{1.881923in}{1.025925in}}%
\pgfpathlineto{\pgfqpoint{1.879102in}{1.031492in}}%
\pgfpathlineto{\pgfqpoint{1.876580in}{1.039536in}}%
\pgfpathlineto{\pgfqpoint{1.868058in}{1.053148in}}%
\pgfpathlineto{\pgfqpoint{1.863445in}{1.058114in}}%
\pgfpathlineto{\pgfqpoint{1.857380in}{1.066759in}}%
\pgfpathlineto{\pgfqpoint{1.847789in}{1.077142in}}%
\pgfpathlineto{\pgfqpoint{1.845289in}{1.080370in}}%
\pgfpathlineto{\pgfqpoint{1.832392in}{1.093981in}}%
\pgfpathlineto{\pgfqpoint{1.832132in}{1.094219in}}%
\pgfpathlineto{\pgfqpoint{1.818954in}{1.107592in}}%
\pgfpathlineto{\pgfqpoint{1.816476in}{1.109854in}}%
\pgfpathlineto{\pgfqpoint{1.804632in}{1.121203in}}%
\pgfpathlineto{\pgfqpoint{1.800819in}{1.124611in}}%
\pgfpathlineto{\pgfqpoint{1.789466in}{1.134814in}}%
\pgfpathlineto{\pgfqpoint{1.785162in}{1.138555in}}%
\pgfpathlineto{\pgfqpoint{1.773426in}{1.148425in}}%
\pgfpathlineto{\pgfqpoint{1.769506in}{1.151740in}}%
\pgfpathlineto{\pgfqpoint{1.756451in}{1.162036in}}%
\pgfpathlineto{\pgfqpoint{1.753849in}{1.164191in}}%
\pgfpathlineto{\pgfqpoint{1.738466in}{1.175647in}}%
\pgfpathlineto{\pgfqpoint{1.738193in}{1.175873in}}%
\pgfpathlineto{\pgfqpoint{1.722536in}{1.187085in}}%
\pgfpathlineto{\pgfqpoint{1.718823in}{1.189259in}}%
\pgfpathlineto{\pgfqpoint{1.706880in}{1.197597in}}%
\pgfpathlineto{\pgfqpoint{1.696936in}{1.202870in}}%
\pgfpathlineto{\pgfqpoint{1.691223in}{1.206880in}}%
\pgfpathlineto{\pgfqpoint{1.675567in}{1.214288in}}%
\pgfpathlineto{\pgfqpoint{1.666313in}{1.216481in}}%
\pgfpathlineto{\pgfqpoint{1.659910in}{1.218933in}}%
\pgfpathlineto{\pgfqpoint{1.644253in}{1.218933in}}%
\pgfpathlineto{\pgfqpoint{1.637850in}{1.216481in}}%
\pgfpathlineto{\pgfqpoint{1.628597in}{1.214288in}}%
\pgfpathlineto{\pgfqpoint{1.612940in}{1.206880in}}%
\pgfpathlineto{\pgfqpoint{1.607227in}{1.202870in}}%
\pgfpathlineto{\pgfqpoint{1.597284in}{1.197597in}}%
\pgfpathlineto{\pgfqpoint{1.585340in}{1.189259in}}%
\pgfpathlineto{\pgfqpoint{1.581627in}{1.187085in}}%
\pgfpathlineto{\pgfqpoint{1.565971in}{1.175873in}}%
\pgfpathlineto{\pgfqpoint{1.565697in}{1.175647in}}%
\pgfpathlineto{\pgfqpoint{1.550314in}{1.164191in}}%
\pgfpathlineto{\pgfqpoint{1.547712in}{1.162036in}}%
\pgfpathlineto{\pgfqpoint{1.534657in}{1.151740in}}%
\pgfpathlineto{\pgfqpoint{1.530737in}{1.148425in}}%
\pgfpathlineto{\pgfqpoint{1.519001in}{1.138555in}}%
\pgfpathlineto{\pgfqpoint{1.514698in}{1.134814in}}%
\pgfpathlineto{\pgfqpoint{1.503344in}{1.124611in}}%
\pgfpathlineto{\pgfqpoint{1.499532in}{1.121203in}}%
\pgfpathlineto{\pgfqpoint{1.487688in}{1.109854in}}%
\pgfpathlineto{\pgfqpoint{1.485209in}{1.107592in}}%
\pgfpathlineto{\pgfqpoint{1.472031in}{1.094219in}}%
\pgfpathlineto{\pgfqpoint{1.471772in}{1.093981in}}%
\pgfpathlineto{\pgfqpoint{1.458874in}{1.080370in}}%
\pgfpathlineto{\pgfqpoint{1.456375in}{1.077142in}}%
\pgfpathlineto{\pgfqpoint{1.446783in}{1.066759in}}%
\pgfpathlineto{\pgfqpoint{1.440718in}{1.058114in}}%
\pgfpathlineto{\pgfqpoint{1.436105in}{1.053148in}}%
\pgfpathlineto{\pgfqpoint{1.427583in}{1.039536in}}%
\pgfpathlineto{\pgfqpoint{1.425061in}{1.031492in}}%
\pgfpathlineto{\pgfqpoint{1.422241in}{1.025925in}}%
\pgfpathlineto{\pgfqpoint{1.422241in}{1.012314in}}%
\pgfpathlineto{\pgfqpoint{1.425061in}{1.006747in}}%
\pgfpathlineto{\pgfqpoint{1.427583in}{0.998703in}}%
\pgfpathlineto{\pgfqpoint{1.436105in}{0.985092in}}%
\pgfpathlineto{\pgfqpoint{1.440718in}{0.980126in}}%
\pgfpathlineto{\pgfqpoint{1.446783in}{0.971481in}}%
\pgfpathlineto{\pgfqpoint{1.456375in}{0.961098in}}%
\pgfpathlineto{\pgfqpoint{1.458874in}{0.957870in}}%
\pgfpathlineto{\pgfqpoint{1.471772in}{0.944259in}}%
\pgfpathlineto{\pgfqpoint{1.472031in}{0.944021in}}%
\pgfpathlineto{\pgfqpoint{1.485209in}{0.930648in}}%
\pgfpathlineto{\pgfqpoint{1.487688in}{0.928386in}}%
\pgfpathlineto{\pgfqpoint{1.499532in}{0.917036in}}%
\pgfpathlineto{\pgfqpoint{1.503344in}{0.913628in}}%
\pgfpathlineto{\pgfqpoint{1.514698in}{0.903425in}}%
\pgfpathlineto{\pgfqpoint{1.519001in}{0.899684in}}%
\pgfpathlineto{\pgfqpoint{1.530737in}{0.889814in}}%
\pgfpathlineto{\pgfqpoint{1.534657in}{0.886500in}}%
\pgfpathlineto{\pgfqpoint{1.547712in}{0.876203in}}%
\pgfpathlineto{\pgfqpoint{1.550314in}{0.874048in}}%
\pgfpathlineto{\pgfqpoint{1.565697in}{0.862592in}}%
\pgfpathlineto{\pgfqpoint{1.565971in}{0.862366in}}%
\pgfpathlineto{\pgfqpoint{1.581627in}{0.851154in}}%
\pgfpathlineto{\pgfqpoint{1.585340in}{0.848981in}}%
\pgfpathlineto{\pgfqpoint{1.597284in}{0.840643in}}%
\pgfpathlineto{\pgfqpoint{1.607227in}{0.835370in}}%
\pgfpathlineto{\pgfqpoint{1.612940in}{0.831359in}}%
\pgfpathlineto{\pgfqpoint{1.628597in}{0.823951in}}%
\pgfpathlineto{\pgfqpoint{1.637850in}{0.821759in}}%
\pgfpathlineto{\pgfqpoint{1.644253in}{0.819306in}}%
\pgfpathclose%
\pgfpathmoveto{\pgfqpoint{1.603299in}{0.876203in}}%
\pgfpathlineto{\pgfqpoint{1.597284in}{0.878527in}}%
\pgfpathlineto{\pgfqpoint{1.581627in}{0.886499in}}%
\pgfpathlineto{\pgfqpoint{1.576301in}{0.889814in}}%
\pgfpathlineto{\pgfqpoint{1.565971in}{0.896138in}}%
\pgfpathlineto{\pgfqpoint{1.555753in}{0.903425in}}%
\pgfpathlineto{\pgfqpoint{1.550314in}{0.907403in}}%
\pgfpathlineto{\pgfqpoint{1.538481in}{0.917036in}}%
\pgfpathlineto{\pgfqpoint{1.534657in}{0.920361in}}%
\pgfpathlineto{\pgfqpoint{1.523577in}{0.930648in}}%
\pgfpathlineto{\pgfqpoint{1.519001in}{0.935376in}}%
\pgfpathlineto{\pgfqpoint{1.510618in}{0.944259in}}%
\pgfpathlineto{\pgfqpoint{1.503344in}{0.953240in}}%
\pgfpathlineto{\pgfqpoint{1.499531in}{0.957870in}}%
\pgfpathlineto{\pgfqpoint{1.490361in}{0.971481in}}%
\pgfpathlineto{\pgfqpoint{1.487688in}{0.976711in}}%
\pgfpathlineto{\pgfqpoint{1.483099in}{0.985092in}}%
\pgfpathlineto{\pgfqpoint{1.478099in}{0.998703in}}%
\pgfpathlineto{\pgfqpoint{1.475603in}{1.012314in}}%
\pgfpathlineto{\pgfqpoint{1.475603in}{1.025925in}}%
\pgfpathlineto{\pgfqpoint{1.478099in}{1.039536in}}%
\pgfpathlineto{\pgfqpoint{1.483099in}{1.053148in}}%
\pgfpathlineto{\pgfqpoint{1.487688in}{1.061529in}}%
\pgfpathlineto{\pgfqpoint{1.490361in}{1.066759in}}%
\pgfpathlineto{\pgfqpoint{1.499531in}{1.080370in}}%
\pgfpathlineto{\pgfqpoint{1.503344in}{1.085000in}}%
\pgfpathlineto{\pgfqpoint{1.510618in}{1.093981in}}%
\pgfpathlineto{\pgfqpoint{1.519001in}{1.102864in}}%
\pgfpathlineto{\pgfqpoint{1.523577in}{1.107592in}}%
\pgfpathlineto{\pgfqpoint{1.534657in}{1.117879in}}%
\pgfpathlineto{\pgfqpoint{1.538481in}{1.121203in}}%
\pgfpathlineto{\pgfqpoint{1.550314in}{1.130836in}}%
\pgfpathlineto{\pgfqpoint{1.555753in}{1.134814in}}%
\pgfpathlineto{\pgfqpoint{1.565971in}{1.142101in}}%
\pgfpathlineto{\pgfqpoint{1.576301in}{1.148425in}}%
\pgfpathlineto{\pgfqpoint{1.581627in}{1.151740in}}%
\pgfpathlineto{\pgfqpoint{1.597284in}{1.159712in}}%
\pgfpathlineto{\pgfqpoint{1.603299in}{1.162036in}}%
\pgfpathlineto{\pgfqpoint{1.612940in}{1.166026in}}%
\pgfpathlineto{\pgfqpoint{1.628597in}{1.170372in}}%
\pgfpathlineto{\pgfqpoint{1.644253in}{1.172542in}}%
\pgfpathlineto{\pgfqpoint{1.659910in}{1.172542in}}%
\pgfpathlineto{\pgfqpoint{1.675567in}{1.170372in}}%
\pgfpathlineto{\pgfqpoint{1.691223in}{1.166026in}}%
\pgfpathlineto{\pgfqpoint{1.700864in}{1.162036in}}%
\pgfpathlineto{\pgfqpoint{1.706880in}{1.159712in}}%
\pgfpathlineto{\pgfqpoint{1.722536in}{1.151740in}}%
\pgfpathlineto{\pgfqpoint{1.727862in}{1.148425in}}%
\pgfpathlineto{\pgfqpoint{1.738193in}{1.142101in}}%
\pgfpathlineto{\pgfqpoint{1.748411in}{1.134814in}}%
\pgfpathlineto{\pgfqpoint{1.753849in}{1.130836in}}%
\pgfpathlineto{\pgfqpoint{1.765682in}{1.121203in}}%
\pgfpathlineto{\pgfqpoint{1.769506in}{1.117879in}}%
\pgfpathlineto{\pgfqpoint{1.780587in}{1.107592in}}%
\pgfpathlineto{\pgfqpoint{1.785162in}{1.102864in}}%
\pgfpathlineto{\pgfqpoint{1.793545in}{1.093981in}}%
\pgfpathlineto{\pgfqpoint{1.800819in}{1.085000in}}%
\pgfpathlineto{\pgfqpoint{1.804632in}{1.080370in}}%
\pgfpathlineto{\pgfqpoint{1.813802in}{1.066759in}}%
\pgfpathlineto{\pgfqpoint{1.816476in}{1.061529in}}%
\pgfpathlineto{\pgfqpoint{1.821065in}{1.053148in}}%
\pgfpathlineto{\pgfqpoint{1.826064in}{1.039536in}}%
\pgfpathlineto{\pgfqpoint{1.828560in}{1.025925in}}%
\pgfpathlineto{\pgfqpoint{1.828560in}{1.012314in}}%
\pgfpathlineto{\pgfqpoint{1.826064in}{0.998703in}}%
\pgfpathlineto{\pgfqpoint{1.821065in}{0.985092in}}%
\pgfpathlineto{\pgfqpoint{1.816476in}{0.976711in}}%
\pgfpathlineto{\pgfqpoint{1.813802in}{0.971481in}}%
\pgfpathlineto{\pgfqpoint{1.804632in}{0.957870in}}%
\pgfpathlineto{\pgfqpoint{1.800819in}{0.953240in}}%
\pgfpathlineto{\pgfqpoint{1.793545in}{0.944259in}}%
\pgfpathlineto{\pgfqpoint{1.785162in}{0.935376in}}%
\pgfpathlineto{\pgfqpoint{1.780587in}{0.930648in}}%
\pgfpathlineto{\pgfqpoint{1.769506in}{0.920361in}}%
\pgfpathlineto{\pgfqpoint{1.765682in}{0.917036in}}%
\pgfpathlineto{\pgfqpoint{1.753849in}{0.907403in}}%
\pgfpathlineto{\pgfqpoint{1.748411in}{0.903425in}}%
\pgfpathlineto{\pgfqpoint{1.738193in}{0.896138in}}%
\pgfpathlineto{\pgfqpoint{1.727862in}{0.889814in}}%
\pgfpathlineto{\pgfqpoint{1.722536in}{0.886499in}}%
\pgfpathlineto{\pgfqpoint{1.706880in}{0.878527in}}%
\pgfpathlineto{\pgfqpoint{1.700864in}{0.876203in}}%
\pgfpathlineto{\pgfqpoint{1.691223in}{0.872213in}}%
\pgfpathlineto{\pgfqpoint{1.675567in}{0.867867in}}%
\pgfpathlineto{\pgfqpoint{1.659910in}{0.865697in}}%
\pgfpathlineto{\pgfqpoint{1.644253in}{0.865697in}}%
\pgfpathlineto{\pgfqpoint{1.628597in}{0.867867in}}%
\pgfpathlineto{\pgfqpoint{1.612940in}{0.872213in}}%
\pgfpathlineto{\pgfqpoint{1.603299in}{0.876203in}}%
\pgfpathclose%
\pgfpathmoveto{\pgfqpoint{0.610920in}{1.268473in}}%
\pgfpathlineto{\pgfqpoint{0.626577in}{1.268473in}}%
\pgfpathlineto{\pgfqpoint{0.632980in}{1.270925in}}%
\pgfpathlineto{\pgfqpoint{0.642233in}{1.273118in}}%
\pgfpathlineto{\pgfqpoint{0.657890in}{1.280526in}}%
\pgfpathlineto{\pgfqpoint{0.663603in}{1.284536in}}%
\pgfpathlineto{\pgfqpoint{0.673546in}{1.289809in}}%
\pgfpathlineto{\pgfqpoint{0.685490in}{1.298148in}}%
\pgfpathlineto{\pgfqpoint{0.689203in}{1.300321in}}%
\pgfpathlineto{\pgfqpoint{0.704859in}{1.311533in}}%
\pgfpathlineto{\pgfqpoint{0.705133in}{1.311759in}}%
\pgfpathlineto{\pgfqpoint{0.720516in}{1.323215in}}%
\pgfpathlineto{\pgfqpoint{0.723118in}{1.325370in}}%
\pgfpathlineto{\pgfqpoint{0.736173in}{1.335666in}}%
\pgfpathlineto{\pgfqpoint{0.740093in}{1.338981in}}%
\pgfpathlineto{\pgfqpoint{0.751829in}{1.348851in}}%
\pgfpathlineto{\pgfqpoint{0.756132in}{1.352592in}}%
\pgfpathlineto{\pgfqpoint{0.767486in}{1.362795in}}%
\pgfpathlineto{\pgfqpoint{0.771298in}{1.366203in}}%
\pgfpathlineto{\pgfqpoint{0.783142in}{1.377553in}}%
\pgfpathlineto{\pgfqpoint{0.785621in}{1.379814in}}%
\pgfpathlineto{\pgfqpoint{0.798799in}{1.393187in}}%
\pgfpathlineto{\pgfqpoint{0.799058in}{1.393425in}}%
\pgfpathlineto{\pgfqpoint{0.811956in}{1.407036in}}%
\pgfpathlineto{\pgfqpoint{0.814455in}{1.410264in}}%
\pgfpathlineto{\pgfqpoint{0.824047in}{1.420648in}}%
\pgfpathlineto{\pgfqpoint{0.830112in}{1.429292in}}%
\pgfpathlineto{\pgfqpoint{0.834725in}{1.434259in}}%
\pgfpathlineto{\pgfqpoint{0.843247in}{1.447870in}}%
\pgfpathlineto{\pgfqpoint{0.845769in}{1.455914in}}%
\pgfpathlineto{\pgfqpoint{0.848589in}{1.461481in}}%
\pgfpathlineto{\pgfqpoint{0.848589in}{1.475092in}}%
\pgfpathlineto{\pgfqpoint{0.845769in}{1.480659in}}%
\pgfpathlineto{\pgfqpoint{0.843247in}{1.488703in}}%
\pgfpathlineto{\pgfqpoint{0.834725in}{1.502314in}}%
\pgfpathlineto{\pgfqpoint{0.830112in}{1.507281in}}%
\pgfpathlineto{\pgfqpoint{0.824047in}{1.515925in}}%
\pgfpathlineto{\pgfqpoint{0.814455in}{1.526308in}}%
\pgfpathlineto{\pgfqpoint{0.811956in}{1.529536in}}%
\pgfpathlineto{\pgfqpoint{0.799058in}{1.543148in}}%
\pgfpathlineto{\pgfqpoint{0.798799in}{1.543385in}}%
\pgfpathlineto{\pgfqpoint{0.785621in}{1.556759in}}%
\pgfpathlineto{\pgfqpoint{0.783142in}{1.559020in}}%
\pgfpathlineto{\pgfqpoint{0.771298in}{1.570370in}}%
\pgfpathlineto{\pgfqpoint{0.767486in}{1.573778in}}%
\pgfpathlineto{\pgfqpoint{0.756132in}{1.583981in}}%
\pgfpathlineto{\pgfqpoint{0.751829in}{1.587722in}}%
\pgfpathlineto{\pgfqpoint{0.740093in}{1.597592in}}%
\pgfpathlineto{\pgfqpoint{0.736173in}{1.600907in}}%
\pgfpathlineto{\pgfqpoint{0.723118in}{1.611203in}}%
\pgfpathlineto{\pgfqpoint{0.720516in}{1.613358in}}%
\pgfpathlineto{\pgfqpoint{0.705133in}{1.624814in}}%
\pgfpathlineto{\pgfqpoint{0.704859in}{1.625040in}}%
\pgfpathlineto{\pgfqpoint{0.689203in}{1.636252in}}%
\pgfpathlineto{\pgfqpoint{0.685490in}{1.638425in}}%
\pgfpathlineto{\pgfqpoint{0.673546in}{1.646763in}}%
\pgfpathlineto{\pgfqpoint{0.663603in}{1.652036in}}%
\pgfpathlineto{\pgfqpoint{0.657890in}{1.656047in}}%
\pgfpathlineto{\pgfqpoint{0.642233in}{1.663455in}}%
\pgfpathlineto{\pgfqpoint{0.632980in}{1.665648in}}%
\pgfpathlineto{\pgfqpoint{0.626577in}{1.668100in}}%
\pgfpathlineto{\pgfqpoint{0.610920in}{1.668100in}}%
\pgfpathlineto{\pgfqpoint{0.604517in}{1.665648in}}%
\pgfpathlineto{\pgfqpoint{0.595263in}{1.663455in}}%
\pgfpathlineto{\pgfqpoint{0.579607in}{1.656047in}}%
\pgfpathlineto{\pgfqpoint{0.573894in}{1.652036in}}%
\pgfpathlineto{\pgfqpoint{0.563950in}{1.646763in}}%
\pgfpathlineto{\pgfqpoint{0.552007in}{1.638425in}}%
\pgfpathlineto{\pgfqpoint{0.548294in}{1.636252in}}%
\pgfpathlineto{\pgfqpoint{0.532637in}{1.625040in}}%
\pgfpathlineto{\pgfqpoint{0.532364in}{1.624814in}}%
\pgfpathlineto{\pgfqpoint{0.516981in}{1.613358in}}%
\pgfpathlineto{\pgfqpoint{0.514379in}{1.611203in}}%
\pgfpathlineto{\pgfqpoint{0.501324in}{1.600907in}}%
\pgfpathlineto{\pgfqpoint{0.497404in}{1.597592in}}%
\pgfpathlineto{\pgfqpoint{0.485668in}{1.587722in}}%
\pgfpathlineto{\pgfqpoint{0.481364in}{1.583981in}}%
\pgfpathlineto{\pgfqpoint{0.470011in}{1.573778in}}%
\pgfpathlineto{\pgfqpoint{0.466198in}{1.570370in}}%
\pgfpathlineto{\pgfqpoint{0.454354in}{1.559020in}}%
\pgfpathlineto{\pgfqpoint{0.451876in}{1.556759in}}%
\pgfpathlineto{\pgfqpoint{0.438698in}{1.543385in}}%
\pgfpathlineto{\pgfqpoint{0.438438in}{1.543148in}}%
\pgfpathlineto{\pgfqpoint{0.425541in}{1.529536in}}%
\pgfpathlineto{\pgfqpoint{0.423041in}{1.526308in}}%
\pgfpathlineto{\pgfqpoint{0.413450in}{1.515925in}}%
\pgfpathlineto{\pgfqpoint{0.407385in}{1.507281in}}%
\pgfpathlineto{\pgfqpoint{0.402772in}{1.502314in}}%
\pgfpathlineto{\pgfqpoint{0.394250in}{1.488703in}}%
\pgfpathlineto{\pgfqpoint{0.391728in}{1.480659in}}%
\pgfpathlineto{\pgfqpoint{0.388907in}{1.475092in}}%
\pgfpathlineto{\pgfqpoint{0.388907in}{1.461481in}}%
\pgfpathlineto{\pgfqpoint{0.391728in}{1.455914in}}%
\pgfpathlineto{\pgfqpoint{0.394250in}{1.447870in}}%
\pgfpathlineto{\pgfqpoint{0.402772in}{1.434259in}}%
\pgfpathlineto{\pgfqpoint{0.407385in}{1.429292in}}%
\pgfpathlineto{\pgfqpoint{0.413450in}{1.420648in}}%
\pgfpathlineto{\pgfqpoint{0.423041in}{1.410264in}}%
\pgfpathlineto{\pgfqpoint{0.425541in}{1.407036in}}%
\pgfpathlineto{\pgfqpoint{0.438438in}{1.393425in}}%
\pgfpathlineto{\pgfqpoint{0.438698in}{1.393187in}}%
\pgfpathlineto{\pgfqpoint{0.451876in}{1.379814in}}%
\pgfpathlineto{\pgfqpoint{0.454354in}{1.377553in}}%
\pgfpathlineto{\pgfqpoint{0.466198in}{1.366203in}}%
\pgfpathlineto{\pgfqpoint{0.470011in}{1.362795in}}%
\pgfpathlineto{\pgfqpoint{0.481364in}{1.352592in}}%
\pgfpathlineto{\pgfqpoint{0.485668in}{1.348851in}}%
\pgfpathlineto{\pgfqpoint{0.497404in}{1.338981in}}%
\pgfpathlineto{\pgfqpoint{0.501324in}{1.335666in}}%
\pgfpathlineto{\pgfqpoint{0.514379in}{1.325370in}}%
\pgfpathlineto{\pgfqpoint{0.516981in}{1.323215in}}%
\pgfpathlineto{\pgfqpoint{0.532364in}{1.311759in}}%
\pgfpathlineto{\pgfqpoint{0.532637in}{1.311533in}}%
\pgfpathlineto{\pgfqpoint{0.548294in}{1.300321in}}%
\pgfpathlineto{\pgfqpoint{0.552007in}{1.298148in}}%
\pgfpathlineto{\pgfqpoint{0.563950in}{1.289809in}}%
\pgfpathlineto{\pgfqpoint{0.573894in}{1.284536in}}%
\pgfpathlineto{\pgfqpoint{0.579607in}{1.280526in}}%
\pgfpathlineto{\pgfqpoint{0.595263in}{1.273118in}}%
\pgfpathlineto{\pgfqpoint{0.604517in}{1.270925in}}%
\pgfpathlineto{\pgfqpoint{0.610920in}{1.268473in}}%
\pgfpathclose%
\pgfpathmoveto{\pgfqpoint{0.569966in}{1.325370in}}%
\pgfpathlineto{\pgfqpoint{0.563950in}{1.327694in}}%
\pgfpathlineto{\pgfqpoint{0.548294in}{1.335666in}}%
\pgfpathlineto{\pgfqpoint{0.542968in}{1.338981in}}%
\pgfpathlineto{\pgfqpoint{0.532637in}{1.345305in}}%
\pgfpathlineto{\pgfqpoint{0.522419in}{1.352592in}}%
\pgfpathlineto{\pgfqpoint{0.516981in}{1.356570in}}%
\pgfpathlineto{\pgfqpoint{0.505148in}{1.366203in}}%
\pgfpathlineto{\pgfqpoint{0.501324in}{1.369527in}}%
\pgfpathlineto{\pgfqpoint{0.490243in}{1.379814in}}%
\pgfpathlineto{\pgfqpoint{0.485668in}{1.384542in}}%
\pgfpathlineto{\pgfqpoint{0.477285in}{1.393425in}}%
\pgfpathlineto{\pgfqpoint{0.470011in}{1.402407in}}%
\pgfpathlineto{\pgfqpoint{0.466198in}{1.407036in}}%
\pgfpathlineto{\pgfqpoint{0.457028in}{1.420648in}}%
\pgfpathlineto{\pgfqpoint{0.454354in}{1.425877in}}%
\pgfpathlineto{\pgfqpoint{0.449765in}{1.434259in}}%
\pgfpathlineto{\pgfqpoint{0.444766in}{1.447870in}}%
\pgfpathlineto{\pgfqpoint{0.442270in}{1.461481in}}%
\pgfpathlineto{\pgfqpoint{0.442270in}{1.475092in}}%
\pgfpathlineto{\pgfqpoint{0.444766in}{1.488703in}}%
\pgfpathlineto{\pgfqpoint{0.449765in}{1.502314in}}%
\pgfpathlineto{\pgfqpoint{0.454354in}{1.510695in}}%
\pgfpathlineto{\pgfqpoint{0.457028in}{1.515925in}}%
\pgfpathlineto{\pgfqpoint{0.466198in}{1.529536in}}%
\pgfpathlineto{\pgfqpoint{0.470011in}{1.534166in}}%
\pgfpathlineto{\pgfqpoint{0.477285in}{1.543148in}}%
\pgfpathlineto{\pgfqpoint{0.485668in}{1.552031in}}%
\pgfpathlineto{\pgfqpoint{0.490243in}{1.556759in}}%
\pgfpathlineto{\pgfqpoint{0.501324in}{1.567045in}}%
\pgfpathlineto{\pgfqpoint{0.505148in}{1.570370in}}%
\pgfpathlineto{\pgfqpoint{0.516981in}{1.580003in}}%
\pgfpathlineto{\pgfqpoint{0.522419in}{1.583981in}}%
\pgfpathlineto{\pgfqpoint{0.532637in}{1.591268in}}%
\pgfpathlineto{\pgfqpoint{0.542968in}{1.597592in}}%
\pgfpathlineto{\pgfqpoint{0.548294in}{1.600907in}}%
\pgfpathlineto{\pgfqpoint{0.563950in}{1.608879in}}%
\pgfpathlineto{\pgfqpoint{0.569966in}{1.611203in}}%
\pgfpathlineto{\pgfqpoint{0.579607in}{1.615193in}}%
\pgfpathlineto{\pgfqpoint{0.595263in}{1.619539in}}%
\pgfpathlineto{\pgfqpoint{0.610920in}{1.621709in}}%
\pgfpathlineto{\pgfqpoint{0.626577in}{1.621709in}}%
\pgfpathlineto{\pgfqpoint{0.642233in}{1.619539in}}%
\pgfpathlineto{\pgfqpoint{0.657890in}{1.615193in}}%
\pgfpathlineto{\pgfqpoint{0.667531in}{1.611203in}}%
\pgfpathlineto{\pgfqpoint{0.673546in}{1.608879in}}%
\pgfpathlineto{\pgfqpoint{0.689203in}{1.600907in}}%
\pgfpathlineto{\pgfqpoint{0.694529in}{1.597592in}}%
\pgfpathlineto{\pgfqpoint{0.704859in}{1.591268in}}%
\pgfpathlineto{\pgfqpoint{0.715077in}{1.583981in}}%
\pgfpathlineto{\pgfqpoint{0.720516in}{1.580003in}}%
\pgfpathlineto{\pgfqpoint{0.732349in}{1.570370in}}%
\pgfpathlineto{\pgfqpoint{0.736173in}{1.567045in}}%
\pgfpathlineto{\pgfqpoint{0.747253in}{1.556759in}}%
\pgfpathlineto{\pgfqpoint{0.751829in}{1.552031in}}%
\pgfpathlineto{\pgfqpoint{0.760212in}{1.543148in}}%
\pgfpathlineto{\pgfqpoint{0.767486in}{1.534166in}}%
\pgfpathlineto{\pgfqpoint{0.771299in}{1.529536in}}%
\pgfpathlineto{\pgfqpoint{0.780469in}{1.515925in}}%
\pgfpathlineto{\pgfqpoint{0.783142in}{1.510695in}}%
\pgfpathlineto{\pgfqpoint{0.787731in}{1.502314in}}%
\pgfpathlineto{\pgfqpoint{0.792731in}{1.488703in}}%
\pgfpathlineto{\pgfqpoint{0.795227in}{1.475092in}}%
\pgfpathlineto{\pgfqpoint{0.795227in}{1.461481in}}%
\pgfpathlineto{\pgfqpoint{0.792731in}{1.447870in}}%
\pgfpathlineto{\pgfqpoint{0.787731in}{1.434259in}}%
\pgfpathlineto{\pgfqpoint{0.783142in}{1.425877in}}%
\pgfpathlineto{\pgfqpoint{0.780469in}{1.420648in}}%
\pgfpathlineto{\pgfqpoint{0.771299in}{1.407036in}}%
\pgfpathlineto{\pgfqpoint{0.767486in}{1.402407in}}%
\pgfpathlineto{\pgfqpoint{0.760212in}{1.393425in}}%
\pgfpathlineto{\pgfqpoint{0.751829in}{1.384542in}}%
\pgfpathlineto{\pgfqpoint{0.747253in}{1.379814in}}%
\pgfpathlineto{\pgfqpoint{0.736173in}{1.369527in}}%
\pgfpathlineto{\pgfqpoint{0.732349in}{1.366203in}}%
\pgfpathlineto{\pgfqpoint{0.720516in}{1.356570in}}%
\pgfpathlineto{\pgfqpoint{0.715077in}{1.352592in}}%
\pgfpathlineto{\pgfqpoint{0.704859in}{1.345305in}}%
\pgfpathlineto{\pgfqpoint{0.694529in}{1.338981in}}%
\pgfpathlineto{\pgfqpoint{0.689203in}{1.335666in}}%
\pgfpathlineto{\pgfqpoint{0.673546in}{1.327694in}}%
\pgfpathlineto{\pgfqpoint{0.667531in}{1.325370in}}%
\pgfpathlineto{\pgfqpoint{0.657890in}{1.321380in}}%
\pgfpathlineto{\pgfqpoint{0.642233in}{1.317034in}}%
\pgfpathlineto{\pgfqpoint{0.626577in}{1.314864in}}%
\pgfpathlineto{\pgfqpoint{0.610920in}{1.314864in}}%
\pgfpathlineto{\pgfqpoint{0.595263in}{1.317034in}}%
\pgfpathlineto{\pgfqpoint{0.579607in}{1.321380in}}%
\pgfpathlineto{\pgfqpoint{0.569966in}{1.325370in}}%
\pgfpathclose%
\pgfpathmoveto{\pgfqpoint{1.127587in}{1.268473in}}%
\pgfpathlineto{\pgfqpoint{1.143243in}{1.268473in}}%
\pgfpathlineto{\pgfqpoint{1.149647in}{1.270925in}}%
\pgfpathlineto{\pgfqpoint{1.158900in}{1.273118in}}%
\pgfpathlineto{\pgfqpoint{1.174556in}{1.280526in}}%
\pgfpathlineto{\pgfqpoint{1.180269in}{1.284536in}}%
\pgfpathlineto{\pgfqpoint{1.190213in}{1.289809in}}%
\pgfpathlineto{\pgfqpoint{1.202156in}{1.298148in}}%
\pgfpathlineto{\pgfqpoint{1.205870in}{1.300321in}}%
\pgfpathlineto{\pgfqpoint{1.221526in}{1.311533in}}%
\pgfpathlineto{\pgfqpoint{1.221800in}{1.311759in}}%
\pgfpathlineto{\pgfqpoint{1.237183in}{1.323215in}}%
\pgfpathlineto{\pgfqpoint{1.239784in}{1.325370in}}%
\pgfpathlineto{\pgfqpoint{1.252839in}{1.335666in}}%
\pgfpathlineto{\pgfqpoint{1.256759in}{1.338981in}}%
\pgfpathlineto{\pgfqpoint{1.268496in}{1.348851in}}%
\pgfpathlineto{\pgfqpoint{1.272799in}{1.352592in}}%
\pgfpathlineto{\pgfqpoint{1.284152in}{1.362795in}}%
\pgfpathlineto{\pgfqpoint{1.287965in}{1.366203in}}%
\pgfpathlineto{\pgfqpoint{1.299809in}{1.377553in}}%
\pgfpathlineto{\pgfqpoint{1.302288in}{1.379814in}}%
\pgfpathlineto{\pgfqpoint{1.315466in}{1.393187in}}%
\pgfpathlineto{\pgfqpoint{1.315725in}{1.393425in}}%
\pgfpathlineto{\pgfqpoint{1.328622in}{1.407036in}}%
\pgfpathlineto{\pgfqpoint{1.331122in}{1.410264in}}%
\pgfpathlineto{\pgfqpoint{1.340713in}{1.420648in}}%
\pgfpathlineto{\pgfqpoint{1.346779in}{1.429292in}}%
\pgfpathlineto{\pgfqpoint{1.351392in}{1.434259in}}%
\pgfpathlineto{\pgfqpoint{1.359913in}{1.447870in}}%
\pgfpathlineto{\pgfqpoint{1.362435in}{1.455914in}}%
\pgfpathlineto{\pgfqpoint{1.365256in}{1.461481in}}%
\pgfpathlineto{\pgfqpoint{1.365256in}{1.475092in}}%
\pgfpathlineto{\pgfqpoint{1.362435in}{1.480659in}}%
\pgfpathlineto{\pgfqpoint{1.359913in}{1.488703in}}%
\pgfpathlineto{\pgfqpoint{1.351392in}{1.502314in}}%
\pgfpathlineto{\pgfqpoint{1.346779in}{1.507281in}}%
\pgfpathlineto{\pgfqpoint{1.340713in}{1.515925in}}%
\pgfpathlineto{\pgfqpoint{1.331122in}{1.526308in}}%
\pgfpathlineto{\pgfqpoint{1.328622in}{1.529536in}}%
\pgfpathlineto{\pgfqpoint{1.315725in}{1.543148in}}%
\pgfpathlineto{\pgfqpoint{1.315466in}{1.543385in}}%
\pgfpathlineto{\pgfqpoint{1.302288in}{1.556759in}}%
\pgfpathlineto{\pgfqpoint{1.299809in}{1.559020in}}%
\pgfpathlineto{\pgfqpoint{1.287965in}{1.570370in}}%
\pgfpathlineto{\pgfqpoint{1.284152in}{1.573778in}}%
\pgfpathlineto{\pgfqpoint{1.272799in}{1.583981in}}%
\pgfpathlineto{\pgfqpoint{1.268496in}{1.587722in}}%
\pgfpathlineto{\pgfqpoint{1.256759in}{1.597592in}}%
\pgfpathlineto{\pgfqpoint{1.252839in}{1.600907in}}%
\pgfpathlineto{\pgfqpoint{1.239784in}{1.611203in}}%
\pgfpathlineto{\pgfqpoint{1.237183in}{1.613358in}}%
\pgfpathlineto{\pgfqpoint{1.221800in}{1.624814in}}%
\pgfpathlineto{\pgfqpoint{1.221526in}{1.625040in}}%
\pgfpathlineto{\pgfqpoint{1.205870in}{1.636252in}}%
\pgfpathlineto{\pgfqpoint{1.202156in}{1.638425in}}%
\pgfpathlineto{\pgfqpoint{1.190213in}{1.646763in}}%
\pgfpathlineto{\pgfqpoint{1.180269in}{1.652036in}}%
\pgfpathlineto{\pgfqpoint{1.174556in}{1.656047in}}%
\pgfpathlineto{\pgfqpoint{1.158900in}{1.663455in}}%
\pgfpathlineto{\pgfqpoint{1.149647in}{1.665648in}}%
\pgfpathlineto{\pgfqpoint{1.143243in}{1.668100in}}%
\pgfpathlineto{\pgfqpoint{1.127587in}{1.668100in}}%
\pgfpathlineto{\pgfqpoint{1.121183in}{1.665648in}}%
\pgfpathlineto{\pgfqpoint{1.111930in}{1.663455in}}%
\pgfpathlineto{\pgfqpoint{1.096274in}{1.656047in}}%
\pgfpathlineto{\pgfqpoint{1.090561in}{1.652036in}}%
\pgfpathlineto{\pgfqpoint{1.080617in}{1.646763in}}%
\pgfpathlineto{\pgfqpoint{1.068674in}{1.638425in}}%
\pgfpathlineto{\pgfqpoint{1.064960in}{1.636252in}}%
\pgfpathlineto{\pgfqpoint{1.049304in}{1.625040in}}%
\pgfpathlineto{\pgfqpoint{1.049030in}{1.624814in}}%
\pgfpathlineto{\pgfqpoint{1.033647in}{1.613358in}}%
\pgfpathlineto{\pgfqpoint{1.031046in}{1.611203in}}%
\pgfpathlineto{\pgfqpoint{1.017991in}{1.600907in}}%
\pgfpathlineto{\pgfqpoint{1.014071in}{1.597592in}}%
\pgfpathlineto{\pgfqpoint{1.002334in}{1.587722in}}%
\pgfpathlineto{\pgfqpoint{0.998031in}{1.583981in}}%
\pgfpathlineto{\pgfqpoint{0.986678in}{1.573778in}}%
\pgfpathlineto{\pgfqpoint{0.982865in}{1.570370in}}%
\pgfpathlineto{\pgfqpoint{0.971021in}{1.559020in}}%
\pgfpathlineto{\pgfqpoint{0.968542in}{1.556759in}}%
\pgfpathlineto{\pgfqpoint{0.955364in}{1.543385in}}%
\pgfpathlineto{\pgfqpoint{0.955105in}{1.543148in}}%
\pgfpathlineto{\pgfqpoint{0.942208in}{1.529536in}}%
\pgfpathlineto{\pgfqpoint{0.939708in}{1.526308in}}%
\pgfpathlineto{\pgfqpoint{0.930117in}{1.515925in}}%
\pgfpathlineto{\pgfqpoint{0.924051in}{1.507281in}}%
\pgfpathlineto{\pgfqpoint{0.919438in}{1.502314in}}%
\pgfpathlineto{\pgfqpoint{0.910917in}{1.488703in}}%
\pgfpathlineto{\pgfqpoint{0.908395in}{1.480659in}}%
\pgfpathlineto{\pgfqpoint{0.905574in}{1.475092in}}%
\pgfpathlineto{\pgfqpoint{0.905574in}{1.461481in}}%
\pgfpathlineto{\pgfqpoint{0.908395in}{1.455914in}}%
\pgfpathlineto{\pgfqpoint{0.910917in}{1.447870in}}%
\pgfpathlineto{\pgfqpoint{0.919438in}{1.434259in}}%
\pgfpathlineto{\pgfqpoint{0.924051in}{1.429292in}}%
\pgfpathlineto{\pgfqpoint{0.930117in}{1.420648in}}%
\pgfpathlineto{\pgfqpoint{0.939708in}{1.410264in}}%
\pgfpathlineto{\pgfqpoint{0.942208in}{1.407036in}}%
\pgfpathlineto{\pgfqpoint{0.955105in}{1.393425in}}%
\pgfpathlineto{\pgfqpoint{0.955364in}{1.393187in}}%
\pgfpathlineto{\pgfqpoint{0.968542in}{1.379814in}}%
\pgfpathlineto{\pgfqpoint{0.971021in}{1.377553in}}%
\pgfpathlineto{\pgfqpoint{0.982865in}{1.366203in}}%
\pgfpathlineto{\pgfqpoint{0.986678in}{1.362795in}}%
\pgfpathlineto{\pgfqpoint{0.998031in}{1.352592in}}%
\pgfpathlineto{\pgfqpoint{1.002334in}{1.348851in}}%
\pgfpathlineto{\pgfqpoint{1.014071in}{1.338981in}}%
\pgfpathlineto{\pgfqpoint{1.017991in}{1.335666in}}%
\pgfpathlineto{\pgfqpoint{1.031046in}{1.325370in}}%
\pgfpathlineto{\pgfqpoint{1.033647in}{1.323215in}}%
\pgfpathlineto{\pgfqpoint{1.049030in}{1.311759in}}%
\pgfpathlineto{\pgfqpoint{1.049304in}{1.311533in}}%
\pgfpathlineto{\pgfqpoint{1.064960in}{1.300321in}}%
\pgfpathlineto{\pgfqpoint{1.068674in}{1.298148in}}%
\pgfpathlineto{\pgfqpoint{1.080617in}{1.289809in}}%
\pgfpathlineto{\pgfqpoint{1.090561in}{1.284536in}}%
\pgfpathlineto{\pgfqpoint{1.096274in}{1.280526in}}%
\pgfpathlineto{\pgfqpoint{1.111930in}{1.273118in}}%
\pgfpathlineto{\pgfqpoint{1.121183in}{1.270925in}}%
\pgfpathlineto{\pgfqpoint{1.127587in}{1.268473in}}%
\pgfpathclose%
\pgfpathmoveto{\pgfqpoint{1.086633in}{1.325370in}}%
\pgfpathlineto{\pgfqpoint{1.080617in}{1.327694in}}%
\pgfpathlineto{\pgfqpoint{1.064960in}{1.335666in}}%
\pgfpathlineto{\pgfqpoint{1.059635in}{1.338981in}}%
\pgfpathlineto{\pgfqpoint{1.049304in}{1.345305in}}%
\pgfpathlineto{\pgfqpoint{1.039086in}{1.352592in}}%
\pgfpathlineto{\pgfqpoint{1.033647in}{1.356570in}}%
\pgfpathlineto{\pgfqpoint{1.021815in}{1.366203in}}%
\pgfpathlineto{\pgfqpoint{1.017991in}{1.369527in}}%
\pgfpathlineto{\pgfqpoint{1.006910in}{1.379814in}}%
\pgfpathlineto{\pgfqpoint{1.002334in}{1.384542in}}%
\pgfpathlineto{\pgfqpoint{0.993952in}{1.393425in}}%
\pgfpathlineto{\pgfqpoint{0.986678in}{1.402407in}}%
\pgfpathlineto{\pgfqpoint{0.982865in}{1.407036in}}%
\pgfpathlineto{\pgfqpoint{0.973694in}{1.420648in}}%
\pgfpathlineto{\pgfqpoint{0.971021in}{1.425877in}}%
\pgfpathlineto{\pgfqpoint{0.966432in}{1.434259in}}%
\pgfpathlineto{\pgfqpoint{0.961433in}{1.447870in}}%
\pgfpathlineto{\pgfqpoint{0.958936in}{1.461481in}}%
\pgfpathlineto{\pgfqpoint{0.958936in}{1.475092in}}%
\pgfpathlineto{\pgfqpoint{0.961433in}{1.488703in}}%
\pgfpathlineto{\pgfqpoint{0.966432in}{1.502314in}}%
\pgfpathlineto{\pgfqpoint{0.971021in}{1.510695in}}%
\pgfpathlineto{\pgfqpoint{0.973694in}{1.515925in}}%
\pgfpathlineto{\pgfqpoint{0.982865in}{1.529536in}}%
\pgfpathlineto{\pgfqpoint{0.986678in}{1.534166in}}%
\pgfpathlineto{\pgfqpoint{0.993952in}{1.543148in}}%
\pgfpathlineto{\pgfqpoint{1.002334in}{1.552031in}}%
\pgfpathlineto{\pgfqpoint{1.006910in}{1.556759in}}%
\pgfpathlineto{\pgfqpoint{1.017991in}{1.567045in}}%
\pgfpathlineto{\pgfqpoint{1.021815in}{1.570370in}}%
\pgfpathlineto{\pgfqpoint{1.033647in}{1.580003in}}%
\pgfpathlineto{\pgfqpoint{1.039086in}{1.583981in}}%
\pgfpathlineto{\pgfqpoint{1.049304in}{1.591268in}}%
\pgfpathlineto{\pgfqpoint{1.059635in}{1.597592in}}%
\pgfpathlineto{\pgfqpoint{1.064960in}{1.600907in}}%
\pgfpathlineto{\pgfqpoint{1.080617in}{1.608879in}}%
\pgfpathlineto{\pgfqpoint{1.086633in}{1.611203in}}%
\pgfpathlineto{\pgfqpoint{1.096274in}{1.615193in}}%
\pgfpathlineto{\pgfqpoint{1.111930in}{1.619539in}}%
\pgfpathlineto{\pgfqpoint{1.127587in}{1.621709in}}%
\pgfpathlineto{\pgfqpoint{1.143243in}{1.621709in}}%
\pgfpathlineto{\pgfqpoint{1.158900in}{1.619539in}}%
\pgfpathlineto{\pgfqpoint{1.174556in}{1.615193in}}%
\pgfpathlineto{\pgfqpoint{1.184197in}{1.611203in}}%
\pgfpathlineto{\pgfqpoint{1.190213in}{1.608879in}}%
\pgfpathlineto{\pgfqpoint{1.205870in}{1.600907in}}%
\pgfpathlineto{\pgfqpoint{1.211195in}{1.597592in}}%
\pgfpathlineto{\pgfqpoint{1.221526in}{1.591268in}}%
\pgfpathlineto{\pgfqpoint{1.231744in}{1.583981in}}%
\pgfpathlineto{\pgfqpoint{1.237183in}{1.580003in}}%
\pgfpathlineto{\pgfqpoint{1.249015in}{1.570370in}}%
\pgfpathlineto{\pgfqpoint{1.252839in}{1.567045in}}%
\pgfpathlineto{\pgfqpoint{1.263920in}{1.556759in}}%
\pgfpathlineto{\pgfqpoint{1.268496in}{1.552031in}}%
\pgfpathlineto{\pgfqpoint{1.276878in}{1.543148in}}%
\pgfpathlineto{\pgfqpoint{1.284152in}{1.534166in}}%
\pgfpathlineto{\pgfqpoint{1.287965in}{1.529536in}}%
\pgfpathlineto{\pgfqpoint{1.297136in}{1.515925in}}%
\pgfpathlineto{\pgfqpoint{1.299809in}{1.510695in}}%
\pgfpathlineto{\pgfqpoint{1.304398in}{1.502314in}}%
\pgfpathlineto{\pgfqpoint{1.309397in}{1.488703in}}%
\pgfpathlineto{\pgfqpoint{1.311894in}{1.475092in}}%
\pgfpathlineto{\pgfqpoint{1.311894in}{1.461481in}}%
\pgfpathlineto{\pgfqpoint{1.309397in}{1.447870in}}%
\pgfpathlineto{\pgfqpoint{1.304398in}{1.434259in}}%
\pgfpathlineto{\pgfqpoint{1.299809in}{1.425877in}}%
\pgfpathlineto{\pgfqpoint{1.297136in}{1.420648in}}%
\pgfpathlineto{\pgfqpoint{1.287965in}{1.407036in}}%
\pgfpathlineto{\pgfqpoint{1.284152in}{1.402407in}}%
\pgfpathlineto{\pgfqpoint{1.276878in}{1.393425in}}%
\pgfpathlineto{\pgfqpoint{1.268496in}{1.384542in}}%
\pgfpathlineto{\pgfqpoint{1.263920in}{1.379814in}}%
\pgfpathlineto{\pgfqpoint{1.252839in}{1.369527in}}%
\pgfpathlineto{\pgfqpoint{1.249015in}{1.366203in}}%
\pgfpathlineto{\pgfqpoint{1.237183in}{1.356570in}}%
\pgfpathlineto{\pgfqpoint{1.231744in}{1.352592in}}%
\pgfpathlineto{\pgfqpoint{1.221526in}{1.345305in}}%
\pgfpathlineto{\pgfqpoint{1.211195in}{1.338981in}}%
\pgfpathlineto{\pgfqpoint{1.205870in}{1.335666in}}%
\pgfpathlineto{\pgfqpoint{1.190213in}{1.327694in}}%
\pgfpathlineto{\pgfqpoint{1.184197in}{1.325370in}}%
\pgfpathlineto{\pgfqpoint{1.174556in}{1.321380in}}%
\pgfpathlineto{\pgfqpoint{1.158900in}{1.317034in}}%
\pgfpathlineto{\pgfqpoint{1.143243in}{1.314864in}}%
\pgfpathlineto{\pgfqpoint{1.127587in}{1.314864in}}%
\pgfpathlineto{\pgfqpoint{1.111930in}{1.317034in}}%
\pgfpathlineto{\pgfqpoint{1.096274in}{1.321380in}}%
\pgfpathlineto{\pgfqpoint{1.086633in}{1.325370in}}%
\pgfpathclose%
\pgfpathmoveto{\pgfqpoint{1.644253in}{1.268473in}}%
\pgfpathlineto{\pgfqpoint{1.659910in}{1.268473in}}%
\pgfpathlineto{\pgfqpoint{1.666313in}{1.270925in}}%
\pgfpathlineto{\pgfqpoint{1.675567in}{1.273118in}}%
\pgfpathlineto{\pgfqpoint{1.691223in}{1.280526in}}%
\pgfpathlineto{\pgfqpoint{1.696936in}{1.284536in}}%
\pgfpathlineto{\pgfqpoint{1.706880in}{1.289809in}}%
\pgfpathlineto{\pgfqpoint{1.718823in}{1.298148in}}%
\pgfpathlineto{\pgfqpoint{1.722536in}{1.300321in}}%
\pgfpathlineto{\pgfqpoint{1.738193in}{1.311533in}}%
\pgfpathlineto{\pgfqpoint{1.738466in}{1.311759in}}%
\pgfpathlineto{\pgfqpoint{1.753849in}{1.323215in}}%
\pgfpathlineto{\pgfqpoint{1.756451in}{1.325370in}}%
\pgfpathlineto{\pgfqpoint{1.769506in}{1.335666in}}%
\pgfpathlineto{\pgfqpoint{1.773426in}{1.338981in}}%
\pgfpathlineto{\pgfqpoint{1.785162in}{1.348851in}}%
\pgfpathlineto{\pgfqpoint{1.789466in}{1.352592in}}%
\pgfpathlineto{\pgfqpoint{1.800819in}{1.362795in}}%
\pgfpathlineto{\pgfqpoint{1.804632in}{1.366203in}}%
\pgfpathlineto{\pgfqpoint{1.816476in}{1.377553in}}%
\pgfpathlineto{\pgfqpoint{1.818954in}{1.379814in}}%
\pgfpathlineto{\pgfqpoint{1.832132in}{1.393187in}}%
\pgfpathlineto{\pgfqpoint{1.832392in}{1.393425in}}%
\pgfpathlineto{\pgfqpoint{1.845289in}{1.407036in}}%
\pgfpathlineto{\pgfqpoint{1.847789in}{1.410264in}}%
\pgfpathlineto{\pgfqpoint{1.857380in}{1.420648in}}%
\pgfpathlineto{\pgfqpoint{1.863445in}{1.429292in}}%
\pgfpathlineto{\pgfqpoint{1.868058in}{1.434259in}}%
\pgfpathlineto{\pgfqpoint{1.876580in}{1.447870in}}%
\pgfpathlineto{\pgfqpoint{1.879102in}{1.455914in}}%
\pgfpathlineto{\pgfqpoint{1.881923in}{1.461481in}}%
\pgfpathlineto{\pgfqpoint{1.881923in}{1.475092in}}%
\pgfpathlineto{\pgfqpoint{1.879102in}{1.480659in}}%
\pgfpathlineto{\pgfqpoint{1.876580in}{1.488703in}}%
\pgfpathlineto{\pgfqpoint{1.868058in}{1.502314in}}%
\pgfpathlineto{\pgfqpoint{1.863445in}{1.507281in}}%
\pgfpathlineto{\pgfqpoint{1.857380in}{1.515925in}}%
\pgfpathlineto{\pgfqpoint{1.847789in}{1.526308in}}%
\pgfpathlineto{\pgfqpoint{1.845289in}{1.529536in}}%
\pgfpathlineto{\pgfqpoint{1.832392in}{1.543148in}}%
\pgfpathlineto{\pgfqpoint{1.832132in}{1.543385in}}%
\pgfpathlineto{\pgfqpoint{1.818954in}{1.556759in}}%
\pgfpathlineto{\pgfqpoint{1.816476in}{1.559020in}}%
\pgfpathlineto{\pgfqpoint{1.804632in}{1.570370in}}%
\pgfpathlineto{\pgfqpoint{1.800819in}{1.573778in}}%
\pgfpathlineto{\pgfqpoint{1.789466in}{1.583981in}}%
\pgfpathlineto{\pgfqpoint{1.785162in}{1.587722in}}%
\pgfpathlineto{\pgfqpoint{1.773426in}{1.597592in}}%
\pgfpathlineto{\pgfqpoint{1.769506in}{1.600907in}}%
\pgfpathlineto{\pgfqpoint{1.756451in}{1.611203in}}%
\pgfpathlineto{\pgfqpoint{1.753849in}{1.613358in}}%
\pgfpathlineto{\pgfqpoint{1.738466in}{1.624814in}}%
\pgfpathlineto{\pgfqpoint{1.738193in}{1.625040in}}%
\pgfpathlineto{\pgfqpoint{1.722536in}{1.636252in}}%
\pgfpathlineto{\pgfqpoint{1.718823in}{1.638425in}}%
\pgfpathlineto{\pgfqpoint{1.706880in}{1.646763in}}%
\pgfpathlineto{\pgfqpoint{1.696936in}{1.652036in}}%
\pgfpathlineto{\pgfqpoint{1.691223in}{1.656047in}}%
\pgfpathlineto{\pgfqpoint{1.675567in}{1.663455in}}%
\pgfpathlineto{\pgfqpoint{1.666313in}{1.665648in}}%
\pgfpathlineto{\pgfqpoint{1.659910in}{1.668100in}}%
\pgfpathlineto{\pgfqpoint{1.644253in}{1.668100in}}%
\pgfpathlineto{\pgfqpoint{1.637850in}{1.665648in}}%
\pgfpathlineto{\pgfqpoint{1.628597in}{1.663455in}}%
\pgfpathlineto{\pgfqpoint{1.612940in}{1.656047in}}%
\pgfpathlineto{\pgfqpoint{1.607227in}{1.652036in}}%
\pgfpathlineto{\pgfqpoint{1.597284in}{1.646763in}}%
\pgfpathlineto{\pgfqpoint{1.585340in}{1.638425in}}%
\pgfpathlineto{\pgfqpoint{1.581627in}{1.636252in}}%
\pgfpathlineto{\pgfqpoint{1.565971in}{1.625040in}}%
\pgfpathlineto{\pgfqpoint{1.565697in}{1.624814in}}%
\pgfpathlineto{\pgfqpoint{1.550314in}{1.613358in}}%
\pgfpathlineto{\pgfqpoint{1.547712in}{1.611203in}}%
\pgfpathlineto{\pgfqpoint{1.534657in}{1.600907in}}%
\pgfpathlineto{\pgfqpoint{1.530737in}{1.597592in}}%
\pgfpathlineto{\pgfqpoint{1.519001in}{1.587722in}}%
\pgfpathlineto{\pgfqpoint{1.514698in}{1.583981in}}%
\pgfpathlineto{\pgfqpoint{1.503344in}{1.573778in}}%
\pgfpathlineto{\pgfqpoint{1.499532in}{1.570370in}}%
\pgfpathlineto{\pgfqpoint{1.487688in}{1.559020in}}%
\pgfpathlineto{\pgfqpoint{1.485209in}{1.556759in}}%
\pgfpathlineto{\pgfqpoint{1.472031in}{1.543385in}}%
\pgfpathlineto{\pgfqpoint{1.471772in}{1.543148in}}%
\pgfpathlineto{\pgfqpoint{1.458874in}{1.529536in}}%
\pgfpathlineto{\pgfqpoint{1.456375in}{1.526308in}}%
\pgfpathlineto{\pgfqpoint{1.446783in}{1.515925in}}%
\pgfpathlineto{\pgfqpoint{1.440718in}{1.507281in}}%
\pgfpathlineto{\pgfqpoint{1.436105in}{1.502314in}}%
\pgfpathlineto{\pgfqpoint{1.427583in}{1.488703in}}%
\pgfpathlineto{\pgfqpoint{1.425061in}{1.480659in}}%
\pgfpathlineto{\pgfqpoint{1.422241in}{1.475092in}}%
\pgfpathlineto{\pgfqpoint{1.422241in}{1.461481in}}%
\pgfpathlineto{\pgfqpoint{1.425061in}{1.455914in}}%
\pgfpathlineto{\pgfqpoint{1.427583in}{1.447870in}}%
\pgfpathlineto{\pgfqpoint{1.436105in}{1.434259in}}%
\pgfpathlineto{\pgfqpoint{1.440718in}{1.429292in}}%
\pgfpathlineto{\pgfqpoint{1.446783in}{1.420648in}}%
\pgfpathlineto{\pgfqpoint{1.456375in}{1.410264in}}%
\pgfpathlineto{\pgfqpoint{1.458874in}{1.407036in}}%
\pgfpathlineto{\pgfqpoint{1.471772in}{1.393425in}}%
\pgfpathlineto{\pgfqpoint{1.472031in}{1.393187in}}%
\pgfpathlineto{\pgfqpoint{1.485209in}{1.379814in}}%
\pgfpathlineto{\pgfqpoint{1.487688in}{1.377553in}}%
\pgfpathlineto{\pgfqpoint{1.499532in}{1.366203in}}%
\pgfpathlineto{\pgfqpoint{1.503344in}{1.362795in}}%
\pgfpathlineto{\pgfqpoint{1.514698in}{1.352592in}}%
\pgfpathlineto{\pgfqpoint{1.519001in}{1.348851in}}%
\pgfpathlineto{\pgfqpoint{1.530737in}{1.338981in}}%
\pgfpathlineto{\pgfqpoint{1.534657in}{1.335666in}}%
\pgfpathlineto{\pgfqpoint{1.547712in}{1.325370in}}%
\pgfpathlineto{\pgfqpoint{1.550314in}{1.323215in}}%
\pgfpathlineto{\pgfqpoint{1.565697in}{1.311759in}}%
\pgfpathlineto{\pgfqpoint{1.565971in}{1.311533in}}%
\pgfpathlineto{\pgfqpoint{1.581627in}{1.300321in}}%
\pgfpathlineto{\pgfqpoint{1.585340in}{1.298148in}}%
\pgfpathlineto{\pgfqpoint{1.597284in}{1.289809in}}%
\pgfpathlineto{\pgfqpoint{1.607227in}{1.284536in}}%
\pgfpathlineto{\pgfqpoint{1.612940in}{1.280526in}}%
\pgfpathlineto{\pgfqpoint{1.628597in}{1.273118in}}%
\pgfpathlineto{\pgfqpoint{1.637850in}{1.270925in}}%
\pgfpathlineto{\pgfqpoint{1.644253in}{1.268473in}}%
\pgfpathclose%
\pgfpathmoveto{\pgfqpoint{1.603299in}{1.325370in}}%
\pgfpathlineto{\pgfqpoint{1.597284in}{1.327694in}}%
\pgfpathlineto{\pgfqpoint{1.581627in}{1.335666in}}%
\pgfpathlineto{\pgfqpoint{1.576301in}{1.338981in}}%
\pgfpathlineto{\pgfqpoint{1.565971in}{1.345305in}}%
\pgfpathlineto{\pgfqpoint{1.555753in}{1.352592in}}%
\pgfpathlineto{\pgfqpoint{1.550314in}{1.356570in}}%
\pgfpathlineto{\pgfqpoint{1.538481in}{1.366203in}}%
\pgfpathlineto{\pgfqpoint{1.534657in}{1.369527in}}%
\pgfpathlineto{\pgfqpoint{1.523577in}{1.379814in}}%
\pgfpathlineto{\pgfqpoint{1.519001in}{1.384542in}}%
\pgfpathlineto{\pgfqpoint{1.510618in}{1.393425in}}%
\pgfpathlineto{\pgfqpoint{1.503344in}{1.402407in}}%
\pgfpathlineto{\pgfqpoint{1.499531in}{1.407036in}}%
\pgfpathlineto{\pgfqpoint{1.490361in}{1.420648in}}%
\pgfpathlineto{\pgfqpoint{1.487688in}{1.425877in}}%
\pgfpathlineto{\pgfqpoint{1.483099in}{1.434259in}}%
\pgfpathlineto{\pgfqpoint{1.478099in}{1.447870in}}%
\pgfpathlineto{\pgfqpoint{1.475603in}{1.461481in}}%
\pgfpathlineto{\pgfqpoint{1.475603in}{1.475092in}}%
\pgfpathlineto{\pgfqpoint{1.478099in}{1.488703in}}%
\pgfpathlineto{\pgfqpoint{1.483099in}{1.502314in}}%
\pgfpathlineto{\pgfqpoint{1.487688in}{1.510695in}}%
\pgfpathlineto{\pgfqpoint{1.490361in}{1.515925in}}%
\pgfpathlineto{\pgfqpoint{1.499531in}{1.529536in}}%
\pgfpathlineto{\pgfqpoint{1.503344in}{1.534166in}}%
\pgfpathlineto{\pgfqpoint{1.510618in}{1.543148in}}%
\pgfpathlineto{\pgfqpoint{1.519001in}{1.552031in}}%
\pgfpathlineto{\pgfqpoint{1.523577in}{1.556759in}}%
\pgfpathlineto{\pgfqpoint{1.534657in}{1.567045in}}%
\pgfpathlineto{\pgfqpoint{1.538481in}{1.570370in}}%
\pgfpathlineto{\pgfqpoint{1.550314in}{1.580003in}}%
\pgfpathlineto{\pgfqpoint{1.555753in}{1.583981in}}%
\pgfpathlineto{\pgfqpoint{1.565971in}{1.591268in}}%
\pgfpathlineto{\pgfqpoint{1.576301in}{1.597592in}}%
\pgfpathlineto{\pgfqpoint{1.581627in}{1.600907in}}%
\pgfpathlineto{\pgfqpoint{1.597284in}{1.608879in}}%
\pgfpathlineto{\pgfqpoint{1.603299in}{1.611203in}}%
\pgfpathlineto{\pgfqpoint{1.612940in}{1.615193in}}%
\pgfpathlineto{\pgfqpoint{1.628597in}{1.619539in}}%
\pgfpathlineto{\pgfqpoint{1.644253in}{1.621709in}}%
\pgfpathlineto{\pgfqpoint{1.659910in}{1.621709in}}%
\pgfpathlineto{\pgfqpoint{1.675567in}{1.619539in}}%
\pgfpathlineto{\pgfqpoint{1.691223in}{1.615193in}}%
\pgfpathlineto{\pgfqpoint{1.700864in}{1.611203in}}%
\pgfpathlineto{\pgfqpoint{1.706880in}{1.608879in}}%
\pgfpathlineto{\pgfqpoint{1.722536in}{1.600907in}}%
\pgfpathlineto{\pgfqpoint{1.727862in}{1.597592in}}%
\pgfpathlineto{\pgfqpoint{1.738193in}{1.591268in}}%
\pgfpathlineto{\pgfqpoint{1.748411in}{1.583981in}}%
\pgfpathlineto{\pgfqpoint{1.753849in}{1.580003in}}%
\pgfpathlineto{\pgfqpoint{1.765682in}{1.570370in}}%
\pgfpathlineto{\pgfqpoint{1.769506in}{1.567045in}}%
\pgfpathlineto{\pgfqpoint{1.780587in}{1.556759in}}%
\pgfpathlineto{\pgfqpoint{1.785162in}{1.552031in}}%
\pgfpathlineto{\pgfqpoint{1.793545in}{1.543148in}}%
\pgfpathlineto{\pgfqpoint{1.800819in}{1.534166in}}%
\pgfpathlineto{\pgfqpoint{1.804632in}{1.529536in}}%
\pgfpathlineto{\pgfqpoint{1.813802in}{1.515925in}}%
\pgfpathlineto{\pgfqpoint{1.816476in}{1.510695in}}%
\pgfpathlineto{\pgfqpoint{1.821065in}{1.502314in}}%
\pgfpathlineto{\pgfqpoint{1.826064in}{1.488703in}}%
\pgfpathlineto{\pgfqpoint{1.828560in}{1.475092in}}%
\pgfpathlineto{\pgfqpoint{1.828560in}{1.461481in}}%
\pgfpathlineto{\pgfqpoint{1.826064in}{1.447870in}}%
\pgfpathlineto{\pgfqpoint{1.821065in}{1.434259in}}%
\pgfpathlineto{\pgfqpoint{1.816476in}{1.425877in}}%
\pgfpathlineto{\pgfqpoint{1.813802in}{1.420648in}}%
\pgfpathlineto{\pgfqpoint{1.804632in}{1.407036in}}%
\pgfpathlineto{\pgfqpoint{1.800819in}{1.402407in}}%
\pgfpathlineto{\pgfqpoint{1.793545in}{1.393425in}}%
\pgfpathlineto{\pgfqpoint{1.785162in}{1.384542in}}%
\pgfpathlineto{\pgfqpoint{1.780587in}{1.379814in}}%
\pgfpathlineto{\pgfqpoint{1.769506in}{1.369527in}}%
\pgfpathlineto{\pgfqpoint{1.765682in}{1.366203in}}%
\pgfpathlineto{\pgfqpoint{1.753849in}{1.356570in}}%
\pgfpathlineto{\pgfqpoint{1.748411in}{1.352592in}}%
\pgfpathlineto{\pgfqpoint{1.738193in}{1.345305in}}%
\pgfpathlineto{\pgfqpoint{1.727862in}{1.338981in}}%
\pgfpathlineto{\pgfqpoint{1.722536in}{1.335666in}}%
\pgfpathlineto{\pgfqpoint{1.706880in}{1.327694in}}%
\pgfpathlineto{\pgfqpoint{1.700864in}{1.325370in}}%
\pgfpathlineto{\pgfqpoint{1.691223in}{1.321380in}}%
\pgfpathlineto{\pgfqpoint{1.675567in}{1.317034in}}%
\pgfpathlineto{\pgfqpoint{1.659910in}{1.314864in}}%
\pgfpathlineto{\pgfqpoint{1.644253in}{1.314864in}}%
\pgfpathlineto{\pgfqpoint{1.628597in}{1.317034in}}%
\pgfpathlineto{\pgfqpoint{1.612940in}{1.321380in}}%
\pgfpathlineto{\pgfqpoint{1.603299in}{1.325370in}}%
\pgfpathclose%
\pgfusepath{fill}%
\end{pgfscope}%
\begin{pgfscope}%
\pgfpathrectangle{\pgfqpoint{0.360415in}{0.345370in}}{\pgfqpoint{1.550000in}{1.347500in}}%
\pgfusepath{clip}%
\pgfsetbuttcap%
\pgfsetroundjoin%
\definecolor{currentfill}{rgb}{0.679160,0.151848,0.575189}%
\pgfsetfillcolor{currentfill}%
\pgfsetlinewidth{0.000000pt}%
\definecolor{currentstroke}{rgb}{0.000000,0.000000,0.000000}%
\pgfsetstrokecolor{currentstroke}%
\pgfsetdash{}{0pt}%
\pgfpathmoveto{\pgfqpoint{0.532637in}{0.345370in}}%
\pgfpathlineto{\pgfqpoint{0.548294in}{0.345370in}}%
\pgfpathlineto{\pgfqpoint{0.563950in}{0.345370in}}%
\pgfpathlineto{\pgfqpoint{0.579607in}{0.345370in}}%
\pgfpathlineto{\pgfqpoint{0.595263in}{0.345370in}}%
\pgfpathlineto{\pgfqpoint{0.610920in}{0.345370in}}%
\pgfpathlineto{\pgfqpoint{0.626577in}{0.345370in}}%
\pgfpathlineto{\pgfqpoint{0.642233in}{0.345370in}}%
\pgfpathlineto{\pgfqpoint{0.657890in}{0.345370in}}%
\pgfpathlineto{\pgfqpoint{0.673546in}{0.345370in}}%
\pgfpathlineto{\pgfqpoint{0.689203in}{0.345370in}}%
\pgfpathlineto{\pgfqpoint{0.704859in}{0.345370in}}%
\pgfpathlineto{\pgfqpoint{0.707123in}{0.345370in}}%
\pgfpathlineto{\pgfqpoint{0.709420in}{0.358981in}}%
\pgfpathlineto{\pgfqpoint{0.716096in}{0.372592in}}%
\pgfpathlineto{\pgfqpoint{0.720516in}{0.378272in}}%
\pgfpathlineto{\pgfqpoint{0.726160in}{0.386203in}}%
\pgfpathlineto{\pgfqpoint{0.736173in}{0.397043in}}%
\pgfpathlineto{\pgfqpoint{0.738588in}{0.399814in}}%
\pgfpathlineto{\pgfqpoint{0.751829in}{0.412763in}}%
\pgfpathlineto{\pgfqpoint{0.752488in}{0.413425in}}%
\pgfpathlineto{\pgfqpoint{0.767368in}{0.427036in}}%
\pgfpathlineto{\pgfqpoint{0.767486in}{0.427139in}}%
\pgfpathlineto{\pgfqpoint{0.783024in}{0.440648in}}%
\pgfpathlineto{\pgfqpoint{0.783142in}{0.440750in}}%
\pgfpathlineto{\pgfqpoint{0.798799in}{0.453686in}}%
\pgfpathlineto{\pgfqpoint{0.799561in}{0.454259in}}%
\pgfpathlineto{\pgfqpoint{0.814455in}{0.465770in}}%
\pgfpathlineto{\pgfqpoint{0.817643in}{0.467870in}}%
\pgfpathlineto{\pgfqpoint{0.830112in}{0.476574in}}%
\pgfpathlineto{\pgfqpoint{0.839234in}{0.481481in}}%
\pgfpathlineto{\pgfqpoint{0.845769in}{0.485323in}}%
\pgfpathlineto{\pgfqpoint{0.861425in}{0.491127in}}%
\pgfpathlineto{\pgfqpoint{0.877082in}{0.493124in}}%
\pgfpathlineto{\pgfqpoint{0.892738in}{0.491127in}}%
\pgfpathlineto{\pgfqpoint{0.908395in}{0.485323in}}%
\pgfpathlineto{\pgfqpoint{0.914929in}{0.481481in}}%
\pgfpathlineto{\pgfqpoint{0.924051in}{0.476574in}}%
\pgfpathlineto{\pgfqpoint{0.936520in}{0.467870in}}%
\pgfpathlineto{\pgfqpoint{0.939708in}{0.465770in}}%
\pgfpathlineto{\pgfqpoint{0.954603in}{0.454259in}}%
\pgfpathlineto{\pgfqpoint{0.955364in}{0.453686in}}%
\pgfpathlineto{\pgfqpoint{0.971021in}{0.440750in}}%
\pgfpathlineto{\pgfqpoint{0.971139in}{0.440648in}}%
\pgfpathlineto{\pgfqpoint{0.986678in}{0.427139in}}%
\pgfpathlineto{\pgfqpoint{0.986795in}{0.427036in}}%
\pgfpathlineto{\pgfqpoint{1.001675in}{0.413425in}}%
\pgfpathlineto{\pgfqpoint{1.002334in}{0.412763in}}%
\pgfpathlineto{\pgfqpoint{1.015575in}{0.399814in}}%
\pgfpathlineto{\pgfqpoint{1.017991in}{0.397043in}}%
\pgfpathlineto{\pgfqpoint{1.028003in}{0.386203in}}%
\pgfpathlineto{\pgfqpoint{1.033647in}{0.378272in}}%
\pgfpathlineto{\pgfqpoint{1.038067in}{0.372592in}}%
\pgfpathlineto{\pgfqpoint{1.044743in}{0.358981in}}%
\pgfpathlineto{\pgfqpoint{1.047041in}{0.345370in}}%
\pgfpathlineto{\pgfqpoint{1.049304in}{0.345370in}}%
\pgfpathlineto{\pgfqpoint{1.064960in}{0.345370in}}%
\pgfpathlineto{\pgfqpoint{1.080617in}{0.345370in}}%
\pgfpathlineto{\pgfqpoint{1.096274in}{0.345370in}}%
\pgfpathlineto{\pgfqpoint{1.111930in}{0.345370in}}%
\pgfpathlineto{\pgfqpoint{1.127587in}{0.345370in}}%
\pgfpathlineto{\pgfqpoint{1.143243in}{0.345370in}}%
\pgfpathlineto{\pgfqpoint{1.158900in}{0.345370in}}%
\pgfpathlineto{\pgfqpoint{1.174556in}{0.345370in}}%
\pgfpathlineto{\pgfqpoint{1.190213in}{0.345370in}}%
\pgfpathlineto{\pgfqpoint{1.205870in}{0.345370in}}%
\pgfpathlineto{\pgfqpoint{1.221526in}{0.345370in}}%
\pgfpathlineto{\pgfqpoint{1.223789in}{0.345370in}}%
\pgfpathlineto{\pgfqpoint{1.226087in}{0.358981in}}%
\pgfpathlineto{\pgfqpoint{1.232763in}{0.372592in}}%
\pgfpathlineto{\pgfqpoint{1.237183in}{0.378272in}}%
\pgfpathlineto{\pgfqpoint{1.242827in}{0.386203in}}%
\pgfpathlineto{\pgfqpoint{1.252839in}{0.397043in}}%
\pgfpathlineto{\pgfqpoint{1.255255in}{0.399814in}}%
\pgfpathlineto{\pgfqpoint{1.268496in}{0.412763in}}%
\pgfpathlineto{\pgfqpoint{1.269155in}{0.413425in}}%
\pgfpathlineto{\pgfqpoint{1.284035in}{0.427036in}}%
\pgfpathlineto{\pgfqpoint{1.284152in}{0.427139in}}%
\pgfpathlineto{\pgfqpoint{1.299691in}{0.440648in}}%
\pgfpathlineto{\pgfqpoint{1.299809in}{0.440750in}}%
\pgfpathlineto{\pgfqpoint{1.315466in}{0.453686in}}%
\pgfpathlineto{\pgfqpoint{1.316227in}{0.454259in}}%
\pgfpathlineto{\pgfqpoint{1.331122in}{0.465770in}}%
\pgfpathlineto{\pgfqpoint{1.334310in}{0.467870in}}%
\pgfpathlineto{\pgfqpoint{1.346779in}{0.476574in}}%
\pgfpathlineto{\pgfqpoint{1.355901in}{0.481481in}}%
\pgfpathlineto{\pgfqpoint{1.362435in}{0.485323in}}%
\pgfpathlineto{\pgfqpoint{1.378092in}{0.491127in}}%
\pgfpathlineto{\pgfqpoint{1.393748in}{0.493124in}}%
\pgfpathlineto{\pgfqpoint{1.409405in}{0.491127in}}%
\pgfpathlineto{\pgfqpoint{1.425061in}{0.485323in}}%
\pgfpathlineto{\pgfqpoint{1.431596in}{0.481481in}}%
\pgfpathlineto{\pgfqpoint{1.440718in}{0.476574in}}%
\pgfpathlineto{\pgfqpoint{1.453187in}{0.467870in}}%
\pgfpathlineto{\pgfqpoint{1.456375in}{0.465770in}}%
\pgfpathlineto{\pgfqpoint{1.471269in}{0.454259in}}%
\pgfpathlineto{\pgfqpoint{1.472031in}{0.453686in}}%
\pgfpathlineto{\pgfqpoint{1.487688in}{0.440750in}}%
\pgfpathlineto{\pgfqpoint{1.487806in}{0.440648in}}%
\pgfpathlineto{\pgfqpoint{1.503344in}{0.427139in}}%
\pgfpathlineto{\pgfqpoint{1.503462in}{0.427036in}}%
\pgfpathlineto{\pgfqpoint{1.518342in}{0.413425in}}%
\pgfpathlineto{\pgfqpoint{1.519001in}{0.412763in}}%
\pgfpathlineto{\pgfqpoint{1.532242in}{0.399814in}}%
\pgfpathlineto{\pgfqpoint{1.534657in}{0.397043in}}%
\pgfpathlineto{\pgfqpoint{1.544670in}{0.386203in}}%
\pgfpathlineto{\pgfqpoint{1.550314in}{0.378272in}}%
\pgfpathlineto{\pgfqpoint{1.554734in}{0.372592in}}%
\pgfpathlineto{\pgfqpoint{1.561410in}{0.358981in}}%
\pgfpathlineto{\pgfqpoint{1.563707in}{0.345370in}}%
\pgfpathlineto{\pgfqpoint{1.565971in}{0.345370in}}%
\pgfpathlineto{\pgfqpoint{1.581627in}{0.345370in}}%
\pgfpathlineto{\pgfqpoint{1.597284in}{0.345370in}}%
\pgfpathlineto{\pgfqpoint{1.612940in}{0.345370in}}%
\pgfpathlineto{\pgfqpoint{1.628597in}{0.345370in}}%
\pgfpathlineto{\pgfqpoint{1.644253in}{0.345370in}}%
\pgfpathlineto{\pgfqpoint{1.659910in}{0.345370in}}%
\pgfpathlineto{\pgfqpoint{1.675567in}{0.345370in}}%
\pgfpathlineto{\pgfqpoint{1.691223in}{0.345370in}}%
\pgfpathlineto{\pgfqpoint{1.706880in}{0.345370in}}%
\pgfpathlineto{\pgfqpoint{1.722536in}{0.345370in}}%
\pgfpathlineto{\pgfqpoint{1.738193in}{0.345370in}}%
\pgfpathlineto{\pgfqpoint{1.740456in}{0.345370in}}%
\pgfpathlineto{\pgfqpoint{1.742754in}{0.358981in}}%
\pgfpathlineto{\pgfqpoint{1.749429in}{0.372592in}}%
\pgfpathlineto{\pgfqpoint{1.753849in}{0.378272in}}%
\pgfpathlineto{\pgfqpoint{1.759494in}{0.386203in}}%
\pgfpathlineto{\pgfqpoint{1.769506in}{0.397043in}}%
\pgfpathlineto{\pgfqpoint{1.771922in}{0.399814in}}%
\pgfpathlineto{\pgfqpoint{1.785162in}{0.412763in}}%
\pgfpathlineto{\pgfqpoint{1.785822in}{0.413425in}}%
\pgfpathlineto{\pgfqpoint{1.800701in}{0.427036in}}%
\pgfpathlineto{\pgfqpoint{1.800819in}{0.427139in}}%
\pgfpathlineto{\pgfqpoint{1.816358in}{0.440648in}}%
\pgfpathlineto{\pgfqpoint{1.816476in}{0.440750in}}%
\pgfpathlineto{\pgfqpoint{1.832132in}{0.453686in}}%
\pgfpathlineto{\pgfqpoint{1.832894in}{0.454259in}}%
\pgfpathlineto{\pgfqpoint{1.847789in}{0.465770in}}%
\pgfpathlineto{\pgfqpoint{1.850977in}{0.467870in}}%
\pgfpathlineto{\pgfqpoint{1.863445in}{0.476574in}}%
\pgfpathlineto{\pgfqpoint{1.872568in}{0.481481in}}%
\pgfpathlineto{\pgfqpoint{1.879102in}{0.485323in}}%
\pgfpathlineto{\pgfqpoint{1.894758in}{0.491127in}}%
\pgfpathlineto{\pgfqpoint{1.910415in}{0.493124in}}%
\pgfpathlineto{\pgfqpoint{1.910415in}{0.495092in}}%
\pgfpathlineto{\pgfqpoint{1.910415in}{0.508703in}}%
\pgfpathlineto{\pgfqpoint{1.910415in}{0.522314in}}%
\pgfpathlineto{\pgfqpoint{1.910415in}{0.535925in}}%
\pgfpathlineto{\pgfqpoint{1.910415in}{0.549536in}}%
\pgfpathlineto{\pgfqpoint{1.910415in}{0.563148in}}%
\pgfpathlineto{\pgfqpoint{1.910415in}{0.576759in}}%
\pgfpathlineto{\pgfqpoint{1.910415in}{0.590370in}}%
\pgfpathlineto{\pgfqpoint{1.910415in}{0.603981in}}%
\pgfpathlineto{\pgfqpoint{1.910415in}{0.617592in}}%
\pgfpathlineto{\pgfqpoint{1.910415in}{0.631203in}}%
\pgfpathlineto{\pgfqpoint{1.910415in}{0.644814in}}%
\pgfpathlineto{\pgfqpoint{1.910415in}{0.646782in}}%
\pgfpathlineto{\pgfqpoint{1.894758in}{0.648779in}}%
\pgfpathlineto{\pgfqpoint{1.879102in}{0.654583in}}%
\pgfpathlineto{\pgfqpoint{1.872568in}{0.658425in}}%
\pgfpathlineto{\pgfqpoint{1.863445in}{0.663332in}}%
\pgfpathlineto{\pgfqpoint{1.850977in}{0.672036in}}%
\pgfpathlineto{\pgfqpoint{1.847789in}{0.674136in}}%
\pgfpathlineto{\pgfqpoint{1.832894in}{0.685648in}}%
\pgfpathlineto{\pgfqpoint{1.832132in}{0.686220in}}%
\pgfpathlineto{\pgfqpoint{1.816476in}{0.699156in}}%
\pgfpathlineto{\pgfqpoint{1.816358in}{0.699259in}}%
\pgfpathlineto{\pgfqpoint{1.800819in}{0.712767in}}%
\pgfpathlineto{\pgfqpoint{1.800701in}{0.712870in}}%
\pgfpathlineto{\pgfqpoint{1.785822in}{0.726481in}}%
\pgfpathlineto{\pgfqpoint{1.785162in}{0.727143in}}%
\pgfpathlineto{\pgfqpoint{1.771922in}{0.740092in}}%
\pgfpathlineto{\pgfqpoint{1.769506in}{0.742863in}}%
\pgfpathlineto{\pgfqpoint{1.759494in}{0.753703in}}%
\pgfpathlineto{\pgfqpoint{1.753849in}{0.761634in}}%
\pgfpathlineto{\pgfqpoint{1.749429in}{0.767314in}}%
\pgfpathlineto{\pgfqpoint{1.742754in}{0.780925in}}%
\pgfpathlineto{\pgfqpoint{1.740456in}{0.794536in}}%
\pgfpathlineto{\pgfqpoint{1.742754in}{0.808148in}}%
\pgfpathlineto{\pgfqpoint{1.749429in}{0.821759in}}%
\pgfpathlineto{\pgfqpoint{1.753849in}{0.827439in}}%
\pgfpathlineto{\pgfqpoint{1.759494in}{0.835370in}}%
\pgfpathlineto{\pgfqpoint{1.769506in}{0.846209in}}%
\pgfpathlineto{\pgfqpoint{1.771922in}{0.848981in}}%
\pgfpathlineto{\pgfqpoint{1.785162in}{0.861930in}}%
\pgfpathlineto{\pgfqpoint{1.785822in}{0.862592in}}%
\pgfpathlineto{\pgfqpoint{1.800701in}{0.876203in}}%
\pgfpathlineto{\pgfqpoint{1.800819in}{0.876306in}}%
\pgfpathlineto{\pgfqpoint{1.816358in}{0.889814in}}%
\pgfpathlineto{\pgfqpoint{1.816476in}{0.889916in}}%
\pgfpathlineto{\pgfqpoint{1.832132in}{0.902852in}}%
\pgfpathlineto{\pgfqpoint{1.832894in}{0.903425in}}%
\pgfpathlineto{\pgfqpoint{1.847789in}{0.914936in}}%
\pgfpathlineto{\pgfqpoint{1.850977in}{0.917036in}}%
\pgfpathlineto{\pgfqpoint{1.863445in}{0.925741in}}%
\pgfpathlineto{\pgfqpoint{1.872568in}{0.930648in}}%
\pgfpathlineto{\pgfqpoint{1.879102in}{0.934490in}}%
\pgfpathlineto{\pgfqpoint{1.894758in}{0.940293in}}%
\pgfpathlineto{\pgfqpoint{1.910415in}{0.942291in}}%
\pgfpathlineto{\pgfqpoint{1.910415in}{0.944259in}}%
\pgfpathlineto{\pgfqpoint{1.910415in}{0.957870in}}%
\pgfpathlineto{\pgfqpoint{1.910415in}{0.971481in}}%
\pgfpathlineto{\pgfqpoint{1.910415in}{0.985092in}}%
\pgfpathlineto{\pgfqpoint{1.910415in}{0.998703in}}%
\pgfpathlineto{\pgfqpoint{1.910415in}{1.012314in}}%
\pgfpathlineto{\pgfqpoint{1.910415in}{1.025925in}}%
\pgfpathlineto{\pgfqpoint{1.910415in}{1.039536in}}%
\pgfpathlineto{\pgfqpoint{1.910415in}{1.053148in}}%
\pgfpathlineto{\pgfqpoint{1.910415in}{1.066759in}}%
\pgfpathlineto{\pgfqpoint{1.910415in}{1.080370in}}%
\pgfpathlineto{\pgfqpoint{1.910415in}{1.093981in}}%
\pgfpathlineto{\pgfqpoint{1.910415in}{1.095948in}}%
\pgfpathlineto{\pgfqpoint{1.894758in}{1.097946in}}%
\pgfpathlineto{\pgfqpoint{1.879102in}{1.103749in}}%
\pgfpathlineto{\pgfqpoint{1.872568in}{1.107592in}}%
\pgfpathlineto{\pgfqpoint{1.863445in}{1.112499in}}%
\pgfpathlineto{\pgfqpoint{1.850977in}{1.121203in}}%
\pgfpathlineto{\pgfqpoint{1.847789in}{1.123303in}}%
\pgfpathlineto{\pgfqpoint{1.832894in}{1.134814in}}%
\pgfpathlineto{\pgfqpoint{1.832132in}{1.135387in}}%
\pgfpathlineto{\pgfqpoint{1.816476in}{1.148323in}}%
\pgfpathlineto{\pgfqpoint{1.816358in}{1.148425in}}%
\pgfpathlineto{\pgfqpoint{1.800819in}{1.161934in}}%
\pgfpathlineto{\pgfqpoint{1.800701in}{1.162036in}}%
\pgfpathlineto{\pgfqpoint{1.785822in}{1.175647in}}%
\pgfpathlineto{\pgfqpoint{1.785162in}{1.176310in}}%
\pgfpathlineto{\pgfqpoint{1.771922in}{1.189259in}}%
\pgfpathlineto{\pgfqpoint{1.769506in}{1.192030in}}%
\pgfpathlineto{\pgfqpoint{1.759494in}{1.202870in}}%
\pgfpathlineto{\pgfqpoint{1.753849in}{1.210800in}}%
\pgfpathlineto{\pgfqpoint{1.749429in}{1.216481in}}%
\pgfpathlineto{\pgfqpoint{1.742754in}{1.230092in}}%
\pgfpathlineto{\pgfqpoint{1.740456in}{1.243703in}}%
\pgfpathlineto{\pgfqpoint{1.742754in}{1.257314in}}%
\pgfpathlineto{\pgfqpoint{1.749429in}{1.270925in}}%
\pgfpathlineto{\pgfqpoint{1.753849in}{1.276606in}}%
\pgfpathlineto{\pgfqpoint{1.759494in}{1.284536in}}%
\pgfpathlineto{\pgfqpoint{1.769506in}{1.295376in}}%
\pgfpathlineto{\pgfqpoint{1.771922in}{1.298148in}}%
\pgfpathlineto{\pgfqpoint{1.785162in}{1.311096in}}%
\pgfpathlineto{\pgfqpoint{1.785822in}{1.311759in}}%
\pgfpathlineto{\pgfqpoint{1.800701in}{1.325370in}}%
\pgfpathlineto{\pgfqpoint{1.800819in}{1.325472in}}%
\pgfpathlineto{\pgfqpoint{1.816358in}{1.338981in}}%
\pgfpathlineto{\pgfqpoint{1.816476in}{1.339083in}}%
\pgfpathlineto{\pgfqpoint{1.832132in}{1.352019in}}%
\pgfpathlineto{\pgfqpoint{1.832894in}{1.352592in}}%
\pgfpathlineto{\pgfqpoint{1.847789in}{1.364103in}}%
\pgfpathlineto{\pgfqpoint{1.850977in}{1.366203in}}%
\pgfpathlineto{\pgfqpoint{1.863445in}{1.374907in}}%
\pgfpathlineto{\pgfqpoint{1.872568in}{1.379814in}}%
\pgfpathlineto{\pgfqpoint{1.879102in}{1.383657in}}%
\pgfpathlineto{\pgfqpoint{1.894758in}{1.389460in}}%
\pgfpathlineto{\pgfqpoint{1.910415in}{1.391458in}}%
\pgfpathlineto{\pgfqpoint{1.910415in}{1.393425in}}%
\pgfpathlineto{\pgfqpoint{1.910415in}{1.407036in}}%
\pgfpathlineto{\pgfqpoint{1.910415in}{1.420648in}}%
\pgfpathlineto{\pgfqpoint{1.910415in}{1.434259in}}%
\pgfpathlineto{\pgfqpoint{1.910415in}{1.447870in}}%
\pgfpathlineto{\pgfqpoint{1.910415in}{1.461481in}}%
\pgfpathlineto{\pgfqpoint{1.910415in}{1.475092in}}%
\pgfpathlineto{\pgfqpoint{1.910415in}{1.488703in}}%
\pgfpathlineto{\pgfqpoint{1.910415in}{1.502314in}}%
\pgfpathlineto{\pgfqpoint{1.910415in}{1.515925in}}%
\pgfpathlineto{\pgfqpoint{1.910415in}{1.529536in}}%
\pgfpathlineto{\pgfqpoint{1.910415in}{1.543148in}}%
\pgfpathlineto{\pgfqpoint{1.910415in}{1.545115in}}%
\pgfpathlineto{\pgfqpoint{1.894758in}{1.547113in}}%
\pgfpathlineto{\pgfqpoint{1.879102in}{1.552916in}}%
\pgfpathlineto{\pgfqpoint{1.872568in}{1.556759in}}%
\pgfpathlineto{\pgfqpoint{1.863445in}{1.561665in}}%
\pgfpathlineto{\pgfqpoint{1.850977in}{1.570370in}}%
\pgfpathlineto{\pgfqpoint{1.847789in}{1.572470in}}%
\pgfpathlineto{\pgfqpoint{1.832894in}{1.583981in}}%
\pgfpathlineto{\pgfqpoint{1.832132in}{1.584554in}}%
\pgfpathlineto{\pgfqpoint{1.816476in}{1.597490in}}%
\pgfpathlineto{\pgfqpoint{1.816358in}{1.597592in}}%
\pgfpathlineto{\pgfqpoint{1.800819in}{1.611100in}}%
\pgfpathlineto{\pgfqpoint{1.800701in}{1.611203in}}%
\pgfpathlineto{\pgfqpoint{1.785822in}{1.624814in}}%
\pgfpathlineto{\pgfqpoint{1.785162in}{1.625476in}}%
\pgfpathlineto{\pgfqpoint{1.771922in}{1.638425in}}%
\pgfpathlineto{\pgfqpoint{1.769506in}{1.641197in}}%
\pgfpathlineto{\pgfqpoint{1.759494in}{1.652036in}}%
\pgfpathlineto{\pgfqpoint{1.753849in}{1.659967in}}%
\pgfpathlineto{\pgfqpoint{1.749429in}{1.665648in}}%
\pgfpathlineto{\pgfqpoint{1.742754in}{1.679259in}}%
\pgfpathlineto{\pgfqpoint{1.740456in}{1.692870in}}%
\pgfpathlineto{\pgfqpoint{1.738193in}{1.692870in}}%
\pgfpathlineto{\pgfqpoint{1.722536in}{1.692870in}}%
\pgfpathlineto{\pgfqpoint{1.706880in}{1.692870in}}%
\pgfpathlineto{\pgfqpoint{1.691223in}{1.692870in}}%
\pgfpathlineto{\pgfqpoint{1.675567in}{1.692870in}}%
\pgfpathlineto{\pgfqpoint{1.659910in}{1.692870in}}%
\pgfpathlineto{\pgfqpoint{1.644253in}{1.692870in}}%
\pgfpathlineto{\pgfqpoint{1.628597in}{1.692870in}}%
\pgfpathlineto{\pgfqpoint{1.612940in}{1.692870in}}%
\pgfpathlineto{\pgfqpoint{1.597284in}{1.692870in}}%
\pgfpathlineto{\pgfqpoint{1.581627in}{1.692870in}}%
\pgfpathlineto{\pgfqpoint{1.565971in}{1.692870in}}%
\pgfpathlineto{\pgfqpoint{1.563707in}{1.692870in}}%
\pgfpathlineto{\pgfqpoint{1.561410in}{1.679259in}}%
\pgfpathlineto{\pgfqpoint{1.554734in}{1.665648in}}%
\pgfpathlineto{\pgfqpoint{1.550314in}{1.659967in}}%
\pgfpathlineto{\pgfqpoint{1.544670in}{1.652036in}}%
\pgfpathlineto{\pgfqpoint{1.534657in}{1.641197in}}%
\pgfpathlineto{\pgfqpoint{1.532242in}{1.638425in}}%
\pgfpathlineto{\pgfqpoint{1.519001in}{1.625476in}}%
\pgfpathlineto{\pgfqpoint{1.518342in}{1.624814in}}%
\pgfpathlineto{\pgfqpoint{1.503462in}{1.611203in}}%
\pgfpathlineto{\pgfqpoint{1.503344in}{1.611100in}}%
\pgfpathlineto{\pgfqpoint{1.487806in}{1.597592in}}%
\pgfpathlineto{\pgfqpoint{1.487688in}{1.597490in}}%
\pgfpathlineto{\pgfqpoint{1.472031in}{1.584554in}}%
\pgfpathlineto{\pgfqpoint{1.471269in}{1.583981in}}%
\pgfpathlineto{\pgfqpoint{1.456375in}{1.572470in}}%
\pgfpathlineto{\pgfqpoint{1.453187in}{1.570370in}}%
\pgfpathlineto{\pgfqpoint{1.440718in}{1.561665in}}%
\pgfpathlineto{\pgfqpoint{1.431596in}{1.556759in}}%
\pgfpathlineto{\pgfqpoint{1.425061in}{1.552916in}}%
\pgfpathlineto{\pgfqpoint{1.409405in}{1.547113in}}%
\pgfpathlineto{\pgfqpoint{1.393748in}{1.545115in}}%
\pgfpathlineto{\pgfqpoint{1.378092in}{1.547113in}}%
\pgfpathlineto{\pgfqpoint{1.362435in}{1.552916in}}%
\pgfpathlineto{\pgfqpoint{1.355901in}{1.556759in}}%
\pgfpathlineto{\pgfqpoint{1.346779in}{1.561665in}}%
\pgfpathlineto{\pgfqpoint{1.334310in}{1.570370in}}%
\pgfpathlineto{\pgfqpoint{1.331122in}{1.572470in}}%
\pgfpathlineto{\pgfqpoint{1.316227in}{1.583981in}}%
\pgfpathlineto{\pgfqpoint{1.315466in}{1.584554in}}%
\pgfpathlineto{\pgfqpoint{1.299809in}{1.597490in}}%
\pgfpathlineto{\pgfqpoint{1.299691in}{1.597592in}}%
\pgfpathlineto{\pgfqpoint{1.284152in}{1.611100in}}%
\pgfpathlineto{\pgfqpoint{1.284035in}{1.611203in}}%
\pgfpathlineto{\pgfqpoint{1.269155in}{1.624814in}}%
\pgfpathlineto{\pgfqpoint{1.268496in}{1.625476in}}%
\pgfpathlineto{\pgfqpoint{1.255255in}{1.638425in}}%
\pgfpathlineto{\pgfqpoint{1.252839in}{1.641197in}}%
\pgfpathlineto{\pgfqpoint{1.242827in}{1.652036in}}%
\pgfpathlineto{\pgfqpoint{1.237183in}{1.659967in}}%
\pgfpathlineto{\pgfqpoint{1.232763in}{1.665648in}}%
\pgfpathlineto{\pgfqpoint{1.226087in}{1.679259in}}%
\pgfpathlineto{\pgfqpoint{1.223789in}{1.692870in}}%
\pgfpathlineto{\pgfqpoint{1.221526in}{1.692870in}}%
\pgfpathlineto{\pgfqpoint{1.205870in}{1.692870in}}%
\pgfpathlineto{\pgfqpoint{1.190213in}{1.692870in}}%
\pgfpathlineto{\pgfqpoint{1.174556in}{1.692870in}}%
\pgfpathlineto{\pgfqpoint{1.158900in}{1.692870in}}%
\pgfpathlineto{\pgfqpoint{1.143243in}{1.692870in}}%
\pgfpathlineto{\pgfqpoint{1.127587in}{1.692870in}}%
\pgfpathlineto{\pgfqpoint{1.111930in}{1.692870in}}%
\pgfpathlineto{\pgfqpoint{1.096274in}{1.692870in}}%
\pgfpathlineto{\pgfqpoint{1.080617in}{1.692870in}}%
\pgfpathlineto{\pgfqpoint{1.064960in}{1.692870in}}%
\pgfpathlineto{\pgfqpoint{1.049304in}{1.692870in}}%
\pgfpathlineto{\pgfqpoint{1.047041in}{1.692870in}}%
\pgfpathlineto{\pgfqpoint{1.044743in}{1.679259in}}%
\pgfpathlineto{\pgfqpoint{1.038067in}{1.665648in}}%
\pgfpathlineto{\pgfqpoint{1.033647in}{1.659967in}}%
\pgfpathlineto{\pgfqpoint{1.028003in}{1.652036in}}%
\pgfpathlineto{\pgfqpoint{1.017991in}{1.641197in}}%
\pgfpathlineto{\pgfqpoint{1.015575in}{1.638425in}}%
\pgfpathlineto{\pgfqpoint{1.002334in}{1.625476in}}%
\pgfpathlineto{\pgfqpoint{1.001675in}{1.624814in}}%
\pgfpathlineto{\pgfqpoint{0.986795in}{1.611203in}}%
\pgfpathlineto{\pgfqpoint{0.986678in}{1.611100in}}%
\pgfpathlineto{\pgfqpoint{0.971139in}{1.597592in}}%
\pgfpathlineto{\pgfqpoint{0.971021in}{1.597490in}}%
\pgfpathlineto{\pgfqpoint{0.955364in}{1.584554in}}%
\pgfpathlineto{\pgfqpoint{0.954603in}{1.583981in}}%
\pgfpathlineto{\pgfqpoint{0.939708in}{1.572470in}}%
\pgfpathlineto{\pgfqpoint{0.936520in}{1.570370in}}%
\pgfpathlineto{\pgfqpoint{0.924051in}{1.561665in}}%
\pgfpathlineto{\pgfqpoint{0.914929in}{1.556759in}}%
\pgfpathlineto{\pgfqpoint{0.908395in}{1.552916in}}%
\pgfpathlineto{\pgfqpoint{0.892738in}{1.547113in}}%
\pgfpathlineto{\pgfqpoint{0.877082in}{1.545115in}}%
\pgfpathlineto{\pgfqpoint{0.861425in}{1.547113in}}%
\pgfpathlineto{\pgfqpoint{0.845769in}{1.552916in}}%
\pgfpathlineto{\pgfqpoint{0.839234in}{1.556759in}}%
\pgfpathlineto{\pgfqpoint{0.830112in}{1.561665in}}%
\pgfpathlineto{\pgfqpoint{0.817643in}{1.570370in}}%
\pgfpathlineto{\pgfqpoint{0.814455in}{1.572470in}}%
\pgfpathlineto{\pgfqpoint{0.799561in}{1.583981in}}%
\pgfpathlineto{\pgfqpoint{0.798799in}{1.584554in}}%
\pgfpathlineto{\pgfqpoint{0.783142in}{1.597490in}}%
\pgfpathlineto{\pgfqpoint{0.783024in}{1.597592in}}%
\pgfpathlineto{\pgfqpoint{0.767486in}{1.611100in}}%
\pgfpathlineto{\pgfqpoint{0.767368in}{1.611203in}}%
\pgfpathlineto{\pgfqpoint{0.752488in}{1.624814in}}%
\pgfpathlineto{\pgfqpoint{0.751829in}{1.625476in}}%
\pgfpathlineto{\pgfqpoint{0.738588in}{1.638425in}}%
\pgfpathlineto{\pgfqpoint{0.736173in}{1.641197in}}%
\pgfpathlineto{\pgfqpoint{0.726160in}{1.652036in}}%
\pgfpathlineto{\pgfqpoint{0.720516in}{1.659967in}}%
\pgfpathlineto{\pgfqpoint{0.716096in}{1.665648in}}%
\pgfpathlineto{\pgfqpoint{0.709420in}{1.679259in}}%
\pgfpathlineto{\pgfqpoint{0.707123in}{1.692870in}}%
\pgfpathlineto{\pgfqpoint{0.704859in}{1.692870in}}%
\pgfpathlineto{\pgfqpoint{0.689203in}{1.692870in}}%
\pgfpathlineto{\pgfqpoint{0.673546in}{1.692870in}}%
\pgfpathlineto{\pgfqpoint{0.657890in}{1.692870in}}%
\pgfpathlineto{\pgfqpoint{0.642233in}{1.692870in}}%
\pgfpathlineto{\pgfqpoint{0.626577in}{1.692870in}}%
\pgfpathlineto{\pgfqpoint{0.610920in}{1.692870in}}%
\pgfpathlineto{\pgfqpoint{0.595263in}{1.692870in}}%
\pgfpathlineto{\pgfqpoint{0.579607in}{1.692870in}}%
\pgfpathlineto{\pgfqpoint{0.563950in}{1.692870in}}%
\pgfpathlineto{\pgfqpoint{0.548294in}{1.692870in}}%
\pgfpathlineto{\pgfqpoint{0.532637in}{1.692870in}}%
\pgfpathlineto{\pgfqpoint{0.530374in}{1.692870in}}%
\pgfpathlineto{\pgfqpoint{0.528076in}{1.679259in}}%
\pgfpathlineto{\pgfqpoint{0.521401in}{1.665648in}}%
\pgfpathlineto{\pgfqpoint{0.516981in}{1.659967in}}%
\pgfpathlineto{\pgfqpoint{0.511336in}{1.652036in}}%
\pgfpathlineto{\pgfqpoint{0.501324in}{1.641197in}}%
\pgfpathlineto{\pgfqpoint{0.498908in}{1.638425in}}%
\pgfpathlineto{\pgfqpoint{0.485668in}{1.625476in}}%
\pgfpathlineto{\pgfqpoint{0.485008in}{1.624814in}}%
\pgfpathlineto{\pgfqpoint{0.470129in}{1.611203in}}%
\pgfpathlineto{\pgfqpoint{0.470011in}{1.611100in}}%
\pgfpathlineto{\pgfqpoint{0.454472in}{1.597592in}}%
\pgfpathlineto{\pgfqpoint{0.454354in}{1.597490in}}%
\pgfpathlineto{\pgfqpoint{0.438698in}{1.584554in}}%
\pgfpathlineto{\pgfqpoint{0.437936in}{1.583981in}}%
\pgfpathlineto{\pgfqpoint{0.423041in}{1.572470in}}%
\pgfpathlineto{\pgfqpoint{0.419853in}{1.570370in}}%
\pgfpathlineto{\pgfqpoint{0.407385in}{1.561665in}}%
\pgfpathlineto{\pgfqpoint{0.398262in}{1.556759in}}%
\pgfpathlineto{\pgfqpoint{0.391728in}{1.552916in}}%
\pgfpathlineto{\pgfqpoint{0.376072in}{1.547113in}}%
\pgfpathlineto{\pgfqpoint{0.360415in}{1.545115in}}%
\pgfpathlineto{\pgfqpoint{0.360415in}{1.543148in}}%
\pgfpathlineto{\pgfqpoint{0.360415in}{1.529536in}}%
\pgfpathlineto{\pgfqpoint{0.360415in}{1.515925in}}%
\pgfpathlineto{\pgfqpoint{0.360415in}{1.502314in}}%
\pgfpathlineto{\pgfqpoint{0.360415in}{1.488703in}}%
\pgfpathlineto{\pgfqpoint{0.360415in}{1.475092in}}%
\pgfpathlineto{\pgfqpoint{0.360415in}{1.461481in}}%
\pgfpathlineto{\pgfqpoint{0.360415in}{1.447870in}}%
\pgfpathlineto{\pgfqpoint{0.360415in}{1.434259in}}%
\pgfpathlineto{\pgfqpoint{0.360415in}{1.420648in}}%
\pgfpathlineto{\pgfqpoint{0.360415in}{1.407036in}}%
\pgfpathlineto{\pgfqpoint{0.360415in}{1.393425in}}%
\pgfpathlineto{\pgfqpoint{0.360415in}{1.391458in}}%
\pgfpathlineto{\pgfqpoint{0.376072in}{1.389460in}}%
\pgfpathlineto{\pgfqpoint{0.391728in}{1.383657in}}%
\pgfpathlineto{\pgfqpoint{0.398262in}{1.379814in}}%
\pgfpathlineto{\pgfqpoint{0.407385in}{1.374907in}}%
\pgfpathlineto{\pgfqpoint{0.419853in}{1.366203in}}%
\pgfpathlineto{\pgfqpoint{0.423041in}{1.364103in}}%
\pgfpathlineto{\pgfqpoint{0.437936in}{1.352592in}}%
\pgfpathlineto{\pgfqpoint{0.438698in}{1.352019in}}%
\pgfpathlineto{\pgfqpoint{0.454354in}{1.339083in}}%
\pgfpathlineto{\pgfqpoint{0.454472in}{1.338981in}}%
\pgfpathlineto{\pgfqpoint{0.470011in}{1.325472in}}%
\pgfpathlineto{\pgfqpoint{0.470129in}{1.325370in}}%
\pgfpathlineto{\pgfqpoint{0.485008in}{1.311759in}}%
\pgfpathlineto{\pgfqpoint{0.485668in}{1.311096in}}%
\pgfpathlineto{\pgfqpoint{0.498908in}{1.298148in}}%
\pgfpathlineto{\pgfqpoint{0.501324in}{1.295376in}}%
\pgfpathlineto{\pgfqpoint{0.511336in}{1.284536in}}%
\pgfpathlineto{\pgfqpoint{0.516981in}{1.276606in}}%
\pgfpathlineto{\pgfqpoint{0.521401in}{1.270925in}}%
\pgfpathlineto{\pgfqpoint{0.528076in}{1.257314in}}%
\pgfpathlineto{\pgfqpoint{0.530374in}{1.243703in}}%
\pgfpathlineto{\pgfqpoint{0.528076in}{1.230092in}}%
\pgfpathlineto{\pgfqpoint{0.521401in}{1.216481in}}%
\pgfpathlineto{\pgfqpoint{0.516981in}{1.210800in}}%
\pgfpathlineto{\pgfqpoint{0.511336in}{1.202870in}}%
\pgfpathlineto{\pgfqpoint{0.501324in}{1.192030in}}%
\pgfpathlineto{\pgfqpoint{0.498908in}{1.189259in}}%
\pgfpathlineto{\pgfqpoint{0.485668in}{1.176310in}}%
\pgfpathlineto{\pgfqpoint{0.485008in}{1.175647in}}%
\pgfpathlineto{\pgfqpoint{0.470129in}{1.162036in}}%
\pgfpathlineto{\pgfqpoint{0.470011in}{1.161934in}}%
\pgfpathlineto{\pgfqpoint{0.454472in}{1.148425in}}%
\pgfpathlineto{\pgfqpoint{0.454354in}{1.148323in}}%
\pgfpathlineto{\pgfqpoint{0.438698in}{1.135387in}}%
\pgfpathlineto{\pgfqpoint{0.437936in}{1.134814in}}%
\pgfpathlineto{\pgfqpoint{0.423041in}{1.123303in}}%
\pgfpathlineto{\pgfqpoint{0.419853in}{1.121203in}}%
\pgfpathlineto{\pgfqpoint{0.407385in}{1.112499in}}%
\pgfpathlineto{\pgfqpoint{0.398262in}{1.107592in}}%
\pgfpathlineto{\pgfqpoint{0.391728in}{1.103749in}}%
\pgfpathlineto{\pgfqpoint{0.376072in}{1.097946in}}%
\pgfpathlineto{\pgfqpoint{0.360415in}{1.095948in}}%
\pgfpathlineto{\pgfqpoint{0.360415in}{1.093981in}}%
\pgfpathlineto{\pgfqpoint{0.360415in}{1.080370in}}%
\pgfpathlineto{\pgfqpoint{0.360415in}{1.066759in}}%
\pgfpathlineto{\pgfqpoint{0.360415in}{1.053148in}}%
\pgfpathlineto{\pgfqpoint{0.360415in}{1.039536in}}%
\pgfpathlineto{\pgfqpoint{0.360415in}{1.025925in}}%
\pgfpathlineto{\pgfqpoint{0.360415in}{1.012314in}}%
\pgfpathlineto{\pgfqpoint{0.360415in}{0.998703in}}%
\pgfpathlineto{\pgfqpoint{0.360415in}{0.985092in}}%
\pgfpathlineto{\pgfqpoint{0.360415in}{0.971481in}}%
\pgfpathlineto{\pgfqpoint{0.360415in}{0.957870in}}%
\pgfpathlineto{\pgfqpoint{0.360415in}{0.944259in}}%
\pgfpathlineto{\pgfqpoint{0.360415in}{0.942291in}}%
\pgfpathlineto{\pgfqpoint{0.376072in}{0.940293in}}%
\pgfpathlineto{\pgfqpoint{0.391728in}{0.934490in}}%
\pgfpathlineto{\pgfqpoint{0.398262in}{0.930648in}}%
\pgfpathlineto{\pgfqpoint{0.407385in}{0.925741in}}%
\pgfpathlineto{\pgfqpoint{0.419853in}{0.917036in}}%
\pgfpathlineto{\pgfqpoint{0.423041in}{0.914936in}}%
\pgfpathlineto{\pgfqpoint{0.437936in}{0.903425in}}%
\pgfpathlineto{\pgfqpoint{0.438698in}{0.902852in}}%
\pgfpathlineto{\pgfqpoint{0.454354in}{0.889916in}}%
\pgfpathlineto{\pgfqpoint{0.454472in}{0.889814in}}%
\pgfpathlineto{\pgfqpoint{0.470011in}{0.876306in}}%
\pgfpathlineto{\pgfqpoint{0.470129in}{0.876203in}}%
\pgfpathlineto{\pgfqpoint{0.485008in}{0.862592in}}%
\pgfpathlineto{\pgfqpoint{0.485668in}{0.861930in}}%
\pgfpathlineto{\pgfqpoint{0.498908in}{0.848981in}}%
\pgfpathlineto{\pgfqpoint{0.501324in}{0.846209in}}%
\pgfpathlineto{\pgfqpoint{0.511336in}{0.835370in}}%
\pgfpathlineto{\pgfqpoint{0.516981in}{0.827439in}}%
\pgfpathlineto{\pgfqpoint{0.521401in}{0.821759in}}%
\pgfpathlineto{\pgfqpoint{0.528076in}{0.808148in}}%
\pgfpathlineto{\pgfqpoint{0.530374in}{0.794536in}}%
\pgfpathlineto{\pgfqpoint{0.528076in}{0.780925in}}%
\pgfpathlineto{\pgfqpoint{0.521401in}{0.767314in}}%
\pgfpathlineto{\pgfqpoint{0.516981in}{0.761634in}}%
\pgfpathlineto{\pgfqpoint{0.511336in}{0.753703in}}%
\pgfpathlineto{\pgfqpoint{0.501324in}{0.742863in}}%
\pgfpathlineto{\pgfqpoint{0.498908in}{0.740092in}}%
\pgfpathlineto{\pgfqpoint{0.485668in}{0.727143in}}%
\pgfpathlineto{\pgfqpoint{0.485008in}{0.726481in}}%
\pgfpathlineto{\pgfqpoint{0.470129in}{0.712870in}}%
\pgfpathlineto{\pgfqpoint{0.470011in}{0.712767in}}%
\pgfpathlineto{\pgfqpoint{0.454472in}{0.699259in}}%
\pgfpathlineto{\pgfqpoint{0.454354in}{0.699156in}}%
\pgfpathlineto{\pgfqpoint{0.438698in}{0.686220in}}%
\pgfpathlineto{\pgfqpoint{0.437936in}{0.685648in}}%
\pgfpathlineto{\pgfqpoint{0.423041in}{0.674136in}}%
\pgfpathlineto{\pgfqpoint{0.419853in}{0.672036in}}%
\pgfpathlineto{\pgfqpoint{0.407385in}{0.663332in}}%
\pgfpathlineto{\pgfqpoint{0.398262in}{0.658425in}}%
\pgfpathlineto{\pgfqpoint{0.391728in}{0.654583in}}%
\pgfpathlineto{\pgfqpoint{0.376072in}{0.648779in}}%
\pgfpathlineto{\pgfqpoint{0.360415in}{0.646782in}}%
\pgfpathlineto{\pgfqpoint{0.360415in}{0.644814in}}%
\pgfpathlineto{\pgfqpoint{0.360415in}{0.631203in}}%
\pgfpathlineto{\pgfqpoint{0.360415in}{0.617592in}}%
\pgfpathlineto{\pgfqpoint{0.360415in}{0.603981in}}%
\pgfpathlineto{\pgfqpoint{0.360415in}{0.590370in}}%
\pgfpathlineto{\pgfqpoint{0.360415in}{0.576759in}}%
\pgfpathlineto{\pgfqpoint{0.360415in}{0.563148in}}%
\pgfpathlineto{\pgfqpoint{0.360415in}{0.549536in}}%
\pgfpathlineto{\pgfqpoint{0.360415in}{0.535925in}}%
\pgfpathlineto{\pgfqpoint{0.360415in}{0.522314in}}%
\pgfpathlineto{\pgfqpoint{0.360415in}{0.508703in}}%
\pgfpathlineto{\pgfqpoint{0.360415in}{0.495092in}}%
\pgfpathlineto{\pgfqpoint{0.360415in}{0.493124in}}%
\pgfpathlineto{\pgfqpoint{0.376072in}{0.491127in}}%
\pgfpathlineto{\pgfqpoint{0.391728in}{0.485323in}}%
\pgfpathlineto{\pgfqpoint{0.398262in}{0.481481in}}%
\pgfpathlineto{\pgfqpoint{0.407385in}{0.476574in}}%
\pgfpathlineto{\pgfqpoint{0.419853in}{0.467870in}}%
\pgfpathlineto{\pgfqpoint{0.423041in}{0.465770in}}%
\pgfpathlineto{\pgfqpoint{0.437936in}{0.454259in}}%
\pgfpathlineto{\pgfqpoint{0.438698in}{0.453686in}}%
\pgfpathlineto{\pgfqpoint{0.454354in}{0.440750in}}%
\pgfpathlineto{\pgfqpoint{0.454472in}{0.440648in}}%
\pgfpathlineto{\pgfqpoint{0.470011in}{0.427139in}}%
\pgfpathlineto{\pgfqpoint{0.470129in}{0.427036in}}%
\pgfpathlineto{\pgfqpoint{0.485008in}{0.413425in}}%
\pgfpathlineto{\pgfqpoint{0.485668in}{0.412763in}}%
\pgfpathlineto{\pgfqpoint{0.498908in}{0.399814in}}%
\pgfpathlineto{\pgfqpoint{0.501324in}{0.397043in}}%
\pgfpathlineto{\pgfqpoint{0.511336in}{0.386203in}}%
\pgfpathlineto{\pgfqpoint{0.516981in}{0.378272in}}%
\pgfpathlineto{\pgfqpoint{0.521401in}{0.372592in}}%
\pgfpathlineto{\pgfqpoint{0.528076in}{0.358981in}}%
\pgfpathlineto{\pgfqpoint{0.530374in}{0.345370in}}%
\pgfpathlineto{\pgfqpoint{0.532637in}{0.345370in}}%
\pgfpathclose%
\pgfpathmoveto{\pgfqpoint{0.604517in}{0.372592in}}%
\pgfpathlineto{\pgfqpoint{0.595263in}{0.374784in}}%
\pgfpathlineto{\pgfqpoint{0.579607in}{0.382193in}}%
\pgfpathlineto{\pgfqpoint{0.573894in}{0.386203in}}%
\pgfpathlineto{\pgfqpoint{0.563950in}{0.391476in}}%
\pgfpathlineto{\pgfqpoint{0.552007in}{0.399814in}}%
\pgfpathlineto{\pgfqpoint{0.548294in}{0.401987in}}%
\pgfpathlineto{\pgfqpoint{0.532637in}{0.413200in}}%
\pgfpathlineto{\pgfqpoint{0.532364in}{0.413425in}}%
\pgfpathlineto{\pgfqpoint{0.516981in}{0.424882in}}%
\pgfpathlineto{\pgfqpoint{0.514379in}{0.427036in}}%
\pgfpathlineto{\pgfqpoint{0.501324in}{0.437333in}}%
\pgfpathlineto{\pgfqpoint{0.497404in}{0.440648in}}%
\pgfpathlineto{\pgfqpoint{0.485668in}{0.450517in}}%
\pgfpathlineto{\pgfqpoint{0.481364in}{0.454259in}}%
\pgfpathlineto{\pgfqpoint{0.470011in}{0.464462in}}%
\pgfpathlineto{\pgfqpoint{0.466198in}{0.467870in}}%
\pgfpathlineto{\pgfqpoint{0.454354in}{0.479219in}}%
\pgfpathlineto{\pgfqpoint{0.451876in}{0.481481in}}%
\pgfpathlineto{\pgfqpoint{0.438698in}{0.494854in}}%
\pgfpathlineto{\pgfqpoint{0.438438in}{0.495092in}}%
\pgfpathlineto{\pgfqpoint{0.425541in}{0.508703in}}%
\pgfpathlineto{\pgfqpoint{0.423041in}{0.511931in}}%
\pgfpathlineto{\pgfqpoint{0.413450in}{0.522314in}}%
\pgfpathlineto{\pgfqpoint{0.407385in}{0.530959in}}%
\pgfpathlineto{\pgfqpoint{0.402772in}{0.535925in}}%
\pgfpathlineto{\pgfqpoint{0.394250in}{0.549536in}}%
\pgfpathlineto{\pgfqpoint{0.391728in}{0.557581in}}%
\pgfpathlineto{\pgfqpoint{0.388907in}{0.563148in}}%
\pgfpathlineto{\pgfqpoint{0.388907in}{0.576759in}}%
\pgfpathlineto{\pgfqpoint{0.391728in}{0.582325in}}%
\pgfpathlineto{\pgfqpoint{0.394250in}{0.590370in}}%
\pgfpathlineto{\pgfqpoint{0.402772in}{0.603981in}}%
\pgfpathlineto{\pgfqpoint{0.407385in}{0.608947in}}%
\pgfpathlineto{\pgfqpoint{0.413450in}{0.617592in}}%
\pgfpathlineto{\pgfqpoint{0.423041in}{0.627975in}}%
\pgfpathlineto{\pgfqpoint{0.425541in}{0.631203in}}%
\pgfpathlineto{\pgfqpoint{0.438438in}{0.644814in}}%
\pgfpathlineto{\pgfqpoint{0.438698in}{0.645052in}}%
\pgfpathlineto{\pgfqpoint{0.451876in}{0.658425in}}%
\pgfpathlineto{\pgfqpoint{0.454354in}{0.660687in}}%
\pgfpathlineto{\pgfqpoint{0.466198in}{0.672036in}}%
\pgfpathlineto{\pgfqpoint{0.470011in}{0.675444in}}%
\pgfpathlineto{\pgfqpoint{0.481364in}{0.685648in}}%
\pgfpathlineto{\pgfqpoint{0.485668in}{0.689389in}}%
\pgfpathlineto{\pgfqpoint{0.497404in}{0.699259in}}%
\pgfpathlineto{\pgfqpoint{0.501324in}{0.702573in}}%
\pgfpathlineto{\pgfqpoint{0.514379in}{0.712870in}}%
\pgfpathlineto{\pgfqpoint{0.516981in}{0.715025in}}%
\pgfpathlineto{\pgfqpoint{0.532364in}{0.726481in}}%
\pgfpathlineto{\pgfqpoint{0.532637in}{0.726706in}}%
\pgfpathlineto{\pgfqpoint{0.548294in}{0.737919in}}%
\pgfpathlineto{\pgfqpoint{0.552007in}{0.740092in}}%
\pgfpathlineto{\pgfqpoint{0.563950in}{0.748430in}}%
\pgfpathlineto{\pgfqpoint{0.573894in}{0.753703in}}%
\pgfpathlineto{\pgfqpoint{0.579607in}{0.757713in}}%
\pgfpathlineto{\pgfqpoint{0.595263in}{0.765122in}}%
\pgfpathlineto{\pgfqpoint{0.604517in}{0.767314in}}%
\pgfpathlineto{\pgfqpoint{0.610920in}{0.769767in}}%
\pgfpathlineto{\pgfqpoint{0.626577in}{0.769767in}}%
\pgfpathlineto{\pgfqpoint{0.632980in}{0.767314in}}%
\pgfpathlineto{\pgfqpoint{0.642233in}{0.765122in}}%
\pgfpathlineto{\pgfqpoint{0.657890in}{0.757713in}}%
\pgfpathlineto{\pgfqpoint{0.663603in}{0.753703in}}%
\pgfpathlineto{\pgfqpoint{0.673546in}{0.748430in}}%
\pgfpathlineto{\pgfqpoint{0.685490in}{0.740092in}}%
\pgfpathlineto{\pgfqpoint{0.689203in}{0.737919in}}%
\pgfpathlineto{\pgfqpoint{0.704859in}{0.726706in}}%
\pgfpathlineto{\pgfqpoint{0.705133in}{0.726481in}}%
\pgfpathlineto{\pgfqpoint{0.720516in}{0.715025in}}%
\pgfpathlineto{\pgfqpoint{0.723118in}{0.712870in}}%
\pgfpathlineto{\pgfqpoint{0.736173in}{0.702573in}}%
\pgfpathlineto{\pgfqpoint{0.740093in}{0.699259in}}%
\pgfpathlineto{\pgfqpoint{0.751829in}{0.689389in}}%
\pgfpathlineto{\pgfqpoint{0.756132in}{0.685648in}}%
\pgfpathlineto{\pgfqpoint{0.767486in}{0.675444in}}%
\pgfpathlineto{\pgfqpoint{0.771298in}{0.672036in}}%
\pgfpathlineto{\pgfqpoint{0.783142in}{0.660687in}}%
\pgfpathlineto{\pgfqpoint{0.785621in}{0.658425in}}%
\pgfpathlineto{\pgfqpoint{0.798799in}{0.645052in}}%
\pgfpathlineto{\pgfqpoint{0.799058in}{0.644814in}}%
\pgfpathlineto{\pgfqpoint{0.811956in}{0.631203in}}%
\pgfpathlineto{\pgfqpoint{0.814455in}{0.627975in}}%
\pgfpathlineto{\pgfqpoint{0.824047in}{0.617592in}}%
\pgfpathlineto{\pgfqpoint{0.830112in}{0.608947in}}%
\pgfpathlineto{\pgfqpoint{0.834725in}{0.603981in}}%
\pgfpathlineto{\pgfqpoint{0.843247in}{0.590370in}}%
\pgfpathlineto{\pgfqpoint{0.845769in}{0.582325in}}%
\pgfpathlineto{\pgfqpoint{0.848589in}{0.576759in}}%
\pgfpathlineto{\pgfqpoint{0.848589in}{0.563148in}}%
\pgfpathlineto{\pgfqpoint{0.845769in}{0.557581in}}%
\pgfpathlineto{\pgfqpoint{0.843247in}{0.549536in}}%
\pgfpathlineto{\pgfqpoint{0.834725in}{0.535925in}}%
\pgfpathlineto{\pgfqpoint{0.830112in}{0.530959in}}%
\pgfpathlineto{\pgfqpoint{0.824047in}{0.522314in}}%
\pgfpathlineto{\pgfqpoint{0.814455in}{0.511931in}}%
\pgfpathlineto{\pgfqpoint{0.811956in}{0.508703in}}%
\pgfpathlineto{\pgfqpoint{0.799058in}{0.495092in}}%
\pgfpathlineto{\pgfqpoint{0.798799in}{0.494854in}}%
\pgfpathlineto{\pgfqpoint{0.785621in}{0.481481in}}%
\pgfpathlineto{\pgfqpoint{0.783142in}{0.479219in}}%
\pgfpathlineto{\pgfqpoint{0.771298in}{0.467870in}}%
\pgfpathlineto{\pgfqpoint{0.767486in}{0.464462in}}%
\pgfpathlineto{\pgfqpoint{0.756132in}{0.454259in}}%
\pgfpathlineto{\pgfqpoint{0.751829in}{0.450517in}}%
\pgfpathlineto{\pgfqpoint{0.740093in}{0.440648in}}%
\pgfpathlineto{\pgfqpoint{0.736173in}{0.437333in}}%
\pgfpathlineto{\pgfqpoint{0.723118in}{0.427036in}}%
\pgfpathlineto{\pgfqpoint{0.720516in}{0.424882in}}%
\pgfpathlineto{\pgfqpoint{0.705133in}{0.413425in}}%
\pgfpathlineto{\pgfqpoint{0.704859in}{0.413200in}}%
\pgfpathlineto{\pgfqpoint{0.689203in}{0.401987in}}%
\pgfpathlineto{\pgfqpoint{0.685490in}{0.399814in}}%
\pgfpathlineto{\pgfqpoint{0.673546in}{0.391476in}}%
\pgfpathlineto{\pgfqpoint{0.663603in}{0.386203in}}%
\pgfpathlineto{\pgfqpoint{0.657890in}{0.382193in}}%
\pgfpathlineto{\pgfqpoint{0.642233in}{0.374784in}}%
\pgfpathlineto{\pgfqpoint{0.632980in}{0.372592in}}%
\pgfpathlineto{\pgfqpoint{0.626577in}{0.370140in}}%
\pgfpathlineto{\pgfqpoint{0.610920in}{0.370140in}}%
\pgfpathlineto{\pgfqpoint{0.604517in}{0.372592in}}%
\pgfpathclose%
\pgfpathmoveto{\pgfqpoint{1.121183in}{0.372592in}}%
\pgfpathlineto{\pgfqpoint{1.111930in}{0.374784in}}%
\pgfpathlineto{\pgfqpoint{1.096274in}{0.382193in}}%
\pgfpathlineto{\pgfqpoint{1.090561in}{0.386203in}}%
\pgfpathlineto{\pgfqpoint{1.080617in}{0.391476in}}%
\pgfpathlineto{\pgfqpoint{1.068674in}{0.399814in}}%
\pgfpathlineto{\pgfqpoint{1.064960in}{0.401987in}}%
\pgfpathlineto{\pgfqpoint{1.049304in}{0.413200in}}%
\pgfpathlineto{\pgfqpoint{1.049030in}{0.413425in}}%
\pgfpathlineto{\pgfqpoint{1.033647in}{0.424882in}}%
\pgfpathlineto{\pgfqpoint{1.031046in}{0.427036in}}%
\pgfpathlineto{\pgfqpoint{1.017991in}{0.437333in}}%
\pgfpathlineto{\pgfqpoint{1.014071in}{0.440648in}}%
\pgfpathlineto{\pgfqpoint{1.002334in}{0.450517in}}%
\pgfpathlineto{\pgfqpoint{0.998031in}{0.454259in}}%
\pgfpathlineto{\pgfqpoint{0.986678in}{0.464462in}}%
\pgfpathlineto{\pgfqpoint{0.982865in}{0.467870in}}%
\pgfpathlineto{\pgfqpoint{0.971021in}{0.479219in}}%
\pgfpathlineto{\pgfqpoint{0.968542in}{0.481481in}}%
\pgfpathlineto{\pgfqpoint{0.955364in}{0.494854in}}%
\pgfpathlineto{\pgfqpoint{0.955105in}{0.495092in}}%
\pgfpathlineto{\pgfqpoint{0.942208in}{0.508703in}}%
\pgfpathlineto{\pgfqpoint{0.939708in}{0.511931in}}%
\pgfpathlineto{\pgfqpoint{0.930117in}{0.522314in}}%
\pgfpathlineto{\pgfqpoint{0.924051in}{0.530959in}}%
\pgfpathlineto{\pgfqpoint{0.919438in}{0.535925in}}%
\pgfpathlineto{\pgfqpoint{0.910917in}{0.549536in}}%
\pgfpathlineto{\pgfqpoint{0.908395in}{0.557581in}}%
\pgfpathlineto{\pgfqpoint{0.905574in}{0.563148in}}%
\pgfpathlineto{\pgfqpoint{0.905574in}{0.576759in}}%
\pgfpathlineto{\pgfqpoint{0.908395in}{0.582325in}}%
\pgfpathlineto{\pgfqpoint{0.910917in}{0.590370in}}%
\pgfpathlineto{\pgfqpoint{0.919438in}{0.603981in}}%
\pgfpathlineto{\pgfqpoint{0.924051in}{0.608947in}}%
\pgfpathlineto{\pgfqpoint{0.930117in}{0.617592in}}%
\pgfpathlineto{\pgfqpoint{0.939708in}{0.627975in}}%
\pgfpathlineto{\pgfqpoint{0.942208in}{0.631203in}}%
\pgfpathlineto{\pgfqpoint{0.955105in}{0.644814in}}%
\pgfpathlineto{\pgfqpoint{0.955364in}{0.645052in}}%
\pgfpathlineto{\pgfqpoint{0.968542in}{0.658425in}}%
\pgfpathlineto{\pgfqpoint{0.971021in}{0.660687in}}%
\pgfpathlineto{\pgfqpoint{0.982865in}{0.672036in}}%
\pgfpathlineto{\pgfqpoint{0.986678in}{0.675444in}}%
\pgfpathlineto{\pgfqpoint{0.998031in}{0.685648in}}%
\pgfpathlineto{\pgfqpoint{1.002334in}{0.689389in}}%
\pgfpathlineto{\pgfqpoint{1.014071in}{0.699259in}}%
\pgfpathlineto{\pgfqpoint{1.017991in}{0.702573in}}%
\pgfpathlineto{\pgfqpoint{1.031046in}{0.712870in}}%
\pgfpathlineto{\pgfqpoint{1.033647in}{0.715025in}}%
\pgfpathlineto{\pgfqpoint{1.049030in}{0.726481in}}%
\pgfpathlineto{\pgfqpoint{1.049304in}{0.726706in}}%
\pgfpathlineto{\pgfqpoint{1.064960in}{0.737919in}}%
\pgfpathlineto{\pgfqpoint{1.068674in}{0.740092in}}%
\pgfpathlineto{\pgfqpoint{1.080617in}{0.748430in}}%
\pgfpathlineto{\pgfqpoint{1.090561in}{0.753703in}}%
\pgfpathlineto{\pgfqpoint{1.096274in}{0.757713in}}%
\pgfpathlineto{\pgfqpoint{1.111930in}{0.765122in}}%
\pgfpathlineto{\pgfqpoint{1.121183in}{0.767314in}}%
\pgfpathlineto{\pgfqpoint{1.127587in}{0.769767in}}%
\pgfpathlineto{\pgfqpoint{1.143243in}{0.769767in}}%
\pgfpathlineto{\pgfqpoint{1.149647in}{0.767314in}}%
\pgfpathlineto{\pgfqpoint{1.158900in}{0.765122in}}%
\pgfpathlineto{\pgfqpoint{1.174556in}{0.757713in}}%
\pgfpathlineto{\pgfqpoint{1.180269in}{0.753703in}}%
\pgfpathlineto{\pgfqpoint{1.190213in}{0.748430in}}%
\pgfpathlineto{\pgfqpoint{1.202156in}{0.740092in}}%
\pgfpathlineto{\pgfqpoint{1.205870in}{0.737919in}}%
\pgfpathlineto{\pgfqpoint{1.221526in}{0.726706in}}%
\pgfpathlineto{\pgfqpoint{1.221800in}{0.726481in}}%
\pgfpathlineto{\pgfqpoint{1.237183in}{0.715025in}}%
\pgfpathlineto{\pgfqpoint{1.239784in}{0.712870in}}%
\pgfpathlineto{\pgfqpoint{1.252839in}{0.702573in}}%
\pgfpathlineto{\pgfqpoint{1.256759in}{0.699259in}}%
\pgfpathlineto{\pgfqpoint{1.268496in}{0.689389in}}%
\pgfpathlineto{\pgfqpoint{1.272799in}{0.685648in}}%
\pgfpathlineto{\pgfqpoint{1.284152in}{0.675444in}}%
\pgfpathlineto{\pgfqpoint{1.287965in}{0.672036in}}%
\pgfpathlineto{\pgfqpoint{1.299809in}{0.660687in}}%
\pgfpathlineto{\pgfqpoint{1.302288in}{0.658425in}}%
\pgfpathlineto{\pgfqpoint{1.315466in}{0.645052in}}%
\pgfpathlineto{\pgfqpoint{1.315725in}{0.644814in}}%
\pgfpathlineto{\pgfqpoint{1.328622in}{0.631203in}}%
\pgfpathlineto{\pgfqpoint{1.331122in}{0.627975in}}%
\pgfpathlineto{\pgfqpoint{1.340713in}{0.617592in}}%
\pgfpathlineto{\pgfqpoint{1.346779in}{0.608947in}}%
\pgfpathlineto{\pgfqpoint{1.351392in}{0.603981in}}%
\pgfpathlineto{\pgfqpoint{1.359913in}{0.590370in}}%
\pgfpathlineto{\pgfqpoint{1.362435in}{0.582325in}}%
\pgfpathlineto{\pgfqpoint{1.365256in}{0.576759in}}%
\pgfpathlineto{\pgfqpoint{1.365256in}{0.563148in}}%
\pgfpathlineto{\pgfqpoint{1.362435in}{0.557581in}}%
\pgfpathlineto{\pgfqpoint{1.359913in}{0.549536in}}%
\pgfpathlineto{\pgfqpoint{1.351392in}{0.535925in}}%
\pgfpathlineto{\pgfqpoint{1.346779in}{0.530959in}}%
\pgfpathlineto{\pgfqpoint{1.340713in}{0.522314in}}%
\pgfpathlineto{\pgfqpoint{1.331122in}{0.511931in}}%
\pgfpathlineto{\pgfqpoint{1.328622in}{0.508703in}}%
\pgfpathlineto{\pgfqpoint{1.315725in}{0.495092in}}%
\pgfpathlineto{\pgfqpoint{1.315466in}{0.494854in}}%
\pgfpathlineto{\pgfqpoint{1.302288in}{0.481481in}}%
\pgfpathlineto{\pgfqpoint{1.299809in}{0.479219in}}%
\pgfpathlineto{\pgfqpoint{1.287965in}{0.467870in}}%
\pgfpathlineto{\pgfqpoint{1.284152in}{0.464462in}}%
\pgfpathlineto{\pgfqpoint{1.272799in}{0.454259in}}%
\pgfpathlineto{\pgfqpoint{1.268496in}{0.450517in}}%
\pgfpathlineto{\pgfqpoint{1.256759in}{0.440648in}}%
\pgfpathlineto{\pgfqpoint{1.252839in}{0.437333in}}%
\pgfpathlineto{\pgfqpoint{1.239784in}{0.427036in}}%
\pgfpathlineto{\pgfqpoint{1.237183in}{0.424882in}}%
\pgfpathlineto{\pgfqpoint{1.221800in}{0.413425in}}%
\pgfpathlineto{\pgfqpoint{1.221526in}{0.413200in}}%
\pgfpathlineto{\pgfqpoint{1.205870in}{0.401987in}}%
\pgfpathlineto{\pgfqpoint{1.202156in}{0.399814in}}%
\pgfpathlineto{\pgfqpoint{1.190213in}{0.391476in}}%
\pgfpathlineto{\pgfqpoint{1.180269in}{0.386203in}}%
\pgfpathlineto{\pgfqpoint{1.174556in}{0.382193in}}%
\pgfpathlineto{\pgfqpoint{1.158900in}{0.374784in}}%
\pgfpathlineto{\pgfqpoint{1.149647in}{0.372592in}}%
\pgfpathlineto{\pgfqpoint{1.143243in}{0.370140in}}%
\pgfpathlineto{\pgfqpoint{1.127587in}{0.370140in}}%
\pgfpathlineto{\pgfqpoint{1.121183in}{0.372592in}}%
\pgfpathclose%
\pgfpathmoveto{\pgfqpoint{1.637850in}{0.372592in}}%
\pgfpathlineto{\pgfqpoint{1.628597in}{0.374784in}}%
\pgfpathlineto{\pgfqpoint{1.612940in}{0.382193in}}%
\pgfpathlineto{\pgfqpoint{1.607227in}{0.386203in}}%
\pgfpathlineto{\pgfqpoint{1.597284in}{0.391476in}}%
\pgfpathlineto{\pgfqpoint{1.585340in}{0.399814in}}%
\pgfpathlineto{\pgfqpoint{1.581627in}{0.401987in}}%
\pgfpathlineto{\pgfqpoint{1.565971in}{0.413200in}}%
\pgfpathlineto{\pgfqpoint{1.565697in}{0.413425in}}%
\pgfpathlineto{\pgfqpoint{1.550314in}{0.424882in}}%
\pgfpathlineto{\pgfqpoint{1.547712in}{0.427036in}}%
\pgfpathlineto{\pgfqpoint{1.534657in}{0.437333in}}%
\pgfpathlineto{\pgfqpoint{1.530737in}{0.440648in}}%
\pgfpathlineto{\pgfqpoint{1.519001in}{0.450517in}}%
\pgfpathlineto{\pgfqpoint{1.514698in}{0.454259in}}%
\pgfpathlineto{\pgfqpoint{1.503344in}{0.464462in}}%
\pgfpathlineto{\pgfqpoint{1.499532in}{0.467870in}}%
\pgfpathlineto{\pgfqpoint{1.487688in}{0.479219in}}%
\pgfpathlineto{\pgfqpoint{1.485209in}{0.481481in}}%
\pgfpathlineto{\pgfqpoint{1.472031in}{0.494854in}}%
\pgfpathlineto{\pgfqpoint{1.471772in}{0.495092in}}%
\pgfpathlineto{\pgfqpoint{1.458874in}{0.508703in}}%
\pgfpathlineto{\pgfqpoint{1.456375in}{0.511931in}}%
\pgfpathlineto{\pgfqpoint{1.446783in}{0.522314in}}%
\pgfpathlineto{\pgfqpoint{1.440718in}{0.530959in}}%
\pgfpathlineto{\pgfqpoint{1.436105in}{0.535925in}}%
\pgfpathlineto{\pgfqpoint{1.427583in}{0.549536in}}%
\pgfpathlineto{\pgfqpoint{1.425061in}{0.557581in}}%
\pgfpathlineto{\pgfqpoint{1.422241in}{0.563148in}}%
\pgfpathlineto{\pgfqpoint{1.422241in}{0.576759in}}%
\pgfpathlineto{\pgfqpoint{1.425061in}{0.582325in}}%
\pgfpathlineto{\pgfqpoint{1.427583in}{0.590370in}}%
\pgfpathlineto{\pgfqpoint{1.436105in}{0.603981in}}%
\pgfpathlineto{\pgfqpoint{1.440718in}{0.608947in}}%
\pgfpathlineto{\pgfqpoint{1.446783in}{0.617592in}}%
\pgfpathlineto{\pgfqpoint{1.456375in}{0.627975in}}%
\pgfpathlineto{\pgfqpoint{1.458874in}{0.631203in}}%
\pgfpathlineto{\pgfqpoint{1.471772in}{0.644814in}}%
\pgfpathlineto{\pgfqpoint{1.472031in}{0.645052in}}%
\pgfpathlineto{\pgfqpoint{1.485209in}{0.658425in}}%
\pgfpathlineto{\pgfqpoint{1.487688in}{0.660687in}}%
\pgfpathlineto{\pgfqpoint{1.499532in}{0.672036in}}%
\pgfpathlineto{\pgfqpoint{1.503344in}{0.675444in}}%
\pgfpathlineto{\pgfqpoint{1.514698in}{0.685648in}}%
\pgfpathlineto{\pgfqpoint{1.519001in}{0.689389in}}%
\pgfpathlineto{\pgfqpoint{1.530737in}{0.699259in}}%
\pgfpathlineto{\pgfqpoint{1.534657in}{0.702573in}}%
\pgfpathlineto{\pgfqpoint{1.547712in}{0.712870in}}%
\pgfpathlineto{\pgfqpoint{1.550314in}{0.715025in}}%
\pgfpathlineto{\pgfqpoint{1.565697in}{0.726481in}}%
\pgfpathlineto{\pgfqpoint{1.565971in}{0.726706in}}%
\pgfpathlineto{\pgfqpoint{1.581627in}{0.737919in}}%
\pgfpathlineto{\pgfqpoint{1.585340in}{0.740092in}}%
\pgfpathlineto{\pgfqpoint{1.597284in}{0.748430in}}%
\pgfpathlineto{\pgfqpoint{1.607227in}{0.753703in}}%
\pgfpathlineto{\pgfqpoint{1.612940in}{0.757713in}}%
\pgfpathlineto{\pgfqpoint{1.628597in}{0.765122in}}%
\pgfpathlineto{\pgfqpoint{1.637850in}{0.767314in}}%
\pgfpathlineto{\pgfqpoint{1.644253in}{0.769767in}}%
\pgfpathlineto{\pgfqpoint{1.659910in}{0.769767in}}%
\pgfpathlineto{\pgfqpoint{1.666313in}{0.767314in}}%
\pgfpathlineto{\pgfqpoint{1.675567in}{0.765122in}}%
\pgfpathlineto{\pgfqpoint{1.691223in}{0.757713in}}%
\pgfpathlineto{\pgfqpoint{1.696936in}{0.753703in}}%
\pgfpathlineto{\pgfqpoint{1.706880in}{0.748430in}}%
\pgfpathlineto{\pgfqpoint{1.718823in}{0.740092in}}%
\pgfpathlineto{\pgfqpoint{1.722536in}{0.737919in}}%
\pgfpathlineto{\pgfqpoint{1.738193in}{0.726706in}}%
\pgfpathlineto{\pgfqpoint{1.738466in}{0.726481in}}%
\pgfpathlineto{\pgfqpoint{1.753849in}{0.715025in}}%
\pgfpathlineto{\pgfqpoint{1.756451in}{0.712870in}}%
\pgfpathlineto{\pgfqpoint{1.769506in}{0.702573in}}%
\pgfpathlineto{\pgfqpoint{1.773426in}{0.699259in}}%
\pgfpathlineto{\pgfqpoint{1.785162in}{0.689389in}}%
\pgfpathlineto{\pgfqpoint{1.789466in}{0.685648in}}%
\pgfpathlineto{\pgfqpoint{1.800819in}{0.675444in}}%
\pgfpathlineto{\pgfqpoint{1.804632in}{0.672036in}}%
\pgfpathlineto{\pgfqpoint{1.816476in}{0.660687in}}%
\pgfpathlineto{\pgfqpoint{1.818954in}{0.658425in}}%
\pgfpathlineto{\pgfqpoint{1.832132in}{0.645052in}}%
\pgfpathlineto{\pgfqpoint{1.832392in}{0.644814in}}%
\pgfpathlineto{\pgfqpoint{1.845289in}{0.631203in}}%
\pgfpathlineto{\pgfqpoint{1.847789in}{0.627975in}}%
\pgfpathlineto{\pgfqpoint{1.857380in}{0.617592in}}%
\pgfpathlineto{\pgfqpoint{1.863445in}{0.608947in}}%
\pgfpathlineto{\pgfqpoint{1.868058in}{0.603981in}}%
\pgfpathlineto{\pgfqpoint{1.876580in}{0.590370in}}%
\pgfpathlineto{\pgfqpoint{1.879102in}{0.582325in}}%
\pgfpathlineto{\pgfqpoint{1.881923in}{0.576759in}}%
\pgfpathlineto{\pgfqpoint{1.881923in}{0.563148in}}%
\pgfpathlineto{\pgfqpoint{1.879102in}{0.557581in}}%
\pgfpathlineto{\pgfqpoint{1.876580in}{0.549536in}}%
\pgfpathlineto{\pgfqpoint{1.868058in}{0.535925in}}%
\pgfpathlineto{\pgfqpoint{1.863445in}{0.530959in}}%
\pgfpathlineto{\pgfqpoint{1.857380in}{0.522314in}}%
\pgfpathlineto{\pgfqpoint{1.847789in}{0.511931in}}%
\pgfpathlineto{\pgfqpoint{1.845289in}{0.508703in}}%
\pgfpathlineto{\pgfqpoint{1.832392in}{0.495092in}}%
\pgfpathlineto{\pgfqpoint{1.832132in}{0.494854in}}%
\pgfpathlineto{\pgfqpoint{1.818954in}{0.481481in}}%
\pgfpathlineto{\pgfqpoint{1.816476in}{0.479219in}}%
\pgfpathlineto{\pgfqpoint{1.804632in}{0.467870in}}%
\pgfpathlineto{\pgfqpoint{1.800819in}{0.464462in}}%
\pgfpathlineto{\pgfqpoint{1.789466in}{0.454259in}}%
\pgfpathlineto{\pgfqpoint{1.785162in}{0.450517in}}%
\pgfpathlineto{\pgfqpoint{1.773426in}{0.440648in}}%
\pgfpathlineto{\pgfqpoint{1.769506in}{0.437333in}}%
\pgfpathlineto{\pgfqpoint{1.756451in}{0.427036in}}%
\pgfpathlineto{\pgfqpoint{1.753849in}{0.424882in}}%
\pgfpathlineto{\pgfqpoint{1.738466in}{0.413425in}}%
\pgfpathlineto{\pgfqpoint{1.738193in}{0.413200in}}%
\pgfpathlineto{\pgfqpoint{1.722536in}{0.401987in}}%
\pgfpathlineto{\pgfqpoint{1.718823in}{0.399814in}}%
\pgfpathlineto{\pgfqpoint{1.706880in}{0.391476in}}%
\pgfpathlineto{\pgfqpoint{1.696936in}{0.386203in}}%
\pgfpathlineto{\pgfqpoint{1.691223in}{0.382193in}}%
\pgfpathlineto{\pgfqpoint{1.675567in}{0.374784in}}%
\pgfpathlineto{\pgfqpoint{1.666313in}{0.372592in}}%
\pgfpathlineto{\pgfqpoint{1.659910in}{0.370140in}}%
\pgfpathlineto{\pgfqpoint{1.644253in}{0.370140in}}%
\pgfpathlineto{\pgfqpoint{1.637850in}{0.372592in}}%
\pgfpathclose%
\pgfpathmoveto{\pgfqpoint{0.839234in}{0.658425in}}%
\pgfpathlineto{\pgfqpoint{0.830112in}{0.663332in}}%
\pgfpathlineto{\pgfqpoint{0.817643in}{0.672036in}}%
\pgfpathlineto{\pgfqpoint{0.814455in}{0.674136in}}%
\pgfpathlineto{\pgfqpoint{0.799561in}{0.685648in}}%
\pgfpathlineto{\pgfqpoint{0.798799in}{0.686220in}}%
\pgfpathlineto{\pgfqpoint{0.783142in}{0.699156in}}%
\pgfpathlineto{\pgfqpoint{0.783024in}{0.699259in}}%
\pgfpathlineto{\pgfqpoint{0.767486in}{0.712767in}}%
\pgfpathlineto{\pgfqpoint{0.767368in}{0.712870in}}%
\pgfpathlineto{\pgfqpoint{0.752488in}{0.726481in}}%
\pgfpathlineto{\pgfqpoint{0.751829in}{0.727143in}}%
\pgfpathlineto{\pgfqpoint{0.738588in}{0.740092in}}%
\pgfpathlineto{\pgfqpoint{0.736173in}{0.742863in}}%
\pgfpathlineto{\pgfqpoint{0.726160in}{0.753703in}}%
\pgfpathlineto{\pgfqpoint{0.720516in}{0.761634in}}%
\pgfpathlineto{\pgfqpoint{0.716096in}{0.767314in}}%
\pgfpathlineto{\pgfqpoint{0.709420in}{0.780925in}}%
\pgfpathlineto{\pgfqpoint{0.707123in}{0.794536in}}%
\pgfpathlineto{\pgfqpoint{0.709420in}{0.808148in}}%
\pgfpathlineto{\pgfqpoint{0.716096in}{0.821759in}}%
\pgfpathlineto{\pgfqpoint{0.720516in}{0.827439in}}%
\pgfpathlineto{\pgfqpoint{0.726160in}{0.835370in}}%
\pgfpathlineto{\pgfqpoint{0.736173in}{0.846209in}}%
\pgfpathlineto{\pgfqpoint{0.738588in}{0.848981in}}%
\pgfpathlineto{\pgfqpoint{0.751829in}{0.861930in}}%
\pgfpathlineto{\pgfqpoint{0.752488in}{0.862592in}}%
\pgfpathlineto{\pgfqpoint{0.767368in}{0.876203in}}%
\pgfpathlineto{\pgfqpoint{0.767486in}{0.876306in}}%
\pgfpathlineto{\pgfqpoint{0.783024in}{0.889814in}}%
\pgfpathlineto{\pgfqpoint{0.783142in}{0.889916in}}%
\pgfpathlineto{\pgfqpoint{0.798799in}{0.902852in}}%
\pgfpathlineto{\pgfqpoint{0.799561in}{0.903425in}}%
\pgfpathlineto{\pgfqpoint{0.814455in}{0.914936in}}%
\pgfpathlineto{\pgfqpoint{0.817643in}{0.917036in}}%
\pgfpathlineto{\pgfqpoint{0.830112in}{0.925741in}}%
\pgfpathlineto{\pgfqpoint{0.839234in}{0.930648in}}%
\pgfpathlineto{\pgfqpoint{0.845769in}{0.934490in}}%
\pgfpathlineto{\pgfqpoint{0.861425in}{0.940293in}}%
\pgfpathlineto{\pgfqpoint{0.877082in}{0.942291in}}%
\pgfpathlineto{\pgfqpoint{0.892738in}{0.940293in}}%
\pgfpathlineto{\pgfqpoint{0.908395in}{0.934490in}}%
\pgfpathlineto{\pgfqpoint{0.914929in}{0.930648in}}%
\pgfpathlineto{\pgfqpoint{0.924051in}{0.925741in}}%
\pgfpathlineto{\pgfqpoint{0.936520in}{0.917036in}}%
\pgfpathlineto{\pgfqpoint{0.939708in}{0.914936in}}%
\pgfpathlineto{\pgfqpoint{0.954603in}{0.903425in}}%
\pgfpathlineto{\pgfqpoint{0.955364in}{0.902852in}}%
\pgfpathlineto{\pgfqpoint{0.971021in}{0.889916in}}%
\pgfpathlineto{\pgfqpoint{0.971139in}{0.889814in}}%
\pgfpathlineto{\pgfqpoint{0.986678in}{0.876306in}}%
\pgfpathlineto{\pgfqpoint{0.986795in}{0.876203in}}%
\pgfpathlineto{\pgfqpoint{1.001675in}{0.862592in}}%
\pgfpathlineto{\pgfqpoint{1.002334in}{0.861930in}}%
\pgfpathlineto{\pgfqpoint{1.015575in}{0.848981in}}%
\pgfpathlineto{\pgfqpoint{1.017991in}{0.846209in}}%
\pgfpathlineto{\pgfqpoint{1.028003in}{0.835370in}}%
\pgfpathlineto{\pgfqpoint{1.033647in}{0.827439in}}%
\pgfpathlineto{\pgfqpoint{1.038067in}{0.821759in}}%
\pgfpathlineto{\pgfqpoint{1.044743in}{0.808148in}}%
\pgfpathlineto{\pgfqpoint{1.047041in}{0.794536in}}%
\pgfpathlineto{\pgfqpoint{1.044743in}{0.780925in}}%
\pgfpathlineto{\pgfqpoint{1.038067in}{0.767314in}}%
\pgfpathlineto{\pgfqpoint{1.033647in}{0.761634in}}%
\pgfpathlineto{\pgfqpoint{1.028003in}{0.753703in}}%
\pgfpathlineto{\pgfqpoint{1.017991in}{0.742863in}}%
\pgfpathlineto{\pgfqpoint{1.015575in}{0.740092in}}%
\pgfpathlineto{\pgfqpoint{1.002334in}{0.727143in}}%
\pgfpathlineto{\pgfqpoint{1.001675in}{0.726481in}}%
\pgfpathlineto{\pgfqpoint{0.986795in}{0.712870in}}%
\pgfpathlineto{\pgfqpoint{0.986678in}{0.712767in}}%
\pgfpathlineto{\pgfqpoint{0.971139in}{0.699259in}}%
\pgfpathlineto{\pgfqpoint{0.971021in}{0.699156in}}%
\pgfpathlineto{\pgfqpoint{0.955364in}{0.686220in}}%
\pgfpathlineto{\pgfqpoint{0.954603in}{0.685648in}}%
\pgfpathlineto{\pgfqpoint{0.939708in}{0.674136in}}%
\pgfpathlineto{\pgfqpoint{0.936520in}{0.672036in}}%
\pgfpathlineto{\pgfqpoint{0.924051in}{0.663332in}}%
\pgfpathlineto{\pgfqpoint{0.914929in}{0.658425in}}%
\pgfpathlineto{\pgfqpoint{0.908395in}{0.654583in}}%
\pgfpathlineto{\pgfqpoint{0.892738in}{0.648779in}}%
\pgfpathlineto{\pgfqpoint{0.877082in}{0.646782in}}%
\pgfpathlineto{\pgfqpoint{0.861425in}{0.648779in}}%
\pgfpathlineto{\pgfqpoint{0.845769in}{0.654583in}}%
\pgfpathlineto{\pgfqpoint{0.839234in}{0.658425in}}%
\pgfpathclose%
\pgfpathmoveto{\pgfqpoint{1.355901in}{0.658425in}}%
\pgfpathlineto{\pgfqpoint{1.346779in}{0.663332in}}%
\pgfpathlineto{\pgfqpoint{1.334310in}{0.672036in}}%
\pgfpathlineto{\pgfqpoint{1.331122in}{0.674136in}}%
\pgfpathlineto{\pgfqpoint{1.316227in}{0.685648in}}%
\pgfpathlineto{\pgfqpoint{1.315466in}{0.686220in}}%
\pgfpathlineto{\pgfqpoint{1.299809in}{0.699156in}}%
\pgfpathlineto{\pgfqpoint{1.299691in}{0.699259in}}%
\pgfpathlineto{\pgfqpoint{1.284152in}{0.712767in}}%
\pgfpathlineto{\pgfqpoint{1.284035in}{0.712870in}}%
\pgfpathlineto{\pgfqpoint{1.269155in}{0.726481in}}%
\pgfpathlineto{\pgfqpoint{1.268496in}{0.727143in}}%
\pgfpathlineto{\pgfqpoint{1.255255in}{0.740092in}}%
\pgfpathlineto{\pgfqpoint{1.252839in}{0.742863in}}%
\pgfpathlineto{\pgfqpoint{1.242827in}{0.753703in}}%
\pgfpathlineto{\pgfqpoint{1.237183in}{0.761634in}}%
\pgfpathlineto{\pgfqpoint{1.232763in}{0.767314in}}%
\pgfpathlineto{\pgfqpoint{1.226087in}{0.780925in}}%
\pgfpathlineto{\pgfqpoint{1.223789in}{0.794536in}}%
\pgfpathlineto{\pgfqpoint{1.226087in}{0.808148in}}%
\pgfpathlineto{\pgfqpoint{1.232763in}{0.821759in}}%
\pgfpathlineto{\pgfqpoint{1.237183in}{0.827439in}}%
\pgfpathlineto{\pgfqpoint{1.242827in}{0.835370in}}%
\pgfpathlineto{\pgfqpoint{1.252839in}{0.846209in}}%
\pgfpathlineto{\pgfqpoint{1.255255in}{0.848981in}}%
\pgfpathlineto{\pgfqpoint{1.268496in}{0.861930in}}%
\pgfpathlineto{\pgfqpoint{1.269155in}{0.862592in}}%
\pgfpathlineto{\pgfqpoint{1.284035in}{0.876203in}}%
\pgfpathlineto{\pgfqpoint{1.284152in}{0.876306in}}%
\pgfpathlineto{\pgfqpoint{1.299691in}{0.889814in}}%
\pgfpathlineto{\pgfqpoint{1.299809in}{0.889916in}}%
\pgfpathlineto{\pgfqpoint{1.315466in}{0.902852in}}%
\pgfpathlineto{\pgfqpoint{1.316227in}{0.903425in}}%
\pgfpathlineto{\pgfqpoint{1.331122in}{0.914936in}}%
\pgfpathlineto{\pgfqpoint{1.334310in}{0.917036in}}%
\pgfpathlineto{\pgfqpoint{1.346779in}{0.925741in}}%
\pgfpathlineto{\pgfqpoint{1.355901in}{0.930648in}}%
\pgfpathlineto{\pgfqpoint{1.362435in}{0.934490in}}%
\pgfpathlineto{\pgfqpoint{1.378092in}{0.940293in}}%
\pgfpathlineto{\pgfqpoint{1.393748in}{0.942291in}}%
\pgfpathlineto{\pgfqpoint{1.409405in}{0.940293in}}%
\pgfpathlineto{\pgfqpoint{1.425061in}{0.934490in}}%
\pgfpathlineto{\pgfqpoint{1.431596in}{0.930648in}}%
\pgfpathlineto{\pgfqpoint{1.440718in}{0.925741in}}%
\pgfpathlineto{\pgfqpoint{1.453187in}{0.917036in}}%
\pgfpathlineto{\pgfqpoint{1.456375in}{0.914936in}}%
\pgfpathlineto{\pgfqpoint{1.471269in}{0.903425in}}%
\pgfpathlineto{\pgfqpoint{1.472031in}{0.902852in}}%
\pgfpathlineto{\pgfqpoint{1.487688in}{0.889916in}}%
\pgfpathlineto{\pgfqpoint{1.487806in}{0.889814in}}%
\pgfpathlineto{\pgfqpoint{1.503344in}{0.876306in}}%
\pgfpathlineto{\pgfqpoint{1.503462in}{0.876203in}}%
\pgfpathlineto{\pgfqpoint{1.518342in}{0.862592in}}%
\pgfpathlineto{\pgfqpoint{1.519001in}{0.861930in}}%
\pgfpathlineto{\pgfqpoint{1.532242in}{0.848981in}}%
\pgfpathlineto{\pgfqpoint{1.534657in}{0.846209in}}%
\pgfpathlineto{\pgfqpoint{1.544670in}{0.835370in}}%
\pgfpathlineto{\pgfqpoint{1.550314in}{0.827439in}}%
\pgfpathlineto{\pgfqpoint{1.554734in}{0.821759in}}%
\pgfpathlineto{\pgfqpoint{1.561410in}{0.808148in}}%
\pgfpathlineto{\pgfqpoint{1.563707in}{0.794536in}}%
\pgfpathlineto{\pgfqpoint{1.561410in}{0.780925in}}%
\pgfpathlineto{\pgfqpoint{1.554734in}{0.767314in}}%
\pgfpathlineto{\pgfqpoint{1.550314in}{0.761634in}}%
\pgfpathlineto{\pgfqpoint{1.544670in}{0.753703in}}%
\pgfpathlineto{\pgfqpoint{1.534657in}{0.742863in}}%
\pgfpathlineto{\pgfqpoint{1.532242in}{0.740092in}}%
\pgfpathlineto{\pgfqpoint{1.519001in}{0.727143in}}%
\pgfpathlineto{\pgfqpoint{1.518342in}{0.726481in}}%
\pgfpathlineto{\pgfqpoint{1.503462in}{0.712870in}}%
\pgfpathlineto{\pgfqpoint{1.503344in}{0.712767in}}%
\pgfpathlineto{\pgfqpoint{1.487806in}{0.699259in}}%
\pgfpathlineto{\pgfqpoint{1.487688in}{0.699156in}}%
\pgfpathlineto{\pgfqpoint{1.472031in}{0.686220in}}%
\pgfpathlineto{\pgfqpoint{1.471269in}{0.685648in}}%
\pgfpathlineto{\pgfqpoint{1.456375in}{0.674136in}}%
\pgfpathlineto{\pgfqpoint{1.453187in}{0.672036in}}%
\pgfpathlineto{\pgfqpoint{1.440718in}{0.663332in}}%
\pgfpathlineto{\pgfqpoint{1.431596in}{0.658425in}}%
\pgfpathlineto{\pgfqpoint{1.425061in}{0.654583in}}%
\pgfpathlineto{\pgfqpoint{1.409405in}{0.648779in}}%
\pgfpathlineto{\pgfqpoint{1.393748in}{0.646782in}}%
\pgfpathlineto{\pgfqpoint{1.378092in}{0.648779in}}%
\pgfpathlineto{\pgfqpoint{1.362435in}{0.654583in}}%
\pgfpathlineto{\pgfqpoint{1.355901in}{0.658425in}}%
\pgfpathclose%
\pgfpathmoveto{\pgfqpoint{0.604517in}{0.821759in}}%
\pgfpathlineto{\pgfqpoint{0.595263in}{0.823951in}}%
\pgfpathlineto{\pgfqpoint{0.579607in}{0.831359in}}%
\pgfpathlineto{\pgfqpoint{0.573894in}{0.835370in}}%
\pgfpathlineto{\pgfqpoint{0.563950in}{0.840643in}}%
\pgfpathlineto{\pgfqpoint{0.552007in}{0.848981in}}%
\pgfpathlineto{\pgfqpoint{0.548294in}{0.851154in}}%
\pgfpathlineto{\pgfqpoint{0.532637in}{0.862366in}}%
\pgfpathlineto{\pgfqpoint{0.532364in}{0.862592in}}%
\pgfpathlineto{\pgfqpoint{0.516981in}{0.874048in}}%
\pgfpathlineto{\pgfqpoint{0.514379in}{0.876203in}}%
\pgfpathlineto{\pgfqpoint{0.501324in}{0.886500in}}%
\pgfpathlineto{\pgfqpoint{0.497404in}{0.889814in}}%
\pgfpathlineto{\pgfqpoint{0.485668in}{0.899684in}}%
\pgfpathlineto{\pgfqpoint{0.481364in}{0.903425in}}%
\pgfpathlineto{\pgfqpoint{0.470011in}{0.913628in}}%
\pgfpathlineto{\pgfqpoint{0.466198in}{0.917036in}}%
\pgfpathlineto{\pgfqpoint{0.454354in}{0.928386in}}%
\pgfpathlineto{\pgfqpoint{0.451876in}{0.930648in}}%
\pgfpathlineto{\pgfqpoint{0.438698in}{0.944021in}}%
\pgfpathlineto{\pgfqpoint{0.438438in}{0.944259in}}%
\pgfpathlineto{\pgfqpoint{0.425541in}{0.957870in}}%
\pgfpathlineto{\pgfqpoint{0.423041in}{0.961098in}}%
\pgfpathlineto{\pgfqpoint{0.413450in}{0.971481in}}%
\pgfpathlineto{\pgfqpoint{0.407385in}{0.980126in}}%
\pgfpathlineto{\pgfqpoint{0.402772in}{0.985092in}}%
\pgfpathlineto{\pgfqpoint{0.394250in}{0.998703in}}%
\pgfpathlineto{\pgfqpoint{0.391728in}{1.006747in}}%
\pgfpathlineto{\pgfqpoint{0.388907in}{1.012314in}}%
\pgfpathlineto{\pgfqpoint{0.388907in}{1.025925in}}%
\pgfpathlineto{\pgfqpoint{0.391728in}{1.031492in}}%
\pgfpathlineto{\pgfqpoint{0.394250in}{1.039536in}}%
\pgfpathlineto{\pgfqpoint{0.402772in}{1.053148in}}%
\pgfpathlineto{\pgfqpoint{0.407385in}{1.058114in}}%
\pgfpathlineto{\pgfqpoint{0.413450in}{1.066759in}}%
\pgfpathlineto{\pgfqpoint{0.423041in}{1.077142in}}%
\pgfpathlineto{\pgfqpoint{0.425541in}{1.080370in}}%
\pgfpathlineto{\pgfqpoint{0.438438in}{1.093981in}}%
\pgfpathlineto{\pgfqpoint{0.438698in}{1.094219in}}%
\pgfpathlineto{\pgfqpoint{0.451876in}{1.107592in}}%
\pgfpathlineto{\pgfqpoint{0.454354in}{1.109854in}}%
\pgfpathlineto{\pgfqpoint{0.466198in}{1.121203in}}%
\pgfpathlineto{\pgfqpoint{0.470011in}{1.124611in}}%
\pgfpathlineto{\pgfqpoint{0.481364in}{1.134814in}}%
\pgfpathlineto{\pgfqpoint{0.485668in}{1.138555in}}%
\pgfpathlineto{\pgfqpoint{0.497404in}{1.148425in}}%
\pgfpathlineto{\pgfqpoint{0.501324in}{1.151740in}}%
\pgfpathlineto{\pgfqpoint{0.514379in}{1.162036in}}%
\pgfpathlineto{\pgfqpoint{0.516981in}{1.164191in}}%
\pgfpathlineto{\pgfqpoint{0.532364in}{1.175647in}}%
\pgfpathlineto{\pgfqpoint{0.532637in}{1.175873in}}%
\pgfpathlineto{\pgfqpoint{0.548294in}{1.187085in}}%
\pgfpathlineto{\pgfqpoint{0.552007in}{1.189259in}}%
\pgfpathlineto{\pgfqpoint{0.563950in}{1.197597in}}%
\pgfpathlineto{\pgfqpoint{0.573894in}{1.202870in}}%
\pgfpathlineto{\pgfqpoint{0.579607in}{1.206880in}}%
\pgfpathlineto{\pgfqpoint{0.595263in}{1.214288in}}%
\pgfpathlineto{\pgfqpoint{0.604517in}{1.216481in}}%
\pgfpathlineto{\pgfqpoint{0.610920in}{1.218933in}}%
\pgfpathlineto{\pgfqpoint{0.626577in}{1.218933in}}%
\pgfpathlineto{\pgfqpoint{0.632980in}{1.216481in}}%
\pgfpathlineto{\pgfqpoint{0.642233in}{1.214288in}}%
\pgfpathlineto{\pgfqpoint{0.657890in}{1.206880in}}%
\pgfpathlineto{\pgfqpoint{0.663603in}{1.202870in}}%
\pgfpathlineto{\pgfqpoint{0.673546in}{1.197597in}}%
\pgfpathlineto{\pgfqpoint{0.685490in}{1.189259in}}%
\pgfpathlineto{\pgfqpoint{0.689203in}{1.187085in}}%
\pgfpathlineto{\pgfqpoint{0.704859in}{1.175873in}}%
\pgfpathlineto{\pgfqpoint{0.705133in}{1.175647in}}%
\pgfpathlineto{\pgfqpoint{0.720516in}{1.164191in}}%
\pgfpathlineto{\pgfqpoint{0.723118in}{1.162036in}}%
\pgfpathlineto{\pgfqpoint{0.736173in}{1.151740in}}%
\pgfpathlineto{\pgfqpoint{0.740093in}{1.148425in}}%
\pgfpathlineto{\pgfqpoint{0.751829in}{1.138555in}}%
\pgfpathlineto{\pgfqpoint{0.756132in}{1.134814in}}%
\pgfpathlineto{\pgfqpoint{0.767486in}{1.124611in}}%
\pgfpathlineto{\pgfqpoint{0.771298in}{1.121203in}}%
\pgfpathlineto{\pgfqpoint{0.783142in}{1.109854in}}%
\pgfpathlineto{\pgfqpoint{0.785621in}{1.107592in}}%
\pgfpathlineto{\pgfqpoint{0.798799in}{1.094219in}}%
\pgfpathlineto{\pgfqpoint{0.799058in}{1.093981in}}%
\pgfpathlineto{\pgfqpoint{0.811956in}{1.080370in}}%
\pgfpathlineto{\pgfqpoint{0.814455in}{1.077142in}}%
\pgfpathlineto{\pgfqpoint{0.824047in}{1.066759in}}%
\pgfpathlineto{\pgfqpoint{0.830112in}{1.058114in}}%
\pgfpathlineto{\pgfqpoint{0.834725in}{1.053148in}}%
\pgfpathlineto{\pgfqpoint{0.843247in}{1.039536in}}%
\pgfpathlineto{\pgfqpoint{0.845769in}{1.031492in}}%
\pgfpathlineto{\pgfqpoint{0.848589in}{1.025925in}}%
\pgfpathlineto{\pgfqpoint{0.848589in}{1.012314in}}%
\pgfpathlineto{\pgfqpoint{0.845769in}{1.006747in}}%
\pgfpathlineto{\pgfqpoint{0.843247in}{0.998703in}}%
\pgfpathlineto{\pgfqpoint{0.834725in}{0.985092in}}%
\pgfpathlineto{\pgfqpoint{0.830112in}{0.980126in}}%
\pgfpathlineto{\pgfqpoint{0.824047in}{0.971481in}}%
\pgfpathlineto{\pgfqpoint{0.814455in}{0.961098in}}%
\pgfpathlineto{\pgfqpoint{0.811956in}{0.957870in}}%
\pgfpathlineto{\pgfqpoint{0.799058in}{0.944259in}}%
\pgfpathlineto{\pgfqpoint{0.798799in}{0.944021in}}%
\pgfpathlineto{\pgfqpoint{0.785621in}{0.930648in}}%
\pgfpathlineto{\pgfqpoint{0.783142in}{0.928386in}}%
\pgfpathlineto{\pgfqpoint{0.771298in}{0.917036in}}%
\pgfpathlineto{\pgfqpoint{0.767486in}{0.913628in}}%
\pgfpathlineto{\pgfqpoint{0.756132in}{0.903425in}}%
\pgfpathlineto{\pgfqpoint{0.751829in}{0.899684in}}%
\pgfpathlineto{\pgfqpoint{0.740093in}{0.889814in}}%
\pgfpathlineto{\pgfqpoint{0.736173in}{0.886500in}}%
\pgfpathlineto{\pgfqpoint{0.723118in}{0.876203in}}%
\pgfpathlineto{\pgfqpoint{0.720516in}{0.874048in}}%
\pgfpathlineto{\pgfqpoint{0.705133in}{0.862592in}}%
\pgfpathlineto{\pgfqpoint{0.704859in}{0.862366in}}%
\pgfpathlineto{\pgfqpoint{0.689203in}{0.851154in}}%
\pgfpathlineto{\pgfqpoint{0.685490in}{0.848981in}}%
\pgfpathlineto{\pgfqpoint{0.673546in}{0.840643in}}%
\pgfpathlineto{\pgfqpoint{0.663603in}{0.835370in}}%
\pgfpathlineto{\pgfqpoint{0.657890in}{0.831359in}}%
\pgfpathlineto{\pgfqpoint{0.642233in}{0.823951in}}%
\pgfpathlineto{\pgfqpoint{0.632980in}{0.821759in}}%
\pgfpathlineto{\pgfqpoint{0.626577in}{0.819306in}}%
\pgfpathlineto{\pgfqpoint{0.610920in}{0.819306in}}%
\pgfpathlineto{\pgfqpoint{0.604517in}{0.821759in}}%
\pgfpathclose%
\pgfpathmoveto{\pgfqpoint{1.121183in}{0.821759in}}%
\pgfpathlineto{\pgfqpoint{1.111930in}{0.823951in}}%
\pgfpathlineto{\pgfqpoint{1.096274in}{0.831359in}}%
\pgfpathlineto{\pgfqpoint{1.090561in}{0.835370in}}%
\pgfpathlineto{\pgfqpoint{1.080617in}{0.840643in}}%
\pgfpathlineto{\pgfqpoint{1.068674in}{0.848981in}}%
\pgfpathlineto{\pgfqpoint{1.064960in}{0.851154in}}%
\pgfpathlineto{\pgfqpoint{1.049304in}{0.862366in}}%
\pgfpathlineto{\pgfqpoint{1.049030in}{0.862592in}}%
\pgfpathlineto{\pgfqpoint{1.033647in}{0.874048in}}%
\pgfpathlineto{\pgfqpoint{1.031046in}{0.876203in}}%
\pgfpathlineto{\pgfqpoint{1.017991in}{0.886500in}}%
\pgfpathlineto{\pgfqpoint{1.014071in}{0.889814in}}%
\pgfpathlineto{\pgfqpoint{1.002334in}{0.899684in}}%
\pgfpathlineto{\pgfqpoint{0.998031in}{0.903425in}}%
\pgfpathlineto{\pgfqpoint{0.986678in}{0.913628in}}%
\pgfpathlineto{\pgfqpoint{0.982865in}{0.917036in}}%
\pgfpathlineto{\pgfqpoint{0.971021in}{0.928386in}}%
\pgfpathlineto{\pgfqpoint{0.968542in}{0.930648in}}%
\pgfpathlineto{\pgfqpoint{0.955364in}{0.944021in}}%
\pgfpathlineto{\pgfqpoint{0.955105in}{0.944259in}}%
\pgfpathlineto{\pgfqpoint{0.942208in}{0.957870in}}%
\pgfpathlineto{\pgfqpoint{0.939708in}{0.961098in}}%
\pgfpathlineto{\pgfqpoint{0.930117in}{0.971481in}}%
\pgfpathlineto{\pgfqpoint{0.924051in}{0.980126in}}%
\pgfpathlineto{\pgfqpoint{0.919438in}{0.985092in}}%
\pgfpathlineto{\pgfqpoint{0.910917in}{0.998703in}}%
\pgfpathlineto{\pgfqpoint{0.908395in}{1.006747in}}%
\pgfpathlineto{\pgfqpoint{0.905574in}{1.012314in}}%
\pgfpathlineto{\pgfqpoint{0.905574in}{1.025925in}}%
\pgfpathlineto{\pgfqpoint{0.908395in}{1.031492in}}%
\pgfpathlineto{\pgfqpoint{0.910917in}{1.039536in}}%
\pgfpathlineto{\pgfqpoint{0.919438in}{1.053148in}}%
\pgfpathlineto{\pgfqpoint{0.924051in}{1.058114in}}%
\pgfpathlineto{\pgfqpoint{0.930117in}{1.066759in}}%
\pgfpathlineto{\pgfqpoint{0.939708in}{1.077142in}}%
\pgfpathlineto{\pgfqpoint{0.942208in}{1.080370in}}%
\pgfpathlineto{\pgfqpoint{0.955105in}{1.093981in}}%
\pgfpathlineto{\pgfqpoint{0.955364in}{1.094219in}}%
\pgfpathlineto{\pgfqpoint{0.968542in}{1.107592in}}%
\pgfpathlineto{\pgfqpoint{0.971021in}{1.109854in}}%
\pgfpathlineto{\pgfqpoint{0.982865in}{1.121203in}}%
\pgfpathlineto{\pgfqpoint{0.986678in}{1.124611in}}%
\pgfpathlineto{\pgfqpoint{0.998031in}{1.134814in}}%
\pgfpathlineto{\pgfqpoint{1.002334in}{1.138555in}}%
\pgfpathlineto{\pgfqpoint{1.014071in}{1.148425in}}%
\pgfpathlineto{\pgfqpoint{1.017991in}{1.151740in}}%
\pgfpathlineto{\pgfqpoint{1.031046in}{1.162036in}}%
\pgfpathlineto{\pgfqpoint{1.033647in}{1.164191in}}%
\pgfpathlineto{\pgfqpoint{1.049030in}{1.175647in}}%
\pgfpathlineto{\pgfqpoint{1.049304in}{1.175873in}}%
\pgfpathlineto{\pgfqpoint{1.064960in}{1.187085in}}%
\pgfpathlineto{\pgfqpoint{1.068674in}{1.189259in}}%
\pgfpathlineto{\pgfqpoint{1.080617in}{1.197597in}}%
\pgfpathlineto{\pgfqpoint{1.090561in}{1.202870in}}%
\pgfpathlineto{\pgfqpoint{1.096274in}{1.206880in}}%
\pgfpathlineto{\pgfqpoint{1.111930in}{1.214288in}}%
\pgfpathlineto{\pgfqpoint{1.121183in}{1.216481in}}%
\pgfpathlineto{\pgfqpoint{1.127587in}{1.218933in}}%
\pgfpathlineto{\pgfqpoint{1.143243in}{1.218933in}}%
\pgfpathlineto{\pgfqpoint{1.149647in}{1.216481in}}%
\pgfpathlineto{\pgfqpoint{1.158900in}{1.214288in}}%
\pgfpathlineto{\pgfqpoint{1.174556in}{1.206880in}}%
\pgfpathlineto{\pgfqpoint{1.180269in}{1.202870in}}%
\pgfpathlineto{\pgfqpoint{1.190213in}{1.197597in}}%
\pgfpathlineto{\pgfqpoint{1.202156in}{1.189259in}}%
\pgfpathlineto{\pgfqpoint{1.205870in}{1.187085in}}%
\pgfpathlineto{\pgfqpoint{1.221526in}{1.175873in}}%
\pgfpathlineto{\pgfqpoint{1.221800in}{1.175647in}}%
\pgfpathlineto{\pgfqpoint{1.237183in}{1.164191in}}%
\pgfpathlineto{\pgfqpoint{1.239784in}{1.162036in}}%
\pgfpathlineto{\pgfqpoint{1.252839in}{1.151740in}}%
\pgfpathlineto{\pgfqpoint{1.256759in}{1.148425in}}%
\pgfpathlineto{\pgfqpoint{1.268496in}{1.138555in}}%
\pgfpathlineto{\pgfqpoint{1.272799in}{1.134814in}}%
\pgfpathlineto{\pgfqpoint{1.284152in}{1.124611in}}%
\pgfpathlineto{\pgfqpoint{1.287965in}{1.121203in}}%
\pgfpathlineto{\pgfqpoint{1.299809in}{1.109854in}}%
\pgfpathlineto{\pgfqpoint{1.302288in}{1.107592in}}%
\pgfpathlineto{\pgfqpoint{1.315466in}{1.094219in}}%
\pgfpathlineto{\pgfqpoint{1.315725in}{1.093981in}}%
\pgfpathlineto{\pgfqpoint{1.328622in}{1.080370in}}%
\pgfpathlineto{\pgfqpoint{1.331122in}{1.077142in}}%
\pgfpathlineto{\pgfqpoint{1.340713in}{1.066759in}}%
\pgfpathlineto{\pgfqpoint{1.346779in}{1.058114in}}%
\pgfpathlineto{\pgfqpoint{1.351392in}{1.053148in}}%
\pgfpathlineto{\pgfqpoint{1.359913in}{1.039536in}}%
\pgfpathlineto{\pgfqpoint{1.362435in}{1.031492in}}%
\pgfpathlineto{\pgfqpoint{1.365256in}{1.025925in}}%
\pgfpathlineto{\pgfqpoint{1.365256in}{1.012314in}}%
\pgfpathlineto{\pgfqpoint{1.362435in}{1.006747in}}%
\pgfpathlineto{\pgfqpoint{1.359913in}{0.998703in}}%
\pgfpathlineto{\pgfqpoint{1.351392in}{0.985092in}}%
\pgfpathlineto{\pgfqpoint{1.346779in}{0.980126in}}%
\pgfpathlineto{\pgfqpoint{1.340713in}{0.971481in}}%
\pgfpathlineto{\pgfqpoint{1.331122in}{0.961098in}}%
\pgfpathlineto{\pgfqpoint{1.328622in}{0.957870in}}%
\pgfpathlineto{\pgfqpoint{1.315725in}{0.944259in}}%
\pgfpathlineto{\pgfqpoint{1.315466in}{0.944021in}}%
\pgfpathlineto{\pgfqpoint{1.302288in}{0.930648in}}%
\pgfpathlineto{\pgfqpoint{1.299809in}{0.928386in}}%
\pgfpathlineto{\pgfqpoint{1.287965in}{0.917036in}}%
\pgfpathlineto{\pgfqpoint{1.284152in}{0.913628in}}%
\pgfpathlineto{\pgfqpoint{1.272799in}{0.903425in}}%
\pgfpathlineto{\pgfqpoint{1.268496in}{0.899684in}}%
\pgfpathlineto{\pgfqpoint{1.256759in}{0.889814in}}%
\pgfpathlineto{\pgfqpoint{1.252839in}{0.886500in}}%
\pgfpathlineto{\pgfqpoint{1.239784in}{0.876203in}}%
\pgfpathlineto{\pgfqpoint{1.237183in}{0.874048in}}%
\pgfpathlineto{\pgfqpoint{1.221800in}{0.862592in}}%
\pgfpathlineto{\pgfqpoint{1.221526in}{0.862366in}}%
\pgfpathlineto{\pgfqpoint{1.205870in}{0.851154in}}%
\pgfpathlineto{\pgfqpoint{1.202156in}{0.848981in}}%
\pgfpathlineto{\pgfqpoint{1.190213in}{0.840643in}}%
\pgfpathlineto{\pgfqpoint{1.180269in}{0.835370in}}%
\pgfpathlineto{\pgfqpoint{1.174556in}{0.831359in}}%
\pgfpathlineto{\pgfqpoint{1.158900in}{0.823951in}}%
\pgfpathlineto{\pgfqpoint{1.149647in}{0.821759in}}%
\pgfpathlineto{\pgfqpoint{1.143243in}{0.819306in}}%
\pgfpathlineto{\pgfqpoint{1.127587in}{0.819306in}}%
\pgfpathlineto{\pgfqpoint{1.121183in}{0.821759in}}%
\pgfpathclose%
\pgfpathmoveto{\pgfqpoint{1.637850in}{0.821759in}}%
\pgfpathlineto{\pgfqpoint{1.628597in}{0.823951in}}%
\pgfpathlineto{\pgfqpoint{1.612940in}{0.831359in}}%
\pgfpathlineto{\pgfqpoint{1.607227in}{0.835370in}}%
\pgfpathlineto{\pgfqpoint{1.597284in}{0.840643in}}%
\pgfpathlineto{\pgfqpoint{1.585340in}{0.848981in}}%
\pgfpathlineto{\pgfqpoint{1.581627in}{0.851154in}}%
\pgfpathlineto{\pgfqpoint{1.565971in}{0.862366in}}%
\pgfpathlineto{\pgfqpoint{1.565697in}{0.862592in}}%
\pgfpathlineto{\pgfqpoint{1.550314in}{0.874048in}}%
\pgfpathlineto{\pgfqpoint{1.547712in}{0.876203in}}%
\pgfpathlineto{\pgfqpoint{1.534657in}{0.886500in}}%
\pgfpathlineto{\pgfqpoint{1.530737in}{0.889814in}}%
\pgfpathlineto{\pgfqpoint{1.519001in}{0.899684in}}%
\pgfpathlineto{\pgfqpoint{1.514698in}{0.903425in}}%
\pgfpathlineto{\pgfqpoint{1.503344in}{0.913628in}}%
\pgfpathlineto{\pgfqpoint{1.499532in}{0.917036in}}%
\pgfpathlineto{\pgfqpoint{1.487688in}{0.928386in}}%
\pgfpathlineto{\pgfqpoint{1.485209in}{0.930648in}}%
\pgfpathlineto{\pgfqpoint{1.472031in}{0.944021in}}%
\pgfpathlineto{\pgfqpoint{1.471772in}{0.944259in}}%
\pgfpathlineto{\pgfqpoint{1.458874in}{0.957870in}}%
\pgfpathlineto{\pgfqpoint{1.456375in}{0.961098in}}%
\pgfpathlineto{\pgfqpoint{1.446783in}{0.971481in}}%
\pgfpathlineto{\pgfqpoint{1.440718in}{0.980126in}}%
\pgfpathlineto{\pgfqpoint{1.436105in}{0.985092in}}%
\pgfpathlineto{\pgfqpoint{1.427583in}{0.998703in}}%
\pgfpathlineto{\pgfqpoint{1.425061in}{1.006747in}}%
\pgfpathlineto{\pgfqpoint{1.422241in}{1.012314in}}%
\pgfpathlineto{\pgfqpoint{1.422241in}{1.025925in}}%
\pgfpathlineto{\pgfqpoint{1.425061in}{1.031492in}}%
\pgfpathlineto{\pgfqpoint{1.427583in}{1.039536in}}%
\pgfpathlineto{\pgfqpoint{1.436105in}{1.053148in}}%
\pgfpathlineto{\pgfqpoint{1.440718in}{1.058114in}}%
\pgfpathlineto{\pgfqpoint{1.446783in}{1.066759in}}%
\pgfpathlineto{\pgfqpoint{1.456375in}{1.077142in}}%
\pgfpathlineto{\pgfqpoint{1.458874in}{1.080370in}}%
\pgfpathlineto{\pgfqpoint{1.471772in}{1.093981in}}%
\pgfpathlineto{\pgfqpoint{1.472031in}{1.094219in}}%
\pgfpathlineto{\pgfqpoint{1.485209in}{1.107592in}}%
\pgfpathlineto{\pgfqpoint{1.487688in}{1.109854in}}%
\pgfpathlineto{\pgfqpoint{1.499532in}{1.121203in}}%
\pgfpathlineto{\pgfqpoint{1.503344in}{1.124611in}}%
\pgfpathlineto{\pgfqpoint{1.514698in}{1.134814in}}%
\pgfpathlineto{\pgfqpoint{1.519001in}{1.138555in}}%
\pgfpathlineto{\pgfqpoint{1.530737in}{1.148425in}}%
\pgfpathlineto{\pgfqpoint{1.534657in}{1.151740in}}%
\pgfpathlineto{\pgfqpoint{1.547712in}{1.162036in}}%
\pgfpathlineto{\pgfqpoint{1.550314in}{1.164191in}}%
\pgfpathlineto{\pgfqpoint{1.565697in}{1.175647in}}%
\pgfpathlineto{\pgfqpoint{1.565971in}{1.175873in}}%
\pgfpathlineto{\pgfqpoint{1.581627in}{1.187085in}}%
\pgfpathlineto{\pgfqpoint{1.585340in}{1.189259in}}%
\pgfpathlineto{\pgfqpoint{1.597284in}{1.197597in}}%
\pgfpathlineto{\pgfqpoint{1.607227in}{1.202870in}}%
\pgfpathlineto{\pgfqpoint{1.612940in}{1.206880in}}%
\pgfpathlineto{\pgfqpoint{1.628597in}{1.214288in}}%
\pgfpathlineto{\pgfqpoint{1.637850in}{1.216481in}}%
\pgfpathlineto{\pgfqpoint{1.644253in}{1.218933in}}%
\pgfpathlineto{\pgfqpoint{1.659910in}{1.218933in}}%
\pgfpathlineto{\pgfqpoint{1.666313in}{1.216481in}}%
\pgfpathlineto{\pgfqpoint{1.675567in}{1.214288in}}%
\pgfpathlineto{\pgfqpoint{1.691223in}{1.206880in}}%
\pgfpathlineto{\pgfqpoint{1.696936in}{1.202870in}}%
\pgfpathlineto{\pgfqpoint{1.706880in}{1.197597in}}%
\pgfpathlineto{\pgfqpoint{1.718823in}{1.189259in}}%
\pgfpathlineto{\pgfqpoint{1.722536in}{1.187085in}}%
\pgfpathlineto{\pgfqpoint{1.738193in}{1.175873in}}%
\pgfpathlineto{\pgfqpoint{1.738466in}{1.175647in}}%
\pgfpathlineto{\pgfqpoint{1.753849in}{1.164191in}}%
\pgfpathlineto{\pgfqpoint{1.756451in}{1.162036in}}%
\pgfpathlineto{\pgfqpoint{1.769506in}{1.151740in}}%
\pgfpathlineto{\pgfqpoint{1.773426in}{1.148425in}}%
\pgfpathlineto{\pgfqpoint{1.785162in}{1.138555in}}%
\pgfpathlineto{\pgfqpoint{1.789466in}{1.134814in}}%
\pgfpathlineto{\pgfqpoint{1.800819in}{1.124611in}}%
\pgfpathlineto{\pgfqpoint{1.804632in}{1.121203in}}%
\pgfpathlineto{\pgfqpoint{1.816476in}{1.109854in}}%
\pgfpathlineto{\pgfqpoint{1.818954in}{1.107592in}}%
\pgfpathlineto{\pgfqpoint{1.832132in}{1.094219in}}%
\pgfpathlineto{\pgfqpoint{1.832392in}{1.093981in}}%
\pgfpathlineto{\pgfqpoint{1.845289in}{1.080370in}}%
\pgfpathlineto{\pgfqpoint{1.847789in}{1.077142in}}%
\pgfpathlineto{\pgfqpoint{1.857380in}{1.066759in}}%
\pgfpathlineto{\pgfqpoint{1.863445in}{1.058114in}}%
\pgfpathlineto{\pgfqpoint{1.868058in}{1.053148in}}%
\pgfpathlineto{\pgfqpoint{1.876580in}{1.039536in}}%
\pgfpathlineto{\pgfqpoint{1.879102in}{1.031492in}}%
\pgfpathlineto{\pgfqpoint{1.881923in}{1.025925in}}%
\pgfpathlineto{\pgfqpoint{1.881923in}{1.012314in}}%
\pgfpathlineto{\pgfqpoint{1.879102in}{1.006747in}}%
\pgfpathlineto{\pgfqpoint{1.876580in}{0.998703in}}%
\pgfpathlineto{\pgfqpoint{1.868058in}{0.985092in}}%
\pgfpathlineto{\pgfqpoint{1.863445in}{0.980126in}}%
\pgfpathlineto{\pgfqpoint{1.857380in}{0.971481in}}%
\pgfpathlineto{\pgfqpoint{1.847789in}{0.961098in}}%
\pgfpathlineto{\pgfqpoint{1.845289in}{0.957870in}}%
\pgfpathlineto{\pgfqpoint{1.832392in}{0.944259in}}%
\pgfpathlineto{\pgfqpoint{1.832132in}{0.944021in}}%
\pgfpathlineto{\pgfqpoint{1.818954in}{0.930648in}}%
\pgfpathlineto{\pgfqpoint{1.816476in}{0.928386in}}%
\pgfpathlineto{\pgfqpoint{1.804632in}{0.917036in}}%
\pgfpathlineto{\pgfqpoint{1.800819in}{0.913628in}}%
\pgfpathlineto{\pgfqpoint{1.789466in}{0.903425in}}%
\pgfpathlineto{\pgfqpoint{1.785162in}{0.899684in}}%
\pgfpathlineto{\pgfqpoint{1.773426in}{0.889814in}}%
\pgfpathlineto{\pgfqpoint{1.769506in}{0.886500in}}%
\pgfpathlineto{\pgfqpoint{1.756451in}{0.876203in}}%
\pgfpathlineto{\pgfqpoint{1.753849in}{0.874048in}}%
\pgfpathlineto{\pgfqpoint{1.738466in}{0.862592in}}%
\pgfpathlineto{\pgfqpoint{1.738193in}{0.862366in}}%
\pgfpathlineto{\pgfqpoint{1.722536in}{0.851154in}}%
\pgfpathlineto{\pgfqpoint{1.718823in}{0.848981in}}%
\pgfpathlineto{\pgfqpoint{1.706880in}{0.840643in}}%
\pgfpathlineto{\pgfqpoint{1.696936in}{0.835370in}}%
\pgfpathlineto{\pgfqpoint{1.691223in}{0.831359in}}%
\pgfpathlineto{\pgfqpoint{1.675567in}{0.823951in}}%
\pgfpathlineto{\pgfqpoint{1.666313in}{0.821759in}}%
\pgfpathlineto{\pgfqpoint{1.659910in}{0.819306in}}%
\pgfpathlineto{\pgfqpoint{1.644253in}{0.819306in}}%
\pgfpathlineto{\pgfqpoint{1.637850in}{0.821759in}}%
\pgfpathclose%
\pgfpathmoveto{\pgfqpoint{0.839234in}{1.107592in}}%
\pgfpathlineto{\pgfqpoint{0.830112in}{1.112499in}}%
\pgfpathlineto{\pgfqpoint{0.817643in}{1.121203in}}%
\pgfpathlineto{\pgfqpoint{0.814455in}{1.123303in}}%
\pgfpathlineto{\pgfqpoint{0.799561in}{1.134814in}}%
\pgfpathlineto{\pgfqpoint{0.798799in}{1.135387in}}%
\pgfpathlineto{\pgfqpoint{0.783142in}{1.148323in}}%
\pgfpathlineto{\pgfqpoint{0.783024in}{1.148425in}}%
\pgfpathlineto{\pgfqpoint{0.767486in}{1.161934in}}%
\pgfpathlineto{\pgfqpoint{0.767368in}{1.162036in}}%
\pgfpathlineto{\pgfqpoint{0.752488in}{1.175647in}}%
\pgfpathlineto{\pgfqpoint{0.751829in}{1.176310in}}%
\pgfpathlineto{\pgfqpoint{0.738588in}{1.189259in}}%
\pgfpathlineto{\pgfqpoint{0.736173in}{1.192030in}}%
\pgfpathlineto{\pgfqpoint{0.726160in}{1.202870in}}%
\pgfpathlineto{\pgfqpoint{0.720516in}{1.210800in}}%
\pgfpathlineto{\pgfqpoint{0.716096in}{1.216481in}}%
\pgfpathlineto{\pgfqpoint{0.709420in}{1.230092in}}%
\pgfpathlineto{\pgfqpoint{0.707123in}{1.243703in}}%
\pgfpathlineto{\pgfqpoint{0.709420in}{1.257314in}}%
\pgfpathlineto{\pgfqpoint{0.716096in}{1.270925in}}%
\pgfpathlineto{\pgfqpoint{0.720516in}{1.276606in}}%
\pgfpathlineto{\pgfqpoint{0.726160in}{1.284536in}}%
\pgfpathlineto{\pgfqpoint{0.736173in}{1.295376in}}%
\pgfpathlineto{\pgfqpoint{0.738588in}{1.298148in}}%
\pgfpathlineto{\pgfqpoint{0.751829in}{1.311096in}}%
\pgfpathlineto{\pgfqpoint{0.752488in}{1.311759in}}%
\pgfpathlineto{\pgfqpoint{0.767368in}{1.325370in}}%
\pgfpathlineto{\pgfqpoint{0.767486in}{1.325472in}}%
\pgfpathlineto{\pgfqpoint{0.783024in}{1.338981in}}%
\pgfpathlineto{\pgfqpoint{0.783142in}{1.339083in}}%
\pgfpathlineto{\pgfqpoint{0.798799in}{1.352019in}}%
\pgfpathlineto{\pgfqpoint{0.799561in}{1.352592in}}%
\pgfpathlineto{\pgfqpoint{0.814455in}{1.364103in}}%
\pgfpathlineto{\pgfqpoint{0.817643in}{1.366203in}}%
\pgfpathlineto{\pgfqpoint{0.830112in}{1.374907in}}%
\pgfpathlineto{\pgfqpoint{0.839234in}{1.379814in}}%
\pgfpathlineto{\pgfqpoint{0.845769in}{1.383657in}}%
\pgfpathlineto{\pgfqpoint{0.861425in}{1.389460in}}%
\pgfpathlineto{\pgfqpoint{0.877082in}{1.391458in}}%
\pgfpathlineto{\pgfqpoint{0.892738in}{1.389460in}}%
\pgfpathlineto{\pgfqpoint{0.908395in}{1.383657in}}%
\pgfpathlineto{\pgfqpoint{0.914929in}{1.379814in}}%
\pgfpathlineto{\pgfqpoint{0.924051in}{1.374907in}}%
\pgfpathlineto{\pgfqpoint{0.936520in}{1.366203in}}%
\pgfpathlineto{\pgfqpoint{0.939708in}{1.364103in}}%
\pgfpathlineto{\pgfqpoint{0.954603in}{1.352592in}}%
\pgfpathlineto{\pgfqpoint{0.955364in}{1.352019in}}%
\pgfpathlineto{\pgfqpoint{0.971021in}{1.339083in}}%
\pgfpathlineto{\pgfqpoint{0.971139in}{1.338981in}}%
\pgfpathlineto{\pgfqpoint{0.986678in}{1.325472in}}%
\pgfpathlineto{\pgfqpoint{0.986795in}{1.325370in}}%
\pgfpathlineto{\pgfqpoint{1.001675in}{1.311759in}}%
\pgfpathlineto{\pgfqpoint{1.002334in}{1.311096in}}%
\pgfpathlineto{\pgfqpoint{1.015575in}{1.298148in}}%
\pgfpathlineto{\pgfqpoint{1.017991in}{1.295376in}}%
\pgfpathlineto{\pgfqpoint{1.028003in}{1.284536in}}%
\pgfpathlineto{\pgfqpoint{1.033647in}{1.276606in}}%
\pgfpathlineto{\pgfqpoint{1.038067in}{1.270925in}}%
\pgfpathlineto{\pgfqpoint{1.044743in}{1.257314in}}%
\pgfpathlineto{\pgfqpoint{1.047041in}{1.243703in}}%
\pgfpathlineto{\pgfqpoint{1.044743in}{1.230092in}}%
\pgfpathlineto{\pgfqpoint{1.038067in}{1.216481in}}%
\pgfpathlineto{\pgfqpoint{1.033647in}{1.210800in}}%
\pgfpathlineto{\pgfqpoint{1.028003in}{1.202870in}}%
\pgfpathlineto{\pgfqpoint{1.017991in}{1.192030in}}%
\pgfpathlineto{\pgfqpoint{1.015575in}{1.189259in}}%
\pgfpathlineto{\pgfqpoint{1.002334in}{1.176310in}}%
\pgfpathlineto{\pgfqpoint{1.001675in}{1.175647in}}%
\pgfpathlineto{\pgfqpoint{0.986795in}{1.162036in}}%
\pgfpathlineto{\pgfqpoint{0.986678in}{1.161934in}}%
\pgfpathlineto{\pgfqpoint{0.971139in}{1.148425in}}%
\pgfpathlineto{\pgfqpoint{0.971021in}{1.148323in}}%
\pgfpathlineto{\pgfqpoint{0.955364in}{1.135387in}}%
\pgfpathlineto{\pgfqpoint{0.954603in}{1.134814in}}%
\pgfpathlineto{\pgfqpoint{0.939708in}{1.123303in}}%
\pgfpathlineto{\pgfqpoint{0.936520in}{1.121203in}}%
\pgfpathlineto{\pgfqpoint{0.924051in}{1.112499in}}%
\pgfpathlineto{\pgfqpoint{0.914929in}{1.107592in}}%
\pgfpathlineto{\pgfqpoint{0.908395in}{1.103749in}}%
\pgfpathlineto{\pgfqpoint{0.892738in}{1.097946in}}%
\pgfpathlineto{\pgfqpoint{0.877082in}{1.095948in}}%
\pgfpathlineto{\pgfqpoint{0.861425in}{1.097946in}}%
\pgfpathlineto{\pgfqpoint{0.845769in}{1.103749in}}%
\pgfpathlineto{\pgfqpoint{0.839234in}{1.107592in}}%
\pgfpathclose%
\pgfpathmoveto{\pgfqpoint{1.355901in}{1.107592in}}%
\pgfpathlineto{\pgfqpoint{1.346779in}{1.112499in}}%
\pgfpathlineto{\pgfqpoint{1.334310in}{1.121203in}}%
\pgfpathlineto{\pgfqpoint{1.331122in}{1.123303in}}%
\pgfpathlineto{\pgfqpoint{1.316227in}{1.134814in}}%
\pgfpathlineto{\pgfqpoint{1.315466in}{1.135387in}}%
\pgfpathlineto{\pgfqpoint{1.299809in}{1.148323in}}%
\pgfpathlineto{\pgfqpoint{1.299691in}{1.148425in}}%
\pgfpathlineto{\pgfqpoint{1.284152in}{1.161934in}}%
\pgfpathlineto{\pgfqpoint{1.284035in}{1.162036in}}%
\pgfpathlineto{\pgfqpoint{1.269155in}{1.175647in}}%
\pgfpathlineto{\pgfqpoint{1.268496in}{1.176310in}}%
\pgfpathlineto{\pgfqpoint{1.255255in}{1.189259in}}%
\pgfpathlineto{\pgfqpoint{1.252839in}{1.192030in}}%
\pgfpathlineto{\pgfqpoint{1.242827in}{1.202870in}}%
\pgfpathlineto{\pgfqpoint{1.237183in}{1.210800in}}%
\pgfpathlineto{\pgfqpoint{1.232763in}{1.216481in}}%
\pgfpathlineto{\pgfqpoint{1.226087in}{1.230092in}}%
\pgfpathlineto{\pgfqpoint{1.223789in}{1.243703in}}%
\pgfpathlineto{\pgfqpoint{1.226087in}{1.257314in}}%
\pgfpathlineto{\pgfqpoint{1.232763in}{1.270925in}}%
\pgfpathlineto{\pgfqpoint{1.237183in}{1.276606in}}%
\pgfpathlineto{\pgfqpoint{1.242827in}{1.284536in}}%
\pgfpathlineto{\pgfqpoint{1.252839in}{1.295376in}}%
\pgfpathlineto{\pgfqpoint{1.255255in}{1.298148in}}%
\pgfpathlineto{\pgfqpoint{1.268496in}{1.311096in}}%
\pgfpathlineto{\pgfqpoint{1.269155in}{1.311759in}}%
\pgfpathlineto{\pgfqpoint{1.284035in}{1.325370in}}%
\pgfpathlineto{\pgfqpoint{1.284152in}{1.325472in}}%
\pgfpathlineto{\pgfqpoint{1.299691in}{1.338981in}}%
\pgfpathlineto{\pgfqpoint{1.299809in}{1.339083in}}%
\pgfpathlineto{\pgfqpoint{1.315466in}{1.352019in}}%
\pgfpathlineto{\pgfqpoint{1.316227in}{1.352592in}}%
\pgfpathlineto{\pgfqpoint{1.331122in}{1.364103in}}%
\pgfpathlineto{\pgfqpoint{1.334310in}{1.366203in}}%
\pgfpathlineto{\pgfqpoint{1.346779in}{1.374907in}}%
\pgfpathlineto{\pgfqpoint{1.355901in}{1.379814in}}%
\pgfpathlineto{\pgfqpoint{1.362435in}{1.383657in}}%
\pgfpathlineto{\pgfqpoint{1.378092in}{1.389460in}}%
\pgfpathlineto{\pgfqpoint{1.393748in}{1.391458in}}%
\pgfpathlineto{\pgfqpoint{1.409405in}{1.389460in}}%
\pgfpathlineto{\pgfqpoint{1.425061in}{1.383657in}}%
\pgfpathlineto{\pgfqpoint{1.431596in}{1.379814in}}%
\pgfpathlineto{\pgfqpoint{1.440718in}{1.374907in}}%
\pgfpathlineto{\pgfqpoint{1.453187in}{1.366203in}}%
\pgfpathlineto{\pgfqpoint{1.456375in}{1.364103in}}%
\pgfpathlineto{\pgfqpoint{1.471269in}{1.352592in}}%
\pgfpathlineto{\pgfqpoint{1.472031in}{1.352019in}}%
\pgfpathlineto{\pgfqpoint{1.487688in}{1.339083in}}%
\pgfpathlineto{\pgfqpoint{1.487806in}{1.338981in}}%
\pgfpathlineto{\pgfqpoint{1.503344in}{1.325472in}}%
\pgfpathlineto{\pgfqpoint{1.503462in}{1.325370in}}%
\pgfpathlineto{\pgfqpoint{1.518342in}{1.311759in}}%
\pgfpathlineto{\pgfqpoint{1.519001in}{1.311096in}}%
\pgfpathlineto{\pgfqpoint{1.532242in}{1.298148in}}%
\pgfpathlineto{\pgfqpoint{1.534657in}{1.295376in}}%
\pgfpathlineto{\pgfqpoint{1.544670in}{1.284536in}}%
\pgfpathlineto{\pgfqpoint{1.550314in}{1.276606in}}%
\pgfpathlineto{\pgfqpoint{1.554734in}{1.270925in}}%
\pgfpathlineto{\pgfqpoint{1.561410in}{1.257314in}}%
\pgfpathlineto{\pgfqpoint{1.563707in}{1.243703in}}%
\pgfpathlineto{\pgfqpoint{1.561410in}{1.230092in}}%
\pgfpathlineto{\pgfqpoint{1.554734in}{1.216481in}}%
\pgfpathlineto{\pgfqpoint{1.550314in}{1.210800in}}%
\pgfpathlineto{\pgfqpoint{1.544670in}{1.202870in}}%
\pgfpathlineto{\pgfqpoint{1.534657in}{1.192030in}}%
\pgfpathlineto{\pgfqpoint{1.532242in}{1.189259in}}%
\pgfpathlineto{\pgfqpoint{1.519001in}{1.176310in}}%
\pgfpathlineto{\pgfqpoint{1.518342in}{1.175647in}}%
\pgfpathlineto{\pgfqpoint{1.503462in}{1.162036in}}%
\pgfpathlineto{\pgfqpoint{1.503344in}{1.161934in}}%
\pgfpathlineto{\pgfqpoint{1.487806in}{1.148425in}}%
\pgfpathlineto{\pgfqpoint{1.487688in}{1.148323in}}%
\pgfpathlineto{\pgfqpoint{1.472031in}{1.135387in}}%
\pgfpathlineto{\pgfqpoint{1.471269in}{1.134814in}}%
\pgfpathlineto{\pgfqpoint{1.456375in}{1.123303in}}%
\pgfpathlineto{\pgfqpoint{1.453187in}{1.121203in}}%
\pgfpathlineto{\pgfqpoint{1.440718in}{1.112499in}}%
\pgfpathlineto{\pgfqpoint{1.431596in}{1.107592in}}%
\pgfpathlineto{\pgfqpoint{1.425061in}{1.103749in}}%
\pgfpathlineto{\pgfqpoint{1.409405in}{1.097946in}}%
\pgfpathlineto{\pgfqpoint{1.393748in}{1.095948in}}%
\pgfpathlineto{\pgfqpoint{1.378092in}{1.097946in}}%
\pgfpathlineto{\pgfqpoint{1.362435in}{1.103749in}}%
\pgfpathlineto{\pgfqpoint{1.355901in}{1.107592in}}%
\pgfpathclose%
\pgfpathmoveto{\pgfqpoint{0.604517in}{1.270925in}}%
\pgfpathlineto{\pgfqpoint{0.595263in}{1.273118in}}%
\pgfpathlineto{\pgfqpoint{0.579607in}{1.280526in}}%
\pgfpathlineto{\pgfqpoint{0.573894in}{1.284536in}}%
\pgfpathlineto{\pgfqpoint{0.563950in}{1.289809in}}%
\pgfpathlineto{\pgfqpoint{0.552007in}{1.298148in}}%
\pgfpathlineto{\pgfqpoint{0.548294in}{1.300321in}}%
\pgfpathlineto{\pgfqpoint{0.532637in}{1.311533in}}%
\pgfpathlineto{\pgfqpoint{0.532364in}{1.311759in}}%
\pgfpathlineto{\pgfqpoint{0.516981in}{1.323215in}}%
\pgfpathlineto{\pgfqpoint{0.514379in}{1.325370in}}%
\pgfpathlineto{\pgfqpoint{0.501324in}{1.335666in}}%
\pgfpathlineto{\pgfqpoint{0.497404in}{1.338981in}}%
\pgfpathlineto{\pgfqpoint{0.485668in}{1.348851in}}%
\pgfpathlineto{\pgfqpoint{0.481364in}{1.352592in}}%
\pgfpathlineto{\pgfqpoint{0.470011in}{1.362795in}}%
\pgfpathlineto{\pgfqpoint{0.466198in}{1.366203in}}%
\pgfpathlineto{\pgfqpoint{0.454354in}{1.377553in}}%
\pgfpathlineto{\pgfqpoint{0.451876in}{1.379814in}}%
\pgfpathlineto{\pgfqpoint{0.438698in}{1.393187in}}%
\pgfpathlineto{\pgfqpoint{0.438438in}{1.393425in}}%
\pgfpathlineto{\pgfqpoint{0.425541in}{1.407036in}}%
\pgfpathlineto{\pgfqpoint{0.423041in}{1.410264in}}%
\pgfpathlineto{\pgfqpoint{0.413450in}{1.420648in}}%
\pgfpathlineto{\pgfqpoint{0.407385in}{1.429292in}}%
\pgfpathlineto{\pgfqpoint{0.402772in}{1.434259in}}%
\pgfpathlineto{\pgfqpoint{0.394250in}{1.447870in}}%
\pgfpathlineto{\pgfqpoint{0.391728in}{1.455914in}}%
\pgfpathlineto{\pgfqpoint{0.388907in}{1.461481in}}%
\pgfpathlineto{\pgfqpoint{0.388907in}{1.475092in}}%
\pgfpathlineto{\pgfqpoint{0.391728in}{1.480659in}}%
\pgfpathlineto{\pgfqpoint{0.394250in}{1.488703in}}%
\pgfpathlineto{\pgfqpoint{0.402772in}{1.502314in}}%
\pgfpathlineto{\pgfqpoint{0.407385in}{1.507281in}}%
\pgfpathlineto{\pgfqpoint{0.413450in}{1.515925in}}%
\pgfpathlineto{\pgfqpoint{0.423041in}{1.526308in}}%
\pgfpathlineto{\pgfqpoint{0.425541in}{1.529536in}}%
\pgfpathlineto{\pgfqpoint{0.438438in}{1.543148in}}%
\pgfpathlineto{\pgfqpoint{0.438698in}{1.543385in}}%
\pgfpathlineto{\pgfqpoint{0.451876in}{1.556759in}}%
\pgfpathlineto{\pgfqpoint{0.454354in}{1.559020in}}%
\pgfpathlineto{\pgfqpoint{0.466198in}{1.570370in}}%
\pgfpathlineto{\pgfqpoint{0.470011in}{1.573778in}}%
\pgfpathlineto{\pgfqpoint{0.481364in}{1.583981in}}%
\pgfpathlineto{\pgfqpoint{0.485668in}{1.587722in}}%
\pgfpathlineto{\pgfqpoint{0.497404in}{1.597592in}}%
\pgfpathlineto{\pgfqpoint{0.501324in}{1.600907in}}%
\pgfpathlineto{\pgfqpoint{0.514379in}{1.611203in}}%
\pgfpathlineto{\pgfqpoint{0.516981in}{1.613358in}}%
\pgfpathlineto{\pgfqpoint{0.532364in}{1.624814in}}%
\pgfpathlineto{\pgfqpoint{0.532637in}{1.625040in}}%
\pgfpathlineto{\pgfqpoint{0.548294in}{1.636252in}}%
\pgfpathlineto{\pgfqpoint{0.552007in}{1.638425in}}%
\pgfpathlineto{\pgfqpoint{0.563950in}{1.646763in}}%
\pgfpathlineto{\pgfqpoint{0.573894in}{1.652036in}}%
\pgfpathlineto{\pgfqpoint{0.579607in}{1.656047in}}%
\pgfpathlineto{\pgfqpoint{0.595263in}{1.663455in}}%
\pgfpathlineto{\pgfqpoint{0.604517in}{1.665648in}}%
\pgfpathlineto{\pgfqpoint{0.610920in}{1.668100in}}%
\pgfpathlineto{\pgfqpoint{0.626577in}{1.668100in}}%
\pgfpathlineto{\pgfqpoint{0.632980in}{1.665648in}}%
\pgfpathlineto{\pgfqpoint{0.642233in}{1.663455in}}%
\pgfpathlineto{\pgfqpoint{0.657890in}{1.656047in}}%
\pgfpathlineto{\pgfqpoint{0.663603in}{1.652036in}}%
\pgfpathlineto{\pgfqpoint{0.673546in}{1.646763in}}%
\pgfpathlineto{\pgfqpoint{0.685490in}{1.638425in}}%
\pgfpathlineto{\pgfqpoint{0.689203in}{1.636252in}}%
\pgfpathlineto{\pgfqpoint{0.704859in}{1.625040in}}%
\pgfpathlineto{\pgfqpoint{0.705133in}{1.624814in}}%
\pgfpathlineto{\pgfqpoint{0.720516in}{1.613358in}}%
\pgfpathlineto{\pgfqpoint{0.723118in}{1.611203in}}%
\pgfpathlineto{\pgfqpoint{0.736173in}{1.600907in}}%
\pgfpathlineto{\pgfqpoint{0.740093in}{1.597592in}}%
\pgfpathlineto{\pgfqpoint{0.751829in}{1.587722in}}%
\pgfpathlineto{\pgfqpoint{0.756132in}{1.583981in}}%
\pgfpathlineto{\pgfqpoint{0.767486in}{1.573778in}}%
\pgfpathlineto{\pgfqpoint{0.771298in}{1.570370in}}%
\pgfpathlineto{\pgfqpoint{0.783142in}{1.559020in}}%
\pgfpathlineto{\pgfqpoint{0.785621in}{1.556759in}}%
\pgfpathlineto{\pgfqpoint{0.798799in}{1.543385in}}%
\pgfpathlineto{\pgfqpoint{0.799058in}{1.543148in}}%
\pgfpathlineto{\pgfqpoint{0.811956in}{1.529536in}}%
\pgfpathlineto{\pgfqpoint{0.814455in}{1.526308in}}%
\pgfpathlineto{\pgfqpoint{0.824047in}{1.515925in}}%
\pgfpathlineto{\pgfqpoint{0.830112in}{1.507281in}}%
\pgfpathlineto{\pgfqpoint{0.834725in}{1.502314in}}%
\pgfpathlineto{\pgfqpoint{0.843247in}{1.488703in}}%
\pgfpathlineto{\pgfqpoint{0.845769in}{1.480659in}}%
\pgfpathlineto{\pgfqpoint{0.848589in}{1.475092in}}%
\pgfpathlineto{\pgfqpoint{0.848589in}{1.461481in}}%
\pgfpathlineto{\pgfqpoint{0.845769in}{1.455914in}}%
\pgfpathlineto{\pgfqpoint{0.843247in}{1.447870in}}%
\pgfpathlineto{\pgfqpoint{0.834725in}{1.434259in}}%
\pgfpathlineto{\pgfqpoint{0.830112in}{1.429292in}}%
\pgfpathlineto{\pgfqpoint{0.824047in}{1.420648in}}%
\pgfpathlineto{\pgfqpoint{0.814455in}{1.410264in}}%
\pgfpathlineto{\pgfqpoint{0.811956in}{1.407036in}}%
\pgfpathlineto{\pgfqpoint{0.799058in}{1.393425in}}%
\pgfpathlineto{\pgfqpoint{0.798799in}{1.393187in}}%
\pgfpathlineto{\pgfqpoint{0.785621in}{1.379814in}}%
\pgfpathlineto{\pgfqpoint{0.783142in}{1.377553in}}%
\pgfpathlineto{\pgfqpoint{0.771298in}{1.366203in}}%
\pgfpathlineto{\pgfqpoint{0.767486in}{1.362795in}}%
\pgfpathlineto{\pgfqpoint{0.756132in}{1.352592in}}%
\pgfpathlineto{\pgfqpoint{0.751829in}{1.348851in}}%
\pgfpathlineto{\pgfqpoint{0.740093in}{1.338981in}}%
\pgfpathlineto{\pgfqpoint{0.736173in}{1.335666in}}%
\pgfpathlineto{\pgfqpoint{0.723118in}{1.325370in}}%
\pgfpathlineto{\pgfqpoint{0.720516in}{1.323215in}}%
\pgfpathlineto{\pgfqpoint{0.705133in}{1.311759in}}%
\pgfpathlineto{\pgfqpoint{0.704859in}{1.311533in}}%
\pgfpathlineto{\pgfqpoint{0.689203in}{1.300321in}}%
\pgfpathlineto{\pgfqpoint{0.685490in}{1.298148in}}%
\pgfpathlineto{\pgfqpoint{0.673546in}{1.289809in}}%
\pgfpathlineto{\pgfqpoint{0.663603in}{1.284536in}}%
\pgfpathlineto{\pgfqpoint{0.657890in}{1.280526in}}%
\pgfpathlineto{\pgfqpoint{0.642233in}{1.273118in}}%
\pgfpathlineto{\pgfqpoint{0.632980in}{1.270925in}}%
\pgfpathlineto{\pgfqpoint{0.626577in}{1.268473in}}%
\pgfpathlineto{\pgfqpoint{0.610920in}{1.268473in}}%
\pgfpathlineto{\pgfqpoint{0.604517in}{1.270925in}}%
\pgfpathclose%
\pgfpathmoveto{\pgfqpoint{1.121183in}{1.270925in}}%
\pgfpathlineto{\pgfqpoint{1.111930in}{1.273118in}}%
\pgfpathlineto{\pgfqpoint{1.096274in}{1.280526in}}%
\pgfpathlineto{\pgfqpoint{1.090561in}{1.284536in}}%
\pgfpathlineto{\pgfqpoint{1.080617in}{1.289809in}}%
\pgfpathlineto{\pgfqpoint{1.068674in}{1.298148in}}%
\pgfpathlineto{\pgfqpoint{1.064960in}{1.300321in}}%
\pgfpathlineto{\pgfqpoint{1.049304in}{1.311533in}}%
\pgfpathlineto{\pgfqpoint{1.049030in}{1.311759in}}%
\pgfpathlineto{\pgfqpoint{1.033647in}{1.323215in}}%
\pgfpathlineto{\pgfqpoint{1.031046in}{1.325370in}}%
\pgfpathlineto{\pgfqpoint{1.017991in}{1.335666in}}%
\pgfpathlineto{\pgfqpoint{1.014071in}{1.338981in}}%
\pgfpathlineto{\pgfqpoint{1.002334in}{1.348851in}}%
\pgfpathlineto{\pgfqpoint{0.998031in}{1.352592in}}%
\pgfpathlineto{\pgfqpoint{0.986678in}{1.362795in}}%
\pgfpathlineto{\pgfqpoint{0.982865in}{1.366203in}}%
\pgfpathlineto{\pgfqpoint{0.971021in}{1.377553in}}%
\pgfpathlineto{\pgfqpoint{0.968542in}{1.379814in}}%
\pgfpathlineto{\pgfqpoint{0.955364in}{1.393187in}}%
\pgfpathlineto{\pgfqpoint{0.955105in}{1.393425in}}%
\pgfpathlineto{\pgfqpoint{0.942208in}{1.407036in}}%
\pgfpathlineto{\pgfqpoint{0.939708in}{1.410264in}}%
\pgfpathlineto{\pgfqpoint{0.930117in}{1.420648in}}%
\pgfpathlineto{\pgfqpoint{0.924051in}{1.429292in}}%
\pgfpathlineto{\pgfqpoint{0.919438in}{1.434259in}}%
\pgfpathlineto{\pgfqpoint{0.910917in}{1.447870in}}%
\pgfpathlineto{\pgfqpoint{0.908395in}{1.455914in}}%
\pgfpathlineto{\pgfqpoint{0.905574in}{1.461481in}}%
\pgfpathlineto{\pgfqpoint{0.905574in}{1.475092in}}%
\pgfpathlineto{\pgfqpoint{0.908395in}{1.480659in}}%
\pgfpathlineto{\pgfqpoint{0.910917in}{1.488703in}}%
\pgfpathlineto{\pgfqpoint{0.919438in}{1.502314in}}%
\pgfpathlineto{\pgfqpoint{0.924051in}{1.507281in}}%
\pgfpathlineto{\pgfqpoint{0.930117in}{1.515925in}}%
\pgfpathlineto{\pgfqpoint{0.939708in}{1.526308in}}%
\pgfpathlineto{\pgfqpoint{0.942208in}{1.529536in}}%
\pgfpathlineto{\pgfqpoint{0.955105in}{1.543148in}}%
\pgfpathlineto{\pgfqpoint{0.955364in}{1.543385in}}%
\pgfpathlineto{\pgfqpoint{0.968542in}{1.556759in}}%
\pgfpathlineto{\pgfqpoint{0.971021in}{1.559020in}}%
\pgfpathlineto{\pgfqpoint{0.982865in}{1.570370in}}%
\pgfpathlineto{\pgfqpoint{0.986678in}{1.573778in}}%
\pgfpathlineto{\pgfqpoint{0.998031in}{1.583981in}}%
\pgfpathlineto{\pgfqpoint{1.002334in}{1.587722in}}%
\pgfpathlineto{\pgfqpoint{1.014071in}{1.597592in}}%
\pgfpathlineto{\pgfqpoint{1.017991in}{1.600907in}}%
\pgfpathlineto{\pgfqpoint{1.031046in}{1.611203in}}%
\pgfpathlineto{\pgfqpoint{1.033647in}{1.613358in}}%
\pgfpathlineto{\pgfqpoint{1.049030in}{1.624814in}}%
\pgfpathlineto{\pgfqpoint{1.049304in}{1.625040in}}%
\pgfpathlineto{\pgfqpoint{1.064960in}{1.636252in}}%
\pgfpathlineto{\pgfqpoint{1.068674in}{1.638425in}}%
\pgfpathlineto{\pgfqpoint{1.080617in}{1.646763in}}%
\pgfpathlineto{\pgfqpoint{1.090561in}{1.652036in}}%
\pgfpathlineto{\pgfqpoint{1.096274in}{1.656047in}}%
\pgfpathlineto{\pgfqpoint{1.111930in}{1.663455in}}%
\pgfpathlineto{\pgfqpoint{1.121183in}{1.665648in}}%
\pgfpathlineto{\pgfqpoint{1.127587in}{1.668100in}}%
\pgfpathlineto{\pgfqpoint{1.143243in}{1.668100in}}%
\pgfpathlineto{\pgfqpoint{1.149647in}{1.665648in}}%
\pgfpathlineto{\pgfqpoint{1.158900in}{1.663455in}}%
\pgfpathlineto{\pgfqpoint{1.174556in}{1.656047in}}%
\pgfpathlineto{\pgfqpoint{1.180269in}{1.652036in}}%
\pgfpathlineto{\pgfqpoint{1.190213in}{1.646763in}}%
\pgfpathlineto{\pgfqpoint{1.202156in}{1.638425in}}%
\pgfpathlineto{\pgfqpoint{1.205870in}{1.636252in}}%
\pgfpathlineto{\pgfqpoint{1.221526in}{1.625040in}}%
\pgfpathlineto{\pgfqpoint{1.221800in}{1.624814in}}%
\pgfpathlineto{\pgfqpoint{1.237183in}{1.613358in}}%
\pgfpathlineto{\pgfqpoint{1.239784in}{1.611203in}}%
\pgfpathlineto{\pgfqpoint{1.252839in}{1.600907in}}%
\pgfpathlineto{\pgfqpoint{1.256759in}{1.597592in}}%
\pgfpathlineto{\pgfqpoint{1.268496in}{1.587722in}}%
\pgfpathlineto{\pgfqpoint{1.272799in}{1.583981in}}%
\pgfpathlineto{\pgfqpoint{1.284152in}{1.573778in}}%
\pgfpathlineto{\pgfqpoint{1.287965in}{1.570370in}}%
\pgfpathlineto{\pgfqpoint{1.299809in}{1.559020in}}%
\pgfpathlineto{\pgfqpoint{1.302288in}{1.556759in}}%
\pgfpathlineto{\pgfqpoint{1.315466in}{1.543385in}}%
\pgfpathlineto{\pgfqpoint{1.315725in}{1.543148in}}%
\pgfpathlineto{\pgfqpoint{1.328622in}{1.529536in}}%
\pgfpathlineto{\pgfqpoint{1.331122in}{1.526308in}}%
\pgfpathlineto{\pgfqpoint{1.340713in}{1.515925in}}%
\pgfpathlineto{\pgfqpoint{1.346779in}{1.507281in}}%
\pgfpathlineto{\pgfqpoint{1.351392in}{1.502314in}}%
\pgfpathlineto{\pgfqpoint{1.359913in}{1.488703in}}%
\pgfpathlineto{\pgfqpoint{1.362435in}{1.480659in}}%
\pgfpathlineto{\pgfqpoint{1.365256in}{1.475092in}}%
\pgfpathlineto{\pgfqpoint{1.365256in}{1.461481in}}%
\pgfpathlineto{\pgfqpoint{1.362435in}{1.455914in}}%
\pgfpathlineto{\pgfqpoint{1.359913in}{1.447870in}}%
\pgfpathlineto{\pgfqpoint{1.351392in}{1.434259in}}%
\pgfpathlineto{\pgfqpoint{1.346779in}{1.429292in}}%
\pgfpathlineto{\pgfqpoint{1.340713in}{1.420648in}}%
\pgfpathlineto{\pgfqpoint{1.331122in}{1.410264in}}%
\pgfpathlineto{\pgfqpoint{1.328622in}{1.407036in}}%
\pgfpathlineto{\pgfqpoint{1.315725in}{1.393425in}}%
\pgfpathlineto{\pgfqpoint{1.315466in}{1.393187in}}%
\pgfpathlineto{\pgfqpoint{1.302288in}{1.379814in}}%
\pgfpathlineto{\pgfqpoint{1.299809in}{1.377553in}}%
\pgfpathlineto{\pgfqpoint{1.287965in}{1.366203in}}%
\pgfpathlineto{\pgfqpoint{1.284152in}{1.362795in}}%
\pgfpathlineto{\pgfqpoint{1.272799in}{1.352592in}}%
\pgfpathlineto{\pgfqpoint{1.268496in}{1.348851in}}%
\pgfpathlineto{\pgfqpoint{1.256759in}{1.338981in}}%
\pgfpathlineto{\pgfqpoint{1.252839in}{1.335666in}}%
\pgfpathlineto{\pgfqpoint{1.239784in}{1.325370in}}%
\pgfpathlineto{\pgfqpoint{1.237183in}{1.323215in}}%
\pgfpathlineto{\pgfqpoint{1.221800in}{1.311759in}}%
\pgfpathlineto{\pgfqpoint{1.221526in}{1.311533in}}%
\pgfpathlineto{\pgfqpoint{1.205870in}{1.300321in}}%
\pgfpathlineto{\pgfqpoint{1.202156in}{1.298148in}}%
\pgfpathlineto{\pgfqpoint{1.190213in}{1.289809in}}%
\pgfpathlineto{\pgfqpoint{1.180269in}{1.284536in}}%
\pgfpathlineto{\pgfqpoint{1.174556in}{1.280526in}}%
\pgfpathlineto{\pgfqpoint{1.158900in}{1.273118in}}%
\pgfpathlineto{\pgfqpoint{1.149647in}{1.270925in}}%
\pgfpathlineto{\pgfqpoint{1.143243in}{1.268473in}}%
\pgfpathlineto{\pgfqpoint{1.127587in}{1.268473in}}%
\pgfpathlineto{\pgfqpoint{1.121183in}{1.270925in}}%
\pgfpathclose%
\pgfpathmoveto{\pgfqpoint{1.637850in}{1.270925in}}%
\pgfpathlineto{\pgfqpoint{1.628597in}{1.273118in}}%
\pgfpathlineto{\pgfqpoint{1.612940in}{1.280526in}}%
\pgfpathlineto{\pgfqpoint{1.607227in}{1.284536in}}%
\pgfpathlineto{\pgfqpoint{1.597284in}{1.289809in}}%
\pgfpathlineto{\pgfqpoint{1.585340in}{1.298148in}}%
\pgfpathlineto{\pgfqpoint{1.581627in}{1.300321in}}%
\pgfpathlineto{\pgfqpoint{1.565971in}{1.311533in}}%
\pgfpathlineto{\pgfqpoint{1.565697in}{1.311759in}}%
\pgfpathlineto{\pgfqpoint{1.550314in}{1.323215in}}%
\pgfpathlineto{\pgfqpoint{1.547712in}{1.325370in}}%
\pgfpathlineto{\pgfqpoint{1.534657in}{1.335666in}}%
\pgfpathlineto{\pgfqpoint{1.530737in}{1.338981in}}%
\pgfpathlineto{\pgfqpoint{1.519001in}{1.348851in}}%
\pgfpathlineto{\pgfqpoint{1.514698in}{1.352592in}}%
\pgfpathlineto{\pgfqpoint{1.503344in}{1.362795in}}%
\pgfpathlineto{\pgfqpoint{1.499532in}{1.366203in}}%
\pgfpathlineto{\pgfqpoint{1.487688in}{1.377553in}}%
\pgfpathlineto{\pgfqpoint{1.485209in}{1.379814in}}%
\pgfpathlineto{\pgfqpoint{1.472031in}{1.393187in}}%
\pgfpathlineto{\pgfqpoint{1.471772in}{1.393425in}}%
\pgfpathlineto{\pgfqpoint{1.458874in}{1.407036in}}%
\pgfpathlineto{\pgfqpoint{1.456375in}{1.410264in}}%
\pgfpathlineto{\pgfqpoint{1.446783in}{1.420648in}}%
\pgfpathlineto{\pgfqpoint{1.440718in}{1.429292in}}%
\pgfpathlineto{\pgfqpoint{1.436105in}{1.434259in}}%
\pgfpathlineto{\pgfqpoint{1.427583in}{1.447870in}}%
\pgfpathlineto{\pgfqpoint{1.425061in}{1.455914in}}%
\pgfpathlineto{\pgfqpoint{1.422241in}{1.461481in}}%
\pgfpathlineto{\pgfqpoint{1.422241in}{1.475092in}}%
\pgfpathlineto{\pgfqpoint{1.425061in}{1.480659in}}%
\pgfpathlineto{\pgfqpoint{1.427583in}{1.488703in}}%
\pgfpathlineto{\pgfqpoint{1.436105in}{1.502314in}}%
\pgfpathlineto{\pgfqpoint{1.440718in}{1.507281in}}%
\pgfpathlineto{\pgfqpoint{1.446783in}{1.515925in}}%
\pgfpathlineto{\pgfqpoint{1.456375in}{1.526308in}}%
\pgfpathlineto{\pgfqpoint{1.458874in}{1.529536in}}%
\pgfpathlineto{\pgfqpoint{1.471772in}{1.543148in}}%
\pgfpathlineto{\pgfqpoint{1.472031in}{1.543385in}}%
\pgfpathlineto{\pgfqpoint{1.485209in}{1.556759in}}%
\pgfpathlineto{\pgfqpoint{1.487688in}{1.559020in}}%
\pgfpathlineto{\pgfqpoint{1.499532in}{1.570370in}}%
\pgfpathlineto{\pgfqpoint{1.503344in}{1.573778in}}%
\pgfpathlineto{\pgfqpoint{1.514698in}{1.583981in}}%
\pgfpathlineto{\pgfqpoint{1.519001in}{1.587722in}}%
\pgfpathlineto{\pgfqpoint{1.530737in}{1.597592in}}%
\pgfpathlineto{\pgfqpoint{1.534657in}{1.600907in}}%
\pgfpathlineto{\pgfqpoint{1.547712in}{1.611203in}}%
\pgfpathlineto{\pgfqpoint{1.550314in}{1.613358in}}%
\pgfpathlineto{\pgfqpoint{1.565697in}{1.624814in}}%
\pgfpathlineto{\pgfqpoint{1.565971in}{1.625040in}}%
\pgfpathlineto{\pgfqpoint{1.581627in}{1.636252in}}%
\pgfpathlineto{\pgfqpoint{1.585340in}{1.638425in}}%
\pgfpathlineto{\pgfqpoint{1.597284in}{1.646763in}}%
\pgfpathlineto{\pgfqpoint{1.607227in}{1.652036in}}%
\pgfpathlineto{\pgfqpoint{1.612940in}{1.656047in}}%
\pgfpathlineto{\pgfqpoint{1.628597in}{1.663455in}}%
\pgfpathlineto{\pgfqpoint{1.637850in}{1.665648in}}%
\pgfpathlineto{\pgfqpoint{1.644253in}{1.668100in}}%
\pgfpathlineto{\pgfqpoint{1.659910in}{1.668100in}}%
\pgfpathlineto{\pgfqpoint{1.666313in}{1.665648in}}%
\pgfpathlineto{\pgfqpoint{1.675567in}{1.663455in}}%
\pgfpathlineto{\pgfqpoint{1.691223in}{1.656047in}}%
\pgfpathlineto{\pgfqpoint{1.696936in}{1.652036in}}%
\pgfpathlineto{\pgfqpoint{1.706880in}{1.646763in}}%
\pgfpathlineto{\pgfqpoint{1.718823in}{1.638425in}}%
\pgfpathlineto{\pgfqpoint{1.722536in}{1.636252in}}%
\pgfpathlineto{\pgfqpoint{1.738193in}{1.625040in}}%
\pgfpathlineto{\pgfqpoint{1.738466in}{1.624814in}}%
\pgfpathlineto{\pgfqpoint{1.753849in}{1.613358in}}%
\pgfpathlineto{\pgfqpoint{1.756451in}{1.611203in}}%
\pgfpathlineto{\pgfqpoint{1.769506in}{1.600907in}}%
\pgfpathlineto{\pgfqpoint{1.773426in}{1.597592in}}%
\pgfpathlineto{\pgfqpoint{1.785162in}{1.587722in}}%
\pgfpathlineto{\pgfqpoint{1.789466in}{1.583981in}}%
\pgfpathlineto{\pgfqpoint{1.800819in}{1.573778in}}%
\pgfpathlineto{\pgfqpoint{1.804632in}{1.570370in}}%
\pgfpathlineto{\pgfqpoint{1.816476in}{1.559020in}}%
\pgfpathlineto{\pgfqpoint{1.818954in}{1.556759in}}%
\pgfpathlineto{\pgfqpoint{1.832132in}{1.543385in}}%
\pgfpathlineto{\pgfqpoint{1.832392in}{1.543148in}}%
\pgfpathlineto{\pgfqpoint{1.845289in}{1.529536in}}%
\pgfpathlineto{\pgfqpoint{1.847789in}{1.526308in}}%
\pgfpathlineto{\pgfqpoint{1.857380in}{1.515925in}}%
\pgfpathlineto{\pgfqpoint{1.863445in}{1.507281in}}%
\pgfpathlineto{\pgfqpoint{1.868058in}{1.502314in}}%
\pgfpathlineto{\pgfqpoint{1.876580in}{1.488703in}}%
\pgfpathlineto{\pgfqpoint{1.879102in}{1.480659in}}%
\pgfpathlineto{\pgfqpoint{1.881923in}{1.475092in}}%
\pgfpathlineto{\pgfqpoint{1.881923in}{1.461481in}}%
\pgfpathlineto{\pgfqpoint{1.879102in}{1.455914in}}%
\pgfpathlineto{\pgfqpoint{1.876580in}{1.447870in}}%
\pgfpathlineto{\pgfqpoint{1.868058in}{1.434259in}}%
\pgfpathlineto{\pgfqpoint{1.863445in}{1.429292in}}%
\pgfpathlineto{\pgfqpoint{1.857380in}{1.420648in}}%
\pgfpathlineto{\pgfqpoint{1.847789in}{1.410264in}}%
\pgfpathlineto{\pgfqpoint{1.845289in}{1.407036in}}%
\pgfpathlineto{\pgfqpoint{1.832392in}{1.393425in}}%
\pgfpathlineto{\pgfqpoint{1.832132in}{1.393187in}}%
\pgfpathlineto{\pgfqpoint{1.818954in}{1.379814in}}%
\pgfpathlineto{\pgfqpoint{1.816476in}{1.377553in}}%
\pgfpathlineto{\pgfqpoint{1.804632in}{1.366203in}}%
\pgfpathlineto{\pgfqpoint{1.800819in}{1.362795in}}%
\pgfpathlineto{\pgfqpoint{1.789466in}{1.352592in}}%
\pgfpathlineto{\pgfqpoint{1.785162in}{1.348851in}}%
\pgfpathlineto{\pgfqpoint{1.773426in}{1.338981in}}%
\pgfpathlineto{\pgfqpoint{1.769506in}{1.335666in}}%
\pgfpathlineto{\pgfqpoint{1.756451in}{1.325370in}}%
\pgfpathlineto{\pgfqpoint{1.753849in}{1.323215in}}%
\pgfpathlineto{\pgfqpoint{1.738466in}{1.311759in}}%
\pgfpathlineto{\pgfqpoint{1.738193in}{1.311533in}}%
\pgfpathlineto{\pgfqpoint{1.722536in}{1.300321in}}%
\pgfpathlineto{\pgfqpoint{1.718823in}{1.298148in}}%
\pgfpathlineto{\pgfqpoint{1.706880in}{1.289809in}}%
\pgfpathlineto{\pgfqpoint{1.696936in}{1.284536in}}%
\pgfpathlineto{\pgfqpoint{1.691223in}{1.280526in}}%
\pgfpathlineto{\pgfqpoint{1.675567in}{1.273118in}}%
\pgfpathlineto{\pgfqpoint{1.666313in}{1.270925in}}%
\pgfpathlineto{\pgfqpoint{1.659910in}{1.268473in}}%
\pgfpathlineto{\pgfqpoint{1.644253in}{1.268473in}}%
\pgfpathlineto{\pgfqpoint{1.637850in}{1.270925in}}%
\pgfpathclose%
\pgfusepath{fill}%
\end{pgfscope}%
\begin{pgfscope}%
\pgfpathrectangle{\pgfqpoint{0.360415in}{0.345370in}}{\pgfqpoint{1.550000in}{1.347500in}}%
\pgfusepath{clip}%
\pgfsetbuttcap%
\pgfsetroundjoin%
\definecolor{currentfill}{rgb}{0.534952,0.031217,0.650165}%
\pgfsetfillcolor{currentfill}%
\pgfsetlinewidth{0.000000pt}%
\definecolor{currentstroke}{rgb}{0.000000,0.000000,0.000000}%
\pgfsetstrokecolor{currentstroke}%
\pgfsetdash{}{0pt}%
\pgfpathmoveto{\pgfqpoint{0.485668in}{0.345370in}}%
\pgfpathlineto{\pgfqpoint{0.501324in}{0.345370in}}%
\pgfpathlineto{\pgfqpoint{0.516981in}{0.345370in}}%
\pgfpathlineto{\pgfqpoint{0.530374in}{0.345370in}}%
\pgfpathlineto{\pgfqpoint{0.528076in}{0.358981in}}%
\pgfpathlineto{\pgfqpoint{0.521401in}{0.372592in}}%
\pgfpathlineto{\pgfqpoint{0.516981in}{0.378272in}}%
\pgfpathlineto{\pgfqpoint{0.511336in}{0.386203in}}%
\pgfpathlineto{\pgfqpoint{0.501324in}{0.397043in}}%
\pgfpathlineto{\pgfqpoint{0.498908in}{0.399814in}}%
\pgfpathlineto{\pgfqpoint{0.485668in}{0.412763in}}%
\pgfpathlineto{\pgfqpoint{0.485008in}{0.413425in}}%
\pgfpathlineto{\pgfqpoint{0.470129in}{0.427036in}}%
\pgfpathlineto{\pgfqpoint{0.470011in}{0.427139in}}%
\pgfpathlineto{\pgfqpoint{0.454472in}{0.440648in}}%
\pgfpathlineto{\pgfqpoint{0.454354in}{0.440750in}}%
\pgfpathlineto{\pgfqpoint{0.438698in}{0.453686in}}%
\pgfpathlineto{\pgfqpoint{0.437936in}{0.454259in}}%
\pgfpathlineto{\pgfqpoint{0.423041in}{0.465770in}}%
\pgfpathlineto{\pgfqpoint{0.419853in}{0.467870in}}%
\pgfpathlineto{\pgfqpoint{0.407385in}{0.476574in}}%
\pgfpathlineto{\pgfqpoint{0.398262in}{0.481481in}}%
\pgfpathlineto{\pgfqpoint{0.391728in}{0.485323in}}%
\pgfpathlineto{\pgfqpoint{0.376072in}{0.491127in}}%
\pgfpathlineto{\pgfqpoint{0.360415in}{0.493124in}}%
\pgfpathlineto{\pgfqpoint{0.360415in}{0.481481in}}%
\pgfpathlineto{\pgfqpoint{0.360415in}{0.467870in}}%
\pgfpathlineto{\pgfqpoint{0.360415in}{0.454259in}}%
\pgfpathlineto{\pgfqpoint{0.360415in}{0.447285in}}%
\pgfpathlineto{\pgfqpoint{0.376072in}{0.445674in}}%
\pgfpathlineto{\pgfqpoint{0.391728in}{0.440993in}}%
\pgfpathlineto{\pgfqpoint{0.392450in}{0.440648in}}%
\pgfpathlineto{\pgfqpoint{0.407385in}{0.433331in}}%
\pgfpathlineto{\pgfqpoint{0.417324in}{0.427036in}}%
\pgfpathlineto{\pgfqpoint{0.423041in}{0.423204in}}%
\pgfpathlineto{\pgfqpoint{0.435469in}{0.413425in}}%
\pgfpathlineto{\pgfqpoint{0.438698in}{0.410618in}}%
\pgfpathlineto{\pgfqpoint{0.449946in}{0.399814in}}%
\pgfpathlineto{\pgfqpoint{0.454354in}{0.394843in}}%
\pgfpathlineto{\pgfqpoint{0.461595in}{0.386203in}}%
\pgfpathlineto{\pgfqpoint{0.470011in}{0.373220in}}%
\pgfpathlineto{\pgfqpoint{0.470408in}{0.372592in}}%
\pgfpathlineto{\pgfqpoint{0.475793in}{0.358981in}}%
\pgfpathlineto{\pgfqpoint{0.477646in}{0.345370in}}%
\pgfpathlineto{\pgfqpoint{0.485668in}{0.345370in}}%
\pgfpathclose%
\pgfpathmoveto{\pgfqpoint{0.720516in}{0.345370in}}%
\pgfpathlineto{\pgfqpoint{0.736173in}{0.345370in}}%
\pgfpathlineto{\pgfqpoint{0.751829in}{0.345370in}}%
\pgfpathlineto{\pgfqpoint{0.759850in}{0.345370in}}%
\pgfpathlineto{\pgfqpoint{0.761704in}{0.358981in}}%
\pgfpathlineto{\pgfqpoint{0.767088in}{0.372592in}}%
\pgfpathlineto{\pgfqpoint{0.767486in}{0.373220in}}%
\pgfpathlineto{\pgfqpoint{0.775902in}{0.386203in}}%
\pgfpathlineto{\pgfqpoint{0.783142in}{0.394843in}}%
\pgfpathlineto{\pgfqpoint{0.787550in}{0.399814in}}%
\pgfpathlineto{\pgfqpoint{0.798799in}{0.410618in}}%
\pgfpathlineto{\pgfqpoint{0.802028in}{0.413425in}}%
\pgfpathlineto{\pgfqpoint{0.814455in}{0.423204in}}%
\pgfpathlineto{\pgfqpoint{0.820173in}{0.427036in}}%
\pgfpathlineto{\pgfqpoint{0.830112in}{0.433331in}}%
\pgfpathlineto{\pgfqpoint{0.845047in}{0.440648in}}%
\pgfpathlineto{\pgfqpoint{0.845769in}{0.440993in}}%
\pgfpathlineto{\pgfqpoint{0.861425in}{0.445674in}}%
\pgfpathlineto{\pgfqpoint{0.877082in}{0.447285in}}%
\pgfpathlineto{\pgfqpoint{0.892738in}{0.445674in}}%
\pgfpathlineto{\pgfqpoint{0.908395in}{0.440993in}}%
\pgfpathlineto{\pgfqpoint{0.909117in}{0.440648in}}%
\pgfpathlineto{\pgfqpoint{0.924051in}{0.433331in}}%
\pgfpathlineto{\pgfqpoint{0.933990in}{0.427036in}}%
\pgfpathlineto{\pgfqpoint{0.939708in}{0.423204in}}%
\pgfpathlineto{\pgfqpoint{0.952136in}{0.413425in}}%
\pgfpathlineto{\pgfqpoint{0.955364in}{0.410618in}}%
\pgfpathlineto{\pgfqpoint{0.966613in}{0.399814in}}%
\pgfpathlineto{\pgfqpoint{0.971021in}{0.394843in}}%
\pgfpathlineto{\pgfqpoint{0.978262in}{0.386203in}}%
\pgfpathlineto{\pgfqpoint{0.986678in}{0.373220in}}%
\pgfpathlineto{\pgfqpoint{0.987075in}{0.372592in}}%
\pgfpathlineto{\pgfqpoint{0.992459in}{0.358981in}}%
\pgfpathlineto{\pgfqpoint{0.994313in}{0.345370in}}%
\pgfpathlineto{\pgfqpoint{1.002334in}{0.345370in}}%
\pgfpathlineto{\pgfqpoint{1.017991in}{0.345370in}}%
\pgfpathlineto{\pgfqpoint{1.033647in}{0.345370in}}%
\pgfpathlineto{\pgfqpoint{1.047041in}{0.345370in}}%
\pgfpathlineto{\pgfqpoint{1.044743in}{0.358981in}}%
\pgfpathlineto{\pgfqpoint{1.038067in}{0.372592in}}%
\pgfpathlineto{\pgfqpoint{1.033647in}{0.378272in}}%
\pgfpathlineto{\pgfqpoint{1.028003in}{0.386203in}}%
\pgfpathlineto{\pgfqpoint{1.017991in}{0.397043in}}%
\pgfpathlineto{\pgfqpoint{1.015575in}{0.399814in}}%
\pgfpathlineto{\pgfqpoint{1.002334in}{0.412763in}}%
\pgfpathlineto{\pgfqpoint{1.001675in}{0.413425in}}%
\pgfpathlineto{\pgfqpoint{0.986795in}{0.427036in}}%
\pgfpathlineto{\pgfqpoint{0.986678in}{0.427139in}}%
\pgfpathlineto{\pgfqpoint{0.971139in}{0.440648in}}%
\pgfpathlineto{\pgfqpoint{0.971021in}{0.440750in}}%
\pgfpathlineto{\pgfqpoint{0.955364in}{0.453686in}}%
\pgfpathlineto{\pgfqpoint{0.954603in}{0.454259in}}%
\pgfpathlineto{\pgfqpoint{0.939708in}{0.465770in}}%
\pgfpathlineto{\pgfqpoint{0.936520in}{0.467870in}}%
\pgfpathlineto{\pgfqpoint{0.924051in}{0.476574in}}%
\pgfpathlineto{\pgfqpoint{0.914929in}{0.481481in}}%
\pgfpathlineto{\pgfqpoint{0.908395in}{0.485323in}}%
\pgfpathlineto{\pgfqpoint{0.892738in}{0.491127in}}%
\pgfpathlineto{\pgfqpoint{0.877082in}{0.493124in}}%
\pgfpathlineto{\pgfqpoint{0.861425in}{0.491127in}}%
\pgfpathlineto{\pgfqpoint{0.845769in}{0.485323in}}%
\pgfpathlineto{\pgfqpoint{0.839234in}{0.481481in}}%
\pgfpathlineto{\pgfqpoint{0.830112in}{0.476574in}}%
\pgfpathlineto{\pgfqpoint{0.817643in}{0.467870in}}%
\pgfpathlineto{\pgfqpoint{0.814455in}{0.465770in}}%
\pgfpathlineto{\pgfqpoint{0.799561in}{0.454259in}}%
\pgfpathlineto{\pgfqpoint{0.798799in}{0.453686in}}%
\pgfpathlineto{\pgfqpoint{0.783142in}{0.440750in}}%
\pgfpathlineto{\pgfqpoint{0.783024in}{0.440648in}}%
\pgfpathlineto{\pgfqpoint{0.767486in}{0.427139in}}%
\pgfpathlineto{\pgfqpoint{0.767368in}{0.427036in}}%
\pgfpathlineto{\pgfqpoint{0.752488in}{0.413425in}}%
\pgfpathlineto{\pgfqpoint{0.751829in}{0.412763in}}%
\pgfpathlineto{\pgfqpoint{0.738588in}{0.399814in}}%
\pgfpathlineto{\pgfqpoint{0.736173in}{0.397043in}}%
\pgfpathlineto{\pgfqpoint{0.726160in}{0.386203in}}%
\pgfpathlineto{\pgfqpoint{0.720516in}{0.378272in}}%
\pgfpathlineto{\pgfqpoint{0.716096in}{0.372592in}}%
\pgfpathlineto{\pgfqpoint{0.709420in}{0.358981in}}%
\pgfpathlineto{\pgfqpoint{0.707123in}{0.345370in}}%
\pgfpathlineto{\pgfqpoint{0.720516in}{0.345370in}}%
\pgfpathclose%
\pgfpathmoveto{\pgfqpoint{1.237183in}{0.345370in}}%
\pgfpathlineto{\pgfqpoint{1.252839in}{0.345370in}}%
\pgfpathlineto{\pgfqpoint{1.268496in}{0.345370in}}%
\pgfpathlineto{\pgfqpoint{1.276517in}{0.345370in}}%
\pgfpathlineto{\pgfqpoint{1.278371in}{0.358981in}}%
\pgfpathlineto{\pgfqpoint{1.283755in}{0.372592in}}%
\pgfpathlineto{\pgfqpoint{1.284152in}{0.373220in}}%
\pgfpathlineto{\pgfqpoint{1.292568in}{0.386203in}}%
\pgfpathlineto{\pgfqpoint{1.299809in}{0.394843in}}%
\pgfpathlineto{\pgfqpoint{1.304217in}{0.399814in}}%
\pgfpathlineto{\pgfqpoint{1.315466in}{0.410618in}}%
\pgfpathlineto{\pgfqpoint{1.318694in}{0.413425in}}%
\pgfpathlineto{\pgfqpoint{1.331122in}{0.423204in}}%
\pgfpathlineto{\pgfqpoint{1.336840in}{0.427036in}}%
\pgfpathlineto{\pgfqpoint{1.346779in}{0.433331in}}%
\pgfpathlineto{\pgfqpoint{1.361713in}{0.440648in}}%
\pgfpathlineto{\pgfqpoint{1.362435in}{0.440993in}}%
\pgfpathlineto{\pgfqpoint{1.378092in}{0.445674in}}%
\pgfpathlineto{\pgfqpoint{1.393748in}{0.447285in}}%
\pgfpathlineto{\pgfqpoint{1.409405in}{0.445674in}}%
\pgfpathlineto{\pgfqpoint{1.425061in}{0.440993in}}%
\pgfpathlineto{\pgfqpoint{1.425783in}{0.440648in}}%
\pgfpathlineto{\pgfqpoint{1.440718in}{0.433331in}}%
\pgfpathlineto{\pgfqpoint{1.450657in}{0.427036in}}%
\pgfpathlineto{\pgfqpoint{1.456375in}{0.423204in}}%
\pgfpathlineto{\pgfqpoint{1.468802in}{0.413425in}}%
\pgfpathlineto{\pgfqpoint{1.472031in}{0.410618in}}%
\pgfpathlineto{\pgfqpoint{1.483280in}{0.399814in}}%
\pgfpathlineto{\pgfqpoint{1.487688in}{0.394843in}}%
\pgfpathlineto{\pgfqpoint{1.494928in}{0.386203in}}%
\pgfpathlineto{\pgfqpoint{1.503344in}{0.373220in}}%
\pgfpathlineto{\pgfqpoint{1.503742in}{0.372592in}}%
\pgfpathlineto{\pgfqpoint{1.509126in}{0.358981in}}%
\pgfpathlineto{\pgfqpoint{1.510980in}{0.345370in}}%
\pgfpathlineto{\pgfqpoint{1.519001in}{0.345370in}}%
\pgfpathlineto{\pgfqpoint{1.534657in}{0.345370in}}%
\pgfpathlineto{\pgfqpoint{1.550314in}{0.345370in}}%
\pgfpathlineto{\pgfqpoint{1.563707in}{0.345370in}}%
\pgfpathlineto{\pgfqpoint{1.561410in}{0.358981in}}%
\pgfpathlineto{\pgfqpoint{1.554734in}{0.372592in}}%
\pgfpathlineto{\pgfqpoint{1.550314in}{0.378272in}}%
\pgfpathlineto{\pgfqpoint{1.544670in}{0.386203in}}%
\pgfpathlineto{\pgfqpoint{1.534657in}{0.397043in}}%
\pgfpathlineto{\pgfqpoint{1.532242in}{0.399814in}}%
\pgfpathlineto{\pgfqpoint{1.519001in}{0.412763in}}%
\pgfpathlineto{\pgfqpoint{1.518342in}{0.413425in}}%
\pgfpathlineto{\pgfqpoint{1.503462in}{0.427036in}}%
\pgfpathlineto{\pgfqpoint{1.503344in}{0.427139in}}%
\pgfpathlineto{\pgfqpoint{1.487806in}{0.440648in}}%
\pgfpathlineto{\pgfqpoint{1.487688in}{0.440750in}}%
\pgfpathlineto{\pgfqpoint{1.472031in}{0.453686in}}%
\pgfpathlineto{\pgfqpoint{1.471269in}{0.454259in}}%
\pgfpathlineto{\pgfqpoint{1.456375in}{0.465770in}}%
\pgfpathlineto{\pgfqpoint{1.453187in}{0.467870in}}%
\pgfpathlineto{\pgfqpoint{1.440718in}{0.476574in}}%
\pgfpathlineto{\pgfqpoint{1.431596in}{0.481481in}}%
\pgfpathlineto{\pgfqpoint{1.425061in}{0.485323in}}%
\pgfpathlineto{\pgfqpoint{1.409405in}{0.491127in}}%
\pgfpathlineto{\pgfqpoint{1.393748in}{0.493124in}}%
\pgfpathlineto{\pgfqpoint{1.378092in}{0.491127in}}%
\pgfpathlineto{\pgfqpoint{1.362435in}{0.485323in}}%
\pgfpathlineto{\pgfqpoint{1.355901in}{0.481481in}}%
\pgfpathlineto{\pgfqpoint{1.346779in}{0.476574in}}%
\pgfpathlineto{\pgfqpoint{1.334310in}{0.467870in}}%
\pgfpathlineto{\pgfqpoint{1.331122in}{0.465770in}}%
\pgfpathlineto{\pgfqpoint{1.316227in}{0.454259in}}%
\pgfpathlineto{\pgfqpoint{1.315466in}{0.453686in}}%
\pgfpathlineto{\pgfqpoint{1.299809in}{0.440750in}}%
\pgfpathlineto{\pgfqpoint{1.299691in}{0.440648in}}%
\pgfpathlineto{\pgfqpoint{1.284152in}{0.427139in}}%
\pgfpathlineto{\pgfqpoint{1.284035in}{0.427036in}}%
\pgfpathlineto{\pgfqpoint{1.269155in}{0.413425in}}%
\pgfpathlineto{\pgfqpoint{1.268496in}{0.412763in}}%
\pgfpathlineto{\pgfqpoint{1.255255in}{0.399814in}}%
\pgfpathlineto{\pgfqpoint{1.252839in}{0.397043in}}%
\pgfpathlineto{\pgfqpoint{1.242827in}{0.386203in}}%
\pgfpathlineto{\pgfqpoint{1.237183in}{0.378272in}}%
\pgfpathlineto{\pgfqpoint{1.232763in}{0.372592in}}%
\pgfpathlineto{\pgfqpoint{1.226087in}{0.358981in}}%
\pgfpathlineto{\pgfqpoint{1.223789in}{0.345370in}}%
\pgfpathlineto{\pgfqpoint{1.237183in}{0.345370in}}%
\pgfpathclose%
\pgfpathmoveto{\pgfqpoint{1.753849in}{0.345370in}}%
\pgfpathlineto{\pgfqpoint{1.769506in}{0.345370in}}%
\pgfpathlineto{\pgfqpoint{1.785162in}{0.345370in}}%
\pgfpathlineto{\pgfqpoint{1.793184in}{0.345370in}}%
\pgfpathlineto{\pgfqpoint{1.795037in}{0.358981in}}%
\pgfpathlineto{\pgfqpoint{1.800422in}{0.372592in}}%
\pgfpathlineto{\pgfqpoint{1.800819in}{0.373220in}}%
\pgfpathlineto{\pgfqpoint{1.809235in}{0.386203in}}%
\pgfpathlineto{\pgfqpoint{1.816476in}{0.394843in}}%
\pgfpathlineto{\pgfqpoint{1.820884in}{0.399814in}}%
\pgfpathlineto{\pgfqpoint{1.832132in}{0.410618in}}%
\pgfpathlineto{\pgfqpoint{1.835361in}{0.413425in}}%
\pgfpathlineto{\pgfqpoint{1.847789in}{0.423204in}}%
\pgfpathlineto{\pgfqpoint{1.853506in}{0.427036in}}%
\pgfpathlineto{\pgfqpoint{1.863445in}{0.433331in}}%
\pgfpathlineto{\pgfqpoint{1.878380in}{0.440648in}}%
\pgfpathlineto{\pgfqpoint{1.879102in}{0.440993in}}%
\pgfpathlineto{\pgfqpoint{1.894758in}{0.445674in}}%
\pgfpathlineto{\pgfqpoint{1.910415in}{0.447285in}}%
\pgfpathlineto{\pgfqpoint{1.910415in}{0.454259in}}%
\pgfpathlineto{\pgfqpoint{1.910415in}{0.467870in}}%
\pgfpathlineto{\pgfqpoint{1.910415in}{0.481481in}}%
\pgfpathlineto{\pgfqpoint{1.910415in}{0.493124in}}%
\pgfpathlineto{\pgfqpoint{1.894758in}{0.491127in}}%
\pgfpathlineto{\pgfqpoint{1.879102in}{0.485323in}}%
\pgfpathlineto{\pgfqpoint{1.872568in}{0.481481in}}%
\pgfpathlineto{\pgfqpoint{1.863445in}{0.476574in}}%
\pgfpathlineto{\pgfqpoint{1.850977in}{0.467870in}}%
\pgfpathlineto{\pgfqpoint{1.847789in}{0.465770in}}%
\pgfpathlineto{\pgfqpoint{1.832894in}{0.454259in}}%
\pgfpathlineto{\pgfqpoint{1.832132in}{0.453686in}}%
\pgfpathlineto{\pgfqpoint{1.816476in}{0.440750in}}%
\pgfpathlineto{\pgfqpoint{1.816358in}{0.440648in}}%
\pgfpathlineto{\pgfqpoint{1.800819in}{0.427139in}}%
\pgfpathlineto{\pgfqpoint{1.800701in}{0.427036in}}%
\pgfpathlineto{\pgfqpoint{1.785822in}{0.413425in}}%
\pgfpathlineto{\pgfqpoint{1.785162in}{0.412763in}}%
\pgfpathlineto{\pgfqpoint{1.771922in}{0.399814in}}%
\pgfpathlineto{\pgfqpoint{1.769506in}{0.397043in}}%
\pgfpathlineto{\pgfqpoint{1.759494in}{0.386203in}}%
\pgfpathlineto{\pgfqpoint{1.753849in}{0.378272in}}%
\pgfpathlineto{\pgfqpoint{1.749429in}{0.372592in}}%
\pgfpathlineto{\pgfqpoint{1.742754in}{0.358981in}}%
\pgfpathlineto{\pgfqpoint{1.740456in}{0.345370in}}%
\pgfpathlineto{\pgfqpoint{1.753849in}{0.345370in}}%
\pgfpathclose%
\pgfpathmoveto{\pgfqpoint{0.376072in}{0.648779in}}%
\pgfpathlineto{\pgfqpoint{0.391728in}{0.654583in}}%
\pgfpathlineto{\pgfqpoint{0.398262in}{0.658425in}}%
\pgfpathlineto{\pgfqpoint{0.407385in}{0.663332in}}%
\pgfpathlineto{\pgfqpoint{0.419853in}{0.672036in}}%
\pgfpathlineto{\pgfqpoint{0.423041in}{0.674136in}}%
\pgfpathlineto{\pgfqpoint{0.437936in}{0.685648in}}%
\pgfpathlineto{\pgfqpoint{0.438698in}{0.686220in}}%
\pgfpathlineto{\pgfqpoint{0.454354in}{0.699156in}}%
\pgfpathlineto{\pgfqpoint{0.454472in}{0.699259in}}%
\pgfpathlineto{\pgfqpoint{0.470011in}{0.712767in}}%
\pgfpathlineto{\pgfqpoint{0.470129in}{0.712870in}}%
\pgfpathlineto{\pgfqpoint{0.485008in}{0.726481in}}%
\pgfpathlineto{\pgfqpoint{0.485668in}{0.727143in}}%
\pgfpathlineto{\pgfqpoint{0.498908in}{0.740092in}}%
\pgfpathlineto{\pgfqpoint{0.501324in}{0.742863in}}%
\pgfpathlineto{\pgfqpoint{0.511336in}{0.753703in}}%
\pgfpathlineto{\pgfqpoint{0.516981in}{0.761634in}}%
\pgfpathlineto{\pgfqpoint{0.521401in}{0.767314in}}%
\pgfpathlineto{\pgfqpoint{0.528076in}{0.780925in}}%
\pgfpathlineto{\pgfqpoint{0.530374in}{0.794536in}}%
\pgfpathlineto{\pgfqpoint{0.528076in}{0.808148in}}%
\pgfpathlineto{\pgfqpoint{0.521401in}{0.821759in}}%
\pgfpathlineto{\pgfqpoint{0.516981in}{0.827439in}}%
\pgfpathlineto{\pgfqpoint{0.511336in}{0.835370in}}%
\pgfpathlineto{\pgfqpoint{0.501324in}{0.846209in}}%
\pgfpathlineto{\pgfqpoint{0.498908in}{0.848981in}}%
\pgfpathlineto{\pgfqpoint{0.485668in}{0.861930in}}%
\pgfpathlineto{\pgfqpoint{0.485008in}{0.862592in}}%
\pgfpathlineto{\pgfqpoint{0.470129in}{0.876203in}}%
\pgfpathlineto{\pgfqpoint{0.470011in}{0.876306in}}%
\pgfpathlineto{\pgfqpoint{0.454472in}{0.889814in}}%
\pgfpathlineto{\pgfqpoint{0.454354in}{0.889916in}}%
\pgfpathlineto{\pgfqpoint{0.438698in}{0.902852in}}%
\pgfpathlineto{\pgfqpoint{0.437936in}{0.903425in}}%
\pgfpathlineto{\pgfqpoint{0.423041in}{0.914936in}}%
\pgfpathlineto{\pgfqpoint{0.419853in}{0.917036in}}%
\pgfpathlineto{\pgfqpoint{0.407385in}{0.925741in}}%
\pgfpathlineto{\pgfqpoint{0.398262in}{0.930648in}}%
\pgfpathlineto{\pgfqpoint{0.391728in}{0.934490in}}%
\pgfpathlineto{\pgfqpoint{0.376072in}{0.940293in}}%
\pgfpathlineto{\pgfqpoint{0.360415in}{0.942291in}}%
\pgfpathlineto{\pgfqpoint{0.360415in}{0.930648in}}%
\pgfpathlineto{\pgfqpoint{0.360415in}{0.917036in}}%
\pgfpathlineto{\pgfqpoint{0.360415in}{0.903425in}}%
\pgfpathlineto{\pgfqpoint{0.360415in}{0.896452in}}%
\pgfpathlineto{\pgfqpoint{0.376072in}{0.894841in}}%
\pgfpathlineto{\pgfqpoint{0.391728in}{0.890160in}}%
\pgfpathlineto{\pgfqpoint{0.392450in}{0.889814in}}%
\pgfpathlineto{\pgfqpoint{0.407385in}{0.882498in}}%
\pgfpathlineto{\pgfqpoint{0.417324in}{0.876203in}}%
\pgfpathlineto{\pgfqpoint{0.423041in}{0.872371in}}%
\pgfpathlineto{\pgfqpoint{0.435469in}{0.862592in}}%
\pgfpathlineto{\pgfqpoint{0.438698in}{0.859785in}}%
\pgfpathlineto{\pgfqpoint{0.449946in}{0.848981in}}%
\pgfpathlineto{\pgfqpoint{0.454354in}{0.844010in}}%
\pgfpathlineto{\pgfqpoint{0.461595in}{0.835370in}}%
\pgfpathlineto{\pgfqpoint{0.470011in}{0.822386in}}%
\pgfpathlineto{\pgfqpoint{0.470408in}{0.821759in}}%
\pgfpathlineto{\pgfqpoint{0.475793in}{0.808148in}}%
\pgfpathlineto{\pgfqpoint{0.477646in}{0.794536in}}%
\pgfpathlineto{\pgfqpoint{0.475793in}{0.780925in}}%
\pgfpathlineto{\pgfqpoint{0.470408in}{0.767314in}}%
\pgfpathlineto{\pgfqpoint{0.470011in}{0.766687in}}%
\pgfpathlineto{\pgfqpoint{0.461595in}{0.753703in}}%
\pgfpathlineto{\pgfqpoint{0.454354in}{0.745063in}}%
\pgfpathlineto{\pgfqpoint{0.449946in}{0.740092in}}%
\pgfpathlineto{\pgfqpoint{0.438698in}{0.729288in}}%
\pgfpathlineto{\pgfqpoint{0.435469in}{0.726481in}}%
\pgfpathlineto{\pgfqpoint{0.423041in}{0.716702in}}%
\pgfpathlineto{\pgfqpoint{0.417324in}{0.712870in}}%
\pgfpathlineto{\pgfqpoint{0.407385in}{0.706575in}}%
\pgfpathlineto{\pgfqpoint{0.392450in}{0.699259in}}%
\pgfpathlineto{\pgfqpoint{0.391728in}{0.698913in}}%
\pgfpathlineto{\pgfqpoint{0.376072in}{0.694232in}}%
\pgfpathlineto{\pgfqpoint{0.360415in}{0.692621in}}%
\pgfpathlineto{\pgfqpoint{0.360415in}{0.685648in}}%
\pgfpathlineto{\pgfqpoint{0.360415in}{0.672036in}}%
\pgfpathlineto{\pgfqpoint{0.360415in}{0.658425in}}%
\pgfpathlineto{\pgfqpoint{0.360415in}{0.646782in}}%
\pgfpathlineto{\pgfqpoint{0.376072in}{0.648779in}}%
\pgfpathclose%
\pgfpathmoveto{\pgfqpoint{0.845769in}{0.654583in}}%
\pgfpathlineto{\pgfqpoint{0.861425in}{0.648779in}}%
\pgfpathlineto{\pgfqpoint{0.877082in}{0.646782in}}%
\pgfpathlineto{\pgfqpoint{0.892738in}{0.648779in}}%
\pgfpathlineto{\pgfqpoint{0.908395in}{0.654583in}}%
\pgfpathlineto{\pgfqpoint{0.914929in}{0.658425in}}%
\pgfpathlineto{\pgfqpoint{0.924051in}{0.663332in}}%
\pgfpathlineto{\pgfqpoint{0.936520in}{0.672036in}}%
\pgfpathlineto{\pgfqpoint{0.939708in}{0.674136in}}%
\pgfpathlineto{\pgfqpoint{0.954603in}{0.685648in}}%
\pgfpathlineto{\pgfqpoint{0.955364in}{0.686220in}}%
\pgfpathlineto{\pgfqpoint{0.971021in}{0.699156in}}%
\pgfpathlineto{\pgfqpoint{0.971139in}{0.699259in}}%
\pgfpathlineto{\pgfqpoint{0.986678in}{0.712767in}}%
\pgfpathlineto{\pgfqpoint{0.986795in}{0.712870in}}%
\pgfpathlineto{\pgfqpoint{1.001675in}{0.726481in}}%
\pgfpathlineto{\pgfqpoint{1.002334in}{0.727143in}}%
\pgfpathlineto{\pgfqpoint{1.015575in}{0.740092in}}%
\pgfpathlineto{\pgfqpoint{1.017991in}{0.742863in}}%
\pgfpathlineto{\pgfqpoint{1.028003in}{0.753703in}}%
\pgfpathlineto{\pgfqpoint{1.033647in}{0.761634in}}%
\pgfpathlineto{\pgfqpoint{1.038067in}{0.767314in}}%
\pgfpathlineto{\pgfqpoint{1.044743in}{0.780925in}}%
\pgfpathlineto{\pgfqpoint{1.047041in}{0.794536in}}%
\pgfpathlineto{\pgfqpoint{1.044743in}{0.808148in}}%
\pgfpathlineto{\pgfqpoint{1.038067in}{0.821759in}}%
\pgfpathlineto{\pgfqpoint{1.033647in}{0.827439in}}%
\pgfpathlineto{\pgfqpoint{1.028003in}{0.835370in}}%
\pgfpathlineto{\pgfqpoint{1.017991in}{0.846209in}}%
\pgfpathlineto{\pgfqpoint{1.015575in}{0.848981in}}%
\pgfpathlineto{\pgfqpoint{1.002334in}{0.861930in}}%
\pgfpathlineto{\pgfqpoint{1.001675in}{0.862592in}}%
\pgfpathlineto{\pgfqpoint{0.986795in}{0.876203in}}%
\pgfpathlineto{\pgfqpoint{0.986678in}{0.876306in}}%
\pgfpathlineto{\pgfqpoint{0.971139in}{0.889814in}}%
\pgfpathlineto{\pgfqpoint{0.971021in}{0.889916in}}%
\pgfpathlineto{\pgfqpoint{0.955364in}{0.902852in}}%
\pgfpathlineto{\pgfqpoint{0.954603in}{0.903425in}}%
\pgfpathlineto{\pgfqpoint{0.939708in}{0.914936in}}%
\pgfpathlineto{\pgfqpoint{0.936520in}{0.917036in}}%
\pgfpathlineto{\pgfqpoint{0.924051in}{0.925741in}}%
\pgfpathlineto{\pgfqpoint{0.914929in}{0.930648in}}%
\pgfpathlineto{\pgfqpoint{0.908395in}{0.934490in}}%
\pgfpathlineto{\pgfqpoint{0.892738in}{0.940293in}}%
\pgfpathlineto{\pgfqpoint{0.877082in}{0.942291in}}%
\pgfpathlineto{\pgfqpoint{0.861425in}{0.940293in}}%
\pgfpathlineto{\pgfqpoint{0.845769in}{0.934490in}}%
\pgfpathlineto{\pgfqpoint{0.839234in}{0.930648in}}%
\pgfpathlineto{\pgfqpoint{0.830112in}{0.925741in}}%
\pgfpathlineto{\pgfqpoint{0.817643in}{0.917036in}}%
\pgfpathlineto{\pgfqpoint{0.814455in}{0.914936in}}%
\pgfpathlineto{\pgfqpoint{0.799561in}{0.903425in}}%
\pgfpathlineto{\pgfqpoint{0.798799in}{0.902852in}}%
\pgfpathlineto{\pgfqpoint{0.783142in}{0.889916in}}%
\pgfpathlineto{\pgfqpoint{0.783024in}{0.889814in}}%
\pgfpathlineto{\pgfqpoint{0.767486in}{0.876306in}}%
\pgfpathlineto{\pgfqpoint{0.767368in}{0.876203in}}%
\pgfpathlineto{\pgfqpoint{0.752488in}{0.862592in}}%
\pgfpathlineto{\pgfqpoint{0.751829in}{0.861930in}}%
\pgfpathlineto{\pgfqpoint{0.738588in}{0.848981in}}%
\pgfpathlineto{\pgfqpoint{0.736173in}{0.846209in}}%
\pgfpathlineto{\pgfqpoint{0.726160in}{0.835370in}}%
\pgfpathlineto{\pgfqpoint{0.720516in}{0.827439in}}%
\pgfpathlineto{\pgfqpoint{0.716096in}{0.821759in}}%
\pgfpathlineto{\pgfqpoint{0.709420in}{0.808148in}}%
\pgfpathlineto{\pgfqpoint{0.707123in}{0.794536in}}%
\pgfpathlineto{\pgfqpoint{0.709420in}{0.780925in}}%
\pgfpathlineto{\pgfqpoint{0.716096in}{0.767314in}}%
\pgfpathlineto{\pgfqpoint{0.720516in}{0.761634in}}%
\pgfpathlineto{\pgfqpoint{0.726160in}{0.753703in}}%
\pgfpathlineto{\pgfqpoint{0.736173in}{0.742863in}}%
\pgfpathlineto{\pgfqpoint{0.738588in}{0.740092in}}%
\pgfpathlineto{\pgfqpoint{0.751829in}{0.727143in}}%
\pgfpathlineto{\pgfqpoint{0.752488in}{0.726481in}}%
\pgfpathlineto{\pgfqpoint{0.767368in}{0.712870in}}%
\pgfpathlineto{\pgfqpoint{0.767486in}{0.712767in}}%
\pgfpathlineto{\pgfqpoint{0.783024in}{0.699259in}}%
\pgfpathlineto{\pgfqpoint{0.783142in}{0.699156in}}%
\pgfpathlineto{\pgfqpoint{0.798799in}{0.686220in}}%
\pgfpathlineto{\pgfqpoint{0.799561in}{0.685648in}}%
\pgfpathlineto{\pgfqpoint{0.814455in}{0.674136in}}%
\pgfpathlineto{\pgfqpoint{0.817643in}{0.672036in}}%
\pgfpathlineto{\pgfqpoint{0.830112in}{0.663332in}}%
\pgfpathlineto{\pgfqpoint{0.839234in}{0.658425in}}%
\pgfpathlineto{\pgfqpoint{0.845769in}{0.654583in}}%
\pgfpathclose%
\pgfpathmoveto{\pgfqpoint{0.845047in}{0.699259in}}%
\pgfpathlineto{\pgfqpoint{0.830112in}{0.706575in}}%
\pgfpathlineto{\pgfqpoint{0.820173in}{0.712870in}}%
\pgfpathlineto{\pgfqpoint{0.814455in}{0.716702in}}%
\pgfpathlineto{\pgfqpoint{0.802028in}{0.726481in}}%
\pgfpathlineto{\pgfqpoint{0.798799in}{0.729288in}}%
\pgfpathlineto{\pgfqpoint{0.787550in}{0.740092in}}%
\pgfpathlineto{\pgfqpoint{0.783142in}{0.745063in}}%
\pgfpathlineto{\pgfqpoint{0.775902in}{0.753703in}}%
\pgfpathlineto{\pgfqpoint{0.767486in}{0.766687in}}%
\pgfpathlineto{\pgfqpoint{0.767088in}{0.767314in}}%
\pgfpathlineto{\pgfqpoint{0.761704in}{0.780925in}}%
\pgfpathlineto{\pgfqpoint{0.759850in}{0.794536in}}%
\pgfpathlineto{\pgfqpoint{0.761704in}{0.808148in}}%
\pgfpathlineto{\pgfqpoint{0.767088in}{0.821759in}}%
\pgfpathlineto{\pgfqpoint{0.767486in}{0.822386in}}%
\pgfpathlineto{\pgfqpoint{0.775902in}{0.835370in}}%
\pgfpathlineto{\pgfqpoint{0.783142in}{0.844010in}}%
\pgfpathlineto{\pgfqpoint{0.787550in}{0.848981in}}%
\pgfpathlineto{\pgfqpoint{0.798799in}{0.859785in}}%
\pgfpathlineto{\pgfqpoint{0.802028in}{0.862592in}}%
\pgfpathlineto{\pgfqpoint{0.814455in}{0.872371in}}%
\pgfpathlineto{\pgfqpoint{0.820173in}{0.876203in}}%
\pgfpathlineto{\pgfqpoint{0.830112in}{0.882498in}}%
\pgfpathlineto{\pgfqpoint{0.845047in}{0.889814in}}%
\pgfpathlineto{\pgfqpoint{0.845769in}{0.890160in}}%
\pgfpathlineto{\pgfqpoint{0.861425in}{0.894841in}}%
\pgfpathlineto{\pgfqpoint{0.877082in}{0.896452in}}%
\pgfpathlineto{\pgfqpoint{0.892738in}{0.894841in}}%
\pgfpathlineto{\pgfqpoint{0.908395in}{0.890160in}}%
\pgfpathlineto{\pgfqpoint{0.909117in}{0.889814in}}%
\pgfpathlineto{\pgfqpoint{0.924051in}{0.882498in}}%
\pgfpathlineto{\pgfqpoint{0.933990in}{0.876203in}}%
\pgfpathlineto{\pgfqpoint{0.939708in}{0.872371in}}%
\pgfpathlineto{\pgfqpoint{0.952136in}{0.862592in}}%
\pgfpathlineto{\pgfqpoint{0.955364in}{0.859785in}}%
\pgfpathlineto{\pgfqpoint{0.966613in}{0.848981in}}%
\pgfpathlineto{\pgfqpoint{0.971021in}{0.844010in}}%
\pgfpathlineto{\pgfqpoint{0.978262in}{0.835370in}}%
\pgfpathlineto{\pgfqpoint{0.986678in}{0.822386in}}%
\pgfpathlineto{\pgfqpoint{0.987075in}{0.821759in}}%
\pgfpathlineto{\pgfqpoint{0.992459in}{0.808148in}}%
\pgfpathlineto{\pgfqpoint{0.994313in}{0.794536in}}%
\pgfpathlineto{\pgfqpoint{0.992459in}{0.780925in}}%
\pgfpathlineto{\pgfqpoint{0.987075in}{0.767314in}}%
\pgfpathlineto{\pgfqpoint{0.986678in}{0.766687in}}%
\pgfpathlineto{\pgfqpoint{0.978262in}{0.753703in}}%
\pgfpathlineto{\pgfqpoint{0.971021in}{0.745063in}}%
\pgfpathlineto{\pgfqpoint{0.966613in}{0.740092in}}%
\pgfpathlineto{\pgfqpoint{0.955364in}{0.729288in}}%
\pgfpathlineto{\pgfqpoint{0.952136in}{0.726481in}}%
\pgfpathlineto{\pgfqpoint{0.939708in}{0.716702in}}%
\pgfpathlineto{\pgfqpoint{0.933990in}{0.712870in}}%
\pgfpathlineto{\pgfqpoint{0.924051in}{0.706575in}}%
\pgfpathlineto{\pgfqpoint{0.909117in}{0.699259in}}%
\pgfpathlineto{\pgfqpoint{0.908395in}{0.698913in}}%
\pgfpathlineto{\pgfqpoint{0.892738in}{0.694232in}}%
\pgfpathlineto{\pgfqpoint{0.877082in}{0.692621in}}%
\pgfpathlineto{\pgfqpoint{0.861425in}{0.694232in}}%
\pgfpathlineto{\pgfqpoint{0.845769in}{0.698913in}}%
\pgfpathlineto{\pgfqpoint{0.845047in}{0.699259in}}%
\pgfpathclose%
\pgfpathmoveto{\pgfqpoint{1.362435in}{0.654583in}}%
\pgfpathlineto{\pgfqpoint{1.378092in}{0.648779in}}%
\pgfpathlineto{\pgfqpoint{1.393748in}{0.646782in}}%
\pgfpathlineto{\pgfqpoint{1.409405in}{0.648779in}}%
\pgfpathlineto{\pgfqpoint{1.425061in}{0.654583in}}%
\pgfpathlineto{\pgfqpoint{1.431596in}{0.658425in}}%
\pgfpathlineto{\pgfqpoint{1.440718in}{0.663332in}}%
\pgfpathlineto{\pgfqpoint{1.453187in}{0.672036in}}%
\pgfpathlineto{\pgfqpoint{1.456375in}{0.674136in}}%
\pgfpathlineto{\pgfqpoint{1.471269in}{0.685648in}}%
\pgfpathlineto{\pgfqpoint{1.472031in}{0.686220in}}%
\pgfpathlineto{\pgfqpoint{1.487688in}{0.699156in}}%
\pgfpathlineto{\pgfqpoint{1.487806in}{0.699259in}}%
\pgfpathlineto{\pgfqpoint{1.503344in}{0.712767in}}%
\pgfpathlineto{\pgfqpoint{1.503462in}{0.712870in}}%
\pgfpathlineto{\pgfqpoint{1.518342in}{0.726481in}}%
\pgfpathlineto{\pgfqpoint{1.519001in}{0.727143in}}%
\pgfpathlineto{\pgfqpoint{1.532242in}{0.740092in}}%
\pgfpathlineto{\pgfqpoint{1.534657in}{0.742863in}}%
\pgfpathlineto{\pgfqpoint{1.544670in}{0.753703in}}%
\pgfpathlineto{\pgfqpoint{1.550314in}{0.761634in}}%
\pgfpathlineto{\pgfqpoint{1.554734in}{0.767314in}}%
\pgfpathlineto{\pgfqpoint{1.561410in}{0.780925in}}%
\pgfpathlineto{\pgfqpoint{1.563707in}{0.794536in}}%
\pgfpathlineto{\pgfqpoint{1.561410in}{0.808148in}}%
\pgfpathlineto{\pgfqpoint{1.554734in}{0.821759in}}%
\pgfpathlineto{\pgfqpoint{1.550314in}{0.827439in}}%
\pgfpathlineto{\pgfqpoint{1.544670in}{0.835370in}}%
\pgfpathlineto{\pgfqpoint{1.534657in}{0.846209in}}%
\pgfpathlineto{\pgfqpoint{1.532242in}{0.848981in}}%
\pgfpathlineto{\pgfqpoint{1.519001in}{0.861930in}}%
\pgfpathlineto{\pgfqpoint{1.518342in}{0.862592in}}%
\pgfpathlineto{\pgfqpoint{1.503462in}{0.876203in}}%
\pgfpathlineto{\pgfqpoint{1.503344in}{0.876306in}}%
\pgfpathlineto{\pgfqpoint{1.487806in}{0.889814in}}%
\pgfpathlineto{\pgfqpoint{1.487688in}{0.889916in}}%
\pgfpathlineto{\pgfqpoint{1.472031in}{0.902852in}}%
\pgfpathlineto{\pgfqpoint{1.471269in}{0.903425in}}%
\pgfpathlineto{\pgfqpoint{1.456375in}{0.914936in}}%
\pgfpathlineto{\pgfqpoint{1.453187in}{0.917036in}}%
\pgfpathlineto{\pgfqpoint{1.440718in}{0.925741in}}%
\pgfpathlineto{\pgfqpoint{1.431596in}{0.930648in}}%
\pgfpathlineto{\pgfqpoint{1.425061in}{0.934490in}}%
\pgfpathlineto{\pgfqpoint{1.409405in}{0.940293in}}%
\pgfpathlineto{\pgfqpoint{1.393748in}{0.942291in}}%
\pgfpathlineto{\pgfqpoint{1.378092in}{0.940293in}}%
\pgfpathlineto{\pgfqpoint{1.362435in}{0.934490in}}%
\pgfpathlineto{\pgfqpoint{1.355901in}{0.930648in}}%
\pgfpathlineto{\pgfqpoint{1.346779in}{0.925741in}}%
\pgfpathlineto{\pgfqpoint{1.334310in}{0.917036in}}%
\pgfpathlineto{\pgfqpoint{1.331122in}{0.914936in}}%
\pgfpathlineto{\pgfqpoint{1.316227in}{0.903425in}}%
\pgfpathlineto{\pgfqpoint{1.315466in}{0.902852in}}%
\pgfpathlineto{\pgfqpoint{1.299809in}{0.889916in}}%
\pgfpathlineto{\pgfqpoint{1.299691in}{0.889814in}}%
\pgfpathlineto{\pgfqpoint{1.284152in}{0.876306in}}%
\pgfpathlineto{\pgfqpoint{1.284035in}{0.876203in}}%
\pgfpathlineto{\pgfqpoint{1.269155in}{0.862592in}}%
\pgfpathlineto{\pgfqpoint{1.268496in}{0.861930in}}%
\pgfpathlineto{\pgfqpoint{1.255255in}{0.848981in}}%
\pgfpathlineto{\pgfqpoint{1.252839in}{0.846209in}}%
\pgfpathlineto{\pgfqpoint{1.242827in}{0.835370in}}%
\pgfpathlineto{\pgfqpoint{1.237183in}{0.827439in}}%
\pgfpathlineto{\pgfqpoint{1.232763in}{0.821759in}}%
\pgfpathlineto{\pgfqpoint{1.226087in}{0.808148in}}%
\pgfpathlineto{\pgfqpoint{1.223789in}{0.794536in}}%
\pgfpathlineto{\pgfqpoint{1.226087in}{0.780925in}}%
\pgfpathlineto{\pgfqpoint{1.232763in}{0.767314in}}%
\pgfpathlineto{\pgfqpoint{1.237183in}{0.761634in}}%
\pgfpathlineto{\pgfqpoint{1.242827in}{0.753703in}}%
\pgfpathlineto{\pgfqpoint{1.252839in}{0.742863in}}%
\pgfpathlineto{\pgfqpoint{1.255255in}{0.740092in}}%
\pgfpathlineto{\pgfqpoint{1.268496in}{0.727143in}}%
\pgfpathlineto{\pgfqpoint{1.269155in}{0.726481in}}%
\pgfpathlineto{\pgfqpoint{1.284035in}{0.712870in}}%
\pgfpathlineto{\pgfqpoint{1.284152in}{0.712767in}}%
\pgfpathlineto{\pgfqpoint{1.299691in}{0.699259in}}%
\pgfpathlineto{\pgfqpoint{1.299809in}{0.699156in}}%
\pgfpathlineto{\pgfqpoint{1.315466in}{0.686220in}}%
\pgfpathlineto{\pgfqpoint{1.316227in}{0.685648in}}%
\pgfpathlineto{\pgfqpoint{1.331122in}{0.674136in}}%
\pgfpathlineto{\pgfqpoint{1.334310in}{0.672036in}}%
\pgfpathlineto{\pgfqpoint{1.346779in}{0.663332in}}%
\pgfpathlineto{\pgfqpoint{1.355901in}{0.658425in}}%
\pgfpathlineto{\pgfqpoint{1.362435in}{0.654583in}}%
\pgfpathclose%
\pgfpathmoveto{\pgfqpoint{1.361713in}{0.699259in}}%
\pgfpathlineto{\pgfqpoint{1.346779in}{0.706575in}}%
\pgfpathlineto{\pgfqpoint{1.336840in}{0.712870in}}%
\pgfpathlineto{\pgfqpoint{1.331122in}{0.716702in}}%
\pgfpathlineto{\pgfqpoint{1.318694in}{0.726481in}}%
\pgfpathlineto{\pgfqpoint{1.315466in}{0.729288in}}%
\pgfpathlineto{\pgfqpoint{1.304217in}{0.740092in}}%
\pgfpathlineto{\pgfqpoint{1.299809in}{0.745063in}}%
\pgfpathlineto{\pgfqpoint{1.292568in}{0.753703in}}%
\pgfpathlineto{\pgfqpoint{1.284152in}{0.766687in}}%
\pgfpathlineto{\pgfqpoint{1.283755in}{0.767314in}}%
\pgfpathlineto{\pgfqpoint{1.278371in}{0.780925in}}%
\pgfpathlineto{\pgfqpoint{1.276517in}{0.794536in}}%
\pgfpathlineto{\pgfqpoint{1.278371in}{0.808148in}}%
\pgfpathlineto{\pgfqpoint{1.283755in}{0.821759in}}%
\pgfpathlineto{\pgfqpoint{1.284152in}{0.822386in}}%
\pgfpathlineto{\pgfqpoint{1.292568in}{0.835370in}}%
\pgfpathlineto{\pgfqpoint{1.299809in}{0.844010in}}%
\pgfpathlineto{\pgfqpoint{1.304217in}{0.848981in}}%
\pgfpathlineto{\pgfqpoint{1.315466in}{0.859785in}}%
\pgfpathlineto{\pgfqpoint{1.318694in}{0.862592in}}%
\pgfpathlineto{\pgfqpoint{1.331122in}{0.872371in}}%
\pgfpathlineto{\pgfqpoint{1.336840in}{0.876203in}}%
\pgfpathlineto{\pgfqpoint{1.346779in}{0.882498in}}%
\pgfpathlineto{\pgfqpoint{1.361713in}{0.889814in}}%
\pgfpathlineto{\pgfqpoint{1.362435in}{0.890160in}}%
\pgfpathlineto{\pgfqpoint{1.378092in}{0.894841in}}%
\pgfpathlineto{\pgfqpoint{1.393748in}{0.896452in}}%
\pgfpathlineto{\pgfqpoint{1.409405in}{0.894841in}}%
\pgfpathlineto{\pgfqpoint{1.425061in}{0.890160in}}%
\pgfpathlineto{\pgfqpoint{1.425783in}{0.889814in}}%
\pgfpathlineto{\pgfqpoint{1.440718in}{0.882498in}}%
\pgfpathlineto{\pgfqpoint{1.450657in}{0.876203in}}%
\pgfpathlineto{\pgfqpoint{1.456375in}{0.872371in}}%
\pgfpathlineto{\pgfqpoint{1.468802in}{0.862592in}}%
\pgfpathlineto{\pgfqpoint{1.472031in}{0.859785in}}%
\pgfpathlineto{\pgfqpoint{1.483280in}{0.848981in}}%
\pgfpathlineto{\pgfqpoint{1.487688in}{0.844010in}}%
\pgfpathlineto{\pgfqpoint{1.494928in}{0.835370in}}%
\pgfpathlineto{\pgfqpoint{1.503344in}{0.822386in}}%
\pgfpathlineto{\pgfqpoint{1.503742in}{0.821759in}}%
\pgfpathlineto{\pgfqpoint{1.509126in}{0.808148in}}%
\pgfpathlineto{\pgfqpoint{1.510980in}{0.794536in}}%
\pgfpathlineto{\pgfqpoint{1.509126in}{0.780925in}}%
\pgfpathlineto{\pgfqpoint{1.503742in}{0.767314in}}%
\pgfpathlineto{\pgfqpoint{1.503344in}{0.766687in}}%
\pgfpathlineto{\pgfqpoint{1.494928in}{0.753703in}}%
\pgfpathlineto{\pgfqpoint{1.487688in}{0.745063in}}%
\pgfpathlineto{\pgfqpoint{1.483280in}{0.740092in}}%
\pgfpathlineto{\pgfqpoint{1.472031in}{0.729288in}}%
\pgfpathlineto{\pgfqpoint{1.468802in}{0.726481in}}%
\pgfpathlineto{\pgfqpoint{1.456375in}{0.716702in}}%
\pgfpathlineto{\pgfqpoint{1.450657in}{0.712870in}}%
\pgfpathlineto{\pgfqpoint{1.440718in}{0.706575in}}%
\pgfpathlineto{\pgfqpoint{1.425783in}{0.699259in}}%
\pgfpathlineto{\pgfqpoint{1.425061in}{0.698913in}}%
\pgfpathlineto{\pgfqpoint{1.409405in}{0.694232in}}%
\pgfpathlineto{\pgfqpoint{1.393748in}{0.692621in}}%
\pgfpathlineto{\pgfqpoint{1.378092in}{0.694232in}}%
\pgfpathlineto{\pgfqpoint{1.362435in}{0.698913in}}%
\pgfpathlineto{\pgfqpoint{1.361713in}{0.699259in}}%
\pgfpathclose%
\pgfpathmoveto{\pgfqpoint{1.879102in}{0.654583in}}%
\pgfpathlineto{\pgfqpoint{1.894758in}{0.648779in}}%
\pgfpathlineto{\pgfqpoint{1.910415in}{0.646782in}}%
\pgfpathlineto{\pgfqpoint{1.910415in}{0.658425in}}%
\pgfpathlineto{\pgfqpoint{1.910415in}{0.672036in}}%
\pgfpathlineto{\pgfqpoint{1.910415in}{0.685648in}}%
\pgfpathlineto{\pgfqpoint{1.910415in}{0.692621in}}%
\pgfpathlineto{\pgfqpoint{1.894758in}{0.694232in}}%
\pgfpathlineto{\pgfqpoint{1.879102in}{0.698913in}}%
\pgfpathlineto{\pgfqpoint{1.878380in}{0.699259in}}%
\pgfpathlineto{\pgfqpoint{1.863445in}{0.706575in}}%
\pgfpathlineto{\pgfqpoint{1.853506in}{0.712870in}}%
\pgfpathlineto{\pgfqpoint{1.847789in}{0.716702in}}%
\pgfpathlineto{\pgfqpoint{1.835361in}{0.726481in}}%
\pgfpathlineto{\pgfqpoint{1.832132in}{0.729288in}}%
\pgfpathlineto{\pgfqpoint{1.820884in}{0.740092in}}%
\pgfpathlineto{\pgfqpoint{1.816476in}{0.745063in}}%
\pgfpathlineto{\pgfqpoint{1.809235in}{0.753703in}}%
\pgfpathlineto{\pgfqpoint{1.800819in}{0.766687in}}%
\pgfpathlineto{\pgfqpoint{1.800422in}{0.767314in}}%
\pgfpathlineto{\pgfqpoint{1.795037in}{0.780925in}}%
\pgfpathlineto{\pgfqpoint{1.793184in}{0.794536in}}%
\pgfpathlineto{\pgfqpoint{1.795037in}{0.808148in}}%
\pgfpathlineto{\pgfqpoint{1.800422in}{0.821759in}}%
\pgfpathlineto{\pgfqpoint{1.800819in}{0.822386in}}%
\pgfpathlineto{\pgfqpoint{1.809235in}{0.835370in}}%
\pgfpathlineto{\pgfqpoint{1.816476in}{0.844010in}}%
\pgfpathlineto{\pgfqpoint{1.820884in}{0.848981in}}%
\pgfpathlineto{\pgfqpoint{1.832132in}{0.859785in}}%
\pgfpathlineto{\pgfqpoint{1.835361in}{0.862592in}}%
\pgfpathlineto{\pgfqpoint{1.847789in}{0.872371in}}%
\pgfpathlineto{\pgfqpoint{1.853506in}{0.876203in}}%
\pgfpathlineto{\pgfqpoint{1.863445in}{0.882498in}}%
\pgfpathlineto{\pgfqpoint{1.878380in}{0.889814in}}%
\pgfpathlineto{\pgfqpoint{1.879102in}{0.890160in}}%
\pgfpathlineto{\pgfqpoint{1.894758in}{0.894841in}}%
\pgfpathlineto{\pgfqpoint{1.910415in}{0.896452in}}%
\pgfpathlineto{\pgfqpoint{1.910415in}{0.903425in}}%
\pgfpathlineto{\pgfqpoint{1.910415in}{0.917036in}}%
\pgfpathlineto{\pgfqpoint{1.910415in}{0.930648in}}%
\pgfpathlineto{\pgfqpoint{1.910415in}{0.942291in}}%
\pgfpathlineto{\pgfqpoint{1.894758in}{0.940293in}}%
\pgfpathlineto{\pgfqpoint{1.879102in}{0.934490in}}%
\pgfpathlineto{\pgfqpoint{1.872568in}{0.930648in}}%
\pgfpathlineto{\pgfqpoint{1.863445in}{0.925741in}}%
\pgfpathlineto{\pgfqpoint{1.850977in}{0.917036in}}%
\pgfpathlineto{\pgfqpoint{1.847789in}{0.914936in}}%
\pgfpathlineto{\pgfqpoint{1.832894in}{0.903425in}}%
\pgfpathlineto{\pgfqpoint{1.832132in}{0.902852in}}%
\pgfpathlineto{\pgfqpoint{1.816476in}{0.889916in}}%
\pgfpathlineto{\pgfqpoint{1.816358in}{0.889814in}}%
\pgfpathlineto{\pgfqpoint{1.800819in}{0.876306in}}%
\pgfpathlineto{\pgfqpoint{1.800701in}{0.876203in}}%
\pgfpathlineto{\pgfqpoint{1.785822in}{0.862592in}}%
\pgfpathlineto{\pgfqpoint{1.785162in}{0.861930in}}%
\pgfpathlineto{\pgfqpoint{1.771922in}{0.848981in}}%
\pgfpathlineto{\pgfqpoint{1.769506in}{0.846209in}}%
\pgfpathlineto{\pgfqpoint{1.759494in}{0.835370in}}%
\pgfpathlineto{\pgfqpoint{1.753849in}{0.827439in}}%
\pgfpathlineto{\pgfqpoint{1.749429in}{0.821759in}}%
\pgfpathlineto{\pgfqpoint{1.742754in}{0.808148in}}%
\pgfpathlineto{\pgfqpoint{1.740456in}{0.794536in}}%
\pgfpathlineto{\pgfqpoint{1.742754in}{0.780925in}}%
\pgfpathlineto{\pgfqpoint{1.749429in}{0.767314in}}%
\pgfpathlineto{\pgfqpoint{1.753849in}{0.761634in}}%
\pgfpathlineto{\pgfqpoint{1.759494in}{0.753703in}}%
\pgfpathlineto{\pgfqpoint{1.769506in}{0.742863in}}%
\pgfpathlineto{\pgfqpoint{1.771922in}{0.740092in}}%
\pgfpathlineto{\pgfqpoint{1.785162in}{0.727143in}}%
\pgfpathlineto{\pgfqpoint{1.785822in}{0.726481in}}%
\pgfpathlineto{\pgfqpoint{1.800701in}{0.712870in}}%
\pgfpathlineto{\pgfqpoint{1.800819in}{0.712767in}}%
\pgfpathlineto{\pgfqpoint{1.816358in}{0.699259in}}%
\pgfpathlineto{\pgfqpoint{1.816476in}{0.699156in}}%
\pgfpathlineto{\pgfqpoint{1.832132in}{0.686220in}}%
\pgfpathlineto{\pgfqpoint{1.832894in}{0.685648in}}%
\pgfpathlineto{\pgfqpoint{1.847789in}{0.674136in}}%
\pgfpathlineto{\pgfqpoint{1.850977in}{0.672036in}}%
\pgfpathlineto{\pgfqpoint{1.863445in}{0.663332in}}%
\pgfpathlineto{\pgfqpoint{1.872568in}{0.658425in}}%
\pgfpathlineto{\pgfqpoint{1.879102in}{0.654583in}}%
\pgfpathclose%
\pgfpathmoveto{\pgfqpoint{0.376072in}{1.097946in}}%
\pgfpathlineto{\pgfqpoint{0.391728in}{1.103749in}}%
\pgfpathlineto{\pgfqpoint{0.398262in}{1.107592in}}%
\pgfpathlineto{\pgfqpoint{0.407385in}{1.112499in}}%
\pgfpathlineto{\pgfqpoint{0.419853in}{1.121203in}}%
\pgfpathlineto{\pgfqpoint{0.423041in}{1.123303in}}%
\pgfpathlineto{\pgfqpoint{0.437936in}{1.134814in}}%
\pgfpathlineto{\pgfqpoint{0.438698in}{1.135387in}}%
\pgfpathlineto{\pgfqpoint{0.454354in}{1.148323in}}%
\pgfpathlineto{\pgfqpoint{0.454472in}{1.148425in}}%
\pgfpathlineto{\pgfqpoint{0.470011in}{1.161934in}}%
\pgfpathlineto{\pgfqpoint{0.470129in}{1.162036in}}%
\pgfpathlineto{\pgfqpoint{0.485008in}{1.175647in}}%
\pgfpathlineto{\pgfqpoint{0.485668in}{1.176310in}}%
\pgfpathlineto{\pgfqpoint{0.498908in}{1.189259in}}%
\pgfpathlineto{\pgfqpoint{0.501324in}{1.192030in}}%
\pgfpathlineto{\pgfqpoint{0.511336in}{1.202870in}}%
\pgfpathlineto{\pgfqpoint{0.516981in}{1.210800in}}%
\pgfpathlineto{\pgfqpoint{0.521401in}{1.216481in}}%
\pgfpathlineto{\pgfqpoint{0.528076in}{1.230092in}}%
\pgfpathlineto{\pgfqpoint{0.530374in}{1.243703in}}%
\pgfpathlineto{\pgfqpoint{0.528076in}{1.257314in}}%
\pgfpathlineto{\pgfqpoint{0.521401in}{1.270925in}}%
\pgfpathlineto{\pgfqpoint{0.516981in}{1.276606in}}%
\pgfpathlineto{\pgfqpoint{0.511336in}{1.284536in}}%
\pgfpathlineto{\pgfqpoint{0.501324in}{1.295376in}}%
\pgfpathlineto{\pgfqpoint{0.498908in}{1.298148in}}%
\pgfpathlineto{\pgfqpoint{0.485668in}{1.311096in}}%
\pgfpathlineto{\pgfqpoint{0.485008in}{1.311759in}}%
\pgfpathlineto{\pgfqpoint{0.470129in}{1.325370in}}%
\pgfpathlineto{\pgfqpoint{0.470011in}{1.325472in}}%
\pgfpathlineto{\pgfqpoint{0.454472in}{1.338981in}}%
\pgfpathlineto{\pgfqpoint{0.454354in}{1.339083in}}%
\pgfpathlineto{\pgfqpoint{0.438698in}{1.352019in}}%
\pgfpathlineto{\pgfqpoint{0.437936in}{1.352592in}}%
\pgfpathlineto{\pgfqpoint{0.423041in}{1.364103in}}%
\pgfpathlineto{\pgfqpoint{0.419853in}{1.366203in}}%
\pgfpathlineto{\pgfqpoint{0.407385in}{1.374907in}}%
\pgfpathlineto{\pgfqpoint{0.398262in}{1.379814in}}%
\pgfpathlineto{\pgfqpoint{0.391728in}{1.383657in}}%
\pgfpathlineto{\pgfqpoint{0.376072in}{1.389460in}}%
\pgfpathlineto{\pgfqpoint{0.360415in}{1.391458in}}%
\pgfpathlineto{\pgfqpoint{0.360415in}{1.379814in}}%
\pgfpathlineto{\pgfqpoint{0.360415in}{1.366203in}}%
\pgfpathlineto{\pgfqpoint{0.360415in}{1.352592in}}%
\pgfpathlineto{\pgfqpoint{0.360415in}{1.345619in}}%
\pgfpathlineto{\pgfqpoint{0.376072in}{1.344007in}}%
\pgfpathlineto{\pgfqpoint{0.391728in}{1.339326in}}%
\pgfpathlineto{\pgfqpoint{0.392450in}{1.338981in}}%
\pgfpathlineto{\pgfqpoint{0.407385in}{1.331664in}}%
\pgfpathlineto{\pgfqpoint{0.417324in}{1.325370in}}%
\pgfpathlineto{\pgfqpoint{0.423041in}{1.321538in}}%
\pgfpathlineto{\pgfqpoint{0.435469in}{1.311759in}}%
\pgfpathlineto{\pgfqpoint{0.438698in}{1.308952in}}%
\pgfpathlineto{\pgfqpoint{0.449946in}{1.298148in}}%
\pgfpathlineto{\pgfqpoint{0.454354in}{1.293177in}}%
\pgfpathlineto{\pgfqpoint{0.461595in}{1.284536in}}%
\pgfpathlineto{\pgfqpoint{0.470011in}{1.271553in}}%
\pgfpathlineto{\pgfqpoint{0.470408in}{1.270925in}}%
\pgfpathlineto{\pgfqpoint{0.475793in}{1.257314in}}%
\pgfpathlineto{\pgfqpoint{0.477646in}{1.243703in}}%
\pgfpathlineto{\pgfqpoint{0.475793in}{1.230092in}}%
\pgfpathlineto{\pgfqpoint{0.470408in}{1.216481in}}%
\pgfpathlineto{\pgfqpoint{0.470011in}{1.215853in}}%
\pgfpathlineto{\pgfqpoint{0.461595in}{1.202870in}}%
\pgfpathlineto{\pgfqpoint{0.454354in}{1.194229in}}%
\pgfpathlineto{\pgfqpoint{0.449946in}{1.189259in}}%
\pgfpathlineto{\pgfqpoint{0.438698in}{1.178455in}}%
\pgfpathlineto{\pgfqpoint{0.435469in}{1.175647in}}%
\pgfpathlineto{\pgfqpoint{0.423041in}{1.165869in}}%
\pgfpathlineto{\pgfqpoint{0.417324in}{1.162036in}}%
\pgfpathlineto{\pgfqpoint{0.407385in}{1.155742in}}%
\pgfpathlineto{\pgfqpoint{0.392450in}{1.148425in}}%
\pgfpathlineto{\pgfqpoint{0.391728in}{1.148080in}}%
\pgfpathlineto{\pgfqpoint{0.376072in}{1.143399in}}%
\pgfpathlineto{\pgfqpoint{0.360415in}{1.141788in}}%
\pgfpathlineto{\pgfqpoint{0.360415in}{1.134814in}}%
\pgfpathlineto{\pgfqpoint{0.360415in}{1.121203in}}%
\pgfpathlineto{\pgfqpoint{0.360415in}{1.107592in}}%
\pgfpathlineto{\pgfqpoint{0.360415in}{1.095948in}}%
\pgfpathlineto{\pgfqpoint{0.376072in}{1.097946in}}%
\pgfpathclose%
\pgfpathmoveto{\pgfqpoint{0.845769in}{1.103749in}}%
\pgfpathlineto{\pgfqpoint{0.861425in}{1.097946in}}%
\pgfpathlineto{\pgfqpoint{0.877082in}{1.095948in}}%
\pgfpathlineto{\pgfqpoint{0.892738in}{1.097946in}}%
\pgfpathlineto{\pgfqpoint{0.908395in}{1.103749in}}%
\pgfpathlineto{\pgfqpoint{0.914929in}{1.107592in}}%
\pgfpathlineto{\pgfqpoint{0.924051in}{1.112499in}}%
\pgfpathlineto{\pgfqpoint{0.936520in}{1.121203in}}%
\pgfpathlineto{\pgfqpoint{0.939708in}{1.123303in}}%
\pgfpathlineto{\pgfqpoint{0.954603in}{1.134814in}}%
\pgfpathlineto{\pgfqpoint{0.955364in}{1.135387in}}%
\pgfpathlineto{\pgfqpoint{0.971021in}{1.148323in}}%
\pgfpathlineto{\pgfqpoint{0.971139in}{1.148425in}}%
\pgfpathlineto{\pgfqpoint{0.986678in}{1.161934in}}%
\pgfpathlineto{\pgfqpoint{0.986795in}{1.162036in}}%
\pgfpathlineto{\pgfqpoint{1.001675in}{1.175647in}}%
\pgfpathlineto{\pgfqpoint{1.002334in}{1.176310in}}%
\pgfpathlineto{\pgfqpoint{1.015575in}{1.189259in}}%
\pgfpathlineto{\pgfqpoint{1.017991in}{1.192030in}}%
\pgfpathlineto{\pgfqpoint{1.028003in}{1.202870in}}%
\pgfpathlineto{\pgfqpoint{1.033647in}{1.210800in}}%
\pgfpathlineto{\pgfqpoint{1.038067in}{1.216481in}}%
\pgfpathlineto{\pgfqpoint{1.044743in}{1.230092in}}%
\pgfpathlineto{\pgfqpoint{1.047041in}{1.243703in}}%
\pgfpathlineto{\pgfqpoint{1.044743in}{1.257314in}}%
\pgfpathlineto{\pgfqpoint{1.038067in}{1.270925in}}%
\pgfpathlineto{\pgfqpoint{1.033647in}{1.276606in}}%
\pgfpathlineto{\pgfqpoint{1.028003in}{1.284536in}}%
\pgfpathlineto{\pgfqpoint{1.017991in}{1.295376in}}%
\pgfpathlineto{\pgfqpoint{1.015575in}{1.298148in}}%
\pgfpathlineto{\pgfqpoint{1.002334in}{1.311096in}}%
\pgfpathlineto{\pgfqpoint{1.001675in}{1.311759in}}%
\pgfpathlineto{\pgfqpoint{0.986795in}{1.325370in}}%
\pgfpathlineto{\pgfqpoint{0.986678in}{1.325472in}}%
\pgfpathlineto{\pgfqpoint{0.971139in}{1.338981in}}%
\pgfpathlineto{\pgfqpoint{0.971021in}{1.339083in}}%
\pgfpathlineto{\pgfqpoint{0.955364in}{1.352019in}}%
\pgfpathlineto{\pgfqpoint{0.954603in}{1.352592in}}%
\pgfpathlineto{\pgfqpoint{0.939708in}{1.364103in}}%
\pgfpathlineto{\pgfqpoint{0.936520in}{1.366203in}}%
\pgfpathlineto{\pgfqpoint{0.924051in}{1.374907in}}%
\pgfpathlineto{\pgfqpoint{0.914929in}{1.379814in}}%
\pgfpathlineto{\pgfqpoint{0.908395in}{1.383657in}}%
\pgfpathlineto{\pgfqpoint{0.892738in}{1.389460in}}%
\pgfpathlineto{\pgfqpoint{0.877082in}{1.391458in}}%
\pgfpathlineto{\pgfqpoint{0.861425in}{1.389460in}}%
\pgfpathlineto{\pgfqpoint{0.845769in}{1.383657in}}%
\pgfpathlineto{\pgfqpoint{0.839234in}{1.379814in}}%
\pgfpathlineto{\pgfqpoint{0.830112in}{1.374907in}}%
\pgfpathlineto{\pgfqpoint{0.817643in}{1.366203in}}%
\pgfpathlineto{\pgfqpoint{0.814455in}{1.364103in}}%
\pgfpathlineto{\pgfqpoint{0.799561in}{1.352592in}}%
\pgfpathlineto{\pgfqpoint{0.798799in}{1.352019in}}%
\pgfpathlineto{\pgfqpoint{0.783142in}{1.339083in}}%
\pgfpathlineto{\pgfqpoint{0.783024in}{1.338981in}}%
\pgfpathlineto{\pgfqpoint{0.767486in}{1.325472in}}%
\pgfpathlineto{\pgfqpoint{0.767368in}{1.325370in}}%
\pgfpathlineto{\pgfqpoint{0.752488in}{1.311759in}}%
\pgfpathlineto{\pgfqpoint{0.751829in}{1.311096in}}%
\pgfpathlineto{\pgfqpoint{0.738588in}{1.298148in}}%
\pgfpathlineto{\pgfqpoint{0.736173in}{1.295376in}}%
\pgfpathlineto{\pgfqpoint{0.726160in}{1.284536in}}%
\pgfpathlineto{\pgfqpoint{0.720516in}{1.276606in}}%
\pgfpathlineto{\pgfqpoint{0.716096in}{1.270925in}}%
\pgfpathlineto{\pgfqpoint{0.709420in}{1.257314in}}%
\pgfpathlineto{\pgfqpoint{0.707123in}{1.243703in}}%
\pgfpathlineto{\pgfqpoint{0.709420in}{1.230092in}}%
\pgfpathlineto{\pgfqpoint{0.716096in}{1.216481in}}%
\pgfpathlineto{\pgfqpoint{0.720516in}{1.210800in}}%
\pgfpathlineto{\pgfqpoint{0.726160in}{1.202870in}}%
\pgfpathlineto{\pgfqpoint{0.736173in}{1.192030in}}%
\pgfpathlineto{\pgfqpoint{0.738588in}{1.189259in}}%
\pgfpathlineto{\pgfqpoint{0.751829in}{1.176310in}}%
\pgfpathlineto{\pgfqpoint{0.752488in}{1.175647in}}%
\pgfpathlineto{\pgfqpoint{0.767368in}{1.162036in}}%
\pgfpathlineto{\pgfqpoint{0.767486in}{1.161934in}}%
\pgfpathlineto{\pgfqpoint{0.783024in}{1.148425in}}%
\pgfpathlineto{\pgfqpoint{0.783142in}{1.148323in}}%
\pgfpathlineto{\pgfqpoint{0.798799in}{1.135387in}}%
\pgfpathlineto{\pgfqpoint{0.799561in}{1.134814in}}%
\pgfpathlineto{\pgfqpoint{0.814455in}{1.123303in}}%
\pgfpathlineto{\pgfqpoint{0.817643in}{1.121203in}}%
\pgfpathlineto{\pgfqpoint{0.830112in}{1.112499in}}%
\pgfpathlineto{\pgfqpoint{0.839234in}{1.107592in}}%
\pgfpathlineto{\pgfqpoint{0.845769in}{1.103749in}}%
\pgfpathclose%
\pgfpathmoveto{\pgfqpoint{0.845047in}{1.148425in}}%
\pgfpathlineto{\pgfqpoint{0.830112in}{1.155742in}}%
\pgfpathlineto{\pgfqpoint{0.820173in}{1.162036in}}%
\pgfpathlineto{\pgfqpoint{0.814455in}{1.165869in}}%
\pgfpathlineto{\pgfqpoint{0.802028in}{1.175647in}}%
\pgfpathlineto{\pgfqpoint{0.798799in}{1.178455in}}%
\pgfpathlineto{\pgfqpoint{0.787550in}{1.189259in}}%
\pgfpathlineto{\pgfqpoint{0.783142in}{1.194229in}}%
\pgfpathlineto{\pgfqpoint{0.775902in}{1.202870in}}%
\pgfpathlineto{\pgfqpoint{0.767486in}{1.215853in}}%
\pgfpathlineto{\pgfqpoint{0.767088in}{1.216481in}}%
\pgfpathlineto{\pgfqpoint{0.761704in}{1.230092in}}%
\pgfpathlineto{\pgfqpoint{0.759850in}{1.243703in}}%
\pgfpathlineto{\pgfqpoint{0.761704in}{1.257314in}}%
\pgfpathlineto{\pgfqpoint{0.767088in}{1.270925in}}%
\pgfpathlineto{\pgfqpoint{0.767486in}{1.271553in}}%
\pgfpathlineto{\pgfqpoint{0.775902in}{1.284536in}}%
\pgfpathlineto{\pgfqpoint{0.783142in}{1.293177in}}%
\pgfpathlineto{\pgfqpoint{0.787550in}{1.298148in}}%
\pgfpathlineto{\pgfqpoint{0.798799in}{1.308952in}}%
\pgfpathlineto{\pgfqpoint{0.802028in}{1.311759in}}%
\pgfpathlineto{\pgfqpoint{0.814455in}{1.321538in}}%
\pgfpathlineto{\pgfqpoint{0.820173in}{1.325370in}}%
\pgfpathlineto{\pgfqpoint{0.830112in}{1.331664in}}%
\pgfpathlineto{\pgfqpoint{0.845047in}{1.338981in}}%
\pgfpathlineto{\pgfqpoint{0.845769in}{1.339326in}}%
\pgfpathlineto{\pgfqpoint{0.861425in}{1.344007in}}%
\pgfpathlineto{\pgfqpoint{0.877082in}{1.345619in}}%
\pgfpathlineto{\pgfqpoint{0.892738in}{1.344007in}}%
\pgfpathlineto{\pgfqpoint{0.908395in}{1.339326in}}%
\pgfpathlineto{\pgfqpoint{0.909117in}{1.338981in}}%
\pgfpathlineto{\pgfqpoint{0.924051in}{1.331664in}}%
\pgfpathlineto{\pgfqpoint{0.933990in}{1.325370in}}%
\pgfpathlineto{\pgfqpoint{0.939708in}{1.321538in}}%
\pgfpathlineto{\pgfqpoint{0.952136in}{1.311759in}}%
\pgfpathlineto{\pgfqpoint{0.955364in}{1.308952in}}%
\pgfpathlineto{\pgfqpoint{0.966613in}{1.298148in}}%
\pgfpathlineto{\pgfqpoint{0.971021in}{1.293177in}}%
\pgfpathlineto{\pgfqpoint{0.978262in}{1.284536in}}%
\pgfpathlineto{\pgfqpoint{0.986678in}{1.271553in}}%
\pgfpathlineto{\pgfqpoint{0.987075in}{1.270925in}}%
\pgfpathlineto{\pgfqpoint{0.992459in}{1.257314in}}%
\pgfpathlineto{\pgfqpoint{0.994313in}{1.243703in}}%
\pgfpathlineto{\pgfqpoint{0.992459in}{1.230092in}}%
\pgfpathlineto{\pgfqpoint{0.987075in}{1.216481in}}%
\pgfpathlineto{\pgfqpoint{0.986678in}{1.215853in}}%
\pgfpathlineto{\pgfqpoint{0.978262in}{1.202870in}}%
\pgfpathlineto{\pgfqpoint{0.971021in}{1.194229in}}%
\pgfpathlineto{\pgfqpoint{0.966613in}{1.189259in}}%
\pgfpathlineto{\pgfqpoint{0.955364in}{1.178455in}}%
\pgfpathlineto{\pgfqpoint{0.952136in}{1.175647in}}%
\pgfpathlineto{\pgfqpoint{0.939708in}{1.165869in}}%
\pgfpathlineto{\pgfqpoint{0.933990in}{1.162036in}}%
\pgfpathlineto{\pgfqpoint{0.924051in}{1.155742in}}%
\pgfpathlineto{\pgfqpoint{0.909117in}{1.148425in}}%
\pgfpathlineto{\pgfqpoint{0.908395in}{1.148080in}}%
\pgfpathlineto{\pgfqpoint{0.892738in}{1.143399in}}%
\pgfpathlineto{\pgfqpoint{0.877082in}{1.141788in}}%
\pgfpathlineto{\pgfqpoint{0.861425in}{1.143399in}}%
\pgfpathlineto{\pgfqpoint{0.845769in}{1.148080in}}%
\pgfpathlineto{\pgfqpoint{0.845047in}{1.148425in}}%
\pgfpathclose%
\pgfpathmoveto{\pgfqpoint{1.362435in}{1.103749in}}%
\pgfpathlineto{\pgfqpoint{1.378092in}{1.097946in}}%
\pgfpathlineto{\pgfqpoint{1.393748in}{1.095948in}}%
\pgfpathlineto{\pgfqpoint{1.409405in}{1.097946in}}%
\pgfpathlineto{\pgfqpoint{1.425061in}{1.103749in}}%
\pgfpathlineto{\pgfqpoint{1.431596in}{1.107592in}}%
\pgfpathlineto{\pgfqpoint{1.440718in}{1.112499in}}%
\pgfpathlineto{\pgfqpoint{1.453187in}{1.121203in}}%
\pgfpathlineto{\pgfqpoint{1.456375in}{1.123303in}}%
\pgfpathlineto{\pgfqpoint{1.471269in}{1.134814in}}%
\pgfpathlineto{\pgfqpoint{1.472031in}{1.135387in}}%
\pgfpathlineto{\pgfqpoint{1.487688in}{1.148323in}}%
\pgfpathlineto{\pgfqpoint{1.487806in}{1.148425in}}%
\pgfpathlineto{\pgfqpoint{1.503344in}{1.161934in}}%
\pgfpathlineto{\pgfqpoint{1.503462in}{1.162036in}}%
\pgfpathlineto{\pgfqpoint{1.518342in}{1.175647in}}%
\pgfpathlineto{\pgfqpoint{1.519001in}{1.176310in}}%
\pgfpathlineto{\pgfqpoint{1.532242in}{1.189259in}}%
\pgfpathlineto{\pgfqpoint{1.534657in}{1.192030in}}%
\pgfpathlineto{\pgfqpoint{1.544670in}{1.202870in}}%
\pgfpathlineto{\pgfqpoint{1.550314in}{1.210800in}}%
\pgfpathlineto{\pgfqpoint{1.554734in}{1.216481in}}%
\pgfpathlineto{\pgfqpoint{1.561410in}{1.230092in}}%
\pgfpathlineto{\pgfqpoint{1.563707in}{1.243703in}}%
\pgfpathlineto{\pgfqpoint{1.561410in}{1.257314in}}%
\pgfpathlineto{\pgfqpoint{1.554734in}{1.270925in}}%
\pgfpathlineto{\pgfqpoint{1.550314in}{1.276606in}}%
\pgfpathlineto{\pgfqpoint{1.544670in}{1.284536in}}%
\pgfpathlineto{\pgfqpoint{1.534657in}{1.295376in}}%
\pgfpathlineto{\pgfqpoint{1.532242in}{1.298148in}}%
\pgfpathlineto{\pgfqpoint{1.519001in}{1.311096in}}%
\pgfpathlineto{\pgfqpoint{1.518342in}{1.311759in}}%
\pgfpathlineto{\pgfqpoint{1.503462in}{1.325370in}}%
\pgfpathlineto{\pgfqpoint{1.503344in}{1.325472in}}%
\pgfpathlineto{\pgfqpoint{1.487806in}{1.338981in}}%
\pgfpathlineto{\pgfqpoint{1.487688in}{1.339083in}}%
\pgfpathlineto{\pgfqpoint{1.472031in}{1.352019in}}%
\pgfpathlineto{\pgfqpoint{1.471269in}{1.352592in}}%
\pgfpathlineto{\pgfqpoint{1.456375in}{1.364103in}}%
\pgfpathlineto{\pgfqpoint{1.453187in}{1.366203in}}%
\pgfpathlineto{\pgfqpoint{1.440718in}{1.374907in}}%
\pgfpathlineto{\pgfqpoint{1.431596in}{1.379814in}}%
\pgfpathlineto{\pgfqpoint{1.425061in}{1.383657in}}%
\pgfpathlineto{\pgfqpoint{1.409405in}{1.389460in}}%
\pgfpathlineto{\pgfqpoint{1.393748in}{1.391458in}}%
\pgfpathlineto{\pgfqpoint{1.378092in}{1.389460in}}%
\pgfpathlineto{\pgfqpoint{1.362435in}{1.383657in}}%
\pgfpathlineto{\pgfqpoint{1.355901in}{1.379814in}}%
\pgfpathlineto{\pgfqpoint{1.346779in}{1.374907in}}%
\pgfpathlineto{\pgfqpoint{1.334310in}{1.366203in}}%
\pgfpathlineto{\pgfqpoint{1.331122in}{1.364103in}}%
\pgfpathlineto{\pgfqpoint{1.316227in}{1.352592in}}%
\pgfpathlineto{\pgfqpoint{1.315466in}{1.352019in}}%
\pgfpathlineto{\pgfqpoint{1.299809in}{1.339083in}}%
\pgfpathlineto{\pgfqpoint{1.299691in}{1.338981in}}%
\pgfpathlineto{\pgfqpoint{1.284152in}{1.325472in}}%
\pgfpathlineto{\pgfqpoint{1.284035in}{1.325370in}}%
\pgfpathlineto{\pgfqpoint{1.269155in}{1.311759in}}%
\pgfpathlineto{\pgfqpoint{1.268496in}{1.311096in}}%
\pgfpathlineto{\pgfqpoint{1.255255in}{1.298148in}}%
\pgfpathlineto{\pgfqpoint{1.252839in}{1.295376in}}%
\pgfpathlineto{\pgfqpoint{1.242827in}{1.284536in}}%
\pgfpathlineto{\pgfqpoint{1.237183in}{1.276606in}}%
\pgfpathlineto{\pgfqpoint{1.232763in}{1.270925in}}%
\pgfpathlineto{\pgfqpoint{1.226087in}{1.257314in}}%
\pgfpathlineto{\pgfqpoint{1.223789in}{1.243703in}}%
\pgfpathlineto{\pgfqpoint{1.226087in}{1.230092in}}%
\pgfpathlineto{\pgfqpoint{1.232763in}{1.216481in}}%
\pgfpathlineto{\pgfqpoint{1.237183in}{1.210800in}}%
\pgfpathlineto{\pgfqpoint{1.242827in}{1.202870in}}%
\pgfpathlineto{\pgfqpoint{1.252839in}{1.192030in}}%
\pgfpathlineto{\pgfqpoint{1.255255in}{1.189259in}}%
\pgfpathlineto{\pgfqpoint{1.268496in}{1.176310in}}%
\pgfpathlineto{\pgfqpoint{1.269155in}{1.175647in}}%
\pgfpathlineto{\pgfqpoint{1.284035in}{1.162036in}}%
\pgfpathlineto{\pgfqpoint{1.284152in}{1.161934in}}%
\pgfpathlineto{\pgfqpoint{1.299691in}{1.148425in}}%
\pgfpathlineto{\pgfqpoint{1.299809in}{1.148323in}}%
\pgfpathlineto{\pgfqpoint{1.315466in}{1.135387in}}%
\pgfpathlineto{\pgfqpoint{1.316227in}{1.134814in}}%
\pgfpathlineto{\pgfqpoint{1.331122in}{1.123303in}}%
\pgfpathlineto{\pgfqpoint{1.334310in}{1.121203in}}%
\pgfpathlineto{\pgfqpoint{1.346779in}{1.112499in}}%
\pgfpathlineto{\pgfqpoint{1.355901in}{1.107592in}}%
\pgfpathlineto{\pgfqpoint{1.362435in}{1.103749in}}%
\pgfpathclose%
\pgfpathmoveto{\pgfqpoint{1.361713in}{1.148425in}}%
\pgfpathlineto{\pgfqpoint{1.346779in}{1.155742in}}%
\pgfpathlineto{\pgfqpoint{1.336840in}{1.162036in}}%
\pgfpathlineto{\pgfqpoint{1.331122in}{1.165869in}}%
\pgfpathlineto{\pgfqpoint{1.318694in}{1.175647in}}%
\pgfpathlineto{\pgfqpoint{1.315466in}{1.178455in}}%
\pgfpathlineto{\pgfqpoint{1.304217in}{1.189259in}}%
\pgfpathlineto{\pgfqpoint{1.299809in}{1.194229in}}%
\pgfpathlineto{\pgfqpoint{1.292568in}{1.202870in}}%
\pgfpathlineto{\pgfqpoint{1.284152in}{1.215853in}}%
\pgfpathlineto{\pgfqpoint{1.283755in}{1.216481in}}%
\pgfpathlineto{\pgfqpoint{1.278371in}{1.230092in}}%
\pgfpathlineto{\pgfqpoint{1.276517in}{1.243703in}}%
\pgfpathlineto{\pgfqpoint{1.278371in}{1.257314in}}%
\pgfpathlineto{\pgfqpoint{1.283755in}{1.270925in}}%
\pgfpathlineto{\pgfqpoint{1.284152in}{1.271553in}}%
\pgfpathlineto{\pgfqpoint{1.292568in}{1.284536in}}%
\pgfpathlineto{\pgfqpoint{1.299809in}{1.293177in}}%
\pgfpathlineto{\pgfqpoint{1.304217in}{1.298148in}}%
\pgfpathlineto{\pgfqpoint{1.315466in}{1.308952in}}%
\pgfpathlineto{\pgfqpoint{1.318694in}{1.311759in}}%
\pgfpathlineto{\pgfqpoint{1.331122in}{1.321538in}}%
\pgfpathlineto{\pgfqpoint{1.336840in}{1.325370in}}%
\pgfpathlineto{\pgfqpoint{1.346779in}{1.331664in}}%
\pgfpathlineto{\pgfqpoint{1.361713in}{1.338981in}}%
\pgfpathlineto{\pgfqpoint{1.362435in}{1.339326in}}%
\pgfpathlineto{\pgfqpoint{1.378092in}{1.344007in}}%
\pgfpathlineto{\pgfqpoint{1.393748in}{1.345619in}}%
\pgfpathlineto{\pgfqpoint{1.409405in}{1.344007in}}%
\pgfpathlineto{\pgfqpoint{1.425061in}{1.339326in}}%
\pgfpathlineto{\pgfqpoint{1.425783in}{1.338981in}}%
\pgfpathlineto{\pgfqpoint{1.440718in}{1.331664in}}%
\pgfpathlineto{\pgfqpoint{1.450657in}{1.325370in}}%
\pgfpathlineto{\pgfqpoint{1.456375in}{1.321538in}}%
\pgfpathlineto{\pgfqpoint{1.468802in}{1.311759in}}%
\pgfpathlineto{\pgfqpoint{1.472031in}{1.308952in}}%
\pgfpathlineto{\pgfqpoint{1.483280in}{1.298148in}}%
\pgfpathlineto{\pgfqpoint{1.487688in}{1.293177in}}%
\pgfpathlineto{\pgfqpoint{1.494928in}{1.284536in}}%
\pgfpathlineto{\pgfqpoint{1.503344in}{1.271553in}}%
\pgfpathlineto{\pgfqpoint{1.503742in}{1.270925in}}%
\pgfpathlineto{\pgfqpoint{1.509126in}{1.257314in}}%
\pgfpathlineto{\pgfqpoint{1.510980in}{1.243703in}}%
\pgfpathlineto{\pgfqpoint{1.509126in}{1.230092in}}%
\pgfpathlineto{\pgfqpoint{1.503742in}{1.216481in}}%
\pgfpathlineto{\pgfqpoint{1.503344in}{1.215853in}}%
\pgfpathlineto{\pgfqpoint{1.494928in}{1.202870in}}%
\pgfpathlineto{\pgfqpoint{1.487688in}{1.194229in}}%
\pgfpathlineto{\pgfqpoint{1.483280in}{1.189259in}}%
\pgfpathlineto{\pgfqpoint{1.472031in}{1.178455in}}%
\pgfpathlineto{\pgfqpoint{1.468802in}{1.175647in}}%
\pgfpathlineto{\pgfqpoint{1.456375in}{1.165869in}}%
\pgfpathlineto{\pgfqpoint{1.450657in}{1.162036in}}%
\pgfpathlineto{\pgfqpoint{1.440718in}{1.155742in}}%
\pgfpathlineto{\pgfqpoint{1.425783in}{1.148425in}}%
\pgfpathlineto{\pgfqpoint{1.425061in}{1.148080in}}%
\pgfpathlineto{\pgfqpoint{1.409405in}{1.143399in}}%
\pgfpathlineto{\pgfqpoint{1.393748in}{1.141788in}}%
\pgfpathlineto{\pgfqpoint{1.378092in}{1.143399in}}%
\pgfpathlineto{\pgfqpoint{1.362435in}{1.148080in}}%
\pgfpathlineto{\pgfqpoint{1.361713in}{1.148425in}}%
\pgfpathclose%
\pgfpathmoveto{\pgfqpoint{1.879102in}{1.103749in}}%
\pgfpathlineto{\pgfqpoint{1.894758in}{1.097946in}}%
\pgfpathlineto{\pgfqpoint{1.910415in}{1.095948in}}%
\pgfpathlineto{\pgfqpoint{1.910415in}{1.107592in}}%
\pgfpathlineto{\pgfqpoint{1.910415in}{1.121203in}}%
\pgfpathlineto{\pgfqpoint{1.910415in}{1.134814in}}%
\pgfpathlineto{\pgfqpoint{1.910415in}{1.141788in}}%
\pgfpathlineto{\pgfqpoint{1.894758in}{1.143399in}}%
\pgfpathlineto{\pgfqpoint{1.879102in}{1.148080in}}%
\pgfpathlineto{\pgfqpoint{1.878380in}{1.148425in}}%
\pgfpathlineto{\pgfqpoint{1.863445in}{1.155742in}}%
\pgfpathlineto{\pgfqpoint{1.853506in}{1.162036in}}%
\pgfpathlineto{\pgfqpoint{1.847789in}{1.165869in}}%
\pgfpathlineto{\pgfqpoint{1.835361in}{1.175647in}}%
\pgfpathlineto{\pgfqpoint{1.832132in}{1.178455in}}%
\pgfpathlineto{\pgfqpoint{1.820884in}{1.189259in}}%
\pgfpathlineto{\pgfqpoint{1.816476in}{1.194229in}}%
\pgfpathlineto{\pgfqpoint{1.809235in}{1.202870in}}%
\pgfpathlineto{\pgfqpoint{1.800819in}{1.215853in}}%
\pgfpathlineto{\pgfqpoint{1.800422in}{1.216481in}}%
\pgfpathlineto{\pgfqpoint{1.795037in}{1.230092in}}%
\pgfpathlineto{\pgfqpoint{1.793184in}{1.243703in}}%
\pgfpathlineto{\pgfqpoint{1.795037in}{1.257314in}}%
\pgfpathlineto{\pgfqpoint{1.800422in}{1.270925in}}%
\pgfpathlineto{\pgfqpoint{1.800819in}{1.271553in}}%
\pgfpathlineto{\pgfqpoint{1.809235in}{1.284536in}}%
\pgfpathlineto{\pgfqpoint{1.816476in}{1.293177in}}%
\pgfpathlineto{\pgfqpoint{1.820884in}{1.298148in}}%
\pgfpathlineto{\pgfqpoint{1.832132in}{1.308952in}}%
\pgfpathlineto{\pgfqpoint{1.835361in}{1.311759in}}%
\pgfpathlineto{\pgfqpoint{1.847789in}{1.321538in}}%
\pgfpathlineto{\pgfqpoint{1.853506in}{1.325370in}}%
\pgfpathlineto{\pgfqpoint{1.863445in}{1.331664in}}%
\pgfpathlineto{\pgfqpoint{1.878380in}{1.338981in}}%
\pgfpathlineto{\pgfqpoint{1.879102in}{1.339326in}}%
\pgfpathlineto{\pgfqpoint{1.894758in}{1.344007in}}%
\pgfpathlineto{\pgfqpoint{1.910415in}{1.345619in}}%
\pgfpathlineto{\pgfqpoint{1.910415in}{1.352592in}}%
\pgfpathlineto{\pgfqpoint{1.910415in}{1.366203in}}%
\pgfpathlineto{\pgfqpoint{1.910415in}{1.379814in}}%
\pgfpathlineto{\pgfqpoint{1.910415in}{1.391458in}}%
\pgfpathlineto{\pgfqpoint{1.894758in}{1.389460in}}%
\pgfpathlineto{\pgfqpoint{1.879102in}{1.383657in}}%
\pgfpathlineto{\pgfqpoint{1.872568in}{1.379814in}}%
\pgfpathlineto{\pgfqpoint{1.863445in}{1.374907in}}%
\pgfpathlineto{\pgfqpoint{1.850977in}{1.366203in}}%
\pgfpathlineto{\pgfqpoint{1.847789in}{1.364103in}}%
\pgfpathlineto{\pgfqpoint{1.832894in}{1.352592in}}%
\pgfpathlineto{\pgfqpoint{1.832132in}{1.352019in}}%
\pgfpathlineto{\pgfqpoint{1.816476in}{1.339083in}}%
\pgfpathlineto{\pgfqpoint{1.816358in}{1.338981in}}%
\pgfpathlineto{\pgfqpoint{1.800819in}{1.325472in}}%
\pgfpathlineto{\pgfqpoint{1.800701in}{1.325370in}}%
\pgfpathlineto{\pgfqpoint{1.785822in}{1.311759in}}%
\pgfpathlineto{\pgfqpoint{1.785162in}{1.311096in}}%
\pgfpathlineto{\pgfqpoint{1.771922in}{1.298148in}}%
\pgfpathlineto{\pgfqpoint{1.769506in}{1.295376in}}%
\pgfpathlineto{\pgfqpoint{1.759494in}{1.284536in}}%
\pgfpathlineto{\pgfqpoint{1.753849in}{1.276606in}}%
\pgfpathlineto{\pgfqpoint{1.749429in}{1.270925in}}%
\pgfpathlineto{\pgfqpoint{1.742754in}{1.257314in}}%
\pgfpathlineto{\pgfqpoint{1.740456in}{1.243703in}}%
\pgfpathlineto{\pgfqpoint{1.742754in}{1.230092in}}%
\pgfpathlineto{\pgfqpoint{1.749429in}{1.216481in}}%
\pgfpathlineto{\pgfqpoint{1.753849in}{1.210800in}}%
\pgfpathlineto{\pgfqpoint{1.759494in}{1.202870in}}%
\pgfpathlineto{\pgfqpoint{1.769506in}{1.192030in}}%
\pgfpathlineto{\pgfqpoint{1.771922in}{1.189259in}}%
\pgfpathlineto{\pgfqpoint{1.785162in}{1.176310in}}%
\pgfpathlineto{\pgfqpoint{1.785822in}{1.175647in}}%
\pgfpathlineto{\pgfqpoint{1.800701in}{1.162036in}}%
\pgfpathlineto{\pgfqpoint{1.800819in}{1.161934in}}%
\pgfpathlineto{\pgfqpoint{1.816358in}{1.148425in}}%
\pgfpathlineto{\pgfqpoint{1.816476in}{1.148323in}}%
\pgfpathlineto{\pgfqpoint{1.832132in}{1.135387in}}%
\pgfpathlineto{\pgfqpoint{1.832894in}{1.134814in}}%
\pgfpathlineto{\pgfqpoint{1.847789in}{1.123303in}}%
\pgfpathlineto{\pgfqpoint{1.850977in}{1.121203in}}%
\pgfpathlineto{\pgfqpoint{1.863445in}{1.112499in}}%
\pgfpathlineto{\pgfqpoint{1.872568in}{1.107592in}}%
\pgfpathlineto{\pgfqpoint{1.879102in}{1.103749in}}%
\pgfpathclose%
\pgfpathmoveto{\pgfqpoint{0.376072in}{1.547113in}}%
\pgfpathlineto{\pgfqpoint{0.391728in}{1.552916in}}%
\pgfpathlineto{\pgfqpoint{0.398262in}{1.556759in}}%
\pgfpathlineto{\pgfqpoint{0.407385in}{1.561665in}}%
\pgfpathlineto{\pgfqpoint{0.419853in}{1.570370in}}%
\pgfpathlineto{\pgfqpoint{0.423041in}{1.572470in}}%
\pgfpathlineto{\pgfqpoint{0.437936in}{1.583981in}}%
\pgfpathlineto{\pgfqpoint{0.438698in}{1.584554in}}%
\pgfpathlineto{\pgfqpoint{0.454354in}{1.597490in}}%
\pgfpathlineto{\pgfqpoint{0.454472in}{1.597592in}}%
\pgfpathlineto{\pgfqpoint{0.470011in}{1.611100in}}%
\pgfpathlineto{\pgfqpoint{0.470129in}{1.611203in}}%
\pgfpathlineto{\pgfqpoint{0.485008in}{1.624814in}}%
\pgfpathlineto{\pgfqpoint{0.485668in}{1.625476in}}%
\pgfpathlineto{\pgfqpoint{0.498908in}{1.638425in}}%
\pgfpathlineto{\pgfqpoint{0.501324in}{1.641197in}}%
\pgfpathlineto{\pgfqpoint{0.511336in}{1.652036in}}%
\pgfpathlineto{\pgfqpoint{0.516981in}{1.659967in}}%
\pgfpathlineto{\pgfqpoint{0.521401in}{1.665648in}}%
\pgfpathlineto{\pgfqpoint{0.528076in}{1.679259in}}%
\pgfpathlineto{\pgfqpoint{0.530374in}{1.692870in}}%
\pgfpathlineto{\pgfqpoint{0.516981in}{1.692870in}}%
\pgfpathlineto{\pgfqpoint{0.501324in}{1.692870in}}%
\pgfpathlineto{\pgfqpoint{0.485668in}{1.692870in}}%
\pgfpathlineto{\pgfqpoint{0.477646in}{1.692870in}}%
\pgfpathlineto{\pgfqpoint{0.475793in}{1.679259in}}%
\pgfpathlineto{\pgfqpoint{0.470408in}{1.665648in}}%
\pgfpathlineto{\pgfqpoint{0.470011in}{1.665020in}}%
\pgfpathlineto{\pgfqpoint{0.461595in}{1.652036in}}%
\pgfpathlineto{\pgfqpoint{0.454354in}{1.643396in}}%
\pgfpathlineto{\pgfqpoint{0.449946in}{1.638425in}}%
\pgfpathlineto{\pgfqpoint{0.438698in}{1.627621in}}%
\pgfpathlineto{\pgfqpoint{0.435469in}{1.624814in}}%
\pgfpathlineto{\pgfqpoint{0.423041in}{1.615035in}}%
\pgfpathlineto{\pgfqpoint{0.417324in}{1.611203in}}%
\pgfpathlineto{\pgfqpoint{0.407385in}{1.604908in}}%
\pgfpathlineto{\pgfqpoint{0.392450in}{1.597592in}}%
\pgfpathlineto{\pgfqpoint{0.391728in}{1.597246in}}%
\pgfpathlineto{\pgfqpoint{0.376072in}{1.592565in}}%
\pgfpathlineto{\pgfqpoint{0.360415in}{1.590954in}}%
\pgfpathlineto{\pgfqpoint{0.360415in}{1.583981in}}%
\pgfpathlineto{\pgfqpoint{0.360415in}{1.570370in}}%
\pgfpathlineto{\pgfqpoint{0.360415in}{1.556759in}}%
\pgfpathlineto{\pgfqpoint{0.360415in}{1.545115in}}%
\pgfpathlineto{\pgfqpoint{0.376072in}{1.547113in}}%
\pgfpathclose%
\pgfpathmoveto{\pgfqpoint{0.845769in}{1.552916in}}%
\pgfpathlineto{\pgfqpoint{0.861425in}{1.547113in}}%
\pgfpathlineto{\pgfqpoint{0.877082in}{1.545115in}}%
\pgfpathlineto{\pgfqpoint{0.892738in}{1.547113in}}%
\pgfpathlineto{\pgfqpoint{0.908395in}{1.552916in}}%
\pgfpathlineto{\pgfqpoint{0.914929in}{1.556759in}}%
\pgfpathlineto{\pgfqpoint{0.924051in}{1.561665in}}%
\pgfpathlineto{\pgfqpoint{0.936520in}{1.570370in}}%
\pgfpathlineto{\pgfqpoint{0.939708in}{1.572470in}}%
\pgfpathlineto{\pgfqpoint{0.954603in}{1.583981in}}%
\pgfpathlineto{\pgfqpoint{0.955364in}{1.584554in}}%
\pgfpathlineto{\pgfqpoint{0.971021in}{1.597490in}}%
\pgfpathlineto{\pgfqpoint{0.971139in}{1.597592in}}%
\pgfpathlineto{\pgfqpoint{0.986678in}{1.611100in}}%
\pgfpathlineto{\pgfqpoint{0.986795in}{1.611203in}}%
\pgfpathlineto{\pgfqpoint{1.001675in}{1.624814in}}%
\pgfpathlineto{\pgfqpoint{1.002334in}{1.625476in}}%
\pgfpathlineto{\pgfqpoint{1.015575in}{1.638425in}}%
\pgfpathlineto{\pgfqpoint{1.017991in}{1.641197in}}%
\pgfpathlineto{\pgfqpoint{1.028003in}{1.652036in}}%
\pgfpathlineto{\pgfqpoint{1.033647in}{1.659967in}}%
\pgfpathlineto{\pgfqpoint{1.038067in}{1.665648in}}%
\pgfpathlineto{\pgfqpoint{1.044743in}{1.679259in}}%
\pgfpathlineto{\pgfqpoint{1.047041in}{1.692870in}}%
\pgfpathlineto{\pgfqpoint{1.033647in}{1.692870in}}%
\pgfpathlineto{\pgfqpoint{1.017991in}{1.692870in}}%
\pgfpathlineto{\pgfqpoint{1.002334in}{1.692870in}}%
\pgfpathlineto{\pgfqpoint{0.994313in}{1.692870in}}%
\pgfpathlineto{\pgfqpoint{0.992459in}{1.679259in}}%
\pgfpathlineto{\pgfqpoint{0.987075in}{1.665648in}}%
\pgfpathlineto{\pgfqpoint{0.986678in}{1.665020in}}%
\pgfpathlineto{\pgfqpoint{0.978262in}{1.652036in}}%
\pgfpathlineto{\pgfqpoint{0.971021in}{1.643396in}}%
\pgfpathlineto{\pgfqpoint{0.966613in}{1.638425in}}%
\pgfpathlineto{\pgfqpoint{0.955364in}{1.627621in}}%
\pgfpathlineto{\pgfqpoint{0.952136in}{1.624814in}}%
\pgfpathlineto{\pgfqpoint{0.939708in}{1.615035in}}%
\pgfpathlineto{\pgfqpoint{0.933990in}{1.611203in}}%
\pgfpathlineto{\pgfqpoint{0.924051in}{1.604908in}}%
\pgfpathlineto{\pgfqpoint{0.909117in}{1.597592in}}%
\pgfpathlineto{\pgfqpoint{0.908395in}{1.597246in}}%
\pgfpathlineto{\pgfqpoint{0.892738in}{1.592565in}}%
\pgfpathlineto{\pgfqpoint{0.877082in}{1.590954in}}%
\pgfpathlineto{\pgfqpoint{0.861425in}{1.592565in}}%
\pgfpathlineto{\pgfqpoint{0.845769in}{1.597246in}}%
\pgfpathlineto{\pgfqpoint{0.845047in}{1.597592in}}%
\pgfpathlineto{\pgfqpoint{0.830112in}{1.604908in}}%
\pgfpathlineto{\pgfqpoint{0.820173in}{1.611203in}}%
\pgfpathlineto{\pgfqpoint{0.814455in}{1.615035in}}%
\pgfpathlineto{\pgfqpoint{0.802028in}{1.624814in}}%
\pgfpathlineto{\pgfqpoint{0.798799in}{1.627621in}}%
\pgfpathlineto{\pgfqpoint{0.787550in}{1.638425in}}%
\pgfpathlineto{\pgfqpoint{0.783142in}{1.643396in}}%
\pgfpathlineto{\pgfqpoint{0.775902in}{1.652036in}}%
\pgfpathlineto{\pgfqpoint{0.767486in}{1.665020in}}%
\pgfpathlineto{\pgfqpoint{0.767088in}{1.665648in}}%
\pgfpathlineto{\pgfqpoint{0.761704in}{1.679259in}}%
\pgfpathlineto{\pgfqpoint{0.759850in}{1.692870in}}%
\pgfpathlineto{\pgfqpoint{0.751829in}{1.692870in}}%
\pgfpathlineto{\pgfqpoint{0.736173in}{1.692870in}}%
\pgfpathlineto{\pgfqpoint{0.720516in}{1.692870in}}%
\pgfpathlineto{\pgfqpoint{0.707123in}{1.692870in}}%
\pgfpathlineto{\pgfqpoint{0.709420in}{1.679259in}}%
\pgfpathlineto{\pgfqpoint{0.716096in}{1.665648in}}%
\pgfpathlineto{\pgfqpoint{0.720516in}{1.659967in}}%
\pgfpathlineto{\pgfqpoint{0.726160in}{1.652036in}}%
\pgfpathlineto{\pgfqpoint{0.736173in}{1.641197in}}%
\pgfpathlineto{\pgfqpoint{0.738588in}{1.638425in}}%
\pgfpathlineto{\pgfqpoint{0.751829in}{1.625476in}}%
\pgfpathlineto{\pgfqpoint{0.752488in}{1.624814in}}%
\pgfpathlineto{\pgfqpoint{0.767368in}{1.611203in}}%
\pgfpathlineto{\pgfqpoint{0.767486in}{1.611100in}}%
\pgfpathlineto{\pgfqpoint{0.783024in}{1.597592in}}%
\pgfpathlineto{\pgfqpoint{0.783142in}{1.597490in}}%
\pgfpathlineto{\pgfqpoint{0.798799in}{1.584554in}}%
\pgfpathlineto{\pgfqpoint{0.799561in}{1.583981in}}%
\pgfpathlineto{\pgfqpoint{0.814455in}{1.572470in}}%
\pgfpathlineto{\pgfqpoint{0.817643in}{1.570370in}}%
\pgfpathlineto{\pgfqpoint{0.830112in}{1.561665in}}%
\pgfpathlineto{\pgfqpoint{0.839234in}{1.556759in}}%
\pgfpathlineto{\pgfqpoint{0.845769in}{1.552916in}}%
\pgfpathclose%
\pgfpathmoveto{\pgfqpoint{1.362435in}{1.552916in}}%
\pgfpathlineto{\pgfqpoint{1.378092in}{1.547113in}}%
\pgfpathlineto{\pgfqpoint{1.393748in}{1.545115in}}%
\pgfpathlineto{\pgfqpoint{1.409405in}{1.547113in}}%
\pgfpathlineto{\pgfqpoint{1.425061in}{1.552916in}}%
\pgfpathlineto{\pgfqpoint{1.431596in}{1.556759in}}%
\pgfpathlineto{\pgfqpoint{1.440718in}{1.561665in}}%
\pgfpathlineto{\pgfqpoint{1.453187in}{1.570370in}}%
\pgfpathlineto{\pgfqpoint{1.456375in}{1.572470in}}%
\pgfpathlineto{\pgfqpoint{1.471269in}{1.583981in}}%
\pgfpathlineto{\pgfqpoint{1.472031in}{1.584554in}}%
\pgfpathlineto{\pgfqpoint{1.487688in}{1.597490in}}%
\pgfpathlineto{\pgfqpoint{1.487806in}{1.597592in}}%
\pgfpathlineto{\pgfqpoint{1.503344in}{1.611100in}}%
\pgfpathlineto{\pgfqpoint{1.503462in}{1.611203in}}%
\pgfpathlineto{\pgfqpoint{1.518342in}{1.624814in}}%
\pgfpathlineto{\pgfqpoint{1.519001in}{1.625476in}}%
\pgfpathlineto{\pgfqpoint{1.532242in}{1.638425in}}%
\pgfpathlineto{\pgfqpoint{1.534657in}{1.641197in}}%
\pgfpathlineto{\pgfqpoint{1.544670in}{1.652036in}}%
\pgfpathlineto{\pgfqpoint{1.550314in}{1.659967in}}%
\pgfpathlineto{\pgfqpoint{1.554734in}{1.665648in}}%
\pgfpathlineto{\pgfqpoint{1.561410in}{1.679259in}}%
\pgfpathlineto{\pgfqpoint{1.563707in}{1.692870in}}%
\pgfpathlineto{\pgfqpoint{1.550314in}{1.692870in}}%
\pgfpathlineto{\pgfqpoint{1.534657in}{1.692870in}}%
\pgfpathlineto{\pgfqpoint{1.519001in}{1.692870in}}%
\pgfpathlineto{\pgfqpoint{1.510980in}{1.692870in}}%
\pgfpathlineto{\pgfqpoint{1.509126in}{1.679259in}}%
\pgfpathlineto{\pgfqpoint{1.503742in}{1.665648in}}%
\pgfpathlineto{\pgfqpoint{1.503344in}{1.665020in}}%
\pgfpathlineto{\pgfqpoint{1.494928in}{1.652036in}}%
\pgfpathlineto{\pgfqpoint{1.487688in}{1.643396in}}%
\pgfpathlineto{\pgfqpoint{1.483280in}{1.638425in}}%
\pgfpathlineto{\pgfqpoint{1.472031in}{1.627621in}}%
\pgfpathlineto{\pgfqpoint{1.468802in}{1.624814in}}%
\pgfpathlineto{\pgfqpoint{1.456375in}{1.615035in}}%
\pgfpathlineto{\pgfqpoint{1.450657in}{1.611203in}}%
\pgfpathlineto{\pgfqpoint{1.440718in}{1.604908in}}%
\pgfpathlineto{\pgfqpoint{1.425783in}{1.597592in}}%
\pgfpathlineto{\pgfqpoint{1.425061in}{1.597246in}}%
\pgfpathlineto{\pgfqpoint{1.409405in}{1.592565in}}%
\pgfpathlineto{\pgfqpoint{1.393748in}{1.590954in}}%
\pgfpathlineto{\pgfqpoint{1.378092in}{1.592565in}}%
\pgfpathlineto{\pgfqpoint{1.362435in}{1.597246in}}%
\pgfpathlineto{\pgfqpoint{1.361713in}{1.597592in}}%
\pgfpathlineto{\pgfqpoint{1.346779in}{1.604908in}}%
\pgfpathlineto{\pgfqpoint{1.336840in}{1.611203in}}%
\pgfpathlineto{\pgfqpoint{1.331122in}{1.615035in}}%
\pgfpathlineto{\pgfqpoint{1.318694in}{1.624814in}}%
\pgfpathlineto{\pgfqpoint{1.315466in}{1.627621in}}%
\pgfpathlineto{\pgfqpoint{1.304217in}{1.638425in}}%
\pgfpathlineto{\pgfqpoint{1.299809in}{1.643396in}}%
\pgfpathlineto{\pgfqpoint{1.292568in}{1.652036in}}%
\pgfpathlineto{\pgfqpoint{1.284152in}{1.665020in}}%
\pgfpathlineto{\pgfqpoint{1.283755in}{1.665648in}}%
\pgfpathlineto{\pgfqpoint{1.278371in}{1.679259in}}%
\pgfpathlineto{\pgfqpoint{1.276517in}{1.692870in}}%
\pgfpathlineto{\pgfqpoint{1.268496in}{1.692870in}}%
\pgfpathlineto{\pgfqpoint{1.252839in}{1.692870in}}%
\pgfpathlineto{\pgfqpoint{1.237183in}{1.692870in}}%
\pgfpathlineto{\pgfqpoint{1.223789in}{1.692870in}}%
\pgfpathlineto{\pgfqpoint{1.226087in}{1.679259in}}%
\pgfpathlineto{\pgfqpoint{1.232763in}{1.665648in}}%
\pgfpathlineto{\pgfqpoint{1.237183in}{1.659967in}}%
\pgfpathlineto{\pgfqpoint{1.242827in}{1.652036in}}%
\pgfpathlineto{\pgfqpoint{1.252839in}{1.641197in}}%
\pgfpathlineto{\pgfqpoint{1.255255in}{1.638425in}}%
\pgfpathlineto{\pgfqpoint{1.268496in}{1.625476in}}%
\pgfpathlineto{\pgfqpoint{1.269155in}{1.624814in}}%
\pgfpathlineto{\pgfqpoint{1.284035in}{1.611203in}}%
\pgfpathlineto{\pgfqpoint{1.284152in}{1.611100in}}%
\pgfpathlineto{\pgfqpoint{1.299691in}{1.597592in}}%
\pgfpathlineto{\pgfqpoint{1.299809in}{1.597490in}}%
\pgfpathlineto{\pgfqpoint{1.315466in}{1.584554in}}%
\pgfpathlineto{\pgfqpoint{1.316227in}{1.583981in}}%
\pgfpathlineto{\pgfqpoint{1.331122in}{1.572470in}}%
\pgfpathlineto{\pgfqpoint{1.334310in}{1.570370in}}%
\pgfpathlineto{\pgfqpoint{1.346779in}{1.561665in}}%
\pgfpathlineto{\pgfqpoint{1.355901in}{1.556759in}}%
\pgfpathlineto{\pgfqpoint{1.362435in}{1.552916in}}%
\pgfpathclose%
\pgfpathmoveto{\pgfqpoint{1.879102in}{1.552916in}}%
\pgfpathlineto{\pgfqpoint{1.894758in}{1.547113in}}%
\pgfpathlineto{\pgfqpoint{1.910415in}{1.545115in}}%
\pgfpathlineto{\pgfqpoint{1.910415in}{1.556759in}}%
\pgfpathlineto{\pgfqpoint{1.910415in}{1.570370in}}%
\pgfpathlineto{\pgfqpoint{1.910415in}{1.583981in}}%
\pgfpathlineto{\pgfqpoint{1.910415in}{1.590954in}}%
\pgfpathlineto{\pgfqpoint{1.894758in}{1.592565in}}%
\pgfpathlineto{\pgfqpoint{1.879102in}{1.597246in}}%
\pgfpathlineto{\pgfqpoint{1.878380in}{1.597592in}}%
\pgfpathlineto{\pgfqpoint{1.863445in}{1.604908in}}%
\pgfpathlineto{\pgfqpoint{1.853506in}{1.611203in}}%
\pgfpathlineto{\pgfqpoint{1.847789in}{1.615035in}}%
\pgfpathlineto{\pgfqpoint{1.835361in}{1.624814in}}%
\pgfpathlineto{\pgfqpoint{1.832132in}{1.627621in}}%
\pgfpathlineto{\pgfqpoint{1.820884in}{1.638425in}}%
\pgfpathlineto{\pgfqpoint{1.816476in}{1.643396in}}%
\pgfpathlineto{\pgfqpoint{1.809235in}{1.652036in}}%
\pgfpathlineto{\pgfqpoint{1.800819in}{1.665020in}}%
\pgfpathlineto{\pgfqpoint{1.800422in}{1.665648in}}%
\pgfpathlineto{\pgfqpoint{1.795037in}{1.679259in}}%
\pgfpathlineto{\pgfqpoint{1.793184in}{1.692870in}}%
\pgfpathlineto{\pgfqpoint{1.785162in}{1.692870in}}%
\pgfpathlineto{\pgfqpoint{1.769506in}{1.692870in}}%
\pgfpathlineto{\pgfqpoint{1.753849in}{1.692870in}}%
\pgfpathlineto{\pgfqpoint{1.740456in}{1.692870in}}%
\pgfpathlineto{\pgfqpoint{1.742754in}{1.679259in}}%
\pgfpathlineto{\pgfqpoint{1.749429in}{1.665648in}}%
\pgfpathlineto{\pgfqpoint{1.753849in}{1.659967in}}%
\pgfpathlineto{\pgfqpoint{1.759494in}{1.652036in}}%
\pgfpathlineto{\pgfqpoint{1.769506in}{1.641197in}}%
\pgfpathlineto{\pgfqpoint{1.771922in}{1.638425in}}%
\pgfpathlineto{\pgfqpoint{1.785162in}{1.625476in}}%
\pgfpathlineto{\pgfqpoint{1.785822in}{1.624814in}}%
\pgfpathlineto{\pgfqpoint{1.800701in}{1.611203in}}%
\pgfpathlineto{\pgfqpoint{1.800819in}{1.611100in}}%
\pgfpathlineto{\pgfqpoint{1.816358in}{1.597592in}}%
\pgfpathlineto{\pgfqpoint{1.816476in}{1.597490in}}%
\pgfpathlineto{\pgfqpoint{1.832132in}{1.584554in}}%
\pgfpathlineto{\pgfqpoint{1.832894in}{1.583981in}}%
\pgfpathlineto{\pgfqpoint{1.847789in}{1.572470in}}%
\pgfpathlineto{\pgfqpoint{1.850977in}{1.570370in}}%
\pgfpathlineto{\pgfqpoint{1.863445in}{1.561665in}}%
\pgfpathlineto{\pgfqpoint{1.872568in}{1.556759in}}%
\pgfpathlineto{\pgfqpoint{1.879102in}{1.552916in}}%
\pgfpathclose%
\pgfusepath{fill}%
\end{pgfscope}%
\begin{pgfscope}%
\pgfpathrectangle{\pgfqpoint{0.360415in}{0.345370in}}{\pgfqpoint{1.550000in}{1.347500in}}%
\pgfusepath{clip}%
\pgfsetbuttcap%
\pgfsetroundjoin%
\definecolor{currentfill}{rgb}{0.362553,0.003243,0.649245}%
\pgfsetfillcolor{currentfill}%
\pgfsetlinewidth{0.000000pt}%
\definecolor{currentstroke}{rgb}{0.000000,0.000000,0.000000}%
\pgfsetstrokecolor{currentstroke}%
\pgfsetdash{}{0pt}%
\pgfpathmoveto{\pgfqpoint{0.423041in}{0.345370in}}%
\pgfpathlineto{\pgfqpoint{0.438698in}{0.345370in}}%
\pgfpathlineto{\pgfqpoint{0.454354in}{0.345370in}}%
\pgfpathlineto{\pgfqpoint{0.470011in}{0.345370in}}%
\pgfpathlineto{\pgfqpoint{0.477646in}{0.345370in}}%
\pgfpathlineto{\pgfqpoint{0.475793in}{0.358981in}}%
\pgfpathlineto{\pgfqpoint{0.470408in}{0.372592in}}%
\pgfpathlineto{\pgfqpoint{0.470011in}{0.373220in}}%
\pgfpathlineto{\pgfqpoint{0.461595in}{0.386203in}}%
\pgfpathlineto{\pgfqpoint{0.454354in}{0.394843in}}%
\pgfpathlineto{\pgfqpoint{0.449946in}{0.399814in}}%
\pgfpathlineto{\pgfqpoint{0.438698in}{0.410618in}}%
\pgfpathlineto{\pgfqpoint{0.435469in}{0.413425in}}%
\pgfpathlineto{\pgfqpoint{0.423041in}{0.423204in}}%
\pgfpathlineto{\pgfqpoint{0.417324in}{0.427036in}}%
\pgfpathlineto{\pgfqpoint{0.407385in}{0.433331in}}%
\pgfpathlineto{\pgfqpoint{0.392450in}{0.440648in}}%
\pgfpathlineto{\pgfqpoint{0.391728in}{0.440993in}}%
\pgfpathlineto{\pgfqpoint{0.376072in}{0.445674in}}%
\pgfpathlineto{\pgfqpoint{0.360415in}{0.447285in}}%
\pgfpathlineto{\pgfqpoint{0.360415in}{0.440648in}}%
\pgfpathlineto{\pgfqpoint{0.360415in}{0.427036in}}%
\pgfpathlineto{\pgfqpoint{0.360415in}{0.413425in}}%
\pgfpathlineto{\pgfqpoint{0.360415in}{0.399814in}}%
\pgfpathlineto{\pgfqpoint{0.360415in}{0.395024in}}%
\pgfpathlineto{\pgfqpoint{0.376072in}{0.392805in}}%
\pgfpathlineto{\pgfqpoint{0.391728in}{0.386357in}}%
\pgfpathlineto{\pgfqpoint{0.391962in}{0.386203in}}%
\pgfpathlineto{\pgfqpoint{0.407385in}{0.372795in}}%
\pgfpathlineto{\pgfqpoint{0.407562in}{0.372592in}}%
\pgfpathlineto{\pgfqpoint{0.414979in}{0.358981in}}%
\pgfpathlineto{\pgfqpoint{0.417532in}{0.345370in}}%
\pgfpathlineto{\pgfqpoint{0.423041in}{0.345370in}}%
\pgfpathclose%
\pgfpathmoveto{\pgfqpoint{0.767486in}{0.345370in}}%
\pgfpathlineto{\pgfqpoint{0.783142in}{0.345370in}}%
\pgfpathlineto{\pgfqpoint{0.798799in}{0.345370in}}%
\pgfpathlineto{\pgfqpoint{0.814455in}{0.345370in}}%
\pgfpathlineto{\pgfqpoint{0.819965in}{0.345370in}}%
\pgfpathlineto{\pgfqpoint{0.822518in}{0.358981in}}%
\pgfpathlineto{\pgfqpoint{0.829934in}{0.372592in}}%
\pgfpathlineto{\pgfqpoint{0.830112in}{0.372795in}}%
\pgfpathlineto{\pgfqpoint{0.845535in}{0.386203in}}%
\pgfpathlineto{\pgfqpoint{0.845769in}{0.386357in}}%
\pgfpathlineto{\pgfqpoint{0.861425in}{0.392805in}}%
\pgfpathlineto{\pgfqpoint{0.877082in}{0.395024in}}%
\pgfpathlineto{\pgfqpoint{0.892738in}{0.392805in}}%
\pgfpathlineto{\pgfqpoint{0.908395in}{0.386357in}}%
\pgfpathlineto{\pgfqpoint{0.908629in}{0.386203in}}%
\pgfpathlineto{\pgfqpoint{0.924051in}{0.372795in}}%
\pgfpathlineto{\pgfqpoint{0.924229in}{0.372592in}}%
\pgfpathlineto{\pgfqpoint{0.931645in}{0.358981in}}%
\pgfpathlineto{\pgfqpoint{0.934198in}{0.345370in}}%
\pgfpathlineto{\pgfqpoint{0.939708in}{0.345370in}}%
\pgfpathlineto{\pgfqpoint{0.955364in}{0.345370in}}%
\pgfpathlineto{\pgfqpoint{0.971021in}{0.345370in}}%
\pgfpathlineto{\pgfqpoint{0.986678in}{0.345370in}}%
\pgfpathlineto{\pgfqpoint{0.994313in}{0.345370in}}%
\pgfpathlineto{\pgfqpoint{0.992459in}{0.358981in}}%
\pgfpathlineto{\pgfqpoint{0.987075in}{0.372592in}}%
\pgfpathlineto{\pgfqpoint{0.986678in}{0.373220in}}%
\pgfpathlineto{\pgfqpoint{0.978262in}{0.386203in}}%
\pgfpathlineto{\pgfqpoint{0.971021in}{0.394843in}}%
\pgfpathlineto{\pgfqpoint{0.966613in}{0.399814in}}%
\pgfpathlineto{\pgfqpoint{0.955364in}{0.410618in}}%
\pgfpathlineto{\pgfqpoint{0.952136in}{0.413425in}}%
\pgfpathlineto{\pgfqpoint{0.939708in}{0.423204in}}%
\pgfpathlineto{\pgfqpoint{0.933990in}{0.427036in}}%
\pgfpathlineto{\pgfqpoint{0.924051in}{0.433331in}}%
\pgfpathlineto{\pgfqpoint{0.909117in}{0.440648in}}%
\pgfpathlineto{\pgfqpoint{0.908395in}{0.440993in}}%
\pgfpathlineto{\pgfqpoint{0.892738in}{0.445674in}}%
\pgfpathlineto{\pgfqpoint{0.877082in}{0.447285in}}%
\pgfpathlineto{\pgfqpoint{0.861425in}{0.445674in}}%
\pgfpathlineto{\pgfqpoint{0.845769in}{0.440993in}}%
\pgfpathlineto{\pgfqpoint{0.845047in}{0.440648in}}%
\pgfpathlineto{\pgfqpoint{0.830112in}{0.433331in}}%
\pgfpathlineto{\pgfqpoint{0.820173in}{0.427036in}}%
\pgfpathlineto{\pgfqpoint{0.814455in}{0.423204in}}%
\pgfpathlineto{\pgfqpoint{0.802028in}{0.413425in}}%
\pgfpathlineto{\pgfqpoint{0.798799in}{0.410618in}}%
\pgfpathlineto{\pgfqpoint{0.787550in}{0.399814in}}%
\pgfpathlineto{\pgfqpoint{0.783142in}{0.394843in}}%
\pgfpathlineto{\pgfqpoint{0.775902in}{0.386203in}}%
\pgfpathlineto{\pgfqpoint{0.767486in}{0.373220in}}%
\pgfpathlineto{\pgfqpoint{0.767088in}{0.372592in}}%
\pgfpathlineto{\pgfqpoint{0.761704in}{0.358981in}}%
\pgfpathlineto{\pgfqpoint{0.759850in}{0.345370in}}%
\pgfpathlineto{\pgfqpoint{0.767486in}{0.345370in}}%
\pgfpathclose%
\pgfpathmoveto{\pgfqpoint{1.284152in}{0.345370in}}%
\pgfpathlineto{\pgfqpoint{1.299809in}{0.345370in}}%
\pgfpathlineto{\pgfqpoint{1.315466in}{0.345370in}}%
\pgfpathlineto{\pgfqpoint{1.331122in}{0.345370in}}%
\pgfpathlineto{\pgfqpoint{1.336632in}{0.345370in}}%
\pgfpathlineto{\pgfqpoint{1.339185in}{0.358981in}}%
\pgfpathlineto{\pgfqpoint{1.346601in}{0.372592in}}%
\pgfpathlineto{\pgfqpoint{1.346779in}{0.372795in}}%
\pgfpathlineto{\pgfqpoint{1.362201in}{0.386203in}}%
\pgfpathlineto{\pgfqpoint{1.362435in}{0.386357in}}%
\pgfpathlineto{\pgfqpoint{1.378092in}{0.392805in}}%
\pgfpathlineto{\pgfqpoint{1.393748in}{0.395024in}}%
\pgfpathlineto{\pgfqpoint{1.409405in}{0.392805in}}%
\pgfpathlineto{\pgfqpoint{1.425061in}{0.386357in}}%
\pgfpathlineto{\pgfqpoint{1.425295in}{0.386203in}}%
\pgfpathlineto{\pgfqpoint{1.440718in}{0.372795in}}%
\pgfpathlineto{\pgfqpoint{1.440896in}{0.372592in}}%
\pgfpathlineto{\pgfqpoint{1.448312in}{0.358981in}}%
\pgfpathlineto{\pgfqpoint{1.450865in}{0.345370in}}%
\pgfpathlineto{\pgfqpoint{1.456375in}{0.345370in}}%
\pgfpathlineto{\pgfqpoint{1.472031in}{0.345370in}}%
\pgfpathlineto{\pgfqpoint{1.487688in}{0.345370in}}%
\pgfpathlineto{\pgfqpoint{1.503344in}{0.345370in}}%
\pgfpathlineto{\pgfqpoint{1.510980in}{0.345370in}}%
\pgfpathlineto{\pgfqpoint{1.509126in}{0.358981in}}%
\pgfpathlineto{\pgfqpoint{1.503742in}{0.372592in}}%
\pgfpathlineto{\pgfqpoint{1.503344in}{0.373220in}}%
\pgfpathlineto{\pgfqpoint{1.494928in}{0.386203in}}%
\pgfpathlineto{\pgfqpoint{1.487688in}{0.394843in}}%
\pgfpathlineto{\pgfqpoint{1.483280in}{0.399814in}}%
\pgfpathlineto{\pgfqpoint{1.472031in}{0.410618in}}%
\pgfpathlineto{\pgfqpoint{1.468802in}{0.413425in}}%
\pgfpathlineto{\pgfqpoint{1.456375in}{0.423204in}}%
\pgfpathlineto{\pgfqpoint{1.450657in}{0.427036in}}%
\pgfpathlineto{\pgfqpoint{1.440718in}{0.433331in}}%
\pgfpathlineto{\pgfqpoint{1.425783in}{0.440648in}}%
\pgfpathlineto{\pgfqpoint{1.425061in}{0.440993in}}%
\pgfpathlineto{\pgfqpoint{1.409405in}{0.445674in}}%
\pgfpathlineto{\pgfqpoint{1.393748in}{0.447285in}}%
\pgfpathlineto{\pgfqpoint{1.378092in}{0.445674in}}%
\pgfpathlineto{\pgfqpoint{1.362435in}{0.440993in}}%
\pgfpathlineto{\pgfqpoint{1.361713in}{0.440648in}}%
\pgfpathlineto{\pgfqpoint{1.346779in}{0.433331in}}%
\pgfpathlineto{\pgfqpoint{1.336840in}{0.427036in}}%
\pgfpathlineto{\pgfqpoint{1.331122in}{0.423204in}}%
\pgfpathlineto{\pgfqpoint{1.318694in}{0.413425in}}%
\pgfpathlineto{\pgfqpoint{1.315466in}{0.410618in}}%
\pgfpathlineto{\pgfqpoint{1.304217in}{0.399814in}}%
\pgfpathlineto{\pgfqpoint{1.299809in}{0.394843in}}%
\pgfpathlineto{\pgfqpoint{1.292568in}{0.386203in}}%
\pgfpathlineto{\pgfqpoint{1.284152in}{0.373220in}}%
\pgfpathlineto{\pgfqpoint{1.283755in}{0.372592in}}%
\pgfpathlineto{\pgfqpoint{1.278371in}{0.358981in}}%
\pgfpathlineto{\pgfqpoint{1.276517in}{0.345370in}}%
\pgfpathlineto{\pgfqpoint{1.284152in}{0.345370in}}%
\pgfpathclose%
\pgfpathmoveto{\pgfqpoint{1.800819in}{0.345370in}}%
\pgfpathlineto{\pgfqpoint{1.816476in}{0.345370in}}%
\pgfpathlineto{\pgfqpoint{1.832132in}{0.345370in}}%
\pgfpathlineto{\pgfqpoint{1.847789in}{0.345370in}}%
\pgfpathlineto{\pgfqpoint{1.853298in}{0.345370in}}%
\pgfpathlineto{\pgfqpoint{1.855851in}{0.358981in}}%
\pgfpathlineto{\pgfqpoint{1.863268in}{0.372592in}}%
\pgfpathlineto{\pgfqpoint{1.863445in}{0.372795in}}%
\pgfpathlineto{\pgfqpoint{1.878868in}{0.386203in}}%
\pgfpathlineto{\pgfqpoint{1.879102in}{0.386357in}}%
\pgfpathlineto{\pgfqpoint{1.894758in}{0.392805in}}%
\pgfpathlineto{\pgfqpoint{1.910415in}{0.395024in}}%
\pgfpathlineto{\pgfqpoint{1.910415in}{0.399814in}}%
\pgfpathlineto{\pgfqpoint{1.910415in}{0.413425in}}%
\pgfpathlineto{\pgfqpoint{1.910415in}{0.427036in}}%
\pgfpathlineto{\pgfqpoint{1.910415in}{0.440648in}}%
\pgfpathlineto{\pgfqpoint{1.910415in}{0.447285in}}%
\pgfpathlineto{\pgfqpoint{1.894758in}{0.445674in}}%
\pgfpathlineto{\pgfqpoint{1.879102in}{0.440993in}}%
\pgfpathlineto{\pgfqpoint{1.878380in}{0.440648in}}%
\pgfpathlineto{\pgfqpoint{1.863445in}{0.433331in}}%
\pgfpathlineto{\pgfqpoint{1.853506in}{0.427036in}}%
\pgfpathlineto{\pgfqpoint{1.847789in}{0.423204in}}%
\pgfpathlineto{\pgfqpoint{1.835361in}{0.413425in}}%
\pgfpathlineto{\pgfqpoint{1.832132in}{0.410618in}}%
\pgfpathlineto{\pgfqpoint{1.820884in}{0.399814in}}%
\pgfpathlineto{\pgfqpoint{1.816476in}{0.394843in}}%
\pgfpathlineto{\pgfqpoint{1.809235in}{0.386203in}}%
\pgfpathlineto{\pgfqpoint{1.800819in}{0.373220in}}%
\pgfpathlineto{\pgfqpoint{1.800422in}{0.372592in}}%
\pgfpathlineto{\pgfqpoint{1.795037in}{0.358981in}}%
\pgfpathlineto{\pgfqpoint{1.793184in}{0.345370in}}%
\pgfpathlineto{\pgfqpoint{1.800819in}{0.345370in}}%
\pgfpathclose%
\pgfpathmoveto{\pgfqpoint{0.376072in}{0.694232in}}%
\pgfpathlineto{\pgfqpoint{0.391728in}{0.698913in}}%
\pgfpathlineto{\pgfqpoint{0.392450in}{0.699259in}}%
\pgfpathlineto{\pgfqpoint{0.407385in}{0.706575in}}%
\pgfpathlineto{\pgfqpoint{0.417324in}{0.712870in}}%
\pgfpathlineto{\pgfqpoint{0.423041in}{0.716702in}}%
\pgfpathlineto{\pgfqpoint{0.435469in}{0.726481in}}%
\pgfpathlineto{\pgfqpoint{0.438698in}{0.729288in}}%
\pgfpathlineto{\pgfqpoint{0.449946in}{0.740092in}}%
\pgfpathlineto{\pgfqpoint{0.454354in}{0.745063in}}%
\pgfpathlineto{\pgfqpoint{0.461595in}{0.753703in}}%
\pgfpathlineto{\pgfqpoint{0.470011in}{0.766687in}}%
\pgfpathlineto{\pgfqpoint{0.470408in}{0.767314in}}%
\pgfpathlineto{\pgfqpoint{0.475793in}{0.780925in}}%
\pgfpathlineto{\pgfqpoint{0.477646in}{0.794536in}}%
\pgfpathlineto{\pgfqpoint{0.475793in}{0.808148in}}%
\pgfpathlineto{\pgfqpoint{0.470408in}{0.821759in}}%
\pgfpathlineto{\pgfqpoint{0.470011in}{0.822386in}}%
\pgfpathlineto{\pgfqpoint{0.461595in}{0.835370in}}%
\pgfpathlineto{\pgfqpoint{0.454354in}{0.844010in}}%
\pgfpathlineto{\pgfqpoint{0.449946in}{0.848981in}}%
\pgfpathlineto{\pgfqpoint{0.438698in}{0.859785in}}%
\pgfpathlineto{\pgfqpoint{0.435469in}{0.862592in}}%
\pgfpathlineto{\pgfqpoint{0.423041in}{0.872371in}}%
\pgfpathlineto{\pgfqpoint{0.417324in}{0.876203in}}%
\pgfpathlineto{\pgfqpoint{0.407385in}{0.882498in}}%
\pgfpathlineto{\pgfqpoint{0.392450in}{0.889814in}}%
\pgfpathlineto{\pgfqpoint{0.391728in}{0.890160in}}%
\pgfpathlineto{\pgfqpoint{0.376072in}{0.894841in}}%
\pgfpathlineto{\pgfqpoint{0.360415in}{0.896452in}}%
\pgfpathlineto{\pgfqpoint{0.360415in}{0.889814in}}%
\pgfpathlineto{\pgfqpoint{0.360415in}{0.876203in}}%
\pgfpathlineto{\pgfqpoint{0.360415in}{0.862592in}}%
\pgfpathlineto{\pgfqpoint{0.360415in}{0.848981in}}%
\pgfpathlineto{\pgfqpoint{0.360415in}{0.844191in}}%
\pgfpathlineto{\pgfqpoint{0.376072in}{0.841972in}}%
\pgfpathlineto{\pgfqpoint{0.391728in}{0.835524in}}%
\pgfpathlineto{\pgfqpoint{0.391962in}{0.835370in}}%
\pgfpathlineto{\pgfqpoint{0.407385in}{0.821962in}}%
\pgfpathlineto{\pgfqpoint{0.407562in}{0.821759in}}%
\pgfpathlineto{\pgfqpoint{0.414979in}{0.808148in}}%
\pgfpathlineto{\pgfqpoint{0.417532in}{0.794536in}}%
\pgfpathlineto{\pgfqpoint{0.414979in}{0.780925in}}%
\pgfpathlineto{\pgfqpoint{0.407562in}{0.767314in}}%
\pgfpathlineto{\pgfqpoint{0.407385in}{0.767111in}}%
\pgfpathlineto{\pgfqpoint{0.391962in}{0.753703in}}%
\pgfpathlineto{\pgfqpoint{0.391728in}{0.753549in}}%
\pgfpathlineto{\pgfqpoint{0.376072in}{0.747101in}}%
\pgfpathlineto{\pgfqpoint{0.360415in}{0.744882in}}%
\pgfpathlineto{\pgfqpoint{0.360415in}{0.740092in}}%
\pgfpathlineto{\pgfqpoint{0.360415in}{0.726481in}}%
\pgfpathlineto{\pgfqpoint{0.360415in}{0.712870in}}%
\pgfpathlineto{\pgfqpoint{0.360415in}{0.699259in}}%
\pgfpathlineto{\pgfqpoint{0.360415in}{0.692621in}}%
\pgfpathlineto{\pgfqpoint{0.376072in}{0.694232in}}%
\pgfpathclose%
\pgfpathmoveto{\pgfqpoint{0.845769in}{0.698913in}}%
\pgfpathlineto{\pgfqpoint{0.861425in}{0.694232in}}%
\pgfpathlineto{\pgfqpoint{0.877082in}{0.692621in}}%
\pgfpathlineto{\pgfqpoint{0.892738in}{0.694232in}}%
\pgfpathlineto{\pgfqpoint{0.908395in}{0.698913in}}%
\pgfpathlineto{\pgfqpoint{0.909117in}{0.699259in}}%
\pgfpathlineto{\pgfqpoint{0.924051in}{0.706575in}}%
\pgfpathlineto{\pgfqpoint{0.933990in}{0.712870in}}%
\pgfpathlineto{\pgfqpoint{0.939708in}{0.716702in}}%
\pgfpathlineto{\pgfqpoint{0.952136in}{0.726481in}}%
\pgfpathlineto{\pgfqpoint{0.955364in}{0.729288in}}%
\pgfpathlineto{\pgfqpoint{0.966613in}{0.740092in}}%
\pgfpathlineto{\pgfqpoint{0.971021in}{0.745063in}}%
\pgfpathlineto{\pgfqpoint{0.978262in}{0.753703in}}%
\pgfpathlineto{\pgfqpoint{0.986678in}{0.766687in}}%
\pgfpathlineto{\pgfqpoint{0.987075in}{0.767314in}}%
\pgfpathlineto{\pgfqpoint{0.992459in}{0.780925in}}%
\pgfpathlineto{\pgfqpoint{0.994313in}{0.794536in}}%
\pgfpathlineto{\pgfqpoint{0.992459in}{0.808148in}}%
\pgfpathlineto{\pgfqpoint{0.987075in}{0.821759in}}%
\pgfpathlineto{\pgfqpoint{0.986678in}{0.822386in}}%
\pgfpathlineto{\pgfqpoint{0.978262in}{0.835370in}}%
\pgfpathlineto{\pgfqpoint{0.971021in}{0.844010in}}%
\pgfpathlineto{\pgfqpoint{0.966613in}{0.848981in}}%
\pgfpathlineto{\pgfqpoint{0.955364in}{0.859785in}}%
\pgfpathlineto{\pgfqpoint{0.952136in}{0.862592in}}%
\pgfpathlineto{\pgfqpoint{0.939708in}{0.872371in}}%
\pgfpathlineto{\pgfqpoint{0.933990in}{0.876203in}}%
\pgfpathlineto{\pgfqpoint{0.924051in}{0.882498in}}%
\pgfpathlineto{\pgfqpoint{0.909117in}{0.889814in}}%
\pgfpathlineto{\pgfqpoint{0.908395in}{0.890160in}}%
\pgfpathlineto{\pgfqpoint{0.892738in}{0.894841in}}%
\pgfpathlineto{\pgfqpoint{0.877082in}{0.896452in}}%
\pgfpathlineto{\pgfqpoint{0.861425in}{0.894841in}}%
\pgfpathlineto{\pgfqpoint{0.845769in}{0.890160in}}%
\pgfpathlineto{\pgfqpoint{0.845047in}{0.889814in}}%
\pgfpathlineto{\pgfqpoint{0.830112in}{0.882498in}}%
\pgfpathlineto{\pgfqpoint{0.820173in}{0.876203in}}%
\pgfpathlineto{\pgfqpoint{0.814455in}{0.872371in}}%
\pgfpathlineto{\pgfqpoint{0.802028in}{0.862592in}}%
\pgfpathlineto{\pgfqpoint{0.798799in}{0.859785in}}%
\pgfpathlineto{\pgfqpoint{0.787550in}{0.848981in}}%
\pgfpathlineto{\pgfqpoint{0.783142in}{0.844010in}}%
\pgfpathlineto{\pgfqpoint{0.775902in}{0.835370in}}%
\pgfpathlineto{\pgfqpoint{0.767486in}{0.822386in}}%
\pgfpathlineto{\pgfqpoint{0.767088in}{0.821759in}}%
\pgfpathlineto{\pgfqpoint{0.761704in}{0.808148in}}%
\pgfpathlineto{\pgfqpoint{0.759850in}{0.794536in}}%
\pgfpathlineto{\pgfqpoint{0.761704in}{0.780925in}}%
\pgfpathlineto{\pgfqpoint{0.767088in}{0.767314in}}%
\pgfpathlineto{\pgfqpoint{0.767486in}{0.766687in}}%
\pgfpathlineto{\pgfqpoint{0.775902in}{0.753703in}}%
\pgfpathlineto{\pgfqpoint{0.783142in}{0.745063in}}%
\pgfpathlineto{\pgfqpoint{0.787550in}{0.740092in}}%
\pgfpathlineto{\pgfqpoint{0.798799in}{0.729288in}}%
\pgfpathlineto{\pgfqpoint{0.802028in}{0.726481in}}%
\pgfpathlineto{\pgfqpoint{0.814455in}{0.716702in}}%
\pgfpathlineto{\pgfqpoint{0.820173in}{0.712870in}}%
\pgfpathlineto{\pgfqpoint{0.830112in}{0.706575in}}%
\pgfpathlineto{\pgfqpoint{0.845047in}{0.699259in}}%
\pgfpathlineto{\pgfqpoint{0.845769in}{0.698913in}}%
\pgfpathclose%
\pgfpathmoveto{\pgfqpoint{0.845535in}{0.753703in}}%
\pgfpathlineto{\pgfqpoint{0.830112in}{0.767111in}}%
\pgfpathlineto{\pgfqpoint{0.829934in}{0.767314in}}%
\pgfpathlineto{\pgfqpoint{0.822518in}{0.780925in}}%
\pgfpathlineto{\pgfqpoint{0.819965in}{0.794536in}}%
\pgfpathlineto{\pgfqpoint{0.822518in}{0.808148in}}%
\pgfpathlineto{\pgfqpoint{0.829934in}{0.821759in}}%
\pgfpathlineto{\pgfqpoint{0.830112in}{0.821962in}}%
\pgfpathlineto{\pgfqpoint{0.845535in}{0.835370in}}%
\pgfpathlineto{\pgfqpoint{0.845769in}{0.835524in}}%
\pgfpathlineto{\pgfqpoint{0.861425in}{0.841972in}}%
\pgfpathlineto{\pgfqpoint{0.877082in}{0.844191in}}%
\pgfpathlineto{\pgfqpoint{0.892738in}{0.841972in}}%
\pgfpathlineto{\pgfqpoint{0.908395in}{0.835524in}}%
\pgfpathlineto{\pgfqpoint{0.908629in}{0.835370in}}%
\pgfpathlineto{\pgfqpoint{0.924051in}{0.821962in}}%
\pgfpathlineto{\pgfqpoint{0.924229in}{0.821759in}}%
\pgfpathlineto{\pgfqpoint{0.931645in}{0.808148in}}%
\pgfpathlineto{\pgfqpoint{0.934198in}{0.794536in}}%
\pgfpathlineto{\pgfqpoint{0.931645in}{0.780925in}}%
\pgfpathlineto{\pgfqpoint{0.924229in}{0.767314in}}%
\pgfpathlineto{\pgfqpoint{0.924051in}{0.767111in}}%
\pgfpathlineto{\pgfqpoint{0.908629in}{0.753703in}}%
\pgfpathlineto{\pgfqpoint{0.908395in}{0.753549in}}%
\pgfpathlineto{\pgfqpoint{0.892738in}{0.747101in}}%
\pgfpathlineto{\pgfqpoint{0.877082in}{0.744882in}}%
\pgfpathlineto{\pgfqpoint{0.861425in}{0.747101in}}%
\pgfpathlineto{\pgfqpoint{0.845769in}{0.753549in}}%
\pgfpathlineto{\pgfqpoint{0.845535in}{0.753703in}}%
\pgfpathclose%
\pgfpathmoveto{\pgfqpoint{1.362435in}{0.698913in}}%
\pgfpathlineto{\pgfqpoint{1.378092in}{0.694232in}}%
\pgfpathlineto{\pgfqpoint{1.393748in}{0.692621in}}%
\pgfpathlineto{\pgfqpoint{1.409405in}{0.694232in}}%
\pgfpathlineto{\pgfqpoint{1.425061in}{0.698913in}}%
\pgfpathlineto{\pgfqpoint{1.425783in}{0.699259in}}%
\pgfpathlineto{\pgfqpoint{1.440718in}{0.706575in}}%
\pgfpathlineto{\pgfqpoint{1.450657in}{0.712870in}}%
\pgfpathlineto{\pgfqpoint{1.456375in}{0.716702in}}%
\pgfpathlineto{\pgfqpoint{1.468802in}{0.726481in}}%
\pgfpathlineto{\pgfqpoint{1.472031in}{0.729288in}}%
\pgfpathlineto{\pgfqpoint{1.483280in}{0.740092in}}%
\pgfpathlineto{\pgfqpoint{1.487688in}{0.745063in}}%
\pgfpathlineto{\pgfqpoint{1.494928in}{0.753703in}}%
\pgfpathlineto{\pgfqpoint{1.503344in}{0.766687in}}%
\pgfpathlineto{\pgfqpoint{1.503742in}{0.767314in}}%
\pgfpathlineto{\pgfqpoint{1.509126in}{0.780925in}}%
\pgfpathlineto{\pgfqpoint{1.510980in}{0.794536in}}%
\pgfpathlineto{\pgfqpoint{1.509126in}{0.808148in}}%
\pgfpathlineto{\pgfqpoint{1.503742in}{0.821759in}}%
\pgfpathlineto{\pgfqpoint{1.503344in}{0.822386in}}%
\pgfpathlineto{\pgfqpoint{1.494928in}{0.835370in}}%
\pgfpathlineto{\pgfqpoint{1.487688in}{0.844010in}}%
\pgfpathlineto{\pgfqpoint{1.483280in}{0.848981in}}%
\pgfpathlineto{\pgfqpoint{1.472031in}{0.859785in}}%
\pgfpathlineto{\pgfqpoint{1.468802in}{0.862592in}}%
\pgfpathlineto{\pgfqpoint{1.456375in}{0.872371in}}%
\pgfpathlineto{\pgfqpoint{1.450657in}{0.876203in}}%
\pgfpathlineto{\pgfqpoint{1.440718in}{0.882498in}}%
\pgfpathlineto{\pgfqpoint{1.425783in}{0.889814in}}%
\pgfpathlineto{\pgfqpoint{1.425061in}{0.890160in}}%
\pgfpathlineto{\pgfqpoint{1.409405in}{0.894841in}}%
\pgfpathlineto{\pgfqpoint{1.393748in}{0.896452in}}%
\pgfpathlineto{\pgfqpoint{1.378092in}{0.894841in}}%
\pgfpathlineto{\pgfqpoint{1.362435in}{0.890160in}}%
\pgfpathlineto{\pgfqpoint{1.361713in}{0.889814in}}%
\pgfpathlineto{\pgfqpoint{1.346779in}{0.882498in}}%
\pgfpathlineto{\pgfqpoint{1.336840in}{0.876203in}}%
\pgfpathlineto{\pgfqpoint{1.331122in}{0.872371in}}%
\pgfpathlineto{\pgfqpoint{1.318694in}{0.862592in}}%
\pgfpathlineto{\pgfqpoint{1.315466in}{0.859785in}}%
\pgfpathlineto{\pgfqpoint{1.304217in}{0.848981in}}%
\pgfpathlineto{\pgfqpoint{1.299809in}{0.844010in}}%
\pgfpathlineto{\pgfqpoint{1.292568in}{0.835370in}}%
\pgfpathlineto{\pgfqpoint{1.284152in}{0.822386in}}%
\pgfpathlineto{\pgfqpoint{1.283755in}{0.821759in}}%
\pgfpathlineto{\pgfqpoint{1.278371in}{0.808148in}}%
\pgfpathlineto{\pgfqpoint{1.276517in}{0.794536in}}%
\pgfpathlineto{\pgfqpoint{1.278371in}{0.780925in}}%
\pgfpathlineto{\pgfqpoint{1.283755in}{0.767314in}}%
\pgfpathlineto{\pgfqpoint{1.284152in}{0.766687in}}%
\pgfpathlineto{\pgfqpoint{1.292568in}{0.753703in}}%
\pgfpathlineto{\pgfqpoint{1.299809in}{0.745063in}}%
\pgfpathlineto{\pgfqpoint{1.304217in}{0.740092in}}%
\pgfpathlineto{\pgfqpoint{1.315466in}{0.729288in}}%
\pgfpathlineto{\pgfqpoint{1.318694in}{0.726481in}}%
\pgfpathlineto{\pgfqpoint{1.331122in}{0.716702in}}%
\pgfpathlineto{\pgfqpoint{1.336840in}{0.712870in}}%
\pgfpathlineto{\pgfqpoint{1.346779in}{0.706575in}}%
\pgfpathlineto{\pgfqpoint{1.361713in}{0.699259in}}%
\pgfpathlineto{\pgfqpoint{1.362435in}{0.698913in}}%
\pgfpathclose%
\pgfpathmoveto{\pgfqpoint{1.362201in}{0.753703in}}%
\pgfpathlineto{\pgfqpoint{1.346779in}{0.767111in}}%
\pgfpathlineto{\pgfqpoint{1.346601in}{0.767314in}}%
\pgfpathlineto{\pgfqpoint{1.339185in}{0.780925in}}%
\pgfpathlineto{\pgfqpoint{1.336632in}{0.794536in}}%
\pgfpathlineto{\pgfqpoint{1.339185in}{0.808148in}}%
\pgfpathlineto{\pgfqpoint{1.346601in}{0.821759in}}%
\pgfpathlineto{\pgfqpoint{1.346779in}{0.821962in}}%
\pgfpathlineto{\pgfqpoint{1.362201in}{0.835370in}}%
\pgfpathlineto{\pgfqpoint{1.362435in}{0.835524in}}%
\pgfpathlineto{\pgfqpoint{1.378092in}{0.841972in}}%
\pgfpathlineto{\pgfqpoint{1.393748in}{0.844191in}}%
\pgfpathlineto{\pgfqpoint{1.409405in}{0.841972in}}%
\pgfpathlineto{\pgfqpoint{1.425061in}{0.835524in}}%
\pgfpathlineto{\pgfqpoint{1.425295in}{0.835370in}}%
\pgfpathlineto{\pgfqpoint{1.440718in}{0.821962in}}%
\pgfpathlineto{\pgfqpoint{1.440896in}{0.821759in}}%
\pgfpathlineto{\pgfqpoint{1.448312in}{0.808148in}}%
\pgfpathlineto{\pgfqpoint{1.450865in}{0.794536in}}%
\pgfpathlineto{\pgfqpoint{1.448312in}{0.780925in}}%
\pgfpathlineto{\pgfqpoint{1.440896in}{0.767314in}}%
\pgfpathlineto{\pgfqpoint{1.440718in}{0.767111in}}%
\pgfpathlineto{\pgfqpoint{1.425295in}{0.753703in}}%
\pgfpathlineto{\pgfqpoint{1.425061in}{0.753549in}}%
\pgfpathlineto{\pgfqpoint{1.409405in}{0.747101in}}%
\pgfpathlineto{\pgfqpoint{1.393748in}{0.744882in}}%
\pgfpathlineto{\pgfqpoint{1.378092in}{0.747101in}}%
\pgfpathlineto{\pgfqpoint{1.362435in}{0.753549in}}%
\pgfpathlineto{\pgfqpoint{1.362201in}{0.753703in}}%
\pgfpathclose%
\pgfpathmoveto{\pgfqpoint{1.879102in}{0.698913in}}%
\pgfpathlineto{\pgfqpoint{1.894758in}{0.694232in}}%
\pgfpathlineto{\pgfqpoint{1.910415in}{0.692621in}}%
\pgfpathlineto{\pgfqpoint{1.910415in}{0.699259in}}%
\pgfpathlineto{\pgfqpoint{1.910415in}{0.712870in}}%
\pgfpathlineto{\pgfqpoint{1.910415in}{0.726481in}}%
\pgfpathlineto{\pgfqpoint{1.910415in}{0.740092in}}%
\pgfpathlineto{\pgfqpoint{1.910415in}{0.744882in}}%
\pgfpathlineto{\pgfqpoint{1.894758in}{0.747101in}}%
\pgfpathlineto{\pgfqpoint{1.879102in}{0.753549in}}%
\pgfpathlineto{\pgfqpoint{1.878868in}{0.753703in}}%
\pgfpathlineto{\pgfqpoint{1.863445in}{0.767111in}}%
\pgfpathlineto{\pgfqpoint{1.863268in}{0.767314in}}%
\pgfpathlineto{\pgfqpoint{1.855851in}{0.780925in}}%
\pgfpathlineto{\pgfqpoint{1.853298in}{0.794536in}}%
\pgfpathlineto{\pgfqpoint{1.855851in}{0.808148in}}%
\pgfpathlineto{\pgfqpoint{1.863268in}{0.821759in}}%
\pgfpathlineto{\pgfqpoint{1.863445in}{0.821962in}}%
\pgfpathlineto{\pgfqpoint{1.878868in}{0.835370in}}%
\pgfpathlineto{\pgfqpoint{1.879102in}{0.835524in}}%
\pgfpathlineto{\pgfqpoint{1.894758in}{0.841972in}}%
\pgfpathlineto{\pgfqpoint{1.910415in}{0.844191in}}%
\pgfpathlineto{\pgfqpoint{1.910415in}{0.848981in}}%
\pgfpathlineto{\pgfqpoint{1.910415in}{0.862592in}}%
\pgfpathlineto{\pgfqpoint{1.910415in}{0.876203in}}%
\pgfpathlineto{\pgfqpoint{1.910415in}{0.889814in}}%
\pgfpathlineto{\pgfqpoint{1.910415in}{0.896452in}}%
\pgfpathlineto{\pgfqpoint{1.894758in}{0.894841in}}%
\pgfpathlineto{\pgfqpoint{1.879102in}{0.890160in}}%
\pgfpathlineto{\pgfqpoint{1.878380in}{0.889814in}}%
\pgfpathlineto{\pgfqpoint{1.863445in}{0.882498in}}%
\pgfpathlineto{\pgfqpoint{1.853506in}{0.876203in}}%
\pgfpathlineto{\pgfqpoint{1.847789in}{0.872371in}}%
\pgfpathlineto{\pgfqpoint{1.835361in}{0.862592in}}%
\pgfpathlineto{\pgfqpoint{1.832132in}{0.859785in}}%
\pgfpathlineto{\pgfqpoint{1.820884in}{0.848981in}}%
\pgfpathlineto{\pgfqpoint{1.816476in}{0.844010in}}%
\pgfpathlineto{\pgfqpoint{1.809235in}{0.835370in}}%
\pgfpathlineto{\pgfqpoint{1.800819in}{0.822386in}}%
\pgfpathlineto{\pgfqpoint{1.800422in}{0.821759in}}%
\pgfpathlineto{\pgfqpoint{1.795037in}{0.808148in}}%
\pgfpathlineto{\pgfqpoint{1.793184in}{0.794536in}}%
\pgfpathlineto{\pgfqpoint{1.795037in}{0.780925in}}%
\pgfpathlineto{\pgfqpoint{1.800422in}{0.767314in}}%
\pgfpathlineto{\pgfqpoint{1.800819in}{0.766687in}}%
\pgfpathlineto{\pgfqpoint{1.809235in}{0.753703in}}%
\pgfpathlineto{\pgfqpoint{1.816476in}{0.745063in}}%
\pgfpathlineto{\pgfqpoint{1.820884in}{0.740092in}}%
\pgfpathlineto{\pgfqpoint{1.832132in}{0.729288in}}%
\pgfpathlineto{\pgfqpoint{1.835361in}{0.726481in}}%
\pgfpathlineto{\pgfqpoint{1.847789in}{0.716702in}}%
\pgfpathlineto{\pgfqpoint{1.853506in}{0.712870in}}%
\pgfpathlineto{\pgfqpoint{1.863445in}{0.706575in}}%
\pgfpathlineto{\pgfqpoint{1.878380in}{0.699259in}}%
\pgfpathlineto{\pgfqpoint{1.879102in}{0.698913in}}%
\pgfpathclose%
\pgfpathmoveto{\pgfqpoint{0.376072in}{1.143399in}}%
\pgfpathlineto{\pgfqpoint{0.391728in}{1.148080in}}%
\pgfpathlineto{\pgfqpoint{0.392450in}{1.148425in}}%
\pgfpathlineto{\pgfqpoint{0.407385in}{1.155742in}}%
\pgfpathlineto{\pgfqpoint{0.417324in}{1.162036in}}%
\pgfpathlineto{\pgfqpoint{0.423041in}{1.165869in}}%
\pgfpathlineto{\pgfqpoint{0.435469in}{1.175647in}}%
\pgfpathlineto{\pgfqpoint{0.438698in}{1.178455in}}%
\pgfpathlineto{\pgfqpoint{0.449946in}{1.189259in}}%
\pgfpathlineto{\pgfqpoint{0.454354in}{1.194229in}}%
\pgfpathlineto{\pgfqpoint{0.461595in}{1.202870in}}%
\pgfpathlineto{\pgfqpoint{0.470011in}{1.215853in}}%
\pgfpathlineto{\pgfqpoint{0.470408in}{1.216481in}}%
\pgfpathlineto{\pgfqpoint{0.475793in}{1.230092in}}%
\pgfpathlineto{\pgfqpoint{0.477646in}{1.243703in}}%
\pgfpathlineto{\pgfqpoint{0.475793in}{1.257314in}}%
\pgfpathlineto{\pgfqpoint{0.470408in}{1.270925in}}%
\pgfpathlineto{\pgfqpoint{0.470011in}{1.271553in}}%
\pgfpathlineto{\pgfqpoint{0.461595in}{1.284536in}}%
\pgfpathlineto{\pgfqpoint{0.454354in}{1.293177in}}%
\pgfpathlineto{\pgfqpoint{0.449946in}{1.298148in}}%
\pgfpathlineto{\pgfqpoint{0.438698in}{1.308952in}}%
\pgfpathlineto{\pgfqpoint{0.435469in}{1.311759in}}%
\pgfpathlineto{\pgfqpoint{0.423041in}{1.321538in}}%
\pgfpathlineto{\pgfqpoint{0.417324in}{1.325370in}}%
\pgfpathlineto{\pgfqpoint{0.407385in}{1.331664in}}%
\pgfpathlineto{\pgfqpoint{0.392450in}{1.338981in}}%
\pgfpathlineto{\pgfqpoint{0.391728in}{1.339326in}}%
\pgfpathlineto{\pgfqpoint{0.376072in}{1.344007in}}%
\pgfpathlineto{\pgfqpoint{0.360415in}{1.345619in}}%
\pgfpathlineto{\pgfqpoint{0.360415in}{1.338981in}}%
\pgfpathlineto{\pgfqpoint{0.360415in}{1.325370in}}%
\pgfpathlineto{\pgfqpoint{0.360415in}{1.311759in}}%
\pgfpathlineto{\pgfqpoint{0.360415in}{1.298148in}}%
\pgfpathlineto{\pgfqpoint{0.360415in}{1.293358in}}%
\pgfpathlineto{\pgfqpoint{0.376072in}{1.291138in}}%
\pgfpathlineto{\pgfqpoint{0.391728in}{1.284691in}}%
\pgfpathlineto{\pgfqpoint{0.391962in}{1.284536in}}%
\pgfpathlineto{\pgfqpoint{0.407385in}{1.271129in}}%
\pgfpathlineto{\pgfqpoint{0.407562in}{1.270925in}}%
\pgfpathlineto{\pgfqpoint{0.414979in}{1.257314in}}%
\pgfpathlineto{\pgfqpoint{0.417532in}{1.243703in}}%
\pgfpathlineto{\pgfqpoint{0.414979in}{1.230092in}}%
\pgfpathlineto{\pgfqpoint{0.407562in}{1.216481in}}%
\pgfpathlineto{\pgfqpoint{0.407385in}{1.216277in}}%
\pgfpathlineto{\pgfqpoint{0.391962in}{1.202870in}}%
\pgfpathlineto{\pgfqpoint{0.391728in}{1.202715in}}%
\pgfpathlineto{\pgfqpoint{0.376072in}{1.196268in}}%
\pgfpathlineto{\pgfqpoint{0.360415in}{1.194048in}}%
\pgfpathlineto{\pgfqpoint{0.360415in}{1.189259in}}%
\pgfpathlineto{\pgfqpoint{0.360415in}{1.175647in}}%
\pgfpathlineto{\pgfqpoint{0.360415in}{1.162036in}}%
\pgfpathlineto{\pgfqpoint{0.360415in}{1.148425in}}%
\pgfpathlineto{\pgfqpoint{0.360415in}{1.141788in}}%
\pgfpathlineto{\pgfqpoint{0.376072in}{1.143399in}}%
\pgfpathclose%
\pgfpathmoveto{\pgfqpoint{0.845769in}{1.148080in}}%
\pgfpathlineto{\pgfqpoint{0.861425in}{1.143399in}}%
\pgfpathlineto{\pgfqpoint{0.877082in}{1.141788in}}%
\pgfpathlineto{\pgfqpoint{0.892738in}{1.143399in}}%
\pgfpathlineto{\pgfqpoint{0.908395in}{1.148080in}}%
\pgfpathlineto{\pgfqpoint{0.909117in}{1.148425in}}%
\pgfpathlineto{\pgfqpoint{0.924051in}{1.155742in}}%
\pgfpathlineto{\pgfqpoint{0.933990in}{1.162036in}}%
\pgfpathlineto{\pgfqpoint{0.939708in}{1.165869in}}%
\pgfpathlineto{\pgfqpoint{0.952136in}{1.175647in}}%
\pgfpathlineto{\pgfqpoint{0.955364in}{1.178455in}}%
\pgfpathlineto{\pgfqpoint{0.966613in}{1.189259in}}%
\pgfpathlineto{\pgfqpoint{0.971021in}{1.194229in}}%
\pgfpathlineto{\pgfqpoint{0.978262in}{1.202870in}}%
\pgfpathlineto{\pgfqpoint{0.986678in}{1.215853in}}%
\pgfpathlineto{\pgfqpoint{0.987075in}{1.216481in}}%
\pgfpathlineto{\pgfqpoint{0.992459in}{1.230092in}}%
\pgfpathlineto{\pgfqpoint{0.994313in}{1.243703in}}%
\pgfpathlineto{\pgfqpoint{0.992459in}{1.257314in}}%
\pgfpathlineto{\pgfqpoint{0.987075in}{1.270925in}}%
\pgfpathlineto{\pgfqpoint{0.986678in}{1.271553in}}%
\pgfpathlineto{\pgfqpoint{0.978262in}{1.284536in}}%
\pgfpathlineto{\pgfqpoint{0.971021in}{1.293177in}}%
\pgfpathlineto{\pgfqpoint{0.966613in}{1.298148in}}%
\pgfpathlineto{\pgfqpoint{0.955364in}{1.308952in}}%
\pgfpathlineto{\pgfqpoint{0.952136in}{1.311759in}}%
\pgfpathlineto{\pgfqpoint{0.939708in}{1.321538in}}%
\pgfpathlineto{\pgfqpoint{0.933990in}{1.325370in}}%
\pgfpathlineto{\pgfqpoint{0.924051in}{1.331664in}}%
\pgfpathlineto{\pgfqpoint{0.909117in}{1.338981in}}%
\pgfpathlineto{\pgfqpoint{0.908395in}{1.339326in}}%
\pgfpathlineto{\pgfqpoint{0.892738in}{1.344007in}}%
\pgfpathlineto{\pgfqpoint{0.877082in}{1.345619in}}%
\pgfpathlineto{\pgfqpoint{0.861425in}{1.344007in}}%
\pgfpathlineto{\pgfqpoint{0.845769in}{1.339326in}}%
\pgfpathlineto{\pgfqpoint{0.845047in}{1.338981in}}%
\pgfpathlineto{\pgfqpoint{0.830112in}{1.331664in}}%
\pgfpathlineto{\pgfqpoint{0.820173in}{1.325370in}}%
\pgfpathlineto{\pgfqpoint{0.814455in}{1.321538in}}%
\pgfpathlineto{\pgfqpoint{0.802028in}{1.311759in}}%
\pgfpathlineto{\pgfqpoint{0.798799in}{1.308952in}}%
\pgfpathlineto{\pgfqpoint{0.787550in}{1.298148in}}%
\pgfpathlineto{\pgfqpoint{0.783142in}{1.293177in}}%
\pgfpathlineto{\pgfqpoint{0.775902in}{1.284536in}}%
\pgfpathlineto{\pgfqpoint{0.767486in}{1.271553in}}%
\pgfpathlineto{\pgfqpoint{0.767088in}{1.270925in}}%
\pgfpathlineto{\pgfqpoint{0.761704in}{1.257314in}}%
\pgfpathlineto{\pgfqpoint{0.759850in}{1.243703in}}%
\pgfpathlineto{\pgfqpoint{0.761704in}{1.230092in}}%
\pgfpathlineto{\pgfqpoint{0.767088in}{1.216481in}}%
\pgfpathlineto{\pgfqpoint{0.767486in}{1.215853in}}%
\pgfpathlineto{\pgfqpoint{0.775902in}{1.202870in}}%
\pgfpathlineto{\pgfqpoint{0.783142in}{1.194229in}}%
\pgfpathlineto{\pgfqpoint{0.787550in}{1.189259in}}%
\pgfpathlineto{\pgfqpoint{0.798799in}{1.178455in}}%
\pgfpathlineto{\pgfqpoint{0.802028in}{1.175647in}}%
\pgfpathlineto{\pgfqpoint{0.814455in}{1.165869in}}%
\pgfpathlineto{\pgfqpoint{0.820173in}{1.162036in}}%
\pgfpathlineto{\pgfqpoint{0.830112in}{1.155742in}}%
\pgfpathlineto{\pgfqpoint{0.845047in}{1.148425in}}%
\pgfpathlineto{\pgfqpoint{0.845769in}{1.148080in}}%
\pgfpathclose%
\pgfpathmoveto{\pgfqpoint{0.845535in}{1.202870in}}%
\pgfpathlineto{\pgfqpoint{0.830112in}{1.216277in}}%
\pgfpathlineto{\pgfqpoint{0.829934in}{1.216481in}}%
\pgfpathlineto{\pgfqpoint{0.822518in}{1.230092in}}%
\pgfpathlineto{\pgfqpoint{0.819965in}{1.243703in}}%
\pgfpathlineto{\pgfqpoint{0.822518in}{1.257314in}}%
\pgfpathlineto{\pgfqpoint{0.829934in}{1.270925in}}%
\pgfpathlineto{\pgfqpoint{0.830112in}{1.271129in}}%
\pgfpathlineto{\pgfqpoint{0.845535in}{1.284536in}}%
\pgfpathlineto{\pgfqpoint{0.845769in}{1.284691in}}%
\pgfpathlineto{\pgfqpoint{0.861425in}{1.291138in}}%
\pgfpathlineto{\pgfqpoint{0.877082in}{1.293358in}}%
\pgfpathlineto{\pgfqpoint{0.892738in}{1.291138in}}%
\pgfpathlineto{\pgfqpoint{0.908395in}{1.284691in}}%
\pgfpathlineto{\pgfqpoint{0.908629in}{1.284536in}}%
\pgfpathlineto{\pgfqpoint{0.924051in}{1.271129in}}%
\pgfpathlineto{\pgfqpoint{0.924229in}{1.270925in}}%
\pgfpathlineto{\pgfqpoint{0.931645in}{1.257314in}}%
\pgfpathlineto{\pgfqpoint{0.934198in}{1.243703in}}%
\pgfpathlineto{\pgfqpoint{0.931645in}{1.230092in}}%
\pgfpathlineto{\pgfqpoint{0.924229in}{1.216481in}}%
\pgfpathlineto{\pgfqpoint{0.924051in}{1.216277in}}%
\pgfpathlineto{\pgfqpoint{0.908629in}{1.202870in}}%
\pgfpathlineto{\pgfqpoint{0.908395in}{1.202715in}}%
\pgfpathlineto{\pgfqpoint{0.892738in}{1.196268in}}%
\pgfpathlineto{\pgfqpoint{0.877082in}{1.194048in}}%
\pgfpathlineto{\pgfqpoint{0.861425in}{1.196268in}}%
\pgfpathlineto{\pgfqpoint{0.845769in}{1.202715in}}%
\pgfpathlineto{\pgfqpoint{0.845535in}{1.202870in}}%
\pgfpathclose%
\pgfpathmoveto{\pgfqpoint{1.362435in}{1.148080in}}%
\pgfpathlineto{\pgfqpoint{1.378092in}{1.143399in}}%
\pgfpathlineto{\pgfqpoint{1.393748in}{1.141788in}}%
\pgfpathlineto{\pgfqpoint{1.409405in}{1.143399in}}%
\pgfpathlineto{\pgfqpoint{1.425061in}{1.148080in}}%
\pgfpathlineto{\pgfqpoint{1.425783in}{1.148425in}}%
\pgfpathlineto{\pgfqpoint{1.440718in}{1.155742in}}%
\pgfpathlineto{\pgfqpoint{1.450657in}{1.162036in}}%
\pgfpathlineto{\pgfqpoint{1.456375in}{1.165869in}}%
\pgfpathlineto{\pgfqpoint{1.468802in}{1.175647in}}%
\pgfpathlineto{\pgfqpoint{1.472031in}{1.178455in}}%
\pgfpathlineto{\pgfqpoint{1.483280in}{1.189259in}}%
\pgfpathlineto{\pgfqpoint{1.487688in}{1.194229in}}%
\pgfpathlineto{\pgfqpoint{1.494928in}{1.202870in}}%
\pgfpathlineto{\pgfqpoint{1.503344in}{1.215853in}}%
\pgfpathlineto{\pgfqpoint{1.503742in}{1.216481in}}%
\pgfpathlineto{\pgfqpoint{1.509126in}{1.230092in}}%
\pgfpathlineto{\pgfqpoint{1.510980in}{1.243703in}}%
\pgfpathlineto{\pgfqpoint{1.509126in}{1.257314in}}%
\pgfpathlineto{\pgfqpoint{1.503742in}{1.270925in}}%
\pgfpathlineto{\pgfqpoint{1.503344in}{1.271553in}}%
\pgfpathlineto{\pgfqpoint{1.494928in}{1.284536in}}%
\pgfpathlineto{\pgfqpoint{1.487688in}{1.293177in}}%
\pgfpathlineto{\pgfqpoint{1.483280in}{1.298148in}}%
\pgfpathlineto{\pgfqpoint{1.472031in}{1.308952in}}%
\pgfpathlineto{\pgfqpoint{1.468802in}{1.311759in}}%
\pgfpathlineto{\pgfqpoint{1.456375in}{1.321538in}}%
\pgfpathlineto{\pgfqpoint{1.450657in}{1.325370in}}%
\pgfpathlineto{\pgfqpoint{1.440718in}{1.331664in}}%
\pgfpathlineto{\pgfqpoint{1.425783in}{1.338981in}}%
\pgfpathlineto{\pgfqpoint{1.425061in}{1.339326in}}%
\pgfpathlineto{\pgfqpoint{1.409405in}{1.344007in}}%
\pgfpathlineto{\pgfqpoint{1.393748in}{1.345619in}}%
\pgfpathlineto{\pgfqpoint{1.378092in}{1.344007in}}%
\pgfpathlineto{\pgfqpoint{1.362435in}{1.339326in}}%
\pgfpathlineto{\pgfqpoint{1.361713in}{1.338981in}}%
\pgfpathlineto{\pgfqpoint{1.346779in}{1.331664in}}%
\pgfpathlineto{\pgfqpoint{1.336840in}{1.325370in}}%
\pgfpathlineto{\pgfqpoint{1.331122in}{1.321538in}}%
\pgfpathlineto{\pgfqpoint{1.318694in}{1.311759in}}%
\pgfpathlineto{\pgfqpoint{1.315466in}{1.308952in}}%
\pgfpathlineto{\pgfqpoint{1.304217in}{1.298148in}}%
\pgfpathlineto{\pgfqpoint{1.299809in}{1.293177in}}%
\pgfpathlineto{\pgfqpoint{1.292568in}{1.284536in}}%
\pgfpathlineto{\pgfqpoint{1.284152in}{1.271553in}}%
\pgfpathlineto{\pgfqpoint{1.283755in}{1.270925in}}%
\pgfpathlineto{\pgfqpoint{1.278371in}{1.257314in}}%
\pgfpathlineto{\pgfqpoint{1.276517in}{1.243703in}}%
\pgfpathlineto{\pgfqpoint{1.278371in}{1.230092in}}%
\pgfpathlineto{\pgfqpoint{1.283755in}{1.216481in}}%
\pgfpathlineto{\pgfqpoint{1.284152in}{1.215853in}}%
\pgfpathlineto{\pgfqpoint{1.292568in}{1.202870in}}%
\pgfpathlineto{\pgfqpoint{1.299809in}{1.194229in}}%
\pgfpathlineto{\pgfqpoint{1.304217in}{1.189259in}}%
\pgfpathlineto{\pgfqpoint{1.315466in}{1.178455in}}%
\pgfpathlineto{\pgfqpoint{1.318694in}{1.175647in}}%
\pgfpathlineto{\pgfqpoint{1.331122in}{1.165869in}}%
\pgfpathlineto{\pgfqpoint{1.336840in}{1.162036in}}%
\pgfpathlineto{\pgfqpoint{1.346779in}{1.155742in}}%
\pgfpathlineto{\pgfqpoint{1.361713in}{1.148425in}}%
\pgfpathlineto{\pgfqpoint{1.362435in}{1.148080in}}%
\pgfpathclose%
\pgfpathmoveto{\pgfqpoint{1.362201in}{1.202870in}}%
\pgfpathlineto{\pgfqpoint{1.346779in}{1.216277in}}%
\pgfpathlineto{\pgfqpoint{1.346601in}{1.216481in}}%
\pgfpathlineto{\pgfqpoint{1.339185in}{1.230092in}}%
\pgfpathlineto{\pgfqpoint{1.336632in}{1.243703in}}%
\pgfpathlineto{\pgfqpoint{1.339185in}{1.257314in}}%
\pgfpathlineto{\pgfqpoint{1.346601in}{1.270925in}}%
\pgfpathlineto{\pgfqpoint{1.346779in}{1.271129in}}%
\pgfpathlineto{\pgfqpoint{1.362201in}{1.284536in}}%
\pgfpathlineto{\pgfqpoint{1.362435in}{1.284691in}}%
\pgfpathlineto{\pgfqpoint{1.378092in}{1.291138in}}%
\pgfpathlineto{\pgfqpoint{1.393748in}{1.293358in}}%
\pgfpathlineto{\pgfqpoint{1.409405in}{1.291138in}}%
\pgfpathlineto{\pgfqpoint{1.425061in}{1.284691in}}%
\pgfpathlineto{\pgfqpoint{1.425295in}{1.284536in}}%
\pgfpathlineto{\pgfqpoint{1.440718in}{1.271129in}}%
\pgfpathlineto{\pgfqpoint{1.440896in}{1.270925in}}%
\pgfpathlineto{\pgfqpoint{1.448312in}{1.257314in}}%
\pgfpathlineto{\pgfqpoint{1.450865in}{1.243703in}}%
\pgfpathlineto{\pgfqpoint{1.448312in}{1.230092in}}%
\pgfpathlineto{\pgfqpoint{1.440896in}{1.216481in}}%
\pgfpathlineto{\pgfqpoint{1.440718in}{1.216277in}}%
\pgfpathlineto{\pgfqpoint{1.425295in}{1.202870in}}%
\pgfpathlineto{\pgfqpoint{1.425061in}{1.202715in}}%
\pgfpathlineto{\pgfqpoint{1.409405in}{1.196268in}}%
\pgfpathlineto{\pgfqpoint{1.393748in}{1.194048in}}%
\pgfpathlineto{\pgfqpoint{1.378092in}{1.196268in}}%
\pgfpathlineto{\pgfqpoint{1.362435in}{1.202715in}}%
\pgfpathlineto{\pgfqpoint{1.362201in}{1.202870in}}%
\pgfpathclose%
\pgfpathmoveto{\pgfqpoint{1.879102in}{1.148080in}}%
\pgfpathlineto{\pgfqpoint{1.894758in}{1.143399in}}%
\pgfpathlineto{\pgfqpoint{1.910415in}{1.141788in}}%
\pgfpathlineto{\pgfqpoint{1.910415in}{1.148425in}}%
\pgfpathlineto{\pgfqpoint{1.910415in}{1.162036in}}%
\pgfpathlineto{\pgfqpoint{1.910415in}{1.175647in}}%
\pgfpathlineto{\pgfqpoint{1.910415in}{1.189259in}}%
\pgfpathlineto{\pgfqpoint{1.910415in}{1.194048in}}%
\pgfpathlineto{\pgfqpoint{1.894758in}{1.196268in}}%
\pgfpathlineto{\pgfqpoint{1.879102in}{1.202715in}}%
\pgfpathlineto{\pgfqpoint{1.878868in}{1.202870in}}%
\pgfpathlineto{\pgfqpoint{1.863445in}{1.216277in}}%
\pgfpathlineto{\pgfqpoint{1.863268in}{1.216481in}}%
\pgfpathlineto{\pgfqpoint{1.855851in}{1.230092in}}%
\pgfpathlineto{\pgfqpoint{1.853298in}{1.243703in}}%
\pgfpathlineto{\pgfqpoint{1.855851in}{1.257314in}}%
\pgfpathlineto{\pgfqpoint{1.863268in}{1.270925in}}%
\pgfpathlineto{\pgfqpoint{1.863445in}{1.271129in}}%
\pgfpathlineto{\pgfqpoint{1.878868in}{1.284536in}}%
\pgfpathlineto{\pgfqpoint{1.879102in}{1.284691in}}%
\pgfpathlineto{\pgfqpoint{1.894758in}{1.291138in}}%
\pgfpathlineto{\pgfqpoint{1.910415in}{1.293358in}}%
\pgfpathlineto{\pgfqpoint{1.910415in}{1.298148in}}%
\pgfpathlineto{\pgfqpoint{1.910415in}{1.311759in}}%
\pgfpathlineto{\pgfqpoint{1.910415in}{1.325370in}}%
\pgfpathlineto{\pgfqpoint{1.910415in}{1.338981in}}%
\pgfpathlineto{\pgfqpoint{1.910415in}{1.345619in}}%
\pgfpathlineto{\pgfqpoint{1.894758in}{1.344007in}}%
\pgfpathlineto{\pgfqpoint{1.879102in}{1.339326in}}%
\pgfpathlineto{\pgfqpoint{1.878380in}{1.338981in}}%
\pgfpathlineto{\pgfqpoint{1.863445in}{1.331664in}}%
\pgfpathlineto{\pgfqpoint{1.853506in}{1.325370in}}%
\pgfpathlineto{\pgfqpoint{1.847789in}{1.321538in}}%
\pgfpathlineto{\pgfqpoint{1.835361in}{1.311759in}}%
\pgfpathlineto{\pgfqpoint{1.832132in}{1.308952in}}%
\pgfpathlineto{\pgfqpoint{1.820884in}{1.298148in}}%
\pgfpathlineto{\pgfqpoint{1.816476in}{1.293177in}}%
\pgfpathlineto{\pgfqpoint{1.809235in}{1.284536in}}%
\pgfpathlineto{\pgfqpoint{1.800819in}{1.271553in}}%
\pgfpathlineto{\pgfqpoint{1.800422in}{1.270925in}}%
\pgfpathlineto{\pgfqpoint{1.795037in}{1.257314in}}%
\pgfpathlineto{\pgfqpoint{1.793184in}{1.243703in}}%
\pgfpathlineto{\pgfqpoint{1.795037in}{1.230092in}}%
\pgfpathlineto{\pgfqpoint{1.800422in}{1.216481in}}%
\pgfpathlineto{\pgfqpoint{1.800819in}{1.215853in}}%
\pgfpathlineto{\pgfqpoint{1.809235in}{1.202870in}}%
\pgfpathlineto{\pgfqpoint{1.816476in}{1.194229in}}%
\pgfpathlineto{\pgfqpoint{1.820884in}{1.189259in}}%
\pgfpathlineto{\pgfqpoint{1.832132in}{1.178455in}}%
\pgfpathlineto{\pgfqpoint{1.835361in}{1.175647in}}%
\pgfpathlineto{\pgfqpoint{1.847789in}{1.165869in}}%
\pgfpathlineto{\pgfqpoint{1.853506in}{1.162036in}}%
\pgfpathlineto{\pgfqpoint{1.863445in}{1.155742in}}%
\pgfpathlineto{\pgfqpoint{1.878380in}{1.148425in}}%
\pgfpathlineto{\pgfqpoint{1.879102in}{1.148080in}}%
\pgfpathclose%
\pgfpathmoveto{\pgfqpoint{0.376072in}{1.592565in}}%
\pgfpathlineto{\pgfqpoint{0.391728in}{1.597246in}}%
\pgfpathlineto{\pgfqpoint{0.392450in}{1.597592in}}%
\pgfpathlineto{\pgfqpoint{0.407385in}{1.604908in}}%
\pgfpathlineto{\pgfqpoint{0.417324in}{1.611203in}}%
\pgfpathlineto{\pgfqpoint{0.423041in}{1.615035in}}%
\pgfpathlineto{\pgfqpoint{0.435469in}{1.624814in}}%
\pgfpathlineto{\pgfqpoint{0.438698in}{1.627621in}}%
\pgfpathlineto{\pgfqpoint{0.449946in}{1.638425in}}%
\pgfpathlineto{\pgfqpoint{0.454354in}{1.643396in}}%
\pgfpathlineto{\pgfqpoint{0.461595in}{1.652036in}}%
\pgfpathlineto{\pgfqpoint{0.470011in}{1.665020in}}%
\pgfpathlineto{\pgfqpoint{0.470408in}{1.665648in}}%
\pgfpathlineto{\pgfqpoint{0.475793in}{1.679259in}}%
\pgfpathlineto{\pgfqpoint{0.477646in}{1.692870in}}%
\pgfpathlineto{\pgfqpoint{0.470011in}{1.692870in}}%
\pgfpathlineto{\pgfqpoint{0.454354in}{1.692870in}}%
\pgfpathlineto{\pgfqpoint{0.438698in}{1.692870in}}%
\pgfpathlineto{\pgfqpoint{0.423041in}{1.692870in}}%
\pgfpathlineto{\pgfqpoint{0.417532in}{1.692870in}}%
\pgfpathlineto{\pgfqpoint{0.414979in}{1.679259in}}%
\pgfpathlineto{\pgfqpoint{0.407562in}{1.665648in}}%
\pgfpathlineto{\pgfqpoint{0.407385in}{1.665444in}}%
\pgfpathlineto{\pgfqpoint{0.391962in}{1.652036in}}%
\pgfpathlineto{\pgfqpoint{0.391728in}{1.651882in}}%
\pgfpathlineto{\pgfqpoint{0.376072in}{1.645435in}}%
\pgfpathlineto{\pgfqpoint{0.360415in}{1.643215in}}%
\pgfpathlineto{\pgfqpoint{0.360415in}{1.638425in}}%
\pgfpathlineto{\pgfqpoint{0.360415in}{1.624814in}}%
\pgfpathlineto{\pgfqpoint{0.360415in}{1.611203in}}%
\pgfpathlineto{\pgfqpoint{0.360415in}{1.597592in}}%
\pgfpathlineto{\pgfqpoint{0.360415in}{1.590954in}}%
\pgfpathlineto{\pgfqpoint{0.376072in}{1.592565in}}%
\pgfpathclose%
\pgfpathmoveto{\pgfqpoint{0.845769in}{1.597246in}}%
\pgfpathlineto{\pgfqpoint{0.861425in}{1.592565in}}%
\pgfpathlineto{\pgfqpoint{0.877082in}{1.590954in}}%
\pgfpathlineto{\pgfqpoint{0.892738in}{1.592565in}}%
\pgfpathlineto{\pgfqpoint{0.908395in}{1.597246in}}%
\pgfpathlineto{\pgfqpoint{0.909117in}{1.597592in}}%
\pgfpathlineto{\pgfqpoint{0.924051in}{1.604908in}}%
\pgfpathlineto{\pgfqpoint{0.933990in}{1.611203in}}%
\pgfpathlineto{\pgfqpoint{0.939708in}{1.615035in}}%
\pgfpathlineto{\pgfqpoint{0.952136in}{1.624814in}}%
\pgfpathlineto{\pgfqpoint{0.955364in}{1.627621in}}%
\pgfpathlineto{\pgfqpoint{0.966613in}{1.638425in}}%
\pgfpathlineto{\pgfqpoint{0.971021in}{1.643396in}}%
\pgfpathlineto{\pgfqpoint{0.978262in}{1.652036in}}%
\pgfpathlineto{\pgfqpoint{0.986678in}{1.665020in}}%
\pgfpathlineto{\pgfqpoint{0.987075in}{1.665648in}}%
\pgfpathlineto{\pgfqpoint{0.992459in}{1.679259in}}%
\pgfpathlineto{\pgfqpoint{0.994313in}{1.692870in}}%
\pgfpathlineto{\pgfqpoint{0.986678in}{1.692870in}}%
\pgfpathlineto{\pgfqpoint{0.971021in}{1.692870in}}%
\pgfpathlineto{\pgfqpoint{0.955364in}{1.692870in}}%
\pgfpathlineto{\pgfqpoint{0.939708in}{1.692870in}}%
\pgfpathlineto{\pgfqpoint{0.934198in}{1.692870in}}%
\pgfpathlineto{\pgfqpoint{0.931645in}{1.679259in}}%
\pgfpathlineto{\pgfqpoint{0.924229in}{1.665648in}}%
\pgfpathlineto{\pgfqpoint{0.924051in}{1.665444in}}%
\pgfpathlineto{\pgfqpoint{0.908629in}{1.652036in}}%
\pgfpathlineto{\pgfqpoint{0.908395in}{1.651882in}}%
\pgfpathlineto{\pgfqpoint{0.892738in}{1.645435in}}%
\pgfpathlineto{\pgfqpoint{0.877082in}{1.643215in}}%
\pgfpathlineto{\pgfqpoint{0.861425in}{1.645435in}}%
\pgfpathlineto{\pgfqpoint{0.845769in}{1.651882in}}%
\pgfpathlineto{\pgfqpoint{0.845535in}{1.652036in}}%
\pgfpathlineto{\pgfqpoint{0.830112in}{1.665444in}}%
\pgfpathlineto{\pgfqpoint{0.829934in}{1.665648in}}%
\pgfpathlineto{\pgfqpoint{0.822518in}{1.679259in}}%
\pgfpathlineto{\pgfqpoint{0.819965in}{1.692870in}}%
\pgfpathlineto{\pgfqpoint{0.814455in}{1.692870in}}%
\pgfpathlineto{\pgfqpoint{0.798799in}{1.692870in}}%
\pgfpathlineto{\pgfqpoint{0.783142in}{1.692870in}}%
\pgfpathlineto{\pgfqpoint{0.767486in}{1.692870in}}%
\pgfpathlineto{\pgfqpoint{0.759850in}{1.692870in}}%
\pgfpathlineto{\pgfqpoint{0.761704in}{1.679259in}}%
\pgfpathlineto{\pgfqpoint{0.767088in}{1.665648in}}%
\pgfpathlineto{\pgfqpoint{0.767486in}{1.665020in}}%
\pgfpathlineto{\pgfqpoint{0.775902in}{1.652036in}}%
\pgfpathlineto{\pgfqpoint{0.783142in}{1.643396in}}%
\pgfpathlineto{\pgfqpoint{0.787550in}{1.638425in}}%
\pgfpathlineto{\pgfqpoint{0.798799in}{1.627621in}}%
\pgfpathlineto{\pgfqpoint{0.802028in}{1.624814in}}%
\pgfpathlineto{\pgfqpoint{0.814455in}{1.615035in}}%
\pgfpathlineto{\pgfqpoint{0.820173in}{1.611203in}}%
\pgfpathlineto{\pgfqpoint{0.830112in}{1.604908in}}%
\pgfpathlineto{\pgfqpoint{0.845047in}{1.597592in}}%
\pgfpathlineto{\pgfqpoint{0.845769in}{1.597246in}}%
\pgfpathclose%
\pgfpathmoveto{\pgfqpoint{1.362435in}{1.597246in}}%
\pgfpathlineto{\pgfqpoint{1.378092in}{1.592565in}}%
\pgfpathlineto{\pgfqpoint{1.393748in}{1.590954in}}%
\pgfpathlineto{\pgfqpoint{1.409405in}{1.592565in}}%
\pgfpathlineto{\pgfqpoint{1.425061in}{1.597246in}}%
\pgfpathlineto{\pgfqpoint{1.425783in}{1.597592in}}%
\pgfpathlineto{\pgfqpoint{1.440718in}{1.604908in}}%
\pgfpathlineto{\pgfqpoint{1.450657in}{1.611203in}}%
\pgfpathlineto{\pgfqpoint{1.456375in}{1.615035in}}%
\pgfpathlineto{\pgfqpoint{1.468802in}{1.624814in}}%
\pgfpathlineto{\pgfqpoint{1.472031in}{1.627621in}}%
\pgfpathlineto{\pgfqpoint{1.483280in}{1.638425in}}%
\pgfpathlineto{\pgfqpoint{1.487688in}{1.643396in}}%
\pgfpathlineto{\pgfqpoint{1.494928in}{1.652036in}}%
\pgfpathlineto{\pgfqpoint{1.503344in}{1.665020in}}%
\pgfpathlineto{\pgfqpoint{1.503742in}{1.665648in}}%
\pgfpathlineto{\pgfqpoint{1.509126in}{1.679259in}}%
\pgfpathlineto{\pgfqpoint{1.510980in}{1.692870in}}%
\pgfpathlineto{\pgfqpoint{1.503344in}{1.692870in}}%
\pgfpathlineto{\pgfqpoint{1.487688in}{1.692870in}}%
\pgfpathlineto{\pgfqpoint{1.472031in}{1.692870in}}%
\pgfpathlineto{\pgfqpoint{1.456375in}{1.692870in}}%
\pgfpathlineto{\pgfqpoint{1.450865in}{1.692870in}}%
\pgfpathlineto{\pgfqpoint{1.448312in}{1.679259in}}%
\pgfpathlineto{\pgfqpoint{1.440896in}{1.665648in}}%
\pgfpathlineto{\pgfqpoint{1.440718in}{1.665444in}}%
\pgfpathlineto{\pgfqpoint{1.425295in}{1.652036in}}%
\pgfpathlineto{\pgfqpoint{1.425061in}{1.651882in}}%
\pgfpathlineto{\pgfqpoint{1.409405in}{1.645435in}}%
\pgfpathlineto{\pgfqpoint{1.393748in}{1.643215in}}%
\pgfpathlineto{\pgfqpoint{1.378092in}{1.645435in}}%
\pgfpathlineto{\pgfqpoint{1.362435in}{1.651882in}}%
\pgfpathlineto{\pgfqpoint{1.362201in}{1.652036in}}%
\pgfpathlineto{\pgfqpoint{1.346779in}{1.665444in}}%
\pgfpathlineto{\pgfqpoint{1.346601in}{1.665648in}}%
\pgfpathlineto{\pgfqpoint{1.339185in}{1.679259in}}%
\pgfpathlineto{\pgfqpoint{1.336632in}{1.692870in}}%
\pgfpathlineto{\pgfqpoint{1.331122in}{1.692870in}}%
\pgfpathlineto{\pgfqpoint{1.315466in}{1.692870in}}%
\pgfpathlineto{\pgfqpoint{1.299809in}{1.692870in}}%
\pgfpathlineto{\pgfqpoint{1.284152in}{1.692870in}}%
\pgfpathlineto{\pgfqpoint{1.276517in}{1.692870in}}%
\pgfpathlineto{\pgfqpoint{1.278371in}{1.679259in}}%
\pgfpathlineto{\pgfqpoint{1.283755in}{1.665648in}}%
\pgfpathlineto{\pgfqpoint{1.284152in}{1.665020in}}%
\pgfpathlineto{\pgfqpoint{1.292568in}{1.652036in}}%
\pgfpathlineto{\pgfqpoint{1.299809in}{1.643396in}}%
\pgfpathlineto{\pgfqpoint{1.304217in}{1.638425in}}%
\pgfpathlineto{\pgfqpoint{1.315466in}{1.627621in}}%
\pgfpathlineto{\pgfqpoint{1.318694in}{1.624814in}}%
\pgfpathlineto{\pgfqpoint{1.331122in}{1.615035in}}%
\pgfpathlineto{\pgfqpoint{1.336840in}{1.611203in}}%
\pgfpathlineto{\pgfqpoint{1.346779in}{1.604908in}}%
\pgfpathlineto{\pgfqpoint{1.361713in}{1.597592in}}%
\pgfpathlineto{\pgfqpoint{1.362435in}{1.597246in}}%
\pgfpathclose%
\pgfpathmoveto{\pgfqpoint{1.879102in}{1.597246in}}%
\pgfpathlineto{\pgfqpoint{1.894758in}{1.592565in}}%
\pgfpathlineto{\pgfqpoint{1.910415in}{1.590954in}}%
\pgfpathlineto{\pgfqpoint{1.910415in}{1.597592in}}%
\pgfpathlineto{\pgfqpoint{1.910415in}{1.611203in}}%
\pgfpathlineto{\pgfqpoint{1.910415in}{1.624814in}}%
\pgfpathlineto{\pgfqpoint{1.910415in}{1.638425in}}%
\pgfpathlineto{\pgfqpoint{1.910415in}{1.643215in}}%
\pgfpathlineto{\pgfqpoint{1.894758in}{1.645435in}}%
\pgfpathlineto{\pgfqpoint{1.879102in}{1.651882in}}%
\pgfpathlineto{\pgfqpoint{1.878868in}{1.652036in}}%
\pgfpathlineto{\pgfqpoint{1.863445in}{1.665444in}}%
\pgfpathlineto{\pgfqpoint{1.863268in}{1.665648in}}%
\pgfpathlineto{\pgfqpoint{1.855851in}{1.679259in}}%
\pgfpathlineto{\pgfqpoint{1.853298in}{1.692870in}}%
\pgfpathlineto{\pgfqpoint{1.847789in}{1.692870in}}%
\pgfpathlineto{\pgfqpoint{1.832132in}{1.692870in}}%
\pgfpathlineto{\pgfqpoint{1.816476in}{1.692870in}}%
\pgfpathlineto{\pgfqpoint{1.800819in}{1.692870in}}%
\pgfpathlineto{\pgfqpoint{1.793184in}{1.692870in}}%
\pgfpathlineto{\pgfqpoint{1.795037in}{1.679259in}}%
\pgfpathlineto{\pgfqpoint{1.800422in}{1.665648in}}%
\pgfpathlineto{\pgfqpoint{1.800819in}{1.665020in}}%
\pgfpathlineto{\pgfqpoint{1.809235in}{1.652036in}}%
\pgfpathlineto{\pgfqpoint{1.816476in}{1.643396in}}%
\pgfpathlineto{\pgfqpoint{1.820884in}{1.638425in}}%
\pgfpathlineto{\pgfqpoint{1.832132in}{1.627621in}}%
\pgfpathlineto{\pgfqpoint{1.835361in}{1.624814in}}%
\pgfpathlineto{\pgfqpoint{1.847789in}{1.615035in}}%
\pgfpathlineto{\pgfqpoint{1.853506in}{1.611203in}}%
\pgfpathlineto{\pgfqpoint{1.863445in}{1.604908in}}%
\pgfpathlineto{\pgfqpoint{1.878380in}{1.597592in}}%
\pgfpathlineto{\pgfqpoint{1.879102in}{1.597246in}}%
\pgfpathclose%
\pgfusepath{fill}%
\end{pgfscope}%
\begin{pgfscope}%
\pgfpathrectangle{\pgfqpoint{0.360415in}{0.345370in}}{\pgfqpoint{1.550000in}{1.347500in}}%
\pgfusepath{clip}%
\pgfsetbuttcap%
\pgfsetroundjoin%
\definecolor{currentfill}{rgb}{0.178950,0.019252,0.584054}%
\pgfsetfillcolor{currentfill}%
\pgfsetlinewidth{0.000000pt}%
\definecolor{currentstroke}{rgb}{0.000000,0.000000,0.000000}%
\pgfsetstrokecolor{currentstroke}%
\pgfsetdash{}{0pt}%
\pgfpathmoveto{\pgfqpoint{0.376072in}{0.345370in}}%
\pgfpathlineto{\pgfqpoint{0.391728in}{0.345370in}}%
\pgfpathlineto{\pgfqpoint{0.407385in}{0.345370in}}%
\pgfpathlineto{\pgfqpoint{0.417532in}{0.345370in}}%
\pgfpathlineto{\pgfqpoint{0.414979in}{0.358981in}}%
\pgfpathlineto{\pgfqpoint{0.407562in}{0.372592in}}%
\pgfpathlineto{\pgfqpoint{0.407385in}{0.372795in}}%
\pgfpathlineto{\pgfqpoint{0.391962in}{0.386203in}}%
\pgfpathlineto{\pgfqpoint{0.391728in}{0.386357in}}%
\pgfpathlineto{\pgfqpoint{0.376072in}{0.392805in}}%
\pgfpathlineto{\pgfqpoint{0.360415in}{0.395024in}}%
\pgfpathlineto{\pgfqpoint{0.360415in}{0.386203in}}%
\pgfpathlineto{\pgfqpoint{0.360415in}{0.372592in}}%
\pgfpathlineto{\pgfqpoint{0.360415in}{0.358981in}}%
\pgfpathlineto{\pgfqpoint{0.360415in}{0.345370in}}%
\pgfpathlineto{\pgfqpoint{0.376072in}{0.345370in}}%
\pgfpathclose%
\pgfpathmoveto{\pgfqpoint{0.830112in}{0.345370in}}%
\pgfpathlineto{\pgfqpoint{0.845769in}{0.345370in}}%
\pgfpathlineto{\pgfqpoint{0.861425in}{0.345370in}}%
\pgfpathlineto{\pgfqpoint{0.877082in}{0.345370in}}%
\pgfpathlineto{\pgfqpoint{0.892738in}{0.345370in}}%
\pgfpathlineto{\pgfqpoint{0.908395in}{0.345370in}}%
\pgfpathlineto{\pgfqpoint{0.924051in}{0.345370in}}%
\pgfpathlineto{\pgfqpoint{0.934198in}{0.345370in}}%
\pgfpathlineto{\pgfqpoint{0.931645in}{0.358981in}}%
\pgfpathlineto{\pgfqpoint{0.924229in}{0.372592in}}%
\pgfpathlineto{\pgfqpoint{0.924051in}{0.372795in}}%
\pgfpathlineto{\pgfqpoint{0.908629in}{0.386203in}}%
\pgfpathlineto{\pgfqpoint{0.908395in}{0.386357in}}%
\pgfpathlineto{\pgfqpoint{0.892738in}{0.392805in}}%
\pgfpathlineto{\pgfqpoint{0.877082in}{0.395024in}}%
\pgfpathlineto{\pgfqpoint{0.861425in}{0.392805in}}%
\pgfpathlineto{\pgfqpoint{0.845769in}{0.386357in}}%
\pgfpathlineto{\pgfqpoint{0.845535in}{0.386203in}}%
\pgfpathlineto{\pgfqpoint{0.830112in}{0.372795in}}%
\pgfpathlineto{\pgfqpoint{0.829934in}{0.372592in}}%
\pgfpathlineto{\pgfqpoint{0.822518in}{0.358981in}}%
\pgfpathlineto{\pgfqpoint{0.819965in}{0.345370in}}%
\pgfpathlineto{\pgfqpoint{0.830112in}{0.345370in}}%
\pgfpathclose%
\pgfpathmoveto{\pgfqpoint{1.346779in}{0.345370in}}%
\pgfpathlineto{\pgfqpoint{1.362435in}{0.345370in}}%
\pgfpathlineto{\pgfqpoint{1.378092in}{0.345370in}}%
\pgfpathlineto{\pgfqpoint{1.393748in}{0.345370in}}%
\pgfpathlineto{\pgfqpoint{1.409405in}{0.345370in}}%
\pgfpathlineto{\pgfqpoint{1.425061in}{0.345370in}}%
\pgfpathlineto{\pgfqpoint{1.440718in}{0.345370in}}%
\pgfpathlineto{\pgfqpoint{1.450865in}{0.345370in}}%
\pgfpathlineto{\pgfqpoint{1.448312in}{0.358981in}}%
\pgfpathlineto{\pgfqpoint{1.440896in}{0.372592in}}%
\pgfpathlineto{\pgfqpoint{1.440718in}{0.372795in}}%
\pgfpathlineto{\pgfqpoint{1.425295in}{0.386203in}}%
\pgfpathlineto{\pgfqpoint{1.425061in}{0.386357in}}%
\pgfpathlineto{\pgfqpoint{1.409405in}{0.392805in}}%
\pgfpathlineto{\pgfqpoint{1.393748in}{0.395024in}}%
\pgfpathlineto{\pgfqpoint{1.378092in}{0.392805in}}%
\pgfpathlineto{\pgfqpoint{1.362435in}{0.386357in}}%
\pgfpathlineto{\pgfqpoint{1.362201in}{0.386203in}}%
\pgfpathlineto{\pgfqpoint{1.346779in}{0.372795in}}%
\pgfpathlineto{\pgfqpoint{1.346601in}{0.372592in}}%
\pgfpathlineto{\pgfqpoint{1.339185in}{0.358981in}}%
\pgfpathlineto{\pgfqpoint{1.336632in}{0.345370in}}%
\pgfpathlineto{\pgfqpoint{1.346779in}{0.345370in}}%
\pgfpathclose%
\pgfpathmoveto{\pgfqpoint{1.863445in}{0.345370in}}%
\pgfpathlineto{\pgfqpoint{1.879102in}{0.345370in}}%
\pgfpathlineto{\pgfqpoint{1.894758in}{0.345370in}}%
\pgfpathlineto{\pgfqpoint{1.910415in}{0.345370in}}%
\pgfpathlineto{\pgfqpoint{1.910415in}{0.358981in}}%
\pgfpathlineto{\pgfqpoint{1.910415in}{0.372592in}}%
\pgfpathlineto{\pgfqpoint{1.910415in}{0.386203in}}%
\pgfpathlineto{\pgfqpoint{1.910415in}{0.395024in}}%
\pgfpathlineto{\pgfqpoint{1.894758in}{0.392805in}}%
\pgfpathlineto{\pgfqpoint{1.879102in}{0.386357in}}%
\pgfpathlineto{\pgfqpoint{1.878868in}{0.386203in}}%
\pgfpathlineto{\pgfqpoint{1.863445in}{0.372795in}}%
\pgfpathlineto{\pgfqpoint{1.863268in}{0.372592in}}%
\pgfpathlineto{\pgfqpoint{1.855851in}{0.358981in}}%
\pgfpathlineto{\pgfqpoint{1.853298in}{0.345370in}}%
\pgfpathlineto{\pgfqpoint{1.863445in}{0.345370in}}%
\pgfpathclose%
\pgfpathmoveto{\pgfqpoint{0.376072in}{0.747101in}}%
\pgfpathlineto{\pgfqpoint{0.391728in}{0.753549in}}%
\pgfpathlineto{\pgfqpoint{0.391962in}{0.753703in}}%
\pgfpathlineto{\pgfqpoint{0.407385in}{0.767111in}}%
\pgfpathlineto{\pgfqpoint{0.407562in}{0.767314in}}%
\pgfpathlineto{\pgfqpoint{0.414979in}{0.780925in}}%
\pgfpathlineto{\pgfqpoint{0.417532in}{0.794536in}}%
\pgfpathlineto{\pgfqpoint{0.414979in}{0.808148in}}%
\pgfpathlineto{\pgfqpoint{0.407562in}{0.821759in}}%
\pgfpathlineto{\pgfqpoint{0.407385in}{0.821962in}}%
\pgfpathlineto{\pgfqpoint{0.391962in}{0.835370in}}%
\pgfpathlineto{\pgfqpoint{0.391728in}{0.835524in}}%
\pgfpathlineto{\pgfqpoint{0.376072in}{0.841972in}}%
\pgfpathlineto{\pgfqpoint{0.360415in}{0.844191in}}%
\pgfpathlineto{\pgfqpoint{0.360415in}{0.835370in}}%
\pgfpathlineto{\pgfqpoint{0.360415in}{0.821759in}}%
\pgfpathlineto{\pgfqpoint{0.360415in}{0.808148in}}%
\pgfpathlineto{\pgfqpoint{0.360415in}{0.794536in}}%
\pgfpathlineto{\pgfqpoint{0.360415in}{0.780925in}}%
\pgfpathlineto{\pgfqpoint{0.360415in}{0.767314in}}%
\pgfpathlineto{\pgfqpoint{0.360415in}{0.753703in}}%
\pgfpathlineto{\pgfqpoint{0.360415in}{0.744882in}}%
\pgfpathlineto{\pgfqpoint{0.376072in}{0.747101in}}%
\pgfpathclose%
\pgfpathmoveto{\pgfqpoint{0.845769in}{0.753549in}}%
\pgfpathlineto{\pgfqpoint{0.861425in}{0.747101in}}%
\pgfpathlineto{\pgfqpoint{0.877082in}{0.744882in}}%
\pgfpathlineto{\pgfqpoint{0.892738in}{0.747101in}}%
\pgfpathlineto{\pgfqpoint{0.908395in}{0.753549in}}%
\pgfpathlineto{\pgfqpoint{0.908629in}{0.753703in}}%
\pgfpathlineto{\pgfqpoint{0.924051in}{0.767111in}}%
\pgfpathlineto{\pgfqpoint{0.924229in}{0.767314in}}%
\pgfpathlineto{\pgfqpoint{0.931645in}{0.780925in}}%
\pgfpathlineto{\pgfqpoint{0.934198in}{0.794536in}}%
\pgfpathlineto{\pgfqpoint{0.931645in}{0.808148in}}%
\pgfpathlineto{\pgfqpoint{0.924229in}{0.821759in}}%
\pgfpathlineto{\pgfqpoint{0.924051in}{0.821962in}}%
\pgfpathlineto{\pgfqpoint{0.908629in}{0.835370in}}%
\pgfpathlineto{\pgfqpoint{0.908395in}{0.835524in}}%
\pgfpathlineto{\pgfqpoint{0.892738in}{0.841972in}}%
\pgfpathlineto{\pgfqpoint{0.877082in}{0.844191in}}%
\pgfpathlineto{\pgfqpoint{0.861425in}{0.841972in}}%
\pgfpathlineto{\pgfqpoint{0.845769in}{0.835524in}}%
\pgfpathlineto{\pgfqpoint{0.845535in}{0.835370in}}%
\pgfpathlineto{\pgfqpoint{0.830112in}{0.821962in}}%
\pgfpathlineto{\pgfqpoint{0.829934in}{0.821759in}}%
\pgfpathlineto{\pgfqpoint{0.822518in}{0.808148in}}%
\pgfpathlineto{\pgfqpoint{0.819965in}{0.794536in}}%
\pgfpathlineto{\pgfqpoint{0.822518in}{0.780925in}}%
\pgfpathlineto{\pgfqpoint{0.829934in}{0.767314in}}%
\pgfpathlineto{\pgfqpoint{0.830112in}{0.767111in}}%
\pgfpathlineto{\pgfqpoint{0.845535in}{0.753703in}}%
\pgfpathlineto{\pgfqpoint{0.845769in}{0.753549in}}%
\pgfpathclose%
\pgfpathmoveto{\pgfqpoint{1.362435in}{0.753549in}}%
\pgfpathlineto{\pgfqpoint{1.378092in}{0.747101in}}%
\pgfpathlineto{\pgfqpoint{1.393748in}{0.744882in}}%
\pgfpathlineto{\pgfqpoint{1.409405in}{0.747101in}}%
\pgfpathlineto{\pgfqpoint{1.425061in}{0.753549in}}%
\pgfpathlineto{\pgfqpoint{1.425295in}{0.753703in}}%
\pgfpathlineto{\pgfqpoint{1.440718in}{0.767111in}}%
\pgfpathlineto{\pgfqpoint{1.440896in}{0.767314in}}%
\pgfpathlineto{\pgfqpoint{1.448312in}{0.780925in}}%
\pgfpathlineto{\pgfqpoint{1.450865in}{0.794536in}}%
\pgfpathlineto{\pgfqpoint{1.448312in}{0.808148in}}%
\pgfpathlineto{\pgfqpoint{1.440896in}{0.821759in}}%
\pgfpathlineto{\pgfqpoint{1.440718in}{0.821962in}}%
\pgfpathlineto{\pgfqpoint{1.425295in}{0.835370in}}%
\pgfpathlineto{\pgfqpoint{1.425061in}{0.835524in}}%
\pgfpathlineto{\pgfqpoint{1.409405in}{0.841972in}}%
\pgfpathlineto{\pgfqpoint{1.393748in}{0.844191in}}%
\pgfpathlineto{\pgfqpoint{1.378092in}{0.841972in}}%
\pgfpathlineto{\pgfqpoint{1.362435in}{0.835524in}}%
\pgfpathlineto{\pgfqpoint{1.362201in}{0.835370in}}%
\pgfpathlineto{\pgfqpoint{1.346779in}{0.821962in}}%
\pgfpathlineto{\pgfqpoint{1.346601in}{0.821759in}}%
\pgfpathlineto{\pgfqpoint{1.339185in}{0.808148in}}%
\pgfpathlineto{\pgfqpoint{1.336632in}{0.794536in}}%
\pgfpathlineto{\pgfqpoint{1.339185in}{0.780925in}}%
\pgfpathlineto{\pgfqpoint{1.346601in}{0.767314in}}%
\pgfpathlineto{\pgfqpoint{1.346779in}{0.767111in}}%
\pgfpathlineto{\pgfqpoint{1.362201in}{0.753703in}}%
\pgfpathlineto{\pgfqpoint{1.362435in}{0.753549in}}%
\pgfpathclose%
\pgfpathmoveto{\pgfqpoint{1.879102in}{0.753549in}}%
\pgfpathlineto{\pgfqpoint{1.894758in}{0.747101in}}%
\pgfpathlineto{\pgfqpoint{1.910415in}{0.744882in}}%
\pgfpathlineto{\pgfqpoint{1.910415in}{0.753703in}}%
\pgfpathlineto{\pgfqpoint{1.910415in}{0.767314in}}%
\pgfpathlineto{\pgfqpoint{1.910415in}{0.780925in}}%
\pgfpathlineto{\pgfqpoint{1.910415in}{0.794536in}}%
\pgfpathlineto{\pgfqpoint{1.910415in}{0.808148in}}%
\pgfpathlineto{\pgfqpoint{1.910415in}{0.821759in}}%
\pgfpathlineto{\pgfqpoint{1.910415in}{0.835370in}}%
\pgfpathlineto{\pgfqpoint{1.910415in}{0.844191in}}%
\pgfpathlineto{\pgfqpoint{1.894758in}{0.841972in}}%
\pgfpathlineto{\pgfqpoint{1.879102in}{0.835524in}}%
\pgfpathlineto{\pgfqpoint{1.878868in}{0.835370in}}%
\pgfpathlineto{\pgfqpoint{1.863445in}{0.821962in}}%
\pgfpathlineto{\pgfqpoint{1.863268in}{0.821759in}}%
\pgfpathlineto{\pgfqpoint{1.855851in}{0.808148in}}%
\pgfpathlineto{\pgfqpoint{1.853298in}{0.794536in}}%
\pgfpathlineto{\pgfqpoint{1.855851in}{0.780925in}}%
\pgfpathlineto{\pgfqpoint{1.863268in}{0.767314in}}%
\pgfpathlineto{\pgfqpoint{1.863445in}{0.767111in}}%
\pgfpathlineto{\pgfqpoint{1.878868in}{0.753703in}}%
\pgfpathlineto{\pgfqpoint{1.879102in}{0.753549in}}%
\pgfpathclose%
\pgfpathmoveto{\pgfqpoint{0.376072in}{1.196268in}}%
\pgfpathlineto{\pgfqpoint{0.391728in}{1.202715in}}%
\pgfpathlineto{\pgfqpoint{0.391962in}{1.202870in}}%
\pgfpathlineto{\pgfqpoint{0.407385in}{1.216277in}}%
\pgfpathlineto{\pgfqpoint{0.407562in}{1.216481in}}%
\pgfpathlineto{\pgfqpoint{0.414979in}{1.230092in}}%
\pgfpathlineto{\pgfqpoint{0.417532in}{1.243703in}}%
\pgfpathlineto{\pgfqpoint{0.414979in}{1.257314in}}%
\pgfpathlineto{\pgfqpoint{0.407562in}{1.270925in}}%
\pgfpathlineto{\pgfqpoint{0.407385in}{1.271129in}}%
\pgfpathlineto{\pgfqpoint{0.391962in}{1.284536in}}%
\pgfpathlineto{\pgfqpoint{0.391728in}{1.284691in}}%
\pgfpathlineto{\pgfqpoint{0.376072in}{1.291138in}}%
\pgfpathlineto{\pgfqpoint{0.360415in}{1.293358in}}%
\pgfpathlineto{\pgfqpoint{0.360415in}{1.284536in}}%
\pgfpathlineto{\pgfqpoint{0.360415in}{1.270925in}}%
\pgfpathlineto{\pgfqpoint{0.360415in}{1.257314in}}%
\pgfpathlineto{\pgfqpoint{0.360415in}{1.243703in}}%
\pgfpathlineto{\pgfqpoint{0.360415in}{1.230092in}}%
\pgfpathlineto{\pgfqpoint{0.360415in}{1.216481in}}%
\pgfpathlineto{\pgfqpoint{0.360415in}{1.202870in}}%
\pgfpathlineto{\pgfqpoint{0.360415in}{1.194048in}}%
\pgfpathlineto{\pgfqpoint{0.376072in}{1.196268in}}%
\pgfpathclose%
\pgfpathmoveto{\pgfqpoint{0.845769in}{1.202715in}}%
\pgfpathlineto{\pgfqpoint{0.861425in}{1.196268in}}%
\pgfpathlineto{\pgfqpoint{0.877082in}{1.194048in}}%
\pgfpathlineto{\pgfqpoint{0.892738in}{1.196268in}}%
\pgfpathlineto{\pgfqpoint{0.908395in}{1.202715in}}%
\pgfpathlineto{\pgfqpoint{0.908629in}{1.202870in}}%
\pgfpathlineto{\pgfqpoint{0.924051in}{1.216277in}}%
\pgfpathlineto{\pgfqpoint{0.924229in}{1.216481in}}%
\pgfpathlineto{\pgfqpoint{0.931645in}{1.230092in}}%
\pgfpathlineto{\pgfqpoint{0.934198in}{1.243703in}}%
\pgfpathlineto{\pgfqpoint{0.931645in}{1.257314in}}%
\pgfpathlineto{\pgfqpoint{0.924229in}{1.270925in}}%
\pgfpathlineto{\pgfqpoint{0.924051in}{1.271129in}}%
\pgfpathlineto{\pgfqpoint{0.908629in}{1.284536in}}%
\pgfpathlineto{\pgfqpoint{0.908395in}{1.284691in}}%
\pgfpathlineto{\pgfqpoint{0.892738in}{1.291138in}}%
\pgfpathlineto{\pgfqpoint{0.877082in}{1.293358in}}%
\pgfpathlineto{\pgfqpoint{0.861425in}{1.291138in}}%
\pgfpathlineto{\pgfqpoint{0.845769in}{1.284691in}}%
\pgfpathlineto{\pgfqpoint{0.845535in}{1.284536in}}%
\pgfpathlineto{\pgfqpoint{0.830112in}{1.271129in}}%
\pgfpathlineto{\pgfqpoint{0.829934in}{1.270925in}}%
\pgfpathlineto{\pgfqpoint{0.822518in}{1.257314in}}%
\pgfpathlineto{\pgfqpoint{0.819965in}{1.243703in}}%
\pgfpathlineto{\pgfqpoint{0.822518in}{1.230092in}}%
\pgfpathlineto{\pgfqpoint{0.829934in}{1.216481in}}%
\pgfpathlineto{\pgfqpoint{0.830112in}{1.216277in}}%
\pgfpathlineto{\pgfqpoint{0.845535in}{1.202870in}}%
\pgfpathlineto{\pgfqpoint{0.845769in}{1.202715in}}%
\pgfpathclose%
\pgfpathmoveto{\pgfqpoint{1.362435in}{1.202715in}}%
\pgfpathlineto{\pgfqpoint{1.378092in}{1.196268in}}%
\pgfpathlineto{\pgfqpoint{1.393748in}{1.194048in}}%
\pgfpathlineto{\pgfqpoint{1.409405in}{1.196268in}}%
\pgfpathlineto{\pgfqpoint{1.425061in}{1.202715in}}%
\pgfpathlineto{\pgfqpoint{1.425295in}{1.202870in}}%
\pgfpathlineto{\pgfqpoint{1.440718in}{1.216277in}}%
\pgfpathlineto{\pgfqpoint{1.440896in}{1.216481in}}%
\pgfpathlineto{\pgfqpoint{1.448312in}{1.230092in}}%
\pgfpathlineto{\pgfqpoint{1.450865in}{1.243703in}}%
\pgfpathlineto{\pgfqpoint{1.448312in}{1.257314in}}%
\pgfpathlineto{\pgfqpoint{1.440896in}{1.270925in}}%
\pgfpathlineto{\pgfqpoint{1.440718in}{1.271129in}}%
\pgfpathlineto{\pgfqpoint{1.425295in}{1.284536in}}%
\pgfpathlineto{\pgfqpoint{1.425061in}{1.284691in}}%
\pgfpathlineto{\pgfqpoint{1.409405in}{1.291138in}}%
\pgfpathlineto{\pgfqpoint{1.393748in}{1.293358in}}%
\pgfpathlineto{\pgfqpoint{1.378092in}{1.291138in}}%
\pgfpathlineto{\pgfqpoint{1.362435in}{1.284691in}}%
\pgfpathlineto{\pgfqpoint{1.362201in}{1.284536in}}%
\pgfpathlineto{\pgfqpoint{1.346779in}{1.271129in}}%
\pgfpathlineto{\pgfqpoint{1.346601in}{1.270925in}}%
\pgfpathlineto{\pgfqpoint{1.339185in}{1.257314in}}%
\pgfpathlineto{\pgfqpoint{1.336632in}{1.243703in}}%
\pgfpathlineto{\pgfqpoint{1.339185in}{1.230092in}}%
\pgfpathlineto{\pgfqpoint{1.346601in}{1.216481in}}%
\pgfpathlineto{\pgfqpoint{1.346779in}{1.216277in}}%
\pgfpathlineto{\pgfqpoint{1.362201in}{1.202870in}}%
\pgfpathlineto{\pgfqpoint{1.362435in}{1.202715in}}%
\pgfpathclose%
\pgfpathmoveto{\pgfqpoint{1.879102in}{1.202715in}}%
\pgfpathlineto{\pgfqpoint{1.894758in}{1.196268in}}%
\pgfpathlineto{\pgfqpoint{1.910415in}{1.194048in}}%
\pgfpathlineto{\pgfqpoint{1.910415in}{1.202870in}}%
\pgfpathlineto{\pgfqpoint{1.910415in}{1.216481in}}%
\pgfpathlineto{\pgfqpoint{1.910415in}{1.230092in}}%
\pgfpathlineto{\pgfqpoint{1.910415in}{1.243703in}}%
\pgfpathlineto{\pgfqpoint{1.910415in}{1.257314in}}%
\pgfpathlineto{\pgfqpoint{1.910415in}{1.270925in}}%
\pgfpathlineto{\pgfqpoint{1.910415in}{1.284536in}}%
\pgfpathlineto{\pgfqpoint{1.910415in}{1.293358in}}%
\pgfpathlineto{\pgfqpoint{1.894758in}{1.291138in}}%
\pgfpathlineto{\pgfqpoint{1.879102in}{1.284691in}}%
\pgfpathlineto{\pgfqpoint{1.878868in}{1.284536in}}%
\pgfpathlineto{\pgfqpoint{1.863445in}{1.271129in}}%
\pgfpathlineto{\pgfqpoint{1.863268in}{1.270925in}}%
\pgfpathlineto{\pgfqpoint{1.855851in}{1.257314in}}%
\pgfpathlineto{\pgfqpoint{1.853298in}{1.243703in}}%
\pgfpathlineto{\pgfqpoint{1.855851in}{1.230092in}}%
\pgfpathlineto{\pgfqpoint{1.863268in}{1.216481in}}%
\pgfpathlineto{\pgfqpoint{1.863445in}{1.216277in}}%
\pgfpathlineto{\pgfqpoint{1.878868in}{1.202870in}}%
\pgfpathlineto{\pgfqpoint{1.879102in}{1.202715in}}%
\pgfpathclose%
\pgfpathmoveto{\pgfqpoint{0.376072in}{1.645435in}}%
\pgfpathlineto{\pgfqpoint{0.391728in}{1.651882in}}%
\pgfpathlineto{\pgfqpoint{0.391962in}{1.652036in}}%
\pgfpathlineto{\pgfqpoint{0.407385in}{1.665444in}}%
\pgfpathlineto{\pgfqpoint{0.407562in}{1.665648in}}%
\pgfpathlineto{\pgfqpoint{0.414979in}{1.679259in}}%
\pgfpathlineto{\pgfqpoint{0.417532in}{1.692870in}}%
\pgfpathlineto{\pgfqpoint{0.407385in}{1.692870in}}%
\pgfpathlineto{\pgfqpoint{0.391728in}{1.692870in}}%
\pgfpathlineto{\pgfqpoint{0.376072in}{1.692870in}}%
\pgfpathlineto{\pgfqpoint{0.360415in}{1.692870in}}%
\pgfpathlineto{\pgfqpoint{0.360415in}{1.679259in}}%
\pgfpathlineto{\pgfqpoint{0.360415in}{1.665648in}}%
\pgfpathlineto{\pgfqpoint{0.360415in}{1.652036in}}%
\pgfpathlineto{\pgfqpoint{0.360415in}{1.643215in}}%
\pgfpathlineto{\pgfqpoint{0.376072in}{1.645435in}}%
\pgfpathclose%
\pgfpathmoveto{\pgfqpoint{0.845769in}{1.651882in}}%
\pgfpathlineto{\pgfqpoint{0.861425in}{1.645435in}}%
\pgfpathlineto{\pgfqpoint{0.877082in}{1.643215in}}%
\pgfpathlineto{\pgfqpoint{0.892738in}{1.645435in}}%
\pgfpathlineto{\pgfqpoint{0.908395in}{1.651882in}}%
\pgfpathlineto{\pgfqpoint{0.908629in}{1.652036in}}%
\pgfpathlineto{\pgfqpoint{0.924051in}{1.665444in}}%
\pgfpathlineto{\pgfqpoint{0.924229in}{1.665648in}}%
\pgfpathlineto{\pgfqpoint{0.931645in}{1.679259in}}%
\pgfpathlineto{\pgfqpoint{0.934198in}{1.692870in}}%
\pgfpathlineto{\pgfqpoint{0.924051in}{1.692870in}}%
\pgfpathlineto{\pgfqpoint{0.908395in}{1.692870in}}%
\pgfpathlineto{\pgfqpoint{0.892738in}{1.692870in}}%
\pgfpathlineto{\pgfqpoint{0.877082in}{1.692870in}}%
\pgfpathlineto{\pgfqpoint{0.861425in}{1.692870in}}%
\pgfpathlineto{\pgfqpoint{0.845769in}{1.692870in}}%
\pgfpathlineto{\pgfqpoint{0.830112in}{1.692870in}}%
\pgfpathlineto{\pgfqpoint{0.819965in}{1.692870in}}%
\pgfpathlineto{\pgfqpoint{0.822518in}{1.679259in}}%
\pgfpathlineto{\pgfqpoint{0.829934in}{1.665648in}}%
\pgfpathlineto{\pgfqpoint{0.830112in}{1.665444in}}%
\pgfpathlineto{\pgfqpoint{0.845535in}{1.652036in}}%
\pgfpathlineto{\pgfqpoint{0.845769in}{1.651882in}}%
\pgfpathclose%
\pgfpathmoveto{\pgfqpoint{1.362435in}{1.651882in}}%
\pgfpathlineto{\pgfqpoint{1.378092in}{1.645435in}}%
\pgfpathlineto{\pgfqpoint{1.393748in}{1.643215in}}%
\pgfpathlineto{\pgfqpoint{1.409405in}{1.645435in}}%
\pgfpathlineto{\pgfqpoint{1.425061in}{1.651882in}}%
\pgfpathlineto{\pgfqpoint{1.425295in}{1.652036in}}%
\pgfpathlineto{\pgfqpoint{1.440718in}{1.665444in}}%
\pgfpathlineto{\pgfqpoint{1.440896in}{1.665648in}}%
\pgfpathlineto{\pgfqpoint{1.448312in}{1.679259in}}%
\pgfpathlineto{\pgfqpoint{1.450865in}{1.692870in}}%
\pgfpathlineto{\pgfqpoint{1.440718in}{1.692870in}}%
\pgfpathlineto{\pgfqpoint{1.425061in}{1.692870in}}%
\pgfpathlineto{\pgfqpoint{1.409405in}{1.692870in}}%
\pgfpathlineto{\pgfqpoint{1.393748in}{1.692870in}}%
\pgfpathlineto{\pgfqpoint{1.378092in}{1.692870in}}%
\pgfpathlineto{\pgfqpoint{1.362435in}{1.692870in}}%
\pgfpathlineto{\pgfqpoint{1.346779in}{1.692870in}}%
\pgfpathlineto{\pgfqpoint{1.336632in}{1.692870in}}%
\pgfpathlineto{\pgfqpoint{1.339185in}{1.679259in}}%
\pgfpathlineto{\pgfqpoint{1.346601in}{1.665648in}}%
\pgfpathlineto{\pgfqpoint{1.346779in}{1.665444in}}%
\pgfpathlineto{\pgfqpoint{1.362201in}{1.652036in}}%
\pgfpathlineto{\pgfqpoint{1.362435in}{1.651882in}}%
\pgfpathclose%
\pgfpathmoveto{\pgfqpoint{1.879102in}{1.651882in}}%
\pgfpathlineto{\pgfqpoint{1.894758in}{1.645435in}}%
\pgfpathlineto{\pgfqpoint{1.910415in}{1.643215in}}%
\pgfpathlineto{\pgfqpoint{1.910415in}{1.652036in}}%
\pgfpathlineto{\pgfqpoint{1.910415in}{1.665648in}}%
\pgfpathlineto{\pgfqpoint{1.910415in}{1.679259in}}%
\pgfpathlineto{\pgfqpoint{1.910415in}{1.692870in}}%
\pgfpathlineto{\pgfqpoint{1.894758in}{1.692870in}}%
\pgfpathlineto{\pgfqpoint{1.879102in}{1.692870in}}%
\pgfpathlineto{\pgfqpoint{1.863445in}{1.692870in}}%
\pgfpathlineto{\pgfqpoint{1.853298in}{1.692870in}}%
\pgfpathlineto{\pgfqpoint{1.855851in}{1.679259in}}%
\pgfpathlineto{\pgfqpoint{1.863268in}{1.665648in}}%
\pgfpathlineto{\pgfqpoint{1.863445in}{1.665444in}}%
\pgfpathlineto{\pgfqpoint{1.878868in}{1.652036in}}%
\pgfpathlineto{\pgfqpoint{1.879102in}{1.651882in}}%
\pgfpathclose%
\pgfusepath{fill}%
\end{pgfscope}%
\begin{pgfscope}%
\pgfsetbuttcap%
\pgfsetroundjoin%
\definecolor{currentfill}{rgb}{0.000000,0.000000,0.000000}%
\pgfsetfillcolor{currentfill}%
\pgfsetlinewidth{0.803000pt}%
\definecolor{currentstroke}{rgb}{0.000000,0.000000,0.000000}%
\pgfsetstrokecolor{currentstroke}%
\pgfsetdash{}{0pt}%
\pgfsys@defobject{currentmarker}{\pgfqpoint{0.000000in}{-0.048611in}}{\pgfqpoint{0.000000in}{0.000000in}}{%
\pgfpathmoveto{\pgfqpoint{0.000000in}{0.000000in}}%
\pgfpathlineto{\pgfqpoint{0.000000in}{-0.048611in}}%
\pgfusepath{stroke,fill}%
}%
\begin{pgfscope}%
\pgfsys@transformshift{0.360415in}{0.345370in}%
\pgfsys@useobject{currentmarker}{}%
\end{pgfscope}%
\end{pgfscope}%
\begin{pgfscope}%
\definecolor{textcolor}{rgb}{0.000000,0.000000,0.000000}%
\pgfsetstrokecolor{textcolor}%
\pgfsetfillcolor{textcolor}%
\pgftext[x=0.360415in,y=0.248148in,,top]{\color{textcolor}{\rmfamily\fontsize{12.000000}{14.400000}\selectfont\catcode`\^=\active\def^{\ifmmode\sp\else\^{}\fi}\catcode`\%=\active\def%{\%}$\mathdefault{0}$}}%
\end{pgfscope}%
\begin{pgfscope}%
\pgfsetbuttcap%
\pgfsetroundjoin%
\definecolor{currentfill}{rgb}{0.000000,0.000000,0.000000}%
\pgfsetfillcolor{currentfill}%
\pgfsetlinewidth{0.803000pt}%
\definecolor{currentstroke}{rgb}{0.000000,0.000000,0.000000}%
\pgfsetstrokecolor{currentstroke}%
\pgfsetdash{}{0pt}%
\pgfsys@defobject{currentmarker}{\pgfqpoint{0.000000in}{-0.048611in}}{\pgfqpoint{0.000000in}{0.000000in}}{%
\pgfpathmoveto{\pgfqpoint{0.000000in}{0.000000in}}%
\pgfpathlineto{\pgfqpoint{0.000000in}{-0.048611in}}%
\pgfusepath{stroke,fill}%
}%
\begin{pgfscope}%
\pgfsys@transformshift{1.221526in}{0.345370in}%
\pgfsys@useobject{currentmarker}{}%
\end{pgfscope}%
\end{pgfscope}%
\begin{pgfscope}%
\definecolor{textcolor}{rgb}{0.000000,0.000000,0.000000}%
\pgfsetstrokecolor{textcolor}%
\pgfsetfillcolor{textcolor}%
\pgftext[x=1.221526in,y=0.248148in,,top]{\color{textcolor}{\rmfamily\fontsize{12.000000}{14.400000}\selectfont\catcode`\^=\active\def^{\ifmmode\sp\else\^{}\fi}\catcode`\%=\active\def%{\%}$\mathdefault{10}$}}%
\end{pgfscope}%
\begin{pgfscope}%
\pgfsetbuttcap%
\pgfsetroundjoin%
\definecolor{currentfill}{rgb}{0.000000,0.000000,0.000000}%
\pgfsetfillcolor{currentfill}%
\pgfsetlinewidth{0.803000pt}%
\definecolor{currentstroke}{rgb}{0.000000,0.000000,0.000000}%
\pgfsetstrokecolor{currentstroke}%
\pgfsetdash{}{0pt}%
\pgfsys@defobject{currentmarker}{\pgfqpoint{-0.048611in}{0.000000in}}{\pgfqpoint{-0.000000in}{0.000000in}}{%
\pgfpathmoveto{\pgfqpoint{-0.000000in}{0.000000in}}%
\pgfpathlineto{\pgfqpoint{-0.048611in}{0.000000in}}%
\pgfusepath{stroke,fill}%
}%
\begin{pgfscope}%
\pgfsys@transformshift{0.360415in}{0.345370in}%
\pgfsys@useobject{currentmarker}{}%
\end{pgfscope}%
\end{pgfscope}%
\begin{pgfscope}%
\definecolor{textcolor}{rgb}{0.000000,0.000000,0.000000}%
\pgfsetstrokecolor{textcolor}%
\pgfsetfillcolor{textcolor}%
\pgftext[x=0.181596in, y=0.287500in, left, base]{\color{textcolor}{\rmfamily\fontsize{12.000000}{14.400000}\selectfont\catcode`\^=\active\def^{\ifmmode\sp\else\^{}\fi}\catcode`\%=\active\def%{\%}$\mathdefault{0}$}}%
\end{pgfscope}%
\begin{pgfscope}%
\pgfsetbuttcap%
\pgfsetroundjoin%
\definecolor{currentfill}{rgb}{0.000000,0.000000,0.000000}%
\pgfsetfillcolor{currentfill}%
\pgfsetlinewidth{0.803000pt}%
\definecolor{currentstroke}{rgb}{0.000000,0.000000,0.000000}%
\pgfsetstrokecolor{currentstroke}%
\pgfsetdash{}{0pt}%
\pgfsys@defobject{currentmarker}{\pgfqpoint{-0.048611in}{0.000000in}}{\pgfqpoint{-0.000000in}{0.000000in}}{%
\pgfpathmoveto{\pgfqpoint{-0.000000in}{0.000000in}}%
\pgfpathlineto{\pgfqpoint{-0.048611in}{0.000000in}}%
\pgfusepath{stroke,fill}%
}%
\begin{pgfscope}%
\pgfsys@transformshift{0.360415in}{0.719675in}%
\pgfsys@useobject{currentmarker}{}%
\end{pgfscope}%
\end{pgfscope}%
\begin{pgfscope}%
\definecolor{textcolor}{rgb}{0.000000,0.000000,0.000000}%
\pgfsetstrokecolor{textcolor}%
\pgfsetfillcolor{textcolor}%
\pgftext[x=0.181596in, y=0.661805in, left, base]{\color{textcolor}{\rmfamily\fontsize{12.000000}{14.400000}\selectfont\catcode`\^=\active\def^{\ifmmode\sp\else\^{}\fi}\catcode`\%=\active\def%{\%}$\mathdefault{5}$}}%
\end{pgfscope}%
\begin{pgfscope}%
\pgfsetbuttcap%
\pgfsetroundjoin%
\definecolor{currentfill}{rgb}{0.000000,0.000000,0.000000}%
\pgfsetfillcolor{currentfill}%
\pgfsetlinewidth{0.803000pt}%
\definecolor{currentstroke}{rgb}{0.000000,0.000000,0.000000}%
\pgfsetstrokecolor{currentstroke}%
\pgfsetdash{}{0pt}%
\pgfsys@defobject{currentmarker}{\pgfqpoint{-0.048611in}{0.000000in}}{\pgfqpoint{-0.000000in}{0.000000in}}{%
\pgfpathmoveto{\pgfqpoint{-0.000000in}{0.000000in}}%
\pgfpathlineto{\pgfqpoint{-0.048611in}{0.000000in}}%
\pgfusepath{stroke,fill}%
}%
\begin{pgfscope}%
\pgfsys@transformshift{0.360415in}{1.093981in}%
\pgfsys@useobject{currentmarker}{}%
\end{pgfscope}%
\end{pgfscope}%
\begin{pgfscope}%
\definecolor{textcolor}{rgb}{0.000000,0.000000,0.000000}%
\pgfsetstrokecolor{textcolor}%
\pgfsetfillcolor{textcolor}%
\pgftext[x=0.100000in, y=1.036111in, left, base]{\color{textcolor}{\rmfamily\fontsize{12.000000}{14.400000}\selectfont\catcode`\^=\active\def^{\ifmmode\sp\else\^{}\fi}\catcode`\%=\active\def%{\%}$\mathdefault{10}$}}%
\end{pgfscope}%
\begin{pgfscope}%
\pgfsetbuttcap%
\pgfsetroundjoin%
\definecolor{currentfill}{rgb}{0.000000,0.000000,0.000000}%
\pgfsetfillcolor{currentfill}%
\pgfsetlinewidth{0.803000pt}%
\definecolor{currentstroke}{rgb}{0.000000,0.000000,0.000000}%
\pgfsetstrokecolor{currentstroke}%
\pgfsetdash{}{0pt}%
\pgfsys@defobject{currentmarker}{\pgfqpoint{-0.048611in}{0.000000in}}{\pgfqpoint{-0.000000in}{0.000000in}}{%
\pgfpathmoveto{\pgfqpoint{-0.000000in}{0.000000in}}%
\pgfpathlineto{\pgfqpoint{-0.048611in}{0.000000in}}%
\pgfusepath{stroke,fill}%
}%
\begin{pgfscope}%
\pgfsys@transformshift{0.360415in}{1.468286in}%
\pgfsys@useobject{currentmarker}{}%
\end{pgfscope}%
\end{pgfscope}%
\begin{pgfscope}%
\definecolor{textcolor}{rgb}{0.000000,0.000000,0.000000}%
\pgfsetstrokecolor{textcolor}%
\pgfsetfillcolor{textcolor}%
\pgftext[x=0.100000in, y=1.410416in, left, base]{\color{textcolor}{\rmfamily\fontsize{12.000000}{14.400000}\selectfont\catcode`\^=\active\def^{\ifmmode\sp\else\^{}\fi}\catcode`\%=\active\def%{\%}$\mathdefault{15}$}}%
\end{pgfscope}%
\begin{pgfscope}%
\pgfsetrectcap%
\pgfsetmiterjoin%
\pgfsetlinewidth{0.803000pt}%
\definecolor{currentstroke}{rgb}{0.000000,0.000000,0.000000}%
\pgfsetstrokecolor{currentstroke}%
\pgfsetdash{}{0pt}%
\pgfpathmoveto{\pgfqpoint{0.360415in}{0.345370in}}%
\pgfpathlineto{\pgfqpoint{0.360415in}{1.692870in}}%
\pgfusepath{stroke}%
\end{pgfscope}%
\begin{pgfscope}%
\pgfsetrectcap%
\pgfsetmiterjoin%
\pgfsetlinewidth{0.803000pt}%
\definecolor{currentstroke}{rgb}{0.000000,0.000000,0.000000}%
\pgfsetstrokecolor{currentstroke}%
\pgfsetdash{}{0pt}%
\pgfpathmoveto{\pgfqpoint{1.910415in}{0.345370in}}%
\pgfpathlineto{\pgfqpoint{1.910415in}{1.692870in}}%
\pgfusepath{stroke}%
\end{pgfscope}%
\begin{pgfscope}%
\pgfsetrectcap%
\pgfsetmiterjoin%
\pgfsetlinewidth{0.803000pt}%
\definecolor{currentstroke}{rgb}{0.000000,0.000000,0.000000}%
\pgfsetstrokecolor{currentstroke}%
\pgfsetdash{}{0pt}%
\pgfpathmoveto{\pgfqpoint{0.360415in}{0.345370in}}%
\pgfpathlineto{\pgfqpoint{1.910415in}{0.345370in}}%
\pgfusepath{stroke}%
\end{pgfscope}%
\begin{pgfscope}%
\pgfsetrectcap%
\pgfsetmiterjoin%
\pgfsetlinewidth{0.803000pt}%
\definecolor{currentstroke}{rgb}{0.000000,0.000000,0.000000}%
\pgfsetstrokecolor{currentstroke}%
\pgfsetdash{}{0pt}%
\pgfpathmoveto{\pgfqpoint{0.360415in}{1.692870in}}%
\pgfpathlineto{\pgfqpoint{1.910415in}{1.692870in}}%
\pgfusepath{stroke}%
\end{pgfscope}%
\end{pgfpicture}%
\makeatother%
\endgroup%
}
        \caption{$n_c=3.$}
        \label{fig:gaussian-well-3}
    \end{subfigure}
    \caption{Cross-sections of the periodic Gaussian wells potential $V$ for different $n_c$.}
    \label{fig:gaussian-well}
\end{figure}

\cref{fig:convergence} displays the convergence behavior of Chebyshev-Nyström++ applied to the matrix $\mtx{A}$ resulting from the discretization of the Hamiltonian for $n_c = 1$ and $n_c = 3$. Unlike \cite{lin-2017-randomized-estimation}, where just a small portion of the spectral density is considered, we approximate the whole spectrum of $\mtx{A}$ (transformed to $[-1, 1]$) in accordance with our theoretical results (\cref{subsubsec:chebyshev-nystrom-analysis}) and to demonstrate the effectiveness of our implementational particularities (\cref{subsubsec:chebyshev-nystrom-implementation}). As a reference, we use the eigenvalues computed by NumPy's Hermitian eigenvalue solver. As expected from \cref{thm:hutchinson}, Girard-Hutchinson alone ($n_{\mtx{\Omega}} = 0$) requires $n_{\mtx{\Psi}} = \mathcal{O}(\varepsilon^{-2})$ samples to achieve an error of order $\varepsilon$. Because the eigenvalues of the matrix are spread out, we observe the better convergence of the Nyström approximation discussed at the end of \cref{sec:application} once $n_{\mtx{\Omega}}$ is sufficiently large. At this point, we can observe that the errors stagnate around $10^{-6}$ as a consequence of the error made in the Chebyshev approximation. Due to the higher eigenvalue density of $\mtx{A}$ for $n_c = 3$, the stronger convergence can be observed to kick in later in this case, because a larger $n_{\mtx{\Omega}}$ is needed for a good Nyström approximation. In fact, for $n_c = 5$ the eigenvalue density is so high that to profit from this effect, $n_{\mtx{\Omega}}$ would need to be chosen so large that the computations become unfeasible.
\begin{figure}[ht]
    \centering
    \begin{subfigure}[b]{0.495\textwidth}
        \scalebox{0.8}{%% Creator: Matplotlib, PGF backend
%%
%% To include the figure in your LaTeX document, write
%%   \input{<filename>.pgf}
%%
%% Make sure the required packages are loaded in your preamble
%%   \usepackage{pgf}
%%
%% Also ensure that all the required font packages are loaded; for instance,
%% the lmodern package is sometimes necessary when using math font.
%%   \usepackage{lmodern}
%%
%% Figures using additional raster images can only be included by \input if
%% they are in the same directory as the main LaTeX file. For loading figures
%% from other directories you can use the `import` package
%%   \usepackage{import}
%%
%% and then include the figures with
%%   \import{<path to file>}{<filename>.pgf}
%%
%% Matplotlib used the following preamble
%%   \def\mathdefault#1{#1}
%%   \everymath=\expandafter{\the\everymath\displaystyle}
%%   
%%   \ifdefined\pdftexversion\else  % non-pdftex case.
%%     \usepackage{fontspec}
%%     \setmainfont{DejaVuSerif.ttf}[Path=\detokenize{/home/matti/Documents/projects/Rand-TRACE/.venv/lib/python3.12/site-packages/matplotlib/mpl-data/fonts/ttf/}]
%%     \setsansfont{DejaVuSans.ttf}[Path=\detokenize{/home/matti/Documents/projects/Rand-TRACE/.venv/lib/python3.12/site-packages/matplotlib/mpl-data/fonts/ttf/}]
%%     \setmonofont{DejaVuSansMono.ttf}[Path=\detokenize{/home/matti/Documents/projects/Rand-TRACE/.venv/lib/python3.12/site-packages/matplotlib/mpl-data/fonts/ttf/}]
%%   \fi
%%   \makeatletter\@ifpackageloaded{underscore}{}{\usepackage[strings]{underscore}}\makeatother
%%
\begingroup%
\makeatletter%
\begin{pgfpicture}%
\pgfpathrectangle{\pgfpointorigin}{\pgfqpoint{3.146913in}{2.959073in}}%
\pgfusepath{use as bounding box, clip}%
\begin{pgfscope}%
\pgfsetbuttcap%
\pgfsetmiterjoin%
\definecolor{currentfill}{rgb}{1.000000,1.000000,1.000000}%
\pgfsetfillcolor{currentfill}%
\pgfsetlinewidth{0.000000pt}%
\definecolor{currentstroke}{rgb}{1.000000,1.000000,1.000000}%
\pgfsetstrokecolor{currentstroke}%
\pgfsetdash{}{0pt}%
\pgfpathmoveto{\pgfqpoint{0.000000in}{-0.000000in}}%
\pgfpathlineto{\pgfqpoint{3.146913in}{-0.000000in}}%
\pgfpathlineto{\pgfqpoint{3.146913in}{2.959073in}}%
\pgfpathlineto{\pgfqpoint{0.000000in}{2.959073in}}%
\pgfpathlineto{\pgfqpoint{0.000000in}{-0.000000in}}%
\pgfpathclose%
\pgfusepath{fill}%
\end{pgfscope}%
\begin{pgfscope}%
\pgfsetbuttcap%
\pgfsetmiterjoin%
\definecolor{currentfill}{rgb}{1.000000,1.000000,1.000000}%
\pgfsetfillcolor{currentfill}%
\pgfsetlinewidth{0.000000pt}%
\definecolor{currentstroke}{rgb}{0.000000,0.000000,0.000000}%
\pgfsetstrokecolor{currentstroke}%
\pgfsetstrokeopacity{0.000000}%
\pgfsetdash{}{0pt}%
\pgfpathmoveto{\pgfqpoint{0.721913in}{0.549073in}}%
\pgfpathlineto{\pgfqpoint{3.046913in}{0.549073in}}%
\pgfpathlineto{\pgfqpoint{3.046913in}{2.859073in}}%
\pgfpathlineto{\pgfqpoint{0.721913in}{2.859073in}}%
\pgfpathlineto{\pgfqpoint{0.721913in}{0.549073in}}%
\pgfpathclose%
\pgfusepath{fill}%
\end{pgfscope}%
\begin{pgfscope}%
\pgfpathrectangle{\pgfqpoint{0.721913in}{0.549073in}}{\pgfqpoint{2.325000in}{2.310000in}}%
\pgfusepath{clip}%
\pgfsetrectcap%
\pgfsetroundjoin%
\pgfsetlinewidth{0.250937pt}%
\definecolor{currentstroke}{rgb}{0.000000,0.000000,0.000000}%
\pgfsetstrokecolor{currentstroke}%
\pgfsetstrokeopacity{0.200000}%
\pgfsetdash{}{0pt}%
\pgfpathmoveto{\pgfqpoint{1.020314in}{0.549073in}}%
\pgfpathlineto{\pgfqpoint{1.020314in}{2.859073in}}%
\pgfusepath{stroke}%
\end{pgfscope}%
\begin{pgfscope}%
\pgfsetbuttcap%
\pgfsetroundjoin%
\definecolor{currentfill}{rgb}{0.000000,0.000000,0.000000}%
\pgfsetfillcolor{currentfill}%
\pgfsetlinewidth{0.803000pt}%
\definecolor{currentstroke}{rgb}{0.000000,0.000000,0.000000}%
\pgfsetstrokecolor{currentstroke}%
\pgfsetdash{}{0pt}%
\pgfsys@defobject{currentmarker}{\pgfqpoint{0.000000in}{-0.048611in}}{\pgfqpoint{0.000000in}{0.000000in}}{%
\pgfpathmoveto{\pgfqpoint{0.000000in}{0.000000in}}%
\pgfpathlineto{\pgfqpoint{0.000000in}{-0.048611in}}%
\pgfusepath{stroke,fill}%
}%
\begin{pgfscope}%
\pgfsys@transformshift{1.020314in}{0.549073in}%
\pgfsys@useobject{currentmarker}{}%
\end{pgfscope}%
\end{pgfscope}%
\begin{pgfscope}%
\definecolor{textcolor}{rgb}{0.000000,0.000000,0.000000}%
\pgfsetstrokecolor{textcolor}%
\pgfsetfillcolor{textcolor}%
\pgftext[x=1.020314in,y=0.451851in,,top]{\color{textcolor}{\rmfamily\fontsize{12.000000}{14.400000}\selectfont\catcode`\^=\active\def^{\ifmmode\sp\else\^{}\fi}\catcode`\%=\active\def%{\%}$\mathdefault{10^{1}}$}}%
\end{pgfscope}%
\begin{pgfscope}%
\pgfpathrectangle{\pgfqpoint{0.721913in}{0.549073in}}{\pgfqpoint{2.325000in}{2.310000in}}%
\pgfusepath{clip}%
\pgfsetrectcap%
\pgfsetroundjoin%
\pgfsetlinewidth{0.250937pt}%
\definecolor{currentstroke}{rgb}{0.000000,0.000000,0.000000}%
\pgfsetstrokecolor{currentstroke}%
\pgfsetstrokeopacity{0.200000}%
\pgfsetdash{}{0pt}%
\pgfpathmoveto{\pgfqpoint{2.115850in}{0.549073in}}%
\pgfpathlineto{\pgfqpoint{2.115850in}{2.859073in}}%
\pgfusepath{stroke}%
\end{pgfscope}%
\begin{pgfscope}%
\pgfsetbuttcap%
\pgfsetroundjoin%
\definecolor{currentfill}{rgb}{0.000000,0.000000,0.000000}%
\pgfsetfillcolor{currentfill}%
\pgfsetlinewidth{0.803000pt}%
\definecolor{currentstroke}{rgb}{0.000000,0.000000,0.000000}%
\pgfsetstrokecolor{currentstroke}%
\pgfsetdash{}{0pt}%
\pgfsys@defobject{currentmarker}{\pgfqpoint{0.000000in}{-0.048611in}}{\pgfqpoint{0.000000in}{0.000000in}}{%
\pgfpathmoveto{\pgfqpoint{0.000000in}{0.000000in}}%
\pgfpathlineto{\pgfqpoint{0.000000in}{-0.048611in}}%
\pgfusepath{stroke,fill}%
}%
\begin{pgfscope}%
\pgfsys@transformshift{2.115850in}{0.549073in}%
\pgfsys@useobject{currentmarker}{}%
\end{pgfscope}%
\end{pgfscope}%
\begin{pgfscope}%
\definecolor{textcolor}{rgb}{0.000000,0.000000,0.000000}%
\pgfsetstrokecolor{textcolor}%
\pgfsetfillcolor{textcolor}%
\pgftext[x=2.115850in,y=0.451851in,,top]{\color{textcolor}{\rmfamily\fontsize{12.000000}{14.400000}\selectfont\catcode`\^=\active\def^{\ifmmode\sp\else\^{}\fi}\catcode`\%=\active\def%{\%}$\mathdefault{10^{2}}$}}%
\end{pgfscope}%
\begin{pgfscope}%
\pgfsetbuttcap%
\pgfsetroundjoin%
\definecolor{currentfill}{rgb}{0.000000,0.000000,0.000000}%
\pgfsetfillcolor{currentfill}%
\pgfsetlinewidth{0.602250pt}%
\definecolor{currentstroke}{rgb}{0.000000,0.000000,0.000000}%
\pgfsetstrokecolor{currentstroke}%
\pgfsetdash{}{0pt}%
\pgfsys@defobject{currentmarker}{\pgfqpoint{0.000000in}{-0.027778in}}{\pgfqpoint{0.000000in}{0.000000in}}{%
\pgfpathmoveto{\pgfqpoint{0.000000in}{0.000000in}}%
\pgfpathlineto{\pgfqpoint{0.000000in}{-0.027778in}}%
\pgfusepath{stroke,fill}%
}%
\begin{pgfscope}%
\pgfsys@transformshift{0.777271in}{0.549073in}%
\pgfsys@useobject{currentmarker}{}%
\end{pgfscope}%
\end{pgfscope}%
\begin{pgfscope}%
\pgfsetbuttcap%
\pgfsetroundjoin%
\definecolor{currentfill}{rgb}{0.000000,0.000000,0.000000}%
\pgfsetfillcolor{currentfill}%
\pgfsetlinewidth{0.602250pt}%
\definecolor{currentstroke}{rgb}{0.000000,0.000000,0.000000}%
\pgfsetstrokecolor{currentstroke}%
\pgfsetdash{}{0pt}%
\pgfsys@defobject{currentmarker}{\pgfqpoint{0.000000in}{-0.027778in}}{\pgfqpoint{0.000000in}{0.000000in}}{%
\pgfpathmoveto{\pgfqpoint{0.000000in}{0.000000in}}%
\pgfpathlineto{\pgfqpoint{0.000000in}{-0.027778in}}%
\pgfusepath{stroke,fill}%
}%
\begin{pgfscope}%
\pgfsys@transformshift{0.850613in}{0.549073in}%
\pgfsys@useobject{currentmarker}{}%
\end{pgfscope}%
\end{pgfscope}%
\begin{pgfscope}%
\pgfsetbuttcap%
\pgfsetroundjoin%
\definecolor{currentfill}{rgb}{0.000000,0.000000,0.000000}%
\pgfsetfillcolor{currentfill}%
\pgfsetlinewidth{0.602250pt}%
\definecolor{currentstroke}{rgb}{0.000000,0.000000,0.000000}%
\pgfsetstrokecolor{currentstroke}%
\pgfsetdash{}{0pt}%
\pgfsys@defobject{currentmarker}{\pgfqpoint{0.000000in}{-0.027778in}}{\pgfqpoint{0.000000in}{0.000000in}}{%
\pgfpathmoveto{\pgfqpoint{0.000000in}{0.000000in}}%
\pgfpathlineto{\pgfqpoint{0.000000in}{-0.027778in}}%
\pgfusepath{stroke,fill}%
}%
\begin{pgfscope}%
\pgfsys@transformshift{0.914146in}{0.549073in}%
\pgfsys@useobject{currentmarker}{}%
\end{pgfscope}%
\end{pgfscope}%
\begin{pgfscope}%
\pgfsetbuttcap%
\pgfsetroundjoin%
\definecolor{currentfill}{rgb}{0.000000,0.000000,0.000000}%
\pgfsetfillcolor{currentfill}%
\pgfsetlinewidth{0.602250pt}%
\definecolor{currentstroke}{rgb}{0.000000,0.000000,0.000000}%
\pgfsetstrokecolor{currentstroke}%
\pgfsetdash{}{0pt}%
\pgfsys@defobject{currentmarker}{\pgfqpoint{0.000000in}{-0.027778in}}{\pgfqpoint{0.000000in}{0.000000in}}{%
\pgfpathmoveto{\pgfqpoint{0.000000in}{0.000000in}}%
\pgfpathlineto{\pgfqpoint{0.000000in}{-0.027778in}}%
\pgfusepath{stroke,fill}%
}%
\begin{pgfscope}%
\pgfsys@transformshift{0.970185in}{0.549073in}%
\pgfsys@useobject{currentmarker}{}%
\end{pgfscope}%
\end{pgfscope}%
\begin{pgfscope}%
\pgfsetbuttcap%
\pgfsetroundjoin%
\definecolor{currentfill}{rgb}{0.000000,0.000000,0.000000}%
\pgfsetfillcolor{currentfill}%
\pgfsetlinewidth{0.602250pt}%
\definecolor{currentstroke}{rgb}{0.000000,0.000000,0.000000}%
\pgfsetstrokecolor{currentstroke}%
\pgfsetdash{}{0pt}%
\pgfsys@defobject{currentmarker}{\pgfqpoint{0.000000in}{-0.027778in}}{\pgfqpoint{0.000000in}{0.000000in}}{%
\pgfpathmoveto{\pgfqpoint{0.000000in}{0.000000in}}%
\pgfpathlineto{\pgfqpoint{0.000000in}{-0.027778in}}%
\pgfusepath{stroke,fill}%
}%
\begin{pgfscope}%
\pgfsys@transformshift{1.350103in}{0.549073in}%
\pgfsys@useobject{currentmarker}{}%
\end{pgfscope}%
\end{pgfscope}%
\begin{pgfscope}%
\pgfsetbuttcap%
\pgfsetroundjoin%
\definecolor{currentfill}{rgb}{0.000000,0.000000,0.000000}%
\pgfsetfillcolor{currentfill}%
\pgfsetlinewidth{0.602250pt}%
\definecolor{currentstroke}{rgb}{0.000000,0.000000,0.000000}%
\pgfsetstrokecolor{currentstroke}%
\pgfsetdash{}{0pt}%
\pgfsys@defobject{currentmarker}{\pgfqpoint{0.000000in}{-0.027778in}}{\pgfqpoint{0.000000in}{0.000000in}}{%
\pgfpathmoveto{\pgfqpoint{0.000000in}{0.000000in}}%
\pgfpathlineto{\pgfqpoint{0.000000in}{-0.027778in}}%
\pgfusepath{stroke,fill}%
}%
\begin{pgfscope}%
\pgfsys@transformshift{1.543017in}{0.549073in}%
\pgfsys@useobject{currentmarker}{}%
\end{pgfscope}%
\end{pgfscope}%
\begin{pgfscope}%
\pgfsetbuttcap%
\pgfsetroundjoin%
\definecolor{currentfill}{rgb}{0.000000,0.000000,0.000000}%
\pgfsetfillcolor{currentfill}%
\pgfsetlinewidth{0.602250pt}%
\definecolor{currentstroke}{rgb}{0.000000,0.000000,0.000000}%
\pgfsetstrokecolor{currentstroke}%
\pgfsetdash{}{0pt}%
\pgfsys@defobject{currentmarker}{\pgfqpoint{0.000000in}{-0.027778in}}{\pgfqpoint{0.000000in}{0.000000in}}{%
\pgfpathmoveto{\pgfqpoint{0.000000in}{0.000000in}}%
\pgfpathlineto{\pgfqpoint{0.000000in}{-0.027778in}}%
\pgfusepath{stroke,fill}%
}%
\begin{pgfscope}%
\pgfsys@transformshift{1.679892in}{0.549073in}%
\pgfsys@useobject{currentmarker}{}%
\end{pgfscope}%
\end{pgfscope}%
\begin{pgfscope}%
\pgfsetbuttcap%
\pgfsetroundjoin%
\definecolor{currentfill}{rgb}{0.000000,0.000000,0.000000}%
\pgfsetfillcolor{currentfill}%
\pgfsetlinewidth{0.602250pt}%
\definecolor{currentstroke}{rgb}{0.000000,0.000000,0.000000}%
\pgfsetstrokecolor{currentstroke}%
\pgfsetdash{}{0pt}%
\pgfsys@defobject{currentmarker}{\pgfqpoint{0.000000in}{-0.027778in}}{\pgfqpoint{0.000000in}{0.000000in}}{%
\pgfpathmoveto{\pgfqpoint{0.000000in}{0.000000in}}%
\pgfpathlineto{\pgfqpoint{0.000000in}{-0.027778in}}%
\pgfusepath{stroke,fill}%
}%
\begin{pgfscope}%
\pgfsys@transformshift{1.786061in}{0.549073in}%
\pgfsys@useobject{currentmarker}{}%
\end{pgfscope}%
\end{pgfscope}%
\begin{pgfscope}%
\pgfsetbuttcap%
\pgfsetroundjoin%
\definecolor{currentfill}{rgb}{0.000000,0.000000,0.000000}%
\pgfsetfillcolor{currentfill}%
\pgfsetlinewidth{0.602250pt}%
\definecolor{currentstroke}{rgb}{0.000000,0.000000,0.000000}%
\pgfsetstrokecolor{currentstroke}%
\pgfsetdash{}{0pt}%
\pgfsys@defobject{currentmarker}{\pgfqpoint{0.000000in}{-0.027778in}}{\pgfqpoint{0.000000in}{0.000000in}}{%
\pgfpathmoveto{\pgfqpoint{0.000000in}{0.000000in}}%
\pgfpathlineto{\pgfqpoint{0.000000in}{-0.027778in}}%
\pgfusepath{stroke,fill}%
}%
\begin{pgfscope}%
\pgfsys@transformshift{1.872807in}{0.549073in}%
\pgfsys@useobject{currentmarker}{}%
\end{pgfscope}%
\end{pgfscope}%
\begin{pgfscope}%
\pgfsetbuttcap%
\pgfsetroundjoin%
\definecolor{currentfill}{rgb}{0.000000,0.000000,0.000000}%
\pgfsetfillcolor{currentfill}%
\pgfsetlinewidth{0.602250pt}%
\definecolor{currentstroke}{rgb}{0.000000,0.000000,0.000000}%
\pgfsetstrokecolor{currentstroke}%
\pgfsetdash{}{0pt}%
\pgfsys@defobject{currentmarker}{\pgfqpoint{0.000000in}{-0.027778in}}{\pgfqpoint{0.000000in}{0.000000in}}{%
\pgfpathmoveto{\pgfqpoint{0.000000in}{0.000000in}}%
\pgfpathlineto{\pgfqpoint{0.000000in}{-0.027778in}}%
\pgfusepath{stroke,fill}%
}%
\begin{pgfscope}%
\pgfsys@transformshift{1.946149in}{0.549073in}%
\pgfsys@useobject{currentmarker}{}%
\end{pgfscope}%
\end{pgfscope}%
\begin{pgfscope}%
\pgfsetbuttcap%
\pgfsetroundjoin%
\definecolor{currentfill}{rgb}{0.000000,0.000000,0.000000}%
\pgfsetfillcolor{currentfill}%
\pgfsetlinewidth{0.602250pt}%
\definecolor{currentstroke}{rgb}{0.000000,0.000000,0.000000}%
\pgfsetstrokecolor{currentstroke}%
\pgfsetdash{}{0pt}%
\pgfsys@defobject{currentmarker}{\pgfqpoint{0.000000in}{-0.027778in}}{\pgfqpoint{0.000000in}{0.000000in}}{%
\pgfpathmoveto{\pgfqpoint{0.000000in}{0.000000in}}%
\pgfpathlineto{\pgfqpoint{0.000000in}{-0.027778in}}%
\pgfusepath{stroke,fill}%
}%
\begin{pgfscope}%
\pgfsys@transformshift{2.009682in}{0.549073in}%
\pgfsys@useobject{currentmarker}{}%
\end{pgfscope}%
\end{pgfscope}%
\begin{pgfscope}%
\pgfsetbuttcap%
\pgfsetroundjoin%
\definecolor{currentfill}{rgb}{0.000000,0.000000,0.000000}%
\pgfsetfillcolor{currentfill}%
\pgfsetlinewidth{0.602250pt}%
\definecolor{currentstroke}{rgb}{0.000000,0.000000,0.000000}%
\pgfsetstrokecolor{currentstroke}%
\pgfsetdash{}{0pt}%
\pgfsys@defobject{currentmarker}{\pgfqpoint{0.000000in}{-0.027778in}}{\pgfqpoint{0.000000in}{0.000000in}}{%
\pgfpathmoveto{\pgfqpoint{0.000000in}{0.000000in}}%
\pgfpathlineto{\pgfqpoint{0.000000in}{-0.027778in}}%
\pgfusepath{stroke,fill}%
}%
\begin{pgfscope}%
\pgfsys@transformshift{2.065721in}{0.549073in}%
\pgfsys@useobject{currentmarker}{}%
\end{pgfscope}%
\end{pgfscope}%
\begin{pgfscope}%
\pgfsetbuttcap%
\pgfsetroundjoin%
\definecolor{currentfill}{rgb}{0.000000,0.000000,0.000000}%
\pgfsetfillcolor{currentfill}%
\pgfsetlinewidth{0.602250pt}%
\definecolor{currentstroke}{rgb}{0.000000,0.000000,0.000000}%
\pgfsetstrokecolor{currentstroke}%
\pgfsetdash{}{0pt}%
\pgfsys@defobject{currentmarker}{\pgfqpoint{0.000000in}{-0.027778in}}{\pgfqpoint{0.000000in}{0.000000in}}{%
\pgfpathmoveto{\pgfqpoint{0.000000in}{0.000000in}}%
\pgfpathlineto{\pgfqpoint{0.000000in}{-0.027778in}}%
\pgfusepath{stroke,fill}%
}%
\begin{pgfscope}%
\pgfsys@transformshift{2.445639in}{0.549073in}%
\pgfsys@useobject{currentmarker}{}%
\end{pgfscope}%
\end{pgfscope}%
\begin{pgfscope}%
\pgfsetbuttcap%
\pgfsetroundjoin%
\definecolor{currentfill}{rgb}{0.000000,0.000000,0.000000}%
\pgfsetfillcolor{currentfill}%
\pgfsetlinewidth{0.602250pt}%
\definecolor{currentstroke}{rgb}{0.000000,0.000000,0.000000}%
\pgfsetstrokecolor{currentstroke}%
\pgfsetdash{}{0pt}%
\pgfsys@defobject{currentmarker}{\pgfqpoint{0.000000in}{-0.027778in}}{\pgfqpoint{0.000000in}{0.000000in}}{%
\pgfpathmoveto{\pgfqpoint{0.000000in}{0.000000in}}%
\pgfpathlineto{\pgfqpoint{0.000000in}{-0.027778in}}%
\pgfusepath{stroke,fill}%
}%
\begin{pgfscope}%
\pgfsys@transformshift{2.638553in}{0.549073in}%
\pgfsys@useobject{currentmarker}{}%
\end{pgfscope}%
\end{pgfscope}%
\begin{pgfscope}%
\pgfsetbuttcap%
\pgfsetroundjoin%
\definecolor{currentfill}{rgb}{0.000000,0.000000,0.000000}%
\pgfsetfillcolor{currentfill}%
\pgfsetlinewidth{0.602250pt}%
\definecolor{currentstroke}{rgb}{0.000000,0.000000,0.000000}%
\pgfsetstrokecolor{currentstroke}%
\pgfsetdash{}{0pt}%
\pgfsys@defobject{currentmarker}{\pgfqpoint{0.000000in}{-0.027778in}}{\pgfqpoint{0.000000in}{0.000000in}}{%
\pgfpathmoveto{\pgfqpoint{0.000000in}{0.000000in}}%
\pgfpathlineto{\pgfqpoint{0.000000in}{-0.027778in}}%
\pgfusepath{stroke,fill}%
}%
\begin{pgfscope}%
\pgfsys@transformshift{2.775428in}{0.549073in}%
\pgfsys@useobject{currentmarker}{}%
\end{pgfscope}%
\end{pgfscope}%
\begin{pgfscope}%
\pgfsetbuttcap%
\pgfsetroundjoin%
\definecolor{currentfill}{rgb}{0.000000,0.000000,0.000000}%
\pgfsetfillcolor{currentfill}%
\pgfsetlinewidth{0.602250pt}%
\definecolor{currentstroke}{rgb}{0.000000,0.000000,0.000000}%
\pgfsetstrokecolor{currentstroke}%
\pgfsetdash{}{0pt}%
\pgfsys@defobject{currentmarker}{\pgfqpoint{0.000000in}{-0.027778in}}{\pgfqpoint{0.000000in}{0.000000in}}{%
\pgfpathmoveto{\pgfqpoint{0.000000in}{0.000000in}}%
\pgfpathlineto{\pgfqpoint{0.000000in}{-0.027778in}}%
\pgfusepath{stroke,fill}%
}%
\begin{pgfscope}%
\pgfsys@transformshift{2.881597in}{0.549073in}%
\pgfsys@useobject{currentmarker}{}%
\end{pgfscope}%
\end{pgfscope}%
\begin{pgfscope}%
\pgfsetbuttcap%
\pgfsetroundjoin%
\definecolor{currentfill}{rgb}{0.000000,0.000000,0.000000}%
\pgfsetfillcolor{currentfill}%
\pgfsetlinewidth{0.602250pt}%
\definecolor{currentstroke}{rgb}{0.000000,0.000000,0.000000}%
\pgfsetstrokecolor{currentstroke}%
\pgfsetdash{}{0pt}%
\pgfsys@defobject{currentmarker}{\pgfqpoint{0.000000in}{-0.027778in}}{\pgfqpoint{0.000000in}{0.000000in}}{%
\pgfpathmoveto{\pgfqpoint{0.000000in}{0.000000in}}%
\pgfpathlineto{\pgfqpoint{0.000000in}{-0.027778in}}%
\pgfusepath{stroke,fill}%
}%
\begin{pgfscope}%
\pgfsys@transformshift{2.968343in}{0.549073in}%
\pgfsys@useobject{currentmarker}{}%
\end{pgfscope}%
\end{pgfscope}%
\begin{pgfscope}%
\pgfsetbuttcap%
\pgfsetroundjoin%
\definecolor{currentfill}{rgb}{0.000000,0.000000,0.000000}%
\pgfsetfillcolor{currentfill}%
\pgfsetlinewidth{0.602250pt}%
\definecolor{currentstroke}{rgb}{0.000000,0.000000,0.000000}%
\pgfsetstrokecolor{currentstroke}%
\pgfsetdash{}{0pt}%
\pgfsys@defobject{currentmarker}{\pgfqpoint{0.000000in}{-0.027778in}}{\pgfqpoint{0.000000in}{0.000000in}}{%
\pgfpathmoveto{\pgfqpoint{0.000000in}{0.000000in}}%
\pgfpathlineto{\pgfqpoint{0.000000in}{-0.027778in}}%
\pgfusepath{stroke,fill}%
}%
\begin{pgfscope}%
\pgfsys@transformshift{3.041685in}{0.549073in}%
\pgfsys@useobject{currentmarker}{}%
\end{pgfscope}%
\end{pgfscope}%
\begin{pgfscope}%
\definecolor{textcolor}{rgb}{0.000000,0.000000,0.000000}%
\pgfsetstrokecolor{textcolor}%
\pgfsetfillcolor{textcolor}%
\pgftext[x=1.884413in,y=0.248148in,,top]{\color{textcolor}{\rmfamily\fontsize{12.000000}{14.400000}\selectfont\catcode`\^=\active\def^{\ifmmode\sp\else\^{}\fi}\catcode`\%=\active\def%{\%}estimator size $n_{\mathbf{\Omega}} + n_{\mathbf{\Psi}}$}}%
\end{pgfscope}%
\begin{pgfscope}%
\pgfpathrectangle{\pgfqpoint{0.721913in}{0.549073in}}{\pgfqpoint{2.325000in}{2.310000in}}%
\pgfusepath{clip}%
\pgfsetrectcap%
\pgfsetroundjoin%
\pgfsetlinewidth{0.250937pt}%
\definecolor{currentstroke}{rgb}{0.000000,0.000000,0.000000}%
\pgfsetstrokecolor{currentstroke}%
\pgfsetstrokeopacity{0.200000}%
\pgfsetdash{}{0pt}%
\pgfpathmoveto{\pgfqpoint{0.721913in}{0.867118in}}%
\pgfpathlineto{\pgfqpoint{3.046913in}{0.867118in}}%
\pgfusepath{stroke}%
\end{pgfscope}%
\begin{pgfscope}%
\pgfsetbuttcap%
\pgfsetroundjoin%
\definecolor{currentfill}{rgb}{0.000000,0.000000,0.000000}%
\pgfsetfillcolor{currentfill}%
\pgfsetlinewidth{0.803000pt}%
\definecolor{currentstroke}{rgb}{0.000000,0.000000,0.000000}%
\pgfsetstrokecolor{currentstroke}%
\pgfsetdash{}{0pt}%
\pgfsys@defobject{currentmarker}{\pgfqpoint{-0.048611in}{0.000000in}}{\pgfqpoint{-0.000000in}{0.000000in}}{%
\pgfpathmoveto{\pgfqpoint{-0.000000in}{0.000000in}}%
\pgfpathlineto{\pgfqpoint{-0.048611in}{0.000000in}}%
\pgfusepath{stroke,fill}%
}%
\begin{pgfscope}%
\pgfsys@transformshift{0.721913in}{0.867118in}%
\pgfsys@useobject{currentmarker}{}%
\end{pgfscope}%
\end{pgfscope}%
\begin{pgfscope}%
\definecolor{textcolor}{rgb}{0.000000,0.000000,0.000000}%
\pgfsetstrokecolor{textcolor}%
\pgfsetfillcolor{textcolor}%
\pgftext[x=0.303703in, y=0.809248in, left, base]{\color{textcolor}{\rmfamily\fontsize{12.000000}{14.400000}\selectfont\catcode`\^=\active\def^{\ifmmode\sp\else\^{}\fi}\catcode`\%=\active\def%{\%}$\mathdefault{10^{-6}}$}}%
\end{pgfscope}%
\begin{pgfscope}%
\pgfpathrectangle{\pgfqpoint{0.721913in}{0.549073in}}{\pgfqpoint{2.325000in}{2.310000in}}%
\pgfusepath{clip}%
\pgfsetrectcap%
\pgfsetroundjoin%
\pgfsetlinewidth{0.250937pt}%
\definecolor{currentstroke}{rgb}{0.000000,0.000000,0.000000}%
\pgfsetstrokecolor{currentstroke}%
\pgfsetstrokeopacity{0.200000}%
\pgfsetdash{}{0pt}%
\pgfpathmoveto{\pgfqpoint{0.721913in}{1.483024in}}%
\pgfpathlineto{\pgfqpoint{3.046913in}{1.483024in}}%
\pgfusepath{stroke}%
\end{pgfscope}%
\begin{pgfscope}%
\pgfsetbuttcap%
\pgfsetroundjoin%
\definecolor{currentfill}{rgb}{0.000000,0.000000,0.000000}%
\pgfsetfillcolor{currentfill}%
\pgfsetlinewidth{0.803000pt}%
\definecolor{currentstroke}{rgb}{0.000000,0.000000,0.000000}%
\pgfsetstrokecolor{currentstroke}%
\pgfsetdash{}{0pt}%
\pgfsys@defobject{currentmarker}{\pgfqpoint{-0.048611in}{0.000000in}}{\pgfqpoint{-0.000000in}{0.000000in}}{%
\pgfpathmoveto{\pgfqpoint{-0.000000in}{0.000000in}}%
\pgfpathlineto{\pgfqpoint{-0.048611in}{0.000000in}}%
\pgfusepath{stroke,fill}%
}%
\begin{pgfscope}%
\pgfsys@transformshift{0.721913in}{1.483024in}%
\pgfsys@useobject{currentmarker}{}%
\end{pgfscope}%
\end{pgfscope}%
\begin{pgfscope}%
\definecolor{textcolor}{rgb}{0.000000,0.000000,0.000000}%
\pgfsetstrokecolor{textcolor}%
\pgfsetfillcolor{textcolor}%
\pgftext[x=0.303703in, y=1.425154in, left, base]{\color{textcolor}{\rmfamily\fontsize{12.000000}{14.400000}\selectfont\catcode`\^=\active\def^{\ifmmode\sp\else\^{}\fi}\catcode`\%=\active\def%{\%}$\mathdefault{10^{-4}}$}}%
\end{pgfscope}%
\begin{pgfscope}%
\pgfpathrectangle{\pgfqpoint{0.721913in}{0.549073in}}{\pgfqpoint{2.325000in}{2.310000in}}%
\pgfusepath{clip}%
\pgfsetrectcap%
\pgfsetroundjoin%
\pgfsetlinewidth{0.250937pt}%
\definecolor{currentstroke}{rgb}{0.000000,0.000000,0.000000}%
\pgfsetstrokecolor{currentstroke}%
\pgfsetstrokeopacity{0.200000}%
\pgfsetdash{}{0pt}%
\pgfpathmoveto{\pgfqpoint{0.721913in}{2.098931in}}%
\pgfpathlineto{\pgfqpoint{3.046913in}{2.098931in}}%
\pgfusepath{stroke}%
\end{pgfscope}%
\begin{pgfscope}%
\pgfsetbuttcap%
\pgfsetroundjoin%
\definecolor{currentfill}{rgb}{0.000000,0.000000,0.000000}%
\pgfsetfillcolor{currentfill}%
\pgfsetlinewidth{0.803000pt}%
\definecolor{currentstroke}{rgb}{0.000000,0.000000,0.000000}%
\pgfsetstrokecolor{currentstroke}%
\pgfsetdash{}{0pt}%
\pgfsys@defobject{currentmarker}{\pgfqpoint{-0.048611in}{0.000000in}}{\pgfqpoint{-0.000000in}{0.000000in}}{%
\pgfpathmoveto{\pgfqpoint{-0.000000in}{0.000000in}}%
\pgfpathlineto{\pgfqpoint{-0.048611in}{0.000000in}}%
\pgfusepath{stroke,fill}%
}%
\begin{pgfscope}%
\pgfsys@transformshift{0.721913in}{2.098931in}%
\pgfsys@useobject{currentmarker}{}%
\end{pgfscope}%
\end{pgfscope}%
\begin{pgfscope}%
\definecolor{textcolor}{rgb}{0.000000,0.000000,0.000000}%
\pgfsetstrokecolor{textcolor}%
\pgfsetfillcolor{textcolor}%
\pgftext[x=0.303703in, y=2.041060in, left, base]{\color{textcolor}{\rmfamily\fontsize{12.000000}{14.400000}\selectfont\catcode`\^=\active\def^{\ifmmode\sp\else\^{}\fi}\catcode`\%=\active\def%{\%}$\mathdefault{10^{-2}}$}}%
\end{pgfscope}%
\begin{pgfscope}%
\pgfpathrectangle{\pgfqpoint{0.721913in}{0.549073in}}{\pgfqpoint{2.325000in}{2.310000in}}%
\pgfusepath{clip}%
\pgfsetrectcap%
\pgfsetroundjoin%
\pgfsetlinewidth{0.250937pt}%
\definecolor{currentstroke}{rgb}{0.000000,0.000000,0.000000}%
\pgfsetstrokecolor{currentstroke}%
\pgfsetstrokeopacity{0.200000}%
\pgfsetdash{}{0pt}%
\pgfpathmoveto{\pgfqpoint{0.721913in}{2.714837in}}%
\pgfpathlineto{\pgfqpoint{3.046913in}{2.714837in}}%
\pgfusepath{stroke}%
\end{pgfscope}%
\begin{pgfscope}%
\pgfsetbuttcap%
\pgfsetroundjoin%
\definecolor{currentfill}{rgb}{0.000000,0.000000,0.000000}%
\pgfsetfillcolor{currentfill}%
\pgfsetlinewidth{0.803000pt}%
\definecolor{currentstroke}{rgb}{0.000000,0.000000,0.000000}%
\pgfsetstrokecolor{currentstroke}%
\pgfsetdash{}{0pt}%
\pgfsys@defobject{currentmarker}{\pgfqpoint{-0.048611in}{0.000000in}}{\pgfqpoint{-0.000000in}{0.000000in}}{%
\pgfpathmoveto{\pgfqpoint{-0.000000in}{0.000000in}}%
\pgfpathlineto{\pgfqpoint{-0.048611in}{0.000000in}}%
\pgfusepath{stroke,fill}%
}%
\begin{pgfscope}%
\pgfsys@transformshift{0.721913in}{2.714837in}%
\pgfsys@useobject{currentmarker}{}%
\end{pgfscope}%
\end{pgfscope}%
\begin{pgfscope}%
\definecolor{textcolor}{rgb}{0.000000,0.000000,0.000000}%
\pgfsetstrokecolor{textcolor}%
\pgfsetfillcolor{textcolor}%
\pgftext[x=0.395525in, y=2.656967in, left, base]{\color{textcolor}{\rmfamily\fontsize{12.000000}{14.400000}\selectfont\catcode`\^=\active\def^{\ifmmode\sp\else\^{}\fi}\catcode`\%=\active\def%{\%}$\mathdefault{10^{0}}$}}%
\end{pgfscope}%
\begin{pgfscope}%
\definecolor{textcolor}{rgb}{0.000000,0.000000,0.000000}%
\pgfsetstrokecolor{textcolor}%
\pgfsetfillcolor{textcolor}%
\pgftext[x=0.248147in,y=1.704073in,,bottom,rotate=90.000000]{\color{textcolor}{\rmfamily\fontsize{12.000000}{14.400000}\selectfont\catcode`\^=\active\def^{\ifmmode\sp\else\^{}\fi}\catcode`\%=\active\def%{\%}$L^1$-error}}%
\end{pgfscope}%
\begin{pgfscope}%
\pgfpathrectangle{\pgfqpoint{0.721913in}{0.549073in}}{\pgfqpoint{2.325000in}{2.310000in}}%
\pgfusepath{clip}%
\pgfsetrectcap%
\pgfsetroundjoin%
\pgfsetlinewidth{1.505625pt}%
\definecolor{currentstroke}{rgb}{0.392157,0.560784,1.000000}%
\pgfsetstrokecolor{currentstroke}%
\pgfsetdash{}{0pt}%
\pgfpathmoveto{\pgfqpoint{0.777271in}{2.426348in}}%
\pgfpathlineto{\pgfqpoint{1.145143in}{2.356873in}}%
\pgfpathlineto{\pgfqpoint{1.526888in}{2.309199in}}%
\pgfpathlineto{\pgfqpoint{1.896020in}{2.264835in}}%
\pgfpathlineto{\pgfqpoint{2.258635in}{2.224581in}}%
\pgfpathlineto{\pgfqpoint{2.625694in}{2.135300in}}%
\pgfpathlineto{\pgfqpoint{2.991556in}{2.122196in}}%
\pgfusepath{stroke}%
\end{pgfscope}%
\begin{pgfscope}%
\pgfpathrectangle{\pgfqpoint{0.721913in}{0.549073in}}{\pgfqpoint{2.325000in}{2.310000in}}%
\pgfusepath{clip}%
\pgfsetbuttcap%
\pgfsetroundjoin%
\definecolor{currentfill}{rgb}{0.392157,0.560784,1.000000}%
\pgfsetfillcolor{currentfill}%
\pgfsetlinewidth{1.003750pt}%
\definecolor{currentstroke}{rgb}{0.392157,0.560784,1.000000}%
\pgfsetstrokecolor{currentstroke}%
\pgfsetdash{}{0pt}%
\pgfsys@defobject{currentmarker}{\pgfqpoint{-0.041667in}{-0.041667in}}{\pgfqpoint{0.041667in}{0.041667in}}{%
\pgfpathmoveto{\pgfqpoint{0.000000in}{-0.041667in}}%
\pgfpathcurveto{\pgfqpoint{0.011050in}{-0.041667in}}{\pgfqpoint{0.021649in}{-0.037276in}}{\pgfqpoint{0.029463in}{-0.029463in}}%
\pgfpathcurveto{\pgfqpoint{0.037276in}{-0.021649in}}{\pgfqpoint{0.041667in}{-0.011050in}}{\pgfqpoint{0.041667in}{0.000000in}}%
\pgfpathcurveto{\pgfqpoint{0.041667in}{0.011050in}}{\pgfqpoint{0.037276in}{0.021649in}}{\pgfqpoint{0.029463in}{0.029463in}}%
\pgfpathcurveto{\pgfqpoint{0.021649in}{0.037276in}}{\pgfqpoint{0.011050in}{0.041667in}}{\pgfqpoint{0.000000in}{0.041667in}}%
\pgfpathcurveto{\pgfqpoint{-0.011050in}{0.041667in}}{\pgfqpoint{-0.021649in}{0.037276in}}{\pgfqpoint{-0.029463in}{0.029463in}}%
\pgfpathcurveto{\pgfqpoint{-0.037276in}{0.021649in}}{\pgfqpoint{-0.041667in}{0.011050in}}{\pgfqpoint{-0.041667in}{0.000000in}}%
\pgfpathcurveto{\pgfqpoint{-0.041667in}{-0.011050in}}{\pgfqpoint{-0.037276in}{-0.021649in}}{\pgfqpoint{-0.029463in}{-0.029463in}}%
\pgfpathcurveto{\pgfqpoint{-0.021649in}{-0.037276in}}{\pgfqpoint{-0.011050in}{-0.041667in}}{\pgfqpoint{0.000000in}{-0.041667in}}%
\pgfpathlineto{\pgfqpoint{0.000000in}{-0.041667in}}%
\pgfpathclose%
\pgfusepath{stroke,fill}%
}%
\begin{pgfscope}%
\pgfsys@transformshift{0.777271in}{2.426348in}%
\pgfsys@useobject{currentmarker}{}%
\end{pgfscope}%
\begin{pgfscope}%
\pgfsys@transformshift{1.145143in}{2.356873in}%
\pgfsys@useobject{currentmarker}{}%
\end{pgfscope}%
\begin{pgfscope}%
\pgfsys@transformshift{1.526888in}{2.309199in}%
\pgfsys@useobject{currentmarker}{}%
\end{pgfscope}%
\begin{pgfscope}%
\pgfsys@transformshift{1.896020in}{2.264835in}%
\pgfsys@useobject{currentmarker}{}%
\end{pgfscope}%
\begin{pgfscope}%
\pgfsys@transformshift{2.258635in}{2.224581in}%
\pgfsys@useobject{currentmarker}{}%
\end{pgfscope}%
\begin{pgfscope}%
\pgfsys@transformshift{2.625694in}{2.135300in}%
\pgfsys@useobject{currentmarker}{}%
\end{pgfscope}%
\begin{pgfscope}%
\pgfsys@transformshift{2.991556in}{2.122196in}%
\pgfsys@useobject{currentmarker}{}%
\end{pgfscope}%
\end{pgfscope}%
\begin{pgfscope}%
\pgfpathrectangle{\pgfqpoint{0.721913in}{0.549073in}}{\pgfqpoint{2.325000in}{2.310000in}}%
\pgfusepath{clip}%
\pgfsetrectcap%
\pgfsetroundjoin%
\pgfsetlinewidth{1.505625pt}%
\definecolor{currentstroke}{rgb}{0.862745,0.149020,0.498039}%
\pgfsetstrokecolor{currentstroke}%
\pgfsetdash{}{0pt}%
\pgfpathmoveto{\pgfqpoint{0.777271in}{2.410049in}}%
\pgfpathlineto{\pgfqpoint{1.145143in}{2.327027in}}%
\pgfpathlineto{\pgfqpoint{1.526888in}{2.247548in}}%
\pgfpathlineto{\pgfqpoint{1.896020in}{2.068359in}}%
\pgfpathlineto{\pgfqpoint{2.258635in}{1.755724in}}%
\pgfpathlineto{\pgfqpoint{2.625694in}{0.772538in}}%
\pgfpathlineto{\pgfqpoint{2.991556in}{0.741573in}}%
\pgfusepath{stroke}%
\end{pgfscope}%
\begin{pgfscope}%
\pgfpathrectangle{\pgfqpoint{0.721913in}{0.549073in}}{\pgfqpoint{2.325000in}{2.310000in}}%
\pgfusepath{clip}%
\pgfsetbuttcap%
\pgfsetmiterjoin%
\definecolor{currentfill}{rgb}{0.862745,0.149020,0.498039}%
\pgfsetfillcolor{currentfill}%
\pgfsetlinewidth{1.003750pt}%
\definecolor{currentstroke}{rgb}{0.862745,0.149020,0.498039}%
\pgfsetstrokecolor{currentstroke}%
\pgfsetdash{}{0pt}%
\pgfsys@defobject{currentmarker}{\pgfqpoint{-0.041667in}{-0.041667in}}{\pgfqpoint{0.041667in}{0.041667in}}{%
\pgfpathmoveto{\pgfqpoint{-0.041667in}{-0.041667in}}%
\pgfpathlineto{\pgfqpoint{0.041667in}{-0.041667in}}%
\pgfpathlineto{\pgfqpoint{0.041667in}{0.041667in}}%
\pgfpathlineto{\pgfqpoint{-0.041667in}{0.041667in}}%
\pgfpathlineto{\pgfqpoint{-0.041667in}{-0.041667in}}%
\pgfpathclose%
\pgfusepath{stroke,fill}%
}%
\begin{pgfscope}%
\pgfsys@transformshift{0.777271in}{2.410049in}%
\pgfsys@useobject{currentmarker}{}%
\end{pgfscope}%
\begin{pgfscope}%
\pgfsys@transformshift{1.145143in}{2.327027in}%
\pgfsys@useobject{currentmarker}{}%
\end{pgfscope}%
\begin{pgfscope}%
\pgfsys@transformshift{1.526888in}{2.247548in}%
\pgfsys@useobject{currentmarker}{}%
\end{pgfscope}%
\begin{pgfscope}%
\pgfsys@transformshift{1.896020in}{2.068359in}%
\pgfsys@useobject{currentmarker}{}%
\end{pgfscope}%
\begin{pgfscope}%
\pgfsys@transformshift{2.258635in}{1.755724in}%
\pgfsys@useobject{currentmarker}{}%
\end{pgfscope}%
\begin{pgfscope}%
\pgfsys@transformshift{2.625694in}{0.772538in}%
\pgfsys@useobject{currentmarker}{}%
\end{pgfscope}%
\begin{pgfscope}%
\pgfsys@transformshift{2.991556in}{0.741573in}%
\pgfsys@useobject{currentmarker}{}%
\end{pgfscope}%
\end{pgfscope}%
\begin{pgfscope}%
\pgfpathrectangle{\pgfqpoint{0.721913in}{0.549073in}}{\pgfqpoint{2.325000in}{2.310000in}}%
\pgfusepath{clip}%
\pgfsetrectcap%
\pgfsetroundjoin%
\pgfsetlinewidth{1.505625pt}%
\definecolor{currentstroke}{rgb}{1.000000,0.690196,0.000000}%
\pgfsetstrokecolor{currentstroke}%
\pgfsetdash{}{0pt}%
\pgfpathmoveto{\pgfqpoint{0.777271in}{2.666573in}}%
\pgfpathlineto{\pgfqpoint{1.145143in}{2.607369in}}%
\pgfpathlineto{\pgfqpoint{1.526888in}{2.493233in}}%
\pgfpathlineto{\pgfqpoint{1.896020in}{2.240947in}}%
\pgfpathlineto{\pgfqpoint{2.258635in}{1.023682in}}%
\pgfpathlineto{\pgfqpoint{2.625694in}{0.900987in}}%
\pgfpathlineto{\pgfqpoint{2.991556in}{0.860469in}}%
\pgfusepath{stroke}%
\end{pgfscope}%
\begin{pgfscope}%
\pgfpathrectangle{\pgfqpoint{0.721913in}{0.549073in}}{\pgfqpoint{2.325000in}{2.310000in}}%
\pgfusepath{clip}%
\pgfsetbuttcap%
\pgfsetmiterjoin%
\definecolor{currentfill}{rgb}{1.000000,0.690196,0.000000}%
\pgfsetfillcolor{currentfill}%
\pgfsetlinewidth{1.003750pt}%
\definecolor{currentstroke}{rgb}{1.000000,0.690196,0.000000}%
\pgfsetstrokecolor{currentstroke}%
\pgfsetdash{}{0pt}%
\pgfsys@defobject{currentmarker}{\pgfqpoint{-0.035355in}{-0.058926in}}{\pgfqpoint{0.035355in}{0.058926in}}{%
\pgfpathmoveto{\pgfqpoint{-0.000000in}{-0.058926in}}%
\pgfpathlineto{\pgfqpoint{0.035355in}{0.000000in}}%
\pgfpathlineto{\pgfqpoint{0.000000in}{0.058926in}}%
\pgfpathlineto{\pgfqpoint{-0.035355in}{0.000000in}}%
\pgfpathlineto{\pgfqpoint{-0.000000in}{-0.058926in}}%
\pgfpathclose%
\pgfusepath{stroke,fill}%
}%
\begin{pgfscope}%
\pgfsys@transformshift{0.777271in}{2.666573in}%
\pgfsys@useobject{currentmarker}{}%
\end{pgfscope}%
\begin{pgfscope}%
\pgfsys@transformshift{1.145143in}{2.607369in}%
\pgfsys@useobject{currentmarker}{}%
\end{pgfscope}%
\begin{pgfscope}%
\pgfsys@transformshift{1.526888in}{2.493233in}%
\pgfsys@useobject{currentmarker}{}%
\end{pgfscope}%
\begin{pgfscope}%
\pgfsys@transformshift{1.896020in}{2.240947in}%
\pgfsys@useobject{currentmarker}{}%
\end{pgfscope}%
\begin{pgfscope}%
\pgfsys@transformshift{2.258635in}{1.023682in}%
\pgfsys@useobject{currentmarker}{}%
\end{pgfscope}%
\begin{pgfscope}%
\pgfsys@transformshift{2.625694in}{0.900987in}%
\pgfsys@useobject{currentmarker}{}%
\end{pgfscope}%
\begin{pgfscope}%
\pgfsys@transformshift{2.991556in}{0.860469in}%
\pgfsys@useobject{currentmarker}{}%
\end{pgfscope}%
\end{pgfscope}%
\begin{pgfscope}%
\pgfpathrectangle{\pgfqpoint{0.721913in}{0.549073in}}{\pgfqpoint{2.325000in}{2.310000in}}%
\pgfusepath{clip}%
\pgfsetbuttcap%
\pgfsetroundjoin%
\pgfsetlinewidth{1.505625pt}%
\definecolor{currentstroke}{rgb}{0.478431,0.478431,0.478431}%
\pgfsetstrokecolor{currentstroke}%
\pgfsetstrokeopacity{0.500000}%
\pgfsetdash{{5.550000pt}{2.400000pt}}{0.000000pt}%
\pgfpathmoveto{\pgfqpoint{0.777271in}{2.334797in}}%
\pgfpathlineto{\pgfqpoint{1.145143in}{2.231389in}}%
\pgfpathlineto{\pgfqpoint{1.526888in}{2.124081in}}%
\pgfpathlineto{\pgfqpoint{1.896020in}{2.020319in}}%
\pgfpathlineto{\pgfqpoint{2.258635in}{1.918388in}}%
\pgfpathlineto{\pgfqpoint{2.625694in}{1.815209in}}%
\pgfpathlineto{\pgfqpoint{2.991556in}{1.712366in}}%
\pgfusepath{stroke}%
\end{pgfscope}%
\begin{pgfscope}%
\pgfpathrectangle{\pgfqpoint{0.721913in}{0.549073in}}{\pgfqpoint{2.325000in}{2.310000in}}%
\pgfusepath{clip}%
\pgfsetbuttcap%
\pgfsetroundjoin%
\pgfsetlinewidth{1.505625pt}%
\definecolor{currentstroke}{rgb}{0.478431,0.478431,0.478431}%
\pgfsetstrokecolor{currentstroke}%
\pgfsetstrokeopacity{0.500000}%
\pgfsetdash{{5.550000pt}{2.400000pt}}{0.000000pt}%
\pgfpathmoveto{\pgfqpoint{0.777271in}{2.488226in}}%
\pgfpathlineto{\pgfqpoint{1.145143in}{2.436521in}}%
\pgfpathlineto{\pgfqpoint{1.526888in}{2.382868in}}%
\pgfpathlineto{\pgfqpoint{1.896020in}{2.330986in}}%
\pgfpathlineto{\pgfqpoint{2.258635in}{2.280021in}}%
\pgfpathlineto{\pgfqpoint{2.625694in}{2.228431in}}%
\pgfpathlineto{\pgfqpoint{2.991556in}{2.177010in}}%
\pgfusepath{stroke}%
\end{pgfscope}%
\begin{pgfscope}%
\pgfsetrectcap%
\pgfsetmiterjoin%
\pgfsetlinewidth{0.803000pt}%
\definecolor{currentstroke}{rgb}{0.000000,0.000000,0.000000}%
\pgfsetstrokecolor{currentstroke}%
\pgfsetdash{}{0pt}%
\pgfpathmoveto{\pgfqpoint{0.721913in}{0.549073in}}%
\pgfpathlineto{\pgfqpoint{0.721913in}{2.859073in}}%
\pgfusepath{stroke}%
\end{pgfscope}%
\begin{pgfscope}%
\pgfsetrectcap%
\pgfsetmiterjoin%
\pgfsetlinewidth{0.803000pt}%
\definecolor{currentstroke}{rgb}{0.000000,0.000000,0.000000}%
\pgfsetstrokecolor{currentstroke}%
\pgfsetdash{}{0pt}%
\pgfpathmoveto{\pgfqpoint{3.046913in}{0.549073in}}%
\pgfpathlineto{\pgfqpoint{3.046913in}{2.859073in}}%
\pgfusepath{stroke}%
\end{pgfscope}%
\begin{pgfscope}%
\pgfsetrectcap%
\pgfsetmiterjoin%
\pgfsetlinewidth{0.803000pt}%
\definecolor{currentstroke}{rgb}{0.000000,0.000000,0.000000}%
\pgfsetstrokecolor{currentstroke}%
\pgfsetdash{}{0pt}%
\pgfpathmoveto{\pgfqpoint{0.721913in}{0.549073in}}%
\pgfpathlineto{\pgfqpoint{3.046913in}{0.549073in}}%
\pgfusepath{stroke}%
\end{pgfscope}%
\begin{pgfscope}%
\pgfsetrectcap%
\pgfsetmiterjoin%
\pgfsetlinewidth{0.803000pt}%
\definecolor{currentstroke}{rgb}{0.000000,0.000000,0.000000}%
\pgfsetstrokecolor{currentstroke}%
\pgfsetdash{}{0pt}%
\pgfpathmoveto{\pgfqpoint{0.721913in}{2.859073in}}%
\pgfpathlineto{\pgfqpoint{3.046913in}{2.859073in}}%
\pgfusepath{stroke}%
\end{pgfscope}%
\begin{pgfscope}%
\definecolor{textcolor}{rgb}{0.478431,0.478431,0.478431}%
\pgfsetstrokecolor{textcolor}%
\pgfsetfillcolor{textcolor}%
\pgftext[x=2.009682in,y=2.006228in,left,base]{\color{textcolor}{\rmfamily\fontsize{12.000000}{14.400000}\selectfont\catcode`\^=\active\def^{\ifmmode\sp\else\^{}\fi}\catcode`\%=\active\def%{\%}$\mathcal{O}(\varepsilon^{-1})$}}%
\end{pgfscope}%
\begin{pgfscope}%
\definecolor{textcolor}{rgb}{0.478431,0.478431,0.478431}%
\pgfsetstrokecolor{textcolor}%
\pgfsetfillcolor{textcolor}%
\pgftext[x=1.872807in,y=2.400024in,left,base]{\color{textcolor}{\rmfamily\fontsize{12.000000}{14.400000}\selectfont\catcode`\^=\active\def^{\ifmmode\sp\else\^{}\fi}\catcode`\%=\active\def%{\%}$\mathcal{O}(\varepsilon^{-2})$}}%
\end{pgfscope}%
\begin{pgfscope}%
\pgfsetbuttcap%
\pgfsetmiterjoin%
\definecolor{currentfill}{rgb}{1.000000,1.000000,1.000000}%
\pgfsetfillcolor{currentfill}%
\pgfsetfillopacity{0.800000}%
\pgfsetlinewidth{1.003750pt}%
\definecolor{currentstroke}{rgb}{0.800000,0.800000,0.800000}%
\pgfsetstrokecolor{currentstroke}%
\pgfsetstrokeopacity{0.800000}%
\pgfsetdash{}{0pt}%
\pgfpathmoveto{\pgfqpoint{0.805247in}{0.632406in}}%
\pgfpathlineto{\pgfqpoint{2.037651in}{0.632406in}}%
\pgfpathlineto{\pgfqpoint{2.037651in}{1.379627in}}%
\pgfpathlineto{\pgfqpoint{0.805247in}{1.379627in}}%
\pgfpathlineto{\pgfqpoint{0.805247in}{0.632406in}}%
\pgfpathclose%
\pgfusepath{stroke,fill}%
\end{pgfscope}%
\begin{pgfscope}%
\pgfsetrectcap%
\pgfsetroundjoin%
\pgfsetlinewidth{1.505625pt}%
\definecolor{currentstroke}{rgb}{0.392157,0.560784,1.000000}%
\pgfsetstrokecolor{currentstroke}%
\pgfsetdash{}{0pt}%
\pgfpathmoveto{\pgfqpoint{0.871913in}{1.254627in}}%
\pgfpathlineto{\pgfqpoint{1.038580in}{1.254627in}}%
\pgfpathlineto{\pgfqpoint{1.205247in}{1.254627in}}%
\pgfusepath{stroke}%
\end{pgfscope}%
\begin{pgfscope}%
\pgfsetbuttcap%
\pgfsetroundjoin%
\definecolor{currentfill}{rgb}{0.392157,0.560784,1.000000}%
\pgfsetfillcolor{currentfill}%
\pgfsetlinewidth{1.003750pt}%
\definecolor{currentstroke}{rgb}{0.392157,0.560784,1.000000}%
\pgfsetstrokecolor{currentstroke}%
\pgfsetdash{}{0pt}%
\pgfsys@defobject{currentmarker}{\pgfqpoint{-0.031250in}{-0.031250in}}{\pgfqpoint{0.031250in}{0.031250in}}{%
\pgfpathmoveto{\pgfqpoint{0.000000in}{-0.031250in}}%
\pgfpathcurveto{\pgfqpoint{0.008288in}{-0.031250in}}{\pgfqpoint{0.016237in}{-0.027957in}}{\pgfqpoint{0.022097in}{-0.022097in}}%
\pgfpathcurveto{\pgfqpoint{0.027957in}{-0.016237in}}{\pgfqpoint{0.031250in}{-0.008288in}}{\pgfqpoint{0.031250in}{0.000000in}}%
\pgfpathcurveto{\pgfqpoint{0.031250in}{0.008288in}}{\pgfqpoint{0.027957in}{0.016237in}}{\pgfqpoint{0.022097in}{0.022097in}}%
\pgfpathcurveto{\pgfqpoint{0.016237in}{0.027957in}}{\pgfqpoint{0.008288in}{0.031250in}}{\pgfqpoint{0.000000in}{0.031250in}}%
\pgfpathcurveto{\pgfqpoint{-0.008288in}{0.031250in}}{\pgfqpoint{-0.016237in}{0.027957in}}{\pgfqpoint{-0.022097in}{0.022097in}}%
\pgfpathcurveto{\pgfqpoint{-0.027957in}{0.016237in}}{\pgfqpoint{-0.031250in}{0.008288in}}{\pgfqpoint{-0.031250in}{0.000000in}}%
\pgfpathcurveto{\pgfqpoint{-0.031250in}{-0.008288in}}{\pgfqpoint{-0.027957in}{-0.016237in}}{\pgfqpoint{-0.022097in}{-0.022097in}}%
\pgfpathcurveto{\pgfqpoint{-0.016237in}{-0.027957in}}{\pgfqpoint{-0.008288in}{-0.031250in}}{\pgfqpoint{0.000000in}{-0.031250in}}%
\pgfpathlineto{\pgfqpoint{0.000000in}{-0.031250in}}%
\pgfpathclose%
\pgfusepath{stroke,fill}%
}%
\begin{pgfscope}%
\pgfsys@transformshift{1.038580in}{1.254627in}%
\pgfsys@useobject{currentmarker}{}%
\end{pgfscope}%
\end{pgfscope}%
\begin{pgfscope}%
\definecolor{textcolor}{rgb}{0.000000,0.000000,0.000000}%
\pgfsetstrokecolor{textcolor}%
\pgfsetfillcolor{textcolor}%
\pgftext[x=1.338580in,y=1.196294in,left,base]{\color{textcolor}{\rmfamily\fontsize{12.000000}{14.400000}\selectfont\catcode`\^=\active\def^{\ifmmode\sp\else\^{}\fi}\catcode`\%=\active\def%{\%}$n_{\mathbf{\Omega}} = 0$}}%
\end{pgfscope}%
\begin{pgfscope}%
\pgfsetrectcap%
\pgfsetroundjoin%
\pgfsetlinewidth{1.505625pt}%
\definecolor{currentstroke}{rgb}{0.862745,0.149020,0.498039}%
\pgfsetstrokecolor{currentstroke}%
\pgfsetdash{}{0pt}%
\pgfpathmoveto{\pgfqpoint{0.871913in}{1.022220in}}%
\pgfpathlineto{\pgfqpoint{1.038580in}{1.022220in}}%
\pgfpathlineto{\pgfqpoint{1.205247in}{1.022220in}}%
\pgfusepath{stroke}%
\end{pgfscope}%
\begin{pgfscope}%
\pgfsetbuttcap%
\pgfsetmiterjoin%
\definecolor{currentfill}{rgb}{0.862745,0.149020,0.498039}%
\pgfsetfillcolor{currentfill}%
\pgfsetlinewidth{1.003750pt}%
\definecolor{currentstroke}{rgb}{0.862745,0.149020,0.498039}%
\pgfsetstrokecolor{currentstroke}%
\pgfsetdash{}{0pt}%
\pgfsys@defobject{currentmarker}{\pgfqpoint{-0.031250in}{-0.031250in}}{\pgfqpoint{0.031250in}{0.031250in}}{%
\pgfpathmoveto{\pgfqpoint{-0.031250in}{-0.031250in}}%
\pgfpathlineto{\pgfqpoint{0.031250in}{-0.031250in}}%
\pgfpathlineto{\pgfqpoint{0.031250in}{0.031250in}}%
\pgfpathlineto{\pgfqpoint{-0.031250in}{0.031250in}}%
\pgfpathlineto{\pgfqpoint{-0.031250in}{-0.031250in}}%
\pgfpathclose%
\pgfusepath{stroke,fill}%
}%
\begin{pgfscope}%
\pgfsys@transformshift{1.038580in}{1.022220in}%
\pgfsys@useobject{currentmarker}{}%
\end{pgfscope}%
\end{pgfscope}%
\begin{pgfscope}%
\definecolor{textcolor}{rgb}{0.000000,0.000000,0.000000}%
\pgfsetstrokecolor{textcolor}%
\pgfsetfillcolor{textcolor}%
\pgftext[x=1.338580in,y=0.963887in,left,base]{\color{textcolor}{\rmfamily\fontsize{12.000000}{14.400000}\selectfont\catcode`\^=\active\def^{\ifmmode\sp\else\^{}\fi}\catcode`\%=\active\def%{\%}$n_{\mathbf{\Omega}} = n_{\mathbf{\Psi}}$}}%
\end{pgfscope}%
\begin{pgfscope}%
\pgfsetrectcap%
\pgfsetroundjoin%
\pgfsetlinewidth{1.505625pt}%
\definecolor{currentstroke}{rgb}{1.000000,0.690196,0.000000}%
\pgfsetstrokecolor{currentstroke}%
\pgfsetdash{}{0pt}%
\pgfpathmoveto{\pgfqpoint{0.871913in}{0.789813in}}%
\pgfpathlineto{\pgfqpoint{1.038580in}{0.789813in}}%
\pgfpathlineto{\pgfqpoint{1.205247in}{0.789813in}}%
\pgfusepath{stroke}%
\end{pgfscope}%
\begin{pgfscope}%
\pgfsetbuttcap%
\pgfsetmiterjoin%
\definecolor{currentfill}{rgb}{1.000000,0.690196,0.000000}%
\pgfsetfillcolor{currentfill}%
\pgfsetlinewidth{1.003750pt}%
\definecolor{currentstroke}{rgb}{1.000000,0.690196,0.000000}%
\pgfsetstrokecolor{currentstroke}%
\pgfsetdash{}{0pt}%
\pgfsys@defobject{currentmarker}{\pgfqpoint{-0.026517in}{-0.044194in}}{\pgfqpoint{0.026517in}{0.044194in}}{%
\pgfpathmoveto{\pgfqpoint{-0.000000in}{-0.044194in}}%
\pgfpathlineto{\pgfqpoint{0.026517in}{0.000000in}}%
\pgfpathlineto{\pgfqpoint{0.000000in}{0.044194in}}%
\pgfpathlineto{\pgfqpoint{-0.026517in}{0.000000in}}%
\pgfpathlineto{\pgfqpoint{-0.000000in}{-0.044194in}}%
\pgfpathclose%
\pgfusepath{stroke,fill}%
}%
\begin{pgfscope}%
\pgfsys@transformshift{1.038580in}{0.789813in}%
\pgfsys@useobject{currentmarker}{}%
\end{pgfscope}%
\end{pgfscope}%
\begin{pgfscope}%
\definecolor{textcolor}{rgb}{0.000000,0.000000,0.000000}%
\pgfsetstrokecolor{textcolor}%
\pgfsetfillcolor{textcolor}%
\pgftext[x=1.338580in,y=0.731480in,left,base]{\color{textcolor}{\rmfamily\fontsize{12.000000}{14.400000}\selectfont\catcode`\^=\active\def^{\ifmmode\sp\else\^{}\fi}\catcode`\%=\active\def%{\%}$n_{\mathbf{\Psi}} = 0$}}%
\end{pgfscope}%
\end{pgfpicture}%
\makeatother%
\endgroup%
}
        \caption{$n_c = 1.$}
        \label{fig:convergence-1}
    \end{subfigure}
    \begin{subfigure}[b]{0.495\textwidth}
        \scalebox{0.8}{%% Creator: Matplotlib, PGF backend
%%
%% To include the figure in your LaTeX document, write
%%   \input{<filename>.pgf}
%%
%% Make sure the required packages are loaded in your preamble
%%   \usepackage{pgf}
%%
%% Also ensure that all the required font packages are loaded; for instance,
%% the lmodern package is sometimes necessary when using math font.
%%   \usepackage{lmodern}
%%
%% Figures using additional raster images can only be included by \input if
%% they are in the same directory as the main LaTeX file. For loading figures
%% from other directories you can use the `import` package
%%   \usepackage{import}
%%
%% and then include the figures with
%%   \import{<path to file>}{<filename>.pgf}
%%
%% Matplotlib used the following preamble
%%   \def\mathdefault#1{#1}
%%   \everymath=\expandafter{\the\everymath\displaystyle}
%%   
%%   \ifdefined\pdftexversion\else  % non-pdftex case.
%%     \usepackage{fontspec}
%%     \setmainfont{DejaVuSerif.ttf}[Path=\detokenize{/home/matti/Documents/projects/Rand-TRACE/.venv/lib/python3.12/site-packages/matplotlib/mpl-data/fonts/ttf/}]
%%     \setsansfont{DejaVuSans.ttf}[Path=\detokenize{/home/matti/Documents/projects/Rand-TRACE/.venv/lib/python3.12/site-packages/matplotlib/mpl-data/fonts/ttf/}]
%%     \setmonofont{DejaVuSansMono.ttf}[Path=\detokenize{/home/matti/Documents/projects/Rand-TRACE/.venv/lib/python3.12/site-packages/matplotlib/mpl-data/fonts/ttf/}]
%%   \fi
%%   \makeatletter\@ifpackageloaded{underscore}{}{\usepackage[strings]{underscore}}\makeatother
%%
\begingroup%
\makeatletter%
\begin{pgfpicture}%
\pgfpathrectangle{\pgfpointorigin}{\pgfqpoint{3.146913in}{2.959073in}}%
\pgfusepath{use as bounding box, clip}%
\begin{pgfscope}%
\pgfsetbuttcap%
\pgfsetmiterjoin%
\definecolor{currentfill}{rgb}{1.000000,1.000000,1.000000}%
\pgfsetfillcolor{currentfill}%
\pgfsetlinewidth{0.000000pt}%
\definecolor{currentstroke}{rgb}{1.000000,1.000000,1.000000}%
\pgfsetstrokecolor{currentstroke}%
\pgfsetdash{}{0pt}%
\pgfpathmoveto{\pgfqpoint{0.000000in}{-0.000000in}}%
\pgfpathlineto{\pgfqpoint{3.146913in}{-0.000000in}}%
\pgfpathlineto{\pgfqpoint{3.146913in}{2.959073in}}%
\pgfpathlineto{\pgfqpoint{0.000000in}{2.959073in}}%
\pgfpathlineto{\pgfqpoint{0.000000in}{-0.000000in}}%
\pgfpathclose%
\pgfusepath{fill}%
\end{pgfscope}%
\begin{pgfscope}%
\pgfsetbuttcap%
\pgfsetmiterjoin%
\definecolor{currentfill}{rgb}{1.000000,1.000000,1.000000}%
\pgfsetfillcolor{currentfill}%
\pgfsetlinewidth{0.000000pt}%
\definecolor{currentstroke}{rgb}{0.000000,0.000000,0.000000}%
\pgfsetstrokecolor{currentstroke}%
\pgfsetstrokeopacity{0.000000}%
\pgfsetdash{}{0pt}%
\pgfpathmoveto{\pgfqpoint{0.721913in}{0.549073in}}%
\pgfpathlineto{\pgfqpoint{3.046913in}{0.549073in}}%
\pgfpathlineto{\pgfqpoint{3.046913in}{2.859073in}}%
\pgfpathlineto{\pgfqpoint{0.721913in}{2.859073in}}%
\pgfpathlineto{\pgfqpoint{0.721913in}{0.549073in}}%
\pgfpathclose%
\pgfusepath{fill}%
\end{pgfscope}%
\begin{pgfscope}%
\pgfpathrectangle{\pgfqpoint{0.721913in}{0.549073in}}{\pgfqpoint{2.325000in}{2.310000in}}%
\pgfusepath{clip}%
\pgfsetrectcap%
\pgfsetroundjoin%
\pgfsetlinewidth{0.250937pt}%
\definecolor{currentstroke}{rgb}{0.000000,0.000000,0.000000}%
\pgfsetstrokecolor{currentstroke}%
\pgfsetstrokeopacity{0.200000}%
\pgfsetdash{}{0pt}%
\pgfpathmoveto{\pgfqpoint{0.964644in}{0.549073in}}%
\pgfpathlineto{\pgfqpoint{0.964644in}{2.859073in}}%
\pgfusepath{stroke}%
\end{pgfscope}%
\begin{pgfscope}%
\pgfsetbuttcap%
\pgfsetroundjoin%
\definecolor{currentfill}{rgb}{0.000000,0.000000,0.000000}%
\pgfsetfillcolor{currentfill}%
\pgfsetlinewidth{0.803000pt}%
\definecolor{currentstroke}{rgb}{0.000000,0.000000,0.000000}%
\pgfsetstrokecolor{currentstroke}%
\pgfsetdash{}{0pt}%
\pgfsys@defobject{currentmarker}{\pgfqpoint{0.000000in}{-0.048611in}}{\pgfqpoint{0.000000in}{0.000000in}}{%
\pgfpathmoveto{\pgfqpoint{0.000000in}{0.000000in}}%
\pgfpathlineto{\pgfqpoint{0.000000in}{-0.048611in}}%
\pgfusepath{stroke,fill}%
}%
\begin{pgfscope}%
\pgfsys@transformshift{0.964644in}{0.549073in}%
\pgfsys@useobject{currentmarker}{}%
\end{pgfscope}%
\end{pgfscope}%
\begin{pgfscope}%
\definecolor{textcolor}{rgb}{0.000000,0.000000,0.000000}%
\pgfsetstrokecolor{textcolor}%
\pgfsetfillcolor{textcolor}%
\pgftext[x=0.964644in,y=0.451851in,,top]{\color{textcolor}{\rmfamily\fontsize{12.000000}{14.400000}\selectfont\catcode`\^=\active\def^{\ifmmode\sp\else\^{}\fi}\catcode`\%=\active\def%{\%}$\mathdefault{10^{1}}$}}%
\end{pgfscope}%
\begin{pgfscope}%
\pgfpathrectangle{\pgfqpoint{0.721913in}{0.549073in}}{\pgfqpoint{2.325000in}{2.310000in}}%
\pgfusepath{clip}%
\pgfsetrectcap%
\pgfsetroundjoin%
\pgfsetlinewidth{0.250937pt}%
\definecolor{currentstroke}{rgb}{0.000000,0.000000,0.000000}%
\pgfsetstrokecolor{currentstroke}%
\pgfsetstrokeopacity{0.200000}%
\pgfsetdash{}{0pt}%
\pgfpathmoveto{\pgfqpoint{1.809245in}{0.549073in}}%
\pgfpathlineto{\pgfqpoint{1.809245in}{2.859073in}}%
\pgfusepath{stroke}%
\end{pgfscope}%
\begin{pgfscope}%
\pgfsetbuttcap%
\pgfsetroundjoin%
\definecolor{currentfill}{rgb}{0.000000,0.000000,0.000000}%
\pgfsetfillcolor{currentfill}%
\pgfsetlinewidth{0.803000pt}%
\definecolor{currentstroke}{rgb}{0.000000,0.000000,0.000000}%
\pgfsetstrokecolor{currentstroke}%
\pgfsetdash{}{0pt}%
\pgfsys@defobject{currentmarker}{\pgfqpoint{0.000000in}{-0.048611in}}{\pgfqpoint{0.000000in}{0.000000in}}{%
\pgfpathmoveto{\pgfqpoint{0.000000in}{0.000000in}}%
\pgfpathlineto{\pgfqpoint{0.000000in}{-0.048611in}}%
\pgfusepath{stroke,fill}%
}%
\begin{pgfscope}%
\pgfsys@transformshift{1.809245in}{0.549073in}%
\pgfsys@useobject{currentmarker}{}%
\end{pgfscope}%
\end{pgfscope}%
\begin{pgfscope}%
\definecolor{textcolor}{rgb}{0.000000,0.000000,0.000000}%
\pgfsetstrokecolor{textcolor}%
\pgfsetfillcolor{textcolor}%
\pgftext[x=1.809245in,y=0.451851in,,top]{\color{textcolor}{\rmfamily\fontsize{12.000000}{14.400000}\selectfont\catcode`\^=\active\def^{\ifmmode\sp\else\^{}\fi}\catcode`\%=\active\def%{\%}$\mathdefault{10^{2}}$}}%
\end{pgfscope}%
\begin{pgfscope}%
\pgfpathrectangle{\pgfqpoint{0.721913in}{0.549073in}}{\pgfqpoint{2.325000in}{2.310000in}}%
\pgfusepath{clip}%
\pgfsetrectcap%
\pgfsetroundjoin%
\pgfsetlinewidth{0.250937pt}%
\definecolor{currentstroke}{rgb}{0.000000,0.000000,0.000000}%
\pgfsetstrokecolor{currentstroke}%
\pgfsetstrokeopacity{0.200000}%
\pgfsetdash{}{0pt}%
\pgfpathmoveto{\pgfqpoint{2.653846in}{0.549073in}}%
\pgfpathlineto{\pgfqpoint{2.653846in}{2.859073in}}%
\pgfusepath{stroke}%
\end{pgfscope}%
\begin{pgfscope}%
\pgfsetbuttcap%
\pgfsetroundjoin%
\definecolor{currentfill}{rgb}{0.000000,0.000000,0.000000}%
\pgfsetfillcolor{currentfill}%
\pgfsetlinewidth{0.803000pt}%
\definecolor{currentstroke}{rgb}{0.000000,0.000000,0.000000}%
\pgfsetstrokecolor{currentstroke}%
\pgfsetdash{}{0pt}%
\pgfsys@defobject{currentmarker}{\pgfqpoint{0.000000in}{-0.048611in}}{\pgfqpoint{0.000000in}{0.000000in}}{%
\pgfpathmoveto{\pgfqpoint{0.000000in}{0.000000in}}%
\pgfpathlineto{\pgfqpoint{0.000000in}{-0.048611in}}%
\pgfusepath{stroke,fill}%
}%
\begin{pgfscope}%
\pgfsys@transformshift{2.653846in}{0.549073in}%
\pgfsys@useobject{currentmarker}{}%
\end{pgfscope}%
\end{pgfscope}%
\begin{pgfscope}%
\definecolor{textcolor}{rgb}{0.000000,0.000000,0.000000}%
\pgfsetstrokecolor{textcolor}%
\pgfsetfillcolor{textcolor}%
\pgftext[x=2.653846in,y=0.451851in,,top]{\color{textcolor}{\rmfamily\fontsize{12.000000}{14.400000}\selectfont\catcode`\^=\active\def^{\ifmmode\sp\else\^{}\fi}\catcode`\%=\active\def%{\%}$\mathdefault{10^{3}}$}}%
\end{pgfscope}%
\begin{pgfscope}%
\pgfsetbuttcap%
\pgfsetroundjoin%
\definecolor{currentfill}{rgb}{0.000000,0.000000,0.000000}%
\pgfsetfillcolor{currentfill}%
\pgfsetlinewidth{0.602250pt}%
\definecolor{currentstroke}{rgb}{0.000000,0.000000,0.000000}%
\pgfsetstrokecolor{currentstroke}%
\pgfsetdash{}{0pt}%
\pgfsys@defobject{currentmarker}{\pgfqpoint{0.000000in}{-0.027778in}}{\pgfqpoint{0.000000in}{0.000000in}}{%
\pgfpathmoveto{\pgfqpoint{0.000000in}{0.000000in}}%
\pgfpathlineto{\pgfqpoint{0.000000in}{-0.027778in}}%
\pgfusepath{stroke,fill}%
}%
\begin{pgfscope}%
\pgfsys@transformshift{0.777271in}{0.549073in}%
\pgfsys@useobject{currentmarker}{}%
\end{pgfscope}%
\end{pgfscope}%
\begin{pgfscope}%
\pgfsetbuttcap%
\pgfsetroundjoin%
\definecolor{currentfill}{rgb}{0.000000,0.000000,0.000000}%
\pgfsetfillcolor{currentfill}%
\pgfsetlinewidth{0.602250pt}%
\definecolor{currentstroke}{rgb}{0.000000,0.000000,0.000000}%
\pgfsetstrokecolor{currentstroke}%
\pgfsetdash{}{0pt}%
\pgfsys@defobject{currentmarker}{\pgfqpoint{0.000000in}{-0.027778in}}{\pgfqpoint{0.000000in}{0.000000in}}{%
\pgfpathmoveto{\pgfqpoint{0.000000in}{0.000000in}}%
\pgfpathlineto{\pgfqpoint{0.000000in}{-0.027778in}}%
\pgfusepath{stroke,fill}%
}%
\begin{pgfscope}%
\pgfsys@transformshift{0.833814in}{0.549073in}%
\pgfsys@useobject{currentmarker}{}%
\end{pgfscope}%
\end{pgfscope}%
\begin{pgfscope}%
\pgfsetbuttcap%
\pgfsetroundjoin%
\definecolor{currentfill}{rgb}{0.000000,0.000000,0.000000}%
\pgfsetfillcolor{currentfill}%
\pgfsetlinewidth{0.602250pt}%
\definecolor{currentstroke}{rgb}{0.000000,0.000000,0.000000}%
\pgfsetstrokecolor{currentstroke}%
\pgfsetdash{}{0pt}%
\pgfsys@defobject{currentmarker}{\pgfqpoint{0.000000in}{-0.027778in}}{\pgfqpoint{0.000000in}{0.000000in}}{%
\pgfpathmoveto{\pgfqpoint{0.000000in}{0.000000in}}%
\pgfpathlineto{\pgfqpoint{0.000000in}{-0.027778in}}%
\pgfusepath{stroke,fill}%
}%
\begin{pgfscope}%
\pgfsys@transformshift{0.882794in}{0.549073in}%
\pgfsys@useobject{currentmarker}{}%
\end{pgfscope}%
\end{pgfscope}%
\begin{pgfscope}%
\pgfsetbuttcap%
\pgfsetroundjoin%
\definecolor{currentfill}{rgb}{0.000000,0.000000,0.000000}%
\pgfsetfillcolor{currentfill}%
\pgfsetlinewidth{0.602250pt}%
\definecolor{currentstroke}{rgb}{0.000000,0.000000,0.000000}%
\pgfsetstrokecolor{currentstroke}%
\pgfsetdash{}{0pt}%
\pgfsys@defobject{currentmarker}{\pgfqpoint{0.000000in}{-0.027778in}}{\pgfqpoint{0.000000in}{0.000000in}}{%
\pgfpathmoveto{\pgfqpoint{0.000000in}{0.000000in}}%
\pgfpathlineto{\pgfqpoint{0.000000in}{-0.027778in}}%
\pgfusepath{stroke,fill}%
}%
\begin{pgfscope}%
\pgfsys@transformshift{0.925997in}{0.549073in}%
\pgfsys@useobject{currentmarker}{}%
\end{pgfscope}%
\end{pgfscope}%
\begin{pgfscope}%
\pgfsetbuttcap%
\pgfsetroundjoin%
\definecolor{currentfill}{rgb}{0.000000,0.000000,0.000000}%
\pgfsetfillcolor{currentfill}%
\pgfsetlinewidth{0.602250pt}%
\definecolor{currentstroke}{rgb}{0.000000,0.000000,0.000000}%
\pgfsetstrokecolor{currentstroke}%
\pgfsetdash{}{0pt}%
\pgfsys@defobject{currentmarker}{\pgfqpoint{0.000000in}{-0.027778in}}{\pgfqpoint{0.000000in}{0.000000in}}{%
\pgfpathmoveto{\pgfqpoint{0.000000in}{0.000000in}}%
\pgfpathlineto{\pgfqpoint{0.000000in}{-0.027778in}}%
\pgfusepath{stroke,fill}%
}%
\begin{pgfscope}%
\pgfsys@transformshift{1.218894in}{0.549073in}%
\pgfsys@useobject{currentmarker}{}%
\end{pgfscope}%
\end{pgfscope}%
\begin{pgfscope}%
\pgfsetbuttcap%
\pgfsetroundjoin%
\definecolor{currentfill}{rgb}{0.000000,0.000000,0.000000}%
\pgfsetfillcolor{currentfill}%
\pgfsetlinewidth{0.602250pt}%
\definecolor{currentstroke}{rgb}{0.000000,0.000000,0.000000}%
\pgfsetstrokecolor{currentstroke}%
\pgfsetdash{}{0pt}%
\pgfsys@defobject{currentmarker}{\pgfqpoint{0.000000in}{-0.027778in}}{\pgfqpoint{0.000000in}{0.000000in}}{%
\pgfpathmoveto{\pgfqpoint{0.000000in}{0.000000in}}%
\pgfpathlineto{\pgfqpoint{0.000000in}{-0.027778in}}%
\pgfusepath{stroke,fill}%
}%
\begin{pgfscope}%
\pgfsys@transformshift{1.367621in}{0.549073in}%
\pgfsys@useobject{currentmarker}{}%
\end{pgfscope}%
\end{pgfscope}%
\begin{pgfscope}%
\pgfsetbuttcap%
\pgfsetroundjoin%
\definecolor{currentfill}{rgb}{0.000000,0.000000,0.000000}%
\pgfsetfillcolor{currentfill}%
\pgfsetlinewidth{0.602250pt}%
\definecolor{currentstroke}{rgb}{0.000000,0.000000,0.000000}%
\pgfsetstrokecolor{currentstroke}%
\pgfsetdash{}{0pt}%
\pgfsys@defobject{currentmarker}{\pgfqpoint{0.000000in}{-0.027778in}}{\pgfqpoint{0.000000in}{0.000000in}}{%
\pgfpathmoveto{\pgfqpoint{0.000000in}{0.000000in}}%
\pgfpathlineto{\pgfqpoint{0.000000in}{-0.027778in}}%
\pgfusepath{stroke,fill}%
}%
\begin{pgfscope}%
\pgfsys@transformshift{1.473144in}{0.549073in}%
\pgfsys@useobject{currentmarker}{}%
\end{pgfscope}%
\end{pgfscope}%
\begin{pgfscope}%
\pgfsetbuttcap%
\pgfsetroundjoin%
\definecolor{currentfill}{rgb}{0.000000,0.000000,0.000000}%
\pgfsetfillcolor{currentfill}%
\pgfsetlinewidth{0.602250pt}%
\definecolor{currentstroke}{rgb}{0.000000,0.000000,0.000000}%
\pgfsetstrokecolor{currentstroke}%
\pgfsetdash{}{0pt}%
\pgfsys@defobject{currentmarker}{\pgfqpoint{0.000000in}{-0.027778in}}{\pgfqpoint{0.000000in}{0.000000in}}{%
\pgfpathmoveto{\pgfqpoint{0.000000in}{0.000000in}}%
\pgfpathlineto{\pgfqpoint{0.000000in}{-0.027778in}}%
\pgfusepath{stroke,fill}%
}%
\begin{pgfscope}%
\pgfsys@transformshift{1.554995in}{0.549073in}%
\pgfsys@useobject{currentmarker}{}%
\end{pgfscope}%
\end{pgfscope}%
\begin{pgfscope}%
\pgfsetbuttcap%
\pgfsetroundjoin%
\definecolor{currentfill}{rgb}{0.000000,0.000000,0.000000}%
\pgfsetfillcolor{currentfill}%
\pgfsetlinewidth{0.602250pt}%
\definecolor{currentstroke}{rgb}{0.000000,0.000000,0.000000}%
\pgfsetstrokecolor{currentstroke}%
\pgfsetdash{}{0pt}%
\pgfsys@defobject{currentmarker}{\pgfqpoint{0.000000in}{-0.027778in}}{\pgfqpoint{0.000000in}{0.000000in}}{%
\pgfpathmoveto{\pgfqpoint{0.000000in}{0.000000in}}%
\pgfpathlineto{\pgfqpoint{0.000000in}{-0.027778in}}%
\pgfusepath{stroke,fill}%
}%
\begin{pgfscope}%
\pgfsys@transformshift{1.621871in}{0.549073in}%
\pgfsys@useobject{currentmarker}{}%
\end{pgfscope}%
\end{pgfscope}%
\begin{pgfscope}%
\pgfsetbuttcap%
\pgfsetroundjoin%
\definecolor{currentfill}{rgb}{0.000000,0.000000,0.000000}%
\pgfsetfillcolor{currentfill}%
\pgfsetlinewidth{0.602250pt}%
\definecolor{currentstroke}{rgb}{0.000000,0.000000,0.000000}%
\pgfsetstrokecolor{currentstroke}%
\pgfsetdash{}{0pt}%
\pgfsys@defobject{currentmarker}{\pgfqpoint{0.000000in}{-0.027778in}}{\pgfqpoint{0.000000in}{0.000000in}}{%
\pgfpathmoveto{\pgfqpoint{0.000000in}{0.000000in}}%
\pgfpathlineto{\pgfqpoint{0.000000in}{-0.027778in}}%
\pgfusepath{stroke,fill}%
}%
\begin{pgfscope}%
\pgfsys@transformshift{1.678415in}{0.549073in}%
\pgfsys@useobject{currentmarker}{}%
\end{pgfscope}%
\end{pgfscope}%
\begin{pgfscope}%
\pgfsetbuttcap%
\pgfsetroundjoin%
\definecolor{currentfill}{rgb}{0.000000,0.000000,0.000000}%
\pgfsetfillcolor{currentfill}%
\pgfsetlinewidth{0.602250pt}%
\definecolor{currentstroke}{rgb}{0.000000,0.000000,0.000000}%
\pgfsetstrokecolor{currentstroke}%
\pgfsetdash{}{0pt}%
\pgfsys@defobject{currentmarker}{\pgfqpoint{0.000000in}{-0.027778in}}{\pgfqpoint{0.000000in}{0.000000in}}{%
\pgfpathmoveto{\pgfqpoint{0.000000in}{0.000000in}}%
\pgfpathlineto{\pgfqpoint{0.000000in}{-0.027778in}}%
\pgfusepath{stroke,fill}%
}%
\begin{pgfscope}%
\pgfsys@transformshift{1.727395in}{0.549073in}%
\pgfsys@useobject{currentmarker}{}%
\end{pgfscope}%
\end{pgfscope}%
\begin{pgfscope}%
\pgfsetbuttcap%
\pgfsetroundjoin%
\definecolor{currentfill}{rgb}{0.000000,0.000000,0.000000}%
\pgfsetfillcolor{currentfill}%
\pgfsetlinewidth{0.602250pt}%
\definecolor{currentstroke}{rgb}{0.000000,0.000000,0.000000}%
\pgfsetstrokecolor{currentstroke}%
\pgfsetdash{}{0pt}%
\pgfsys@defobject{currentmarker}{\pgfqpoint{0.000000in}{-0.027778in}}{\pgfqpoint{0.000000in}{0.000000in}}{%
\pgfpathmoveto{\pgfqpoint{0.000000in}{0.000000in}}%
\pgfpathlineto{\pgfqpoint{0.000000in}{-0.027778in}}%
\pgfusepath{stroke,fill}%
}%
\begin{pgfscope}%
\pgfsys@transformshift{1.770598in}{0.549073in}%
\pgfsys@useobject{currentmarker}{}%
\end{pgfscope}%
\end{pgfscope}%
\begin{pgfscope}%
\pgfsetbuttcap%
\pgfsetroundjoin%
\definecolor{currentfill}{rgb}{0.000000,0.000000,0.000000}%
\pgfsetfillcolor{currentfill}%
\pgfsetlinewidth{0.602250pt}%
\definecolor{currentstroke}{rgb}{0.000000,0.000000,0.000000}%
\pgfsetstrokecolor{currentstroke}%
\pgfsetdash{}{0pt}%
\pgfsys@defobject{currentmarker}{\pgfqpoint{0.000000in}{-0.027778in}}{\pgfqpoint{0.000000in}{0.000000in}}{%
\pgfpathmoveto{\pgfqpoint{0.000000in}{0.000000in}}%
\pgfpathlineto{\pgfqpoint{0.000000in}{-0.027778in}}%
\pgfusepath{stroke,fill}%
}%
\begin{pgfscope}%
\pgfsys@transformshift{2.063495in}{0.549073in}%
\pgfsys@useobject{currentmarker}{}%
\end{pgfscope}%
\end{pgfscope}%
\begin{pgfscope}%
\pgfsetbuttcap%
\pgfsetroundjoin%
\definecolor{currentfill}{rgb}{0.000000,0.000000,0.000000}%
\pgfsetfillcolor{currentfill}%
\pgfsetlinewidth{0.602250pt}%
\definecolor{currentstroke}{rgb}{0.000000,0.000000,0.000000}%
\pgfsetstrokecolor{currentstroke}%
\pgfsetdash{}{0pt}%
\pgfsys@defobject{currentmarker}{\pgfqpoint{0.000000in}{-0.027778in}}{\pgfqpoint{0.000000in}{0.000000in}}{%
\pgfpathmoveto{\pgfqpoint{0.000000in}{0.000000in}}%
\pgfpathlineto{\pgfqpoint{0.000000in}{-0.027778in}}%
\pgfusepath{stroke,fill}%
}%
\begin{pgfscope}%
\pgfsys@transformshift{2.212222in}{0.549073in}%
\pgfsys@useobject{currentmarker}{}%
\end{pgfscope}%
\end{pgfscope}%
\begin{pgfscope}%
\pgfsetbuttcap%
\pgfsetroundjoin%
\definecolor{currentfill}{rgb}{0.000000,0.000000,0.000000}%
\pgfsetfillcolor{currentfill}%
\pgfsetlinewidth{0.602250pt}%
\definecolor{currentstroke}{rgb}{0.000000,0.000000,0.000000}%
\pgfsetstrokecolor{currentstroke}%
\pgfsetdash{}{0pt}%
\pgfsys@defobject{currentmarker}{\pgfqpoint{0.000000in}{-0.027778in}}{\pgfqpoint{0.000000in}{0.000000in}}{%
\pgfpathmoveto{\pgfqpoint{0.000000in}{0.000000in}}%
\pgfpathlineto{\pgfqpoint{0.000000in}{-0.027778in}}%
\pgfusepath{stroke,fill}%
}%
\begin{pgfscope}%
\pgfsys@transformshift{2.317745in}{0.549073in}%
\pgfsys@useobject{currentmarker}{}%
\end{pgfscope}%
\end{pgfscope}%
\begin{pgfscope}%
\pgfsetbuttcap%
\pgfsetroundjoin%
\definecolor{currentfill}{rgb}{0.000000,0.000000,0.000000}%
\pgfsetfillcolor{currentfill}%
\pgfsetlinewidth{0.602250pt}%
\definecolor{currentstroke}{rgb}{0.000000,0.000000,0.000000}%
\pgfsetstrokecolor{currentstroke}%
\pgfsetdash{}{0pt}%
\pgfsys@defobject{currentmarker}{\pgfqpoint{0.000000in}{-0.027778in}}{\pgfqpoint{0.000000in}{0.000000in}}{%
\pgfpathmoveto{\pgfqpoint{0.000000in}{0.000000in}}%
\pgfpathlineto{\pgfqpoint{0.000000in}{-0.027778in}}%
\pgfusepath{stroke,fill}%
}%
\begin{pgfscope}%
\pgfsys@transformshift{2.399595in}{0.549073in}%
\pgfsys@useobject{currentmarker}{}%
\end{pgfscope}%
\end{pgfscope}%
\begin{pgfscope}%
\pgfsetbuttcap%
\pgfsetroundjoin%
\definecolor{currentfill}{rgb}{0.000000,0.000000,0.000000}%
\pgfsetfillcolor{currentfill}%
\pgfsetlinewidth{0.602250pt}%
\definecolor{currentstroke}{rgb}{0.000000,0.000000,0.000000}%
\pgfsetstrokecolor{currentstroke}%
\pgfsetdash{}{0pt}%
\pgfsys@defobject{currentmarker}{\pgfqpoint{0.000000in}{-0.027778in}}{\pgfqpoint{0.000000in}{0.000000in}}{%
\pgfpathmoveto{\pgfqpoint{0.000000in}{0.000000in}}%
\pgfpathlineto{\pgfqpoint{0.000000in}{-0.027778in}}%
\pgfusepath{stroke,fill}%
}%
\begin{pgfscope}%
\pgfsys@transformshift{2.466472in}{0.549073in}%
\pgfsys@useobject{currentmarker}{}%
\end{pgfscope}%
\end{pgfscope}%
\begin{pgfscope}%
\pgfsetbuttcap%
\pgfsetroundjoin%
\definecolor{currentfill}{rgb}{0.000000,0.000000,0.000000}%
\pgfsetfillcolor{currentfill}%
\pgfsetlinewidth{0.602250pt}%
\definecolor{currentstroke}{rgb}{0.000000,0.000000,0.000000}%
\pgfsetstrokecolor{currentstroke}%
\pgfsetdash{}{0pt}%
\pgfsys@defobject{currentmarker}{\pgfqpoint{0.000000in}{-0.027778in}}{\pgfqpoint{0.000000in}{0.000000in}}{%
\pgfpathmoveto{\pgfqpoint{0.000000in}{0.000000in}}%
\pgfpathlineto{\pgfqpoint{0.000000in}{-0.027778in}}%
\pgfusepath{stroke,fill}%
}%
\begin{pgfscope}%
\pgfsys@transformshift{2.523015in}{0.549073in}%
\pgfsys@useobject{currentmarker}{}%
\end{pgfscope}%
\end{pgfscope}%
\begin{pgfscope}%
\pgfsetbuttcap%
\pgfsetroundjoin%
\definecolor{currentfill}{rgb}{0.000000,0.000000,0.000000}%
\pgfsetfillcolor{currentfill}%
\pgfsetlinewidth{0.602250pt}%
\definecolor{currentstroke}{rgb}{0.000000,0.000000,0.000000}%
\pgfsetstrokecolor{currentstroke}%
\pgfsetdash{}{0pt}%
\pgfsys@defobject{currentmarker}{\pgfqpoint{0.000000in}{-0.027778in}}{\pgfqpoint{0.000000in}{0.000000in}}{%
\pgfpathmoveto{\pgfqpoint{0.000000in}{0.000000in}}%
\pgfpathlineto{\pgfqpoint{0.000000in}{-0.027778in}}%
\pgfusepath{stroke,fill}%
}%
\begin{pgfscope}%
\pgfsys@transformshift{2.571995in}{0.549073in}%
\pgfsys@useobject{currentmarker}{}%
\end{pgfscope}%
\end{pgfscope}%
\begin{pgfscope}%
\pgfsetbuttcap%
\pgfsetroundjoin%
\definecolor{currentfill}{rgb}{0.000000,0.000000,0.000000}%
\pgfsetfillcolor{currentfill}%
\pgfsetlinewidth{0.602250pt}%
\definecolor{currentstroke}{rgb}{0.000000,0.000000,0.000000}%
\pgfsetstrokecolor{currentstroke}%
\pgfsetdash{}{0pt}%
\pgfsys@defobject{currentmarker}{\pgfqpoint{0.000000in}{-0.027778in}}{\pgfqpoint{0.000000in}{0.000000in}}{%
\pgfpathmoveto{\pgfqpoint{0.000000in}{0.000000in}}%
\pgfpathlineto{\pgfqpoint{0.000000in}{-0.027778in}}%
\pgfusepath{stroke,fill}%
}%
\begin{pgfscope}%
\pgfsys@transformshift{2.615199in}{0.549073in}%
\pgfsys@useobject{currentmarker}{}%
\end{pgfscope}%
\end{pgfscope}%
\begin{pgfscope}%
\pgfsetbuttcap%
\pgfsetroundjoin%
\definecolor{currentfill}{rgb}{0.000000,0.000000,0.000000}%
\pgfsetfillcolor{currentfill}%
\pgfsetlinewidth{0.602250pt}%
\definecolor{currentstroke}{rgb}{0.000000,0.000000,0.000000}%
\pgfsetstrokecolor{currentstroke}%
\pgfsetdash{}{0pt}%
\pgfsys@defobject{currentmarker}{\pgfqpoint{0.000000in}{-0.027778in}}{\pgfqpoint{0.000000in}{0.000000in}}{%
\pgfpathmoveto{\pgfqpoint{0.000000in}{0.000000in}}%
\pgfpathlineto{\pgfqpoint{0.000000in}{-0.027778in}}%
\pgfusepath{stroke,fill}%
}%
\begin{pgfscope}%
\pgfsys@transformshift{2.908096in}{0.549073in}%
\pgfsys@useobject{currentmarker}{}%
\end{pgfscope}%
\end{pgfscope}%
\begin{pgfscope}%
\definecolor{textcolor}{rgb}{0.000000,0.000000,0.000000}%
\pgfsetstrokecolor{textcolor}%
\pgfsetfillcolor{textcolor}%
\pgftext[x=1.884413in,y=0.248148in,,top]{\color{textcolor}{\rmfamily\fontsize{12.000000}{14.400000}\selectfont\catcode`\^=\active\def^{\ifmmode\sp\else\^{}\fi}\catcode`\%=\active\def%{\%}estimator size $n_{\mathbf{\Omega}} + n_{\mathbf{\Psi}}$}}%
\end{pgfscope}%
\begin{pgfscope}%
\pgfpathrectangle{\pgfqpoint{0.721913in}{0.549073in}}{\pgfqpoint{2.325000in}{2.310000in}}%
\pgfusepath{clip}%
\pgfsetrectcap%
\pgfsetroundjoin%
\pgfsetlinewidth{0.250937pt}%
\definecolor{currentstroke}{rgb}{0.000000,0.000000,0.000000}%
\pgfsetstrokecolor{currentstroke}%
\pgfsetstrokeopacity{0.200000}%
\pgfsetdash{}{0pt}%
\pgfpathmoveto{\pgfqpoint{0.721913in}{0.856046in}}%
\pgfpathlineto{\pgfqpoint{3.046913in}{0.856046in}}%
\pgfusepath{stroke}%
\end{pgfscope}%
\begin{pgfscope}%
\pgfsetbuttcap%
\pgfsetroundjoin%
\definecolor{currentfill}{rgb}{0.000000,0.000000,0.000000}%
\pgfsetfillcolor{currentfill}%
\pgfsetlinewidth{0.803000pt}%
\definecolor{currentstroke}{rgb}{0.000000,0.000000,0.000000}%
\pgfsetstrokecolor{currentstroke}%
\pgfsetdash{}{0pt}%
\pgfsys@defobject{currentmarker}{\pgfqpoint{-0.048611in}{0.000000in}}{\pgfqpoint{-0.000000in}{0.000000in}}{%
\pgfpathmoveto{\pgfqpoint{-0.000000in}{0.000000in}}%
\pgfpathlineto{\pgfqpoint{-0.048611in}{0.000000in}}%
\pgfusepath{stroke,fill}%
}%
\begin{pgfscope}%
\pgfsys@transformshift{0.721913in}{0.856046in}%
\pgfsys@useobject{currentmarker}{}%
\end{pgfscope}%
\end{pgfscope}%
\begin{pgfscope}%
\definecolor{textcolor}{rgb}{0.000000,0.000000,0.000000}%
\pgfsetstrokecolor{textcolor}%
\pgfsetfillcolor{textcolor}%
\pgftext[x=0.303703in, y=0.798176in, left, base]{\color{textcolor}{\rmfamily\fontsize{12.000000}{14.400000}\selectfont\catcode`\^=\active\def^{\ifmmode\sp\else\^{}\fi}\catcode`\%=\active\def%{\%}$\mathdefault{10^{-6}}$}}%
\end{pgfscope}%
\begin{pgfscope}%
\pgfpathrectangle{\pgfqpoint{0.721913in}{0.549073in}}{\pgfqpoint{2.325000in}{2.310000in}}%
\pgfusepath{clip}%
\pgfsetrectcap%
\pgfsetroundjoin%
\pgfsetlinewidth{0.250937pt}%
\definecolor{currentstroke}{rgb}{0.000000,0.000000,0.000000}%
\pgfsetstrokecolor{currentstroke}%
\pgfsetstrokeopacity{0.200000}%
\pgfsetdash{}{0pt}%
\pgfpathmoveto{\pgfqpoint{0.721913in}{1.460880in}}%
\pgfpathlineto{\pgfqpoint{3.046913in}{1.460880in}}%
\pgfusepath{stroke}%
\end{pgfscope}%
\begin{pgfscope}%
\pgfsetbuttcap%
\pgfsetroundjoin%
\definecolor{currentfill}{rgb}{0.000000,0.000000,0.000000}%
\pgfsetfillcolor{currentfill}%
\pgfsetlinewidth{0.803000pt}%
\definecolor{currentstroke}{rgb}{0.000000,0.000000,0.000000}%
\pgfsetstrokecolor{currentstroke}%
\pgfsetdash{}{0pt}%
\pgfsys@defobject{currentmarker}{\pgfqpoint{-0.048611in}{0.000000in}}{\pgfqpoint{-0.000000in}{0.000000in}}{%
\pgfpathmoveto{\pgfqpoint{-0.000000in}{0.000000in}}%
\pgfpathlineto{\pgfqpoint{-0.048611in}{0.000000in}}%
\pgfusepath{stroke,fill}%
}%
\begin{pgfscope}%
\pgfsys@transformshift{0.721913in}{1.460880in}%
\pgfsys@useobject{currentmarker}{}%
\end{pgfscope}%
\end{pgfscope}%
\begin{pgfscope}%
\definecolor{textcolor}{rgb}{0.000000,0.000000,0.000000}%
\pgfsetstrokecolor{textcolor}%
\pgfsetfillcolor{textcolor}%
\pgftext[x=0.303703in, y=1.403010in, left, base]{\color{textcolor}{\rmfamily\fontsize{12.000000}{14.400000}\selectfont\catcode`\^=\active\def^{\ifmmode\sp\else\^{}\fi}\catcode`\%=\active\def%{\%}$\mathdefault{10^{-4}}$}}%
\end{pgfscope}%
\begin{pgfscope}%
\pgfpathrectangle{\pgfqpoint{0.721913in}{0.549073in}}{\pgfqpoint{2.325000in}{2.310000in}}%
\pgfusepath{clip}%
\pgfsetrectcap%
\pgfsetroundjoin%
\pgfsetlinewidth{0.250937pt}%
\definecolor{currentstroke}{rgb}{0.000000,0.000000,0.000000}%
\pgfsetstrokecolor{currentstroke}%
\pgfsetstrokeopacity{0.200000}%
\pgfsetdash{}{0pt}%
\pgfpathmoveto{\pgfqpoint{0.721913in}{2.065714in}}%
\pgfpathlineto{\pgfqpoint{3.046913in}{2.065714in}}%
\pgfusepath{stroke}%
\end{pgfscope}%
\begin{pgfscope}%
\pgfsetbuttcap%
\pgfsetroundjoin%
\definecolor{currentfill}{rgb}{0.000000,0.000000,0.000000}%
\pgfsetfillcolor{currentfill}%
\pgfsetlinewidth{0.803000pt}%
\definecolor{currentstroke}{rgb}{0.000000,0.000000,0.000000}%
\pgfsetstrokecolor{currentstroke}%
\pgfsetdash{}{0pt}%
\pgfsys@defobject{currentmarker}{\pgfqpoint{-0.048611in}{0.000000in}}{\pgfqpoint{-0.000000in}{0.000000in}}{%
\pgfpathmoveto{\pgfqpoint{-0.000000in}{0.000000in}}%
\pgfpathlineto{\pgfqpoint{-0.048611in}{0.000000in}}%
\pgfusepath{stroke,fill}%
}%
\begin{pgfscope}%
\pgfsys@transformshift{0.721913in}{2.065714in}%
\pgfsys@useobject{currentmarker}{}%
\end{pgfscope}%
\end{pgfscope}%
\begin{pgfscope}%
\definecolor{textcolor}{rgb}{0.000000,0.000000,0.000000}%
\pgfsetstrokecolor{textcolor}%
\pgfsetfillcolor{textcolor}%
\pgftext[x=0.303703in, y=2.007844in, left, base]{\color{textcolor}{\rmfamily\fontsize{12.000000}{14.400000}\selectfont\catcode`\^=\active\def^{\ifmmode\sp\else\^{}\fi}\catcode`\%=\active\def%{\%}$\mathdefault{10^{-2}}$}}%
\end{pgfscope}%
\begin{pgfscope}%
\pgfpathrectangle{\pgfqpoint{0.721913in}{0.549073in}}{\pgfqpoint{2.325000in}{2.310000in}}%
\pgfusepath{clip}%
\pgfsetrectcap%
\pgfsetroundjoin%
\pgfsetlinewidth{0.250937pt}%
\definecolor{currentstroke}{rgb}{0.000000,0.000000,0.000000}%
\pgfsetstrokecolor{currentstroke}%
\pgfsetstrokeopacity{0.200000}%
\pgfsetdash{}{0pt}%
\pgfpathmoveto{\pgfqpoint{0.721913in}{2.670548in}}%
\pgfpathlineto{\pgfqpoint{3.046913in}{2.670548in}}%
\pgfusepath{stroke}%
\end{pgfscope}%
\begin{pgfscope}%
\pgfsetbuttcap%
\pgfsetroundjoin%
\definecolor{currentfill}{rgb}{0.000000,0.000000,0.000000}%
\pgfsetfillcolor{currentfill}%
\pgfsetlinewidth{0.803000pt}%
\definecolor{currentstroke}{rgb}{0.000000,0.000000,0.000000}%
\pgfsetstrokecolor{currentstroke}%
\pgfsetdash{}{0pt}%
\pgfsys@defobject{currentmarker}{\pgfqpoint{-0.048611in}{0.000000in}}{\pgfqpoint{-0.000000in}{0.000000in}}{%
\pgfpathmoveto{\pgfqpoint{-0.000000in}{0.000000in}}%
\pgfpathlineto{\pgfqpoint{-0.048611in}{0.000000in}}%
\pgfusepath{stroke,fill}%
}%
\begin{pgfscope}%
\pgfsys@transformshift{0.721913in}{2.670548in}%
\pgfsys@useobject{currentmarker}{}%
\end{pgfscope}%
\end{pgfscope}%
\begin{pgfscope}%
\definecolor{textcolor}{rgb}{0.000000,0.000000,0.000000}%
\pgfsetstrokecolor{textcolor}%
\pgfsetfillcolor{textcolor}%
\pgftext[x=0.395525in, y=2.612678in, left, base]{\color{textcolor}{\rmfamily\fontsize{12.000000}{14.400000}\selectfont\catcode`\^=\active\def^{\ifmmode\sp\else\^{}\fi}\catcode`\%=\active\def%{\%}$\mathdefault{10^{0}}$}}%
\end{pgfscope}%
\begin{pgfscope}%
\definecolor{textcolor}{rgb}{0.000000,0.000000,0.000000}%
\pgfsetstrokecolor{textcolor}%
\pgfsetfillcolor{textcolor}%
\pgftext[x=0.248147in,y=1.704073in,,bottom,rotate=90.000000]{\color{textcolor}{\rmfamily\fontsize{12.000000}{14.400000}\selectfont\catcode`\^=\active\def^{\ifmmode\sp\else\^{}\fi}\catcode`\%=\active\def%{\%}$L^1$-error}}%
\end{pgfscope}%
\begin{pgfscope}%
\pgfpathrectangle{\pgfqpoint{0.721913in}{0.549073in}}{\pgfqpoint{2.325000in}{2.310000in}}%
\pgfusepath{clip}%
\pgfsetrectcap%
\pgfsetroundjoin%
\pgfsetlinewidth{1.505625pt}%
\definecolor{currentstroke}{rgb}{0.392157,0.560784,1.000000}%
\pgfsetstrokecolor{currentstroke}%
\pgfsetdash{}{0pt}%
\pgfpathmoveto{\pgfqpoint{0.777271in}{2.204699in}}%
\pgfpathlineto{\pgfqpoint{1.159282in}{2.114072in}}%
\pgfpathlineto{\pgfqpoint{1.524410in}{2.059564in}}%
\pgfpathlineto{\pgfqpoint{1.891095in}{2.010133in}}%
\pgfpathlineto{\pgfqpoint{2.259210in}{1.934435in}}%
\pgfpathlineto{\pgfqpoint{2.625645in}{1.861712in}}%
\pgfpathlineto{\pgfqpoint{2.991556in}{1.804047in}}%
\pgfusepath{stroke}%
\end{pgfscope}%
\begin{pgfscope}%
\pgfpathrectangle{\pgfqpoint{0.721913in}{0.549073in}}{\pgfqpoint{2.325000in}{2.310000in}}%
\pgfusepath{clip}%
\pgfsetbuttcap%
\pgfsetroundjoin%
\definecolor{currentfill}{rgb}{0.392157,0.560784,1.000000}%
\pgfsetfillcolor{currentfill}%
\pgfsetlinewidth{1.003750pt}%
\definecolor{currentstroke}{rgb}{0.392157,0.560784,1.000000}%
\pgfsetstrokecolor{currentstroke}%
\pgfsetdash{}{0pt}%
\pgfsys@defobject{currentmarker}{\pgfqpoint{-0.041667in}{-0.041667in}}{\pgfqpoint{0.041667in}{0.041667in}}{%
\pgfpathmoveto{\pgfqpoint{0.000000in}{-0.041667in}}%
\pgfpathcurveto{\pgfqpoint{0.011050in}{-0.041667in}}{\pgfqpoint{0.021649in}{-0.037276in}}{\pgfqpoint{0.029463in}{-0.029463in}}%
\pgfpathcurveto{\pgfqpoint{0.037276in}{-0.021649in}}{\pgfqpoint{0.041667in}{-0.011050in}}{\pgfqpoint{0.041667in}{0.000000in}}%
\pgfpathcurveto{\pgfqpoint{0.041667in}{0.011050in}}{\pgfqpoint{0.037276in}{0.021649in}}{\pgfqpoint{0.029463in}{0.029463in}}%
\pgfpathcurveto{\pgfqpoint{0.021649in}{0.037276in}}{\pgfqpoint{0.011050in}{0.041667in}}{\pgfqpoint{0.000000in}{0.041667in}}%
\pgfpathcurveto{\pgfqpoint{-0.011050in}{0.041667in}}{\pgfqpoint{-0.021649in}{0.037276in}}{\pgfqpoint{-0.029463in}{0.029463in}}%
\pgfpathcurveto{\pgfqpoint{-0.037276in}{0.021649in}}{\pgfqpoint{-0.041667in}{0.011050in}}{\pgfqpoint{-0.041667in}{0.000000in}}%
\pgfpathcurveto{\pgfqpoint{-0.041667in}{-0.011050in}}{\pgfqpoint{-0.037276in}{-0.021649in}}{\pgfqpoint{-0.029463in}{-0.029463in}}%
\pgfpathcurveto{\pgfqpoint{-0.021649in}{-0.037276in}}{\pgfqpoint{-0.011050in}{-0.041667in}}{\pgfqpoint{0.000000in}{-0.041667in}}%
\pgfpathlineto{\pgfqpoint{0.000000in}{-0.041667in}}%
\pgfpathclose%
\pgfusepath{stroke,fill}%
}%
\begin{pgfscope}%
\pgfsys@transformshift{0.777271in}{2.204699in}%
\pgfsys@useobject{currentmarker}{}%
\end{pgfscope}%
\begin{pgfscope}%
\pgfsys@transformshift{1.159282in}{2.114072in}%
\pgfsys@useobject{currentmarker}{}%
\end{pgfscope}%
\begin{pgfscope}%
\pgfsys@transformshift{1.524410in}{2.059564in}%
\pgfsys@useobject{currentmarker}{}%
\end{pgfscope}%
\begin{pgfscope}%
\pgfsys@transformshift{1.891095in}{2.010133in}%
\pgfsys@useobject{currentmarker}{}%
\end{pgfscope}%
\begin{pgfscope}%
\pgfsys@transformshift{2.259210in}{1.934435in}%
\pgfsys@useobject{currentmarker}{}%
\end{pgfscope}%
\begin{pgfscope}%
\pgfsys@transformshift{2.625645in}{1.861712in}%
\pgfsys@useobject{currentmarker}{}%
\end{pgfscope}%
\begin{pgfscope}%
\pgfsys@transformshift{2.991556in}{1.804047in}%
\pgfsys@useobject{currentmarker}{}%
\end{pgfscope}%
\end{pgfscope}%
\begin{pgfscope}%
\pgfpathrectangle{\pgfqpoint{0.721913in}{0.549073in}}{\pgfqpoint{2.325000in}{2.310000in}}%
\pgfusepath{clip}%
\pgfsetrectcap%
\pgfsetroundjoin%
\pgfsetlinewidth{1.505625pt}%
\definecolor{currentstroke}{rgb}{0.862745,0.149020,0.498039}%
\pgfsetstrokecolor{currentstroke}%
\pgfsetdash{}{0pt}%
\pgfpathmoveto{\pgfqpoint{0.777271in}{2.240870in}}%
\pgfpathlineto{\pgfqpoint{1.159282in}{2.156063in}}%
\pgfpathlineto{\pgfqpoint{1.524410in}{2.106084in}}%
\pgfpathlineto{\pgfqpoint{1.891095in}{2.020666in}}%
\pgfpathlineto{\pgfqpoint{2.259210in}{1.894368in}}%
\pgfpathlineto{\pgfqpoint{2.625645in}{1.657607in}}%
\pgfpathlineto{\pgfqpoint{2.991556in}{0.741573in}}%
\pgfusepath{stroke}%
\end{pgfscope}%
\begin{pgfscope}%
\pgfpathrectangle{\pgfqpoint{0.721913in}{0.549073in}}{\pgfqpoint{2.325000in}{2.310000in}}%
\pgfusepath{clip}%
\pgfsetbuttcap%
\pgfsetmiterjoin%
\definecolor{currentfill}{rgb}{0.862745,0.149020,0.498039}%
\pgfsetfillcolor{currentfill}%
\pgfsetlinewidth{1.003750pt}%
\definecolor{currentstroke}{rgb}{0.862745,0.149020,0.498039}%
\pgfsetstrokecolor{currentstroke}%
\pgfsetdash{}{0pt}%
\pgfsys@defobject{currentmarker}{\pgfqpoint{-0.041667in}{-0.041667in}}{\pgfqpoint{0.041667in}{0.041667in}}{%
\pgfpathmoveto{\pgfqpoint{-0.041667in}{-0.041667in}}%
\pgfpathlineto{\pgfqpoint{0.041667in}{-0.041667in}}%
\pgfpathlineto{\pgfqpoint{0.041667in}{0.041667in}}%
\pgfpathlineto{\pgfqpoint{-0.041667in}{0.041667in}}%
\pgfpathlineto{\pgfqpoint{-0.041667in}{-0.041667in}}%
\pgfpathclose%
\pgfusepath{stroke,fill}%
}%
\begin{pgfscope}%
\pgfsys@transformshift{0.777271in}{2.240870in}%
\pgfsys@useobject{currentmarker}{}%
\end{pgfscope}%
\begin{pgfscope}%
\pgfsys@transformshift{1.159282in}{2.156063in}%
\pgfsys@useobject{currentmarker}{}%
\end{pgfscope}%
\begin{pgfscope}%
\pgfsys@transformshift{1.524410in}{2.106084in}%
\pgfsys@useobject{currentmarker}{}%
\end{pgfscope}%
\begin{pgfscope}%
\pgfsys@transformshift{1.891095in}{2.020666in}%
\pgfsys@useobject{currentmarker}{}%
\end{pgfscope}%
\begin{pgfscope}%
\pgfsys@transformshift{2.259210in}{1.894368in}%
\pgfsys@useobject{currentmarker}{}%
\end{pgfscope}%
\begin{pgfscope}%
\pgfsys@transformshift{2.625645in}{1.657607in}%
\pgfsys@useobject{currentmarker}{}%
\end{pgfscope}%
\begin{pgfscope}%
\pgfsys@transformshift{2.991556in}{0.741573in}%
\pgfsys@useobject{currentmarker}{}%
\end{pgfscope}%
\end{pgfscope}%
\begin{pgfscope}%
\pgfpathrectangle{\pgfqpoint{0.721913in}{0.549073in}}{\pgfqpoint{2.325000in}{2.310000in}}%
\pgfusepath{clip}%
\pgfsetrectcap%
\pgfsetroundjoin%
\pgfsetlinewidth{1.505625pt}%
\definecolor{currentstroke}{rgb}{1.000000,0.690196,0.000000}%
\pgfsetstrokecolor{currentstroke}%
\pgfsetdash{}{0pt}%
\pgfpathmoveto{\pgfqpoint{0.777271in}{2.666573in}}%
\pgfpathlineto{\pgfqpoint{1.159282in}{2.660893in}}%
\pgfpathlineto{\pgfqpoint{1.524410in}{2.645765in}}%
\pgfpathlineto{\pgfqpoint{1.891095in}{2.603867in}}%
\pgfpathlineto{\pgfqpoint{2.259210in}{2.484962in}}%
\pgfpathlineto{\pgfqpoint{2.625645in}{1.897006in}}%
\pgfpathlineto{\pgfqpoint{2.991556in}{1.100817in}}%
\pgfusepath{stroke}%
\end{pgfscope}%
\begin{pgfscope}%
\pgfpathrectangle{\pgfqpoint{0.721913in}{0.549073in}}{\pgfqpoint{2.325000in}{2.310000in}}%
\pgfusepath{clip}%
\pgfsetbuttcap%
\pgfsetmiterjoin%
\definecolor{currentfill}{rgb}{1.000000,0.690196,0.000000}%
\pgfsetfillcolor{currentfill}%
\pgfsetlinewidth{1.003750pt}%
\definecolor{currentstroke}{rgb}{1.000000,0.690196,0.000000}%
\pgfsetstrokecolor{currentstroke}%
\pgfsetdash{}{0pt}%
\pgfsys@defobject{currentmarker}{\pgfqpoint{-0.035355in}{-0.058926in}}{\pgfqpoint{0.035355in}{0.058926in}}{%
\pgfpathmoveto{\pgfqpoint{-0.000000in}{-0.058926in}}%
\pgfpathlineto{\pgfqpoint{0.035355in}{0.000000in}}%
\pgfpathlineto{\pgfqpoint{0.000000in}{0.058926in}}%
\pgfpathlineto{\pgfqpoint{-0.035355in}{0.000000in}}%
\pgfpathlineto{\pgfqpoint{-0.000000in}{-0.058926in}}%
\pgfpathclose%
\pgfusepath{stroke,fill}%
}%
\begin{pgfscope}%
\pgfsys@transformshift{0.777271in}{2.666573in}%
\pgfsys@useobject{currentmarker}{}%
\end{pgfscope}%
\begin{pgfscope}%
\pgfsys@transformshift{1.159282in}{2.660893in}%
\pgfsys@useobject{currentmarker}{}%
\end{pgfscope}%
\begin{pgfscope}%
\pgfsys@transformshift{1.524410in}{2.645765in}%
\pgfsys@useobject{currentmarker}{}%
\end{pgfscope}%
\begin{pgfscope}%
\pgfsys@transformshift{1.891095in}{2.603867in}%
\pgfsys@useobject{currentmarker}{}%
\end{pgfscope}%
\begin{pgfscope}%
\pgfsys@transformshift{2.259210in}{2.484962in}%
\pgfsys@useobject{currentmarker}{}%
\end{pgfscope}%
\begin{pgfscope}%
\pgfsys@transformshift{2.625645in}{1.897006in}%
\pgfsys@useobject{currentmarker}{}%
\end{pgfscope}%
\begin{pgfscope}%
\pgfsys@transformshift{2.991556in}{1.100817in}%
\pgfsys@useobject{currentmarker}{}%
\end{pgfscope}%
\end{pgfscope}%
\begin{pgfscope}%
\pgfsetrectcap%
\pgfsetmiterjoin%
\pgfsetlinewidth{0.803000pt}%
\definecolor{currentstroke}{rgb}{0.000000,0.000000,0.000000}%
\pgfsetstrokecolor{currentstroke}%
\pgfsetdash{}{0pt}%
\pgfpathmoveto{\pgfqpoint{0.721913in}{0.549073in}}%
\pgfpathlineto{\pgfqpoint{0.721913in}{2.859073in}}%
\pgfusepath{stroke}%
\end{pgfscope}%
\begin{pgfscope}%
\pgfsetrectcap%
\pgfsetmiterjoin%
\pgfsetlinewidth{0.803000pt}%
\definecolor{currentstroke}{rgb}{0.000000,0.000000,0.000000}%
\pgfsetstrokecolor{currentstroke}%
\pgfsetdash{}{0pt}%
\pgfpathmoveto{\pgfqpoint{3.046913in}{0.549073in}}%
\pgfpathlineto{\pgfqpoint{3.046913in}{2.859073in}}%
\pgfusepath{stroke}%
\end{pgfscope}%
\begin{pgfscope}%
\pgfsetrectcap%
\pgfsetmiterjoin%
\pgfsetlinewidth{0.803000pt}%
\definecolor{currentstroke}{rgb}{0.000000,0.000000,0.000000}%
\pgfsetstrokecolor{currentstroke}%
\pgfsetdash{}{0pt}%
\pgfpathmoveto{\pgfqpoint{0.721913in}{0.549073in}}%
\pgfpathlineto{\pgfqpoint{3.046913in}{0.549073in}}%
\pgfusepath{stroke}%
\end{pgfscope}%
\begin{pgfscope}%
\pgfsetrectcap%
\pgfsetmiterjoin%
\pgfsetlinewidth{0.803000pt}%
\definecolor{currentstroke}{rgb}{0.000000,0.000000,0.000000}%
\pgfsetstrokecolor{currentstroke}%
\pgfsetdash{}{0pt}%
\pgfpathmoveto{\pgfqpoint{0.721913in}{2.859073in}}%
\pgfpathlineto{\pgfqpoint{3.046913in}{2.859073in}}%
\pgfusepath{stroke}%
\end{pgfscope}%
\begin{pgfscope}%
\pgfsetbuttcap%
\pgfsetmiterjoin%
\definecolor{currentfill}{rgb}{1.000000,1.000000,1.000000}%
\pgfsetfillcolor{currentfill}%
\pgfsetfillopacity{0.800000}%
\pgfsetlinewidth{1.003750pt}%
\definecolor{currentstroke}{rgb}{0.800000,0.800000,0.800000}%
\pgfsetstrokecolor{currentstroke}%
\pgfsetstrokeopacity{0.800000}%
\pgfsetdash{}{0pt}%
\pgfpathmoveto{\pgfqpoint{0.805247in}{0.632406in}}%
\pgfpathlineto{\pgfqpoint{2.037651in}{0.632406in}}%
\pgfpathlineto{\pgfqpoint{2.037651in}{1.379627in}}%
\pgfpathlineto{\pgfqpoint{0.805247in}{1.379627in}}%
\pgfpathlineto{\pgfqpoint{0.805247in}{0.632406in}}%
\pgfpathclose%
\pgfusepath{stroke,fill}%
\end{pgfscope}%
\begin{pgfscope}%
\pgfsetrectcap%
\pgfsetroundjoin%
\pgfsetlinewidth{1.505625pt}%
\definecolor{currentstroke}{rgb}{0.392157,0.560784,1.000000}%
\pgfsetstrokecolor{currentstroke}%
\pgfsetdash{}{0pt}%
\pgfpathmoveto{\pgfqpoint{0.871913in}{1.254627in}}%
\pgfpathlineto{\pgfqpoint{1.038580in}{1.254627in}}%
\pgfpathlineto{\pgfqpoint{1.205247in}{1.254627in}}%
\pgfusepath{stroke}%
\end{pgfscope}%
\begin{pgfscope}%
\pgfsetbuttcap%
\pgfsetroundjoin%
\definecolor{currentfill}{rgb}{0.392157,0.560784,1.000000}%
\pgfsetfillcolor{currentfill}%
\pgfsetlinewidth{1.003750pt}%
\definecolor{currentstroke}{rgb}{0.392157,0.560784,1.000000}%
\pgfsetstrokecolor{currentstroke}%
\pgfsetdash{}{0pt}%
\pgfsys@defobject{currentmarker}{\pgfqpoint{-0.031250in}{-0.031250in}}{\pgfqpoint{0.031250in}{0.031250in}}{%
\pgfpathmoveto{\pgfqpoint{0.000000in}{-0.031250in}}%
\pgfpathcurveto{\pgfqpoint{0.008288in}{-0.031250in}}{\pgfqpoint{0.016237in}{-0.027957in}}{\pgfqpoint{0.022097in}{-0.022097in}}%
\pgfpathcurveto{\pgfqpoint{0.027957in}{-0.016237in}}{\pgfqpoint{0.031250in}{-0.008288in}}{\pgfqpoint{0.031250in}{0.000000in}}%
\pgfpathcurveto{\pgfqpoint{0.031250in}{0.008288in}}{\pgfqpoint{0.027957in}{0.016237in}}{\pgfqpoint{0.022097in}{0.022097in}}%
\pgfpathcurveto{\pgfqpoint{0.016237in}{0.027957in}}{\pgfqpoint{0.008288in}{0.031250in}}{\pgfqpoint{0.000000in}{0.031250in}}%
\pgfpathcurveto{\pgfqpoint{-0.008288in}{0.031250in}}{\pgfqpoint{-0.016237in}{0.027957in}}{\pgfqpoint{-0.022097in}{0.022097in}}%
\pgfpathcurveto{\pgfqpoint{-0.027957in}{0.016237in}}{\pgfqpoint{-0.031250in}{0.008288in}}{\pgfqpoint{-0.031250in}{0.000000in}}%
\pgfpathcurveto{\pgfqpoint{-0.031250in}{-0.008288in}}{\pgfqpoint{-0.027957in}{-0.016237in}}{\pgfqpoint{-0.022097in}{-0.022097in}}%
\pgfpathcurveto{\pgfqpoint{-0.016237in}{-0.027957in}}{\pgfqpoint{-0.008288in}{-0.031250in}}{\pgfqpoint{0.000000in}{-0.031250in}}%
\pgfpathlineto{\pgfqpoint{0.000000in}{-0.031250in}}%
\pgfpathclose%
\pgfusepath{stroke,fill}%
}%
\begin{pgfscope}%
\pgfsys@transformshift{1.038580in}{1.254627in}%
\pgfsys@useobject{currentmarker}{}%
\end{pgfscope}%
\end{pgfscope}%
\begin{pgfscope}%
\definecolor{textcolor}{rgb}{0.000000,0.000000,0.000000}%
\pgfsetstrokecolor{textcolor}%
\pgfsetfillcolor{textcolor}%
\pgftext[x=1.338580in,y=1.196294in,left,base]{\color{textcolor}{\rmfamily\fontsize{12.000000}{14.400000}\selectfont\catcode`\^=\active\def^{\ifmmode\sp\else\^{}\fi}\catcode`\%=\active\def%{\%}$n_{\mathbf{\Omega}} = 0$}}%
\end{pgfscope}%
\begin{pgfscope}%
\pgfsetrectcap%
\pgfsetroundjoin%
\pgfsetlinewidth{1.505625pt}%
\definecolor{currentstroke}{rgb}{0.862745,0.149020,0.498039}%
\pgfsetstrokecolor{currentstroke}%
\pgfsetdash{}{0pt}%
\pgfpathmoveto{\pgfqpoint{0.871913in}{1.022220in}}%
\pgfpathlineto{\pgfqpoint{1.038580in}{1.022220in}}%
\pgfpathlineto{\pgfqpoint{1.205247in}{1.022220in}}%
\pgfusepath{stroke}%
\end{pgfscope}%
\begin{pgfscope}%
\pgfsetbuttcap%
\pgfsetmiterjoin%
\definecolor{currentfill}{rgb}{0.862745,0.149020,0.498039}%
\pgfsetfillcolor{currentfill}%
\pgfsetlinewidth{1.003750pt}%
\definecolor{currentstroke}{rgb}{0.862745,0.149020,0.498039}%
\pgfsetstrokecolor{currentstroke}%
\pgfsetdash{}{0pt}%
\pgfsys@defobject{currentmarker}{\pgfqpoint{-0.031250in}{-0.031250in}}{\pgfqpoint{0.031250in}{0.031250in}}{%
\pgfpathmoveto{\pgfqpoint{-0.031250in}{-0.031250in}}%
\pgfpathlineto{\pgfqpoint{0.031250in}{-0.031250in}}%
\pgfpathlineto{\pgfqpoint{0.031250in}{0.031250in}}%
\pgfpathlineto{\pgfqpoint{-0.031250in}{0.031250in}}%
\pgfpathlineto{\pgfqpoint{-0.031250in}{-0.031250in}}%
\pgfpathclose%
\pgfusepath{stroke,fill}%
}%
\begin{pgfscope}%
\pgfsys@transformshift{1.038580in}{1.022220in}%
\pgfsys@useobject{currentmarker}{}%
\end{pgfscope}%
\end{pgfscope}%
\begin{pgfscope}%
\definecolor{textcolor}{rgb}{0.000000,0.000000,0.000000}%
\pgfsetstrokecolor{textcolor}%
\pgfsetfillcolor{textcolor}%
\pgftext[x=1.338580in,y=0.963887in,left,base]{\color{textcolor}{\rmfamily\fontsize{12.000000}{14.400000}\selectfont\catcode`\^=\active\def^{\ifmmode\sp\else\^{}\fi}\catcode`\%=\active\def%{\%}$n_{\mathbf{\Omega}} = n_{\mathbf{\Psi}}$}}%
\end{pgfscope}%
\begin{pgfscope}%
\pgfsetrectcap%
\pgfsetroundjoin%
\pgfsetlinewidth{1.505625pt}%
\definecolor{currentstroke}{rgb}{1.000000,0.690196,0.000000}%
\pgfsetstrokecolor{currentstroke}%
\pgfsetdash{}{0pt}%
\pgfpathmoveto{\pgfqpoint{0.871913in}{0.789813in}}%
\pgfpathlineto{\pgfqpoint{1.038580in}{0.789813in}}%
\pgfpathlineto{\pgfqpoint{1.205247in}{0.789813in}}%
\pgfusepath{stroke}%
\end{pgfscope}%
\begin{pgfscope}%
\pgfsetbuttcap%
\pgfsetmiterjoin%
\definecolor{currentfill}{rgb}{1.000000,0.690196,0.000000}%
\pgfsetfillcolor{currentfill}%
\pgfsetlinewidth{1.003750pt}%
\definecolor{currentstroke}{rgb}{1.000000,0.690196,0.000000}%
\pgfsetstrokecolor{currentstroke}%
\pgfsetdash{}{0pt}%
\pgfsys@defobject{currentmarker}{\pgfqpoint{-0.026517in}{-0.044194in}}{\pgfqpoint{0.026517in}{0.044194in}}{%
\pgfpathmoveto{\pgfqpoint{-0.000000in}{-0.044194in}}%
\pgfpathlineto{\pgfqpoint{0.026517in}{0.000000in}}%
\pgfpathlineto{\pgfqpoint{0.000000in}{0.044194in}}%
\pgfpathlineto{\pgfqpoint{-0.026517in}{0.000000in}}%
\pgfpathlineto{\pgfqpoint{-0.000000in}{-0.044194in}}%
\pgfpathclose%
\pgfusepath{stroke,fill}%
}%
\begin{pgfscope}%
\pgfsys@transformshift{1.038580in}{0.789813in}%
\pgfsys@useobject{currentmarker}{}%
\end{pgfscope}%
\end{pgfscope}%
\begin{pgfscope}%
\definecolor{textcolor}{rgb}{0.000000,0.000000,0.000000}%
\pgfsetstrokecolor{textcolor}%
\pgfsetfillcolor{textcolor}%
\pgftext[x=1.338580in,y=0.731480in,left,base]{\color{textcolor}{\rmfamily\fontsize{12.000000}{14.400000}\selectfont\catcode`\^=\active\def^{\ifmmode\sp\else\^{}\fi}\catcode`\%=\active\def%{\%}$n_{\mathbf{\Psi}} = 0$}}%
\end{pgfscope}%
\end{pgfpicture}%
\makeatother%
\endgroup%
}
        \caption{$n_c = 3.$}
        \label{fig:convergence-3}
    \end{subfigure}
    \caption{Error vs. number of random vectors $n_{\mtx{\Omega}} + n_{\mtx{\Psi}}$ when applying  Chebyshev-Nyström++ to the Hamiltonian matrix from~\cref{subsec:hamiltonian}, with fixed $\sigma=0.005$ and $m=1\,000$, such that the error of the Chebyshev approximation remains sufficiently small according to \cref{fig:chebyshev-heatmap}.}
    \label{fig:convergence}
\end{figure}

For a fixed budget of $n_{\mtx{\Omega}} + n_{\mtx{\Psi}} = 80$ of random vectors, \cref{fig:distribution} shows how the value of the smoothing parameter $\sigma$ impacts the choice between Nyström ($n_{\mtx{\Omega}}$) and 
Girard-Hutchinson ($n_{\mtx{\Psi}}$). As expected, for small $\sigma$ the Nyström approximation alone suffices, but 
this is not an effective choice for large $\sigma$. Choosing $n_{\mtx{\Omega}} = n_{\mtx{\Psi}}$ constitutes a good compromise.


\begin{figure}[ht]
    \centering
    \scalebox{0.8}{ %% Creator: Matplotlib, PGF backend
%%
%% To include the figure in your LaTeX document, write
%%   \input{<filename>.pgf}
%%
%% Make sure the required packages are loaded in your preamble
%%   \usepackage{pgf}
%%
%% Also ensure that all the required font packages are loaded; for instance,
%% the lmodern package is sometimes necessary when using math font.
%%   \usepackage{lmodern}
%%
%% Figures using additional raster images can only be included by \input if
%% they are in the same directory as the main LaTeX file. For loading figures
%% from other directories you can use the `import` package
%%   \usepackage{import}
%%
%% and then include the figures with
%%   \import{<path to file>}{<filename>.pgf}
%%
%% Matplotlib used the following preamble
%%   \def\mathdefault#1{#1}
%%   \everymath=\expandafter{\the\everymath\displaystyle}
%%   
%%   \makeatletter\@ifpackageloaded{underscore}{}{\usepackage[strings]{underscore}}\makeatother
%%
\begingroup%
\makeatletter%
\begin{pgfpicture}%
\pgfpathrectangle{\pgfpointorigin}{\pgfqpoint{5.393080in}{2.959073in}}%
\pgfusepath{use as bounding box, clip}%
\begin{pgfscope}%
\pgfsetbuttcap%
\pgfsetmiterjoin%
\definecolor{currentfill}{rgb}{1.000000,1.000000,1.000000}%
\pgfsetfillcolor{currentfill}%
\pgfsetlinewidth{0.000000pt}%
\definecolor{currentstroke}{rgb}{1.000000,1.000000,1.000000}%
\pgfsetstrokecolor{currentstroke}%
\pgfsetdash{}{0pt}%
\pgfpathmoveto{\pgfqpoint{0.000000in}{-0.000000in}}%
\pgfpathlineto{\pgfqpoint{5.393080in}{-0.000000in}}%
\pgfpathlineto{\pgfqpoint{5.393080in}{2.959073in}}%
\pgfpathlineto{\pgfqpoint{0.000000in}{2.959073in}}%
\pgfpathlineto{\pgfqpoint{0.000000in}{-0.000000in}}%
\pgfpathclose%
\pgfusepath{fill}%
\end{pgfscope}%
\begin{pgfscope}%
\pgfsetbuttcap%
\pgfsetmiterjoin%
\definecolor{currentfill}{rgb}{1.000000,1.000000,1.000000}%
\pgfsetfillcolor{currentfill}%
\pgfsetlinewidth{0.000000pt}%
\definecolor{currentstroke}{rgb}{0.000000,0.000000,0.000000}%
\pgfsetstrokecolor{currentstroke}%
\pgfsetstrokeopacity{0.000000}%
\pgfsetdash{}{0pt}%
\pgfpathmoveto{\pgfqpoint{0.721913in}{0.549073in}}%
\pgfpathlineto{\pgfqpoint{5.240163in}{0.549073in}}%
\pgfpathlineto{\pgfqpoint{5.240163in}{2.859073in}}%
\pgfpathlineto{\pgfqpoint{0.721913in}{2.859073in}}%
\pgfpathlineto{\pgfqpoint{0.721913in}{0.549073in}}%
\pgfpathclose%
\pgfusepath{fill}%
\end{pgfscope}%
\begin{pgfscope}%
\pgfpathrectangle{\pgfqpoint{0.721913in}{0.549073in}}{\pgfqpoint{4.518250in}{2.310000in}}%
\pgfusepath{clip}%
\pgfsetrectcap%
\pgfsetroundjoin%
\pgfsetlinewidth{0.250937pt}%
\definecolor{currentstroke}{rgb}{0.000000,0.000000,0.000000}%
\pgfsetstrokecolor{currentstroke}%
\pgfsetstrokeopacity{0.200000}%
\pgfsetdash{}{0pt}%
\pgfpathmoveto{\pgfqpoint{0.829491in}{0.549073in}}%
\pgfpathlineto{\pgfqpoint{0.829491in}{2.859073in}}%
\pgfusepath{stroke}%
\end{pgfscope}%
\begin{pgfscope}%
\pgfsetbuttcap%
\pgfsetroundjoin%
\definecolor{currentfill}{rgb}{0.000000,0.000000,0.000000}%
\pgfsetfillcolor{currentfill}%
\pgfsetlinewidth{0.803000pt}%
\definecolor{currentstroke}{rgb}{0.000000,0.000000,0.000000}%
\pgfsetstrokecolor{currentstroke}%
\pgfsetdash{}{0pt}%
\pgfsys@defobject{currentmarker}{\pgfqpoint{0.000000in}{-0.048611in}}{\pgfqpoint{0.000000in}{0.000000in}}{%
\pgfpathmoveto{\pgfqpoint{0.000000in}{0.000000in}}%
\pgfpathlineto{\pgfqpoint{0.000000in}{-0.048611in}}%
\pgfusepath{stroke,fill}%
}%
\begin{pgfscope}%
\pgfsys@transformshift{0.829491in}{0.549073in}%
\pgfsys@useobject{currentmarker}{}%
\end{pgfscope}%
\end{pgfscope}%
\begin{pgfscope}%
\definecolor{textcolor}{rgb}{0.000000,0.000000,0.000000}%
\pgfsetstrokecolor{textcolor}%
\pgfsetfillcolor{textcolor}%
\pgftext[x=0.829491in,y=0.451851in,,top]{\color{textcolor}{\rmfamily\fontsize{12.000000}{14.400000}\selectfont\catcode`\^=\active\def^{\ifmmode\sp\else\^{}\fi}\catcode`\%=\active\def%{\%}$\mathdefault{10^{-3}}$}}%
\end{pgfscope}%
\begin{pgfscope}%
\pgfpathrectangle{\pgfqpoint{0.721913in}{0.549073in}}{\pgfqpoint{4.518250in}{2.310000in}}%
\pgfusepath{clip}%
\pgfsetrectcap%
\pgfsetroundjoin%
\pgfsetlinewidth{0.250937pt}%
\definecolor{currentstroke}{rgb}{0.000000,0.000000,0.000000}%
\pgfsetstrokecolor{currentstroke}%
\pgfsetstrokeopacity{0.200000}%
\pgfsetdash{}{0pt}%
\pgfpathmoveto{\pgfqpoint{2.981038in}{0.549073in}}%
\pgfpathlineto{\pgfqpoint{2.981038in}{2.859073in}}%
\pgfusepath{stroke}%
\end{pgfscope}%
\begin{pgfscope}%
\pgfsetbuttcap%
\pgfsetroundjoin%
\definecolor{currentfill}{rgb}{0.000000,0.000000,0.000000}%
\pgfsetfillcolor{currentfill}%
\pgfsetlinewidth{0.803000pt}%
\definecolor{currentstroke}{rgb}{0.000000,0.000000,0.000000}%
\pgfsetstrokecolor{currentstroke}%
\pgfsetdash{}{0pt}%
\pgfsys@defobject{currentmarker}{\pgfqpoint{0.000000in}{-0.048611in}}{\pgfqpoint{0.000000in}{0.000000in}}{%
\pgfpathmoveto{\pgfqpoint{0.000000in}{0.000000in}}%
\pgfpathlineto{\pgfqpoint{0.000000in}{-0.048611in}}%
\pgfusepath{stroke,fill}%
}%
\begin{pgfscope}%
\pgfsys@transformshift{2.981038in}{0.549073in}%
\pgfsys@useobject{currentmarker}{}%
\end{pgfscope}%
\end{pgfscope}%
\begin{pgfscope}%
\definecolor{textcolor}{rgb}{0.000000,0.000000,0.000000}%
\pgfsetstrokecolor{textcolor}%
\pgfsetfillcolor{textcolor}%
\pgftext[x=2.981038in,y=0.451851in,,top]{\color{textcolor}{\rmfamily\fontsize{12.000000}{14.400000}\selectfont\catcode`\^=\active\def^{\ifmmode\sp\else\^{}\fi}\catcode`\%=\active\def%{\%}$\mathdefault{10^{-2}}$}}%
\end{pgfscope}%
\begin{pgfscope}%
\pgfpathrectangle{\pgfqpoint{0.721913in}{0.549073in}}{\pgfqpoint{4.518250in}{2.310000in}}%
\pgfusepath{clip}%
\pgfsetrectcap%
\pgfsetroundjoin%
\pgfsetlinewidth{0.250937pt}%
\definecolor{currentstroke}{rgb}{0.000000,0.000000,0.000000}%
\pgfsetstrokecolor{currentstroke}%
\pgfsetstrokeopacity{0.200000}%
\pgfsetdash{}{0pt}%
\pgfpathmoveto{\pgfqpoint{5.132586in}{0.549073in}}%
\pgfpathlineto{\pgfqpoint{5.132586in}{2.859073in}}%
\pgfusepath{stroke}%
\end{pgfscope}%
\begin{pgfscope}%
\pgfsetbuttcap%
\pgfsetroundjoin%
\definecolor{currentfill}{rgb}{0.000000,0.000000,0.000000}%
\pgfsetfillcolor{currentfill}%
\pgfsetlinewidth{0.803000pt}%
\definecolor{currentstroke}{rgb}{0.000000,0.000000,0.000000}%
\pgfsetstrokecolor{currentstroke}%
\pgfsetdash{}{0pt}%
\pgfsys@defobject{currentmarker}{\pgfqpoint{0.000000in}{-0.048611in}}{\pgfqpoint{0.000000in}{0.000000in}}{%
\pgfpathmoveto{\pgfqpoint{0.000000in}{0.000000in}}%
\pgfpathlineto{\pgfqpoint{0.000000in}{-0.048611in}}%
\pgfusepath{stroke,fill}%
}%
\begin{pgfscope}%
\pgfsys@transformshift{5.132586in}{0.549073in}%
\pgfsys@useobject{currentmarker}{}%
\end{pgfscope}%
\end{pgfscope}%
\begin{pgfscope}%
\definecolor{textcolor}{rgb}{0.000000,0.000000,0.000000}%
\pgfsetstrokecolor{textcolor}%
\pgfsetfillcolor{textcolor}%
\pgftext[x=5.132586in,y=0.451851in,,top]{\color{textcolor}{\rmfamily\fontsize{12.000000}{14.400000}\selectfont\catcode`\^=\active\def^{\ifmmode\sp\else\^{}\fi}\catcode`\%=\active\def%{\%}$\mathdefault{10^{-1}}$}}%
\end{pgfscope}%
\begin{pgfscope}%
\pgfpathrectangle{\pgfqpoint{0.721913in}{0.549073in}}{\pgfqpoint{4.518250in}{2.310000in}}%
\pgfusepath{clip}%
\pgfsetrectcap%
\pgfsetroundjoin%
\pgfsetlinewidth{0.250937pt}%
\definecolor{currentstroke}{rgb}{0.000000,0.000000,0.000000}%
\pgfsetstrokecolor{currentstroke}%
\pgfsetstrokeopacity{0.200000}%
\pgfsetdash{}{0pt}%
\pgfpathmoveto{\pgfqpoint{0.731041in}{0.549073in}}%
\pgfpathlineto{\pgfqpoint{0.731041in}{2.859073in}}%
\pgfusepath{stroke}%
\end{pgfscope}%
\begin{pgfscope}%
\pgfsetbuttcap%
\pgfsetroundjoin%
\definecolor{currentfill}{rgb}{0.000000,0.000000,0.000000}%
\pgfsetfillcolor{currentfill}%
\pgfsetlinewidth{0.602250pt}%
\definecolor{currentstroke}{rgb}{0.000000,0.000000,0.000000}%
\pgfsetstrokecolor{currentstroke}%
\pgfsetdash{}{0pt}%
\pgfsys@defobject{currentmarker}{\pgfqpoint{0.000000in}{-0.027778in}}{\pgfqpoint{0.000000in}{0.000000in}}{%
\pgfpathmoveto{\pgfqpoint{0.000000in}{0.000000in}}%
\pgfpathlineto{\pgfqpoint{0.000000in}{-0.027778in}}%
\pgfusepath{stroke,fill}%
}%
\begin{pgfscope}%
\pgfsys@transformshift{0.731041in}{0.549073in}%
\pgfsys@useobject{currentmarker}{}%
\end{pgfscope}%
\end{pgfscope}%
\begin{pgfscope}%
\pgfpathrectangle{\pgfqpoint{0.721913in}{0.549073in}}{\pgfqpoint{4.518250in}{2.310000in}}%
\pgfusepath{clip}%
\pgfsetrectcap%
\pgfsetroundjoin%
\pgfsetlinewidth{0.250937pt}%
\definecolor{currentstroke}{rgb}{0.000000,0.000000,0.000000}%
\pgfsetstrokecolor{currentstroke}%
\pgfsetstrokeopacity{0.200000}%
\pgfsetdash{}{0pt}%
\pgfpathmoveto{\pgfqpoint{1.477171in}{0.549073in}}%
\pgfpathlineto{\pgfqpoint{1.477171in}{2.859073in}}%
\pgfusepath{stroke}%
\end{pgfscope}%
\begin{pgfscope}%
\pgfsetbuttcap%
\pgfsetroundjoin%
\definecolor{currentfill}{rgb}{0.000000,0.000000,0.000000}%
\pgfsetfillcolor{currentfill}%
\pgfsetlinewidth{0.602250pt}%
\definecolor{currentstroke}{rgb}{0.000000,0.000000,0.000000}%
\pgfsetstrokecolor{currentstroke}%
\pgfsetdash{}{0pt}%
\pgfsys@defobject{currentmarker}{\pgfqpoint{0.000000in}{-0.027778in}}{\pgfqpoint{0.000000in}{0.000000in}}{%
\pgfpathmoveto{\pgfqpoint{0.000000in}{0.000000in}}%
\pgfpathlineto{\pgfqpoint{0.000000in}{-0.027778in}}%
\pgfusepath{stroke,fill}%
}%
\begin{pgfscope}%
\pgfsys@transformshift{1.477171in}{0.549073in}%
\pgfsys@useobject{currentmarker}{}%
\end{pgfscope}%
\end{pgfscope}%
\begin{pgfscope}%
\pgfpathrectangle{\pgfqpoint{0.721913in}{0.549073in}}{\pgfqpoint{4.518250in}{2.310000in}}%
\pgfusepath{clip}%
\pgfsetrectcap%
\pgfsetroundjoin%
\pgfsetlinewidth{0.250937pt}%
\definecolor{currentstroke}{rgb}{0.000000,0.000000,0.000000}%
\pgfsetstrokecolor{currentstroke}%
\pgfsetstrokeopacity{0.200000}%
\pgfsetdash{}{0pt}%
\pgfpathmoveto{\pgfqpoint{1.856040in}{0.549073in}}%
\pgfpathlineto{\pgfqpoint{1.856040in}{2.859073in}}%
\pgfusepath{stroke}%
\end{pgfscope}%
\begin{pgfscope}%
\pgfsetbuttcap%
\pgfsetroundjoin%
\definecolor{currentfill}{rgb}{0.000000,0.000000,0.000000}%
\pgfsetfillcolor{currentfill}%
\pgfsetlinewidth{0.602250pt}%
\definecolor{currentstroke}{rgb}{0.000000,0.000000,0.000000}%
\pgfsetstrokecolor{currentstroke}%
\pgfsetdash{}{0pt}%
\pgfsys@defobject{currentmarker}{\pgfqpoint{0.000000in}{-0.027778in}}{\pgfqpoint{0.000000in}{0.000000in}}{%
\pgfpathmoveto{\pgfqpoint{0.000000in}{0.000000in}}%
\pgfpathlineto{\pgfqpoint{0.000000in}{-0.027778in}}%
\pgfusepath{stroke,fill}%
}%
\begin{pgfscope}%
\pgfsys@transformshift{1.856040in}{0.549073in}%
\pgfsys@useobject{currentmarker}{}%
\end{pgfscope}%
\end{pgfscope}%
\begin{pgfscope}%
\pgfpathrectangle{\pgfqpoint{0.721913in}{0.549073in}}{\pgfqpoint{4.518250in}{2.310000in}}%
\pgfusepath{clip}%
\pgfsetrectcap%
\pgfsetroundjoin%
\pgfsetlinewidth{0.250937pt}%
\definecolor{currentstroke}{rgb}{0.000000,0.000000,0.000000}%
\pgfsetstrokecolor{currentstroke}%
\pgfsetstrokeopacity{0.200000}%
\pgfsetdash{}{0pt}%
\pgfpathmoveto{\pgfqpoint{2.124852in}{0.549073in}}%
\pgfpathlineto{\pgfqpoint{2.124852in}{2.859073in}}%
\pgfusepath{stroke}%
\end{pgfscope}%
\begin{pgfscope}%
\pgfsetbuttcap%
\pgfsetroundjoin%
\definecolor{currentfill}{rgb}{0.000000,0.000000,0.000000}%
\pgfsetfillcolor{currentfill}%
\pgfsetlinewidth{0.602250pt}%
\definecolor{currentstroke}{rgb}{0.000000,0.000000,0.000000}%
\pgfsetstrokecolor{currentstroke}%
\pgfsetdash{}{0pt}%
\pgfsys@defobject{currentmarker}{\pgfqpoint{0.000000in}{-0.027778in}}{\pgfqpoint{0.000000in}{0.000000in}}{%
\pgfpathmoveto{\pgfqpoint{0.000000in}{0.000000in}}%
\pgfpathlineto{\pgfqpoint{0.000000in}{-0.027778in}}%
\pgfusepath{stroke,fill}%
}%
\begin{pgfscope}%
\pgfsys@transformshift{2.124852in}{0.549073in}%
\pgfsys@useobject{currentmarker}{}%
\end{pgfscope}%
\end{pgfscope}%
\begin{pgfscope}%
\pgfpathrectangle{\pgfqpoint{0.721913in}{0.549073in}}{\pgfqpoint{4.518250in}{2.310000in}}%
\pgfusepath{clip}%
\pgfsetrectcap%
\pgfsetroundjoin%
\pgfsetlinewidth{0.250937pt}%
\definecolor{currentstroke}{rgb}{0.000000,0.000000,0.000000}%
\pgfsetstrokecolor{currentstroke}%
\pgfsetstrokeopacity{0.200000}%
\pgfsetdash{}{0pt}%
\pgfpathmoveto{\pgfqpoint{2.333358in}{0.549073in}}%
\pgfpathlineto{\pgfqpoint{2.333358in}{2.859073in}}%
\pgfusepath{stroke}%
\end{pgfscope}%
\begin{pgfscope}%
\pgfsetbuttcap%
\pgfsetroundjoin%
\definecolor{currentfill}{rgb}{0.000000,0.000000,0.000000}%
\pgfsetfillcolor{currentfill}%
\pgfsetlinewidth{0.602250pt}%
\definecolor{currentstroke}{rgb}{0.000000,0.000000,0.000000}%
\pgfsetstrokecolor{currentstroke}%
\pgfsetdash{}{0pt}%
\pgfsys@defobject{currentmarker}{\pgfqpoint{0.000000in}{-0.027778in}}{\pgfqpoint{0.000000in}{0.000000in}}{%
\pgfpathmoveto{\pgfqpoint{0.000000in}{0.000000in}}%
\pgfpathlineto{\pgfqpoint{0.000000in}{-0.027778in}}%
\pgfusepath{stroke,fill}%
}%
\begin{pgfscope}%
\pgfsys@transformshift{2.333358in}{0.549073in}%
\pgfsys@useobject{currentmarker}{}%
\end{pgfscope}%
\end{pgfscope}%
\begin{pgfscope}%
\pgfpathrectangle{\pgfqpoint{0.721913in}{0.549073in}}{\pgfqpoint{4.518250in}{2.310000in}}%
\pgfusepath{clip}%
\pgfsetrectcap%
\pgfsetroundjoin%
\pgfsetlinewidth{0.250937pt}%
\definecolor{currentstroke}{rgb}{0.000000,0.000000,0.000000}%
\pgfsetstrokecolor{currentstroke}%
\pgfsetstrokeopacity{0.200000}%
\pgfsetdash{}{0pt}%
\pgfpathmoveto{\pgfqpoint{2.503720in}{0.549073in}}%
\pgfpathlineto{\pgfqpoint{2.503720in}{2.859073in}}%
\pgfusepath{stroke}%
\end{pgfscope}%
\begin{pgfscope}%
\pgfsetbuttcap%
\pgfsetroundjoin%
\definecolor{currentfill}{rgb}{0.000000,0.000000,0.000000}%
\pgfsetfillcolor{currentfill}%
\pgfsetlinewidth{0.602250pt}%
\definecolor{currentstroke}{rgb}{0.000000,0.000000,0.000000}%
\pgfsetstrokecolor{currentstroke}%
\pgfsetdash{}{0pt}%
\pgfsys@defobject{currentmarker}{\pgfqpoint{0.000000in}{-0.027778in}}{\pgfqpoint{0.000000in}{0.000000in}}{%
\pgfpathmoveto{\pgfqpoint{0.000000in}{0.000000in}}%
\pgfpathlineto{\pgfqpoint{0.000000in}{-0.027778in}}%
\pgfusepath{stroke,fill}%
}%
\begin{pgfscope}%
\pgfsys@transformshift{2.503720in}{0.549073in}%
\pgfsys@useobject{currentmarker}{}%
\end{pgfscope}%
\end{pgfscope}%
\begin{pgfscope}%
\pgfpathrectangle{\pgfqpoint{0.721913in}{0.549073in}}{\pgfqpoint{4.518250in}{2.310000in}}%
\pgfusepath{clip}%
\pgfsetrectcap%
\pgfsetroundjoin%
\pgfsetlinewidth{0.250937pt}%
\definecolor{currentstroke}{rgb}{0.000000,0.000000,0.000000}%
\pgfsetstrokecolor{currentstroke}%
\pgfsetstrokeopacity{0.200000}%
\pgfsetdash{}{0pt}%
\pgfpathmoveto{\pgfqpoint{2.647760in}{0.549073in}}%
\pgfpathlineto{\pgfqpoint{2.647760in}{2.859073in}}%
\pgfusepath{stroke}%
\end{pgfscope}%
\begin{pgfscope}%
\pgfsetbuttcap%
\pgfsetroundjoin%
\definecolor{currentfill}{rgb}{0.000000,0.000000,0.000000}%
\pgfsetfillcolor{currentfill}%
\pgfsetlinewidth{0.602250pt}%
\definecolor{currentstroke}{rgb}{0.000000,0.000000,0.000000}%
\pgfsetstrokecolor{currentstroke}%
\pgfsetdash{}{0pt}%
\pgfsys@defobject{currentmarker}{\pgfqpoint{0.000000in}{-0.027778in}}{\pgfqpoint{0.000000in}{0.000000in}}{%
\pgfpathmoveto{\pgfqpoint{0.000000in}{0.000000in}}%
\pgfpathlineto{\pgfqpoint{0.000000in}{-0.027778in}}%
\pgfusepath{stroke,fill}%
}%
\begin{pgfscope}%
\pgfsys@transformshift{2.647760in}{0.549073in}%
\pgfsys@useobject{currentmarker}{}%
\end{pgfscope}%
\end{pgfscope}%
\begin{pgfscope}%
\pgfpathrectangle{\pgfqpoint{0.721913in}{0.549073in}}{\pgfqpoint{4.518250in}{2.310000in}}%
\pgfusepath{clip}%
\pgfsetrectcap%
\pgfsetroundjoin%
\pgfsetlinewidth{0.250937pt}%
\definecolor{currentstroke}{rgb}{0.000000,0.000000,0.000000}%
\pgfsetstrokecolor{currentstroke}%
\pgfsetstrokeopacity{0.200000}%
\pgfsetdash{}{0pt}%
\pgfpathmoveto{\pgfqpoint{2.772532in}{0.549073in}}%
\pgfpathlineto{\pgfqpoint{2.772532in}{2.859073in}}%
\pgfusepath{stroke}%
\end{pgfscope}%
\begin{pgfscope}%
\pgfsetbuttcap%
\pgfsetroundjoin%
\definecolor{currentfill}{rgb}{0.000000,0.000000,0.000000}%
\pgfsetfillcolor{currentfill}%
\pgfsetlinewidth{0.602250pt}%
\definecolor{currentstroke}{rgb}{0.000000,0.000000,0.000000}%
\pgfsetstrokecolor{currentstroke}%
\pgfsetdash{}{0pt}%
\pgfsys@defobject{currentmarker}{\pgfqpoint{0.000000in}{-0.027778in}}{\pgfqpoint{0.000000in}{0.000000in}}{%
\pgfpathmoveto{\pgfqpoint{0.000000in}{0.000000in}}%
\pgfpathlineto{\pgfqpoint{0.000000in}{-0.027778in}}%
\pgfusepath{stroke,fill}%
}%
\begin{pgfscope}%
\pgfsys@transformshift{2.772532in}{0.549073in}%
\pgfsys@useobject{currentmarker}{}%
\end{pgfscope}%
\end{pgfscope}%
\begin{pgfscope}%
\pgfpathrectangle{\pgfqpoint{0.721913in}{0.549073in}}{\pgfqpoint{4.518250in}{2.310000in}}%
\pgfusepath{clip}%
\pgfsetrectcap%
\pgfsetroundjoin%
\pgfsetlinewidth{0.250937pt}%
\definecolor{currentstroke}{rgb}{0.000000,0.000000,0.000000}%
\pgfsetstrokecolor{currentstroke}%
\pgfsetstrokeopacity{0.200000}%
\pgfsetdash{}{0pt}%
\pgfpathmoveto{\pgfqpoint{2.882589in}{0.549073in}}%
\pgfpathlineto{\pgfqpoint{2.882589in}{2.859073in}}%
\pgfusepath{stroke}%
\end{pgfscope}%
\begin{pgfscope}%
\pgfsetbuttcap%
\pgfsetroundjoin%
\definecolor{currentfill}{rgb}{0.000000,0.000000,0.000000}%
\pgfsetfillcolor{currentfill}%
\pgfsetlinewidth{0.602250pt}%
\definecolor{currentstroke}{rgb}{0.000000,0.000000,0.000000}%
\pgfsetstrokecolor{currentstroke}%
\pgfsetdash{}{0pt}%
\pgfsys@defobject{currentmarker}{\pgfqpoint{0.000000in}{-0.027778in}}{\pgfqpoint{0.000000in}{0.000000in}}{%
\pgfpathmoveto{\pgfqpoint{0.000000in}{0.000000in}}%
\pgfpathlineto{\pgfqpoint{0.000000in}{-0.027778in}}%
\pgfusepath{stroke,fill}%
}%
\begin{pgfscope}%
\pgfsys@transformshift{2.882589in}{0.549073in}%
\pgfsys@useobject{currentmarker}{}%
\end{pgfscope}%
\end{pgfscope}%
\begin{pgfscope}%
\pgfpathrectangle{\pgfqpoint{0.721913in}{0.549073in}}{\pgfqpoint{4.518250in}{2.310000in}}%
\pgfusepath{clip}%
\pgfsetrectcap%
\pgfsetroundjoin%
\pgfsetlinewidth{0.250937pt}%
\definecolor{currentstroke}{rgb}{0.000000,0.000000,0.000000}%
\pgfsetstrokecolor{currentstroke}%
\pgfsetstrokeopacity{0.200000}%
\pgfsetdash{}{0pt}%
\pgfpathmoveto{\pgfqpoint{3.628719in}{0.549073in}}%
\pgfpathlineto{\pgfqpoint{3.628719in}{2.859073in}}%
\pgfusepath{stroke}%
\end{pgfscope}%
\begin{pgfscope}%
\pgfsetbuttcap%
\pgfsetroundjoin%
\definecolor{currentfill}{rgb}{0.000000,0.000000,0.000000}%
\pgfsetfillcolor{currentfill}%
\pgfsetlinewidth{0.602250pt}%
\definecolor{currentstroke}{rgb}{0.000000,0.000000,0.000000}%
\pgfsetstrokecolor{currentstroke}%
\pgfsetdash{}{0pt}%
\pgfsys@defobject{currentmarker}{\pgfqpoint{0.000000in}{-0.027778in}}{\pgfqpoint{0.000000in}{0.000000in}}{%
\pgfpathmoveto{\pgfqpoint{0.000000in}{0.000000in}}%
\pgfpathlineto{\pgfqpoint{0.000000in}{-0.027778in}}%
\pgfusepath{stroke,fill}%
}%
\begin{pgfscope}%
\pgfsys@transformshift{3.628719in}{0.549073in}%
\pgfsys@useobject{currentmarker}{}%
\end{pgfscope}%
\end{pgfscope}%
\begin{pgfscope}%
\pgfpathrectangle{\pgfqpoint{0.721913in}{0.549073in}}{\pgfqpoint{4.518250in}{2.310000in}}%
\pgfusepath{clip}%
\pgfsetrectcap%
\pgfsetroundjoin%
\pgfsetlinewidth{0.250937pt}%
\definecolor{currentstroke}{rgb}{0.000000,0.000000,0.000000}%
\pgfsetstrokecolor{currentstroke}%
\pgfsetstrokeopacity{0.200000}%
\pgfsetdash{}{0pt}%
\pgfpathmoveto{\pgfqpoint{4.007588in}{0.549073in}}%
\pgfpathlineto{\pgfqpoint{4.007588in}{2.859073in}}%
\pgfusepath{stroke}%
\end{pgfscope}%
\begin{pgfscope}%
\pgfsetbuttcap%
\pgfsetroundjoin%
\definecolor{currentfill}{rgb}{0.000000,0.000000,0.000000}%
\pgfsetfillcolor{currentfill}%
\pgfsetlinewidth{0.602250pt}%
\definecolor{currentstroke}{rgb}{0.000000,0.000000,0.000000}%
\pgfsetstrokecolor{currentstroke}%
\pgfsetdash{}{0pt}%
\pgfsys@defobject{currentmarker}{\pgfqpoint{0.000000in}{-0.027778in}}{\pgfqpoint{0.000000in}{0.000000in}}{%
\pgfpathmoveto{\pgfqpoint{0.000000in}{0.000000in}}%
\pgfpathlineto{\pgfqpoint{0.000000in}{-0.027778in}}%
\pgfusepath{stroke,fill}%
}%
\begin{pgfscope}%
\pgfsys@transformshift{4.007588in}{0.549073in}%
\pgfsys@useobject{currentmarker}{}%
\end{pgfscope}%
\end{pgfscope}%
\begin{pgfscope}%
\pgfpathrectangle{\pgfqpoint{0.721913in}{0.549073in}}{\pgfqpoint{4.518250in}{2.310000in}}%
\pgfusepath{clip}%
\pgfsetrectcap%
\pgfsetroundjoin%
\pgfsetlinewidth{0.250937pt}%
\definecolor{currentstroke}{rgb}{0.000000,0.000000,0.000000}%
\pgfsetstrokecolor{currentstroke}%
\pgfsetstrokeopacity{0.200000}%
\pgfsetdash{}{0pt}%
\pgfpathmoveto{\pgfqpoint{4.276399in}{0.549073in}}%
\pgfpathlineto{\pgfqpoint{4.276399in}{2.859073in}}%
\pgfusepath{stroke}%
\end{pgfscope}%
\begin{pgfscope}%
\pgfsetbuttcap%
\pgfsetroundjoin%
\definecolor{currentfill}{rgb}{0.000000,0.000000,0.000000}%
\pgfsetfillcolor{currentfill}%
\pgfsetlinewidth{0.602250pt}%
\definecolor{currentstroke}{rgb}{0.000000,0.000000,0.000000}%
\pgfsetstrokecolor{currentstroke}%
\pgfsetdash{}{0pt}%
\pgfsys@defobject{currentmarker}{\pgfqpoint{0.000000in}{-0.027778in}}{\pgfqpoint{0.000000in}{0.000000in}}{%
\pgfpathmoveto{\pgfqpoint{0.000000in}{0.000000in}}%
\pgfpathlineto{\pgfqpoint{0.000000in}{-0.027778in}}%
\pgfusepath{stroke,fill}%
}%
\begin{pgfscope}%
\pgfsys@transformshift{4.276399in}{0.549073in}%
\pgfsys@useobject{currentmarker}{}%
\end{pgfscope}%
\end{pgfscope}%
\begin{pgfscope}%
\pgfpathrectangle{\pgfqpoint{0.721913in}{0.549073in}}{\pgfqpoint{4.518250in}{2.310000in}}%
\pgfusepath{clip}%
\pgfsetrectcap%
\pgfsetroundjoin%
\pgfsetlinewidth{0.250937pt}%
\definecolor{currentstroke}{rgb}{0.000000,0.000000,0.000000}%
\pgfsetstrokecolor{currentstroke}%
\pgfsetstrokeopacity{0.200000}%
\pgfsetdash{}{0pt}%
\pgfpathmoveto{\pgfqpoint{4.484906in}{0.549073in}}%
\pgfpathlineto{\pgfqpoint{4.484906in}{2.859073in}}%
\pgfusepath{stroke}%
\end{pgfscope}%
\begin{pgfscope}%
\pgfsetbuttcap%
\pgfsetroundjoin%
\definecolor{currentfill}{rgb}{0.000000,0.000000,0.000000}%
\pgfsetfillcolor{currentfill}%
\pgfsetlinewidth{0.602250pt}%
\definecolor{currentstroke}{rgb}{0.000000,0.000000,0.000000}%
\pgfsetstrokecolor{currentstroke}%
\pgfsetdash{}{0pt}%
\pgfsys@defobject{currentmarker}{\pgfqpoint{0.000000in}{-0.027778in}}{\pgfqpoint{0.000000in}{0.000000in}}{%
\pgfpathmoveto{\pgfqpoint{0.000000in}{0.000000in}}%
\pgfpathlineto{\pgfqpoint{0.000000in}{-0.027778in}}%
\pgfusepath{stroke,fill}%
}%
\begin{pgfscope}%
\pgfsys@transformshift{4.484906in}{0.549073in}%
\pgfsys@useobject{currentmarker}{}%
\end{pgfscope}%
\end{pgfscope}%
\begin{pgfscope}%
\pgfpathrectangle{\pgfqpoint{0.721913in}{0.549073in}}{\pgfqpoint{4.518250in}{2.310000in}}%
\pgfusepath{clip}%
\pgfsetrectcap%
\pgfsetroundjoin%
\pgfsetlinewidth{0.250937pt}%
\definecolor{currentstroke}{rgb}{0.000000,0.000000,0.000000}%
\pgfsetstrokecolor{currentstroke}%
\pgfsetstrokeopacity{0.200000}%
\pgfsetdash{}{0pt}%
\pgfpathmoveto{\pgfqpoint{4.655268in}{0.549073in}}%
\pgfpathlineto{\pgfqpoint{4.655268in}{2.859073in}}%
\pgfusepath{stroke}%
\end{pgfscope}%
\begin{pgfscope}%
\pgfsetbuttcap%
\pgfsetroundjoin%
\definecolor{currentfill}{rgb}{0.000000,0.000000,0.000000}%
\pgfsetfillcolor{currentfill}%
\pgfsetlinewidth{0.602250pt}%
\definecolor{currentstroke}{rgb}{0.000000,0.000000,0.000000}%
\pgfsetstrokecolor{currentstroke}%
\pgfsetdash{}{0pt}%
\pgfsys@defobject{currentmarker}{\pgfqpoint{0.000000in}{-0.027778in}}{\pgfqpoint{0.000000in}{0.000000in}}{%
\pgfpathmoveto{\pgfqpoint{0.000000in}{0.000000in}}%
\pgfpathlineto{\pgfqpoint{0.000000in}{-0.027778in}}%
\pgfusepath{stroke,fill}%
}%
\begin{pgfscope}%
\pgfsys@transformshift{4.655268in}{0.549073in}%
\pgfsys@useobject{currentmarker}{}%
\end{pgfscope}%
\end{pgfscope}%
\begin{pgfscope}%
\pgfpathrectangle{\pgfqpoint{0.721913in}{0.549073in}}{\pgfqpoint{4.518250in}{2.310000in}}%
\pgfusepath{clip}%
\pgfsetrectcap%
\pgfsetroundjoin%
\pgfsetlinewidth{0.250937pt}%
\definecolor{currentstroke}{rgb}{0.000000,0.000000,0.000000}%
\pgfsetstrokecolor{currentstroke}%
\pgfsetstrokeopacity{0.200000}%
\pgfsetdash{}{0pt}%
\pgfpathmoveto{\pgfqpoint{4.799307in}{0.549073in}}%
\pgfpathlineto{\pgfqpoint{4.799307in}{2.859073in}}%
\pgfusepath{stroke}%
\end{pgfscope}%
\begin{pgfscope}%
\pgfsetbuttcap%
\pgfsetroundjoin%
\definecolor{currentfill}{rgb}{0.000000,0.000000,0.000000}%
\pgfsetfillcolor{currentfill}%
\pgfsetlinewidth{0.602250pt}%
\definecolor{currentstroke}{rgb}{0.000000,0.000000,0.000000}%
\pgfsetstrokecolor{currentstroke}%
\pgfsetdash{}{0pt}%
\pgfsys@defobject{currentmarker}{\pgfqpoint{0.000000in}{-0.027778in}}{\pgfqpoint{0.000000in}{0.000000in}}{%
\pgfpathmoveto{\pgfqpoint{0.000000in}{0.000000in}}%
\pgfpathlineto{\pgfqpoint{0.000000in}{-0.027778in}}%
\pgfusepath{stroke,fill}%
}%
\begin{pgfscope}%
\pgfsys@transformshift{4.799307in}{0.549073in}%
\pgfsys@useobject{currentmarker}{}%
\end{pgfscope}%
\end{pgfscope}%
\begin{pgfscope}%
\pgfpathrectangle{\pgfqpoint{0.721913in}{0.549073in}}{\pgfqpoint{4.518250in}{2.310000in}}%
\pgfusepath{clip}%
\pgfsetrectcap%
\pgfsetroundjoin%
\pgfsetlinewidth{0.250937pt}%
\definecolor{currentstroke}{rgb}{0.000000,0.000000,0.000000}%
\pgfsetstrokecolor{currentstroke}%
\pgfsetstrokeopacity{0.200000}%
\pgfsetdash{}{0pt}%
\pgfpathmoveto{\pgfqpoint{4.924080in}{0.549073in}}%
\pgfpathlineto{\pgfqpoint{4.924080in}{2.859073in}}%
\pgfusepath{stroke}%
\end{pgfscope}%
\begin{pgfscope}%
\pgfsetbuttcap%
\pgfsetroundjoin%
\definecolor{currentfill}{rgb}{0.000000,0.000000,0.000000}%
\pgfsetfillcolor{currentfill}%
\pgfsetlinewidth{0.602250pt}%
\definecolor{currentstroke}{rgb}{0.000000,0.000000,0.000000}%
\pgfsetstrokecolor{currentstroke}%
\pgfsetdash{}{0pt}%
\pgfsys@defobject{currentmarker}{\pgfqpoint{0.000000in}{-0.027778in}}{\pgfqpoint{0.000000in}{0.000000in}}{%
\pgfpathmoveto{\pgfqpoint{0.000000in}{0.000000in}}%
\pgfpathlineto{\pgfqpoint{0.000000in}{-0.027778in}}%
\pgfusepath{stroke,fill}%
}%
\begin{pgfscope}%
\pgfsys@transformshift{4.924080in}{0.549073in}%
\pgfsys@useobject{currentmarker}{}%
\end{pgfscope}%
\end{pgfscope}%
\begin{pgfscope}%
\pgfpathrectangle{\pgfqpoint{0.721913in}{0.549073in}}{\pgfqpoint{4.518250in}{2.310000in}}%
\pgfusepath{clip}%
\pgfsetrectcap%
\pgfsetroundjoin%
\pgfsetlinewidth{0.250937pt}%
\definecolor{currentstroke}{rgb}{0.000000,0.000000,0.000000}%
\pgfsetstrokecolor{currentstroke}%
\pgfsetstrokeopacity{0.200000}%
\pgfsetdash{}{0pt}%
\pgfpathmoveto{\pgfqpoint{5.034137in}{0.549073in}}%
\pgfpathlineto{\pgfqpoint{5.034137in}{2.859073in}}%
\pgfusepath{stroke}%
\end{pgfscope}%
\begin{pgfscope}%
\pgfsetbuttcap%
\pgfsetroundjoin%
\definecolor{currentfill}{rgb}{0.000000,0.000000,0.000000}%
\pgfsetfillcolor{currentfill}%
\pgfsetlinewidth{0.602250pt}%
\definecolor{currentstroke}{rgb}{0.000000,0.000000,0.000000}%
\pgfsetstrokecolor{currentstroke}%
\pgfsetdash{}{0pt}%
\pgfsys@defobject{currentmarker}{\pgfqpoint{0.000000in}{-0.027778in}}{\pgfqpoint{0.000000in}{0.000000in}}{%
\pgfpathmoveto{\pgfqpoint{0.000000in}{0.000000in}}%
\pgfpathlineto{\pgfqpoint{0.000000in}{-0.027778in}}%
\pgfusepath{stroke,fill}%
}%
\begin{pgfscope}%
\pgfsys@transformshift{5.034137in}{0.549073in}%
\pgfsys@useobject{currentmarker}{}%
\end{pgfscope}%
\end{pgfscope}%
\begin{pgfscope}%
\definecolor{textcolor}{rgb}{0.000000,0.000000,0.000000}%
\pgfsetstrokecolor{textcolor}%
\pgfsetfillcolor{textcolor}%
\pgftext[x=2.981038in,y=0.248148in,,top]{\color{textcolor}{\rmfamily\fontsize{12.000000}{14.400000}\selectfont\catcode`\^=\active\def^{\ifmmode\sp\else\^{}\fi}\catcode`\%=\active\def%{\%}smoothing parameter $\sigma$}}%
\end{pgfscope}%
\begin{pgfscope}%
\pgfpathrectangle{\pgfqpoint{0.721913in}{0.549073in}}{\pgfqpoint{4.518250in}{2.310000in}}%
\pgfusepath{clip}%
\pgfsetrectcap%
\pgfsetroundjoin%
\pgfsetlinewidth{0.250937pt}%
\definecolor{currentstroke}{rgb}{0.000000,0.000000,0.000000}%
\pgfsetstrokecolor{currentstroke}%
\pgfsetstrokeopacity{0.200000}%
\pgfsetdash{}{0pt}%
\pgfpathmoveto{\pgfqpoint{0.721913in}{1.106088in}}%
\pgfpathlineto{\pgfqpoint{5.240163in}{1.106088in}}%
\pgfusepath{stroke}%
\end{pgfscope}%
\begin{pgfscope}%
\pgfsetbuttcap%
\pgfsetroundjoin%
\definecolor{currentfill}{rgb}{0.000000,0.000000,0.000000}%
\pgfsetfillcolor{currentfill}%
\pgfsetlinewidth{0.803000pt}%
\definecolor{currentstroke}{rgb}{0.000000,0.000000,0.000000}%
\pgfsetstrokecolor{currentstroke}%
\pgfsetdash{}{0pt}%
\pgfsys@defobject{currentmarker}{\pgfqpoint{-0.048611in}{0.000000in}}{\pgfqpoint{-0.000000in}{0.000000in}}{%
\pgfpathmoveto{\pgfqpoint{-0.000000in}{0.000000in}}%
\pgfpathlineto{\pgfqpoint{-0.048611in}{0.000000in}}%
\pgfusepath{stroke,fill}%
}%
\begin{pgfscope}%
\pgfsys@transformshift{0.721913in}{1.106088in}%
\pgfsys@useobject{currentmarker}{}%
\end{pgfscope}%
\end{pgfscope}%
\begin{pgfscope}%
\definecolor{textcolor}{rgb}{0.000000,0.000000,0.000000}%
\pgfsetstrokecolor{textcolor}%
\pgfsetfillcolor{textcolor}%
\pgftext[x=0.303703in, y=1.048218in, left, base]{\color{textcolor}{\rmfamily\fontsize{12.000000}{14.400000}\selectfont\catcode`\^=\active\def^{\ifmmode\sp\else\^{}\fi}\catcode`\%=\active\def%{\%}$\mathdefault{10^{-5}}$}}%
\end{pgfscope}%
\begin{pgfscope}%
\pgfpathrectangle{\pgfqpoint{0.721913in}{0.549073in}}{\pgfqpoint{4.518250in}{2.310000in}}%
\pgfusepath{clip}%
\pgfsetrectcap%
\pgfsetroundjoin%
\pgfsetlinewidth{0.250937pt}%
\definecolor{currentstroke}{rgb}{0.000000,0.000000,0.000000}%
\pgfsetstrokecolor{currentstroke}%
\pgfsetstrokeopacity{0.200000}%
\pgfsetdash{}{0pt}%
\pgfpathmoveto{\pgfqpoint{0.721913in}{1.757546in}}%
\pgfpathlineto{\pgfqpoint{5.240163in}{1.757546in}}%
\pgfusepath{stroke}%
\end{pgfscope}%
\begin{pgfscope}%
\pgfsetbuttcap%
\pgfsetroundjoin%
\definecolor{currentfill}{rgb}{0.000000,0.000000,0.000000}%
\pgfsetfillcolor{currentfill}%
\pgfsetlinewidth{0.803000pt}%
\definecolor{currentstroke}{rgb}{0.000000,0.000000,0.000000}%
\pgfsetstrokecolor{currentstroke}%
\pgfsetdash{}{0pt}%
\pgfsys@defobject{currentmarker}{\pgfqpoint{-0.048611in}{0.000000in}}{\pgfqpoint{-0.000000in}{0.000000in}}{%
\pgfpathmoveto{\pgfqpoint{-0.000000in}{0.000000in}}%
\pgfpathlineto{\pgfqpoint{-0.048611in}{0.000000in}}%
\pgfusepath{stroke,fill}%
}%
\begin{pgfscope}%
\pgfsys@transformshift{0.721913in}{1.757546in}%
\pgfsys@useobject{currentmarker}{}%
\end{pgfscope}%
\end{pgfscope}%
\begin{pgfscope}%
\definecolor{textcolor}{rgb}{0.000000,0.000000,0.000000}%
\pgfsetstrokecolor{textcolor}%
\pgfsetfillcolor{textcolor}%
\pgftext[x=0.303703in, y=1.699676in, left, base]{\color{textcolor}{\rmfamily\fontsize{12.000000}{14.400000}\selectfont\catcode`\^=\active\def^{\ifmmode\sp\else\^{}\fi}\catcode`\%=\active\def%{\%}$\mathdefault{10^{-3}}$}}%
\end{pgfscope}%
\begin{pgfscope}%
\pgfpathrectangle{\pgfqpoint{0.721913in}{0.549073in}}{\pgfqpoint{4.518250in}{2.310000in}}%
\pgfusepath{clip}%
\pgfsetrectcap%
\pgfsetroundjoin%
\pgfsetlinewidth{0.250937pt}%
\definecolor{currentstroke}{rgb}{0.000000,0.000000,0.000000}%
\pgfsetstrokecolor{currentstroke}%
\pgfsetstrokeopacity{0.200000}%
\pgfsetdash{}{0pt}%
\pgfpathmoveto{\pgfqpoint{0.721913in}{2.409005in}}%
\pgfpathlineto{\pgfqpoint{5.240163in}{2.409005in}}%
\pgfusepath{stroke}%
\end{pgfscope}%
\begin{pgfscope}%
\pgfsetbuttcap%
\pgfsetroundjoin%
\definecolor{currentfill}{rgb}{0.000000,0.000000,0.000000}%
\pgfsetfillcolor{currentfill}%
\pgfsetlinewidth{0.803000pt}%
\definecolor{currentstroke}{rgb}{0.000000,0.000000,0.000000}%
\pgfsetstrokecolor{currentstroke}%
\pgfsetdash{}{0pt}%
\pgfsys@defobject{currentmarker}{\pgfqpoint{-0.048611in}{0.000000in}}{\pgfqpoint{-0.000000in}{0.000000in}}{%
\pgfpathmoveto{\pgfqpoint{-0.000000in}{0.000000in}}%
\pgfpathlineto{\pgfqpoint{-0.048611in}{0.000000in}}%
\pgfusepath{stroke,fill}%
}%
\begin{pgfscope}%
\pgfsys@transformshift{0.721913in}{2.409005in}%
\pgfsys@useobject{currentmarker}{}%
\end{pgfscope}%
\end{pgfscope}%
\begin{pgfscope}%
\definecolor{textcolor}{rgb}{0.000000,0.000000,0.000000}%
\pgfsetstrokecolor{textcolor}%
\pgfsetfillcolor{textcolor}%
\pgftext[x=0.303703in, y=2.351135in, left, base]{\color{textcolor}{\rmfamily\fontsize{12.000000}{14.400000}\selectfont\catcode`\^=\active\def^{\ifmmode\sp\else\^{}\fi}\catcode`\%=\active\def%{\%}$\mathdefault{10^{-1}}$}}%
\end{pgfscope}%
\begin{pgfscope}%
\definecolor{textcolor}{rgb}{0.000000,0.000000,0.000000}%
\pgfsetstrokecolor{textcolor}%
\pgfsetfillcolor{textcolor}%
\pgftext[x=0.248148in,y=1.704073in,,bottom,rotate=90.000000]{\color{textcolor}{\rmfamily\fontsize{12.000000}{14.400000}\selectfont\catcode`\^=\active\def^{\ifmmode\sp\else\^{}\fi}\catcode`\%=\active\def%{\%}$L^1$ error}}%
\end{pgfscope}%
\begin{pgfscope}%
\pgfpathrectangle{\pgfqpoint{0.721913in}{0.549073in}}{\pgfqpoint{4.518250in}{2.310000in}}%
\pgfusepath{clip}%
\pgfsetrectcap%
\pgfsetroundjoin%
\pgfsetlinewidth{1.505625pt}%
\definecolor{currentstroke}{rgb}{0.996369,0.791167,0.553499}%
\pgfsetstrokecolor{currentstroke}%
\pgfsetdash{}{0pt}%
\pgfpathmoveto{\pgfqpoint{0.829491in}{2.334844in}}%
\pgfpathlineto{\pgfqpoint{1.546673in}{2.285486in}}%
\pgfpathlineto{\pgfqpoint{2.263856in}{2.236461in}}%
\pgfpathlineto{\pgfqpoint{2.981038in}{2.212275in}}%
\pgfpathlineto{\pgfqpoint{3.698221in}{2.189423in}}%
\pgfpathlineto{\pgfqpoint{4.415404in}{2.142727in}}%
\pgfpathlineto{\pgfqpoint{5.132586in}{2.077965in}}%
\pgfusepath{stroke}%
\end{pgfscope}%
\begin{pgfscope}%
\pgfpathrectangle{\pgfqpoint{0.721913in}{0.549073in}}{\pgfqpoint{4.518250in}{2.310000in}}%
\pgfusepath{clip}%
\pgfsetbuttcap%
\pgfsetmiterjoin%
\definecolor{currentfill}{rgb}{0.996369,0.791167,0.553499}%
\pgfsetfillcolor{currentfill}%
\pgfsetlinewidth{1.003750pt}%
\definecolor{currentstroke}{rgb}{0.996369,0.791167,0.553499}%
\pgfsetstrokecolor{currentstroke}%
\pgfsetdash{}{0pt}%
\pgfsys@defobject{currentmarker}{\pgfqpoint{-0.047140in}{-0.078567in}}{\pgfqpoint{0.047140in}{0.078567in}}{%
\pgfpathmoveto{\pgfqpoint{-0.000000in}{-0.078567in}}%
\pgfpathlineto{\pgfqpoint{0.047140in}{0.000000in}}%
\pgfpathlineto{\pgfqpoint{0.000000in}{0.078567in}}%
\pgfpathlineto{\pgfqpoint{-0.047140in}{0.000000in}}%
\pgfpathlineto{\pgfqpoint{-0.000000in}{-0.078567in}}%
\pgfpathclose%
\pgfusepath{stroke,fill}%
}%
\begin{pgfscope}%
\pgfsys@transformshift{0.829491in}{2.334844in}%
\pgfsys@useobject{currentmarker}{}%
\end{pgfscope}%
\begin{pgfscope}%
\pgfsys@transformshift{1.546673in}{2.285486in}%
\pgfsys@useobject{currentmarker}{}%
\end{pgfscope}%
\begin{pgfscope}%
\pgfsys@transformshift{2.263856in}{2.236461in}%
\pgfsys@useobject{currentmarker}{}%
\end{pgfscope}%
\begin{pgfscope}%
\pgfsys@transformshift{2.981038in}{2.212275in}%
\pgfsys@useobject{currentmarker}{}%
\end{pgfscope}%
\begin{pgfscope}%
\pgfsys@transformshift{3.698221in}{2.189423in}%
\pgfsys@useobject{currentmarker}{}%
\end{pgfscope}%
\begin{pgfscope}%
\pgfsys@transformshift{4.415404in}{2.142727in}%
\pgfsys@useobject{currentmarker}{}%
\end{pgfscope}%
\begin{pgfscope}%
\pgfsys@transformshift{5.132586in}{2.077965in}%
\pgfsys@useobject{currentmarker}{}%
\end{pgfscope}%
\end{pgfscope}%
\begin{pgfscope}%
\pgfpathrectangle{\pgfqpoint{0.721913in}{0.549073in}}{\pgfqpoint{4.518250in}{2.310000in}}%
\pgfusepath{clip}%
\pgfsetrectcap%
\pgfsetroundjoin%
\pgfsetlinewidth{1.505625pt}%
\definecolor{currentstroke}{rgb}{0.944006,0.377643,0.365136}%
\pgfsetstrokecolor{currentstroke}%
\pgfsetdash{}{0pt}%
\pgfpathmoveto{\pgfqpoint{0.829491in}{1.867322in}}%
\pgfpathlineto{\pgfqpoint{1.546673in}{1.927324in}}%
\pgfpathlineto{\pgfqpoint{2.263856in}{2.065271in}}%
\pgfpathlineto{\pgfqpoint{2.981038in}{2.118461in}}%
\pgfpathlineto{\pgfqpoint{3.698221in}{2.145188in}}%
\pgfpathlineto{\pgfqpoint{4.415404in}{2.123513in}}%
\pgfpathlineto{\pgfqpoint{5.132586in}{2.076224in}}%
\pgfusepath{stroke}%
\end{pgfscope}%
\begin{pgfscope}%
\pgfpathrectangle{\pgfqpoint{0.721913in}{0.549073in}}{\pgfqpoint{4.518250in}{2.310000in}}%
\pgfusepath{clip}%
\pgfsetbuttcap%
\pgfsetmiterjoin%
\definecolor{currentfill}{rgb}{0.944006,0.377643,0.365136}%
\pgfsetfillcolor{currentfill}%
\pgfsetlinewidth{1.003750pt}%
\definecolor{currentstroke}{rgb}{0.944006,0.377643,0.365136}%
\pgfsetstrokecolor{currentstroke}%
\pgfsetdash{}{0pt}%
\pgfsys@defobject{currentmarker}{\pgfqpoint{-0.052836in}{-0.044945in}}{\pgfqpoint{0.052836in}{0.055556in}}{%
\pgfpathmoveto{\pgfqpoint{0.000000in}{0.055556in}}%
\pgfpathlineto{\pgfqpoint{-0.052836in}{0.017168in}}%
\pgfpathlineto{\pgfqpoint{-0.032655in}{-0.044945in}}%
\pgfpathlineto{\pgfqpoint{0.032655in}{-0.044945in}}%
\pgfpathlineto{\pgfqpoint{0.052836in}{0.017168in}}%
\pgfpathlineto{\pgfqpoint{0.000000in}{0.055556in}}%
\pgfpathclose%
\pgfusepath{stroke,fill}%
}%
\begin{pgfscope}%
\pgfsys@transformshift{0.829491in}{1.867322in}%
\pgfsys@useobject{currentmarker}{}%
\end{pgfscope}%
\begin{pgfscope}%
\pgfsys@transformshift{1.546673in}{1.927324in}%
\pgfsys@useobject{currentmarker}{}%
\end{pgfscope}%
\begin{pgfscope}%
\pgfsys@transformshift{2.263856in}{2.065271in}%
\pgfsys@useobject{currentmarker}{}%
\end{pgfscope}%
\begin{pgfscope}%
\pgfsys@transformshift{2.981038in}{2.118461in}%
\pgfsys@useobject{currentmarker}{}%
\end{pgfscope}%
\begin{pgfscope}%
\pgfsys@transformshift{3.698221in}{2.145188in}%
\pgfsys@useobject{currentmarker}{}%
\end{pgfscope}%
\begin{pgfscope}%
\pgfsys@transformshift{4.415404in}{2.123513in}%
\pgfsys@useobject{currentmarker}{}%
\end{pgfscope}%
\begin{pgfscope}%
\pgfsys@transformshift{5.132586in}{2.076224in}%
\pgfsys@useobject{currentmarker}{}%
\end{pgfscope}%
\end{pgfscope}%
\begin{pgfscope}%
\pgfpathrectangle{\pgfqpoint{0.721913in}{0.549073in}}{\pgfqpoint{4.518250in}{2.310000in}}%
\pgfusepath{clip}%
\pgfsetrectcap%
\pgfsetroundjoin%
\pgfsetlinewidth{1.505625pt}%
\definecolor{currentstroke}{rgb}{0.620005,0.183840,0.497524}%
\pgfsetstrokecolor{currentstroke}%
\pgfsetdash{}{0pt}%
\pgfpathmoveto{\pgfqpoint{0.829491in}{1.025419in}}%
\pgfpathlineto{\pgfqpoint{1.546673in}{1.804421in}}%
\pgfpathlineto{\pgfqpoint{2.263856in}{2.032570in}}%
\pgfpathlineto{\pgfqpoint{2.981038in}{2.017955in}}%
\pgfpathlineto{\pgfqpoint{3.698221in}{2.059145in}}%
\pgfpathlineto{\pgfqpoint{4.415404in}{2.091041in}}%
\pgfpathlineto{\pgfqpoint{5.132586in}{2.122345in}}%
\pgfusepath{stroke}%
\end{pgfscope}%
\begin{pgfscope}%
\pgfpathrectangle{\pgfqpoint{0.721913in}{0.549073in}}{\pgfqpoint{4.518250in}{2.310000in}}%
\pgfusepath{clip}%
\pgfsetbuttcap%
\pgfsetmiterjoin%
\definecolor{currentfill}{rgb}{0.620005,0.183840,0.497524}%
\pgfsetfillcolor{currentfill}%
\pgfsetlinewidth{1.003750pt}%
\definecolor{currentstroke}{rgb}{0.620005,0.183840,0.497524}%
\pgfsetstrokecolor{currentstroke}%
\pgfsetdash{}{0pt}%
\pgfsys@defobject{currentmarker}{\pgfqpoint{-0.055556in}{-0.055556in}}{\pgfqpoint{0.055556in}{0.055556in}}{%
\pgfpathmoveto{\pgfqpoint{-0.055556in}{-0.055556in}}%
\pgfpathlineto{\pgfqpoint{0.055556in}{-0.055556in}}%
\pgfpathlineto{\pgfqpoint{0.055556in}{0.055556in}}%
\pgfpathlineto{\pgfqpoint{-0.055556in}{0.055556in}}%
\pgfpathlineto{\pgfqpoint{-0.055556in}{-0.055556in}}%
\pgfpathclose%
\pgfusepath{stroke,fill}%
}%
\begin{pgfscope}%
\pgfsys@transformshift{0.829491in}{1.025419in}%
\pgfsys@useobject{currentmarker}{}%
\end{pgfscope}%
\begin{pgfscope}%
\pgfsys@transformshift{1.546673in}{1.804421in}%
\pgfsys@useobject{currentmarker}{}%
\end{pgfscope}%
\begin{pgfscope}%
\pgfsys@transformshift{2.263856in}{2.032570in}%
\pgfsys@useobject{currentmarker}{}%
\end{pgfscope}%
\begin{pgfscope}%
\pgfsys@transformshift{2.981038in}{2.017955in}%
\pgfsys@useobject{currentmarker}{}%
\end{pgfscope}%
\begin{pgfscope}%
\pgfsys@transformshift{3.698221in}{2.059145in}%
\pgfsys@useobject{currentmarker}{}%
\end{pgfscope}%
\begin{pgfscope}%
\pgfsys@transformshift{4.415404in}{2.091041in}%
\pgfsys@useobject{currentmarker}{}%
\end{pgfscope}%
\begin{pgfscope}%
\pgfsys@transformshift{5.132586in}{2.122345in}%
\pgfsys@useobject{currentmarker}{}%
\end{pgfscope}%
\end{pgfscope}%
\begin{pgfscope}%
\pgfpathrectangle{\pgfqpoint{0.721913in}{0.549073in}}{\pgfqpoint{4.518250in}{2.310000in}}%
\pgfusepath{clip}%
\pgfsetrectcap%
\pgfsetroundjoin%
\pgfsetlinewidth{1.505625pt}%
\definecolor{currentstroke}{rgb}{0.265447,0.060237,0.461840}%
\pgfsetstrokecolor{currentstroke}%
\pgfsetdash{}{0pt}%
\pgfpathmoveto{\pgfqpoint{0.829491in}{0.741573in}}%
\pgfpathlineto{\pgfqpoint{1.546673in}{1.299068in}}%
\pgfpathlineto{\pgfqpoint{2.263856in}{1.828909in}}%
\pgfpathlineto{\pgfqpoint{2.981038in}{1.909198in}}%
\pgfpathlineto{\pgfqpoint{3.698221in}{2.012388in}}%
\pgfpathlineto{\pgfqpoint{4.415404in}{2.085098in}}%
\pgfpathlineto{\pgfqpoint{5.132586in}{2.137620in}}%
\pgfusepath{stroke}%
\end{pgfscope}%
\begin{pgfscope}%
\pgfpathrectangle{\pgfqpoint{0.721913in}{0.549073in}}{\pgfqpoint{4.518250in}{2.310000in}}%
\pgfusepath{clip}%
\pgfsetbuttcap%
\pgfsetmiterjoin%
\definecolor{currentfill}{rgb}{0.265447,0.060237,0.461840}%
\pgfsetfillcolor{currentfill}%
\pgfsetlinewidth{1.003750pt}%
\definecolor{currentstroke}{rgb}{0.265447,0.060237,0.461840}%
\pgfsetstrokecolor{currentstroke}%
\pgfsetdash{}{0pt}%
\pgfsys@defobject{currentmarker}{\pgfqpoint{-0.055556in}{-0.055556in}}{\pgfqpoint{0.055556in}{0.055556in}}{%
\pgfpathmoveto{\pgfqpoint{0.000000in}{0.055556in}}%
\pgfpathlineto{\pgfqpoint{-0.055556in}{-0.055556in}}%
\pgfpathlineto{\pgfqpoint{0.055556in}{-0.055556in}}%
\pgfpathlineto{\pgfqpoint{0.000000in}{0.055556in}}%
\pgfpathclose%
\pgfusepath{stroke,fill}%
}%
\begin{pgfscope}%
\pgfsys@transformshift{0.829491in}{0.741573in}%
\pgfsys@useobject{currentmarker}{}%
\end{pgfscope}%
\begin{pgfscope}%
\pgfsys@transformshift{1.546673in}{1.299068in}%
\pgfsys@useobject{currentmarker}{}%
\end{pgfscope}%
\begin{pgfscope}%
\pgfsys@transformshift{2.263856in}{1.828909in}%
\pgfsys@useobject{currentmarker}{}%
\end{pgfscope}%
\begin{pgfscope}%
\pgfsys@transformshift{2.981038in}{1.909198in}%
\pgfsys@useobject{currentmarker}{}%
\end{pgfscope}%
\begin{pgfscope}%
\pgfsys@transformshift{3.698221in}{2.012388in}%
\pgfsys@useobject{currentmarker}{}%
\end{pgfscope}%
\begin{pgfscope}%
\pgfsys@transformshift{4.415404in}{2.085098in}%
\pgfsys@useobject{currentmarker}{}%
\end{pgfscope}%
\begin{pgfscope}%
\pgfsys@transformshift{5.132586in}{2.137620in}%
\pgfsys@useobject{currentmarker}{}%
\end{pgfscope}%
\end{pgfscope}%
\begin{pgfscope}%
\pgfpathrectangle{\pgfqpoint{0.721913in}{0.549073in}}{\pgfqpoint{4.518250in}{2.310000in}}%
\pgfusepath{clip}%
\pgfsetrectcap%
\pgfsetroundjoin%
\pgfsetlinewidth{1.505625pt}%
\definecolor{currentstroke}{rgb}{0.001462,0.000466,0.013866}%
\pgfsetstrokecolor{currentstroke}%
\pgfsetdash{}{0pt}%
\pgfpathmoveto{\pgfqpoint{0.829491in}{0.924089in}}%
\pgfpathlineto{\pgfqpoint{1.546673in}{1.155129in}}%
\pgfpathlineto{\pgfqpoint{2.263856in}{1.801493in}}%
\pgfpathlineto{\pgfqpoint{2.981038in}{2.213882in}}%
\pgfpathlineto{\pgfqpoint{3.698221in}{2.434195in}}%
\pgfpathlineto{\pgfqpoint{4.415404in}{2.590517in}}%
\pgfpathlineto{\pgfqpoint{5.132586in}{2.666573in}}%
\pgfusepath{stroke}%
\end{pgfscope}%
\begin{pgfscope}%
\pgfpathrectangle{\pgfqpoint{0.721913in}{0.549073in}}{\pgfqpoint{4.518250in}{2.310000in}}%
\pgfusepath{clip}%
\pgfsetbuttcap%
\pgfsetroundjoin%
\definecolor{currentfill}{rgb}{0.001462,0.000466,0.013866}%
\pgfsetfillcolor{currentfill}%
\pgfsetlinewidth{1.003750pt}%
\definecolor{currentstroke}{rgb}{0.001462,0.000466,0.013866}%
\pgfsetstrokecolor{currentstroke}%
\pgfsetdash{}{0pt}%
\pgfsys@defobject{currentmarker}{\pgfqpoint{-0.055556in}{-0.055556in}}{\pgfqpoint{0.055556in}{0.055556in}}{%
\pgfpathmoveto{\pgfqpoint{0.000000in}{-0.055556in}}%
\pgfpathcurveto{\pgfqpoint{0.014734in}{-0.055556in}}{\pgfqpoint{0.028866in}{-0.049702in}}{\pgfqpoint{0.039284in}{-0.039284in}}%
\pgfpathcurveto{\pgfqpoint{0.049702in}{-0.028866in}}{\pgfqpoint{0.055556in}{-0.014734in}}{\pgfqpoint{0.055556in}{0.000000in}}%
\pgfpathcurveto{\pgfqpoint{0.055556in}{0.014734in}}{\pgfqpoint{0.049702in}{0.028866in}}{\pgfqpoint{0.039284in}{0.039284in}}%
\pgfpathcurveto{\pgfqpoint{0.028866in}{0.049702in}}{\pgfqpoint{0.014734in}{0.055556in}}{\pgfqpoint{0.000000in}{0.055556in}}%
\pgfpathcurveto{\pgfqpoint{-0.014734in}{0.055556in}}{\pgfqpoint{-0.028866in}{0.049702in}}{\pgfqpoint{-0.039284in}{0.039284in}}%
\pgfpathcurveto{\pgfqpoint{-0.049702in}{0.028866in}}{\pgfqpoint{-0.055556in}{0.014734in}}{\pgfqpoint{-0.055556in}{0.000000in}}%
\pgfpathcurveto{\pgfqpoint{-0.055556in}{-0.014734in}}{\pgfqpoint{-0.049702in}{-0.028866in}}{\pgfqpoint{-0.039284in}{-0.039284in}}%
\pgfpathcurveto{\pgfqpoint{-0.028866in}{-0.049702in}}{\pgfqpoint{-0.014734in}{-0.055556in}}{\pgfqpoint{0.000000in}{-0.055556in}}%
\pgfpathlineto{\pgfqpoint{0.000000in}{-0.055556in}}%
\pgfpathclose%
\pgfusepath{stroke,fill}%
}%
\begin{pgfscope}%
\pgfsys@transformshift{0.829491in}{0.924089in}%
\pgfsys@useobject{currentmarker}{}%
\end{pgfscope}%
\begin{pgfscope}%
\pgfsys@transformshift{1.546673in}{1.155129in}%
\pgfsys@useobject{currentmarker}{}%
\end{pgfscope}%
\begin{pgfscope}%
\pgfsys@transformshift{2.263856in}{1.801493in}%
\pgfsys@useobject{currentmarker}{}%
\end{pgfscope}%
\begin{pgfscope}%
\pgfsys@transformshift{2.981038in}{2.213882in}%
\pgfsys@useobject{currentmarker}{}%
\end{pgfscope}%
\begin{pgfscope}%
\pgfsys@transformshift{3.698221in}{2.434195in}%
\pgfsys@useobject{currentmarker}{}%
\end{pgfscope}%
\begin{pgfscope}%
\pgfsys@transformshift{4.415404in}{2.590517in}%
\pgfsys@useobject{currentmarker}{}%
\end{pgfscope}%
\begin{pgfscope}%
\pgfsys@transformshift{5.132586in}{2.666573in}%
\pgfsys@useobject{currentmarker}{}%
\end{pgfscope}%
\end{pgfscope}%
\begin{pgfscope}%
\pgfsetrectcap%
\pgfsetmiterjoin%
\pgfsetlinewidth{0.803000pt}%
\definecolor{currentstroke}{rgb}{0.000000,0.000000,0.000000}%
\pgfsetstrokecolor{currentstroke}%
\pgfsetdash{}{0pt}%
\pgfpathmoveto{\pgfqpoint{0.721913in}{0.549073in}}%
\pgfpathlineto{\pgfqpoint{0.721913in}{2.859073in}}%
\pgfusepath{stroke}%
\end{pgfscope}%
\begin{pgfscope}%
\pgfsetrectcap%
\pgfsetmiterjoin%
\pgfsetlinewidth{0.803000pt}%
\definecolor{currentstroke}{rgb}{0.000000,0.000000,0.000000}%
\pgfsetstrokecolor{currentstroke}%
\pgfsetdash{}{0pt}%
\pgfpathmoveto{\pgfqpoint{5.240163in}{0.549073in}}%
\pgfpathlineto{\pgfqpoint{5.240163in}{2.859073in}}%
\pgfusepath{stroke}%
\end{pgfscope}%
\begin{pgfscope}%
\pgfsetrectcap%
\pgfsetmiterjoin%
\pgfsetlinewidth{0.803000pt}%
\definecolor{currentstroke}{rgb}{0.000000,0.000000,0.000000}%
\pgfsetstrokecolor{currentstroke}%
\pgfsetdash{}{0pt}%
\pgfpathmoveto{\pgfqpoint{0.721913in}{0.549073in}}%
\pgfpathlineto{\pgfqpoint{5.240163in}{0.549073in}}%
\pgfusepath{stroke}%
\end{pgfscope}%
\begin{pgfscope}%
\pgfsetrectcap%
\pgfsetmiterjoin%
\pgfsetlinewidth{0.803000pt}%
\definecolor{currentstroke}{rgb}{0.000000,0.000000,0.000000}%
\pgfsetstrokecolor{currentstroke}%
\pgfsetdash{}{0pt}%
\pgfpathmoveto{\pgfqpoint{0.721913in}{2.859073in}}%
\pgfpathlineto{\pgfqpoint{5.240163in}{2.859073in}}%
\pgfusepath{stroke}%
\end{pgfscope}%
\begin{pgfscope}%
\pgfsetbuttcap%
\pgfsetmiterjoin%
\definecolor{currentfill}{rgb}{1.000000,1.000000,1.000000}%
\pgfsetfillcolor{currentfill}%
\pgfsetfillopacity{0.800000}%
\pgfsetlinewidth{1.003750pt}%
\definecolor{currentstroke}{rgb}{0.800000,0.800000,0.800000}%
\pgfsetstrokecolor{currentstroke}%
\pgfsetstrokeopacity{0.800000}%
\pgfsetdash{}{0pt}%
\pgfpathmoveto{\pgfqpoint{3.278793in}{0.632406in}}%
\pgfpathlineto{\pgfqpoint{5.156830in}{0.632406in}}%
\pgfpathlineto{\pgfqpoint{5.156830in}{1.844442in}}%
\pgfpathlineto{\pgfqpoint{3.278793in}{1.844442in}}%
\pgfpathlineto{\pgfqpoint{3.278793in}{0.632406in}}%
\pgfpathclose%
\pgfusepath{stroke,fill}%
\end{pgfscope}%
\begin{pgfscope}%
\pgfsetrectcap%
\pgfsetroundjoin%
\pgfsetlinewidth{1.505625pt}%
\definecolor{currentstroke}{rgb}{0.996369,0.791167,0.553499}%
\pgfsetstrokecolor{currentstroke}%
\pgfsetdash{}{0pt}%
\pgfpathmoveto{\pgfqpoint{3.345459in}{1.719442in}}%
\pgfpathlineto{\pgfqpoint{3.512126in}{1.719442in}}%
\pgfpathlineto{\pgfqpoint{3.678793in}{1.719442in}}%
\pgfusepath{stroke}%
\end{pgfscope}%
\begin{pgfscope}%
\pgfsetbuttcap%
\pgfsetmiterjoin%
\definecolor{currentfill}{rgb}{0.996369,0.791167,0.553499}%
\pgfsetfillcolor{currentfill}%
\pgfsetlinewidth{1.003750pt}%
\definecolor{currentstroke}{rgb}{0.996369,0.791167,0.553499}%
\pgfsetstrokecolor{currentstroke}%
\pgfsetdash{}{0pt}%
\pgfsys@defobject{currentmarker}{\pgfqpoint{-0.035355in}{-0.058926in}}{\pgfqpoint{0.035355in}{0.058926in}}{%
\pgfpathmoveto{\pgfqpoint{-0.000000in}{-0.058926in}}%
\pgfpathlineto{\pgfqpoint{0.035355in}{0.000000in}}%
\pgfpathlineto{\pgfqpoint{0.000000in}{0.058926in}}%
\pgfpathlineto{\pgfqpoint{-0.035355in}{0.000000in}}%
\pgfpathlineto{\pgfqpoint{-0.000000in}{-0.058926in}}%
\pgfpathclose%
\pgfusepath{stroke,fill}%
}%
\begin{pgfscope}%
\pgfsys@transformshift{3.512126in}{1.719442in}%
\pgfsys@useobject{currentmarker}{}%
\end{pgfscope}%
\end{pgfscope}%
\begin{pgfscope}%
\definecolor{textcolor}{rgb}{0.000000,0.000000,0.000000}%
\pgfsetstrokecolor{textcolor}%
\pgfsetfillcolor{textcolor}%
\pgftext[x=3.812126in,y=1.661108in,left,base]{\color{textcolor}{\rmfamily\fontsize{12.000000}{14.400000}\selectfont\catcode`\^=\active\def^{\ifmmode\sp\else\^{}\fi}\catcode`\%=\active\def%{\%}$n_{\mathbf{\Psi}} = 80$, $n_{\mathbf{\Omega}} = 0$}}%
\end{pgfscope}%
\begin{pgfscope}%
\pgfsetrectcap%
\pgfsetroundjoin%
\pgfsetlinewidth{1.505625pt}%
\definecolor{currentstroke}{rgb}{0.944006,0.377643,0.365136}%
\pgfsetstrokecolor{currentstroke}%
\pgfsetdash{}{0pt}%
\pgfpathmoveto{\pgfqpoint{3.345459in}{1.487034in}}%
\pgfpathlineto{\pgfqpoint{3.512126in}{1.487034in}}%
\pgfpathlineto{\pgfqpoint{3.678793in}{1.487034in}}%
\pgfusepath{stroke}%
\end{pgfscope}%
\begin{pgfscope}%
\pgfsetbuttcap%
\pgfsetmiterjoin%
\definecolor{currentfill}{rgb}{0.944006,0.377643,0.365136}%
\pgfsetfillcolor{currentfill}%
\pgfsetlinewidth{1.003750pt}%
\definecolor{currentstroke}{rgb}{0.944006,0.377643,0.365136}%
\pgfsetstrokecolor{currentstroke}%
\pgfsetdash{}{0pt}%
\pgfsys@defobject{currentmarker}{\pgfqpoint{-0.039627in}{-0.033709in}}{\pgfqpoint{0.039627in}{0.041667in}}{%
\pgfpathmoveto{\pgfqpoint{0.000000in}{0.041667in}}%
\pgfpathlineto{\pgfqpoint{-0.039627in}{0.012876in}}%
\pgfpathlineto{\pgfqpoint{-0.024491in}{-0.033709in}}%
\pgfpathlineto{\pgfqpoint{0.024491in}{-0.033709in}}%
\pgfpathlineto{\pgfqpoint{0.039627in}{0.012876in}}%
\pgfpathlineto{\pgfqpoint{0.000000in}{0.041667in}}%
\pgfpathclose%
\pgfusepath{stroke,fill}%
}%
\begin{pgfscope}%
\pgfsys@transformshift{3.512126in}{1.487034in}%
\pgfsys@useobject{currentmarker}{}%
\end{pgfscope}%
\end{pgfscope}%
\begin{pgfscope}%
\definecolor{textcolor}{rgb}{0.000000,0.000000,0.000000}%
\pgfsetstrokecolor{textcolor}%
\pgfsetfillcolor{textcolor}%
\pgftext[x=3.812126in,y=1.428701in,left,base]{\color{textcolor}{\rmfamily\fontsize{12.000000}{14.400000}\selectfont\catcode`\^=\active\def^{\ifmmode\sp\else\^{}\fi}\catcode`\%=\active\def%{\%}$n_{\mathbf{\Psi}} = 60$, $n_{\mathbf{\Omega}} = 20$}}%
\end{pgfscope}%
\begin{pgfscope}%
\pgfsetrectcap%
\pgfsetroundjoin%
\pgfsetlinewidth{1.505625pt}%
\definecolor{currentstroke}{rgb}{0.620005,0.183840,0.497524}%
\pgfsetstrokecolor{currentstroke}%
\pgfsetdash{}{0pt}%
\pgfpathmoveto{\pgfqpoint{3.345459in}{1.254627in}}%
\pgfpathlineto{\pgfqpoint{3.512126in}{1.254627in}}%
\pgfpathlineto{\pgfqpoint{3.678793in}{1.254627in}}%
\pgfusepath{stroke}%
\end{pgfscope}%
\begin{pgfscope}%
\pgfsetbuttcap%
\pgfsetmiterjoin%
\definecolor{currentfill}{rgb}{0.620005,0.183840,0.497524}%
\pgfsetfillcolor{currentfill}%
\pgfsetlinewidth{1.003750pt}%
\definecolor{currentstroke}{rgb}{0.620005,0.183840,0.497524}%
\pgfsetstrokecolor{currentstroke}%
\pgfsetdash{}{0pt}%
\pgfsys@defobject{currentmarker}{\pgfqpoint{-0.041667in}{-0.041667in}}{\pgfqpoint{0.041667in}{0.041667in}}{%
\pgfpathmoveto{\pgfqpoint{-0.041667in}{-0.041667in}}%
\pgfpathlineto{\pgfqpoint{0.041667in}{-0.041667in}}%
\pgfpathlineto{\pgfqpoint{0.041667in}{0.041667in}}%
\pgfpathlineto{\pgfqpoint{-0.041667in}{0.041667in}}%
\pgfpathlineto{\pgfqpoint{-0.041667in}{-0.041667in}}%
\pgfpathclose%
\pgfusepath{stroke,fill}%
}%
\begin{pgfscope}%
\pgfsys@transformshift{3.512126in}{1.254627in}%
\pgfsys@useobject{currentmarker}{}%
\end{pgfscope}%
\end{pgfscope}%
\begin{pgfscope}%
\definecolor{textcolor}{rgb}{0.000000,0.000000,0.000000}%
\pgfsetstrokecolor{textcolor}%
\pgfsetfillcolor{textcolor}%
\pgftext[x=3.812126in,y=1.196294in,left,base]{\color{textcolor}{\rmfamily\fontsize{12.000000}{14.400000}\selectfont\catcode`\^=\active\def^{\ifmmode\sp\else\^{}\fi}\catcode`\%=\active\def%{\%}$n_{\mathbf{\Psi}} = 40$, $n_{\mathbf{\Omega}} = 40$}}%
\end{pgfscope}%
\begin{pgfscope}%
\pgfsetrectcap%
\pgfsetroundjoin%
\pgfsetlinewidth{1.505625pt}%
\definecolor{currentstroke}{rgb}{0.265447,0.060237,0.461840}%
\pgfsetstrokecolor{currentstroke}%
\pgfsetdash{}{0pt}%
\pgfpathmoveto{\pgfqpoint{3.345459in}{1.022220in}}%
\pgfpathlineto{\pgfqpoint{3.512126in}{1.022220in}}%
\pgfpathlineto{\pgfqpoint{3.678793in}{1.022220in}}%
\pgfusepath{stroke}%
\end{pgfscope}%
\begin{pgfscope}%
\pgfsetbuttcap%
\pgfsetmiterjoin%
\definecolor{currentfill}{rgb}{0.265447,0.060237,0.461840}%
\pgfsetfillcolor{currentfill}%
\pgfsetlinewidth{1.003750pt}%
\definecolor{currentstroke}{rgb}{0.265447,0.060237,0.461840}%
\pgfsetstrokecolor{currentstroke}%
\pgfsetdash{}{0pt}%
\pgfsys@defobject{currentmarker}{\pgfqpoint{-0.041667in}{-0.041667in}}{\pgfqpoint{0.041667in}{0.041667in}}{%
\pgfpathmoveto{\pgfqpoint{0.000000in}{0.041667in}}%
\pgfpathlineto{\pgfqpoint{-0.041667in}{-0.041667in}}%
\pgfpathlineto{\pgfqpoint{0.041667in}{-0.041667in}}%
\pgfpathlineto{\pgfqpoint{0.000000in}{0.041667in}}%
\pgfpathclose%
\pgfusepath{stroke,fill}%
}%
\begin{pgfscope}%
\pgfsys@transformshift{3.512126in}{1.022220in}%
\pgfsys@useobject{currentmarker}{}%
\end{pgfscope}%
\end{pgfscope}%
\begin{pgfscope}%
\definecolor{textcolor}{rgb}{0.000000,0.000000,0.000000}%
\pgfsetstrokecolor{textcolor}%
\pgfsetfillcolor{textcolor}%
\pgftext[x=3.812126in,y=0.963887in,left,base]{\color{textcolor}{\rmfamily\fontsize{12.000000}{14.400000}\selectfont\catcode`\^=\active\def^{\ifmmode\sp\else\^{}\fi}\catcode`\%=\active\def%{\%}$n_{\mathbf{\Psi}} = 20$, $n_{\mathbf{\Omega}} = 60$}}%
\end{pgfscope}%
\begin{pgfscope}%
\pgfsetrectcap%
\pgfsetroundjoin%
\pgfsetlinewidth{1.505625pt}%
\definecolor{currentstroke}{rgb}{0.001462,0.000466,0.013866}%
\pgfsetstrokecolor{currentstroke}%
\pgfsetdash{}{0pt}%
\pgfpathmoveto{\pgfqpoint{3.345459in}{0.789813in}}%
\pgfpathlineto{\pgfqpoint{3.512126in}{0.789813in}}%
\pgfpathlineto{\pgfqpoint{3.678793in}{0.789813in}}%
\pgfusepath{stroke}%
\end{pgfscope}%
\begin{pgfscope}%
\pgfsetbuttcap%
\pgfsetroundjoin%
\definecolor{currentfill}{rgb}{0.001462,0.000466,0.013866}%
\pgfsetfillcolor{currentfill}%
\pgfsetlinewidth{1.003750pt}%
\definecolor{currentstroke}{rgb}{0.001462,0.000466,0.013866}%
\pgfsetstrokecolor{currentstroke}%
\pgfsetdash{}{0pt}%
\pgfsys@defobject{currentmarker}{\pgfqpoint{-0.041667in}{-0.041667in}}{\pgfqpoint{0.041667in}{0.041667in}}{%
\pgfpathmoveto{\pgfqpoint{0.000000in}{-0.041667in}}%
\pgfpathcurveto{\pgfqpoint{0.011050in}{-0.041667in}}{\pgfqpoint{0.021649in}{-0.037276in}}{\pgfqpoint{0.029463in}{-0.029463in}}%
\pgfpathcurveto{\pgfqpoint{0.037276in}{-0.021649in}}{\pgfqpoint{0.041667in}{-0.011050in}}{\pgfqpoint{0.041667in}{0.000000in}}%
\pgfpathcurveto{\pgfqpoint{0.041667in}{0.011050in}}{\pgfqpoint{0.037276in}{0.021649in}}{\pgfqpoint{0.029463in}{0.029463in}}%
\pgfpathcurveto{\pgfqpoint{0.021649in}{0.037276in}}{\pgfqpoint{0.011050in}{0.041667in}}{\pgfqpoint{0.000000in}{0.041667in}}%
\pgfpathcurveto{\pgfqpoint{-0.011050in}{0.041667in}}{\pgfqpoint{-0.021649in}{0.037276in}}{\pgfqpoint{-0.029463in}{0.029463in}}%
\pgfpathcurveto{\pgfqpoint{-0.037276in}{0.021649in}}{\pgfqpoint{-0.041667in}{0.011050in}}{\pgfqpoint{-0.041667in}{0.000000in}}%
\pgfpathcurveto{\pgfqpoint{-0.041667in}{-0.011050in}}{\pgfqpoint{-0.037276in}{-0.021649in}}{\pgfqpoint{-0.029463in}{-0.029463in}}%
\pgfpathcurveto{\pgfqpoint{-0.021649in}{-0.037276in}}{\pgfqpoint{-0.011050in}{-0.041667in}}{\pgfqpoint{0.000000in}{-0.041667in}}%
\pgfpathlineto{\pgfqpoint{0.000000in}{-0.041667in}}%
\pgfpathclose%
\pgfusepath{stroke,fill}%
}%
\begin{pgfscope}%
\pgfsys@transformshift{3.512126in}{0.789813in}%
\pgfsys@useobject{currentmarker}{}%
\end{pgfscope}%
\end{pgfscope}%
\begin{pgfscope}%
\definecolor{textcolor}{rgb}{0.000000,0.000000,0.000000}%
\pgfsetstrokecolor{textcolor}%
\pgfsetfillcolor{textcolor}%
\pgftext[x=3.812126in,y=0.731480in,left,base]{\color{textcolor}{\rmfamily\fontsize{12.000000}{14.400000}\selectfont\catcode`\^=\active\def^{\ifmmode\sp\else\^{}\fi}\catcode`\%=\active\def%{\%}$n_{\mathbf{\Psi}} = 0$, $n_{\mathbf{\Omega}} = 80$}}%
\end{pgfscope}%
\end{pgfpicture}%
\makeatother%
\endgroup%
}
    \caption{Error vs. $\sigma$ for a fixed budget of $80$ random vectors 
    when applying  Chebyshev-Nyström++ to the Hamiltonian matrix from~\cref{subsec:hamiltonian} with $n_c = 1$.
The Chebyshev approximation error is made negligible by setting $m=16 / \sigma$ (see \cref{lem:non-negative-chebyshev-error}).}
    \label{fig:distribution}
\end{figure}

\subsection{Comparison to Krylov-aware trace estimator}
\label{subsec:krylov-aware}

We notice that the Krylov-aware stochastic trace estimator from \cite[Algorithm 3.1]{chen-2023-krylovaware-stochastic} can also effectively be used in the setting of spectral density approximation. It also samples two Gaussian random matrices $\mtx{\Omega} \in \mathbb{R}^{n \times n_{\mtx{\Omega}}}$ and $\mtx{\Psi} \in \mathbb{R}^{n \times n_{\mtx{\Psi}}}$. It first runs block Lanczos algorithm on $\mtx{A}$ with starting block $\mtx{\Omega}$ for $r$ iterations with reorthogonalization and subsequently $q$ without. The columns of $\mtx{\Psi}$ are then projected onto the complement of the block Krylov subspace generated by the Lanczos algorithm, and are afterwards used as starting vectors for $r$ iterations of the Lanczos algorithm on $\mtx{A}$. What is remarkable is that all these heavy computations are completely independent of the smoothing kernel $g_{\sigma}$ and the parameter $t$. Further, some algebraic manipulations show that similarly to the stochastic Lanczos quadrature \cite[Section 3]{ubaru-2017-fast-estimation}, we can compress the estimator into a quadrature. Since this quadrature is independent of $g_{\sigma}$ and $t$, it can be applied to different smoothing kernels $g_{\sigma}$ and to as many values of $t$ as needed --- at little additional cost.

We run our own, faithfully optimized implementation of the Krylov-aware stochastic trace estimator \cite[Algorithm 3.1]{chen-2023-krylovaware-stochastic} for different parameter settings on the example from \cref{subsec:hamiltonian} and plot the approximation errors for logarithmically spaced values of the smoothing parameter $\sigma$ in \cref{fig:krylov-aware-density}. For reference, we also report the error of Chebyshev-Nyström++ for parameters that lead to a comparable run-time. The non-monotonic behavior of the Chebyshev-Nyström++ method in this case arises because, for small values of $\sigma$, the Chebyshev interpolation requires a very high polynomial degree to be accurate, whereas for large $\sigma$, the low-rank approximation becomes less effective. Due to the fixed parameters, the sweet spot lies around $\sigma \approx 0.002$: the low-rank approximation benefits from some underlying low-rank structure in $g_{\sigma}(t \mtx{I} - \mtx{A})$, while the Chebyshev interpolation remains good because $g_{\sigma}$ is not too sharply peaked. While the Krylov-aware estimator is clearly an attractive alternative, we argue that the parameters of the Chebyshev-Nyström++ method can be chosen in a more interpretable way, due to the explicit requirement for $m$ based on $\sigma$ (\cref{fig:chebyshev-heatmap}) and for $n_{\mtx{\Omega}}$ based on the distribution of the eigenvalues (discussed towards the end of \cref{subsubsec:chebyshev-nystrom-analysis}).

\begin{figure}[ht]
    \begin{minipage}[c]{.515\linewidth}
        \centering
        \scalebox{0.8}{%% Creator: Matplotlib, PGF backend
%%
%% To include the figure in your LaTeX document, write
%%   \input{<filename>.pgf}
%%
%% Make sure the required packages are loaded in your preamble
%%   \usepackage{pgf}
%%
%% Also ensure that all the required font packages are loaded; for instance,
%% the lmodern package is sometimes necessary when using math font.
%%   \usepackage{lmodern}
%%
%% Figures using additional raster images can only be included by \input if
%% they are in the same directory as the main LaTeX file. For loading figures
%% from other directories you can use the `import` package
%%   \usepackage{import}
%%
%% and then include the figures with
%%   \import{<path to file>}{<filename>.pgf}
%%
%% Matplotlib used the following preamble
%%   \def\mathdefault#1{#1}
%%   \everymath=\expandafter{\the\everymath\displaystyle}
%%   
%%   \ifdefined\pdftexversion\else  % non-pdftex case.
%%     \usepackage{fontspec}
%%     \setmainfont{DejaVuSerif.ttf}[Path=\detokenize{/home/matti/Documents/.venv/lib/python3.12/site-packages/matplotlib/mpl-data/fonts/ttf/}]
%%     \setsansfont{DejaVuSans.ttf}[Path=\detokenize{/home/matti/Documents/.venv/lib/python3.12/site-packages/matplotlib/mpl-data/fonts/ttf/}]
%%     \setmonofont{DejaVuSansMono.ttf}[Path=\detokenize{/home/matti/Documents/.venv/lib/python3.12/site-packages/matplotlib/mpl-data/fonts/ttf/}]
%%   \fi
%%   \makeatletter\@ifpackageloaded{underscore}{}{\usepackage[strings]{underscore}}\makeatother
%%
\begingroup%
\makeatletter%
\begin{pgfpicture}%
\pgfpathrectangle{\pgfpointorigin}{\pgfqpoint{3.252050in}{2.959073in}}%
\pgfusepath{use as bounding box, clip}%
\begin{pgfscope}%
\pgfsetbuttcap%
\pgfsetmiterjoin%
\definecolor{currentfill}{rgb}{1.000000,1.000000,1.000000}%
\pgfsetfillcolor{currentfill}%
\pgfsetlinewidth{0.000000pt}%
\definecolor{currentstroke}{rgb}{1.000000,1.000000,1.000000}%
\pgfsetstrokecolor{currentstroke}%
\pgfsetdash{}{0pt}%
\pgfpathmoveto{\pgfqpoint{0.000000in}{-0.000000in}}%
\pgfpathlineto{\pgfqpoint{3.252050in}{-0.000000in}}%
\pgfpathlineto{\pgfqpoint{3.252050in}{2.959073in}}%
\pgfpathlineto{\pgfqpoint{0.000000in}{2.959073in}}%
\pgfpathlineto{\pgfqpoint{0.000000in}{-0.000000in}}%
\pgfpathclose%
\pgfusepath{fill}%
\end{pgfscope}%
\begin{pgfscope}%
\pgfsetbuttcap%
\pgfsetmiterjoin%
\definecolor{currentfill}{rgb}{1.000000,1.000000,1.000000}%
\pgfsetfillcolor{currentfill}%
\pgfsetlinewidth{0.000000pt}%
\definecolor{currentstroke}{rgb}{0.000000,0.000000,0.000000}%
\pgfsetstrokecolor{currentstroke}%
\pgfsetstrokeopacity{0.000000}%
\pgfsetdash{}{0pt}%
\pgfpathmoveto{\pgfqpoint{0.721913in}{0.549073in}}%
\pgfpathlineto{\pgfqpoint{3.046913in}{0.549073in}}%
\pgfpathlineto{\pgfqpoint{3.046913in}{2.859073in}}%
\pgfpathlineto{\pgfqpoint{0.721913in}{2.859073in}}%
\pgfpathlineto{\pgfqpoint{0.721913in}{0.549073in}}%
\pgfpathclose%
\pgfusepath{fill}%
\end{pgfscope}%
\begin{pgfscope}%
\pgfpathrectangle{\pgfqpoint{0.721913in}{0.549073in}}{\pgfqpoint{2.325000in}{2.310000in}}%
\pgfusepath{clip}%
\pgfsetrectcap%
\pgfsetroundjoin%
\pgfsetlinewidth{0.250937pt}%
\definecolor{currentstroke}{rgb}{0.000000,0.000000,0.000000}%
\pgfsetstrokecolor{currentstroke}%
\pgfsetstrokeopacity{0.200000}%
\pgfsetdash{}{0pt}%
\pgfpathmoveto{\pgfqpoint{1.220128in}{0.549073in}}%
\pgfpathlineto{\pgfqpoint{1.220128in}{2.859073in}}%
\pgfusepath{stroke}%
\end{pgfscope}%
\begin{pgfscope}%
\pgfsetbuttcap%
\pgfsetroundjoin%
\definecolor{currentfill}{rgb}{0.000000,0.000000,0.000000}%
\pgfsetfillcolor{currentfill}%
\pgfsetlinewidth{0.803000pt}%
\definecolor{currentstroke}{rgb}{0.000000,0.000000,0.000000}%
\pgfsetstrokecolor{currentstroke}%
\pgfsetdash{}{0pt}%
\pgfsys@defobject{currentmarker}{\pgfqpoint{0.000000in}{-0.048611in}}{\pgfqpoint{0.000000in}{0.000000in}}{%
\pgfpathmoveto{\pgfqpoint{0.000000in}{0.000000in}}%
\pgfpathlineto{\pgfqpoint{0.000000in}{-0.048611in}}%
\pgfusepath{stroke,fill}%
}%
\begin{pgfscope}%
\pgfsys@transformshift{1.220128in}{0.549073in}%
\pgfsys@useobject{currentmarker}{}%
\end{pgfscope}%
\end{pgfscope}%
\begin{pgfscope}%
\definecolor{textcolor}{rgb}{0.000000,0.000000,0.000000}%
\pgfsetstrokecolor{textcolor}%
\pgfsetfillcolor{textcolor}%
\pgftext[x=1.220128in,y=0.451851in,,top]{\color{textcolor}{\rmfamily\fontsize{12.000000}{14.400000}\selectfont\catcode`\^=\active\def^{\ifmmode\sp\else\^{}\fi}\catcode`\%=\active\def%{\%}$\mathdefault{10^{-3}}$}}%
\end{pgfscope}%
\begin{pgfscope}%
\pgfpathrectangle{\pgfqpoint{0.721913in}{0.549073in}}{\pgfqpoint{2.325000in}{2.310000in}}%
\pgfusepath{clip}%
\pgfsetrectcap%
\pgfsetroundjoin%
\pgfsetlinewidth{0.250937pt}%
\definecolor{currentstroke}{rgb}{0.000000,0.000000,0.000000}%
\pgfsetstrokecolor{currentstroke}%
\pgfsetstrokeopacity{0.200000}%
\pgfsetdash{}{0pt}%
\pgfpathmoveto{\pgfqpoint{2.105842in}{0.549073in}}%
\pgfpathlineto{\pgfqpoint{2.105842in}{2.859073in}}%
\pgfusepath{stroke}%
\end{pgfscope}%
\begin{pgfscope}%
\pgfsetbuttcap%
\pgfsetroundjoin%
\definecolor{currentfill}{rgb}{0.000000,0.000000,0.000000}%
\pgfsetfillcolor{currentfill}%
\pgfsetlinewidth{0.803000pt}%
\definecolor{currentstroke}{rgb}{0.000000,0.000000,0.000000}%
\pgfsetstrokecolor{currentstroke}%
\pgfsetdash{}{0pt}%
\pgfsys@defobject{currentmarker}{\pgfqpoint{0.000000in}{-0.048611in}}{\pgfqpoint{0.000000in}{0.000000in}}{%
\pgfpathmoveto{\pgfqpoint{0.000000in}{0.000000in}}%
\pgfpathlineto{\pgfqpoint{0.000000in}{-0.048611in}}%
\pgfusepath{stroke,fill}%
}%
\begin{pgfscope}%
\pgfsys@transformshift{2.105842in}{0.549073in}%
\pgfsys@useobject{currentmarker}{}%
\end{pgfscope}%
\end{pgfscope}%
\begin{pgfscope}%
\definecolor{textcolor}{rgb}{0.000000,0.000000,0.000000}%
\pgfsetstrokecolor{textcolor}%
\pgfsetfillcolor{textcolor}%
\pgftext[x=2.105842in,y=0.451851in,,top]{\color{textcolor}{\rmfamily\fontsize{12.000000}{14.400000}\selectfont\catcode`\^=\active\def^{\ifmmode\sp\else\^{}\fi}\catcode`\%=\active\def%{\%}$\mathdefault{10^{-2}}$}}%
\end{pgfscope}%
\begin{pgfscope}%
\pgfpathrectangle{\pgfqpoint{0.721913in}{0.549073in}}{\pgfqpoint{2.325000in}{2.310000in}}%
\pgfusepath{clip}%
\pgfsetrectcap%
\pgfsetroundjoin%
\pgfsetlinewidth{0.250937pt}%
\definecolor{currentstroke}{rgb}{0.000000,0.000000,0.000000}%
\pgfsetstrokecolor{currentstroke}%
\pgfsetstrokeopacity{0.200000}%
\pgfsetdash{}{0pt}%
\pgfpathmoveto{\pgfqpoint{2.991556in}{0.549073in}}%
\pgfpathlineto{\pgfqpoint{2.991556in}{2.859073in}}%
\pgfusepath{stroke}%
\end{pgfscope}%
\begin{pgfscope}%
\pgfsetbuttcap%
\pgfsetroundjoin%
\definecolor{currentfill}{rgb}{0.000000,0.000000,0.000000}%
\pgfsetfillcolor{currentfill}%
\pgfsetlinewidth{0.803000pt}%
\definecolor{currentstroke}{rgb}{0.000000,0.000000,0.000000}%
\pgfsetstrokecolor{currentstroke}%
\pgfsetdash{}{0pt}%
\pgfsys@defobject{currentmarker}{\pgfqpoint{0.000000in}{-0.048611in}}{\pgfqpoint{0.000000in}{0.000000in}}{%
\pgfpathmoveto{\pgfqpoint{0.000000in}{0.000000in}}%
\pgfpathlineto{\pgfqpoint{0.000000in}{-0.048611in}}%
\pgfusepath{stroke,fill}%
}%
\begin{pgfscope}%
\pgfsys@transformshift{2.991556in}{0.549073in}%
\pgfsys@useobject{currentmarker}{}%
\end{pgfscope}%
\end{pgfscope}%
\begin{pgfscope}%
\definecolor{textcolor}{rgb}{0.000000,0.000000,0.000000}%
\pgfsetstrokecolor{textcolor}%
\pgfsetfillcolor{textcolor}%
\pgftext[x=2.991556in,y=0.451851in,,top]{\color{textcolor}{\rmfamily\fontsize{12.000000}{14.400000}\selectfont\catcode`\^=\active\def^{\ifmmode\sp\else\^{}\fi}\catcode`\%=\active\def%{\%}$\mathdefault{10^{-1}}$}}%
\end{pgfscope}%
\begin{pgfscope}%
\pgfpathrectangle{\pgfqpoint{0.721913in}{0.549073in}}{\pgfqpoint{2.325000in}{2.310000in}}%
\pgfusepath{clip}%
\pgfsetrectcap%
\pgfsetroundjoin%
\pgfsetlinewidth{0.250937pt}%
\definecolor{currentstroke}{rgb}{0.000000,0.000000,0.000000}%
\pgfsetstrokecolor{currentstroke}%
\pgfsetstrokeopacity{0.200000}%
\pgfsetdash{}{0pt}%
\pgfpathmoveto{\pgfqpoint{0.757007in}{0.549073in}}%
\pgfpathlineto{\pgfqpoint{0.757007in}{2.859073in}}%
\pgfusepath{stroke}%
\end{pgfscope}%
\begin{pgfscope}%
\pgfsetbuttcap%
\pgfsetroundjoin%
\definecolor{currentfill}{rgb}{0.000000,0.000000,0.000000}%
\pgfsetfillcolor{currentfill}%
\pgfsetlinewidth{0.602250pt}%
\definecolor{currentstroke}{rgb}{0.000000,0.000000,0.000000}%
\pgfsetstrokecolor{currentstroke}%
\pgfsetdash{}{0pt}%
\pgfsys@defobject{currentmarker}{\pgfqpoint{0.000000in}{-0.027778in}}{\pgfqpoint{0.000000in}{0.000000in}}{%
\pgfpathmoveto{\pgfqpoint{0.000000in}{0.000000in}}%
\pgfpathlineto{\pgfqpoint{0.000000in}{-0.027778in}}%
\pgfusepath{stroke,fill}%
}%
\begin{pgfscope}%
\pgfsys@transformshift{0.757007in}{0.549073in}%
\pgfsys@useobject{currentmarker}{}%
\end{pgfscope}%
\end{pgfscope}%
\begin{pgfscope}%
\pgfpathrectangle{\pgfqpoint{0.721913in}{0.549073in}}{\pgfqpoint{2.325000in}{2.310000in}}%
\pgfusepath{clip}%
\pgfsetrectcap%
\pgfsetroundjoin%
\pgfsetlinewidth{0.250937pt}%
\definecolor{currentstroke}{rgb}{0.000000,0.000000,0.000000}%
\pgfsetstrokecolor{currentstroke}%
\pgfsetstrokeopacity{0.200000}%
\pgfsetdash{}{0pt}%
\pgfpathmoveto{\pgfqpoint{0.867667in}{0.549073in}}%
\pgfpathlineto{\pgfqpoint{0.867667in}{2.859073in}}%
\pgfusepath{stroke}%
\end{pgfscope}%
\begin{pgfscope}%
\pgfsetbuttcap%
\pgfsetroundjoin%
\definecolor{currentfill}{rgb}{0.000000,0.000000,0.000000}%
\pgfsetfillcolor{currentfill}%
\pgfsetlinewidth{0.602250pt}%
\definecolor{currentstroke}{rgb}{0.000000,0.000000,0.000000}%
\pgfsetstrokecolor{currentstroke}%
\pgfsetdash{}{0pt}%
\pgfsys@defobject{currentmarker}{\pgfqpoint{0.000000in}{-0.027778in}}{\pgfqpoint{0.000000in}{0.000000in}}{%
\pgfpathmoveto{\pgfqpoint{0.000000in}{0.000000in}}%
\pgfpathlineto{\pgfqpoint{0.000000in}{-0.027778in}}%
\pgfusepath{stroke,fill}%
}%
\begin{pgfscope}%
\pgfsys@transformshift{0.867667in}{0.549073in}%
\pgfsys@useobject{currentmarker}{}%
\end{pgfscope}%
\end{pgfscope}%
\begin{pgfscope}%
\pgfpathrectangle{\pgfqpoint{0.721913in}{0.549073in}}{\pgfqpoint{2.325000in}{2.310000in}}%
\pgfusepath{clip}%
\pgfsetrectcap%
\pgfsetroundjoin%
\pgfsetlinewidth{0.250937pt}%
\definecolor{currentstroke}{rgb}{0.000000,0.000000,0.000000}%
\pgfsetstrokecolor{currentstroke}%
\pgfsetstrokeopacity{0.200000}%
\pgfsetdash{}{0pt}%
\pgfpathmoveto{\pgfqpoint{0.953501in}{0.549073in}}%
\pgfpathlineto{\pgfqpoint{0.953501in}{2.859073in}}%
\pgfusepath{stroke}%
\end{pgfscope}%
\begin{pgfscope}%
\pgfsetbuttcap%
\pgfsetroundjoin%
\definecolor{currentfill}{rgb}{0.000000,0.000000,0.000000}%
\pgfsetfillcolor{currentfill}%
\pgfsetlinewidth{0.602250pt}%
\definecolor{currentstroke}{rgb}{0.000000,0.000000,0.000000}%
\pgfsetstrokecolor{currentstroke}%
\pgfsetdash{}{0pt}%
\pgfsys@defobject{currentmarker}{\pgfqpoint{0.000000in}{-0.027778in}}{\pgfqpoint{0.000000in}{0.000000in}}{%
\pgfpathmoveto{\pgfqpoint{0.000000in}{0.000000in}}%
\pgfpathlineto{\pgfqpoint{0.000000in}{-0.027778in}}%
\pgfusepath{stroke,fill}%
}%
\begin{pgfscope}%
\pgfsys@transformshift{0.953501in}{0.549073in}%
\pgfsys@useobject{currentmarker}{}%
\end{pgfscope}%
\end{pgfscope}%
\begin{pgfscope}%
\pgfpathrectangle{\pgfqpoint{0.721913in}{0.549073in}}{\pgfqpoint{2.325000in}{2.310000in}}%
\pgfusepath{clip}%
\pgfsetrectcap%
\pgfsetroundjoin%
\pgfsetlinewidth{0.250937pt}%
\definecolor{currentstroke}{rgb}{0.000000,0.000000,0.000000}%
\pgfsetstrokecolor{currentstroke}%
\pgfsetstrokeopacity{0.200000}%
\pgfsetdash{}{0pt}%
\pgfpathmoveto{\pgfqpoint{1.023633in}{0.549073in}}%
\pgfpathlineto{\pgfqpoint{1.023633in}{2.859073in}}%
\pgfusepath{stroke}%
\end{pgfscope}%
\begin{pgfscope}%
\pgfsetbuttcap%
\pgfsetroundjoin%
\definecolor{currentfill}{rgb}{0.000000,0.000000,0.000000}%
\pgfsetfillcolor{currentfill}%
\pgfsetlinewidth{0.602250pt}%
\definecolor{currentstroke}{rgb}{0.000000,0.000000,0.000000}%
\pgfsetstrokecolor{currentstroke}%
\pgfsetdash{}{0pt}%
\pgfsys@defobject{currentmarker}{\pgfqpoint{0.000000in}{-0.027778in}}{\pgfqpoint{0.000000in}{0.000000in}}{%
\pgfpathmoveto{\pgfqpoint{0.000000in}{0.000000in}}%
\pgfpathlineto{\pgfqpoint{0.000000in}{-0.027778in}}%
\pgfusepath{stroke,fill}%
}%
\begin{pgfscope}%
\pgfsys@transformshift{1.023633in}{0.549073in}%
\pgfsys@useobject{currentmarker}{}%
\end{pgfscope}%
\end{pgfscope}%
\begin{pgfscope}%
\pgfpathrectangle{\pgfqpoint{0.721913in}{0.549073in}}{\pgfqpoint{2.325000in}{2.310000in}}%
\pgfusepath{clip}%
\pgfsetrectcap%
\pgfsetroundjoin%
\pgfsetlinewidth{0.250937pt}%
\definecolor{currentstroke}{rgb}{0.000000,0.000000,0.000000}%
\pgfsetstrokecolor{currentstroke}%
\pgfsetstrokeopacity{0.200000}%
\pgfsetdash{}{0pt}%
\pgfpathmoveto{\pgfqpoint{1.082929in}{0.549073in}}%
\pgfpathlineto{\pgfqpoint{1.082929in}{2.859073in}}%
\pgfusepath{stroke}%
\end{pgfscope}%
\begin{pgfscope}%
\pgfsetbuttcap%
\pgfsetroundjoin%
\definecolor{currentfill}{rgb}{0.000000,0.000000,0.000000}%
\pgfsetfillcolor{currentfill}%
\pgfsetlinewidth{0.602250pt}%
\definecolor{currentstroke}{rgb}{0.000000,0.000000,0.000000}%
\pgfsetstrokecolor{currentstroke}%
\pgfsetdash{}{0pt}%
\pgfsys@defobject{currentmarker}{\pgfqpoint{0.000000in}{-0.027778in}}{\pgfqpoint{0.000000in}{0.000000in}}{%
\pgfpathmoveto{\pgfqpoint{0.000000in}{0.000000in}}%
\pgfpathlineto{\pgfqpoint{0.000000in}{-0.027778in}}%
\pgfusepath{stroke,fill}%
}%
\begin{pgfscope}%
\pgfsys@transformshift{1.082929in}{0.549073in}%
\pgfsys@useobject{currentmarker}{}%
\end{pgfscope}%
\end{pgfscope}%
\begin{pgfscope}%
\pgfpathrectangle{\pgfqpoint{0.721913in}{0.549073in}}{\pgfqpoint{2.325000in}{2.310000in}}%
\pgfusepath{clip}%
\pgfsetrectcap%
\pgfsetroundjoin%
\pgfsetlinewidth{0.250937pt}%
\definecolor{currentstroke}{rgb}{0.000000,0.000000,0.000000}%
\pgfsetstrokecolor{currentstroke}%
\pgfsetstrokeopacity{0.200000}%
\pgfsetdash{}{0pt}%
\pgfpathmoveto{\pgfqpoint{1.134293in}{0.549073in}}%
\pgfpathlineto{\pgfqpoint{1.134293in}{2.859073in}}%
\pgfusepath{stroke}%
\end{pgfscope}%
\begin{pgfscope}%
\pgfsetbuttcap%
\pgfsetroundjoin%
\definecolor{currentfill}{rgb}{0.000000,0.000000,0.000000}%
\pgfsetfillcolor{currentfill}%
\pgfsetlinewidth{0.602250pt}%
\definecolor{currentstroke}{rgb}{0.000000,0.000000,0.000000}%
\pgfsetstrokecolor{currentstroke}%
\pgfsetdash{}{0pt}%
\pgfsys@defobject{currentmarker}{\pgfqpoint{0.000000in}{-0.027778in}}{\pgfqpoint{0.000000in}{0.000000in}}{%
\pgfpathmoveto{\pgfqpoint{0.000000in}{0.000000in}}%
\pgfpathlineto{\pgfqpoint{0.000000in}{-0.027778in}}%
\pgfusepath{stroke,fill}%
}%
\begin{pgfscope}%
\pgfsys@transformshift{1.134293in}{0.549073in}%
\pgfsys@useobject{currentmarker}{}%
\end{pgfscope}%
\end{pgfscope}%
\begin{pgfscope}%
\pgfpathrectangle{\pgfqpoint{0.721913in}{0.549073in}}{\pgfqpoint{2.325000in}{2.310000in}}%
\pgfusepath{clip}%
\pgfsetrectcap%
\pgfsetroundjoin%
\pgfsetlinewidth{0.250937pt}%
\definecolor{currentstroke}{rgb}{0.000000,0.000000,0.000000}%
\pgfsetstrokecolor{currentstroke}%
\pgfsetstrokeopacity{0.200000}%
\pgfsetdash{}{0pt}%
\pgfpathmoveto{\pgfqpoint{1.179600in}{0.549073in}}%
\pgfpathlineto{\pgfqpoint{1.179600in}{2.859073in}}%
\pgfusepath{stroke}%
\end{pgfscope}%
\begin{pgfscope}%
\pgfsetbuttcap%
\pgfsetroundjoin%
\definecolor{currentfill}{rgb}{0.000000,0.000000,0.000000}%
\pgfsetfillcolor{currentfill}%
\pgfsetlinewidth{0.602250pt}%
\definecolor{currentstroke}{rgb}{0.000000,0.000000,0.000000}%
\pgfsetstrokecolor{currentstroke}%
\pgfsetdash{}{0pt}%
\pgfsys@defobject{currentmarker}{\pgfqpoint{0.000000in}{-0.027778in}}{\pgfqpoint{0.000000in}{0.000000in}}{%
\pgfpathmoveto{\pgfqpoint{0.000000in}{0.000000in}}%
\pgfpathlineto{\pgfqpoint{0.000000in}{-0.027778in}}%
\pgfusepath{stroke,fill}%
}%
\begin{pgfscope}%
\pgfsys@transformshift{1.179600in}{0.549073in}%
\pgfsys@useobject{currentmarker}{}%
\end{pgfscope}%
\end{pgfscope}%
\begin{pgfscope}%
\pgfpathrectangle{\pgfqpoint{0.721913in}{0.549073in}}{\pgfqpoint{2.325000in}{2.310000in}}%
\pgfusepath{clip}%
\pgfsetrectcap%
\pgfsetroundjoin%
\pgfsetlinewidth{0.250937pt}%
\definecolor{currentstroke}{rgb}{0.000000,0.000000,0.000000}%
\pgfsetstrokecolor{currentstroke}%
\pgfsetstrokeopacity{0.200000}%
\pgfsetdash{}{0pt}%
\pgfpathmoveto{\pgfqpoint{1.486754in}{0.549073in}}%
\pgfpathlineto{\pgfqpoint{1.486754in}{2.859073in}}%
\pgfusepath{stroke}%
\end{pgfscope}%
\begin{pgfscope}%
\pgfsetbuttcap%
\pgfsetroundjoin%
\definecolor{currentfill}{rgb}{0.000000,0.000000,0.000000}%
\pgfsetfillcolor{currentfill}%
\pgfsetlinewidth{0.602250pt}%
\definecolor{currentstroke}{rgb}{0.000000,0.000000,0.000000}%
\pgfsetstrokecolor{currentstroke}%
\pgfsetdash{}{0pt}%
\pgfsys@defobject{currentmarker}{\pgfqpoint{0.000000in}{-0.027778in}}{\pgfqpoint{0.000000in}{0.000000in}}{%
\pgfpathmoveto{\pgfqpoint{0.000000in}{0.000000in}}%
\pgfpathlineto{\pgfqpoint{0.000000in}{-0.027778in}}%
\pgfusepath{stroke,fill}%
}%
\begin{pgfscope}%
\pgfsys@transformshift{1.486754in}{0.549073in}%
\pgfsys@useobject{currentmarker}{}%
\end{pgfscope}%
\end{pgfscope}%
\begin{pgfscope}%
\pgfpathrectangle{\pgfqpoint{0.721913in}{0.549073in}}{\pgfqpoint{2.325000in}{2.310000in}}%
\pgfusepath{clip}%
\pgfsetrectcap%
\pgfsetroundjoin%
\pgfsetlinewidth{0.250937pt}%
\definecolor{currentstroke}{rgb}{0.000000,0.000000,0.000000}%
\pgfsetstrokecolor{currentstroke}%
\pgfsetstrokeopacity{0.200000}%
\pgfsetdash{}{0pt}%
\pgfpathmoveto{\pgfqpoint{1.642721in}{0.549073in}}%
\pgfpathlineto{\pgfqpoint{1.642721in}{2.859073in}}%
\pgfusepath{stroke}%
\end{pgfscope}%
\begin{pgfscope}%
\pgfsetbuttcap%
\pgfsetroundjoin%
\definecolor{currentfill}{rgb}{0.000000,0.000000,0.000000}%
\pgfsetfillcolor{currentfill}%
\pgfsetlinewidth{0.602250pt}%
\definecolor{currentstroke}{rgb}{0.000000,0.000000,0.000000}%
\pgfsetstrokecolor{currentstroke}%
\pgfsetdash{}{0pt}%
\pgfsys@defobject{currentmarker}{\pgfqpoint{0.000000in}{-0.027778in}}{\pgfqpoint{0.000000in}{0.000000in}}{%
\pgfpathmoveto{\pgfqpoint{0.000000in}{0.000000in}}%
\pgfpathlineto{\pgfqpoint{0.000000in}{-0.027778in}}%
\pgfusepath{stroke,fill}%
}%
\begin{pgfscope}%
\pgfsys@transformshift{1.642721in}{0.549073in}%
\pgfsys@useobject{currentmarker}{}%
\end{pgfscope}%
\end{pgfscope}%
\begin{pgfscope}%
\pgfpathrectangle{\pgfqpoint{0.721913in}{0.549073in}}{\pgfqpoint{2.325000in}{2.310000in}}%
\pgfusepath{clip}%
\pgfsetrectcap%
\pgfsetroundjoin%
\pgfsetlinewidth{0.250937pt}%
\definecolor{currentstroke}{rgb}{0.000000,0.000000,0.000000}%
\pgfsetstrokecolor{currentstroke}%
\pgfsetstrokeopacity{0.200000}%
\pgfsetdash{}{0pt}%
\pgfpathmoveto{\pgfqpoint{1.753381in}{0.549073in}}%
\pgfpathlineto{\pgfqpoint{1.753381in}{2.859073in}}%
\pgfusepath{stroke}%
\end{pgfscope}%
\begin{pgfscope}%
\pgfsetbuttcap%
\pgfsetroundjoin%
\definecolor{currentfill}{rgb}{0.000000,0.000000,0.000000}%
\pgfsetfillcolor{currentfill}%
\pgfsetlinewidth{0.602250pt}%
\definecolor{currentstroke}{rgb}{0.000000,0.000000,0.000000}%
\pgfsetstrokecolor{currentstroke}%
\pgfsetdash{}{0pt}%
\pgfsys@defobject{currentmarker}{\pgfqpoint{0.000000in}{-0.027778in}}{\pgfqpoint{0.000000in}{0.000000in}}{%
\pgfpathmoveto{\pgfqpoint{0.000000in}{0.000000in}}%
\pgfpathlineto{\pgfqpoint{0.000000in}{-0.027778in}}%
\pgfusepath{stroke,fill}%
}%
\begin{pgfscope}%
\pgfsys@transformshift{1.753381in}{0.549073in}%
\pgfsys@useobject{currentmarker}{}%
\end{pgfscope}%
\end{pgfscope}%
\begin{pgfscope}%
\pgfpathrectangle{\pgfqpoint{0.721913in}{0.549073in}}{\pgfqpoint{2.325000in}{2.310000in}}%
\pgfusepath{clip}%
\pgfsetrectcap%
\pgfsetroundjoin%
\pgfsetlinewidth{0.250937pt}%
\definecolor{currentstroke}{rgb}{0.000000,0.000000,0.000000}%
\pgfsetstrokecolor{currentstroke}%
\pgfsetstrokeopacity{0.200000}%
\pgfsetdash{}{0pt}%
\pgfpathmoveto{\pgfqpoint{1.839215in}{0.549073in}}%
\pgfpathlineto{\pgfqpoint{1.839215in}{2.859073in}}%
\pgfusepath{stroke}%
\end{pgfscope}%
\begin{pgfscope}%
\pgfsetbuttcap%
\pgfsetroundjoin%
\definecolor{currentfill}{rgb}{0.000000,0.000000,0.000000}%
\pgfsetfillcolor{currentfill}%
\pgfsetlinewidth{0.602250pt}%
\definecolor{currentstroke}{rgb}{0.000000,0.000000,0.000000}%
\pgfsetstrokecolor{currentstroke}%
\pgfsetdash{}{0pt}%
\pgfsys@defobject{currentmarker}{\pgfqpoint{0.000000in}{-0.027778in}}{\pgfqpoint{0.000000in}{0.000000in}}{%
\pgfpathmoveto{\pgfqpoint{0.000000in}{0.000000in}}%
\pgfpathlineto{\pgfqpoint{0.000000in}{-0.027778in}}%
\pgfusepath{stroke,fill}%
}%
\begin{pgfscope}%
\pgfsys@transformshift{1.839215in}{0.549073in}%
\pgfsys@useobject{currentmarker}{}%
\end{pgfscope}%
\end{pgfscope}%
\begin{pgfscope}%
\pgfpathrectangle{\pgfqpoint{0.721913in}{0.549073in}}{\pgfqpoint{2.325000in}{2.310000in}}%
\pgfusepath{clip}%
\pgfsetrectcap%
\pgfsetroundjoin%
\pgfsetlinewidth{0.250937pt}%
\definecolor{currentstroke}{rgb}{0.000000,0.000000,0.000000}%
\pgfsetstrokecolor{currentstroke}%
\pgfsetstrokeopacity{0.200000}%
\pgfsetdash{}{0pt}%
\pgfpathmoveto{\pgfqpoint{1.909347in}{0.549073in}}%
\pgfpathlineto{\pgfqpoint{1.909347in}{2.859073in}}%
\pgfusepath{stroke}%
\end{pgfscope}%
\begin{pgfscope}%
\pgfsetbuttcap%
\pgfsetroundjoin%
\definecolor{currentfill}{rgb}{0.000000,0.000000,0.000000}%
\pgfsetfillcolor{currentfill}%
\pgfsetlinewidth{0.602250pt}%
\definecolor{currentstroke}{rgb}{0.000000,0.000000,0.000000}%
\pgfsetstrokecolor{currentstroke}%
\pgfsetdash{}{0pt}%
\pgfsys@defobject{currentmarker}{\pgfqpoint{0.000000in}{-0.027778in}}{\pgfqpoint{0.000000in}{0.000000in}}{%
\pgfpathmoveto{\pgfqpoint{0.000000in}{0.000000in}}%
\pgfpathlineto{\pgfqpoint{0.000000in}{-0.027778in}}%
\pgfusepath{stroke,fill}%
}%
\begin{pgfscope}%
\pgfsys@transformshift{1.909347in}{0.549073in}%
\pgfsys@useobject{currentmarker}{}%
\end{pgfscope}%
\end{pgfscope}%
\begin{pgfscope}%
\pgfpathrectangle{\pgfqpoint{0.721913in}{0.549073in}}{\pgfqpoint{2.325000in}{2.310000in}}%
\pgfusepath{clip}%
\pgfsetrectcap%
\pgfsetroundjoin%
\pgfsetlinewidth{0.250937pt}%
\definecolor{currentstroke}{rgb}{0.000000,0.000000,0.000000}%
\pgfsetstrokecolor{currentstroke}%
\pgfsetstrokeopacity{0.200000}%
\pgfsetdash{}{0pt}%
\pgfpathmoveto{\pgfqpoint{1.968643in}{0.549073in}}%
\pgfpathlineto{\pgfqpoint{1.968643in}{2.859073in}}%
\pgfusepath{stroke}%
\end{pgfscope}%
\begin{pgfscope}%
\pgfsetbuttcap%
\pgfsetroundjoin%
\definecolor{currentfill}{rgb}{0.000000,0.000000,0.000000}%
\pgfsetfillcolor{currentfill}%
\pgfsetlinewidth{0.602250pt}%
\definecolor{currentstroke}{rgb}{0.000000,0.000000,0.000000}%
\pgfsetstrokecolor{currentstroke}%
\pgfsetdash{}{0pt}%
\pgfsys@defobject{currentmarker}{\pgfqpoint{0.000000in}{-0.027778in}}{\pgfqpoint{0.000000in}{0.000000in}}{%
\pgfpathmoveto{\pgfqpoint{0.000000in}{0.000000in}}%
\pgfpathlineto{\pgfqpoint{0.000000in}{-0.027778in}}%
\pgfusepath{stroke,fill}%
}%
\begin{pgfscope}%
\pgfsys@transformshift{1.968643in}{0.549073in}%
\pgfsys@useobject{currentmarker}{}%
\end{pgfscope}%
\end{pgfscope}%
\begin{pgfscope}%
\pgfpathrectangle{\pgfqpoint{0.721913in}{0.549073in}}{\pgfqpoint{2.325000in}{2.310000in}}%
\pgfusepath{clip}%
\pgfsetrectcap%
\pgfsetroundjoin%
\pgfsetlinewidth{0.250937pt}%
\definecolor{currentstroke}{rgb}{0.000000,0.000000,0.000000}%
\pgfsetstrokecolor{currentstroke}%
\pgfsetstrokeopacity{0.200000}%
\pgfsetdash{}{0pt}%
\pgfpathmoveto{\pgfqpoint{2.020007in}{0.549073in}}%
\pgfpathlineto{\pgfqpoint{2.020007in}{2.859073in}}%
\pgfusepath{stroke}%
\end{pgfscope}%
\begin{pgfscope}%
\pgfsetbuttcap%
\pgfsetroundjoin%
\definecolor{currentfill}{rgb}{0.000000,0.000000,0.000000}%
\pgfsetfillcolor{currentfill}%
\pgfsetlinewidth{0.602250pt}%
\definecolor{currentstroke}{rgb}{0.000000,0.000000,0.000000}%
\pgfsetstrokecolor{currentstroke}%
\pgfsetdash{}{0pt}%
\pgfsys@defobject{currentmarker}{\pgfqpoint{0.000000in}{-0.027778in}}{\pgfqpoint{0.000000in}{0.000000in}}{%
\pgfpathmoveto{\pgfqpoint{0.000000in}{0.000000in}}%
\pgfpathlineto{\pgfqpoint{0.000000in}{-0.027778in}}%
\pgfusepath{stroke,fill}%
}%
\begin{pgfscope}%
\pgfsys@transformshift{2.020007in}{0.549073in}%
\pgfsys@useobject{currentmarker}{}%
\end{pgfscope}%
\end{pgfscope}%
\begin{pgfscope}%
\pgfpathrectangle{\pgfqpoint{0.721913in}{0.549073in}}{\pgfqpoint{2.325000in}{2.310000in}}%
\pgfusepath{clip}%
\pgfsetrectcap%
\pgfsetroundjoin%
\pgfsetlinewidth{0.250937pt}%
\definecolor{currentstroke}{rgb}{0.000000,0.000000,0.000000}%
\pgfsetstrokecolor{currentstroke}%
\pgfsetstrokeopacity{0.200000}%
\pgfsetdash{}{0pt}%
\pgfpathmoveto{\pgfqpoint{2.065314in}{0.549073in}}%
\pgfpathlineto{\pgfqpoint{2.065314in}{2.859073in}}%
\pgfusepath{stroke}%
\end{pgfscope}%
\begin{pgfscope}%
\pgfsetbuttcap%
\pgfsetroundjoin%
\definecolor{currentfill}{rgb}{0.000000,0.000000,0.000000}%
\pgfsetfillcolor{currentfill}%
\pgfsetlinewidth{0.602250pt}%
\definecolor{currentstroke}{rgb}{0.000000,0.000000,0.000000}%
\pgfsetstrokecolor{currentstroke}%
\pgfsetdash{}{0pt}%
\pgfsys@defobject{currentmarker}{\pgfqpoint{0.000000in}{-0.027778in}}{\pgfqpoint{0.000000in}{0.000000in}}{%
\pgfpathmoveto{\pgfqpoint{0.000000in}{0.000000in}}%
\pgfpathlineto{\pgfqpoint{0.000000in}{-0.027778in}}%
\pgfusepath{stroke,fill}%
}%
\begin{pgfscope}%
\pgfsys@transformshift{2.065314in}{0.549073in}%
\pgfsys@useobject{currentmarker}{}%
\end{pgfscope}%
\end{pgfscope}%
\begin{pgfscope}%
\pgfpathrectangle{\pgfqpoint{0.721913in}{0.549073in}}{\pgfqpoint{2.325000in}{2.310000in}}%
\pgfusepath{clip}%
\pgfsetrectcap%
\pgfsetroundjoin%
\pgfsetlinewidth{0.250937pt}%
\definecolor{currentstroke}{rgb}{0.000000,0.000000,0.000000}%
\pgfsetstrokecolor{currentstroke}%
\pgfsetstrokeopacity{0.200000}%
\pgfsetdash{}{0pt}%
\pgfpathmoveto{\pgfqpoint{2.372469in}{0.549073in}}%
\pgfpathlineto{\pgfqpoint{2.372469in}{2.859073in}}%
\pgfusepath{stroke}%
\end{pgfscope}%
\begin{pgfscope}%
\pgfsetbuttcap%
\pgfsetroundjoin%
\definecolor{currentfill}{rgb}{0.000000,0.000000,0.000000}%
\pgfsetfillcolor{currentfill}%
\pgfsetlinewidth{0.602250pt}%
\definecolor{currentstroke}{rgb}{0.000000,0.000000,0.000000}%
\pgfsetstrokecolor{currentstroke}%
\pgfsetdash{}{0pt}%
\pgfsys@defobject{currentmarker}{\pgfqpoint{0.000000in}{-0.027778in}}{\pgfqpoint{0.000000in}{0.000000in}}{%
\pgfpathmoveto{\pgfqpoint{0.000000in}{0.000000in}}%
\pgfpathlineto{\pgfqpoint{0.000000in}{-0.027778in}}%
\pgfusepath{stroke,fill}%
}%
\begin{pgfscope}%
\pgfsys@transformshift{2.372469in}{0.549073in}%
\pgfsys@useobject{currentmarker}{}%
\end{pgfscope}%
\end{pgfscope}%
\begin{pgfscope}%
\pgfpathrectangle{\pgfqpoint{0.721913in}{0.549073in}}{\pgfqpoint{2.325000in}{2.310000in}}%
\pgfusepath{clip}%
\pgfsetrectcap%
\pgfsetroundjoin%
\pgfsetlinewidth{0.250937pt}%
\definecolor{currentstroke}{rgb}{0.000000,0.000000,0.000000}%
\pgfsetstrokecolor{currentstroke}%
\pgfsetstrokeopacity{0.200000}%
\pgfsetdash{}{0pt}%
\pgfpathmoveto{\pgfqpoint{2.528435in}{0.549073in}}%
\pgfpathlineto{\pgfqpoint{2.528435in}{2.859073in}}%
\pgfusepath{stroke}%
\end{pgfscope}%
\begin{pgfscope}%
\pgfsetbuttcap%
\pgfsetroundjoin%
\definecolor{currentfill}{rgb}{0.000000,0.000000,0.000000}%
\pgfsetfillcolor{currentfill}%
\pgfsetlinewidth{0.602250pt}%
\definecolor{currentstroke}{rgb}{0.000000,0.000000,0.000000}%
\pgfsetstrokecolor{currentstroke}%
\pgfsetdash{}{0pt}%
\pgfsys@defobject{currentmarker}{\pgfqpoint{0.000000in}{-0.027778in}}{\pgfqpoint{0.000000in}{0.000000in}}{%
\pgfpathmoveto{\pgfqpoint{0.000000in}{0.000000in}}%
\pgfpathlineto{\pgfqpoint{0.000000in}{-0.027778in}}%
\pgfusepath{stroke,fill}%
}%
\begin{pgfscope}%
\pgfsys@transformshift{2.528435in}{0.549073in}%
\pgfsys@useobject{currentmarker}{}%
\end{pgfscope}%
\end{pgfscope}%
\begin{pgfscope}%
\pgfpathrectangle{\pgfqpoint{0.721913in}{0.549073in}}{\pgfqpoint{2.325000in}{2.310000in}}%
\pgfusepath{clip}%
\pgfsetrectcap%
\pgfsetroundjoin%
\pgfsetlinewidth{0.250937pt}%
\definecolor{currentstroke}{rgb}{0.000000,0.000000,0.000000}%
\pgfsetstrokecolor{currentstroke}%
\pgfsetstrokeopacity{0.200000}%
\pgfsetdash{}{0pt}%
\pgfpathmoveto{\pgfqpoint{2.639095in}{0.549073in}}%
\pgfpathlineto{\pgfqpoint{2.639095in}{2.859073in}}%
\pgfusepath{stroke}%
\end{pgfscope}%
\begin{pgfscope}%
\pgfsetbuttcap%
\pgfsetroundjoin%
\definecolor{currentfill}{rgb}{0.000000,0.000000,0.000000}%
\pgfsetfillcolor{currentfill}%
\pgfsetlinewidth{0.602250pt}%
\definecolor{currentstroke}{rgb}{0.000000,0.000000,0.000000}%
\pgfsetstrokecolor{currentstroke}%
\pgfsetdash{}{0pt}%
\pgfsys@defobject{currentmarker}{\pgfqpoint{0.000000in}{-0.027778in}}{\pgfqpoint{0.000000in}{0.000000in}}{%
\pgfpathmoveto{\pgfqpoint{0.000000in}{0.000000in}}%
\pgfpathlineto{\pgfqpoint{0.000000in}{-0.027778in}}%
\pgfusepath{stroke,fill}%
}%
\begin{pgfscope}%
\pgfsys@transformshift{2.639095in}{0.549073in}%
\pgfsys@useobject{currentmarker}{}%
\end{pgfscope}%
\end{pgfscope}%
\begin{pgfscope}%
\pgfpathrectangle{\pgfqpoint{0.721913in}{0.549073in}}{\pgfqpoint{2.325000in}{2.310000in}}%
\pgfusepath{clip}%
\pgfsetrectcap%
\pgfsetroundjoin%
\pgfsetlinewidth{0.250937pt}%
\definecolor{currentstroke}{rgb}{0.000000,0.000000,0.000000}%
\pgfsetstrokecolor{currentstroke}%
\pgfsetstrokeopacity{0.200000}%
\pgfsetdash{}{0pt}%
\pgfpathmoveto{\pgfqpoint{2.724930in}{0.549073in}}%
\pgfpathlineto{\pgfqpoint{2.724930in}{2.859073in}}%
\pgfusepath{stroke}%
\end{pgfscope}%
\begin{pgfscope}%
\pgfsetbuttcap%
\pgfsetroundjoin%
\definecolor{currentfill}{rgb}{0.000000,0.000000,0.000000}%
\pgfsetfillcolor{currentfill}%
\pgfsetlinewidth{0.602250pt}%
\definecolor{currentstroke}{rgb}{0.000000,0.000000,0.000000}%
\pgfsetstrokecolor{currentstroke}%
\pgfsetdash{}{0pt}%
\pgfsys@defobject{currentmarker}{\pgfqpoint{0.000000in}{-0.027778in}}{\pgfqpoint{0.000000in}{0.000000in}}{%
\pgfpathmoveto{\pgfqpoint{0.000000in}{0.000000in}}%
\pgfpathlineto{\pgfqpoint{0.000000in}{-0.027778in}}%
\pgfusepath{stroke,fill}%
}%
\begin{pgfscope}%
\pgfsys@transformshift{2.724930in}{0.549073in}%
\pgfsys@useobject{currentmarker}{}%
\end{pgfscope}%
\end{pgfscope}%
\begin{pgfscope}%
\pgfpathrectangle{\pgfqpoint{0.721913in}{0.549073in}}{\pgfqpoint{2.325000in}{2.310000in}}%
\pgfusepath{clip}%
\pgfsetrectcap%
\pgfsetroundjoin%
\pgfsetlinewidth{0.250937pt}%
\definecolor{currentstroke}{rgb}{0.000000,0.000000,0.000000}%
\pgfsetstrokecolor{currentstroke}%
\pgfsetstrokeopacity{0.200000}%
\pgfsetdash{}{0pt}%
\pgfpathmoveto{\pgfqpoint{2.795062in}{0.549073in}}%
\pgfpathlineto{\pgfqpoint{2.795062in}{2.859073in}}%
\pgfusepath{stroke}%
\end{pgfscope}%
\begin{pgfscope}%
\pgfsetbuttcap%
\pgfsetroundjoin%
\definecolor{currentfill}{rgb}{0.000000,0.000000,0.000000}%
\pgfsetfillcolor{currentfill}%
\pgfsetlinewidth{0.602250pt}%
\definecolor{currentstroke}{rgb}{0.000000,0.000000,0.000000}%
\pgfsetstrokecolor{currentstroke}%
\pgfsetdash{}{0pt}%
\pgfsys@defobject{currentmarker}{\pgfqpoint{0.000000in}{-0.027778in}}{\pgfqpoint{0.000000in}{0.000000in}}{%
\pgfpathmoveto{\pgfqpoint{0.000000in}{0.000000in}}%
\pgfpathlineto{\pgfqpoint{0.000000in}{-0.027778in}}%
\pgfusepath{stroke,fill}%
}%
\begin{pgfscope}%
\pgfsys@transformshift{2.795062in}{0.549073in}%
\pgfsys@useobject{currentmarker}{}%
\end{pgfscope}%
\end{pgfscope}%
\begin{pgfscope}%
\pgfpathrectangle{\pgfqpoint{0.721913in}{0.549073in}}{\pgfqpoint{2.325000in}{2.310000in}}%
\pgfusepath{clip}%
\pgfsetrectcap%
\pgfsetroundjoin%
\pgfsetlinewidth{0.250937pt}%
\definecolor{currentstroke}{rgb}{0.000000,0.000000,0.000000}%
\pgfsetstrokecolor{currentstroke}%
\pgfsetstrokeopacity{0.200000}%
\pgfsetdash{}{0pt}%
\pgfpathmoveto{\pgfqpoint{2.854357in}{0.549073in}}%
\pgfpathlineto{\pgfqpoint{2.854357in}{2.859073in}}%
\pgfusepath{stroke}%
\end{pgfscope}%
\begin{pgfscope}%
\pgfsetbuttcap%
\pgfsetroundjoin%
\definecolor{currentfill}{rgb}{0.000000,0.000000,0.000000}%
\pgfsetfillcolor{currentfill}%
\pgfsetlinewidth{0.602250pt}%
\definecolor{currentstroke}{rgb}{0.000000,0.000000,0.000000}%
\pgfsetstrokecolor{currentstroke}%
\pgfsetdash{}{0pt}%
\pgfsys@defobject{currentmarker}{\pgfqpoint{0.000000in}{-0.027778in}}{\pgfqpoint{0.000000in}{0.000000in}}{%
\pgfpathmoveto{\pgfqpoint{0.000000in}{0.000000in}}%
\pgfpathlineto{\pgfqpoint{0.000000in}{-0.027778in}}%
\pgfusepath{stroke,fill}%
}%
\begin{pgfscope}%
\pgfsys@transformshift{2.854357in}{0.549073in}%
\pgfsys@useobject{currentmarker}{}%
\end{pgfscope}%
\end{pgfscope}%
\begin{pgfscope}%
\pgfpathrectangle{\pgfqpoint{0.721913in}{0.549073in}}{\pgfqpoint{2.325000in}{2.310000in}}%
\pgfusepath{clip}%
\pgfsetrectcap%
\pgfsetroundjoin%
\pgfsetlinewidth{0.250937pt}%
\definecolor{currentstroke}{rgb}{0.000000,0.000000,0.000000}%
\pgfsetstrokecolor{currentstroke}%
\pgfsetstrokeopacity{0.200000}%
\pgfsetdash{}{0pt}%
\pgfpathmoveto{\pgfqpoint{2.905722in}{0.549073in}}%
\pgfpathlineto{\pgfqpoint{2.905722in}{2.859073in}}%
\pgfusepath{stroke}%
\end{pgfscope}%
\begin{pgfscope}%
\pgfsetbuttcap%
\pgfsetroundjoin%
\definecolor{currentfill}{rgb}{0.000000,0.000000,0.000000}%
\pgfsetfillcolor{currentfill}%
\pgfsetlinewidth{0.602250pt}%
\definecolor{currentstroke}{rgb}{0.000000,0.000000,0.000000}%
\pgfsetstrokecolor{currentstroke}%
\pgfsetdash{}{0pt}%
\pgfsys@defobject{currentmarker}{\pgfqpoint{0.000000in}{-0.027778in}}{\pgfqpoint{0.000000in}{0.000000in}}{%
\pgfpathmoveto{\pgfqpoint{0.000000in}{0.000000in}}%
\pgfpathlineto{\pgfqpoint{0.000000in}{-0.027778in}}%
\pgfusepath{stroke,fill}%
}%
\begin{pgfscope}%
\pgfsys@transformshift{2.905722in}{0.549073in}%
\pgfsys@useobject{currentmarker}{}%
\end{pgfscope}%
\end{pgfscope}%
\begin{pgfscope}%
\pgfpathrectangle{\pgfqpoint{0.721913in}{0.549073in}}{\pgfqpoint{2.325000in}{2.310000in}}%
\pgfusepath{clip}%
\pgfsetrectcap%
\pgfsetroundjoin%
\pgfsetlinewidth{0.250937pt}%
\definecolor{currentstroke}{rgb}{0.000000,0.000000,0.000000}%
\pgfsetstrokecolor{currentstroke}%
\pgfsetstrokeopacity{0.200000}%
\pgfsetdash{}{0pt}%
\pgfpathmoveto{\pgfqpoint{2.951028in}{0.549073in}}%
\pgfpathlineto{\pgfqpoint{2.951028in}{2.859073in}}%
\pgfusepath{stroke}%
\end{pgfscope}%
\begin{pgfscope}%
\pgfsetbuttcap%
\pgfsetroundjoin%
\definecolor{currentfill}{rgb}{0.000000,0.000000,0.000000}%
\pgfsetfillcolor{currentfill}%
\pgfsetlinewidth{0.602250pt}%
\definecolor{currentstroke}{rgb}{0.000000,0.000000,0.000000}%
\pgfsetstrokecolor{currentstroke}%
\pgfsetdash{}{0pt}%
\pgfsys@defobject{currentmarker}{\pgfqpoint{0.000000in}{-0.027778in}}{\pgfqpoint{0.000000in}{0.000000in}}{%
\pgfpathmoveto{\pgfqpoint{0.000000in}{0.000000in}}%
\pgfpathlineto{\pgfqpoint{0.000000in}{-0.027778in}}%
\pgfusepath{stroke,fill}%
}%
\begin{pgfscope}%
\pgfsys@transformshift{2.951028in}{0.549073in}%
\pgfsys@useobject{currentmarker}{}%
\end{pgfscope}%
\end{pgfscope}%
\begin{pgfscope}%
\definecolor{textcolor}{rgb}{0.000000,0.000000,0.000000}%
\pgfsetstrokecolor{textcolor}%
\pgfsetfillcolor{textcolor}%
\pgftext[x=1.884413in,y=0.248148in,,top]{\color{textcolor}{\rmfamily\fontsize{12.000000}{14.400000}\selectfont\catcode`\^=\active\def^{\ifmmode\sp\else\^{}\fi}\catcode`\%=\active\def%{\%}smoothing parameter $\sigma$}}%
\end{pgfscope}%
\begin{pgfscope}%
\pgfpathrectangle{\pgfqpoint{0.721913in}{0.549073in}}{\pgfqpoint{2.325000in}{2.310000in}}%
\pgfusepath{clip}%
\pgfsetrectcap%
\pgfsetroundjoin%
\pgfsetlinewidth{0.250937pt}%
\definecolor{currentstroke}{rgb}{0.000000,0.000000,0.000000}%
\pgfsetstrokecolor{currentstroke}%
\pgfsetstrokeopacity{0.200000}%
\pgfsetdash{}{0pt}%
\pgfpathmoveto{\pgfqpoint{0.721913in}{0.647289in}}%
\pgfpathlineto{\pgfqpoint{3.046913in}{0.647289in}}%
\pgfusepath{stroke}%
\end{pgfscope}%
\begin{pgfscope}%
\pgfsetbuttcap%
\pgfsetroundjoin%
\definecolor{currentfill}{rgb}{0.000000,0.000000,0.000000}%
\pgfsetfillcolor{currentfill}%
\pgfsetlinewidth{0.803000pt}%
\definecolor{currentstroke}{rgb}{0.000000,0.000000,0.000000}%
\pgfsetstrokecolor{currentstroke}%
\pgfsetdash{}{0pt}%
\pgfsys@defobject{currentmarker}{\pgfqpoint{-0.048611in}{0.000000in}}{\pgfqpoint{-0.000000in}{0.000000in}}{%
\pgfpathmoveto{\pgfqpoint{-0.000000in}{0.000000in}}%
\pgfpathlineto{\pgfqpoint{-0.048611in}{0.000000in}}%
\pgfusepath{stroke,fill}%
}%
\begin{pgfscope}%
\pgfsys@transformshift{0.721913in}{0.647289in}%
\pgfsys@useobject{currentmarker}{}%
\end{pgfscope}%
\end{pgfscope}%
\begin{pgfscope}%
\definecolor{textcolor}{rgb}{0.000000,0.000000,0.000000}%
\pgfsetstrokecolor{textcolor}%
\pgfsetfillcolor{textcolor}%
\pgftext[x=0.303703in, y=0.589419in, left, base]{\color{textcolor}{\rmfamily\fontsize{12.000000}{14.400000}\selectfont\catcode`\^=\active\def^{\ifmmode\sp\else\^{}\fi}\catcode`\%=\active\def%{\%}$\mathdefault{10^{-3}}$}}%
\end{pgfscope}%
\begin{pgfscope}%
\pgfpathrectangle{\pgfqpoint{0.721913in}{0.549073in}}{\pgfqpoint{2.325000in}{2.310000in}}%
\pgfusepath{clip}%
\pgfsetrectcap%
\pgfsetroundjoin%
\pgfsetlinewidth{0.250937pt}%
\definecolor{currentstroke}{rgb}{0.000000,0.000000,0.000000}%
\pgfsetstrokecolor{currentstroke}%
\pgfsetstrokeopacity{0.200000}%
\pgfsetdash{}{0pt}%
\pgfpathmoveto{\pgfqpoint{0.721913in}{1.288210in}}%
\pgfpathlineto{\pgfqpoint{3.046913in}{1.288210in}}%
\pgfusepath{stroke}%
\end{pgfscope}%
\begin{pgfscope}%
\pgfsetbuttcap%
\pgfsetroundjoin%
\definecolor{currentfill}{rgb}{0.000000,0.000000,0.000000}%
\pgfsetfillcolor{currentfill}%
\pgfsetlinewidth{0.803000pt}%
\definecolor{currentstroke}{rgb}{0.000000,0.000000,0.000000}%
\pgfsetstrokecolor{currentstroke}%
\pgfsetdash{}{0pt}%
\pgfsys@defobject{currentmarker}{\pgfqpoint{-0.048611in}{0.000000in}}{\pgfqpoint{-0.000000in}{0.000000in}}{%
\pgfpathmoveto{\pgfqpoint{-0.000000in}{0.000000in}}%
\pgfpathlineto{\pgfqpoint{-0.048611in}{0.000000in}}%
\pgfusepath{stroke,fill}%
}%
\begin{pgfscope}%
\pgfsys@transformshift{0.721913in}{1.288210in}%
\pgfsys@useobject{currentmarker}{}%
\end{pgfscope}%
\end{pgfscope}%
\begin{pgfscope}%
\definecolor{textcolor}{rgb}{0.000000,0.000000,0.000000}%
\pgfsetstrokecolor{textcolor}%
\pgfsetfillcolor{textcolor}%
\pgftext[x=0.303703in, y=1.230340in, left, base]{\color{textcolor}{\rmfamily\fontsize{12.000000}{14.400000}\selectfont\catcode`\^=\active\def^{\ifmmode\sp\else\^{}\fi}\catcode`\%=\active\def%{\%}$\mathdefault{10^{-2}}$}}%
\end{pgfscope}%
\begin{pgfscope}%
\pgfpathrectangle{\pgfqpoint{0.721913in}{0.549073in}}{\pgfqpoint{2.325000in}{2.310000in}}%
\pgfusepath{clip}%
\pgfsetrectcap%
\pgfsetroundjoin%
\pgfsetlinewidth{0.250937pt}%
\definecolor{currentstroke}{rgb}{0.000000,0.000000,0.000000}%
\pgfsetstrokecolor{currentstroke}%
\pgfsetstrokeopacity{0.200000}%
\pgfsetdash{}{0pt}%
\pgfpathmoveto{\pgfqpoint{0.721913in}{1.929131in}}%
\pgfpathlineto{\pgfqpoint{3.046913in}{1.929131in}}%
\pgfusepath{stroke}%
\end{pgfscope}%
\begin{pgfscope}%
\pgfsetbuttcap%
\pgfsetroundjoin%
\definecolor{currentfill}{rgb}{0.000000,0.000000,0.000000}%
\pgfsetfillcolor{currentfill}%
\pgfsetlinewidth{0.803000pt}%
\definecolor{currentstroke}{rgb}{0.000000,0.000000,0.000000}%
\pgfsetstrokecolor{currentstroke}%
\pgfsetdash{}{0pt}%
\pgfsys@defobject{currentmarker}{\pgfqpoint{-0.048611in}{0.000000in}}{\pgfqpoint{-0.000000in}{0.000000in}}{%
\pgfpathmoveto{\pgfqpoint{-0.000000in}{0.000000in}}%
\pgfpathlineto{\pgfqpoint{-0.048611in}{0.000000in}}%
\pgfusepath{stroke,fill}%
}%
\begin{pgfscope}%
\pgfsys@transformshift{0.721913in}{1.929131in}%
\pgfsys@useobject{currentmarker}{}%
\end{pgfscope}%
\end{pgfscope}%
\begin{pgfscope}%
\definecolor{textcolor}{rgb}{0.000000,0.000000,0.000000}%
\pgfsetstrokecolor{textcolor}%
\pgfsetfillcolor{textcolor}%
\pgftext[x=0.303703in, y=1.871260in, left, base]{\color{textcolor}{\rmfamily\fontsize{12.000000}{14.400000}\selectfont\catcode`\^=\active\def^{\ifmmode\sp\else\^{}\fi}\catcode`\%=\active\def%{\%}$\mathdefault{10^{-1}}$}}%
\end{pgfscope}%
\begin{pgfscope}%
\pgfpathrectangle{\pgfqpoint{0.721913in}{0.549073in}}{\pgfqpoint{2.325000in}{2.310000in}}%
\pgfusepath{clip}%
\pgfsetrectcap%
\pgfsetroundjoin%
\pgfsetlinewidth{0.250937pt}%
\definecolor{currentstroke}{rgb}{0.000000,0.000000,0.000000}%
\pgfsetstrokecolor{currentstroke}%
\pgfsetstrokeopacity{0.200000}%
\pgfsetdash{}{0pt}%
\pgfpathmoveto{\pgfqpoint{0.721913in}{2.570052in}}%
\pgfpathlineto{\pgfqpoint{3.046913in}{2.570052in}}%
\pgfusepath{stroke}%
\end{pgfscope}%
\begin{pgfscope}%
\pgfsetbuttcap%
\pgfsetroundjoin%
\definecolor{currentfill}{rgb}{0.000000,0.000000,0.000000}%
\pgfsetfillcolor{currentfill}%
\pgfsetlinewidth{0.803000pt}%
\definecolor{currentstroke}{rgb}{0.000000,0.000000,0.000000}%
\pgfsetstrokecolor{currentstroke}%
\pgfsetdash{}{0pt}%
\pgfsys@defobject{currentmarker}{\pgfqpoint{-0.048611in}{0.000000in}}{\pgfqpoint{-0.000000in}{0.000000in}}{%
\pgfpathmoveto{\pgfqpoint{-0.000000in}{0.000000in}}%
\pgfpathlineto{\pgfqpoint{-0.048611in}{0.000000in}}%
\pgfusepath{stroke,fill}%
}%
\begin{pgfscope}%
\pgfsys@transformshift{0.721913in}{2.570052in}%
\pgfsys@useobject{currentmarker}{}%
\end{pgfscope}%
\end{pgfscope}%
\begin{pgfscope}%
\definecolor{textcolor}{rgb}{0.000000,0.000000,0.000000}%
\pgfsetstrokecolor{textcolor}%
\pgfsetfillcolor{textcolor}%
\pgftext[x=0.395525in, y=2.512181in, left, base]{\color{textcolor}{\rmfamily\fontsize{12.000000}{14.400000}\selectfont\catcode`\^=\active\def^{\ifmmode\sp\else\^{}\fi}\catcode`\%=\active\def%{\%}$\mathdefault{10^{0}}$}}%
\end{pgfscope}%
\begin{pgfscope}%
\pgfpathrectangle{\pgfqpoint{0.721913in}{0.549073in}}{\pgfqpoint{2.325000in}{2.310000in}}%
\pgfusepath{clip}%
\pgfsetrectcap%
\pgfsetroundjoin%
\pgfsetlinewidth{0.250937pt}%
\definecolor{currentstroke}{rgb}{0.000000,0.000000,0.000000}%
\pgfsetstrokecolor{currentstroke}%
\pgfsetstrokeopacity{0.200000}%
\pgfsetdash{}{0pt}%
\pgfpathmoveto{\pgfqpoint{0.721913in}{0.585177in}}%
\pgfpathlineto{\pgfqpoint{3.046913in}{0.585177in}}%
\pgfusepath{stroke}%
\end{pgfscope}%
\begin{pgfscope}%
\pgfsetbuttcap%
\pgfsetroundjoin%
\definecolor{currentfill}{rgb}{0.000000,0.000000,0.000000}%
\pgfsetfillcolor{currentfill}%
\pgfsetlinewidth{0.602250pt}%
\definecolor{currentstroke}{rgb}{0.000000,0.000000,0.000000}%
\pgfsetstrokecolor{currentstroke}%
\pgfsetdash{}{0pt}%
\pgfsys@defobject{currentmarker}{\pgfqpoint{-0.027778in}{0.000000in}}{\pgfqpoint{-0.000000in}{0.000000in}}{%
\pgfpathmoveto{\pgfqpoint{-0.000000in}{0.000000in}}%
\pgfpathlineto{\pgfqpoint{-0.027778in}{0.000000in}}%
\pgfusepath{stroke,fill}%
}%
\begin{pgfscope}%
\pgfsys@transformshift{0.721913in}{0.585177in}%
\pgfsys@useobject{currentmarker}{}%
\end{pgfscope}%
\end{pgfscope}%
\begin{pgfscope}%
\pgfpathrectangle{\pgfqpoint{0.721913in}{0.549073in}}{\pgfqpoint{2.325000in}{2.310000in}}%
\pgfusepath{clip}%
\pgfsetrectcap%
\pgfsetroundjoin%
\pgfsetlinewidth{0.250937pt}%
\definecolor{currentstroke}{rgb}{0.000000,0.000000,0.000000}%
\pgfsetstrokecolor{currentstroke}%
\pgfsetstrokeopacity{0.200000}%
\pgfsetdash{}{0pt}%
\pgfpathmoveto{\pgfqpoint{0.721913in}{0.617962in}}%
\pgfpathlineto{\pgfqpoint{3.046913in}{0.617962in}}%
\pgfusepath{stroke}%
\end{pgfscope}%
\begin{pgfscope}%
\pgfsetbuttcap%
\pgfsetroundjoin%
\definecolor{currentfill}{rgb}{0.000000,0.000000,0.000000}%
\pgfsetfillcolor{currentfill}%
\pgfsetlinewidth{0.602250pt}%
\definecolor{currentstroke}{rgb}{0.000000,0.000000,0.000000}%
\pgfsetstrokecolor{currentstroke}%
\pgfsetdash{}{0pt}%
\pgfsys@defobject{currentmarker}{\pgfqpoint{-0.027778in}{0.000000in}}{\pgfqpoint{-0.000000in}{0.000000in}}{%
\pgfpathmoveto{\pgfqpoint{-0.000000in}{0.000000in}}%
\pgfpathlineto{\pgfqpoint{-0.027778in}{0.000000in}}%
\pgfusepath{stroke,fill}%
}%
\begin{pgfscope}%
\pgfsys@transformshift{0.721913in}{0.617962in}%
\pgfsys@useobject{currentmarker}{}%
\end{pgfscope}%
\end{pgfscope}%
\begin{pgfscope}%
\pgfpathrectangle{\pgfqpoint{0.721913in}{0.549073in}}{\pgfqpoint{2.325000in}{2.310000in}}%
\pgfusepath{clip}%
\pgfsetrectcap%
\pgfsetroundjoin%
\pgfsetlinewidth{0.250937pt}%
\definecolor{currentstroke}{rgb}{0.000000,0.000000,0.000000}%
\pgfsetstrokecolor{currentstroke}%
\pgfsetstrokeopacity{0.200000}%
\pgfsetdash{}{0pt}%
\pgfpathmoveto{\pgfqpoint{0.721913in}{0.840225in}}%
\pgfpathlineto{\pgfqpoint{3.046913in}{0.840225in}}%
\pgfusepath{stroke}%
\end{pgfscope}%
\begin{pgfscope}%
\pgfsetbuttcap%
\pgfsetroundjoin%
\definecolor{currentfill}{rgb}{0.000000,0.000000,0.000000}%
\pgfsetfillcolor{currentfill}%
\pgfsetlinewidth{0.602250pt}%
\definecolor{currentstroke}{rgb}{0.000000,0.000000,0.000000}%
\pgfsetstrokecolor{currentstroke}%
\pgfsetdash{}{0pt}%
\pgfsys@defobject{currentmarker}{\pgfqpoint{-0.027778in}{0.000000in}}{\pgfqpoint{-0.000000in}{0.000000in}}{%
\pgfpathmoveto{\pgfqpoint{-0.000000in}{0.000000in}}%
\pgfpathlineto{\pgfqpoint{-0.027778in}{0.000000in}}%
\pgfusepath{stroke,fill}%
}%
\begin{pgfscope}%
\pgfsys@transformshift{0.721913in}{0.840225in}%
\pgfsys@useobject{currentmarker}{}%
\end{pgfscope}%
\end{pgfscope}%
\begin{pgfscope}%
\pgfpathrectangle{\pgfqpoint{0.721913in}{0.549073in}}{\pgfqpoint{2.325000in}{2.310000in}}%
\pgfusepath{clip}%
\pgfsetrectcap%
\pgfsetroundjoin%
\pgfsetlinewidth{0.250937pt}%
\definecolor{currentstroke}{rgb}{0.000000,0.000000,0.000000}%
\pgfsetstrokecolor{currentstroke}%
\pgfsetstrokeopacity{0.200000}%
\pgfsetdash{}{0pt}%
\pgfpathmoveto{\pgfqpoint{0.721913in}{0.953086in}}%
\pgfpathlineto{\pgfqpoint{3.046913in}{0.953086in}}%
\pgfusepath{stroke}%
\end{pgfscope}%
\begin{pgfscope}%
\pgfsetbuttcap%
\pgfsetroundjoin%
\definecolor{currentfill}{rgb}{0.000000,0.000000,0.000000}%
\pgfsetfillcolor{currentfill}%
\pgfsetlinewidth{0.602250pt}%
\definecolor{currentstroke}{rgb}{0.000000,0.000000,0.000000}%
\pgfsetstrokecolor{currentstroke}%
\pgfsetdash{}{0pt}%
\pgfsys@defobject{currentmarker}{\pgfqpoint{-0.027778in}{0.000000in}}{\pgfqpoint{-0.000000in}{0.000000in}}{%
\pgfpathmoveto{\pgfqpoint{-0.000000in}{0.000000in}}%
\pgfpathlineto{\pgfqpoint{-0.027778in}{0.000000in}}%
\pgfusepath{stroke,fill}%
}%
\begin{pgfscope}%
\pgfsys@transformshift{0.721913in}{0.953086in}%
\pgfsys@useobject{currentmarker}{}%
\end{pgfscope}%
\end{pgfscope}%
\begin{pgfscope}%
\pgfpathrectangle{\pgfqpoint{0.721913in}{0.549073in}}{\pgfqpoint{2.325000in}{2.310000in}}%
\pgfusepath{clip}%
\pgfsetrectcap%
\pgfsetroundjoin%
\pgfsetlinewidth{0.250937pt}%
\definecolor{currentstroke}{rgb}{0.000000,0.000000,0.000000}%
\pgfsetstrokecolor{currentstroke}%
\pgfsetstrokeopacity{0.200000}%
\pgfsetdash{}{0pt}%
\pgfpathmoveto{\pgfqpoint{0.721913in}{1.033162in}}%
\pgfpathlineto{\pgfqpoint{3.046913in}{1.033162in}}%
\pgfusepath{stroke}%
\end{pgfscope}%
\begin{pgfscope}%
\pgfsetbuttcap%
\pgfsetroundjoin%
\definecolor{currentfill}{rgb}{0.000000,0.000000,0.000000}%
\pgfsetfillcolor{currentfill}%
\pgfsetlinewidth{0.602250pt}%
\definecolor{currentstroke}{rgb}{0.000000,0.000000,0.000000}%
\pgfsetstrokecolor{currentstroke}%
\pgfsetdash{}{0pt}%
\pgfsys@defobject{currentmarker}{\pgfqpoint{-0.027778in}{0.000000in}}{\pgfqpoint{-0.000000in}{0.000000in}}{%
\pgfpathmoveto{\pgfqpoint{-0.000000in}{0.000000in}}%
\pgfpathlineto{\pgfqpoint{-0.027778in}{0.000000in}}%
\pgfusepath{stroke,fill}%
}%
\begin{pgfscope}%
\pgfsys@transformshift{0.721913in}{1.033162in}%
\pgfsys@useobject{currentmarker}{}%
\end{pgfscope}%
\end{pgfscope}%
\begin{pgfscope}%
\pgfpathrectangle{\pgfqpoint{0.721913in}{0.549073in}}{\pgfqpoint{2.325000in}{2.310000in}}%
\pgfusepath{clip}%
\pgfsetrectcap%
\pgfsetroundjoin%
\pgfsetlinewidth{0.250937pt}%
\definecolor{currentstroke}{rgb}{0.000000,0.000000,0.000000}%
\pgfsetstrokecolor{currentstroke}%
\pgfsetstrokeopacity{0.200000}%
\pgfsetdash{}{0pt}%
\pgfpathmoveto{\pgfqpoint{0.721913in}{1.095273in}}%
\pgfpathlineto{\pgfqpoint{3.046913in}{1.095273in}}%
\pgfusepath{stroke}%
\end{pgfscope}%
\begin{pgfscope}%
\pgfsetbuttcap%
\pgfsetroundjoin%
\definecolor{currentfill}{rgb}{0.000000,0.000000,0.000000}%
\pgfsetfillcolor{currentfill}%
\pgfsetlinewidth{0.602250pt}%
\definecolor{currentstroke}{rgb}{0.000000,0.000000,0.000000}%
\pgfsetstrokecolor{currentstroke}%
\pgfsetdash{}{0pt}%
\pgfsys@defobject{currentmarker}{\pgfqpoint{-0.027778in}{0.000000in}}{\pgfqpoint{-0.000000in}{0.000000in}}{%
\pgfpathmoveto{\pgfqpoint{-0.000000in}{0.000000in}}%
\pgfpathlineto{\pgfqpoint{-0.027778in}{0.000000in}}%
\pgfusepath{stroke,fill}%
}%
\begin{pgfscope}%
\pgfsys@transformshift{0.721913in}{1.095273in}%
\pgfsys@useobject{currentmarker}{}%
\end{pgfscope}%
\end{pgfscope}%
\begin{pgfscope}%
\pgfpathrectangle{\pgfqpoint{0.721913in}{0.549073in}}{\pgfqpoint{2.325000in}{2.310000in}}%
\pgfusepath{clip}%
\pgfsetrectcap%
\pgfsetroundjoin%
\pgfsetlinewidth{0.250937pt}%
\definecolor{currentstroke}{rgb}{0.000000,0.000000,0.000000}%
\pgfsetstrokecolor{currentstroke}%
\pgfsetstrokeopacity{0.200000}%
\pgfsetdash{}{0pt}%
\pgfpathmoveto{\pgfqpoint{0.721913in}{1.146022in}}%
\pgfpathlineto{\pgfqpoint{3.046913in}{1.146022in}}%
\pgfusepath{stroke}%
\end{pgfscope}%
\begin{pgfscope}%
\pgfsetbuttcap%
\pgfsetroundjoin%
\definecolor{currentfill}{rgb}{0.000000,0.000000,0.000000}%
\pgfsetfillcolor{currentfill}%
\pgfsetlinewidth{0.602250pt}%
\definecolor{currentstroke}{rgb}{0.000000,0.000000,0.000000}%
\pgfsetstrokecolor{currentstroke}%
\pgfsetdash{}{0pt}%
\pgfsys@defobject{currentmarker}{\pgfqpoint{-0.027778in}{0.000000in}}{\pgfqpoint{-0.000000in}{0.000000in}}{%
\pgfpathmoveto{\pgfqpoint{-0.000000in}{0.000000in}}%
\pgfpathlineto{\pgfqpoint{-0.027778in}{0.000000in}}%
\pgfusepath{stroke,fill}%
}%
\begin{pgfscope}%
\pgfsys@transformshift{0.721913in}{1.146022in}%
\pgfsys@useobject{currentmarker}{}%
\end{pgfscope}%
\end{pgfscope}%
\begin{pgfscope}%
\pgfpathrectangle{\pgfqpoint{0.721913in}{0.549073in}}{\pgfqpoint{2.325000in}{2.310000in}}%
\pgfusepath{clip}%
\pgfsetrectcap%
\pgfsetroundjoin%
\pgfsetlinewidth{0.250937pt}%
\definecolor{currentstroke}{rgb}{0.000000,0.000000,0.000000}%
\pgfsetstrokecolor{currentstroke}%
\pgfsetstrokeopacity{0.200000}%
\pgfsetdash{}{0pt}%
\pgfpathmoveto{\pgfqpoint{0.721913in}{1.188930in}}%
\pgfpathlineto{\pgfqpoint{3.046913in}{1.188930in}}%
\pgfusepath{stroke}%
\end{pgfscope}%
\begin{pgfscope}%
\pgfsetbuttcap%
\pgfsetroundjoin%
\definecolor{currentfill}{rgb}{0.000000,0.000000,0.000000}%
\pgfsetfillcolor{currentfill}%
\pgfsetlinewidth{0.602250pt}%
\definecolor{currentstroke}{rgb}{0.000000,0.000000,0.000000}%
\pgfsetstrokecolor{currentstroke}%
\pgfsetdash{}{0pt}%
\pgfsys@defobject{currentmarker}{\pgfqpoint{-0.027778in}{0.000000in}}{\pgfqpoint{-0.000000in}{0.000000in}}{%
\pgfpathmoveto{\pgfqpoint{-0.000000in}{0.000000in}}%
\pgfpathlineto{\pgfqpoint{-0.027778in}{0.000000in}}%
\pgfusepath{stroke,fill}%
}%
\begin{pgfscope}%
\pgfsys@transformshift{0.721913in}{1.188930in}%
\pgfsys@useobject{currentmarker}{}%
\end{pgfscope}%
\end{pgfscope}%
\begin{pgfscope}%
\pgfpathrectangle{\pgfqpoint{0.721913in}{0.549073in}}{\pgfqpoint{2.325000in}{2.310000in}}%
\pgfusepath{clip}%
\pgfsetrectcap%
\pgfsetroundjoin%
\pgfsetlinewidth{0.250937pt}%
\definecolor{currentstroke}{rgb}{0.000000,0.000000,0.000000}%
\pgfsetstrokecolor{currentstroke}%
\pgfsetstrokeopacity{0.200000}%
\pgfsetdash{}{0pt}%
\pgfpathmoveto{\pgfqpoint{0.721913in}{1.226098in}}%
\pgfpathlineto{\pgfqpoint{3.046913in}{1.226098in}}%
\pgfusepath{stroke}%
\end{pgfscope}%
\begin{pgfscope}%
\pgfsetbuttcap%
\pgfsetroundjoin%
\definecolor{currentfill}{rgb}{0.000000,0.000000,0.000000}%
\pgfsetfillcolor{currentfill}%
\pgfsetlinewidth{0.602250pt}%
\definecolor{currentstroke}{rgb}{0.000000,0.000000,0.000000}%
\pgfsetstrokecolor{currentstroke}%
\pgfsetdash{}{0pt}%
\pgfsys@defobject{currentmarker}{\pgfqpoint{-0.027778in}{0.000000in}}{\pgfqpoint{-0.000000in}{0.000000in}}{%
\pgfpathmoveto{\pgfqpoint{-0.000000in}{0.000000in}}%
\pgfpathlineto{\pgfqpoint{-0.027778in}{0.000000in}}%
\pgfusepath{stroke,fill}%
}%
\begin{pgfscope}%
\pgfsys@transformshift{0.721913in}{1.226098in}%
\pgfsys@useobject{currentmarker}{}%
\end{pgfscope}%
\end{pgfscope}%
\begin{pgfscope}%
\pgfpathrectangle{\pgfqpoint{0.721913in}{0.549073in}}{\pgfqpoint{2.325000in}{2.310000in}}%
\pgfusepath{clip}%
\pgfsetrectcap%
\pgfsetroundjoin%
\pgfsetlinewidth{0.250937pt}%
\definecolor{currentstroke}{rgb}{0.000000,0.000000,0.000000}%
\pgfsetstrokecolor{currentstroke}%
\pgfsetstrokeopacity{0.200000}%
\pgfsetdash{}{0pt}%
\pgfpathmoveto{\pgfqpoint{0.721913in}{1.258883in}}%
\pgfpathlineto{\pgfqpoint{3.046913in}{1.258883in}}%
\pgfusepath{stroke}%
\end{pgfscope}%
\begin{pgfscope}%
\pgfsetbuttcap%
\pgfsetroundjoin%
\definecolor{currentfill}{rgb}{0.000000,0.000000,0.000000}%
\pgfsetfillcolor{currentfill}%
\pgfsetlinewidth{0.602250pt}%
\definecolor{currentstroke}{rgb}{0.000000,0.000000,0.000000}%
\pgfsetstrokecolor{currentstroke}%
\pgfsetdash{}{0pt}%
\pgfsys@defobject{currentmarker}{\pgfqpoint{-0.027778in}{0.000000in}}{\pgfqpoint{-0.000000in}{0.000000in}}{%
\pgfpathmoveto{\pgfqpoint{-0.000000in}{0.000000in}}%
\pgfpathlineto{\pgfqpoint{-0.027778in}{0.000000in}}%
\pgfusepath{stroke,fill}%
}%
\begin{pgfscope}%
\pgfsys@transformshift{0.721913in}{1.258883in}%
\pgfsys@useobject{currentmarker}{}%
\end{pgfscope}%
\end{pgfscope}%
\begin{pgfscope}%
\pgfpathrectangle{\pgfqpoint{0.721913in}{0.549073in}}{\pgfqpoint{2.325000in}{2.310000in}}%
\pgfusepath{clip}%
\pgfsetrectcap%
\pgfsetroundjoin%
\pgfsetlinewidth{0.250937pt}%
\definecolor{currentstroke}{rgb}{0.000000,0.000000,0.000000}%
\pgfsetstrokecolor{currentstroke}%
\pgfsetstrokeopacity{0.200000}%
\pgfsetdash{}{0pt}%
\pgfpathmoveto{\pgfqpoint{0.721913in}{1.481146in}}%
\pgfpathlineto{\pgfqpoint{3.046913in}{1.481146in}}%
\pgfusepath{stroke}%
\end{pgfscope}%
\begin{pgfscope}%
\pgfsetbuttcap%
\pgfsetroundjoin%
\definecolor{currentfill}{rgb}{0.000000,0.000000,0.000000}%
\pgfsetfillcolor{currentfill}%
\pgfsetlinewidth{0.602250pt}%
\definecolor{currentstroke}{rgb}{0.000000,0.000000,0.000000}%
\pgfsetstrokecolor{currentstroke}%
\pgfsetdash{}{0pt}%
\pgfsys@defobject{currentmarker}{\pgfqpoint{-0.027778in}{0.000000in}}{\pgfqpoint{-0.000000in}{0.000000in}}{%
\pgfpathmoveto{\pgfqpoint{-0.000000in}{0.000000in}}%
\pgfpathlineto{\pgfqpoint{-0.027778in}{0.000000in}}%
\pgfusepath{stroke,fill}%
}%
\begin{pgfscope}%
\pgfsys@transformshift{0.721913in}{1.481146in}%
\pgfsys@useobject{currentmarker}{}%
\end{pgfscope}%
\end{pgfscope}%
\begin{pgfscope}%
\pgfpathrectangle{\pgfqpoint{0.721913in}{0.549073in}}{\pgfqpoint{2.325000in}{2.310000in}}%
\pgfusepath{clip}%
\pgfsetrectcap%
\pgfsetroundjoin%
\pgfsetlinewidth{0.250937pt}%
\definecolor{currentstroke}{rgb}{0.000000,0.000000,0.000000}%
\pgfsetstrokecolor{currentstroke}%
\pgfsetstrokeopacity{0.200000}%
\pgfsetdash{}{0pt}%
\pgfpathmoveto{\pgfqpoint{0.721913in}{1.594007in}}%
\pgfpathlineto{\pgfqpoint{3.046913in}{1.594007in}}%
\pgfusepath{stroke}%
\end{pgfscope}%
\begin{pgfscope}%
\pgfsetbuttcap%
\pgfsetroundjoin%
\definecolor{currentfill}{rgb}{0.000000,0.000000,0.000000}%
\pgfsetfillcolor{currentfill}%
\pgfsetlinewidth{0.602250pt}%
\definecolor{currentstroke}{rgb}{0.000000,0.000000,0.000000}%
\pgfsetstrokecolor{currentstroke}%
\pgfsetdash{}{0pt}%
\pgfsys@defobject{currentmarker}{\pgfqpoint{-0.027778in}{0.000000in}}{\pgfqpoint{-0.000000in}{0.000000in}}{%
\pgfpathmoveto{\pgfqpoint{-0.000000in}{0.000000in}}%
\pgfpathlineto{\pgfqpoint{-0.027778in}{0.000000in}}%
\pgfusepath{stroke,fill}%
}%
\begin{pgfscope}%
\pgfsys@transformshift{0.721913in}{1.594007in}%
\pgfsys@useobject{currentmarker}{}%
\end{pgfscope}%
\end{pgfscope}%
\begin{pgfscope}%
\pgfpathrectangle{\pgfqpoint{0.721913in}{0.549073in}}{\pgfqpoint{2.325000in}{2.310000in}}%
\pgfusepath{clip}%
\pgfsetrectcap%
\pgfsetroundjoin%
\pgfsetlinewidth{0.250937pt}%
\definecolor{currentstroke}{rgb}{0.000000,0.000000,0.000000}%
\pgfsetstrokecolor{currentstroke}%
\pgfsetstrokeopacity{0.200000}%
\pgfsetdash{}{0pt}%
\pgfpathmoveto{\pgfqpoint{0.721913in}{1.674083in}}%
\pgfpathlineto{\pgfqpoint{3.046913in}{1.674083in}}%
\pgfusepath{stroke}%
\end{pgfscope}%
\begin{pgfscope}%
\pgfsetbuttcap%
\pgfsetroundjoin%
\definecolor{currentfill}{rgb}{0.000000,0.000000,0.000000}%
\pgfsetfillcolor{currentfill}%
\pgfsetlinewidth{0.602250pt}%
\definecolor{currentstroke}{rgb}{0.000000,0.000000,0.000000}%
\pgfsetstrokecolor{currentstroke}%
\pgfsetdash{}{0pt}%
\pgfsys@defobject{currentmarker}{\pgfqpoint{-0.027778in}{0.000000in}}{\pgfqpoint{-0.000000in}{0.000000in}}{%
\pgfpathmoveto{\pgfqpoint{-0.000000in}{0.000000in}}%
\pgfpathlineto{\pgfqpoint{-0.027778in}{0.000000in}}%
\pgfusepath{stroke,fill}%
}%
\begin{pgfscope}%
\pgfsys@transformshift{0.721913in}{1.674083in}%
\pgfsys@useobject{currentmarker}{}%
\end{pgfscope}%
\end{pgfscope}%
\begin{pgfscope}%
\pgfpathrectangle{\pgfqpoint{0.721913in}{0.549073in}}{\pgfqpoint{2.325000in}{2.310000in}}%
\pgfusepath{clip}%
\pgfsetrectcap%
\pgfsetroundjoin%
\pgfsetlinewidth{0.250937pt}%
\definecolor{currentstroke}{rgb}{0.000000,0.000000,0.000000}%
\pgfsetstrokecolor{currentstroke}%
\pgfsetstrokeopacity{0.200000}%
\pgfsetdash{}{0pt}%
\pgfpathmoveto{\pgfqpoint{0.721913in}{1.736194in}}%
\pgfpathlineto{\pgfqpoint{3.046913in}{1.736194in}}%
\pgfusepath{stroke}%
\end{pgfscope}%
\begin{pgfscope}%
\pgfsetbuttcap%
\pgfsetroundjoin%
\definecolor{currentfill}{rgb}{0.000000,0.000000,0.000000}%
\pgfsetfillcolor{currentfill}%
\pgfsetlinewidth{0.602250pt}%
\definecolor{currentstroke}{rgb}{0.000000,0.000000,0.000000}%
\pgfsetstrokecolor{currentstroke}%
\pgfsetdash{}{0pt}%
\pgfsys@defobject{currentmarker}{\pgfqpoint{-0.027778in}{0.000000in}}{\pgfqpoint{-0.000000in}{0.000000in}}{%
\pgfpathmoveto{\pgfqpoint{-0.000000in}{0.000000in}}%
\pgfpathlineto{\pgfqpoint{-0.027778in}{0.000000in}}%
\pgfusepath{stroke,fill}%
}%
\begin{pgfscope}%
\pgfsys@transformshift{0.721913in}{1.736194in}%
\pgfsys@useobject{currentmarker}{}%
\end{pgfscope}%
\end{pgfscope}%
\begin{pgfscope}%
\pgfpathrectangle{\pgfqpoint{0.721913in}{0.549073in}}{\pgfqpoint{2.325000in}{2.310000in}}%
\pgfusepath{clip}%
\pgfsetrectcap%
\pgfsetroundjoin%
\pgfsetlinewidth{0.250937pt}%
\definecolor{currentstroke}{rgb}{0.000000,0.000000,0.000000}%
\pgfsetstrokecolor{currentstroke}%
\pgfsetstrokeopacity{0.200000}%
\pgfsetdash{}{0pt}%
\pgfpathmoveto{\pgfqpoint{0.721913in}{1.786943in}}%
\pgfpathlineto{\pgfqpoint{3.046913in}{1.786943in}}%
\pgfusepath{stroke}%
\end{pgfscope}%
\begin{pgfscope}%
\pgfsetbuttcap%
\pgfsetroundjoin%
\definecolor{currentfill}{rgb}{0.000000,0.000000,0.000000}%
\pgfsetfillcolor{currentfill}%
\pgfsetlinewidth{0.602250pt}%
\definecolor{currentstroke}{rgb}{0.000000,0.000000,0.000000}%
\pgfsetstrokecolor{currentstroke}%
\pgfsetdash{}{0pt}%
\pgfsys@defobject{currentmarker}{\pgfqpoint{-0.027778in}{0.000000in}}{\pgfqpoint{-0.000000in}{0.000000in}}{%
\pgfpathmoveto{\pgfqpoint{-0.000000in}{0.000000in}}%
\pgfpathlineto{\pgfqpoint{-0.027778in}{0.000000in}}%
\pgfusepath{stroke,fill}%
}%
\begin{pgfscope}%
\pgfsys@transformshift{0.721913in}{1.786943in}%
\pgfsys@useobject{currentmarker}{}%
\end{pgfscope}%
\end{pgfscope}%
\begin{pgfscope}%
\pgfpathrectangle{\pgfqpoint{0.721913in}{0.549073in}}{\pgfqpoint{2.325000in}{2.310000in}}%
\pgfusepath{clip}%
\pgfsetrectcap%
\pgfsetroundjoin%
\pgfsetlinewidth{0.250937pt}%
\definecolor{currentstroke}{rgb}{0.000000,0.000000,0.000000}%
\pgfsetstrokecolor{currentstroke}%
\pgfsetstrokeopacity{0.200000}%
\pgfsetdash{}{0pt}%
\pgfpathmoveto{\pgfqpoint{0.721913in}{1.829851in}}%
\pgfpathlineto{\pgfqpoint{3.046913in}{1.829851in}}%
\pgfusepath{stroke}%
\end{pgfscope}%
\begin{pgfscope}%
\pgfsetbuttcap%
\pgfsetroundjoin%
\definecolor{currentfill}{rgb}{0.000000,0.000000,0.000000}%
\pgfsetfillcolor{currentfill}%
\pgfsetlinewidth{0.602250pt}%
\definecolor{currentstroke}{rgb}{0.000000,0.000000,0.000000}%
\pgfsetstrokecolor{currentstroke}%
\pgfsetdash{}{0pt}%
\pgfsys@defobject{currentmarker}{\pgfqpoint{-0.027778in}{0.000000in}}{\pgfqpoint{-0.000000in}{0.000000in}}{%
\pgfpathmoveto{\pgfqpoint{-0.000000in}{0.000000in}}%
\pgfpathlineto{\pgfqpoint{-0.027778in}{0.000000in}}%
\pgfusepath{stroke,fill}%
}%
\begin{pgfscope}%
\pgfsys@transformshift{0.721913in}{1.829851in}%
\pgfsys@useobject{currentmarker}{}%
\end{pgfscope}%
\end{pgfscope}%
\begin{pgfscope}%
\pgfpathrectangle{\pgfqpoint{0.721913in}{0.549073in}}{\pgfqpoint{2.325000in}{2.310000in}}%
\pgfusepath{clip}%
\pgfsetrectcap%
\pgfsetroundjoin%
\pgfsetlinewidth{0.250937pt}%
\definecolor{currentstroke}{rgb}{0.000000,0.000000,0.000000}%
\pgfsetstrokecolor{currentstroke}%
\pgfsetstrokeopacity{0.200000}%
\pgfsetdash{}{0pt}%
\pgfpathmoveto{\pgfqpoint{0.721913in}{1.867019in}}%
\pgfpathlineto{\pgfqpoint{3.046913in}{1.867019in}}%
\pgfusepath{stroke}%
\end{pgfscope}%
\begin{pgfscope}%
\pgfsetbuttcap%
\pgfsetroundjoin%
\definecolor{currentfill}{rgb}{0.000000,0.000000,0.000000}%
\pgfsetfillcolor{currentfill}%
\pgfsetlinewidth{0.602250pt}%
\definecolor{currentstroke}{rgb}{0.000000,0.000000,0.000000}%
\pgfsetstrokecolor{currentstroke}%
\pgfsetdash{}{0pt}%
\pgfsys@defobject{currentmarker}{\pgfqpoint{-0.027778in}{0.000000in}}{\pgfqpoint{-0.000000in}{0.000000in}}{%
\pgfpathmoveto{\pgfqpoint{-0.000000in}{0.000000in}}%
\pgfpathlineto{\pgfqpoint{-0.027778in}{0.000000in}}%
\pgfusepath{stroke,fill}%
}%
\begin{pgfscope}%
\pgfsys@transformshift{0.721913in}{1.867019in}%
\pgfsys@useobject{currentmarker}{}%
\end{pgfscope}%
\end{pgfscope}%
\begin{pgfscope}%
\pgfpathrectangle{\pgfqpoint{0.721913in}{0.549073in}}{\pgfqpoint{2.325000in}{2.310000in}}%
\pgfusepath{clip}%
\pgfsetrectcap%
\pgfsetroundjoin%
\pgfsetlinewidth{0.250937pt}%
\definecolor{currentstroke}{rgb}{0.000000,0.000000,0.000000}%
\pgfsetstrokecolor{currentstroke}%
\pgfsetstrokeopacity{0.200000}%
\pgfsetdash{}{0pt}%
\pgfpathmoveto{\pgfqpoint{0.721913in}{1.899804in}}%
\pgfpathlineto{\pgfqpoint{3.046913in}{1.899804in}}%
\pgfusepath{stroke}%
\end{pgfscope}%
\begin{pgfscope}%
\pgfsetbuttcap%
\pgfsetroundjoin%
\definecolor{currentfill}{rgb}{0.000000,0.000000,0.000000}%
\pgfsetfillcolor{currentfill}%
\pgfsetlinewidth{0.602250pt}%
\definecolor{currentstroke}{rgb}{0.000000,0.000000,0.000000}%
\pgfsetstrokecolor{currentstroke}%
\pgfsetdash{}{0pt}%
\pgfsys@defobject{currentmarker}{\pgfqpoint{-0.027778in}{0.000000in}}{\pgfqpoint{-0.000000in}{0.000000in}}{%
\pgfpathmoveto{\pgfqpoint{-0.000000in}{0.000000in}}%
\pgfpathlineto{\pgfqpoint{-0.027778in}{0.000000in}}%
\pgfusepath{stroke,fill}%
}%
\begin{pgfscope}%
\pgfsys@transformshift{0.721913in}{1.899804in}%
\pgfsys@useobject{currentmarker}{}%
\end{pgfscope}%
\end{pgfscope}%
\begin{pgfscope}%
\pgfpathrectangle{\pgfqpoint{0.721913in}{0.549073in}}{\pgfqpoint{2.325000in}{2.310000in}}%
\pgfusepath{clip}%
\pgfsetrectcap%
\pgfsetroundjoin%
\pgfsetlinewidth{0.250937pt}%
\definecolor{currentstroke}{rgb}{0.000000,0.000000,0.000000}%
\pgfsetstrokecolor{currentstroke}%
\pgfsetstrokeopacity{0.200000}%
\pgfsetdash{}{0pt}%
\pgfpathmoveto{\pgfqpoint{0.721913in}{2.122067in}}%
\pgfpathlineto{\pgfqpoint{3.046913in}{2.122067in}}%
\pgfusepath{stroke}%
\end{pgfscope}%
\begin{pgfscope}%
\pgfsetbuttcap%
\pgfsetroundjoin%
\definecolor{currentfill}{rgb}{0.000000,0.000000,0.000000}%
\pgfsetfillcolor{currentfill}%
\pgfsetlinewidth{0.602250pt}%
\definecolor{currentstroke}{rgb}{0.000000,0.000000,0.000000}%
\pgfsetstrokecolor{currentstroke}%
\pgfsetdash{}{0pt}%
\pgfsys@defobject{currentmarker}{\pgfqpoint{-0.027778in}{0.000000in}}{\pgfqpoint{-0.000000in}{0.000000in}}{%
\pgfpathmoveto{\pgfqpoint{-0.000000in}{0.000000in}}%
\pgfpathlineto{\pgfqpoint{-0.027778in}{0.000000in}}%
\pgfusepath{stroke,fill}%
}%
\begin{pgfscope}%
\pgfsys@transformshift{0.721913in}{2.122067in}%
\pgfsys@useobject{currentmarker}{}%
\end{pgfscope}%
\end{pgfscope}%
\begin{pgfscope}%
\pgfpathrectangle{\pgfqpoint{0.721913in}{0.549073in}}{\pgfqpoint{2.325000in}{2.310000in}}%
\pgfusepath{clip}%
\pgfsetrectcap%
\pgfsetroundjoin%
\pgfsetlinewidth{0.250937pt}%
\definecolor{currentstroke}{rgb}{0.000000,0.000000,0.000000}%
\pgfsetstrokecolor{currentstroke}%
\pgfsetstrokeopacity{0.200000}%
\pgfsetdash{}{0pt}%
\pgfpathmoveto{\pgfqpoint{0.721913in}{2.234928in}}%
\pgfpathlineto{\pgfqpoint{3.046913in}{2.234928in}}%
\pgfusepath{stroke}%
\end{pgfscope}%
\begin{pgfscope}%
\pgfsetbuttcap%
\pgfsetroundjoin%
\definecolor{currentfill}{rgb}{0.000000,0.000000,0.000000}%
\pgfsetfillcolor{currentfill}%
\pgfsetlinewidth{0.602250pt}%
\definecolor{currentstroke}{rgb}{0.000000,0.000000,0.000000}%
\pgfsetstrokecolor{currentstroke}%
\pgfsetdash{}{0pt}%
\pgfsys@defobject{currentmarker}{\pgfqpoint{-0.027778in}{0.000000in}}{\pgfqpoint{-0.000000in}{0.000000in}}{%
\pgfpathmoveto{\pgfqpoint{-0.000000in}{0.000000in}}%
\pgfpathlineto{\pgfqpoint{-0.027778in}{0.000000in}}%
\pgfusepath{stroke,fill}%
}%
\begin{pgfscope}%
\pgfsys@transformshift{0.721913in}{2.234928in}%
\pgfsys@useobject{currentmarker}{}%
\end{pgfscope}%
\end{pgfscope}%
\begin{pgfscope}%
\pgfpathrectangle{\pgfqpoint{0.721913in}{0.549073in}}{\pgfqpoint{2.325000in}{2.310000in}}%
\pgfusepath{clip}%
\pgfsetrectcap%
\pgfsetroundjoin%
\pgfsetlinewidth{0.250937pt}%
\definecolor{currentstroke}{rgb}{0.000000,0.000000,0.000000}%
\pgfsetstrokecolor{currentstroke}%
\pgfsetstrokeopacity{0.200000}%
\pgfsetdash{}{0pt}%
\pgfpathmoveto{\pgfqpoint{0.721913in}{2.315004in}}%
\pgfpathlineto{\pgfqpoint{3.046913in}{2.315004in}}%
\pgfusepath{stroke}%
\end{pgfscope}%
\begin{pgfscope}%
\pgfsetbuttcap%
\pgfsetroundjoin%
\definecolor{currentfill}{rgb}{0.000000,0.000000,0.000000}%
\pgfsetfillcolor{currentfill}%
\pgfsetlinewidth{0.602250pt}%
\definecolor{currentstroke}{rgb}{0.000000,0.000000,0.000000}%
\pgfsetstrokecolor{currentstroke}%
\pgfsetdash{}{0pt}%
\pgfsys@defobject{currentmarker}{\pgfqpoint{-0.027778in}{0.000000in}}{\pgfqpoint{-0.000000in}{0.000000in}}{%
\pgfpathmoveto{\pgfqpoint{-0.000000in}{0.000000in}}%
\pgfpathlineto{\pgfqpoint{-0.027778in}{0.000000in}}%
\pgfusepath{stroke,fill}%
}%
\begin{pgfscope}%
\pgfsys@transformshift{0.721913in}{2.315004in}%
\pgfsys@useobject{currentmarker}{}%
\end{pgfscope}%
\end{pgfscope}%
\begin{pgfscope}%
\pgfpathrectangle{\pgfqpoint{0.721913in}{0.549073in}}{\pgfqpoint{2.325000in}{2.310000in}}%
\pgfusepath{clip}%
\pgfsetrectcap%
\pgfsetroundjoin%
\pgfsetlinewidth{0.250937pt}%
\definecolor{currentstroke}{rgb}{0.000000,0.000000,0.000000}%
\pgfsetstrokecolor{currentstroke}%
\pgfsetstrokeopacity{0.200000}%
\pgfsetdash{}{0pt}%
\pgfpathmoveto{\pgfqpoint{0.721913in}{2.377115in}}%
\pgfpathlineto{\pgfqpoint{3.046913in}{2.377115in}}%
\pgfusepath{stroke}%
\end{pgfscope}%
\begin{pgfscope}%
\pgfsetbuttcap%
\pgfsetroundjoin%
\definecolor{currentfill}{rgb}{0.000000,0.000000,0.000000}%
\pgfsetfillcolor{currentfill}%
\pgfsetlinewidth{0.602250pt}%
\definecolor{currentstroke}{rgb}{0.000000,0.000000,0.000000}%
\pgfsetstrokecolor{currentstroke}%
\pgfsetdash{}{0pt}%
\pgfsys@defobject{currentmarker}{\pgfqpoint{-0.027778in}{0.000000in}}{\pgfqpoint{-0.000000in}{0.000000in}}{%
\pgfpathmoveto{\pgfqpoint{-0.000000in}{0.000000in}}%
\pgfpathlineto{\pgfqpoint{-0.027778in}{0.000000in}}%
\pgfusepath{stroke,fill}%
}%
\begin{pgfscope}%
\pgfsys@transformshift{0.721913in}{2.377115in}%
\pgfsys@useobject{currentmarker}{}%
\end{pgfscope}%
\end{pgfscope}%
\begin{pgfscope}%
\pgfpathrectangle{\pgfqpoint{0.721913in}{0.549073in}}{\pgfqpoint{2.325000in}{2.310000in}}%
\pgfusepath{clip}%
\pgfsetrectcap%
\pgfsetroundjoin%
\pgfsetlinewidth{0.250937pt}%
\definecolor{currentstroke}{rgb}{0.000000,0.000000,0.000000}%
\pgfsetstrokecolor{currentstroke}%
\pgfsetstrokeopacity{0.200000}%
\pgfsetdash{}{0pt}%
\pgfpathmoveto{\pgfqpoint{0.721913in}{2.427864in}}%
\pgfpathlineto{\pgfqpoint{3.046913in}{2.427864in}}%
\pgfusepath{stroke}%
\end{pgfscope}%
\begin{pgfscope}%
\pgfsetbuttcap%
\pgfsetroundjoin%
\definecolor{currentfill}{rgb}{0.000000,0.000000,0.000000}%
\pgfsetfillcolor{currentfill}%
\pgfsetlinewidth{0.602250pt}%
\definecolor{currentstroke}{rgb}{0.000000,0.000000,0.000000}%
\pgfsetstrokecolor{currentstroke}%
\pgfsetdash{}{0pt}%
\pgfsys@defobject{currentmarker}{\pgfqpoint{-0.027778in}{0.000000in}}{\pgfqpoint{-0.000000in}{0.000000in}}{%
\pgfpathmoveto{\pgfqpoint{-0.000000in}{0.000000in}}%
\pgfpathlineto{\pgfqpoint{-0.027778in}{0.000000in}}%
\pgfusepath{stroke,fill}%
}%
\begin{pgfscope}%
\pgfsys@transformshift{0.721913in}{2.427864in}%
\pgfsys@useobject{currentmarker}{}%
\end{pgfscope}%
\end{pgfscope}%
\begin{pgfscope}%
\pgfpathrectangle{\pgfqpoint{0.721913in}{0.549073in}}{\pgfqpoint{2.325000in}{2.310000in}}%
\pgfusepath{clip}%
\pgfsetrectcap%
\pgfsetroundjoin%
\pgfsetlinewidth{0.250937pt}%
\definecolor{currentstroke}{rgb}{0.000000,0.000000,0.000000}%
\pgfsetstrokecolor{currentstroke}%
\pgfsetstrokeopacity{0.200000}%
\pgfsetdash{}{0pt}%
\pgfpathmoveto{\pgfqpoint{0.721913in}{2.470772in}}%
\pgfpathlineto{\pgfqpoint{3.046913in}{2.470772in}}%
\pgfusepath{stroke}%
\end{pgfscope}%
\begin{pgfscope}%
\pgfsetbuttcap%
\pgfsetroundjoin%
\definecolor{currentfill}{rgb}{0.000000,0.000000,0.000000}%
\pgfsetfillcolor{currentfill}%
\pgfsetlinewidth{0.602250pt}%
\definecolor{currentstroke}{rgb}{0.000000,0.000000,0.000000}%
\pgfsetstrokecolor{currentstroke}%
\pgfsetdash{}{0pt}%
\pgfsys@defobject{currentmarker}{\pgfqpoint{-0.027778in}{0.000000in}}{\pgfqpoint{-0.000000in}{0.000000in}}{%
\pgfpathmoveto{\pgfqpoint{-0.000000in}{0.000000in}}%
\pgfpathlineto{\pgfqpoint{-0.027778in}{0.000000in}}%
\pgfusepath{stroke,fill}%
}%
\begin{pgfscope}%
\pgfsys@transformshift{0.721913in}{2.470772in}%
\pgfsys@useobject{currentmarker}{}%
\end{pgfscope}%
\end{pgfscope}%
\begin{pgfscope}%
\pgfpathrectangle{\pgfqpoint{0.721913in}{0.549073in}}{\pgfqpoint{2.325000in}{2.310000in}}%
\pgfusepath{clip}%
\pgfsetrectcap%
\pgfsetroundjoin%
\pgfsetlinewidth{0.250937pt}%
\definecolor{currentstroke}{rgb}{0.000000,0.000000,0.000000}%
\pgfsetstrokecolor{currentstroke}%
\pgfsetstrokeopacity{0.200000}%
\pgfsetdash{}{0pt}%
\pgfpathmoveto{\pgfqpoint{0.721913in}{2.507940in}}%
\pgfpathlineto{\pgfqpoint{3.046913in}{2.507940in}}%
\pgfusepath{stroke}%
\end{pgfscope}%
\begin{pgfscope}%
\pgfsetbuttcap%
\pgfsetroundjoin%
\definecolor{currentfill}{rgb}{0.000000,0.000000,0.000000}%
\pgfsetfillcolor{currentfill}%
\pgfsetlinewidth{0.602250pt}%
\definecolor{currentstroke}{rgb}{0.000000,0.000000,0.000000}%
\pgfsetstrokecolor{currentstroke}%
\pgfsetdash{}{0pt}%
\pgfsys@defobject{currentmarker}{\pgfqpoint{-0.027778in}{0.000000in}}{\pgfqpoint{-0.000000in}{0.000000in}}{%
\pgfpathmoveto{\pgfqpoint{-0.000000in}{0.000000in}}%
\pgfpathlineto{\pgfqpoint{-0.027778in}{0.000000in}}%
\pgfusepath{stroke,fill}%
}%
\begin{pgfscope}%
\pgfsys@transformshift{0.721913in}{2.507940in}%
\pgfsys@useobject{currentmarker}{}%
\end{pgfscope}%
\end{pgfscope}%
\begin{pgfscope}%
\pgfpathrectangle{\pgfqpoint{0.721913in}{0.549073in}}{\pgfqpoint{2.325000in}{2.310000in}}%
\pgfusepath{clip}%
\pgfsetrectcap%
\pgfsetroundjoin%
\pgfsetlinewidth{0.250937pt}%
\definecolor{currentstroke}{rgb}{0.000000,0.000000,0.000000}%
\pgfsetstrokecolor{currentstroke}%
\pgfsetstrokeopacity{0.200000}%
\pgfsetdash{}{0pt}%
\pgfpathmoveto{\pgfqpoint{0.721913in}{2.540725in}}%
\pgfpathlineto{\pgfqpoint{3.046913in}{2.540725in}}%
\pgfusepath{stroke}%
\end{pgfscope}%
\begin{pgfscope}%
\pgfsetbuttcap%
\pgfsetroundjoin%
\definecolor{currentfill}{rgb}{0.000000,0.000000,0.000000}%
\pgfsetfillcolor{currentfill}%
\pgfsetlinewidth{0.602250pt}%
\definecolor{currentstroke}{rgb}{0.000000,0.000000,0.000000}%
\pgfsetstrokecolor{currentstroke}%
\pgfsetdash{}{0pt}%
\pgfsys@defobject{currentmarker}{\pgfqpoint{-0.027778in}{0.000000in}}{\pgfqpoint{-0.000000in}{0.000000in}}{%
\pgfpathmoveto{\pgfqpoint{-0.000000in}{0.000000in}}%
\pgfpathlineto{\pgfqpoint{-0.027778in}{0.000000in}}%
\pgfusepath{stroke,fill}%
}%
\begin{pgfscope}%
\pgfsys@transformshift{0.721913in}{2.540725in}%
\pgfsys@useobject{currentmarker}{}%
\end{pgfscope}%
\end{pgfscope}%
\begin{pgfscope}%
\pgfpathrectangle{\pgfqpoint{0.721913in}{0.549073in}}{\pgfqpoint{2.325000in}{2.310000in}}%
\pgfusepath{clip}%
\pgfsetrectcap%
\pgfsetroundjoin%
\pgfsetlinewidth{0.250937pt}%
\definecolor{currentstroke}{rgb}{0.000000,0.000000,0.000000}%
\pgfsetstrokecolor{currentstroke}%
\pgfsetstrokeopacity{0.200000}%
\pgfsetdash{}{0pt}%
\pgfpathmoveto{\pgfqpoint{0.721913in}{2.762988in}}%
\pgfpathlineto{\pgfqpoint{3.046913in}{2.762988in}}%
\pgfusepath{stroke}%
\end{pgfscope}%
\begin{pgfscope}%
\pgfsetbuttcap%
\pgfsetroundjoin%
\definecolor{currentfill}{rgb}{0.000000,0.000000,0.000000}%
\pgfsetfillcolor{currentfill}%
\pgfsetlinewidth{0.602250pt}%
\definecolor{currentstroke}{rgb}{0.000000,0.000000,0.000000}%
\pgfsetstrokecolor{currentstroke}%
\pgfsetdash{}{0pt}%
\pgfsys@defobject{currentmarker}{\pgfqpoint{-0.027778in}{0.000000in}}{\pgfqpoint{-0.000000in}{0.000000in}}{%
\pgfpathmoveto{\pgfqpoint{-0.000000in}{0.000000in}}%
\pgfpathlineto{\pgfqpoint{-0.027778in}{0.000000in}}%
\pgfusepath{stroke,fill}%
}%
\begin{pgfscope}%
\pgfsys@transformshift{0.721913in}{2.762988in}%
\pgfsys@useobject{currentmarker}{}%
\end{pgfscope}%
\end{pgfscope}%
\begin{pgfscope}%
\definecolor{textcolor}{rgb}{0.000000,0.000000,0.000000}%
\pgfsetstrokecolor{textcolor}%
\pgfsetfillcolor{textcolor}%
\pgftext[x=0.248147in,y=1.704073in,,bottom,rotate=90.000000]{\color{textcolor}{\rmfamily\fontsize{12.000000}{14.400000}\selectfont\catcode`\^=\active\def^{\ifmmode\sp\else\^{}\fi}\catcode`\%=\active\def%{\%}$L^1$-error}}%
\end{pgfscope}%
\begin{pgfscope}%
\pgfpathrectangle{\pgfqpoint{0.721913in}{0.549073in}}{\pgfqpoint{2.325000in}{2.310000in}}%
\pgfusepath{clip}%
\pgfsetrectcap%
\pgfsetroundjoin%
\pgfsetlinewidth{1.505625pt}%
\definecolor{currentstroke}{rgb}{1.000000,0.690196,0.000000}%
\pgfsetstrokecolor{currentstroke}%
\pgfsetdash{}{0pt}%
\pgfpathmoveto{\pgfqpoint{0.777271in}{2.319601in}}%
\pgfpathlineto{\pgfqpoint{1.146318in}{2.311660in}}%
\pgfpathlineto{\pgfqpoint{1.515366in}{2.273746in}}%
\pgfpathlineto{\pgfqpoint{1.884413in}{2.244715in}}%
\pgfpathlineto{\pgfqpoint{2.253461in}{2.148673in}}%
\pgfpathlineto{\pgfqpoint{2.622509in}{2.028923in}}%
\pgfpathlineto{\pgfqpoint{2.991556in}{1.956603in}}%
\pgfusepath{stroke}%
\end{pgfscope}%
\begin{pgfscope}%
\pgfpathrectangle{\pgfqpoint{0.721913in}{0.549073in}}{\pgfqpoint{2.325000in}{2.310000in}}%
\pgfusepath{clip}%
\pgfsetbuttcap%
\pgfsetmiterjoin%
\definecolor{currentfill}{rgb}{1.000000,0.690196,0.000000}%
\pgfsetfillcolor{currentfill}%
\pgfsetlinewidth{1.003750pt}%
\definecolor{currentstroke}{rgb}{1.000000,0.690196,0.000000}%
\pgfsetstrokecolor{currentstroke}%
\pgfsetdash{}{0pt}%
\pgfsys@defobject{currentmarker}{\pgfqpoint{-0.035355in}{-0.058926in}}{\pgfqpoint{0.035355in}{0.058926in}}{%
\pgfpathmoveto{\pgfqpoint{-0.000000in}{-0.058926in}}%
\pgfpathlineto{\pgfqpoint{0.035355in}{0.000000in}}%
\pgfpathlineto{\pgfqpoint{0.000000in}{0.058926in}}%
\pgfpathlineto{\pgfqpoint{-0.035355in}{0.000000in}}%
\pgfpathlineto{\pgfqpoint{-0.000000in}{-0.058926in}}%
\pgfpathclose%
\pgfusepath{stroke,fill}%
}%
\begin{pgfscope}%
\pgfsys@transformshift{0.777271in}{2.319601in}%
\pgfsys@useobject{currentmarker}{}%
\end{pgfscope}%
\begin{pgfscope}%
\pgfsys@transformshift{1.146318in}{2.311660in}%
\pgfsys@useobject{currentmarker}{}%
\end{pgfscope}%
\begin{pgfscope}%
\pgfsys@transformshift{1.515366in}{2.273746in}%
\pgfsys@useobject{currentmarker}{}%
\end{pgfscope}%
\begin{pgfscope}%
\pgfsys@transformshift{1.884413in}{2.244715in}%
\pgfsys@useobject{currentmarker}{}%
\end{pgfscope}%
\begin{pgfscope}%
\pgfsys@transformshift{2.253461in}{2.148673in}%
\pgfsys@useobject{currentmarker}{}%
\end{pgfscope}%
\begin{pgfscope}%
\pgfsys@transformshift{2.622509in}{2.028923in}%
\pgfsys@useobject{currentmarker}{}%
\end{pgfscope}%
\begin{pgfscope}%
\pgfsys@transformshift{2.991556in}{1.956603in}%
\pgfsys@useobject{currentmarker}{}%
\end{pgfscope}%
\end{pgfscope}%
\begin{pgfscope}%
\pgfpathrectangle{\pgfqpoint{0.721913in}{0.549073in}}{\pgfqpoint{2.325000in}{2.310000in}}%
\pgfusepath{clip}%
\pgfsetrectcap%
\pgfsetroundjoin%
\pgfsetlinewidth{1.505625pt}%
\definecolor{currentstroke}{rgb}{0.996078,0.380392,0.000000}%
\pgfsetstrokecolor{currentstroke}%
\pgfsetdash{}{0pt}%
\pgfpathmoveto{\pgfqpoint{0.777271in}{2.666573in}}%
\pgfpathlineto{\pgfqpoint{1.146318in}{2.484348in}}%
\pgfpathlineto{\pgfqpoint{1.515366in}{2.345954in}}%
\pgfpathlineto{\pgfqpoint{1.884413in}{2.141982in}}%
\pgfpathlineto{\pgfqpoint{2.253461in}{1.787536in}}%
\pgfpathlineto{\pgfqpoint{2.622509in}{1.308344in}}%
\pgfpathlineto{\pgfqpoint{2.991556in}{1.057695in}}%
\pgfusepath{stroke}%
\end{pgfscope}%
\begin{pgfscope}%
\pgfpathrectangle{\pgfqpoint{0.721913in}{0.549073in}}{\pgfqpoint{2.325000in}{2.310000in}}%
\pgfusepath{clip}%
\pgfsetbuttcap%
\pgfsetmiterjoin%
\definecolor{currentfill}{rgb}{0.996078,0.380392,0.000000}%
\pgfsetfillcolor{currentfill}%
\pgfsetlinewidth{1.003750pt}%
\definecolor{currentstroke}{rgb}{0.996078,0.380392,0.000000}%
\pgfsetstrokecolor{currentstroke}%
\pgfsetdash{}{0pt}%
\pgfsys@defobject{currentmarker}{\pgfqpoint{-0.039627in}{-0.033709in}}{\pgfqpoint{0.039627in}{0.041667in}}{%
\pgfpathmoveto{\pgfqpoint{0.000000in}{0.041667in}}%
\pgfpathlineto{\pgfqpoint{-0.039627in}{0.012876in}}%
\pgfpathlineto{\pgfqpoint{-0.024491in}{-0.033709in}}%
\pgfpathlineto{\pgfqpoint{0.024491in}{-0.033709in}}%
\pgfpathlineto{\pgfqpoint{0.039627in}{0.012876in}}%
\pgfpathlineto{\pgfqpoint{0.000000in}{0.041667in}}%
\pgfpathclose%
\pgfusepath{stroke,fill}%
}%
\begin{pgfscope}%
\pgfsys@transformshift{0.777271in}{2.666573in}%
\pgfsys@useobject{currentmarker}{}%
\end{pgfscope}%
\begin{pgfscope}%
\pgfsys@transformshift{1.146318in}{2.484348in}%
\pgfsys@useobject{currentmarker}{}%
\end{pgfscope}%
\begin{pgfscope}%
\pgfsys@transformshift{1.515366in}{2.345954in}%
\pgfsys@useobject{currentmarker}{}%
\end{pgfscope}%
\begin{pgfscope}%
\pgfsys@transformshift{1.884413in}{2.141982in}%
\pgfsys@useobject{currentmarker}{}%
\end{pgfscope}%
\begin{pgfscope}%
\pgfsys@transformshift{2.253461in}{1.787536in}%
\pgfsys@useobject{currentmarker}{}%
\end{pgfscope}%
\begin{pgfscope}%
\pgfsys@transformshift{2.622509in}{1.308344in}%
\pgfsys@useobject{currentmarker}{}%
\end{pgfscope}%
\begin{pgfscope}%
\pgfsys@transformshift{2.991556in}{1.057695in}%
\pgfsys@useobject{currentmarker}{}%
\end{pgfscope}%
\end{pgfscope}%
\begin{pgfscope}%
\pgfpathrectangle{\pgfqpoint{0.721913in}{0.549073in}}{\pgfqpoint{2.325000in}{2.310000in}}%
\pgfusepath{clip}%
\pgfsetrectcap%
\pgfsetroundjoin%
\pgfsetlinewidth{1.505625pt}%
\definecolor{currentstroke}{rgb}{0.470588,0.368627,0.941176}%
\pgfsetstrokecolor{currentstroke}%
\pgfsetdash{}{0pt}%
\pgfpathmoveto{\pgfqpoint{0.777271in}{2.641399in}}%
\pgfpathlineto{\pgfqpoint{1.146318in}{2.497398in}}%
\pgfpathlineto{\pgfqpoint{1.515366in}{2.356308in}}%
\pgfpathlineto{\pgfqpoint{1.884413in}{2.074123in}}%
\pgfpathlineto{\pgfqpoint{2.253461in}{1.775551in}}%
\pgfpathlineto{\pgfqpoint{2.622509in}{1.417490in}}%
\pgfpathlineto{\pgfqpoint{2.991556in}{1.357921in}}%
\pgfusepath{stroke}%
\end{pgfscope}%
\begin{pgfscope}%
\pgfpathrectangle{\pgfqpoint{0.721913in}{0.549073in}}{\pgfqpoint{2.325000in}{2.310000in}}%
\pgfusepath{clip}%
\pgfsetbuttcap%
\pgfsetmiterjoin%
\definecolor{currentfill}{rgb}{0.470588,0.368627,0.941176}%
\pgfsetfillcolor{currentfill}%
\pgfsetlinewidth{1.003750pt}%
\definecolor{currentstroke}{rgb}{0.470588,0.368627,0.941176}%
\pgfsetstrokecolor{currentstroke}%
\pgfsetdash{}{0pt}%
\pgfsys@defobject{currentmarker}{\pgfqpoint{-0.041667in}{-0.041667in}}{\pgfqpoint{0.041667in}{0.041667in}}{%
\pgfpathmoveto{\pgfqpoint{0.000000in}{0.041667in}}%
\pgfpathlineto{\pgfqpoint{-0.041667in}{-0.041667in}}%
\pgfpathlineto{\pgfqpoint{0.041667in}{-0.041667in}}%
\pgfpathlineto{\pgfqpoint{0.000000in}{0.041667in}}%
\pgfpathclose%
\pgfusepath{stroke,fill}%
}%
\begin{pgfscope}%
\pgfsys@transformshift{0.777271in}{2.641399in}%
\pgfsys@useobject{currentmarker}{}%
\end{pgfscope}%
\begin{pgfscope}%
\pgfsys@transformshift{1.146318in}{2.497398in}%
\pgfsys@useobject{currentmarker}{}%
\end{pgfscope}%
\begin{pgfscope}%
\pgfsys@transformshift{1.515366in}{2.356308in}%
\pgfsys@useobject{currentmarker}{}%
\end{pgfscope}%
\begin{pgfscope}%
\pgfsys@transformshift{1.884413in}{2.074123in}%
\pgfsys@useobject{currentmarker}{}%
\end{pgfscope}%
\begin{pgfscope}%
\pgfsys@transformshift{2.253461in}{1.775551in}%
\pgfsys@useobject{currentmarker}{}%
\end{pgfscope}%
\begin{pgfscope}%
\pgfsys@transformshift{2.622509in}{1.417490in}%
\pgfsys@useobject{currentmarker}{}%
\end{pgfscope}%
\begin{pgfscope}%
\pgfsys@transformshift{2.991556in}{1.357921in}%
\pgfsys@useobject{currentmarker}{}%
\end{pgfscope}%
\end{pgfscope}%
\begin{pgfscope}%
\pgfpathrectangle{\pgfqpoint{0.721913in}{0.549073in}}{\pgfqpoint{2.325000in}{2.310000in}}%
\pgfusepath{clip}%
\pgfsetrectcap%
\pgfsetroundjoin%
\pgfsetlinewidth{1.505625pt}%
\definecolor{currentstroke}{rgb}{0.392157,0.560784,1.000000}%
\pgfsetstrokecolor{currentstroke}%
\pgfsetdash{}{0pt}%
\pgfpathmoveto{\pgfqpoint{0.777271in}{2.291555in}}%
\pgfpathlineto{\pgfqpoint{1.146318in}{2.273614in}}%
\pgfpathlineto{\pgfqpoint{1.515366in}{2.189084in}}%
\pgfpathlineto{\pgfqpoint{1.884413in}{2.276410in}}%
\pgfpathlineto{\pgfqpoint{2.253461in}{2.273980in}}%
\pgfpathlineto{\pgfqpoint{2.622509in}{2.258660in}}%
\pgfpathlineto{\pgfqpoint{2.991556in}{2.258660in}}%
\pgfusepath{stroke}%
\end{pgfscope}%
\begin{pgfscope}%
\pgfpathrectangle{\pgfqpoint{0.721913in}{0.549073in}}{\pgfqpoint{2.325000in}{2.310000in}}%
\pgfusepath{clip}%
\pgfsetbuttcap%
\pgfsetroundjoin%
\definecolor{currentfill}{rgb}{0.392157,0.560784,1.000000}%
\pgfsetfillcolor{currentfill}%
\pgfsetlinewidth{1.003750pt}%
\definecolor{currentstroke}{rgb}{0.392157,0.560784,1.000000}%
\pgfsetstrokecolor{currentstroke}%
\pgfsetdash{}{0pt}%
\pgfsys@defobject{currentmarker}{\pgfqpoint{-0.041667in}{-0.041667in}}{\pgfqpoint{0.041667in}{0.041667in}}{%
\pgfpathmoveto{\pgfqpoint{0.000000in}{-0.041667in}}%
\pgfpathcurveto{\pgfqpoint{0.011050in}{-0.041667in}}{\pgfqpoint{0.021649in}{-0.037276in}}{\pgfqpoint{0.029463in}{-0.029463in}}%
\pgfpathcurveto{\pgfqpoint{0.037276in}{-0.021649in}}{\pgfqpoint{0.041667in}{-0.011050in}}{\pgfqpoint{0.041667in}{0.000000in}}%
\pgfpathcurveto{\pgfqpoint{0.041667in}{0.011050in}}{\pgfqpoint{0.037276in}{0.021649in}}{\pgfqpoint{0.029463in}{0.029463in}}%
\pgfpathcurveto{\pgfqpoint{0.021649in}{0.037276in}}{\pgfqpoint{0.011050in}{0.041667in}}{\pgfqpoint{0.000000in}{0.041667in}}%
\pgfpathcurveto{\pgfqpoint{-0.011050in}{0.041667in}}{\pgfqpoint{-0.021649in}{0.037276in}}{\pgfqpoint{-0.029463in}{0.029463in}}%
\pgfpathcurveto{\pgfqpoint{-0.037276in}{0.021649in}}{\pgfqpoint{-0.041667in}{0.011050in}}{\pgfqpoint{-0.041667in}{0.000000in}}%
\pgfpathcurveto{\pgfqpoint{-0.041667in}{-0.011050in}}{\pgfqpoint{-0.037276in}{-0.021649in}}{\pgfqpoint{-0.029463in}{-0.029463in}}%
\pgfpathcurveto{\pgfqpoint{-0.021649in}{-0.037276in}}{\pgfqpoint{-0.011050in}{-0.041667in}}{\pgfqpoint{0.000000in}{-0.041667in}}%
\pgfpathlineto{\pgfqpoint{0.000000in}{-0.041667in}}%
\pgfpathclose%
\pgfusepath{stroke,fill}%
}%
\begin{pgfscope}%
\pgfsys@transformshift{0.777271in}{2.291555in}%
\pgfsys@useobject{currentmarker}{}%
\end{pgfscope}%
\begin{pgfscope}%
\pgfsys@transformshift{1.146318in}{2.273614in}%
\pgfsys@useobject{currentmarker}{}%
\end{pgfscope}%
\begin{pgfscope}%
\pgfsys@transformshift{1.515366in}{2.189084in}%
\pgfsys@useobject{currentmarker}{}%
\end{pgfscope}%
\begin{pgfscope}%
\pgfsys@transformshift{1.884413in}{2.276410in}%
\pgfsys@useobject{currentmarker}{}%
\end{pgfscope}%
\begin{pgfscope}%
\pgfsys@transformshift{2.253461in}{2.273980in}%
\pgfsys@useobject{currentmarker}{}%
\end{pgfscope}%
\begin{pgfscope}%
\pgfsys@transformshift{2.622509in}{2.258660in}%
\pgfsys@useobject{currentmarker}{}%
\end{pgfscope}%
\begin{pgfscope}%
\pgfsys@transformshift{2.991556in}{2.258660in}%
\pgfsys@useobject{currentmarker}{}%
\end{pgfscope}%
\end{pgfscope}%
\begin{pgfscope}%
\pgfpathrectangle{\pgfqpoint{0.721913in}{0.549073in}}{\pgfqpoint{2.325000in}{2.310000in}}%
\pgfusepath{clip}%
\pgfsetrectcap%
\pgfsetroundjoin%
\pgfsetlinewidth{1.505625pt}%
\definecolor{currentstroke}{rgb}{0.862745,0.149020,0.498039}%
\pgfsetstrokecolor{currentstroke}%
\pgfsetdash{}{0pt}%
\pgfpathmoveto{\pgfqpoint{0.777271in}{2.627731in}}%
\pgfpathlineto{\pgfqpoint{1.146318in}{2.129478in}}%
\pgfpathlineto{\pgfqpoint{1.515366in}{0.741573in}}%
\pgfpathlineto{\pgfqpoint{1.884413in}{1.185921in}}%
\pgfpathlineto{\pgfqpoint{2.253461in}{1.203630in}}%
\pgfpathlineto{\pgfqpoint{2.622509in}{1.293039in}}%
\pgfpathlineto{\pgfqpoint{2.991556in}{1.365084in}}%
\pgfusepath{stroke}%
\end{pgfscope}%
\begin{pgfscope}%
\pgfpathrectangle{\pgfqpoint{0.721913in}{0.549073in}}{\pgfqpoint{2.325000in}{2.310000in}}%
\pgfusepath{clip}%
\pgfsetbuttcap%
\pgfsetmiterjoin%
\definecolor{currentfill}{rgb}{0.862745,0.149020,0.498039}%
\pgfsetfillcolor{currentfill}%
\pgfsetlinewidth{1.003750pt}%
\definecolor{currentstroke}{rgb}{0.862745,0.149020,0.498039}%
\pgfsetstrokecolor{currentstroke}%
\pgfsetdash{}{0pt}%
\pgfsys@defobject{currentmarker}{\pgfqpoint{-0.041667in}{-0.041667in}}{\pgfqpoint{0.041667in}{0.041667in}}{%
\pgfpathmoveto{\pgfqpoint{-0.041667in}{-0.041667in}}%
\pgfpathlineto{\pgfqpoint{0.041667in}{-0.041667in}}%
\pgfpathlineto{\pgfqpoint{0.041667in}{0.041667in}}%
\pgfpathlineto{\pgfqpoint{-0.041667in}{0.041667in}}%
\pgfpathlineto{\pgfqpoint{-0.041667in}{-0.041667in}}%
\pgfpathclose%
\pgfusepath{stroke,fill}%
}%
\begin{pgfscope}%
\pgfsys@transformshift{0.777271in}{2.627731in}%
\pgfsys@useobject{currentmarker}{}%
\end{pgfscope}%
\begin{pgfscope}%
\pgfsys@transformshift{1.146318in}{2.129478in}%
\pgfsys@useobject{currentmarker}{}%
\end{pgfscope}%
\begin{pgfscope}%
\pgfsys@transformshift{1.515366in}{0.741573in}%
\pgfsys@useobject{currentmarker}{}%
\end{pgfscope}%
\begin{pgfscope}%
\pgfsys@transformshift{1.884413in}{1.185921in}%
\pgfsys@useobject{currentmarker}{}%
\end{pgfscope}%
\begin{pgfscope}%
\pgfsys@transformshift{2.253461in}{1.203630in}%
\pgfsys@useobject{currentmarker}{}%
\end{pgfscope}%
\begin{pgfscope}%
\pgfsys@transformshift{2.622509in}{1.293039in}%
\pgfsys@useobject{currentmarker}{}%
\end{pgfscope}%
\begin{pgfscope}%
\pgfsys@transformshift{2.991556in}{1.365084in}%
\pgfsys@useobject{currentmarker}{}%
\end{pgfscope}%
\end{pgfscope}%
\begin{pgfscope}%
\pgfsetrectcap%
\pgfsetmiterjoin%
\pgfsetlinewidth{0.803000pt}%
\definecolor{currentstroke}{rgb}{0.000000,0.000000,0.000000}%
\pgfsetstrokecolor{currentstroke}%
\pgfsetdash{}{0pt}%
\pgfpathmoveto{\pgfqpoint{0.721913in}{0.549073in}}%
\pgfpathlineto{\pgfqpoint{0.721913in}{2.859073in}}%
\pgfusepath{stroke}%
\end{pgfscope}%
\begin{pgfscope}%
\pgfsetrectcap%
\pgfsetmiterjoin%
\pgfsetlinewidth{0.803000pt}%
\definecolor{currentstroke}{rgb}{0.000000,0.000000,0.000000}%
\pgfsetstrokecolor{currentstroke}%
\pgfsetdash{}{0pt}%
\pgfpathmoveto{\pgfqpoint{3.046913in}{0.549073in}}%
\pgfpathlineto{\pgfqpoint{3.046913in}{2.859073in}}%
\pgfusepath{stroke}%
\end{pgfscope}%
\begin{pgfscope}%
\pgfsetrectcap%
\pgfsetmiterjoin%
\pgfsetlinewidth{0.803000pt}%
\definecolor{currentstroke}{rgb}{0.000000,0.000000,0.000000}%
\pgfsetstrokecolor{currentstroke}%
\pgfsetdash{}{0pt}%
\pgfpathmoveto{\pgfqpoint{0.721913in}{0.549073in}}%
\pgfpathlineto{\pgfqpoint{3.046913in}{0.549073in}}%
\pgfusepath{stroke}%
\end{pgfscope}%
\begin{pgfscope}%
\pgfsetrectcap%
\pgfsetmiterjoin%
\pgfsetlinewidth{0.803000pt}%
\definecolor{currentstroke}{rgb}{0.000000,0.000000,0.000000}%
\pgfsetstrokecolor{currentstroke}%
\pgfsetdash{}{0pt}%
\pgfpathmoveto{\pgfqpoint{0.721913in}{2.859073in}}%
\pgfpathlineto{\pgfqpoint{3.046913in}{2.859073in}}%
\pgfusepath{stroke}%
\end{pgfscope}%
\begin{pgfscope}%
\pgfsetbuttcap%
\pgfsetmiterjoin%
\definecolor{currentfill}{rgb}{1.000000,1.000000,1.000000}%
\pgfsetfillcolor{currentfill}%
\pgfsetfillopacity{0.800000}%
\pgfsetlinewidth{1.003750pt}%
\definecolor{currentstroke}{rgb}{0.800000,0.800000,0.800000}%
\pgfsetstrokecolor{currentstroke}%
\pgfsetstrokeopacity{0.800000}%
\pgfsetdash{}{0pt}%
\pgfpathmoveto{\pgfqpoint{0.805247in}{0.632406in}}%
\pgfpathlineto{\pgfqpoint{2.021463in}{0.632406in}}%
\pgfpathlineto{\pgfqpoint{2.021463in}{1.914813in}}%
\pgfpathlineto{\pgfqpoint{0.805247in}{1.914813in}}%
\pgfpathlineto{\pgfqpoint{0.805247in}{0.632406in}}%
\pgfpathclose%
\pgfusepath{stroke,fill}%
\end{pgfscope}%
\begin{pgfscope}%
\pgfsetrectcap%
\pgfsetroundjoin%
\pgfsetlinewidth{1.505625pt}%
\definecolor{currentstroke}{rgb}{1.000000,0.690196,0.000000}%
\pgfsetstrokecolor{currentstroke}%
\pgfsetdash{}{0pt}%
\pgfpathmoveto{\pgfqpoint{0.871913in}{1.781480in}}%
\pgfpathlineto{\pgfqpoint{1.038580in}{1.781480in}}%
\pgfpathlineto{\pgfqpoint{1.205247in}{1.781480in}}%
\pgfusepath{stroke}%
\end{pgfscope}%
\begin{pgfscope}%
\pgfsetbuttcap%
\pgfsetmiterjoin%
\definecolor{currentfill}{rgb}{1.000000,0.690196,0.000000}%
\pgfsetfillcolor{currentfill}%
\pgfsetlinewidth{1.003750pt}%
\definecolor{currentstroke}{rgb}{1.000000,0.690196,0.000000}%
\pgfsetstrokecolor{currentstroke}%
\pgfsetdash{}{0pt}%
\pgfsys@defobject{currentmarker}{\pgfqpoint{-0.026517in}{-0.044194in}}{\pgfqpoint{0.026517in}{0.044194in}}{%
\pgfpathmoveto{\pgfqpoint{-0.000000in}{-0.044194in}}%
\pgfpathlineto{\pgfqpoint{0.026517in}{0.000000in}}%
\pgfpathlineto{\pgfqpoint{0.000000in}{0.044194in}}%
\pgfpathlineto{\pgfqpoint{-0.026517in}{0.000000in}}%
\pgfpathlineto{\pgfqpoint{-0.000000in}{-0.044194in}}%
\pgfpathclose%
\pgfusepath{stroke,fill}%
}%
\begin{pgfscope}%
\pgfsys@transformshift{1.038580in}{1.781480in}%
\pgfsys@useobject{currentmarker}{}%
\end{pgfscope}%
\end{pgfscope}%
\begin{pgfscope}%
\definecolor{textcolor}{rgb}{0.000000,0.000000,0.000000}%
\pgfsetstrokecolor{textcolor}%
\pgfsetfillcolor{textcolor}%
\pgftext[x=1.338580in,y=1.723147in,left,base]{\color{textcolor}{\rmfamily\fontsize{12.000000}{14.400000}\selectfont\catcode`\^=\active\def^{\ifmmode\sp\else\^{}\fi}\catcode`\%=\active\def%{\%}KA (I)}}%
\end{pgfscope}%
\begin{pgfscope}%
\pgfsetrectcap%
\pgfsetroundjoin%
\pgfsetlinewidth{1.505625pt}%
\definecolor{currentstroke}{rgb}{0.996078,0.380392,0.000000}%
\pgfsetstrokecolor{currentstroke}%
\pgfsetdash{}{0pt}%
\pgfpathmoveto{\pgfqpoint{0.871913in}{1.531480in}}%
\pgfpathlineto{\pgfqpoint{1.038580in}{1.531480in}}%
\pgfpathlineto{\pgfqpoint{1.205247in}{1.531480in}}%
\pgfusepath{stroke}%
\end{pgfscope}%
\begin{pgfscope}%
\pgfsetbuttcap%
\pgfsetmiterjoin%
\definecolor{currentfill}{rgb}{0.996078,0.380392,0.000000}%
\pgfsetfillcolor{currentfill}%
\pgfsetlinewidth{1.003750pt}%
\definecolor{currentstroke}{rgb}{0.996078,0.380392,0.000000}%
\pgfsetstrokecolor{currentstroke}%
\pgfsetdash{}{0pt}%
\pgfsys@defobject{currentmarker}{\pgfqpoint{-0.029721in}{-0.025282in}}{\pgfqpoint{0.029721in}{0.031250in}}{%
\pgfpathmoveto{\pgfqpoint{0.000000in}{0.031250in}}%
\pgfpathlineto{\pgfqpoint{-0.029721in}{0.009657in}}%
\pgfpathlineto{\pgfqpoint{-0.018368in}{-0.025282in}}%
\pgfpathlineto{\pgfqpoint{0.018368in}{-0.025282in}}%
\pgfpathlineto{\pgfqpoint{0.029721in}{0.009657in}}%
\pgfpathlineto{\pgfqpoint{0.000000in}{0.031250in}}%
\pgfpathclose%
\pgfusepath{stroke,fill}%
}%
\begin{pgfscope}%
\pgfsys@transformshift{1.038580in}{1.531480in}%
\pgfsys@useobject{currentmarker}{}%
\end{pgfscope}%
\end{pgfscope}%
\begin{pgfscope}%
\definecolor{textcolor}{rgb}{0.000000,0.000000,0.000000}%
\pgfsetstrokecolor{textcolor}%
\pgfsetfillcolor{textcolor}%
\pgftext[x=1.338580in,y=1.473147in,left,base]{\color{textcolor}{\rmfamily\fontsize{12.000000}{14.400000}\selectfont\catcode`\^=\active\def^{\ifmmode\sp\else\^{}\fi}\catcode`\%=\active\def%{\%}KA (II)}}%
\end{pgfscope}%
\begin{pgfscope}%
\pgfsetrectcap%
\pgfsetroundjoin%
\pgfsetlinewidth{1.505625pt}%
\definecolor{currentstroke}{rgb}{0.470588,0.368627,0.941176}%
\pgfsetstrokecolor{currentstroke}%
\pgfsetdash{}{0pt}%
\pgfpathmoveto{\pgfqpoint{0.871913in}{1.281480in}}%
\pgfpathlineto{\pgfqpoint{1.038580in}{1.281480in}}%
\pgfpathlineto{\pgfqpoint{1.205247in}{1.281480in}}%
\pgfusepath{stroke}%
\end{pgfscope}%
\begin{pgfscope}%
\pgfsetbuttcap%
\pgfsetmiterjoin%
\definecolor{currentfill}{rgb}{0.470588,0.368627,0.941176}%
\pgfsetfillcolor{currentfill}%
\pgfsetlinewidth{1.003750pt}%
\definecolor{currentstroke}{rgb}{0.470588,0.368627,0.941176}%
\pgfsetstrokecolor{currentstroke}%
\pgfsetdash{}{0pt}%
\pgfsys@defobject{currentmarker}{\pgfqpoint{-0.031250in}{-0.031250in}}{\pgfqpoint{0.031250in}{0.031250in}}{%
\pgfpathmoveto{\pgfqpoint{0.000000in}{0.031250in}}%
\pgfpathlineto{\pgfqpoint{-0.031250in}{-0.031250in}}%
\pgfpathlineto{\pgfqpoint{0.031250in}{-0.031250in}}%
\pgfpathlineto{\pgfqpoint{0.000000in}{0.031250in}}%
\pgfpathclose%
\pgfusepath{stroke,fill}%
}%
\begin{pgfscope}%
\pgfsys@transformshift{1.038580in}{1.281480in}%
\pgfsys@useobject{currentmarker}{}%
\end{pgfscope}%
\end{pgfscope}%
\begin{pgfscope}%
\definecolor{textcolor}{rgb}{0.000000,0.000000,0.000000}%
\pgfsetstrokecolor{textcolor}%
\pgfsetfillcolor{textcolor}%
\pgftext[x=1.338580in,y=1.223147in,left,base]{\color{textcolor}{\rmfamily\fontsize{12.000000}{14.400000}\selectfont\catcode`\^=\active\def^{\ifmmode\sp\else\^{}\fi}\catcode`\%=\active\def%{\%}KA (III)}}%
\end{pgfscope}%
\begin{pgfscope}%
\pgfsetrectcap%
\pgfsetroundjoin%
\pgfsetlinewidth{1.505625pt}%
\definecolor{currentstroke}{rgb}{0.392157,0.560784,1.000000}%
\pgfsetstrokecolor{currentstroke}%
\pgfsetdash{}{0pt}%
\pgfpathmoveto{\pgfqpoint{0.871913in}{1.031480in}}%
\pgfpathlineto{\pgfqpoint{1.038580in}{1.031480in}}%
\pgfpathlineto{\pgfqpoint{1.205247in}{1.031480in}}%
\pgfusepath{stroke}%
\end{pgfscope}%
\begin{pgfscope}%
\pgfsetbuttcap%
\pgfsetroundjoin%
\definecolor{currentfill}{rgb}{0.392157,0.560784,1.000000}%
\pgfsetfillcolor{currentfill}%
\pgfsetlinewidth{1.003750pt}%
\definecolor{currentstroke}{rgb}{0.392157,0.560784,1.000000}%
\pgfsetstrokecolor{currentstroke}%
\pgfsetdash{}{0pt}%
\pgfsys@defobject{currentmarker}{\pgfqpoint{-0.031250in}{-0.031250in}}{\pgfqpoint{0.031250in}{0.031250in}}{%
\pgfpathmoveto{\pgfqpoint{0.000000in}{-0.031250in}}%
\pgfpathcurveto{\pgfqpoint{0.008288in}{-0.031250in}}{\pgfqpoint{0.016237in}{-0.027957in}}{\pgfqpoint{0.022097in}{-0.022097in}}%
\pgfpathcurveto{\pgfqpoint{0.027957in}{-0.016237in}}{\pgfqpoint{0.031250in}{-0.008288in}}{\pgfqpoint{0.031250in}{0.000000in}}%
\pgfpathcurveto{\pgfqpoint{0.031250in}{0.008288in}}{\pgfqpoint{0.027957in}{0.016237in}}{\pgfqpoint{0.022097in}{0.022097in}}%
\pgfpathcurveto{\pgfqpoint{0.016237in}{0.027957in}}{\pgfqpoint{0.008288in}{0.031250in}}{\pgfqpoint{0.000000in}{0.031250in}}%
\pgfpathcurveto{\pgfqpoint{-0.008288in}{0.031250in}}{\pgfqpoint{-0.016237in}{0.027957in}}{\pgfqpoint{-0.022097in}{0.022097in}}%
\pgfpathcurveto{\pgfqpoint{-0.027957in}{0.016237in}}{\pgfqpoint{-0.031250in}{0.008288in}}{\pgfqpoint{-0.031250in}{0.000000in}}%
\pgfpathcurveto{\pgfqpoint{-0.031250in}{-0.008288in}}{\pgfqpoint{-0.027957in}{-0.016237in}}{\pgfqpoint{-0.022097in}{-0.022097in}}%
\pgfpathcurveto{\pgfqpoint{-0.016237in}{-0.027957in}}{\pgfqpoint{-0.008288in}{-0.031250in}}{\pgfqpoint{0.000000in}{-0.031250in}}%
\pgfpathlineto{\pgfqpoint{0.000000in}{-0.031250in}}%
\pgfpathclose%
\pgfusepath{stroke,fill}%
}%
\begin{pgfscope}%
\pgfsys@transformshift{1.038580in}{1.031480in}%
\pgfsys@useobject{currentmarker}{}%
\end{pgfscope}%
\end{pgfscope}%
\begin{pgfscope}%
\definecolor{textcolor}{rgb}{0.000000,0.000000,0.000000}%
\pgfsetstrokecolor{textcolor}%
\pgfsetfillcolor{textcolor}%
\pgftext[x=1.338580in,y=0.973147in,left,base]{\color{textcolor}{\rmfamily\fontsize{12.000000}{14.400000}\selectfont\catcode`\^=\active\def^{\ifmmode\sp\else\^{}\fi}\catcode`\%=\active\def%{\%}KA (IV)}}%
\end{pgfscope}%
\begin{pgfscope}%
\pgfsetrectcap%
\pgfsetroundjoin%
\pgfsetlinewidth{1.505625pt}%
\definecolor{currentstroke}{rgb}{0.862745,0.149020,0.498039}%
\pgfsetstrokecolor{currentstroke}%
\pgfsetdash{}{0pt}%
\pgfpathmoveto{\pgfqpoint{0.871913in}{0.789813in}}%
\pgfpathlineto{\pgfqpoint{1.038580in}{0.789813in}}%
\pgfpathlineto{\pgfqpoint{1.205247in}{0.789813in}}%
\pgfusepath{stroke}%
\end{pgfscope}%
\begin{pgfscope}%
\pgfsetbuttcap%
\pgfsetmiterjoin%
\definecolor{currentfill}{rgb}{0.862745,0.149020,0.498039}%
\pgfsetfillcolor{currentfill}%
\pgfsetlinewidth{1.003750pt}%
\definecolor{currentstroke}{rgb}{0.862745,0.149020,0.498039}%
\pgfsetstrokecolor{currentstroke}%
\pgfsetdash{}{0pt}%
\pgfsys@defobject{currentmarker}{\pgfqpoint{-0.031250in}{-0.031250in}}{\pgfqpoint{0.031250in}{0.031250in}}{%
\pgfpathmoveto{\pgfqpoint{-0.031250in}{-0.031250in}}%
\pgfpathlineto{\pgfqpoint{0.031250in}{-0.031250in}}%
\pgfpathlineto{\pgfqpoint{0.031250in}{0.031250in}}%
\pgfpathlineto{\pgfqpoint{-0.031250in}{0.031250in}}%
\pgfpathlineto{\pgfqpoint{-0.031250in}{-0.031250in}}%
\pgfpathclose%
\pgfusepath{stroke,fill}%
}%
\begin{pgfscope}%
\pgfsys@transformshift{1.038580in}{0.789813in}%
\pgfsys@useobject{currentmarker}{}%
\end{pgfscope}%
\end{pgfscope}%
\begin{pgfscope}%
\definecolor{textcolor}{rgb}{0.000000,0.000000,0.000000}%
\pgfsetstrokecolor{textcolor}%
\pgfsetfillcolor{textcolor}%
\pgftext[x=1.338580in,y=0.731480in,left,base]{\color{textcolor}{\rmfamily\fontsize{12.000000}{14.400000}\selectfont\catcode`\^=\active\def^{\ifmmode\sp\else\^{}\fi}\catcode`\%=\active\def%{\%}CN++}}%
\end{pgfscope}%
\end{pgfpicture}%
\makeatother%
\endgroup%
}
    \end{minipage}\hfill%
    \begin{minipage}[c]{.475\linewidth}
        \vspace{-35pt}
        \scalebox{0.8}{\centering
\renewcommand{\arraystretch}{1.2}
\begin{tabular}{@{}lccccc@{}}
\toprule
 & $n_{\mtx{\Omega}}$ & $n_{\mtx{\Psi}}$ & $q$ & $n$ & time (s)\\
\midrule
KA (I) & $10$ & $60$ & $30$ & $20$ & $39.68$ \\
KA (II) & $20$ & $30$ & $30$ & $20$ & $11.00$ \\
KA (III) & $10$ & $30$ & $60$ & $20$ & $13.21$ \\
KA (IV) & $10$ & $30$ & $30$ & $40$ & $9.96$ \\
\bottomrule
\end{tabular}
}
        \newline
        \vspace{15pt}
        \newline
        \scalebox{0.8}{\input{tables/krylov_aware_density_CN.tex}}
    \end{minipage}
    \caption{For the example from \cref{subsec:hamiltonian}, error vs. $\sigma$ the Krylov-Aware (KA) stochastic trace estimator with different parameter settings, compared to the Chebyshev-Nyström++ (CN++) method.}
    \label{fig:krylov-aware-density}
\end{figure}

\subsection{Spectral density of Hessian matrix of neural network}
\label{subsec:hessian}

When fitting the parameters of a neural network, it can be of interest to determine whether a (local) minimum has been found and whether this minimum is robust, i.e., whether a small perturbation of the parameters leads to a significant decrease in the fit of the neural network. Both properties are reflected in the eigenvalues of the Hessian matrix $\mtx{A}$ with respect to some loss function: if all eigenvalues are non-negative, then the loss attains a local minimum, and if, additionally, all of them are small, we talk about a flat minimum; one where small perturbations of the parameters do not cause large increases of the loss and are therefore associated with better generalization \cite{yao-2020-pyhessian-neural}. On the other hand, large eigenvalues in the spectrum indicate high sensitivity of the loss to small changes in the parameters. Those we want to avoid, which is the reason why we monitor them.

Since neural networks are usually parametrized by a large number of parameters, assembling the Hessian matrix explicitly and then computing its eigenvalues is clearly infeasible. On the other hand, one can cheaply compute matrix-vector products $\mtx{A} \vct{x}$ with the Hessian, at a cost that scales proportionally with the number of parameters \cite{pearlmutter-1994-fast-exact}.

To demonstrate that our algorithm can effectively be applied in the setting of neural network optimization, we approximate the spectral density of a small fully connected convolutional neural network with $6\,782$ parameters. We train this network on the handwritten digit classification task given by the MNIST dataset\footnote{Handwritten digit classification; taken from \url{http://yann.lecun.com/exdb/mnist}.} in PyTorch 2.4.1 in a standard way. We use the bound from \cite[Theorem 1]{zhou-2011-bounding-spectrum} to estimate bounds on the spectrum.

We plot the spectral density of the Hessian matrix of the untrained neural network, and at different stages of training in \cref{fig:hessian-density}, as well as the corresponding mean squared error loss. It can be observed that the eigenvalues creep towards zero as the training proceeds. Furthermore, despite the loss steadily decreasing, the presence of relatively large eigenvalues in some epochs indicates a sharp minimum of the network, hence, unfavorable generalization properties.

\begin{figure}
    \begin{minipage}[c]{.49\linewidth}
        \centering
        \scalebox{0.8}{%% Creator: Matplotlib, PGF backend
%%
%% To include the figure in your LaTeX document, write
%%   \input{<filename>.pgf}
%%
%% Make sure the required packages are loaded in your preamble
%%   \usepackage{pgf}
%%
%% Also ensure that all the required font packages are loaded; for instance,
%% the lmodern package is sometimes necessary when using math font.
%%   \usepackage{lmodern}
%%
%% Figures using additional raster images can only be included by \input if
%% they are in the same directory as the main LaTeX file. For loading figures
%% from other directories you can use the `import` package
%%   \usepackage{import}
%%
%% and then include the figures with
%%   \import{<path to file>}{<filename>.pgf}
%%
%% Matplotlib used the following preamble
%%   \def\mathdefault#1{#1}
%%   \everymath=\expandafter{\the\everymath\displaystyle}
%%   \IfFileExists{scrextend.sty}{
%%     \usepackage[fontsize=12.000000pt]{scrextend}
%%   }{
%%     \renewcommand{\normalsize}{\fontsize{12.000000}{14.400000}\selectfont}
%%     \normalsize
%%   }
%%   
%%   \ifdefined\pdftexversion\else  % non-pdftex case.
%%     \usepackage{fontspec}
%%     \setmainfont{DejaVuSerif.ttf}[Path=\detokenize{/opt/hostedtoolcache/Python/3.12.9/x64/lib/python3.12/site-packages/matplotlib/mpl-data/fonts/ttf/}]
%%     \setsansfont{DejaVuSans.ttf}[Path=\detokenize{/opt/hostedtoolcache/Python/3.12.9/x64/lib/python3.12/site-packages/matplotlib/mpl-data/fonts/ttf/}]
%%     \setmonofont{DejaVuSansMono.ttf}[Path=\detokenize{/opt/hostedtoolcache/Python/3.12.9/x64/lib/python3.12/site-packages/matplotlib/mpl-data/fonts/ttf/}]
%%   \fi
%%   \makeatletter\@ifpackageloaded{underscore}{}{\usepackage[strings]{underscore}}\makeatother
%%
\begingroup%
\makeatletter%
\begin{pgfpicture}%
\pgfpathrectangle{\pgfpointorigin}{\pgfqpoint{3.134565in}{3.003943in}}%
\pgfusepath{use as bounding box, clip}%
\begin{pgfscope}%
\pgfsetbuttcap%
\pgfsetmiterjoin%
\definecolor{currentfill}{rgb}{1.000000,1.000000,1.000000}%
\pgfsetfillcolor{currentfill}%
\pgfsetlinewidth{0.000000pt}%
\definecolor{currentstroke}{rgb}{1.000000,1.000000,1.000000}%
\pgfsetstrokecolor{currentstroke}%
\pgfsetdash{}{0pt}%
\pgfpathmoveto{\pgfqpoint{0.000000in}{-0.000000in}}%
\pgfpathlineto{\pgfqpoint{3.134565in}{-0.000000in}}%
\pgfpathlineto{\pgfqpoint{3.134565in}{3.003943in}}%
\pgfpathlineto{\pgfqpoint{0.000000in}{3.003943in}}%
\pgfpathlineto{\pgfqpoint{0.000000in}{-0.000000in}}%
\pgfpathclose%
\pgfusepath{fill}%
\end{pgfscope}%
\begin{pgfscope}%
\pgfsetbuttcap%
\pgfsetmiterjoin%
\definecolor{currentfill}{rgb}{1.000000,1.000000,1.000000}%
\pgfsetfillcolor{currentfill}%
\pgfsetlinewidth{0.000000pt}%
\definecolor{currentstroke}{rgb}{0.000000,0.000000,0.000000}%
\pgfsetstrokecolor{currentstroke}%
\pgfsetstrokeopacity{0.000000}%
\pgfsetdash{}{0pt}%
\pgfpathmoveto{\pgfqpoint{0.709565in}{0.549073in}}%
\pgfpathlineto{\pgfqpoint{3.034565in}{0.549073in}}%
\pgfpathlineto{\pgfqpoint{3.034565in}{2.859073in}}%
\pgfpathlineto{\pgfqpoint{0.709565in}{2.859073in}}%
\pgfpathlineto{\pgfqpoint{0.709565in}{0.549073in}}%
\pgfpathclose%
\pgfusepath{fill}%
\end{pgfscope}%
\begin{pgfscope}%
\pgfpathrectangle{\pgfqpoint{0.709565in}{0.549073in}}{\pgfqpoint{2.325000in}{2.310000in}}%
\pgfusepath{clip}%
\pgfsetrectcap%
\pgfsetroundjoin%
\pgfsetlinewidth{0.250937pt}%
\definecolor{currentstroke}{rgb}{0.000000,0.000000,0.000000}%
\pgfsetstrokecolor{currentstroke}%
\pgfsetstrokeopacity{0.200000}%
\pgfsetdash{}{0pt}%
\pgfpathmoveto{\pgfqpoint{1.232240in}{0.549073in}}%
\pgfpathlineto{\pgfqpoint{1.232240in}{2.859073in}}%
\pgfusepath{stroke}%
\end{pgfscope}%
\begin{pgfscope}%
\pgfsetbuttcap%
\pgfsetroundjoin%
\definecolor{currentfill}{rgb}{0.000000,0.000000,0.000000}%
\pgfsetfillcolor{currentfill}%
\pgfsetlinewidth{0.803000pt}%
\definecolor{currentstroke}{rgb}{0.000000,0.000000,0.000000}%
\pgfsetstrokecolor{currentstroke}%
\pgfsetdash{}{0pt}%
\pgfsys@defobject{currentmarker}{\pgfqpoint{0.000000in}{-0.048611in}}{\pgfqpoint{0.000000in}{0.000000in}}{%
\pgfpathmoveto{\pgfqpoint{0.000000in}{0.000000in}}%
\pgfpathlineto{\pgfqpoint{0.000000in}{-0.048611in}}%
\pgfusepath{stroke,fill}%
}%
\begin{pgfscope}%
\pgfsys@transformshift{1.232240in}{0.549073in}%
\pgfsys@useobject{currentmarker}{}%
\end{pgfscope}%
\end{pgfscope}%
\begin{pgfscope}%
\definecolor{textcolor}{rgb}{0.000000,0.000000,0.000000}%
\pgfsetstrokecolor{textcolor}%
\pgfsetfillcolor{textcolor}%
\pgftext[x=1.232240in,y=0.451851in,,top]{\color{textcolor}{\rmfamily\fontsize{12.000000}{14.400000}\selectfont\catcode`\^=\active\def^{\ifmmode\sp\else\^{}\fi}\catcode`\%=\active\def%{\%}$\mathdefault{2}$}}%
\end{pgfscope}%
\begin{pgfscope}%
\pgfpathrectangle{\pgfqpoint{0.709565in}{0.549073in}}{\pgfqpoint{2.325000in}{2.310000in}}%
\pgfusepath{clip}%
\pgfsetrectcap%
\pgfsetroundjoin%
\pgfsetlinewidth{0.250937pt}%
\definecolor{currentstroke}{rgb}{0.000000,0.000000,0.000000}%
\pgfsetstrokecolor{currentstroke}%
\pgfsetstrokeopacity{0.200000}%
\pgfsetdash{}{0pt}%
\pgfpathmoveto{\pgfqpoint{1.816100in}{0.549073in}}%
\pgfpathlineto{\pgfqpoint{1.816100in}{2.859073in}}%
\pgfusepath{stroke}%
\end{pgfscope}%
\begin{pgfscope}%
\pgfsetbuttcap%
\pgfsetroundjoin%
\definecolor{currentfill}{rgb}{0.000000,0.000000,0.000000}%
\pgfsetfillcolor{currentfill}%
\pgfsetlinewidth{0.803000pt}%
\definecolor{currentstroke}{rgb}{0.000000,0.000000,0.000000}%
\pgfsetstrokecolor{currentstroke}%
\pgfsetdash{}{0pt}%
\pgfsys@defobject{currentmarker}{\pgfqpoint{0.000000in}{-0.048611in}}{\pgfqpoint{0.000000in}{0.000000in}}{%
\pgfpathmoveto{\pgfqpoint{0.000000in}{0.000000in}}%
\pgfpathlineto{\pgfqpoint{0.000000in}{-0.048611in}}%
\pgfusepath{stroke,fill}%
}%
\begin{pgfscope}%
\pgfsys@transformshift{1.816100in}{0.549073in}%
\pgfsys@useobject{currentmarker}{}%
\end{pgfscope}%
\end{pgfscope}%
\begin{pgfscope}%
\definecolor{textcolor}{rgb}{0.000000,0.000000,0.000000}%
\pgfsetstrokecolor{textcolor}%
\pgfsetfillcolor{textcolor}%
\pgftext[x=1.816100in,y=0.451851in,,top]{\color{textcolor}{\rmfamily\fontsize{12.000000}{14.400000}\selectfont\catcode`\^=\active\def^{\ifmmode\sp\else\^{}\fi}\catcode`\%=\active\def%{\%}$\mathdefault{4}$}}%
\end{pgfscope}%
\begin{pgfscope}%
\pgfpathrectangle{\pgfqpoint{0.709565in}{0.549073in}}{\pgfqpoint{2.325000in}{2.310000in}}%
\pgfusepath{clip}%
\pgfsetrectcap%
\pgfsetroundjoin%
\pgfsetlinewidth{0.250937pt}%
\definecolor{currentstroke}{rgb}{0.000000,0.000000,0.000000}%
\pgfsetstrokecolor{currentstroke}%
\pgfsetstrokeopacity{0.200000}%
\pgfsetdash{}{0pt}%
\pgfpathmoveto{\pgfqpoint{2.399959in}{0.549073in}}%
\pgfpathlineto{\pgfqpoint{2.399959in}{2.859073in}}%
\pgfusepath{stroke}%
\end{pgfscope}%
\begin{pgfscope}%
\pgfsetbuttcap%
\pgfsetroundjoin%
\definecolor{currentfill}{rgb}{0.000000,0.000000,0.000000}%
\pgfsetfillcolor{currentfill}%
\pgfsetlinewidth{0.803000pt}%
\definecolor{currentstroke}{rgb}{0.000000,0.000000,0.000000}%
\pgfsetstrokecolor{currentstroke}%
\pgfsetdash{}{0pt}%
\pgfsys@defobject{currentmarker}{\pgfqpoint{0.000000in}{-0.048611in}}{\pgfqpoint{0.000000in}{0.000000in}}{%
\pgfpathmoveto{\pgfqpoint{0.000000in}{0.000000in}}%
\pgfpathlineto{\pgfqpoint{0.000000in}{-0.048611in}}%
\pgfusepath{stroke,fill}%
}%
\begin{pgfscope}%
\pgfsys@transformshift{2.399959in}{0.549073in}%
\pgfsys@useobject{currentmarker}{}%
\end{pgfscope}%
\end{pgfscope}%
\begin{pgfscope}%
\definecolor{textcolor}{rgb}{0.000000,0.000000,0.000000}%
\pgfsetstrokecolor{textcolor}%
\pgfsetfillcolor{textcolor}%
\pgftext[x=2.399959in,y=0.451851in,,top]{\color{textcolor}{\rmfamily\fontsize{12.000000}{14.400000}\selectfont\catcode`\^=\active\def^{\ifmmode\sp\else\^{}\fi}\catcode`\%=\active\def%{\%}$\mathdefault{6}$}}%
\end{pgfscope}%
\begin{pgfscope}%
\pgfpathrectangle{\pgfqpoint{0.709565in}{0.549073in}}{\pgfqpoint{2.325000in}{2.310000in}}%
\pgfusepath{clip}%
\pgfsetrectcap%
\pgfsetroundjoin%
\pgfsetlinewidth{0.250937pt}%
\definecolor{currentstroke}{rgb}{0.000000,0.000000,0.000000}%
\pgfsetstrokecolor{currentstroke}%
\pgfsetstrokeopacity{0.200000}%
\pgfsetdash{}{0pt}%
\pgfpathmoveto{\pgfqpoint{2.983819in}{0.549073in}}%
\pgfpathlineto{\pgfqpoint{2.983819in}{2.859073in}}%
\pgfusepath{stroke}%
\end{pgfscope}%
\begin{pgfscope}%
\pgfsetbuttcap%
\pgfsetroundjoin%
\definecolor{currentfill}{rgb}{0.000000,0.000000,0.000000}%
\pgfsetfillcolor{currentfill}%
\pgfsetlinewidth{0.803000pt}%
\definecolor{currentstroke}{rgb}{0.000000,0.000000,0.000000}%
\pgfsetstrokecolor{currentstroke}%
\pgfsetdash{}{0pt}%
\pgfsys@defobject{currentmarker}{\pgfqpoint{0.000000in}{-0.048611in}}{\pgfqpoint{0.000000in}{0.000000in}}{%
\pgfpathmoveto{\pgfqpoint{0.000000in}{0.000000in}}%
\pgfpathlineto{\pgfqpoint{0.000000in}{-0.048611in}}%
\pgfusepath{stroke,fill}%
}%
\begin{pgfscope}%
\pgfsys@transformshift{2.983819in}{0.549073in}%
\pgfsys@useobject{currentmarker}{}%
\end{pgfscope}%
\end{pgfscope}%
\begin{pgfscope}%
\definecolor{textcolor}{rgb}{0.000000,0.000000,0.000000}%
\pgfsetstrokecolor{textcolor}%
\pgfsetfillcolor{textcolor}%
\pgftext[x=2.983819in,y=0.451851in,,top]{\color{textcolor}{\rmfamily\fontsize{12.000000}{14.400000}\selectfont\catcode`\^=\active\def^{\ifmmode\sp\else\^{}\fi}\catcode`\%=\active\def%{\%}$\mathdefault{8}$}}%
\end{pgfscope}%
\begin{pgfscope}%
\definecolor{textcolor}{rgb}{0.000000,0.000000,0.000000}%
\pgfsetstrokecolor{textcolor}%
\pgfsetfillcolor{textcolor}%
\pgftext[x=1.872065in,y=0.248148in,,top]{\color{textcolor}{\rmfamily\fontsize{12.000000}{14.400000}\selectfont\catcode`\^=\active\def^{\ifmmode\sp\else\^{}\fi}\catcode`\%=\active\def%{\%}spectral parameter $t$}}%
\end{pgfscope}%
\begin{pgfscope}%
\pgfpathrectangle{\pgfqpoint{0.709565in}{0.549073in}}{\pgfqpoint{2.325000in}{2.310000in}}%
\pgfusepath{clip}%
\pgfsetrectcap%
\pgfsetroundjoin%
\pgfsetlinewidth{0.250937pt}%
\definecolor{currentstroke}{rgb}{0.000000,0.000000,0.000000}%
\pgfsetstrokecolor{currentstroke}%
\pgfsetstrokeopacity{0.200000}%
\pgfsetdash{}{0pt}%
\pgfpathmoveto{\pgfqpoint{0.709565in}{0.768469in}}%
\pgfpathlineto{\pgfqpoint{3.034565in}{0.768469in}}%
\pgfusepath{stroke}%
\end{pgfscope}%
\begin{pgfscope}%
\pgfsetbuttcap%
\pgfsetroundjoin%
\definecolor{currentfill}{rgb}{0.000000,0.000000,0.000000}%
\pgfsetfillcolor{currentfill}%
\pgfsetlinewidth{0.803000pt}%
\definecolor{currentstroke}{rgb}{0.000000,0.000000,0.000000}%
\pgfsetstrokecolor{currentstroke}%
\pgfsetdash{}{0pt}%
\pgfsys@defobject{currentmarker}{\pgfqpoint{-0.048611in}{0.000000in}}{\pgfqpoint{-0.000000in}{0.000000in}}{%
\pgfpathmoveto{\pgfqpoint{-0.000000in}{0.000000in}}%
\pgfpathlineto{\pgfqpoint{-0.048611in}{0.000000in}}%
\pgfusepath{stroke,fill}%
}%
\begin{pgfscope}%
\pgfsys@transformshift{0.709565in}{0.768469in}%
\pgfsys@useobject{currentmarker}{}%
\end{pgfscope}%
\end{pgfscope}%
\begin{pgfscope}%
\definecolor{textcolor}{rgb}{0.000000,0.000000,0.000000}%
\pgfsetstrokecolor{textcolor}%
\pgfsetfillcolor{textcolor}%
\pgftext[x=0.322222in, y=0.710599in, left, base]{\color{textcolor}{\rmfamily\fontsize{12.000000}{14.400000}\selectfont\catcode`\^=\active\def^{\ifmmode\sp\else\^{}\fi}\catcode`\%=\active\def%{\%}$\mathdefault{0.00}$}}%
\end{pgfscope}%
\begin{pgfscope}%
\pgfpathrectangle{\pgfqpoint{0.709565in}{0.549073in}}{\pgfqpoint{2.325000in}{2.310000in}}%
\pgfusepath{clip}%
\pgfsetrectcap%
\pgfsetroundjoin%
\pgfsetlinewidth{0.250937pt}%
\definecolor{currentstroke}{rgb}{0.000000,0.000000,0.000000}%
\pgfsetstrokecolor{currentstroke}%
\pgfsetstrokeopacity{0.200000}%
\pgfsetdash{}{0pt}%
\pgfpathmoveto{\pgfqpoint{0.709565in}{1.287870in}}%
\pgfpathlineto{\pgfqpoint{3.034565in}{1.287870in}}%
\pgfusepath{stroke}%
\end{pgfscope}%
\begin{pgfscope}%
\pgfsetbuttcap%
\pgfsetroundjoin%
\definecolor{currentfill}{rgb}{0.000000,0.000000,0.000000}%
\pgfsetfillcolor{currentfill}%
\pgfsetlinewidth{0.803000pt}%
\definecolor{currentstroke}{rgb}{0.000000,0.000000,0.000000}%
\pgfsetstrokecolor{currentstroke}%
\pgfsetdash{}{0pt}%
\pgfsys@defobject{currentmarker}{\pgfqpoint{-0.048611in}{0.000000in}}{\pgfqpoint{-0.000000in}{0.000000in}}{%
\pgfpathmoveto{\pgfqpoint{-0.000000in}{0.000000in}}%
\pgfpathlineto{\pgfqpoint{-0.048611in}{0.000000in}}%
\pgfusepath{stroke,fill}%
}%
\begin{pgfscope}%
\pgfsys@transformshift{0.709565in}{1.287870in}%
\pgfsys@useobject{currentmarker}{}%
\end{pgfscope}%
\end{pgfscope}%
\begin{pgfscope}%
\definecolor{textcolor}{rgb}{0.000000,0.000000,0.000000}%
\pgfsetstrokecolor{textcolor}%
\pgfsetfillcolor{textcolor}%
\pgftext[x=0.322222in, y=1.230000in, left, base]{\color{textcolor}{\rmfamily\fontsize{12.000000}{14.400000}\selectfont\catcode`\^=\active\def^{\ifmmode\sp\else\^{}\fi}\catcode`\%=\active\def%{\%}$\mathdefault{0.05}$}}%
\end{pgfscope}%
\begin{pgfscope}%
\pgfpathrectangle{\pgfqpoint{0.709565in}{0.549073in}}{\pgfqpoint{2.325000in}{2.310000in}}%
\pgfusepath{clip}%
\pgfsetrectcap%
\pgfsetroundjoin%
\pgfsetlinewidth{0.250937pt}%
\definecolor{currentstroke}{rgb}{0.000000,0.000000,0.000000}%
\pgfsetstrokecolor{currentstroke}%
\pgfsetstrokeopacity{0.200000}%
\pgfsetdash{}{0pt}%
\pgfpathmoveto{\pgfqpoint{0.709565in}{1.807271in}}%
\pgfpathlineto{\pgfqpoint{3.034565in}{1.807271in}}%
\pgfusepath{stroke}%
\end{pgfscope}%
\begin{pgfscope}%
\pgfsetbuttcap%
\pgfsetroundjoin%
\definecolor{currentfill}{rgb}{0.000000,0.000000,0.000000}%
\pgfsetfillcolor{currentfill}%
\pgfsetlinewidth{0.803000pt}%
\definecolor{currentstroke}{rgb}{0.000000,0.000000,0.000000}%
\pgfsetstrokecolor{currentstroke}%
\pgfsetdash{}{0pt}%
\pgfsys@defobject{currentmarker}{\pgfqpoint{-0.048611in}{0.000000in}}{\pgfqpoint{-0.000000in}{0.000000in}}{%
\pgfpathmoveto{\pgfqpoint{-0.000000in}{0.000000in}}%
\pgfpathlineto{\pgfqpoint{-0.048611in}{0.000000in}}%
\pgfusepath{stroke,fill}%
}%
\begin{pgfscope}%
\pgfsys@transformshift{0.709565in}{1.807271in}%
\pgfsys@useobject{currentmarker}{}%
\end{pgfscope}%
\end{pgfscope}%
\begin{pgfscope}%
\definecolor{textcolor}{rgb}{0.000000,0.000000,0.000000}%
\pgfsetstrokecolor{textcolor}%
\pgfsetfillcolor{textcolor}%
\pgftext[x=0.322222in, y=1.749401in, left, base]{\color{textcolor}{\rmfamily\fontsize{12.000000}{14.400000}\selectfont\catcode`\^=\active\def^{\ifmmode\sp\else\^{}\fi}\catcode`\%=\active\def%{\%}$\mathdefault{0.10}$}}%
\end{pgfscope}%
\begin{pgfscope}%
\pgfpathrectangle{\pgfqpoint{0.709565in}{0.549073in}}{\pgfqpoint{2.325000in}{2.310000in}}%
\pgfusepath{clip}%
\pgfsetrectcap%
\pgfsetroundjoin%
\pgfsetlinewidth{0.250937pt}%
\definecolor{currentstroke}{rgb}{0.000000,0.000000,0.000000}%
\pgfsetstrokecolor{currentstroke}%
\pgfsetstrokeopacity{0.200000}%
\pgfsetdash{}{0pt}%
\pgfpathmoveto{\pgfqpoint{0.709565in}{2.326672in}}%
\pgfpathlineto{\pgfqpoint{3.034565in}{2.326672in}}%
\pgfusepath{stroke}%
\end{pgfscope}%
\begin{pgfscope}%
\pgfsetbuttcap%
\pgfsetroundjoin%
\definecolor{currentfill}{rgb}{0.000000,0.000000,0.000000}%
\pgfsetfillcolor{currentfill}%
\pgfsetlinewidth{0.803000pt}%
\definecolor{currentstroke}{rgb}{0.000000,0.000000,0.000000}%
\pgfsetstrokecolor{currentstroke}%
\pgfsetdash{}{0pt}%
\pgfsys@defobject{currentmarker}{\pgfqpoint{-0.048611in}{0.000000in}}{\pgfqpoint{-0.000000in}{0.000000in}}{%
\pgfpathmoveto{\pgfqpoint{-0.000000in}{0.000000in}}%
\pgfpathlineto{\pgfqpoint{-0.048611in}{0.000000in}}%
\pgfusepath{stroke,fill}%
}%
\begin{pgfscope}%
\pgfsys@transformshift{0.709565in}{2.326672in}%
\pgfsys@useobject{currentmarker}{}%
\end{pgfscope}%
\end{pgfscope}%
\begin{pgfscope}%
\definecolor{textcolor}{rgb}{0.000000,0.000000,0.000000}%
\pgfsetstrokecolor{textcolor}%
\pgfsetfillcolor{textcolor}%
\pgftext[x=0.322222in, y=2.268802in, left, base]{\color{textcolor}{\rmfamily\fontsize{12.000000}{14.400000}\selectfont\catcode`\^=\active\def^{\ifmmode\sp\else\^{}\fi}\catcode`\%=\active\def%{\%}$\mathdefault{0.15}$}}%
\end{pgfscope}%
\begin{pgfscope}%
\pgfpathrectangle{\pgfqpoint{0.709565in}{0.549073in}}{\pgfqpoint{2.325000in}{2.310000in}}%
\pgfusepath{clip}%
\pgfsetrectcap%
\pgfsetroundjoin%
\pgfsetlinewidth{0.250937pt}%
\definecolor{currentstroke}{rgb}{0.000000,0.000000,0.000000}%
\pgfsetstrokecolor{currentstroke}%
\pgfsetstrokeopacity{0.200000}%
\pgfsetdash{}{0pt}%
\pgfpathmoveto{\pgfqpoint{0.709565in}{2.846073in}}%
\pgfpathlineto{\pgfqpoint{3.034565in}{2.846073in}}%
\pgfusepath{stroke}%
\end{pgfscope}%
\begin{pgfscope}%
\pgfsetbuttcap%
\pgfsetroundjoin%
\definecolor{currentfill}{rgb}{0.000000,0.000000,0.000000}%
\pgfsetfillcolor{currentfill}%
\pgfsetlinewidth{0.803000pt}%
\definecolor{currentstroke}{rgb}{0.000000,0.000000,0.000000}%
\pgfsetstrokecolor{currentstroke}%
\pgfsetdash{}{0pt}%
\pgfsys@defobject{currentmarker}{\pgfqpoint{-0.048611in}{0.000000in}}{\pgfqpoint{-0.000000in}{0.000000in}}{%
\pgfpathmoveto{\pgfqpoint{-0.000000in}{0.000000in}}%
\pgfpathlineto{\pgfqpoint{-0.048611in}{0.000000in}}%
\pgfusepath{stroke,fill}%
}%
\begin{pgfscope}%
\pgfsys@transformshift{0.709565in}{2.846073in}%
\pgfsys@useobject{currentmarker}{}%
\end{pgfscope}%
\end{pgfscope}%
\begin{pgfscope}%
\definecolor{textcolor}{rgb}{0.000000,0.000000,0.000000}%
\pgfsetstrokecolor{textcolor}%
\pgfsetfillcolor{textcolor}%
\pgftext[x=0.322222in, y=2.788203in, left, base]{\color{textcolor}{\rmfamily\fontsize{12.000000}{14.400000}\selectfont\catcode`\^=\active\def^{\ifmmode\sp\else\^{}\fi}\catcode`\%=\active\def%{\%}$\mathdefault{0.20}$}}%
\end{pgfscope}%
\begin{pgfscope}%
\definecolor{textcolor}{rgb}{0.000000,0.000000,0.000000}%
\pgfsetstrokecolor{textcolor}%
\pgfsetfillcolor{textcolor}%
\pgftext[x=0.266667in,y=1.704073in,,bottom,rotate=90.000000]{\color{textcolor}{\rmfamily\fontsize{12.000000}{14.400000}\selectfont\catcode`\^=\active\def^{\ifmmode\sp\else\^{}\fi}\catcode`\%=\active\def%{\%}smoothed spectral density $\phi_{\sigma}(t)$}}%
\end{pgfscope}%
\begin{pgfscope}%
\pgfpathrectangle{\pgfqpoint{0.709565in}{0.549073in}}{\pgfqpoint{2.325000in}{2.310000in}}%
\pgfusepath{clip}%
\pgfsetrectcap%
\pgfsetroundjoin%
\pgfsetlinewidth{1.505625pt}%
\definecolor{currentstroke}{rgb}{0.392157,0.560784,1.000000}%
\pgfsetstrokecolor{currentstroke}%
\pgfsetdash{}{0pt}%
\pgfpathmoveto{\pgfqpoint{0.764922in}{2.666573in}}%
\pgfpathlineto{\pgfqpoint{0.779783in}{2.349133in}}%
\pgfpathlineto{\pgfqpoint{0.794644in}{2.064030in}}%
\pgfpathlineto{\pgfqpoint{0.809505in}{1.560939in}}%
\pgfpathlineto{\pgfqpoint{0.824366in}{1.422134in}}%
\pgfpathlineto{\pgfqpoint{0.839227in}{1.333923in}}%
\pgfpathlineto{\pgfqpoint{0.854088in}{1.058531in}}%
\pgfpathlineto{\pgfqpoint{0.868949in}{0.843291in}}%
\pgfpathlineto{\pgfqpoint{0.883810in}{0.822419in}}%
\pgfpathlineto{\pgfqpoint{0.898671in}{0.995799in}}%
\pgfpathlineto{\pgfqpoint{0.913532in}{0.768469in}}%
\pgfpathlineto{\pgfqpoint{0.972976in}{0.768469in}}%
\pgfpathlineto{\pgfqpoint{0.987837in}{1.325158in}}%
\pgfpathlineto{\pgfqpoint{1.002698in}{0.785742in}}%
\pgfpathlineto{\pgfqpoint{1.017559in}{0.863922in}}%
\pgfpathlineto{\pgfqpoint{1.032420in}{0.872231in}}%
\pgfpathlineto{\pgfqpoint{1.047281in}{0.790733in}}%
\pgfpathlineto{\pgfqpoint{1.062142in}{0.925698in}}%
\pgfpathlineto{\pgfqpoint{1.077003in}{0.797016in}}%
\pgfpathlineto{\pgfqpoint{1.091864in}{0.880129in}}%
\pgfpathlineto{\pgfqpoint{1.106724in}{0.853613in}}%
\pgfpathlineto{\pgfqpoint{1.121585in}{0.781240in}}%
\pgfpathlineto{\pgfqpoint{1.136446in}{1.107847in}}%
\pgfpathlineto{\pgfqpoint{1.151307in}{0.768469in}}%
\pgfpathlineto{\pgfqpoint{1.181029in}{0.768469in}}%
\pgfpathlineto{\pgfqpoint{1.195890in}{1.247312in}}%
\pgfpathlineto{\pgfqpoint{1.210751in}{0.791069in}}%
\pgfpathlineto{\pgfqpoint{1.225612in}{0.872821in}}%
\pgfpathlineto{\pgfqpoint{1.240473in}{0.863249in}}%
\pgfpathlineto{\pgfqpoint{1.255334in}{0.785402in}}%
\pgfpathlineto{\pgfqpoint{1.270195in}{1.011292in}}%
\pgfpathlineto{\pgfqpoint{1.285056in}{0.768469in}}%
\pgfpathlineto{\pgfqpoint{2.830598in}{0.768469in}}%
\pgfpathlineto{\pgfqpoint{2.845459in}{0.771363in}}%
\pgfpathlineto{\pgfqpoint{2.860320in}{0.810525in}}%
\pgfpathlineto{\pgfqpoint{2.875181in}{0.888597in}}%
\pgfpathlineto{\pgfqpoint{2.890042in}{0.835965in}}%
\pgfpathlineto{\pgfqpoint{2.904903in}{0.775933in}}%
\pgfpathlineto{\pgfqpoint{2.919764in}{1.891634in}}%
\pgfpathlineto{\pgfqpoint{2.934625in}{0.768469in}}%
\pgfpathlineto{\pgfqpoint{2.979208in}{0.768469in}}%
\pgfpathlineto{\pgfqpoint{2.979208in}{0.768469in}}%
\pgfusepath{stroke}%
\end{pgfscope}%
\begin{pgfscope}%
\pgfpathrectangle{\pgfqpoint{0.709565in}{0.549073in}}{\pgfqpoint{2.325000in}{2.310000in}}%
\pgfusepath{clip}%
\pgfsetrectcap%
\pgfsetroundjoin%
\pgfsetlinewidth{1.505625pt}%
\definecolor{currentstroke}{rgb}{0.470588,0.368627,0.941176}%
\pgfsetstrokecolor{currentstroke}%
\pgfsetdash{}{0pt}%
\pgfpathmoveto{\pgfqpoint{0.764922in}{1.203677in}}%
\pgfpathlineto{\pgfqpoint{0.779783in}{1.027614in}}%
\pgfpathlineto{\pgfqpoint{0.794644in}{0.867176in}}%
\pgfpathlineto{\pgfqpoint{0.809505in}{0.818751in}}%
\pgfpathlineto{\pgfqpoint{0.824366in}{0.911260in}}%
\pgfpathlineto{\pgfqpoint{0.839227in}{0.947720in}}%
\pgfpathlineto{\pgfqpoint{0.854088in}{0.869407in}}%
\pgfpathlineto{\pgfqpoint{0.868949in}{0.785052in}}%
\pgfpathlineto{\pgfqpoint{0.883810in}{0.789086in}}%
\pgfpathlineto{\pgfqpoint{0.898671in}{0.768469in}}%
\pgfpathlineto{\pgfqpoint{0.913532in}{0.768469in}}%
\pgfpathlineto{\pgfqpoint{0.928393in}{0.773367in}}%
\pgfpathlineto{\pgfqpoint{0.943254in}{0.823282in}}%
\pgfpathlineto{\pgfqpoint{0.958115in}{0.890888in}}%
\pgfpathlineto{\pgfqpoint{0.972976in}{0.831017in}}%
\pgfpathlineto{\pgfqpoint{0.987837in}{0.847884in}}%
\pgfpathlineto{\pgfqpoint{1.002698in}{0.917431in}}%
\pgfpathlineto{\pgfqpoint{1.017559in}{0.919445in}}%
\pgfpathlineto{\pgfqpoint{1.047281in}{0.782003in}}%
\pgfpathlineto{\pgfqpoint{1.062142in}{0.788761in}}%
\pgfpathlineto{\pgfqpoint{1.077003in}{0.887636in}}%
\pgfpathlineto{\pgfqpoint{1.091864in}{1.004448in}}%
\pgfpathlineto{\pgfqpoint{1.106724in}{1.011842in}}%
\pgfpathlineto{\pgfqpoint{1.121585in}{0.861468in}}%
\pgfpathlineto{\pgfqpoint{1.136446in}{0.792471in}}%
\pgfpathlineto{\pgfqpoint{1.151307in}{0.859823in}}%
\pgfpathlineto{\pgfqpoint{1.166168in}{0.880131in}}%
\pgfpathlineto{\pgfqpoint{1.181029in}{0.844926in}}%
\pgfpathlineto{\pgfqpoint{1.195890in}{0.891726in}}%
\pgfpathlineto{\pgfqpoint{1.210751in}{0.825387in}}%
\pgfpathlineto{\pgfqpoint{1.225612in}{0.783673in}}%
\pgfpathlineto{\pgfqpoint{1.240473in}{0.768469in}}%
\pgfpathlineto{\pgfqpoint{1.448527in}{0.768469in}}%
\pgfpathlineto{\pgfqpoint{1.463388in}{0.833969in}}%
\pgfpathlineto{\pgfqpoint{1.478249in}{0.777169in}}%
\pgfpathlineto{\pgfqpoint{1.493110in}{0.840836in}}%
\pgfpathlineto{\pgfqpoint{1.507971in}{0.886879in}}%
\pgfpathlineto{\pgfqpoint{1.522832in}{0.806580in}}%
\pgfpathlineto{\pgfqpoint{1.537693in}{0.770882in}}%
\pgfpathlineto{\pgfqpoint{1.552554in}{0.768469in}}%
\pgfpathlineto{\pgfqpoint{2.979208in}{0.768469in}}%
\pgfpathlineto{\pgfqpoint{2.979208in}{0.768469in}}%
\pgfusepath{stroke}%
\end{pgfscope}%
\begin{pgfscope}%
\pgfpathrectangle{\pgfqpoint{0.709565in}{0.549073in}}{\pgfqpoint{2.325000in}{2.310000in}}%
\pgfusepath{clip}%
\pgfsetrectcap%
\pgfsetroundjoin%
\pgfsetlinewidth{1.505625pt}%
\definecolor{currentstroke}{rgb}{0.862745,0.149020,0.498039}%
\pgfsetstrokecolor{currentstroke}%
\pgfsetdash{}{0pt}%
\pgfpathmoveto{\pgfqpoint{0.764922in}{1.244152in}}%
\pgfpathlineto{\pgfqpoint{0.794644in}{0.879739in}}%
\pgfpathlineto{\pgfqpoint{0.809505in}{0.962537in}}%
\pgfpathlineto{\pgfqpoint{0.824366in}{0.945958in}}%
\pgfpathlineto{\pgfqpoint{0.839227in}{0.817214in}}%
\pgfpathlineto{\pgfqpoint{0.854088in}{0.814212in}}%
\pgfpathlineto{\pgfqpoint{0.868949in}{0.888950in}}%
\pgfpathlineto{\pgfqpoint{0.883810in}{0.834877in}}%
\pgfpathlineto{\pgfqpoint{0.898671in}{0.774503in}}%
\pgfpathlineto{\pgfqpoint{0.913532in}{0.759406in}}%
\pgfpathlineto{\pgfqpoint{0.928393in}{0.768469in}}%
\pgfpathlineto{\pgfqpoint{0.943254in}{0.782768in}}%
\pgfpathlineto{\pgfqpoint{0.958115in}{0.830863in}}%
\pgfpathlineto{\pgfqpoint{0.972976in}{0.890788in}}%
\pgfpathlineto{\pgfqpoint{0.987837in}{0.838444in}}%
\pgfpathlineto{\pgfqpoint{1.002698in}{0.908739in}}%
\pgfpathlineto{\pgfqpoint{1.017559in}{1.010421in}}%
\pgfpathlineto{\pgfqpoint{1.032420in}{1.023496in}}%
\pgfpathlineto{\pgfqpoint{1.047281in}{0.988867in}}%
\pgfpathlineto{\pgfqpoint{1.062142in}{0.847822in}}%
\pgfpathlineto{\pgfqpoint{1.077003in}{0.822917in}}%
\pgfpathlineto{\pgfqpoint{1.091864in}{0.890307in}}%
\pgfpathlineto{\pgfqpoint{1.106724in}{0.829504in}}%
\pgfpathlineto{\pgfqpoint{1.121585in}{0.776248in}}%
\pgfpathlineto{\pgfqpoint{1.136446in}{0.741573in}}%
\pgfpathlineto{\pgfqpoint{1.151307in}{0.768469in}}%
\pgfpathlineto{\pgfqpoint{1.210751in}{0.768469in}}%
\pgfpathlineto{\pgfqpoint{1.225612in}{0.979388in}}%
\pgfpathlineto{\pgfqpoint{1.240473in}{0.778817in}}%
\pgfpathlineto{\pgfqpoint{1.255334in}{0.846522in}}%
\pgfpathlineto{\pgfqpoint{1.270195in}{0.884273in}}%
\pgfpathlineto{\pgfqpoint{1.285056in}{0.802266in}}%
\pgfpathlineto{\pgfqpoint{1.299917in}{0.779226in}}%
\pgfpathlineto{\pgfqpoint{1.314778in}{0.768469in}}%
\pgfpathlineto{\pgfqpoint{1.463388in}{0.768469in}}%
\pgfpathlineto{\pgfqpoint{1.478249in}{0.773371in}}%
\pgfpathlineto{\pgfqpoint{1.493110in}{0.823649in}}%
\pgfpathlineto{\pgfqpoint{1.507971in}{0.890669in}}%
\pgfpathlineto{\pgfqpoint{1.522832in}{0.821702in}}%
\pgfpathlineto{\pgfqpoint{1.537693in}{0.773031in}}%
\pgfpathlineto{\pgfqpoint{1.552554in}{0.768469in}}%
\pgfpathlineto{\pgfqpoint{2.979208in}{0.768469in}}%
\pgfpathlineto{\pgfqpoint{2.979208in}{0.768469in}}%
\pgfusepath{stroke}%
\end{pgfscope}%
\begin{pgfscope}%
\pgfpathrectangle{\pgfqpoint{0.709565in}{0.549073in}}{\pgfqpoint{2.325000in}{2.310000in}}%
\pgfusepath{clip}%
\pgfsetrectcap%
\pgfsetroundjoin%
\pgfsetlinewidth{1.505625pt}%
\definecolor{currentstroke}{rgb}{0.996078,0.380392,0.000000}%
\pgfsetstrokecolor{currentstroke}%
\pgfsetdash{}{0pt}%
\pgfpathmoveto{\pgfqpoint{0.764922in}{1.028894in}}%
\pgfpathlineto{\pgfqpoint{0.779783in}{1.126497in}}%
\pgfpathlineto{\pgfqpoint{0.794644in}{1.045254in}}%
\pgfpathlineto{\pgfqpoint{0.809505in}{1.116331in}}%
\pgfpathlineto{\pgfqpoint{0.824366in}{1.053667in}}%
\pgfpathlineto{\pgfqpoint{0.839227in}{0.926514in}}%
\pgfpathlineto{\pgfqpoint{0.854088in}{0.857373in}}%
\pgfpathlineto{\pgfqpoint{0.868949in}{0.782441in}}%
\pgfpathlineto{\pgfqpoint{0.883810in}{0.782371in}}%
\pgfpathlineto{\pgfqpoint{0.898671in}{0.786701in}}%
\pgfpathlineto{\pgfqpoint{0.913532in}{0.789333in}}%
\pgfpathlineto{\pgfqpoint{0.928393in}{0.867976in}}%
\pgfpathlineto{\pgfqpoint{0.943254in}{0.868279in}}%
\pgfpathlineto{\pgfqpoint{0.958115in}{0.791266in}}%
\pgfpathlineto{\pgfqpoint{0.972976in}{0.823010in}}%
\pgfpathlineto{\pgfqpoint{0.987837in}{0.966065in}}%
\pgfpathlineto{\pgfqpoint{1.002698in}{1.007445in}}%
\pgfpathlineto{\pgfqpoint{1.017559in}{0.959191in}}%
\pgfpathlineto{\pgfqpoint{1.032420in}{0.929563in}}%
\pgfpathlineto{\pgfqpoint{1.047281in}{0.883014in}}%
\pgfpathlineto{\pgfqpoint{1.062142in}{0.877809in}}%
\pgfpathlineto{\pgfqpoint{1.077003in}{0.875173in}}%
\pgfpathlineto{\pgfqpoint{1.091864in}{0.792376in}}%
\pgfpathlineto{\pgfqpoint{1.106724in}{0.761147in}}%
\pgfpathlineto{\pgfqpoint{1.121585in}{0.768469in}}%
\pgfpathlineto{\pgfqpoint{1.210751in}{0.768469in}}%
\pgfpathlineto{\pgfqpoint{1.225612in}{0.771953in}}%
\pgfpathlineto{\pgfqpoint{1.240473in}{0.814833in}}%
\pgfpathlineto{\pgfqpoint{1.255334in}{0.889846in}}%
\pgfpathlineto{\pgfqpoint{1.285056in}{0.774800in}}%
\pgfpathlineto{\pgfqpoint{1.299917in}{0.768469in}}%
\pgfpathlineto{\pgfqpoint{1.448527in}{0.768469in}}%
\pgfpathlineto{\pgfqpoint{1.463388in}{0.771431in}}%
\pgfpathlineto{\pgfqpoint{1.478249in}{0.811028in}}%
\pgfpathlineto{\pgfqpoint{1.493110in}{0.888775in}}%
\pgfpathlineto{\pgfqpoint{1.507971in}{0.835364in}}%
\pgfpathlineto{\pgfqpoint{1.522832in}{0.775786in}}%
\pgfpathlineto{\pgfqpoint{1.537693in}{1.193024in}}%
\pgfpathlineto{\pgfqpoint{1.552554in}{0.768469in}}%
\pgfpathlineto{\pgfqpoint{2.979208in}{0.768469in}}%
\pgfpathlineto{\pgfqpoint{2.979208in}{0.768469in}}%
\pgfusepath{stroke}%
\end{pgfscope}%
\begin{pgfscope}%
\pgfpathrectangle{\pgfqpoint{0.709565in}{0.549073in}}{\pgfqpoint{2.325000in}{2.310000in}}%
\pgfusepath{clip}%
\pgfsetrectcap%
\pgfsetroundjoin%
\pgfsetlinewidth{1.505625pt}%
\definecolor{currentstroke}{rgb}{1.000000,0.690196,0.000000}%
\pgfsetstrokecolor{currentstroke}%
\pgfsetdash{}{0pt}%
\pgfpathmoveto{\pgfqpoint{0.764922in}{0.953074in}}%
\pgfpathlineto{\pgfqpoint{0.794644in}{1.357035in}}%
\pgfpathlineto{\pgfqpoint{0.809505in}{1.140422in}}%
\pgfpathlineto{\pgfqpoint{0.824366in}{0.852135in}}%
\pgfpathlineto{\pgfqpoint{0.839227in}{0.772656in}}%
\pgfpathlineto{\pgfqpoint{0.854088in}{0.769790in}}%
\pgfpathlineto{\pgfqpoint{0.868949in}{0.771374in}}%
\pgfpathlineto{\pgfqpoint{0.883810in}{0.850837in}}%
\pgfpathlineto{\pgfqpoint{0.898671in}{1.002252in}}%
\pgfpathlineto{\pgfqpoint{0.913532in}{0.909114in}}%
\pgfpathlineto{\pgfqpoint{0.928393in}{0.803491in}}%
\pgfpathlineto{\pgfqpoint{0.943254in}{0.864396in}}%
\pgfpathlineto{\pgfqpoint{0.958115in}{0.872512in}}%
\pgfpathlineto{\pgfqpoint{0.972976in}{0.807153in}}%
\pgfpathlineto{\pgfqpoint{1.002698in}{1.006179in}}%
\pgfpathlineto{\pgfqpoint{1.017559in}{0.886501in}}%
\pgfpathlineto{\pgfqpoint{1.032420in}{0.901696in}}%
\pgfpathlineto{\pgfqpoint{1.047281in}{0.927214in}}%
\pgfpathlineto{\pgfqpoint{1.062142in}{0.893161in}}%
\pgfpathlineto{\pgfqpoint{1.077003in}{0.800630in}}%
\pgfpathlineto{\pgfqpoint{1.106724in}{0.768469in}}%
\pgfpathlineto{\pgfqpoint{1.195890in}{0.768469in}}%
\pgfpathlineto{\pgfqpoint{1.210751in}{0.914129in}}%
\pgfpathlineto{\pgfqpoint{1.225612in}{0.790916in}}%
\pgfpathlineto{\pgfqpoint{1.240473in}{0.872603in}}%
\pgfpathlineto{\pgfqpoint{1.255334in}{0.863497in}}%
\pgfpathlineto{\pgfqpoint{1.270195in}{0.785527in}}%
\pgfpathlineto{\pgfqpoint{1.285056in}{0.792579in}}%
\pgfpathlineto{\pgfqpoint{1.299917in}{0.768469in}}%
\pgfpathlineto{\pgfqpoint{1.448527in}{0.768469in}}%
\pgfpathlineto{\pgfqpoint{1.463388in}{0.772345in}}%
\pgfpathlineto{\pgfqpoint{1.478249in}{0.817465in}}%
\pgfpathlineto{\pgfqpoint{1.493110in}{0.890321in}}%
\pgfpathlineto{\pgfqpoint{1.507971in}{0.828079in}}%
\pgfpathlineto{\pgfqpoint{1.522832in}{0.774205in}}%
\pgfpathlineto{\pgfqpoint{1.537693in}{0.768469in}}%
\pgfpathlineto{\pgfqpoint{2.979208in}{0.768469in}}%
\pgfpathlineto{\pgfqpoint{2.979208in}{0.768469in}}%
\pgfusepath{stroke}%
\end{pgfscope}%
\begin{pgfscope}%
\pgfsetrectcap%
\pgfsetmiterjoin%
\pgfsetlinewidth{0.803000pt}%
\definecolor{currentstroke}{rgb}{0.000000,0.000000,0.000000}%
\pgfsetstrokecolor{currentstroke}%
\pgfsetdash{}{0pt}%
\pgfpathmoveto{\pgfqpoint{0.709565in}{0.549073in}}%
\pgfpathlineto{\pgfqpoint{0.709565in}{2.859073in}}%
\pgfusepath{stroke}%
\end{pgfscope}%
\begin{pgfscope}%
\pgfsetrectcap%
\pgfsetmiterjoin%
\pgfsetlinewidth{0.803000pt}%
\definecolor{currentstroke}{rgb}{0.000000,0.000000,0.000000}%
\pgfsetstrokecolor{currentstroke}%
\pgfsetdash{}{0pt}%
\pgfpathmoveto{\pgfqpoint{3.034565in}{0.549073in}}%
\pgfpathlineto{\pgfqpoint{3.034565in}{2.859073in}}%
\pgfusepath{stroke}%
\end{pgfscope}%
\begin{pgfscope}%
\pgfsetrectcap%
\pgfsetmiterjoin%
\pgfsetlinewidth{0.803000pt}%
\definecolor{currentstroke}{rgb}{0.000000,0.000000,0.000000}%
\pgfsetstrokecolor{currentstroke}%
\pgfsetdash{}{0pt}%
\pgfpathmoveto{\pgfqpoint{0.709565in}{0.549073in}}%
\pgfpathlineto{\pgfqpoint{3.034565in}{0.549073in}}%
\pgfusepath{stroke}%
\end{pgfscope}%
\begin{pgfscope}%
\pgfsetrectcap%
\pgfsetmiterjoin%
\pgfsetlinewidth{0.803000pt}%
\definecolor{currentstroke}{rgb}{0.000000,0.000000,0.000000}%
\pgfsetstrokecolor{currentstroke}%
\pgfsetdash{}{0pt}%
\pgfpathmoveto{\pgfqpoint{0.709565in}{2.859073in}}%
\pgfpathlineto{\pgfqpoint{3.034565in}{2.859073in}}%
\pgfusepath{stroke}%
\end{pgfscope}%
\begin{pgfscope}%
\pgfsetbuttcap%
\pgfsetmiterjoin%
\definecolor{currentfill}{rgb}{1.000000,1.000000,1.000000}%
\pgfsetfillcolor{currentfill}%
\pgfsetfillopacity{0.800000}%
\pgfsetlinewidth{1.003750pt}%
\definecolor{currentstroke}{rgb}{0.800000,0.800000,0.800000}%
\pgfsetstrokecolor{currentstroke}%
\pgfsetstrokeopacity{0.800000}%
\pgfsetdash{}{0pt}%
\pgfpathmoveto{\pgfqpoint{1.229813in}{1.563704in}}%
\pgfpathlineto{\pgfqpoint{2.514316in}{1.563704in}}%
\pgfpathlineto{\pgfqpoint{2.514316in}{2.775739in}}%
\pgfpathlineto{\pgfqpoint{1.229813in}{2.775739in}}%
\pgfpathlineto{\pgfqpoint{1.229813in}{1.563704in}}%
\pgfpathclose%
\pgfusepath{stroke,fill}%
\end{pgfscope}%
\begin{pgfscope}%
\pgfsetrectcap%
\pgfsetroundjoin%
\pgfsetlinewidth{1.505625pt}%
\definecolor{currentstroke}{rgb}{0.392157,0.560784,1.000000}%
\pgfsetstrokecolor{currentstroke}%
\pgfsetdash{}{0pt}%
\pgfpathmoveto{\pgfqpoint{1.296480in}{2.650739in}}%
\pgfpathlineto{\pgfqpoint{1.463147in}{2.650739in}}%
\pgfpathlineto{\pgfqpoint{1.629813in}{2.650739in}}%
\pgfusepath{stroke}%
\end{pgfscope}%
\begin{pgfscope}%
\definecolor{textcolor}{rgb}{0.000000,0.000000,0.000000}%
\pgfsetstrokecolor{textcolor}%
\pgfsetfillcolor{textcolor}%
\pgftext[x=1.763147in,y=2.592406in,left,base]{\color{textcolor}{\rmfamily\fontsize{12.000000}{14.400000}\selectfont\catcode`\^=\active\def^{\ifmmode\sp\else\^{}\fi}\catcode`\%=\active\def%{\%}untrained}}%
\end{pgfscope}%
\begin{pgfscope}%
\pgfsetrectcap%
\pgfsetroundjoin%
\pgfsetlinewidth{1.505625pt}%
\definecolor{currentstroke}{rgb}{0.470588,0.368627,0.941176}%
\pgfsetstrokecolor{currentstroke}%
\pgfsetdash{}{0pt}%
\pgfpathmoveto{\pgfqpoint{1.296480in}{2.418332in}}%
\pgfpathlineto{\pgfqpoint{1.463147in}{2.418332in}}%
\pgfpathlineto{\pgfqpoint{1.629813in}{2.418332in}}%
\pgfusepath{stroke}%
\end{pgfscope}%
\begin{pgfscope}%
\definecolor{textcolor}{rgb}{0.000000,0.000000,0.000000}%
\pgfsetstrokecolor{textcolor}%
\pgfsetfillcolor{textcolor}%
\pgftext[x=1.763147in,y=2.359999in,left,base]{\color{textcolor}{\rmfamily\fontsize{12.000000}{14.400000}\selectfont\catcode`\^=\active\def^{\ifmmode\sp\else\^{}\fi}\catcode`\%=\active\def%{\%}epoch $2$}}%
\end{pgfscope}%
\begin{pgfscope}%
\pgfsetrectcap%
\pgfsetroundjoin%
\pgfsetlinewidth{1.505625pt}%
\definecolor{currentstroke}{rgb}{0.862745,0.149020,0.498039}%
\pgfsetstrokecolor{currentstroke}%
\pgfsetdash{}{0pt}%
\pgfpathmoveto{\pgfqpoint{1.296480in}{2.185925in}}%
\pgfpathlineto{\pgfqpoint{1.463147in}{2.185925in}}%
\pgfpathlineto{\pgfqpoint{1.629813in}{2.185925in}}%
\pgfusepath{stroke}%
\end{pgfscope}%
\begin{pgfscope}%
\definecolor{textcolor}{rgb}{0.000000,0.000000,0.000000}%
\pgfsetstrokecolor{textcolor}%
\pgfsetfillcolor{textcolor}%
\pgftext[x=1.763147in,y=2.127592in,left,base]{\color{textcolor}{\rmfamily\fontsize{12.000000}{14.400000}\selectfont\catcode`\^=\active\def^{\ifmmode\sp\else\^{}\fi}\catcode`\%=\active\def%{\%}epoch $4$}}%
\end{pgfscope}%
\begin{pgfscope}%
\pgfsetrectcap%
\pgfsetroundjoin%
\pgfsetlinewidth{1.505625pt}%
\definecolor{currentstroke}{rgb}{0.996078,0.380392,0.000000}%
\pgfsetstrokecolor{currentstroke}%
\pgfsetdash{}{0pt}%
\pgfpathmoveto{\pgfqpoint{1.296480in}{1.953518in}}%
\pgfpathlineto{\pgfqpoint{1.463147in}{1.953518in}}%
\pgfpathlineto{\pgfqpoint{1.629813in}{1.953518in}}%
\pgfusepath{stroke}%
\end{pgfscope}%
\begin{pgfscope}%
\definecolor{textcolor}{rgb}{0.000000,0.000000,0.000000}%
\pgfsetstrokecolor{textcolor}%
\pgfsetfillcolor{textcolor}%
\pgftext[x=1.763147in,y=1.895185in,left,base]{\color{textcolor}{\rmfamily\fontsize{12.000000}{14.400000}\selectfont\catcode`\^=\active\def^{\ifmmode\sp\else\^{}\fi}\catcode`\%=\active\def%{\%}epoch $6$}}%
\end{pgfscope}%
\begin{pgfscope}%
\pgfsetrectcap%
\pgfsetroundjoin%
\pgfsetlinewidth{1.505625pt}%
\definecolor{currentstroke}{rgb}{1.000000,0.690196,0.000000}%
\pgfsetstrokecolor{currentstroke}%
\pgfsetdash{}{0pt}%
\pgfpathmoveto{\pgfqpoint{1.296480in}{1.721111in}}%
\pgfpathlineto{\pgfqpoint{1.463147in}{1.721111in}}%
\pgfpathlineto{\pgfqpoint{1.629813in}{1.721111in}}%
\pgfusepath{stroke}%
\end{pgfscope}%
\begin{pgfscope}%
\definecolor{textcolor}{rgb}{0.000000,0.000000,0.000000}%
\pgfsetstrokecolor{textcolor}%
\pgfsetfillcolor{textcolor}%
\pgftext[x=1.763147in,y=1.662778in,left,base]{\color{textcolor}{\rmfamily\fontsize{12.000000}{14.400000}\selectfont\catcode`\^=\active\def^{\ifmmode\sp\else\^{}\fi}\catcode`\%=\active\def%{\%}epoch $8$}}%
\end{pgfscope}%
\end{pgfpicture}%
\makeatother%
\endgroup%
}
    \end{minipage}\hfill%
    \begin{minipage}[c]{.49\linewidth}
        \centering
        \scalebox{0.8}{%% Creator: Matplotlib, PGF backend
%%
%% To include the figure in your LaTeX document, write
%%   \input{<filename>.pgf}
%%
%% Make sure the required packages are loaded in your preamble
%%   \usepackage{pgf}
%%
%% Also ensure that all the required font packages are loaded; for instance,
%% the lmodern package is sometimes necessary when using math font.
%%   \usepackage{lmodern}
%%
%% Figures using additional raster images can only be included by \input if
%% they are in the same directory as the main LaTeX file. For loading figures
%% from other directories you can use the `import` package
%%   \usepackage{import}
%%
%% and then include the figures with
%%   \import{<path to file>}{<filename>.pgf}
%%
%% Matplotlib used the following preamble
%%   \def\mathdefault#1{#1}
%%   \everymath=\expandafter{\the\everymath\displaystyle}
%%   
%%   \ifdefined\pdftexversion\else  % non-pdftex case.
%%     \usepackage{fontspec}
%%     \setmainfont{DejaVuSerif.ttf}[Path=\detokenize{/opt/hostedtoolcache/Python/3.12.7/x64/lib/python3.12/site-packages/matplotlib/mpl-data/fonts/ttf/}]
%%     \setsansfont{DejaVuSans.ttf}[Path=\detokenize{/opt/hostedtoolcache/Python/3.12.7/x64/lib/python3.12/site-packages/matplotlib/mpl-data/fonts/ttf/}]
%%     \setmonofont{DejaVuSansMono.ttf}[Path=\detokenize{/opt/hostedtoolcache/Python/3.12.7/x64/lib/python3.12/site-packages/matplotlib/mpl-data/fonts/ttf/}]
%%   \fi
%%   \makeatletter\@ifpackageloaded{underscore}{}{\usepackage[strings]{underscore}}\makeatother
%%
\begingroup%
\makeatletter%
\begin{pgfpicture}%
\pgfpathrectangle{\pgfpointorigin}{\pgfqpoint{3.034449in}{2.959073in}}%
\pgfusepath{use as bounding box, clip}%
\begin{pgfscope}%
\pgfsetbuttcap%
\pgfsetmiterjoin%
\definecolor{currentfill}{rgb}{1.000000,1.000000,1.000000}%
\pgfsetfillcolor{currentfill}%
\pgfsetlinewidth{0.000000pt}%
\definecolor{currentstroke}{rgb}{1.000000,1.000000,1.000000}%
\pgfsetstrokecolor{currentstroke}%
\pgfsetdash{}{0pt}%
\pgfpathmoveto{\pgfqpoint{0.000000in}{-0.000000in}}%
\pgfpathlineto{\pgfqpoint{3.034449in}{-0.000000in}}%
\pgfpathlineto{\pgfqpoint{3.034449in}{2.959073in}}%
\pgfpathlineto{\pgfqpoint{0.000000in}{2.959073in}}%
\pgfpathlineto{\pgfqpoint{0.000000in}{-0.000000in}}%
\pgfpathclose%
\pgfusepath{fill}%
\end{pgfscope}%
\begin{pgfscope}%
\pgfsetbuttcap%
\pgfsetmiterjoin%
\definecolor{currentfill}{rgb}{1.000000,1.000000,1.000000}%
\pgfsetfillcolor{currentfill}%
\pgfsetlinewidth{0.000000pt}%
\definecolor{currentstroke}{rgb}{0.000000,0.000000,0.000000}%
\pgfsetstrokecolor{currentstroke}%
\pgfsetstrokeopacity{0.000000}%
\pgfsetdash{}{0pt}%
\pgfpathmoveto{\pgfqpoint{0.609449in}{0.549073in}}%
\pgfpathlineto{\pgfqpoint{2.934449in}{0.549073in}}%
\pgfpathlineto{\pgfqpoint{2.934449in}{2.859073in}}%
\pgfpathlineto{\pgfqpoint{0.609449in}{2.859073in}}%
\pgfpathlineto{\pgfqpoint{0.609449in}{0.549073in}}%
\pgfpathclose%
\pgfusepath{fill}%
\end{pgfscope}%
\begin{pgfscope}%
\pgfpathrectangle{\pgfqpoint{0.609449in}{0.549073in}}{\pgfqpoint{2.325000in}{2.310000in}}%
\pgfusepath{clip}%
\pgfsetrectcap%
\pgfsetroundjoin%
\pgfsetlinewidth{0.250937pt}%
\definecolor{currentstroke}{rgb}{0.000000,0.000000,0.000000}%
\pgfsetstrokecolor{currentstroke}%
\pgfsetstrokeopacity{0.200000}%
\pgfsetdash{}{0pt}%
\pgfpathmoveto{\pgfqpoint{0.664806in}{0.549073in}}%
\pgfpathlineto{\pgfqpoint{0.664806in}{2.859073in}}%
\pgfusepath{stroke}%
\end{pgfscope}%
\begin{pgfscope}%
\pgfsetbuttcap%
\pgfsetroundjoin%
\definecolor{currentfill}{rgb}{0.000000,0.000000,0.000000}%
\pgfsetfillcolor{currentfill}%
\pgfsetlinewidth{0.803000pt}%
\definecolor{currentstroke}{rgb}{0.000000,0.000000,0.000000}%
\pgfsetstrokecolor{currentstroke}%
\pgfsetdash{}{0pt}%
\pgfsys@defobject{currentmarker}{\pgfqpoint{0.000000in}{-0.048611in}}{\pgfqpoint{0.000000in}{0.000000in}}{%
\pgfpathmoveto{\pgfqpoint{0.000000in}{0.000000in}}%
\pgfpathlineto{\pgfqpoint{0.000000in}{-0.048611in}}%
\pgfusepath{stroke,fill}%
}%
\begin{pgfscope}%
\pgfsys@transformshift{0.664806in}{0.549073in}%
\pgfsys@useobject{currentmarker}{}%
\end{pgfscope}%
\end{pgfscope}%
\begin{pgfscope}%
\definecolor{textcolor}{rgb}{0.000000,0.000000,0.000000}%
\pgfsetstrokecolor{textcolor}%
\pgfsetfillcolor{textcolor}%
\pgftext[x=0.664806in,y=0.451851in,,top]{\color{textcolor}{\rmfamily\fontsize{12.000000}{14.400000}\selectfont\catcode`\^=\active\def^{\ifmmode\sp\else\^{}\fi}\catcode`\%=\active\def%{\%}0}}%
\end{pgfscope}%
\begin{pgfscope}%
\pgfpathrectangle{\pgfqpoint{0.609449in}{0.549073in}}{\pgfqpoint{2.325000in}{2.310000in}}%
\pgfusepath{clip}%
\pgfsetrectcap%
\pgfsetroundjoin%
\pgfsetlinewidth{0.250937pt}%
\definecolor{currentstroke}{rgb}{0.000000,0.000000,0.000000}%
\pgfsetstrokecolor{currentstroke}%
\pgfsetstrokeopacity{0.200000}%
\pgfsetdash{}{0pt}%
\pgfpathmoveto{\pgfqpoint{1.218378in}{0.549073in}}%
\pgfpathlineto{\pgfqpoint{1.218378in}{2.859073in}}%
\pgfusepath{stroke}%
\end{pgfscope}%
\begin{pgfscope}%
\pgfsetbuttcap%
\pgfsetroundjoin%
\definecolor{currentfill}{rgb}{0.000000,0.000000,0.000000}%
\pgfsetfillcolor{currentfill}%
\pgfsetlinewidth{0.803000pt}%
\definecolor{currentstroke}{rgb}{0.000000,0.000000,0.000000}%
\pgfsetstrokecolor{currentstroke}%
\pgfsetdash{}{0pt}%
\pgfsys@defobject{currentmarker}{\pgfqpoint{0.000000in}{-0.048611in}}{\pgfqpoint{0.000000in}{0.000000in}}{%
\pgfpathmoveto{\pgfqpoint{0.000000in}{0.000000in}}%
\pgfpathlineto{\pgfqpoint{0.000000in}{-0.048611in}}%
\pgfusepath{stroke,fill}%
}%
\begin{pgfscope}%
\pgfsys@transformshift{1.218378in}{0.549073in}%
\pgfsys@useobject{currentmarker}{}%
\end{pgfscope}%
\end{pgfscope}%
\begin{pgfscope}%
\definecolor{textcolor}{rgb}{0.000000,0.000000,0.000000}%
\pgfsetstrokecolor{textcolor}%
\pgfsetfillcolor{textcolor}%
\pgftext[x=1.218378in,y=0.451851in,,top]{\color{textcolor}{\rmfamily\fontsize{12.000000}{14.400000}\selectfont\catcode`\^=\active\def^{\ifmmode\sp\else\^{}\fi}\catcode`\%=\active\def%{\%}2}}%
\end{pgfscope}%
\begin{pgfscope}%
\pgfpathrectangle{\pgfqpoint{0.609449in}{0.549073in}}{\pgfqpoint{2.325000in}{2.310000in}}%
\pgfusepath{clip}%
\pgfsetrectcap%
\pgfsetroundjoin%
\pgfsetlinewidth{0.250937pt}%
\definecolor{currentstroke}{rgb}{0.000000,0.000000,0.000000}%
\pgfsetstrokecolor{currentstroke}%
\pgfsetstrokeopacity{0.200000}%
\pgfsetdash{}{0pt}%
\pgfpathmoveto{\pgfqpoint{1.771949in}{0.549073in}}%
\pgfpathlineto{\pgfqpoint{1.771949in}{2.859073in}}%
\pgfusepath{stroke}%
\end{pgfscope}%
\begin{pgfscope}%
\pgfsetbuttcap%
\pgfsetroundjoin%
\definecolor{currentfill}{rgb}{0.000000,0.000000,0.000000}%
\pgfsetfillcolor{currentfill}%
\pgfsetlinewidth{0.803000pt}%
\definecolor{currentstroke}{rgb}{0.000000,0.000000,0.000000}%
\pgfsetstrokecolor{currentstroke}%
\pgfsetdash{}{0pt}%
\pgfsys@defobject{currentmarker}{\pgfqpoint{0.000000in}{-0.048611in}}{\pgfqpoint{0.000000in}{0.000000in}}{%
\pgfpathmoveto{\pgfqpoint{0.000000in}{0.000000in}}%
\pgfpathlineto{\pgfqpoint{0.000000in}{-0.048611in}}%
\pgfusepath{stroke,fill}%
}%
\begin{pgfscope}%
\pgfsys@transformshift{1.771949in}{0.549073in}%
\pgfsys@useobject{currentmarker}{}%
\end{pgfscope}%
\end{pgfscope}%
\begin{pgfscope}%
\definecolor{textcolor}{rgb}{0.000000,0.000000,0.000000}%
\pgfsetstrokecolor{textcolor}%
\pgfsetfillcolor{textcolor}%
\pgftext[x=1.771949in,y=0.451851in,,top]{\color{textcolor}{\rmfamily\fontsize{12.000000}{14.400000}\selectfont\catcode`\^=\active\def^{\ifmmode\sp\else\^{}\fi}\catcode`\%=\active\def%{\%}4}}%
\end{pgfscope}%
\begin{pgfscope}%
\pgfpathrectangle{\pgfqpoint{0.609449in}{0.549073in}}{\pgfqpoint{2.325000in}{2.310000in}}%
\pgfusepath{clip}%
\pgfsetrectcap%
\pgfsetroundjoin%
\pgfsetlinewidth{0.250937pt}%
\definecolor{currentstroke}{rgb}{0.000000,0.000000,0.000000}%
\pgfsetstrokecolor{currentstroke}%
\pgfsetstrokeopacity{0.200000}%
\pgfsetdash{}{0pt}%
\pgfpathmoveto{\pgfqpoint{2.325521in}{0.549073in}}%
\pgfpathlineto{\pgfqpoint{2.325521in}{2.859073in}}%
\pgfusepath{stroke}%
\end{pgfscope}%
\begin{pgfscope}%
\pgfsetbuttcap%
\pgfsetroundjoin%
\definecolor{currentfill}{rgb}{0.000000,0.000000,0.000000}%
\pgfsetfillcolor{currentfill}%
\pgfsetlinewidth{0.803000pt}%
\definecolor{currentstroke}{rgb}{0.000000,0.000000,0.000000}%
\pgfsetstrokecolor{currentstroke}%
\pgfsetdash{}{0pt}%
\pgfsys@defobject{currentmarker}{\pgfqpoint{0.000000in}{-0.048611in}}{\pgfqpoint{0.000000in}{0.000000in}}{%
\pgfpathmoveto{\pgfqpoint{0.000000in}{0.000000in}}%
\pgfpathlineto{\pgfqpoint{0.000000in}{-0.048611in}}%
\pgfusepath{stroke,fill}%
}%
\begin{pgfscope}%
\pgfsys@transformshift{2.325521in}{0.549073in}%
\pgfsys@useobject{currentmarker}{}%
\end{pgfscope}%
\end{pgfscope}%
\begin{pgfscope}%
\definecolor{textcolor}{rgb}{0.000000,0.000000,0.000000}%
\pgfsetstrokecolor{textcolor}%
\pgfsetfillcolor{textcolor}%
\pgftext[x=2.325521in,y=0.451851in,,top]{\color{textcolor}{\rmfamily\fontsize{12.000000}{14.400000}\selectfont\catcode`\^=\active\def^{\ifmmode\sp\else\^{}\fi}\catcode`\%=\active\def%{\%}6}}%
\end{pgfscope}%
\begin{pgfscope}%
\pgfpathrectangle{\pgfqpoint{0.609449in}{0.549073in}}{\pgfqpoint{2.325000in}{2.310000in}}%
\pgfusepath{clip}%
\pgfsetrectcap%
\pgfsetroundjoin%
\pgfsetlinewidth{0.250937pt}%
\definecolor{currentstroke}{rgb}{0.000000,0.000000,0.000000}%
\pgfsetstrokecolor{currentstroke}%
\pgfsetstrokeopacity{0.200000}%
\pgfsetdash{}{0pt}%
\pgfpathmoveto{\pgfqpoint{2.879092in}{0.549073in}}%
\pgfpathlineto{\pgfqpoint{2.879092in}{2.859073in}}%
\pgfusepath{stroke}%
\end{pgfscope}%
\begin{pgfscope}%
\pgfsetbuttcap%
\pgfsetroundjoin%
\definecolor{currentfill}{rgb}{0.000000,0.000000,0.000000}%
\pgfsetfillcolor{currentfill}%
\pgfsetlinewidth{0.803000pt}%
\definecolor{currentstroke}{rgb}{0.000000,0.000000,0.000000}%
\pgfsetstrokecolor{currentstroke}%
\pgfsetdash{}{0pt}%
\pgfsys@defobject{currentmarker}{\pgfqpoint{0.000000in}{-0.048611in}}{\pgfqpoint{0.000000in}{0.000000in}}{%
\pgfpathmoveto{\pgfqpoint{0.000000in}{0.000000in}}%
\pgfpathlineto{\pgfqpoint{0.000000in}{-0.048611in}}%
\pgfusepath{stroke,fill}%
}%
\begin{pgfscope}%
\pgfsys@transformshift{2.879092in}{0.549073in}%
\pgfsys@useobject{currentmarker}{}%
\end{pgfscope}%
\end{pgfscope}%
\begin{pgfscope}%
\definecolor{textcolor}{rgb}{0.000000,0.000000,0.000000}%
\pgfsetstrokecolor{textcolor}%
\pgfsetfillcolor{textcolor}%
\pgftext[x=2.879092in,y=0.451851in,,top]{\color{textcolor}{\rmfamily\fontsize{12.000000}{14.400000}\selectfont\catcode`\^=\active\def^{\ifmmode\sp\else\^{}\fi}\catcode`\%=\active\def%{\%}8}}%
\end{pgfscope}%
\begin{pgfscope}%
\definecolor{textcolor}{rgb}{0.000000,0.000000,0.000000}%
\pgfsetstrokecolor{textcolor}%
\pgfsetfillcolor{textcolor}%
\pgftext[x=1.771949in,y=0.248148in,,top]{\color{textcolor}{\rmfamily\fontsize{12.000000}{14.400000}\selectfont\catcode`\^=\active\def^{\ifmmode\sp\else\^{}\fi}\catcode`\%=\active\def%{\%}epoch}}%
\end{pgfscope}%
\begin{pgfscope}%
\pgfpathrectangle{\pgfqpoint{0.609449in}{0.549073in}}{\pgfqpoint{2.325000in}{2.310000in}}%
\pgfusepath{clip}%
\pgfsetrectcap%
\pgfsetroundjoin%
\pgfsetlinewidth{0.250937pt}%
\definecolor{currentstroke}{rgb}{0.000000,0.000000,0.000000}%
\pgfsetstrokecolor{currentstroke}%
\pgfsetstrokeopacity{0.200000}%
\pgfsetdash{}{0pt}%
\pgfpathmoveto{\pgfqpoint{0.609449in}{0.581051in}}%
\pgfpathlineto{\pgfqpoint{2.934449in}{0.581051in}}%
\pgfusepath{stroke}%
\end{pgfscope}%
\begin{pgfscope}%
\pgfsetbuttcap%
\pgfsetroundjoin%
\definecolor{currentfill}{rgb}{0.000000,0.000000,0.000000}%
\pgfsetfillcolor{currentfill}%
\pgfsetlinewidth{0.803000pt}%
\definecolor{currentstroke}{rgb}{0.000000,0.000000,0.000000}%
\pgfsetstrokecolor{currentstroke}%
\pgfsetdash{}{0pt}%
\pgfsys@defobject{currentmarker}{\pgfqpoint{-0.048611in}{0.000000in}}{\pgfqpoint{-0.000000in}{0.000000in}}{%
\pgfpathmoveto{\pgfqpoint{-0.000000in}{0.000000in}}%
\pgfpathlineto{\pgfqpoint{-0.048611in}{0.000000in}}%
\pgfusepath{stroke,fill}%
}%
\begin{pgfscope}%
\pgfsys@transformshift{0.609449in}{0.581051in}%
\pgfsys@useobject{currentmarker}{}%
\end{pgfscope}%
\end{pgfscope}%
\begin{pgfscope}%
\definecolor{textcolor}{rgb}{0.000000,0.000000,0.000000}%
\pgfsetstrokecolor{textcolor}%
\pgfsetfillcolor{textcolor}%
\pgftext[x=0.303703in, y=0.523181in, left, base]{\color{textcolor}{\rmfamily\fontsize{12.000000}{14.400000}\selectfont\catcode`\^=\active\def^{\ifmmode\sp\else\^{}\fi}\catcode`\%=\active\def%{\%}$\mathdefault{0.0}$}}%
\end{pgfscope}%
\begin{pgfscope}%
\pgfpathrectangle{\pgfqpoint{0.609449in}{0.549073in}}{\pgfqpoint{2.325000in}{2.310000in}}%
\pgfusepath{clip}%
\pgfsetrectcap%
\pgfsetroundjoin%
\pgfsetlinewidth{0.250937pt}%
\definecolor{currentstroke}{rgb}{0.000000,0.000000,0.000000}%
\pgfsetstrokecolor{currentstroke}%
\pgfsetstrokeopacity{0.200000}%
\pgfsetdash{}{0pt}%
\pgfpathmoveto{\pgfqpoint{0.609449in}{1.280225in}}%
\pgfpathlineto{\pgfqpoint{2.934449in}{1.280225in}}%
\pgfusepath{stroke}%
\end{pgfscope}%
\begin{pgfscope}%
\pgfsetbuttcap%
\pgfsetroundjoin%
\definecolor{currentfill}{rgb}{0.000000,0.000000,0.000000}%
\pgfsetfillcolor{currentfill}%
\pgfsetlinewidth{0.803000pt}%
\definecolor{currentstroke}{rgb}{0.000000,0.000000,0.000000}%
\pgfsetstrokecolor{currentstroke}%
\pgfsetdash{}{0pt}%
\pgfsys@defobject{currentmarker}{\pgfqpoint{-0.048611in}{0.000000in}}{\pgfqpoint{-0.000000in}{0.000000in}}{%
\pgfpathmoveto{\pgfqpoint{-0.000000in}{0.000000in}}%
\pgfpathlineto{\pgfqpoint{-0.048611in}{0.000000in}}%
\pgfusepath{stroke,fill}%
}%
\begin{pgfscope}%
\pgfsys@transformshift{0.609449in}{1.280225in}%
\pgfsys@useobject{currentmarker}{}%
\end{pgfscope}%
\end{pgfscope}%
\begin{pgfscope}%
\definecolor{textcolor}{rgb}{0.000000,0.000000,0.000000}%
\pgfsetstrokecolor{textcolor}%
\pgfsetfillcolor{textcolor}%
\pgftext[x=0.303703in, y=1.222355in, left, base]{\color{textcolor}{\rmfamily\fontsize{12.000000}{14.400000}\selectfont\catcode`\^=\active\def^{\ifmmode\sp\else\^{}\fi}\catcode`\%=\active\def%{\%}$\mathdefault{0.1}$}}%
\end{pgfscope}%
\begin{pgfscope}%
\pgfpathrectangle{\pgfqpoint{0.609449in}{0.549073in}}{\pgfqpoint{2.325000in}{2.310000in}}%
\pgfusepath{clip}%
\pgfsetrectcap%
\pgfsetroundjoin%
\pgfsetlinewidth{0.250937pt}%
\definecolor{currentstroke}{rgb}{0.000000,0.000000,0.000000}%
\pgfsetstrokecolor{currentstroke}%
\pgfsetstrokeopacity{0.200000}%
\pgfsetdash{}{0pt}%
\pgfpathmoveto{\pgfqpoint{0.609449in}{1.979399in}}%
\pgfpathlineto{\pgfqpoint{2.934449in}{1.979399in}}%
\pgfusepath{stroke}%
\end{pgfscope}%
\begin{pgfscope}%
\pgfsetbuttcap%
\pgfsetroundjoin%
\definecolor{currentfill}{rgb}{0.000000,0.000000,0.000000}%
\pgfsetfillcolor{currentfill}%
\pgfsetlinewidth{0.803000pt}%
\definecolor{currentstroke}{rgb}{0.000000,0.000000,0.000000}%
\pgfsetstrokecolor{currentstroke}%
\pgfsetdash{}{0pt}%
\pgfsys@defobject{currentmarker}{\pgfqpoint{-0.048611in}{0.000000in}}{\pgfqpoint{-0.000000in}{0.000000in}}{%
\pgfpathmoveto{\pgfqpoint{-0.000000in}{0.000000in}}%
\pgfpathlineto{\pgfqpoint{-0.048611in}{0.000000in}}%
\pgfusepath{stroke,fill}%
}%
\begin{pgfscope}%
\pgfsys@transformshift{0.609449in}{1.979399in}%
\pgfsys@useobject{currentmarker}{}%
\end{pgfscope}%
\end{pgfscope}%
\begin{pgfscope}%
\definecolor{textcolor}{rgb}{0.000000,0.000000,0.000000}%
\pgfsetstrokecolor{textcolor}%
\pgfsetfillcolor{textcolor}%
\pgftext[x=0.303703in, y=1.921529in, left, base]{\color{textcolor}{\rmfamily\fontsize{12.000000}{14.400000}\selectfont\catcode`\^=\active\def^{\ifmmode\sp\else\^{}\fi}\catcode`\%=\active\def%{\%}$\mathdefault{0.2}$}}%
\end{pgfscope}%
\begin{pgfscope}%
\pgfpathrectangle{\pgfqpoint{0.609449in}{0.549073in}}{\pgfqpoint{2.325000in}{2.310000in}}%
\pgfusepath{clip}%
\pgfsetrectcap%
\pgfsetroundjoin%
\pgfsetlinewidth{0.250937pt}%
\definecolor{currentstroke}{rgb}{0.000000,0.000000,0.000000}%
\pgfsetstrokecolor{currentstroke}%
\pgfsetstrokeopacity{0.200000}%
\pgfsetdash{}{0pt}%
\pgfpathmoveto{\pgfqpoint{0.609449in}{2.678573in}}%
\pgfpathlineto{\pgfqpoint{2.934449in}{2.678573in}}%
\pgfusepath{stroke}%
\end{pgfscope}%
\begin{pgfscope}%
\pgfsetbuttcap%
\pgfsetroundjoin%
\definecolor{currentfill}{rgb}{0.000000,0.000000,0.000000}%
\pgfsetfillcolor{currentfill}%
\pgfsetlinewidth{0.803000pt}%
\definecolor{currentstroke}{rgb}{0.000000,0.000000,0.000000}%
\pgfsetstrokecolor{currentstroke}%
\pgfsetdash{}{0pt}%
\pgfsys@defobject{currentmarker}{\pgfqpoint{-0.048611in}{0.000000in}}{\pgfqpoint{-0.000000in}{0.000000in}}{%
\pgfpathmoveto{\pgfqpoint{-0.000000in}{0.000000in}}%
\pgfpathlineto{\pgfqpoint{-0.048611in}{0.000000in}}%
\pgfusepath{stroke,fill}%
}%
\begin{pgfscope}%
\pgfsys@transformshift{0.609449in}{2.678573in}%
\pgfsys@useobject{currentmarker}{}%
\end{pgfscope}%
\end{pgfscope}%
\begin{pgfscope}%
\definecolor{textcolor}{rgb}{0.000000,0.000000,0.000000}%
\pgfsetstrokecolor{textcolor}%
\pgfsetfillcolor{textcolor}%
\pgftext[x=0.303703in, y=2.620702in, left, base]{\color{textcolor}{\rmfamily\fontsize{12.000000}{14.400000}\selectfont\catcode`\^=\active\def^{\ifmmode\sp\else\^{}\fi}\catcode`\%=\active\def%{\%}$\mathdefault{0.3}$}}%
\end{pgfscope}%
\begin{pgfscope}%
\definecolor{textcolor}{rgb}{0.000000,0.000000,0.000000}%
\pgfsetstrokecolor{textcolor}%
\pgfsetfillcolor{textcolor}%
\pgftext[x=0.248148in,y=1.704073in,,bottom,rotate=90.000000]{\color{textcolor}{\rmfamily\fontsize{12.000000}{14.400000}\selectfont\catcode`\^=\active\def^{\ifmmode\sp\else\^{}\fi}\catcode`\%=\active\def%{\%}training loss}}%
\end{pgfscope}%
\begin{pgfscope}%
\pgfpathrectangle{\pgfqpoint{0.609449in}{0.549073in}}{\pgfqpoint{2.325000in}{2.310000in}}%
\pgfusepath{clip}%
\pgfsetrectcap%
\pgfsetroundjoin%
\pgfsetlinewidth{1.505625pt}%
\definecolor{currentstroke}{rgb}{0.000000,0.000000,0.000000}%
\pgfsetstrokecolor{currentstroke}%
\pgfsetdash{}{0pt}%
\pgfpathmoveto{\pgfqpoint{0.664806in}{2.666573in}}%
\pgfpathlineto{\pgfqpoint{0.941592in}{1.089056in}}%
\pgfpathlineto{\pgfqpoint{1.218378in}{0.864258in}}%
\pgfpathlineto{\pgfqpoint{1.495164in}{0.805542in}}%
\pgfpathlineto{\pgfqpoint{1.771949in}{0.779204in}}%
\pgfpathlineto{\pgfqpoint{2.048735in}{0.764110in}}%
\pgfpathlineto{\pgfqpoint{2.325521in}{0.754160in}}%
\pgfpathlineto{\pgfqpoint{2.602306in}{0.747012in}}%
\pgfpathlineto{\pgfqpoint{2.879092in}{0.741573in}}%
\pgfusepath{stroke}%
\end{pgfscope}%
\begin{pgfscope}%
\pgfpathrectangle{\pgfqpoint{0.609449in}{0.549073in}}{\pgfqpoint{2.325000in}{2.310000in}}%
\pgfusepath{clip}%
\pgfsetbuttcap%
\pgfsetmiterjoin%
\definecolor{currentfill}{rgb}{0.000000,0.000000,0.000000}%
\pgfsetfillcolor{currentfill}%
\pgfsetlinewidth{1.003750pt}%
\definecolor{currentstroke}{rgb}{0.000000,0.000000,0.000000}%
\pgfsetstrokecolor{currentstroke}%
\pgfsetdash{}{0pt}%
\pgfsys@defobject{currentmarker}{\pgfqpoint{-0.035355in}{-0.058926in}}{\pgfqpoint{0.035355in}{0.058926in}}{%
\pgfpathmoveto{\pgfqpoint{-0.000000in}{-0.058926in}}%
\pgfpathlineto{\pgfqpoint{0.035355in}{0.000000in}}%
\pgfpathlineto{\pgfqpoint{0.000000in}{0.058926in}}%
\pgfpathlineto{\pgfqpoint{-0.035355in}{0.000000in}}%
\pgfpathlineto{\pgfqpoint{-0.000000in}{-0.058926in}}%
\pgfpathclose%
\pgfusepath{stroke,fill}%
}%
\begin{pgfscope}%
\pgfsys@transformshift{0.664806in}{2.666573in}%
\pgfsys@useobject{currentmarker}{}%
\end{pgfscope}%
\begin{pgfscope}%
\pgfsys@transformshift{0.941592in}{1.089056in}%
\pgfsys@useobject{currentmarker}{}%
\end{pgfscope}%
\begin{pgfscope}%
\pgfsys@transformshift{1.218378in}{0.864258in}%
\pgfsys@useobject{currentmarker}{}%
\end{pgfscope}%
\begin{pgfscope}%
\pgfsys@transformshift{1.495164in}{0.805542in}%
\pgfsys@useobject{currentmarker}{}%
\end{pgfscope}%
\begin{pgfscope}%
\pgfsys@transformshift{1.771949in}{0.779204in}%
\pgfsys@useobject{currentmarker}{}%
\end{pgfscope}%
\begin{pgfscope}%
\pgfsys@transformshift{2.048735in}{0.764110in}%
\pgfsys@useobject{currentmarker}{}%
\end{pgfscope}%
\begin{pgfscope}%
\pgfsys@transformshift{2.325521in}{0.754160in}%
\pgfsys@useobject{currentmarker}{}%
\end{pgfscope}%
\begin{pgfscope}%
\pgfsys@transformshift{2.602306in}{0.747012in}%
\pgfsys@useobject{currentmarker}{}%
\end{pgfscope}%
\begin{pgfscope}%
\pgfsys@transformshift{2.879092in}{0.741573in}%
\pgfsys@useobject{currentmarker}{}%
\end{pgfscope}%
\end{pgfscope}%
\begin{pgfscope}%
\pgfsetrectcap%
\pgfsetmiterjoin%
\pgfsetlinewidth{0.803000pt}%
\definecolor{currentstroke}{rgb}{0.000000,0.000000,0.000000}%
\pgfsetstrokecolor{currentstroke}%
\pgfsetdash{}{0pt}%
\pgfpathmoveto{\pgfqpoint{0.609449in}{0.549073in}}%
\pgfpathlineto{\pgfqpoint{0.609449in}{2.859073in}}%
\pgfusepath{stroke}%
\end{pgfscope}%
\begin{pgfscope}%
\pgfsetrectcap%
\pgfsetmiterjoin%
\pgfsetlinewidth{0.803000pt}%
\definecolor{currentstroke}{rgb}{0.000000,0.000000,0.000000}%
\pgfsetstrokecolor{currentstroke}%
\pgfsetdash{}{0pt}%
\pgfpathmoveto{\pgfqpoint{2.934449in}{0.549073in}}%
\pgfpathlineto{\pgfqpoint{2.934449in}{2.859073in}}%
\pgfusepath{stroke}%
\end{pgfscope}%
\begin{pgfscope}%
\pgfsetrectcap%
\pgfsetmiterjoin%
\pgfsetlinewidth{0.803000pt}%
\definecolor{currentstroke}{rgb}{0.000000,0.000000,0.000000}%
\pgfsetstrokecolor{currentstroke}%
\pgfsetdash{}{0pt}%
\pgfpathmoveto{\pgfqpoint{0.609449in}{0.549073in}}%
\pgfpathlineto{\pgfqpoint{2.934449in}{0.549073in}}%
\pgfusepath{stroke}%
\end{pgfscope}%
\begin{pgfscope}%
\pgfsetrectcap%
\pgfsetmiterjoin%
\pgfsetlinewidth{0.803000pt}%
\definecolor{currentstroke}{rgb}{0.000000,0.000000,0.000000}%
\pgfsetstrokecolor{currentstroke}%
\pgfsetdash{}{0pt}%
\pgfpathmoveto{\pgfqpoint{0.609449in}{2.859073in}}%
\pgfpathlineto{\pgfqpoint{2.934449in}{2.859073in}}%
\pgfusepath{stroke}%
\end{pgfscope}%
\end{pgfpicture}%
\makeatother%
\endgroup%
}
    \end{minipage}
    \caption{Mean squared error training loss (right) and corresponding approximate spectral density of the Hessian matrix of a fully connected convolutional neural network in different epochs of training on the MNIST dataset (left). The spectral density is approximated in $n_t=150$ uniformly spaced points using Chebyshev-Nyström++ with parameters $n_{\mtx{\Omega}} = 10$, $n_{\mtx{\Psi}} = 10$, $m = 1000$, and $\sigma = 0.005$.}
    \label{fig:hessian-density}
\end{figure}
