
\section{Numerical experiments}
\label{sec:results}

\color{black}

In this last section, we test the Chebyshev-Nyström++ algorithm in multiple scenarios leading from electronic structure interaction, statistical thermodynamics, and neural network optimization.

While theoretically an important tool, for the remainder of this section, we waive using the non-negative Chebyshev approximation, since the (slight) indefiniteness of $g_{\sigma}^{(m)}(t \mtx{I}_n - \mtx{A})$ \refequ{equ:matrix-approximation} does not seem to matter when computing the Nyström++ estimator in finite precision arithmetic. Unless otherwise mentioned, we transform the matrices such that all their eigenvalues are contained in $[-1, 1]$ by approximating the spectrum with NumPy's Hessian eigenvalue solver. We compute the spectral densities at $n_t = 100$ uniformly distributed values of the parameter $t$ in $[-1, 1]$. The integral involved in computing the $L^1$-errors is approximated with the composite midpoint quadrature rule using the $n_t$ values of $t$ as nodes. Both parameters introduced at the end \refsec{subsubsec:chebyshev-nystrom-implementation},the eigenvalue truncation threshold and the parameter for detecting a vanishing spectral density, are both set to $10^{-5}$.

All our implementations are written in Python 3.12.3 using the packages NumPy 2.0.0 and SciPy 1.14.1. They are executed on a single thread of a GitHub-hosted Ubuntu runner with a 64-bit processor and 16 GB of RAM. 

\subsection{Spectral density for Hamiltonian of electronic structure}
\label{subsec:hamiltonian}

Our first numerical example comes from electronic structure interaction \cite[Section 6]{lin-2017-randomized-estimation}, more precisely from the second order finite difference discretization of the Hamiltonian
\begin{equation}
    \mathcal{H} = - \Delta + V
    \label{equ:5-experiments-electronic-hamiltonian}
\end{equation}
in three dimensions. The potential $V$ interacting with the electrons is generated by Gaussian wells
\begin{equation}
    v(r) = v_0 e^{-\lambda r^2}
    \label{equ:5-experiments-gaussian-cell}
\end{equation}
with $v_0 = -4$ and $\lambda = 8$, centered in cells of side length $L=6$ which are stacked $n_c \in \mathbb{N}$ times in each spatial dimension (cf. \reffig{fig:gaussian-well}). The grid width is fixed to be $h=0.6$. For example, for $n_c = 1$, this leads to a matrix of size $1000 \times 1000$. This is an idealized model for the interaction of nuclei on a regular grid with electrons for a $k$-vector in the center of the first Brillouin zone. The distribution of the eigenvalues of the Hamiltonian --- its spectral density --- allows us to interpret the system's energy levels.

\begin{figure}[ht]
    \begin{subfigure}[b]{0.32\columnwidth}
        %% Creator: Matplotlib, PGF backend
%%
%% To include the figure in your LaTeX document, write
%%   \input{<filename>.pgf}
%%
%% Make sure the required packages are loaded in your preamble
%%   \usepackage{pgf}
%%
%% Also ensure that all the required font packages are loaded; for instance,
%% the lmodern package is sometimes necessary when using math font.
%%   \usepackage{lmodern}
%%
%% Figures using additional raster images can only be included by \input if
%% they are in the same directory as the main LaTeX file. For loading figures
%% from other directories you can use the `import` package
%%   \usepackage{import}
%%
%% and then include the figures with
%%   \import{<path to file>}{<filename>.pgf}
%%
%% Matplotlib used the following preamble
%%   \def\mathdefault#1{#1}
%%   \everymath=\expandafter{\the\everymath\displaystyle}
%%   
%%   \ifdefined\pdftexversion\else  % non-pdftex case.
%%     \usepackage{fontspec}
%%     \setmainfont{DejaVuSerif.ttf}[Path=\detokenize{/opt/hostedtoolcache/Python/3.12.5/x64/lib/python3.12/site-packages/matplotlib/mpl-data/fonts/ttf/}]
%%     \setsansfont{DejaVuSans.ttf}[Path=\detokenize{/opt/hostedtoolcache/Python/3.12.5/x64/lib/python3.12/site-packages/matplotlib/mpl-data/fonts/ttf/}]
%%     \setmonofont{DejaVuSansMono.ttf}[Path=\detokenize{/opt/hostedtoolcache/Python/3.12.5/x64/lib/python3.12/site-packages/matplotlib/mpl-data/fonts/ttf/}]
%%   \fi
%%   \makeatletter\@ifpackageloaded{underscore}{}{\usepackage[strings]{underscore}}\makeatother
%%
\begingroup%
\makeatletter%
\begin{pgfpicture}%
\pgfpathrectangle{\pgfpointorigin}{\pgfqpoint{1.969617in}{1.850740in}}%
\pgfusepath{use as bounding box, clip}%
\begin{pgfscope}%
\pgfsetbuttcap%
\pgfsetmiterjoin%
\definecolor{currentfill}{rgb}{1.000000,1.000000,1.000000}%
\pgfsetfillcolor{currentfill}%
\pgfsetlinewidth{0.000000pt}%
\definecolor{currentstroke}{rgb}{1.000000,1.000000,1.000000}%
\pgfsetstrokecolor{currentstroke}%
\pgfsetdash{}{0pt}%
\pgfpathmoveto{\pgfqpoint{0.000000in}{0.000000in}}%
\pgfpathlineto{\pgfqpoint{1.969617in}{0.000000in}}%
\pgfpathlineto{\pgfqpoint{1.969617in}{1.850740in}}%
\pgfpathlineto{\pgfqpoint{0.000000in}{1.850740in}}%
\pgfpathlineto{\pgfqpoint{0.000000in}{0.000000in}}%
\pgfpathclose%
\pgfusepath{fill}%
\end{pgfscope}%
\begin{pgfscope}%
\pgfsetbuttcap%
\pgfsetmiterjoin%
\definecolor{currentfill}{rgb}{1.000000,1.000000,1.000000}%
\pgfsetfillcolor{currentfill}%
\pgfsetlinewidth{0.000000pt}%
\definecolor{currentstroke}{rgb}{0.000000,0.000000,0.000000}%
\pgfsetstrokecolor{currentstroke}%
\pgfsetstrokeopacity{0.000000}%
\pgfsetdash{}{0pt}%
\pgfpathmoveto{\pgfqpoint{0.278819in}{0.345370in}}%
\pgfpathlineto{\pgfqpoint{1.828819in}{0.345370in}}%
\pgfpathlineto{\pgfqpoint{1.828819in}{1.692870in}}%
\pgfpathlineto{\pgfqpoint{0.278819in}{1.692870in}}%
\pgfpathlineto{\pgfqpoint{0.278819in}{0.345370in}}%
\pgfpathclose%
\pgfusepath{fill}%
\end{pgfscope}%
\begin{pgfscope}%
\pgfpathrectangle{\pgfqpoint{0.278819in}{0.345370in}}{\pgfqpoint{1.550000in}{1.347500in}}%
\pgfusepath{clip}%
\pgfsetbuttcap%
\pgfsetroundjoin%
\definecolor{currentfill}{rgb}{0.972530,0.881250,0.144923}%
\pgfsetfillcolor{currentfill}%
\pgfsetlinewidth{0.000000pt}%
\definecolor{currentstroke}{rgb}{0.000000,0.000000,0.000000}%
\pgfsetstrokecolor{currentstroke}%
\pgfsetdash{}{0pt}%
\pgfpathmoveto{\pgfqpoint{1.014677in}{0.833047in}}%
\pgfpathlineto{\pgfqpoint{1.030334in}{0.830965in}}%
\pgfpathlineto{\pgfqpoint{1.045990in}{0.829924in}}%
\pgfpathlineto{\pgfqpoint{1.061647in}{0.829924in}}%
\pgfpathlineto{\pgfqpoint{1.077303in}{0.830965in}}%
\pgfpathlineto{\pgfqpoint{1.092960in}{0.833047in}}%
\pgfpathlineto{\pgfqpoint{1.104609in}{0.835370in}}%
\pgfpathlineto{\pgfqpoint{1.108617in}{0.836216in}}%
\pgfpathlineto{\pgfqpoint{1.124273in}{0.840617in}}%
\pgfpathlineto{\pgfqpoint{1.139930in}{0.846121in}}%
\pgfpathlineto{\pgfqpoint{1.146722in}{0.848981in}}%
\pgfpathlineto{\pgfqpoint{1.155586in}{0.852970in}}%
\pgfpathlineto{\pgfqpoint{1.171243in}{0.861179in}}%
\pgfpathlineto{\pgfqpoint{1.173612in}{0.862592in}}%
\pgfpathlineto{\pgfqpoint{1.186899in}{0.871149in}}%
\pgfpathlineto{\pgfqpoint{1.193903in}{0.876203in}}%
\pgfpathlineto{\pgfqpoint{1.202556in}{0.883015in}}%
\pgfpathlineto{\pgfqpoint{1.210377in}{0.889814in}}%
\pgfpathlineto{\pgfqpoint{1.218213in}{0.897337in}}%
\pgfpathlineto{\pgfqpoint{1.224026in}{0.903425in}}%
\pgfpathlineto{\pgfqpoint{1.233869in}{0.914977in}}%
\pgfpathlineto{\pgfqpoint{1.235495in}{0.917036in}}%
\pgfpathlineto{\pgfqpoint{1.244937in}{0.930648in}}%
\pgfpathlineto{\pgfqpoint{1.249526in}{0.938354in}}%
\pgfpathlineto{\pgfqpoint{1.252815in}{0.944259in}}%
\pgfpathlineto{\pgfqpoint{1.259146in}{0.957870in}}%
\pgfpathlineto{\pgfqpoint{1.264209in}{0.971481in}}%
\pgfpathlineto{\pgfqpoint{1.265182in}{0.974965in}}%
\pgfpathlineto{\pgfqpoint{1.267854in}{0.985092in}}%
\pgfpathlineto{\pgfqpoint{1.270249in}{0.998703in}}%
\pgfpathlineto{\pgfqpoint{1.271447in}{1.012314in}}%
\pgfpathlineto{\pgfqpoint{1.271447in}{1.025925in}}%
\pgfpathlineto{\pgfqpoint{1.270249in}{1.039536in}}%
\pgfpathlineto{\pgfqpoint{1.267854in}{1.053148in}}%
\pgfpathlineto{\pgfqpoint{1.265182in}{1.063275in}}%
\pgfpathlineto{\pgfqpoint{1.264209in}{1.066759in}}%
\pgfpathlineto{\pgfqpoint{1.259146in}{1.080370in}}%
\pgfpathlineto{\pgfqpoint{1.252815in}{1.093981in}}%
\pgfpathlineto{\pgfqpoint{1.249526in}{1.099886in}}%
\pgfpathlineto{\pgfqpoint{1.244937in}{1.107592in}}%
\pgfpathlineto{\pgfqpoint{1.235495in}{1.121203in}}%
\pgfpathlineto{\pgfqpoint{1.233869in}{1.123262in}}%
\pgfpathlineto{\pgfqpoint{1.224026in}{1.134814in}}%
\pgfpathlineto{\pgfqpoint{1.218213in}{1.140902in}}%
\pgfpathlineto{\pgfqpoint{1.210377in}{1.148425in}}%
\pgfpathlineto{\pgfqpoint{1.202556in}{1.155224in}}%
\pgfpathlineto{\pgfqpoint{1.193903in}{1.162036in}}%
\pgfpathlineto{\pgfqpoint{1.186899in}{1.167090in}}%
\pgfpathlineto{\pgfqpoint{1.173612in}{1.175647in}}%
\pgfpathlineto{\pgfqpoint{1.171243in}{1.177061in}}%
\pgfpathlineto{\pgfqpoint{1.155586in}{1.185269in}}%
\pgfpathlineto{\pgfqpoint{1.146722in}{1.189259in}}%
\pgfpathlineto{\pgfqpoint{1.139930in}{1.192118in}}%
\pgfpathlineto{\pgfqpoint{1.124273in}{1.197622in}}%
\pgfpathlineto{\pgfqpoint{1.108617in}{1.202024in}}%
\pgfpathlineto{\pgfqpoint{1.104609in}{1.202870in}}%
\pgfpathlineto{\pgfqpoint{1.092960in}{1.205193in}}%
\pgfpathlineto{\pgfqpoint{1.077303in}{1.207275in}}%
\pgfpathlineto{\pgfqpoint{1.061647in}{1.208316in}}%
\pgfpathlineto{\pgfqpoint{1.045990in}{1.208316in}}%
\pgfpathlineto{\pgfqpoint{1.030334in}{1.207275in}}%
\pgfpathlineto{\pgfqpoint{1.014677in}{1.205193in}}%
\pgfpathlineto{\pgfqpoint{1.003028in}{1.202870in}}%
\pgfpathlineto{\pgfqpoint{0.999021in}{1.202024in}}%
\pgfpathlineto{\pgfqpoint{0.983364in}{1.197622in}}%
\pgfpathlineto{\pgfqpoint{0.967708in}{1.192118in}}%
\pgfpathlineto{\pgfqpoint{0.960915in}{1.189259in}}%
\pgfpathlineto{\pgfqpoint{0.952051in}{1.185269in}}%
\pgfpathlineto{\pgfqpoint{0.936394in}{1.177061in}}%
\pgfpathlineto{\pgfqpoint{0.934026in}{1.175647in}}%
\pgfpathlineto{\pgfqpoint{0.920738in}{1.167090in}}%
\pgfpathlineto{\pgfqpoint{0.913735in}{1.162036in}}%
\pgfpathlineto{\pgfqpoint{0.905081in}{1.155224in}}%
\pgfpathlineto{\pgfqpoint{0.897261in}{1.148425in}}%
\pgfpathlineto{\pgfqpoint{0.889425in}{1.140902in}}%
\pgfpathlineto{\pgfqpoint{0.883612in}{1.134814in}}%
\pgfpathlineto{\pgfqpoint{0.873768in}{1.123262in}}%
\pgfpathlineto{\pgfqpoint{0.872143in}{1.121203in}}%
\pgfpathlineto{\pgfqpoint{0.862700in}{1.107592in}}%
\pgfpathlineto{\pgfqpoint{0.858112in}{1.099886in}}%
\pgfpathlineto{\pgfqpoint{0.854822in}{1.093981in}}%
\pgfpathlineto{\pgfqpoint{0.848491in}{1.080370in}}%
\pgfpathlineto{\pgfqpoint{0.843428in}{1.066759in}}%
\pgfpathlineto{\pgfqpoint{0.842455in}{1.063275in}}%
\pgfpathlineto{\pgfqpoint{0.839783in}{1.053147in}}%
\pgfpathlineto{\pgfqpoint{0.837388in}{1.039536in}}%
\pgfpathlineto{\pgfqpoint{0.836190in}{1.025925in}}%
\pgfpathlineto{\pgfqpoint{0.836190in}{1.012314in}}%
\pgfpathlineto{\pgfqpoint{0.837388in}{0.998703in}}%
\pgfpathlineto{\pgfqpoint{0.839783in}{0.985092in}}%
\pgfpathlineto{\pgfqpoint{0.842455in}{0.974965in}}%
\pgfpathlineto{\pgfqpoint{0.843428in}{0.971481in}}%
\pgfpathlineto{\pgfqpoint{0.848491in}{0.957870in}}%
\pgfpathlineto{\pgfqpoint{0.854822in}{0.944259in}}%
\pgfpathlineto{\pgfqpoint{0.858112in}{0.938354in}}%
\pgfpathlineto{\pgfqpoint{0.862700in}{0.930648in}}%
\pgfpathlineto{\pgfqpoint{0.872143in}{0.917036in}}%
\pgfpathlineto{\pgfqpoint{0.873768in}{0.914977in}}%
\pgfpathlineto{\pgfqpoint{0.883612in}{0.903425in}}%
\pgfpathlineto{\pgfqpoint{0.889425in}{0.897337in}}%
\pgfpathlineto{\pgfqpoint{0.897261in}{0.889814in}}%
\pgfpathlineto{\pgfqpoint{0.905081in}{0.883015in}}%
\pgfpathlineto{\pgfqpoint{0.913735in}{0.876203in}}%
\pgfpathlineto{\pgfqpoint{0.920738in}{0.871149in}}%
\pgfpathlineto{\pgfqpoint{0.934026in}{0.862592in}}%
\pgfpathlineto{\pgfqpoint{0.936394in}{0.861179in}}%
\pgfpathlineto{\pgfqpoint{0.952051in}{0.852970in}}%
\pgfpathlineto{\pgfqpoint{0.960915in}{0.848981in}}%
\pgfpathlineto{\pgfqpoint{0.967708in}{0.846121in}}%
\pgfpathlineto{\pgfqpoint{0.983364in}{0.840617in}}%
\pgfpathlineto{\pgfqpoint{0.999021in}{0.836216in}}%
\pgfpathlineto{\pgfqpoint{1.003028in}{0.835370in}}%
\pgfpathlineto{\pgfqpoint{1.014677in}{0.833047in}}%
\pgfpathclose%
\pgfusepath{fill}%
\end{pgfscope}%
\begin{pgfscope}%
\pgfpathrectangle{\pgfqpoint{0.278819in}{0.345370in}}{\pgfqpoint{1.550000in}{1.347500in}}%
\pgfusepath{clip}%
\pgfsetbuttcap%
\pgfsetroundjoin%
\definecolor{currentfill}{rgb}{0.993814,0.704741,0.183043}%
\pgfsetfillcolor{currentfill}%
\pgfsetlinewidth{0.000000pt}%
\definecolor{currentstroke}{rgb}{0.000000,0.000000,0.000000}%
\pgfsetstrokecolor{currentstroke}%
\pgfsetdash{}{0pt}%
\pgfpathmoveto{\pgfqpoint{0.999021in}{0.739415in}}%
\pgfpathlineto{\pgfqpoint{1.014677in}{0.737106in}}%
\pgfpathlineto{\pgfqpoint{1.030334in}{0.735568in}}%
\pgfpathlineto{\pgfqpoint{1.045990in}{0.734799in}}%
\pgfpathlineto{\pgfqpoint{1.061647in}{0.734799in}}%
\pgfpathlineto{\pgfqpoint{1.077303in}{0.735568in}}%
\pgfpathlineto{\pgfqpoint{1.092960in}{0.737106in}}%
\pgfpathlineto{\pgfqpoint{1.108617in}{0.739415in}}%
\pgfpathlineto{\pgfqpoint{1.112069in}{0.740092in}}%
\pgfpathlineto{\pgfqpoint{1.124273in}{0.742535in}}%
\pgfpathlineto{\pgfqpoint{1.139930in}{0.746452in}}%
\pgfpathlineto{\pgfqpoint{1.155586in}{0.751154in}}%
\pgfpathlineto{\pgfqpoint{1.162877in}{0.753703in}}%
\pgfpathlineto{\pgfqpoint{1.171243in}{0.756707in}}%
\pgfpathlineto{\pgfqpoint{1.186899in}{0.763121in}}%
\pgfpathlineto{\pgfqpoint{1.196020in}{0.767314in}}%
\pgfpathlineto{\pgfqpoint{1.202556in}{0.770419in}}%
\pgfpathlineto{\pgfqpoint{1.218213in}{0.778656in}}%
\pgfpathlineto{\pgfqpoint{1.222157in}{0.780925in}}%
\pgfpathlineto{\pgfqpoint{1.233869in}{0.787933in}}%
\pgfpathlineto{\pgfqpoint{1.244037in}{0.794536in}}%
\pgfpathlineto{\pgfqpoint{1.249526in}{0.798265in}}%
\pgfpathlineto{\pgfqpoint{1.263051in}{0.808148in}}%
\pgfpathlineto{\pgfqpoint{1.265182in}{0.809786in}}%
\pgfpathlineto{\pgfqpoint{1.279772in}{0.821759in}}%
\pgfpathlineto{\pgfqpoint{1.280839in}{0.822686in}}%
\pgfpathlineto{\pgfqpoint{1.294610in}{0.835370in}}%
\pgfpathlineto{\pgfqpoint{1.296495in}{0.837223in}}%
\pgfpathlineto{\pgfqpoint{1.307864in}{0.848981in}}%
\pgfpathlineto{\pgfqpoint{1.312152in}{0.853753in}}%
\pgfpathlineto{\pgfqpoint{1.319748in}{0.862592in}}%
\pgfpathlineto{\pgfqpoint{1.327809in}{0.872774in}}%
\pgfpathlineto{\pgfqpoint{1.330419in}{0.876203in}}%
\pgfpathlineto{\pgfqpoint{1.339893in}{0.889814in}}%
\pgfpathlineto{\pgfqpoint{1.343465in}{0.895496in}}%
\pgfpathlineto{\pgfqpoint{1.348289in}{0.903425in}}%
\pgfpathlineto{\pgfqpoint{1.355666in}{0.917036in}}%
\pgfpathlineto{\pgfqpoint{1.359122in}{0.924309in}}%
\pgfpathlineto{\pgfqpoint{1.362054in}{0.930648in}}%
\pgfpathlineto{\pgfqpoint{1.367463in}{0.944259in}}%
\pgfpathlineto{\pgfqpoint{1.371968in}{0.957870in}}%
\pgfpathlineto{\pgfqpoint{1.374778in}{0.968479in}}%
\pgfpathlineto{\pgfqpoint{1.375557in}{0.971481in}}%
\pgfpathlineto{\pgfqpoint{1.378212in}{0.985092in}}%
\pgfpathlineto{\pgfqpoint{1.379982in}{0.998703in}}%
\pgfpathlineto{\pgfqpoint{1.380866in}{1.012314in}}%
\pgfpathlineto{\pgfqpoint{1.380866in}{1.025925in}}%
\pgfpathlineto{\pgfqpoint{1.379982in}{1.039536in}}%
\pgfpathlineto{\pgfqpoint{1.378212in}{1.053148in}}%
\pgfpathlineto{\pgfqpoint{1.375557in}{1.066759in}}%
\pgfpathlineto{\pgfqpoint{1.374778in}{1.069760in}}%
\pgfpathlineto{\pgfqpoint{1.371968in}{1.080370in}}%
\pgfpathlineto{\pgfqpoint{1.367463in}{1.093981in}}%
\pgfpathlineto{\pgfqpoint{1.362054in}{1.107592in}}%
\pgfpathlineto{\pgfqpoint{1.359122in}{1.113930in}}%
\pgfpathlineto{\pgfqpoint{1.355666in}{1.121203in}}%
\pgfpathlineto{\pgfqpoint{1.348289in}{1.134814in}}%
\pgfpathlineto{\pgfqpoint{1.343465in}{1.142743in}}%
\pgfpathlineto{\pgfqpoint{1.339893in}{1.148425in}}%
\pgfpathlineto{\pgfqpoint{1.330419in}{1.162036in}}%
\pgfpathlineto{\pgfqpoint{1.327809in}{1.165465in}}%
\pgfpathlineto{\pgfqpoint{1.319748in}{1.175647in}}%
\pgfpathlineto{\pgfqpoint{1.312152in}{1.184487in}}%
\pgfpathlineto{\pgfqpoint{1.307864in}{1.189259in}}%
\pgfpathlineto{\pgfqpoint{1.296495in}{1.201017in}}%
\pgfpathlineto{\pgfqpoint{1.294610in}{1.202870in}}%
\pgfpathlineto{\pgfqpoint{1.280839in}{1.215553in}}%
\pgfpathlineto{\pgfqpoint{1.279772in}{1.216481in}}%
\pgfpathlineto{\pgfqpoint{1.265182in}{1.228453in}}%
\pgfpathlineto{\pgfqpoint{1.263051in}{1.230092in}}%
\pgfpathlineto{\pgfqpoint{1.249526in}{1.239975in}}%
\pgfpathlineto{\pgfqpoint{1.244037in}{1.243703in}}%
\pgfpathlineto{\pgfqpoint{1.233869in}{1.250307in}}%
\pgfpathlineto{\pgfqpoint{1.222157in}{1.257314in}}%
\pgfpathlineto{\pgfqpoint{1.218213in}{1.259584in}}%
\pgfpathlineto{\pgfqpoint{1.202556in}{1.267820in}}%
\pgfpathlineto{\pgfqpoint{1.196020in}{1.270925in}}%
\pgfpathlineto{\pgfqpoint{1.186899in}{1.275119in}}%
\pgfpathlineto{\pgfqpoint{1.171243in}{1.281532in}}%
\pgfpathlineto{\pgfqpoint{1.162877in}{1.284536in}}%
\pgfpathlineto{\pgfqpoint{1.155586in}{1.287085in}}%
\pgfpathlineto{\pgfqpoint{1.139930in}{1.291788in}}%
\pgfpathlineto{\pgfqpoint{1.124273in}{1.295705in}}%
\pgfpathlineto{\pgfqpoint{1.112069in}{1.298148in}}%
\pgfpathlineto{\pgfqpoint{1.108617in}{1.298825in}}%
\pgfpathlineto{\pgfqpoint{1.092960in}{1.301133in}}%
\pgfpathlineto{\pgfqpoint{1.077303in}{1.302671in}}%
\pgfpathlineto{\pgfqpoint{1.061647in}{1.303440in}}%
\pgfpathlineto{\pgfqpoint{1.045990in}{1.303440in}}%
\pgfpathlineto{\pgfqpoint{1.030334in}{1.302671in}}%
\pgfpathlineto{\pgfqpoint{1.014677in}{1.301133in}}%
\pgfpathlineto{\pgfqpoint{0.999021in}{1.298825in}}%
\pgfpathlineto{\pgfqpoint{0.995568in}{1.298148in}}%
\pgfpathlineto{\pgfqpoint{0.983364in}{1.295705in}}%
\pgfpathlineto{\pgfqpoint{0.967708in}{1.291788in}}%
\pgfpathlineto{\pgfqpoint{0.952051in}{1.287085in}}%
\pgfpathlineto{\pgfqpoint{0.944760in}{1.284536in}}%
\pgfpathlineto{\pgfqpoint{0.936394in}{1.281532in}}%
\pgfpathlineto{\pgfqpoint{0.920738in}{1.275119in}}%
\pgfpathlineto{\pgfqpoint{0.911617in}{1.270925in}}%
\pgfpathlineto{\pgfqpoint{0.905081in}{1.267820in}}%
\pgfpathlineto{\pgfqpoint{0.889425in}{1.259584in}}%
\pgfpathlineto{\pgfqpoint{0.885481in}{1.257314in}}%
\pgfpathlineto{\pgfqpoint{0.873768in}{1.250307in}}%
\pgfpathlineto{\pgfqpoint{0.863600in}{1.243703in}}%
\pgfpathlineto{\pgfqpoint{0.858112in}{1.239975in}}%
\pgfpathlineto{\pgfqpoint{0.844586in}{1.230092in}}%
\pgfpathlineto{\pgfqpoint{0.842455in}{1.228453in}}%
\pgfpathlineto{\pgfqpoint{0.827865in}{1.216481in}}%
\pgfpathlineto{\pgfqpoint{0.826798in}{1.215553in}}%
\pgfpathlineto{\pgfqpoint{0.813027in}{1.202870in}}%
\pgfpathlineto{\pgfqpoint{0.811142in}{1.201017in}}%
\pgfpathlineto{\pgfqpoint{0.799774in}{1.189259in}}%
\pgfpathlineto{\pgfqpoint{0.795485in}{1.184487in}}%
\pgfpathlineto{\pgfqpoint{0.787889in}{1.175647in}}%
\pgfpathlineto{\pgfqpoint{0.779829in}{1.165465in}}%
\pgfpathlineto{\pgfqpoint{0.777218in}{1.162036in}}%
\pgfpathlineto{\pgfqpoint{0.767744in}{1.148425in}}%
\pgfpathlineto{\pgfqpoint{0.764172in}{1.142743in}}%
\pgfpathlineto{\pgfqpoint{0.759348in}{1.134814in}}%
\pgfpathlineto{\pgfqpoint{0.751971in}{1.121203in}}%
\pgfpathlineto{\pgfqpoint{0.748516in}{1.113930in}}%
\pgfpathlineto{\pgfqpoint{0.745583in}{1.107592in}}%
\pgfpathlineto{\pgfqpoint{0.740174in}{1.093981in}}%
\pgfpathlineto{\pgfqpoint{0.735669in}{1.080370in}}%
\pgfpathlineto{\pgfqpoint{0.732859in}{1.069760in}}%
\pgfpathlineto{\pgfqpoint{0.732080in}{1.066759in}}%
\pgfpathlineto{\pgfqpoint{0.729425in}{1.053148in}}%
\pgfpathlineto{\pgfqpoint{0.727655in}{1.039536in}}%
\pgfpathlineto{\pgfqpoint{0.726771in}{1.025925in}}%
\pgfpathlineto{\pgfqpoint{0.726771in}{1.012314in}}%
\pgfpathlineto{\pgfqpoint{0.727655in}{0.998703in}}%
\pgfpathlineto{\pgfqpoint{0.729425in}{0.985092in}}%
\pgfpathlineto{\pgfqpoint{0.732080in}{0.971481in}}%
\pgfpathlineto{\pgfqpoint{0.732859in}{0.968479in}}%
\pgfpathlineto{\pgfqpoint{0.735669in}{0.957870in}}%
\pgfpathlineto{\pgfqpoint{0.740174in}{0.944259in}}%
\pgfpathlineto{\pgfqpoint{0.745583in}{0.930648in}}%
\pgfpathlineto{\pgfqpoint{0.748516in}{0.924309in}}%
\pgfpathlineto{\pgfqpoint{0.751971in}{0.917036in}}%
\pgfpathlineto{\pgfqpoint{0.759348in}{0.903425in}}%
\pgfpathlineto{\pgfqpoint{0.764172in}{0.895496in}}%
\pgfpathlineto{\pgfqpoint{0.767744in}{0.889814in}}%
\pgfpathlineto{\pgfqpoint{0.777218in}{0.876203in}}%
\pgfpathlineto{\pgfqpoint{0.779829in}{0.872774in}}%
\pgfpathlineto{\pgfqpoint{0.787889in}{0.862592in}}%
\pgfpathlineto{\pgfqpoint{0.795485in}{0.853753in}}%
\pgfpathlineto{\pgfqpoint{0.799774in}{0.848981in}}%
\pgfpathlineto{\pgfqpoint{0.811142in}{0.837223in}}%
\pgfpathlineto{\pgfqpoint{0.813027in}{0.835370in}}%
\pgfpathlineto{\pgfqpoint{0.826798in}{0.822686in}}%
\pgfpathlineto{\pgfqpoint{0.827865in}{0.821759in}}%
\pgfpathlineto{\pgfqpoint{0.842455in}{0.809786in}}%
\pgfpathlineto{\pgfqpoint{0.844586in}{0.808148in}}%
\pgfpathlineto{\pgfqpoint{0.858112in}{0.798265in}}%
\pgfpathlineto{\pgfqpoint{0.863600in}{0.794536in}}%
\pgfpathlineto{\pgfqpoint{0.873768in}{0.787933in}}%
\pgfpathlineto{\pgfqpoint{0.885481in}{0.780925in}}%
\pgfpathlineto{\pgfqpoint{0.889425in}{0.778656in}}%
\pgfpathlineto{\pgfqpoint{0.905081in}{0.770419in}}%
\pgfpathlineto{\pgfqpoint{0.911617in}{0.767314in}}%
\pgfpathlineto{\pgfqpoint{0.920738in}{0.763121in}}%
\pgfpathlineto{\pgfqpoint{0.936394in}{0.756707in}}%
\pgfpathlineto{\pgfqpoint{0.944760in}{0.753703in}}%
\pgfpathlineto{\pgfqpoint{0.952051in}{0.751154in}}%
\pgfpathlineto{\pgfqpoint{0.967708in}{0.746452in}}%
\pgfpathlineto{\pgfqpoint{0.983364in}{0.742535in}}%
\pgfpathlineto{\pgfqpoint{0.995568in}{0.740092in}}%
\pgfpathlineto{\pgfqpoint{0.999021in}{0.739415in}}%
\pgfpathclose%
\pgfpathmoveto{\pgfqpoint{1.003028in}{0.835370in}}%
\pgfpathlineto{\pgfqpoint{0.999021in}{0.836216in}}%
\pgfpathlineto{\pgfqpoint{0.983364in}{0.840617in}}%
\pgfpathlineto{\pgfqpoint{0.967708in}{0.846121in}}%
\pgfpathlineto{\pgfqpoint{0.960915in}{0.848981in}}%
\pgfpathlineto{\pgfqpoint{0.952051in}{0.852970in}}%
\pgfpathlineto{\pgfqpoint{0.936394in}{0.861179in}}%
\pgfpathlineto{\pgfqpoint{0.934026in}{0.862592in}}%
\pgfpathlineto{\pgfqpoint{0.920738in}{0.871149in}}%
\pgfpathlineto{\pgfqpoint{0.913735in}{0.876203in}}%
\pgfpathlineto{\pgfqpoint{0.905081in}{0.883015in}}%
\pgfpathlineto{\pgfqpoint{0.897261in}{0.889814in}}%
\pgfpathlineto{\pgfqpoint{0.889425in}{0.897337in}}%
\pgfpathlineto{\pgfqpoint{0.883612in}{0.903425in}}%
\pgfpathlineto{\pgfqpoint{0.873768in}{0.914977in}}%
\pgfpathlineto{\pgfqpoint{0.872143in}{0.917036in}}%
\pgfpathlineto{\pgfqpoint{0.862700in}{0.930648in}}%
\pgfpathlineto{\pgfqpoint{0.858112in}{0.938354in}}%
\pgfpathlineto{\pgfqpoint{0.854822in}{0.944259in}}%
\pgfpathlineto{\pgfqpoint{0.848491in}{0.957870in}}%
\pgfpathlineto{\pgfqpoint{0.843428in}{0.971481in}}%
\pgfpathlineto{\pgfqpoint{0.842455in}{0.974965in}}%
\pgfpathlineto{\pgfqpoint{0.839783in}{0.985092in}}%
\pgfpathlineto{\pgfqpoint{0.837388in}{0.998703in}}%
\pgfpathlineto{\pgfqpoint{0.836190in}{1.012314in}}%
\pgfpathlineto{\pgfqpoint{0.836190in}{1.025925in}}%
\pgfpathlineto{\pgfqpoint{0.837388in}{1.039536in}}%
\pgfpathlineto{\pgfqpoint{0.839783in}{1.053148in}}%
\pgfpathlineto{\pgfqpoint{0.842455in}{1.063275in}}%
\pgfpathlineto{\pgfqpoint{0.843428in}{1.066759in}}%
\pgfpathlineto{\pgfqpoint{0.848491in}{1.080370in}}%
\pgfpathlineto{\pgfqpoint{0.854822in}{1.093981in}}%
\pgfpathlineto{\pgfqpoint{0.858112in}{1.099886in}}%
\pgfpathlineto{\pgfqpoint{0.862700in}{1.107592in}}%
\pgfpathlineto{\pgfqpoint{0.872143in}{1.121203in}}%
\pgfpathlineto{\pgfqpoint{0.873768in}{1.123262in}}%
\pgfpathlineto{\pgfqpoint{0.883612in}{1.134814in}}%
\pgfpathlineto{\pgfqpoint{0.889425in}{1.140902in}}%
\pgfpathlineto{\pgfqpoint{0.897261in}{1.148425in}}%
\pgfpathlineto{\pgfqpoint{0.905081in}{1.155224in}}%
\pgfpathlineto{\pgfqpoint{0.913735in}{1.162036in}}%
\pgfpathlineto{\pgfqpoint{0.920738in}{1.167090in}}%
\pgfpathlineto{\pgfqpoint{0.934026in}{1.175647in}}%
\pgfpathlineto{\pgfqpoint{0.936394in}{1.177061in}}%
\pgfpathlineto{\pgfqpoint{0.952051in}{1.185269in}}%
\pgfpathlineto{\pgfqpoint{0.960915in}{1.189259in}}%
\pgfpathlineto{\pgfqpoint{0.967708in}{1.192118in}}%
\pgfpathlineto{\pgfqpoint{0.983364in}{1.197622in}}%
\pgfpathlineto{\pgfqpoint{0.999021in}{1.202024in}}%
\pgfpathlineto{\pgfqpoint{1.003028in}{1.202870in}}%
\pgfpathlineto{\pgfqpoint{1.014677in}{1.205193in}}%
\pgfpathlineto{\pgfqpoint{1.030334in}{1.207275in}}%
\pgfpathlineto{\pgfqpoint{1.045990in}{1.208316in}}%
\pgfpathlineto{\pgfqpoint{1.061647in}{1.208316in}}%
\pgfpathlineto{\pgfqpoint{1.077303in}{1.207275in}}%
\pgfpathlineto{\pgfqpoint{1.092960in}{1.205193in}}%
\pgfpathlineto{\pgfqpoint{1.104609in}{1.202870in}}%
\pgfpathlineto{\pgfqpoint{1.108617in}{1.202024in}}%
\pgfpathlineto{\pgfqpoint{1.124273in}{1.197622in}}%
\pgfpathlineto{\pgfqpoint{1.139930in}{1.192118in}}%
\pgfpathlineto{\pgfqpoint{1.146722in}{1.189259in}}%
\pgfpathlineto{\pgfqpoint{1.155586in}{1.185269in}}%
\pgfpathlineto{\pgfqpoint{1.171243in}{1.177061in}}%
\pgfpathlineto{\pgfqpoint{1.173612in}{1.175647in}}%
\pgfpathlineto{\pgfqpoint{1.186899in}{1.167090in}}%
\pgfpathlineto{\pgfqpoint{1.193903in}{1.162036in}}%
\pgfpathlineto{\pgfqpoint{1.202556in}{1.155224in}}%
\pgfpathlineto{\pgfqpoint{1.210377in}{1.148425in}}%
\pgfpathlineto{\pgfqpoint{1.218213in}{1.140902in}}%
\pgfpathlineto{\pgfqpoint{1.224026in}{1.134814in}}%
\pgfpathlineto{\pgfqpoint{1.233869in}{1.123262in}}%
\pgfpathlineto{\pgfqpoint{1.235495in}{1.121203in}}%
\pgfpathlineto{\pgfqpoint{1.244937in}{1.107592in}}%
\pgfpathlineto{\pgfqpoint{1.249526in}{1.099886in}}%
\pgfpathlineto{\pgfqpoint{1.252815in}{1.093981in}}%
\pgfpathlineto{\pgfqpoint{1.259146in}{1.080370in}}%
\pgfpathlineto{\pgfqpoint{1.264209in}{1.066759in}}%
\pgfpathlineto{\pgfqpoint{1.265182in}{1.063275in}}%
\pgfpathlineto{\pgfqpoint{1.267854in}{1.053148in}}%
\pgfpathlineto{\pgfqpoint{1.270249in}{1.039536in}}%
\pgfpathlineto{\pgfqpoint{1.271447in}{1.025925in}}%
\pgfpathlineto{\pgfqpoint{1.271447in}{1.012314in}}%
\pgfpathlineto{\pgfqpoint{1.270249in}{0.998703in}}%
\pgfpathlineto{\pgfqpoint{1.267854in}{0.985092in}}%
\pgfpathlineto{\pgfqpoint{1.265182in}{0.974965in}}%
\pgfpathlineto{\pgfqpoint{1.264209in}{0.971481in}}%
\pgfpathlineto{\pgfqpoint{1.259146in}{0.957870in}}%
\pgfpathlineto{\pgfqpoint{1.252815in}{0.944259in}}%
\pgfpathlineto{\pgfqpoint{1.249526in}{0.938354in}}%
\pgfpathlineto{\pgfqpoint{1.244937in}{0.930648in}}%
\pgfpathlineto{\pgfqpoint{1.235495in}{0.917036in}}%
\pgfpathlineto{\pgfqpoint{1.233869in}{0.914977in}}%
\pgfpathlineto{\pgfqpoint{1.224026in}{0.903425in}}%
\pgfpathlineto{\pgfqpoint{1.218213in}{0.897337in}}%
\pgfpathlineto{\pgfqpoint{1.210377in}{0.889814in}}%
\pgfpathlineto{\pgfqpoint{1.202556in}{0.883015in}}%
\pgfpathlineto{\pgfqpoint{1.193903in}{0.876203in}}%
\pgfpathlineto{\pgfqpoint{1.186899in}{0.871149in}}%
\pgfpathlineto{\pgfqpoint{1.173612in}{0.862592in}}%
\pgfpathlineto{\pgfqpoint{1.171243in}{0.861179in}}%
\pgfpathlineto{\pgfqpoint{1.155586in}{0.852970in}}%
\pgfpathlineto{\pgfqpoint{1.146722in}{0.848981in}}%
\pgfpathlineto{\pgfqpoint{1.139930in}{0.846121in}}%
\pgfpathlineto{\pgfqpoint{1.124273in}{0.840617in}}%
\pgfpathlineto{\pgfqpoint{1.108617in}{0.836216in}}%
\pgfpathlineto{\pgfqpoint{1.104609in}{0.835370in}}%
\pgfpathlineto{\pgfqpoint{1.092960in}{0.833047in}}%
\pgfpathlineto{\pgfqpoint{1.077303in}{0.830965in}}%
\pgfpathlineto{\pgfqpoint{1.061647in}{0.829924in}}%
\pgfpathlineto{\pgfqpoint{1.045990in}{0.829924in}}%
\pgfpathlineto{\pgfqpoint{1.030334in}{0.830965in}}%
\pgfpathlineto{\pgfqpoint{1.014677in}{0.833047in}}%
\pgfpathlineto{\pgfqpoint{1.003028in}{0.835370in}}%
\pgfpathclose%
\pgfusepath{fill}%
\end{pgfscope}%
\begin{pgfscope}%
\pgfpathrectangle{\pgfqpoint{0.278819in}{0.345370in}}{\pgfqpoint{1.550000in}{1.347500in}}%
\pgfusepath{clip}%
\pgfsetbuttcap%
\pgfsetroundjoin%
\definecolor{currentfill}{rgb}{0.959424,0.543431,0.278701}%
\pgfsetfillcolor{currentfill}%
\pgfsetlinewidth{0.000000pt}%
\definecolor{currentstroke}{rgb}{0.000000,0.000000,0.000000}%
\pgfsetstrokecolor{currentstroke}%
\pgfsetdash{}{0pt}%
\pgfpathmoveto{\pgfqpoint{0.983364in}{0.656656in}}%
\pgfpathlineto{\pgfqpoint{0.999021in}{0.653880in}}%
\pgfpathlineto{\pgfqpoint{1.014677in}{0.651798in}}%
\pgfpathlineto{\pgfqpoint{1.030334in}{0.650411in}}%
\pgfpathlineto{\pgfqpoint{1.045990in}{0.649717in}}%
\pgfpathlineto{\pgfqpoint{1.061647in}{0.649717in}}%
\pgfpathlineto{\pgfqpoint{1.077303in}{0.650411in}}%
\pgfpathlineto{\pgfqpoint{1.092960in}{0.651798in}}%
\pgfpathlineto{\pgfqpoint{1.108617in}{0.653880in}}%
\pgfpathlineto{\pgfqpoint{1.124273in}{0.656656in}}%
\pgfpathlineto{\pgfqpoint{1.132264in}{0.658425in}}%
\pgfpathlineto{\pgfqpoint{1.139930in}{0.660116in}}%
\pgfpathlineto{\pgfqpoint{1.155586in}{0.664254in}}%
\pgfpathlineto{\pgfqpoint{1.171243in}{0.669083in}}%
\pgfpathlineto{\pgfqpoint{1.179638in}{0.672036in}}%
\pgfpathlineto{\pgfqpoint{1.186899in}{0.674597in}}%
\pgfpathlineto{\pgfqpoint{1.202556in}{0.680792in}}%
\pgfpathlineto{\pgfqpoint{1.213621in}{0.685648in}}%
\pgfpathlineto{\pgfqpoint{1.218213in}{0.687678in}}%
\pgfpathlineto{\pgfqpoint{1.233869in}{0.695258in}}%
\pgfpathlineto{\pgfqpoint{1.241476in}{0.699259in}}%
\pgfpathlineto{\pgfqpoint{1.249526in}{0.703548in}}%
\pgfpathlineto{\pgfqpoint{1.265182in}{0.712546in}}%
\pgfpathlineto{\pgfqpoint{1.265711in}{0.712870in}}%
\pgfpathlineto{\pgfqpoint{1.280839in}{0.722313in}}%
\pgfpathlineto{\pgfqpoint{1.287118in}{0.726481in}}%
\pgfpathlineto{\pgfqpoint{1.296495in}{0.732851in}}%
\pgfpathlineto{\pgfqpoint{1.306582in}{0.740092in}}%
\pgfpathlineto{\pgfqpoint{1.312152in}{0.744203in}}%
\pgfpathlineto{\pgfqpoint{1.324405in}{0.753703in}}%
\pgfpathlineto{\pgfqpoint{1.327809in}{0.756431in}}%
\pgfpathlineto{\pgfqpoint{1.340814in}{0.767314in}}%
\pgfpathlineto{\pgfqpoint{1.343465in}{0.769619in}}%
\pgfpathlineto{\pgfqpoint{1.355984in}{0.780925in}}%
\pgfpathlineto{\pgfqpoint{1.359122in}{0.783885in}}%
\pgfpathlineto{\pgfqpoint{1.370049in}{0.794536in}}%
\pgfpathlineto{\pgfqpoint{1.374778in}{0.799378in}}%
\pgfpathlineto{\pgfqpoint{1.383107in}{0.808148in}}%
\pgfpathlineto{\pgfqpoint{1.390435in}{0.816300in}}%
\pgfpathlineto{\pgfqpoint{1.395229in}{0.821759in}}%
\pgfpathlineto{\pgfqpoint{1.406091in}{0.834910in}}%
\pgfpathlineto{\pgfqpoint{1.406464in}{0.835370in}}%
\pgfpathlineto{\pgfqpoint{1.416814in}{0.848981in}}%
\pgfpathlineto{\pgfqpoint{1.421748in}{0.855979in}}%
\pgfpathlineto{\pgfqpoint{1.426349in}{0.862592in}}%
\pgfpathlineto{\pgfqpoint{1.435069in}{0.876203in}}%
\pgfpathlineto{\pgfqpoint{1.437404in}{0.880194in}}%
\pgfpathlineto{\pgfqpoint{1.442990in}{0.889814in}}%
\pgfpathlineto{\pgfqpoint{1.450116in}{0.903425in}}%
\pgfpathlineto{\pgfqpoint{1.453061in}{0.909738in}}%
\pgfpathlineto{\pgfqpoint{1.456458in}{0.917036in}}%
\pgfpathlineto{\pgfqpoint{1.462014in}{0.930648in}}%
\pgfpathlineto{\pgfqpoint{1.466773in}{0.944259in}}%
\pgfpathlineto{\pgfqpoint{1.468718in}{0.950923in}}%
\pgfpathlineto{\pgfqpoint{1.470752in}{0.957870in}}%
\pgfpathlineto{\pgfqpoint{1.473946in}{0.971481in}}%
\pgfpathlineto{\pgfqpoint{1.476341in}{0.985092in}}%
\pgfpathlineto{\pgfqpoint{1.477937in}{0.998703in}}%
\pgfpathlineto{\pgfqpoint{1.478734in}{1.012314in}}%
\pgfpathlineto{\pgfqpoint{1.478734in}{1.025925in}}%
\pgfpathlineto{\pgfqpoint{1.477937in}{1.039536in}}%
\pgfpathlineto{\pgfqpoint{1.476341in}{1.053148in}}%
\pgfpathlineto{\pgfqpoint{1.473946in}{1.066759in}}%
\pgfpathlineto{\pgfqpoint{1.470752in}{1.080370in}}%
\pgfpathlineto{\pgfqpoint{1.468718in}{1.087317in}}%
\pgfpathlineto{\pgfqpoint{1.466773in}{1.093981in}}%
\pgfpathlineto{\pgfqpoint{1.462014in}{1.107592in}}%
\pgfpathlineto{\pgfqpoint{1.456458in}{1.121203in}}%
\pgfpathlineto{\pgfqpoint{1.453061in}{1.128501in}}%
\pgfpathlineto{\pgfqpoint{1.450116in}{1.134814in}}%
\pgfpathlineto{\pgfqpoint{1.442990in}{1.148425in}}%
\pgfpathlineto{\pgfqpoint{1.437404in}{1.158045in}}%
\pgfpathlineto{\pgfqpoint{1.435069in}{1.162036in}}%
\pgfpathlineto{\pgfqpoint{1.426349in}{1.175647in}}%
\pgfpathlineto{\pgfqpoint{1.421748in}{1.182261in}}%
\pgfpathlineto{\pgfqpoint{1.416814in}{1.189259in}}%
\pgfpathlineto{\pgfqpoint{1.406464in}{1.202870in}}%
\pgfpathlineto{\pgfqpoint{1.406091in}{1.203329in}}%
\pgfpathlineto{\pgfqpoint{1.395229in}{1.216481in}}%
\pgfpathlineto{\pgfqpoint{1.390435in}{1.221940in}}%
\pgfpathlineto{\pgfqpoint{1.383107in}{1.230092in}}%
\pgfpathlineto{\pgfqpoint{1.374778in}{1.238861in}}%
\pgfpathlineto{\pgfqpoint{1.370049in}{1.243703in}}%
\pgfpathlineto{\pgfqpoint{1.359122in}{1.254355in}}%
\pgfpathlineto{\pgfqpoint{1.355984in}{1.257314in}}%
\pgfpathlineto{\pgfqpoint{1.343465in}{1.268620in}}%
\pgfpathlineto{\pgfqpoint{1.340814in}{1.270925in}}%
\pgfpathlineto{\pgfqpoint{1.327809in}{1.281809in}}%
\pgfpathlineto{\pgfqpoint{1.324405in}{1.284536in}}%
\pgfpathlineto{\pgfqpoint{1.312152in}{1.294036in}}%
\pgfpathlineto{\pgfqpoint{1.306582in}{1.298148in}}%
\pgfpathlineto{\pgfqpoint{1.296495in}{1.305388in}}%
\pgfpathlineto{\pgfqpoint{1.287118in}{1.311759in}}%
\pgfpathlineto{\pgfqpoint{1.280839in}{1.315927in}}%
\pgfpathlineto{\pgfqpoint{1.265711in}{1.325370in}}%
\pgfpathlineto{\pgfqpoint{1.265182in}{1.325694in}}%
\pgfpathlineto{\pgfqpoint{1.249526in}{1.334692in}}%
\pgfpathlineto{\pgfqpoint{1.241476in}{1.338981in}}%
\pgfpathlineto{\pgfqpoint{1.233869in}{1.342981in}}%
\pgfpathlineto{\pgfqpoint{1.218213in}{1.350561in}}%
\pgfpathlineto{\pgfqpoint{1.213621in}{1.352592in}}%
\pgfpathlineto{\pgfqpoint{1.202556in}{1.357448in}}%
\pgfpathlineto{\pgfqpoint{1.186899in}{1.363642in}}%
\pgfpathlineto{\pgfqpoint{1.179638in}{1.366203in}}%
\pgfpathlineto{\pgfqpoint{1.171243in}{1.369156in}}%
\pgfpathlineto{\pgfqpoint{1.155586in}{1.373986in}}%
\pgfpathlineto{\pgfqpoint{1.139930in}{1.378124in}}%
\pgfpathlineto{\pgfqpoint{1.132264in}{1.379814in}}%
\pgfpathlineto{\pgfqpoint{1.124273in}{1.381583in}}%
\pgfpathlineto{\pgfqpoint{1.108617in}{1.384360in}}%
\pgfpathlineto{\pgfqpoint{1.092960in}{1.386441in}}%
\pgfpathlineto{\pgfqpoint{1.077303in}{1.387829in}}%
\pgfpathlineto{\pgfqpoint{1.061647in}{1.388522in}}%
\pgfpathlineto{\pgfqpoint{1.045990in}{1.388522in}}%
\pgfpathlineto{\pgfqpoint{1.030334in}{1.387829in}}%
\pgfpathlineto{\pgfqpoint{1.014677in}{1.386441in}}%
\pgfpathlineto{\pgfqpoint{0.999021in}{1.384360in}}%
\pgfpathlineto{\pgfqpoint{0.983364in}{1.381583in}}%
\pgfpathlineto{\pgfqpoint{0.975373in}{1.379814in}}%
\pgfpathlineto{\pgfqpoint{0.967708in}{1.378124in}}%
\pgfpathlineto{\pgfqpoint{0.952051in}{1.373986in}}%
\pgfpathlineto{\pgfqpoint{0.936394in}{1.369156in}}%
\pgfpathlineto{\pgfqpoint{0.927999in}{1.366203in}}%
\pgfpathlineto{\pgfqpoint{0.920738in}{1.363642in}}%
\pgfpathlineto{\pgfqpoint{0.905081in}{1.357448in}}%
\pgfpathlineto{\pgfqpoint{0.894016in}{1.352592in}}%
\pgfpathlineto{\pgfqpoint{0.889425in}{1.350561in}}%
\pgfpathlineto{\pgfqpoint{0.873768in}{1.342981in}}%
\pgfpathlineto{\pgfqpoint{0.866161in}{1.338981in}}%
\pgfpathlineto{\pgfqpoint{0.858112in}{1.334692in}}%
\pgfpathlineto{\pgfqpoint{0.842455in}{1.325694in}}%
\pgfpathlineto{\pgfqpoint{0.841926in}{1.325370in}}%
\pgfpathlineto{\pgfqpoint{0.826798in}{1.315927in}}%
\pgfpathlineto{\pgfqpoint{0.820519in}{1.311759in}}%
\pgfpathlineto{\pgfqpoint{0.811142in}{1.305388in}}%
\pgfpathlineto{\pgfqpoint{0.801055in}{1.298148in}}%
\pgfpathlineto{\pgfqpoint{0.795485in}{1.294036in}}%
\pgfpathlineto{\pgfqpoint{0.783233in}{1.284536in}}%
\pgfpathlineto{\pgfqpoint{0.779829in}{1.281809in}}%
\pgfpathlineto{\pgfqpoint{0.766824in}{1.270925in}}%
\pgfpathlineto{\pgfqpoint{0.764172in}{1.268620in}}%
\pgfpathlineto{\pgfqpoint{0.751653in}{1.257314in}}%
\pgfpathlineto{\pgfqpoint{0.748516in}{1.254355in}}%
\pgfpathlineto{\pgfqpoint{0.737588in}{1.243703in}}%
\pgfpathlineto{\pgfqpoint{0.732859in}{1.238861in}}%
\pgfpathlineto{\pgfqpoint{0.724530in}{1.230092in}}%
\pgfpathlineto{\pgfqpoint{0.717202in}{1.221940in}}%
\pgfpathlineto{\pgfqpoint{0.712408in}{1.216481in}}%
\pgfpathlineto{\pgfqpoint{0.701546in}{1.203329in}}%
\pgfpathlineto{\pgfqpoint{0.701173in}{1.202870in}}%
\pgfpathlineto{\pgfqpoint{0.690823in}{1.189259in}}%
\pgfpathlineto{\pgfqpoint{0.685889in}{1.182261in}}%
\pgfpathlineto{\pgfqpoint{0.681288in}{1.175647in}}%
\pgfpathlineto{\pgfqpoint{0.672568in}{1.162036in}}%
\pgfpathlineto{\pgfqpoint{0.670233in}{1.158045in}}%
\pgfpathlineto{\pgfqpoint{0.664647in}{1.148425in}}%
\pgfpathlineto{\pgfqpoint{0.657522in}{1.134814in}}%
\pgfpathlineto{\pgfqpoint{0.654576in}{1.128501in}}%
\pgfpathlineto{\pgfqpoint{0.651179in}{1.121203in}}%
\pgfpathlineto{\pgfqpoint{0.645624in}{1.107592in}}%
\pgfpathlineto{\pgfqpoint{0.640864in}{1.093981in}}%
\pgfpathlineto{\pgfqpoint{0.638920in}{1.087317in}}%
\pgfpathlineto{\pgfqpoint{0.636885in}{1.080370in}}%
\pgfpathlineto{\pgfqpoint{0.633691in}{1.066759in}}%
\pgfpathlineto{\pgfqpoint{0.631296in}{1.053148in}}%
\pgfpathlineto{\pgfqpoint{0.629701in}{1.039536in}}%
\pgfpathlineto{\pgfqpoint{0.628903in}{1.025925in}}%
\pgfpathlineto{\pgfqpoint{0.628903in}{1.012314in}}%
\pgfpathlineto{\pgfqpoint{0.629701in}{0.998703in}}%
\pgfpathlineto{\pgfqpoint{0.631296in}{0.985092in}}%
\pgfpathlineto{\pgfqpoint{0.633691in}{0.971481in}}%
\pgfpathlineto{\pgfqpoint{0.636885in}{0.957870in}}%
\pgfpathlineto{\pgfqpoint{0.638920in}{0.950923in}}%
\pgfpathlineto{\pgfqpoint{0.640864in}{0.944259in}}%
\pgfpathlineto{\pgfqpoint{0.645624in}{0.930648in}}%
\pgfpathlineto{\pgfqpoint{0.651179in}{0.917036in}}%
\pgfpathlineto{\pgfqpoint{0.654576in}{0.909738in}}%
\pgfpathlineto{\pgfqpoint{0.657522in}{0.903425in}}%
\pgfpathlineto{\pgfqpoint{0.664647in}{0.889814in}}%
\pgfpathlineto{\pgfqpoint{0.670233in}{0.880194in}}%
\pgfpathlineto{\pgfqpoint{0.672568in}{0.876203in}}%
\pgfpathlineto{\pgfqpoint{0.681288in}{0.862592in}}%
\pgfpathlineto{\pgfqpoint{0.685889in}{0.855979in}}%
\pgfpathlineto{\pgfqpoint{0.690823in}{0.848981in}}%
\pgfpathlineto{\pgfqpoint{0.701173in}{0.835370in}}%
\pgfpathlineto{\pgfqpoint{0.701546in}{0.834910in}}%
\pgfpathlineto{\pgfqpoint{0.712408in}{0.821759in}}%
\pgfpathlineto{\pgfqpoint{0.717202in}{0.816300in}}%
\pgfpathlineto{\pgfqpoint{0.724530in}{0.808148in}}%
\pgfpathlineto{\pgfqpoint{0.732859in}{0.799378in}}%
\pgfpathlineto{\pgfqpoint{0.737588in}{0.794536in}}%
\pgfpathlineto{\pgfqpoint{0.748516in}{0.783885in}}%
\pgfpathlineto{\pgfqpoint{0.751653in}{0.780925in}}%
\pgfpathlineto{\pgfqpoint{0.764172in}{0.769619in}}%
\pgfpathlineto{\pgfqpoint{0.766824in}{0.767314in}}%
\pgfpathlineto{\pgfqpoint{0.779829in}{0.756431in}}%
\pgfpathlineto{\pgfqpoint{0.783233in}{0.753703in}}%
\pgfpathlineto{\pgfqpoint{0.795485in}{0.744203in}}%
\pgfpathlineto{\pgfqpoint{0.801055in}{0.740092in}}%
\pgfpathlineto{\pgfqpoint{0.811142in}{0.732851in}}%
\pgfpathlineto{\pgfqpoint{0.820519in}{0.726481in}}%
\pgfpathlineto{\pgfqpoint{0.826798in}{0.722313in}}%
\pgfpathlineto{\pgfqpoint{0.841926in}{0.712870in}}%
\pgfpathlineto{\pgfqpoint{0.842455in}{0.712546in}}%
\pgfpathlineto{\pgfqpoint{0.858112in}{0.703548in}}%
\pgfpathlineto{\pgfqpoint{0.866161in}{0.699259in}}%
\pgfpathlineto{\pgfqpoint{0.873768in}{0.695258in}}%
\pgfpathlineto{\pgfqpoint{0.889425in}{0.687678in}}%
\pgfpathlineto{\pgfqpoint{0.894016in}{0.685648in}}%
\pgfpathlineto{\pgfqpoint{0.905081in}{0.680792in}}%
\pgfpathlineto{\pgfqpoint{0.920738in}{0.674597in}}%
\pgfpathlineto{\pgfqpoint{0.927999in}{0.672036in}}%
\pgfpathlineto{\pgfqpoint{0.936394in}{0.669083in}}%
\pgfpathlineto{\pgfqpoint{0.952051in}{0.664254in}}%
\pgfpathlineto{\pgfqpoint{0.967708in}{0.660116in}}%
\pgfpathlineto{\pgfqpoint{0.975373in}{0.658425in}}%
\pgfpathlineto{\pgfqpoint{0.983364in}{0.656656in}}%
\pgfpathclose%
\pgfpathmoveto{\pgfqpoint{0.995568in}{0.740092in}}%
\pgfpathlineto{\pgfqpoint{0.983364in}{0.742535in}}%
\pgfpathlineto{\pgfqpoint{0.967708in}{0.746452in}}%
\pgfpathlineto{\pgfqpoint{0.952051in}{0.751154in}}%
\pgfpathlineto{\pgfqpoint{0.944760in}{0.753703in}}%
\pgfpathlineto{\pgfqpoint{0.936394in}{0.756707in}}%
\pgfpathlineto{\pgfqpoint{0.920738in}{0.763121in}}%
\pgfpathlineto{\pgfqpoint{0.911617in}{0.767314in}}%
\pgfpathlineto{\pgfqpoint{0.905081in}{0.770419in}}%
\pgfpathlineto{\pgfqpoint{0.889425in}{0.778656in}}%
\pgfpathlineto{\pgfqpoint{0.885481in}{0.780925in}}%
\pgfpathlineto{\pgfqpoint{0.873768in}{0.787933in}}%
\pgfpathlineto{\pgfqpoint{0.863600in}{0.794536in}}%
\pgfpathlineto{\pgfqpoint{0.858112in}{0.798265in}}%
\pgfpathlineto{\pgfqpoint{0.844586in}{0.808148in}}%
\pgfpathlineto{\pgfqpoint{0.842455in}{0.809786in}}%
\pgfpathlineto{\pgfqpoint{0.827865in}{0.821759in}}%
\pgfpathlineto{\pgfqpoint{0.826798in}{0.822686in}}%
\pgfpathlineto{\pgfqpoint{0.813027in}{0.835370in}}%
\pgfpathlineto{\pgfqpoint{0.811142in}{0.837223in}}%
\pgfpathlineto{\pgfqpoint{0.799774in}{0.848981in}}%
\pgfpathlineto{\pgfqpoint{0.795485in}{0.853753in}}%
\pgfpathlineto{\pgfqpoint{0.787889in}{0.862592in}}%
\pgfpathlineto{\pgfqpoint{0.779829in}{0.872774in}}%
\pgfpathlineto{\pgfqpoint{0.777218in}{0.876203in}}%
\pgfpathlineto{\pgfqpoint{0.767744in}{0.889814in}}%
\pgfpathlineto{\pgfqpoint{0.764172in}{0.895496in}}%
\pgfpathlineto{\pgfqpoint{0.759348in}{0.903425in}}%
\pgfpathlineto{\pgfqpoint{0.751971in}{0.917036in}}%
\pgfpathlineto{\pgfqpoint{0.748516in}{0.924309in}}%
\pgfpathlineto{\pgfqpoint{0.745583in}{0.930648in}}%
\pgfpathlineto{\pgfqpoint{0.740174in}{0.944259in}}%
\pgfpathlineto{\pgfqpoint{0.735669in}{0.957870in}}%
\pgfpathlineto{\pgfqpoint{0.732859in}{0.968479in}}%
\pgfpathlineto{\pgfqpoint{0.732080in}{0.971481in}}%
\pgfpathlineto{\pgfqpoint{0.729425in}{0.985092in}}%
\pgfpathlineto{\pgfqpoint{0.727655in}{0.998703in}}%
\pgfpathlineto{\pgfqpoint{0.726771in}{1.012314in}}%
\pgfpathlineto{\pgfqpoint{0.726771in}{1.025925in}}%
\pgfpathlineto{\pgfqpoint{0.727655in}{1.039536in}}%
\pgfpathlineto{\pgfqpoint{0.729425in}{1.053148in}}%
\pgfpathlineto{\pgfqpoint{0.732080in}{1.066759in}}%
\pgfpathlineto{\pgfqpoint{0.732859in}{1.069760in}}%
\pgfpathlineto{\pgfqpoint{0.735669in}{1.080370in}}%
\pgfpathlineto{\pgfqpoint{0.740174in}{1.093981in}}%
\pgfpathlineto{\pgfqpoint{0.745583in}{1.107592in}}%
\pgfpathlineto{\pgfqpoint{0.748516in}{1.113930in}}%
\pgfpathlineto{\pgfqpoint{0.751971in}{1.121203in}}%
\pgfpathlineto{\pgfqpoint{0.759348in}{1.134814in}}%
\pgfpathlineto{\pgfqpoint{0.764172in}{1.142743in}}%
\pgfpathlineto{\pgfqpoint{0.767744in}{1.148425in}}%
\pgfpathlineto{\pgfqpoint{0.777218in}{1.162036in}}%
\pgfpathlineto{\pgfqpoint{0.779829in}{1.165465in}}%
\pgfpathlineto{\pgfqpoint{0.787889in}{1.175647in}}%
\pgfpathlineto{\pgfqpoint{0.795485in}{1.184487in}}%
\pgfpathlineto{\pgfqpoint{0.799774in}{1.189259in}}%
\pgfpathlineto{\pgfqpoint{0.811142in}{1.201017in}}%
\pgfpathlineto{\pgfqpoint{0.813027in}{1.202870in}}%
\pgfpathlineto{\pgfqpoint{0.826798in}{1.215553in}}%
\pgfpathlineto{\pgfqpoint{0.827865in}{1.216481in}}%
\pgfpathlineto{\pgfqpoint{0.842455in}{1.228453in}}%
\pgfpathlineto{\pgfqpoint{0.844586in}{1.230092in}}%
\pgfpathlineto{\pgfqpoint{0.858112in}{1.239975in}}%
\pgfpathlineto{\pgfqpoint{0.863600in}{1.243703in}}%
\pgfpathlineto{\pgfqpoint{0.873768in}{1.250307in}}%
\pgfpathlineto{\pgfqpoint{0.885481in}{1.257314in}}%
\pgfpathlineto{\pgfqpoint{0.889425in}{1.259584in}}%
\pgfpathlineto{\pgfqpoint{0.905081in}{1.267820in}}%
\pgfpathlineto{\pgfqpoint{0.911617in}{1.270925in}}%
\pgfpathlineto{\pgfqpoint{0.920738in}{1.275119in}}%
\pgfpathlineto{\pgfqpoint{0.936394in}{1.281532in}}%
\pgfpathlineto{\pgfqpoint{0.944760in}{1.284536in}}%
\pgfpathlineto{\pgfqpoint{0.952051in}{1.287085in}}%
\pgfpathlineto{\pgfqpoint{0.967708in}{1.291788in}}%
\pgfpathlineto{\pgfqpoint{0.983364in}{1.295705in}}%
\pgfpathlineto{\pgfqpoint{0.995568in}{1.298148in}}%
\pgfpathlineto{\pgfqpoint{0.999021in}{1.298825in}}%
\pgfpathlineto{\pgfqpoint{1.014677in}{1.301133in}}%
\pgfpathlineto{\pgfqpoint{1.030334in}{1.302671in}}%
\pgfpathlineto{\pgfqpoint{1.045990in}{1.303440in}}%
\pgfpathlineto{\pgfqpoint{1.061647in}{1.303440in}}%
\pgfpathlineto{\pgfqpoint{1.077303in}{1.302671in}}%
\pgfpathlineto{\pgfqpoint{1.092960in}{1.301133in}}%
\pgfpathlineto{\pgfqpoint{1.108617in}{1.298825in}}%
\pgfpathlineto{\pgfqpoint{1.112069in}{1.298148in}}%
\pgfpathlineto{\pgfqpoint{1.124273in}{1.295705in}}%
\pgfpathlineto{\pgfqpoint{1.139930in}{1.291788in}}%
\pgfpathlineto{\pgfqpoint{1.155586in}{1.287085in}}%
\pgfpathlineto{\pgfqpoint{1.162877in}{1.284536in}}%
\pgfpathlineto{\pgfqpoint{1.171243in}{1.281532in}}%
\pgfpathlineto{\pgfqpoint{1.186899in}{1.275119in}}%
\pgfpathlineto{\pgfqpoint{1.196020in}{1.270925in}}%
\pgfpathlineto{\pgfqpoint{1.202556in}{1.267820in}}%
\pgfpathlineto{\pgfqpoint{1.218213in}{1.259584in}}%
\pgfpathlineto{\pgfqpoint{1.222157in}{1.257314in}}%
\pgfpathlineto{\pgfqpoint{1.233869in}{1.250307in}}%
\pgfpathlineto{\pgfqpoint{1.244037in}{1.243703in}}%
\pgfpathlineto{\pgfqpoint{1.249526in}{1.239975in}}%
\pgfpathlineto{\pgfqpoint{1.263051in}{1.230092in}}%
\pgfpathlineto{\pgfqpoint{1.265182in}{1.228453in}}%
\pgfpathlineto{\pgfqpoint{1.279772in}{1.216481in}}%
\pgfpathlineto{\pgfqpoint{1.280839in}{1.215553in}}%
\pgfpathlineto{\pgfqpoint{1.294610in}{1.202870in}}%
\pgfpathlineto{\pgfqpoint{1.296495in}{1.201017in}}%
\pgfpathlineto{\pgfqpoint{1.307864in}{1.189259in}}%
\pgfpathlineto{\pgfqpoint{1.312152in}{1.184487in}}%
\pgfpathlineto{\pgfqpoint{1.319748in}{1.175647in}}%
\pgfpathlineto{\pgfqpoint{1.327809in}{1.165465in}}%
\pgfpathlineto{\pgfqpoint{1.330419in}{1.162036in}}%
\pgfpathlineto{\pgfqpoint{1.339893in}{1.148425in}}%
\pgfpathlineto{\pgfqpoint{1.343465in}{1.142743in}}%
\pgfpathlineto{\pgfqpoint{1.348289in}{1.134814in}}%
\pgfpathlineto{\pgfqpoint{1.355666in}{1.121203in}}%
\pgfpathlineto{\pgfqpoint{1.359122in}{1.113930in}}%
\pgfpathlineto{\pgfqpoint{1.362054in}{1.107592in}}%
\pgfpathlineto{\pgfqpoint{1.367463in}{1.093981in}}%
\pgfpathlineto{\pgfqpoint{1.371968in}{1.080370in}}%
\pgfpathlineto{\pgfqpoint{1.374778in}{1.069760in}}%
\pgfpathlineto{\pgfqpoint{1.375557in}{1.066759in}}%
\pgfpathlineto{\pgfqpoint{1.378212in}{1.053148in}}%
\pgfpathlineto{\pgfqpoint{1.379982in}{1.039536in}}%
\pgfpathlineto{\pgfqpoint{1.380866in}{1.025925in}}%
\pgfpathlineto{\pgfqpoint{1.380866in}{1.012314in}}%
\pgfpathlineto{\pgfqpoint{1.379982in}{0.998703in}}%
\pgfpathlineto{\pgfqpoint{1.378212in}{0.985092in}}%
\pgfpathlineto{\pgfqpoint{1.375557in}{0.971481in}}%
\pgfpathlineto{\pgfqpoint{1.374778in}{0.968479in}}%
\pgfpathlineto{\pgfqpoint{1.371968in}{0.957870in}}%
\pgfpathlineto{\pgfqpoint{1.367463in}{0.944259in}}%
\pgfpathlineto{\pgfqpoint{1.362054in}{0.930648in}}%
\pgfpathlineto{\pgfqpoint{1.359122in}{0.924309in}}%
\pgfpathlineto{\pgfqpoint{1.355666in}{0.917036in}}%
\pgfpathlineto{\pgfqpoint{1.348289in}{0.903425in}}%
\pgfpathlineto{\pgfqpoint{1.343465in}{0.895496in}}%
\pgfpathlineto{\pgfqpoint{1.339893in}{0.889814in}}%
\pgfpathlineto{\pgfqpoint{1.330419in}{0.876203in}}%
\pgfpathlineto{\pgfqpoint{1.327809in}{0.872774in}}%
\pgfpathlineto{\pgfqpoint{1.319748in}{0.862592in}}%
\pgfpathlineto{\pgfqpoint{1.312152in}{0.853753in}}%
\pgfpathlineto{\pgfqpoint{1.307864in}{0.848981in}}%
\pgfpathlineto{\pgfqpoint{1.296495in}{0.837223in}}%
\pgfpathlineto{\pgfqpoint{1.294610in}{0.835370in}}%
\pgfpathlineto{\pgfqpoint{1.280839in}{0.822686in}}%
\pgfpathlineto{\pgfqpoint{1.279772in}{0.821759in}}%
\pgfpathlineto{\pgfqpoint{1.265182in}{0.809786in}}%
\pgfpathlineto{\pgfqpoint{1.263051in}{0.808148in}}%
\pgfpathlineto{\pgfqpoint{1.249526in}{0.798265in}}%
\pgfpathlineto{\pgfqpoint{1.244037in}{0.794536in}}%
\pgfpathlineto{\pgfqpoint{1.233869in}{0.787933in}}%
\pgfpathlineto{\pgfqpoint{1.222157in}{0.780925in}}%
\pgfpathlineto{\pgfqpoint{1.218213in}{0.778656in}}%
\pgfpathlineto{\pgfqpoint{1.202556in}{0.770419in}}%
\pgfpathlineto{\pgfqpoint{1.196020in}{0.767314in}}%
\pgfpathlineto{\pgfqpoint{1.186899in}{0.763121in}}%
\pgfpathlineto{\pgfqpoint{1.171243in}{0.756707in}}%
\pgfpathlineto{\pgfqpoint{1.162877in}{0.753703in}}%
\pgfpathlineto{\pgfqpoint{1.155586in}{0.751154in}}%
\pgfpathlineto{\pgfqpoint{1.139930in}{0.746452in}}%
\pgfpathlineto{\pgfqpoint{1.124273in}{0.742535in}}%
\pgfpathlineto{\pgfqpoint{1.112069in}{0.740092in}}%
\pgfpathlineto{\pgfqpoint{1.108617in}{0.739415in}}%
\pgfpathlineto{\pgfqpoint{1.092960in}{0.737106in}}%
\pgfpathlineto{\pgfqpoint{1.077303in}{0.735568in}}%
\pgfpathlineto{\pgfqpoint{1.061647in}{0.734799in}}%
\pgfpathlineto{\pgfqpoint{1.045990in}{0.734799in}}%
\pgfpathlineto{\pgfqpoint{1.030334in}{0.735568in}}%
\pgfpathlineto{\pgfqpoint{1.014677in}{0.737106in}}%
\pgfpathlineto{\pgfqpoint{0.999021in}{0.739415in}}%
\pgfpathlineto{\pgfqpoint{0.995568in}{0.740092in}}%
\pgfpathclose%
\pgfusepath{fill}%
\end{pgfscope}%
\begin{pgfscope}%
\pgfpathrectangle{\pgfqpoint{0.278819in}{0.345370in}}{\pgfqpoint{1.550000in}{1.347500in}}%
\pgfusepath{clip}%
\pgfsetbuttcap%
\pgfsetroundjoin%
\definecolor{currentfill}{rgb}{0.890340,0.406398,0.373130}%
\pgfsetfillcolor{currentfill}%
\pgfsetlinewidth{0.000000pt}%
\definecolor{currentstroke}{rgb}{0.000000,0.000000,0.000000}%
\pgfsetstrokecolor{currentstroke}%
\pgfsetdash{}{0pt}%
\pgfpathmoveto{\pgfqpoint{0.999021in}{0.562974in}}%
\pgfpathlineto{\pgfqpoint{1.014677in}{0.560719in}}%
\pgfpathlineto{\pgfqpoint{1.030334in}{0.559216in}}%
\pgfpathlineto{\pgfqpoint{1.045990in}{0.558465in}}%
\pgfpathlineto{\pgfqpoint{1.061647in}{0.558465in}}%
\pgfpathlineto{\pgfqpoint{1.077303in}{0.559216in}}%
\pgfpathlineto{\pgfqpoint{1.092960in}{0.560719in}}%
\pgfpathlineto{\pgfqpoint{1.108617in}{0.562974in}}%
\pgfpathlineto{\pgfqpoint{1.109522in}{0.563148in}}%
\pgfpathlineto{\pgfqpoint{1.124273in}{0.565871in}}%
\pgfpathlineto{\pgfqpoint{1.139930in}{0.569486in}}%
\pgfpathlineto{\pgfqpoint{1.155586in}{0.573826in}}%
\pgfpathlineto{\pgfqpoint{1.164668in}{0.576759in}}%
\pgfpathlineto{\pgfqpoint{1.171243in}{0.578820in}}%
\pgfpathlineto{\pgfqpoint{1.186899in}{0.584417in}}%
\pgfpathlineto{\pgfqpoint{1.201700in}{0.590370in}}%
\pgfpathlineto{\pgfqpoint{1.202556in}{0.590706in}}%
\pgfpathlineto{\pgfqpoint{1.218213in}{0.597506in}}%
\pgfpathlineto{\pgfqpoint{1.231779in}{0.603981in}}%
\pgfpathlineto{\pgfqpoint{1.233869in}{0.604962in}}%
\pgfpathlineto{\pgfqpoint{1.249526in}{0.612924in}}%
\pgfpathlineto{\pgfqpoint{1.258035in}{0.617592in}}%
\pgfpathlineto{\pgfqpoint{1.265182in}{0.621470in}}%
\pgfpathlineto{\pgfqpoint{1.280839in}{0.630567in}}%
\pgfpathlineto{\pgfqpoint{1.281872in}{0.631203in}}%
\pgfpathlineto{\pgfqpoint{1.296495in}{0.640161in}}%
\pgfpathlineto{\pgfqpoint{1.303674in}{0.644814in}}%
\pgfpathlineto{\pgfqpoint{1.312152in}{0.650305in}}%
\pgfpathlineto{\pgfqpoint{1.324067in}{0.658425in}}%
\pgfpathlineto{\pgfqpoint{1.327809in}{0.660985in}}%
\pgfpathlineto{\pgfqpoint{1.343245in}{0.672036in}}%
\pgfpathlineto{\pgfqpoint{1.343465in}{0.672196in}}%
\pgfpathlineto{\pgfqpoint{1.359122in}{0.683947in}}%
\pgfpathlineto{\pgfqpoint{1.361308in}{0.685648in}}%
\pgfpathlineto{\pgfqpoint{1.374778in}{0.696265in}}%
\pgfpathlineto{\pgfqpoint{1.378459in}{0.699259in}}%
\pgfpathlineto{\pgfqpoint{1.390435in}{0.709173in}}%
\pgfpathlineto{\pgfqpoint{1.394782in}{0.712870in}}%
\pgfpathlineto{\pgfqpoint{1.406091in}{0.722701in}}%
\pgfpathlineto{\pgfqpoint{1.410343in}{0.726481in}}%
\pgfpathlineto{\pgfqpoint{1.421748in}{0.736892in}}%
\pgfpathlineto{\pgfqpoint{1.425191in}{0.740092in}}%
\pgfpathlineto{\pgfqpoint{1.437404in}{0.751802in}}%
\pgfpathlineto{\pgfqpoint{1.439361in}{0.753703in}}%
\pgfpathlineto{\pgfqpoint{1.452878in}{0.767314in}}%
\pgfpathlineto{\pgfqpoint{1.453061in}{0.767506in}}%
\pgfpathlineto{\pgfqpoint{1.465774in}{0.780925in}}%
\pgfpathlineto{\pgfqpoint{1.468718in}{0.784178in}}%
\pgfpathlineto{\pgfqpoint{1.478058in}{0.794536in}}%
\pgfpathlineto{\pgfqpoint{1.484374in}{0.801907in}}%
\pgfpathlineto{\pgfqpoint{1.489726in}{0.808148in}}%
\pgfpathlineto{\pgfqpoint{1.500031in}{0.820860in}}%
\pgfpathlineto{\pgfqpoint{1.500763in}{0.821759in}}%
\pgfpathlineto{\pgfqpoint{1.511227in}{0.835370in}}%
\pgfpathlineto{\pgfqpoint{1.515687in}{0.841583in}}%
\pgfpathlineto{\pgfqpoint{1.521056in}{0.848981in}}%
\pgfpathlineto{\pgfqpoint{1.530215in}{0.862592in}}%
\pgfpathlineto{\pgfqpoint{1.531344in}{0.864409in}}%
\pgfpathlineto{\pgfqpoint{1.538792in}{0.876203in}}%
\pgfpathlineto{\pgfqpoint{1.546613in}{0.889814in}}%
\pgfpathlineto{\pgfqpoint{1.547000in}{0.890558in}}%
\pgfpathlineto{\pgfqpoint{1.553848in}{0.903425in}}%
\pgfpathlineto{\pgfqpoint{1.560286in}{0.917036in}}%
\pgfpathlineto{\pgfqpoint{1.562657in}{0.922752in}}%
\pgfpathlineto{\pgfqpoint{1.566030in}{0.930648in}}%
\pgfpathlineto{\pgfqpoint{1.571022in}{0.944259in}}%
\pgfpathlineto{\pgfqpoint{1.575180in}{0.957870in}}%
\pgfpathlineto{\pgfqpoint{1.578314in}{0.970693in}}%
\pgfpathlineto{\pgfqpoint{1.578513in}{0.971481in}}%
\pgfpathlineto{\pgfqpoint{1.581107in}{0.985092in}}%
\pgfpathlineto{\pgfqpoint{1.582836in}{0.998703in}}%
\pgfpathlineto{\pgfqpoint{1.583700in}{1.012314in}}%
\pgfpathlineto{\pgfqpoint{1.583700in}{1.025925in}}%
\pgfpathlineto{\pgfqpoint{1.582836in}{1.039536in}}%
\pgfpathlineto{\pgfqpoint{1.581107in}{1.053148in}}%
\pgfpathlineto{\pgfqpoint{1.578513in}{1.066759in}}%
\pgfpathlineto{\pgfqpoint{1.578314in}{1.067546in}}%
\pgfpathlineto{\pgfqpoint{1.575180in}{1.080370in}}%
\pgfpathlineto{\pgfqpoint{1.571022in}{1.093981in}}%
\pgfpathlineto{\pgfqpoint{1.566030in}{1.107592in}}%
\pgfpathlineto{\pgfqpoint{1.562657in}{1.115488in}}%
\pgfpathlineto{\pgfqpoint{1.560286in}{1.121203in}}%
\pgfpathlineto{\pgfqpoint{1.553848in}{1.134814in}}%
\pgfpathlineto{\pgfqpoint{1.547000in}{1.147681in}}%
\pgfpathlineto{\pgfqpoint{1.546613in}{1.148425in}}%
\pgfpathlineto{\pgfqpoint{1.538792in}{1.162036in}}%
\pgfpathlineto{\pgfqpoint{1.531344in}{1.173830in}}%
\pgfpathlineto{\pgfqpoint{1.530215in}{1.175647in}}%
\pgfpathlineto{\pgfqpoint{1.521056in}{1.189259in}}%
\pgfpathlineto{\pgfqpoint{1.515687in}{1.196656in}}%
\pgfpathlineto{\pgfqpoint{1.511227in}{1.202870in}}%
\pgfpathlineto{\pgfqpoint{1.500763in}{1.216481in}}%
\pgfpathlineto{\pgfqpoint{1.500031in}{1.217379in}}%
\pgfpathlineto{\pgfqpoint{1.489726in}{1.230092in}}%
\pgfpathlineto{\pgfqpoint{1.484374in}{1.236333in}}%
\pgfpathlineto{\pgfqpoint{1.478058in}{1.243703in}}%
\pgfpathlineto{\pgfqpoint{1.468718in}{1.254062in}}%
\pgfpathlineto{\pgfqpoint{1.465774in}{1.257314in}}%
\pgfpathlineto{\pgfqpoint{1.453061in}{1.270734in}}%
\pgfpathlineto{\pgfqpoint{1.452878in}{1.270925in}}%
\pgfpathlineto{\pgfqpoint{1.439361in}{1.284536in}}%
\pgfpathlineto{\pgfqpoint{1.437404in}{1.286437in}}%
\pgfpathlineto{\pgfqpoint{1.425191in}{1.298148in}}%
\pgfpathlineto{\pgfqpoint{1.421748in}{1.301347in}}%
\pgfpathlineto{\pgfqpoint{1.410343in}{1.311759in}}%
\pgfpathlineto{\pgfqpoint{1.406091in}{1.315538in}}%
\pgfpathlineto{\pgfqpoint{1.394782in}{1.325370in}}%
\pgfpathlineto{\pgfqpoint{1.390435in}{1.329066in}}%
\pgfpathlineto{\pgfqpoint{1.378459in}{1.338981in}}%
\pgfpathlineto{\pgfqpoint{1.374778in}{1.341974in}}%
\pgfpathlineto{\pgfqpoint{1.361308in}{1.352592in}}%
\pgfpathlineto{\pgfqpoint{1.359122in}{1.354293in}}%
\pgfpathlineto{\pgfqpoint{1.343465in}{1.366044in}}%
\pgfpathlineto{\pgfqpoint{1.343245in}{1.366203in}}%
\pgfpathlineto{\pgfqpoint{1.327809in}{1.377255in}}%
\pgfpathlineto{\pgfqpoint{1.324067in}{1.379814in}}%
\pgfpathlineto{\pgfqpoint{1.312152in}{1.387934in}}%
\pgfpathlineto{\pgfqpoint{1.303674in}{1.393425in}}%
\pgfpathlineto{\pgfqpoint{1.296495in}{1.398078in}}%
\pgfpathlineto{\pgfqpoint{1.281872in}{1.407036in}}%
\pgfpathlineto{\pgfqpoint{1.280839in}{1.407673in}}%
\pgfpathlineto{\pgfqpoint{1.265182in}{1.416770in}}%
\pgfpathlineto{\pgfqpoint{1.258035in}{1.420648in}}%
\pgfpathlineto{\pgfqpoint{1.249526in}{1.425315in}}%
\pgfpathlineto{\pgfqpoint{1.233869in}{1.433278in}}%
\pgfpathlineto{\pgfqpoint{1.231779in}{1.434259in}}%
\pgfpathlineto{\pgfqpoint{1.218213in}{1.440734in}}%
\pgfpathlineto{\pgfqpoint{1.202556in}{1.447533in}}%
\pgfpathlineto{\pgfqpoint{1.201700in}{1.447870in}}%
\pgfpathlineto{\pgfqpoint{1.186899in}{1.453822in}}%
\pgfpathlineto{\pgfqpoint{1.171243in}{1.459419in}}%
\pgfpathlineto{\pgfqpoint{1.164668in}{1.461481in}}%
\pgfpathlineto{\pgfqpoint{1.155586in}{1.464413in}}%
\pgfpathlineto{\pgfqpoint{1.139930in}{1.468753in}}%
\pgfpathlineto{\pgfqpoint{1.124273in}{1.472368in}}%
\pgfpathlineto{\pgfqpoint{1.109522in}{1.475092in}}%
\pgfpathlineto{\pgfqpoint{1.108617in}{1.475265in}}%
\pgfpathlineto{\pgfqpoint{1.092960in}{1.477520in}}%
\pgfpathlineto{\pgfqpoint{1.077303in}{1.479023in}}%
\pgfpathlineto{\pgfqpoint{1.061647in}{1.479775in}}%
\pgfpathlineto{\pgfqpoint{1.045990in}{1.479775in}}%
\pgfpathlineto{\pgfqpoint{1.030334in}{1.479023in}}%
\pgfpathlineto{\pgfqpoint{1.014677in}{1.477520in}}%
\pgfpathlineto{\pgfqpoint{0.999021in}{1.475265in}}%
\pgfpathlineto{\pgfqpoint{0.998115in}{1.475092in}}%
\pgfpathlineto{\pgfqpoint{0.983364in}{1.472368in}}%
\pgfpathlineto{\pgfqpoint{0.967708in}{1.468753in}}%
\pgfpathlineto{\pgfqpoint{0.952051in}{1.464413in}}%
\pgfpathlineto{\pgfqpoint{0.942969in}{1.461481in}}%
\pgfpathlineto{\pgfqpoint{0.936394in}{1.459419in}}%
\pgfpathlineto{\pgfqpoint{0.920738in}{1.453822in}}%
\pgfpathlineto{\pgfqpoint{0.905937in}{1.447870in}}%
\pgfpathlineto{\pgfqpoint{0.905081in}{1.447533in}}%
\pgfpathlineto{\pgfqpoint{0.889425in}{1.440734in}}%
\pgfpathlineto{\pgfqpoint{0.875858in}{1.434259in}}%
\pgfpathlineto{\pgfqpoint{0.873768in}{1.433278in}}%
\pgfpathlineto{\pgfqpoint{0.858112in}{1.425315in}}%
\pgfpathlineto{\pgfqpoint{0.849602in}{1.420648in}}%
\pgfpathlineto{\pgfqpoint{0.842455in}{1.416770in}}%
\pgfpathlineto{\pgfqpoint{0.826798in}{1.407673in}}%
\pgfpathlineto{\pgfqpoint{0.825765in}{1.407036in}}%
\pgfpathlineto{\pgfqpoint{0.811142in}{1.398078in}}%
\pgfpathlineto{\pgfqpoint{0.803963in}{1.393425in}}%
\pgfpathlineto{\pgfqpoint{0.795485in}{1.387934in}}%
\pgfpathlineto{\pgfqpoint{0.783570in}{1.379814in}}%
\pgfpathlineto{\pgfqpoint{0.779829in}{1.377255in}}%
\pgfpathlineto{\pgfqpoint{0.764393in}{1.366203in}}%
\pgfpathlineto{\pgfqpoint{0.764172in}{1.366044in}}%
\pgfpathlineto{\pgfqpoint{0.748516in}{1.354293in}}%
\pgfpathlineto{\pgfqpoint{0.746329in}{1.352592in}}%
\pgfpathlineto{\pgfqpoint{0.732859in}{1.341974in}}%
\pgfpathlineto{\pgfqpoint{0.729179in}{1.338981in}}%
\pgfpathlineto{\pgfqpoint{0.717202in}{1.329066in}}%
\pgfpathlineto{\pgfqpoint{0.712855in}{1.325370in}}%
\pgfpathlineto{\pgfqpoint{0.701546in}{1.315538in}}%
\pgfpathlineto{\pgfqpoint{0.697294in}{1.311759in}}%
\pgfpathlineto{\pgfqpoint{0.685889in}{1.301347in}}%
\pgfpathlineto{\pgfqpoint{0.682446in}{1.298148in}}%
\pgfpathlineto{\pgfqpoint{0.670233in}{1.286437in}}%
\pgfpathlineto{\pgfqpoint{0.668276in}{1.284536in}}%
\pgfpathlineto{\pgfqpoint{0.654759in}{1.270925in}}%
\pgfpathlineto{\pgfqpoint{0.654576in}{1.270734in}}%
\pgfpathlineto{\pgfqpoint{0.641864in}{1.257314in}}%
\pgfpathlineto{\pgfqpoint{0.638920in}{1.254062in}}%
\pgfpathlineto{\pgfqpoint{0.629579in}{1.243703in}}%
\pgfpathlineto{\pgfqpoint{0.623263in}{1.236333in}}%
\pgfpathlineto{\pgfqpoint{0.617911in}{1.230092in}}%
\pgfpathlineto{\pgfqpoint{0.607606in}{1.217379in}}%
\pgfpathlineto{\pgfqpoint{0.606874in}{1.216481in}}%
\pgfpathlineto{\pgfqpoint{0.596410in}{1.202870in}}%
\pgfpathlineto{\pgfqpoint{0.591950in}{1.196656in}}%
\pgfpathlineto{\pgfqpoint{0.586581in}{1.189259in}}%
\pgfpathlineto{\pgfqpoint{0.577422in}{1.175647in}}%
\pgfpathlineto{\pgfqpoint{0.576293in}{1.173830in}}%
\pgfpathlineto{\pgfqpoint{0.568845in}{1.162036in}}%
\pgfpathlineto{\pgfqpoint{0.561024in}{1.148425in}}%
\pgfpathlineto{\pgfqpoint{0.560637in}{1.147681in}}%
\pgfpathlineto{\pgfqpoint{0.553789in}{1.134814in}}%
\pgfpathlineto{\pgfqpoint{0.547351in}{1.121203in}}%
\pgfpathlineto{\pgfqpoint{0.544980in}{1.115488in}}%
\pgfpathlineto{\pgfqpoint{0.541607in}{1.107592in}}%
\pgfpathlineto{\pgfqpoint{0.536615in}{1.093981in}}%
\pgfpathlineto{\pgfqpoint{0.532457in}{1.080370in}}%
\pgfpathlineto{\pgfqpoint{0.529324in}{1.067546in}}%
\pgfpathlineto{\pgfqpoint{0.529124in}{1.066759in}}%
\pgfpathlineto{\pgfqpoint{0.526530in}{1.053148in}}%
\pgfpathlineto{\pgfqpoint{0.524801in}{1.039536in}}%
\pgfpathlineto{\pgfqpoint{0.523937in}{1.025925in}}%
\pgfpathlineto{\pgfqpoint{0.523937in}{1.012314in}}%
\pgfpathlineto{\pgfqpoint{0.524801in}{0.998703in}}%
\pgfpathlineto{\pgfqpoint{0.526530in}{0.985092in}}%
\pgfpathlineto{\pgfqpoint{0.529124in}{0.971481in}}%
\pgfpathlineto{\pgfqpoint{0.529324in}{0.970693in}}%
\pgfpathlineto{\pgfqpoint{0.532457in}{0.957870in}}%
\pgfpathlineto{\pgfqpoint{0.536615in}{0.944259in}}%
\pgfpathlineto{\pgfqpoint{0.541607in}{0.930648in}}%
\pgfpathlineto{\pgfqpoint{0.544980in}{0.922752in}}%
\pgfpathlineto{\pgfqpoint{0.547351in}{0.917036in}}%
\pgfpathlineto{\pgfqpoint{0.553789in}{0.903425in}}%
\pgfpathlineto{\pgfqpoint{0.560637in}{0.890558in}}%
\pgfpathlineto{\pgfqpoint{0.561024in}{0.889814in}}%
\pgfpathlineto{\pgfqpoint{0.568845in}{0.876203in}}%
\pgfpathlineto{\pgfqpoint{0.576293in}{0.864409in}}%
\pgfpathlineto{\pgfqpoint{0.577422in}{0.862592in}}%
\pgfpathlineto{\pgfqpoint{0.586581in}{0.848981in}}%
\pgfpathlineto{\pgfqpoint{0.591950in}{0.841583in}}%
\pgfpathlineto{\pgfqpoint{0.596410in}{0.835370in}}%
\pgfpathlineto{\pgfqpoint{0.606874in}{0.821759in}}%
\pgfpathlineto{\pgfqpoint{0.607606in}{0.820860in}}%
\pgfpathlineto{\pgfqpoint{0.617911in}{0.808148in}}%
\pgfpathlineto{\pgfqpoint{0.623263in}{0.801907in}}%
\pgfpathlineto{\pgfqpoint{0.629579in}{0.794536in}}%
\pgfpathlineto{\pgfqpoint{0.638920in}{0.784178in}}%
\pgfpathlineto{\pgfqpoint{0.641864in}{0.780925in}}%
\pgfpathlineto{\pgfqpoint{0.654576in}{0.767506in}}%
\pgfpathlineto{\pgfqpoint{0.654759in}{0.767314in}}%
\pgfpathlineto{\pgfqpoint{0.668276in}{0.753703in}}%
\pgfpathlineto{\pgfqpoint{0.670233in}{0.751802in}}%
\pgfpathlineto{\pgfqpoint{0.682446in}{0.740092in}}%
\pgfpathlineto{\pgfqpoint{0.685889in}{0.736892in}}%
\pgfpathlineto{\pgfqpoint{0.697294in}{0.726481in}}%
\pgfpathlineto{\pgfqpoint{0.701546in}{0.722701in}}%
\pgfpathlineto{\pgfqpoint{0.712855in}{0.712870in}}%
\pgfpathlineto{\pgfqpoint{0.717202in}{0.709173in}}%
\pgfpathlineto{\pgfqpoint{0.729179in}{0.699259in}}%
\pgfpathlineto{\pgfqpoint{0.732859in}{0.696265in}}%
\pgfpathlineto{\pgfqpoint{0.746329in}{0.685648in}}%
\pgfpathlineto{\pgfqpoint{0.748516in}{0.683947in}}%
\pgfpathlineto{\pgfqpoint{0.764172in}{0.672196in}}%
\pgfpathlineto{\pgfqpoint{0.764393in}{0.672036in}}%
\pgfpathlineto{\pgfqpoint{0.779829in}{0.660985in}}%
\pgfpathlineto{\pgfqpoint{0.783570in}{0.658425in}}%
\pgfpathlineto{\pgfqpoint{0.795485in}{0.650305in}}%
\pgfpathlineto{\pgfqpoint{0.803963in}{0.644814in}}%
\pgfpathlineto{\pgfqpoint{0.811142in}{0.640161in}}%
\pgfpathlineto{\pgfqpoint{0.825765in}{0.631203in}}%
\pgfpathlineto{\pgfqpoint{0.826798in}{0.630567in}}%
\pgfpathlineto{\pgfqpoint{0.842455in}{0.621470in}}%
\pgfpathlineto{\pgfqpoint{0.849602in}{0.617592in}}%
\pgfpathlineto{\pgfqpoint{0.858112in}{0.612924in}}%
\pgfpathlineto{\pgfqpoint{0.873768in}{0.604962in}}%
\pgfpathlineto{\pgfqpoint{0.875858in}{0.603981in}}%
\pgfpathlineto{\pgfqpoint{0.889425in}{0.597506in}}%
\pgfpathlineto{\pgfqpoint{0.905081in}{0.590706in}}%
\pgfpathlineto{\pgfqpoint{0.905937in}{0.590370in}}%
\pgfpathlineto{\pgfqpoint{0.920738in}{0.584417in}}%
\pgfpathlineto{\pgfqpoint{0.936394in}{0.578820in}}%
\pgfpathlineto{\pgfqpoint{0.942969in}{0.576759in}}%
\pgfpathlineto{\pgfqpoint{0.952051in}{0.573826in}}%
\pgfpathlineto{\pgfqpoint{0.967708in}{0.569486in}}%
\pgfpathlineto{\pgfqpoint{0.983364in}{0.565871in}}%
\pgfpathlineto{\pgfqpoint{0.998115in}{0.563148in}}%
\pgfpathlineto{\pgfqpoint{0.999021in}{0.562974in}}%
\pgfpathclose%
\pgfpathmoveto{\pgfqpoint{0.975373in}{0.658425in}}%
\pgfpathlineto{\pgfqpoint{0.967708in}{0.660116in}}%
\pgfpathlineto{\pgfqpoint{0.952051in}{0.664254in}}%
\pgfpathlineto{\pgfqpoint{0.936394in}{0.669083in}}%
\pgfpathlineto{\pgfqpoint{0.927999in}{0.672036in}}%
\pgfpathlineto{\pgfqpoint{0.920738in}{0.674597in}}%
\pgfpathlineto{\pgfqpoint{0.905081in}{0.680792in}}%
\pgfpathlineto{\pgfqpoint{0.894016in}{0.685648in}}%
\pgfpathlineto{\pgfqpoint{0.889425in}{0.687678in}}%
\pgfpathlineto{\pgfqpoint{0.873768in}{0.695258in}}%
\pgfpathlineto{\pgfqpoint{0.866161in}{0.699259in}}%
\pgfpathlineto{\pgfqpoint{0.858112in}{0.703548in}}%
\pgfpathlineto{\pgfqpoint{0.842455in}{0.712546in}}%
\pgfpathlineto{\pgfqpoint{0.841926in}{0.712870in}}%
\pgfpathlineto{\pgfqpoint{0.826798in}{0.722313in}}%
\pgfpathlineto{\pgfqpoint{0.820519in}{0.726481in}}%
\pgfpathlineto{\pgfqpoint{0.811142in}{0.732851in}}%
\pgfpathlineto{\pgfqpoint{0.801055in}{0.740092in}}%
\pgfpathlineto{\pgfqpoint{0.795485in}{0.744203in}}%
\pgfpathlineto{\pgfqpoint{0.783233in}{0.753703in}}%
\pgfpathlineto{\pgfqpoint{0.779829in}{0.756431in}}%
\pgfpathlineto{\pgfqpoint{0.766824in}{0.767314in}}%
\pgfpathlineto{\pgfqpoint{0.764172in}{0.769619in}}%
\pgfpathlineto{\pgfqpoint{0.751653in}{0.780925in}}%
\pgfpathlineto{\pgfqpoint{0.748516in}{0.783885in}}%
\pgfpathlineto{\pgfqpoint{0.737588in}{0.794536in}}%
\pgfpathlineto{\pgfqpoint{0.732859in}{0.799378in}}%
\pgfpathlineto{\pgfqpoint{0.724530in}{0.808148in}}%
\pgfpathlineto{\pgfqpoint{0.717202in}{0.816300in}}%
\pgfpathlineto{\pgfqpoint{0.712408in}{0.821759in}}%
\pgfpathlineto{\pgfqpoint{0.701546in}{0.834910in}}%
\pgfpathlineto{\pgfqpoint{0.701173in}{0.835370in}}%
\pgfpathlineto{\pgfqpoint{0.690823in}{0.848981in}}%
\pgfpathlineto{\pgfqpoint{0.685889in}{0.855979in}}%
\pgfpathlineto{\pgfqpoint{0.681288in}{0.862592in}}%
\pgfpathlineto{\pgfqpoint{0.672568in}{0.876203in}}%
\pgfpathlineto{\pgfqpoint{0.670233in}{0.880194in}}%
\pgfpathlineto{\pgfqpoint{0.664647in}{0.889814in}}%
\pgfpathlineto{\pgfqpoint{0.657522in}{0.903425in}}%
\pgfpathlineto{\pgfqpoint{0.654576in}{0.909738in}}%
\pgfpathlineto{\pgfqpoint{0.651179in}{0.917036in}}%
\pgfpathlineto{\pgfqpoint{0.645624in}{0.930648in}}%
\pgfpathlineto{\pgfqpoint{0.640864in}{0.944259in}}%
\pgfpathlineto{\pgfqpoint{0.638920in}{0.950923in}}%
\pgfpathlineto{\pgfqpoint{0.636885in}{0.957870in}}%
\pgfpathlineto{\pgfqpoint{0.633691in}{0.971481in}}%
\pgfpathlineto{\pgfqpoint{0.631296in}{0.985092in}}%
\pgfpathlineto{\pgfqpoint{0.629701in}{0.998703in}}%
\pgfpathlineto{\pgfqpoint{0.628903in}{1.012314in}}%
\pgfpathlineto{\pgfqpoint{0.628903in}{1.025925in}}%
\pgfpathlineto{\pgfqpoint{0.629701in}{1.039536in}}%
\pgfpathlineto{\pgfqpoint{0.631296in}{1.053148in}}%
\pgfpathlineto{\pgfqpoint{0.633691in}{1.066759in}}%
\pgfpathlineto{\pgfqpoint{0.636885in}{1.080370in}}%
\pgfpathlineto{\pgfqpoint{0.638920in}{1.087317in}}%
\pgfpathlineto{\pgfqpoint{0.640864in}{1.093981in}}%
\pgfpathlineto{\pgfqpoint{0.645624in}{1.107592in}}%
\pgfpathlineto{\pgfqpoint{0.651179in}{1.121203in}}%
\pgfpathlineto{\pgfqpoint{0.654576in}{1.128501in}}%
\pgfpathlineto{\pgfqpoint{0.657522in}{1.134814in}}%
\pgfpathlineto{\pgfqpoint{0.664647in}{1.148425in}}%
\pgfpathlineto{\pgfqpoint{0.670233in}{1.158045in}}%
\pgfpathlineto{\pgfqpoint{0.672568in}{1.162036in}}%
\pgfpathlineto{\pgfqpoint{0.681288in}{1.175647in}}%
\pgfpathlineto{\pgfqpoint{0.685889in}{1.182261in}}%
\pgfpathlineto{\pgfqpoint{0.690823in}{1.189259in}}%
\pgfpathlineto{\pgfqpoint{0.701173in}{1.202870in}}%
\pgfpathlineto{\pgfqpoint{0.701546in}{1.203329in}}%
\pgfpathlineto{\pgfqpoint{0.712408in}{1.216481in}}%
\pgfpathlineto{\pgfqpoint{0.717202in}{1.221940in}}%
\pgfpathlineto{\pgfqpoint{0.724530in}{1.230092in}}%
\pgfpathlineto{\pgfqpoint{0.732859in}{1.238861in}}%
\pgfpathlineto{\pgfqpoint{0.737588in}{1.243703in}}%
\pgfpathlineto{\pgfqpoint{0.748516in}{1.254355in}}%
\pgfpathlineto{\pgfqpoint{0.751653in}{1.257314in}}%
\pgfpathlineto{\pgfqpoint{0.764172in}{1.268620in}}%
\pgfpathlineto{\pgfqpoint{0.766824in}{1.270925in}}%
\pgfpathlineto{\pgfqpoint{0.779829in}{1.281809in}}%
\pgfpathlineto{\pgfqpoint{0.783233in}{1.284536in}}%
\pgfpathlineto{\pgfqpoint{0.795485in}{1.294036in}}%
\pgfpathlineto{\pgfqpoint{0.801055in}{1.298148in}}%
\pgfpathlineto{\pgfqpoint{0.811142in}{1.305388in}}%
\pgfpathlineto{\pgfqpoint{0.820519in}{1.311759in}}%
\pgfpathlineto{\pgfqpoint{0.826798in}{1.315927in}}%
\pgfpathlineto{\pgfqpoint{0.841926in}{1.325370in}}%
\pgfpathlineto{\pgfqpoint{0.842455in}{1.325694in}}%
\pgfpathlineto{\pgfqpoint{0.858112in}{1.334692in}}%
\pgfpathlineto{\pgfqpoint{0.866161in}{1.338981in}}%
\pgfpathlineto{\pgfqpoint{0.873768in}{1.342981in}}%
\pgfpathlineto{\pgfqpoint{0.889425in}{1.350561in}}%
\pgfpathlineto{\pgfqpoint{0.894016in}{1.352592in}}%
\pgfpathlineto{\pgfqpoint{0.905081in}{1.357448in}}%
\pgfpathlineto{\pgfqpoint{0.920738in}{1.363642in}}%
\pgfpathlineto{\pgfqpoint{0.927999in}{1.366203in}}%
\pgfpathlineto{\pgfqpoint{0.936394in}{1.369156in}}%
\pgfpathlineto{\pgfqpoint{0.952051in}{1.373986in}}%
\pgfpathlineto{\pgfqpoint{0.967708in}{1.378124in}}%
\pgfpathlineto{\pgfqpoint{0.975373in}{1.379814in}}%
\pgfpathlineto{\pgfqpoint{0.983364in}{1.381583in}}%
\pgfpathlineto{\pgfqpoint{0.999021in}{1.384360in}}%
\pgfpathlineto{\pgfqpoint{1.014677in}{1.386441in}}%
\pgfpathlineto{\pgfqpoint{1.030334in}{1.387829in}}%
\pgfpathlineto{\pgfqpoint{1.045990in}{1.388522in}}%
\pgfpathlineto{\pgfqpoint{1.061647in}{1.388522in}}%
\pgfpathlineto{\pgfqpoint{1.077303in}{1.387829in}}%
\pgfpathlineto{\pgfqpoint{1.092960in}{1.386441in}}%
\pgfpathlineto{\pgfqpoint{1.108617in}{1.384360in}}%
\pgfpathlineto{\pgfqpoint{1.124273in}{1.381583in}}%
\pgfpathlineto{\pgfqpoint{1.132264in}{1.379814in}}%
\pgfpathlineto{\pgfqpoint{1.139930in}{1.378124in}}%
\pgfpathlineto{\pgfqpoint{1.155586in}{1.373986in}}%
\pgfpathlineto{\pgfqpoint{1.171243in}{1.369156in}}%
\pgfpathlineto{\pgfqpoint{1.179638in}{1.366203in}}%
\pgfpathlineto{\pgfqpoint{1.186899in}{1.363642in}}%
\pgfpathlineto{\pgfqpoint{1.202556in}{1.357448in}}%
\pgfpathlineto{\pgfqpoint{1.213621in}{1.352592in}}%
\pgfpathlineto{\pgfqpoint{1.218213in}{1.350561in}}%
\pgfpathlineto{\pgfqpoint{1.233869in}{1.342981in}}%
\pgfpathlineto{\pgfqpoint{1.241476in}{1.338981in}}%
\pgfpathlineto{\pgfqpoint{1.249526in}{1.334692in}}%
\pgfpathlineto{\pgfqpoint{1.265182in}{1.325694in}}%
\pgfpathlineto{\pgfqpoint{1.265711in}{1.325370in}}%
\pgfpathlineto{\pgfqpoint{1.280839in}{1.315927in}}%
\pgfpathlineto{\pgfqpoint{1.287118in}{1.311759in}}%
\pgfpathlineto{\pgfqpoint{1.296495in}{1.305388in}}%
\pgfpathlineto{\pgfqpoint{1.306582in}{1.298148in}}%
\pgfpathlineto{\pgfqpoint{1.312152in}{1.294036in}}%
\pgfpathlineto{\pgfqpoint{1.324405in}{1.284536in}}%
\pgfpathlineto{\pgfqpoint{1.327809in}{1.281809in}}%
\pgfpathlineto{\pgfqpoint{1.340814in}{1.270925in}}%
\pgfpathlineto{\pgfqpoint{1.343465in}{1.268620in}}%
\pgfpathlineto{\pgfqpoint{1.355984in}{1.257314in}}%
\pgfpathlineto{\pgfqpoint{1.359122in}{1.254355in}}%
\pgfpathlineto{\pgfqpoint{1.370049in}{1.243703in}}%
\pgfpathlineto{\pgfqpoint{1.374778in}{1.238861in}}%
\pgfpathlineto{\pgfqpoint{1.383107in}{1.230092in}}%
\pgfpathlineto{\pgfqpoint{1.390435in}{1.221940in}}%
\pgfpathlineto{\pgfqpoint{1.395229in}{1.216481in}}%
\pgfpathlineto{\pgfqpoint{1.406091in}{1.203329in}}%
\pgfpathlineto{\pgfqpoint{1.406464in}{1.202870in}}%
\pgfpathlineto{\pgfqpoint{1.416814in}{1.189259in}}%
\pgfpathlineto{\pgfqpoint{1.421748in}{1.182261in}}%
\pgfpathlineto{\pgfqpoint{1.426349in}{1.175647in}}%
\pgfpathlineto{\pgfqpoint{1.435069in}{1.162036in}}%
\pgfpathlineto{\pgfqpoint{1.437404in}{1.158045in}}%
\pgfpathlineto{\pgfqpoint{1.442990in}{1.148425in}}%
\pgfpathlineto{\pgfqpoint{1.450116in}{1.134814in}}%
\pgfpathlineto{\pgfqpoint{1.453061in}{1.128501in}}%
\pgfpathlineto{\pgfqpoint{1.456458in}{1.121203in}}%
\pgfpathlineto{\pgfqpoint{1.462014in}{1.107592in}}%
\pgfpathlineto{\pgfqpoint{1.466773in}{1.093981in}}%
\pgfpathlineto{\pgfqpoint{1.468718in}{1.087317in}}%
\pgfpathlineto{\pgfqpoint{1.470752in}{1.080370in}}%
\pgfpathlineto{\pgfqpoint{1.473946in}{1.066759in}}%
\pgfpathlineto{\pgfqpoint{1.476341in}{1.053148in}}%
\pgfpathlineto{\pgfqpoint{1.477937in}{1.039536in}}%
\pgfpathlineto{\pgfqpoint{1.478734in}{1.025925in}}%
\pgfpathlineto{\pgfqpoint{1.478734in}{1.012314in}}%
\pgfpathlineto{\pgfqpoint{1.477937in}{0.998703in}}%
\pgfpathlineto{\pgfqpoint{1.476341in}{0.985092in}}%
\pgfpathlineto{\pgfqpoint{1.473946in}{0.971481in}}%
\pgfpathlineto{\pgfqpoint{1.470752in}{0.957870in}}%
\pgfpathlineto{\pgfqpoint{1.468718in}{0.950923in}}%
\pgfpathlineto{\pgfqpoint{1.466773in}{0.944259in}}%
\pgfpathlineto{\pgfqpoint{1.462014in}{0.930647in}}%
\pgfpathlineto{\pgfqpoint{1.456458in}{0.917036in}}%
\pgfpathlineto{\pgfqpoint{1.453061in}{0.909738in}}%
\pgfpathlineto{\pgfqpoint{1.450116in}{0.903425in}}%
\pgfpathlineto{\pgfqpoint{1.442990in}{0.889814in}}%
\pgfpathlineto{\pgfqpoint{1.437404in}{0.880194in}}%
\pgfpathlineto{\pgfqpoint{1.435069in}{0.876203in}}%
\pgfpathlineto{\pgfqpoint{1.426349in}{0.862592in}}%
\pgfpathlineto{\pgfqpoint{1.421748in}{0.855979in}}%
\pgfpathlineto{\pgfqpoint{1.416814in}{0.848981in}}%
\pgfpathlineto{\pgfqpoint{1.406464in}{0.835370in}}%
\pgfpathlineto{\pgfqpoint{1.406091in}{0.834910in}}%
\pgfpathlineto{\pgfqpoint{1.395229in}{0.821759in}}%
\pgfpathlineto{\pgfqpoint{1.390435in}{0.816300in}}%
\pgfpathlineto{\pgfqpoint{1.383107in}{0.808148in}}%
\pgfpathlineto{\pgfqpoint{1.374778in}{0.799378in}}%
\pgfpathlineto{\pgfqpoint{1.370049in}{0.794536in}}%
\pgfpathlineto{\pgfqpoint{1.359122in}{0.783885in}}%
\pgfpathlineto{\pgfqpoint{1.355984in}{0.780925in}}%
\pgfpathlineto{\pgfqpoint{1.343465in}{0.769619in}}%
\pgfpathlineto{\pgfqpoint{1.340814in}{0.767314in}}%
\pgfpathlineto{\pgfqpoint{1.327809in}{0.756431in}}%
\pgfpathlineto{\pgfqpoint{1.324405in}{0.753703in}}%
\pgfpathlineto{\pgfqpoint{1.312152in}{0.744203in}}%
\pgfpathlineto{\pgfqpoint{1.306582in}{0.740092in}}%
\pgfpathlineto{\pgfqpoint{1.296495in}{0.732851in}}%
\pgfpathlineto{\pgfqpoint{1.287118in}{0.726481in}}%
\pgfpathlineto{\pgfqpoint{1.280839in}{0.722313in}}%
\pgfpathlineto{\pgfqpoint{1.265711in}{0.712870in}}%
\pgfpathlineto{\pgfqpoint{1.265182in}{0.712546in}}%
\pgfpathlineto{\pgfqpoint{1.249526in}{0.703548in}}%
\pgfpathlineto{\pgfqpoint{1.241476in}{0.699259in}}%
\pgfpathlineto{\pgfqpoint{1.233869in}{0.695258in}}%
\pgfpathlineto{\pgfqpoint{1.218213in}{0.687678in}}%
\pgfpathlineto{\pgfqpoint{1.213621in}{0.685648in}}%
\pgfpathlineto{\pgfqpoint{1.202556in}{0.680792in}}%
\pgfpathlineto{\pgfqpoint{1.186899in}{0.674597in}}%
\pgfpathlineto{\pgfqpoint{1.179638in}{0.672036in}}%
\pgfpathlineto{\pgfqpoint{1.171243in}{0.669083in}}%
\pgfpathlineto{\pgfqpoint{1.155586in}{0.664254in}}%
\pgfpathlineto{\pgfqpoint{1.139930in}{0.660116in}}%
\pgfpathlineto{\pgfqpoint{1.132264in}{0.658425in}}%
\pgfpathlineto{\pgfqpoint{1.124273in}{0.656656in}}%
\pgfpathlineto{\pgfqpoint{1.108617in}{0.653880in}}%
\pgfpathlineto{\pgfqpoint{1.092960in}{0.651798in}}%
\pgfpathlineto{\pgfqpoint{1.077303in}{0.650411in}}%
\pgfpathlineto{\pgfqpoint{1.061647in}{0.649717in}}%
\pgfpathlineto{\pgfqpoint{1.045990in}{0.649717in}}%
\pgfpathlineto{\pgfqpoint{1.030334in}{0.650411in}}%
\pgfpathlineto{\pgfqpoint{1.014677in}{0.651798in}}%
\pgfpathlineto{\pgfqpoint{0.999021in}{0.653880in}}%
\pgfpathlineto{\pgfqpoint{0.983364in}{0.656656in}}%
\pgfpathlineto{\pgfqpoint{0.975373in}{0.658425in}}%
\pgfpathclose%
\pgfusepath{fill}%
\end{pgfscope}%
\begin{pgfscope}%
\pgfpathrectangle{\pgfqpoint{0.278819in}{0.345370in}}{\pgfqpoint{1.550000in}{1.347500in}}%
\pgfusepath{clip}%
\pgfsetbuttcap%
\pgfsetroundjoin%
\definecolor{currentfill}{rgb}{0.794549,0.275770,0.473117}%
\pgfsetfillcolor{currentfill}%
\pgfsetlinewidth{0.000000pt}%
\definecolor{currentstroke}{rgb}{0.000000,0.000000,0.000000}%
\pgfsetstrokecolor{currentstroke}%
\pgfsetdash{}{0pt}%
\pgfpathmoveto{\pgfqpoint{1.014677in}{0.424285in}}%
\pgfpathlineto{\pgfqpoint{1.030334in}{0.420985in}}%
\pgfpathlineto{\pgfqpoint{1.045990in}{0.419335in}}%
\pgfpathlineto{\pgfqpoint{1.061647in}{0.419335in}}%
\pgfpathlineto{\pgfqpoint{1.077303in}{0.420985in}}%
\pgfpathlineto{\pgfqpoint{1.092960in}{0.424285in}}%
\pgfpathlineto{\pgfqpoint{1.101671in}{0.427036in}}%
\pgfpathlineto{\pgfqpoint{1.108617in}{0.428912in}}%
\pgfpathlineto{\pgfqpoint{1.124273in}{0.434545in}}%
\pgfpathlineto{\pgfqpoint{1.137845in}{0.440648in}}%
\pgfpathlineto{\pgfqpoint{1.139930in}{0.441471in}}%
\pgfpathlineto{\pgfqpoint{1.155586in}{0.448869in}}%
\pgfpathlineto{\pgfqpoint{1.165377in}{0.454259in}}%
\pgfpathlineto{\pgfqpoint{1.171243in}{0.457153in}}%
\pgfpathlineto{\pgfqpoint{1.186899in}{0.465960in}}%
\pgfpathlineto{\pgfqpoint{1.189932in}{0.467870in}}%
\pgfpathlineto{\pgfqpoint{1.202556in}{0.475117in}}%
\pgfpathlineto{\pgfqpoint{1.212568in}{0.481481in}}%
\pgfpathlineto{\pgfqpoint{1.218213in}{0.484795in}}%
\pgfpathlineto{\pgfqpoint{1.233869in}{0.494863in}}%
\pgfpathlineto{\pgfqpoint{1.234198in}{0.495092in}}%
\pgfpathlineto{\pgfqpoint{1.249526in}{0.505061in}}%
\pgfpathlineto{\pgfqpoint{1.254729in}{0.508703in}}%
\pgfpathlineto{\pgfqpoint{1.265182in}{0.515609in}}%
\pgfpathlineto{\pgfqpoint{1.274677in}{0.522314in}}%
\pgfpathlineto{\pgfqpoint{1.280839in}{0.526457in}}%
\pgfpathlineto{\pgfqpoint{1.294096in}{0.535925in}}%
\pgfpathlineto{\pgfqpoint{1.296495in}{0.537569in}}%
\pgfpathlineto{\pgfqpoint{1.312152in}{0.548895in}}%
\pgfpathlineto{\pgfqpoint{1.312997in}{0.549536in}}%
\pgfpathlineto{\pgfqpoint{1.327809in}{0.560397in}}%
\pgfpathlineto{\pgfqpoint{1.331396in}{0.563148in}}%
\pgfpathlineto{\pgfqpoint{1.343465in}{0.572146in}}%
\pgfpathlineto{\pgfqpoint{1.349409in}{0.576759in}}%
\pgfpathlineto{\pgfqpoint{1.359122in}{0.584132in}}%
\pgfpathlineto{\pgfqpoint{1.367051in}{0.590370in}}%
\pgfpathlineto{\pgfqpoint{1.374778in}{0.596349in}}%
\pgfpathlineto{\pgfqpoint{1.384337in}{0.603981in}}%
\pgfpathlineto{\pgfqpoint{1.390435in}{0.608794in}}%
\pgfpathlineto{\pgfqpoint{1.401279in}{0.617592in}}%
\pgfpathlineto{\pgfqpoint{1.406091in}{0.621470in}}%
\pgfpathlineto{\pgfqpoint{1.417891in}{0.631203in}}%
\pgfpathlineto{\pgfqpoint{1.421748in}{0.634378in}}%
\pgfpathlineto{\pgfqpoint{1.434185in}{0.644814in}}%
\pgfpathlineto{\pgfqpoint{1.437404in}{0.647523in}}%
\pgfpathlineto{\pgfqpoint{1.450170in}{0.658425in}}%
\pgfpathlineto{\pgfqpoint{1.453061in}{0.660911in}}%
\pgfpathlineto{\pgfqpoint{1.465858in}{0.672036in}}%
\pgfpathlineto{\pgfqpoint{1.468718in}{0.674550in}}%
\pgfpathlineto{\pgfqpoint{1.481258in}{0.685648in}}%
\pgfpathlineto{\pgfqpoint{1.484374in}{0.688447in}}%
\pgfpathlineto{\pgfqpoint{1.496379in}{0.699259in}}%
\pgfpathlineto{\pgfqpoint{1.500031in}{0.702611in}}%
\pgfpathlineto{\pgfqpoint{1.511227in}{0.712870in}}%
\pgfpathlineto{\pgfqpoint{1.515687in}{0.717053in}}%
\pgfpathlineto{\pgfqpoint{1.525807in}{0.726481in}}%
\pgfpathlineto{\pgfqpoint{1.531344in}{0.731782in}}%
\pgfpathlineto{\pgfqpoint{1.540123in}{0.740092in}}%
\pgfpathlineto{\pgfqpoint{1.547000in}{0.746810in}}%
\pgfpathlineto{\pgfqpoint{1.554176in}{0.753703in}}%
\pgfpathlineto{\pgfqpoint{1.562657in}{0.762147in}}%
\pgfpathlineto{\pgfqpoint{1.567963in}{0.767314in}}%
\pgfpathlineto{\pgfqpoint{1.578314in}{0.777806in}}%
\pgfpathlineto{\pgfqpoint{1.581478in}{0.780925in}}%
\pgfpathlineto{\pgfqpoint{1.593970in}{0.793801in}}%
\pgfpathlineto{\pgfqpoint{1.594708in}{0.794536in}}%
\pgfpathlineto{\pgfqpoint{1.607735in}{0.808148in}}%
\pgfpathlineto{\pgfqpoint{1.609627in}{0.810234in}}%
\pgfpathlineto{\pgfqpoint{1.620518in}{0.821759in}}%
\pgfpathlineto{\pgfqpoint{1.625283in}{0.827116in}}%
\pgfpathlineto{\pgfqpoint{1.632996in}{0.835370in}}%
\pgfpathlineto{\pgfqpoint{1.640940in}{0.844457in}}%
\pgfpathlineto{\pgfqpoint{1.645130in}{0.848981in}}%
\pgfpathlineto{\pgfqpoint{1.656596in}{0.862306in}}%
\pgfpathlineto{\pgfqpoint{1.656859in}{0.862592in}}%
\pgfpathlineto{\pgfqpoint{1.668440in}{0.876203in}}%
\pgfpathlineto{\pgfqpoint{1.672253in}{0.881110in}}%
\pgfpathlineto{\pgfqpoint{1.679573in}{0.889814in}}%
\pgfpathlineto{\pgfqpoint{1.687910in}{0.900789in}}%
\pgfpathlineto{\pgfqpoint{1.690106in}{0.903425in}}%
\pgfpathlineto{\pgfqpoint{1.700236in}{0.917036in}}%
\pgfpathlineto{\pgfqpoint{1.703566in}{0.922136in}}%
\pgfpathlineto{\pgfqpoint{1.709766in}{0.930648in}}%
\pgfpathlineto{\pgfqpoint{1.718276in}{0.944259in}}%
\pgfpathlineto{\pgfqpoint{1.719223in}{0.946071in}}%
\pgfpathlineto{\pgfqpoint{1.726242in}{0.957870in}}%
\pgfpathlineto{\pgfqpoint{1.732722in}{0.971481in}}%
\pgfpathlineto{\pgfqpoint{1.734879in}{0.977519in}}%
\pgfpathlineto{\pgfqpoint{1.738045in}{0.985092in}}%
\pgfpathlineto{\pgfqpoint{1.741840in}{0.998703in}}%
\pgfpathlineto{\pgfqpoint{1.743738in}{1.012314in}}%
\pgfpathlineto{\pgfqpoint{1.743738in}{1.025925in}}%
\pgfpathlineto{\pgfqpoint{1.741840in}{1.039536in}}%
\pgfpathlineto{\pgfqpoint{1.738045in}{1.053148in}}%
\pgfpathlineto{\pgfqpoint{1.734879in}{1.060720in}}%
\pgfpathlineto{\pgfqpoint{1.732722in}{1.066759in}}%
\pgfpathlineto{\pgfqpoint{1.726242in}{1.080370in}}%
\pgfpathlineto{\pgfqpoint{1.719223in}{1.092168in}}%
\pgfpathlineto{\pgfqpoint{1.718276in}{1.093981in}}%
\pgfpathlineto{\pgfqpoint{1.709766in}{1.107592in}}%
\pgfpathlineto{\pgfqpoint{1.703566in}{1.116103in}}%
\pgfpathlineto{\pgfqpoint{1.700236in}{1.121203in}}%
\pgfpathlineto{\pgfqpoint{1.690106in}{1.134814in}}%
\pgfpathlineto{\pgfqpoint{1.687910in}{1.137451in}}%
\pgfpathlineto{\pgfqpoint{1.679573in}{1.148425in}}%
\pgfpathlineto{\pgfqpoint{1.672253in}{1.157129in}}%
\pgfpathlineto{\pgfqpoint{1.668440in}{1.162036in}}%
\pgfpathlineto{\pgfqpoint{1.656859in}{1.175647in}}%
\pgfpathlineto{\pgfqpoint{1.656596in}{1.175933in}}%
\pgfpathlineto{\pgfqpoint{1.645130in}{1.189259in}}%
\pgfpathlineto{\pgfqpoint{1.640940in}{1.193783in}}%
\pgfpathlineto{\pgfqpoint{1.632996in}{1.202870in}}%
\pgfpathlineto{\pgfqpoint{1.625283in}{1.211124in}}%
\pgfpathlineto{\pgfqpoint{1.620518in}{1.216481in}}%
\pgfpathlineto{\pgfqpoint{1.609627in}{1.228006in}}%
\pgfpathlineto{\pgfqpoint{1.607735in}{1.230092in}}%
\pgfpathlineto{\pgfqpoint{1.594708in}{1.243703in}}%
\pgfpathlineto{\pgfqpoint{1.593970in}{1.244438in}}%
\pgfpathlineto{\pgfqpoint{1.581478in}{1.257314in}}%
\pgfpathlineto{\pgfqpoint{1.578314in}{1.260433in}}%
\pgfpathlineto{\pgfqpoint{1.567963in}{1.270925in}}%
\pgfpathlineto{\pgfqpoint{1.562657in}{1.276093in}}%
\pgfpathlineto{\pgfqpoint{1.554176in}{1.284536in}}%
\pgfpathlineto{\pgfqpoint{1.547000in}{1.291430in}}%
\pgfpathlineto{\pgfqpoint{1.540123in}{1.298148in}}%
\pgfpathlineto{\pgfqpoint{1.531344in}{1.306457in}}%
\pgfpathlineto{\pgfqpoint{1.525807in}{1.311759in}}%
\pgfpathlineto{\pgfqpoint{1.515687in}{1.321186in}}%
\pgfpathlineto{\pgfqpoint{1.511227in}{1.325370in}}%
\pgfpathlineto{\pgfqpoint{1.500031in}{1.335628in}}%
\pgfpathlineto{\pgfqpoint{1.496379in}{1.338981in}}%
\pgfpathlineto{\pgfqpoint{1.484374in}{1.349793in}}%
\pgfpathlineto{\pgfqpoint{1.481258in}{1.352592in}}%
\pgfpathlineto{\pgfqpoint{1.468718in}{1.363690in}}%
\pgfpathlineto{\pgfqpoint{1.465858in}{1.366203in}}%
\pgfpathlineto{\pgfqpoint{1.453061in}{1.377328in}}%
\pgfpathlineto{\pgfqpoint{1.450170in}{1.379814in}}%
\pgfpathlineto{\pgfqpoint{1.437404in}{1.390717in}}%
\pgfpathlineto{\pgfqpoint{1.434185in}{1.393425in}}%
\pgfpathlineto{\pgfqpoint{1.421748in}{1.403862in}}%
\pgfpathlineto{\pgfqpoint{1.417891in}{1.407036in}}%
\pgfpathlineto{\pgfqpoint{1.406091in}{1.416770in}}%
\pgfpathlineto{\pgfqpoint{1.401279in}{1.420648in}}%
\pgfpathlineto{\pgfqpoint{1.390435in}{1.429445in}}%
\pgfpathlineto{\pgfqpoint{1.384337in}{1.434259in}}%
\pgfpathlineto{\pgfqpoint{1.374778in}{1.441891in}}%
\pgfpathlineto{\pgfqpoint{1.367051in}{1.447870in}}%
\pgfpathlineto{\pgfqpoint{1.359122in}{1.454108in}}%
\pgfpathlineto{\pgfqpoint{1.349409in}{1.461481in}}%
\pgfpathlineto{\pgfqpoint{1.343465in}{1.466094in}}%
\pgfpathlineto{\pgfqpoint{1.331396in}{1.475092in}}%
\pgfpathlineto{\pgfqpoint{1.327809in}{1.477843in}}%
\pgfpathlineto{\pgfqpoint{1.312997in}{1.488703in}}%
\pgfpathlineto{\pgfqpoint{1.312152in}{1.489345in}}%
\pgfpathlineto{\pgfqpoint{1.296495in}{1.500670in}}%
\pgfpathlineto{\pgfqpoint{1.294096in}{1.502314in}}%
\pgfpathlineto{\pgfqpoint{1.280839in}{1.511783in}}%
\pgfpathlineto{\pgfqpoint{1.274677in}{1.515925in}}%
\pgfpathlineto{\pgfqpoint{1.265182in}{1.522631in}}%
\pgfpathlineto{\pgfqpoint{1.254729in}{1.529536in}}%
\pgfpathlineto{\pgfqpoint{1.249526in}{1.533179in}}%
\pgfpathlineto{\pgfqpoint{1.234198in}{1.543148in}}%
\pgfpathlineto{\pgfqpoint{1.233869in}{1.543376in}}%
\pgfpathlineto{\pgfqpoint{1.218213in}{1.553444in}}%
\pgfpathlineto{\pgfqpoint{1.212568in}{1.556759in}}%
\pgfpathlineto{\pgfqpoint{1.202556in}{1.563123in}}%
\pgfpathlineto{\pgfqpoint{1.189932in}{1.570370in}}%
\pgfpathlineto{\pgfqpoint{1.186899in}{1.572280in}}%
\pgfpathlineto{\pgfqpoint{1.171243in}{1.581086in}}%
\pgfpathlineto{\pgfqpoint{1.165377in}{1.583981in}}%
\pgfpathlineto{\pgfqpoint{1.155586in}{1.589371in}}%
\pgfpathlineto{\pgfqpoint{1.139930in}{1.596769in}}%
\pgfpathlineto{\pgfqpoint{1.137845in}{1.597592in}}%
\pgfpathlineto{\pgfqpoint{1.124273in}{1.603695in}}%
\pgfpathlineto{\pgfqpoint{1.108617in}{1.609327in}}%
\pgfpathlineto{\pgfqpoint{1.101671in}{1.611203in}}%
\pgfpathlineto{\pgfqpoint{1.092960in}{1.613955in}}%
\pgfpathlineto{\pgfqpoint{1.077303in}{1.617255in}}%
\pgfpathlineto{\pgfqpoint{1.061647in}{1.618904in}}%
\pgfpathlineto{\pgfqpoint{1.045990in}{1.618904in}}%
\pgfpathlineto{\pgfqpoint{1.030334in}{1.617255in}}%
\pgfpathlineto{\pgfqpoint{1.014677in}{1.613955in}}%
\pgfpathlineto{\pgfqpoint{1.005967in}{1.611203in}}%
\pgfpathlineto{\pgfqpoint{0.999021in}{1.609327in}}%
\pgfpathlineto{\pgfqpoint{0.983364in}{1.603695in}}%
\pgfpathlineto{\pgfqpoint{0.969793in}{1.597592in}}%
\pgfpathlineto{\pgfqpoint{0.967708in}{1.596769in}}%
\pgfpathlineto{\pgfqpoint{0.952051in}{1.589371in}}%
\pgfpathlineto{\pgfqpoint{0.942260in}{1.583981in}}%
\pgfpathlineto{\pgfqpoint{0.936394in}{1.581086in}}%
\pgfpathlineto{\pgfqpoint{0.920738in}{1.572280in}}%
\pgfpathlineto{\pgfqpoint{0.917705in}{1.570370in}}%
\pgfpathlineto{\pgfqpoint{0.905081in}{1.563123in}}%
\pgfpathlineto{\pgfqpoint{0.895070in}{1.556759in}}%
\pgfpathlineto{\pgfqpoint{0.889425in}{1.553444in}}%
\pgfpathlineto{\pgfqpoint{0.873768in}{1.543376in}}%
\pgfpathlineto{\pgfqpoint{0.873440in}{1.543148in}}%
\pgfpathlineto{\pgfqpoint{0.858112in}{1.533179in}}%
\pgfpathlineto{\pgfqpoint{0.852908in}{1.529536in}}%
\pgfpathlineto{\pgfqpoint{0.842455in}{1.522631in}}%
\pgfpathlineto{\pgfqpoint{0.832960in}{1.515925in}}%
\pgfpathlineto{\pgfqpoint{0.826798in}{1.511783in}}%
\pgfpathlineto{\pgfqpoint{0.813542in}{1.502314in}}%
\pgfpathlineto{\pgfqpoint{0.811142in}{1.500670in}}%
\pgfpathlineto{\pgfqpoint{0.795485in}{1.489345in}}%
\pgfpathlineto{\pgfqpoint{0.794640in}{1.488703in}}%
\pgfpathlineto{\pgfqpoint{0.779829in}{1.477843in}}%
\pgfpathlineto{\pgfqpoint{0.776241in}{1.475092in}}%
\pgfpathlineto{\pgfqpoint{0.764172in}{1.466094in}}%
\pgfpathlineto{\pgfqpoint{0.758228in}{1.461481in}}%
\pgfpathlineto{\pgfqpoint{0.748516in}{1.454108in}}%
\pgfpathlineto{\pgfqpoint{0.740586in}{1.447870in}}%
\pgfpathlineto{\pgfqpoint{0.732859in}{1.441891in}}%
\pgfpathlineto{\pgfqpoint{0.723301in}{1.434259in}}%
\pgfpathlineto{\pgfqpoint{0.717202in}{1.429445in}}%
\pgfpathlineto{\pgfqpoint{0.706358in}{1.420648in}}%
\pgfpathlineto{\pgfqpoint{0.701546in}{1.416770in}}%
\pgfpathlineto{\pgfqpoint{0.689746in}{1.407036in}}%
\pgfpathlineto{\pgfqpoint{0.685889in}{1.403862in}}%
\pgfpathlineto{\pgfqpoint{0.673453in}{1.393425in}}%
\pgfpathlineto{\pgfqpoint{0.670233in}{1.390717in}}%
\pgfpathlineto{\pgfqpoint{0.657467in}{1.379814in}}%
\pgfpathlineto{\pgfqpoint{0.654576in}{1.377328in}}%
\pgfpathlineto{\pgfqpoint{0.641779in}{1.366203in}}%
\pgfpathlineto{\pgfqpoint{0.638920in}{1.363690in}}%
\pgfpathlineto{\pgfqpoint{0.626379in}{1.352592in}}%
\pgfpathlineto{\pgfqpoint{0.623263in}{1.349793in}}%
\pgfpathlineto{\pgfqpoint{0.611258in}{1.338981in}}%
\pgfpathlineto{\pgfqpoint{0.607606in}{1.335628in}}%
\pgfpathlineto{\pgfqpoint{0.596411in}{1.325370in}}%
\pgfpathlineto{\pgfqpoint{0.591950in}{1.321186in}}%
\pgfpathlineto{\pgfqpoint{0.581830in}{1.311759in}}%
\pgfpathlineto{\pgfqpoint{0.576293in}{1.306457in}}%
\pgfpathlineto{\pgfqpoint{0.567514in}{1.298148in}}%
\pgfpathlineto{\pgfqpoint{0.560637in}{1.291430in}}%
\pgfpathlineto{\pgfqpoint{0.553461in}{1.284536in}}%
\pgfpathlineto{\pgfqpoint{0.544980in}{1.276093in}}%
\pgfpathlineto{\pgfqpoint{0.539674in}{1.270925in}}%
\pgfpathlineto{\pgfqpoint{0.529324in}{1.260433in}}%
\pgfpathlineto{\pgfqpoint{0.526160in}{1.257314in}}%
\pgfpathlineto{\pgfqpoint{0.513667in}{1.244438in}}%
\pgfpathlineto{\pgfqpoint{0.512929in}{1.243703in}}%
\pgfpathlineto{\pgfqpoint{0.499902in}{1.230092in}}%
\pgfpathlineto{\pgfqpoint{0.498011in}{1.228006in}}%
\pgfpathlineto{\pgfqpoint{0.487119in}{1.216481in}}%
\pgfpathlineto{\pgfqpoint{0.482354in}{1.211124in}}%
\pgfpathlineto{\pgfqpoint{0.474641in}{1.202870in}}%
\pgfpathlineto{\pgfqpoint{0.466697in}{1.193783in}}%
\pgfpathlineto{\pgfqpoint{0.462508in}{1.189259in}}%
\pgfpathlineto{\pgfqpoint{0.451041in}{1.175933in}}%
\pgfpathlineto{\pgfqpoint{0.450778in}{1.175647in}}%
\pgfpathlineto{\pgfqpoint{0.439197in}{1.162036in}}%
\pgfpathlineto{\pgfqpoint{0.435384in}{1.157129in}}%
\pgfpathlineto{\pgfqpoint{0.428064in}{1.148425in}}%
\pgfpathlineto{\pgfqpoint{0.419728in}{1.137451in}}%
\pgfpathlineto{\pgfqpoint{0.417531in}{1.134814in}}%
\pgfpathlineto{\pgfqpoint{0.407401in}{1.121203in}}%
\pgfpathlineto{\pgfqpoint{0.404071in}{1.116103in}}%
\pgfpathlineto{\pgfqpoint{0.397871in}{1.107592in}}%
\pgfpathlineto{\pgfqpoint{0.389362in}{1.093981in}}%
\pgfpathlineto{\pgfqpoint{0.388415in}{1.092168in}}%
\pgfpathlineto{\pgfqpoint{0.381395in}{1.080370in}}%
\pgfpathlineto{\pgfqpoint{0.374916in}{1.066759in}}%
\pgfpathlineto{\pgfqpoint{0.372758in}{1.060720in}}%
\pgfpathlineto{\pgfqpoint{0.369593in}{1.053148in}}%
\pgfpathlineto{\pgfqpoint{0.365797in}{1.039536in}}%
\pgfpathlineto{\pgfqpoint{0.363899in}{1.025925in}}%
\pgfpathlineto{\pgfqpoint{0.363899in}{1.012314in}}%
\pgfpathlineto{\pgfqpoint{0.365797in}{0.998703in}}%
\pgfpathlineto{\pgfqpoint{0.369593in}{0.985092in}}%
\pgfpathlineto{\pgfqpoint{0.372758in}{0.977519in}}%
\pgfpathlineto{\pgfqpoint{0.374916in}{0.971481in}}%
\pgfpathlineto{\pgfqpoint{0.381395in}{0.957870in}}%
\pgfpathlineto{\pgfqpoint{0.388415in}{0.946071in}}%
\pgfpathlineto{\pgfqpoint{0.389362in}{0.944259in}}%
\pgfpathlineto{\pgfqpoint{0.397871in}{0.930648in}}%
\pgfpathlineto{\pgfqpoint{0.404071in}{0.922136in}}%
\pgfpathlineto{\pgfqpoint{0.407401in}{0.917036in}}%
\pgfpathlineto{\pgfqpoint{0.417531in}{0.903425in}}%
\pgfpathlineto{\pgfqpoint{0.419728in}{0.900789in}}%
\pgfpathlineto{\pgfqpoint{0.428064in}{0.889814in}}%
\pgfpathlineto{\pgfqpoint{0.435384in}{0.881110in}}%
\pgfpathlineto{\pgfqpoint{0.439197in}{0.876203in}}%
\pgfpathlineto{\pgfqpoint{0.450778in}{0.862592in}}%
\pgfpathlineto{\pgfqpoint{0.451041in}{0.862306in}}%
\pgfpathlineto{\pgfqpoint{0.462508in}{0.848981in}}%
\pgfpathlineto{\pgfqpoint{0.466697in}{0.844457in}}%
\pgfpathlineto{\pgfqpoint{0.474641in}{0.835370in}}%
\pgfpathlineto{\pgfqpoint{0.482354in}{0.827116in}}%
\pgfpathlineto{\pgfqpoint{0.487119in}{0.821759in}}%
\pgfpathlineto{\pgfqpoint{0.498011in}{0.810234in}}%
\pgfpathlineto{\pgfqpoint{0.499902in}{0.808148in}}%
\pgfpathlineto{\pgfqpoint{0.512929in}{0.794536in}}%
\pgfpathlineto{\pgfqpoint{0.513667in}{0.793801in}}%
\pgfpathlineto{\pgfqpoint{0.526160in}{0.780925in}}%
\pgfpathlineto{\pgfqpoint{0.529324in}{0.777806in}}%
\pgfpathlineto{\pgfqpoint{0.539674in}{0.767314in}}%
\pgfpathlineto{\pgfqpoint{0.544980in}{0.762147in}}%
\pgfpathlineto{\pgfqpoint{0.553461in}{0.753703in}}%
\pgfpathlineto{\pgfqpoint{0.560637in}{0.746810in}}%
\pgfpathlineto{\pgfqpoint{0.567514in}{0.740092in}}%
\pgfpathlineto{\pgfqpoint{0.576293in}{0.731782in}}%
\pgfpathlineto{\pgfqpoint{0.581830in}{0.726481in}}%
\pgfpathlineto{\pgfqpoint{0.591950in}{0.717053in}}%
\pgfpathlineto{\pgfqpoint{0.596411in}{0.712870in}}%
\pgfpathlineto{\pgfqpoint{0.607606in}{0.702611in}}%
\pgfpathlineto{\pgfqpoint{0.611258in}{0.699259in}}%
\pgfpathlineto{\pgfqpoint{0.623263in}{0.688447in}}%
\pgfpathlineto{\pgfqpoint{0.626379in}{0.685648in}}%
\pgfpathlineto{\pgfqpoint{0.638920in}{0.674550in}}%
\pgfpathlineto{\pgfqpoint{0.641779in}{0.672036in}}%
\pgfpathlineto{\pgfqpoint{0.654576in}{0.660911in}}%
\pgfpathlineto{\pgfqpoint{0.657467in}{0.658425in}}%
\pgfpathlineto{\pgfqpoint{0.670233in}{0.647523in}}%
\pgfpathlineto{\pgfqpoint{0.673453in}{0.644814in}}%
\pgfpathlineto{\pgfqpoint{0.685889in}{0.634378in}}%
\pgfpathlineto{\pgfqpoint{0.689746in}{0.631203in}}%
\pgfpathlineto{\pgfqpoint{0.701546in}{0.621470in}}%
\pgfpathlineto{\pgfqpoint{0.706358in}{0.617592in}}%
\pgfpathlineto{\pgfqpoint{0.717202in}{0.608794in}}%
\pgfpathlineto{\pgfqpoint{0.723301in}{0.603981in}}%
\pgfpathlineto{\pgfqpoint{0.732859in}{0.596349in}}%
\pgfpathlineto{\pgfqpoint{0.740586in}{0.590370in}}%
\pgfpathlineto{\pgfqpoint{0.748516in}{0.584132in}}%
\pgfpathlineto{\pgfqpoint{0.758228in}{0.576759in}}%
\pgfpathlineto{\pgfqpoint{0.764172in}{0.572146in}}%
\pgfpathlineto{\pgfqpoint{0.776241in}{0.563148in}}%
\pgfpathlineto{\pgfqpoint{0.779829in}{0.560397in}}%
\pgfpathlineto{\pgfqpoint{0.794640in}{0.549536in}}%
\pgfpathlineto{\pgfqpoint{0.795485in}{0.548895in}}%
\pgfpathlineto{\pgfqpoint{0.811142in}{0.537569in}}%
\pgfpathlineto{\pgfqpoint{0.813542in}{0.535925in}}%
\pgfpathlineto{\pgfqpoint{0.826798in}{0.526457in}}%
\pgfpathlineto{\pgfqpoint{0.832960in}{0.522314in}}%
\pgfpathlineto{\pgfqpoint{0.842455in}{0.515609in}}%
\pgfpathlineto{\pgfqpoint{0.852908in}{0.508703in}}%
\pgfpathlineto{\pgfqpoint{0.858112in}{0.505061in}}%
\pgfpathlineto{\pgfqpoint{0.873440in}{0.495092in}}%
\pgfpathlineto{\pgfqpoint{0.873768in}{0.494863in}}%
\pgfpathlineto{\pgfqpoint{0.889425in}{0.484795in}}%
\pgfpathlineto{\pgfqpoint{0.895070in}{0.481481in}}%
\pgfpathlineto{\pgfqpoint{0.905081in}{0.475117in}}%
\pgfpathlineto{\pgfqpoint{0.917705in}{0.467870in}}%
\pgfpathlineto{\pgfqpoint{0.920738in}{0.465960in}}%
\pgfpathlineto{\pgfqpoint{0.936394in}{0.457153in}}%
\pgfpathlineto{\pgfqpoint{0.942260in}{0.454259in}}%
\pgfpathlineto{\pgfqpoint{0.952051in}{0.448869in}}%
\pgfpathlineto{\pgfqpoint{0.967708in}{0.441471in}}%
\pgfpathlineto{\pgfqpoint{0.969793in}{0.440648in}}%
\pgfpathlineto{\pgfqpoint{0.983364in}{0.434545in}}%
\pgfpathlineto{\pgfqpoint{0.999021in}{0.428912in}}%
\pgfpathlineto{\pgfqpoint{1.005967in}{0.427036in}}%
\pgfpathlineto{\pgfqpoint{1.014677in}{0.424285in}}%
\pgfpathclose%
\pgfpathmoveto{\pgfqpoint{0.998115in}{0.563148in}}%
\pgfpathlineto{\pgfqpoint{0.983364in}{0.565871in}}%
\pgfpathlineto{\pgfqpoint{0.967708in}{0.569486in}}%
\pgfpathlineto{\pgfqpoint{0.952051in}{0.573826in}}%
\pgfpathlineto{\pgfqpoint{0.942969in}{0.576759in}}%
\pgfpathlineto{\pgfqpoint{0.936394in}{0.578820in}}%
\pgfpathlineto{\pgfqpoint{0.920738in}{0.584417in}}%
\pgfpathlineto{\pgfqpoint{0.905937in}{0.590370in}}%
\pgfpathlineto{\pgfqpoint{0.905081in}{0.590706in}}%
\pgfpathlineto{\pgfqpoint{0.889425in}{0.597506in}}%
\pgfpathlineto{\pgfqpoint{0.875858in}{0.603981in}}%
\pgfpathlineto{\pgfqpoint{0.873768in}{0.604962in}}%
\pgfpathlineto{\pgfqpoint{0.858112in}{0.612924in}}%
\pgfpathlineto{\pgfqpoint{0.849602in}{0.617592in}}%
\pgfpathlineto{\pgfqpoint{0.842455in}{0.621470in}}%
\pgfpathlineto{\pgfqpoint{0.826798in}{0.630567in}}%
\pgfpathlineto{\pgfqpoint{0.825765in}{0.631203in}}%
\pgfpathlineto{\pgfqpoint{0.811142in}{0.640161in}}%
\pgfpathlineto{\pgfqpoint{0.803963in}{0.644814in}}%
\pgfpathlineto{\pgfqpoint{0.795485in}{0.650305in}}%
\pgfpathlineto{\pgfqpoint{0.783570in}{0.658425in}}%
\pgfpathlineto{\pgfqpoint{0.779829in}{0.660985in}}%
\pgfpathlineto{\pgfqpoint{0.764393in}{0.672036in}}%
\pgfpathlineto{\pgfqpoint{0.764172in}{0.672196in}}%
\pgfpathlineto{\pgfqpoint{0.748516in}{0.683947in}}%
\pgfpathlineto{\pgfqpoint{0.746329in}{0.685648in}}%
\pgfpathlineto{\pgfqpoint{0.732859in}{0.696265in}}%
\pgfpathlineto{\pgfqpoint{0.729179in}{0.699259in}}%
\pgfpathlineto{\pgfqpoint{0.717202in}{0.709173in}}%
\pgfpathlineto{\pgfqpoint{0.712855in}{0.712870in}}%
\pgfpathlineto{\pgfqpoint{0.701546in}{0.722701in}}%
\pgfpathlineto{\pgfqpoint{0.697294in}{0.726481in}}%
\pgfpathlineto{\pgfqpoint{0.685889in}{0.736892in}}%
\pgfpathlineto{\pgfqpoint{0.682446in}{0.740092in}}%
\pgfpathlineto{\pgfqpoint{0.670233in}{0.751802in}}%
\pgfpathlineto{\pgfqpoint{0.668276in}{0.753703in}}%
\pgfpathlineto{\pgfqpoint{0.654759in}{0.767314in}}%
\pgfpathlineto{\pgfqpoint{0.654576in}{0.767506in}}%
\pgfpathlineto{\pgfqpoint{0.641864in}{0.780925in}}%
\pgfpathlineto{\pgfqpoint{0.638920in}{0.784178in}}%
\pgfpathlineto{\pgfqpoint{0.629579in}{0.794536in}}%
\pgfpathlineto{\pgfqpoint{0.623263in}{0.801907in}}%
\pgfpathlineto{\pgfqpoint{0.617911in}{0.808148in}}%
\pgfpathlineto{\pgfqpoint{0.607606in}{0.820860in}}%
\pgfpathlineto{\pgfqpoint{0.606874in}{0.821759in}}%
\pgfpathlineto{\pgfqpoint{0.596410in}{0.835370in}}%
\pgfpathlineto{\pgfqpoint{0.591950in}{0.841583in}}%
\pgfpathlineto{\pgfqpoint{0.586581in}{0.848981in}}%
\pgfpathlineto{\pgfqpoint{0.577422in}{0.862592in}}%
\pgfpathlineto{\pgfqpoint{0.576293in}{0.864409in}}%
\pgfpathlineto{\pgfqpoint{0.568845in}{0.876203in}}%
\pgfpathlineto{\pgfqpoint{0.561024in}{0.889814in}}%
\pgfpathlineto{\pgfqpoint{0.560637in}{0.890558in}}%
\pgfpathlineto{\pgfqpoint{0.553789in}{0.903425in}}%
\pgfpathlineto{\pgfqpoint{0.547351in}{0.917036in}}%
\pgfpathlineto{\pgfqpoint{0.544980in}{0.922752in}}%
\pgfpathlineto{\pgfqpoint{0.541607in}{0.930648in}}%
\pgfpathlineto{\pgfqpoint{0.536615in}{0.944259in}}%
\pgfpathlineto{\pgfqpoint{0.532457in}{0.957870in}}%
\pgfpathlineto{\pgfqpoint{0.529324in}{0.970693in}}%
\pgfpathlineto{\pgfqpoint{0.529124in}{0.971481in}}%
\pgfpathlineto{\pgfqpoint{0.526530in}{0.985092in}}%
\pgfpathlineto{\pgfqpoint{0.524801in}{0.998703in}}%
\pgfpathlineto{\pgfqpoint{0.523937in}{1.012314in}}%
\pgfpathlineto{\pgfqpoint{0.523937in}{1.025925in}}%
\pgfpathlineto{\pgfqpoint{0.524801in}{1.039536in}}%
\pgfpathlineto{\pgfqpoint{0.526530in}{1.053148in}}%
\pgfpathlineto{\pgfqpoint{0.529124in}{1.066759in}}%
\pgfpathlineto{\pgfqpoint{0.529324in}{1.067546in}}%
\pgfpathlineto{\pgfqpoint{0.532457in}{1.080370in}}%
\pgfpathlineto{\pgfqpoint{0.536615in}{1.093981in}}%
\pgfpathlineto{\pgfqpoint{0.541607in}{1.107592in}}%
\pgfpathlineto{\pgfqpoint{0.544980in}{1.115488in}}%
\pgfpathlineto{\pgfqpoint{0.547351in}{1.121203in}}%
\pgfpathlineto{\pgfqpoint{0.553789in}{1.134814in}}%
\pgfpathlineto{\pgfqpoint{0.560637in}{1.147681in}}%
\pgfpathlineto{\pgfqpoint{0.561024in}{1.148425in}}%
\pgfpathlineto{\pgfqpoint{0.568845in}{1.162036in}}%
\pgfpathlineto{\pgfqpoint{0.576293in}{1.173830in}}%
\pgfpathlineto{\pgfqpoint{0.577422in}{1.175647in}}%
\pgfpathlineto{\pgfqpoint{0.586581in}{1.189259in}}%
\pgfpathlineto{\pgfqpoint{0.591950in}{1.196656in}}%
\pgfpathlineto{\pgfqpoint{0.596410in}{1.202870in}}%
\pgfpathlineto{\pgfqpoint{0.606874in}{1.216481in}}%
\pgfpathlineto{\pgfqpoint{0.607606in}{1.217379in}}%
\pgfpathlineto{\pgfqpoint{0.617911in}{1.230092in}}%
\pgfpathlineto{\pgfqpoint{0.623263in}{1.236333in}}%
\pgfpathlineto{\pgfqpoint{0.629579in}{1.243703in}}%
\pgfpathlineto{\pgfqpoint{0.638920in}{1.254062in}}%
\pgfpathlineto{\pgfqpoint{0.641864in}{1.257314in}}%
\pgfpathlineto{\pgfqpoint{0.654576in}{1.270734in}}%
\pgfpathlineto{\pgfqpoint{0.654759in}{1.270925in}}%
\pgfpathlineto{\pgfqpoint{0.668276in}{1.284536in}}%
\pgfpathlineto{\pgfqpoint{0.670233in}{1.286437in}}%
\pgfpathlineto{\pgfqpoint{0.682446in}{1.298148in}}%
\pgfpathlineto{\pgfqpoint{0.685889in}{1.301347in}}%
\pgfpathlineto{\pgfqpoint{0.697294in}{1.311759in}}%
\pgfpathlineto{\pgfqpoint{0.701546in}{1.315538in}}%
\pgfpathlineto{\pgfqpoint{0.712855in}{1.325370in}}%
\pgfpathlineto{\pgfqpoint{0.717202in}{1.329066in}}%
\pgfpathlineto{\pgfqpoint{0.729179in}{1.338981in}}%
\pgfpathlineto{\pgfqpoint{0.732859in}{1.341974in}}%
\pgfpathlineto{\pgfqpoint{0.746329in}{1.352592in}}%
\pgfpathlineto{\pgfqpoint{0.748516in}{1.354293in}}%
\pgfpathlineto{\pgfqpoint{0.764172in}{1.366044in}}%
\pgfpathlineto{\pgfqpoint{0.764393in}{1.366203in}}%
\pgfpathlineto{\pgfqpoint{0.779829in}{1.377255in}}%
\pgfpathlineto{\pgfqpoint{0.783570in}{1.379814in}}%
\pgfpathlineto{\pgfqpoint{0.795485in}{1.387934in}}%
\pgfpathlineto{\pgfqpoint{0.803963in}{1.393425in}}%
\pgfpathlineto{\pgfqpoint{0.811142in}{1.398078in}}%
\pgfpathlineto{\pgfqpoint{0.825765in}{1.407036in}}%
\pgfpathlineto{\pgfqpoint{0.826798in}{1.407673in}}%
\pgfpathlineto{\pgfqpoint{0.842455in}{1.416770in}}%
\pgfpathlineto{\pgfqpoint{0.849602in}{1.420648in}}%
\pgfpathlineto{\pgfqpoint{0.858112in}{1.425315in}}%
\pgfpathlineto{\pgfqpoint{0.873768in}{1.433278in}}%
\pgfpathlineto{\pgfqpoint{0.875858in}{1.434259in}}%
\pgfpathlineto{\pgfqpoint{0.889425in}{1.440734in}}%
\pgfpathlineto{\pgfqpoint{0.905081in}{1.447533in}}%
\pgfpathlineto{\pgfqpoint{0.905937in}{1.447870in}}%
\pgfpathlineto{\pgfqpoint{0.920738in}{1.453822in}}%
\pgfpathlineto{\pgfqpoint{0.936394in}{1.459419in}}%
\pgfpathlineto{\pgfqpoint{0.942969in}{1.461481in}}%
\pgfpathlineto{\pgfqpoint{0.952051in}{1.464413in}}%
\pgfpathlineto{\pgfqpoint{0.967708in}{1.468753in}}%
\pgfpathlineto{\pgfqpoint{0.983364in}{1.472368in}}%
\pgfpathlineto{\pgfqpoint{0.998115in}{1.475092in}}%
\pgfpathlineto{\pgfqpoint{0.999021in}{1.475265in}}%
\pgfpathlineto{\pgfqpoint{1.014677in}{1.477520in}}%
\pgfpathlineto{\pgfqpoint{1.030334in}{1.479023in}}%
\pgfpathlineto{\pgfqpoint{1.045990in}{1.479775in}}%
\pgfpathlineto{\pgfqpoint{1.061647in}{1.479775in}}%
\pgfpathlineto{\pgfqpoint{1.077303in}{1.479023in}}%
\pgfpathlineto{\pgfqpoint{1.092960in}{1.477520in}}%
\pgfpathlineto{\pgfqpoint{1.108617in}{1.475265in}}%
\pgfpathlineto{\pgfqpoint{1.109522in}{1.475092in}}%
\pgfpathlineto{\pgfqpoint{1.124273in}{1.472368in}}%
\pgfpathlineto{\pgfqpoint{1.139930in}{1.468753in}}%
\pgfpathlineto{\pgfqpoint{1.155586in}{1.464413in}}%
\pgfpathlineto{\pgfqpoint{1.164668in}{1.461481in}}%
\pgfpathlineto{\pgfqpoint{1.171243in}{1.459419in}}%
\pgfpathlineto{\pgfqpoint{1.186899in}{1.453822in}}%
\pgfpathlineto{\pgfqpoint{1.201700in}{1.447870in}}%
\pgfpathlineto{\pgfqpoint{1.202556in}{1.447533in}}%
\pgfpathlineto{\pgfqpoint{1.218213in}{1.440734in}}%
\pgfpathlineto{\pgfqpoint{1.231779in}{1.434259in}}%
\pgfpathlineto{\pgfqpoint{1.233869in}{1.433278in}}%
\pgfpathlineto{\pgfqpoint{1.249526in}{1.425315in}}%
\pgfpathlineto{\pgfqpoint{1.258035in}{1.420648in}}%
\pgfpathlineto{\pgfqpoint{1.265182in}{1.416770in}}%
\pgfpathlineto{\pgfqpoint{1.280839in}{1.407673in}}%
\pgfpathlineto{\pgfqpoint{1.281872in}{1.407036in}}%
\pgfpathlineto{\pgfqpoint{1.296495in}{1.398078in}}%
\pgfpathlineto{\pgfqpoint{1.303674in}{1.393425in}}%
\pgfpathlineto{\pgfqpoint{1.312152in}{1.387934in}}%
\pgfpathlineto{\pgfqpoint{1.324067in}{1.379814in}}%
\pgfpathlineto{\pgfqpoint{1.327809in}{1.377255in}}%
\pgfpathlineto{\pgfqpoint{1.343245in}{1.366203in}}%
\pgfpathlineto{\pgfqpoint{1.343465in}{1.366044in}}%
\pgfpathlineto{\pgfqpoint{1.359122in}{1.354293in}}%
\pgfpathlineto{\pgfqpoint{1.361308in}{1.352592in}}%
\pgfpathlineto{\pgfqpoint{1.374778in}{1.341974in}}%
\pgfpathlineto{\pgfqpoint{1.378459in}{1.338981in}}%
\pgfpathlineto{\pgfqpoint{1.390435in}{1.329066in}}%
\pgfpathlineto{\pgfqpoint{1.394782in}{1.325370in}}%
\pgfpathlineto{\pgfqpoint{1.406091in}{1.315538in}}%
\pgfpathlineto{\pgfqpoint{1.410343in}{1.311759in}}%
\pgfpathlineto{\pgfqpoint{1.421748in}{1.301347in}}%
\pgfpathlineto{\pgfqpoint{1.425191in}{1.298148in}}%
\pgfpathlineto{\pgfqpoint{1.437404in}{1.286437in}}%
\pgfpathlineto{\pgfqpoint{1.439361in}{1.284536in}}%
\pgfpathlineto{\pgfqpoint{1.452878in}{1.270925in}}%
\pgfpathlineto{\pgfqpoint{1.453061in}{1.270734in}}%
\pgfpathlineto{\pgfqpoint{1.465774in}{1.257314in}}%
\pgfpathlineto{\pgfqpoint{1.468718in}{1.254062in}}%
\pgfpathlineto{\pgfqpoint{1.478058in}{1.243703in}}%
\pgfpathlineto{\pgfqpoint{1.484374in}{1.236333in}}%
\pgfpathlineto{\pgfqpoint{1.489726in}{1.230092in}}%
\pgfpathlineto{\pgfqpoint{1.500031in}{1.217379in}}%
\pgfpathlineto{\pgfqpoint{1.500763in}{1.216481in}}%
\pgfpathlineto{\pgfqpoint{1.511227in}{1.202870in}}%
\pgfpathlineto{\pgfqpoint{1.515687in}{1.196656in}}%
\pgfpathlineto{\pgfqpoint{1.521056in}{1.189259in}}%
\pgfpathlineto{\pgfqpoint{1.530215in}{1.175647in}}%
\pgfpathlineto{\pgfqpoint{1.531344in}{1.173830in}}%
\pgfpathlineto{\pgfqpoint{1.538792in}{1.162036in}}%
\pgfpathlineto{\pgfqpoint{1.546613in}{1.148425in}}%
\pgfpathlineto{\pgfqpoint{1.547000in}{1.147681in}}%
\pgfpathlineto{\pgfqpoint{1.553848in}{1.134814in}}%
\pgfpathlineto{\pgfqpoint{1.560286in}{1.121203in}}%
\pgfpathlineto{\pgfqpoint{1.562657in}{1.115488in}}%
\pgfpathlineto{\pgfqpoint{1.566030in}{1.107592in}}%
\pgfpathlineto{\pgfqpoint{1.571022in}{1.093981in}}%
\pgfpathlineto{\pgfqpoint{1.575180in}{1.080370in}}%
\pgfpathlineto{\pgfqpoint{1.578314in}{1.067546in}}%
\pgfpathlineto{\pgfqpoint{1.578513in}{1.066759in}}%
\pgfpathlineto{\pgfqpoint{1.581107in}{1.053148in}}%
\pgfpathlineto{\pgfqpoint{1.582836in}{1.039536in}}%
\pgfpathlineto{\pgfqpoint{1.583700in}{1.025925in}}%
\pgfpathlineto{\pgfqpoint{1.583700in}{1.012314in}}%
\pgfpathlineto{\pgfqpoint{1.582836in}{0.998703in}}%
\pgfpathlineto{\pgfqpoint{1.581107in}{0.985092in}}%
\pgfpathlineto{\pgfqpoint{1.578513in}{0.971481in}}%
\pgfpathlineto{\pgfqpoint{1.578314in}{0.970693in}}%
\pgfpathlineto{\pgfqpoint{1.575180in}{0.957870in}}%
\pgfpathlineto{\pgfqpoint{1.571022in}{0.944259in}}%
\pgfpathlineto{\pgfqpoint{1.566030in}{0.930647in}}%
\pgfpathlineto{\pgfqpoint{1.562657in}{0.922752in}}%
\pgfpathlineto{\pgfqpoint{1.560286in}{0.917036in}}%
\pgfpathlineto{\pgfqpoint{1.553848in}{0.903425in}}%
\pgfpathlineto{\pgfqpoint{1.547000in}{0.890558in}}%
\pgfpathlineto{\pgfqpoint{1.546613in}{0.889814in}}%
\pgfpathlineto{\pgfqpoint{1.538792in}{0.876203in}}%
\pgfpathlineto{\pgfqpoint{1.531344in}{0.864409in}}%
\pgfpathlineto{\pgfqpoint{1.530215in}{0.862592in}}%
\pgfpathlineto{\pgfqpoint{1.521056in}{0.848981in}}%
\pgfpathlineto{\pgfqpoint{1.515687in}{0.841583in}}%
\pgfpathlineto{\pgfqpoint{1.511227in}{0.835370in}}%
\pgfpathlineto{\pgfqpoint{1.500763in}{0.821759in}}%
\pgfpathlineto{\pgfqpoint{1.500031in}{0.820860in}}%
\pgfpathlineto{\pgfqpoint{1.489726in}{0.808148in}}%
\pgfpathlineto{\pgfqpoint{1.484374in}{0.801907in}}%
\pgfpathlineto{\pgfqpoint{1.478058in}{0.794536in}}%
\pgfpathlineto{\pgfqpoint{1.468718in}{0.784178in}}%
\pgfpathlineto{\pgfqpoint{1.465774in}{0.780925in}}%
\pgfpathlineto{\pgfqpoint{1.453061in}{0.767506in}}%
\pgfpathlineto{\pgfqpoint{1.452878in}{0.767314in}}%
\pgfpathlineto{\pgfqpoint{1.439361in}{0.753703in}}%
\pgfpathlineto{\pgfqpoint{1.437404in}{0.751802in}}%
\pgfpathlineto{\pgfqpoint{1.425191in}{0.740092in}}%
\pgfpathlineto{\pgfqpoint{1.421748in}{0.736892in}}%
\pgfpathlineto{\pgfqpoint{1.410343in}{0.726481in}}%
\pgfpathlineto{\pgfqpoint{1.406091in}{0.722701in}}%
\pgfpathlineto{\pgfqpoint{1.394782in}{0.712870in}}%
\pgfpathlineto{\pgfqpoint{1.390435in}{0.709173in}}%
\pgfpathlineto{\pgfqpoint{1.378459in}{0.699259in}}%
\pgfpathlineto{\pgfqpoint{1.374778in}{0.696265in}}%
\pgfpathlineto{\pgfqpoint{1.361308in}{0.685648in}}%
\pgfpathlineto{\pgfqpoint{1.359122in}{0.683947in}}%
\pgfpathlineto{\pgfqpoint{1.343465in}{0.672196in}}%
\pgfpathlineto{\pgfqpoint{1.343245in}{0.672036in}}%
\pgfpathlineto{\pgfqpoint{1.327809in}{0.660985in}}%
\pgfpathlineto{\pgfqpoint{1.324067in}{0.658425in}}%
\pgfpathlineto{\pgfqpoint{1.312152in}{0.650305in}}%
\pgfpathlineto{\pgfqpoint{1.303674in}{0.644814in}}%
\pgfpathlineto{\pgfqpoint{1.296495in}{0.640161in}}%
\pgfpathlineto{\pgfqpoint{1.281872in}{0.631203in}}%
\pgfpathlineto{\pgfqpoint{1.280839in}{0.630567in}}%
\pgfpathlineto{\pgfqpoint{1.265182in}{0.621470in}}%
\pgfpathlineto{\pgfqpoint{1.258035in}{0.617592in}}%
\pgfpathlineto{\pgfqpoint{1.249526in}{0.612924in}}%
\pgfpathlineto{\pgfqpoint{1.233869in}{0.604962in}}%
\pgfpathlineto{\pgfqpoint{1.231779in}{0.603981in}}%
\pgfpathlineto{\pgfqpoint{1.218213in}{0.597506in}}%
\pgfpathlineto{\pgfqpoint{1.202556in}{0.590706in}}%
\pgfpathlineto{\pgfqpoint{1.201700in}{0.590370in}}%
\pgfpathlineto{\pgfqpoint{1.186899in}{0.584417in}}%
\pgfpathlineto{\pgfqpoint{1.171243in}{0.578820in}}%
\pgfpathlineto{\pgfqpoint{1.164668in}{0.576759in}}%
\pgfpathlineto{\pgfqpoint{1.155586in}{0.573826in}}%
\pgfpathlineto{\pgfqpoint{1.139930in}{0.569486in}}%
\pgfpathlineto{\pgfqpoint{1.124273in}{0.565871in}}%
\pgfpathlineto{\pgfqpoint{1.109522in}{0.563148in}}%
\pgfpathlineto{\pgfqpoint{1.108617in}{0.562974in}}%
\pgfpathlineto{\pgfqpoint{1.092960in}{0.560719in}}%
\pgfpathlineto{\pgfqpoint{1.077303in}{0.559216in}}%
\pgfpathlineto{\pgfqpoint{1.061647in}{0.558465in}}%
\pgfpathlineto{\pgfqpoint{1.045990in}{0.558465in}}%
\pgfpathlineto{\pgfqpoint{1.030334in}{0.559216in}}%
\pgfpathlineto{\pgfqpoint{1.014677in}{0.560719in}}%
\pgfpathlineto{\pgfqpoint{0.999021in}{0.562974in}}%
\pgfpathlineto{\pgfqpoint{0.998115in}{0.563148in}}%
\pgfpathclose%
\pgfusepath{fill}%
\end{pgfscope}%
\begin{pgfscope}%
\pgfpathrectangle{\pgfqpoint{0.278819in}{0.345370in}}{\pgfqpoint{1.550000in}{1.347500in}}%
\pgfusepath{clip}%
\pgfsetbuttcap%
\pgfsetroundjoin%
\definecolor{currentfill}{rgb}{0.679160,0.151848,0.575189}%
\pgfsetfillcolor{currentfill}%
\pgfsetlinewidth{0.000000pt}%
\definecolor{currentstroke}{rgb}{0.000000,0.000000,0.000000}%
\pgfsetstrokecolor{currentstroke}%
\pgfsetdash{}{0pt}%
\pgfpathmoveto{\pgfqpoint{0.795485in}{0.345370in}}%
\pgfpathlineto{\pgfqpoint{0.811142in}{0.345370in}}%
\pgfpathlineto{\pgfqpoint{0.826798in}{0.345370in}}%
\pgfpathlineto{\pgfqpoint{0.842455in}{0.345370in}}%
\pgfpathlineto{\pgfqpoint{0.858112in}{0.345370in}}%
\pgfpathlineto{\pgfqpoint{0.873768in}{0.345370in}}%
\pgfpathlineto{\pgfqpoint{0.889425in}{0.345370in}}%
\pgfpathlineto{\pgfqpoint{0.905081in}{0.345370in}}%
\pgfpathlineto{\pgfqpoint{0.920738in}{0.345370in}}%
\pgfpathlineto{\pgfqpoint{0.936394in}{0.345370in}}%
\pgfpathlineto{\pgfqpoint{0.952051in}{0.345370in}}%
\pgfpathlineto{\pgfqpoint{0.967708in}{0.345370in}}%
\pgfpathlineto{\pgfqpoint{0.983364in}{0.345370in}}%
\pgfpathlineto{\pgfqpoint{0.999021in}{0.345370in}}%
\pgfpathlineto{\pgfqpoint{1.014677in}{0.345370in}}%
\pgfpathlineto{\pgfqpoint{1.030334in}{0.345370in}}%
\pgfpathlineto{\pgfqpoint{1.045990in}{0.345370in}}%
\pgfpathlineto{\pgfqpoint{1.061647in}{0.345370in}}%
\pgfpathlineto{\pgfqpoint{1.077303in}{0.345370in}}%
\pgfpathlineto{\pgfqpoint{1.092960in}{0.345370in}}%
\pgfpathlineto{\pgfqpoint{1.108617in}{0.345370in}}%
\pgfpathlineto{\pgfqpoint{1.124273in}{0.345370in}}%
\pgfpathlineto{\pgfqpoint{1.139930in}{0.345370in}}%
\pgfpathlineto{\pgfqpoint{1.155586in}{0.345370in}}%
\pgfpathlineto{\pgfqpoint{1.171243in}{0.345370in}}%
\pgfpathlineto{\pgfqpoint{1.186899in}{0.345370in}}%
\pgfpathlineto{\pgfqpoint{1.202556in}{0.345370in}}%
\pgfpathlineto{\pgfqpoint{1.218213in}{0.345370in}}%
\pgfpathlineto{\pgfqpoint{1.233869in}{0.345370in}}%
\pgfpathlineto{\pgfqpoint{1.249526in}{0.345370in}}%
\pgfpathlineto{\pgfqpoint{1.265182in}{0.345370in}}%
\pgfpathlineto{\pgfqpoint{1.280839in}{0.345370in}}%
\pgfpathlineto{\pgfqpoint{1.296495in}{0.345370in}}%
\pgfpathlineto{\pgfqpoint{1.312152in}{0.345370in}}%
\pgfpathlineto{\pgfqpoint{1.319147in}{0.345370in}}%
\pgfpathlineto{\pgfqpoint{1.319942in}{0.358981in}}%
\pgfpathlineto{\pgfqpoint{1.322318in}{0.372592in}}%
\pgfpathlineto{\pgfqpoint{1.326249in}{0.386203in}}%
\pgfpathlineto{\pgfqpoint{1.327809in}{0.390091in}}%
\pgfpathlineto{\pgfqpoint{1.331575in}{0.399814in}}%
\pgfpathlineto{\pgfqpoint{1.338264in}{0.413425in}}%
\pgfpathlineto{\pgfqpoint{1.343465in}{0.422220in}}%
\pgfpathlineto{\pgfqpoint{1.346223in}{0.427036in}}%
\pgfpathlineto{\pgfqpoint{1.355275in}{0.440648in}}%
\pgfpathlineto{\pgfqpoint{1.359122in}{0.445759in}}%
\pgfpathlineto{\pgfqpoint{1.365331in}{0.454259in}}%
\pgfpathlineto{\pgfqpoint{1.374778in}{0.465940in}}%
\pgfpathlineto{\pgfqpoint{1.376299in}{0.467870in}}%
\pgfpathlineto{\pgfqpoint{1.388009in}{0.481481in}}%
\pgfpathlineto{\pgfqpoint{1.390435in}{0.484098in}}%
\pgfpathlineto{\pgfqpoint{1.400390in}{0.495092in}}%
\pgfpathlineto{\pgfqpoint{1.406091in}{0.501007in}}%
\pgfpathlineto{\pgfqpoint{1.413363in}{0.508703in}}%
\pgfpathlineto{\pgfqpoint{1.421748in}{0.517120in}}%
\pgfpathlineto{\pgfqpoint{1.426838in}{0.522314in}}%
\pgfpathlineto{\pgfqpoint{1.437404in}{0.532624in}}%
\pgfpathlineto{\pgfqpoint{1.440744in}{0.535925in}}%
\pgfpathlineto{\pgfqpoint{1.453061in}{0.547652in}}%
\pgfpathlineto{\pgfqpoint{1.455021in}{0.549536in}}%
\pgfpathlineto{\pgfqpoint{1.468718in}{0.562299in}}%
\pgfpathlineto{\pgfqpoint{1.469623in}{0.563148in}}%
\pgfpathlineto{\pgfqpoint{1.484374in}{0.576631in}}%
\pgfpathlineto{\pgfqpoint{1.484514in}{0.576759in}}%
\pgfpathlineto{\pgfqpoint{1.499674in}{0.590370in}}%
\pgfpathlineto{\pgfqpoint{1.500031in}{0.590685in}}%
\pgfpathlineto{\pgfqpoint{1.515086in}{0.603981in}}%
\pgfpathlineto{\pgfqpoint{1.515687in}{0.604507in}}%
\pgfpathlineto{\pgfqpoint{1.530739in}{0.617592in}}%
\pgfpathlineto{\pgfqpoint{1.531344in}{0.618115in}}%
\pgfpathlineto{\pgfqpoint{1.546638in}{0.631203in}}%
\pgfpathlineto{\pgfqpoint{1.547000in}{0.631513in}}%
\pgfpathlineto{\pgfqpoint{1.562657in}{0.644693in}}%
\pgfpathlineto{\pgfqpoint{1.562804in}{0.644814in}}%
\pgfpathlineto{\pgfqpoint{1.578314in}{0.657638in}}%
\pgfpathlineto{\pgfqpoint{1.579290in}{0.658425in}}%
\pgfpathlineto{\pgfqpoint{1.593970in}{0.670333in}}%
\pgfpathlineto{\pgfqpoint{1.596137in}{0.672036in}}%
\pgfpathlineto{\pgfqpoint{1.609627in}{0.682744in}}%
\pgfpathlineto{\pgfqpoint{1.613424in}{0.685648in}}%
\pgfpathlineto{\pgfqpoint{1.625283in}{0.694834in}}%
\pgfpathlineto{\pgfqpoint{1.631259in}{0.699259in}}%
\pgfpathlineto{\pgfqpoint{1.640940in}{0.706548in}}%
\pgfpathlineto{\pgfqpoint{1.649793in}{0.712870in}}%
\pgfpathlineto{\pgfqpoint{1.656596in}{0.717826in}}%
\pgfpathlineto{\pgfqpoint{1.669243in}{0.726481in}}%
\pgfpathlineto{\pgfqpoint{1.672253in}{0.728590in}}%
\pgfpathlineto{\pgfqpoint{1.687910in}{0.738770in}}%
\pgfpathlineto{\pgfqpoint{1.690129in}{0.740092in}}%
\pgfpathlineto{\pgfqpoint{1.703566in}{0.748305in}}%
\pgfpathlineto{\pgfqpoint{1.713343in}{0.753703in}}%
\pgfpathlineto{\pgfqpoint{1.719223in}{0.757047in}}%
\pgfpathlineto{\pgfqpoint{1.734879in}{0.764917in}}%
\pgfpathlineto{\pgfqpoint{1.740419in}{0.767314in}}%
\pgfpathlineto{\pgfqpoint{1.750536in}{0.771836in}}%
\pgfpathlineto{\pgfqpoint{1.766192in}{0.777651in}}%
\pgfpathlineto{\pgfqpoint{1.777377in}{0.780925in}}%
\pgfpathlineto{\pgfqpoint{1.781849in}{0.782281in}}%
\pgfpathlineto{\pgfqpoint{1.797505in}{0.785699in}}%
\pgfpathlineto{\pgfqpoint{1.813162in}{0.787764in}}%
\pgfpathlineto{\pgfqpoint{1.828819in}{0.788455in}}%
\pgfpathlineto{\pgfqpoint{1.828819in}{0.794536in}}%
\pgfpathlineto{\pgfqpoint{1.828819in}{0.808148in}}%
\pgfpathlineto{\pgfqpoint{1.828819in}{0.821759in}}%
\pgfpathlineto{\pgfqpoint{1.828819in}{0.835370in}}%
\pgfpathlineto{\pgfqpoint{1.828819in}{0.848981in}}%
\pgfpathlineto{\pgfqpoint{1.828819in}{0.862592in}}%
\pgfpathlineto{\pgfqpoint{1.828819in}{0.876203in}}%
\pgfpathlineto{\pgfqpoint{1.828819in}{0.889814in}}%
\pgfpathlineto{\pgfqpoint{1.828819in}{0.903425in}}%
\pgfpathlineto{\pgfqpoint{1.828819in}{0.917036in}}%
\pgfpathlineto{\pgfqpoint{1.828819in}{0.930648in}}%
\pgfpathlineto{\pgfqpoint{1.828819in}{0.944259in}}%
\pgfpathlineto{\pgfqpoint{1.828819in}{0.957870in}}%
\pgfpathlineto{\pgfqpoint{1.828819in}{0.971481in}}%
\pgfpathlineto{\pgfqpoint{1.828819in}{0.985092in}}%
\pgfpathlineto{\pgfqpoint{1.828819in}{0.998703in}}%
\pgfpathlineto{\pgfqpoint{1.828819in}{1.012314in}}%
\pgfpathlineto{\pgfqpoint{1.828819in}{1.025925in}}%
\pgfpathlineto{\pgfqpoint{1.828819in}{1.039536in}}%
\pgfpathlineto{\pgfqpoint{1.828819in}{1.053148in}}%
\pgfpathlineto{\pgfqpoint{1.828819in}{1.066759in}}%
\pgfpathlineto{\pgfqpoint{1.828819in}{1.080370in}}%
\pgfpathlineto{\pgfqpoint{1.828819in}{1.093981in}}%
\pgfpathlineto{\pgfqpoint{1.828819in}{1.107592in}}%
\pgfpathlineto{\pgfqpoint{1.828819in}{1.121203in}}%
\pgfpathlineto{\pgfqpoint{1.828819in}{1.134814in}}%
\pgfpathlineto{\pgfqpoint{1.828819in}{1.148425in}}%
\pgfpathlineto{\pgfqpoint{1.828819in}{1.162036in}}%
\pgfpathlineto{\pgfqpoint{1.828819in}{1.175647in}}%
\pgfpathlineto{\pgfqpoint{1.828819in}{1.189259in}}%
\pgfpathlineto{\pgfqpoint{1.828819in}{1.202870in}}%
\pgfpathlineto{\pgfqpoint{1.828819in}{1.216481in}}%
\pgfpathlineto{\pgfqpoint{1.828819in}{1.230092in}}%
\pgfpathlineto{\pgfqpoint{1.828819in}{1.243703in}}%
\pgfpathlineto{\pgfqpoint{1.828819in}{1.249784in}}%
\pgfpathlineto{\pgfqpoint{1.813162in}{1.250475in}}%
\pgfpathlineto{\pgfqpoint{1.797505in}{1.252541in}}%
\pgfpathlineto{\pgfqpoint{1.781849in}{1.255958in}}%
\pgfpathlineto{\pgfqpoint{1.777377in}{1.257314in}}%
\pgfpathlineto{\pgfqpoint{1.766192in}{1.260588in}}%
\pgfpathlineto{\pgfqpoint{1.750536in}{1.266404in}}%
\pgfpathlineto{\pgfqpoint{1.740419in}{1.270925in}}%
\pgfpathlineto{\pgfqpoint{1.734879in}{1.273323in}}%
\pgfpathlineto{\pgfqpoint{1.719223in}{1.281192in}}%
\pgfpathlineto{\pgfqpoint{1.713343in}{1.284536in}}%
\pgfpathlineto{\pgfqpoint{1.703566in}{1.289935in}}%
\pgfpathlineto{\pgfqpoint{1.690129in}{1.298148in}}%
\pgfpathlineto{\pgfqpoint{1.687910in}{1.299469in}}%
\pgfpathlineto{\pgfqpoint{1.672253in}{1.309650in}}%
\pgfpathlineto{\pgfqpoint{1.669243in}{1.311759in}}%
\pgfpathlineto{\pgfqpoint{1.656596in}{1.320413in}}%
\pgfpathlineto{\pgfqpoint{1.649793in}{1.325370in}}%
\pgfpathlineto{\pgfqpoint{1.640940in}{1.331691in}}%
\pgfpathlineto{\pgfqpoint{1.631259in}{1.338981in}}%
\pgfpathlineto{\pgfqpoint{1.625283in}{1.343406in}}%
\pgfpathlineto{\pgfqpoint{1.613424in}{1.352592in}}%
\pgfpathlineto{\pgfqpoint{1.609627in}{1.355495in}}%
\pgfpathlineto{\pgfqpoint{1.596137in}{1.366203in}}%
\pgfpathlineto{\pgfqpoint{1.593970in}{1.367907in}}%
\pgfpathlineto{\pgfqpoint{1.579290in}{1.379814in}}%
\pgfpathlineto{\pgfqpoint{1.578314in}{1.380601in}}%
\pgfpathlineto{\pgfqpoint{1.562804in}{1.393425in}}%
\pgfpathlineto{\pgfqpoint{1.562657in}{1.393547in}}%
\pgfpathlineto{\pgfqpoint{1.547000in}{1.406727in}}%
\pgfpathlineto{\pgfqpoint{1.546638in}{1.407036in}}%
\pgfpathlineto{\pgfqpoint{1.531344in}{1.420124in}}%
\pgfpathlineto{\pgfqpoint{1.530739in}{1.420648in}}%
\pgfpathlineto{\pgfqpoint{1.515687in}{1.433733in}}%
\pgfpathlineto{\pgfqpoint{1.515086in}{1.434259in}}%
\pgfpathlineto{\pgfqpoint{1.500031in}{1.447555in}}%
\pgfpathlineto{\pgfqpoint{1.499674in}{1.447870in}}%
\pgfpathlineto{\pgfqpoint{1.484514in}{1.461481in}}%
\pgfpathlineto{\pgfqpoint{1.484374in}{1.461609in}}%
\pgfpathlineto{\pgfqpoint{1.469623in}{1.475092in}}%
\pgfpathlineto{\pgfqpoint{1.468718in}{1.475940in}}%
\pgfpathlineto{\pgfqpoint{1.455021in}{1.488703in}}%
\pgfpathlineto{\pgfqpoint{1.453061in}{1.490587in}}%
\pgfpathlineto{\pgfqpoint{1.440744in}{1.502314in}}%
\pgfpathlineto{\pgfqpoint{1.437404in}{1.505615in}}%
\pgfpathlineto{\pgfqpoint{1.426838in}{1.515925in}}%
\pgfpathlineto{\pgfqpoint{1.421748in}{1.521120in}}%
\pgfpathlineto{\pgfqpoint{1.413363in}{1.529536in}}%
\pgfpathlineto{\pgfqpoint{1.406091in}{1.537233in}}%
\pgfpathlineto{\pgfqpoint{1.400390in}{1.543148in}}%
\pgfpathlineto{\pgfqpoint{1.390435in}{1.554142in}}%
\pgfpathlineto{\pgfqpoint{1.388009in}{1.556759in}}%
\pgfpathlineto{\pgfqpoint{1.376299in}{1.570370in}}%
\pgfpathlineto{\pgfqpoint{1.374778in}{1.572300in}}%
\pgfpathlineto{\pgfqpoint{1.365331in}{1.583981in}}%
\pgfpathlineto{\pgfqpoint{1.359122in}{1.592480in}}%
\pgfpathlineto{\pgfqpoint{1.355275in}{1.597592in}}%
\pgfpathlineto{\pgfqpoint{1.346223in}{1.611203in}}%
\pgfpathlineto{\pgfqpoint{1.343465in}{1.616020in}}%
\pgfpathlineto{\pgfqpoint{1.338264in}{1.624814in}}%
\pgfpathlineto{\pgfqpoint{1.331575in}{1.638425in}}%
\pgfpathlineto{\pgfqpoint{1.327809in}{1.648149in}}%
\pgfpathlineto{\pgfqpoint{1.326249in}{1.652036in}}%
\pgfpathlineto{\pgfqpoint{1.322318in}{1.665648in}}%
\pgfpathlineto{\pgfqpoint{1.319942in}{1.679259in}}%
\pgfpathlineto{\pgfqpoint{1.319147in}{1.692870in}}%
\pgfpathlineto{\pgfqpoint{1.312152in}{1.692870in}}%
\pgfpathlineto{\pgfqpoint{1.296495in}{1.692870in}}%
\pgfpathlineto{\pgfqpoint{1.280839in}{1.692870in}}%
\pgfpathlineto{\pgfqpoint{1.265182in}{1.692870in}}%
\pgfpathlineto{\pgfqpoint{1.249526in}{1.692870in}}%
\pgfpathlineto{\pgfqpoint{1.233869in}{1.692870in}}%
\pgfpathlineto{\pgfqpoint{1.218213in}{1.692870in}}%
\pgfpathlineto{\pgfqpoint{1.202556in}{1.692870in}}%
\pgfpathlineto{\pgfqpoint{1.186899in}{1.692870in}}%
\pgfpathlineto{\pgfqpoint{1.171243in}{1.692870in}}%
\pgfpathlineto{\pgfqpoint{1.155586in}{1.692870in}}%
\pgfpathlineto{\pgfqpoint{1.139930in}{1.692870in}}%
\pgfpathlineto{\pgfqpoint{1.124273in}{1.692870in}}%
\pgfpathlineto{\pgfqpoint{1.108617in}{1.692870in}}%
\pgfpathlineto{\pgfqpoint{1.092960in}{1.692870in}}%
\pgfpathlineto{\pgfqpoint{1.077303in}{1.692870in}}%
\pgfpathlineto{\pgfqpoint{1.061647in}{1.692870in}}%
\pgfpathlineto{\pgfqpoint{1.045990in}{1.692870in}}%
\pgfpathlineto{\pgfqpoint{1.030334in}{1.692870in}}%
\pgfpathlineto{\pgfqpoint{1.014677in}{1.692870in}}%
\pgfpathlineto{\pgfqpoint{0.999021in}{1.692870in}}%
\pgfpathlineto{\pgfqpoint{0.983364in}{1.692870in}}%
\pgfpathlineto{\pgfqpoint{0.967708in}{1.692870in}}%
\pgfpathlineto{\pgfqpoint{0.952051in}{1.692870in}}%
\pgfpathlineto{\pgfqpoint{0.936394in}{1.692870in}}%
\pgfpathlineto{\pgfqpoint{0.920738in}{1.692870in}}%
\pgfpathlineto{\pgfqpoint{0.905081in}{1.692870in}}%
\pgfpathlineto{\pgfqpoint{0.889425in}{1.692870in}}%
\pgfpathlineto{\pgfqpoint{0.873768in}{1.692870in}}%
\pgfpathlineto{\pgfqpoint{0.858112in}{1.692870in}}%
\pgfpathlineto{\pgfqpoint{0.842455in}{1.692870in}}%
\pgfpathlineto{\pgfqpoint{0.826798in}{1.692870in}}%
\pgfpathlineto{\pgfqpoint{0.811142in}{1.692870in}}%
\pgfpathlineto{\pgfqpoint{0.795485in}{1.692870in}}%
\pgfpathlineto{\pgfqpoint{0.788490in}{1.692870in}}%
\pgfpathlineto{\pgfqpoint{0.787696in}{1.679259in}}%
\pgfpathlineto{\pgfqpoint{0.785320in}{1.665648in}}%
\pgfpathlineto{\pgfqpoint{0.781388in}{1.652036in}}%
\pgfpathlineto{\pgfqpoint{0.779829in}{1.648149in}}%
\pgfpathlineto{\pgfqpoint{0.776063in}{1.638425in}}%
\pgfpathlineto{\pgfqpoint{0.769373in}{1.624814in}}%
\pgfpathlineto{\pgfqpoint{0.764172in}{1.616020in}}%
\pgfpathlineto{\pgfqpoint{0.761415in}{1.611203in}}%
\pgfpathlineto{\pgfqpoint{0.752362in}{1.597592in}}%
\pgfpathlineto{\pgfqpoint{0.748516in}{1.592480in}}%
\pgfpathlineto{\pgfqpoint{0.742306in}{1.583981in}}%
\pgfpathlineto{\pgfqpoint{0.732859in}{1.572300in}}%
\pgfpathlineto{\pgfqpoint{0.731339in}{1.570370in}}%
\pgfpathlineto{\pgfqpoint{0.719628in}{1.556759in}}%
\pgfpathlineto{\pgfqpoint{0.717202in}{1.554142in}}%
\pgfpathlineto{\pgfqpoint{0.707247in}{1.543148in}}%
\pgfpathlineto{\pgfqpoint{0.701546in}{1.537233in}}%
\pgfpathlineto{\pgfqpoint{0.694275in}{1.529536in}}%
\pgfpathlineto{\pgfqpoint{0.685889in}{1.521120in}}%
\pgfpathlineto{\pgfqpoint{0.680799in}{1.515925in}}%
\pgfpathlineto{\pgfqpoint{0.670233in}{1.505615in}}%
\pgfpathlineto{\pgfqpoint{0.666893in}{1.502314in}}%
\pgfpathlineto{\pgfqpoint{0.654576in}{1.490587in}}%
\pgfpathlineto{\pgfqpoint{0.652616in}{1.488703in}}%
\pgfpathlineto{\pgfqpoint{0.638920in}{1.475940in}}%
\pgfpathlineto{\pgfqpoint{0.638014in}{1.475092in}}%
\pgfpathlineto{\pgfqpoint{0.623263in}{1.461609in}}%
\pgfpathlineto{\pgfqpoint{0.623124in}{1.461481in}}%
\pgfpathlineto{\pgfqpoint{0.607963in}{1.447870in}}%
\pgfpathlineto{\pgfqpoint{0.607606in}{1.447555in}}%
\pgfpathlineto{\pgfqpoint{0.592552in}{1.434259in}}%
\pgfpathlineto{\pgfqpoint{0.591950in}{1.433733in}}%
\pgfpathlineto{\pgfqpoint{0.576899in}{1.420648in}}%
\pgfpathlineto{\pgfqpoint{0.576293in}{1.420124in}}%
\pgfpathlineto{\pgfqpoint{0.560999in}{1.407036in}}%
\pgfpathlineto{\pgfqpoint{0.560637in}{1.406727in}}%
\pgfpathlineto{\pgfqpoint{0.544980in}{1.393547in}}%
\pgfpathlineto{\pgfqpoint{0.544833in}{1.393425in}}%
\pgfpathlineto{\pgfqpoint{0.529324in}{1.380601in}}%
\pgfpathlineto{\pgfqpoint{0.528348in}{1.379814in}}%
\pgfpathlineto{\pgfqpoint{0.513667in}{1.367907in}}%
\pgfpathlineto{\pgfqpoint{0.511500in}{1.366203in}}%
\pgfpathlineto{\pgfqpoint{0.498011in}{1.355495in}}%
\pgfpathlineto{\pgfqpoint{0.494213in}{1.352592in}}%
\pgfpathlineto{\pgfqpoint{0.482354in}{1.343406in}}%
\pgfpathlineto{\pgfqpoint{0.476379in}{1.338981in}}%
\pgfpathlineto{\pgfqpoint{0.466697in}{1.331691in}}%
\pgfpathlineto{\pgfqpoint{0.457844in}{1.325370in}}%
\pgfpathlineto{\pgfqpoint{0.451041in}{1.320413in}}%
\pgfpathlineto{\pgfqpoint{0.438394in}{1.311759in}}%
\pgfpathlineto{\pgfqpoint{0.435384in}{1.309650in}}%
\pgfpathlineto{\pgfqpoint{0.419728in}{1.299469in}}%
\pgfpathlineto{\pgfqpoint{0.417508in}{1.298148in}}%
\pgfpathlineto{\pgfqpoint{0.404071in}{1.289935in}}%
\pgfpathlineto{\pgfqpoint{0.394294in}{1.284536in}}%
\pgfpathlineto{\pgfqpoint{0.388415in}{1.281192in}}%
\pgfpathlineto{\pgfqpoint{0.372758in}{1.273323in}}%
\pgfpathlineto{\pgfqpoint{0.367218in}{1.270925in}}%
\pgfpathlineto{\pgfqpoint{0.357101in}{1.266404in}}%
\pgfpathlineto{\pgfqpoint{0.341445in}{1.260588in}}%
\pgfpathlineto{\pgfqpoint{0.330260in}{1.257314in}}%
\pgfpathlineto{\pgfqpoint{0.325788in}{1.255958in}}%
\pgfpathlineto{\pgfqpoint{0.310132in}{1.252541in}}%
\pgfpathlineto{\pgfqpoint{0.294475in}{1.250475in}}%
\pgfpathlineto{\pgfqpoint{0.278819in}{1.249784in}}%
\pgfpathlineto{\pgfqpoint{0.278819in}{1.243703in}}%
\pgfpathlineto{\pgfqpoint{0.278819in}{1.230092in}}%
\pgfpathlineto{\pgfqpoint{0.278819in}{1.216481in}}%
\pgfpathlineto{\pgfqpoint{0.278819in}{1.202870in}}%
\pgfpathlineto{\pgfqpoint{0.278819in}{1.189259in}}%
\pgfpathlineto{\pgfqpoint{0.278819in}{1.175647in}}%
\pgfpathlineto{\pgfqpoint{0.278819in}{1.162036in}}%
\pgfpathlineto{\pgfqpoint{0.278819in}{1.148425in}}%
\pgfpathlineto{\pgfqpoint{0.278819in}{1.134814in}}%
\pgfpathlineto{\pgfqpoint{0.278819in}{1.121203in}}%
\pgfpathlineto{\pgfqpoint{0.278819in}{1.107592in}}%
\pgfpathlineto{\pgfqpoint{0.278819in}{1.093981in}}%
\pgfpathlineto{\pgfqpoint{0.278819in}{1.080370in}}%
\pgfpathlineto{\pgfqpoint{0.278819in}{1.066759in}}%
\pgfpathlineto{\pgfqpoint{0.278819in}{1.053148in}}%
\pgfpathlineto{\pgfqpoint{0.278819in}{1.039536in}}%
\pgfpathlineto{\pgfqpoint{0.278819in}{1.025925in}}%
\pgfpathlineto{\pgfqpoint{0.278819in}{1.012314in}}%
\pgfpathlineto{\pgfqpoint{0.278819in}{0.998703in}}%
\pgfpathlineto{\pgfqpoint{0.278819in}{0.985092in}}%
\pgfpathlineto{\pgfqpoint{0.278819in}{0.971481in}}%
\pgfpathlineto{\pgfqpoint{0.278819in}{0.957870in}}%
\pgfpathlineto{\pgfqpoint{0.278819in}{0.944259in}}%
\pgfpathlineto{\pgfqpoint{0.278819in}{0.930648in}}%
\pgfpathlineto{\pgfqpoint{0.278819in}{0.917036in}}%
\pgfpathlineto{\pgfqpoint{0.278819in}{0.903425in}}%
\pgfpathlineto{\pgfqpoint{0.278819in}{0.889814in}}%
\pgfpathlineto{\pgfqpoint{0.278819in}{0.876203in}}%
\pgfpathlineto{\pgfqpoint{0.278819in}{0.862592in}}%
\pgfpathlineto{\pgfqpoint{0.278819in}{0.848981in}}%
\pgfpathlineto{\pgfqpoint{0.278819in}{0.835370in}}%
\pgfpathlineto{\pgfqpoint{0.278819in}{0.821759in}}%
\pgfpathlineto{\pgfqpoint{0.278819in}{0.808148in}}%
\pgfpathlineto{\pgfqpoint{0.278819in}{0.794536in}}%
\pgfpathlineto{\pgfqpoint{0.278819in}{0.788455in}}%
\pgfpathlineto{\pgfqpoint{0.294475in}{0.787764in}}%
\pgfpathlineto{\pgfqpoint{0.310132in}{0.785699in}}%
\pgfpathlineto{\pgfqpoint{0.325788in}{0.782281in}}%
\pgfpathlineto{\pgfqpoint{0.330260in}{0.780925in}}%
\pgfpathlineto{\pgfqpoint{0.341445in}{0.777651in}}%
\pgfpathlineto{\pgfqpoint{0.357101in}{0.771836in}}%
\pgfpathlineto{\pgfqpoint{0.367218in}{0.767314in}}%
\pgfpathlineto{\pgfqpoint{0.372758in}{0.764917in}}%
\pgfpathlineto{\pgfqpoint{0.388415in}{0.757047in}}%
\pgfpathlineto{\pgfqpoint{0.394294in}{0.753703in}}%
\pgfpathlineto{\pgfqpoint{0.404071in}{0.748305in}}%
\pgfpathlineto{\pgfqpoint{0.417508in}{0.740092in}}%
\pgfpathlineto{\pgfqpoint{0.419728in}{0.738770in}}%
\pgfpathlineto{\pgfqpoint{0.435384in}{0.728590in}}%
\pgfpathlineto{\pgfqpoint{0.438394in}{0.726481in}}%
\pgfpathlineto{\pgfqpoint{0.451041in}{0.717826in}}%
\pgfpathlineto{\pgfqpoint{0.457844in}{0.712870in}}%
\pgfpathlineto{\pgfqpoint{0.466697in}{0.706548in}}%
\pgfpathlineto{\pgfqpoint{0.476379in}{0.699259in}}%
\pgfpathlineto{\pgfqpoint{0.482354in}{0.694834in}}%
\pgfpathlineto{\pgfqpoint{0.494213in}{0.685648in}}%
\pgfpathlineto{\pgfqpoint{0.498011in}{0.682744in}}%
\pgfpathlineto{\pgfqpoint{0.511500in}{0.672036in}}%
\pgfpathlineto{\pgfqpoint{0.513667in}{0.670333in}}%
\pgfpathlineto{\pgfqpoint{0.528348in}{0.658425in}}%
\pgfpathlineto{\pgfqpoint{0.529324in}{0.657638in}}%
\pgfpathlineto{\pgfqpoint{0.544833in}{0.644814in}}%
\pgfpathlineto{\pgfqpoint{0.544980in}{0.644693in}}%
\pgfpathlineto{\pgfqpoint{0.560637in}{0.631513in}}%
\pgfpathlineto{\pgfqpoint{0.560999in}{0.631203in}}%
\pgfpathlineto{\pgfqpoint{0.576293in}{0.618115in}}%
\pgfpathlineto{\pgfqpoint{0.576899in}{0.617592in}}%
\pgfpathlineto{\pgfqpoint{0.591950in}{0.604507in}}%
\pgfpathlineto{\pgfqpoint{0.592552in}{0.603981in}}%
\pgfpathlineto{\pgfqpoint{0.607606in}{0.590685in}}%
\pgfpathlineto{\pgfqpoint{0.607963in}{0.590370in}}%
\pgfpathlineto{\pgfqpoint{0.623124in}{0.576759in}}%
\pgfpathlineto{\pgfqpoint{0.623263in}{0.576631in}}%
\pgfpathlineto{\pgfqpoint{0.638014in}{0.563148in}}%
\pgfpathlineto{\pgfqpoint{0.638920in}{0.562299in}}%
\pgfpathlineto{\pgfqpoint{0.652616in}{0.549536in}}%
\pgfpathlineto{\pgfqpoint{0.654576in}{0.547652in}}%
\pgfpathlineto{\pgfqpoint{0.666893in}{0.535925in}}%
\pgfpathlineto{\pgfqpoint{0.670233in}{0.532624in}}%
\pgfpathlineto{\pgfqpoint{0.680799in}{0.522314in}}%
\pgfpathlineto{\pgfqpoint{0.685889in}{0.517120in}}%
\pgfpathlineto{\pgfqpoint{0.694275in}{0.508703in}}%
\pgfpathlineto{\pgfqpoint{0.701546in}{0.501007in}}%
\pgfpathlineto{\pgfqpoint{0.707247in}{0.495092in}}%
\pgfpathlineto{\pgfqpoint{0.717202in}{0.484098in}}%
\pgfpathlineto{\pgfqpoint{0.719628in}{0.481481in}}%
\pgfpathlineto{\pgfqpoint{0.731339in}{0.467870in}}%
\pgfpathlineto{\pgfqpoint{0.732859in}{0.465940in}}%
\pgfpathlineto{\pgfqpoint{0.742306in}{0.454259in}}%
\pgfpathlineto{\pgfqpoint{0.748516in}{0.445759in}}%
\pgfpathlineto{\pgfqpoint{0.752362in}{0.440648in}}%
\pgfpathlineto{\pgfqpoint{0.761415in}{0.427036in}}%
\pgfpathlineto{\pgfqpoint{0.764172in}{0.422220in}}%
\pgfpathlineto{\pgfqpoint{0.769373in}{0.413425in}}%
\pgfpathlineto{\pgfqpoint{0.776063in}{0.399814in}}%
\pgfpathlineto{\pgfqpoint{0.779829in}{0.390091in}}%
\pgfpathlineto{\pgfqpoint{0.781388in}{0.386203in}}%
\pgfpathlineto{\pgfqpoint{0.785320in}{0.372592in}}%
\pgfpathlineto{\pgfqpoint{0.787696in}{0.358981in}}%
\pgfpathlineto{\pgfqpoint{0.788490in}{0.345370in}}%
\pgfpathlineto{\pgfqpoint{0.795485in}{0.345370in}}%
\pgfpathclose%
\pgfpathmoveto{\pgfqpoint{1.005967in}{0.427036in}}%
\pgfpathlineto{\pgfqpoint{0.999021in}{0.428912in}}%
\pgfpathlineto{\pgfqpoint{0.983364in}{0.434545in}}%
\pgfpathlineto{\pgfqpoint{0.969793in}{0.440648in}}%
\pgfpathlineto{\pgfqpoint{0.967708in}{0.441471in}}%
\pgfpathlineto{\pgfqpoint{0.952051in}{0.448869in}}%
\pgfpathlineto{\pgfqpoint{0.942260in}{0.454259in}}%
\pgfpathlineto{\pgfqpoint{0.936394in}{0.457153in}}%
\pgfpathlineto{\pgfqpoint{0.920738in}{0.465960in}}%
\pgfpathlineto{\pgfqpoint{0.917705in}{0.467870in}}%
\pgfpathlineto{\pgfqpoint{0.905081in}{0.475117in}}%
\pgfpathlineto{\pgfqpoint{0.895070in}{0.481481in}}%
\pgfpathlineto{\pgfqpoint{0.889425in}{0.484795in}}%
\pgfpathlineto{\pgfqpoint{0.873768in}{0.494863in}}%
\pgfpathlineto{\pgfqpoint{0.873440in}{0.495092in}}%
\pgfpathlineto{\pgfqpoint{0.858112in}{0.505061in}}%
\pgfpathlineto{\pgfqpoint{0.852908in}{0.508703in}}%
\pgfpathlineto{\pgfqpoint{0.842455in}{0.515609in}}%
\pgfpathlineto{\pgfqpoint{0.832960in}{0.522314in}}%
\pgfpathlineto{\pgfqpoint{0.826798in}{0.526457in}}%
\pgfpathlineto{\pgfqpoint{0.813542in}{0.535925in}}%
\pgfpathlineto{\pgfqpoint{0.811142in}{0.537569in}}%
\pgfpathlineto{\pgfqpoint{0.795485in}{0.548895in}}%
\pgfpathlineto{\pgfqpoint{0.794640in}{0.549536in}}%
\pgfpathlineto{\pgfqpoint{0.779829in}{0.560397in}}%
\pgfpathlineto{\pgfqpoint{0.776241in}{0.563148in}}%
\pgfpathlineto{\pgfqpoint{0.764172in}{0.572146in}}%
\pgfpathlineto{\pgfqpoint{0.758228in}{0.576759in}}%
\pgfpathlineto{\pgfqpoint{0.748516in}{0.584132in}}%
\pgfpathlineto{\pgfqpoint{0.740586in}{0.590370in}}%
\pgfpathlineto{\pgfqpoint{0.732859in}{0.596349in}}%
\pgfpathlineto{\pgfqpoint{0.723301in}{0.603981in}}%
\pgfpathlineto{\pgfqpoint{0.717202in}{0.608794in}}%
\pgfpathlineto{\pgfqpoint{0.706358in}{0.617592in}}%
\pgfpathlineto{\pgfqpoint{0.701546in}{0.621470in}}%
\pgfpathlineto{\pgfqpoint{0.689746in}{0.631203in}}%
\pgfpathlineto{\pgfqpoint{0.685889in}{0.634378in}}%
\pgfpathlineto{\pgfqpoint{0.673453in}{0.644814in}}%
\pgfpathlineto{\pgfqpoint{0.670233in}{0.647523in}}%
\pgfpathlineto{\pgfqpoint{0.657467in}{0.658425in}}%
\pgfpathlineto{\pgfqpoint{0.654576in}{0.660911in}}%
\pgfpathlineto{\pgfqpoint{0.641779in}{0.672036in}}%
\pgfpathlineto{\pgfqpoint{0.638920in}{0.674550in}}%
\pgfpathlineto{\pgfqpoint{0.626379in}{0.685648in}}%
\pgfpathlineto{\pgfqpoint{0.623263in}{0.688447in}}%
\pgfpathlineto{\pgfqpoint{0.611258in}{0.699259in}}%
\pgfpathlineto{\pgfqpoint{0.607606in}{0.702611in}}%
\pgfpathlineto{\pgfqpoint{0.596411in}{0.712870in}}%
\pgfpathlineto{\pgfqpoint{0.591950in}{0.717053in}}%
\pgfpathlineto{\pgfqpoint{0.581830in}{0.726481in}}%
\pgfpathlineto{\pgfqpoint{0.576293in}{0.731782in}}%
\pgfpathlineto{\pgfqpoint{0.567514in}{0.740092in}}%
\pgfpathlineto{\pgfqpoint{0.560637in}{0.746810in}}%
\pgfpathlineto{\pgfqpoint{0.553461in}{0.753703in}}%
\pgfpathlineto{\pgfqpoint{0.544980in}{0.762147in}}%
\pgfpathlineto{\pgfqpoint{0.539674in}{0.767314in}}%
\pgfpathlineto{\pgfqpoint{0.529324in}{0.777806in}}%
\pgfpathlineto{\pgfqpoint{0.526160in}{0.780925in}}%
\pgfpathlineto{\pgfqpoint{0.513667in}{0.793801in}}%
\pgfpathlineto{\pgfqpoint{0.512929in}{0.794536in}}%
\pgfpathlineto{\pgfqpoint{0.499902in}{0.808148in}}%
\pgfpathlineto{\pgfqpoint{0.498011in}{0.810234in}}%
\pgfpathlineto{\pgfqpoint{0.487119in}{0.821759in}}%
\pgfpathlineto{\pgfqpoint{0.482354in}{0.827116in}}%
\pgfpathlineto{\pgfqpoint{0.474641in}{0.835370in}}%
\pgfpathlineto{\pgfqpoint{0.466697in}{0.844457in}}%
\pgfpathlineto{\pgfqpoint{0.462508in}{0.848981in}}%
\pgfpathlineto{\pgfqpoint{0.451041in}{0.862306in}}%
\pgfpathlineto{\pgfqpoint{0.450778in}{0.862592in}}%
\pgfpathlineto{\pgfqpoint{0.439197in}{0.876203in}}%
\pgfpathlineto{\pgfqpoint{0.435384in}{0.881110in}}%
\pgfpathlineto{\pgfqpoint{0.428064in}{0.889814in}}%
\pgfpathlineto{\pgfqpoint{0.419728in}{0.900789in}}%
\pgfpathlineto{\pgfqpoint{0.417531in}{0.903425in}}%
\pgfpathlineto{\pgfqpoint{0.407401in}{0.917036in}}%
\pgfpathlineto{\pgfqpoint{0.404071in}{0.922136in}}%
\pgfpathlineto{\pgfqpoint{0.397871in}{0.930648in}}%
\pgfpathlineto{\pgfqpoint{0.389362in}{0.944259in}}%
\pgfpathlineto{\pgfqpoint{0.388415in}{0.946071in}}%
\pgfpathlineto{\pgfqpoint{0.381395in}{0.957870in}}%
\pgfpathlineto{\pgfqpoint{0.374916in}{0.971481in}}%
\pgfpathlineto{\pgfqpoint{0.372758in}{0.977519in}}%
\pgfpathlineto{\pgfqpoint{0.369593in}{0.985092in}}%
\pgfpathlineto{\pgfqpoint{0.365797in}{0.998703in}}%
\pgfpathlineto{\pgfqpoint{0.363899in}{1.012314in}}%
\pgfpathlineto{\pgfqpoint{0.363899in}{1.025925in}}%
\pgfpathlineto{\pgfqpoint{0.365797in}{1.039536in}}%
\pgfpathlineto{\pgfqpoint{0.369593in}{1.053148in}}%
\pgfpathlineto{\pgfqpoint{0.372758in}{1.060720in}}%
\pgfpathlineto{\pgfqpoint{0.374916in}{1.066759in}}%
\pgfpathlineto{\pgfqpoint{0.381395in}{1.080370in}}%
\pgfpathlineto{\pgfqpoint{0.388415in}{1.092168in}}%
\pgfpathlineto{\pgfqpoint{0.389362in}{1.093981in}}%
\pgfpathlineto{\pgfqpoint{0.397871in}{1.107592in}}%
\pgfpathlineto{\pgfqpoint{0.404071in}{1.116103in}}%
\pgfpathlineto{\pgfqpoint{0.407401in}{1.121203in}}%
\pgfpathlineto{\pgfqpoint{0.417531in}{1.134814in}}%
\pgfpathlineto{\pgfqpoint{0.419728in}{1.137451in}}%
\pgfpathlineto{\pgfqpoint{0.428064in}{1.148425in}}%
\pgfpathlineto{\pgfqpoint{0.435384in}{1.157129in}}%
\pgfpathlineto{\pgfqpoint{0.439197in}{1.162036in}}%
\pgfpathlineto{\pgfqpoint{0.450778in}{1.175647in}}%
\pgfpathlineto{\pgfqpoint{0.451041in}{1.175933in}}%
\pgfpathlineto{\pgfqpoint{0.462508in}{1.189259in}}%
\pgfpathlineto{\pgfqpoint{0.466697in}{1.193783in}}%
\pgfpathlineto{\pgfqpoint{0.474641in}{1.202870in}}%
\pgfpathlineto{\pgfqpoint{0.482354in}{1.211124in}}%
\pgfpathlineto{\pgfqpoint{0.487119in}{1.216481in}}%
\pgfpathlineto{\pgfqpoint{0.498011in}{1.228006in}}%
\pgfpathlineto{\pgfqpoint{0.499902in}{1.230092in}}%
\pgfpathlineto{\pgfqpoint{0.512929in}{1.243703in}}%
\pgfpathlineto{\pgfqpoint{0.513667in}{1.244438in}}%
\pgfpathlineto{\pgfqpoint{0.526160in}{1.257314in}}%
\pgfpathlineto{\pgfqpoint{0.529324in}{1.260433in}}%
\pgfpathlineto{\pgfqpoint{0.539674in}{1.270925in}}%
\pgfpathlineto{\pgfqpoint{0.544980in}{1.276093in}}%
\pgfpathlineto{\pgfqpoint{0.553461in}{1.284536in}}%
\pgfpathlineto{\pgfqpoint{0.560637in}{1.291430in}}%
\pgfpathlineto{\pgfqpoint{0.567514in}{1.298148in}}%
\pgfpathlineto{\pgfqpoint{0.576293in}{1.306457in}}%
\pgfpathlineto{\pgfqpoint{0.581830in}{1.311759in}}%
\pgfpathlineto{\pgfqpoint{0.591950in}{1.321186in}}%
\pgfpathlineto{\pgfqpoint{0.596411in}{1.325370in}}%
\pgfpathlineto{\pgfqpoint{0.607606in}{1.335628in}}%
\pgfpathlineto{\pgfqpoint{0.611258in}{1.338981in}}%
\pgfpathlineto{\pgfqpoint{0.623263in}{1.349793in}}%
\pgfpathlineto{\pgfqpoint{0.626379in}{1.352592in}}%
\pgfpathlineto{\pgfqpoint{0.638920in}{1.363690in}}%
\pgfpathlineto{\pgfqpoint{0.641779in}{1.366203in}}%
\pgfpathlineto{\pgfqpoint{0.654576in}{1.377328in}}%
\pgfpathlineto{\pgfqpoint{0.657467in}{1.379814in}}%
\pgfpathlineto{\pgfqpoint{0.670233in}{1.390717in}}%
\pgfpathlineto{\pgfqpoint{0.673453in}{1.393425in}}%
\pgfpathlineto{\pgfqpoint{0.685889in}{1.403862in}}%
\pgfpathlineto{\pgfqpoint{0.689746in}{1.407036in}}%
\pgfpathlineto{\pgfqpoint{0.701546in}{1.416770in}}%
\pgfpathlineto{\pgfqpoint{0.706358in}{1.420648in}}%
\pgfpathlineto{\pgfqpoint{0.717202in}{1.429445in}}%
\pgfpathlineto{\pgfqpoint{0.723301in}{1.434259in}}%
\pgfpathlineto{\pgfqpoint{0.732859in}{1.441891in}}%
\pgfpathlineto{\pgfqpoint{0.740586in}{1.447870in}}%
\pgfpathlineto{\pgfqpoint{0.748516in}{1.454108in}}%
\pgfpathlineto{\pgfqpoint{0.758228in}{1.461481in}}%
\pgfpathlineto{\pgfqpoint{0.764172in}{1.466094in}}%
\pgfpathlineto{\pgfqpoint{0.776241in}{1.475092in}}%
\pgfpathlineto{\pgfqpoint{0.779829in}{1.477843in}}%
\pgfpathlineto{\pgfqpoint{0.794640in}{1.488703in}}%
\pgfpathlineto{\pgfqpoint{0.795485in}{1.489345in}}%
\pgfpathlineto{\pgfqpoint{0.811142in}{1.500670in}}%
\pgfpathlineto{\pgfqpoint{0.813542in}{1.502314in}}%
\pgfpathlineto{\pgfqpoint{0.826798in}{1.511783in}}%
\pgfpathlineto{\pgfqpoint{0.832960in}{1.515925in}}%
\pgfpathlineto{\pgfqpoint{0.842455in}{1.522631in}}%
\pgfpathlineto{\pgfqpoint{0.852908in}{1.529536in}}%
\pgfpathlineto{\pgfqpoint{0.858112in}{1.533179in}}%
\pgfpathlineto{\pgfqpoint{0.873440in}{1.543148in}}%
\pgfpathlineto{\pgfqpoint{0.873768in}{1.543376in}}%
\pgfpathlineto{\pgfqpoint{0.889425in}{1.553444in}}%
\pgfpathlineto{\pgfqpoint{0.895070in}{1.556759in}}%
\pgfpathlineto{\pgfqpoint{0.905081in}{1.563123in}}%
\pgfpathlineto{\pgfqpoint{0.917705in}{1.570370in}}%
\pgfpathlineto{\pgfqpoint{0.920738in}{1.572280in}}%
\pgfpathlineto{\pgfqpoint{0.936394in}{1.581086in}}%
\pgfpathlineto{\pgfqpoint{0.942260in}{1.583981in}}%
\pgfpathlineto{\pgfqpoint{0.952051in}{1.589371in}}%
\pgfpathlineto{\pgfqpoint{0.967708in}{1.596769in}}%
\pgfpathlineto{\pgfqpoint{0.969793in}{1.597592in}}%
\pgfpathlineto{\pgfqpoint{0.983364in}{1.603695in}}%
\pgfpathlineto{\pgfqpoint{0.999021in}{1.609327in}}%
\pgfpathlineto{\pgfqpoint{1.005967in}{1.611203in}}%
\pgfpathlineto{\pgfqpoint{1.014677in}{1.613955in}}%
\pgfpathlineto{\pgfqpoint{1.030334in}{1.617255in}}%
\pgfpathlineto{\pgfqpoint{1.045990in}{1.618904in}}%
\pgfpathlineto{\pgfqpoint{1.061647in}{1.618904in}}%
\pgfpathlineto{\pgfqpoint{1.077303in}{1.617255in}}%
\pgfpathlineto{\pgfqpoint{1.092960in}{1.613955in}}%
\pgfpathlineto{\pgfqpoint{1.101671in}{1.611203in}}%
\pgfpathlineto{\pgfqpoint{1.108617in}{1.609327in}}%
\pgfpathlineto{\pgfqpoint{1.124273in}{1.603695in}}%
\pgfpathlineto{\pgfqpoint{1.137845in}{1.597592in}}%
\pgfpathlineto{\pgfqpoint{1.139930in}{1.596769in}}%
\pgfpathlineto{\pgfqpoint{1.155586in}{1.589371in}}%
\pgfpathlineto{\pgfqpoint{1.165377in}{1.583981in}}%
\pgfpathlineto{\pgfqpoint{1.171243in}{1.581086in}}%
\pgfpathlineto{\pgfqpoint{1.186899in}{1.572280in}}%
\pgfpathlineto{\pgfqpoint{1.189932in}{1.570370in}}%
\pgfpathlineto{\pgfqpoint{1.202556in}{1.563123in}}%
\pgfpathlineto{\pgfqpoint{1.212568in}{1.556759in}}%
\pgfpathlineto{\pgfqpoint{1.218213in}{1.553444in}}%
\pgfpathlineto{\pgfqpoint{1.233869in}{1.543376in}}%
\pgfpathlineto{\pgfqpoint{1.234198in}{1.543148in}}%
\pgfpathlineto{\pgfqpoint{1.249526in}{1.533179in}}%
\pgfpathlineto{\pgfqpoint{1.254729in}{1.529536in}}%
\pgfpathlineto{\pgfqpoint{1.265182in}{1.522631in}}%
\pgfpathlineto{\pgfqpoint{1.274677in}{1.515925in}}%
\pgfpathlineto{\pgfqpoint{1.280839in}{1.511783in}}%
\pgfpathlineto{\pgfqpoint{1.294096in}{1.502314in}}%
\pgfpathlineto{\pgfqpoint{1.296495in}{1.500670in}}%
\pgfpathlineto{\pgfqpoint{1.312152in}{1.489345in}}%
\pgfpathlineto{\pgfqpoint{1.312997in}{1.488703in}}%
\pgfpathlineto{\pgfqpoint{1.327809in}{1.477843in}}%
\pgfpathlineto{\pgfqpoint{1.331396in}{1.475092in}}%
\pgfpathlineto{\pgfqpoint{1.343465in}{1.466094in}}%
\pgfpathlineto{\pgfqpoint{1.349409in}{1.461481in}}%
\pgfpathlineto{\pgfqpoint{1.359122in}{1.454108in}}%
\pgfpathlineto{\pgfqpoint{1.367051in}{1.447870in}}%
\pgfpathlineto{\pgfqpoint{1.374778in}{1.441891in}}%
\pgfpathlineto{\pgfqpoint{1.384337in}{1.434259in}}%
\pgfpathlineto{\pgfqpoint{1.390435in}{1.429445in}}%
\pgfpathlineto{\pgfqpoint{1.401279in}{1.420648in}}%
\pgfpathlineto{\pgfqpoint{1.406091in}{1.416770in}}%
\pgfpathlineto{\pgfqpoint{1.417891in}{1.407036in}}%
\pgfpathlineto{\pgfqpoint{1.421748in}{1.403862in}}%
\pgfpathlineto{\pgfqpoint{1.434185in}{1.393425in}}%
\pgfpathlineto{\pgfqpoint{1.437404in}{1.390717in}}%
\pgfpathlineto{\pgfqpoint{1.450170in}{1.379814in}}%
\pgfpathlineto{\pgfqpoint{1.453061in}{1.377328in}}%
\pgfpathlineto{\pgfqpoint{1.465858in}{1.366203in}}%
\pgfpathlineto{\pgfqpoint{1.468718in}{1.363690in}}%
\pgfpathlineto{\pgfqpoint{1.481258in}{1.352592in}}%
\pgfpathlineto{\pgfqpoint{1.484374in}{1.349793in}}%
\pgfpathlineto{\pgfqpoint{1.496379in}{1.338981in}}%
\pgfpathlineto{\pgfqpoint{1.500031in}{1.335628in}}%
\pgfpathlineto{\pgfqpoint{1.511227in}{1.325370in}}%
\pgfpathlineto{\pgfqpoint{1.515687in}{1.321186in}}%
\pgfpathlineto{\pgfqpoint{1.525807in}{1.311759in}}%
\pgfpathlineto{\pgfqpoint{1.531344in}{1.306457in}}%
\pgfpathlineto{\pgfqpoint{1.540123in}{1.298148in}}%
\pgfpathlineto{\pgfqpoint{1.547000in}{1.291430in}}%
\pgfpathlineto{\pgfqpoint{1.554176in}{1.284536in}}%
\pgfpathlineto{\pgfqpoint{1.562657in}{1.276093in}}%
\pgfpathlineto{\pgfqpoint{1.567963in}{1.270925in}}%
\pgfpathlineto{\pgfqpoint{1.578314in}{1.260433in}}%
\pgfpathlineto{\pgfqpoint{1.581478in}{1.257314in}}%
\pgfpathlineto{\pgfqpoint{1.593970in}{1.244438in}}%
\pgfpathlineto{\pgfqpoint{1.594708in}{1.243703in}}%
\pgfpathlineto{\pgfqpoint{1.607735in}{1.230092in}}%
\pgfpathlineto{\pgfqpoint{1.609627in}{1.228006in}}%
\pgfpathlineto{\pgfqpoint{1.620518in}{1.216481in}}%
\pgfpathlineto{\pgfqpoint{1.625283in}{1.211124in}}%
\pgfpathlineto{\pgfqpoint{1.632996in}{1.202870in}}%
\pgfpathlineto{\pgfqpoint{1.640940in}{1.193783in}}%
\pgfpathlineto{\pgfqpoint{1.645130in}{1.189259in}}%
\pgfpathlineto{\pgfqpoint{1.656596in}{1.175933in}}%
\pgfpathlineto{\pgfqpoint{1.656859in}{1.175647in}}%
\pgfpathlineto{\pgfqpoint{1.668440in}{1.162036in}}%
\pgfpathlineto{\pgfqpoint{1.672253in}{1.157129in}}%
\pgfpathlineto{\pgfqpoint{1.679573in}{1.148425in}}%
\pgfpathlineto{\pgfqpoint{1.687910in}{1.137451in}}%
\pgfpathlineto{\pgfqpoint{1.690106in}{1.134814in}}%
\pgfpathlineto{\pgfqpoint{1.700236in}{1.121203in}}%
\pgfpathlineto{\pgfqpoint{1.703566in}{1.116103in}}%
\pgfpathlineto{\pgfqpoint{1.709766in}{1.107592in}}%
\pgfpathlineto{\pgfqpoint{1.718276in}{1.093981in}}%
\pgfpathlineto{\pgfqpoint{1.719223in}{1.092168in}}%
\pgfpathlineto{\pgfqpoint{1.726242in}{1.080370in}}%
\pgfpathlineto{\pgfqpoint{1.732722in}{1.066759in}}%
\pgfpathlineto{\pgfqpoint{1.734879in}{1.060720in}}%
\pgfpathlineto{\pgfqpoint{1.738045in}{1.053148in}}%
\pgfpathlineto{\pgfqpoint{1.741840in}{1.039536in}}%
\pgfpathlineto{\pgfqpoint{1.743738in}{1.025925in}}%
\pgfpathlineto{\pgfqpoint{1.743738in}{1.012314in}}%
\pgfpathlineto{\pgfqpoint{1.741840in}{0.998703in}}%
\pgfpathlineto{\pgfqpoint{1.738045in}{0.985092in}}%
\pgfpathlineto{\pgfqpoint{1.734879in}{0.977519in}}%
\pgfpathlineto{\pgfqpoint{1.732722in}{0.971481in}}%
\pgfpathlineto{\pgfqpoint{1.726242in}{0.957870in}}%
\pgfpathlineto{\pgfqpoint{1.719223in}{0.946071in}}%
\pgfpathlineto{\pgfqpoint{1.718276in}{0.944259in}}%
\pgfpathlineto{\pgfqpoint{1.709766in}{0.930648in}}%
\pgfpathlineto{\pgfqpoint{1.703566in}{0.922136in}}%
\pgfpathlineto{\pgfqpoint{1.700236in}{0.917036in}}%
\pgfpathlineto{\pgfqpoint{1.690106in}{0.903425in}}%
\pgfpathlineto{\pgfqpoint{1.687910in}{0.900789in}}%
\pgfpathlineto{\pgfqpoint{1.679573in}{0.889814in}}%
\pgfpathlineto{\pgfqpoint{1.672253in}{0.881110in}}%
\pgfpathlineto{\pgfqpoint{1.668440in}{0.876203in}}%
\pgfpathlineto{\pgfqpoint{1.656859in}{0.862592in}}%
\pgfpathlineto{\pgfqpoint{1.656596in}{0.862306in}}%
\pgfpathlineto{\pgfqpoint{1.645130in}{0.848981in}}%
\pgfpathlineto{\pgfqpoint{1.640940in}{0.844457in}}%
\pgfpathlineto{\pgfqpoint{1.632996in}{0.835370in}}%
\pgfpathlineto{\pgfqpoint{1.625283in}{0.827116in}}%
\pgfpathlineto{\pgfqpoint{1.620518in}{0.821759in}}%
\pgfpathlineto{\pgfqpoint{1.609627in}{0.810234in}}%
\pgfpathlineto{\pgfqpoint{1.607735in}{0.808148in}}%
\pgfpathlineto{\pgfqpoint{1.594708in}{0.794536in}}%
\pgfpathlineto{\pgfqpoint{1.593970in}{0.793801in}}%
\pgfpathlineto{\pgfqpoint{1.581478in}{0.780925in}}%
\pgfpathlineto{\pgfqpoint{1.578314in}{0.777806in}}%
\pgfpathlineto{\pgfqpoint{1.567963in}{0.767314in}}%
\pgfpathlineto{\pgfqpoint{1.562657in}{0.762147in}}%
\pgfpathlineto{\pgfqpoint{1.554176in}{0.753703in}}%
\pgfpathlineto{\pgfqpoint{1.547000in}{0.746810in}}%
\pgfpathlineto{\pgfqpoint{1.540123in}{0.740092in}}%
\pgfpathlineto{\pgfqpoint{1.531344in}{0.731782in}}%
\pgfpathlineto{\pgfqpoint{1.525807in}{0.726481in}}%
\pgfpathlineto{\pgfqpoint{1.515687in}{0.717053in}}%
\pgfpathlineto{\pgfqpoint{1.511227in}{0.712870in}}%
\pgfpathlineto{\pgfqpoint{1.500031in}{0.702611in}}%
\pgfpathlineto{\pgfqpoint{1.496379in}{0.699259in}}%
\pgfpathlineto{\pgfqpoint{1.484374in}{0.688447in}}%
\pgfpathlineto{\pgfqpoint{1.481258in}{0.685648in}}%
\pgfpathlineto{\pgfqpoint{1.468718in}{0.674550in}}%
\pgfpathlineto{\pgfqpoint{1.465858in}{0.672036in}}%
\pgfpathlineto{\pgfqpoint{1.453061in}{0.660911in}}%
\pgfpathlineto{\pgfqpoint{1.450170in}{0.658425in}}%
\pgfpathlineto{\pgfqpoint{1.437404in}{0.647523in}}%
\pgfpathlineto{\pgfqpoint{1.434185in}{0.644814in}}%
\pgfpathlineto{\pgfqpoint{1.421748in}{0.634378in}}%
\pgfpathlineto{\pgfqpoint{1.417891in}{0.631203in}}%
\pgfpathlineto{\pgfqpoint{1.406091in}{0.621470in}}%
\pgfpathlineto{\pgfqpoint{1.401279in}{0.617592in}}%
\pgfpathlineto{\pgfqpoint{1.390435in}{0.608794in}}%
\pgfpathlineto{\pgfqpoint{1.384337in}{0.603981in}}%
\pgfpathlineto{\pgfqpoint{1.374778in}{0.596349in}}%
\pgfpathlineto{\pgfqpoint{1.367051in}{0.590370in}}%
\pgfpathlineto{\pgfqpoint{1.359122in}{0.584132in}}%
\pgfpathlineto{\pgfqpoint{1.349409in}{0.576759in}}%
\pgfpathlineto{\pgfqpoint{1.343465in}{0.572146in}}%
\pgfpathlineto{\pgfqpoint{1.331396in}{0.563148in}}%
\pgfpathlineto{\pgfqpoint{1.327809in}{0.560397in}}%
\pgfpathlineto{\pgfqpoint{1.312997in}{0.549536in}}%
\pgfpathlineto{\pgfqpoint{1.312152in}{0.548895in}}%
\pgfpathlineto{\pgfqpoint{1.296495in}{0.537569in}}%
\pgfpathlineto{\pgfqpoint{1.294096in}{0.535925in}}%
\pgfpathlineto{\pgfqpoint{1.280839in}{0.526457in}}%
\pgfpathlineto{\pgfqpoint{1.274677in}{0.522314in}}%
\pgfpathlineto{\pgfqpoint{1.265182in}{0.515609in}}%
\pgfpathlineto{\pgfqpoint{1.254729in}{0.508703in}}%
\pgfpathlineto{\pgfqpoint{1.249526in}{0.505061in}}%
\pgfpathlineto{\pgfqpoint{1.234198in}{0.495092in}}%
\pgfpathlineto{\pgfqpoint{1.233869in}{0.494863in}}%
\pgfpathlineto{\pgfqpoint{1.218213in}{0.484795in}}%
\pgfpathlineto{\pgfqpoint{1.212568in}{0.481481in}}%
\pgfpathlineto{\pgfqpoint{1.202556in}{0.475117in}}%
\pgfpathlineto{\pgfqpoint{1.189932in}{0.467870in}}%
\pgfpathlineto{\pgfqpoint{1.186899in}{0.465960in}}%
\pgfpathlineto{\pgfqpoint{1.171243in}{0.457153in}}%
\pgfpathlineto{\pgfqpoint{1.165377in}{0.454259in}}%
\pgfpathlineto{\pgfqpoint{1.155586in}{0.448869in}}%
\pgfpathlineto{\pgfqpoint{1.139930in}{0.441471in}}%
\pgfpathlineto{\pgfqpoint{1.137845in}{0.440648in}}%
\pgfpathlineto{\pgfqpoint{1.124273in}{0.434545in}}%
\pgfpathlineto{\pgfqpoint{1.108617in}{0.428912in}}%
\pgfpathlineto{\pgfqpoint{1.101671in}{0.427036in}}%
\pgfpathlineto{\pgfqpoint{1.092960in}{0.424285in}}%
\pgfpathlineto{\pgfqpoint{1.077303in}{0.420985in}}%
\pgfpathlineto{\pgfqpoint{1.061647in}{0.419335in}}%
\pgfpathlineto{\pgfqpoint{1.045990in}{0.419335in}}%
\pgfpathlineto{\pgfqpoint{1.030334in}{0.420985in}}%
\pgfpathlineto{\pgfqpoint{1.014677in}{0.424285in}}%
\pgfpathlineto{\pgfqpoint{1.005967in}{0.427036in}}%
\pgfpathclose%
\pgfusepath{fill}%
\end{pgfscope}%
\begin{pgfscope}%
\pgfpathrectangle{\pgfqpoint{0.278819in}{0.345370in}}{\pgfqpoint{1.550000in}{1.347500in}}%
\pgfusepath{clip}%
\pgfsetbuttcap%
\pgfsetroundjoin%
\definecolor{currentfill}{rgb}{0.534952,0.031217,0.650165}%
\pgfsetfillcolor{currentfill}%
\pgfsetlinewidth{0.000000pt}%
\definecolor{currentstroke}{rgb}{0.000000,0.000000,0.000000}%
\pgfsetstrokecolor{currentstroke}%
\pgfsetdash{}{0pt}%
\pgfpathmoveto{\pgfqpoint{0.638920in}{0.345370in}}%
\pgfpathlineto{\pgfqpoint{0.654576in}{0.345370in}}%
\pgfpathlineto{\pgfqpoint{0.670233in}{0.345370in}}%
\pgfpathlineto{\pgfqpoint{0.685889in}{0.345370in}}%
\pgfpathlineto{\pgfqpoint{0.701546in}{0.345370in}}%
\pgfpathlineto{\pgfqpoint{0.717202in}{0.345370in}}%
\pgfpathlineto{\pgfqpoint{0.732859in}{0.345370in}}%
\pgfpathlineto{\pgfqpoint{0.748516in}{0.345370in}}%
\pgfpathlineto{\pgfqpoint{0.764172in}{0.345370in}}%
\pgfpathlineto{\pgfqpoint{0.779829in}{0.345370in}}%
\pgfpathlineto{\pgfqpoint{0.788490in}{0.345370in}}%
\pgfpathlineto{\pgfqpoint{0.787696in}{0.358981in}}%
\pgfpathlineto{\pgfqpoint{0.785320in}{0.372592in}}%
\pgfpathlineto{\pgfqpoint{0.781388in}{0.386203in}}%
\pgfpathlineto{\pgfqpoint{0.779829in}{0.390091in}}%
\pgfpathlineto{\pgfqpoint{0.776063in}{0.399814in}}%
\pgfpathlineto{\pgfqpoint{0.769373in}{0.413425in}}%
\pgfpathlineto{\pgfqpoint{0.764172in}{0.422220in}}%
\pgfpathlineto{\pgfqpoint{0.761415in}{0.427036in}}%
\pgfpathlineto{\pgfqpoint{0.752362in}{0.440648in}}%
\pgfpathlineto{\pgfqpoint{0.748516in}{0.445759in}}%
\pgfpathlineto{\pgfqpoint{0.742306in}{0.454259in}}%
\pgfpathlineto{\pgfqpoint{0.732859in}{0.465940in}}%
\pgfpathlineto{\pgfqpoint{0.731339in}{0.467870in}}%
\pgfpathlineto{\pgfqpoint{0.719628in}{0.481481in}}%
\pgfpathlineto{\pgfqpoint{0.717202in}{0.484098in}}%
\pgfpathlineto{\pgfqpoint{0.707247in}{0.495092in}}%
\pgfpathlineto{\pgfqpoint{0.701546in}{0.501007in}}%
\pgfpathlineto{\pgfqpoint{0.694275in}{0.508703in}}%
\pgfpathlineto{\pgfqpoint{0.685889in}{0.517120in}}%
\pgfpathlineto{\pgfqpoint{0.680799in}{0.522314in}}%
\pgfpathlineto{\pgfqpoint{0.670233in}{0.532624in}}%
\pgfpathlineto{\pgfqpoint{0.666893in}{0.535925in}}%
\pgfpathlineto{\pgfqpoint{0.654576in}{0.547652in}}%
\pgfpathlineto{\pgfqpoint{0.652616in}{0.549536in}}%
\pgfpathlineto{\pgfqpoint{0.638920in}{0.562299in}}%
\pgfpathlineto{\pgfqpoint{0.638014in}{0.563148in}}%
\pgfpathlineto{\pgfqpoint{0.623263in}{0.576631in}}%
\pgfpathlineto{\pgfqpoint{0.623124in}{0.576759in}}%
\pgfpathlineto{\pgfqpoint{0.607963in}{0.590370in}}%
\pgfpathlineto{\pgfqpoint{0.607606in}{0.590685in}}%
\pgfpathlineto{\pgfqpoint{0.592552in}{0.603981in}}%
\pgfpathlineto{\pgfqpoint{0.591950in}{0.604507in}}%
\pgfpathlineto{\pgfqpoint{0.576899in}{0.617592in}}%
\pgfpathlineto{\pgfqpoint{0.576293in}{0.618115in}}%
\pgfpathlineto{\pgfqpoint{0.560999in}{0.631203in}}%
\pgfpathlineto{\pgfqpoint{0.560637in}{0.631513in}}%
\pgfpathlineto{\pgfqpoint{0.544980in}{0.644693in}}%
\pgfpathlineto{\pgfqpoint{0.544833in}{0.644814in}}%
\pgfpathlineto{\pgfqpoint{0.529324in}{0.657638in}}%
\pgfpathlineto{\pgfqpoint{0.528348in}{0.658425in}}%
\pgfpathlineto{\pgfqpoint{0.513667in}{0.670333in}}%
\pgfpathlineto{\pgfqpoint{0.511500in}{0.672036in}}%
\pgfpathlineto{\pgfqpoint{0.498011in}{0.682744in}}%
\pgfpathlineto{\pgfqpoint{0.494213in}{0.685648in}}%
\pgfpathlineto{\pgfqpoint{0.482354in}{0.694834in}}%
\pgfpathlineto{\pgfqpoint{0.476379in}{0.699259in}}%
\pgfpathlineto{\pgfqpoint{0.466697in}{0.706548in}}%
\pgfpathlineto{\pgfqpoint{0.457844in}{0.712870in}}%
\pgfpathlineto{\pgfqpoint{0.451041in}{0.717826in}}%
\pgfpathlineto{\pgfqpoint{0.438394in}{0.726481in}}%
\pgfpathlineto{\pgfqpoint{0.435384in}{0.728590in}}%
\pgfpathlineto{\pgfqpoint{0.419728in}{0.738770in}}%
\pgfpathlineto{\pgfqpoint{0.417508in}{0.740092in}}%
\pgfpathlineto{\pgfqpoint{0.404071in}{0.748305in}}%
\pgfpathlineto{\pgfqpoint{0.394294in}{0.753703in}}%
\pgfpathlineto{\pgfqpoint{0.388415in}{0.757047in}}%
\pgfpathlineto{\pgfqpoint{0.372758in}{0.764917in}}%
\pgfpathlineto{\pgfqpoint{0.367218in}{0.767314in}}%
\pgfpathlineto{\pgfqpoint{0.357101in}{0.771836in}}%
\pgfpathlineto{\pgfqpoint{0.341445in}{0.777651in}}%
\pgfpathlineto{\pgfqpoint{0.330260in}{0.780925in}}%
\pgfpathlineto{\pgfqpoint{0.325788in}{0.782281in}}%
\pgfpathlineto{\pgfqpoint{0.310132in}{0.785699in}}%
\pgfpathlineto{\pgfqpoint{0.294475in}{0.787764in}}%
\pgfpathlineto{\pgfqpoint{0.278819in}{0.788455in}}%
\pgfpathlineto{\pgfqpoint{0.278819in}{0.780925in}}%
\pgfpathlineto{\pgfqpoint{0.278819in}{0.767314in}}%
\pgfpathlineto{\pgfqpoint{0.278819in}{0.753703in}}%
\pgfpathlineto{\pgfqpoint{0.278819in}{0.740092in}}%
\pgfpathlineto{\pgfqpoint{0.278819in}{0.726481in}}%
\pgfpathlineto{\pgfqpoint{0.278819in}{0.712870in}}%
\pgfpathlineto{\pgfqpoint{0.278819in}{0.699259in}}%
\pgfpathlineto{\pgfqpoint{0.278819in}{0.685648in}}%
\pgfpathlineto{\pgfqpoint{0.278819in}{0.672036in}}%
\pgfpathlineto{\pgfqpoint{0.278819in}{0.658425in}}%
\pgfpathlineto{\pgfqpoint{0.278819in}{0.651246in}}%
\pgfpathlineto{\pgfqpoint{0.294475in}{0.650706in}}%
\pgfpathlineto{\pgfqpoint{0.310132in}{0.649091in}}%
\pgfpathlineto{\pgfqpoint{0.325788in}{0.646418in}}%
\pgfpathlineto{\pgfqpoint{0.332560in}{0.644814in}}%
\pgfpathlineto{\pgfqpoint{0.341445in}{0.642693in}}%
\pgfpathlineto{\pgfqpoint{0.357101in}{0.637948in}}%
\pgfpathlineto{\pgfqpoint{0.372758in}{0.632250in}}%
\pgfpathlineto{\pgfqpoint{0.375230in}{0.631203in}}%
\pgfpathlineto{\pgfqpoint{0.388415in}{0.625570in}}%
\pgfpathlineto{\pgfqpoint{0.404071in}{0.618030in}}%
\pgfpathlineto{\pgfqpoint{0.404889in}{0.617592in}}%
\pgfpathlineto{\pgfqpoint{0.419728in}{0.609559in}}%
\pgfpathlineto{\pgfqpoint{0.429195in}{0.603981in}}%
\pgfpathlineto{\pgfqpoint{0.435384in}{0.600280in}}%
\pgfpathlineto{\pgfqpoint{0.450802in}{0.590370in}}%
\pgfpathlineto{\pgfqpoint{0.451041in}{0.590213in}}%
\pgfpathlineto{\pgfqpoint{0.466697in}{0.579276in}}%
\pgfpathlineto{\pgfqpoint{0.470115in}{0.576759in}}%
\pgfpathlineto{\pgfqpoint{0.482354in}{0.567540in}}%
\pgfpathlineto{\pgfqpoint{0.487928in}{0.563148in}}%
\pgfpathlineto{\pgfqpoint{0.498011in}{0.554982in}}%
\pgfpathlineto{\pgfqpoint{0.504482in}{0.549536in}}%
\pgfpathlineto{\pgfqpoint{0.513667in}{0.541551in}}%
\pgfpathlineto{\pgfqpoint{0.519932in}{0.535925in}}%
\pgfpathlineto{\pgfqpoint{0.529324in}{0.527160in}}%
\pgfpathlineto{\pgfqpoint{0.534376in}{0.522314in}}%
\pgfpathlineto{\pgfqpoint{0.544980in}{0.511674in}}%
\pgfpathlineto{\pgfqpoint{0.547876in}{0.508703in}}%
\pgfpathlineto{\pgfqpoint{0.560457in}{0.495092in}}%
\pgfpathlineto{\pgfqpoint{0.560637in}{0.494884in}}%
\pgfpathlineto{\pgfqpoint{0.572036in}{0.481481in}}%
\pgfpathlineto{\pgfqpoint{0.576293in}{0.476100in}}%
\pgfpathlineto{\pgfqpoint{0.582710in}{0.467870in}}%
\pgfpathlineto{\pgfqpoint{0.591950in}{0.454970in}}%
\pgfpathlineto{\pgfqpoint{0.592454in}{0.454259in}}%
\pgfpathlineto{\pgfqpoint{0.601127in}{0.440648in}}%
\pgfpathlineto{\pgfqpoint{0.607606in}{0.429186in}}%
\pgfpathlineto{\pgfqpoint{0.608811in}{0.427036in}}%
\pgfpathlineto{\pgfqpoint{0.615365in}{0.413425in}}%
\pgfpathlineto{\pgfqpoint{0.620823in}{0.399814in}}%
\pgfpathlineto{\pgfqpoint{0.623263in}{0.392090in}}%
\pgfpathlineto{\pgfqpoint{0.625108in}{0.386203in}}%
\pgfpathlineto{\pgfqpoint{0.628182in}{0.372592in}}%
\pgfpathlineto{\pgfqpoint{0.630040in}{0.358981in}}%
\pgfpathlineto{\pgfqpoint{0.630661in}{0.345370in}}%
\pgfpathlineto{\pgfqpoint{0.638920in}{0.345370in}}%
\pgfpathclose%
\pgfpathmoveto{\pgfqpoint{1.327809in}{0.345370in}}%
\pgfpathlineto{\pgfqpoint{1.343465in}{0.345370in}}%
\pgfpathlineto{\pgfqpoint{1.359122in}{0.345370in}}%
\pgfpathlineto{\pgfqpoint{1.374778in}{0.345370in}}%
\pgfpathlineto{\pgfqpoint{1.390435in}{0.345370in}}%
\pgfpathlineto{\pgfqpoint{1.406091in}{0.345370in}}%
\pgfpathlineto{\pgfqpoint{1.421748in}{0.345370in}}%
\pgfpathlineto{\pgfqpoint{1.437404in}{0.345370in}}%
\pgfpathlineto{\pgfqpoint{1.453061in}{0.345370in}}%
\pgfpathlineto{\pgfqpoint{1.468718in}{0.345370in}}%
\pgfpathlineto{\pgfqpoint{1.476976in}{0.345370in}}%
\pgfpathlineto{\pgfqpoint{1.477597in}{0.358981in}}%
\pgfpathlineto{\pgfqpoint{1.479455in}{0.372592in}}%
\pgfpathlineto{\pgfqpoint{1.482529in}{0.386203in}}%
\pgfpathlineto{\pgfqpoint{1.484374in}{0.392090in}}%
\pgfpathlineto{\pgfqpoint{1.486814in}{0.399814in}}%
\pgfpathlineto{\pgfqpoint{1.492272in}{0.413425in}}%
\pgfpathlineto{\pgfqpoint{1.498826in}{0.427036in}}%
\pgfpathlineto{\pgfqpoint{1.500031in}{0.429186in}}%
\pgfpathlineto{\pgfqpoint{1.506510in}{0.440648in}}%
\pgfpathlineto{\pgfqpoint{1.515184in}{0.454259in}}%
\pgfpathlineto{\pgfqpoint{1.515687in}{0.454970in}}%
\pgfpathlineto{\pgfqpoint{1.524928in}{0.467870in}}%
\pgfpathlineto{\pgfqpoint{1.531344in}{0.476100in}}%
\pgfpathlineto{\pgfqpoint{1.535601in}{0.481481in}}%
\pgfpathlineto{\pgfqpoint{1.547000in}{0.494884in}}%
\pgfpathlineto{\pgfqpoint{1.547180in}{0.495092in}}%
\pgfpathlineto{\pgfqpoint{1.559761in}{0.508703in}}%
\pgfpathlineto{\pgfqpoint{1.562657in}{0.511674in}}%
\pgfpathlineto{\pgfqpoint{1.573261in}{0.522314in}}%
\pgfpathlineto{\pgfqpoint{1.578314in}{0.527160in}}%
\pgfpathlineto{\pgfqpoint{1.587706in}{0.535925in}}%
\pgfpathlineto{\pgfqpoint{1.593970in}{0.541551in}}%
\pgfpathlineto{\pgfqpoint{1.603155in}{0.549536in}}%
\pgfpathlineto{\pgfqpoint{1.609627in}{0.554982in}}%
\pgfpathlineto{\pgfqpoint{1.619710in}{0.563148in}}%
\pgfpathlineto{\pgfqpoint{1.625283in}{0.567540in}}%
\pgfpathlineto{\pgfqpoint{1.637522in}{0.576759in}}%
\pgfpathlineto{\pgfqpoint{1.640940in}{0.579276in}}%
\pgfpathlineto{\pgfqpoint{1.656596in}{0.590213in}}%
\pgfpathlineto{\pgfqpoint{1.656835in}{0.590370in}}%
\pgfpathlineto{\pgfqpoint{1.672253in}{0.600280in}}%
\pgfpathlineto{\pgfqpoint{1.678442in}{0.603981in}}%
\pgfpathlineto{\pgfqpoint{1.687910in}{0.609559in}}%
\pgfpathlineto{\pgfqpoint{1.702748in}{0.617592in}}%
\pgfpathlineto{\pgfqpoint{1.703566in}{0.618030in}}%
\pgfpathlineto{\pgfqpoint{1.719223in}{0.625570in}}%
\pgfpathlineto{\pgfqpoint{1.732407in}{0.631203in}}%
\pgfpathlineto{\pgfqpoint{1.734879in}{0.632250in}}%
\pgfpathlineto{\pgfqpoint{1.750536in}{0.637948in}}%
\pgfpathlineto{\pgfqpoint{1.766192in}{0.642693in}}%
\pgfpathlineto{\pgfqpoint{1.775078in}{0.644814in}}%
\pgfpathlineto{\pgfqpoint{1.781849in}{0.646418in}}%
\pgfpathlineto{\pgfqpoint{1.797505in}{0.649091in}}%
\pgfpathlineto{\pgfqpoint{1.813162in}{0.650706in}}%
\pgfpathlineto{\pgfqpoint{1.828819in}{0.651246in}}%
\pgfpathlineto{\pgfqpoint{1.828819in}{0.658425in}}%
\pgfpathlineto{\pgfqpoint{1.828819in}{0.672036in}}%
\pgfpathlineto{\pgfqpoint{1.828819in}{0.685648in}}%
\pgfpathlineto{\pgfqpoint{1.828819in}{0.699259in}}%
\pgfpathlineto{\pgfqpoint{1.828819in}{0.712870in}}%
\pgfpathlineto{\pgfqpoint{1.828819in}{0.726481in}}%
\pgfpathlineto{\pgfqpoint{1.828819in}{0.740092in}}%
\pgfpathlineto{\pgfqpoint{1.828819in}{0.753703in}}%
\pgfpathlineto{\pgfqpoint{1.828819in}{0.767314in}}%
\pgfpathlineto{\pgfqpoint{1.828819in}{0.780925in}}%
\pgfpathlineto{\pgfqpoint{1.828819in}{0.788455in}}%
\pgfpathlineto{\pgfqpoint{1.813162in}{0.787764in}}%
\pgfpathlineto{\pgfqpoint{1.797505in}{0.785699in}}%
\pgfpathlineto{\pgfqpoint{1.781849in}{0.782281in}}%
\pgfpathlineto{\pgfqpoint{1.777377in}{0.780925in}}%
\pgfpathlineto{\pgfqpoint{1.766192in}{0.777651in}}%
\pgfpathlineto{\pgfqpoint{1.750536in}{0.771836in}}%
\pgfpathlineto{\pgfqpoint{1.740419in}{0.767314in}}%
\pgfpathlineto{\pgfqpoint{1.734879in}{0.764917in}}%
\pgfpathlineto{\pgfqpoint{1.719223in}{0.757047in}}%
\pgfpathlineto{\pgfqpoint{1.713343in}{0.753703in}}%
\pgfpathlineto{\pgfqpoint{1.703566in}{0.748305in}}%
\pgfpathlineto{\pgfqpoint{1.690129in}{0.740092in}}%
\pgfpathlineto{\pgfqpoint{1.687910in}{0.738770in}}%
\pgfpathlineto{\pgfqpoint{1.672253in}{0.728590in}}%
\pgfpathlineto{\pgfqpoint{1.669243in}{0.726481in}}%
\pgfpathlineto{\pgfqpoint{1.656596in}{0.717826in}}%
\pgfpathlineto{\pgfqpoint{1.649793in}{0.712870in}}%
\pgfpathlineto{\pgfqpoint{1.640940in}{0.706548in}}%
\pgfpathlineto{\pgfqpoint{1.631259in}{0.699259in}}%
\pgfpathlineto{\pgfqpoint{1.625283in}{0.694834in}}%
\pgfpathlineto{\pgfqpoint{1.613424in}{0.685648in}}%
\pgfpathlineto{\pgfqpoint{1.609627in}{0.682744in}}%
\pgfpathlineto{\pgfqpoint{1.596137in}{0.672036in}}%
\pgfpathlineto{\pgfqpoint{1.593970in}{0.670333in}}%
\pgfpathlineto{\pgfqpoint{1.579290in}{0.658425in}}%
\pgfpathlineto{\pgfqpoint{1.578314in}{0.657638in}}%
\pgfpathlineto{\pgfqpoint{1.562804in}{0.644814in}}%
\pgfpathlineto{\pgfqpoint{1.562657in}{0.644693in}}%
\pgfpathlineto{\pgfqpoint{1.547000in}{0.631513in}}%
\pgfpathlineto{\pgfqpoint{1.546638in}{0.631203in}}%
\pgfpathlineto{\pgfqpoint{1.531344in}{0.618115in}}%
\pgfpathlineto{\pgfqpoint{1.530739in}{0.617592in}}%
\pgfpathlineto{\pgfqpoint{1.515687in}{0.604507in}}%
\pgfpathlineto{\pgfqpoint{1.515086in}{0.603981in}}%
\pgfpathlineto{\pgfqpoint{1.500031in}{0.590685in}}%
\pgfpathlineto{\pgfqpoint{1.499674in}{0.590370in}}%
\pgfpathlineto{\pgfqpoint{1.484514in}{0.576759in}}%
\pgfpathlineto{\pgfqpoint{1.484374in}{0.576631in}}%
\pgfpathlineto{\pgfqpoint{1.469623in}{0.563148in}}%
\pgfpathlineto{\pgfqpoint{1.468718in}{0.562299in}}%
\pgfpathlineto{\pgfqpoint{1.455021in}{0.549536in}}%
\pgfpathlineto{\pgfqpoint{1.453061in}{0.547652in}}%
\pgfpathlineto{\pgfqpoint{1.440744in}{0.535925in}}%
\pgfpathlineto{\pgfqpoint{1.437404in}{0.532624in}}%
\pgfpathlineto{\pgfqpoint{1.426838in}{0.522314in}}%
\pgfpathlineto{\pgfqpoint{1.421748in}{0.517120in}}%
\pgfpathlineto{\pgfqpoint{1.413363in}{0.508703in}}%
\pgfpathlineto{\pgfqpoint{1.406091in}{0.501007in}}%
\pgfpathlineto{\pgfqpoint{1.400390in}{0.495092in}}%
\pgfpathlineto{\pgfqpoint{1.390435in}{0.484098in}}%
\pgfpathlineto{\pgfqpoint{1.388009in}{0.481481in}}%
\pgfpathlineto{\pgfqpoint{1.376299in}{0.467870in}}%
\pgfpathlineto{\pgfqpoint{1.374778in}{0.465940in}}%
\pgfpathlineto{\pgfqpoint{1.365331in}{0.454259in}}%
\pgfpathlineto{\pgfqpoint{1.359122in}{0.445759in}}%
\pgfpathlineto{\pgfqpoint{1.355275in}{0.440648in}}%
\pgfpathlineto{\pgfqpoint{1.346223in}{0.427036in}}%
\pgfpathlineto{\pgfqpoint{1.343465in}{0.422220in}}%
\pgfpathlineto{\pgfqpoint{1.338264in}{0.413425in}}%
\pgfpathlineto{\pgfqpoint{1.331575in}{0.399814in}}%
\pgfpathlineto{\pgfqpoint{1.327809in}{0.390091in}}%
\pgfpathlineto{\pgfqpoint{1.326249in}{0.386203in}}%
\pgfpathlineto{\pgfqpoint{1.322318in}{0.372592in}}%
\pgfpathlineto{\pgfqpoint{1.319942in}{0.358981in}}%
\pgfpathlineto{\pgfqpoint{1.319147in}{0.345370in}}%
\pgfpathlineto{\pgfqpoint{1.327809in}{0.345370in}}%
\pgfpathclose%
\pgfpathmoveto{\pgfqpoint{0.294475in}{1.250475in}}%
\pgfpathlineto{\pgfqpoint{0.310132in}{1.252541in}}%
\pgfpathlineto{\pgfqpoint{0.325788in}{1.255958in}}%
\pgfpathlineto{\pgfqpoint{0.330260in}{1.257314in}}%
\pgfpathlineto{\pgfqpoint{0.341445in}{1.260588in}}%
\pgfpathlineto{\pgfqpoint{0.357101in}{1.266404in}}%
\pgfpathlineto{\pgfqpoint{0.367218in}{1.270925in}}%
\pgfpathlineto{\pgfqpoint{0.372758in}{1.273323in}}%
\pgfpathlineto{\pgfqpoint{0.388415in}{1.281192in}}%
\pgfpathlineto{\pgfqpoint{0.394294in}{1.284536in}}%
\pgfpathlineto{\pgfqpoint{0.404071in}{1.289935in}}%
\pgfpathlineto{\pgfqpoint{0.417508in}{1.298148in}}%
\pgfpathlineto{\pgfqpoint{0.419728in}{1.299469in}}%
\pgfpathlineto{\pgfqpoint{0.435384in}{1.309650in}}%
\pgfpathlineto{\pgfqpoint{0.438394in}{1.311759in}}%
\pgfpathlineto{\pgfqpoint{0.451041in}{1.320413in}}%
\pgfpathlineto{\pgfqpoint{0.457844in}{1.325370in}}%
\pgfpathlineto{\pgfqpoint{0.466697in}{1.331691in}}%
\pgfpathlineto{\pgfqpoint{0.476379in}{1.338981in}}%
\pgfpathlineto{\pgfqpoint{0.482354in}{1.343406in}}%
\pgfpathlineto{\pgfqpoint{0.494213in}{1.352592in}}%
\pgfpathlineto{\pgfqpoint{0.498011in}{1.355495in}}%
\pgfpathlineto{\pgfqpoint{0.511500in}{1.366203in}}%
\pgfpathlineto{\pgfqpoint{0.513667in}{1.367907in}}%
\pgfpathlineto{\pgfqpoint{0.528348in}{1.379814in}}%
\pgfpathlineto{\pgfqpoint{0.529324in}{1.380601in}}%
\pgfpathlineto{\pgfqpoint{0.544833in}{1.393425in}}%
\pgfpathlineto{\pgfqpoint{0.544980in}{1.393547in}}%
\pgfpathlineto{\pgfqpoint{0.560637in}{1.406727in}}%
\pgfpathlineto{\pgfqpoint{0.560999in}{1.407036in}}%
\pgfpathlineto{\pgfqpoint{0.576293in}{1.420124in}}%
\pgfpathlineto{\pgfqpoint{0.576899in}{1.420648in}}%
\pgfpathlineto{\pgfqpoint{0.591950in}{1.433733in}}%
\pgfpathlineto{\pgfqpoint{0.592552in}{1.434259in}}%
\pgfpathlineto{\pgfqpoint{0.607606in}{1.447555in}}%
\pgfpathlineto{\pgfqpoint{0.607963in}{1.447870in}}%
\pgfpathlineto{\pgfqpoint{0.623124in}{1.461481in}}%
\pgfpathlineto{\pgfqpoint{0.623263in}{1.461609in}}%
\pgfpathlineto{\pgfqpoint{0.638014in}{1.475092in}}%
\pgfpathlineto{\pgfqpoint{0.638920in}{1.475940in}}%
\pgfpathlineto{\pgfqpoint{0.652616in}{1.488703in}}%
\pgfpathlineto{\pgfqpoint{0.654576in}{1.490587in}}%
\pgfpathlineto{\pgfqpoint{0.666893in}{1.502314in}}%
\pgfpathlineto{\pgfqpoint{0.670233in}{1.505615in}}%
\pgfpathlineto{\pgfqpoint{0.680799in}{1.515925in}}%
\pgfpathlineto{\pgfqpoint{0.685889in}{1.521120in}}%
\pgfpathlineto{\pgfqpoint{0.694275in}{1.529536in}}%
\pgfpathlineto{\pgfqpoint{0.701546in}{1.537233in}}%
\pgfpathlineto{\pgfqpoint{0.707247in}{1.543148in}}%
\pgfpathlineto{\pgfqpoint{0.717202in}{1.554142in}}%
\pgfpathlineto{\pgfqpoint{0.719628in}{1.556759in}}%
\pgfpathlineto{\pgfqpoint{0.731339in}{1.570370in}}%
\pgfpathlineto{\pgfqpoint{0.732859in}{1.572300in}}%
\pgfpathlineto{\pgfqpoint{0.742306in}{1.583981in}}%
\pgfpathlineto{\pgfqpoint{0.748516in}{1.592480in}}%
\pgfpathlineto{\pgfqpoint{0.752362in}{1.597592in}}%
\pgfpathlineto{\pgfqpoint{0.761415in}{1.611203in}}%
\pgfpathlineto{\pgfqpoint{0.764172in}{1.616020in}}%
\pgfpathlineto{\pgfqpoint{0.769373in}{1.624814in}}%
\pgfpathlineto{\pgfqpoint{0.776063in}{1.638425in}}%
\pgfpathlineto{\pgfqpoint{0.779829in}{1.648149in}}%
\pgfpathlineto{\pgfqpoint{0.781388in}{1.652036in}}%
\pgfpathlineto{\pgfqpoint{0.785320in}{1.665648in}}%
\pgfpathlineto{\pgfqpoint{0.787696in}{1.679259in}}%
\pgfpathlineto{\pgfqpoint{0.788490in}{1.692870in}}%
\pgfpathlineto{\pgfqpoint{0.779829in}{1.692870in}}%
\pgfpathlineto{\pgfqpoint{0.764172in}{1.692870in}}%
\pgfpathlineto{\pgfqpoint{0.748516in}{1.692870in}}%
\pgfpathlineto{\pgfqpoint{0.732859in}{1.692870in}}%
\pgfpathlineto{\pgfqpoint{0.717202in}{1.692870in}}%
\pgfpathlineto{\pgfqpoint{0.701546in}{1.692870in}}%
\pgfpathlineto{\pgfqpoint{0.685889in}{1.692870in}}%
\pgfpathlineto{\pgfqpoint{0.670233in}{1.692870in}}%
\pgfpathlineto{\pgfqpoint{0.654576in}{1.692870in}}%
\pgfpathlineto{\pgfqpoint{0.638920in}{1.692870in}}%
\pgfpathlineto{\pgfqpoint{0.630661in}{1.692870in}}%
\pgfpathlineto{\pgfqpoint{0.630040in}{1.679259in}}%
\pgfpathlineto{\pgfqpoint{0.628182in}{1.665648in}}%
\pgfpathlineto{\pgfqpoint{0.625108in}{1.652036in}}%
\pgfpathlineto{\pgfqpoint{0.623263in}{1.646150in}}%
\pgfpathlineto{\pgfqpoint{0.620823in}{1.638425in}}%
\pgfpathlineto{\pgfqpoint{0.615365in}{1.624814in}}%
\pgfpathlineto{\pgfqpoint{0.608811in}{1.611203in}}%
\pgfpathlineto{\pgfqpoint{0.607606in}{1.609054in}}%
\pgfpathlineto{\pgfqpoint{0.601127in}{1.597592in}}%
\pgfpathlineto{\pgfqpoint{0.592454in}{1.583981in}}%
\pgfpathlineto{\pgfqpoint{0.591950in}{1.583270in}}%
\pgfpathlineto{\pgfqpoint{0.582710in}{1.570370in}}%
\pgfpathlineto{\pgfqpoint{0.576293in}{1.562139in}}%
\pgfpathlineto{\pgfqpoint{0.572036in}{1.556759in}}%
\pgfpathlineto{\pgfqpoint{0.560637in}{1.543355in}}%
\pgfpathlineto{\pgfqpoint{0.560457in}{1.543148in}}%
\pgfpathlineto{\pgfqpoint{0.547876in}{1.529536in}}%
\pgfpathlineto{\pgfqpoint{0.544980in}{1.526565in}}%
\pgfpathlineto{\pgfqpoint{0.534376in}{1.515925in}}%
\pgfpathlineto{\pgfqpoint{0.529324in}{1.511080in}}%
\pgfpathlineto{\pgfqpoint{0.519932in}{1.502314in}}%
\pgfpathlineto{\pgfqpoint{0.513667in}{1.496688in}}%
\pgfpathlineto{\pgfqpoint{0.504482in}{1.488703in}}%
\pgfpathlineto{\pgfqpoint{0.498011in}{1.483257in}}%
\pgfpathlineto{\pgfqpoint{0.487928in}{1.475092in}}%
\pgfpathlineto{\pgfqpoint{0.482354in}{1.470700in}}%
\pgfpathlineto{\pgfqpoint{0.470115in}{1.461481in}}%
\pgfpathlineto{\pgfqpoint{0.466697in}{1.458963in}}%
\pgfpathlineto{\pgfqpoint{0.451041in}{1.448026in}}%
\pgfpathlineto{\pgfqpoint{0.450802in}{1.447870in}}%
\pgfpathlineto{\pgfqpoint{0.435384in}{1.437960in}}%
\pgfpathlineto{\pgfqpoint{0.429195in}{1.434259in}}%
\pgfpathlineto{\pgfqpoint{0.419728in}{1.428681in}}%
\pgfpathlineto{\pgfqpoint{0.404889in}{1.420648in}}%
\pgfpathlineto{\pgfqpoint{0.404071in}{1.420210in}}%
\pgfpathlineto{\pgfqpoint{0.388415in}{1.412670in}}%
\pgfpathlineto{\pgfqpoint{0.375230in}{1.407036in}}%
\pgfpathlineto{\pgfqpoint{0.372758in}{1.405989in}}%
\pgfpathlineto{\pgfqpoint{0.357101in}{1.400292in}}%
\pgfpathlineto{\pgfqpoint{0.341445in}{1.395546in}}%
\pgfpathlineto{\pgfqpoint{0.332560in}{1.393425in}}%
\pgfpathlineto{\pgfqpoint{0.325788in}{1.391821in}}%
\pgfpathlineto{\pgfqpoint{0.310132in}{1.389149in}}%
\pgfpathlineto{\pgfqpoint{0.294475in}{1.387534in}}%
\pgfpathlineto{\pgfqpoint{0.278819in}{1.386994in}}%
\pgfpathlineto{\pgfqpoint{0.278819in}{1.379814in}}%
\pgfpathlineto{\pgfqpoint{0.278819in}{1.366203in}}%
\pgfpathlineto{\pgfqpoint{0.278819in}{1.352592in}}%
\pgfpathlineto{\pgfqpoint{0.278819in}{1.338981in}}%
\pgfpathlineto{\pgfqpoint{0.278819in}{1.325370in}}%
\pgfpathlineto{\pgfqpoint{0.278819in}{1.311759in}}%
\pgfpathlineto{\pgfqpoint{0.278819in}{1.298148in}}%
\pgfpathlineto{\pgfqpoint{0.278819in}{1.284536in}}%
\pgfpathlineto{\pgfqpoint{0.278819in}{1.270925in}}%
\pgfpathlineto{\pgfqpoint{0.278819in}{1.257314in}}%
\pgfpathlineto{\pgfqpoint{0.278819in}{1.249784in}}%
\pgfpathlineto{\pgfqpoint{0.294475in}{1.250475in}}%
\pgfpathclose%
\pgfpathmoveto{\pgfqpoint{1.781849in}{1.255958in}}%
\pgfpathlineto{\pgfqpoint{1.797505in}{1.252541in}}%
\pgfpathlineto{\pgfqpoint{1.813162in}{1.250475in}}%
\pgfpathlineto{\pgfqpoint{1.828819in}{1.249784in}}%
\pgfpathlineto{\pgfqpoint{1.828819in}{1.257314in}}%
\pgfpathlineto{\pgfqpoint{1.828819in}{1.270925in}}%
\pgfpathlineto{\pgfqpoint{1.828819in}{1.284536in}}%
\pgfpathlineto{\pgfqpoint{1.828819in}{1.298148in}}%
\pgfpathlineto{\pgfqpoint{1.828819in}{1.311759in}}%
\pgfpathlineto{\pgfqpoint{1.828819in}{1.325370in}}%
\pgfpathlineto{\pgfqpoint{1.828819in}{1.338981in}}%
\pgfpathlineto{\pgfqpoint{1.828819in}{1.352592in}}%
\pgfpathlineto{\pgfqpoint{1.828819in}{1.366203in}}%
\pgfpathlineto{\pgfqpoint{1.828819in}{1.379814in}}%
\pgfpathlineto{\pgfqpoint{1.828819in}{1.386994in}}%
\pgfpathlineto{\pgfqpoint{1.813162in}{1.387534in}}%
\pgfpathlineto{\pgfqpoint{1.797505in}{1.389149in}}%
\pgfpathlineto{\pgfqpoint{1.781849in}{1.391821in}}%
\pgfpathlineto{\pgfqpoint{1.775078in}{1.393425in}}%
\pgfpathlineto{\pgfqpoint{1.766192in}{1.395546in}}%
\pgfpathlineto{\pgfqpoint{1.750536in}{1.400292in}}%
\pgfpathlineto{\pgfqpoint{1.734879in}{1.405989in}}%
\pgfpathlineto{\pgfqpoint{1.732407in}{1.407036in}}%
\pgfpathlineto{\pgfqpoint{1.719223in}{1.412670in}}%
\pgfpathlineto{\pgfqpoint{1.703566in}{1.420210in}}%
\pgfpathlineto{\pgfqpoint{1.702748in}{1.420648in}}%
\pgfpathlineto{\pgfqpoint{1.687910in}{1.428681in}}%
\pgfpathlineto{\pgfqpoint{1.678442in}{1.434259in}}%
\pgfpathlineto{\pgfqpoint{1.672253in}{1.437960in}}%
\pgfpathlineto{\pgfqpoint{1.656835in}{1.447870in}}%
\pgfpathlineto{\pgfqpoint{1.656596in}{1.448026in}}%
\pgfpathlineto{\pgfqpoint{1.640940in}{1.458963in}}%
\pgfpathlineto{\pgfqpoint{1.637522in}{1.461481in}}%
\pgfpathlineto{\pgfqpoint{1.625283in}{1.470700in}}%
\pgfpathlineto{\pgfqpoint{1.619710in}{1.475092in}}%
\pgfpathlineto{\pgfqpoint{1.609627in}{1.483257in}}%
\pgfpathlineto{\pgfqpoint{1.603155in}{1.488703in}}%
\pgfpathlineto{\pgfqpoint{1.593970in}{1.496688in}}%
\pgfpathlineto{\pgfqpoint{1.587706in}{1.502314in}}%
\pgfpathlineto{\pgfqpoint{1.578314in}{1.511080in}}%
\pgfpathlineto{\pgfqpoint{1.573261in}{1.515925in}}%
\pgfpathlineto{\pgfqpoint{1.562657in}{1.526565in}}%
\pgfpathlineto{\pgfqpoint{1.559761in}{1.529536in}}%
\pgfpathlineto{\pgfqpoint{1.547180in}{1.543148in}}%
\pgfpathlineto{\pgfqpoint{1.547000in}{1.543355in}}%
\pgfpathlineto{\pgfqpoint{1.535601in}{1.556759in}}%
\pgfpathlineto{\pgfqpoint{1.531344in}{1.562139in}}%
\pgfpathlineto{\pgfqpoint{1.524928in}{1.570370in}}%
\pgfpathlineto{\pgfqpoint{1.515687in}{1.583270in}}%
\pgfpathlineto{\pgfqpoint{1.515184in}{1.583981in}}%
\pgfpathlineto{\pgfqpoint{1.506510in}{1.597592in}}%
\pgfpathlineto{\pgfqpoint{1.500031in}{1.609054in}}%
\pgfpathlineto{\pgfqpoint{1.498826in}{1.611203in}}%
\pgfpathlineto{\pgfqpoint{1.492272in}{1.624814in}}%
\pgfpathlineto{\pgfqpoint{1.486814in}{1.638425in}}%
\pgfpathlineto{\pgfqpoint{1.484374in}{1.646150in}}%
\pgfpathlineto{\pgfqpoint{1.482529in}{1.652036in}}%
\pgfpathlineto{\pgfqpoint{1.479455in}{1.665648in}}%
\pgfpathlineto{\pgfqpoint{1.477597in}{1.679259in}}%
\pgfpathlineto{\pgfqpoint{1.476976in}{1.692870in}}%
\pgfpathlineto{\pgfqpoint{1.468718in}{1.692870in}}%
\pgfpathlineto{\pgfqpoint{1.453061in}{1.692870in}}%
\pgfpathlineto{\pgfqpoint{1.437404in}{1.692870in}}%
\pgfpathlineto{\pgfqpoint{1.421748in}{1.692870in}}%
\pgfpathlineto{\pgfqpoint{1.406091in}{1.692870in}}%
\pgfpathlineto{\pgfqpoint{1.390435in}{1.692870in}}%
\pgfpathlineto{\pgfqpoint{1.374778in}{1.692870in}}%
\pgfpathlineto{\pgfqpoint{1.359122in}{1.692870in}}%
\pgfpathlineto{\pgfqpoint{1.343465in}{1.692870in}}%
\pgfpathlineto{\pgfqpoint{1.327809in}{1.692870in}}%
\pgfpathlineto{\pgfqpoint{1.319147in}{1.692870in}}%
\pgfpathlineto{\pgfqpoint{1.319942in}{1.679259in}}%
\pgfpathlineto{\pgfqpoint{1.322318in}{1.665648in}}%
\pgfpathlineto{\pgfqpoint{1.326249in}{1.652036in}}%
\pgfpathlineto{\pgfqpoint{1.327809in}{1.648149in}}%
\pgfpathlineto{\pgfqpoint{1.331575in}{1.638425in}}%
\pgfpathlineto{\pgfqpoint{1.338264in}{1.624814in}}%
\pgfpathlineto{\pgfqpoint{1.343465in}{1.616020in}}%
\pgfpathlineto{\pgfqpoint{1.346223in}{1.611203in}}%
\pgfpathlineto{\pgfqpoint{1.355275in}{1.597592in}}%
\pgfpathlineto{\pgfqpoint{1.359122in}{1.592480in}}%
\pgfpathlineto{\pgfqpoint{1.365331in}{1.583981in}}%
\pgfpathlineto{\pgfqpoint{1.374778in}{1.572300in}}%
\pgfpathlineto{\pgfqpoint{1.376299in}{1.570370in}}%
\pgfpathlineto{\pgfqpoint{1.388009in}{1.556759in}}%
\pgfpathlineto{\pgfqpoint{1.390435in}{1.554142in}}%
\pgfpathlineto{\pgfqpoint{1.400390in}{1.543148in}}%
\pgfpathlineto{\pgfqpoint{1.406091in}{1.537233in}}%
\pgfpathlineto{\pgfqpoint{1.413363in}{1.529536in}}%
\pgfpathlineto{\pgfqpoint{1.421748in}{1.521120in}}%
\pgfpathlineto{\pgfqpoint{1.426838in}{1.515925in}}%
\pgfpathlineto{\pgfqpoint{1.437404in}{1.505615in}}%
\pgfpathlineto{\pgfqpoint{1.440744in}{1.502314in}}%
\pgfpathlineto{\pgfqpoint{1.453061in}{1.490587in}}%
\pgfpathlineto{\pgfqpoint{1.455021in}{1.488703in}}%
\pgfpathlineto{\pgfqpoint{1.468718in}{1.475940in}}%
\pgfpathlineto{\pgfqpoint{1.469623in}{1.475092in}}%
\pgfpathlineto{\pgfqpoint{1.484374in}{1.461609in}}%
\pgfpathlineto{\pgfqpoint{1.484514in}{1.461481in}}%
\pgfpathlineto{\pgfqpoint{1.499674in}{1.447870in}}%
\pgfpathlineto{\pgfqpoint{1.500031in}{1.447555in}}%
\pgfpathlineto{\pgfqpoint{1.515086in}{1.434259in}}%
\pgfpathlineto{\pgfqpoint{1.515687in}{1.433733in}}%
\pgfpathlineto{\pgfqpoint{1.530739in}{1.420648in}}%
\pgfpathlineto{\pgfqpoint{1.531344in}{1.420124in}}%
\pgfpathlineto{\pgfqpoint{1.546638in}{1.407036in}}%
\pgfpathlineto{\pgfqpoint{1.547000in}{1.406727in}}%
\pgfpathlineto{\pgfqpoint{1.562657in}{1.393547in}}%
\pgfpathlineto{\pgfqpoint{1.562804in}{1.393425in}}%
\pgfpathlineto{\pgfqpoint{1.578314in}{1.380601in}}%
\pgfpathlineto{\pgfqpoint{1.579290in}{1.379814in}}%
\pgfpathlineto{\pgfqpoint{1.593970in}{1.367907in}}%
\pgfpathlineto{\pgfqpoint{1.596137in}{1.366203in}}%
\pgfpathlineto{\pgfqpoint{1.609627in}{1.355495in}}%
\pgfpathlineto{\pgfqpoint{1.613424in}{1.352592in}}%
\pgfpathlineto{\pgfqpoint{1.625283in}{1.343406in}}%
\pgfpathlineto{\pgfqpoint{1.631259in}{1.338981in}}%
\pgfpathlineto{\pgfqpoint{1.640940in}{1.331691in}}%
\pgfpathlineto{\pgfqpoint{1.649793in}{1.325370in}}%
\pgfpathlineto{\pgfqpoint{1.656596in}{1.320413in}}%
\pgfpathlineto{\pgfqpoint{1.669243in}{1.311759in}}%
\pgfpathlineto{\pgfqpoint{1.672253in}{1.309650in}}%
\pgfpathlineto{\pgfqpoint{1.687910in}{1.299469in}}%
\pgfpathlineto{\pgfqpoint{1.690129in}{1.298148in}}%
\pgfpathlineto{\pgfqpoint{1.703566in}{1.289935in}}%
\pgfpathlineto{\pgfqpoint{1.713343in}{1.284536in}}%
\pgfpathlineto{\pgfqpoint{1.719223in}{1.281192in}}%
\pgfpathlineto{\pgfqpoint{1.734879in}{1.273323in}}%
\pgfpathlineto{\pgfqpoint{1.740419in}{1.270925in}}%
\pgfpathlineto{\pgfqpoint{1.750536in}{1.266404in}}%
\pgfpathlineto{\pgfqpoint{1.766192in}{1.260588in}}%
\pgfpathlineto{\pgfqpoint{1.777377in}{1.257314in}}%
\pgfpathlineto{\pgfqpoint{1.781849in}{1.255958in}}%
\pgfpathclose%
\pgfusepath{fill}%
\end{pgfscope}%
\begin{pgfscope}%
\pgfpathrectangle{\pgfqpoint{0.278819in}{0.345370in}}{\pgfqpoint{1.550000in}{1.347500in}}%
\pgfusepath{clip}%
\pgfsetbuttcap%
\pgfsetroundjoin%
\definecolor{currentfill}{rgb}{0.362553,0.003243,0.649245}%
\pgfsetfillcolor{currentfill}%
\pgfsetlinewidth{0.000000pt}%
\definecolor{currentstroke}{rgb}{0.000000,0.000000,0.000000}%
\pgfsetstrokecolor{currentstroke}%
\pgfsetdash{}{0pt}%
\pgfpathmoveto{\pgfqpoint{0.451041in}{0.351484in}}%
\pgfpathlineto{\pgfqpoint{0.451397in}{0.345370in}}%
\pgfpathlineto{\pgfqpoint{0.466697in}{0.345370in}}%
\pgfpathlineto{\pgfqpoint{0.482354in}{0.345370in}}%
\pgfpathlineto{\pgfqpoint{0.498011in}{0.345370in}}%
\pgfpathlineto{\pgfqpoint{0.513667in}{0.345370in}}%
\pgfpathlineto{\pgfqpoint{0.529324in}{0.345370in}}%
\pgfpathlineto{\pgfqpoint{0.544980in}{0.345370in}}%
\pgfpathlineto{\pgfqpoint{0.560637in}{0.345370in}}%
\pgfpathlineto{\pgfqpoint{0.576293in}{0.345370in}}%
\pgfpathlineto{\pgfqpoint{0.591950in}{0.345370in}}%
\pgfpathlineto{\pgfqpoint{0.607606in}{0.345370in}}%
\pgfpathlineto{\pgfqpoint{0.623263in}{0.345370in}}%
\pgfpathlineto{\pgfqpoint{0.630661in}{0.345370in}}%
\pgfpathlineto{\pgfqpoint{0.630040in}{0.358981in}}%
\pgfpathlineto{\pgfqpoint{0.628182in}{0.372592in}}%
\pgfpathlineto{\pgfqpoint{0.625108in}{0.386203in}}%
\pgfpathlineto{\pgfqpoint{0.623263in}{0.392090in}}%
\pgfpathlineto{\pgfqpoint{0.620823in}{0.399814in}}%
\pgfpathlineto{\pgfqpoint{0.615365in}{0.413425in}}%
\pgfpathlineto{\pgfqpoint{0.608811in}{0.427036in}}%
\pgfpathlineto{\pgfqpoint{0.607606in}{0.429186in}}%
\pgfpathlineto{\pgfqpoint{0.601127in}{0.440648in}}%
\pgfpathlineto{\pgfqpoint{0.592454in}{0.454259in}}%
\pgfpathlineto{\pgfqpoint{0.591950in}{0.454970in}}%
\pgfpathlineto{\pgfqpoint{0.582710in}{0.467870in}}%
\pgfpathlineto{\pgfqpoint{0.576293in}{0.476100in}}%
\pgfpathlineto{\pgfqpoint{0.572036in}{0.481481in}}%
\pgfpathlineto{\pgfqpoint{0.560637in}{0.494884in}}%
\pgfpathlineto{\pgfqpoint{0.560457in}{0.495092in}}%
\pgfpathlineto{\pgfqpoint{0.547876in}{0.508703in}}%
\pgfpathlineto{\pgfqpoint{0.544980in}{0.511674in}}%
\pgfpathlineto{\pgfqpoint{0.534376in}{0.522314in}}%
\pgfpathlineto{\pgfqpoint{0.529324in}{0.527160in}}%
\pgfpathlineto{\pgfqpoint{0.519932in}{0.535925in}}%
\pgfpathlineto{\pgfqpoint{0.513667in}{0.541551in}}%
\pgfpathlineto{\pgfqpoint{0.504482in}{0.549536in}}%
\pgfpathlineto{\pgfqpoint{0.498011in}{0.554982in}}%
\pgfpathlineto{\pgfqpoint{0.487928in}{0.563148in}}%
\pgfpathlineto{\pgfqpoint{0.482354in}{0.567540in}}%
\pgfpathlineto{\pgfqpoint{0.470115in}{0.576759in}}%
\pgfpathlineto{\pgfqpoint{0.466697in}{0.579276in}}%
\pgfpathlineto{\pgfqpoint{0.451041in}{0.590213in}}%
\pgfpathlineto{\pgfqpoint{0.450802in}{0.590370in}}%
\pgfpathlineto{\pgfqpoint{0.435384in}{0.600280in}}%
\pgfpathlineto{\pgfqpoint{0.429195in}{0.603981in}}%
\pgfpathlineto{\pgfqpoint{0.419728in}{0.609559in}}%
\pgfpathlineto{\pgfqpoint{0.404889in}{0.617592in}}%
\pgfpathlineto{\pgfqpoint{0.404071in}{0.618030in}}%
\pgfpathlineto{\pgfqpoint{0.388415in}{0.625570in}}%
\pgfpathlineto{\pgfqpoint{0.375230in}{0.631203in}}%
\pgfpathlineto{\pgfqpoint{0.372758in}{0.632250in}}%
\pgfpathlineto{\pgfqpoint{0.357101in}{0.637948in}}%
\pgfpathlineto{\pgfqpoint{0.341445in}{0.642693in}}%
\pgfpathlineto{\pgfqpoint{0.332560in}{0.644814in}}%
\pgfpathlineto{\pgfqpoint{0.325788in}{0.646418in}}%
\pgfpathlineto{\pgfqpoint{0.310132in}{0.649091in}}%
\pgfpathlineto{\pgfqpoint{0.294475in}{0.650706in}}%
\pgfpathlineto{\pgfqpoint{0.278819in}{0.651246in}}%
\pgfpathlineto{\pgfqpoint{0.278819in}{0.644814in}}%
\pgfpathlineto{\pgfqpoint{0.278819in}{0.631203in}}%
\pgfpathlineto{\pgfqpoint{0.278819in}{0.617592in}}%
\pgfpathlineto{\pgfqpoint{0.278819in}{0.603981in}}%
\pgfpathlineto{\pgfqpoint{0.278819in}{0.590370in}}%
\pgfpathlineto{\pgfqpoint{0.278819in}{0.576759in}}%
\pgfpathlineto{\pgfqpoint{0.278819in}{0.563148in}}%
\pgfpathlineto{\pgfqpoint{0.278819in}{0.549536in}}%
\pgfpathlineto{\pgfqpoint{0.278819in}{0.535925in}}%
\pgfpathlineto{\pgfqpoint{0.278819in}{0.522314in}}%
\pgfpathlineto{\pgfqpoint{0.278819in}{0.508703in}}%
\pgfpathlineto{\pgfqpoint{0.278819in}{0.495402in}}%
\pgfpathlineto{\pgfqpoint{0.285852in}{0.495092in}}%
\pgfpathlineto{\pgfqpoint{0.294475in}{0.494682in}}%
\pgfpathlineto{\pgfqpoint{0.310132in}{0.492458in}}%
\pgfpathlineto{\pgfqpoint{0.325788in}{0.488777in}}%
\pgfpathlineto{\pgfqpoint{0.341445in}{0.483679in}}%
\pgfpathlineto{\pgfqpoint{0.346754in}{0.481481in}}%
\pgfpathlineto{\pgfqpoint{0.357101in}{0.476833in}}%
\pgfpathlineto{\pgfqpoint{0.372758in}{0.468374in}}%
\pgfpathlineto{\pgfqpoint{0.373559in}{0.467870in}}%
\pgfpathlineto{\pgfqpoint{0.388415in}{0.457620in}}%
\pgfpathlineto{\pgfqpoint{0.392715in}{0.454259in}}%
\pgfpathlineto{\pgfqpoint{0.404071in}{0.444386in}}%
\pgfpathlineto{\pgfqpoint{0.407938in}{0.440648in}}%
\pgfpathlineto{\pgfqpoint{0.419728in}{0.427733in}}%
\pgfpathlineto{\pgfqpoint{0.420308in}{0.427036in}}%
\pgfpathlineto{\pgfqpoint{0.430038in}{0.413425in}}%
\pgfpathlineto{\pgfqpoint{0.435384in}{0.404429in}}%
\pgfpathlineto{\pgfqpoint{0.437913in}{0.399814in}}%
\pgfpathlineto{\pgfqpoint{0.443777in}{0.386203in}}%
\pgfpathlineto{\pgfqpoint{0.448011in}{0.372592in}}%
\pgfpathlineto{\pgfqpoint{0.450570in}{0.358981in}}%
\pgfpathlineto{\pgfqpoint{0.451041in}{0.351484in}}%
\pgfpathclose%
\pgfpathmoveto{\pgfqpoint{1.484374in}{0.345370in}}%
\pgfpathlineto{\pgfqpoint{1.500031in}{0.345370in}}%
\pgfpathlineto{\pgfqpoint{1.515687in}{0.345370in}}%
\pgfpathlineto{\pgfqpoint{1.531344in}{0.345370in}}%
\pgfpathlineto{\pgfqpoint{1.547000in}{0.345370in}}%
\pgfpathlineto{\pgfqpoint{1.562657in}{0.345370in}}%
\pgfpathlineto{\pgfqpoint{1.578314in}{0.345370in}}%
\pgfpathlineto{\pgfqpoint{1.593970in}{0.345370in}}%
\pgfpathlineto{\pgfqpoint{1.609627in}{0.345370in}}%
\pgfpathlineto{\pgfqpoint{1.625283in}{0.345370in}}%
\pgfpathlineto{\pgfqpoint{1.640940in}{0.345370in}}%
\pgfpathlineto{\pgfqpoint{1.656240in}{0.345370in}}%
\pgfpathlineto{\pgfqpoint{1.656596in}{0.351484in}}%
\pgfpathlineto{\pgfqpoint{1.657068in}{0.358981in}}%
\pgfpathlineto{\pgfqpoint{1.659626in}{0.372592in}}%
\pgfpathlineto{\pgfqpoint{1.663860in}{0.386203in}}%
\pgfpathlineto{\pgfqpoint{1.669724in}{0.399814in}}%
\pgfpathlineto{\pgfqpoint{1.672253in}{0.404429in}}%
\pgfpathlineto{\pgfqpoint{1.677600in}{0.413425in}}%
\pgfpathlineto{\pgfqpoint{1.687329in}{0.427036in}}%
\pgfpathlineto{\pgfqpoint{1.687910in}{0.427733in}}%
\pgfpathlineto{\pgfqpoint{1.699699in}{0.440648in}}%
\pgfpathlineto{\pgfqpoint{1.703566in}{0.444386in}}%
\pgfpathlineto{\pgfqpoint{1.714922in}{0.454259in}}%
\pgfpathlineto{\pgfqpoint{1.719223in}{0.457620in}}%
\pgfpathlineto{\pgfqpoint{1.734078in}{0.467870in}}%
\pgfpathlineto{\pgfqpoint{1.734879in}{0.468374in}}%
\pgfpathlineto{\pgfqpoint{1.750536in}{0.476833in}}%
\pgfpathlineto{\pgfqpoint{1.760883in}{0.481481in}}%
\pgfpathlineto{\pgfqpoint{1.766192in}{0.483679in}}%
\pgfpathlineto{\pgfqpoint{1.781849in}{0.488777in}}%
\pgfpathlineto{\pgfqpoint{1.797505in}{0.492458in}}%
\pgfpathlineto{\pgfqpoint{1.813162in}{0.494682in}}%
\pgfpathlineto{\pgfqpoint{1.821785in}{0.495092in}}%
\pgfpathlineto{\pgfqpoint{1.828819in}{0.495402in}}%
\pgfpathlineto{\pgfqpoint{1.828819in}{0.508703in}}%
\pgfpathlineto{\pgfqpoint{1.828819in}{0.522314in}}%
\pgfpathlineto{\pgfqpoint{1.828819in}{0.535925in}}%
\pgfpathlineto{\pgfqpoint{1.828819in}{0.549536in}}%
\pgfpathlineto{\pgfqpoint{1.828819in}{0.563148in}}%
\pgfpathlineto{\pgfqpoint{1.828819in}{0.576759in}}%
\pgfpathlineto{\pgfqpoint{1.828819in}{0.590370in}}%
\pgfpathlineto{\pgfqpoint{1.828819in}{0.603981in}}%
\pgfpathlineto{\pgfqpoint{1.828819in}{0.617592in}}%
\pgfpathlineto{\pgfqpoint{1.828819in}{0.631203in}}%
\pgfpathlineto{\pgfqpoint{1.828819in}{0.644814in}}%
\pgfpathlineto{\pgfqpoint{1.828819in}{0.651246in}}%
\pgfpathlineto{\pgfqpoint{1.813162in}{0.650706in}}%
\pgfpathlineto{\pgfqpoint{1.797505in}{0.649091in}}%
\pgfpathlineto{\pgfqpoint{1.781849in}{0.646418in}}%
\pgfpathlineto{\pgfqpoint{1.775078in}{0.644814in}}%
\pgfpathlineto{\pgfqpoint{1.766192in}{0.642693in}}%
\pgfpathlineto{\pgfqpoint{1.750536in}{0.637948in}}%
\pgfpathlineto{\pgfqpoint{1.734879in}{0.632250in}}%
\pgfpathlineto{\pgfqpoint{1.732407in}{0.631203in}}%
\pgfpathlineto{\pgfqpoint{1.719223in}{0.625570in}}%
\pgfpathlineto{\pgfqpoint{1.703566in}{0.618030in}}%
\pgfpathlineto{\pgfqpoint{1.702748in}{0.617592in}}%
\pgfpathlineto{\pgfqpoint{1.687910in}{0.609559in}}%
\pgfpathlineto{\pgfqpoint{1.678442in}{0.603981in}}%
\pgfpathlineto{\pgfqpoint{1.672253in}{0.600280in}}%
\pgfpathlineto{\pgfqpoint{1.656835in}{0.590370in}}%
\pgfpathlineto{\pgfqpoint{1.656596in}{0.590213in}}%
\pgfpathlineto{\pgfqpoint{1.640940in}{0.579276in}}%
\pgfpathlineto{\pgfqpoint{1.637522in}{0.576759in}}%
\pgfpathlineto{\pgfqpoint{1.625283in}{0.567540in}}%
\pgfpathlineto{\pgfqpoint{1.619710in}{0.563148in}}%
\pgfpathlineto{\pgfqpoint{1.609627in}{0.554982in}}%
\pgfpathlineto{\pgfqpoint{1.603155in}{0.549536in}}%
\pgfpathlineto{\pgfqpoint{1.593970in}{0.541551in}}%
\pgfpathlineto{\pgfqpoint{1.587706in}{0.535925in}}%
\pgfpathlineto{\pgfqpoint{1.578314in}{0.527160in}}%
\pgfpathlineto{\pgfqpoint{1.573261in}{0.522314in}}%
\pgfpathlineto{\pgfqpoint{1.562657in}{0.511674in}}%
\pgfpathlineto{\pgfqpoint{1.559761in}{0.508703in}}%
\pgfpathlineto{\pgfqpoint{1.547180in}{0.495092in}}%
\pgfpathlineto{\pgfqpoint{1.547000in}{0.494884in}}%
\pgfpathlineto{\pgfqpoint{1.535601in}{0.481481in}}%
\pgfpathlineto{\pgfqpoint{1.531344in}{0.476100in}}%
\pgfpathlineto{\pgfqpoint{1.524928in}{0.467870in}}%
\pgfpathlineto{\pgfqpoint{1.515687in}{0.454970in}}%
\pgfpathlineto{\pgfqpoint{1.515184in}{0.454259in}}%
\pgfpathlineto{\pgfqpoint{1.506510in}{0.440648in}}%
\pgfpathlineto{\pgfqpoint{1.500031in}{0.429186in}}%
\pgfpathlineto{\pgfqpoint{1.498826in}{0.427036in}}%
\pgfpathlineto{\pgfqpoint{1.492272in}{0.413425in}}%
\pgfpathlineto{\pgfqpoint{1.486814in}{0.399814in}}%
\pgfpathlineto{\pgfqpoint{1.484374in}{0.392090in}}%
\pgfpathlineto{\pgfqpoint{1.482529in}{0.386203in}}%
\pgfpathlineto{\pgfqpoint{1.479455in}{0.372592in}}%
\pgfpathlineto{\pgfqpoint{1.477597in}{0.358981in}}%
\pgfpathlineto{\pgfqpoint{1.476976in}{0.345370in}}%
\pgfpathlineto{\pgfqpoint{1.484374in}{0.345370in}}%
\pgfpathclose%
\pgfpathmoveto{\pgfqpoint{0.294475in}{1.387534in}}%
\pgfpathlineto{\pgfqpoint{0.310132in}{1.389149in}}%
\pgfpathlineto{\pgfqpoint{0.325788in}{1.391821in}}%
\pgfpathlineto{\pgfqpoint{0.332560in}{1.393425in}}%
\pgfpathlineto{\pgfqpoint{0.341445in}{1.395546in}}%
\pgfpathlineto{\pgfqpoint{0.357101in}{1.400292in}}%
\pgfpathlineto{\pgfqpoint{0.372758in}{1.405989in}}%
\pgfpathlineto{\pgfqpoint{0.375230in}{1.407036in}}%
\pgfpathlineto{\pgfqpoint{0.388415in}{1.412670in}}%
\pgfpathlineto{\pgfqpoint{0.404071in}{1.420210in}}%
\pgfpathlineto{\pgfqpoint{0.404889in}{1.420648in}}%
\pgfpathlineto{\pgfqpoint{0.419728in}{1.428681in}}%
\pgfpathlineto{\pgfqpoint{0.429195in}{1.434259in}}%
\pgfpathlineto{\pgfqpoint{0.435384in}{1.437960in}}%
\pgfpathlineto{\pgfqpoint{0.450802in}{1.447870in}}%
\pgfpathlineto{\pgfqpoint{0.451041in}{1.448026in}}%
\pgfpathlineto{\pgfqpoint{0.466697in}{1.458963in}}%
\pgfpathlineto{\pgfqpoint{0.470115in}{1.461481in}}%
\pgfpathlineto{\pgfqpoint{0.482354in}{1.470700in}}%
\pgfpathlineto{\pgfqpoint{0.487928in}{1.475092in}}%
\pgfpathlineto{\pgfqpoint{0.498011in}{1.483257in}}%
\pgfpathlineto{\pgfqpoint{0.504482in}{1.488703in}}%
\pgfpathlineto{\pgfqpoint{0.513667in}{1.496688in}}%
\pgfpathlineto{\pgfqpoint{0.519932in}{1.502314in}}%
\pgfpathlineto{\pgfqpoint{0.529324in}{1.511080in}}%
\pgfpathlineto{\pgfqpoint{0.534376in}{1.515925in}}%
\pgfpathlineto{\pgfqpoint{0.544980in}{1.526565in}}%
\pgfpathlineto{\pgfqpoint{0.547876in}{1.529536in}}%
\pgfpathlineto{\pgfqpoint{0.560457in}{1.543148in}}%
\pgfpathlineto{\pgfqpoint{0.560637in}{1.543355in}}%
\pgfpathlineto{\pgfqpoint{0.572036in}{1.556759in}}%
\pgfpathlineto{\pgfqpoint{0.576293in}{1.562139in}}%
\pgfpathlineto{\pgfqpoint{0.582710in}{1.570370in}}%
\pgfpathlineto{\pgfqpoint{0.591950in}{1.583270in}}%
\pgfpathlineto{\pgfqpoint{0.592454in}{1.583981in}}%
\pgfpathlineto{\pgfqpoint{0.601127in}{1.597592in}}%
\pgfpathlineto{\pgfqpoint{0.607606in}{1.609054in}}%
\pgfpathlineto{\pgfqpoint{0.608811in}{1.611203in}}%
\pgfpathlineto{\pgfqpoint{0.615365in}{1.624814in}}%
\pgfpathlineto{\pgfqpoint{0.620823in}{1.638425in}}%
\pgfpathlineto{\pgfqpoint{0.623263in}{1.646150in}}%
\pgfpathlineto{\pgfqpoint{0.625108in}{1.652036in}}%
\pgfpathlineto{\pgfqpoint{0.628182in}{1.665647in}}%
\pgfpathlineto{\pgfqpoint{0.630040in}{1.679259in}}%
\pgfpathlineto{\pgfqpoint{0.630661in}{1.692870in}}%
\pgfpathlineto{\pgfqpoint{0.623263in}{1.692870in}}%
\pgfpathlineto{\pgfqpoint{0.607606in}{1.692870in}}%
\pgfpathlineto{\pgfqpoint{0.591950in}{1.692870in}}%
\pgfpathlineto{\pgfqpoint{0.576293in}{1.692870in}}%
\pgfpathlineto{\pgfqpoint{0.560637in}{1.692870in}}%
\pgfpathlineto{\pgfqpoint{0.544980in}{1.692870in}}%
\pgfpathlineto{\pgfqpoint{0.529324in}{1.692870in}}%
\pgfpathlineto{\pgfqpoint{0.513667in}{1.692870in}}%
\pgfpathlineto{\pgfqpoint{0.498011in}{1.692870in}}%
\pgfpathlineto{\pgfqpoint{0.482354in}{1.692870in}}%
\pgfpathlineto{\pgfqpoint{0.466697in}{1.692870in}}%
\pgfpathlineto{\pgfqpoint{0.451397in}{1.692870in}}%
\pgfpathlineto{\pgfqpoint{0.451041in}{1.686755in}}%
\pgfpathlineto{\pgfqpoint{0.450570in}{1.679259in}}%
\pgfpathlineto{\pgfqpoint{0.448011in}{1.665648in}}%
\pgfpathlineto{\pgfqpoint{0.443777in}{1.652036in}}%
\pgfpathlineto{\pgfqpoint{0.437913in}{1.638425in}}%
\pgfpathlineto{\pgfqpoint{0.435384in}{1.633810in}}%
\pgfpathlineto{\pgfqpoint{0.430038in}{1.624814in}}%
\pgfpathlineto{\pgfqpoint{0.420308in}{1.611203in}}%
\pgfpathlineto{\pgfqpoint{0.419728in}{1.610506in}}%
\pgfpathlineto{\pgfqpoint{0.407938in}{1.597592in}}%
\pgfpathlineto{\pgfqpoint{0.404071in}{1.593853in}}%
\pgfpathlineto{\pgfqpoint{0.392715in}{1.583981in}}%
\pgfpathlineto{\pgfqpoint{0.388415in}{1.580619in}}%
\pgfpathlineto{\pgfqpoint{0.373559in}{1.570370in}}%
\pgfpathlineto{\pgfqpoint{0.372758in}{1.569865in}}%
\pgfpathlineto{\pgfqpoint{0.357101in}{1.561407in}}%
\pgfpathlineto{\pgfqpoint{0.346754in}{1.556759in}}%
\pgfpathlineto{\pgfqpoint{0.341445in}{1.554560in}}%
\pgfpathlineto{\pgfqpoint{0.325788in}{1.549463in}}%
\pgfpathlineto{\pgfqpoint{0.310132in}{1.545782in}}%
\pgfpathlineto{\pgfqpoint{0.294475in}{1.543557in}}%
\pgfpathlineto{\pgfqpoint{0.285852in}{1.543148in}}%
\pgfpathlineto{\pgfqpoint{0.278819in}{1.542838in}}%
\pgfpathlineto{\pgfqpoint{0.278819in}{1.529536in}}%
\pgfpathlineto{\pgfqpoint{0.278819in}{1.515925in}}%
\pgfpathlineto{\pgfqpoint{0.278819in}{1.502314in}}%
\pgfpathlineto{\pgfqpoint{0.278819in}{1.488703in}}%
\pgfpathlineto{\pgfqpoint{0.278819in}{1.475092in}}%
\pgfpathlineto{\pgfqpoint{0.278819in}{1.461481in}}%
\pgfpathlineto{\pgfqpoint{0.278819in}{1.447870in}}%
\pgfpathlineto{\pgfqpoint{0.278819in}{1.434259in}}%
\pgfpathlineto{\pgfqpoint{0.278819in}{1.420648in}}%
\pgfpathlineto{\pgfqpoint{0.278819in}{1.407036in}}%
\pgfpathlineto{\pgfqpoint{0.278819in}{1.393425in}}%
\pgfpathlineto{\pgfqpoint{0.278819in}{1.386994in}}%
\pgfpathlineto{\pgfqpoint{0.294475in}{1.387534in}}%
\pgfpathclose%
\pgfpathmoveto{\pgfqpoint{1.781849in}{1.391821in}}%
\pgfpathlineto{\pgfqpoint{1.797505in}{1.389149in}}%
\pgfpathlineto{\pgfqpoint{1.813162in}{1.387534in}}%
\pgfpathlineto{\pgfqpoint{1.828819in}{1.386994in}}%
\pgfpathlineto{\pgfqpoint{1.828819in}{1.393425in}}%
\pgfpathlineto{\pgfqpoint{1.828819in}{1.407036in}}%
\pgfpathlineto{\pgfqpoint{1.828819in}{1.420648in}}%
\pgfpathlineto{\pgfqpoint{1.828819in}{1.434259in}}%
\pgfpathlineto{\pgfqpoint{1.828819in}{1.447870in}}%
\pgfpathlineto{\pgfqpoint{1.828819in}{1.461481in}}%
\pgfpathlineto{\pgfqpoint{1.828819in}{1.475092in}}%
\pgfpathlineto{\pgfqpoint{1.828819in}{1.488703in}}%
\pgfpathlineto{\pgfqpoint{1.828819in}{1.502314in}}%
\pgfpathlineto{\pgfqpoint{1.828819in}{1.515925in}}%
\pgfpathlineto{\pgfqpoint{1.828819in}{1.529536in}}%
\pgfpathlineto{\pgfqpoint{1.828819in}{1.542838in}}%
\pgfpathlineto{\pgfqpoint{1.821785in}{1.543148in}}%
\pgfpathlineto{\pgfqpoint{1.813162in}{1.543557in}}%
\pgfpathlineto{\pgfqpoint{1.797505in}{1.545782in}}%
\pgfpathlineto{\pgfqpoint{1.781849in}{1.549463in}}%
\pgfpathlineto{\pgfqpoint{1.766192in}{1.554560in}}%
\pgfpathlineto{\pgfqpoint{1.760883in}{1.556759in}}%
\pgfpathlineto{\pgfqpoint{1.750536in}{1.561407in}}%
\pgfpathlineto{\pgfqpoint{1.734879in}{1.569865in}}%
\pgfpathlineto{\pgfqpoint{1.734078in}{1.570370in}}%
\pgfpathlineto{\pgfqpoint{1.719223in}{1.580619in}}%
\pgfpathlineto{\pgfqpoint{1.714922in}{1.583981in}}%
\pgfpathlineto{\pgfqpoint{1.703566in}{1.593853in}}%
\pgfpathlineto{\pgfqpoint{1.699699in}{1.597592in}}%
\pgfpathlineto{\pgfqpoint{1.687910in}{1.610506in}}%
\pgfpathlineto{\pgfqpoint{1.687329in}{1.611203in}}%
\pgfpathlineto{\pgfqpoint{1.677600in}{1.624814in}}%
\pgfpathlineto{\pgfqpoint{1.672253in}{1.633810in}}%
\pgfpathlineto{\pgfqpoint{1.669724in}{1.638425in}}%
\pgfpathlineto{\pgfqpoint{1.663860in}{1.652036in}}%
\pgfpathlineto{\pgfqpoint{1.659626in}{1.665648in}}%
\pgfpathlineto{\pgfqpoint{1.657068in}{1.679259in}}%
\pgfpathlineto{\pgfqpoint{1.656596in}{1.686755in}}%
\pgfpathlineto{\pgfqpoint{1.656240in}{1.692870in}}%
\pgfpathlineto{\pgfqpoint{1.640940in}{1.692870in}}%
\pgfpathlineto{\pgfqpoint{1.625283in}{1.692870in}}%
\pgfpathlineto{\pgfqpoint{1.609627in}{1.692870in}}%
\pgfpathlineto{\pgfqpoint{1.593970in}{1.692870in}}%
\pgfpathlineto{\pgfqpoint{1.578314in}{1.692870in}}%
\pgfpathlineto{\pgfqpoint{1.562657in}{1.692870in}}%
\pgfpathlineto{\pgfqpoint{1.547000in}{1.692870in}}%
\pgfpathlineto{\pgfqpoint{1.531344in}{1.692870in}}%
\pgfpathlineto{\pgfqpoint{1.515687in}{1.692870in}}%
\pgfpathlineto{\pgfqpoint{1.500031in}{1.692870in}}%
\pgfpathlineto{\pgfqpoint{1.484374in}{1.692870in}}%
\pgfpathlineto{\pgfqpoint{1.476976in}{1.692870in}}%
\pgfpathlineto{\pgfqpoint{1.477597in}{1.679259in}}%
\pgfpathlineto{\pgfqpoint{1.479455in}{1.665647in}}%
\pgfpathlineto{\pgfqpoint{1.482529in}{1.652036in}}%
\pgfpathlineto{\pgfqpoint{1.484374in}{1.646150in}}%
\pgfpathlineto{\pgfqpoint{1.486814in}{1.638425in}}%
\pgfpathlineto{\pgfqpoint{1.492272in}{1.624814in}}%
\pgfpathlineto{\pgfqpoint{1.498826in}{1.611203in}}%
\pgfpathlineto{\pgfqpoint{1.500031in}{1.609054in}}%
\pgfpathlineto{\pgfqpoint{1.506510in}{1.597592in}}%
\pgfpathlineto{\pgfqpoint{1.515184in}{1.583981in}}%
\pgfpathlineto{\pgfqpoint{1.515687in}{1.583270in}}%
\pgfpathlineto{\pgfqpoint{1.524928in}{1.570370in}}%
\pgfpathlineto{\pgfqpoint{1.531344in}{1.562139in}}%
\pgfpathlineto{\pgfqpoint{1.535601in}{1.556759in}}%
\pgfpathlineto{\pgfqpoint{1.547000in}{1.543355in}}%
\pgfpathlineto{\pgfqpoint{1.547180in}{1.543148in}}%
\pgfpathlineto{\pgfqpoint{1.559761in}{1.529536in}}%
\pgfpathlineto{\pgfqpoint{1.562657in}{1.526565in}}%
\pgfpathlineto{\pgfqpoint{1.573261in}{1.515925in}}%
\pgfpathlineto{\pgfqpoint{1.578314in}{1.511080in}}%
\pgfpathlineto{\pgfqpoint{1.587706in}{1.502314in}}%
\pgfpathlineto{\pgfqpoint{1.593970in}{1.496688in}}%
\pgfpathlineto{\pgfqpoint{1.603155in}{1.488703in}}%
\pgfpathlineto{\pgfqpoint{1.609627in}{1.483257in}}%
\pgfpathlineto{\pgfqpoint{1.619710in}{1.475092in}}%
\pgfpathlineto{\pgfqpoint{1.625283in}{1.470700in}}%
\pgfpathlineto{\pgfqpoint{1.637522in}{1.461481in}}%
\pgfpathlineto{\pgfqpoint{1.640940in}{1.458963in}}%
\pgfpathlineto{\pgfqpoint{1.656596in}{1.448026in}}%
\pgfpathlineto{\pgfqpoint{1.656835in}{1.447870in}}%
\pgfpathlineto{\pgfqpoint{1.672253in}{1.437960in}}%
\pgfpathlineto{\pgfqpoint{1.678442in}{1.434259in}}%
\pgfpathlineto{\pgfqpoint{1.687910in}{1.428681in}}%
\pgfpathlineto{\pgfqpoint{1.702748in}{1.420648in}}%
\pgfpathlineto{\pgfqpoint{1.703566in}{1.420210in}}%
\pgfpathlineto{\pgfqpoint{1.719223in}{1.412670in}}%
\pgfpathlineto{\pgfqpoint{1.732407in}{1.407036in}}%
\pgfpathlineto{\pgfqpoint{1.734879in}{1.405989in}}%
\pgfpathlineto{\pgfqpoint{1.750536in}{1.400292in}}%
\pgfpathlineto{\pgfqpoint{1.766192in}{1.395546in}}%
\pgfpathlineto{\pgfqpoint{1.775078in}{1.393425in}}%
\pgfpathlineto{\pgfqpoint{1.781849in}{1.391821in}}%
\pgfpathclose%
\pgfusepath{fill}%
\end{pgfscope}%
\begin{pgfscope}%
\pgfpathrectangle{\pgfqpoint{0.278819in}{0.345370in}}{\pgfqpoint{1.550000in}{1.347500in}}%
\pgfusepath{clip}%
\pgfsetbuttcap%
\pgfsetroundjoin%
\definecolor{currentfill}{rgb}{0.178950,0.019252,0.584054}%
\pgfsetfillcolor{currentfill}%
\pgfsetlinewidth{0.000000pt}%
\definecolor{currentstroke}{rgb}{0.000000,0.000000,0.000000}%
\pgfsetstrokecolor{currentstroke}%
\pgfsetdash{}{0pt}%
\pgfpathmoveto{\pgfqpoint{0.294475in}{0.345370in}}%
\pgfpathlineto{\pgfqpoint{0.310132in}{0.345370in}}%
\pgfpathlineto{\pgfqpoint{0.325788in}{0.345370in}}%
\pgfpathlineto{\pgfqpoint{0.341445in}{0.345370in}}%
\pgfpathlineto{\pgfqpoint{0.357101in}{0.345370in}}%
\pgfpathlineto{\pgfqpoint{0.372758in}{0.345370in}}%
\pgfpathlineto{\pgfqpoint{0.388415in}{0.345370in}}%
\pgfpathlineto{\pgfqpoint{0.404071in}{0.345370in}}%
\pgfpathlineto{\pgfqpoint{0.419728in}{0.345370in}}%
\pgfpathlineto{\pgfqpoint{0.435384in}{0.345370in}}%
\pgfpathlineto{\pgfqpoint{0.451041in}{0.345370in}}%
\pgfpathlineto{\pgfqpoint{0.451397in}{0.345370in}}%
\pgfpathlineto{\pgfqpoint{0.451041in}{0.351484in}}%
\pgfpathlineto{\pgfqpoint{0.450570in}{0.358981in}}%
\pgfpathlineto{\pgfqpoint{0.448011in}{0.372592in}}%
\pgfpathlineto{\pgfqpoint{0.443777in}{0.386203in}}%
\pgfpathlineto{\pgfqpoint{0.437913in}{0.399814in}}%
\pgfpathlineto{\pgfqpoint{0.435384in}{0.404429in}}%
\pgfpathlineto{\pgfqpoint{0.430038in}{0.413425in}}%
\pgfpathlineto{\pgfqpoint{0.420308in}{0.427036in}}%
\pgfpathlineto{\pgfqpoint{0.419728in}{0.427733in}}%
\pgfpathlineto{\pgfqpoint{0.407938in}{0.440648in}}%
\pgfpathlineto{\pgfqpoint{0.404071in}{0.444386in}}%
\pgfpathlineto{\pgfqpoint{0.392715in}{0.454259in}}%
\pgfpathlineto{\pgfqpoint{0.388415in}{0.457620in}}%
\pgfpathlineto{\pgfqpoint{0.373559in}{0.467870in}}%
\pgfpathlineto{\pgfqpoint{0.372758in}{0.468374in}}%
\pgfpathlineto{\pgfqpoint{0.357101in}{0.476833in}}%
\pgfpathlineto{\pgfqpoint{0.346754in}{0.481481in}}%
\pgfpathlineto{\pgfqpoint{0.341445in}{0.483679in}}%
\pgfpathlineto{\pgfqpoint{0.325788in}{0.488777in}}%
\pgfpathlineto{\pgfqpoint{0.310132in}{0.492458in}}%
\pgfpathlineto{\pgfqpoint{0.294475in}{0.494682in}}%
\pgfpathlineto{\pgfqpoint{0.285852in}{0.495092in}}%
\pgfpathlineto{\pgfqpoint{0.278819in}{0.495402in}}%
\pgfpathlineto{\pgfqpoint{0.278819in}{0.495092in}}%
\pgfpathlineto{\pgfqpoint{0.278819in}{0.481481in}}%
\pgfpathlineto{\pgfqpoint{0.278819in}{0.467870in}}%
\pgfpathlineto{\pgfqpoint{0.278819in}{0.454259in}}%
\pgfpathlineto{\pgfqpoint{0.278819in}{0.440648in}}%
\pgfpathlineto{\pgfqpoint{0.278819in}{0.427036in}}%
\pgfpathlineto{\pgfqpoint{0.278819in}{0.413425in}}%
\pgfpathlineto{\pgfqpoint{0.278819in}{0.399814in}}%
\pgfpathlineto{\pgfqpoint{0.278819in}{0.386203in}}%
\pgfpathlineto{\pgfqpoint{0.278819in}{0.372592in}}%
\pgfpathlineto{\pgfqpoint{0.278819in}{0.358981in}}%
\pgfpathlineto{\pgfqpoint{0.278819in}{0.345370in}}%
\pgfpathlineto{\pgfqpoint{0.294475in}{0.345370in}}%
\pgfpathclose%
\pgfpathmoveto{\pgfqpoint{1.656596in}{0.345370in}}%
\pgfpathlineto{\pgfqpoint{1.672253in}{0.345370in}}%
\pgfpathlineto{\pgfqpoint{1.687910in}{0.345370in}}%
\pgfpathlineto{\pgfqpoint{1.703566in}{0.345370in}}%
\pgfpathlineto{\pgfqpoint{1.719223in}{0.345370in}}%
\pgfpathlineto{\pgfqpoint{1.734879in}{0.345370in}}%
\pgfpathlineto{\pgfqpoint{1.750536in}{0.345370in}}%
\pgfpathlineto{\pgfqpoint{1.766192in}{0.345370in}}%
\pgfpathlineto{\pgfqpoint{1.781849in}{0.345370in}}%
\pgfpathlineto{\pgfqpoint{1.797505in}{0.345370in}}%
\pgfpathlineto{\pgfqpoint{1.813162in}{0.345370in}}%
\pgfpathlineto{\pgfqpoint{1.828819in}{0.345370in}}%
\pgfpathlineto{\pgfqpoint{1.828819in}{0.358981in}}%
\pgfpathlineto{\pgfqpoint{1.828819in}{0.372592in}}%
\pgfpathlineto{\pgfqpoint{1.828819in}{0.386203in}}%
\pgfpathlineto{\pgfqpoint{1.828819in}{0.399814in}}%
\pgfpathlineto{\pgfqpoint{1.828819in}{0.413425in}}%
\pgfpathlineto{\pgfqpoint{1.828819in}{0.427036in}}%
\pgfpathlineto{\pgfqpoint{1.828819in}{0.440648in}}%
\pgfpathlineto{\pgfqpoint{1.828819in}{0.454259in}}%
\pgfpathlineto{\pgfqpoint{1.828819in}{0.467870in}}%
\pgfpathlineto{\pgfqpoint{1.828819in}{0.481481in}}%
\pgfpathlineto{\pgfqpoint{1.828819in}{0.495092in}}%
\pgfpathlineto{\pgfqpoint{1.828819in}{0.495402in}}%
\pgfpathlineto{\pgfqpoint{1.821785in}{0.495092in}}%
\pgfpathlineto{\pgfqpoint{1.813162in}{0.494682in}}%
\pgfpathlineto{\pgfqpoint{1.797505in}{0.492458in}}%
\pgfpathlineto{\pgfqpoint{1.781849in}{0.488777in}}%
\pgfpathlineto{\pgfqpoint{1.766192in}{0.483679in}}%
\pgfpathlineto{\pgfqpoint{1.760883in}{0.481481in}}%
\pgfpathlineto{\pgfqpoint{1.750536in}{0.476833in}}%
\pgfpathlineto{\pgfqpoint{1.734879in}{0.468374in}}%
\pgfpathlineto{\pgfqpoint{1.734078in}{0.467870in}}%
\pgfpathlineto{\pgfqpoint{1.719223in}{0.457620in}}%
\pgfpathlineto{\pgfqpoint{1.714922in}{0.454259in}}%
\pgfpathlineto{\pgfqpoint{1.703566in}{0.444386in}}%
\pgfpathlineto{\pgfqpoint{1.699699in}{0.440648in}}%
\pgfpathlineto{\pgfqpoint{1.687910in}{0.427733in}}%
\pgfpathlineto{\pgfqpoint{1.687329in}{0.427036in}}%
\pgfpathlineto{\pgfqpoint{1.677600in}{0.413425in}}%
\pgfpathlineto{\pgfqpoint{1.672253in}{0.404429in}}%
\pgfpathlineto{\pgfqpoint{1.669724in}{0.399814in}}%
\pgfpathlineto{\pgfqpoint{1.663860in}{0.386203in}}%
\pgfpathlineto{\pgfqpoint{1.659626in}{0.372592in}}%
\pgfpathlineto{\pgfqpoint{1.657068in}{0.358981in}}%
\pgfpathlineto{\pgfqpoint{1.656596in}{0.351484in}}%
\pgfpathlineto{\pgfqpoint{1.656240in}{0.345370in}}%
\pgfpathlineto{\pgfqpoint{1.656596in}{0.345370in}}%
\pgfpathclose%
\pgfpathmoveto{\pgfqpoint{0.285852in}{1.543148in}}%
\pgfpathlineto{\pgfqpoint{0.294475in}{1.543557in}}%
\pgfpathlineto{\pgfqpoint{0.310132in}{1.545782in}}%
\pgfpathlineto{\pgfqpoint{0.325788in}{1.549463in}}%
\pgfpathlineto{\pgfqpoint{0.341445in}{1.554560in}}%
\pgfpathlineto{\pgfqpoint{0.346754in}{1.556759in}}%
\pgfpathlineto{\pgfqpoint{0.357101in}{1.561407in}}%
\pgfpathlineto{\pgfqpoint{0.372758in}{1.569865in}}%
\pgfpathlineto{\pgfqpoint{0.373559in}{1.570370in}}%
\pgfpathlineto{\pgfqpoint{0.388415in}{1.580619in}}%
\pgfpathlineto{\pgfqpoint{0.392715in}{1.583981in}}%
\pgfpathlineto{\pgfqpoint{0.404071in}{1.593853in}}%
\pgfpathlineto{\pgfqpoint{0.407938in}{1.597592in}}%
\pgfpathlineto{\pgfqpoint{0.419728in}{1.610506in}}%
\pgfpathlineto{\pgfqpoint{0.420308in}{1.611203in}}%
\pgfpathlineto{\pgfqpoint{0.430038in}{1.624814in}}%
\pgfpathlineto{\pgfqpoint{0.435384in}{1.633810in}}%
\pgfpathlineto{\pgfqpoint{0.437913in}{1.638425in}}%
\pgfpathlineto{\pgfqpoint{0.443777in}{1.652036in}}%
\pgfpathlineto{\pgfqpoint{0.448011in}{1.665648in}}%
\pgfpathlineto{\pgfqpoint{0.450570in}{1.679259in}}%
\pgfpathlineto{\pgfqpoint{0.451041in}{1.686755in}}%
\pgfpathlineto{\pgfqpoint{0.451397in}{1.692870in}}%
\pgfpathlineto{\pgfqpoint{0.451041in}{1.692870in}}%
\pgfpathlineto{\pgfqpoint{0.435384in}{1.692870in}}%
\pgfpathlineto{\pgfqpoint{0.419728in}{1.692870in}}%
\pgfpathlineto{\pgfqpoint{0.404071in}{1.692870in}}%
\pgfpathlineto{\pgfqpoint{0.388415in}{1.692870in}}%
\pgfpathlineto{\pgfqpoint{0.372758in}{1.692870in}}%
\pgfpathlineto{\pgfqpoint{0.357101in}{1.692870in}}%
\pgfpathlineto{\pgfqpoint{0.341445in}{1.692870in}}%
\pgfpathlineto{\pgfqpoint{0.325788in}{1.692870in}}%
\pgfpathlineto{\pgfqpoint{0.310132in}{1.692870in}}%
\pgfpathlineto{\pgfqpoint{0.294475in}{1.692870in}}%
\pgfpathlineto{\pgfqpoint{0.278819in}{1.692870in}}%
\pgfpathlineto{\pgfqpoint{0.278819in}{1.679259in}}%
\pgfpathlineto{\pgfqpoint{0.278819in}{1.665648in}}%
\pgfpathlineto{\pgfqpoint{0.278819in}{1.652036in}}%
\pgfpathlineto{\pgfqpoint{0.278819in}{1.638425in}}%
\pgfpathlineto{\pgfqpoint{0.278819in}{1.624814in}}%
\pgfpathlineto{\pgfqpoint{0.278819in}{1.611203in}}%
\pgfpathlineto{\pgfqpoint{0.278819in}{1.597592in}}%
\pgfpathlineto{\pgfqpoint{0.278819in}{1.583981in}}%
\pgfpathlineto{\pgfqpoint{0.278819in}{1.570370in}}%
\pgfpathlineto{\pgfqpoint{0.278819in}{1.556759in}}%
\pgfpathlineto{\pgfqpoint{0.278819in}{1.543148in}}%
\pgfpathlineto{\pgfqpoint{0.278819in}{1.542838in}}%
\pgfpathlineto{\pgfqpoint{0.285852in}{1.543148in}}%
\pgfpathclose%
\pgfpathmoveto{\pgfqpoint{1.828819in}{1.542838in}}%
\pgfpathlineto{\pgfqpoint{1.828819in}{1.543148in}}%
\pgfpathlineto{\pgfqpoint{1.828819in}{1.556759in}}%
\pgfpathlineto{\pgfqpoint{1.828819in}{1.570370in}}%
\pgfpathlineto{\pgfqpoint{1.828819in}{1.583981in}}%
\pgfpathlineto{\pgfqpoint{1.828819in}{1.597592in}}%
\pgfpathlineto{\pgfqpoint{1.828819in}{1.611203in}}%
\pgfpathlineto{\pgfqpoint{1.828819in}{1.624814in}}%
\pgfpathlineto{\pgfqpoint{1.828819in}{1.638425in}}%
\pgfpathlineto{\pgfqpoint{1.828819in}{1.652036in}}%
\pgfpathlineto{\pgfqpoint{1.828819in}{1.665648in}}%
\pgfpathlineto{\pgfqpoint{1.828819in}{1.679259in}}%
\pgfpathlineto{\pgfqpoint{1.828819in}{1.692870in}}%
\pgfpathlineto{\pgfqpoint{1.813162in}{1.692870in}}%
\pgfpathlineto{\pgfqpoint{1.797505in}{1.692870in}}%
\pgfpathlineto{\pgfqpoint{1.781849in}{1.692870in}}%
\pgfpathlineto{\pgfqpoint{1.766192in}{1.692870in}}%
\pgfpathlineto{\pgfqpoint{1.750536in}{1.692870in}}%
\pgfpathlineto{\pgfqpoint{1.734879in}{1.692870in}}%
\pgfpathlineto{\pgfqpoint{1.719223in}{1.692870in}}%
\pgfpathlineto{\pgfqpoint{1.703566in}{1.692870in}}%
\pgfpathlineto{\pgfqpoint{1.687910in}{1.692870in}}%
\pgfpathlineto{\pgfqpoint{1.672253in}{1.692870in}}%
\pgfpathlineto{\pgfqpoint{1.656596in}{1.692870in}}%
\pgfpathlineto{\pgfqpoint{1.656240in}{1.692870in}}%
\pgfpathlineto{\pgfqpoint{1.656596in}{1.686755in}}%
\pgfpathlineto{\pgfqpoint{1.657068in}{1.679259in}}%
\pgfpathlineto{\pgfqpoint{1.659626in}{1.665648in}}%
\pgfpathlineto{\pgfqpoint{1.663860in}{1.652036in}}%
\pgfpathlineto{\pgfqpoint{1.669724in}{1.638425in}}%
\pgfpathlineto{\pgfqpoint{1.672253in}{1.633810in}}%
\pgfpathlineto{\pgfqpoint{1.677600in}{1.624814in}}%
\pgfpathlineto{\pgfqpoint{1.687329in}{1.611203in}}%
\pgfpathlineto{\pgfqpoint{1.687910in}{1.610506in}}%
\pgfpathlineto{\pgfqpoint{1.699699in}{1.597592in}}%
\pgfpathlineto{\pgfqpoint{1.703566in}{1.593853in}}%
\pgfpathlineto{\pgfqpoint{1.714922in}{1.583981in}}%
\pgfpathlineto{\pgfqpoint{1.719223in}{1.580619in}}%
\pgfpathlineto{\pgfqpoint{1.734078in}{1.570370in}}%
\pgfpathlineto{\pgfqpoint{1.734879in}{1.569865in}}%
\pgfpathlineto{\pgfqpoint{1.750536in}{1.561407in}}%
\pgfpathlineto{\pgfqpoint{1.760883in}{1.556759in}}%
\pgfpathlineto{\pgfqpoint{1.766192in}{1.554560in}}%
\pgfpathlineto{\pgfqpoint{1.781849in}{1.549463in}}%
\pgfpathlineto{\pgfqpoint{1.797505in}{1.545782in}}%
\pgfpathlineto{\pgfqpoint{1.813162in}{1.543557in}}%
\pgfpathlineto{\pgfqpoint{1.821785in}{1.543148in}}%
\pgfpathlineto{\pgfqpoint{1.828819in}{1.542838in}}%
\pgfpathclose%
\pgfusepath{fill}%
\end{pgfscope}%
\begin{pgfscope}%
\pgfsetbuttcap%
\pgfsetroundjoin%
\definecolor{currentfill}{rgb}{0.000000,0.000000,0.000000}%
\pgfsetfillcolor{currentfill}%
\pgfsetlinewidth{0.803000pt}%
\definecolor{currentstroke}{rgb}{0.000000,0.000000,0.000000}%
\pgfsetstrokecolor{currentstroke}%
\pgfsetdash{}{0pt}%
\pgfsys@defobject{currentmarker}{\pgfqpoint{0.000000in}{-0.048611in}}{\pgfqpoint{0.000000in}{0.000000in}}{%
\pgfpathmoveto{\pgfqpoint{0.000000in}{0.000000in}}%
\pgfpathlineto{\pgfqpoint{0.000000in}{-0.048611in}}%
\pgfusepath{stroke,fill}%
}%
\begin{pgfscope}%
\pgfsys@transformshift{0.278819in}{0.345370in}%
\pgfsys@useobject{currentmarker}{}%
\end{pgfscope}%
\end{pgfscope}%
\begin{pgfscope}%
\definecolor{textcolor}{rgb}{0.000000,0.000000,0.000000}%
\pgfsetstrokecolor{textcolor}%
\pgfsetfillcolor{textcolor}%
\pgftext[x=0.278819in,y=0.248148in,,top]{\color{textcolor}{\rmfamily\fontsize{12.000000}{14.400000}\selectfont\catcode`\^=\active\def^{\ifmmode\sp\else\^{}\fi}\catcode`\%=\active\def%{\%}$\mathdefault{0}$}}%
\end{pgfscope}%
\begin{pgfscope}%
\pgfsetbuttcap%
\pgfsetroundjoin%
\definecolor{currentfill}{rgb}{0.000000,0.000000,0.000000}%
\pgfsetfillcolor{currentfill}%
\pgfsetlinewidth{0.803000pt}%
\definecolor{currentstroke}{rgb}{0.000000,0.000000,0.000000}%
\pgfsetstrokecolor{currentstroke}%
\pgfsetdash{}{0pt}%
\pgfsys@defobject{currentmarker}{\pgfqpoint{0.000000in}{-0.048611in}}{\pgfqpoint{0.000000in}{0.000000in}}{%
\pgfpathmoveto{\pgfqpoint{0.000000in}{0.000000in}}%
\pgfpathlineto{\pgfqpoint{0.000000in}{-0.048611in}}%
\pgfusepath{stroke,fill}%
}%
\begin{pgfscope}%
\pgfsys@transformshift{0.795485in}{0.345370in}%
\pgfsys@useobject{currentmarker}{}%
\end{pgfscope}%
\end{pgfscope}%
\begin{pgfscope}%
\definecolor{textcolor}{rgb}{0.000000,0.000000,0.000000}%
\pgfsetstrokecolor{textcolor}%
\pgfsetfillcolor{textcolor}%
\pgftext[x=0.795485in,y=0.248148in,,top]{\color{textcolor}{\rmfamily\fontsize{12.000000}{14.400000}\selectfont\catcode`\^=\active\def^{\ifmmode\sp\else\^{}\fi}\catcode`\%=\active\def%{\%}$\mathdefault{2}$}}%
\end{pgfscope}%
\begin{pgfscope}%
\pgfsetbuttcap%
\pgfsetroundjoin%
\definecolor{currentfill}{rgb}{0.000000,0.000000,0.000000}%
\pgfsetfillcolor{currentfill}%
\pgfsetlinewidth{0.803000pt}%
\definecolor{currentstroke}{rgb}{0.000000,0.000000,0.000000}%
\pgfsetstrokecolor{currentstroke}%
\pgfsetdash{}{0pt}%
\pgfsys@defobject{currentmarker}{\pgfqpoint{0.000000in}{-0.048611in}}{\pgfqpoint{0.000000in}{0.000000in}}{%
\pgfpathmoveto{\pgfqpoint{0.000000in}{0.000000in}}%
\pgfpathlineto{\pgfqpoint{0.000000in}{-0.048611in}}%
\pgfusepath{stroke,fill}%
}%
\begin{pgfscope}%
\pgfsys@transformshift{1.312152in}{0.345370in}%
\pgfsys@useobject{currentmarker}{}%
\end{pgfscope}%
\end{pgfscope}%
\begin{pgfscope}%
\definecolor{textcolor}{rgb}{0.000000,0.000000,0.000000}%
\pgfsetstrokecolor{textcolor}%
\pgfsetfillcolor{textcolor}%
\pgftext[x=1.312152in,y=0.248148in,,top]{\color{textcolor}{\rmfamily\fontsize{12.000000}{14.400000}\selectfont\catcode`\^=\active\def^{\ifmmode\sp\else\^{}\fi}\catcode`\%=\active\def%{\%}$\mathdefault{4}$}}%
\end{pgfscope}%
\begin{pgfscope}%
\pgfsetbuttcap%
\pgfsetroundjoin%
\definecolor{currentfill}{rgb}{0.000000,0.000000,0.000000}%
\pgfsetfillcolor{currentfill}%
\pgfsetlinewidth{0.803000pt}%
\definecolor{currentstroke}{rgb}{0.000000,0.000000,0.000000}%
\pgfsetstrokecolor{currentstroke}%
\pgfsetdash{}{0pt}%
\pgfsys@defobject{currentmarker}{\pgfqpoint{0.000000in}{-0.048611in}}{\pgfqpoint{0.000000in}{0.000000in}}{%
\pgfpathmoveto{\pgfqpoint{0.000000in}{0.000000in}}%
\pgfpathlineto{\pgfqpoint{0.000000in}{-0.048611in}}%
\pgfusepath{stroke,fill}%
}%
\begin{pgfscope}%
\pgfsys@transformshift{1.828819in}{0.345370in}%
\pgfsys@useobject{currentmarker}{}%
\end{pgfscope}%
\end{pgfscope}%
\begin{pgfscope}%
\definecolor{textcolor}{rgb}{0.000000,0.000000,0.000000}%
\pgfsetstrokecolor{textcolor}%
\pgfsetfillcolor{textcolor}%
\pgftext[x=1.828819in,y=0.248148in,,top]{\color{textcolor}{\rmfamily\fontsize{12.000000}{14.400000}\selectfont\catcode`\^=\active\def^{\ifmmode\sp\else\^{}\fi}\catcode`\%=\active\def%{\%}$\mathdefault{6}$}}%
\end{pgfscope}%
\begin{pgfscope}%
\pgfsetbuttcap%
\pgfsetroundjoin%
\definecolor{currentfill}{rgb}{0.000000,0.000000,0.000000}%
\pgfsetfillcolor{currentfill}%
\pgfsetlinewidth{0.803000pt}%
\definecolor{currentstroke}{rgb}{0.000000,0.000000,0.000000}%
\pgfsetstrokecolor{currentstroke}%
\pgfsetdash{}{0pt}%
\pgfsys@defobject{currentmarker}{\pgfqpoint{-0.048611in}{0.000000in}}{\pgfqpoint{-0.000000in}{0.000000in}}{%
\pgfpathmoveto{\pgfqpoint{-0.000000in}{0.000000in}}%
\pgfpathlineto{\pgfqpoint{-0.048611in}{0.000000in}}%
\pgfusepath{stroke,fill}%
}%
\begin{pgfscope}%
\pgfsys@transformshift{0.278819in}{0.345370in}%
\pgfsys@useobject{currentmarker}{}%
\end{pgfscope}%
\end{pgfscope}%
\begin{pgfscope}%
\definecolor{textcolor}{rgb}{0.000000,0.000000,0.000000}%
\pgfsetstrokecolor{textcolor}%
\pgfsetfillcolor{textcolor}%
\pgftext[x=0.100000in, y=0.287500in, left, base]{\color{textcolor}{\rmfamily\fontsize{12.000000}{14.400000}\selectfont\catcode`\^=\active\def^{\ifmmode\sp\else\^{}\fi}\catcode`\%=\active\def%{\%}$\mathdefault{0}$}}%
\end{pgfscope}%
\begin{pgfscope}%
\pgfsetbuttcap%
\pgfsetroundjoin%
\definecolor{currentfill}{rgb}{0.000000,0.000000,0.000000}%
\pgfsetfillcolor{currentfill}%
\pgfsetlinewidth{0.803000pt}%
\definecolor{currentstroke}{rgb}{0.000000,0.000000,0.000000}%
\pgfsetstrokecolor{currentstroke}%
\pgfsetdash{}{0pt}%
\pgfsys@defobject{currentmarker}{\pgfqpoint{-0.048611in}{0.000000in}}{\pgfqpoint{-0.000000in}{0.000000in}}{%
\pgfpathmoveto{\pgfqpoint{-0.000000in}{0.000000in}}%
\pgfpathlineto{\pgfqpoint{-0.048611in}{0.000000in}}%
\pgfusepath{stroke,fill}%
}%
\begin{pgfscope}%
\pgfsys@transformshift{0.278819in}{0.794536in}%
\pgfsys@useobject{currentmarker}{}%
\end{pgfscope}%
\end{pgfscope}%
\begin{pgfscope}%
\definecolor{textcolor}{rgb}{0.000000,0.000000,0.000000}%
\pgfsetstrokecolor{textcolor}%
\pgfsetfillcolor{textcolor}%
\pgftext[x=0.100000in, y=0.736666in, left, base]{\color{textcolor}{\rmfamily\fontsize{12.000000}{14.400000}\selectfont\catcode`\^=\active\def^{\ifmmode\sp\else\^{}\fi}\catcode`\%=\active\def%{\%}$\mathdefault{2}$}}%
\end{pgfscope}%
\begin{pgfscope}%
\pgfsetbuttcap%
\pgfsetroundjoin%
\definecolor{currentfill}{rgb}{0.000000,0.000000,0.000000}%
\pgfsetfillcolor{currentfill}%
\pgfsetlinewidth{0.803000pt}%
\definecolor{currentstroke}{rgb}{0.000000,0.000000,0.000000}%
\pgfsetstrokecolor{currentstroke}%
\pgfsetdash{}{0pt}%
\pgfsys@defobject{currentmarker}{\pgfqpoint{-0.048611in}{0.000000in}}{\pgfqpoint{-0.000000in}{0.000000in}}{%
\pgfpathmoveto{\pgfqpoint{-0.000000in}{0.000000in}}%
\pgfpathlineto{\pgfqpoint{-0.048611in}{0.000000in}}%
\pgfusepath{stroke,fill}%
}%
\begin{pgfscope}%
\pgfsys@transformshift{0.278819in}{1.243703in}%
\pgfsys@useobject{currentmarker}{}%
\end{pgfscope}%
\end{pgfscope}%
\begin{pgfscope}%
\definecolor{textcolor}{rgb}{0.000000,0.000000,0.000000}%
\pgfsetstrokecolor{textcolor}%
\pgfsetfillcolor{textcolor}%
\pgftext[x=0.100000in, y=1.185833in, left, base]{\color{textcolor}{\rmfamily\fontsize{12.000000}{14.400000}\selectfont\catcode`\^=\active\def^{\ifmmode\sp\else\^{}\fi}\catcode`\%=\active\def%{\%}$\mathdefault{4}$}}%
\end{pgfscope}%
\begin{pgfscope}%
\pgfsetbuttcap%
\pgfsetroundjoin%
\definecolor{currentfill}{rgb}{0.000000,0.000000,0.000000}%
\pgfsetfillcolor{currentfill}%
\pgfsetlinewidth{0.803000pt}%
\definecolor{currentstroke}{rgb}{0.000000,0.000000,0.000000}%
\pgfsetstrokecolor{currentstroke}%
\pgfsetdash{}{0pt}%
\pgfsys@defobject{currentmarker}{\pgfqpoint{-0.048611in}{0.000000in}}{\pgfqpoint{-0.000000in}{0.000000in}}{%
\pgfpathmoveto{\pgfqpoint{-0.000000in}{0.000000in}}%
\pgfpathlineto{\pgfqpoint{-0.048611in}{0.000000in}}%
\pgfusepath{stroke,fill}%
}%
\begin{pgfscope}%
\pgfsys@transformshift{0.278819in}{1.692870in}%
\pgfsys@useobject{currentmarker}{}%
\end{pgfscope}%
\end{pgfscope}%
\begin{pgfscope}%
\definecolor{textcolor}{rgb}{0.000000,0.000000,0.000000}%
\pgfsetstrokecolor{textcolor}%
\pgfsetfillcolor{textcolor}%
\pgftext[x=0.100000in, y=1.635000in, left, base]{\color{textcolor}{\rmfamily\fontsize{12.000000}{14.400000}\selectfont\catcode`\^=\active\def^{\ifmmode\sp\else\^{}\fi}\catcode`\%=\active\def%{\%}$\mathdefault{6}$}}%
\end{pgfscope}%
\begin{pgfscope}%
\pgfsetrectcap%
\pgfsetmiterjoin%
\pgfsetlinewidth{0.803000pt}%
\definecolor{currentstroke}{rgb}{0.000000,0.000000,0.000000}%
\pgfsetstrokecolor{currentstroke}%
\pgfsetdash{}{0pt}%
\pgfpathmoveto{\pgfqpoint{0.278819in}{0.345370in}}%
\pgfpathlineto{\pgfqpoint{0.278819in}{1.692870in}}%
\pgfusepath{stroke}%
\end{pgfscope}%
\begin{pgfscope}%
\pgfsetrectcap%
\pgfsetmiterjoin%
\pgfsetlinewidth{0.803000pt}%
\definecolor{currentstroke}{rgb}{0.000000,0.000000,0.000000}%
\pgfsetstrokecolor{currentstroke}%
\pgfsetdash{}{0pt}%
\pgfpathmoveto{\pgfqpoint{1.828819in}{0.345370in}}%
\pgfpathlineto{\pgfqpoint{1.828819in}{1.692870in}}%
\pgfusepath{stroke}%
\end{pgfscope}%
\begin{pgfscope}%
\pgfsetrectcap%
\pgfsetmiterjoin%
\pgfsetlinewidth{0.803000pt}%
\definecolor{currentstroke}{rgb}{0.000000,0.000000,0.000000}%
\pgfsetstrokecolor{currentstroke}%
\pgfsetdash{}{0pt}%
\pgfpathmoveto{\pgfqpoint{0.278819in}{0.345370in}}%
\pgfpathlineto{\pgfqpoint{1.828819in}{0.345370in}}%
\pgfusepath{stroke}%
\end{pgfscope}%
\begin{pgfscope}%
\pgfsetrectcap%
\pgfsetmiterjoin%
\pgfsetlinewidth{0.803000pt}%
\definecolor{currentstroke}{rgb}{0.000000,0.000000,0.000000}%
\pgfsetstrokecolor{currentstroke}%
\pgfsetdash{}{0pt}%
\pgfpathmoveto{\pgfqpoint{0.278819in}{1.692870in}}%
\pgfpathlineto{\pgfqpoint{1.828819in}{1.692870in}}%
\pgfusepath{stroke}%
\end{pgfscope}%
\end{pgfpicture}%
\makeatother%
\endgroup%

        \caption{$n_c=1$}
        \label{fig:gaussian-well-1}
    \end{subfigure}
    \begin{subfigure}[b]{0.32\columnwidth}
        %% Creator: Matplotlib, PGF backend
%%
%% To include the figure in your LaTeX document, write
%%   \input{<filename>.pgf}
%%
%% Make sure the required packages are loaded in your preamble
%%   \usepackage{pgf}
%%
%% Also ensure that all the required font packages are loaded; for instance,
%% the lmodern package is sometimes necessary when using math font.
%%   \usepackage{lmodern}
%%
%% Figures using additional raster images can only be included by \input if
%% they are in the same directory as the main LaTeX file. For loading figures
%% from other directories you can use the `import` package
%%   \usepackage{import}
%%
%% and then include the figures with
%%   \import{<path to file>}{<filename>.pgf}
%%
%% Matplotlib used the following preamble
%%   \def\mathdefault#1{#1}
%%   \everymath=\expandafter{\the\everymath\displaystyle}
%%   \IfFileExists{scrextend.sty}{
%%     \usepackage[fontsize=12.000000pt]{scrextend}
%%   }{
%%     \renewcommand{\normalsize}{\fontsize{12.000000}{14.400000}\selectfont}
%%     \normalsize
%%   }
%%   
%%   \ifdefined\pdftexversion\else  % non-pdftex case.
%%     \usepackage{fontspec}
%%     \setmainfont{DejaVuSerif.ttf}[Path=\detokenize{/opt/hostedtoolcache/Python/3.12.9/x64/lib/python3.12/site-packages/matplotlib/mpl-data/fonts/ttf/}]
%%     \setsansfont{DejaVuSans.ttf}[Path=\detokenize{/opt/hostedtoolcache/Python/3.12.9/x64/lib/python3.12/site-packages/matplotlib/mpl-data/fonts/ttf/}]
%%     \setmonofont{DejaVuSansMono.ttf}[Path=\detokenize{/opt/hostedtoolcache/Python/3.12.9/x64/lib/python3.12/site-packages/matplotlib/mpl-data/fonts/ttf/}]
%%   \fi
%%   \makeatletter\@ifpackageloaded{underscore}{}{\usepackage[strings]{underscore}}\makeatother
%%
\begingroup%
\makeatletter%
\begin{pgfpicture}%
\pgfpathrectangle{\pgfpointorigin}{\pgfqpoint{2.010415in}{1.792870in}}%
\pgfusepath{use as bounding box, clip}%
\begin{pgfscope}%
\pgfsetbuttcap%
\pgfsetmiterjoin%
\definecolor{currentfill}{rgb}{1.000000,1.000000,1.000000}%
\pgfsetfillcolor{currentfill}%
\pgfsetlinewidth{0.000000pt}%
\definecolor{currentstroke}{rgb}{1.000000,1.000000,1.000000}%
\pgfsetstrokecolor{currentstroke}%
\pgfsetdash{}{0pt}%
\pgfpathmoveto{\pgfqpoint{0.000000in}{0.000000in}}%
\pgfpathlineto{\pgfqpoint{2.010415in}{0.000000in}}%
\pgfpathlineto{\pgfqpoint{2.010415in}{1.792870in}}%
\pgfpathlineto{\pgfqpoint{0.000000in}{1.792870in}}%
\pgfpathlineto{\pgfqpoint{0.000000in}{0.000000in}}%
\pgfpathclose%
\pgfusepath{fill}%
\end{pgfscope}%
\begin{pgfscope}%
\pgfsetbuttcap%
\pgfsetmiterjoin%
\definecolor{currentfill}{rgb}{1.000000,1.000000,1.000000}%
\pgfsetfillcolor{currentfill}%
\pgfsetlinewidth{0.000000pt}%
\definecolor{currentstroke}{rgb}{0.000000,0.000000,0.000000}%
\pgfsetstrokecolor{currentstroke}%
\pgfsetstrokeopacity{0.000000}%
\pgfsetdash{}{0pt}%
\pgfpathmoveto{\pgfqpoint{0.360415in}{0.345370in}}%
\pgfpathlineto{\pgfqpoint{1.910415in}{0.345370in}}%
\pgfpathlineto{\pgfqpoint{1.910415in}{1.692870in}}%
\pgfpathlineto{\pgfqpoint{0.360415in}{1.692870in}}%
\pgfpathlineto{\pgfqpoint{0.360415in}{0.345370in}}%
\pgfpathclose%
\pgfusepath{fill}%
\end{pgfscope}%
\begin{pgfscope}%
\pgfpathrectangle{\pgfqpoint{0.360415in}{0.345370in}}{\pgfqpoint{1.550000in}{1.347500in}}%
\pgfusepath{clip}%
\pgfsetbuttcap%
\pgfsetroundjoin%
\definecolor{currentfill}{rgb}{0.972530,0.881250,0.144923}%
\pgfsetfillcolor{currentfill}%
\pgfsetlinewidth{0.000000pt}%
\definecolor{currentstroke}{rgb}{0.000000,0.000000,0.000000}%
\pgfsetstrokecolor{currentstroke}%
\pgfsetdash{}{0pt}%
\pgfpathmoveto{\pgfqpoint{0.736173in}{0.588220in}}%
\pgfpathlineto{\pgfqpoint{0.751829in}{0.587712in}}%
\pgfpathlineto{\pgfqpoint{0.767486in}{0.589236in}}%
\pgfpathlineto{\pgfqpoint{0.772492in}{0.590370in}}%
\pgfpathlineto{\pgfqpoint{0.783142in}{0.593075in}}%
\pgfpathlineto{\pgfqpoint{0.798799in}{0.599320in}}%
\pgfpathlineto{\pgfqpoint{0.807403in}{0.603981in}}%
\pgfpathlineto{\pgfqpoint{0.814455in}{0.608432in}}%
\pgfpathlineto{\pgfqpoint{0.826014in}{0.617592in}}%
\pgfpathlineto{\pgfqpoint{0.830112in}{0.621552in}}%
\pgfpathlineto{\pgfqpoint{0.838505in}{0.631203in}}%
\pgfpathlineto{\pgfqpoint{0.845769in}{0.642076in}}%
\pgfpathlineto{\pgfqpoint{0.847368in}{0.644814in}}%
\pgfpathlineto{\pgfqpoint{0.852909in}{0.658425in}}%
\pgfpathlineto{\pgfqpoint{0.855985in}{0.672036in}}%
\pgfpathlineto{\pgfqpoint{0.856601in}{0.685648in}}%
\pgfpathlineto{\pgfqpoint{0.854755in}{0.699259in}}%
\pgfpathlineto{\pgfqpoint{0.850447in}{0.712870in}}%
\pgfpathlineto{\pgfqpoint{0.845769in}{0.722292in}}%
\pgfpathlineto{\pgfqpoint{0.843393in}{0.726481in}}%
\pgfpathlineto{\pgfqpoint{0.832916in}{0.740092in}}%
\pgfpathlineto{\pgfqpoint{0.830112in}{0.742996in}}%
\pgfpathlineto{\pgfqpoint{0.817796in}{0.753703in}}%
\pgfpathlineto{\pgfqpoint{0.814455in}{0.756141in}}%
\pgfpathlineto{\pgfqpoint{0.798799in}{0.765249in}}%
\pgfpathlineto{\pgfqpoint{0.793980in}{0.767314in}}%
\pgfpathlineto{\pgfqpoint{0.783142in}{0.771382in}}%
\pgfpathlineto{\pgfqpoint{0.767486in}{0.775127in}}%
\pgfpathlineto{\pgfqpoint{0.751829in}{0.776731in}}%
\pgfpathlineto{\pgfqpoint{0.736173in}{0.776196in}}%
\pgfpathlineto{\pgfqpoint{0.720516in}{0.773522in}}%
\pgfpathlineto{\pgfqpoint{0.704859in}{0.768705in}}%
\pgfpathlineto{\pgfqpoint{0.701709in}{0.767314in}}%
\pgfpathlineto{\pgfqpoint{0.689203in}{0.760999in}}%
\pgfpathlineto{\pgfqpoint{0.678102in}{0.753703in}}%
\pgfpathlineto{\pgfqpoint{0.673546in}{0.750140in}}%
\pgfpathlineto{\pgfqpoint{0.663010in}{0.740092in}}%
\pgfpathlineto{\pgfqpoint{0.657890in}{0.733961in}}%
\pgfpathlineto{\pgfqpoint{0.652528in}{0.726481in}}%
\pgfpathlineto{\pgfqpoint{0.645345in}{0.712870in}}%
\pgfpathlineto{\pgfqpoint{0.642233in}{0.703611in}}%
\pgfpathlineto{\pgfqpoint{0.640929in}{0.699259in}}%
\pgfpathlineto{\pgfqpoint{0.639176in}{0.685648in}}%
\pgfpathlineto{\pgfqpoint{0.639760in}{0.672036in}}%
\pgfpathlineto{\pgfqpoint{0.642233in}{0.660513in}}%
\pgfpathlineto{\pgfqpoint{0.642735in}{0.658425in}}%
\pgfpathlineto{\pgfqpoint{0.648609in}{0.644814in}}%
\pgfpathlineto{\pgfqpoint{0.657102in}{0.631203in}}%
\pgfpathlineto{\pgfqpoint{0.657890in}{0.630228in}}%
\pgfpathlineto{\pgfqpoint{0.669793in}{0.617592in}}%
\pgfpathlineto{\pgfqpoint{0.673546in}{0.614329in}}%
\pgfpathlineto{\pgfqpoint{0.688081in}{0.603981in}}%
\pgfpathlineto{\pgfqpoint{0.689203in}{0.603296in}}%
\pgfpathlineto{\pgfqpoint{0.704859in}{0.595913in}}%
\pgfpathlineto{\pgfqpoint{0.720516in}{0.590806in}}%
\pgfpathlineto{\pgfqpoint{0.722917in}{0.590370in}}%
\pgfpathlineto{\pgfqpoint{0.736173in}{0.588220in}}%
\pgfpathclose%
\pgfpathmoveto{\pgfqpoint{1.503344in}{0.589236in}}%
\pgfpathlineto{\pgfqpoint{1.519001in}{0.587712in}}%
\pgfpathlineto{\pgfqpoint{1.534657in}{0.588220in}}%
\pgfpathlineto{\pgfqpoint{1.547913in}{0.590370in}}%
\pgfpathlineto{\pgfqpoint{1.550314in}{0.590806in}}%
\pgfpathlineto{\pgfqpoint{1.565971in}{0.595913in}}%
\pgfpathlineto{\pgfqpoint{1.581627in}{0.603296in}}%
\pgfpathlineto{\pgfqpoint{1.582749in}{0.603981in}}%
\pgfpathlineto{\pgfqpoint{1.597284in}{0.614329in}}%
\pgfpathlineto{\pgfqpoint{1.601037in}{0.617592in}}%
\pgfpathlineto{\pgfqpoint{1.612940in}{0.630228in}}%
\pgfpathlineto{\pgfqpoint{1.613728in}{0.631203in}}%
\pgfpathlineto{\pgfqpoint{1.622221in}{0.644814in}}%
\pgfpathlineto{\pgfqpoint{1.628095in}{0.658425in}}%
\pgfpathlineto{\pgfqpoint{1.628597in}{0.660513in}}%
\pgfpathlineto{\pgfqpoint{1.631070in}{0.672036in}}%
\pgfpathlineto{\pgfqpoint{1.631654in}{0.685648in}}%
\pgfpathlineto{\pgfqpoint{1.629901in}{0.699259in}}%
\pgfpathlineto{\pgfqpoint{1.628597in}{0.703611in}}%
\pgfpathlineto{\pgfqpoint{1.625485in}{0.712870in}}%
\pgfpathlineto{\pgfqpoint{1.618302in}{0.726481in}}%
\pgfpathlineto{\pgfqpoint{1.612940in}{0.733961in}}%
\pgfpathlineto{\pgfqpoint{1.607820in}{0.740092in}}%
\pgfpathlineto{\pgfqpoint{1.597284in}{0.750140in}}%
\pgfpathlineto{\pgfqpoint{1.592728in}{0.753703in}}%
\pgfpathlineto{\pgfqpoint{1.581627in}{0.760999in}}%
\pgfpathlineto{\pgfqpoint{1.569121in}{0.767314in}}%
\pgfpathlineto{\pgfqpoint{1.565971in}{0.768705in}}%
\pgfpathlineto{\pgfqpoint{1.550314in}{0.773522in}}%
\pgfpathlineto{\pgfqpoint{1.534657in}{0.776196in}}%
\pgfpathlineto{\pgfqpoint{1.519001in}{0.776731in}}%
\pgfpathlineto{\pgfqpoint{1.503344in}{0.775127in}}%
\pgfpathlineto{\pgfqpoint{1.487688in}{0.771382in}}%
\pgfpathlineto{\pgfqpoint{1.476850in}{0.767314in}}%
\pgfpathlineto{\pgfqpoint{1.472031in}{0.765249in}}%
\pgfpathlineto{\pgfqpoint{1.456375in}{0.756141in}}%
\pgfpathlineto{\pgfqpoint{1.453034in}{0.753703in}}%
\pgfpathlineto{\pgfqpoint{1.440718in}{0.742996in}}%
\pgfpathlineto{\pgfqpoint{1.437914in}{0.740092in}}%
\pgfpathlineto{\pgfqpoint{1.427437in}{0.726481in}}%
\pgfpathlineto{\pgfqpoint{1.425061in}{0.722292in}}%
\pgfpathlineto{\pgfqpoint{1.420383in}{0.712870in}}%
\pgfpathlineto{\pgfqpoint{1.416075in}{0.699259in}}%
\pgfpathlineto{\pgfqpoint{1.414229in}{0.685648in}}%
\pgfpathlineto{\pgfqpoint{1.414845in}{0.672036in}}%
\pgfpathlineto{\pgfqpoint{1.417921in}{0.658425in}}%
\pgfpathlineto{\pgfqpoint{1.423462in}{0.644814in}}%
\pgfpathlineto{\pgfqpoint{1.425061in}{0.642076in}}%
\pgfpathlineto{\pgfqpoint{1.432325in}{0.631203in}}%
\pgfpathlineto{\pgfqpoint{1.440718in}{0.621552in}}%
\pgfpathlineto{\pgfqpoint{1.444816in}{0.617592in}}%
\pgfpathlineto{\pgfqpoint{1.456375in}{0.608432in}}%
\pgfpathlineto{\pgfqpoint{1.463427in}{0.603981in}}%
\pgfpathlineto{\pgfqpoint{1.472031in}{0.599320in}}%
\pgfpathlineto{\pgfqpoint{1.487688in}{0.593075in}}%
\pgfpathlineto{\pgfqpoint{1.498338in}{0.590370in}}%
\pgfpathlineto{\pgfqpoint{1.503344in}{0.589236in}}%
\pgfpathclose%
\pgfpathmoveto{\pgfqpoint{0.704859in}{1.269535in}}%
\pgfpathlineto{\pgfqpoint{0.720516in}{1.264717in}}%
\pgfpathlineto{\pgfqpoint{0.736173in}{1.262043in}}%
\pgfpathlineto{\pgfqpoint{0.751829in}{1.261508in}}%
\pgfpathlineto{\pgfqpoint{0.767486in}{1.263113in}}%
\pgfpathlineto{\pgfqpoint{0.783142in}{1.266858in}}%
\pgfpathlineto{\pgfqpoint{0.793980in}{1.270925in}}%
\pgfpathlineto{\pgfqpoint{0.798799in}{1.272990in}}%
\pgfpathlineto{\pgfqpoint{0.814455in}{1.282099in}}%
\pgfpathlineto{\pgfqpoint{0.817796in}{1.284536in}}%
\pgfpathlineto{\pgfqpoint{0.830112in}{1.295243in}}%
\pgfpathlineto{\pgfqpoint{0.832916in}{1.298148in}}%
\pgfpathlineto{\pgfqpoint{0.843393in}{1.311759in}}%
\pgfpathlineto{\pgfqpoint{0.845769in}{1.315948in}}%
\pgfpathlineto{\pgfqpoint{0.850447in}{1.325370in}}%
\pgfpathlineto{\pgfqpoint{0.854755in}{1.338981in}}%
\pgfpathlineto{\pgfqpoint{0.856601in}{1.352592in}}%
\pgfpathlineto{\pgfqpoint{0.855985in}{1.366203in}}%
\pgfpathlineto{\pgfqpoint{0.852909in}{1.379814in}}%
\pgfpathlineto{\pgfqpoint{0.847368in}{1.393425in}}%
\pgfpathlineto{\pgfqpoint{0.845769in}{1.396164in}}%
\pgfpathlineto{\pgfqpoint{0.838505in}{1.407036in}}%
\pgfpathlineto{\pgfqpoint{0.830112in}{1.416687in}}%
\pgfpathlineto{\pgfqpoint{0.826014in}{1.420648in}}%
\pgfpathlineto{\pgfqpoint{0.814455in}{1.429808in}}%
\pgfpathlineto{\pgfqpoint{0.807403in}{1.434259in}}%
\pgfpathlineto{\pgfqpoint{0.798799in}{1.438920in}}%
\pgfpathlineto{\pgfqpoint{0.783142in}{1.445164in}}%
\pgfpathlineto{\pgfqpoint{0.772492in}{1.447870in}}%
\pgfpathlineto{\pgfqpoint{0.767486in}{1.449003in}}%
\pgfpathlineto{\pgfqpoint{0.751829in}{1.450527in}}%
\pgfpathlineto{\pgfqpoint{0.736173in}{1.450019in}}%
\pgfpathlineto{\pgfqpoint{0.722917in}{1.447870in}}%
\pgfpathlineto{\pgfqpoint{0.720516in}{1.447434in}}%
\pgfpathlineto{\pgfqpoint{0.704859in}{1.442327in}}%
\pgfpathlineto{\pgfqpoint{0.689203in}{1.434943in}}%
\pgfpathlineto{\pgfqpoint{0.688081in}{1.434259in}}%
\pgfpathlineto{\pgfqpoint{0.673546in}{1.423910in}}%
\pgfpathlineto{\pgfqpoint{0.669793in}{1.420648in}}%
\pgfpathlineto{\pgfqpoint{0.657890in}{1.408012in}}%
\pgfpathlineto{\pgfqpoint{0.657102in}{1.407036in}}%
\pgfpathlineto{\pgfqpoint{0.648609in}{1.393425in}}%
\pgfpathlineto{\pgfqpoint{0.642735in}{1.379814in}}%
\pgfpathlineto{\pgfqpoint{0.642233in}{1.377727in}}%
\pgfpathlineto{\pgfqpoint{0.639760in}{1.366203in}}%
\pgfpathlineto{\pgfqpoint{0.639176in}{1.352592in}}%
\pgfpathlineto{\pgfqpoint{0.640929in}{1.338981in}}%
\pgfpathlineto{\pgfqpoint{0.642233in}{1.334629in}}%
\pgfpathlineto{\pgfqpoint{0.645345in}{1.325370in}}%
\pgfpathlineto{\pgfqpoint{0.652528in}{1.311759in}}%
\pgfpathlineto{\pgfqpoint{0.657890in}{1.304279in}}%
\pgfpathlineto{\pgfqpoint{0.663010in}{1.298148in}}%
\pgfpathlineto{\pgfqpoint{0.673546in}{1.288099in}}%
\pgfpathlineto{\pgfqpoint{0.678102in}{1.284536in}}%
\pgfpathlineto{\pgfqpoint{0.689203in}{1.277240in}}%
\pgfpathlineto{\pgfqpoint{0.701709in}{1.270925in}}%
\pgfpathlineto{\pgfqpoint{0.704859in}{1.269535in}}%
\pgfpathclose%
\pgfpathmoveto{\pgfqpoint{1.487688in}{1.266858in}}%
\pgfpathlineto{\pgfqpoint{1.503344in}{1.263113in}}%
\pgfpathlineto{\pgfqpoint{1.519001in}{1.261508in}}%
\pgfpathlineto{\pgfqpoint{1.534657in}{1.262043in}}%
\pgfpathlineto{\pgfqpoint{1.550314in}{1.264717in}}%
\pgfpathlineto{\pgfqpoint{1.565971in}{1.269535in}}%
\pgfpathlineto{\pgfqpoint{1.569121in}{1.270925in}}%
\pgfpathlineto{\pgfqpoint{1.581627in}{1.277240in}}%
\pgfpathlineto{\pgfqpoint{1.592728in}{1.284536in}}%
\pgfpathlineto{\pgfqpoint{1.597284in}{1.288099in}}%
\pgfpathlineto{\pgfqpoint{1.607820in}{1.298148in}}%
\pgfpathlineto{\pgfqpoint{1.612940in}{1.304279in}}%
\pgfpathlineto{\pgfqpoint{1.618302in}{1.311759in}}%
\pgfpathlineto{\pgfqpoint{1.625485in}{1.325370in}}%
\pgfpathlineto{\pgfqpoint{1.628597in}{1.334629in}}%
\pgfpathlineto{\pgfqpoint{1.629901in}{1.338981in}}%
\pgfpathlineto{\pgfqpoint{1.631654in}{1.352592in}}%
\pgfpathlineto{\pgfqpoint{1.631070in}{1.366203in}}%
\pgfpathlineto{\pgfqpoint{1.628597in}{1.377727in}}%
\pgfpathlineto{\pgfqpoint{1.628095in}{1.379814in}}%
\pgfpathlineto{\pgfqpoint{1.622221in}{1.393425in}}%
\pgfpathlineto{\pgfqpoint{1.613728in}{1.407036in}}%
\pgfpathlineto{\pgfqpoint{1.612940in}{1.408012in}}%
\pgfpathlineto{\pgfqpoint{1.601037in}{1.420648in}}%
\pgfpathlineto{\pgfqpoint{1.597284in}{1.423910in}}%
\pgfpathlineto{\pgfqpoint{1.582749in}{1.434259in}}%
\pgfpathlineto{\pgfqpoint{1.581627in}{1.434943in}}%
\pgfpathlineto{\pgfqpoint{1.565971in}{1.442327in}}%
\pgfpathlineto{\pgfqpoint{1.550314in}{1.447434in}}%
\pgfpathlineto{\pgfqpoint{1.547913in}{1.447870in}}%
\pgfpathlineto{\pgfqpoint{1.534657in}{1.450019in}}%
\pgfpathlineto{\pgfqpoint{1.519001in}{1.450527in}}%
\pgfpathlineto{\pgfqpoint{1.503344in}{1.449003in}}%
\pgfpathlineto{\pgfqpoint{1.498338in}{1.447870in}}%
\pgfpathlineto{\pgfqpoint{1.487688in}{1.445164in}}%
\pgfpathlineto{\pgfqpoint{1.472031in}{1.438920in}}%
\pgfpathlineto{\pgfqpoint{1.463427in}{1.434259in}}%
\pgfpathlineto{\pgfqpoint{1.456375in}{1.429808in}}%
\pgfpathlineto{\pgfqpoint{1.444816in}{1.420648in}}%
\pgfpathlineto{\pgfqpoint{1.440718in}{1.416687in}}%
\pgfpathlineto{\pgfqpoint{1.432325in}{1.407036in}}%
\pgfpathlineto{\pgfqpoint{1.425061in}{1.396164in}}%
\pgfpathlineto{\pgfqpoint{1.423462in}{1.393425in}}%
\pgfpathlineto{\pgfqpoint{1.417921in}{1.379814in}}%
\pgfpathlineto{\pgfqpoint{1.414845in}{1.366203in}}%
\pgfpathlineto{\pgfqpoint{1.414229in}{1.352592in}}%
\pgfpathlineto{\pgfqpoint{1.416075in}{1.338981in}}%
\pgfpathlineto{\pgfqpoint{1.420383in}{1.325370in}}%
\pgfpathlineto{\pgfqpoint{1.425061in}{1.315948in}}%
\pgfpathlineto{\pgfqpoint{1.427437in}{1.311759in}}%
\pgfpathlineto{\pgfqpoint{1.437914in}{1.298148in}}%
\pgfpathlineto{\pgfqpoint{1.440718in}{1.295243in}}%
\pgfpathlineto{\pgfqpoint{1.453034in}{1.284536in}}%
\pgfpathlineto{\pgfqpoint{1.456375in}{1.282099in}}%
\pgfpathlineto{\pgfqpoint{1.472031in}{1.272990in}}%
\pgfpathlineto{\pgfqpoint{1.476850in}{1.270925in}}%
\pgfpathlineto{\pgfqpoint{1.487688in}{1.266858in}}%
\pgfpathclose%
\pgfusepath{fill}%
\end{pgfscope}%
\begin{pgfscope}%
\pgfpathrectangle{\pgfqpoint{0.360415in}{0.345370in}}{\pgfqpoint{1.550000in}{1.347500in}}%
\pgfusepath{clip}%
\pgfsetbuttcap%
\pgfsetroundjoin%
\definecolor{currentfill}{rgb}{0.993814,0.704741,0.183043}%
\pgfsetfillcolor{currentfill}%
\pgfsetlinewidth{0.000000pt}%
\definecolor{currentstroke}{rgb}{0.000000,0.000000,0.000000}%
\pgfsetstrokecolor{currentstroke}%
\pgfsetdash{}{0pt}%
\pgfpathmoveto{\pgfqpoint{0.704859in}{0.545942in}}%
\pgfpathlineto{\pgfqpoint{0.720516in}{0.542448in}}%
\pgfpathlineto{\pgfqpoint{0.736173in}{0.540508in}}%
\pgfpathlineto{\pgfqpoint{0.751829in}{0.540120in}}%
\pgfpathlineto{\pgfqpoint{0.767486in}{0.541284in}}%
\pgfpathlineto{\pgfqpoint{0.783142in}{0.544000in}}%
\pgfpathlineto{\pgfqpoint{0.798799in}{0.548273in}}%
\pgfpathlineto{\pgfqpoint{0.802220in}{0.549536in}}%
\pgfpathlineto{\pgfqpoint{0.814455in}{0.554307in}}%
\pgfpathlineto{\pgfqpoint{0.830112in}{0.562028in}}%
\pgfpathlineto{\pgfqpoint{0.832014in}{0.563148in}}%
\pgfpathlineto{\pgfqpoint{0.845769in}{0.571907in}}%
\pgfpathlineto{\pgfqpoint{0.852346in}{0.576759in}}%
\pgfpathlineto{\pgfqpoint{0.861425in}{0.584180in}}%
\pgfpathlineto{\pgfqpoint{0.868166in}{0.590370in}}%
\pgfpathlineto{\pgfqpoint{0.877082in}{0.599699in}}%
\pgfpathlineto{\pgfqpoint{0.880821in}{0.603981in}}%
\pgfpathlineto{\pgfqpoint{0.890925in}{0.617592in}}%
\pgfpathlineto{\pgfqpoint{0.892738in}{0.620583in}}%
\pgfpathlineto{\pgfqpoint{0.898773in}{0.631203in}}%
\pgfpathlineto{\pgfqpoint{0.904697in}{0.644814in}}%
\pgfpathlineto{\pgfqpoint{0.908395in}{0.657087in}}%
\pgfpathlineto{\pgfqpoint{0.908782in}{0.658425in}}%
\pgfpathlineto{\pgfqpoint{0.910980in}{0.672036in}}%
\pgfpathlineto{\pgfqpoint{0.911419in}{0.685648in}}%
\pgfpathlineto{\pgfqpoint{0.910101in}{0.699259in}}%
\pgfpathlineto{\pgfqpoint{0.908395in}{0.706820in}}%
\pgfpathlineto{\pgfqpoint{0.906975in}{0.712870in}}%
\pgfpathlineto{\pgfqpoint{0.901964in}{0.726481in}}%
\pgfpathlineto{\pgfqpoint{0.895124in}{0.740092in}}%
\pgfpathlineto{\pgfqpoint{0.892738in}{0.743874in}}%
\pgfpathlineto{\pgfqpoint{0.886114in}{0.753703in}}%
\pgfpathlineto{\pgfqpoint{0.877082in}{0.764844in}}%
\pgfpathlineto{\pgfqpoint{0.874892in}{0.767314in}}%
\pgfpathlineto{\pgfqpoint{0.861425in}{0.780449in}}%
\pgfpathlineto{\pgfqpoint{0.860877in}{0.780925in}}%
\pgfpathlineto{\pgfqpoint{0.845769in}{0.792633in}}%
\pgfpathlineto{\pgfqpoint{0.842927in}{0.794536in}}%
\pgfpathlineto{\pgfqpoint{0.830112in}{0.802388in}}%
\pgfpathlineto{\pgfqpoint{0.818806in}{0.808148in}}%
\pgfpathlineto{\pgfqpoint{0.814455in}{0.810222in}}%
\pgfpathlineto{\pgfqpoint{0.798799in}{0.816168in}}%
\pgfpathlineto{\pgfqpoint{0.783142in}{0.820524in}}%
\pgfpathlineto{\pgfqpoint{0.776183in}{0.821759in}}%
\pgfpathlineto{\pgfqpoint{0.767486in}{0.823242in}}%
\pgfpathlineto{\pgfqpoint{0.751829in}{0.824388in}}%
\pgfpathlineto{\pgfqpoint{0.736173in}{0.824006in}}%
\pgfpathlineto{\pgfqpoint{0.720516in}{0.822095in}}%
\pgfpathlineto{\pgfqpoint{0.718976in}{0.821759in}}%
\pgfpathlineto{\pgfqpoint{0.704859in}{0.818544in}}%
\pgfpathlineto{\pgfqpoint{0.689203in}{0.813394in}}%
\pgfpathlineto{\pgfqpoint{0.676987in}{0.808148in}}%
\pgfpathlineto{\pgfqpoint{0.673546in}{0.806571in}}%
\pgfpathlineto{\pgfqpoint{0.657890in}{0.797787in}}%
\pgfpathlineto{\pgfqpoint{0.652964in}{0.794536in}}%
\pgfpathlineto{\pgfqpoint{0.642233in}{0.786786in}}%
\pgfpathlineto{\pgfqpoint{0.635114in}{0.780925in}}%
\pgfpathlineto{\pgfqpoint{0.626577in}{0.773032in}}%
\pgfpathlineto{\pgfqpoint{0.620996in}{0.767314in}}%
\pgfpathlineto{\pgfqpoint{0.610920in}{0.755357in}}%
\pgfpathlineto{\pgfqpoint{0.609632in}{0.753703in}}%
\pgfpathlineto{\pgfqpoint{0.600751in}{0.740092in}}%
\pgfpathlineto{\pgfqpoint{0.595263in}{0.729455in}}%
\pgfpathlineto{\pgfqpoint{0.593810in}{0.726481in}}%
\pgfpathlineto{\pgfqpoint{0.588896in}{0.712870in}}%
\pgfpathlineto{\pgfqpoint{0.585771in}{0.699259in}}%
\pgfpathlineto{\pgfqpoint{0.584432in}{0.685648in}}%
\pgfpathlineto{\pgfqpoint{0.584878in}{0.672036in}}%
\pgfpathlineto{\pgfqpoint{0.587110in}{0.658425in}}%
\pgfpathlineto{\pgfqpoint{0.591129in}{0.644814in}}%
\pgfpathlineto{\pgfqpoint{0.595263in}{0.635103in}}%
\pgfpathlineto{\pgfqpoint{0.597014in}{0.631203in}}%
\pgfpathlineto{\pgfqpoint{0.604957in}{0.617592in}}%
\pgfpathlineto{\pgfqpoint{0.610920in}{0.609277in}}%
\pgfpathlineto{\pgfqpoint{0.615024in}{0.603981in}}%
\pgfpathlineto{\pgfqpoint{0.626577in}{0.591325in}}%
\pgfpathlineto{\pgfqpoint{0.627542in}{0.590370in}}%
\pgfpathlineto{\pgfqpoint{0.642233in}{0.577598in}}%
\pgfpathlineto{\pgfqpoint{0.643332in}{0.576759in}}%
\pgfpathlineto{\pgfqpoint{0.657890in}{0.566715in}}%
\pgfpathlineto{\pgfqpoint{0.663982in}{0.563148in}}%
\pgfpathlineto{\pgfqpoint{0.673546in}{0.557963in}}%
\pgfpathlineto{\pgfqpoint{0.689203in}{0.551058in}}%
\pgfpathlineto{\pgfqpoint{0.693689in}{0.549536in}}%
\pgfpathlineto{\pgfqpoint{0.704859in}{0.545942in}}%
\pgfpathclose%
\pgfpathmoveto{\pgfqpoint{0.722917in}{0.590370in}}%
\pgfpathlineto{\pgfqpoint{0.720516in}{0.590806in}}%
\pgfpathlineto{\pgfqpoint{0.704859in}{0.595913in}}%
\pgfpathlineto{\pgfqpoint{0.689203in}{0.603296in}}%
\pgfpathlineto{\pgfqpoint{0.688081in}{0.603981in}}%
\pgfpathlineto{\pgfqpoint{0.673546in}{0.614329in}}%
\pgfpathlineto{\pgfqpoint{0.669793in}{0.617592in}}%
\pgfpathlineto{\pgfqpoint{0.657890in}{0.630228in}}%
\pgfpathlineto{\pgfqpoint{0.657102in}{0.631203in}}%
\pgfpathlineto{\pgfqpoint{0.648609in}{0.644814in}}%
\pgfpathlineto{\pgfqpoint{0.642735in}{0.658425in}}%
\pgfpathlineto{\pgfqpoint{0.642233in}{0.660513in}}%
\pgfpathlineto{\pgfqpoint{0.639760in}{0.672036in}}%
\pgfpathlineto{\pgfqpoint{0.639176in}{0.685648in}}%
\pgfpathlineto{\pgfqpoint{0.640929in}{0.699259in}}%
\pgfpathlineto{\pgfqpoint{0.642233in}{0.703611in}}%
\pgfpathlineto{\pgfqpoint{0.645345in}{0.712870in}}%
\pgfpathlineto{\pgfqpoint{0.652528in}{0.726481in}}%
\pgfpathlineto{\pgfqpoint{0.657890in}{0.733961in}}%
\pgfpathlineto{\pgfqpoint{0.663010in}{0.740092in}}%
\pgfpathlineto{\pgfqpoint{0.673546in}{0.750140in}}%
\pgfpathlineto{\pgfqpoint{0.678102in}{0.753703in}}%
\pgfpathlineto{\pgfqpoint{0.689203in}{0.760999in}}%
\pgfpathlineto{\pgfqpoint{0.701709in}{0.767314in}}%
\pgfpathlineto{\pgfqpoint{0.704859in}{0.768705in}}%
\pgfpathlineto{\pgfqpoint{0.720516in}{0.773522in}}%
\pgfpathlineto{\pgfqpoint{0.736173in}{0.776196in}}%
\pgfpathlineto{\pgfqpoint{0.751829in}{0.776731in}}%
\pgfpathlineto{\pgfqpoint{0.767486in}{0.775127in}}%
\pgfpathlineto{\pgfqpoint{0.783142in}{0.771382in}}%
\pgfpathlineto{\pgfqpoint{0.793980in}{0.767314in}}%
\pgfpathlineto{\pgfqpoint{0.798799in}{0.765249in}}%
\pgfpathlineto{\pgfqpoint{0.814455in}{0.756141in}}%
\pgfpathlineto{\pgfqpoint{0.817796in}{0.753703in}}%
\pgfpathlineto{\pgfqpoint{0.830112in}{0.742996in}}%
\pgfpathlineto{\pgfqpoint{0.832916in}{0.740092in}}%
\pgfpathlineto{\pgfqpoint{0.843393in}{0.726481in}}%
\pgfpathlineto{\pgfqpoint{0.845769in}{0.722292in}}%
\pgfpathlineto{\pgfqpoint{0.850447in}{0.712870in}}%
\pgfpathlineto{\pgfqpoint{0.854755in}{0.699259in}}%
\pgfpathlineto{\pgfqpoint{0.856601in}{0.685648in}}%
\pgfpathlineto{\pgfqpoint{0.855985in}{0.672036in}}%
\pgfpathlineto{\pgfqpoint{0.852909in}{0.658425in}}%
\pgfpathlineto{\pgfqpoint{0.847368in}{0.644814in}}%
\pgfpathlineto{\pgfqpoint{0.845769in}{0.642076in}}%
\pgfpathlineto{\pgfqpoint{0.838505in}{0.631203in}}%
\pgfpathlineto{\pgfqpoint{0.830112in}{0.621552in}}%
\pgfpathlineto{\pgfqpoint{0.826014in}{0.617592in}}%
\pgfpathlineto{\pgfqpoint{0.814455in}{0.608432in}}%
\pgfpathlineto{\pgfqpoint{0.807403in}{0.603981in}}%
\pgfpathlineto{\pgfqpoint{0.798799in}{0.599320in}}%
\pgfpathlineto{\pgfqpoint{0.783142in}{0.593075in}}%
\pgfpathlineto{\pgfqpoint{0.772492in}{0.590370in}}%
\pgfpathlineto{\pgfqpoint{0.767486in}{0.589236in}}%
\pgfpathlineto{\pgfqpoint{0.751829in}{0.587712in}}%
\pgfpathlineto{\pgfqpoint{0.736173in}{0.588220in}}%
\pgfpathlineto{\pgfqpoint{0.722917in}{0.590370in}}%
\pgfpathclose%
\pgfpathmoveto{\pgfqpoint{1.472031in}{0.548273in}}%
\pgfpathlineto{\pgfqpoint{1.487688in}{0.544000in}}%
\pgfpathlineto{\pgfqpoint{1.503344in}{0.541284in}}%
\pgfpathlineto{\pgfqpoint{1.519001in}{0.540120in}}%
\pgfpathlineto{\pgfqpoint{1.534657in}{0.540508in}}%
\pgfpathlineto{\pgfqpoint{1.550314in}{0.542448in}}%
\pgfpathlineto{\pgfqpoint{1.565971in}{0.545942in}}%
\pgfpathlineto{\pgfqpoint{1.577141in}{0.549536in}}%
\pgfpathlineto{\pgfqpoint{1.581627in}{0.551058in}}%
\pgfpathlineto{\pgfqpoint{1.597284in}{0.557963in}}%
\pgfpathlineto{\pgfqpoint{1.606848in}{0.563148in}}%
\pgfpathlineto{\pgfqpoint{1.612940in}{0.566715in}}%
\pgfpathlineto{\pgfqpoint{1.627498in}{0.576759in}}%
\pgfpathlineto{\pgfqpoint{1.628597in}{0.577598in}}%
\pgfpathlineto{\pgfqpoint{1.643288in}{0.590370in}}%
\pgfpathlineto{\pgfqpoint{1.644253in}{0.591325in}}%
\pgfpathlineto{\pgfqpoint{1.655806in}{0.603981in}}%
\pgfpathlineto{\pgfqpoint{1.659910in}{0.609277in}}%
\pgfpathlineto{\pgfqpoint{1.665873in}{0.617592in}}%
\pgfpathlineto{\pgfqpoint{1.673816in}{0.631203in}}%
\pgfpathlineto{\pgfqpoint{1.675567in}{0.635103in}}%
\pgfpathlineto{\pgfqpoint{1.679701in}{0.644814in}}%
\pgfpathlineto{\pgfqpoint{1.683720in}{0.658425in}}%
\pgfpathlineto{\pgfqpoint{1.685952in}{0.672036in}}%
\pgfpathlineto{\pgfqpoint{1.686398in}{0.685648in}}%
\pgfpathlineto{\pgfqpoint{1.685059in}{0.699259in}}%
\pgfpathlineto{\pgfqpoint{1.681934in}{0.712870in}}%
\pgfpathlineto{\pgfqpoint{1.677020in}{0.726481in}}%
\pgfpathlineto{\pgfqpoint{1.675567in}{0.729455in}}%
\pgfpathlineto{\pgfqpoint{1.670079in}{0.740092in}}%
\pgfpathlineto{\pgfqpoint{1.661198in}{0.753703in}}%
\pgfpathlineto{\pgfqpoint{1.659910in}{0.755357in}}%
\pgfpathlineto{\pgfqpoint{1.649834in}{0.767314in}}%
\pgfpathlineto{\pgfqpoint{1.644253in}{0.773032in}}%
\pgfpathlineto{\pgfqpoint{1.635716in}{0.780925in}}%
\pgfpathlineto{\pgfqpoint{1.628597in}{0.786786in}}%
\pgfpathlineto{\pgfqpoint{1.617866in}{0.794536in}}%
\pgfpathlineto{\pgfqpoint{1.612940in}{0.797787in}}%
\pgfpathlineto{\pgfqpoint{1.597284in}{0.806571in}}%
\pgfpathlineto{\pgfqpoint{1.593843in}{0.808148in}}%
\pgfpathlineto{\pgfqpoint{1.581627in}{0.813394in}}%
\pgfpathlineto{\pgfqpoint{1.565971in}{0.818544in}}%
\pgfpathlineto{\pgfqpoint{1.551854in}{0.821759in}}%
\pgfpathlineto{\pgfqpoint{1.550314in}{0.822095in}}%
\pgfpathlineto{\pgfqpoint{1.534657in}{0.824006in}}%
\pgfpathlineto{\pgfqpoint{1.519001in}{0.824388in}}%
\pgfpathlineto{\pgfqpoint{1.503344in}{0.823242in}}%
\pgfpathlineto{\pgfqpoint{1.494647in}{0.821759in}}%
\pgfpathlineto{\pgfqpoint{1.487688in}{0.820524in}}%
\pgfpathlineto{\pgfqpoint{1.472031in}{0.816168in}}%
\pgfpathlineto{\pgfqpoint{1.456375in}{0.810222in}}%
\pgfpathlineto{\pgfqpoint{1.452024in}{0.808148in}}%
\pgfpathlineto{\pgfqpoint{1.440718in}{0.802388in}}%
\pgfpathlineto{\pgfqpoint{1.427903in}{0.794536in}}%
\pgfpathlineto{\pgfqpoint{1.425061in}{0.792633in}}%
\pgfpathlineto{\pgfqpoint{1.409953in}{0.780925in}}%
\pgfpathlineto{\pgfqpoint{1.409405in}{0.780449in}}%
\pgfpathlineto{\pgfqpoint{1.395938in}{0.767314in}}%
\pgfpathlineto{\pgfqpoint{1.393748in}{0.764844in}}%
\pgfpathlineto{\pgfqpoint{1.384716in}{0.753703in}}%
\pgfpathlineto{\pgfqpoint{1.378092in}{0.743874in}}%
\pgfpathlineto{\pgfqpoint{1.375706in}{0.740092in}}%
\pgfpathlineto{\pgfqpoint{1.368866in}{0.726481in}}%
\pgfpathlineto{\pgfqpoint{1.363855in}{0.712870in}}%
\pgfpathlineto{\pgfqpoint{1.362435in}{0.706820in}}%
\pgfpathlineto{\pgfqpoint{1.360729in}{0.699259in}}%
\pgfpathlineto{\pgfqpoint{1.359411in}{0.685648in}}%
\pgfpathlineto{\pgfqpoint{1.359850in}{0.672036in}}%
\pgfpathlineto{\pgfqpoint{1.362048in}{0.658425in}}%
\pgfpathlineto{\pgfqpoint{1.362435in}{0.657087in}}%
\pgfpathlineto{\pgfqpoint{1.366133in}{0.644814in}}%
\pgfpathlineto{\pgfqpoint{1.372057in}{0.631203in}}%
\pgfpathlineto{\pgfqpoint{1.378092in}{0.620583in}}%
\pgfpathlineto{\pgfqpoint{1.379905in}{0.617592in}}%
\pgfpathlineto{\pgfqpoint{1.390009in}{0.603981in}}%
\pgfpathlineto{\pgfqpoint{1.393748in}{0.599699in}}%
\pgfpathlineto{\pgfqpoint{1.402664in}{0.590370in}}%
\pgfpathlineto{\pgfqpoint{1.409405in}{0.584180in}}%
\pgfpathlineto{\pgfqpoint{1.418484in}{0.576759in}}%
\pgfpathlineto{\pgfqpoint{1.425061in}{0.571907in}}%
\pgfpathlineto{\pgfqpoint{1.438816in}{0.563148in}}%
\pgfpathlineto{\pgfqpoint{1.440718in}{0.562028in}}%
\pgfpathlineto{\pgfqpoint{1.456375in}{0.554307in}}%
\pgfpathlineto{\pgfqpoint{1.468610in}{0.549536in}}%
\pgfpathlineto{\pgfqpoint{1.472031in}{0.548273in}}%
\pgfpathclose%
\pgfpathmoveto{\pgfqpoint{1.498338in}{0.590370in}}%
\pgfpathlineto{\pgfqpoint{1.487688in}{0.593075in}}%
\pgfpathlineto{\pgfqpoint{1.472031in}{0.599320in}}%
\pgfpathlineto{\pgfqpoint{1.463427in}{0.603981in}}%
\pgfpathlineto{\pgfqpoint{1.456375in}{0.608432in}}%
\pgfpathlineto{\pgfqpoint{1.444816in}{0.617592in}}%
\pgfpathlineto{\pgfqpoint{1.440718in}{0.621552in}}%
\pgfpathlineto{\pgfqpoint{1.432325in}{0.631203in}}%
\pgfpathlineto{\pgfqpoint{1.425061in}{0.642076in}}%
\pgfpathlineto{\pgfqpoint{1.423462in}{0.644814in}}%
\pgfpathlineto{\pgfqpoint{1.417921in}{0.658425in}}%
\pgfpathlineto{\pgfqpoint{1.414845in}{0.672036in}}%
\pgfpathlineto{\pgfqpoint{1.414229in}{0.685648in}}%
\pgfpathlineto{\pgfqpoint{1.416075in}{0.699259in}}%
\pgfpathlineto{\pgfqpoint{1.420383in}{0.712870in}}%
\pgfpathlineto{\pgfqpoint{1.425061in}{0.722292in}}%
\pgfpathlineto{\pgfqpoint{1.427437in}{0.726481in}}%
\pgfpathlineto{\pgfqpoint{1.437914in}{0.740092in}}%
\pgfpathlineto{\pgfqpoint{1.440718in}{0.742996in}}%
\pgfpathlineto{\pgfqpoint{1.453034in}{0.753703in}}%
\pgfpathlineto{\pgfqpoint{1.456375in}{0.756141in}}%
\pgfpathlineto{\pgfqpoint{1.472031in}{0.765249in}}%
\pgfpathlineto{\pgfqpoint{1.476850in}{0.767314in}}%
\pgfpathlineto{\pgfqpoint{1.487688in}{0.771382in}}%
\pgfpathlineto{\pgfqpoint{1.503344in}{0.775127in}}%
\pgfpathlineto{\pgfqpoint{1.519001in}{0.776731in}}%
\pgfpathlineto{\pgfqpoint{1.534657in}{0.776196in}}%
\pgfpathlineto{\pgfqpoint{1.550314in}{0.773522in}}%
\pgfpathlineto{\pgfqpoint{1.565971in}{0.768705in}}%
\pgfpathlineto{\pgfqpoint{1.569121in}{0.767314in}}%
\pgfpathlineto{\pgfqpoint{1.581627in}{0.760999in}}%
\pgfpathlineto{\pgfqpoint{1.592728in}{0.753703in}}%
\pgfpathlineto{\pgfqpoint{1.597284in}{0.750140in}}%
\pgfpathlineto{\pgfqpoint{1.607820in}{0.740092in}}%
\pgfpathlineto{\pgfqpoint{1.612940in}{0.733961in}}%
\pgfpathlineto{\pgfqpoint{1.618302in}{0.726481in}}%
\pgfpathlineto{\pgfqpoint{1.625485in}{0.712870in}}%
\pgfpathlineto{\pgfqpoint{1.628597in}{0.703611in}}%
\pgfpathlineto{\pgfqpoint{1.629901in}{0.699259in}}%
\pgfpathlineto{\pgfqpoint{1.631654in}{0.685648in}}%
\pgfpathlineto{\pgfqpoint{1.631070in}{0.672036in}}%
\pgfpathlineto{\pgfqpoint{1.628597in}{0.660513in}}%
\pgfpathlineto{\pgfqpoint{1.628095in}{0.658425in}}%
\pgfpathlineto{\pgfqpoint{1.622221in}{0.644814in}}%
\pgfpathlineto{\pgfqpoint{1.613728in}{0.631203in}}%
\pgfpathlineto{\pgfqpoint{1.612940in}{0.630228in}}%
\pgfpathlineto{\pgfqpoint{1.601037in}{0.617592in}}%
\pgfpathlineto{\pgfqpoint{1.597284in}{0.614329in}}%
\pgfpathlineto{\pgfqpoint{1.582749in}{0.603981in}}%
\pgfpathlineto{\pgfqpoint{1.581627in}{0.603296in}}%
\pgfpathlineto{\pgfqpoint{1.565971in}{0.595913in}}%
\pgfpathlineto{\pgfqpoint{1.550314in}{0.590806in}}%
\pgfpathlineto{\pgfqpoint{1.547913in}{0.590370in}}%
\pgfpathlineto{\pgfqpoint{1.534657in}{0.588220in}}%
\pgfpathlineto{\pgfqpoint{1.519001in}{0.587712in}}%
\pgfpathlineto{\pgfqpoint{1.503344in}{0.589236in}}%
\pgfpathlineto{\pgfqpoint{1.498338in}{0.590370in}}%
\pgfpathclose%
\pgfpathmoveto{\pgfqpoint{0.720516in}{1.216144in}}%
\pgfpathlineto{\pgfqpoint{0.736173in}{1.214234in}}%
\pgfpathlineto{\pgfqpoint{0.751829in}{1.213852in}}%
\pgfpathlineto{\pgfqpoint{0.767486in}{1.214998in}}%
\pgfpathlineto{\pgfqpoint{0.776183in}{1.216481in}}%
\pgfpathlineto{\pgfqpoint{0.783142in}{1.217716in}}%
\pgfpathlineto{\pgfqpoint{0.798799in}{1.222072in}}%
\pgfpathlineto{\pgfqpoint{0.814455in}{1.228018in}}%
\pgfpathlineto{\pgfqpoint{0.818806in}{1.230092in}}%
\pgfpathlineto{\pgfqpoint{0.830112in}{1.235851in}}%
\pgfpathlineto{\pgfqpoint{0.842927in}{1.243703in}}%
\pgfpathlineto{\pgfqpoint{0.845769in}{1.245606in}}%
\pgfpathlineto{\pgfqpoint{0.860877in}{1.257314in}}%
\pgfpathlineto{\pgfqpoint{0.861425in}{1.257791in}}%
\pgfpathlineto{\pgfqpoint{0.874892in}{1.270925in}}%
\pgfpathlineto{\pgfqpoint{0.877082in}{1.273396in}}%
\pgfpathlineto{\pgfqpoint{0.886114in}{1.284536in}}%
\pgfpathlineto{\pgfqpoint{0.892738in}{1.294365in}}%
\pgfpathlineto{\pgfqpoint{0.895124in}{1.298148in}}%
\pgfpathlineto{\pgfqpoint{0.901964in}{1.311759in}}%
\pgfpathlineto{\pgfqpoint{0.906975in}{1.325370in}}%
\pgfpathlineto{\pgfqpoint{0.908395in}{1.331420in}}%
\pgfpathlineto{\pgfqpoint{0.910101in}{1.338981in}}%
\pgfpathlineto{\pgfqpoint{0.911419in}{1.352592in}}%
\pgfpathlineto{\pgfqpoint{0.910980in}{1.366203in}}%
\pgfpathlineto{\pgfqpoint{0.908782in}{1.379814in}}%
\pgfpathlineto{\pgfqpoint{0.908395in}{1.381153in}}%
\pgfpathlineto{\pgfqpoint{0.904697in}{1.393425in}}%
\pgfpathlineto{\pgfqpoint{0.898773in}{1.407036in}}%
\pgfpathlineto{\pgfqpoint{0.892738in}{1.417656in}}%
\pgfpathlineto{\pgfqpoint{0.890925in}{1.420648in}}%
\pgfpathlineto{\pgfqpoint{0.880821in}{1.434259in}}%
\pgfpathlineto{\pgfqpoint{0.877082in}{1.438541in}}%
\pgfpathlineto{\pgfqpoint{0.868166in}{1.447870in}}%
\pgfpathlineto{\pgfqpoint{0.861425in}{1.454059in}}%
\pgfpathlineto{\pgfqpoint{0.852346in}{1.461481in}}%
\pgfpathlineto{\pgfqpoint{0.845769in}{1.466333in}}%
\pgfpathlineto{\pgfqpoint{0.832014in}{1.475092in}}%
\pgfpathlineto{\pgfqpoint{0.830112in}{1.476212in}}%
\pgfpathlineto{\pgfqpoint{0.814455in}{1.483932in}}%
\pgfpathlineto{\pgfqpoint{0.802220in}{1.488703in}}%
\pgfpathlineto{\pgfqpoint{0.798799in}{1.489967in}}%
\pgfpathlineto{\pgfqpoint{0.783142in}{1.494239in}}%
\pgfpathlineto{\pgfqpoint{0.767486in}{1.496956in}}%
\pgfpathlineto{\pgfqpoint{0.751829in}{1.498119in}}%
\pgfpathlineto{\pgfqpoint{0.736173in}{1.497731in}}%
\pgfpathlineto{\pgfqpoint{0.720516in}{1.495792in}}%
\pgfpathlineto{\pgfqpoint{0.704859in}{1.492298in}}%
\pgfpathlineto{\pgfqpoint{0.693689in}{1.488703in}}%
\pgfpathlineto{\pgfqpoint{0.689203in}{1.487181in}}%
\pgfpathlineto{\pgfqpoint{0.673546in}{1.480276in}}%
\pgfpathlineto{\pgfqpoint{0.663982in}{1.475092in}}%
\pgfpathlineto{\pgfqpoint{0.657890in}{1.471525in}}%
\pgfpathlineto{\pgfqpoint{0.643332in}{1.461481in}}%
\pgfpathlineto{\pgfqpoint{0.642233in}{1.460642in}}%
\pgfpathlineto{\pgfqpoint{0.627542in}{1.447870in}}%
\pgfpathlineto{\pgfqpoint{0.626577in}{1.446914in}}%
\pgfpathlineto{\pgfqpoint{0.615024in}{1.434259in}}%
\pgfpathlineto{\pgfqpoint{0.610920in}{1.428962in}}%
\pgfpathlineto{\pgfqpoint{0.604957in}{1.420648in}}%
\pgfpathlineto{\pgfqpoint{0.597014in}{1.407036in}}%
\pgfpathlineto{\pgfqpoint{0.595263in}{1.403136in}}%
\pgfpathlineto{\pgfqpoint{0.591129in}{1.393425in}}%
\pgfpathlineto{\pgfqpoint{0.587110in}{1.379814in}}%
\pgfpathlineto{\pgfqpoint{0.584878in}{1.366203in}}%
\pgfpathlineto{\pgfqpoint{0.584432in}{1.352592in}}%
\pgfpathlineto{\pgfqpoint{0.585771in}{1.338981in}}%
\pgfpathlineto{\pgfqpoint{0.588896in}{1.325370in}}%
\pgfpathlineto{\pgfqpoint{0.593810in}{1.311759in}}%
\pgfpathlineto{\pgfqpoint{0.595263in}{1.308784in}}%
\pgfpathlineto{\pgfqpoint{0.600751in}{1.298148in}}%
\pgfpathlineto{\pgfqpoint{0.609632in}{1.284536in}}%
\pgfpathlineto{\pgfqpoint{0.610920in}{1.282883in}}%
\pgfpathlineto{\pgfqpoint{0.620996in}{1.270925in}}%
\pgfpathlineto{\pgfqpoint{0.626577in}{1.265207in}}%
\pgfpathlineto{\pgfqpoint{0.635114in}{1.257314in}}%
\pgfpathlineto{\pgfqpoint{0.642233in}{1.251454in}}%
\pgfpathlineto{\pgfqpoint{0.652964in}{1.243703in}}%
\pgfpathlineto{\pgfqpoint{0.657890in}{1.240453in}}%
\pgfpathlineto{\pgfqpoint{0.673546in}{1.231669in}}%
\pgfpathlineto{\pgfqpoint{0.676987in}{1.230092in}}%
\pgfpathlineto{\pgfqpoint{0.689203in}{1.224846in}}%
\pgfpathlineto{\pgfqpoint{0.704859in}{1.219695in}}%
\pgfpathlineto{\pgfqpoint{0.718976in}{1.216481in}}%
\pgfpathlineto{\pgfqpoint{0.720516in}{1.216144in}}%
\pgfpathclose%
\pgfpathmoveto{\pgfqpoint{0.701709in}{1.270925in}}%
\pgfpathlineto{\pgfqpoint{0.689203in}{1.277240in}}%
\pgfpathlineto{\pgfqpoint{0.678102in}{1.284536in}}%
\pgfpathlineto{\pgfqpoint{0.673546in}{1.288099in}}%
\pgfpathlineto{\pgfqpoint{0.663010in}{1.298148in}}%
\pgfpathlineto{\pgfqpoint{0.657890in}{1.304279in}}%
\pgfpathlineto{\pgfqpoint{0.652528in}{1.311759in}}%
\pgfpathlineto{\pgfqpoint{0.645345in}{1.325370in}}%
\pgfpathlineto{\pgfqpoint{0.642233in}{1.334629in}}%
\pgfpathlineto{\pgfqpoint{0.640929in}{1.338981in}}%
\pgfpathlineto{\pgfqpoint{0.639176in}{1.352592in}}%
\pgfpathlineto{\pgfqpoint{0.639760in}{1.366203in}}%
\pgfpathlineto{\pgfqpoint{0.642233in}{1.377727in}}%
\pgfpathlineto{\pgfqpoint{0.642735in}{1.379814in}}%
\pgfpathlineto{\pgfqpoint{0.648609in}{1.393425in}}%
\pgfpathlineto{\pgfqpoint{0.657102in}{1.407036in}}%
\pgfpathlineto{\pgfqpoint{0.657890in}{1.408012in}}%
\pgfpathlineto{\pgfqpoint{0.669793in}{1.420648in}}%
\pgfpathlineto{\pgfqpoint{0.673546in}{1.423910in}}%
\pgfpathlineto{\pgfqpoint{0.688081in}{1.434259in}}%
\pgfpathlineto{\pgfqpoint{0.689203in}{1.434943in}}%
\pgfpathlineto{\pgfqpoint{0.704859in}{1.442327in}}%
\pgfpathlineto{\pgfqpoint{0.720516in}{1.447434in}}%
\pgfpathlineto{\pgfqpoint{0.722917in}{1.447870in}}%
\pgfpathlineto{\pgfqpoint{0.736173in}{1.450019in}}%
\pgfpathlineto{\pgfqpoint{0.751829in}{1.450527in}}%
\pgfpathlineto{\pgfqpoint{0.767486in}{1.449003in}}%
\pgfpathlineto{\pgfqpoint{0.772492in}{1.447870in}}%
\pgfpathlineto{\pgfqpoint{0.783142in}{1.445164in}}%
\pgfpathlineto{\pgfqpoint{0.798799in}{1.438920in}}%
\pgfpathlineto{\pgfqpoint{0.807403in}{1.434259in}}%
\pgfpathlineto{\pgfqpoint{0.814455in}{1.429808in}}%
\pgfpathlineto{\pgfqpoint{0.826014in}{1.420648in}}%
\pgfpathlineto{\pgfqpoint{0.830112in}{1.416687in}}%
\pgfpathlineto{\pgfqpoint{0.838505in}{1.407036in}}%
\pgfpathlineto{\pgfqpoint{0.845769in}{1.396164in}}%
\pgfpathlineto{\pgfqpoint{0.847368in}{1.393425in}}%
\pgfpathlineto{\pgfqpoint{0.852909in}{1.379814in}}%
\pgfpathlineto{\pgfqpoint{0.855985in}{1.366203in}}%
\pgfpathlineto{\pgfqpoint{0.856601in}{1.352592in}}%
\pgfpathlineto{\pgfqpoint{0.854755in}{1.338981in}}%
\pgfpathlineto{\pgfqpoint{0.850447in}{1.325370in}}%
\pgfpathlineto{\pgfqpoint{0.845769in}{1.315948in}}%
\pgfpathlineto{\pgfqpoint{0.843393in}{1.311759in}}%
\pgfpathlineto{\pgfqpoint{0.832916in}{1.298148in}}%
\pgfpathlineto{\pgfqpoint{0.830112in}{1.295243in}}%
\pgfpathlineto{\pgfqpoint{0.817796in}{1.284536in}}%
\pgfpathlineto{\pgfqpoint{0.814455in}{1.282099in}}%
\pgfpathlineto{\pgfqpoint{0.798799in}{1.272990in}}%
\pgfpathlineto{\pgfqpoint{0.793980in}{1.270925in}}%
\pgfpathlineto{\pgfqpoint{0.783142in}{1.266858in}}%
\pgfpathlineto{\pgfqpoint{0.767486in}{1.263113in}}%
\pgfpathlineto{\pgfqpoint{0.751829in}{1.261508in}}%
\pgfpathlineto{\pgfqpoint{0.736173in}{1.262043in}}%
\pgfpathlineto{\pgfqpoint{0.720516in}{1.264717in}}%
\pgfpathlineto{\pgfqpoint{0.704859in}{1.269535in}}%
\pgfpathlineto{\pgfqpoint{0.701709in}{1.270925in}}%
\pgfpathclose%
\pgfpathmoveto{\pgfqpoint{1.503344in}{1.214998in}}%
\pgfpathlineto{\pgfqpoint{1.519001in}{1.213852in}}%
\pgfpathlineto{\pgfqpoint{1.534657in}{1.214234in}}%
\pgfpathlineto{\pgfqpoint{1.550314in}{1.216144in}}%
\pgfpathlineto{\pgfqpoint{1.551854in}{1.216481in}}%
\pgfpathlineto{\pgfqpoint{1.565971in}{1.219695in}}%
\pgfpathlineto{\pgfqpoint{1.581627in}{1.224846in}}%
\pgfpathlineto{\pgfqpoint{1.593843in}{1.230092in}}%
\pgfpathlineto{\pgfqpoint{1.597284in}{1.231669in}}%
\pgfpathlineto{\pgfqpoint{1.612940in}{1.240453in}}%
\pgfpathlineto{\pgfqpoint{1.617866in}{1.243703in}}%
\pgfpathlineto{\pgfqpoint{1.628597in}{1.251454in}}%
\pgfpathlineto{\pgfqpoint{1.635716in}{1.257314in}}%
\pgfpathlineto{\pgfqpoint{1.644253in}{1.265207in}}%
\pgfpathlineto{\pgfqpoint{1.649834in}{1.270925in}}%
\pgfpathlineto{\pgfqpoint{1.659910in}{1.282883in}}%
\pgfpathlineto{\pgfqpoint{1.661198in}{1.284536in}}%
\pgfpathlineto{\pgfqpoint{1.670079in}{1.298148in}}%
\pgfpathlineto{\pgfqpoint{1.675567in}{1.308784in}}%
\pgfpathlineto{\pgfqpoint{1.677020in}{1.311759in}}%
\pgfpathlineto{\pgfqpoint{1.681934in}{1.325370in}}%
\pgfpathlineto{\pgfqpoint{1.685059in}{1.338981in}}%
\pgfpathlineto{\pgfqpoint{1.686398in}{1.352592in}}%
\pgfpathlineto{\pgfqpoint{1.685952in}{1.366203in}}%
\pgfpathlineto{\pgfqpoint{1.683720in}{1.379814in}}%
\pgfpathlineto{\pgfqpoint{1.679701in}{1.393425in}}%
\pgfpathlineto{\pgfqpoint{1.675567in}{1.403136in}}%
\pgfpathlineto{\pgfqpoint{1.673816in}{1.407036in}}%
\pgfpathlineto{\pgfqpoint{1.665873in}{1.420647in}}%
\pgfpathlineto{\pgfqpoint{1.659910in}{1.428962in}}%
\pgfpathlineto{\pgfqpoint{1.655806in}{1.434259in}}%
\pgfpathlineto{\pgfqpoint{1.644253in}{1.446914in}}%
\pgfpathlineto{\pgfqpoint{1.643288in}{1.447870in}}%
\pgfpathlineto{\pgfqpoint{1.628597in}{1.460642in}}%
\pgfpathlineto{\pgfqpoint{1.627498in}{1.461481in}}%
\pgfpathlineto{\pgfqpoint{1.612940in}{1.471525in}}%
\pgfpathlineto{\pgfqpoint{1.606848in}{1.475092in}}%
\pgfpathlineto{\pgfqpoint{1.597284in}{1.480276in}}%
\pgfpathlineto{\pgfqpoint{1.581627in}{1.487181in}}%
\pgfpathlineto{\pgfqpoint{1.577141in}{1.488703in}}%
\pgfpathlineto{\pgfqpoint{1.565971in}{1.492298in}}%
\pgfpathlineto{\pgfqpoint{1.550314in}{1.495792in}}%
\pgfpathlineto{\pgfqpoint{1.534657in}{1.497731in}}%
\pgfpathlineto{\pgfqpoint{1.519001in}{1.498119in}}%
\pgfpathlineto{\pgfqpoint{1.503344in}{1.496956in}}%
\pgfpathlineto{\pgfqpoint{1.487688in}{1.494239in}}%
\pgfpathlineto{\pgfqpoint{1.472031in}{1.489967in}}%
\pgfpathlineto{\pgfqpoint{1.468610in}{1.488703in}}%
\pgfpathlineto{\pgfqpoint{1.456375in}{1.483932in}}%
\pgfpathlineto{\pgfqpoint{1.440718in}{1.476212in}}%
\pgfpathlineto{\pgfqpoint{1.438816in}{1.475092in}}%
\pgfpathlineto{\pgfqpoint{1.425061in}{1.466333in}}%
\pgfpathlineto{\pgfqpoint{1.418484in}{1.461481in}}%
\pgfpathlineto{\pgfqpoint{1.409405in}{1.454059in}}%
\pgfpathlineto{\pgfqpoint{1.402664in}{1.447870in}}%
\pgfpathlineto{\pgfqpoint{1.393748in}{1.438541in}}%
\pgfpathlineto{\pgfqpoint{1.390009in}{1.434259in}}%
\pgfpathlineto{\pgfqpoint{1.379905in}{1.420648in}}%
\pgfpathlineto{\pgfqpoint{1.378092in}{1.417656in}}%
\pgfpathlineto{\pgfqpoint{1.372057in}{1.407036in}}%
\pgfpathlineto{\pgfqpoint{1.366133in}{1.393425in}}%
\pgfpathlineto{\pgfqpoint{1.362435in}{1.381153in}}%
\pgfpathlineto{\pgfqpoint{1.362048in}{1.379814in}}%
\pgfpathlineto{\pgfqpoint{1.359850in}{1.366203in}}%
\pgfpathlineto{\pgfqpoint{1.359411in}{1.352592in}}%
\pgfpathlineto{\pgfqpoint{1.360729in}{1.338981in}}%
\pgfpathlineto{\pgfqpoint{1.362435in}{1.331420in}}%
\pgfpathlineto{\pgfqpoint{1.363855in}{1.325370in}}%
\pgfpathlineto{\pgfqpoint{1.368866in}{1.311759in}}%
\pgfpathlineto{\pgfqpoint{1.375706in}{1.298148in}}%
\pgfpathlineto{\pgfqpoint{1.378092in}{1.294365in}}%
\pgfpathlineto{\pgfqpoint{1.384716in}{1.284536in}}%
\pgfpathlineto{\pgfqpoint{1.393748in}{1.273396in}}%
\pgfpathlineto{\pgfqpoint{1.395938in}{1.270925in}}%
\pgfpathlineto{\pgfqpoint{1.409405in}{1.257791in}}%
\pgfpathlineto{\pgfqpoint{1.409953in}{1.257314in}}%
\pgfpathlineto{\pgfqpoint{1.425061in}{1.245606in}}%
\pgfpathlineto{\pgfqpoint{1.427903in}{1.243703in}}%
\pgfpathlineto{\pgfqpoint{1.440718in}{1.235851in}}%
\pgfpathlineto{\pgfqpoint{1.452024in}{1.230092in}}%
\pgfpathlineto{\pgfqpoint{1.456375in}{1.228018in}}%
\pgfpathlineto{\pgfqpoint{1.472031in}{1.222072in}}%
\pgfpathlineto{\pgfqpoint{1.487688in}{1.217716in}}%
\pgfpathlineto{\pgfqpoint{1.494647in}{1.216481in}}%
\pgfpathlineto{\pgfqpoint{1.503344in}{1.214998in}}%
\pgfpathclose%
\pgfpathmoveto{\pgfqpoint{1.476850in}{1.270925in}}%
\pgfpathlineto{\pgfqpoint{1.472031in}{1.272990in}}%
\pgfpathlineto{\pgfqpoint{1.456375in}{1.282099in}}%
\pgfpathlineto{\pgfqpoint{1.453034in}{1.284536in}}%
\pgfpathlineto{\pgfqpoint{1.440718in}{1.295243in}}%
\pgfpathlineto{\pgfqpoint{1.437914in}{1.298148in}}%
\pgfpathlineto{\pgfqpoint{1.427437in}{1.311759in}}%
\pgfpathlineto{\pgfqpoint{1.425061in}{1.315948in}}%
\pgfpathlineto{\pgfqpoint{1.420383in}{1.325370in}}%
\pgfpathlineto{\pgfqpoint{1.416075in}{1.338981in}}%
\pgfpathlineto{\pgfqpoint{1.414229in}{1.352592in}}%
\pgfpathlineto{\pgfqpoint{1.414845in}{1.366203in}}%
\pgfpathlineto{\pgfqpoint{1.417921in}{1.379814in}}%
\pgfpathlineto{\pgfqpoint{1.423462in}{1.393425in}}%
\pgfpathlineto{\pgfqpoint{1.425061in}{1.396164in}}%
\pgfpathlineto{\pgfqpoint{1.432325in}{1.407036in}}%
\pgfpathlineto{\pgfqpoint{1.440718in}{1.416687in}}%
\pgfpathlineto{\pgfqpoint{1.444816in}{1.420648in}}%
\pgfpathlineto{\pgfqpoint{1.456375in}{1.429808in}}%
\pgfpathlineto{\pgfqpoint{1.463427in}{1.434259in}}%
\pgfpathlineto{\pgfqpoint{1.472031in}{1.438920in}}%
\pgfpathlineto{\pgfqpoint{1.487688in}{1.445164in}}%
\pgfpathlineto{\pgfqpoint{1.498338in}{1.447870in}}%
\pgfpathlineto{\pgfqpoint{1.503344in}{1.449003in}}%
\pgfpathlineto{\pgfqpoint{1.519001in}{1.450527in}}%
\pgfpathlineto{\pgfqpoint{1.534657in}{1.450019in}}%
\pgfpathlineto{\pgfqpoint{1.547913in}{1.447870in}}%
\pgfpathlineto{\pgfqpoint{1.550314in}{1.447434in}}%
\pgfpathlineto{\pgfqpoint{1.565971in}{1.442327in}}%
\pgfpathlineto{\pgfqpoint{1.581627in}{1.434943in}}%
\pgfpathlineto{\pgfqpoint{1.582749in}{1.434259in}}%
\pgfpathlineto{\pgfqpoint{1.597284in}{1.423910in}}%
\pgfpathlineto{\pgfqpoint{1.601037in}{1.420648in}}%
\pgfpathlineto{\pgfqpoint{1.612940in}{1.408012in}}%
\pgfpathlineto{\pgfqpoint{1.613728in}{1.407036in}}%
\pgfpathlineto{\pgfqpoint{1.622221in}{1.393425in}}%
\pgfpathlineto{\pgfqpoint{1.628095in}{1.379814in}}%
\pgfpathlineto{\pgfqpoint{1.628597in}{1.377727in}}%
\pgfpathlineto{\pgfqpoint{1.631070in}{1.366203in}}%
\pgfpathlineto{\pgfqpoint{1.631654in}{1.352592in}}%
\pgfpathlineto{\pgfqpoint{1.629901in}{1.338981in}}%
\pgfpathlineto{\pgfqpoint{1.628597in}{1.334629in}}%
\pgfpathlineto{\pgfqpoint{1.625485in}{1.325370in}}%
\pgfpathlineto{\pgfqpoint{1.618302in}{1.311759in}}%
\pgfpathlineto{\pgfqpoint{1.612940in}{1.304279in}}%
\pgfpathlineto{\pgfqpoint{1.607820in}{1.298148in}}%
\pgfpathlineto{\pgfqpoint{1.597284in}{1.288099in}}%
\pgfpathlineto{\pgfqpoint{1.592728in}{1.284536in}}%
\pgfpathlineto{\pgfqpoint{1.581627in}{1.277240in}}%
\pgfpathlineto{\pgfqpoint{1.569121in}{1.270925in}}%
\pgfpathlineto{\pgfqpoint{1.565971in}{1.269535in}}%
\pgfpathlineto{\pgfqpoint{1.550314in}{1.264717in}}%
\pgfpathlineto{\pgfqpoint{1.534657in}{1.262043in}}%
\pgfpathlineto{\pgfqpoint{1.519001in}{1.261508in}}%
\pgfpathlineto{\pgfqpoint{1.503344in}{1.263113in}}%
\pgfpathlineto{\pgfqpoint{1.487688in}{1.266858in}}%
\pgfpathlineto{\pgfqpoint{1.476850in}{1.270925in}}%
\pgfpathclose%
\pgfusepath{fill}%
\end{pgfscope}%
\begin{pgfscope}%
\pgfpathrectangle{\pgfqpoint{0.360415in}{0.345370in}}{\pgfqpoint{1.550000in}{1.347500in}}%
\pgfusepath{clip}%
\pgfsetbuttcap%
\pgfsetroundjoin%
\definecolor{currentfill}{rgb}{0.959424,0.543431,0.278701}%
\pgfsetfillcolor{currentfill}%
\pgfsetlinewidth{0.000000pt}%
\definecolor{currentstroke}{rgb}{0.000000,0.000000,0.000000}%
\pgfsetstrokecolor{currentstroke}%
\pgfsetdash{}{0pt}%
\pgfpathmoveto{\pgfqpoint{0.689203in}{0.507221in}}%
\pgfpathlineto{\pgfqpoint{0.704859in}{0.502721in}}%
\pgfpathlineto{\pgfqpoint{0.720516in}{0.499608in}}%
\pgfpathlineto{\pgfqpoint{0.736173in}{0.497880in}}%
\pgfpathlineto{\pgfqpoint{0.751829in}{0.497534in}}%
\pgfpathlineto{\pgfqpoint{0.767486in}{0.498571in}}%
\pgfpathlineto{\pgfqpoint{0.783142in}{0.500991in}}%
\pgfpathlineto{\pgfqpoint{0.798799in}{0.504797in}}%
\pgfpathlineto{\pgfqpoint{0.810601in}{0.508703in}}%
\pgfpathlineto{\pgfqpoint{0.814455in}{0.509984in}}%
\pgfpathlineto{\pgfqpoint{0.830112in}{0.516526in}}%
\pgfpathlineto{\pgfqpoint{0.841601in}{0.522314in}}%
\pgfpathlineto{\pgfqpoint{0.845769in}{0.524468in}}%
\pgfpathlineto{\pgfqpoint{0.861425in}{0.533850in}}%
\pgfpathlineto{\pgfqpoint{0.864499in}{0.535925in}}%
\pgfpathlineto{\pgfqpoint{0.877082in}{0.544827in}}%
\pgfpathlineto{\pgfqpoint{0.883113in}{0.549536in}}%
\pgfpathlineto{\pgfqpoint{0.892738in}{0.557569in}}%
\pgfpathlineto{\pgfqpoint{0.898930in}{0.563148in}}%
\pgfpathlineto{\pgfqpoint{0.908395in}{0.572454in}}%
\pgfpathlineto{\pgfqpoint{0.912532in}{0.576759in}}%
\pgfpathlineto{\pgfqpoint{0.924051in}{0.590145in}}%
\pgfpathlineto{\pgfqpoint{0.924237in}{0.590370in}}%
\pgfpathlineto{\pgfqpoint{0.934167in}{0.603981in}}%
\pgfpathlineto{\pgfqpoint{0.939708in}{0.612976in}}%
\pgfpathlineto{\pgfqpoint{0.942505in}{0.617592in}}%
\pgfpathlineto{\pgfqpoint{0.949245in}{0.631203in}}%
\pgfpathlineto{\pgfqpoint{0.954394in}{0.644814in}}%
\pgfpathlineto{\pgfqpoint{0.955364in}{0.648506in}}%
\pgfpathlineto{\pgfqpoint{0.957994in}{0.658425in}}%
\pgfpathlineto{\pgfqpoint{0.960001in}{0.672036in}}%
\pgfpathlineto{\pgfqpoint{0.960402in}{0.685648in}}%
\pgfpathlineto{\pgfqpoint{0.959198in}{0.699259in}}%
\pgfpathlineto{\pgfqpoint{0.956388in}{0.712870in}}%
\pgfpathlineto{\pgfqpoint{0.955364in}{0.716041in}}%
\pgfpathlineto{\pgfqpoint{0.952018in}{0.726481in}}%
\pgfpathlineto{\pgfqpoint{0.946074in}{0.740092in}}%
\pgfpathlineto{\pgfqpoint{0.939708in}{0.751616in}}%
\pgfpathlineto{\pgfqpoint{0.938537in}{0.753703in}}%
\pgfpathlineto{\pgfqpoint{0.929400in}{0.767314in}}%
\pgfpathlineto{\pgfqpoint{0.924051in}{0.774169in}}%
\pgfpathlineto{\pgfqpoint{0.918585in}{0.780925in}}%
\pgfpathlineto{\pgfqpoint{0.908395in}{0.792065in}}%
\pgfpathlineto{\pgfqpoint{0.906005in}{0.794536in}}%
\pgfpathlineto{\pgfqpoint{0.892738in}{0.806977in}}%
\pgfpathlineto{\pgfqpoint{0.891392in}{0.808148in}}%
\pgfpathlineto{\pgfqpoint{0.877082in}{0.819681in}}%
\pgfpathlineto{\pgfqpoint{0.874238in}{0.821759in}}%
\pgfpathlineto{\pgfqpoint{0.861425in}{0.830618in}}%
\pgfpathlineto{\pgfqpoint{0.853654in}{0.835370in}}%
\pgfpathlineto{\pgfqpoint{0.845769in}{0.840020in}}%
\pgfpathlineto{\pgfqpoint{0.830112in}{0.847963in}}%
\pgfpathlineto{\pgfqpoint{0.827711in}{0.848981in}}%
\pgfpathlineto{\pgfqpoint{0.814455in}{0.854515in}}%
\pgfpathlineto{\pgfqpoint{0.798799in}{0.859683in}}%
\pgfpathlineto{\pgfqpoint{0.786790in}{0.862592in}}%
\pgfpathlineto{\pgfqpoint{0.783142in}{0.863482in}}%
\pgfpathlineto{\pgfqpoint{0.767486in}{0.865925in}}%
\pgfpathlineto{\pgfqpoint{0.751829in}{0.866971in}}%
\pgfpathlineto{\pgfqpoint{0.736173in}{0.866622in}}%
\pgfpathlineto{\pgfqpoint{0.720516in}{0.864878in}}%
\pgfpathlineto{\pgfqpoint{0.709106in}{0.862592in}}%
\pgfpathlineto{\pgfqpoint{0.704859in}{0.861748in}}%
\pgfpathlineto{\pgfqpoint{0.689203in}{0.857272in}}%
\pgfpathlineto{\pgfqpoint{0.673546in}{0.851413in}}%
\pgfpathlineto{\pgfqpoint{0.668236in}{0.848981in}}%
\pgfpathlineto{\pgfqpoint{0.657890in}{0.844164in}}%
\pgfpathlineto{\pgfqpoint{0.642233in}{0.835531in}}%
\pgfpathlineto{\pgfqpoint{0.641975in}{0.835370in}}%
\pgfpathlineto{\pgfqpoint{0.626577in}{0.825355in}}%
\pgfpathlineto{\pgfqpoint{0.621625in}{0.821759in}}%
\pgfpathlineto{\pgfqpoint{0.610920in}{0.813530in}}%
\pgfpathlineto{\pgfqpoint{0.604503in}{0.808148in}}%
\pgfpathlineto{\pgfqpoint{0.595263in}{0.799779in}}%
\pgfpathlineto{\pgfqpoint{0.589847in}{0.794536in}}%
\pgfpathlineto{\pgfqpoint{0.579607in}{0.783597in}}%
\pgfpathlineto{\pgfqpoint{0.577220in}{0.780925in}}%
\pgfpathlineto{\pgfqpoint{0.566428in}{0.767314in}}%
\pgfpathlineto{\pgfqpoint{0.563950in}{0.763691in}}%
\pgfpathlineto{\pgfqpoint{0.557292in}{0.753703in}}%
\pgfpathlineto{\pgfqpoint{0.549767in}{0.740092in}}%
\pgfpathlineto{\pgfqpoint{0.548294in}{0.736741in}}%
\pgfpathlineto{\pgfqpoint{0.543801in}{0.726481in}}%
\pgfpathlineto{\pgfqpoint{0.539423in}{0.712870in}}%
\pgfpathlineto{\pgfqpoint{0.536639in}{0.699259in}}%
\pgfpathlineto{\pgfqpoint{0.535447in}{0.685648in}}%
\pgfpathlineto{\pgfqpoint{0.535844in}{0.672036in}}%
\pgfpathlineto{\pgfqpoint{0.537832in}{0.658425in}}%
\pgfpathlineto{\pgfqpoint{0.541412in}{0.644814in}}%
\pgfpathlineto{\pgfqpoint{0.546589in}{0.631203in}}%
\pgfpathlineto{\pgfqpoint{0.548294in}{0.627747in}}%
\pgfpathlineto{\pgfqpoint{0.553331in}{0.617592in}}%
\pgfpathlineto{\pgfqpoint{0.561650in}{0.603981in}}%
\pgfpathlineto{\pgfqpoint{0.563950in}{0.600777in}}%
\pgfpathlineto{\pgfqpoint{0.571626in}{0.590370in}}%
\pgfpathlineto{\pgfqpoint{0.579607in}{0.580933in}}%
\pgfpathlineto{\pgfqpoint{0.583302in}{0.576759in}}%
\pgfpathlineto{\pgfqpoint{0.595263in}{0.564654in}}%
\pgfpathlineto{\pgfqpoint{0.596853in}{0.563148in}}%
\pgfpathlineto{\pgfqpoint{0.610920in}{0.550918in}}%
\pgfpathlineto{\pgfqpoint{0.612653in}{0.549536in}}%
\pgfpathlineto{\pgfqpoint{0.626577in}{0.539138in}}%
\pgfpathlineto{\pgfqpoint{0.631378in}{0.535925in}}%
\pgfpathlineto{\pgfqpoint{0.642233in}{0.528987in}}%
\pgfpathlineto{\pgfqpoint{0.654205in}{0.522314in}}%
\pgfpathlineto{\pgfqpoint{0.657890in}{0.520315in}}%
\pgfpathlineto{\pgfqpoint{0.673546in}{0.513082in}}%
\pgfpathlineto{\pgfqpoint{0.685228in}{0.508703in}}%
\pgfpathlineto{\pgfqpoint{0.689203in}{0.507221in}}%
\pgfpathclose%
\pgfpathmoveto{\pgfqpoint{0.693689in}{0.549536in}}%
\pgfpathlineto{\pgfqpoint{0.689203in}{0.551058in}}%
\pgfpathlineto{\pgfqpoint{0.673546in}{0.557963in}}%
\pgfpathlineto{\pgfqpoint{0.663982in}{0.563148in}}%
\pgfpathlineto{\pgfqpoint{0.657890in}{0.566715in}}%
\pgfpathlineto{\pgfqpoint{0.643332in}{0.576759in}}%
\pgfpathlineto{\pgfqpoint{0.642233in}{0.577598in}}%
\pgfpathlineto{\pgfqpoint{0.627542in}{0.590370in}}%
\pgfpathlineto{\pgfqpoint{0.626577in}{0.591325in}}%
\pgfpathlineto{\pgfqpoint{0.615024in}{0.603981in}}%
\pgfpathlineto{\pgfqpoint{0.610920in}{0.609277in}}%
\pgfpathlineto{\pgfqpoint{0.604957in}{0.617592in}}%
\pgfpathlineto{\pgfqpoint{0.597014in}{0.631203in}}%
\pgfpathlineto{\pgfqpoint{0.595263in}{0.635103in}}%
\pgfpathlineto{\pgfqpoint{0.591129in}{0.644814in}}%
\pgfpathlineto{\pgfqpoint{0.587110in}{0.658425in}}%
\pgfpathlineto{\pgfqpoint{0.584878in}{0.672036in}}%
\pgfpathlineto{\pgfqpoint{0.584432in}{0.685648in}}%
\pgfpathlineto{\pgfqpoint{0.585771in}{0.699259in}}%
\pgfpathlineto{\pgfqpoint{0.588896in}{0.712870in}}%
\pgfpathlineto{\pgfqpoint{0.593810in}{0.726481in}}%
\pgfpathlineto{\pgfqpoint{0.595263in}{0.729455in}}%
\pgfpathlineto{\pgfqpoint{0.600751in}{0.740092in}}%
\pgfpathlineto{\pgfqpoint{0.609632in}{0.753703in}}%
\pgfpathlineto{\pgfqpoint{0.610920in}{0.755357in}}%
\pgfpathlineto{\pgfqpoint{0.620996in}{0.767314in}}%
\pgfpathlineto{\pgfqpoint{0.626577in}{0.773032in}}%
\pgfpathlineto{\pgfqpoint{0.635114in}{0.780925in}}%
\pgfpathlineto{\pgfqpoint{0.642233in}{0.786786in}}%
\pgfpathlineto{\pgfqpoint{0.652964in}{0.794536in}}%
\pgfpathlineto{\pgfqpoint{0.657890in}{0.797787in}}%
\pgfpathlineto{\pgfqpoint{0.673546in}{0.806571in}}%
\pgfpathlineto{\pgfqpoint{0.676987in}{0.808148in}}%
\pgfpathlineto{\pgfqpoint{0.689203in}{0.813394in}}%
\pgfpathlineto{\pgfqpoint{0.704859in}{0.818544in}}%
\pgfpathlineto{\pgfqpoint{0.718976in}{0.821759in}}%
\pgfpathlineto{\pgfqpoint{0.720516in}{0.822095in}}%
\pgfpathlineto{\pgfqpoint{0.736173in}{0.824006in}}%
\pgfpathlineto{\pgfqpoint{0.751829in}{0.824388in}}%
\pgfpathlineto{\pgfqpoint{0.767486in}{0.823242in}}%
\pgfpathlineto{\pgfqpoint{0.776183in}{0.821759in}}%
\pgfpathlineto{\pgfqpoint{0.783142in}{0.820524in}}%
\pgfpathlineto{\pgfqpoint{0.798799in}{0.816168in}}%
\pgfpathlineto{\pgfqpoint{0.814455in}{0.810222in}}%
\pgfpathlineto{\pgfqpoint{0.818806in}{0.808148in}}%
\pgfpathlineto{\pgfqpoint{0.830112in}{0.802388in}}%
\pgfpathlineto{\pgfqpoint{0.842927in}{0.794536in}}%
\pgfpathlineto{\pgfqpoint{0.845769in}{0.792633in}}%
\pgfpathlineto{\pgfqpoint{0.860877in}{0.780925in}}%
\pgfpathlineto{\pgfqpoint{0.861425in}{0.780449in}}%
\pgfpathlineto{\pgfqpoint{0.874892in}{0.767314in}}%
\pgfpathlineto{\pgfqpoint{0.877082in}{0.764844in}}%
\pgfpathlineto{\pgfqpoint{0.886114in}{0.753703in}}%
\pgfpathlineto{\pgfqpoint{0.892738in}{0.743874in}}%
\pgfpathlineto{\pgfqpoint{0.895124in}{0.740092in}}%
\pgfpathlineto{\pgfqpoint{0.901964in}{0.726481in}}%
\pgfpathlineto{\pgfqpoint{0.906975in}{0.712870in}}%
\pgfpathlineto{\pgfqpoint{0.908395in}{0.706820in}}%
\pgfpathlineto{\pgfqpoint{0.910101in}{0.699259in}}%
\pgfpathlineto{\pgfqpoint{0.911419in}{0.685648in}}%
\pgfpathlineto{\pgfqpoint{0.910980in}{0.672036in}}%
\pgfpathlineto{\pgfqpoint{0.908782in}{0.658425in}}%
\pgfpathlineto{\pgfqpoint{0.908395in}{0.657087in}}%
\pgfpathlineto{\pgfqpoint{0.904697in}{0.644814in}}%
\pgfpathlineto{\pgfqpoint{0.898773in}{0.631203in}}%
\pgfpathlineto{\pgfqpoint{0.892738in}{0.620583in}}%
\pgfpathlineto{\pgfqpoint{0.890925in}{0.617592in}}%
\pgfpathlineto{\pgfqpoint{0.880821in}{0.603981in}}%
\pgfpathlineto{\pgfqpoint{0.877082in}{0.599699in}}%
\pgfpathlineto{\pgfqpoint{0.868166in}{0.590370in}}%
\pgfpathlineto{\pgfqpoint{0.861425in}{0.584180in}}%
\pgfpathlineto{\pgfqpoint{0.852346in}{0.576759in}}%
\pgfpathlineto{\pgfqpoint{0.845769in}{0.571907in}}%
\pgfpathlineto{\pgfqpoint{0.832014in}{0.563148in}}%
\pgfpathlineto{\pgfqpoint{0.830112in}{0.562028in}}%
\pgfpathlineto{\pgfqpoint{0.814455in}{0.554307in}}%
\pgfpathlineto{\pgfqpoint{0.802220in}{0.549536in}}%
\pgfpathlineto{\pgfqpoint{0.798799in}{0.548273in}}%
\pgfpathlineto{\pgfqpoint{0.783142in}{0.544000in}}%
\pgfpathlineto{\pgfqpoint{0.767486in}{0.541284in}}%
\pgfpathlineto{\pgfqpoint{0.751829in}{0.540120in}}%
\pgfpathlineto{\pgfqpoint{0.736173in}{0.540508in}}%
\pgfpathlineto{\pgfqpoint{0.720516in}{0.542448in}}%
\pgfpathlineto{\pgfqpoint{0.704859in}{0.545942in}}%
\pgfpathlineto{\pgfqpoint{0.693689in}{0.549536in}}%
\pgfpathclose%
\pgfpathmoveto{\pgfqpoint{1.472031in}{0.504797in}}%
\pgfpathlineto{\pgfqpoint{1.487688in}{0.500991in}}%
\pgfpathlineto{\pgfqpoint{1.503344in}{0.498571in}}%
\pgfpathlineto{\pgfqpoint{1.519001in}{0.497534in}}%
\pgfpathlineto{\pgfqpoint{1.534657in}{0.497880in}}%
\pgfpathlineto{\pgfqpoint{1.550314in}{0.499608in}}%
\pgfpathlineto{\pgfqpoint{1.565971in}{0.502721in}}%
\pgfpathlineto{\pgfqpoint{1.581627in}{0.507221in}}%
\pgfpathlineto{\pgfqpoint{1.585602in}{0.508703in}}%
\pgfpathlineto{\pgfqpoint{1.597284in}{0.513082in}}%
\pgfpathlineto{\pgfqpoint{1.612940in}{0.520315in}}%
\pgfpathlineto{\pgfqpoint{1.616625in}{0.522314in}}%
\pgfpathlineto{\pgfqpoint{1.628597in}{0.528987in}}%
\pgfpathlineto{\pgfqpoint{1.639452in}{0.535925in}}%
\pgfpathlineto{\pgfqpoint{1.644253in}{0.539138in}}%
\pgfpathlineto{\pgfqpoint{1.658177in}{0.549536in}}%
\pgfpathlineto{\pgfqpoint{1.659910in}{0.550918in}}%
\pgfpathlineto{\pgfqpoint{1.673977in}{0.563148in}}%
\pgfpathlineto{\pgfqpoint{1.675567in}{0.564654in}}%
\pgfpathlineto{\pgfqpoint{1.687528in}{0.576759in}}%
\pgfpathlineto{\pgfqpoint{1.691223in}{0.580933in}}%
\pgfpathlineto{\pgfqpoint{1.699204in}{0.590370in}}%
\pgfpathlineto{\pgfqpoint{1.706880in}{0.600777in}}%
\pgfpathlineto{\pgfqpoint{1.709180in}{0.603981in}}%
\pgfpathlineto{\pgfqpoint{1.717499in}{0.617592in}}%
\pgfpathlineto{\pgfqpoint{1.722536in}{0.627747in}}%
\pgfpathlineto{\pgfqpoint{1.724241in}{0.631203in}}%
\pgfpathlineto{\pgfqpoint{1.729418in}{0.644814in}}%
\pgfpathlineto{\pgfqpoint{1.732998in}{0.658425in}}%
\pgfpathlineto{\pgfqpoint{1.734986in}{0.672036in}}%
\pgfpathlineto{\pgfqpoint{1.735383in}{0.685648in}}%
\pgfpathlineto{\pgfqpoint{1.734191in}{0.699259in}}%
\pgfpathlineto{\pgfqpoint{1.731407in}{0.712870in}}%
\pgfpathlineto{\pgfqpoint{1.727029in}{0.726481in}}%
\pgfpathlineto{\pgfqpoint{1.722536in}{0.736741in}}%
\pgfpathlineto{\pgfqpoint{1.721063in}{0.740092in}}%
\pgfpathlineto{\pgfqpoint{1.713538in}{0.753703in}}%
\pgfpathlineto{\pgfqpoint{1.706880in}{0.763691in}}%
\pgfpathlineto{\pgfqpoint{1.704402in}{0.767314in}}%
\pgfpathlineto{\pgfqpoint{1.693610in}{0.780925in}}%
\pgfpathlineto{\pgfqpoint{1.691223in}{0.783597in}}%
\pgfpathlineto{\pgfqpoint{1.680983in}{0.794536in}}%
\pgfpathlineto{\pgfqpoint{1.675567in}{0.799779in}}%
\pgfpathlineto{\pgfqpoint{1.666327in}{0.808148in}}%
\pgfpathlineto{\pgfqpoint{1.659910in}{0.813530in}}%
\pgfpathlineto{\pgfqpoint{1.649205in}{0.821759in}}%
\pgfpathlineto{\pgfqpoint{1.644253in}{0.825355in}}%
\pgfpathlineto{\pgfqpoint{1.628855in}{0.835370in}}%
\pgfpathlineto{\pgfqpoint{1.628597in}{0.835531in}}%
\pgfpathlineto{\pgfqpoint{1.612940in}{0.844164in}}%
\pgfpathlineto{\pgfqpoint{1.602594in}{0.848981in}}%
\pgfpathlineto{\pgfqpoint{1.597284in}{0.851413in}}%
\pgfpathlineto{\pgfqpoint{1.581627in}{0.857272in}}%
\pgfpathlineto{\pgfqpoint{1.565971in}{0.861748in}}%
\pgfpathlineto{\pgfqpoint{1.561724in}{0.862592in}}%
\pgfpathlineto{\pgfqpoint{1.550314in}{0.864878in}}%
\pgfpathlineto{\pgfqpoint{1.534657in}{0.866622in}}%
\pgfpathlineto{\pgfqpoint{1.519001in}{0.866971in}}%
\pgfpathlineto{\pgfqpoint{1.503344in}{0.865925in}}%
\pgfpathlineto{\pgfqpoint{1.487688in}{0.863482in}}%
\pgfpathlineto{\pgfqpoint{1.484040in}{0.862592in}}%
\pgfpathlineto{\pgfqpoint{1.472031in}{0.859683in}}%
\pgfpathlineto{\pgfqpoint{1.456375in}{0.854515in}}%
\pgfpathlineto{\pgfqpoint{1.443119in}{0.848981in}}%
\pgfpathlineto{\pgfqpoint{1.440718in}{0.847963in}}%
\pgfpathlineto{\pgfqpoint{1.425061in}{0.840020in}}%
\pgfpathlineto{\pgfqpoint{1.417176in}{0.835370in}}%
\pgfpathlineto{\pgfqpoint{1.409405in}{0.830618in}}%
\pgfpathlineto{\pgfqpoint{1.396592in}{0.821759in}}%
\pgfpathlineto{\pgfqpoint{1.393748in}{0.819681in}}%
\pgfpathlineto{\pgfqpoint{1.379438in}{0.808148in}}%
\pgfpathlineto{\pgfqpoint{1.378092in}{0.806977in}}%
\pgfpathlineto{\pgfqpoint{1.364825in}{0.794536in}}%
\pgfpathlineto{\pgfqpoint{1.362435in}{0.792065in}}%
\pgfpathlineto{\pgfqpoint{1.352245in}{0.780925in}}%
\pgfpathlineto{\pgfqpoint{1.346779in}{0.774169in}}%
\pgfpathlineto{\pgfqpoint{1.341430in}{0.767314in}}%
\pgfpathlineto{\pgfqpoint{1.332293in}{0.753703in}}%
\pgfpathlineto{\pgfqpoint{1.331122in}{0.751616in}}%
\pgfpathlineto{\pgfqpoint{1.324756in}{0.740092in}}%
\pgfpathlineto{\pgfqpoint{1.318812in}{0.726481in}}%
\pgfpathlineto{\pgfqpoint{1.315466in}{0.716041in}}%
\pgfpathlineto{\pgfqpoint{1.314442in}{0.712870in}}%
\pgfpathlineto{\pgfqpoint{1.311632in}{0.699259in}}%
\pgfpathlineto{\pgfqpoint{1.310428in}{0.685648in}}%
\pgfpathlineto{\pgfqpoint{1.310829in}{0.672036in}}%
\pgfpathlineto{\pgfqpoint{1.312836in}{0.658425in}}%
\pgfpathlineto{\pgfqpoint{1.315466in}{0.648506in}}%
\pgfpathlineto{\pgfqpoint{1.316436in}{0.644814in}}%
\pgfpathlineto{\pgfqpoint{1.321585in}{0.631203in}}%
\pgfpathlineto{\pgfqpoint{1.328325in}{0.617592in}}%
\pgfpathlineto{\pgfqpoint{1.331122in}{0.612976in}}%
\pgfpathlineto{\pgfqpoint{1.336663in}{0.603981in}}%
\pgfpathlineto{\pgfqpoint{1.346593in}{0.590370in}}%
\pgfpathlineto{\pgfqpoint{1.346779in}{0.590145in}}%
\pgfpathlineto{\pgfqpoint{1.358298in}{0.576759in}}%
\pgfpathlineto{\pgfqpoint{1.362435in}{0.572454in}}%
\pgfpathlineto{\pgfqpoint{1.371900in}{0.563148in}}%
\pgfpathlineto{\pgfqpoint{1.378092in}{0.557569in}}%
\pgfpathlineto{\pgfqpoint{1.387717in}{0.549536in}}%
\pgfpathlineto{\pgfqpoint{1.393748in}{0.544827in}}%
\pgfpathlineto{\pgfqpoint{1.406331in}{0.535925in}}%
\pgfpathlineto{\pgfqpoint{1.409405in}{0.533850in}}%
\pgfpathlineto{\pgfqpoint{1.425061in}{0.524468in}}%
\pgfpathlineto{\pgfqpoint{1.429229in}{0.522314in}}%
\pgfpathlineto{\pgfqpoint{1.440718in}{0.516526in}}%
\pgfpathlineto{\pgfqpoint{1.456375in}{0.509984in}}%
\pgfpathlineto{\pgfqpoint{1.460229in}{0.508703in}}%
\pgfpathlineto{\pgfqpoint{1.472031in}{0.504797in}}%
\pgfpathclose%
\pgfpathmoveto{\pgfqpoint{1.468610in}{0.549536in}}%
\pgfpathlineto{\pgfqpoint{1.456375in}{0.554307in}}%
\pgfpathlineto{\pgfqpoint{1.440718in}{0.562028in}}%
\pgfpathlineto{\pgfqpoint{1.438816in}{0.563148in}}%
\pgfpathlineto{\pgfqpoint{1.425061in}{0.571907in}}%
\pgfpathlineto{\pgfqpoint{1.418484in}{0.576759in}}%
\pgfpathlineto{\pgfqpoint{1.409405in}{0.584180in}}%
\pgfpathlineto{\pgfqpoint{1.402664in}{0.590370in}}%
\pgfpathlineto{\pgfqpoint{1.393748in}{0.599699in}}%
\pgfpathlineto{\pgfqpoint{1.390009in}{0.603981in}}%
\pgfpathlineto{\pgfqpoint{1.379905in}{0.617592in}}%
\pgfpathlineto{\pgfqpoint{1.378092in}{0.620583in}}%
\pgfpathlineto{\pgfqpoint{1.372057in}{0.631203in}}%
\pgfpathlineto{\pgfqpoint{1.366133in}{0.644814in}}%
\pgfpathlineto{\pgfqpoint{1.362435in}{0.657087in}}%
\pgfpathlineto{\pgfqpoint{1.362048in}{0.658425in}}%
\pgfpathlineto{\pgfqpoint{1.359850in}{0.672036in}}%
\pgfpathlineto{\pgfqpoint{1.359411in}{0.685648in}}%
\pgfpathlineto{\pgfqpoint{1.360729in}{0.699259in}}%
\pgfpathlineto{\pgfqpoint{1.362435in}{0.706820in}}%
\pgfpathlineto{\pgfqpoint{1.363855in}{0.712870in}}%
\pgfpathlineto{\pgfqpoint{1.368866in}{0.726481in}}%
\pgfpathlineto{\pgfqpoint{1.375706in}{0.740092in}}%
\pgfpathlineto{\pgfqpoint{1.378092in}{0.743874in}}%
\pgfpathlineto{\pgfqpoint{1.384716in}{0.753703in}}%
\pgfpathlineto{\pgfqpoint{1.393748in}{0.764844in}}%
\pgfpathlineto{\pgfqpoint{1.395938in}{0.767314in}}%
\pgfpathlineto{\pgfqpoint{1.409405in}{0.780449in}}%
\pgfpathlineto{\pgfqpoint{1.409953in}{0.780925in}}%
\pgfpathlineto{\pgfqpoint{1.425061in}{0.792633in}}%
\pgfpathlineto{\pgfqpoint{1.427903in}{0.794536in}}%
\pgfpathlineto{\pgfqpoint{1.440718in}{0.802388in}}%
\pgfpathlineto{\pgfqpoint{1.452024in}{0.808148in}}%
\pgfpathlineto{\pgfqpoint{1.456375in}{0.810222in}}%
\pgfpathlineto{\pgfqpoint{1.472031in}{0.816168in}}%
\pgfpathlineto{\pgfqpoint{1.487688in}{0.820524in}}%
\pgfpathlineto{\pgfqpoint{1.494647in}{0.821759in}}%
\pgfpathlineto{\pgfqpoint{1.503344in}{0.823242in}}%
\pgfpathlineto{\pgfqpoint{1.519001in}{0.824388in}}%
\pgfpathlineto{\pgfqpoint{1.534657in}{0.824006in}}%
\pgfpathlineto{\pgfqpoint{1.550314in}{0.822095in}}%
\pgfpathlineto{\pgfqpoint{1.551854in}{0.821759in}}%
\pgfpathlineto{\pgfqpoint{1.565971in}{0.818544in}}%
\pgfpathlineto{\pgfqpoint{1.581627in}{0.813394in}}%
\pgfpathlineto{\pgfqpoint{1.593843in}{0.808148in}}%
\pgfpathlineto{\pgfqpoint{1.597284in}{0.806571in}}%
\pgfpathlineto{\pgfqpoint{1.612940in}{0.797787in}}%
\pgfpathlineto{\pgfqpoint{1.617866in}{0.794536in}}%
\pgfpathlineto{\pgfqpoint{1.628597in}{0.786786in}}%
\pgfpathlineto{\pgfqpoint{1.635716in}{0.780925in}}%
\pgfpathlineto{\pgfqpoint{1.644253in}{0.773032in}}%
\pgfpathlineto{\pgfqpoint{1.649834in}{0.767314in}}%
\pgfpathlineto{\pgfqpoint{1.659910in}{0.755357in}}%
\pgfpathlineto{\pgfqpoint{1.661198in}{0.753703in}}%
\pgfpathlineto{\pgfqpoint{1.670079in}{0.740092in}}%
\pgfpathlineto{\pgfqpoint{1.675567in}{0.729455in}}%
\pgfpathlineto{\pgfqpoint{1.677020in}{0.726481in}}%
\pgfpathlineto{\pgfqpoint{1.681934in}{0.712870in}}%
\pgfpathlineto{\pgfqpoint{1.685059in}{0.699259in}}%
\pgfpathlineto{\pgfqpoint{1.686398in}{0.685648in}}%
\pgfpathlineto{\pgfqpoint{1.685952in}{0.672036in}}%
\pgfpathlineto{\pgfqpoint{1.683720in}{0.658425in}}%
\pgfpathlineto{\pgfqpoint{1.679701in}{0.644814in}}%
\pgfpathlineto{\pgfqpoint{1.675567in}{0.635103in}}%
\pgfpathlineto{\pgfqpoint{1.673816in}{0.631203in}}%
\pgfpathlineto{\pgfqpoint{1.665873in}{0.617592in}}%
\pgfpathlineto{\pgfqpoint{1.659910in}{0.609277in}}%
\pgfpathlineto{\pgfqpoint{1.655806in}{0.603981in}}%
\pgfpathlineto{\pgfqpoint{1.644253in}{0.591325in}}%
\pgfpathlineto{\pgfqpoint{1.643288in}{0.590370in}}%
\pgfpathlineto{\pgfqpoint{1.628597in}{0.577598in}}%
\pgfpathlineto{\pgfqpoint{1.627498in}{0.576759in}}%
\pgfpathlineto{\pgfqpoint{1.612940in}{0.566715in}}%
\pgfpathlineto{\pgfqpoint{1.606848in}{0.563148in}}%
\pgfpathlineto{\pgfqpoint{1.597284in}{0.557963in}}%
\pgfpathlineto{\pgfqpoint{1.581627in}{0.551058in}}%
\pgfpathlineto{\pgfqpoint{1.577141in}{0.549536in}}%
\pgfpathlineto{\pgfqpoint{1.565971in}{0.545942in}}%
\pgfpathlineto{\pgfqpoint{1.550314in}{0.542448in}}%
\pgfpathlineto{\pgfqpoint{1.534657in}{0.540508in}}%
\pgfpathlineto{\pgfqpoint{1.519001in}{0.540120in}}%
\pgfpathlineto{\pgfqpoint{1.503344in}{0.541284in}}%
\pgfpathlineto{\pgfqpoint{1.487688in}{0.544000in}}%
\pgfpathlineto{\pgfqpoint{1.472031in}{0.548273in}}%
\pgfpathlineto{\pgfqpoint{1.468610in}{0.549536in}}%
\pgfpathclose%
\pgfpathmoveto{\pgfqpoint{0.720516in}{1.173362in}}%
\pgfpathlineto{\pgfqpoint{0.736173in}{1.171617in}}%
\pgfpathlineto{\pgfqpoint{0.751829in}{1.171268in}}%
\pgfpathlineto{\pgfqpoint{0.767486in}{1.172315in}}%
\pgfpathlineto{\pgfqpoint{0.783142in}{1.174758in}}%
\pgfpathlineto{\pgfqpoint{0.786790in}{1.175647in}}%
\pgfpathlineto{\pgfqpoint{0.798799in}{1.178557in}}%
\pgfpathlineto{\pgfqpoint{0.814455in}{1.183724in}}%
\pgfpathlineto{\pgfqpoint{0.827711in}{1.189259in}}%
\pgfpathlineto{\pgfqpoint{0.830112in}{1.190276in}}%
\pgfpathlineto{\pgfqpoint{0.845769in}{1.198220in}}%
\pgfpathlineto{\pgfqpoint{0.853654in}{1.202870in}}%
\pgfpathlineto{\pgfqpoint{0.861425in}{1.207622in}}%
\pgfpathlineto{\pgfqpoint{0.874238in}{1.216481in}}%
\pgfpathlineto{\pgfqpoint{0.877082in}{1.218559in}}%
\pgfpathlineto{\pgfqpoint{0.891392in}{1.230092in}}%
\pgfpathlineto{\pgfqpoint{0.892738in}{1.231262in}}%
\pgfpathlineto{\pgfqpoint{0.906005in}{1.243703in}}%
\pgfpathlineto{\pgfqpoint{0.908395in}{1.246175in}}%
\pgfpathlineto{\pgfqpoint{0.918585in}{1.257314in}}%
\pgfpathlineto{\pgfqpoint{0.924051in}{1.264070in}}%
\pgfpathlineto{\pgfqpoint{0.929400in}{1.270925in}}%
\pgfpathlineto{\pgfqpoint{0.938537in}{1.284536in}}%
\pgfpathlineto{\pgfqpoint{0.939708in}{1.286624in}}%
\pgfpathlineto{\pgfqpoint{0.946074in}{1.298148in}}%
\pgfpathlineto{\pgfqpoint{0.952018in}{1.311759in}}%
\pgfpathlineto{\pgfqpoint{0.955364in}{1.322199in}}%
\pgfpathlineto{\pgfqpoint{0.956388in}{1.325370in}}%
\pgfpathlineto{\pgfqpoint{0.959198in}{1.338981in}}%
\pgfpathlineto{\pgfqpoint{0.960402in}{1.352592in}}%
\pgfpathlineto{\pgfqpoint{0.960001in}{1.366203in}}%
\pgfpathlineto{\pgfqpoint{0.957994in}{1.379814in}}%
\pgfpathlineto{\pgfqpoint{0.955364in}{1.389733in}}%
\pgfpathlineto{\pgfqpoint{0.954394in}{1.393425in}}%
\pgfpathlineto{\pgfqpoint{0.949245in}{1.407036in}}%
\pgfpathlineto{\pgfqpoint{0.942505in}{1.420648in}}%
\pgfpathlineto{\pgfqpoint{0.939708in}{1.425264in}}%
\pgfpathlineto{\pgfqpoint{0.934167in}{1.434259in}}%
\pgfpathlineto{\pgfqpoint{0.924237in}{1.447870in}}%
\pgfpathlineto{\pgfqpoint{0.924051in}{1.448094in}}%
\pgfpathlineto{\pgfqpoint{0.912532in}{1.461481in}}%
\pgfpathlineto{\pgfqpoint{0.908395in}{1.465786in}}%
\pgfpathlineto{\pgfqpoint{0.898930in}{1.475092in}}%
\pgfpathlineto{\pgfqpoint{0.892738in}{1.480671in}}%
\pgfpathlineto{\pgfqpoint{0.883113in}{1.488703in}}%
\pgfpathlineto{\pgfqpoint{0.877082in}{1.493412in}}%
\pgfpathlineto{\pgfqpoint{0.864499in}{1.502314in}}%
\pgfpathlineto{\pgfqpoint{0.861425in}{1.504389in}}%
\pgfpathlineto{\pgfqpoint{0.845769in}{1.513771in}}%
\pgfpathlineto{\pgfqpoint{0.841601in}{1.515925in}}%
\pgfpathlineto{\pgfqpoint{0.830112in}{1.521714in}}%
\pgfpathlineto{\pgfqpoint{0.814455in}{1.528256in}}%
\pgfpathlineto{\pgfqpoint{0.810601in}{1.529536in}}%
\pgfpathlineto{\pgfqpoint{0.798799in}{1.533442in}}%
\pgfpathlineto{\pgfqpoint{0.783142in}{1.537248in}}%
\pgfpathlineto{\pgfqpoint{0.767486in}{1.539669in}}%
\pgfpathlineto{\pgfqpoint{0.751829in}{1.540705in}}%
\pgfpathlineto{\pgfqpoint{0.736173in}{1.540360in}}%
\pgfpathlineto{\pgfqpoint{0.720516in}{1.538632in}}%
\pgfpathlineto{\pgfqpoint{0.704859in}{1.535519in}}%
\pgfpathlineto{\pgfqpoint{0.689203in}{1.531018in}}%
\pgfpathlineto{\pgfqpoint{0.685228in}{1.529536in}}%
\pgfpathlineto{\pgfqpoint{0.673546in}{1.525157in}}%
\pgfpathlineto{\pgfqpoint{0.657890in}{1.517925in}}%
\pgfpathlineto{\pgfqpoint{0.654205in}{1.515925in}}%
\pgfpathlineto{\pgfqpoint{0.642233in}{1.509253in}}%
\pgfpathlineto{\pgfqpoint{0.631378in}{1.502314in}}%
\pgfpathlineto{\pgfqpoint{0.626577in}{1.499101in}}%
\pgfpathlineto{\pgfqpoint{0.612653in}{1.488703in}}%
\pgfpathlineto{\pgfqpoint{0.610920in}{1.487321in}}%
\pgfpathlineto{\pgfqpoint{0.596853in}{1.475092in}}%
\pgfpathlineto{\pgfqpoint{0.595263in}{1.473585in}}%
\pgfpathlineto{\pgfqpoint{0.583302in}{1.461481in}}%
\pgfpathlineto{\pgfqpoint{0.579607in}{1.457307in}}%
\pgfpathlineto{\pgfqpoint{0.571626in}{1.447870in}}%
\pgfpathlineto{\pgfqpoint{0.563950in}{1.437462in}}%
\pgfpathlineto{\pgfqpoint{0.561650in}{1.434259in}}%
\pgfpathlineto{\pgfqpoint{0.553331in}{1.420648in}}%
\pgfpathlineto{\pgfqpoint{0.548294in}{1.410492in}}%
\pgfpathlineto{\pgfqpoint{0.546589in}{1.407036in}}%
\pgfpathlineto{\pgfqpoint{0.541412in}{1.393425in}}%
\pgfpathlineto{\pgfqpoint{0.537832in}{1.379814in}}%
\pgfpathlineto{\pgfqpoint{0.535844in}{1.366203in}}%
\pgfpathlineto{\pgfqpoint{0.535447in}{1.352592in}}%
\pgfpathlineto{\pgfqpoint{0.536639in}{1.338981in}}%
\pgfpathlineto{\pgfqpoint{0.539423in}{1.325370in}}%
\pgfpathlineto{\pgfqpoint{0.543801in}{1.311759in}}%
\pgfpathlineto{\pgfqpoint{0.548294in}{1.301498in}}%
\pgfpathlineto{\pgfqpoint{0.549767in}{1.298148in}}%
\pgfpathlineto{\pgfqpoint{0.557292in}{1.284536in}}%
\pgfpathlineto{\pgfqpoint{0.563950in}{1.274548in}}%
\pgfpathlineto{\pgfqpoint{0.566428in}{1.270925in}}%
\pgfpathlineto{\pgfqpoint{0.577220in}{1.257314in}}%
\pgfpathlineto{\pgfqpoint{0.579607in}{1.254642in}}%
\pgfpathlineto{\pgfqpoint{0.589847in}{1.243703in}}%
\pgfpathlineto{\pgfqpoint{0.595263in}{1.238460in}}%
\pgfpathlineto{\pgfqpoint{0.604503in}{1.230092in}}%
\pgfpathlineto{\pgfqpoint{0.610920in}{1.224709in}}%
\pgfpathlineto{\pgfqpoint{0.621625in}{1.216481in}}%
\pgfpathlineto{\pgfqpoint{0.626577in}{1.212884in}}%
\pgfpathlineto{\pgfqpoint{0.641975in}{1.202870in}}%
\pgfpathlineto{\pgfqpoint{0.642233in}{1.202708in}}%
\pgfpathlineto{\pgfqpoint{0.657890in}{1.194075in}}%
\pgfpathlineto{\pgfqpoint{0.668236in}{1.189259in}}%
\pgfpathlineto{\pgfqpoint{0.673546in}{1.186827in}}%
\pgfpathlineto{\pgfqpoint{0.689203in}{1.180967in}}%
\pgfpathlineto{\pgfqpoint{0.704859in}{1.176491in}}%
\pgfpathlineto{\pgfqpoint{0.709106in}{1.175647in}}%
\pgfpathlineto{\pgfqpoint{0.720516in}{1.173362in}}%
\pgfpathclose%
\pgfpathmoveto{\pgfqpoint{0.718976in}{1.216481in}}%
\pgfpathlineto{\pgfqpoint{0.704859in}{1.219695in}}%
\pgfpathlineto{\pgfqpoint{0.689203in}{1.224846in}}%
\pgfpathlineto{\pgfqpoint{0.676987in}{1.230092in}}%
\pgfpathlineto{\pgfqpoint{0.673546in}{1.231669in}}%
\pgfpathlineto{\pgfqpoint{0.657890in}{1.240453in}}%
\pgfpathlineto{\pgfqpoint{0.652964in}{1.243703in}}%
\pgfpathlineto{\pgfqpoint{0.642233in}{1.251454in}}%
\pgfpathlineto{\pgfqpoint{0.635114in}{1.257314in}}%
\pgfpathlineto{\pgfqpoint{0.626577in}{1.265207in}}%
\pgfpathlineto{\pgfqpoint{0.620996in}{1.270925in}}%
\pgfpathlineto{\pgfqpoint{0.610920in}{1.282883in}}%
\pgfpathlineto{\pgfqpoint{0.609632in}{1.284536in}}%
\pgfpathlineto{\pgfqpoint{0.600751in}{1.298148in}}%
\pgfpathlineto{\pgfqpoint{0.595263in}{1.308784in}}%
\pgfpathlineto{\pgfqpoint{0.593810in}{1.311759in}}%
\pgfpathlineto{\pgfqpoint{0.588896in}{1.325370in}}%
\pgfpathlineto{\pgfqpoint{0.585771in}{1.338981in}}%
\pgfpathlineto{\pgfqpoint{0.584432in}{1.352592in}}%
\pgfpathlineto{\pgfqpoint{0.584878in}{1.366203in}}%
\pgfpathlineto{\pgfqpoint{0.587110in}{1.379814in}}%
\pgfpathlineto{\pgfqpoint{0.591129in}{1.393425in}}%
\pgfpathlineto{\pgfqpoint{0.595263in}{1.403136in}}%
\pgfpathlineto{\pgfqpoint{0.597014in}{1.407036in}}%
\pgfpathlineto{\pgfqpoint{0.604957in}{1.420648in}}%
\pgfpathlineto{\pgfqpoint{0.610920in}{1.428962in}}%
\pgfpathlineto{\pgfqpoint{0.615024in}{1.434259in}}%
\pgfpathlineto{\pgfqpoint{0.626577in}{1.446914in}}%
\pgfpathlineto{\pgfqpoint{0.627542in}{1.447870in}}%
\pgfpathlineto{\pgfqpoint{0.642233in}{1.460642in}}%
\pgfpathlineto{\pgfqpoint{0.643332in}{1.461481in}}%
\pgfpathlineto{\pgfqpoint{0.657890in}{1.471525in}}%
\pgfpathlineto{\pgfqpoint{0.663982in}{1.475092in}}%
\pgfpathlineto{\pgfqpoint{0.673546in}{1.480276in}}%
\pgfpathlineto{\pgfqpoint{0.689203in}{1.487181in}}%
\pgfpathlineto{\pgfqpoint{0.693689in}{1.488703in}}%
\pgfpathlineto{\pgfqpoint{0.704859in}{1.492298in}}%
\pgfpathlineto{\pgfqpoint{0.720516in}{1.495792in}}%
\pgfpathlineto{\pgfqpoint{0.736173in}{1.497731in}}%
\pgfpathlineto{\pgfqpoint{0.751829in}{1.498119in}}%
\pgfpathlineto{\pgfqpoint{0.767486in}{1.496956in}}%
\pgfpathlineto{\pgfqpoint{0.783142in}{1.494239in}}%
\pgfpathlineto{\pgfqpoint{0.798799in}{1.489967in}}%
\pgfpathlineto{\pgfqpoint{0.802220in}{1.488703in}}%
\pgfpathlineto{\pgfqpoint{0.814455in}{1.483932in}}%
\pgfpathlineto{\pgfqpoint{0.830112in}{1.476212in}}%
\pgfpathlineto{\pgfqpoint{0.832014in}{1.475092in}}%
\pgfpathlineto{\pgfqpoint{0.845769in}{1.466333in}}%
\pgfpathlineto{\pgfqpoint{0.852346in}{1.461481in}}%
\pgfpathlineto{\pgfqpoint{0.861425in}{1.454059in}}%
\pgfpathlineto{\pgfqpoint{0.868166in}{1.447870in}}%
\pgfpathlineto{\pgfqpoint{0.877082in}{1.438541in}}%
\pgfpathlineto{\pgfqpoint{0.880821in}{1.434259in}}%
\pgfpathlineto{\pgfqpoint{0.890925in}{1.420648in}}%
\pgfpathlineto{\pgfqpoint{0.892738in}{1.417656in}}%
\pgfpathlineto{\pgfqpoint{0.898773in}{1.407036in}}%
\pgfpathlineto{\pgfqpoint{0.904697in}{1.393425in}}%
\pgfpathlineto{\pgfqpoint{0.908395in}{1.381153in}}%
\pgfpathlineto{\pgfqpoint{0.908782in}{1.379814in}}%
\pgfpathlineto{\pgfqpoint{0.910980in}{1.366203in}}%
\pgfpathlineto{\pgfqpoint{0.911419in}{1.352592in}}%
\pgfpathlineto{\pgfqpoint{0.910101in}{1.338981in}}%
\pgfpathlineto{\pgfqpoint{0.908395in}{1.331420in}}%
\pgfpathlineto{\pgfqpoint{0.906975in}{1.325370in}}%
\pgfpathlineto{\pgfqpoint{0.901964in}{1.311759in}}%
\pgfpathlineto{\pgfqpoint{0.895124in}{1.298148in}}%
\pgfpathlineto{\pgfqpoint{0.892738in}{1.294365in}}%
\pgfpathlineto{\pgfqpoint{0.886114in}{1.284536in}}%
\pgfpathlineto{\pgfqpoint{0.877082in}{1.273396in}}%
\pgfpathlineto{\pgfqpoint{0.874892in}{1.270925in}}%
\pgfpathlineto{\pgfqpoint{0.861425in}{1.257791in}}%
\pgfpathlineto{\pgfqpoint{0.860877in}{1.257314in}}%
\pgfpathlineto{\pgfqpoint{0.845769in}{1.245606in}}%
\pgfpathlineto{\pgfqpoint{0.842927in}{1.243703in}}%
\pgfpathlineto{\pgfqpoint{0.830112in}{1.235851in}}%
\pgfpathlineto{\pgfqpoint{0.818806in}{1.230092in}}%
\pgfpathlineto{\pgfqpoint{0.814455in}{1.228018in}}%
\pgfpathlineto{\pgfqpoint{0.798799in}{1.222072in}}%
\pgfpathlineto{\pgfqpoint{0.783142in}{1.217716in}}%
\pgfpathlineto{\pgfqpoint{0.776183in}{1.216481in}}%
\pgfpathlineto{\pgfqpoint{0.767486in}{1.214998in}}%
\pgfpathlineto{\pgfqpoint{0.751829in}{1.213852in}}%
\pgfpathlineto{\pgfqpoint{0.736173in}{1.214234in}}%
\pgfpathlineto{\pgfqpoint{0.720516in}{1.216144in}}%
\pgfpathlineto{\pgfqpoint{0.718976in}{1.216481in}}%
\pgfpathclose%
\pgfpathmoveto{\pgfqpoint{1.487688in}{1.174758in}}%
\pgfpathlineto{\pgfqpoint{1.503344in}{1.172315in}}%
\pgfpathlineto{\pgfqpoint{1.519001in}{1.171268in}}%
\pgfpathlineto{\pgfqpoint{1.534657in}{1.171617in}}%
\pgfpathlineto{\pgfqpoint{1.550314in}{1.173362in}}%
\pgfpathlineto{\pgfqpoint{1.561724in}{1.175647in}}%
\pgfpathlineto{\pgfqpoint{1.565971in}{1.176491in}}%
\pgfpathlineto{\pgfqpoint{1.581627in}{1.180967in}}%
\pgfpathlineto{\pgfqpoint{1.597284in}{1.186827in}}%
\pgfpathlineto{\pgfqpoint{1.602594in}{1.189259in}}%
\pgfpathlineto{\pgfqpoint{1.612940in}{1.194075in}}%
\pgfpathlineto{\pgfqpoint{1.628597in}{1.202708in}}%
\pgfpathlineto{\pgfqpoint{1.628855in}{1.202870in}}%
\pgfpathlineto{\pgfqpoint{1.644253in}{1.212884in}}%
\pgfpathlineto{\pgfqpoint{1.649205in}{1.216481in}}%
\pgfpathlineto{\pgfqpoint{1.659910in}{1.224709in}}%
\pgfpathlineto{\pgfqpoint{1.666327in}{1.230092in}}%
\pgfpathlineto{\pgfqpoint{1.675567in}{1.238460in}}%
\pgfpathlineto{\pgfqpoint{1.680983in}{1.243703in}}%
\pgfpathlineto{\pgfqpoint{1.691223in}{1.254642in}}%
\pgfpathlineto{\pgfqpoint{1.693610in}{1.257314in}}%
\pgfpathlineto{\pgfqpoint{1.704402in}{1.270925in}}%
\pgfpathlineto{\pgfqpoint{1.706880in}{1.274548in}}%
\pgfpathlineto{\pgfqpoint{1.713538in}{1.284536in}}%
\pgfpathlineto{\pgfqpoint{1.721063in}{1.298148in}}%
\pgfpathlineto{\pgfqpoint{1.722536in}{1.301498in}}%
\pgfpathlineto{\pgfqpoint{1.727029in}{1.311759in}}%
\pgfpathlineto{\pgfqpoint{1.731407in}{1.325370in}}%
\pgfpathlineto{\pgfqpoint{1.734191in}{1.338981in}}%
\pgfpathlineto{\pgfqpoint{1.735383in}{1.352592in}}%
\pgfpathlineto{\pgfqpoint{1.734986in}{1.366203in}}%
\pgfpathlineto{\pgfqpoint{1.732998in}{1.379814in}}%
\pgfpathlineto{\pgfqpoint{1.729418in}{1.393425in}}%
\pgfpathlineto{\pgfqpoint{1.724241in}{1.407036in}}%
\pgfpathlineto{\pgfqpoint{1.722536in}{1.410492in}}%
\pgfpathlineto{\pgfqpoint{1.717499in}{1.420648in}}%
\pgfpathlineto{\pgfqpoint{1.709180in}{1.434259in}}%
\pgfpathlineto{\pgfqpoint{1.706880in}{1.437462in}}%
\pgfpathlineto{\pgfqpoint{1.699204in}{1.447870in}}%
\pgfpathlineto{\pgfqpoint{1.691223in}{1.457307in}}%
\pgfpathlineto{\pgfqpoint{1.687528in}{1.461481in}}%
\pgfpathlineto{\pgfqpoint{1.675567in}{1.473585in}}%
\pgfpathlineto{\pgfqpoint{1.673977in}{1.475092in}}%
\pgfpathlineto{\pgfqpoint{1.659910in}{1.487321in}}%
\pgfpathlineto{\pgfqpoint{1.658177in}{1.488703in}}%
\pgfpathlineto{\pgfqpoint{1.644253in}{1.499101in}}%
\pgfpathlineto{\pgfqpoint{1.639452in}{1.502314in}}%
\pgfpathlineto{\pgfqpoint{1.628597in}{1.509253in}}%
\pgfpathlineto{\pgfqpoint{1.616625in}{1.515925in}}%
\pgfpathlineto{\pgfqpoint{1.612940in}{1.517925in}}%
\pgfpathlineto{\pgfqpoint{1.597284in}{1.525157in}}%
\pgfpathlineto{\pgfqpoint{1.585602in}{1.529536in}}%
\pgfpathlineto{\pgfqpoint{1.581627in}{1.531018in}}%
\pgfpathlineto{\pgfqpoint{1.565971in}{1.535519in}}%
\pgfpathlineto{\pgfqpoint{1.550314in}{1.538632in}}%
\pgfpathlineto{\pgfqpoint{1.534657in}{1.540360in}}%
\pgfpathlineto{\pgfqpoint{1.519001in}{1.540705in}}%
\pgfpathlineto{\pgfqpoint{1.503344in}{1.539669in}}%
\pgfpathlineto{\pgfqpoint{1.487688in}{1.537248in}}%
\pgfpathlineto{\pgfqpoint{1.472031in}{1.533442in}}%
\pgfpathlineto{\pgfqpoint{1.460229in}{1.529536in}}%
\pgfpathlineto{\pgfqpoint{1.456375in}{1.528256in}}%
\pgfpathlineto{\pgfqpoint{1.440718in}{1.521714in}}%
\pgfpathlineto{\pgfqpoint{1.429229in}{1.515925in}}%
\pgfpathlineto{\pgfqpoint{1.425061in}{1.513771in}}%
\pgfpathlineto{\pgfqpoint{1.409405in}{1.504389in}}%
\pgfpathlineto{\pgfqpoint{1.406331in}{1.502314in}}%
\pgfpathlineto{\pgfqpoint{1.393748in}{1.493412in}}%
\pgfpathlineto{\pgfqpoint{1.387717in}{1.488703in}}%
\pgfpathlineto{\pgfqpoint{1.378092in}{1.480671in}}%
\pgfpathlineto{\pgfqpoint{1.371900in}{1.475092in}}%
\pgfpathlineto{\pgfqpoint{1.362435in}{1.465786in}}%
\pgfpathlineto{\pgfqpoint{1.358298in}{1.461481in}}%
\pgfpathlineto{\pgfqpoint{1.346779in}{1.448094in}}%
\pgfpathlineto{\pgfqpoint{1.346593in}{1.447870in}}%
\pgfpathlineto{\pgfqpoint{1.336663in}{1.434259in}}%
\pgfpathlineto{\pgfqpoint{1.331122in}{1.425264in}}%
\pgfpathlineto{\pgfqpoint{1.328325in}{1.420648in}}%
\pgfpathlineto{\pgfqpoint{1.321585in}{1.407036in}}%
\pgfpathlineto{\pgfqpoint{1.316436in}{1.393425in}}%
\pgfpathlineto{\pgfqpoint{1.315466in}{1.389733in}}%
\pgfpathlineto{\pgfqpoint{1.312836in}{1.379814in}}%
\pgfpathlineto{\pgfqpoint{1.310829in}{1.366203in}}%
\pgfpathlineto{\pgfqpoint{1.310428in}{1.352592in}}%
\pgfpathlineto{\pgfqpoint{1.311632in}{1.338981in}}%
\pgfpathlineto{\pgfqpoint{1.314442in}{1.325370in}}%
\pgfpathlineto{\pgfqpoint{1.315466in}{1.322199in}}%
\pgfpathlineto{\pgfqpoint{1.318812in}{1.311759in}}%
\pgfpathlineto{\pgfqpoint{1.324756in}{1.298148in}}%
\pgfpathlineto{\pgfqpoint{1.331122in}{1.286624in}}%
\pgfpathlineto{\pgfqpoint{1.332293in}{1.284536in}}%
\pgfpathlineto{\pgfqpoint{1.341430in}{1.270925in}}%
\pgfpathlineto{\pgfqpoint{1.346779in}{1.264070in}}%
\pgfpathlineto{\pgfqpoint{1.352245in}{1.257314in}}%
\pgfpathlineto{\pgfqpoint{1.362435in}{1.246175in}}%
\pgfpathlineto{\pgfqpoint{1.364825in}{1.243703in}}%
\pgfpathlineto{\pgfqpoint{1.378092in}{1.231262in}}%
\pgfpathlineto{\pgfqpoint{1.379438in}{1.230092in}}%
\pgfpathlineto{\pgfqpoint{1.393748in}{1.218559in}}%
\pgfpathlineto{\pgfqpoint{1.396592in}{1.216481in}}%
\pgfpathlineto{\pgfqpoint{1.409405in}{1.207622in}}%
\pgfpathlineto{\pgfqpoint{1.417176in}{1.202870in}}%
\pgfpathlineto{\pgfqpoint{1.425061in}{1.198220in}}%
\pgfpathlineto{\pgfqpoint{1.440718in}{1.190276in}}%
\pgfpathlineto{\pgfqpoint{1.443119in}{1.189259in}}%
\pgfpathlineto{\pgfqpoint{1.456375in}{1.183724in}}%
\pgfpathlineto{\pgfqpoint{1.472031in}{1.178557in}}%
\pgfpathlineto{\pgfqpoint{1.484040in}{1.175647in}}%
\pgfpathlineto{\pgfqpoint{1.487688in}{1.174758in}}%
\pgfpathclose%
\pgfpathmoveto{\pgfqpoint{1.494647in}{1.216481in}}%
\pgfpathlineto{\pgfqpoint{1.487688in}{1.217716in}}%
\pgfpathlineto{\pgfqpoint{1.472031in}{1.222072in}}%
\pgfpathlineto{\pgfqpoint{1.456375in}{1.228018in}}%
\pgfpathlineto{\pgfqpoint{1.452024in}{1.230092in}}%
\pgfpathlineto{\pgfqpoint{1.440718in}{1.235851in}}%
\pgfpathlineto{\pgfqpoint{1.427903in}{1.243703in}}%
\pgfpathlineto{\pgfqpoint{1.425061in}{1.245606in}}%
\pgfpathlineto{\pgfqpoint{1.409953in}{1.257314in}}%
\pgfpathlineto{\pgfqpoint{1.409405in}{1.257791in}}%
\pgfpathlineto{\pgfqpoint{1.395938in}{1.270925in}}%
\pgfpathlineto{\pgfqpoint{1.393748in}{1.273396in}}%
\pgfpathlineto{\pgfqpoint{1.384716in}{1.284536in}}%
\pgfpathlineto{\pgfqpoint{1.378092in}{1.294365in}}%
\pgfpathlineto{\pgfqpoint{1.375706in}{1.298148in}}%
\pgfpathlineto{\pgfqpoint{1.368866in}{1.311759in}}%
\pgfpathlineto{\pgfqpoint{1.363855in}{1.325370in}}%
\pgfpathlineto{\pgfqpoint{1.362435in}{1.331420in}}%
\pgfpathlineto{\pgfqpoint{1.360729in}{1.338981in}}%
\pgfpathlineto{\pgfqpoint{1.359411in}{1.352592in}}%
\pgfpathlineto{\pgfqpoint{1.359850in}{1.366203in}}%
\pgfpathlineto{\pgfqpoint{1.362048in}{1.379814in}}%
\pgfpathlineto{\pgfqpoint{1.362435in}{1.381153in}}%
\pgfpathlineto{\pgfqpoint{1.366133in}{1.393425in}}%
\pgfpathlineto{\pgfqpoint{1.372057in}{1.407036in}}%
\pgfpathlineto{\pgfqpoint{1.378092in}{1.417656in}}%
\pgfpathlineto{\pgfqpoint{1.379905in}{1.420648in}}%
\pgfpathlineto{\pgfqpoint{1.390009in}{1.434259in}}%
\pgfpathlineto{\pgfqpoint{1.393748in}{1.438541in}}%
\pgfpathlineto{\pgfqpoint{1.402664in}{1.447870in}}%
\pgfpathlineto{\pgfqpoint{1.409405in}{1.454059in}}%
\pgfpathlineto{\pgfqpoint{1.418484in}{1.461481in}}%
\pgfpathlineto{\pgfqpoint{1.425061in}{1.466333in}}%
\pgfpathlineto{\pgfqpoint{1.438816in}{1.475092in}}%
\pgfpathlineto{\pgfqpoint{1.440718in}{1.476212in}}%
\pgfpathlineto{\pgfqpoint{1.456375in}{1.483932in}}%
\pgfpathlineto{\pgfqpoint{1.468610in}{1.488703in}}%
\pgfpathlineto{\pgfqpoint{1.472031in}{1.489967in}}%
\pgfpathlineto{\pgfqpoint{1.487688in}{1.494239in}}%
\pgfpathlineto{\pgfqpoint{1.503344in}{1.496956in}}%
\pgfpathlineto{\pgfqpoint{1.519001in}{1.498119in}}%
\pgfpathlineto{\pgfqpoint{1.534657in}{1.497731in}}%
\pgfpathlineto{\pgfqpoint{1.550314in}{1.495792in}}%
\pgfpathlineto{\pgfqpoint{1.565971in}{1.492298in}}%
\pgfpathlineto{\pgfqpoint{1.577141in}{1.488703in}}%
\pgfpathlineto{\pgfqpoint{1.581627in}{1.487181in}}%
\pgfpathlineto{\pgfqpoint{1.597284in}{1.480276in}}%
\pgfpathlineto{\pgfqpoint{1.606848in}{1.475092in}}%
\pgfpathlineto{\pgfqpoint{1.612940in}{1.471525in}}%
\pgfpathlineto{\pgfqpoint{1.627498in}{1.461481in}}%
\pgfpathlineto{\pgfqpoint{1.628597in}{1.460642in}}%
\pgfpathlineto{\pgfqpoint{1.643288in}{1.447870in}}%
\pgfpathlineto{\pgfqpoint{1.644253in}{1.446914in}}%
\pgfpathlineto{\pgfqpoint{1.655806in}{1.434259in}}%
\pgfpathlineto{\pgfqpoint{1.659910in}{1.428962in}}%
\pgfpathlineto{\pgfqpoint{1.665873in}{1.420648in}}%
\pgfpathlineto{\pgfqpoint{1.673816in}{1.407036in}}%
\pgfpathlineto{\pgfqpoint{1.675567in}{1.403136in}}%
\pgfpathlineto{\pgfqpoint{1.679701in}{1.393425in}}%
\pgfpathlineto{\pgfqpoint{1.683720in}{1.379814in}}%
\pgfpathlineto{\pgfqpoint{1.685952in}{1.366203in}}%
\pgfpathlineto{\pgfqpoint{1.686398in}{1.352592in}}%
\pgfpathlineto{\pgfqpoint{1.685059in}{1.338981in}}%
\pgfpathlineto{\pgfqpoint{1.681934in}{1.325370in}}%
\pgfpathlineto{\pgfqpoint{1.677020in}{1.311759in}}%
\pgfpathlineto{\pgfqpoint{1.675567in}{1.308784in}}%
\pgfpathlineto{\pgfqpoint{1.670079in}{1.298148in}}%
\pgfpathlineto{\pgfqpoint{1.661198in}{1.284536in}}%
\pgfpathlineto{\pgfqpoint{1.659910in}{1.282883in}}%
\pgfpathlineto{\pgfqpoint{1.649834in}{1.270925in}}%
\pgfpathlineto{\pgfqpoint{1.644253in}{1.265207in}}%
\pgfpathlineto{\pgfqpoint{1.635716in}{1.257314in}}%
\pgfpathlineto{\pgfqpoint{1.628597in}{1.251454in}}%
\pgfpathlineto{\pgfqpoint{1.617866in}{1.243703in}}%
\pgfpathlineto{\pgfqpoint{1.612940in}{1.240453in}}%
\pgfpathlineto{\pgfqpoint{1.597284in}{1.231669in}}%
\pgfpathlineto{\pgfqpoint{1.593843in}{1.230092in}}%
\pgfpathlineto{\pgfqpoint{1.581627in}{1.224846in}}%
\pgfpathlineto{\pgfqpoint{1.565971in}{1.219695in}}%
\pgfpathlineto{\pgfqpoint{1.551854in}{1.216481in}}%
\pgfpathlineto{\pgfqpoint{1.550314in}{1.216144in}}%
\pgfpathlineto{\pgfqpoint{1.534657in}{1.214234in}}%
\pgfpathlineto{\pgfqpoint{1.519001in}{1.213852in}}%
\pgfpathlineto{\pgfqpoint{1.503344in}{1.214998in}}%
\pgfpathlineto{\pgfqpoint{1.494647in}{1.216481in}}%
\pgfpathclose%
\pgfusepath{fill}%
\end{pgfscope}%
\begin{pgfscope}%
\pgfpathrectangle{\pgfqpoint{0.360415in}{0.345370in}}{\pgfqpoint{1.550000in}{1.347500in}}%
\pgfusepath{clip}%
\pgfsetbuttcap%
\pgfsetroundjoin%
\definecolor{currentfill}{rgb}{0.890340,0.406398,0.373130}%
\pgfsetfillcolor{currentfill}%
\pgfsetlinewidth{0.000000pt}%
\definecolor{currentstroke}{rgb}{0.000000,0.000000,0.000000}%
\pgfsetstrokecolor{currentstroke}%
\pgfsetdash{}{0pt}%
\pgfpathmoveto{\pgfqpoint{0.720516in}{0.454170in}}%
\pgfpathlineto{\pgfqpoint{0.736173in}{0.452248in}}%
\pgfpathlineto{\pgfqpoint{0.751829in}{0.451864in}}%
\pgfpathlineto{\pgfqpoint{0.767486in}{0.453017in}}%
\pgfpathlineto{\pgfqpoint{0.774731in}{0.454259in}}%
\pgfpathlineto{\pgfqpoint{0.783142in}{0.455597in}}%
\pgfpathlineto{\pgfqpoint{0.798799in}{0.459506in}}%
\pgfpathlineto{\pgfqpoint{0.814455in}{0.464842in}}%
\pgfpathlineto{\pgfqpoint{0.821515in}{0.467870in}}%
\pgfpathlineto{\pgfqpoint{0.830112in}{0.471394in}}%
\pgfpathlineto{\pgfqpoint{0.845769in}{0.479118in}}%
\pgfpathlineto{\pgfqpoint{0.849907in}{0.481481in}}%
\pgfpathlineto{\pgfqpoint{0.861425in}{0.487917in}}%
\pgfpathlineto{\pgfqpoint{0.872754in}{0.495092in}}%
\pgfpathlineto{\pgfqpoint{0.877082in}{0.497827in}}%
\pgfpathlineto{\pgfqpoint{0.892626in}{0.508703in}}%
\pgfpathlineto{\pgfqpoint{0.892738in}{0.508783in}}%
\pgfpathlineto{\pgfqpoint{0.908395in}{0.520818in}}%
\pgfpathlineto{\pgfqpoint{0.910223in}{0.522314in}}%
\pgfpathlineto{\pgfqpoint{0.924051in}{0.534044in}}%
\pgfpathlineto{\pgfqpoint{0.926172in}{0.535925in}}%
\pgfpathlineto{\pgfqpoint{0.939708in}{0.548594in}}%
\pgfpathlineto{\pgfqpoint{0.940688in}{0.549536in}}%
\pgfpathlineto{\pgfqpoint{0.953895in}{0.563148in}}%
\pgfpathlineto{\pgfqpoint{0.955364in}{0.564803in}}%
\pgfpathlineto{\pgfqpoint{0.965899in}{0.576759in}}%
\pgfpathlineto{\pgfqpoint{0.971021in}{0.583268in}}%
\pgfpathlineto{\pgfqpoint{0.976675in}{0.590370in}}%
\pgfpathlineto{\pgfqpoint{0.986119in}{0.603981in}}%
\pgfpathlineto{\pgfqpoint{0.986678in}{0.604926in}}%
\pgfpathlineto{\pgfqpoint{0.994424in}{0.617592in}}%
\pgfpathlineto{\pgfqpoint{1.001166in}{0.631203in}}%
\pgfpathlineto{\pgfqpoint{1.002334in}{0.634268in}}%
\pgfpathlineto{\pgfqpoint{1.006600in}{0.644814in}}%
\pgfpathlineto{\pgfqpoint{1.010416in}{0.658425in}}%
\pgfpathlineto{\pgfqpoint{1.012534in}{0.672036in}}%
\pgfpathlineto{\pgfqpoint{1.012958in}{0.685648in}}%
\pgfpathlineto{\pgfqpoint{1.011687in}{0.699259in}}%
\pgfpathlineto{\pgfqpoint{1.008720in}{0.712870in}}%
\pgfpathlineto{\pgfqpoint{1.004054in}{0.726481in}}%
\pgfpathlineto{\pgfqpoint{1.002334in}{0.730186in}}%
\pgfpathlineto{\pgfqpoint{0.997994in}{0.740092in}}%
\pgfpathlineto{\pgfqpoint{0.990456in}{0.753703in}}%
\pgfpathlineto{\pgfqpoint{0.986678in}{0.759392in}}%
\pgfpathlineto{\pgfqpoint{0.981585in}{0.767314in}}%
\pgfpathlineto{\pgfqpoint{0.971391in}{0.780925in}}%
\pgfpathlineto{\pgfqpoint{0.971021in}{0.781365in}}%
\pgfpathlineto{\pgfqpoint{0.960062in}{0.794536in}}%
\pgfpathlineto{\pgfqpoint{0.955364in}{0.799635in}}%
\pgfpathlineto{\pgfqpoint{0.947458in}{0.808148in}}%
\pgfpathlineto{\pgfqpoint{0.939708in}{0.815840in}}%
\pgfpathlineto{\pgfqpoint{0.933587in}{0.821759in}}%
\pgfpathlineto{\pgfqpoint{0.924051in}{0.830422in}}%
\pgfpathlineto{\pgfqpoint{0.918360in}{0.835370in}}%
\pgfpathlineto{\pgfqpoint{0.908395in}{0.843659in}}%
\pgfpathlineto{\pgfqpoint{0.901587in}{0.848981in}}%
\pgfpathlineto{\pgfqpoint{0.892738in}{0.855718in}}%
\pgfpathlineto{\pgfqpoint{0.882947in}{0.862592in}}%
\pgfpathlineto{\pgfqpoint{0.877082in}{0.866675in}}%
\pgfpathlineto{\pgfqpoint{0.861931in}{0.876203in}}%
\pgfpathlineto{\pgfqpoint{0.861425in}{0.876524in}}%
\pgfpathlineto{\pgfqpoint{0.845769in}{0.885387in}}%
\pgfpathlineto{\pgfqpoint{0.836655in}{0.889814in}}%
\pgfpathlineto{\pgfqpoint{0.830112in}{0.893099in}}%
\pgfpathlineto{\pgfqpoint{0.814455in}{0.899652in}}%
\pgfpathlineto{\pgfqpoint{0.803061in}{0.903425in}}%
\pgfpathlineto{\pgfqpoint{0.798799in}{0.904921in}}%
\pgfpathlineto{\pgfqpoint{0.783142in}{0.908977in}}%
\pgfpathlineto{\pgfqpoint{0.767486in}{0.911556in}}%
\pgfpathlineto{\pgfqpoint{0.751829in}{0.912661in}}%
\pgfpathlineto{\pgfqpoint{0.736173in}{0.912293in}}%
\pgfpathlineto{\pgfqpoint{0.720516in}{0.910451in}}%
\pgfpathlineto{\pgfqpoint{0.704859in}{0.907134in}}%
\pgfpathlineto{\pgfqpoint{0.692728in}{0.903425in}}%
\pgfpathlineto{\pgfqpoint{0.689203in}{0.902410in}}%
\pgfpathlineto{\pgfqpoint{0.673546in}{0.896549in}}%
\pgfpathlineto{\pgfqpoint{0.658977in}{0.889814in}}%
\pgfpathlineto{\pgfqpoint{0.657890in}{0.889329in}}%
\pgfpathlineto{\pgfqpoint{0.642233in}{0.881119in}}%
\pgfpathlineto{\pgfqpoint{0.634064in}{0.876203in}}%
\pgfpathlineto{\pgfqpoint{0.626577in}{0.871750in}}%
\pgfpathlineto{\pgfqpoint{0.612825in}{0.862592in}}%
\pgfpathlineto{\pgfqpoint{0.610920in}{0.861314in}}%
\pgfpathlineto{\pgfqpoint{0.595263in}{0.849833in}}%
\pgfpathlineto{\pgfqpoint{0.594180in}{0.848981in}}%
\pgfpathlineto{\pgfqpoint{0.579607in}{0.837214in}}%
\pgfpathlineto{\pgfqpoint{0.577443in}{0.835370in}}%
\pgfpathlineto{\pgfqpoint{0.563950in}{0.823348in}}%
\pgfpathlineto{\pgfqpoint{0.562229in}{0.821759in}}%
\pgfpathlineto{\pgfqpoint{0.548385in}{0.808148in}}%
\pgfpathlineto{\pgfqpoint{0.548294in}{0.808050in}}%
\pgfpathlineto{\pgfqpoint{0.535784in}{0.794536in}}%
\pgfpathlineto{\pgfqpoint{0.532637in}{0.790774in}}%
\pgfpathlineto{\pgfqpoint{0.524384in}{0.780925in}}%
\pgfpathlineto{\pgfqpoint{0.516981in}{0.770912in}}%
\pgfpathlineto{\pgfqpoint{0.514263in}{0.767314in}}%
\pgfpathlineto{\pgfqpoint{0.505378in}{0.753703in}}%
\pgfpathlineto{\pgfqpoint{0.501324in}{0.746229in}}%
\pgfpathlineto{\pgfqpoint{0.497842in}{0.740092in}}%
\pgfpathlineto{\pgfqpoint{0.491704in}{0.726481in}}%
\pgfpathlineto{\pgfqpoint{0.487207in}{0.712870in}}%
\pgfpathlineto{\pgfqpoint{0.485668in}{0.705557in}}%
\pgfpathlineto{\pgfqpoint{0.484239in}{0.699259in}}%
\pgfpathlineto{\pgfqpoint{0.482913in}{0.685648in}}%
\pgfpathlineto{\pgfqpoint{0.483355in}{0.672036in}}%
\pgfpathlineto{\pgfqpoint{0.485565in}{0.658425in}}%
\pgfpathlineto{\pgfqpoint{0.485668in}{0.658074in}}%
\pgfpathlineto{\pgfqpoint{0.489251in}{0.644814in}}%
\pgfpathlineto{\pgfqpoint{0.494567in}{0.631203in}}%
\pgfpathlineto{\pgfqpoint{0.501324in}{0.617984in}}%
\pgfpathlineto{\pgfqpoint{0.501515in}{0.617592in}}%
\pgfpathlineto{\pgfqpoint{0.509627in}{0.603981in}}%
\pgfpathlineto{\pgfqpoint{0.516981in}{0.593573in}}%
\pgfpathlineto{\pgfqpoint{0.519193in}{0.590370in}}%
\pgfpathlineto{\pgfqpoint{0.529940in}{0.576759in}}%
\pgfpathlineto{\pgfqpoint{0.532637in}{0.573695in}}%
\pgfpathlineto{\pgfqpoint{0.541914in}{0.563148in}}%
\pgfpathlineto{\pgfqpoint{0.548294in}{0.556526in}}%
\pgfpathlineto{\pgfqpoint{0.555146in}{0.549536in}}%
\pgfpathlineto{\pgfqpoint{0.563950in}{0.541176in}}%
\pgfpathlineto{\pgfqpoint{0.569678in}{0.535925in}}%
\pgfpathlineto{\pgfqpoint{0.579607in}{0.527293in}}%
\pgfpathlineto{\pgfqpoint{0.585646in}{0.522314in}}%
\pgfpathlineto{\pgfqpoint{0.595263in}{0.514660in}}%
\pgfpathlineto{\pgfqpoint{0.603303in}{0.508703in}}%
\pgfpathlineto{\pgfqpoint{0.610920in}{0.503157in}}%
\pgfpathlineto{\pgfqpoint{0.623053in}{0.495092in}}%
\pgfpathlineto{\pgfqpoint{0.626577in}{0.492747in}}%
\pgfpathlineto{\pgfqpoint{0.642233in}{0.483404in}}%
\pgfpathlineto{\pgfqpoint{0.645918in}{0.481481in}}%
\pgfpathlineto{\pgfqpoint{0.657890in}{0.475088in}}%
\pgfpathlineto{\pgfqpoint{0.673546in}{0.468036in}}%
\pgfpathlineto{\pgfqpoint{0.673997in}{0.467870in}}%
\pgfpathlineto{\pgfqpoint{0.689203in}{0.461996in}}%
\pgfpathlineto{\pgfqpoint{0.704859in}{0.457374in}}%
\pgfpathlineto{\pgfqpoint{0.720112in}{0.454259in}}%
\pgfpathlineto{\pgfqpoint{0.720516in}{0.454170in}}%
\pgfpathclose%
\pgfpathmoveto{\pgfqpoint{0.685228in}{0.508703in}}%
\pgfpathlineto{\pgfqpoint{0.673546in}{0.513082in}}%
\pgfpathlineto{\pgfqpoint{0.657890in}{0.520315in}}%
\pgfpathlineto{\pgfqpoint{0.654205in}{0.522314in}}%
\pgfpathlineto{\pgfqpoint{0.642233in}{0.528987in}}%
\pgfpathlineto{\pgfqpoint{0.631378in}{0.535925in}}%
\pgfpathlineto{\pgfqpoint{0.626577in}{0.539138in}}%
\pgfpathlineto{\pgfqpoint{0.612653in}{0.549536in}}%
\pgfpathlineto{\pgfqpoint{0.610920in}{0.550918in}}%
\pgfpathlineto{\pgfqpoint{0.596853in}{0.563148in}}%
\pgfpathlineto{\pgfqpoint{0.595263in}{0.564654in}}%
\pgfpathlineto{\pgfqpoint{0.583302in}{0.576759in}}%
\pgfpathlineto{\pgfqpoint{0.579607in}{0.580933in}}%
\pgfpathlineto{\pgfqpoint{0.571626in}{0.590370in}}%
\pgfpathlineto{\pgfqpoint{0.563950in}{0.600777in}}%
\pgfpathlineto{\pgfqpoint{0.561650in}{0.603981in}}%
\pgfpathlineto{\pgfqpoint{0.553331in}{0.617592in}}%
\pgfpathlineto{\pgfqpoint{0.548294in}{0.627747in}}%
\pgfpathlineto{\pgfqpoint{0.546589in}{0.631203in}}%
\pgfpathlineto{\pgfqpoint{0.541412in}{0.644814in}}%
\pgfpathlineto{\pgfqpoint{0.537832in}{0.658425in}}%
\pgfpathlineto{\pgfqpoint{0.535844in}{0.672036in}}%
\pgfpathlineto{\pgfqpoint{0.535447in}{0.685648in}}%
\pgfpathlineto{\pgfqpoint{0.536639in}{0.699259in}}%
\pgfpathlineto{\pgfqpoint{0.539423in}{0.712870in}}%
\pgfpathlineto{\pgfqpoint{0.543801in}{0.726481in}}%
\pgfpathlineto{\pgfqpoint{0.548294in}{0.736741in}}%
\pgfpathlineto{\pgfqpoint{0.549767in}{0.740092in}}%
\pgfpathlineto{\pgfqpoint{0.557292in}{0.753703in}}%
\pgfpathlineto{\pgfqpoint{0.563950in}{0.763691in}}%
\pgfpathlineto{\pgfqpoint{0.566428in}{0.767314in}}%
\pgfpathlineto{\pgfqpoint{0.577220in}{0.780925in}}%
\pgfpathlineto{\pgfqpoint{0.579607in}{0.783597in}}%
\pgfpathlineto{\pgfqpoint{0.589847in}{0.794536in}}%
\pgfpathlineto{\pgfqpoint{0.595263in}{0.799779in}}%
\pgfpathlineto{\pgfqpoint{0.604503in}{0.808148in}}%
\pgfpathlineto{\pgfqpoint{0.610920in}{0.813530in}}%
\pgfpathlineto{\pgfqpoint{0.621625in}{0.821759in}}%
\pgfpathlineto{\pgfqpoint{0.626577in}{0.825355in}}%
\pgfpathlineto{\pgfqpoint{0.641975in}{0.835370in}}%
\pgfpathlineto{\pgfqpoint{0.642233in}{0.835531in}}%
\pgfpathlineto{\pgfqpoint{0.657890in}{0.844164in}}%
\pgfpathlineto{\pgfqpoint{0.668236in}{0.848981in}}%
\pgfpathlineto{\pgfqpoint{0.673546in}{0.851413in}}%
\pgfpathlineto{\pgfqpoint{0.689203in}{0.857272in}}%
\pgfpathlineto{\pgfqpoint{0.704859in}{0.861748in}}%
\pgfpathlineto{\pgfqpoint{0.709106in}{0.862592in}}%
\pgfpathlineto{\pgfqpoint{0.720516in}{0.864878in}}%
\pgfpathlineto{\pgfqpoint{0.736173in}{0.866622in}}%
\pgfpathlineto{\pgfqpoint{0.751829in}{0.866971in}}%
\pgfpathlineto{\pgfqpoint{0.767486in}{0.865925in}}%
\pgfpathlineto{\pgfqpoint{0.783142in}{0.863482in}}%
\pgfpathlineto{\pgfqpoint{0.786790in}{0.862592in}}%
\pgfpathlineto{\pgfqpoint{0.798799in}{0.859683in}}%
\pgfpathlineto{\pgfqpoint{0.814455in}{0.854515in}}%
\pgfpathlineto{\pgfqpoint{0.827711in}{0.848981in}}%
\pgfpathlineto{\pgfqpoint{0.830112in}{0.847963in}}%
\pgfpathlineto{\pgfqpoint{0.845769in}{0.840020in}}%
\pgfpathlineto{\pgfqpoint{0.853654in}{0.835370in}}%
\pgfpathlineto{\pgfqpoint{0.861425in}{0.830618in}}%
\pgfpathlineto{\pgfqpoint{0.874238in}{0.821759in}}%
\pgfpathlineto{\pgfqpoint{0.877082in}{0.819681in}}%
\pgfpathlineto{\pgfqpoint{0.891392in}{0.808148in}}%
\pgfpathlineto{\pgfqpoint{0.892738in}{0.806977in}}%
\pgfpathlineto{\pgfqpoint{0.906005in}{0.794536in}}%
\pgfpathlineto{\pgfqpoint{0.908395in}{0.792065in}}%
\pgfpathlineto{\pgfqpoint{0.918585in}{0.780925in}}%
\pgfpathlineto{\pgfqpoint{0.924051in}{0.774169in}}%
\pgfpathlineto{\pgfqpoint{0.929400in}{0.767314in}}%
\pgfpathlineto{\pgfqpoint{0.938537in}{0.753703in}}%
\pgfpathlineto{\pgfqpoint{0.939708in}{0.751616in}}%
\pgfpathlineto{\pgfqpoint{0.946074in}{0.740092in}}%
\pgfpathlineto{\pgfqpoint{0.952018in}{0.726481in}}%
\pgfpathlineto{\pgfqpoint{0.955364in}{0.716041in}}%
\pgfpathlineto{\pgfqpoint{0.956388in}{0.712870in}}%
\pgfpathlineto{\pgfqpoint{0.959198in}{0.699259in}}%
\pgfpathlineto{\pgfqpoint{0.960402in}{0.685648in}}%
\pgfpathlineto{\pgfqpoint{0.960001in}{0.672036in}}%
\pgfpathlineto{\pgfqpoint{0.957994in}{0.658425in}}%
\pgfpathlineto{\pgfqpoint{0.955364in}{0.648506in}}%
\pgfpathlineto{\pgfqpoint{0.954394in}{0.644814in}}%
\pgfpathlineto{\pgfqpoint{0.949245in}{0.631203in}}%
\pgfpathlineto{\pgfqpoint{0.942505in}{0.617592in}}%
\pgfpathlineto{\pgfqpoint{0.939708in}{0.612976in}}%
\pgfpathlineto{\pgfqpoint{0.934167in}{0.603981in}}%
\pgfpathlineto{\pgfqpoint{0.924237in}{0.590370in}}%
\pgfpathlineto{\pgfqpoint{0.924051in}{0.590145in}}%
\pgfpathlineto{\pgfqpoint{0.912532in}{0.576759in}}%
\pgfpathlineto{\pgfqpoint{0.908395in}{0.572454in}}%
\pgfpathlineto{\pgfqpoint{0.898930in}{0.563148in}}%
\pgfpathlineto{\pgfqpoint{0.892738in}{0.557569in}}%
\pgfpathlineto{\pgfqpoint{0.883113in}{0.549536in}}%
\pgfpathlineto{\pgfqpoint{0.877082in}{0.544827in}}%
\pgfpathlineto{\pgfqpoint{0.864499in}{0.535925in}}%
\pgfpathlineto{\pgfqpoint{0.861425in}{0.533850in}}%
\pgfpathlineto{\pgfqpoint{0.845769in}{0.524468in}}%
\pgfpathlineto{\pgfqpoint{0.841601in}{0.522314in}}%
\pgfpathlineto{\pgfqpoint{0.830112in}{0.516526in}}%
\pgfpathlineto{\pgfqpoint{0.814455in}{0.509984in}}%
\pgfpathlineto{\pgfqpoint{0.810601in}{0.508703in}}%
\pgfpathlineto{\pgfqpoint{0.798799in}{0.504797in}}%
\pgfpathlineto{\pgfqpoint{0.783142in}{0.500991in}}%
\pgfpathlineto{\pgfqpoint{0.767486in}{0.498571in}}%
\pgfpathlineto{\pgfqpoint{0.751829in}{0.497534in}}%
\pgfpathlineto{\pgfqpoint{0.736173in}{0.497880in}}%
\pgfpathlineto{\pgfqpoint{0.720516in}{0.499608in}}%
\pgfpathlineto{\pgfqpoint{0.704859in}{0.502721in}}%
\pgfpathlineto{\pgfqpoint{0.689203in}{0.507221in}}%
\pgfpathlineto{\pgfqpoint{0.685228in}{0.508703in}}%
\pgfpathclose%
\pgfpathmoveto{\pgfqpoint{1.503344in}{0.453017in}}%
\pgfpathlineto{\pgfqpoint{1.519001in}{0.451864in}}%
\pgfpathlineto{\pgfqpoint{1.534657in}{0.452248in}}%
\pgfpathlineto{\pgfqpoint{1.550314in}{0.454170in}}%
\pgfpathlineto{\pgfqpoint{1.550718in}{0.454259in}}%
\pgfpathlineto{\pgfqpoint{1.565971in}{0.457374in}}%
\pgfpathlineto{\pgfqpoint{1.581627in}{0.461996in}}%
\pgfpathlineto{\pgfqpoint{1.596833in}{0.467870in}}%
\pgfpathlineto{\pgfqpoint{1.597284in}{0.468036in}}%
\pgfpathlineto{\pgfqpoint{1.612940in}{0.475088in}}%
\pgfpathlineto{\pgfqpoint{1.624912in}{0.481481in}}%
\pgfpathlineto{\pgfqpoint{1.628597in}{0.483404in}}%
\pgfpathlineto{\pgfqpoint{1.644253in}{0.492747in}}%
\pgfpathlineto{\pgfqpoint{1.647777in}{0.495092in}}%
\pgfpathlineto{\pgfqpoint{1.659910in}{0.503157in}}%
\pgfpathlineto{\pgfqpoint{1.667527in}{0.508703in}}%
\pgfpathlineto{\pgfqpoint{1.675567in}{0.514660in}}%
\pgfpathlineto{\pgfqpoint{1.685184in}{0.522314in}}%
\pgfpathlineto{\pgfqpoint{1.691223in}{0.527293in}}%
\pgfpathlineto{\pgfqpoint{1.701152in}{0.535925in}}%
\pgfpathlineto{\pgfqpoint{1.706880in}{0.541176in}}%
\pgfpathlineto{\pgfqpoint{1.715684in}{0.549536in}}%
\pgfpathlineto{\pgfqpoint{1.722536in}{0.556526in}}%
\pgfpathlineto{\pgfqpoint{1.728916in}{0.563148in}}%
\pgfpathlineto{\pgfqpoint{1.738193in}{0.573695in}}%
\pgfpathlineto{\pgfqpoint{1.740890in}{0.576759in}}%
\pgfpathlineto{\pgfqpoint{1.751637in}{0.590370in}}%
\pgfpathlineto{\pgfqpoint{1.753849in}{0.593573in}}%
\pgfpathlineto{\pgfqpoint{1.761203in}{0.603981in}}%
\pgfpathlineto{\pgfqpoint{1.769315in}{0.617592in}}%
\pgfpathlineto{\pgfqpoint{1.769506in}{0.617984in}}%
\pgfpathlineto{\pgfqpoint{1.776263in}{0.631203in}}%
\pgfpathlineto{\pgfqpoint{1.781579in}{0.644814in}}%
\pgfpathlineto{\pgfqpoint{1.785162in}{0.658074in}}%
\pgfpathlineto{\pgfqpoint{1.785265in}{0.658425in}}%
\pgfpathlineto{\pgfqpoint{1.787475in}{0.672036in}}%
\pgfpathlineto{\pgfqpoint{1.787917in}{0.685648in}}%
\pgfpathlineto{\pgfqpoint{1.786591in}{0.699259in}}%
\pgfpathlineto{\pgfqpoint{1.785162in}{0.705557in}}%
\pgfpathlineto{\pgfqpoint{1.783623in}{0.712870in}}%
\pgfpathlineto{\pgfqpoint{1.779126in}{0.726481in}}%
\pgfpathlineto{\pgfqpoint{1.772988in}{0.740092in}}%
\pgfpathlineto{\pgfqpoint{1.769506in}{0.746229in}}%
\pgfpathlineto{\pgfqpoint{1.765452in}{0.753703in}}%
\pgfpathlineto{\pgfqpoint{1.756567in}{0.767314in}}%
\pgfpathlineto{\pgfqpoint{1.753849in}{0.770912in}}%
\pgfpathlineto{\pgfqpoint{1.746446in}{0.780925in}}%
\pgfpathlineto{\pgfqpoint{1.738193in}{0.790774in}}%
\pgfpathlineto{\pgfqpoint{1.735046in}{0.794536in}}%
\pgfpathlineto{\pgfqpoint{1.722536in}{0.808050in}}%
\pgfpathlineto{\pgfqpoint{1.722445in}{0.808148in}}%
\pgfpathlineto{\pgfqpoint{1.708601in}{0.821759in}}%
\pgfpathlineto{\pgfqpoint{1.706880in}{0.823348in}}%
\pgfpathlineto{\pgfqpoint{1.693387in}{0.835370in}}%
\pgfpathlineto{\pgfqpoint{1.691223in}{0.837214in}}%
\pgfpathlineto{\pgfqpoint{1.676650in}{0.848981in}}%
\pgfpathlineto{\pgfqpoint{1.675567in}{0.849833in}}%
\pgfpathlineto{\pgfqpoint{1.659910in}{0.861314in}}%
\pgfpathlineto{\pgfqpoint{1.658005in}{0.862592in}}%
\pgfpathlineto{\pgfqpoint{1.644253in}{0.871750in}}%
\pgfpathlineto{\pgfqpoint{1.636766in}{0.876203in}}%
\pgfpathlineto{\pgfqpoint{1.628597in}{0.881119in}}%
\pgfpathlineto{\pgfqpoint{1.612940in}{0.889329in}}%
\pgfpathlineto{\pgfqpoint{1.611853in}{0.889814in}}%
\pgfpathlineto{\pgfqpoint{1.597284in}{0.896549in}}%
\pgfpathlineto{\pgfqpoint{1.581627in}{0.902410in}}%
\pgfpathlineto{\pgfqpoint{1.578102in}{0.903425in}}%
\pgfpathlineto{\pgfqpoint{1.565971in}{0.907134in}}%
\pgfpathlineto{\pgfqpoint{1.550314in}{0.910451in}}%
\pgfpathlineto{\pgfqpoint{1.534657in}{0.912293in}}%
\pgfpathlineto{\pgfqpoint{1.519001in}{0.912661in}}%
\pgfpathlineto{\pgfqpoint{1.503344in}{0.911556in}}%
\pgfpathlineto{\pgfqpoint{1.487688in}{0.908977in}}%
\pgfpathlineto{\pgfqpoint{1.472031in}{0.904921in}}%
\pgfpathlineto{\pgfqpoint{1.467769in}{0.903425in}}%
\pgfpathlineto{\pgfqpoint{1.456375in}{0.899652in}}%
\pgfpathlineto{\pgfqpoint{1.440718in}{0.893099in}}%
\pgfpathlineto{\pgfqpoint{1.434175in}{0.889814in}}%
\pgfpathlineto{\pgfqpoint{1.425061in}{0.885387in}}%
\pgfpathlineto{\pgfqpoint{1.409405in}{0.876524in}}%
\pgfpathlineto{\pgfqpoint{1.408899in}{0.876203in}}%
\pgfpathlineto{\pgfqpoint{1.393748in}{0.866675in}}%
\pgfpathlineto{\pgfqpoint{1.387883in}{0.862592in}}%
\pgfpathlineto{\pgfqpoint{1.378092in}{0.855718in}}%
\pgfpathlineto{\pgfqpoint{1.369243in}{0.848981in}}%
\pgfpathlineto{\pgfqpoint{1.362435in}{0.843659in}}%
\pgfpathlineto{\pgfqpoint{1.352470in}{0.835370in}}%
\pgfpathlineto{\pgfqpoint{1.346779in}{0.830422in}}%
\pgfpathlineto{\pgfqpoint{1.337243in}{0.821759in}}%
\pgfpathlineto{\pgfqpoint{1.331122in}{0.815840in}}%
\pgfpathlineto{\pgfqpoint{1.323372in}{0.808148in}}%
\pgfpathlineto{\pgfqpoint{1.315466in}{0.799635in}}%
\pgfpathlineto{\pgfqpoint{1.310768in}{0.794536in}}%
\pgfpathlineto{\pgfqpoint{1.299809in}{0.781365in}}%
\pgfpathlineto{\pgfqpoint{1.299439in}{0.780925in}}%
\pgfpathlineto{\pgfqpoint{1.289245in}{0.767314in}}%
\pgfpathlineto{\pgfqpoint{1.284152in}{0.759392in}}%
\pgfpathlineto{\pgfqpoint{1.280374in}{0.753703in}}%
\pgfpathlineto{\pgfqpoint{1.272836in}{0.740092in}}%
\pgfpathlineto{\pgfqpoint{1.268496in}{0.730186in}}%
\pgfpathlineto{\pgfqpoint{1.266776in}{0.726481in}}%
\pgfpathlineto{\pgfqpoint{1.262110in}{0.712870in}}%
\pgfpathlineto{\pgfqpoint{1.259143in}{0.699259in}}%
\pgfpathlineto{\pgfqpoint{1.257872in}{0.685648in}}%
\pgfpathlineto{\pgfqpoint{1.258296in}{0.672036in}}%
\pgfpathlineto{\pgfqpoint{1.260414in}{0.658425in}}%
\pgfpathlineto{\pgfqpoint{1.264230in}{0.644814in}}%
\pgfpathlineto{\pgfqpoint{1.268496in}{0.634268in}}%
\pgfpathlineto{\pgfqpoint{1.269664in}{0.631203in}}%
\pgfpathlineto{\pgfqpoint{1.276406in}{0.617592in}}%
\pgfpathlineto{\pgfqpoint{1.284152in}{0.604926in}}%
\pgfpathlineto{\pgfqpoint{1.284711in}{0.603981in}}%
\pgfpathlineto{\pgfqpoint{1.294155in}{0.590370in}}%
\pgfpathlineto{\pgfqpoint{1.299809in}{0.583268in}}%
\pgfpathlineto{\pgfqpoint{1.304931in}{0.576759in}}%
\pgfpathlineto{\pgfqpoint{1.315466in}{0.564803in}}%
\pgfpathlineto{\pgfqpoint{1.316935in}{0.563148in}}%
\pgfpathlineto{\pgfqpoint{1.330142in}{0.549536in}}%
\pgfpathlineto{\pgfqpoint{1.331122in}{0.548594in}}%
\pgfpathlineto{\pgfqpoint{1.344658in}{0.535925in}}%
\pgfpathlineto{\pgfqpoint{1.346779in}{0.534044in}}%
\pgfpathlineto{\pgfqpoint{1.360607in}{0.522314in}}%
\pgfpathlineto{\pgfqpoint{1.362435in}{0.520818in}}%
\pgfpathlineto{\pgfqpoint{1.378092in}{0.508783in}}%
\pgfpathlineto{\pgfqpoint{1.378204in}{0.508703in}}%
\pgfpathlineto{\pgfqpoint{1.393748in}{0.497827in}}%
\pgfpathlineto{\pgfqpoint{1.398076in}{0.495092in}}%
\pgfpathlineto{\pgfqpoint{1.409405in}{0.487917in}}%
\pgfpathlineto{\pgfqpoint{1.420923in}{0.481481in}}%
\pgfpathlineto{\pgfqpoint{1.425061in}{0.479118in}}%
\pgfpathlineto{\pgfqpoint{1.440718in}{0.471394in}}%
\pgfpathlineto{\pgfqpoint{1.449315in}{0.467870in}}%
\pgfpathlineto{\pgfqpoint{1.456375in}{0.464842in}}%
\pgfpathlineto{\pgfqpoint{1.472031in}{0.459506in}}%
\pgfpathlineto{\pgfqpoint{1.487688in}{0.455597in}}%
\pgfpathlineto{\pgfqpoint{1.496099in}{0.454259in}}%
\pgfpathlineto{\pgfqpoint{1.503344in}{0.453017in}}%
\pgfpathclose%
\pgfpathmoveto{\pgfqpoint{1.460229in}{0.508703in}}%
\pgfpathlineto{\pgfqpoint{1.456375in}{0.509984in}}%
\pgfpathlineto{\pgfqpoint{1.440718in}{0.516526in}}%
\pgfpathlineto{\pgfqpoint{1.429229in}{0.522314in}}%
\pgfpathlineto{\pgfqpoint{1.425061in}{0.524468in}}%
\pgfpathlineto{\pgfqpoint{1.409405in}{0.533850in}}%
\pgfpathlineto{\pgfqpoint{1.406331in}{0.535925in}}%
\pgfpathlineto{\pgfqpoint{1.393748in}{0.544827in}}%
\pgfpathlineto{\pgfqpoint{1.387717in}{0.549536in}}%
\pgfpathlineto{\pgfqpoint{1.378092in}{0.557569in}}%
\pgfpathlineto{\pgfqpoint{1.371900in}{0.563148in}}%
\pgfpathlineto{\pgfqpoint{1.362435in}{0.572454in}}%
\pgfpathlineto{\pgfqpoint{1.358298in}{0.576759in}}%
\pgfpathlineto{\pgfqpoint{1.346779in}{0.590145in}}%
\pgfpathlineto{\pgfqpoint{1.346593in}{0.590370in}}%
\pgfpathlineto{\pgfqpoint{1.336663in}{0.603981in}}%
\pgfpathlineto{\pgfqpoint{1.331122in}{0.612976in}}%
\pgfpathlineto{\pgfqpoint{1.328325in}{0.617592in}}%
\pgfpathlineto{\pgfqpoint{1.321585in}{0.631203in}}%
\pgfpathlineto{\pgfqpoint{1.316436in}{0.644814in}}%
\pgfpathlineto{\pgfqpoint{1.315466in}{0.648506in}}%
\pgfpathlineto{\pgfqpoint{1.312836in}{0.658425in}}%
\pgfpathlineto{\pgfqpoint{1.310829in}{0.672036in}}%
\pgfpathlineto{\pgfqpoint{1.310428in}{0.685648in}}%
\pgfpathlineto{\pgfqpoint{1.311632in}{0.699259in}}%
\pgfpathlineto{\pgfqpoint{1.314442in}{0.712870in}}%
\pgfpathlineto{\pgfqpoint{1.315466in}{0.716041in}}%
\pgfpathlineto{\pgfqpoint{1.318812in}{0.726481in}}%
\pgfpathlineto{\pgfqpoint{1.324756in}{0.740092in}}%
\pgfpathlineto{\pgfqpoint{1.331122in}{0.751616in}}%
\pgfpathlineto{\pgfqpoint{1.332293in}{0.753703in}}%
\pgfpathlineto{\pgfqpoint{1.341430in}{0.767314in}}%
\pgfpathlineto{\pgfqpoint{1.346779in}{0.774169in}}%
\pgfpathlineto{\pgfqpoint{1.352245in}{0.780925in}}%
\pgfpathlineto{\pgfqpoint{1.362435in}{0.792065in}}%
\pgfpathlineto{\pgfqpoint{1.364825in}{0.794536in}}%
\pgfpathlineto{\pgfqpoint{1.378092in}{0.806977in}}%
\pgfpathlineto{\pgfqpoint{1.379438in}{0.808148in}}%
\pgfpathlineto{\pgfqpoint{1.393748in}{0.819681in}}%
\pgfpathlineto{\pgfqpoint{1.396592in}{0.821759in}}%
\pgfpathlineto{\pgfqpoint{1.409405in}{0.830618in}}%
\pgfpathlineto{\pgfqpoint{1.417176in}{0.835370in}}%
\pgfpathlineto{\pgfqpoint{1.425061in}{0.840020in}}%
\pgfpathlineto{\pgfqpoint{1.440718in}{0.847963in}}%
\pgfpathlineto{\pgfqpoint{1.443119in}{0.848981in}}%
\pgfpathlineto{\pgfqpoint{1.456375in}{0.854515in}}%
\pgfpathlineto{\pgfqpoint{1.472031in}{0.859683in}}%
\pgfpathlineto{\pgfqpoint{1.484040in}{0.862592in}}%
\pgfpathlineto{\pgfqpoint{1.487688in}{0.863482in}}%
\pgfpathlineto{\pgfqpoint{1.503344in}{0.865925in}}%
\pgfpathlineto{\pgfqpoint{1.519001in}{0.866971in}}%
\pgfpathlineto{\pgfqpoint{1.534657in}{0.866622in}}%
\pgfpathlineto{\pgfqpoint{1.550314in}{0.864878in}}%
\pgfpathlineto{\pgfqpoint{1.561724in}{0.862592in}}%
\pgfpathlineto{\pgfqpoint{1.565971in}{0.861748in}}%
\pgfpathlineto{\pgfqpoint{1.581627in}{0.857272in}}%
\pgfpathlineto{\pgfqpoint{1.597284in}{0.851413in}}%
\pgfpathlineto{\pgfqpoint{1.602594in}{0.848981in}}%
\pgfpathlineto{\pgfqpoint{1.612940in}{0.844164in}}%
\pgfpathlineto{\pgfqpoint{1.628597in}{0.835531in}}%
\pgfpathlineto{\pgfqpoint{1.628855in}{0.835370in}}%
\pgfpathlineto{\pgfqpoint{1.644253in}{0.825355in}}%
\pgfpathlineto{\pgfqpoint{1.649205in}{0.821759in}}%
\pgfpathlineto{\pgfqpoint{1.659910in}{0.813530in}}%
\pgfpathlineto{\pgfqpoint{1.666327in}{0.808148in}}%
\pgfpathlineto{\pgfqpoint{1.675567in}{0.799779in}}%
\pgfpathlineto{\pgfqpoint{1.680983in}{0.794536in}}%
\pgfpathlineto{\pgfqpoint{1.691223in}{0.783597in}}%
\pgfpathlineto{\pgfqpoint{1.693610in}{0.780925in}}%
\pgfpathlineto{\pgfqpoint{1.704402in}{0.767314in}}%
\pgfpathlineto{\pgfqpoint{1.706880in}{0.763691in}}%
\pgfpathlineto{\pgfqpoint{1.713538in}{0.753703in}}%
\pgfpathlineto{\pgfqpoint{1.721063in}{0.740092in}}%
\pgfpathlineto{\pgfqpoint{1.722536in}{0.736741in}}%
\pgfpathlineto{\pgfqpoint{1.727029in}{0.726481in}}%
\pgfpathlineto{\pgfqpoint{1.731407in}{0.712870in}}%
\pgfpathlineto{\pgfqpoint{1.734191in}{0.699259in}}%
\pgfpathlineto{\pgfqpoint{1.735383in}{0.685648in}}%
\pgfpathlineto{\pgfqpoint{1.734986in}{0.672036in}}%
\pgfpathlineto{\pgfqpoint{1.732998in}{0.658425in}}%
\pgfpathlineto{\pgfqpoint{1.729418in}{0.644814in}}%
\pgfpathlineto{\pgfqpoint{1.724241in}{0.631203in}}%
\pgfpathlineto{\pgfqpoint{1.722536in}{0.627747in}}%
\pgfpathlineto{\pgfqpoint{1.717499in}{0.617592in}}%
\pgfpathlineto{\pgfqpoint{1.709180in}{0.603981in}}%
\pgfpathlineto{\pgfqpoint{1.706880in}{0.600777in}}%
\pgfpathlineto{\pgfqpoint{1.699204in}{0.590370in}}%
\pgfpathlineto{\pgfqpoint{1.691223in}{0.580933in}}%
\pgfpathlineto{\pgfqpoint{1.687528in}{0.576759in}}%
\pgfpathlineto{\pgfqpoint{1.675567in}{0.564654in}}%
\pgfpathlineto{\pgfqpoint{1.673977in}{0.563148in}}%
\pgfpathlineto{\pgfqpoint{1.659910in}{0.550918in}}%
\pgfpathlineto{\pgfqpoint{1.658177in}{0.549536in}}%
\pgfpathlineto{\pgfqpoint{1.644253in}{0.539138in}}%
\pgfpathlineto{\pgfqpoint{1.639452in}{0.535925in}}%
\pgfpathlineto{\pgfqpoint{1.628597in}{0.528987in}}%
\pgfpathlineto{\pgfqpoint{1.616625in}{0.522314in}}%
\pgfpathlineto{\pgfqpoint{1.612940in}{0.520315in}}%
\pgfpathlineto{\pgfqpoint{1.597284in}{0.513082in}}%
\pgfpathlineto{\pgfqpoint{1.585602in}{0.508703in}}%
\pgfpathlineto{\pgfqpoint{1.581627in}{0.507221in}}%
\pgfpathlineto{\pgfqpoint{1.565971in}{0.502721in}}%
\pgfpathlineto{\pgfqpoint{1.550314in}{0.499608in}}%
\pgfpathlineto{\pgfqpoint{1.534657in}{0.497880in}}%
\pgfpathlineto{\pgfqpoint{1.519001in}{0.497534in}}%
\pgfpathlineto{\pgfqpoint{1.503344in}{0.498571in}}%
\pgfpathlineto{\pgfqpoint{1.487688in}{0.500991in}}%
\pgfpathlineto{\pgfqpoint{1.472031in}{0.504797in}}%
\pgfpathlineto{\pgfqpoint{1.460229in}{0.508703in}}%
\pgfpathclose%
\pgfpathmoveto{\pgfqpoint{0.704859in}{1.131106in}}%
\pgfpathlineto{\pgfqpoint{0.720516in}{1.127789in}}%
\pgfpathlineto{\pgfqpoint{0.736173in}{1.125947in}}%
\pgfpathlineto{\pgfqpoint{0.751829in}{1.125579in}}%
\pgfpathlineto{\pgfqpoint{0.767486in}{1.126683in}}%
\pgfpathlineto{\pgfqpoint{0.783142in}{1.129263in}}%
\pgfpathlineto{\pgfqpoint{0.798799in}{1.133319in}}%
\pgfpathlineto{\pgfqpoint{0.803061in}{1.134814in}}%
\pgfpathlineto{\pgfqpoint{0.814455in}{1.138587in}}%
\pgfpathlineto{\pgfqpoint{0.830112in}{1.145140in}}%
\pgfpathlineto{\pgfqpoint{0.836655in}{1.148425in}}%
\pgfpathlineto{\pgfqpoint{0.845769in}{1.152852in}}%
\pgfpathlineto{\pgfqpoint{0.861425in}{1.161715in}}%
\pgfpathlineto{\pgfqpoint{0.861931in}{1.162036in}}%
\pgfpathlineto{\pgfqpoint{0.877082in}{1.171564in}}%
\pgfpathlineto{\pgfqpoint{0.882947in}{1.175647in}}%
\pgfpathlineto{\pgfqpoint{0.892738in}{1.182521in}}%
\pgfpathlineto{\pgfqpoint{0.901587in}{1.189259in}}%
\pgfpathlineto{\pgfqpoint{0.908395in}{1.194580in}}%
\pgfpathlineto{\pgfqpoint{0.918360in}{1.202870in}}%
\pgfpathlineto{\pgfqpoint{0.924051in}{1.207817in}}%
\pgfpathlineto{\pgfqpoint{0.933587in}{1.216481in}}%
\pgfpathlineto{\pgfqpoint{0.939708in}{1.222399in}}%
\pgfpathlineto{\pgfqpoint{0.947458in}{1.230092in}}%
\pgfpathlineto{\pgfqpoint{0.955364in}{1.238604in}}%
\pgfpathlineto{\pgfqpoint{0.960062in}{1.243703in}}%
\pgfpathlineto{\pgfqpoint{0.971021in}{1.256874in}}%
\pgfpathlineto{\pgfqpoint{0.971391in}{1.257314in}}%
\pgfpathlineto{\pgfqpoint{0.981585in}{1.270925in}}%
\pgfpathlineto{\pgfqpoint{0.986678in}{1.278848in}}%
\pgfpathlineto{\pgfqpoint{0.990456in}{1.284536in}}%
\pgfpathlineto{\pgfqpoint{0.997994in}{1.298148in}}%
\pgfpathlineto{\pgfqpoint{1.002334in}{1.308054in}}%
\pgfpathlineto{\pgfqpoint{1.004054in}{1.311759in}}%
\pgfpathlineto{\pgfqpoint{1.008720in}{1.325370in}}%
\pgfpathlineto{\pgfqpoint{1.011687in}{1.338981in}}%
\pgfpathlineto{\pgfqpoint{1.012958in}{1.352592in}}%
\pgfpathlineto{\pgfqpoint{1.012534in}{1.366203in}}%
\pgfpathlineto{\pgfqpoint{1.010416in}{1.379814in}}%
\pgfpathlineto{\pgfqpoint{1.006600in}{1.393425in}}%
\pgfpathlineto{\pgfqpoint{1.002334in}{1.403972in}}%
\pgfpathlineto{\pgfqpoint{1.001166in}{1.407036in}}%
\pgfpathlineto{\pgfqpoint{0.994424in}{1.420648in}}%
\pgfpathlineto{\pgfqpoint{0.986678in}{1.433314in}}%
\pgfpathlineto{\pgfqpoint{0.986119in}{1.434259in}}%
\pgfpathlineto{\pgfqpoint{0.976675in}{1.447870in}}%
\pgfpathlineto{\pgfqpoint{0.971021in}{1.454972in}}%
\pgfpathlineto{\pgfqpoint{0.965899in}{1.461481in}}%
\pgfpathlineto{\pgfqpoint{0.955364in}{1.473436in}}%
\pgfpathlineto{\pgfqpoint{0.953895in}{1.475092in}}%
\pgfpathlineto{\pgfqpoint{0.940688in}{1.488703in}}%
\pgfpathlineto{\pgfqpoint{0.939708in}{1.489645in}}%
\pgfpathlineto{\pgfqpoint{0.926172in}{1.502314in}}%
\pgfpathlineto{\pgfqpoint{0.924051in}{1.504196in}}%
\pgfpathlineto{\pgfqpoint{0.910223in}{1.515925in}}%
\pgfpathlineto{\pgfqpoint{0.908395in}{1.517421in}}%
\pgfpathlineto{\pgfqpoint{0.892738in}{1.529457in}}%
\pgfpathlineto{\pgfqpoint{0.892626in}{1.529536in}}%
\pgfpathlineto{\pgfqpoint{0.877082in}{1.540412in}}%
\pgfpathlineto{\pgfqpoint{0.872754in}{1.543148in}}%
\pgfpathlineto{\pgfqpoint{0.861425in}{1.550322in}}%
\pgfpathlineto{\pgfqpoint{0.849907in}{1.556759in}}%
\pgfpathlineto{\pgfqpoint{0.845769in}{1.559121in}}%
\pgfpathlineto{\pgfqpoint{0.830112in}{1.566846in}}%
\pgfpathlineto{\pgfqpoint{0.821515in}{1.570370in}}%
\pgfpathlineto{\pgfqpoint{0.814455in}{1.573397in}}%
\pgfpathlineto{\pgfqpoint{0.798799in}{1.578733in}}%
\pgfpathlineto{\pgfqpoint{0.783142in}{1.582642in}}%
\pgfpathlineto{\pgfqpoint{0.774731in}{1.583981in}}%
\pgfpathlineto{\pgfqpoint{0.767486in}{1.585223in}}%
\pgfpathlineto{\pgfqpoint{0.751829in}{1.586375in}}%
\pgfpathlineto{\pgfqpoint{0.736173in}{1.585991in}}%
\pgfpathlineto{\pgfqpoint{0.720516in}{1.584070in}}%
\pgfpathlineto{\pgfqpoint{0.720112in}{1.583981in}}%
\pgfpathlineto{\pgfqpoint{0.704859in}{1.580866in}}%
\pgfpathlineto{\pgfqpoint{0.689203in}{1.576244in}}%
\pgfpathlineto{\pgfqpoint{0.673997in}{1.570370in}}%
\pgfpathlineto{\pgfqpoint{0.673546in}{1.570204in}}%
\pgfpathlineto{\pgfqpoint{0.657890in}{1.563151in}}%
\pgfpathlineto{\pgfqpoint{0.645918in}{1.556759in}}%
\pgfpathlineto{\pgfqpoint{0.642233in}{1.554835in}}%
\pgfpathlineto{\pgfqpoint{0.626577in}{1.545492in}}%
\pgfpathlineto{\pgfqpoint{0.623053in}{1.543148in}}%
\pgfpathlineto{\pgfqpoint{0.610920in}{1.535083in}}%
\pgfpathlineto{\pgfqpoint{0.603303in}{1.529536in}}%
\pgfpathlineto{\pgfqpoint{0.595263in}{1.523579in}}%
\pgfpathlineto{\pgfqpoint{0.585646in}{1.515925in}}%
\pgfpathlineto{\pgfqpoint{0.579607in}{1.510946in}}%
\pgfpathlineto{\pgfqpoint{0.569678in}{1.502314in}}%
\pgfpathlineto{\pgfqpoint{0.563950in}{1.497064in}}%
\pgfpathlineto{\pgfqpoint{0.555146in}{1.488703in}}%
\pgfpathlineto{\pgfqpoint{0.548294in}{1.481714in}}%
\pgfpathlineto{\pgfqpoint{0.541914in}{1.475092in}}%
\pgfpathlineto{\pgfqpoint{0.532637in}{1.464544in}}%
\pgfpathlineto{\pgfqpoint{0.529940in}{1.461481in}}%
\pgfpathlineto{\pgfqpoint{0.519193in}{1.447870in}}%
\pgfpathlineto{\pgfqpoint{0.516981in}{1.444666in}}%
\pgfpathlineto{\pgfqpoint{0.509627in}{1.434259in}}%
\pgfpathlineto{\pgfqpoint{0.501515in}{1.420648in}}%
\pgfpathlineto{\pgfqpoint{0.501324in}{1.420256in}}%
\pgfpathlineto{\pgfqpoint{0.494567in}{1.407036in}}%
\pgfpathlineto{\pgfqpoint{0.489251in}{1.393425in}}%
\pgfpathlineto{\pgfqpoint{0.485668in}{1.380165in}}%
\pgfpathlineto{\pgfqpoint{0.485565in}{1.379814in}}%
\pgfpathlineto{\pgfqpoint{0.483355in}{1.366203in}}%
\pgfpathlineto{\pgfqpoint{0.482913in}{1.352592in}}%
\pgfpathlineto{\pgfqpoint{0.484239in}{1.338981in}}%
\pgfpathlineto{\pgfqpoint{0.485668in}{1.332683in}}%
\pgfpathlineto{\pgfqpoint{0.487207in}{1.325370in}}%
\pgfpathlineto{\pgfqpoint{0.491704in}{1.311759in}}%
\pgfpathlineto{\pgfqpoint{0.497842in}{1.298148in}}%
\pgfpathlineto{\pgfqpoint{0.501324in}{1.292010in}}%
\pgfpathlineto{\pgfqpoint{0.505378in}{1.284536in}}%
\pgfpathlineto{\pgfqpoint{0.514263in}{1.270925in}}%
\pgfpathlineto{\pgfqpoint{0.516981in}{1.267328in}}%
\pgfpathlineto{\pgfqpoint{0.524384in}{1.257314in}}%
\pgfpathlineto{\pgfqpoint{0.532637in}{1.247466in}}%
\pgfpathlineto{\pgfqpoint{0.535784in}{1.243703in}}%
\pgfpathlineto{\pgfqpoint{0.548294in}{1.230189in}}%
\pgfpathlineto{\pgfqpoint{0.548385in}{1.230092in}}%
\pgfpathlineto{\pgfqpoint{0.562229in}{1.216481in}}%
\pgfpathlineto{\pgfqpoint{0.563950in}{1.214891in}}%
\pgfpathlineto{\pgfqpoint{0.577443in}{1.202870in}}%
\pgfpathlineto{\pgfqpoint{0.579607in}{1.201026in}}%
\pgfpathlineto{\pgfqpoint{0.594180in}{1.189259in}}%
\pgfpathlineto{\pgfqpoint{0.595263in}{1.188407in}}%
\pgfpathlineto{\pgfqpoint{0.610920in}{1.176925in}}%
\pgfpathlineto{\pgfqpoint{0.612825in}{1.175647in}}%
\pgfpathlineto{\pgfqpoint{0.626577in}{1.166490in}}%
\pgfpathlineto{\pgfqpoint{0.634064in}{1.162036in}}%
\pgfpathlineto{\pgfqpoint{0.642233in}{1.157121in}}%
\pgfpathlineto{\pgfqpoint{0.657890in}{1.148911in}}%
\pgfpathlineto{\pgfqpoint{0.658977in}{1.148425in}}%
\pgfpathlineto{\pgfqpoint{0.673546in}{1.141691in}}%
\pgfpathlineto{\pgfqpoint{0.689203in}{1.135830in}}%
\pgfpathlineto{\pgfqpoint{0.692728in}{1.134814in}}%
\pgfpathlineto{\pgfqpoint{0.704859in}{1.131106in}}%
\pgfpathclose%
\pgfpathmoveto{\pgfqpoint{0.709106in}{1.175647in}}%
\pgfpathlineto{\pgfqpoint{0.704859in}{1.176491in}}%
\pgfpathlineto{\pgfqpoint{0.689203in}{1.180967in}}%
\pgfpathlineto{\pgfqpoint{0.673546in}{1.186827in}}%
\pgfpathlineto{\pgfqpoint{0.668236in}{1.189259in}}%
\pgfpathlineto{\pgfqpoint{0.657890in}{1.194075in}}%
\pgfpathlineto{\pgfqpoint{0.642233in}{1.202708in}}%
\pgfpathlineto{\pgfqpoint{0.641975in}{1.202870in}}%
\pgfpathlineto{\pgfqpoint{0.626577in}{1.212884in}}%
\pgfpathlineto{\pgfqpoint{0.621625in}{1.216481in}}%
\pgfpathlineto{\pgfqpoint{0.610920in}{1.224709in}}%
\pgfpathlineto{\pgfqpoint{0.604503in}{1.230092in}}%
\pgfpathlineto{\pgfqpoint{0.595263in}{1.238460in}}%
\pgfpathlineto{\pgfqpoint{0.589847in}{1.243703in}}%
\pgfpathlineto{\pgfqpoint{0.579607in}{1.254642in}}%
\pgfpathlineto{\pgfqpoint{0.577220in}{1.257314in}}%
\pgfpathlineto{\pgfqpoint{0.566428in}{1.270925in}}%
\pgfpathlineto{\pgfqpoint{0.563950in}{1.274548in}}%
\pgfpathlineto{\pgfqpoint{0.557292in}{1.284536in}}%
\pgfpathlineto{\pgfqpoint{0.549767in}{1.298148in}}%
\pgfpathlineto{\pgfqpoint{0.548294in}{1.301498in}}%
\pgfpathlineto{\pgfqpoint{0.543801in}{1.311759in}}%
\pgfpathlineto{\pgfqpoint{0.539423in}{1.325370in}}%
\pgfpathlineto{\pgfqpoint{0.536639in}{1.338981in}}%
\pgfpathlineto{\pgfqpoint{0.535447in}{1.352592in}}%
\pgfpathlineto{\pgfqpoint{0.535844in}{1.366203in}}%
\pgfpathlineto{\pgfqpoint{0.537832in}{1.379814in}}%
\pgfpathlineto{\pgfqpoint{0.541412in}{1.393425in}}%
\pgfpathlineto{\pgfqpoint{0.546589in}{1.407036in}}%
\pgfpathlineto{\pgfqpoint{0.548294in}{1.410492in}}%
\pgfpathlineto{\pgfqpoint{0.553331in}{1.420648in}}%
\pgfpathlineto{\pgfqpoint{0.561650in}{1.434259in}}%
\pgfpathlineto{\pgfqpoint{0.563950in}{1.437462in}}%
\pgfpathlineto{\pgfqpoint{0.571626in}{1.447870in}}%
\pgfpathlineto{\pgfqpoint{0.579607in}{1.457307in}}%
\pgfpathlineto{\pgfqpoint{0.583302in}{1.461481in}}%
\pgfpathlineto{\pgfqpoint{0.595263in}{1.473585in}}%
\pgfpathlineto{\pgfqpoint{0.596853in}{1.475092in}}%
\pgfpathlineto{\pgfqpoint{0.610920in}{1.487321in}}%
\pgfpathlineto{\pgfqpoint{0.612653in}{1.488703in}}%
\pgfpathlineto{\pgfqpoint{0.626577in}{1.499101in}}%
\pgfpathlineto{\pgfqpoint{0.631378in}{1.502314in}}%
\pgfpathlineto{\pgfqpoint{0.642233in}{1.509253in}}%
\pgfpathlineto{\pgfqpoint{0.654205in}{1.515925in}}%
\pgfpathlineto{\pgfqpoint{0.657890in}{1.517925in}}%
\pgfpathlineto{\pgfqpoint{0.673546in}{1.525157in}}%
\pgfpathlineto{\pgfqpoint{0.685228in}{1.529536in}}%
\pgfpathlineto{\pgfqpoint{0.689203in}{1.531018in}}%
\pgfpathlineto{\pgfqpoint{0.704859in}{1.535519in}}%
\pgfpathlineto{\pgfqpoint{0.720516in}{1.538632in}}%
\pgfpathlineto{\pgfqpoint{0.736173in}{1.540360in}}%
\pgfpathlineto{\pgfqpoint{0.751829in}{1.540705in}}%
\pgfpathlineto{\pgfqpoint{0.767486in}{1.539669in}}%
\pgfpathlineto{\pgfqpoint{0.783142in}{1.537248in}}%
\pgfpathlineto{\pgfqpoint{0.798799in}{1.533442in}}%
\pgfpathlineto{\pgfqpoint{0.810601in}{1.529536in}}%
\pgfpathlineto{\pgfqpoint{0.814455in}{1.528256in}}%
\pgfpathlineto{\pgfqpoint{0.830112in}{1.521714in}}%
\pgfpathlineto{\pgfqpoint{0.841601in}{1.515925in}}%
\pgfpathlineto{\pgfqpoint{0.845769in}{1.513771in}}%
\pgfpathlineto{\pgfqpoint{0.861425in}{1.504389in}}%
\pgfpathlineto{\pgfqpoint{0.864499in}{1.502314in}}%
\pgfpathlineto{\pgfqpoint{0.877082in}{1.493412in}}%
\pgfpathlineto{\pgfqpoint{0.883113in}{1.488703in}}%
\pgfpathlineto{\pgfqpoint{0.892738in}{1.480671in}}%
\pgfpathlineto{\pgfqpoint{0.898930in}{1.475092in}}%
\pgfpathlineto{\pgfqpoint{0.908395in}{1.465786in}}%
\pgfpathlineto{\pgfqpoint{0.912532in}{1.461481in}}%
\pgfpathlineto{\pgfqpoint{0.924051in}{1.448094in}}%
\pgfpathlineto{\pgfqpoint{0.924237in}{1.447870in}}%
\pgfpathlineto{\pgfqpoint{0.934167in}{1.434259in}}%
\pgfpathlineto{\pgfqpoint{0.939708in}{1.425264in}}%
\pgfpathlineto{\pgfqpoint{0.942505in}{1.420648in}}%
\pgfpathlineto{\pgfqpoint{0.949245in}{1.407036in}}%
\pgfpathlineto{\pgfqpoint{0.954394in}{1.393425in}}%
\pgfpathlineto{\pgfqpoint{0.955364in}{1.389733in}}%
\pgfpathlineto{\pgfqpoint{0.957994in}{1.379814in}}%
\pgfpathlineto{\pgfqpoint{0.960001in}{1.366203in}}%
\pgfpathlineto{\pgfqpoint{0.960402in}{1.352592in}}%
\pgfpathlineto{\pgfqpoint{0.959198in}{1.338981in}}%
\pgfpathlineto{\pgfqpoint{0.956388in}{1.325370in}}%
\pgfpathlineto{\pgfqpoint{0.955364in}{1.322199in}}%
\pgfpathlineto{\pgfqpoint{0.952018in}{1.311759in}}%
\pgfpathlineto{\pgfqpoint{0.946074in}{1.298148in}}%
\pgfpathlineto{\pgfqpoint{0.939708in}{1.286624in}}%
\pgfpathlineto{\pgfqpoint{0.938537in}{1.284536in}}%
\pgfpathlineto{\pgfqpoint{0.929400in}{1.270925in}}%
\pgfpathlineto{\pgfqpoint{0.924051in}{1.264070in}}%
\pgfpathlineto{\pgfqpoint{0.918585in}{1.257314in}}%
\pgfpathlineto{\pgfqpoint{0.908395in}{1.246175in}}%
\pgfpathlineto{\pgfqpoint{0.906005in}{1.243703in}}%
\pgfpathlineto{\pgfqpoint{0.892738in}{1.231262in}}%
\pgfpathlineto{\pgfqpoint{0.891392in}{1.230092in}}%
\pgfpathlineto{\pgfqpoint{0.877082in}{1.218559in}}%
\pgfpathlineto{\pgfqpoint{0.874238in}{1.216481in}}%
\pgfpathlineto{\pgfqpoint{0.861425in}{1.207622in}}%
\pgfpathlineto{\pgfqpoint{0.853654in}{1.202870in}}%
\pgfpathlineto{\pgfqpoint{0.845769in}{1.198220in}}%
\pgfpathlineto{\pgfqpoint{0.830112in}{1.190276in}}%
\pgfpathlineto{\pgfqpoint{0.827711in}{1.189259in}}%
\pgfpathlineto{\pgfqpoint{0.814455in}{1.183724in}}%
\pgfpathlineto{\pgfqpoint{0.798799in}{1.178557in}}%
\pgfpathlineto{\pgfqpoint{0.786790in}{1.175647in}}%
\pgfpathlineto{\pgfqpoint{0.783142in}{1.174758in}}%
\pgfpathlineto{\pgfqpoint{0.767486in}{1.172315in}}%
\pgfpathlineto{\pgfqpoint{0.751829in}{1.171268in}}%
\pgfpathlineto{\pgfqpoint{0.736173in}{1.171617in}}%
\pgfpathlineto{\pgfqpoint{0.720516in}{1.173362in}}%
\pgfpathlineto{\pgfqpoint{0.709106in}{1.175647in}}%
\pgfpathclose%
\pgfpathmoveto{\pgfqpoint{1.472031in}{1.133319in}}%
\pgfpathlineto{\pgfqpoint{1.487688in}{1.129263in}}%
\pgfpathlineto{\pgfqpoint{1.503344in}{1.126683in}}%
\pgfpathlineto{\pgfqpoint{1.519001in}{1.125579in}}%
\pgfpathlineto{\pgfqpoint{1.534657in}{1.125947in}}%
\pgfpathlineto{\pgfqpoint{1.550314in}{1.127789in}}%
\pgfpathlineto{\pgfqpoint{1.565971in}{1.131106in}}%
\pgfpathlineto{\pgfqpoint{1.578102in}{1.134814in}}%
\pgfpathlineto{\pgfqpoint{1.581627in}{1.135830in}}%
\pgfpathlineto{\pgfqpoint{1.597284in}{1.141691in}}%
\pgfpathlineto{\pgfqpoint{1.611853in}{1.148425in}}%
\pgfpathlineto{\pgfqpoint{1.612940in}{1.148911in}}%
\pgfpathlineto{\pgfqpoint{1.628597in}{1.157121in}}%
\pgfpathlineto{\pgfqpoint{1.636766in}{1.162036in}}%
\pgfpathlineto{\pgfqpoint{1.644253in}{1.166490in}}%
\pgfpathlineto{\pgfqpoint{1.658005in}{1.175647in}}%
\pgfpathlineto{\pgfqpoint{1.659910in}{1.176925in}}%
\pgfpathlineto{\pgfqpoint{1.675567in}{1.188407in}}%
\pgfpathlineto{\pgfqpoint{1.676650in}{1.189259in}}%
\pgfpathlineto{\pgfqpoint{1.691223in}{1.201026in}}%
\pgfpathlineto{\pgfqpoint{1.693387in}{1.202870in}}%
\pgfpathlineto{\pgfqpoint{1.706880in}{1.214891in}}%
\pgfpathlineto{\pgfqpoint{1.708601in}{1.216481in}}%
\pgfpathlineto{\pgfqpoint{1.722445in}{1.230092in}}%
\pgfpathlineto{\pgfqpoint{1.722536in}{1.230189in}}%
\pgfpathlineto{\pgfqpoint{1.735046in}{1.243703in}}%
\pgfpathlineto{\pgfqpoint{1.738193in}{1.247466in}}%
\pgfpathlineto{\pgfqpoint{1.746446in}{1.257314in}}%
\pgfpathlineto{\pgfqpoint{1.753849in}{1.267328in}}%
\pgfpathlineto{\pgfqpoint{1.756567in}{1.270925in}}%
\pgfpathlineto{\pgfqpoint{1.765452in}{1.284536in}}%
\pgfpathlineto{\pgfqpoint{1.769506in}{1.292010in}}%
\pgfpathlineto{\pgfqpoint{1.772988in}{1.298148in}}%
\pgfpathlineto{\pgfqpoint{1.779126in}{1.311759in}}%
\pgfpathlineto{\pgfqpoint{1.783623in}{1.325370in}}%
\pgfpathlineto{\pgfqpoint{1.785162in}{1.332683in}}%
\pgfpathlineto{\pgfqpoint{1.786591in}{1.338981in}}%
\pgfpathlineto{\pgfqpoint{1.787917in}{1.352592in}}%
\pgfpathlineto{\pgfqpoint{1.787475in}{1.366203in}}%
\pgfpathlineto{\pgfqpoint{1.785265in}{1.379814in}}%
\pgfpathlineto{\pgfqpoint{1.785162in}{1.380165in}}%
\pgfpathlineto{\pgfqpoint{1.781579in}{1.393425in}}%
\pgfpathlineto{\pgfqpoint{1.776263in}{1.407036in}}%
\pgfpathlineto{\pgfqpoint{1.769506in}{1.420256in}}%
\pgfpathlineto{\pgfqpoint{1.769315in}{1.420648in}}%
\pgfpathlineto{\pgfqpoint{1.761203in}{1.434259in}}%
\pgfpathlineto{\pgfqpoint{1.753849in}{1.444666in}}%
\pgfpathlineto{\pgfqpoint{1.751637in}{1.447870in}}%
\pgfpathlineto{\pgfqpoint{1.740890in}{1.461481in}}%
\pgfpathlineto{\pgfqpoint{1.738193in}{1.464544in}}%
\pgfpathlineto{\pgfqpoint{1.728916in}{1.475092in}}%
\pgfpathlineto{\pgfqpoint{1.722536in}{1.481714in}}%
\pgfpathlineto{\pgfqpoint{1.715684in}{1.488703in}}%
\pgfpathlineto{\pgfqpoint{1.706880in}{1.497064in}}%
\pgfpathlineto{\pgfqpoint{1.701152in}{1.502314in}}%
\pgfpathlineto{\pgfqpoint{1.691223in}{1.510946in}}%
\pgfpathlineto{\pgfqpoint{1.685184in}{1.515925in}}%
\pgfpathlineto{\pgfqpoint{1.675567in}{1.523579in}}%
\pgfpathlineto{\pgfqpoint{1.667527in}{1.529536in}}%
\pgfpathlineto{\pgfqpoint{1.659910in}{1.535083in}}%
\pgfpathlineto{\pgfqpoint{1.647777in}{1.543148in}}%
\pgfpathlineto{\pgfqpoint{1.644253in}{1.545492in}}%
\pgfpathlineto{\pgfqpoint{1.628597in}{1.554835in}}%
\pgfpathlineto{\pgfqpoint{1.624912in}{1.556759in}}%
\pgfpathlineto{\pgfqpoint{1.612940in}{1.563151in}}%
\pgfpathlineto{\pgfqpoint{1.597284in}{1.570204in}}%
\pgfpathlineto{\pgfqpoint{1.596833in}{1.570370in}}%
\pgfpathlineto{\pgfqpoint{1.581627in}{1.576244in}}%
\pgfpathlineto{\pgfqpoint{1.565971in}{1.580866in}}%
\pgfpathlineto{\pgfqpoint{1.550718in}{1.583981in}}%
\pgfpathlineto{\pgfqpoint{1.550314in}{1.584070in}}%
\pgfpathlineto{\pgfqpoint{1.534657in}{1.585991in}}%
\pgfpathlineto{\pgfqpoint{1.519001in}{1.586375in}}%
\pgfpathlineto{\pgfqpoint{1.503344in}{1.585223in}}%
\pgfpathlineto{\pgfqpoint{1.496099in}{1.583981in}}%
\pgfpathlineto{\pgfqpoint{1.487688in}{1.582642in}}%
\pgfpathlineto{\pgfqpoint{1.472031in}{1.578733in}}%
\pgfpathlineto{\pgfqpoint{1.456375in}{1.573397in}}%
\pgfpathlineto{\pgfqpoint{1.449315in}{1.570370in}}%
\pgfpathlineto{\pgfqpoint{1.440718in}{1.566846in}}%
\pgfpathlineto{\pgfqpoint{1.425061in}{1.559121in}}%
\pgfpathlineto{\pgfqpoint{1.420923in}{1.556759in}}%
\pgfpathlineto{\pgfqpoint{1.409405in}{1.550322in}}%
\pgfpathlineto{\pgfqpoint{1.398076in}{1.543148in}}%
\pgfpathlineto{\pgfqpoint{1.393748in}{1.540412in}}%
\pgfpathlineto{\pgfqpoint{1.378204in}{1.529536in}}%
\pgfpathlineto{\pgfqpoint{1.378092in}{1.529457in}}%
\pgfpathlineto{\pgfqpoint{1.362435in}{1.517421in}}%
\pgfpathlineto{\pgfqpoint{1.360607in}{1.515925in}}%
\pgfpathlineto{\pgfqpoint{1.346779in}{1.504196in}}%
\pgfpathlineto{\pgfqpoint{1.344658in}{1.502314in}}%
\pgfpathlineto{\pgfqpoint{1.331122in}{1.489645in}}%
\pgfpathlineto{\pgfqpoint{1.330142in}{1.488703in}}%
\pgfpathlineto{\pgfqpoint{1.316935in}{1.475092in}}%
\pgfpathlineto{\pgfqpoint{1.315466in}{1.473436in}}%
\pgfpathlineto{\pgfqpoint{1.304931in}{1.461481in}}%
\pgfpathlineto{\pgfqpoint{1.299809in}{1.454972in}}%
\pgfpathlineto{\pgfqpoint{1.294155in}{1.447870in}}%
\pgfpathlineto{\pgfqpoint{1.284711in}{1.434259in}}%
\pgfpathlineto{\pgfqpoint{1.284152in}{1.433314in}}%
\pgfpathlineto{\pgfqpoint{1.276406in}{1.420648in}}%
\pgfpathlineto{\pgfqpoint{1.269664in}{1.407036in}}%
\pgfpathlineto{\pgfqpoint{1.268496in}{1.403972in}}%
\pgfpathlineto{\pgfqpoint{1.264230in}{1.393425in}}%
\pgfpathlineto{\pgfqpoint{1.260414in}{1.379814in}}%
\pgfpathlineto{\pgfqpoint{1.258296in}{1.366203in}}%
\pgfpathlineto{\pgfqpoint{1.257872in}{1.352592in}}%
\pgfpathlineto{\pgfqpoint{1.259143in}{1.338981in}}%
\pgfpathlineto{\pgfqpoint{1.262110in}{1.325370in}}%
\pgfpathlineto{\pgfqpoint{1.266776in}{1.311759in}}%
\pgfpathlineto{\pgfqpoint{1.268496in}{1.308054in}}%
\pgfpathlineto{\pgfqpoint{1.272836in}{1.298148in}}%
\pgfpathlineto{\pgfqpoint{1.280374in}{1.284536in}}%
\pgfpathlineto{\pgfqpoint{1.284152in}{1.278848in}}%
\pgfpathlineto{\pgfqpoint{1.289245in}{1.270925in}}%
\pgfpathlineto{\pgfqpoint{1.299439in}{1.257314in}}%
\pgfpathlineto{\pgfqpoint{1.299809in}{1.256874in}}%
\pgfpathlineto{\pgfqpoint{1.310768in}{1.243703in}}%
\pgfpathlineto{\pgfqpoint{1.315466in}{1.238604in}}%
\pgfpathlineto{\pgfqpoint{1.323372in}{1.230092in}}%
\pgfpathlineto{\pgfqpoint{1.331122in}{1.222399in}}%
\pgfpathlineto{\pgfqpoint{1.337243in}{1.216481in}}%
\pgfpathlineto{\pgfqpoint{1.346779in}{1.207817in}}%
\pgfpathlineto{\pgfqpoint{1.352470in}{1.202870in}}%
\pgfpathlineto{\pgfqpoint{1.362435in}{1.194580in}}%
\pgfpathlineto{\pgfqpoint{1.369243in}{1.189259in}}%
\pgfpathlineto{\pgfqpoint{1.378092in}{1.182521in}}%
\pgfpathlineto{\pgfqpoint{1.387883in}{1.175647in}}%
\pgfpathlineto{\pgfqpoint{1.393748in}{1.171564in}}%
\pgfpathlineto{\pgfqpoint{1.408899in}{1.162036in}}%
\pgfpathlineto{\pgfqpoint{1.409405in}{1.161715in}}%
\pgfpathlineto{\pgfqpoint{1.425061in}{1.152852in}}%
\pgfpathlineto{\pgfqpoint{1.434175in}{1.148425in}}%
\pgfpathlineto{\pgfqpoint{1.440718in}{1.145140in}}%
\pgfpathlineto{\pgfqpoint{1.456375in}{1.138587in}}%
\pgfpathlineto{\pgfqpoint{1.467769in}{1.134814in}}%
\pgfpathlineto{\pgfqpoint{1.472031in}{1.133319in}}%
\pgfpathclose%
\pgfpathmoveto{\pgfqpoint{1.484040in}{1.175647in}}%
\pgfpathlineto{\pgfqpoint{1.472031in}{1.178557in}}%
\pgfpathlineto{\pgfqpoint{1.456375in}{1.183724in}}%
\pgfpathlineto{\pgfqpoint{1.443119in}{1.189259in}}%
\pgfpathlineto{\pgfqpoint{1.440718in}{1.190276in}}%
\pgfpathlineto{\pgfqpoint{1.425061in}{1.198220in}}%
\pgfpathlineto{\pgfqpoint{1.417176in}{1.202870in}}%
\pgfpathlineto{\pgfqpoint{1.409405in}{1.207622in}}%
\pgfpathlineto{\pgfqpoint{1.396592in}{1.216481in}}%
\pgfpathlineto{\pgfqpoint{1.393748in}{1.218559in}}%
\pgfpathlineto{\pgfqpoint{1.379438in}{1.230092in}}%
\pgfpathlineto{\pgfqpoint{1.378092in}{1.231262in}}%
\pgfpathlineto{\pgfqpoint{1.364825in}{1.243703in}}%
\pgfpathlineto{\pgfqpoint{1.362435in}{1.246175in}}%
\pgfpathlineto{\pgfqpoint{1.352245in}{1.257314in}}%
\pgfpathlineto{\pgfqpoint{1.346779in}{1.264070in}}%
\pgfpathlineto{\pgfqpoint{1.341430in}{1.270925in}}%
\pgfpathlineto{\pgfqpoint{1.332293in}{1.284536in}}%
\pgfpathlineto{\pgfqpoint{1.331122in}{1.286624in}}%
\pgfpathlineto{\pgfqpoint{1.324756in}{1.298148in}}%
\pgfpathlineto{\pgfqpoint{1.318812in}{1.311759in}}%
\pgfpathlineto{\pgfqpoint{1.315466in}{1.322199in}}%
\pgfpathlineto{\pgfqpoint{1.314442in}{1.325370in}}%
\pgfpathlineto{\pgfqpoint{1.311632in}{1.338981in}}%
\pgfpathlineto{\pgfqpoint{1.310428in}{1.352592in}}%
\pgfpathlineto{\pgfqpoint{1.310829in}{1.366203in}}%
\pgfpathlineto{\pgfqpoint{1.312836in}{1.379814in}}%
\pgfpathlineto{\pgfqpoint{1.315466in}{1.389733in}}%
\pgfpathlineto{\pgfqpoint{1.316436in}{1.393425in}}%
\pgfpathlineto{\pgfqpoint{1.321585in}{1.407036in}}%
\pgfpathlineto{\pgfqpoint{1.328325in}{1.420648in}}%
\pgfpathlineto{\pgfqpoint{1.331122in}{1.425264in}}%
\pgfpathlineto{\pgfqpoint{1.336663in}{1.434259in}}%
\pgfpathlineto{\pgfqpoint{1.346593in}{1.447870in}}%
\pgfpathlineto{\pgfqpoint{1.346779in}{1.448094in}}%
\pgfpathlineto{\pgfqpoint{1.358298in}{1.461481in}}%
\pgfpathlineto{\pgfqpoint{1.362435in}{1.465786in}}%
\pgfpathlineto{\pgfqpoint{1.371900in}{1.475092in}}%
\pgfpathlineto{\pgfqpoint{1.378092in}{1.480671in}}%
\pgfpathlineto{\pgfqpoint{1.387717in}{1.488703in}}%
\pgfpathlineto{\pgfqpoint{1.393748in}{1.493412in}}%
\pgfpathlineto{\pgfqpoint{1.406331in}{1.502314in}}%
\pgfpathlineto{\pgfqpoint{1.409405in}{1.504389in}}%
\pgfpathlineto{\pgfqpoint{1.425061in}{1.513771in}}%
\pgfpathlineto{\pgfqpoint{1.429229in}{1.515925in}}%
\pgfpathlineto{\pgfqpoint{1.440718in}{1.521714in}}%
\pgfpathlineto{\pgfqpoint{1.456375in}{1.528256in}}%
\pgfpathlineto{\pgfqpoint{1.460229in}{1.529536in}}%
\pgfpathlineto{\pgfqpoint{1.472031in}{1.533442in}}%
\pgfpathlineto{\pgfqpoint{1.487688in}{1.537248in}}%
\pgfpathlineto{\pgfqpoint{1.503344in}{1.539669in}}%
\pgfpathlineto{\pgfqpoint{1.519001in}{1.540705in}}%
\pgfpathlineto{\pgfqpoint{1.534657in}{1.540360in}}%
\pgfpathlineto{\pgfqpoint{1.550314in}{1.538632in}}%
\pgfpathlineto{\pgfqpoint{1.565971in}{1.535519in}}%
\pgfpathlineto{\pgfqpoint{1.581627in}{1.531018in}}%
\pgfpathlineto{\pgfqpoint{1.585602in}{1.529536in}}%
\pgfpathlineto{\pgfqpoint{1.597284in}{1.525157in}}%
\pgfpathlineto{\pgfqpoint{1.612940in}{1.517925in}}%
\pgfpathlineto{\pgfqpoint{1.616625in}{1.515925in}}%
\pgfpathlineto{\pgfqpoint{1.628597in}{1.509253in}}%
\pgfpathlineto{\pgfqpoint{1.639452in}{1.502314in}}%
\pgfpathlineto{\pgfqpoint{1.644253in}{1.499101in}}%
\pgfpathlineto{\pgfqpoint{1.658177in}{1.488703in}}%
\pgfpathlineto{\pgfqpoint{1.659910in}{1.487321in}}%
\pgfpathlineto{\pgfqpoint{1.673977in}{1.475092in}}%
\pgfpathlineto{\pgfqpoint{1.675567in}{1.473585in}}%
\pgfpathlineto{\pgfqpoint{1.687528in}{1.461481in}}%
\pgfpathlineto{\pgfqpoint{1.691223in}{1.457307in}}%
\pgfpathlineto{\pgfqpoint{1.699204in}{1.447870in}}%
\pgfpathlineto{\pgfqpoint{1.706880in}{1.437462in}}%
\pgfpathlineto{\pgfqpoint{1.709180in}{1.434259in}}%
\pgfpathlineto{\pgfqpoint{1.717499in}{1.420648in}}%
\pgfpathlineto{\pgfqpoint{1.722536in}{1.410492in}}%
\pgfpathlineto{\pgfqpoint{1.724241in}{1.407036in}}%
\pgfpathlineto{\pgfqpoint{1.729418in}{1.393425in}}%
\pgfpathlineto{\pgfqpoint{1.732998in}{1.379814in}}%
\pgfpathlineto{\pgfqpoint{1.734986in}{1.366203in}}%
\pgfpathlineto{\pgfqpoint{1.735383in}{1.352592in}}%
\pgfpathlineto{\pgfqpoint{1.734191in}{1.338981in}}%
\pgfpathlineto{\pgfqpoint{1.731407in}{1.325370in}}%
\pgfpathlineto{\pgfqpoint{1.727029in}{1.311759in}}%
\pgfpathlineto{\pgfqpoint{1.722536in}{1.301498in}}%
\pgfpathlineto{\pgfqpoint{1.721063in}{1.298148in}}%
\pgfpathlineto{\pgfqpoint{1.713538in}{1.284536in}}%
\pgfpathlineto{\pgfqpoint{1.706880in}{1.274548in}}%
\pgfpathlineto{\pgfqpoint{1.704402in}{1.270925in}}%
\pgfpathlineto{\pgfqpoint{1.693610in}{1.257314in}}%
\pgfpathlineto{\pgfqpoint{1.691223in}{1.254642in}}%
\pgfpathlineto{\pgfqpoint{1.680983in}{1.243703in}}%
\pgfpathlineto{\pgfqpoint{1.675567in}{1.238460in}}%
\pgfpathlineto{\pgfqpoint{1.666327in}{1.230092in}}%
\pgfpathlineto{\pgfqpoint{1.659910in}{1.224709in}}%
\pgfpathlineto{\pgfqpoint{1.649205in}{1.216481in}}%
\pgfpathlineto{\pgfqpoint{1.644253in}{1.212884in}}%
\pgfpathlineto{\pgfqpoint{1.628855in}{1.202870in}}%
\pgfpathlineto{\pgfqpoint{1.628597in}{1.202708in}}%
\pgfpathlineto{\pgfqpoint{1.612940in}{1.194075in}}%
\pgfpathlineto{\pgfqpoint{1.602594in}{1.189259in}}%
\pgfpathlineto{\pgfqpoint{1.597284in}{1.186827in}}%
\pgfpathlineto{\pgfqpoint{1.581627in}{1.180967in}}%
\pgfpathlineto{\pgfqpoint{1.565971in}{1.176491in}}%
\pgfpathlineto{\pgfqpoint{1.561724in}{1.175647in}}%
\pgfpathlineto{\pgfqpoint{1.550314in}{1.173362in}}%
\pgfpathlineto{\pgfqpoint{1.534657in}{1.171617in}}%
\pgfpathlineto{\pgfqpoint{1.519001in}{1.171268in}}%
\pgfpathlineto{\pgfqpoint{1.503344in}{1.172315in}}%
\pgfpathlineto{\pgfqpoint{1.487688in}{1.174758in}}%
\pgfpathlineto{\pgfqpoint{1.484040in}{1.175647in}}%
\pgfpathclose%
\pgfusepath{fill}%
\end{pgfscope}%
\begin{pgfscope}%
\pgfpathrectangle{\pgfqpoint{0.360415in}{0.345370in}}{\pgfqpoint{1.550000in}{1.347500in}}%
\pgfusepath{clip}%
\pgfsetbuttcap%
\pgfsetroundjoin%
\definecolor{currentfill}{rgb}{0.794549,0.275770,0.473117}%
\pgfsetfillcolor{currentfill}%
\pgfsetlinewidth{0.000000pt}%
\definecolor{currentstroke}{rgb}{0.000000,0.000000,0.000000}%
\pgfsetstrokecolor{currentstroke}%
\pgfsetdash{}{0pt}%
\pgfpathmoveto{\pgfqpoint{0.736173in}{0.382885in}}%
\pgfpathlineto{\pgfqpoint{0.751829in}{0.381980in}}%
\pgfpathlineto{\pgfqpoint{0.767486in}{0.384694in}}%
\pgfpathlineto{\pgfqpoint{0.771229in}{0.386203in}}%
\pgfpathlineto{\pgfqpoint{0.783142in}{0.389707in}}%
\pgfpathlineto{\pgfqpoint{0.798799in}{0.396939in}}%
\pgfpathlineto{\pgfqpoint{0.803393in}{0.399814in}}%
\pgfpathlineto{\pgfqpoint{0.814455in}{0.405375in}}%
\pgfpathlineto{\pgfqpoint{0.827160in}{0.413425in}}%
\pgfpathlineto{\pgfqpoint{0.830112in}{0.415020in}}%
\pgfpathlineto{\pgfqpoint{0.845769in}{0.425146in}}%
\pgfpathlineto{\pgfqpoint{0.848299in}{0.427036in}}%
\pgfpathlineto{\pgfqpoint{0.861425in}{0.435774in}}%
\pgfpathlineto{\pgfqpoint{0.867914in}{0.440648in}}%
\pgfpathlineto{\pgfqpoint{0.877082in}{0.446985in}}%
\pgfpathlineto{\pgfqpoint{0.886610in}{0.454259in}}%
\pgfpathlineto{\pgfqpoint{0.892738in}{0.458681in}}%
\pgfpathlineto{\pgfqpoint{0.904489in}{0.467870in}}%
\pgfpathlineto{\pgfqpoint{0.908395in}{0.470822in}}%
\pgfpathlineto{\pgfqpoint{0.921630in}{0.481481in}}%
\pgfpathlineto{\pgfqpoint{0.924051in}{0.483405in}}%
\pgfpathlineto{\pgfqpoint{0.938094in}{0.495092in}}%
\pgfpathlineto{\pgfqpoint{0.939708in}{0.496441in}}%
\pgfpathlineto{\pgfqpoint{0.953937in}{0.508703in}}%
\pgfpathlineto{\pgfqpoint{0.955364in}{0.509960in}}%
\pgfpathlineto{\pgfqpoint{0.969206in}{0.522314in}}%
\pgfpathlineto{\pgfqpoint{0.971021in}{0.523998in}}%
\pgfpathlineto{\pgfqpoint{0.983937in}{0.535925in}}%
\pgfpathlineto{\pgfqpoint{0.986678in}{0.538599in}}%
\pgfpathlineto{\pgfqpoint{0.998156in}{0.549536in}}%
\pgfpathlineto{\pgfqpoint{1.002334in}{0.553814in}}%
\pgfpathlineto{\pgfqpoint{1.011868in}{0.563148in}}%
\pgfpathlineto{\pgfqpoint{1.017991in}{0.569703in}}%
\pgfpathlineto{\pgfqpoint{1.025052in}{0.576759in}}%
\pgfpathlineto{\pgfqpoint{1.033647in}{0.586342in}}%
\pgfpathlineto{\pgfqpoint{1.037633in}{0.590370in}}%
\pgfpathlineto{\pgfqpoint{1.049304in}{0.603843in}}%
\pgfpathlineto{\pgfqpoint{1.049441in}{0.603981in}}%
\pgfpathlineto{\pgfqpoint{1.060972in}{0.617592in}}%
\pgfpathlineto{\pgfqpoint{1.064960in}{0.623370in}}%
\pgfpathlineto{\pgfqpoint{1.071476in}{0.631203in}}%
\pgfpathlineto{\pgfqpoint{1.080171in}{0.644814in}}%
\pgfpathlineto{\pgfqpoint{1.080617in}{0.645818in}}%
\pgfpathlineto{\pgfqpoint{1.087902in}{0.658425in}}%
\pgfpathlineto{\pgfqpoint{1.092271in}{0.672036in}}%
\pgfpathlineto{\pgfqpoint{1.093144in}{0.685648in}}%
\pgfpathlineto{\pgfqpoint{1.090524in}{0.699259in}}%
\pgfpathlineto{\pgfqpoint{1.084405in}{0.712870in}}%
\pgfpathlineto{\pgfqpoint{1.080617in}{0.718255in}}%
\pgfpathlineto{\pgfqpoint{1.076159in}{0.726481in}}%
\pgfpathlineto{\pgfqpoint{1.066122in}{0.740092in}}%
\pgfpathlineto{\pgfqpoint{1.064960in}{0.741349in}}%
\pgfpathlineto{\pgfqpoint{1.055482in}{0.753703in}}%
\pgfpathlineto{\pgfqpoint{1.049304in}{0.760416in}}%
\pgfpathlineto{\pgfqpoint{1.043762in}{0.767314in}}%
\pgfpathlineto{\pgfqpoint{1.033647in}{0.778173in}}%
\pgfpathlineto{\pgfqpoint{1.031327in}{0.780925in}}%
\pgfpathlineto{\pgfqpoint{1.018369in}{0.794536in}}%
\pgfpathlineto{\pgfqpoint{1.017991in}{0.794898in}}%
\pgfpathlineto{\pgfqpoint{1.005040in}{0.808148in}}%
\pgfpathlineto{\pgfqpoint{1.002334in}{0.810703in}}%
\pgfpathlineto{\pgfqpoint{0.991132in}{0.821759in}}%
\pgfpathlineto{\pgfqpoint{0.986678in}{0.825887in}}%
\pgfpathlineto{\pgfqpoint{0.976675in}{0.835370in}}%
\pgfpathlineto{\pgfqpoint{0.971021in}{0.840487in}}%
\pgfpathlineto{\pgfqpoint{0.961671in}{0.848981in}}%
\pgfpathlineto{\pgfqpoint{0.955364in}{0.854539in}}%
\pgfpathlineto{\pgfqpoint{0.946102in}{0.862592in}}%
\pgfpathlineto{\pgfqpoint{0.939708in}{0.868075in}}%
\pgfpathlineto{\pgfqpoint{0.929938in}{0.876203in}}%
\pgfpathlineto{\pgfqpoint{0.924051in}{0.881119in}}%
\pgfpathlineto{\pgfqpoint{0.913143in}{0.889814in}}%
\pgfpathlineto{\pgfqpoint{0.908395in}{0.893686in}}%
\pgfpathlineto{\pgfqpoint{0.895678in}{0.903425in}}%
\pgfpathlineto{\pgfqpoint{0.892738in}{0.905777in}}%
\pgfpathlineto{\pgfqpoint{0.877498in}{0.917036in}}%
\pgfpathlineto{\pgfqpoint{0.877082in}{0.917365in}}%
\pgfpathlineto{\pgfqpoint{0.861425in}{0.928630in}}%
\pgfpathlineto{\pgfqpoint{0.858260in}{0.930648in}}%
\pgfpathlineto{\pgfqpoint{0.845769in}{0.939441in}}%
\pgfpathlineto{\pgfqpoint{0.837833in}{0.944259in}}%
\pgfpathlineto{\pgfqpoint{0.830112in}{0.949629in}}%
\pgfpathlineto{\pgfqpoint{0.815902in}{0.957870in}}%
\pgfpathlineto{\pgfqpoint{0.814455in}{0.958879in}}%
\pgfpathlineto{\pgfqpoint{0.798799in}{0.967605in}}%
\pgfpathlineto{\pgfqpoint{0.789337in}{0.971481in}}%
\pgfpathlineto{\pgfqpoint{0.783142in}{0.974774in}}%
\pgfpathlineto{\pgfqpoint{0.767486in}{0.980093in}}%
\pgfpathlineto{\pgfqpoint{0.751829in}{0.982371in}}%
\pgfpathlineto{\pgfqpoint{0.736173in}{0.981612in}}%
\pgfpathlineto{\pgfqpoint{0.720516in}{0.977814in}}%
\pgfpathlineto{\pgfqpoint{0.706015in}{0.971481in}}%
\pgfpathlineto{\pgfqpoint{0.704859in}{0.971093in}}%
\pgfpathlineto{\pgfqpoint{0.689203in}{0.963534in}}%
\pgfpathlineto{\pgfqpoint{0.680192in}{0.957870in}}%
\pgfpathlineto{\pgfqpoint{0.673546in}{0.954403in}}%
\pgfpathlineto{\pgfqpoint{0.657890in}{0.944378in}}%
\pgfpathlineto{\pgfqpoint{0.657731in}{0.944259in}}%
\pgfpathlineto{\pgfqpoint{0.642233in}{0.934112in}}%
\pgfpathlineto{\pgfqpoint{0.637600in}{0.930647in}}%
\pgfpathlineto{\pgfqpoint{0.626577in}{0.923175in}}%
\pgfpathlineto{\pgfqpoint{0.618460in}{0.917036in}}%
\pgfpathlineto{\pgfqpoint{0.610920in}{0.911714in}}%
\pgfpathlineto{\pgfqpoint{0.600184in}{0.903425in}}%
\pgfpathlineto{\pgfqpoint{0.595263in}{0.899793in}}%
\pgfpathlineto{\pgfqpoint{0.582682in}{0.889814in}}%
\pgfpathlineto{\pgfqpoint{0.579607in}{0.887432in}}%
\pgfpathlineto{\pgfqpoint{0.565887in}{0.876203in}}%
\pgfpathlineto{\pgfqpoint{0.563950in}{0.874625in}}%
\pgfpathlineto{\pgfqpoint{0.549740in}{0.862592in}}%
\pgfpathlineto{\pgfqpoint{0.548294in}{0.861351in}}%
\pgfpathlineto{\pgfqpoint{0.534189in}{0.848981in}}%
\pgfpathlineto{\pgfqpoint{0.532637in}{0.847578in}}%
\pgfpathlineto{\pgfqpoint{0.519193in}{0.835370in}}%
\pgfpathlineto{\pgfqpoint{0.516981in}{0.833265in}}%
\pgfpathlineto{\pgfqpoint{0.504720in}{0.821759in}}%
\pgfpathlineto{\pgfqpoint{0.501324in}{0.818363in}}%
\pgfpathlineto{\pgfqpoint{0.490754in}{0.808148in}}%
\pgfpathlineto{\pgfqpoint{0.485668in}{0.802820in}}%
\pgfpathlineto{\pgfqpoint{0.477301in}{0.794536in}}%
\pgfpathlineto{\pgfqpoint{0.470011in}{0.786567in}}%
\pgfpathlineto{\pgfqpoint{0.464405in}{0.780925in}}%
\pgfpathlineto{\pgfqpoint{0.454354in}{0.769514in}}%
\pgfpathlineto{\pgfqpoint{0.452180in}{0.767314in}}%
\pgfpathlineto{\pgfqpoint{0.440532in}{0.753703in}}%
\pgfpathlineto{\pgfqpoint{0.438698in}{0.751136in}}%
\pgfpathlineto{\pgfqpoint{0.429438in}{0.740092in}}%
\pgfpathlineto{\pgfqpoint{0.423041in}{0.730475in}}%
\pgfpathlineto{\pgfqpoint{0.419734in}{0.726481in}}%
\pgfpathlineto{\pgfqpoint{0.411415in}{0.712870in}}%
\pgfpathlineto{\pgfqpoint{0.407385in}{0.702513in}}%
\pgfpathlineto{\pgfqpoint{0.405649in}{0.699259in}}%
\pgfpathlineto{\pgfqpoint{0.402527in}{0.685648in}}%
\pgfpathlineto{\pgfqpoint{0.403568in}{0.672036in}}%
\pgfpathlineto{\pgfqpoint{0.407385in}{0.662043in}}%
\pgfpathlineto{\pgfqpoint{0.408392in}{0.658425in}}%
\pgfpathlineto{\pgfqpoint{0.415195in}{0.644814in}}%
\pgfpathlineto{\pgfqpoint{0.423041in}{0.633938in}}%
\pgfpathlineto{\pgfqpoint{0.424624in}{0.631203in}}%
\pgfpathlineto{\pgfqpoint{0.434857in}{0.617592in}}%
\pgfpathlineto{\pgfqpoint{0.438698in}{0.613415in}}%
\pgfpathlineto{\pgfqpoint{0.446103in}{0.603981in}}%
\pgfpathlineto{\pgfqpoint{0.454354in}{0.595055in}}%
\pgfpathlineto{\pgfqpoint{0.458209in}{0.590370in}}%
\pgfpathlineto{\pgfqpoint{0.470011in}{0.577827in}}%
\pgfpathlineto{\pgfqpoint{0.470935in}{0.576759in}}%
\pgfpathlineto{\pgfqpoint{0.484049in}{0.563148in}}%
\pgfpathlineto{\pgfqpoint{0.485668in}{0.561608in}}%
\pgfpathlineto{\pgfqpoint{0.497675in}{0.549536in}}%
\pgfpathlineto{\pgfqpoint{0.501324in}{0.546119in}}%
\pgfpathlineto{\pgfqpoint{0.511859in}{0.535925in}}%
\pgfpathlineto{\pgfqpoint{0.516981in}{0.531232in}}%
\pgfpathlineto{\pgfqpoint{0.526588in}{0.522314in}}%
\pgfpathlineto{\pgfqpoint{0.532637in}{0.516910in}}%
\pgfpathlineto{\pgfqpoint{0.541871in}{0.508703in}}%
\pgfpathlineto{\pgfqpoint{0.548294in}{0.503120in}}%
\pgfpathlineto{\pgfqpoint{0.557734in}{0.495092in}}%
\pgfpathlineto{\pgfqpoint{0.563950in}{0.489833in}}%
\pgfpathlineto{\pgfqpoint{0.574208in}{0.481481in}}%
\pgfpathlineto{\pgfqpoint{0.579607in}{0.477028in}}%
\pgfpathlineto{\pgfqpoint{0.591333in}{0.467870in}}%
\pgfpathlineto{\pgfqpoint{0.595263in}{0.464697in}}%
\pgfpathlineto{\pgfqpoint{0.609150in}{0.454259in}}%
\pgfpathlineto{\pgfqpoint{0.610920in}{0.452852in}}%
\pgfpathlineto{\pgfqpoint{0.626577in}{0.441451in}}%
\pgfpathlineto{\pgfqpoint{0.627805in}{0.440648in}}%
\pgfpathlineto{\pgfqpoint{0.642233in}{0.430388in}}%
\pgfpathlineto{\pgfqpoint{0.647622in}{0.427036in}}%
\pgfpathlineto{\pgfqpoint{0.657890in}{0.419863in}}%
\pgfpathlineto{\pgfqpoint{0.668742in}{0.413425in}}%
\pgfpathlineto{\pgfqpoint{0.673546in}{0.410086in}}%
\pgfpathlineto{\pgfqpoint{0.689203in}{0.401190in}}%
\pgfpathlineto{\pgfqpoint{0.692348in}{0.399814in}}%
\pgfpathlineto{\pgfqpoint{0.704859in}{0.392993in}}%
\pgfpathlineto{\pgfqpoint{0.720516in}{0.387078in}}%
\pgfpathlineto{\pgfqpoint{0.724678in}{0.386203in}}%
\pgfpathlineto{\pgfqpoint{0.736173in}{0.382885in}}%
\pgfpathclose%
\pgfpathmoveto{\pgfqpoint{0.720112in}{0.454259in}}%
\pgfpathlineto{\pgfqpoint{0.704859in}{0.457374in}}%
\pgfpathlineto{\pgfqpoint{0.689203in}{0.461996in}}%
\pgfpathlineto{\pgfqpoint{0.673997in}{0.467870in}}%
\pgfpathlineto{\pgfqpoint{0.673546in}{0.468036in}}%
\pgfpathlineto{\pgfqpoint{0.657890in}{0.475088in}}%
\pgfpathlineto{\pgfqpoint{0.645918in}{0.481481in}}%
\pgfpathlineto{\pgfqpoint{0.642233in}{0.483404in}}%
\pgfpathlineto{\pgfqpoint{0.626577in}{0.492747in}}%
\pgfpathlineto{\pgfqpoint{0.623053in}{0.495092in}}%
\pgfpathlineto{\pgfqpoint{0.610920in}{0.503157in}}%
\pgfpathlineto{\pgfqpoint{0.603303in}{0.508703in}}%
\pgfpathlineto{\pgfqpoint{0.595263in}{0.514660in}}%
\pgfpathlineto{\pgfqpoint{0.585646in}{0.522314in}}%
\pgfpathlineto{\pgfqpoint{0.579607in}{0.527293in}}%
\pgfpathlineto{\pgfqpoint{0.569678in}{0.535925in}}%
\pgfpathlineto{\pgfqpoint{0.563950in}{0.541176in}}%
\pgfpathlineto{\pgfqpoint{0.555146in}{0.549536in}}%
\pgfpathlineto{\pgfqpoint{0.548294in}{0.556526in}}%
\pgfpathlineto{\pgfqpoint{0.541914in}{0.563148in}}%
\pgfpathlineto{\pgfqpoint{0.532637in}{0.573695in}}%
\pgfpathlineto{\pgfqpoint{0.529940in}{0.576759in}}%
\pgfpathlineto{\pgfqpoint{0.519193in}{0.590370in}}%
\pgfpathlineto{\pgfqpoint{0.516981in}{0.593573in}}%
\pgfpathlineto{\pgfqpoint{0.509627in}{0.603981in}}%
\pgfpathlineto{\pgfqpoint{0.501515in}{0.617592in}}%
\pgfpathlineto{\pgfqpoint{0.501324in}{0.617984in}}%
\pgfpathlineto{\pgfqpoint{0.494567in}{0.631203in}}%
\pgfpathlineto{\pgfqpoint{0.489251in}{0.644814in}}%
\pgfpathlineto{\pgfqpoint{0.485668in}{0.658074in}}%
\pgfpathlineto{\pgfqpoint{0.485565in}{0.658425in}}%
\pgfpathlineto{\pgfqpoint{0.483355in}{0.672036in}}%
\pgfpathlineto{\pgfqpoint{0.482913in}{0.685648in}}%
\pgfpathlineto{\pgfqpoint{0.484239in}{0.699259in}}%
\pgfpathlineto{\pgfqpoint{0.485668in}{0.705557in}}%
\pgfpathlineto{\pgfqpoint{0.487207in}{0.712870in}}%
\pgfpathlineto{\pgfqpoint{0.491704in}{0.726481in}}%
\pgfpathlineto{\pgfqpoint{0.497842in}{0.740092in}}%
\pgfpathlineto{\pgfqpoint{0.501324in}{0.746229in}}%
\pgfpathlineto{\pgfqpoint{0.505378in}{0.753703in}}%
\pgfpathlineto{\pgfqpoint{0.514263in}{0.767314in}}%
\pgfpathlineto{\pgfqpoint{0.516981in}{0.770912in}}%
\pgfpathlineto{\pgfqpoint{0.524384in}{0.780925in}}%
\pgfpathlineto{\pgfqpoint{0.532637in}{0.790774in}}%
\pgfpathlineto{\pgfqpoint{0.535784in}{0.794536in}}%
\pgfpathlineto{\pgfqpoint{0.548294in}{0.808050in}}%
\pgfpathlineto{\pgfqpoint{0.548385in}{0.808148in}}%
\pgfpathlineto{\pgfqpoint{0.562229in}{0.821759in}}%
\pgfpathlineto{\pgfqpoint{0.563950in}{0.823348in}}%
\pgfpathlineto{\pgfqpoint{0.577443in}{0.835370in}}%
\pgfpathlineto{\pgfqpoint{0.579607in}{0.837214in}}%
\pgfpathlineto{\pgfqpoint{0.594180in}{0.848981in}}%
\pgfpathlineto{\pgfqpoint{0.595263in}{0.849833in}}%
\pgfpathlineto{\pgfqpoint{0.610920in}{0.861314in}}%
\pgfpathlineto{\pgfqpoint{0.612825in}{0.862592in}}%
\pgfpathlineto{\pgfqpoint{0.626577in}{0.871750in}}%
\pgfpathlineto{\pgfqpoint{0.634064in}{0.876203in}}%
\pgfpathlineto{\pgfqpoint{0.642233in}{0.881119in}}%
\pgfpathlineto{\pgfqpoint{0.657890in}{0.889329in}}%
\pgfpathlineto{\pgfqpoint{0.658977in}{0.889814in}}%
\pgfpathlineto{\pgfqpoint{0.673546in}{0.896549in}}%
\pgfpathlineto{\pgfqpoint{0.689203in}{0.902410in}}%
\pgfpathlineto{\pgfqpoint{0.692728in}{0.903425in}}%
\pgfpathlineto{\pgfqpoint{0.704859in}{0.907134in}}%
\pgfpathlineto{\pgfqpoint{0.720516in}{0.910451in}}%
\pgfpathlineto{\pgfqpoint{0.736173in}{0.912293in}}%
\pgfpathlineto{\pgfqpoint{0.751829in}{0.912661in}}%
\pgfpathlineto{\pgfqpoint{0.767486in}{0.911556in}}%
\pgfpathlineto{\pgfqpoint{0.783142in}{0.908977in}}%
\pgfpathlineto{\pgfqpoint{0.798799in}{0.904921in}}%
\pgfpathlineto{\pgfqpoint{0.803061in}{0.903425in}}%
\pgfpathlineto{\pgfqpoint{0.814455in}{0.899652in}}%
\pgfpathlineto{\pgfqpoint{0.830112in}{0.893099in}}%
\pgfpathlineto{\pgfqpoint{0.836655in}{0.889814in}}%
\pgfpathlineto{\pgfqpoint{0.845769in}{0.885387in}}%
\pgfpathlineto{\pgfqpoint{0.861425in}{0.876524in}}%
\pgfpathlineto{\pgfqpoint{0.861931in}{0.876203in}}%
\pgfpathlineto{\pgfqpoint{0.877082in}{0.866675in}}%
\pgfpathlineto{\pgfqpoint{0.882947in}{0.862592in}}%
\pgfpathlineto{\pgfqpoint{0.892738in}{0.855718in}}%
\pgfpathlineto{\pgfqpoint{0.901587in}{0.848981in}}%
\pgfpathlineto{\pgfqpoint{0.908395in}{0.843659in}}%
\pgfpathlineto{\pgfqpoint{0.918360in}{0.835370in}}%
\pgfpathlineto{\pgfqpoint{0.924051in}{0.830422in}}%
\pgfpathlineto{\pgfqpoint{0.933587in}{0.821759in}}%
\pgfpathlineto{\pgfqpoint{0.939708in}{0.815840in}}%
\pgfpathlineto{\pgfqpoint{0.947458in}{0.808148in}}%
\pgfpathlineto{\pgfqpoint{0.955364in}{0.799635in}}%
\pgfpathlineto{\pgfqpoint{0.960062in}{0.794536in}}%
\pgfpathlineto{\pgfqpoint{0.971021in}{0.781365in}}%
\pgfpathlineto{\pgfqpoint{0.971391in}{0.780925in}}%
\pgfpathlineto{\pgfqpoint{0.981585in}{0.767314in}}%
\pgfpathlineto{\pgfqpoint{0.986678in}{0.759392in}}%
\pgfpathlineto{\pgfqpoint{0.990456in}{0.753703in}}%
\pgfpathlineto{\pgfqpoint{0.997994in}{0.740092in}}%
\pgfpathlineto{\pgfqpoint{1.002334in}{0.730186in}}%
\pgfpathlineto{\pgfqpoint{1.004054in}{0.726481in}}%
\pgfpathlineto{\pgfqpoint{1.008720in}{0.712870in}}%
\pgfpathlineto{\pgfqpoint{1.011687in}{0.699259in}}%
\pgfpathlineto{\pgfqpoint{1.012958in}{0.685648in}}%
\pgfpathlineto{\pgfqpoint{1.012534in}{0.672036in}}%
\pgfpathlineto{\pgfqpoint{1.010416in}{0.658425in}}%
\pgfpathlineto{\pgfqpoint{1.006600in}{0.644814in}}%
\pgfpathlineto{\pgfqpoint{1.002334in}{0.634268in}}%
\pgfpathlineto{\pgfqpoint{1.001166in}{0.631203in}}%
\pgfpathlineto{\pgfqpoint{0.994424in}{0.617592in}}%
\pgfpathlineto{\pgfqpoint{0.986678in}{0.604926in}}%
\pgfpathlineto{\pgfqpoint{0.986119in}{0.603981in}}%
\pgfpathlineto{\pgfqpoint{0.976675in}{0.590370in}}%
\pgfpathlineto{\pgfqpoint{0.971021in}{0.583268in}}%
\pgfpathlineto{\pgfqpoint{0.965899in}{0.576759in}}%
\pgfpathlineto{\pgfqpoint{0.955364in}{0.564803in}}%
\pgfpathlineto{\pgfqpoint{0.953895in}{0.563148in}}%
\pgfpathlineto{\pgfqpoint{0.940688in}{0.549536in}}%
\pgfpathlineto{\pgfqpoint{0.939708in}{0.548594in}}%
\pgfpathlineto{\pgfqpoint{0.926172in}{0.535925in}}%
\pgfpathlineto{\pgfqpoint{0.924051in}{0.534044in}}%
\pgfpathlineto{\pgfqpoint{0.910223in}{0.522314in}}%
\pgfpathlineto{\pgfqpoint{0.908395in}{0.520818in}}%
\pgfpathlineto{\pgfqpoint{0.892738in}{0.508783in}}%
\pgfpathlineto{\pgfqpoint{0.892626in}{0.508703in}}%
\pgfpathlineto{\pgfqpoint{0.877082in}{0.497827in}}%
\pgfpathlineto{\pgfqpoint{0.872754in}{0.495092in}}%
\pgfpathlineto{\pgfqpoint{0.861425in}{0.487917in}}%
\pgfpathlineto{\pgfqpoint{0.849907in}{0.481481in}}%
\pgfpathlineto{\pgfqpoint{0.845769in}{0.479118in}}%
\pgfpathlineto{\pgfqpoint{0.830112in}{0.471394in}}%
\pgfpathlineto{\pgfqpoint{0.821515in}{0.467870in}}%
\pgfpathlineto{\pgfqpoint{0.814455in}{0.464842in}}%
\pgfpathlineto{\pgfqpoint{0.798799in}{0.459506in}}%
\pgfpathlineto{\pgfqpoint{0.783142in}{0.455597in}}%
\pgfpathlineto{\pgfqpoint{0.774731in}{0.454259in}}%
\pgfpathlineto{\pgfqpoint{0.767486in}{0.453017in}}%
\pgfpathlineto{\pgfqpoint{0.751829in}{0.451864in}}%
\pgfpathlineto{\pgfqpoint{0.736173in}{0.452248in}}%
\pgfpathlineto{\pgfqpoint{0.720516in}{0.454170in}}%
\pgfpathlineto{\pgfqpoint{0.720112in}{0.454259in}}%
\pgfpathclose%
\pgfpathmoveto{\pgfqpoint{1.503344in}{0.384694in}}%
\pgfpathlineto{\pgfqpoint{1.519001in}{0.381980in}}%
\pgfpathlineto{\pgfqpoint{1.534657in}{0.382885in}}%
\pgfpathlineto{\pgfqpoint{1.546152in}{0.386203in}}%
\pgfpathlineto{\pgfqpoint{1.550314in}{0.387078in}}%
\pgfpathlineto{\pgfqpoint{1.565971in}{0.392993in}}%
\pgfpathlineto{\pgfqpoint{1.578482in}{0.399814in}}%
\pgfpathlineto{\pgfqpoint{1.581627in}{0.401190in}}%
\pgfpathlineto{\pgfqpoint{1.597284in}{0.410086in}}%
\pgfpathlineto{\pgfqpoint{1.602088in}{0.413425in}}%
\pgfpathlineto{\pgfqpoint{1.612940in}{0.419863in}}%
\pgfpathlineto{\pgfqpoint{1.623208in}{0.427036in}}%
\pgfpathlineto{\pgfqpoint{1.628597in}{0.430388in}}%
\pgfpathlineto{\pgfqpoint{1.643025in}{0.440648in}}%
\pgfpathlineto{\pgfqpoint{1.644253in}{0.441451in}}%
\pgfpathlineto{\pgfqpoint{1.659910in}{0.452852in}}%
\pgfpathlineto{\pgfqpoint{1.661680in}{0.454259in}}%
\pgfpathlineto{\pgfqpoint{1.675567in}{0.464697in}}%
\pgfpathlineto{\pgfqpoint{1.679497in}{0.467870in}}%
\pgfpathlineto{\pgfqpoint{1.691223in}{0.477028in}}%
\pgfpathlineto{\pgfqpoint{1.696622in}{0.481481in}}%
\pgfpathlineto{\pgfqpoint{1.706880in}{0.489833in}}%
\pgfpathlineto{\pgfqpoint{1.713096in}{0.495092in}}%
\pgfpathlineto{\pgfqpoint{1.722536in}{0.503120in}}%
\pgfpathlineto{\pgfqpoint{1.728959in}{0.508703in}}%
\pgfpathlineto{\pgfqpoint{1.738193in}{0.516910in}}%
\pgfpathlineto{\pgfqpoint{1.744242in}{0.522314in}}%
\pgfpathlineto{\pgfqpoint{1.753849in}{0.531232in}}%
\pgfpathlineto{\pgfqpoint{1.758971in}{0.535925in}}%
\pgfpathlineto{\pgfqpoint{1.769506in}{0.546119in}}%
\pgfpathlineto{\pgfqpoint{1.773155in}{0.549536in}}%
\pgfpathlineto{\pgfqpoint{1.785162in}{0.561608in}}%
\pgfpathlineto{\pgfqpoint{1.786781in}{0.563148in}}%
\pgfpathlineto{\pgfqpoint{1.799895in}{0.576759in}}%
\pgfpathlineto{\pgfqpoint{1.800819in}{0.577827in}}%
\pgfpathlineto{\pgfqpoint{1.812621in}{0.590370in}}%
\pgfpathlineto{\pgfqpoint{1.816476in}{0.595055in}}%
\pgfpathlineto{\pgfqpoint{1.824727in}{0.603981in}}%
\pgfpathlineto{\pgfqpoint{1.832132in}{0.613415in}}%
\pgfpathlineto{\pgfqpoint{1.835973in}{0.617592in}}%
\pgfpathlineto{\pgfqpoint{1.846206in}{0.631203in}}%
\pgfpathlineto{\pgfqpoint{1.847789in}{0.633938in}}%
\pgfpathlineto{\pgfqpoint{1.855635in}{0.644814in}}%
\pgfpathlineto{\pgfqpoint{1.862438in}{0.658425in}}%
\pgfpathlineto{\pgfqpoint{1.863445in}{0.662043in}}%
\pgfpathlineto{\pgfqpoint{1.867262in}{0.672036in}}%
\pgfpathlineto{\pgfqpoint{1.868303in}{0.685648in}}%
\pgfpathlineto{\pgfqpoint{1.865181in}{0.699259in}}%
\pgfpathlineto{\pgfqpoint{1.863445in}{0.702513in}}%
\pgfpathlineto{\pgfqpoint{1.859415in}{0.712870in}}%
\pgfpathlineto{\pgfqpoint{1.851096in}{0.726481in}}%
\pgfpathlineto{\pgfqpoint{1.847789in}{0.730475in}}%
\pgfpathlineto{\pgfqpoint{1.841392in}{0.740092in}}%
\pgfpathlineto{\pgfqpoint{1.832132in}{0.751136in}}%
\pgfpathlineto{\pgfqpoint{1.830298in}{0.753703in}}%
\pgfpathlineto{\pgfqpoint{1.818650in}{0.767314in}}%
\pgfpathlineto{\pgfqpoint{1.816476in}{0.769514in}}%
\pgfpathlineto{\pgfqpoint{1.806425in}{0.780925in}}%
\pgfpathlineto{\pgfqpoint{1.800819in}{0.786567in}}%
\pgfpathlineto{\pgfqpoint{1.793529in}{0.794536in}}%
\pgfpathlineto{\pgfqpoint{1.785162in}{0.802820in}}%
\pgfpathlineto{\pgfqpoint{1.780076in}{0.808148in}}%
\pgfpathlineto{\pgfqpoint{1.769506in}{0.818363in}}%
\pgfpathlineto{\pgfqpoint{1.766110in}{0.821759in}}%
\pgfpathlineto{\pgfqpoint{1.753849in}{0.833265in}}%
\pgfpathlineto{\pgfqpoint{1.751637in}{0.835370in}}%
\pgfpathlineto{\pgfqpoint{1.738193in}{0.847578in}}%
\pgfpathlineto{\pgfqpoint{1.736641in}{0.848981in}}%
\pgfpathlineto{\pgfqpoint{1.722536in}{0.861351in}}%
\pgfpathlineto{\pgfqpoint{1.721090in}{0.862592in}}%
\pgfpathlineto{\pgfqpoint{1.706880in}{0.874625in}}%
\pgfpathlineto{\pgfqpoint{1.704943in}{0.876203in}}%
\pgfpathlineto{\pgfqpoint{1.691223in}{0.887432in}}%
\pgfpathlineto{\pgfqpoint{1.688148in}{0.889814in}}%
\pgfpathlineto{\pgfqpoint{1.675567in}{0.899793in}}%
\pgfpathlineto{\pgfqpoint{1.670646in}{0.903425in}}%
\pgfpathlineto{\pgfqpoint{1.659910in}{0.911714in}}%
\pgfpathlineto{\pgfqpoint{1.652370in}{0.917036in}}%
\pgfpathlineto{\pgfqpoint{1.644253in}{0.923175in}}%
\pgfpathlineto{\pgfqpoint{1.633230in}{0.930648in}}%
\pgfpathlineto{\pgfqpoint{1.628597in}{0.934112in}}%
\pgfpathlineto{\pgfqpoint{1.613099in}{0.944259in}}%
\pgfpathlineto{\pgfqpoint{1.612940in}{0.944378in}}%
\pgfpathlineto{\pgfqpoint{1.597284in}{0.954403in}}%
\pgfpathlineto{\pgfqpoint{1.590638in}{0.957870in}}%
\pgfpathlineto{\pgfqpoint{1.581627in}{0.963534in}}%
\pgfpathlineto{\pgfqpoint{1.565971in}{0.971093in}}%
\pgfpathlineto{\pgfqpoint{1.564815in}{0.971481in}}%
\pgfpathlineto{\pgfqpoint{1.550314in}{0.977814in}}%
\pgfpathlineto{\pgfqpoint{1.534657in}{0.981612in}}%
\pgfpathlineto{\pgfqpoint{1.519001in}{0.982371in}}%
\pgfpathlineto{\pgfqpoint{1.503344in}{0.980093in}}%
\pgfpathlineto{\pgfqpoint{1.487688in}{0.974774in}}%
\pgfpathlineto{\pgfqpoint{1.481493in}{0.971481in}}%
\pgfpathlineto{\pgfqpoint{1.472031in}{0.967605in}}%
\pgfpathlineto{\pgfqpoint{1.456375in}{0.958879in}}%
\pgfpathlineto{\pgfqpoint{1.454928in}{0.957870in}}%
\pgfpathlineto{\pgfqpoint{1.440718in}{0.949629in}}%
\pgfpathlineto{\pgfqpoint{1.432997in}{0.944259in}}%
\pgfpathlineto{\pgfqpoint{1.425061in}{0.939441in}}%
\pgfpathlineto{\pgfqpoint{1.412570in}{0.930648in}}%
\pgfpathlineto{\pgfqpoint{1.409405in}{0.928630in}}%
\pgfpathlineto{\pgfqpoint{1.393748in}{0.917365in}}%
\pgfpathlineto{\pgfqpoint{1.393332in}{0.917036in}}%
\pgfpathlineto{\pgfqpoint{1.378092in}{0.905777in}}%
\pgfpathlineto{\pgfqpoint{1.375152in}{0.903425in}}%
\pgfpathlineto{\pgfqpoint{1.362435in}{0.893686in}}%
\pgfpathlineto{\pgfqpoint{1.357687in}{0.889814in}}%
\pgfpathlineto{\pgfqpoint{1.346779in}{0.881119in}}%
\pgfpathlineto{\pgfqpoint{1.340892in}{0.876203in}}%
\pgfpathlineto{\pgfqpoint{1.331122in}{0.868075in}}%
\pgfpathlineto{\pgfqpoint{1.324728in}{0.862592in}}%
\pgfpathlineto{\pgfqpoint{1.315466in}{0.854539in}}%
\pgfpathlineto{\pgfqpoint{1.309159in}{0.848981in}}%
\pgfpathlineto{\pgfqpoint{1.299809in}{0.840487in}}%
\pgfpathlineto{\pgfqpoint{1.294155in}{0.835370in}}%
\pgfpathlineto{\pgfqpoint{1.284152in}{0.825887in}}%
\pgfpathlineto{\pgfqpoint{1.279698in}{0.821759in}}%
\pgfpathlineto{\pgfqpoint{1.268496in}{0.810703in}}%
\pgfpathlineto{\pgfqpoint{1.265790in}{0.808148in}}%
\pgfpathlineto{\pgfqpoint{1.252839in}{0.794898in}}%
\pgfpathlineto{\pgfqpoint{1.252461in}{0.794536in}}%
\pgfpathlineto{\pgfqpoint{1.239503in}{0.780925in}}%
\pgfpathlineto{\pgfqpoint{1.237183in}{0.778173in}}%
\pgfpathlineto{\pgfqpoint{1.227068in}{0.767314in}}%
\pgfpathlineto{\pgfqpoint{1.221526in}{0.760416in}}%
\pgfpathlineto{\pgfqpoint{1.215348in}{0.753703in}}%
\pgfpathlineto{\pgfqpoint{1.205870in}{0.741349in}}%
\pgfpathlineto{\pgfqpoint{1.204708in}{0.740092in}}%
\pgfpathlineto{\pgfqpoint{1.194671in}{0.726481in}}%
\pgfpathlineto{\pgfqpoint{1.190213in}{0.718255in}}%
\pgfpathlineto{\pgfqpoint{1.186425in}{0.712870in}}%
\pgfpathlineto{\pgfqpoint{1.180306in}{0.699259in}}%
\pgfpathlineto{\pgfqpoint{1.177686in}{0.685648in}}%
\pgfpathlineto{\pgfqpoint{1.178559in}{0.672036in}}%
\pgfpathlineto{\pgfqpoint{1.182928in}{0.658425in}}%
\pgfpathlineto{\pgfqpoint{1.190213in}{0.645818in}}%
\pgfpathlineto{\pgfqpoint{1.190659in}{0.644814in}}%
\pgfpathlineto{\pgfqpoint{1.199354in}{0.631203in}}%
\pgfpathlineto{\pgfqpoint{1.205870in}{0.623370in}}%
\pgfpathlineto{\pgfqpoint{1.209858in}{0.617592in}}%
\pgfpathlineto{\pgfqpoint{1.221389in}{0.603981in}}%
\pgfpathlineto{\pgfqpoint{1.221526in}{0.603843in}}%
\pgfpathlineto{\pgfqpoint{1.233197in}{0.590370in}}%
\pgfpathlineto{\pgfqpoint{1.237183in}{0.586342in}}%
\pgfpathlineto{\pgfqpoint{1.245778in}{0.576759in}}%
\pgfpathlineto{\pgfqpoint{1.252839in}{0.569703in}}%
\pgfpathlineto{\pgfqpoint{1.258962in}{0.563148in}}%
\pgfpathlineto{\pgfqpoint{1.268496in}{0.553814in}}%
\pgfpathlineto{\pgfqpoint{1.272674in}{0.549536in}}%
\pgfpathlineto{\pgfqpoint{1.284152in}{0.538599in}}%
\pgfpathlineto{\pgfqpoint{1.286893in}{0.535925in}}%
\pgfpathlineto{\pgfqpoint{1.299809in}{0.523998in}}%
\pgfpathlineto{\pgfqpoint{1.301624in}{0.522314in}}%
\pgfpathlineto{\pgfqpoint{1.315466in}{0.509960in}}%
\pgfpathlineto{\pgfqpoint{1.316893in}{0.508703in}}%
\pgfpathlineto{\pgfqpoint{1.331122in}{0.496441in}}%
\pgfpathlineto{\pgfqpoint{1.332736in}{0.495092in}}%
\pgfpathlineto{\pgfqpoint{1.346779in}{0.483405in}}%
\pgfpathlineto{\pgfqpoint{1.349200in}{0.481481in}}%
\pgfpathlineto{\pgfqpoint{1.362435in}{0.470822in}}%
\pgfpathlineto{\pgfqpoint{1.366341in}{0.467870in}}%
\pgfpathlineto{\pgfqpoint{1.378092in}{0.458681in}}%
\pgfpathlineto{\pgfqpoint{1.384220in}{0.454259in}}%
\pgfpathlineto{\pgfqpoint{1.393748in}{0.446985in}}%
\pgfpathlineto{\pgfqpoint{1.402916in}{0.440648in}}%
\pgfpathlineto{\pgfqpoint{1.409405in}{0.435774in}}%
\pgfpathlineto{\pgfqpoint{1.422531in}{0.427036in}}%
\pgfpathlineto{\pgfqpoint{1.425061in}{0.425146in}}%
\pgfpathlineto{\pgfqpoint{1.440718in}{0.415020in}}%
\pgfpathlineto{\pgfqpoint{1.443670in}{0.413425in}}%
\pgfpathlineto{\pgfqpoint{1.456375in}{0.405375in}}%
\pgfpathlineto{\pgfqpoint{1.467437in}{0.399814in}}%
\pgfpathlineto{\pgfqpoint{1.472031in}{0.396939in}}%
\pgfpathlineto{\pgfqpoint{1.487688in}{0.389707in}}%
\pgfpathlineto{\pgfqpoint{1.499601in}{0.386203in}}%
\pgfpathlineto{\pgfqpoint{1.503344in}{0.384694in}}%
\pgfpathclose%
\pgfpathmoveto{\pgfqpoint{1.496099in}{0.454259in}}%
\pgfpathlineto{\pgfqpoint{1.487688in}{0.455597in}}%
\pgfpathlineto{\pgfqpoint{1.472031in}{0.459506in}}%
\pgfpathlineto{\pgfqpoint{1.456375in}{0.464842in}}%
\pgfpathlineto{\pgfqpoint{1.449315in}{0.467870in}}%
\pgfpathlineto{\pgfqpoint{1.440718in}{0.471394in}}%
\pgfpathlineto{\pgfqpoint{1.425061in}{0.479118in}}%
\pgfpathlineto{\pgfqpoint{1.420923in}{0.481481in}}%
\pgfpathlineto{\pgfqpoint{1.409405in}{0.487917in}}%
\pgfpathlineto{\pgfqpoint{1.398076in}{0.495092in}}%
\pgfpathlineto{\pgfqpoint{1.393748in}{0.497827in}}%
\pgfpathlineto{\pgfqpoint{1.378204in}{0.508703in}}%
\pgfpathlineto{\pgfqpoint{1.378092in}{0.508783in}}%
\pgfpathlineto{\pgfqpoint{1.362435in}{0.520818in}}%
\pgfpathlineto{\pgfqpoint{1.360607in}{0.522314in}}%
\pgfpathlineto{\pgfqpoint{1.346779in}{0.534044in}}%
\pgfpathlineto{\pgfqpoint{1.344658in}{0.535925in}}%
\pgfpathlineto{\pgfqpoint{1.331122in}{0.548594in}}%
\pgfpathlineto{\pgfqpoint{1.330142in}{0.549536in}}%
\pgfpathlineto{\pgfqpoint{1.316935in}{0.563148in}}%
\pgfpathlineto{\pgfqpoint{1.315466in}{0.564803in}}%
\pgfpathlineto{\pgfqpoint{1.304931in}{0.576759in}}%
\pgfpathlineto{\pgfqpoint{1.299809in}{0.583268in}}%
\pgfpathlineto{\pgfqpoint{1.294155in}{0.590370in}}%
\pgfpathlineto{\pgfqpoint{1.284711in}{0.603981in}}%
\pgfpathlineto{\pgfqpoint{1.284152in}{0.604926in}}%
\pgfpathlineto{\pgfqpoint{1.276406in}{0.617592in}}%
\pgfpathlineto{\pgfqpoint{1.269664in}{0.631203in}}%
\pgfpathlineto{\pgfqpoint{1.268496in}{0.634268in}}%
\pgfpathlineto{\pgfqpoint{1.264230in}{0.644814in}}%
\pgfpathlineto{\pgfqpoint{1.260414in}{0.658425in}}%
\pgfpathlineto{\pgfqpoint{1.258296in}{0.672036in}}%
\pgfpathlineto{\pgfqpoint{1.257872in}{0.685648in}}%
\pgfpathlineto{\pgfqpoint{1.259143in}{0.699259in}}%
\pgfpathlineto{\pgfqpoint{1.262110in}{0.712870in}}%
\pgfpathlineto{\pgfqpoint{1.266776in}{0.726481in}}%
\pgfpathlineto{\pgfqpoint{1.268496in}{0.730186in}}%
\pgfpathlineto{\pgfqpoint{1.272836in}{0.740092in}}%
\pgfpathlineto{\pgfqpoint{1.280374in}{0.753703in}}%
\pgfpathlineto{\pgfqpoint{1.284152in}{0.759392in}}%
\pgfpathlineto{\pgfqpoint{1.289245in}{0.767314in}}%
\pgfpathlineto{\pgfqpoint{1.299439in}{0.780925in}}%
\pgfpathlineto{\pgfqpoint{1.299809in}{0.781365in}}%
\pgfpathlineto{\pgfqpoint{1.310768in}{0.794536in}}%
\pgfpathlineto{\pgfqpoint{1.315466in}{0.799635in}}%
\pgfpathlineto{\pgfqpoint{1.323372in}{0.808148in}}%
\pgfpathlineto{\pgfqpoint{1.331122in}{0.815840in}}%
\pgfpathlineto{\pgfqpoint{1.337243in}{0.821759in}}%
\pgfpathlineto{\pgfqpoint{1.346779in}{0.830422in}}%
\pgfpathlineto{\pgfqpoint{1.352470in}{0.835370in}}%
\pgfpathlineto{\pgfqpoint{1.362435in}{0.843659in}}%
\pgfpathlineto{\pgfqpoint{1.369243in}{0.848981in}}%
\pgfpathlineto{\pgfqpoint{1.378092in}{0.855718in}}%
\pgfpathlineto{\pgfqpoint{1.387883in}{0.862592in}}%
\pgfpathlineto{\pgfqpoint{1.393748in}{0.866675in}}%
\pgfpathlineto{\pgfqpoint{1.408899in}{0.876203in}}%
\pgfpathlineto{\pgfqpoint{1.409405in}{0.876524in}}%
\pgfpathlineto{\pgfqpoint{1.425061in}{0.885387in}}%
\pgfpathlineto{\pgfqpoint{1.434175in}{0.889814in}}%
\pgfpathlineto{\pgfqpoint{1.440718in}{0.893099in}}%
\pgfpathlineto{\pgfqpoint{1.456375in}{0.899652in}}%
\pgfpathlineto{\pgfqpoint{1.467769in}{0.903425in}}%
\pgfpathlineto{\pgfqpoint{1.472031in}{0.904921in}}%
\pgfpathlineto{\pgfqpoint{1.487688in}{0.908977in}}%
\pgfpathlineto{\pgfqpoint{1.503344in}{0.911556in}}%
\pgfpathlineto{\pgfqpoint{1.519001in}{0.912661in}}%
\pgfpathlineto{\pgfqpoint{1.534657in}{0.912293in}}%
\pgfpathlineto{\pgfqpoint{1.550314in}{0.910451in}}%
\pgfpathlineto{\pgfqpoint{1.565971in}{0.907134in}}%
\pgfpathlineto{\pgfqpoint{1.578102in}{0.903425in}}%
\pgfpathlineto{\pgfqpoint{1.581627in}{0.902410in}}%
\pgfpathlineto{\pgfqpoint{1.597284in}{0.896549in}}%
\pgfpathlineto{\pgfqpoint{1.611853in}{0.889814in}}%
\pgfpathlineto{\pgfqpoint{1.612940in}{0.889329in}}%
\pgfpathlineto{\pgfqpoint{1.628597in}{0.881119in}}%
\pgfpathlineto{\pgfqpoint{1.636766in}{0.876203in}}%
\pgfpathlineto{\pgfqpoint{1.644253in}{0.871750in}}%
\pgfpathlineto{\pgfqpoint{1.658005in}{0.862592in}}%
\pgfpathlineto{\pgfqpoint{1.659910in}{0.861314in}}%
\pgfpathlineto{\pgfqpoint{1.675567in}{0.849833in}}%
\pgfpathlineto{\pgfqpoint{1.676650in}{0.848981in}}%
\pgfpathlineto{\pgfqpoint{1.691223in}{0.837214in}}%
\pgfpathlineto{\pgfqpoint{1.693387in}{0.835370in}}%
\pgfpathlineto{\pgfqpoint{1.706880in}{0.823348in}}%
\pgfpathlineto{\pgfqpoint{1.708601in}{0.821759in}}%
\pgfpathlineto{\pgfqpoint{1.722445in}{0.808148in}}%
\pgfpathlineto{\pgfqpoint{1.722536in}{0.808050in}}%
\pgfpathlineto{\pgfqpoint{1.735046in}{0.794536in}}%
\pgfpathlineto{\pgfqpoint{1.738193in}{0.790774in}}%
\pgfpathlineto{\pgfqpoint{1.746446in}{0.780925in}}%
\pgfpathlineto{\pgfqpoint{1.753849in}{0.770912in}}%
\pgfpathlineto{\pgfqpoint{1.756567in}{0.767314in}}%
\pgfpathlineto{\pgfqpoint{1.765452in}{0.753703in}}%
\pgfpathlineto{\pgfqpoint{1.769506in}{0.746229in}}%
\pgfpathlineto{\pgfqpoint{1.772988in}{0.740092in}}%
\pgfpathlineto{\pgfqpoint{1.779126in}{0.726481in}}%
\pgfpathlineto{\pgfqpoint{1.783623in}{0.712870in}}%
\pgfpathlineto{\pgfqpoint{1.785162in}{0.705557in}}%
\pgfpathlineto{\pgfqpoint{1.786591in}{0.699259in}}%
\pgfpathlineto{\pgfqpoint{1.787917in}{0.685648in}}%
\pgfpathlineto{\pgfqpoint{1.787475in}{0.672036in}}%
\pgfpathlineto{\pgfqpoint{1.785265in}{0.658425in}}%
\pgfpathlineto{\pgfqpoint{1.785162in}{0.658074in}}%
\pgfpathlineto{\pgfqpoint{1.781579in}{0.644814in}}%
\pgfpathlineto{\pgfqpoint{1.776263in}{0.631203in}}%
\pgfpathlineto{\pgfqpoint{1.769506in}{0.617984in}}%
\pgfpathlineto{\pgfqpoint{1.769315in}{0.617592in}}%
\pgfpathlineto{\pgfqpoint{1.761203in}{0.603981in}}%
\pgfpathlineto{\pgfqpoint{1.753849in}{0.593573in}}%
\pgfpathlineto{\pgfqpoint{1.751637in}{0.590370in}}%
\pgfpathlineto{\pgfqpoint{1.740890in}{0.576759in}}%
\pgfpathlineto{\pgfqpoint{1.738193in}{0.573695in}}%
\pgfpathlineto{\pgfqpoint{1.728916in}{0.563148in}}%
\pgfpathlineto{\pgfqpoint{1.722536in}{0.556526in}}%
\pgfpathlineto{\pgfqpoint{1.715684in}{0.549536in}}%
\pgfpathlineto{\pgfqpoint{1.706880in}{0.541176in}}%
\pgfpathlineto{\pgfqpoint{1.701152in}{0.535925in}}%
\pgfpathlineto{\pgfqpoint{1.691223in}{0.527293in}}%
\pgfpathlineto{\pgfqpoint{1.685184in}{0.522314in}}%
\pgfpathlineto{\pgfqpoint{1.675567in}{0.514660in}}%
\pgfpathlineto{\pgfqpoint{1.667527in}{0.508703in}}%
\pgfpathlineto{\pgfqpoint{1.659910in}{0.503157in}}%
\pgfpathlineto{\pgfqpoint{1.647777in}{0.495092in}}%
\pgfpathlineto{\pgfqpoint{1.644253in}{0.492747in}}%
\pgfpathlineto{\pgfqpoint{1.628597in}{0.483404in}}%
\pgfpathlineto{\pgfqpoint{1.624912in}{0.481481in}}%
\pgfpathlineto{\pgfqpoint{1.612940in}{0.475088in}}%
\pgfpathlineto{\pgfqpoint{1.597284in}{0.468036in}}%
\pgfpathlineto{\pgfqpoint{1.596833in}{0.467870in}}%
\pgfpathlineto{\pgfqpoint{1.581627in}{0.461996in}}%
\pgfpathlineto{\pgfqpoint{1.565971in}{0.457374in}}%
\pgfpathlineto{\pgfqpoint{1.550718in}{0.454259in}}%
\pgfpathlineto{\pgfqpoint{1.550314in}{0.454170in}}%
\pgfpathlineto{\pgfqpoint{1.534657in}{0.452248in}}%
\pgfpathlineto{\pgfqpoint{1.519001in}{0.451864in}}%
\pgfpathlineto{\pgfqpoint{1.503344in}{0.453017in}}%
\pgfpathlineto{\pgfqpoint{1.496099in}{0.454259in}}%
\pgfpathclose%
\pgfpathmoveto{\pgfqpoint{0.720516in}{1.060425in}}%
\pgfpathlineto{\pgfqpoint{0.736173in}{1.056627in}}%
\pgfpathlineto{\pgfqpoint{0.751829in}{1.055868in}}%
\pgfpathlineto{\pgfqpoint{0.767486in}{1.058146in}}%
\pgfpathlineto{\pgfqpoint{0.783142in}{1.063465in}}%
\pgfpathlineto{\pgfqpoint{0.789337in}{1.066759in}}%
\pgfpathlineto{\pgfqpoint{0.798799in}{1.070634in}}%
\pgfpathlineto{\pgfqpoint{0.814455in}{1.079360in}}%
\pgfpathlineto{\pgfqpoint{0.815902in}{1.080370in}}%
\pgfpathlineto{\pgfqpoint{0.830112in}{1.088610in}}%
\pgfpathlineto{\pgfqpoint{0.837833in}{1.093981in}}%
\pgfpathlineto{\pgfqpoint{0.845769in}{1.098799in}}%
\pgfpathlineto{\pgfqpoint{0.858260in}{1.107592in}}%
\pgfpathlineto{\pgfqpoint{0.861425in}{1.109610in}}%
\pgfpathlineto{\pgfqpoint{0.877082in}{1.120874in}}%
\pgfpathlineto{\pgfqpoint{0.877498in}{1.121203in}}%
\pgfpathlineto{\pgfqpoint{0.892738in}{1.132462in}}%
\pgfpathlineto{\pgfqpoint{0.895678in}{1.134814in}}%
\pgfpathlineto{\pgfqpoint{0.908395in}{1.144553in}}%
\pgfpathlineto{\pgfqpoint{0.913143in}{1.148425in}}%
\pgfpathlineto{\pgfqpoint{0.924051in}{1.157121in}}%
\pgfpathlineto{\pgfqpoint{0.929938in}{1.162036in}}%
\pgfpathlineto{\pgfqpoint{0.939708in}{1.170165in}}%
\pgfpathlineto{\pgfqpoint{0.946102in}{1.175647in}}%
\pgfpathlineto{\pgfqpoint{0.955364in}{1.183700in}}%
\pgfpathlineto{\pgfqpoint{0.961671in}{1.189259in}}%
\pgfpathlineto{\pgfqpoint{0.971021in}{1.197753in}}%
\pgfpathlineto{\pgfqpoint{0.976675in}{1.202870in}}%
\pgfpathlineto{\pgfqpoint{0.986678in}{1.212353in}}%
\pgfpathlineto{\pgfqpoint{0.991132in}{1.216481in}}%
\pgfpathlineto{\pgfqpoint{1.002334in}{1.227536in}}%
\pgfpathlineto{\pgfqpoint{1.005040in}{1.230092in}}%
\pgfpathlineto{\pgfqpoint{1.017991in}{1.243341in}}%
\pgfpathlineto{\pgfqpoint{1.018369in}{1.243703in}}%
\pgfpathlineto{\pgfqpoint{1.031327in}{1.257314in}}%
\pgfpathlineto{\pgfqpoint{1.033647in}{1.260066in}}%
\pgfpathlineto{\pgfqpoint{1.043762in}{1.270925in}}%
\pgfpathlineto{\pgfqpoint{1.049304in}{1.277824in}}%
\pgfpathlineto{\pgfqpoint{1.055482in}{1.284536in}}%
\pgfpathlineto{\pgfqpoint{1.064960in}{1.296890in}}%
\pgfpathlineto{\pgfqpoint{1.066122in}{1.298148in}}%
\pgfpathlineto{\pgfqpoint{1.076159in}{1.311759in}}%
\pgfpathlineto{\pgfqpoint{1.080617in}{1.319985in}}%
\pgfpathlineto{\pgfqpoint{1.084405in}{1.325370in}}%
\pgfpathlineto{\pgfqpoint{1.090524in}{1.338981in}}%
\pgfpathlineto{\pgfqpoint{1.093144in}{1.352592in}}%
\pgfpathlineto{\pgfqpoint{1.092271in}{1.366203in}}%
\pgfpathlineto{\pgfqpoint{1.087902in}{1.379814in}}%
\pgfpathlineto{\pgfqpoint{1.080617in}{1.392421in}}%
\pgfpathlineto{\pgfqpoint{1.080171in}{1.393425in}}%
\pgfpathlineto{\pgfqpoint{1.071476in}{1.407036in}}%
\pgfpathlineto{\pgfqpoint{1.064960in}{1.414870in}}%
\pgfpathlineto{\pgfqpoint{1.060972in}{1.420648in}}%
\pgfpathlineto{\pgfqpoint{1.049441in}{1.434259in}}%
\pgfpathlineto{\pgfqpoint{1.049304in}{1.434397in}}%
\pgfpathlineto{\pgfqpoint{1.037633in}{1.447870in}}%
\pgfpathlineto{\pgfqpoint{1.033647in}{1.451898in}}%
\pgfpathlineto{\pgfqpoint{1.025052in}{1.461481in}}%
\pgfpathlineto{\pgfqpoint{1.017991in}{1.468537in}}%
\pgfpathlineto{\pgfqpoint{1.011868in}{1.475092in}}%
\pgfpathlineto{\pgfqpoint{1.002334in}{1.484426in}}%
\pgfpathlineto{\pgfqpoint{0.998156in}{1.488703in}}%
\pgfpathlineto{\pgfqpoint{0.986678in}{1.499641in}}%
\pgfpathlineto{\pgfqpoint{0.983937in}{1.502314in}}%
\pgfpathlineto{\pgfqpoint{0.971021in}{1.514242in}}%
\pgfpathlineto{\pgfqpoint{0.969206in}{1.515925in}}%
\pgfpathlineto{\pgfqpoint{0.955364in}{1.528279in}}%
\pgfpathlineto{\pgfqpoint{0.953937in}{1.529536in}}%
\pgfpathlineto{\pgfqpoint{0.939708in}{1.541798in}}%
\pgfpathlineto{\pgfqpoint{0.938094in}{1.543148in}}%
\pgfpathlineto{\pgfqpoint{0.924051in}{1.554835in}}%
\pgfpathlineto{\pgfqpoint{0.921630in}{1.556759in}}%
\pgfpathlineto{\pgfqpoint{0.908395in}{1.567417in}}%
\pgfpathlineto{\pgfqpoint{0.904489in}{1.570370in}}%
\pgfpathlineto{\pgfqpoint{0.892738in}{1.579559in}}%
\pgfpathlineto{\pgfqpoint{0.886610in}{1.583981in}}%
\pgfpathlineto{\pgfqpoint{0.877082in}{1.591255in}}%
\pgfpathlineto{\pgfqpoint{0.867914in}{1.597592in}}%
\pgfpathlineto{\pgfqpoint{0.861425in}{1.602466in}}%
\pgfpathlineto{\pgfqpoint{0.848299in}{1.611203in}}%
\pgfpathlineto{\pgfqpoint{0.845769in}{1.613093in}}%
\pgfpathlineto{\pgfqpoint{0.830112in}{1.623220in}}%
\pgfpathlineto{\pgfqpoint{0.827160in}{1.624814in}}%
\pgfpathlineto{\pgfqpoint{0.814455in}{1.632864in}}%
\pgfpathlineto{\pgfqpoint{0.803393in}{1.638425in}}%
\pgfpathlineto{\pgfqpoint{0.798799in}{1.641300in}}%
\pgfpathlineto{\pgfqpoint{0.783142in}{1.648533in}}%
\pgfpathlineto{\pgfqpoint{0.771229in}{1.652036in}}%
\pgfpathlineto{\pgfqpoint{0.767486in}{1.653545in}}%
\pgfpathlineto{\pgfqpoint{0.751829in}{1.656259in}}%
\pgfpathlineto{\pgfqpoint{0.736173in}{1.655355in}}%
\pgfpathlineto{\pgfqpoint{0.724678in}{1.652036in}}%
\pgfpathlineto{\pgfqpoint{0.720516in}{1.651161in}}%
\pgfpathlineto{\pgfqpoint{0.704859in}{1.645246in}}%
\pgfpathlineto{\pgfqpoint{0.692348in}{1.638425in}}%
\pgfpathlineto{\pgfqpoint{0.689203in}{1.637050in}}%
\pgfpathlineto{\pgfqpoint{0.673546in}{1.628154in}}%
\pgfpathlineto{\pgfqpoint{0.668742in}{1.624814in}}%
\pgfpathlineto{\pgfqpoint{0.657890in}{1.618377in}}%
\pgfpathlineto{\pgfqpoint{0.647622in}{1.611203in}}%
\pgfpathlineto{\pgfqpoint{0.642233in}{1.607852in}}%
\pgfpathlineto{\pgfqpoint{0.627805in}{1.597592in}}%
\pgfpathlineto{\pgfqpoint{0.626577in}{1.596789in}}%
\pgfpathlineto{\pgfqpoint{0.610920in}{1.585387in}}%
\pgfpathlineto{\pgfqpoint{0.609150in}{1.583981in}}%
\pgfpathlineto{\pgfqpoint{0.595263in}{1.573542in}}%
\pgfpathlineto{\pgfqpoint{0.591333in}{1.570370in}}%
\pgfpathlineto{\pgfqpoint{0.579607in}{1.561211in}}%
\pgfpathlineto{\pgfqpoint{0.574208in}{1.556759in}}%
\pgfpathlineto{\pgfqpoint{0.563950in}{1.548407in}}%
\pgfpathlineto{\pgfqpoint{0.557734in}{1.543148in}}%
\pgfpathlineto{\pgfqpoint{0.548294in}{1.535120in}}%
\pgfpathlineto{\pgfqpoint{0.541871in}{1.529536in}}%
\pgfpathlineto{\pgfqpoint{0.532637in}{1.521329in}}%
\pgfpathlineto{\pgfqpoint{0.526588in}{1.515925in}}%
\pgfpathlineto{\pgfqpoint{0.516981in}{1.507007in}}%
\pgfpathlineto{\pgfqpoint{0.511859in}{1.502314in}}%
\pgfpathlineto{\pgfqpoint{0.501324in}{1.492120in}}%
\pgfpathlineto{\pgfqpoint{0.497675in}{1.488703in}}%
\pgfpathlineto{\pgfqpoint{0.485668in}{1.476631in}}%
\pgfpathlineto{\pgfqpoint{0.484049in}{1.475092in}}%
\pgfpathlineto{\pgfqpoint{0.470935in}{1.461481in}}%
\pgfpathlineto{\pgfqpoint{0.470011in}{1.460413in}}%
\pgfpathlineto{\pgfqpoint{0.458209in}{1.447870in}}%
\pgfpathlineto{\pgfqpoint{0.454354in}{1.443185in}}%
\pgfpathlineto{\pgfqpoint{0.446103in}{1.434259in}}%
\pgfpathlineto{\pgfqpoint{0.438698in}{1.424824in}}%
\pgfpathlineto{\pgfqpoint{0.434857in}{1.420648in}}%
\pgfpathlineto{\pgfqpoint{0.424624in}{1.407036in}}%
\pgfpathlineto{\pgfqpoint{0.423041in}{1.404302in}}%
\pgfpathlineto{\pgfqpoint{0.415195in}{1.393425in}}%
\pgfpathlineto{\pgfqpoint{0.408392in}{1.379814in}}%
\pgfpathlineto{\pgfqpoint{0.407385in}{1.376196in}}%
\pgfpathlineto{\pgfqpoint{0.403568in}{1.366203in}}%
\pgfpathlineto{\pgfqpoint{0.402527in}{1.352592in}}%
\pgfpathlineto{\pgfqpoint{0.405649in}{1.338981in}}%
\pgfpathlineto{\pgfqpoint{0.407385in}{1.335727in}}%
\pgfpathlineto{\pgfqpoint{0.411415in}{1.325370in}}%
\pgfpathlineto{\pgfqpoint{0.419734in}{1.311759in}}%
\pgfpathlineto{\pgfqpoint{0.423041in}{1.307765in}}%
\pgfpathlineto{\pgfqpoint{0.429438in}{1.298148in}}%
\pgfpathlineto{\pgfqpoint{0.438698in}{1.287103in}}%
\pgfpathlineto{\pgfqpoint{0.440532in}{1.284536in}}%
\pgfpathlineto{\pgfqpoint{0.452180in}{1.270925in}}%
\pgfpathlineto{\pgfqpoint{0.454354in}{1.268725in}}%
\pgfpathlineto{\pgfqpoint{0.464405in}{1.257314in}}%
\pgfpathlineto{\pgfqpoint{0.470011in}{1.251673in}}%
\pgfpathlineto{\pgfqpoint{0.477301in}{1.243703in}}%
\pgfpathlineto{\pgfqpoint{0.485668in}{1.235420in}}%
\pgfpathlineto{\pgfqpoint{0.490754in}{1.230092in}}%
\pgfpathlineto{\pgfqpoint{0.501324in}{1.219876in}}%
\pgfpathlineto{\pgfqpoint{0.504720in}{1.216481in}}%
\pgfpathlineto{\pgfqpoint{0.516981in}{1.204975in}}%
\pgfpathlineto{\pgfqpoint{0.519193in}{1.202870in}}%
\pgfpathlineto{\pgfqpoint{0.532637in}{1.190661in}}%
\pgfpathlineto{\pgfqpoint{0.534189in}{1.189259in}}%
\pgfpathlineto{\pgfqpoint{0.548294in}{1.176888in}}%
\pgfpathlineto{\pgfqpoint{0.549740in}{1.175647in}}%
\pgfpathlineto{\pgfqpoint{0.563950in}{1.163615in}}%
\pgfpathlineto{\pgfqpoint{0.565887in}{1.162036in}}%
\pgfpathlineto{\pgfqpoint{0.579607in}{1.150808in}}%
\pgfpathlineto{\pgfqpoint{0.582682in}{1.148425in}}%
\pgfpathlineto{\pgfqpoint{0.595263in}{1.138447in}}%
\pgfpathlineto{\pgfqpoint{0.600184in}{1.134814in}}%
\pgfpathlineto{\pgfqpoint{0.610920in}{1.126525in}}%
\pgfpathlineto{\pgfqpoint{0.618460in}{1.121203in}}%
\pgfpathlineto{\pgfqpoint{0.626577in}{1.115064in}}%
\pgfpathlineto{\pgfqpoint{0.637600in}{1.107592in}}%
\pgfpathlineto{\pgfqpoint{0.642233in}{1.104127in}}%
\pgfpathlineto{\pgfqpoint{0.657731in}{1.093981in}}%
\pgfpathlineto{\pgfqpoint{0.657890in}{1.093862in}}%
\pgfpathlineto{\pgfqpoint{0.673546in}{1.083837in}}%
\pgfpathlineto{\pgfqpoint{0.680192in}{1.080370in}}%
\pgfpathlineto{\pgfqpoint{0.689203in}{1.074705in}}%
\pgfpathlineto{\pgfqpoint{0.704859in}{1.067147in}}%
\pgfpathlineto{\pgfqpoint{0.706015in}{1.066759in}}%
\pgfpathlineto{\pgfqpoint{0.720516in}{1.060425in}}%
\pgfpathclose%
\pgfpathmoveto{\pgfqpoint{0.692728in}{1.134814in}}%
\pgfpathlineto{\pgfqpoint{0.689203in}{1.135830in}}%
\pgfpathlineto{\pgfqpoint{0.673546in}{1.141691in}}%
\pgfpathlineto{\pgfqpoint{0.658977in}{1.148425in}}%
\pgfpathlineto{\pgfqpoint{0.657890in}{1.148911in}}%
\pgfpathlineto{\pgfqpoint{0.642233in}{1.157121in}}%
\pgfpathlineto{\pgfqpoint{0.634064in}{1.162036in}}%
\pgfpathlineto{\pgfqpoint{0.626577in}{1.166490in}}%
\pgfpathlineto{\pgfqpoint{0.612825in}{1.175647in}}%
\pgfpathlineto{\pgfqpoint{0.610920in}{1.176925in}}%
\pgfpathlineto{\pgfqpoint{0.595263in}{1.188407in}}%
\pgfpathlineto{\pgfqpoint{0.594180in}{1.189259in}}%
\pgfpathlineto{\pgfqpoint{0.579607in}{1.201026in}}%
\pgfpathlineto{\pgfqpoint{0.577443in}{1.202870in}}%
\pgfpathlineto{\pgfqpoint{0.563950in}{1.214891in}}%
\pgfpathlineto{\pgfqpoint{0.562229in}{1.216481in}}%
\pgfpathlineto{\pgfqpoint{0.548385in}{1.230092in}}%
\pgfpathlineto{\pgfqpoint{0.548294in}{1.230189in}}%
\pgfpathlineto{\pgfqpoint{0.535784in}{1.243703in}}%
\pgfpathlineto{\pgfqpoint{0.532637in}{1.247466in}}%
\pgfpathlineto{\pgfqpoint{0.524384in}{1.257314in}}%
\pgfpathlineto{\pgfqpoint{0.516981in}{1.267328in}}%
\pgfpathlineto{\pgfqpoint{0.514263in}{1.270925in}}%
\pgfpathlineto{\pgfqpoint{0.505378in}{1.284536in}}%
\pgfpathlineto{\pgfqpoint{0.501324in}{1.292010in}}%
\pgfpathlineto{\pgfqpoint{0.497842in}{1.298148in}}%
\pgfpathlineto{\pgfqpoint{0.491704in}{1.311759in}}%
\pgfpathlineto{\pgfqpoint{0.487207in}{1.325370in}}%
\pgfpathlineto{\pgfqpoint{0.485668in}{1.332683in}}%
\pgfpathlineto{\pgfqpoint{0.484239in}{1.338981in}}%
\pgfpathlineto{\pgfqpoint{0.482913in}{1.352592in}}%
\pgfpathlineto{\pgfqpoint{0.483355in}{1.366203in}}%
\pgfpathlineto{\pgfqpoint{0.485565in}{1.379814in}}%
\pgfpathlineto{\pgfqpoint{0.485668in}{1.380165in}}%
\pgfpathlineto{\pgfqpoint{0.489251in}{1.393425in}}%
\pgfpathlineto{\pgfqpoint{0.494567in}{1.407036in}}%
\pgfpathlineto{\pgfqpoint{0.501324in}{1.420256in}}%
\pgfpathlineto{\pgfqpoint{0.501515in}{1.420648in}}%
\pgfpathlineto{\pgfqpoint{0.509627in}{1.434259in}}%
\pgfpathlineto{\pgfqpoint{0.516981in}{1.444666in}}%
\pgfpathlineto{\pgfqpoint{0.519193in}{1.447870in}}%
\pgfpathlineto{\pgfqpoint{0.529940in}{1.461481in}}%
\pgfpathlineto{\pgfqpoint{0.532637in}{1.464544in}}%
\pgfpathlineto{\pgfqpoint{0.541914in}{1.475092in}}%
\pgfpathlineto{\pgfqpoint{0.548294in}{1.481714in}}%
\pgfpathlineto{\pgfqpoint{0.555146in}{1.488703in}}%
\pgfpathlineto{\pgfqpoint{0.563950in}{1.497064in}}%
\pgfpathlineto{\pgfqpoint{0.569678in}{1.502314in}}%
\pgfpathlineto{\pgfqpoint{0.579607in}{1.510946in}}%
\pgfpathlineto{\pgfqpoint{0.585646in}{1.515925in}}%
\pgfpathlineto{\pgfqpoint{0.595263in}{1.523579in}}%
\pgfpathlineto{\pgfqpoint{0.603303in}{1.529536in}}%
\pgfpathlineto{\pgfqpoint{0.610920in}{1.535083in}}%
\pgfpathlineto{\pgfqpoint{0.623053in}{1.543148in}}%
\pgfpathlineto{\pgfqpoint{0.626577in}{1.545492in}}%
\pgfpathlineto{\pgfqpoint{0.642233in}{1.554835in}}%
\pgfpathlineto{\pgfqpoint{0.645918in}{1.556759in}}%
\pgfpathlineto{\pgfqpoint{0.657890in}{1.563151in}}%
\pgfpathlineto{\pgfqpoint{0.673546in}{1.570204in}}%
\pgfpathlineto{\pgfqpoint{0.673997in}{1.570370in}}%
\pgfpathlineto{\pgfqpoint{0.689203in}{1.576244in}}%
\pgfpathlineto{\pgfqpoint{0.704859in}{1.580866in}}%
\pgfpathlineto{\pgfqpoint{0.720112in}{1.583981in}}%
\pgfpathlineto{\pgfqpoint{0.720516in}{1.584070in}}%
\pgfpathlineto{\pgfqpoint{0.736173in}{1.585991in}}%
\pgfpathlineto{\pgfqpoint{0.751829in}{1.586375in}}%
\pgfpathlineto{\pgfqpoint{0.767486in}{1.585223in}}%
\pgfpathlineto{\pgfqpoint{0.774731in}{1.583981in}}%
\pgfpathlineto{\pgfqpoint{0.783142in}{1.582642in}}%
\pgfpathlineto{\pgfqpoint{0.798799in}{1.578733in}}%
\pgfpathlineto{\pgfqpoint{0.814455in}{1.573397in}}%
\pgfpathlineto{\pgfqpoint{0.821515in}{1.570370in}}%
\pgfpathlineto{\pgfqpoint{0.830112in}{1.566846in}}%
\pgfpathlineto{\pgfqpoint{0.845769in}{1.559121in}}%
\pgfpathlineto{\pgfqpoint{0.849907in}{1.556759in}}%
\pgfpathlineto{\pgfqpoint{0.861425in}{1.550322in}}%
\pgfpathlineto{\pgfqpoint{0.872754in}{1.543148in}}%
\pgfpathlineto{\pgfqpoint{0.877082in}{1.540412in}}%
\pgfpathlineto{\pgfqpoint{0.892626in}{1.529536in}}%
\pgfpathlineto{\pgfqpoint{0.892738in}{1.529457in}}%
\pgfpathlineto{\pgfqpoint{0.908395in}{1.517421in}}%
\pgfpathlineto{\pgfqpoint{0.910223in}{1.515925in}}%
\pgfpathlineto{\pgfqpoint{0.924051in}{1.504196in}}%
\pgfpathlineto{\pgfqpoint{0.926172in}{1.502314in}}%
\pgfpathlineto{\pgfqpoint{0.939708in}{1.489645in}}%
\pgfpathlineto{\pgfqpoint{0.940688in}{1.488703in}}%
\pgfpathlineto{\pgfqpoint{0.953895in}{1.475092in}}%
\pgfpathlineto{\pgfqpoint{0.955364in}{1.473436in}}%
\pgfpathlineto{\pgfqpoint{0.965899in}{1.461481in}}%
\pgfpathlineto{\pgfqpoint{0.971021in}{1.454972in}}%
\pgfpathlineto{\pgfqpoint{0.976675in}{1.447870in}}%
\pgfpathlineto{\pgfqpoint{0.986119in}{1.434259in}}%
\pgfpathlineto{\pgfqpoint{0.986678in}{1.433314in}}%
\pgfpathlineto{\pgfqpoint{0.994424in}{1.420648in}}%
\pgfpathlineto{\pgfqpoint{1.001166in}{1.407036in}}%
\pgfpathlineto{\pgfqpoint{1.002334in}{1.403972in}}%
\pgfpathlineto{\pgfqpoint{1.006600in}{1.393425in}}%
\pgfpathlineto{\pgfqpoint{1.010416in}{1.379814in}}%
\pgfpathlineto{\pgfqpoint{1.012534in}{1.366203in}}%
\pgfpathlineto{\pgfqpoint{1.012958in}{1.352592in}}%
\pgfpathlineto{\pgfqpoint{1.011687in}{1.338981in}}%
\pgfpathlineto{\pgfqpoint{1.008720in}{1.325370in}}%
\pgfpathlineto{\pgfqpoint{1.004054in}{1.311759in}}%
\pgfpathlineto{\pgfqpoint{1.002334in}{1.308054in}}%
\pgfpathlineto{\pgfqpoint{0.997994in}{1.298148in}}%
\pgfpathlineto{\pgfqpoint{0.990456in}{1.284536in}}%
\pgfpathlineto{\pgfqpoint{0.986678in}{1.278848in}}%
\pgfpathlineto{\pgfqpoint{0.981585in}{1.270925in}}%
\pgfpathlineto{\pgfqpoint{0.971391in}{1.257314in}}%
\pgfpathlineto{\pgfqpoint{0.971021in}{1.256874in}}%
\pgfpathlineto{\pgfqpoint{0.960062in}{1.243703in}}%
\pgfpathlineto{\pgfqpoint{0.955364in}{1.238604in}}%
\pgfpathlineto{\pgfqpoint{0.947458in}{1.230092in}}%
\pgfpathlineto{\pgfqpoint{0.939708in}{1.222399in}}%
\pgfpathlineto{\pgfqpoint{0.933587in}{1.216481in}}%
\pgfpathlineto{\pgfqpoint{0.924051in}{1.207817in}}%
\pgfpathlineto{\pgfqpoint{0.918360in}{1.202870in}}%
\pgfpathlineto{\pgfqpoint{0.908395in}{1.194580in}}%
\pgfpathlineto{\pgfqpoint{0.901587in}{1.189259in}}%
\pgfpathlineto{\pgfqpoint{0.892738in}{1.182521in}}%
\pgfpathlineto{\pgfqpoint{0.882947in}{1.175647in}}%
\pgfpathlineto{\pgfqpoint{0.877082in}{1.171564in}}%
\pgfpathlineto{\pgfqpoint{0.861931in}{1.162036in}}%
\pgfpathlineto{\pgfqpoint{0.861425in}{1.161715in}}%
\pgfpathlineto{\pgfqpoint{0.845769in}{1.152852in}}%
\pgfpathlineto{\pgfqpoint{0.836655in}{1.148425in}}%
\pgfpathlineto{\pgfqpoint{0.830112in}{1.145140in}}%
\pgfpathlineto{\pgfqpoint{0.814455in}{1.138587in}}%
\pgfpathlineto{\pgfqpoint{0.803061in}{1.134814in}}%
\pgfpathlineto{\pgfqpoint{0.798799in}{1.133319in}}%
\pgfpathlineto{\pgfqpoint{0.783142in}{1.129263in}}%
\pgfpathlineto{\pgfqpoint{0.767486in}{1.126683in}}%
\pgfpathlineto{\pgfqpoint{0.751829in}{1.125579in}}%
\pgfpathlineto{\pgfqpoint{0.736173in}{1.125947in}}%
\pgfpathlineto{\pgfqpoint{0.720516in}{1.127789in}}%
\pgfpathlineto{\pgfqpoint{0.704859in}{1.131106in}}%
\pgfpathlineto{\pgfqpoint{0.692728in}{1.134814in}}%
\pgfpathclose%
\pgfpathmoveto{\pgfqpoint{1.487688in}{1.063465in}}%
\pgfpathlineto{\pgfqpoint{1.503344in}{1.058146in}}%
\pgfpathlineto{\pgfqpoint{1.519001in}{1.055868in}}%
\pgfpathlineto{\pgfqpoint{1.534657in}{1.056627in}}%
\pgfpathlineto{\pgfqpoint{1.550314in}{1.060425in}}%
\pgfpathlineto{\pgfqpoint{1.564815in}{1.066759in}}%
\pgfpathlineto{\pgfqpoint{1.565971in}{1.067147in}}%
\pgfpathlineto{\pgfqpoint{1.581627in}{1.074705in}}%
\pgfpathlineto{\pgfqpoint{1.590638in}{1.080370in}}%
\pgfpathlineto{\pgfqpoint{1.597284in}{1.083837in}}%
\pgfpathlineto{\pgfqpoint{1.612940in}{1.093862in}}%
\pgfpathlineto{\pgfqpoint{1.613099in}{1.093981in}}%
\pgfpathlineto{\pgfqpoint{1.628597in}{1.104127in}}%
\pgfpathlineto{\pgfqpoint{1.633230in}{1.107592in}}%
\pgfpathlineto{\pgfqpoint{1.644253in}{1.115064in}}%
\pgfpathlineto{\pgfqpoint{1.652370in}{1.121203in}}%
\pgfpathlineto{\pgfqpoint{1.659910in}{1.126525in}}%
\pgfpathlineto{\pgfqpoint{1.670646in}{1.134814in}}%
\pgfpathlineto{\pgfqpoint{1.675567in}{1.138447in}}%
\pgfpathlineto{\pgfqpoint{1.688148in}{1.148425in}}%
\pgfpathlineto{\pgfqpoint{1.691223in}{1.150808in}}%
\pgfpathlineto{\pgfqpoint{1.704943in}{1.162036in}}%
\pgfpathlineto{\pgfqpoint{1.706880in}{1.163615in}}%
\pgfpathlineto{\pgfqpoint{1.721090in}{1.175647in}}%
\pgfpathlineto{\pgfqpoint{1.722536in}{1.176888in}}%
\pgfpathlineto{\pgfqpoint{1.736641in}{1.189259in}}%
\pgfpathlineto{\pgfqpoint{1.738193in}{1.190661in}}%
\pgfpathlineto{\pgfqpoint{1.751637in}{1.202870in}}%
\pgfpathlineto{\pgfqpoint{1.753849in}{1.204975in}}%
\pgfpathlineto{\pgfqpoint{1.766110in}{1.216481in}}%
\pgfpathlineto{\pgfqpoint{1.769506in}{1.219876in}}%
\pgfpathlineto{\pgfqpoint{1.780076in}{1.230092in}}%
\pgfpathlineto{\pgfqpoint{1.785162in}{1.235420in}}%
\pgfpathlineto{\pgfqpoint{1.793529in}{1.243703in}}%
\pgfpathlineto{\pgfqpoint{1.800819in}{1.251673in}}%
\pgfpathlineto{\pgfqpoint{1.806425in}{1.257314in}}%
\pgfpathlineto{\pgfqpoint{1.816476in}{1.268725in}}%
\pgfpathlineto{\pgfqpoint{1.818650in}{1.270925in}}%
\pgfpathlineto{\pgfqpoint{1.830298in}{1.284536in}}%
\pgfpathlineto{\pgfqpoint{1.832132in}{1.287103in}}%
\pgfpathlineto{\pgfqpoint{1.841392in}{1.298148in}}%
\pgfpathlineto{\pgfqpoint{1.847789in}{1.307765in}}%
\pgfpathlineto{\pgfqpoint{1.851096in}{1.311759in}}%
\pgfpathlineto{\pgfqpoint{1.859415in}{1.325370in}}%
\pgfpathlineto{\pgfqpoint{1.863445in}{1.335727in}}%
\pgfpathlineto{\pgfqpoint{1.865181in}{1.338981in}}%
\pgfpathlineto{\pgfqpoint{1.868303in}{1.352592in}}%
\pgfpathlineto{\pgfqpoint{1.867262in}{1.366203in}}%
\pgfpathlineto{\pgfqpoint{1.863445in}{1.376196in}}%
\pgfpathlineto{\pgfqpoint{1.862438in}{1.379814in}}%
\pgfpathlineto{\pgfqpoint{1.855635in}{1.393425in}}%
\pgfpathlineto{\pgfqpoint{1.847789in}{1.404302in}}%
\pgfpathlineto{\pgfqpoint{1.846206in}{1.407036in}}%
\pgfpathlineto{\pgfqpoint{1.835973in}{1.420648in}}%
\pgfpathlineto{\pgfqpoint{1.832132in}{1.424824in}}%
\pgfpathlineto{\pgfqpoint{1.824727in}{1.434259in}}%
\pgfpathlineto{\pgfqpoint{1.816476in}{1.443185in}}%
\pgfpathlineto{\pgfqpoint{1.812621in}{1.447870in}}%
\pgfpathlineto{\pgfqpoint{1.800819in}{1.460413in}}%
\pgfpathlineto{\pgfqpoint{1.799895in}{1.461481in}}%
\pgfpathlineto{\pgfqpoint{1.786781in}{1.475092in}}%
\pgfpathlineto{\pgfqpoint{1.785162in}{1.476631in}}%
\pgfpathlineto{\pgfqpoint{1.773155in}{1.488703in}}%
\pgfpathlineto{\pgfqpoint{1.769506in}{1.492120in}}%
\pgfpathlineto{\pgfqpoint{1.758971in}{1.502314in}}%
\pgfpathlineto{\pgfqpoint{1.753849in}{1.507007in}}%
\pgfpathlineto{\pgfqpoint{1.744242in}{1.515925in}}%
\pgfpathlineto{\pgfqpoint{1.738193in}{1.521329in}}%
\pgfpathlineto{\pgfqpoint{1.728959in}{1.529536in}}%
\pgfpathlineto{\pgfqpoint{1.722536in}{1.535120in}}%
\pgfpathlineto{\pgfqpoint{1.713096in}{1.543148in}}%
\pgfpathlineto{\pgfqpoint{1.706880in}{1.548407in}}%
\pgfpathlineto{\pgfqpoint{1.696622in}{1.556759in}}%
\pgfpathlineto{\pgfqpoint{1.691223in}{1.561211in}}%
\pgfpathlineto{\pgfqpoint{1.679497in}{1.570370in}}%
\pgfpathlineto{\pgfqpoint{1.675567in}{1.573542in}}%
\pgfpathlineto{\pgfqpoint{1.661680in}{1.583981in}}%
\pgfpathlineto{\pgfqpoint{1.659910in}{1.585387in}}%
\pgfpathlineto{\pgfqpoint{1.644253in}{1.596789in}}%
\pgfpathlineto{\pgfqpoint{1.643025in}{1.597592in}}%
\pgfpathlineto{\pgfqpoint{1.628597in}{1.607852in}}%
\pgfpathlineto{\pgfqpoint{1.623208in}{1.611203in}}%
\pgfpathlineto{\pgfqpoint{1.612940in}{1.618377in}}%
\pgfpathlineto{\pgfqpoint{1.602088in}{1.624814in}}%
\pgfpathlineto{\pgfqpoint{1.597284in}{1.628154in}}%
\pgfpathlineto{\pgfqpoint{1.581627in}{1.637050in}}%
\pgfpathlineto{\pgfqpoint{1.578482in}{1.638425in}}%
\pgfpathlineto{\pgfqpoint{1.565971in}{1.645246in}}%
\pgfpathlineto{\pgfqpoint{1.550314in}{1.651161in}}%
\pgfpathlineto{\pgfqpoint{1.546152in}{1.652036in}}%
\pgfpathlineto{\pgfqpoint{1.534657in}{1.655355in}}%
\pgfpathlineto{\pgfqpoint{1.519001in}{1.656259in}}%
\pgfpathlineto{\pgfqpoint{1.503344in}{1.653545in}}%
\pgfpathlineto{\pgfqpoint{1.499601in}{1.652036in}}%
\pgfpathlineto{\pgfqpoint{1.487688in}{1.648533in}}%
\pgfpathlineto{\pgfqpoint{1.472031in}{1.641300in}}%
\pgfpathlineto{\pgfqpoint{1.467437in}{1.638425in}}%
\pgfpathlineto{\pgfqpoint{1.456375in}{1.632864in}}%
\pgfpathlineto{\pgfqpoint{1.443670in}{1.624814in}}%
\pgfpathlineto{\pgfqpoint{1.440718in}{1.623220in}}%
\pgfpathlineto{\pgfqpoint{1.425061in}{1.613093in}}%
\pgfpathlineto{\pgfqpoint{1.422531in}{1.611203in}}%
\pgfpathlineto{\pgfqpoint{1.409405in}{1.602466in}}%
\pgfpathlineto{\pgfqpoint{1.402916in}{1.597592in}}%
\pgfpathlineto{\pgfqpoint{1.393748in}{1.591255in}}%
\pgfpathlineto{\pgfqpoint{1.384220in}{1.583981in}}%
\pgfpathlineto{\pgfqpoint{1.378092in}{1.579559in}}%
\pgfpathlineto{\pgfqpoint{1.366341in}{1.570370in}}%
\pgfpathlineto{\pgfqpoint{1.362435in}{1.567417in}}%
\pgfpathlineto{\pgfqpoint{1.349200in}{1.556759in}}%
\pgfpathlineto{\pgfqpoint{1.346779in}{1.554835in}}%
\pgfpathlineto{\pgfqpoint{1.332736in}{1.543148in}}%
\pgfpathlineto{\pgfqpoint{1.331122in}{1.541798in}}%
\pgfpathlineto{\pgfqpoint{1.316893in}{1.529536in}}%
\pgfpathlineto{\pgfqpoint{1.315466in}{1.528279in}}%
\pgfpathlineto{\pgfqpoint{1.301624in}{1.515925in}}%
\pgfpathlineto{\pgfqpoint{1.299809in}{1.514242in}}%
\pgfpathlineto{\pgfqpoint{1.286893in}{1.502314in}}%
\pgfpathlineto{\pgfqpoint{1.284152in}{1.499641in}}%
\pgfpathlineto{\pgfqpoint{1.272674in}{1.488703in}}%
\pgfpathlineto{\pgfqpoint{1.268496in}{1.484426in}}%
\pgfpathlineto{\pgfqpoint{1.258962in}{1.475092in}}%
\pgfpathlineto{\pgfqpoint{1.252839in}{1.468537in}}%
\pgfpathlineto{\pgfqpoint{1.245778in}{1.461481in}}%
\pgfpathlineto{\pgfqpoint{1.237183in}{1.451898in}}%
\pgfpathlineto{\pgfqpoint{1.233197in}{1.447870in}}%
\pgfpathlineto{\pgfqpoint{1.221526in}{1.434397in}}%
\pgfpathlineto{\pgfqpoint{1.221389in}{1.434259in}}%
\pgfpathlineto{\pgfqpoint{1.209858in}{1.420648in}}%
\pgfpathlineto{\pgfqpoint{1.205870in}{1.414870in}}%
\pgfpathlineto{\pgfqpoint{1.199354in}{1.407036in}}%
\pgfpathlineto{\pgfqpoint{1.190659in}{1.393425in}}%
\pgfpathlineto{\pgfqpoint{1.190213in}{1.392421in}}%
\pgfpathlineto{\pgfqpoint{1.182928in}{1.379814in}}%
\pgfpathlineto{\pgfqpoint{1.178559in}{1.366203in}}%
\pgfpathlineto{\pgfqpoint{1.177686in}{1.352592in}}%
\pgfpathlineto{\pgfqpoint{1.180306in}{1.338981in}}%
\pgfpathlineto{\pgfqpoint{1.186425in}{1.325370in}}%
\pgfpathlineto{\pgfqpoint{1.190213in}{1.319985in}}%
\pgfpathlineto{\pgfqpoint{1.194671in}{1.311759in}}%
\pgfpathlineto{\pgfqpoint{1.204708in}{1.298148in}}%
\pgfpathlineto{\pgfqpoint{1.205870in}{1.296890in}}%
\pgfpathlineto{\pgfqpoint{1.215348in}{1.284536in}}%
\pgfpathlineto{\pgfqpoint{1.221526in}{1.277824in}}%
\pgfpathlineto{\pgfqpoint{1.227068in}{1.270925in}}%
\pgfpathlineto{\pgfqpoint{1.237183in}{1.260066in}}%
\pgfpathlineto{\pgfqpoint{1.239503in}{1.257314in}}%
\pgfpathlineto{\pgfqpoint{1.252461in}{1.243703in}}%
\pgfpathlineto{\pgfqpoint{1.252839in}{1.243341in}}%
\pgfpathlineto{\pgfqpoint{1.265790in}{1.230092in}}%
\pgfpathlineto{\pgfqpoint{1.268496in}{1.227536in}}%
\pgfpathlineto{\pgfqpoint{1.279698in}{1.216481in}}%
\pgfpathlineto{\pgfqpoint{1.284152in}{1.212353in}}%
\pgfpathlineto{\pgfqpoint{1.294155in}{1.202870in}}%
\pgfpathlineto{\pgfqpoint{1.299809in}{1.197753in}}%
\pgfpathlineto{\pgfqpoint{1.309159in}{1.189259in}}%
\pgfpathlineto{\pgfqpoint{1.315466in}{1.183700in}}%
\pgfpathlineto{\pgfqpoint{1.324728in}{1.175647in}}%
\pgfpathlineto{\pgfqpoint{1.331122in}{1.170165in}}%
\pgfpathlineto{\pgfqpoint{1.340892in}{1.162036in}}%
\pgfpathlineto{\pgfqpoint{1.346779in}{1.157121in}}%
\pgfpathlineto{\pgfqpoint{1.357687in}{1.148425in}}%
\pgfpathlineto{\pgfqpoint{1.362435in}{1.144553in}}%
\pgfpathlineto{\pgfqpoint{1.375152in}{1.134814in}}%
\pgfpathlineto{\pgfqpoint{1.378092in}{1.132462in}}%
\pgfpathlineto{\pgfqpoint{1.393332in}{1.121203in}}%
\pgfpathlineto{\pgfqpoint{1.393748in}{1.120874in}}%
\pgfpathlineto{\pgfqpoint{1.409405in}{1.109610in}}%
\pgfpathlineto{\pgfqpoint{1.412570in}{1.107592in}}%
\pgfpathlineto{\pgfqpoint{1.425061in}{1.098799in}}%
\pgfpathlineto{\pgfqpoint{1.432997in}{1.093981in}}%
\pgfpathlineto{\pgfqpoint{1.440718in}{1.088610in}}%
\pgfpathlineto{\pgfqpoint{1.454928in}{1.080370in}}%
\pgfpathlineto{\pgfqpoint{1.456375in}{1.079360in}}%
\pgfpathlineto{\pgfqpoint{1.472031in}{1.070634in}}%
\pgfpathlineto{\pgfqpoint{1.481493in}{1.066759in}}%
\pgfpathlineto{\pgfqpoint{1.487688in}{1.063465in}}%
\pgfpathclose%
\pgfpathmoveto{\pgfqpoint{1.467769in}{1.134814in}}%
\pgfpathlineto{\pgfqpoint{1.456375in}{1.138587in}}%
\pgfpathlineto{\pgfqpoint{1.440718in}{1.145140in}}%
\pgfpathlineto{\pgfqpoint{1.434175in}{1.148425in}}%
\pgfpathlineto{\pgfqpoint{1.425061in}{1.152852in}}%
\pgfpathlineto{\pgfqpoint{1.409405in}{1.161715in}}%
\pgfpathlineto{\pgfqpoint{1.408899in}{1.162036in}}%
\pgfpathlineto{\pgfqpoint{1.393748in}{1.171564in}}%
\pgfpathlineto{\pgfqpoint{1.387883in}{1.175647in}}%
\pgfpathlineto{\pgfqpoint{1.378092in}{1.182521in}}%
\pgfpathlineto{\pgfqpoint{1.369243in}{1.189259in}}%
\pgfpathlineto{\pgfqpoint{1.362435in}{1.194580in}}%
\pgfpathlineto{\pgfqpoint{1.352470in}{1.202870in}}%
\pgfpathlineto{\pgfqpoint{1.346779in}{1.207817in}}%
\pgfpathlineto{\pgfqpoint{1.337243in}{1.216481in}}%
\pgfpathlineto{\pgfqpoint{1.331122in}{1.222399in}}%
\pgfpathlineto{\pgfqpoint{1.323372in}{1.230092in}}%
\pgfpathlineto{\pgfqpoint{1.315466in}{1.238604in}}%
\pgfpathlineto{\pgfqpoint{1.310768in}{1.243703in}}%
\pgfpathlineto{\pgfqpoint{1.299809in}{1.256874in}}%
\pgfpathlineto{\pgfqpoint{1.299439in}{1.257314in}}%
\pgfpathlineto{\pgfqpoint{1.289245in}{1.270925in}}%
\pgfpathlineto{\pgfqpoint{1.284152in}{1.278848in}}%
\pgfpathlineto{\pgfqpoint{1.280374in}{1.284536in}}%
\pgfpathlineto{\pgfqpoint{1.272836in}{1.298148in}}%
\pgfpathlineto{\pgfqpoint{1.268496in}{1.308054in}}%
\pgfpathlineto{\pgfqpoint{1.266776in}{1.311759in}}%
\pgfpathlineto{\pgfqpoint{1.262110in}{1.325370in}}%
\pgfpathlineto{\pgfqpoint{1.259143in}{1.338981in}}%
\pgfpathlineto{\pgfqpoint{1.257872in}{1.352592in}}%
\pgfpathlineto{\pgfqpoint{1.258296in}{1.366203in}}%
\pgfpathlineto{\pgfqpoint{1.260414in}{1.379814in}}%
\pgfpathlineto{\pgfqpoint{1.264230in}{1.393425in}}%
\pgfpathlineto{\pgfqpoint{1.268496in}{1.403972in}}%
\pgfpathlineto{\pgfqpoint{1.269664in}{1.407036in}}%
\pgfpathlineto{\pgfqpoint{1.276406in}{1.420648in}}%
\pgfpathlineto{\pgfqpoint{1.284152in}{1.433314in}}%
\pgfpathlineto{\pgfqpoint{1.284711in}{1.434259in}}%
\pgfpathlineto{\pgfqpoint{1.294155in}{1.447870in}}%
\pgfpathlineto{\pgfqpoint{1.299809in}{1.454972in}}%
\pgfpathlineto{\pgfqpoint{1.304931in}{1.461481in}}%
\pgfpathlineto{\pgfqpoint{1.315466in}{1.473436in}}%
\pgfpathlineto{\pgfqpoint{1.316935in}{1.475092in}}%
\pgfpathlineto{\pgfqpoint{1.330142in}{1.488703in}}%
\pgfpathlineto{\pgfqpoint{1.331122in}{1.489645in}}%
\pgfpathlineto{\pgfqpoint{1.344658in}{1.502314in}}%
\pgfpathlineto{\pgfqpoint{1.346779in}{1.504196in}}%
\pgfpathlineto{\pgfqpoint{1.360607in}{1.515925in}}%
\pgfpathlineto{\pgfqpoint{1.362435in}{1.517421in}}%
\pgfpathlineto{\pgfqpoint{1.378092in}{1.529457in}}%
\pgfpathlineto{\pgfqpoint{1.378204in}{1.529536in}}%
\pgfpathlineto{\pgfqpoint{1.393748in}{1.540412in}}%
\pgfpathlineto{\pgfqpoint{1.398076in}{1.543148in}}%
\pgfpathlineto{\pgfqpoint{1.409405in}{1.550322in}}%
\pgfpathlineto{\pgfqpoint{1.420923in}{1.556759in}}%
\pgfpathlineto{\pgfqpoint{1.425061in}{1.559121in}}%
\pgfpathlineto{\pgfqpoint{1.440718in}{1.566846in}}%
\pgfpathlineto{\pgfqpoint{1.449315in}{1.570370in}}%
\pgfpathlineto{\pgfqpoint{1.456375in}{1.573397in}}%
\pgfpathlineto{\pgfqpoint{1.472031in}{1.578733in}}%
\pgfpathlineto{\pgfqpoint{1.487688in}{1.582642in}}%
\pgfpathlineto{\pgfqpoint{1.496099in}{1.583981in}}%
\pgfpathlineto{\pgfqpoint{1.503344in}{1.585223in}}%
\pgfpathlineto{\pgfqpoint{1.519001in}{1.586375in}}%
\pgfpathlineto{\pgfqpoint{1.534657in}{1.585991in}}%
\pgfpathlineto{\pgfqpoint{1.550314in}{1.584070in}}%
\pgfpathlineto{\pgfqpoint{1.550718in}{1.583981in}}%
\pgfpathlineto{\pgfqpoint{1.565971in}{1.580866in}}%
\pgfpathlineto{\pgfqpoint{1.581627in}{1.576244in}}%
\pgfpathlineto{\pgfqpoint{1.596833in}{1.570370in}}%
\pgfpathlineto{\pgfqpoint{1.597284in}{1.570204in}}%
\pgfpathlineto{\pgfqpoint{1.612940in}{1.563151in}}%
\pgfpathlineto{\pgfqpoint{1.624912in}{1.556759in}}%
\pgfpathlineto{\pgfqpoint{1.628597in}{1.554835in}}%
\pgfpathlineto{\pgfqpoint{1.644253in}{1.545492in}}%
\pgfpathlineto{\pgfqpoint{1.647777in}{1.543148in}}%
\pgfpathlineto{\pgfqpoint{1.659910in}{1.535083in}}%
\pgfpathlineto{\pgfqpoint{1.667527in}{1.529536in}}%
\pgfpathlineto{\pgfqpoint{1.675567in}{1.523579in}}%
\pgfpathlineto{\pgfqpoint{1.685184in}{1.515925in}}%
\pgfpathlineto{\pgfqpoint{1.691223in}{1.510946in}}%
\pgfpathlineto{\pgfqpoint{1.701152in}{1.502314in}}%
\pgfpathlineto{\pgfqpoint{1.706880in}{1.497064in}}%
\pgfpathlineto{\pgfqpoint{1.715684in}{1.488703in}}%
\pgfpathlineto{\pgfqpoint{1.722536in}{1.481714in}}%
\pgfpathlineto{\pgfqpoint{1.728916in}{1.475092in}}%
\pgfpathlineto{\pgfqpoint{1.738193in}{1.464544in}}%
\pgfpathlineto{\pgfqpoint{1.740890in}{1.461481in}}%
\pgfpathlineto{\pgfqpoint{1.751637in}{1.447870in}}%
\pgfpathlineto{\pgfqpoint{1.753849in}{1.444666in}}%
\pgfpathlineto{\pgfqpoint{1.761203in}{1.434259in}}%
\pgfpathlineto{\pgfqpoint{1.769315in}{1.420648in}}%
\pgfpathlineto{\pgfqpoint{1.769506in}{1.420256in}}%
\pgfpathlineto{\pgfqpoint{1.776263in}{1.407036in}}%
\pgfpathlineto{\pgfqpoint{1.781579in}{1.393425in}}%
\pgfpathlineto{\pgfqpoint{1.785162in}{1.380165in}}%
\pgfpathlineto{\pgfqpoint{1.785265in}{1.379814in}}%
\pgfpathlineto{\pgfqpoint{1.787475in}{1.366203in}}%
\pgfpathlineto{\pgfqpoint{1.787917in}{1.352592in}}%
\pgfpathlineto{\pgfqpoint{1.786591in}{1.338981in}}%
\pgfpathlineto{\pgfqpoint{1.785162in}{1.332683in}}%
\pgfpathlineto{\pgfqpoint{1.783623in}{1.325370in}}%
\pgfpathlineto{\pgfqpoint{1.779126in}{1.311759in}}%
\pgfpathlineto{\pgfqpoint{1.772988in}{1.298148in}}%
\pgfpathlineto{\pgfqpoint{1.769506in}{1.292010in}}%
\pgfpathlineto{\pgfqpoint{1.765452in}{1.284536in}}%
\pgfpathlineto{\pgfqpoint{1.756567in}{1.270925in}}%
\pgfpathlineto{\pgfqpoint{1.753849in}{1.267328in}}%
\pgfpathlineto{\pgfqpoint{1.746446in}{1.257314in}}%
\pgfpathlineto{\pgfqpoint{1.738193in}{1.247466in}}%
\pgfpathlineto{\pgfqpoint{1.735046in}{1.243703in}}%
\pgfpathlineto{\pgfqpoint{1.722536in}{1.230189in}}%
\pgfpathlineto{\pgfqpoint{1.722445in}{1.230092in}}%
\pgfpathlineto{\pgfqpoint{1.708601in}{1.216481in}}%
\pgfpathlineto{\pgfqpoint{1.706880in}{1.214891in}}%
\pgfpathlineto{\pgfqpoint{1.693387in}{1.202870in}}%
\pgfpathlineto{\pgfqpoint{1.691223in}{1.201026in}}%
\pgfpathlineto{\pgfqpoint{1.676650in}{1.189259in}}%
\pgfpathlineto{\pgfqpoint{1.675567in}{1.188407in}}%
\pgfpathlineto{\pgfqpoint{1.659910in}{1.176925in}}%
\pgfpathlineto{\pgfqpoint{1.658005in}{1.175647in}}%
\pgfpathlineto{\pgfqpoint{1.644253in}{1.166490in}}%
\pgfpathlineto{\pgfqpoint{1.636766in}{1.162036in}}%
\pgfpathlineto{\pgfqpoint{1.628597in}{1.157121in}}%
\pgfpathlineto{\pgfqpoint{1.612940in}{1.148911in}}%
\pgfpathlineto{\pgfqpoint{1.611853in}{1.148425in}}%
\pgfpathlineto{\pgfqpoint{1.597284in}{1.141691in}}%
\pgfpathlineto{\pgfqpoint{1.581627in}{1.135830in}}%
\pgfpathlineto{\pgfqpoint{1.578102in}{1.134814in}}%
\pgfpathlineto{\pgfqpoint{1.565971in}{1.131106in}}%
\pgfpathlineto{\pgfqpoint{1.550314in}{1.127789in}}%
\pgfpathlineto{\pgfqpoint{1.534657in}{1.125947in}}%
\pgfpathlineto{\pgfqpoint{1.519001in}{1.125579in}}%
\pgfpathlineto{\pgfqpoint{1.503344in}{1.126683in}}%
\pgfpathlineto{\pgfqpoint{1.487688in}{1.129263in}}%
\pgfpathlineto{\pgfqpoint{1.472031in}{1.133319in}}%
\pgfpathlineto{\pgfqpoint{1.467769in}{1.134814in}}%
\pgfpathclose%
\pgfusepath{fill}%
\end{pgfscope}%
\begin{pgfscope}%
\pgfpathrectangle{\pgfqpoint{0.360415in}{0.345370in}}{\pgfqpoint{1.550000in}{1.347500in}}%
\pgfusepath{clip}%
\pgfsetbuttcap%
\pgfsetroundjoin%
\definecolor{currentfill}{rgb}{0.679160,0.151848,0.575189}%
\pgfsetfillcolor{currentfill}%
\pgfsetlinewidth{0.000000pt}%
\definecolor{currentstroke}{rgb}{0.000000,0.000000,0.000000}%
\pgfsetstrokecolor{currentstroke}%
\pgfsetdash{}{0pt}%
\pgfpathmoveto{\pgfqpoint{0.626577in}{0.345370in}}%
\pgfpathlineto{\pgfqpoint{0.642233in}{0.345370in}}%
\pgfpathlineto{\pgfqpoint{0.657890in}{0.345370in}}%
\pgfpathlineto{\pgfqpoint{0.673546in}{0.345370in}}%
\pgfpathlineto{\pgfqpoint{0.689203in}{0.345370in}}%
\pgfpathlineto{\pgfqpoint{0.704859in}{0.345370in}}%
\pgfpathlineto{\pgfqpoint{0.720516in}{0.345370in}}%
\pgfpathlineto{\pgfqpoint{0.736173in}{0.345370in}}%
\pgfpathlineto{\pgfqpoint{0.751829in}{0.345370in}}%
\pgfpathlineto{\pgfqpoint{0.767486in}{0.345370in}}%
\pgfpathlineto{\pgfqpoint{0.783142in}{0.345370in}}%
\pgfpathlineto{\pgfqpoint{0.798799in}{0.345370in}}%
\pgfpathlineto{\pgfqpoint{0.814455in}{0.345370in}}%
\pgfpathlineto{\pgfqpoint{0.830112in}{0.345370in}}%
\pgfpathlineto{\pgfqpoint{0.845769in}{0.345370in}}%
\pgfpathlineto{\pgfqpoint{0.861425in}{0.345370in}}%
\pgfpathlineto{\pgfqpoint{0.877082in}{0.345370in}}%
\pgfpathlineto{\pgfqpoint{0.880525in}{0.345370in}}%
\pgfpathlineto{\pgfqpoint{0.882085in}{0.358981in}}%
\pgfpathlineto{\pgfqpoint{0.886700in}{0.372592in}}%
\pgfpathlineto{\pgfqpoint{0.892738in}{0.383562in}}%
\pgfpathlineto{\pgfqpoint{0.894102in}{0.386203in}}%
\pgfpathlineto{\pgfqpoint{0.903620in}{0.399814in}}%
\pgfpathlineto{\pgfqpoint{0.908395in}{0.405366in}}%
\pgfpathlineto{\pgfqpoint{0.914967in}{0.413425in}}%
\pgfpathlineto{\pgfqpoint{0.924051in}{0.423042in}}%
\pgfpathlineto{\pgfqpoint{0.927678in}{0.427036in}}%
\pgfpathlineto{\pgfqpoint{0.939708in}{0.438948in}}%
\pgfpathlineto{\pgfqpoint{0.941380in}{0.440648in}}%
\pgfpathlineto{\pgfqpoint{0.955364in}{0.453825in}}%
\pgfpathlineto{\pgfqpoint{0.955820in}{0.454259in}}%
\pgfpathlineto{\pgfqpoint{0.970844in}{0.467870in}}%
\pgfpathlineto{\pgfqpoint{0.971021in}{0.468025in}}%
\pgfpathlineto{\pgfqpoint{0.986378in}{0.481481in}}%
\pgfpathlineto{\pgfqpoint{0.986678in}{0.481740in}}%
\pgfpathlineto{\pgfqpoint{1.002334in}{0.495031in}}%
\pgfpathlineto{\pgfqpoint{1.002409in}{0.495092in}}%
\pgfpathlineto{\pgfqpoint{1.017991in}{0.507848in}}%
\pgfpathlineto{\pgfqpoint{1.019102in}{0.508703in}}%
\pgfpathlineto{\pgfqpoint{1.033647in}{0.520102in}}%
\pgfpathlineto{\pgfqpoint{1.036735in}{0.522314in}}%
\pgfpathlineto{\pgfqpoint{1.049304in}{0.531617in}}%
\pgfpathlineto{\pgfqpoint{1.055900in}{0.535925in}}%
\pgfpathlineto{\pgfqpoint{1.064960in}{0.542123in}}%
\pgfpathlineto{\pgfqpoint{1.077845in}{0.549536in}}%
\pgfpathlineto{\pgfqpoint{1.080617in}{0.551230in}}%
\pgfpathlineto{\pgfqpoint{1.096274in}{0.558662in}}%
\pgfpathlineto{\pgfqpoint{1.109968in}{0.563148in}}%
\pgfpathlineto{\pgfqpoint{1.111930in}{0.563838in}}%
\pgfpathlineto{\pgfqpoint{1.127587in}{0.566629in}}%
\pgfpathlineto{\pgfqpoint{1.143243in}{0.566629in}}%
\pgfpathlineto{\pgfqpoint{1.158900in}{0.563838in}}%
\pgfpathlineto{\pgfqpoint{1.160862in}{0.563148in}}%
\pgfpathlineto{\pgfqpoint{1.174556in}{0.558662in}}%
\pgfpathlineto{\pgfqpoint{1.190213in}{0.551230in}}%
\pgfpathlineto{\pgfqpoint{1.192985in}{0.549536in}}%
\pgfpathlineto{\pgfqpoint{1.205870in}{0.542123in}}%
\pgfpathlineto{\pgfqpoint{1.214930in}{0.535925in}}%
\pgfpathlineto{\pgfqpoint{1.221526in}{0.531617in}}%
\pgfpathlineto{\pgfqpoint{1.234095in}{0.522314in}}%
\pgfpathlineto{\pgfqpoint{1.237183in}{0.520102in}}%
\pgfpathlineto{\pgfqpoint{1.251728in}{0.508703in}}%
\pgfpathlineto{\pgfqpoint{1.252839in}{0.507848in}}%
\pgfpathlineto{\pgfqpoint{1.268421in}{0.495092in}}%
\pgfpathlineto{\pgfqpoint{1.268496in}{0.495031in}}%
\pgfpathlineto{\pgfqpoint{1.284152in}{0.481740in}}%
\pgfpathlineto{\pgfqpoint{1.284452in}{0.481481in}}%
\pgfpathlineto{\pgfqpoint{1.299809in}{0.468025in}}%
\pgfpathlineto{\pgfqpoint{1.299986in}{0.467870in}}%
\pgfpathlineto{\pgfqpoint{1.315010in}{0.454259in}}%
\pgfpathlineto{\pgfqpoint{1.315466in}{0.453825in}}%
\pgfpathlineto{\pgfqpoint{1.329450in}{0.440648in}}%
\pgfpathlineto{\pgfqpoint{1.331122in}{0.438948in}}%
\pgfpathlineto{\pgfqpoint{1.343152in}{0.427036in}}%
\pgfpathlineto{\pgfqpoint{1.346779in}{0.423042in}}%
\pgfpathlineto{\pgfqpoint{1.355863in}{0.413425in}}%
\pgfpathlineto{\pgfqpoint{1.362435in}{0.405366in}}%
\pgfpathlineto{\pgfqpoint{1.367210in}{0.399814in}}%
\pgfpathlineto{\pgfqpoint{1.376728in}{0.386203in}}%
\pgfpathlineto{\pgfqpoint{1.378092in}{0.383562in}}%
\pgfpathlineto{\pgfqpoint{1.384130in}{0.372592in}}%
\pgfpathlineto{\pgfqpoint{1.388745in}{0.358981in}}%
\pgfpathlineto{\pgfqpoint{1.390305in}{0.345370in}}%
\pgfpathlineto{\pgfqpoint{1.393748in}{0.345370in}}%
\pgfpathlineto{\pgfqpoint{1.409405in}{0.345370in}}%
\pgfpathlineto{\pgfqpoint{1.425061in}{0.345370in}}%
\pgfpathlineto{\pgfqpoint{1.440718in}{0.345370in}}%
\pgfpathlineto{\pgfqpoint{1.456375in}{0.345370in}}%
\pgfpathlineto{\pgfqpoint{1.472031in}{0.345370in}}%
\pgfpathlineto{\pgfqpoint{1.487688in}{0.345370in}}%
\pgfpathlineto{\pgfqpoint{1.503344in}{0.345370in}}%
\pgfpathlineto{\pgfqpoint{1.519001in}{0.345370in}}%
\pgfpathlineto{\pgfqpoint{1.534657in}{0.345370in}}%
\pgfpathlineto{\pgfqpoint{1.550314in}{0.345370in}}%
\pgfpathlineto{\pgfqpoint{1.565971in}{0.345370in}}%
\pgfpathlineto{\pgfqpoint{1.581627in}{0.345370in}}%
\pgfpathlineto{\pgfqpoint{1.597284in}{0.345370in}}%
\pgfpathlineto{\pgfqpoint{1.612940in}{0.345370in}}%
\pgfpathlineto{\pgfqpoint{1.628597in}{0.345370in}}%
\pgfpathlineto{\pgfqpoint{1.644253in}{0.345370in}}%
\pgfpathlineto{\pgfqpoint{1.655500in}{0.345370in}}%
\pgfpathlineto{\pgfqpoint{1.657115in}{0.358981in}}%
\pgfpathlineto{\pgfqpoint{1.659910in}{0.366929in}}%
\pgfpathlineto{\pgfqpoint{1.661768in}{0.372592in}}%
\pgfpathlineto{\pgfqpoint{1.669032in}{0.386203in}}%
\pgfpathlineto{\pgfqpoint{1.675567in}{0.395281in}}%
\pgfpathlineto{\pgfqpoint{1.678644in}{0.399814in}}%
\pgfpathlineto{\pgfqpoint{1.690000in}{0.413425in}}%
\pgfpathlineto{\pgfqpoint{1.691223in}{0.414684in}}%
\pgfpathlineto{\pgfqpoint{1.702669in}{0.427036in}}%
\pgfpathlineto{\pgfqpoint{1.706880in}{0.431121in}}%
\pgfpathlineto{\pgfqpoint{1.716376in}{0.440648in}}%
\pgfpathlineto{\pgfqpoint{1.722536in}{0.446378in}}%
\pgfpathlineto{\pgfqpoint{1.730844in}{0.454259in}}%
\pgfpathlineto{\pgfqpoint{1.738193in}{0.460885in}}%
\pgfpathlineto{\pgfqpoint{1.745903in}{0.467870in}}%
\pgfpathlineto{\pgfqpoint{1.753849in}{0.474858in}}%
\pgfpathlineto{\pgfqpoint{1.761468in}{0.481481in}}%
\pgfpathlineto{\pgfqpoint{1.769506in}{0.488389in}}%
\pgfpathlineto{\pgfqpoint{1.777541in}{0.495092in}}%
\pgfpathlineto{\pgfqpoint{1.785162in}{0.501480in}}%
\pgfpathlineto{\pgfqpoint{1.794228in}{0.508703in}}%
\pgfpathlineto{\pgfqpoint{1.800819in}{0.514059in}}%
\pgfpathlineto{\pgfqpoint{1.811778in}{0.522314in}}%
\pgfpathlineto{\pgfqpoint{1.816476in}{0.525975in}}%
\pgfpathlineto{\pgfqpoint{1.830684in}{0.535925in}}%
\pgfpathlineto{\pgfqpoint{1.832132in}{0.536989in}}%
\pgfpathlineto{\pgfqpoint{1.847789in}{0.546861in}}%
\pgfpathlineto{\pgfqpoint{1.853004in}{0.549536in}}%
\pgfpathlineto{\pgfqpoint{1.863445in}{0.555217in}}%
\pgfpathlineto{\pgfqpoint{1.879102in}{0.561532in}}%
\pgfpathlineto{\pgfqpoint{1.885616in}{0.563148in}}%
\pgfpathlineto{\pgfqpoint{1.894758in}{0.565578in}}%
\pgfpathlineto{\pgfqpoint{1.910415in}{0.566981in}}%
\pgfpathlineto{\pgfqpoint{1.910415in}{0.576759in}}%
\pgfpathlineto{\pgfqpoint{1.910415in}{0.590370in}}%
\pgfpathlineto{\pgfqpoint{1.910415in}{0.603981in}}%
\pgfpathlineto{\pgfqpoint{1.910415in}{0.617592in}}%
\pgfpathlineto{\pgfqpoint{1.910415in}{0.631203in}}%
\pgfpathlineto{\pgfqpoint{1.910415in}{0.644814in}}%
\pgfpathlineto{\pgfqpoint{1.910415in}{0.658425in}}%
\pgfpathlineto{\pgfqpoint{1.910415in}{0.672036in}}%
\pgfpathlineto{\pgfqpoint{1.910415in}{0.685648in}}%
\pgfpathlineto{\pgfqpoint{1.910415in}{0.699259in}}%
\pgfpathlineto{\pgfqpoint{1.910415in}{0.712870in}}%
\pgfpathlineto{\pgfqpoint{1.910415in}{0.726481in}}%
\pgfpathlineto{\pgfqpoint{1.910415in}{0.740092in}}%
\pgfpathlineto{\pgfqpoint{1.910415in}{0.753703in}}%
\pgfpathlineto{\pgfqpoint{1.910415in}{0.767314in}}%
\pgfpathlineto{\pgfqpoint{1.910415in}{0.780925in}}%
\pgfpathlineto{\pgfqpoint{1.910415in}{0.794536in}}%
\pgfpathlineto{\pgfqpoint{1.910415in}{0.797530in}}%
\pgfpathlineto{\pgfqpoint{1.894758in}{0.798886in}}%
\pgfpathlineto{\pgfqpoint{1.879102in}{0.802899in}}%
\pgfpathlineto{\pgfqpoint{1.866483in}{0.808148in}}%
\pgfpathlineto{\pgfqpoint{1.863445in}{0.809333in}}%
\pgfpathlineto{\pgfqpoint{1.847789in}{0.817608in}}%
\pgfpathlineto{\pgfqpoint{1.841403in}{0.821759in}}%
\pgfpathlineto{\pgfqpoint{1.832132in}{0.827473in}}%
\pgfpathlineto{\pgfqpoint{1.821070in}{0.835370in}}%
\pgfpathlineto{\pgfqpoint{1.816476in}{0.838523in}}%
\pgfpathlineto{\pgfqpoint{1.802774in}{0.848981in}}%
\pgfpathlineto{\pgfqpoint{1.800819in}{0.850435in}}%
\pgfpathlineto{\pgfqpoint{1.785662in}{0.862592in}}%
\pgfpathlineto{\pgfqpoint{1.785162in}{0.862988in}}%
\pgfpathlineto{\pgfqpoint{1.769506in}{0.876049in}}%
\pgfpathlineto{\pgfqpoint{1.769327in}{0.876203in}}%
\pgfpathlineto{\pgfqpoint{1.753849in}{0.889554in}}%
\pgfpathlineto{\pgfqpoint{1.753551in}{0.889814in}}%
\pgfpathlineto{\pgfqpoint{1.738263in}{0.903425in}}%
\pgfpathlineto{\pgfqpoint{1.738193in}{0.903490in}}%
\pgfpathlineto{\pgfqpoint{1.723520in}{0.917036in}}%
\pgfpathlineto{\pgfqpoint{1.722536in}{0.918003in}}%
\pgfpathlineto{\pgfqpoint{1.709424in}{0.930648in}}%
\pgfpathlineto{\pgfqpoint{1.706880in}{0.933332in}}%
\pgfpathlineto{\pgfqpoint{1.696179in}{0.944259in}}%
\pgfpathlineto{\pgfqpoint{1.691223in}{0.949993in}}%
\pgfpathlineto{\pgfqpoint{1.684094in}{0.957870in}}%
\pgfpathlineto{\pgfqpoint{1.675567in}{0.969071in}}%
\pgfpathlineto{\pgfqpoint{1.673618in}{0.971481in}}%
\pgfpathlineto{\pgfqpoint{1.665069in}{0.985092in}}%
\pgfpathlineto{\pgfqpoint{1.659910in}{0.996997in}}%
\pgfpathlineto{\pgfqpoint{1.659116in}{0.998703in}}%
\pgfpathlineto{\pgfqpoint{1.655905in}{1.012314in}}%
\pgfpathlineto{\pgfqpoint{1.655905in}{1.025925in}}%
\pgfpathlineto{\pgfqpoint{1.659116in}{1.039536in}}%
\pgfpathlineto{\pgfqpoint{1.659910in}{1.041242in}}%
\pgfpathlineto{\pgfqpoint{1.665069in}{1.053148in}}%
\pgfpathlineto{\pgfqpoint{1.673618in}{1.066759in}}%
\pgfpathlineto{\pgfqpoint{1.675567in}{1.069168in}}%
\pgfpathlineto{\pgfqpoint{1.684094in}{1.080370in}}%
\pgfpathlineto{\pgfqpoint{1.691223in}{1.088246in}}%
\pgfpathlineto{\pgfqpoint{1.696179in}{1.093981in}}%
\pgfpathlineto{\pgfqpoint{1.706880in}{1.104908in}}%
\pgfpathlineto{\pgfqpoint{1.709424in}{1.107592in}}%
\pgfpathlineto{\pgfqpoint{1.722536in}{1.120237in}}%
\pgfpathlineto{\pgfqpoint{1.723520in}{1.121203in}}%
\pgfpathlineto{\pgfqpoint{1.738193in}{1.134749in}}%
\pgfpathlineto{\pgfqpoint{1.738263in}{1.134814in}}%
\pgfpathlineto{\pgfqpoint{1.753551in}{1.148425in}}%
\pgfpathlineto{\pgfqpoint{1.753849in}{1.148686in}}%
\pgfpathlineto{\pgfqpoint{1.769327in}{1.162036in}}%
\pgfpathlineto{\pgfqpoint{1.769506in}{1.162190in}}%
\pgfpathlineto{\pgfqpoint{1.785162in}{1.175252in}}%
\pgfpathlineto{\pgfqpoint{1.785662in}{1.175647in}}%
\pgfpathlineto{\pgfqpoint{1.800819in}{1.187805in}}%
\pgfpathlineto{\pgfqpoint{1.802774in}{1.189259in}}%
\pgfpathlineto{\pgfqpoint{1.816476in}{1.199716in}}%
\pgfpathlineto{\pgfqpoint{1.821070in}{1.202870in}}%
\pgfpathlineto{\pgfqpoint{1.832132in}{1.210767in}}%
\pgfpathlineto{\pgfqpoint{1.841403in}{1.216481in}}%
\pgfpathlineto{\pgfqpoint{1.847789in}{1.220632in}}%
\pgfpathlineto{\pgfqpoint{1.863445in}{1.228906in}}%
\pgfpathlineto{\pgfqpoint{1.866483in}{1.230092in}}%
\pgfpathlineto{\pgfqpoint{1.879102in}{1.235341in}}%
\pgfpathlineto{\pgfqpoint{1.894758in}{1.239353in}}%
\pgfpathlineto{\pgfqpoint{1.910415in}{1.240710in}}%
\pgfpathlineto{\pgfqpoint{1.910415in}{1.243703in}}%
\pgfpathlineto{\pgfqpoint{1.910415in}{1.257314in}}%
\pgfpathlineto{\pgfqpoint{1.910415in}{1.270925in}}%
\pgfpathlineto{\pgfqpoint{1.910415in}{1.284536in}}%
\pgfpathlineto{\pgfqpoint{1.910415in}{1.298148in}}%
\pgfpathlineto{\pgfqpoint{1.910415in}{1.311759in}}%
\pgfpathlineto{\pgfqpoint{1.910415in}{1.325370in}}%
\pgfpathlineto{\pgfqpoint{1.910415in}{1.338981in}}%
\pgfpathlineto{\pgfqpoint{1.910415in}{1.352592in}}%
\pgfpathlineto{\pgfqpoint{1.910415in}{1.366203in}}%
\pgfpathlineto{\pgfqpoint{1.910415in}{1.379814in}}%
\pgfpathlineto{\pgfqpoint{1.910415in}{1.393425in}}%
\pgfpathlineto{\pgfqpoint{1.910415in}{1.407036in}}%
\pgfpathlineto{\pgfqpoint{1.910415in}{1.420648in}}%
\pgfpathlineto{\pgfqpoint{1.910415in}{1.434259in}}%
\pgfpathlineto{\pgfqpoint{1.910415in}{1.447870in}}%
\pgfpathlineto{\pgfqpoint{1.910415in}{1.461481in}}%
\pgfpathlineto{\pgfqpoint{1.910415in}{1.471258in}}%
\pgfpathlineto{\pgfqpoint{1.894758in}{1.472662in}}%
\pgfpathlineto{\pgfqpoint{1.885616in}{1.475092in}}%
\pgfpathlineto{\pgfqpoint{1.879102in}{1.476708in}}%
\pgfpathlineto{\pgfqpoint{1.863445in}{1.483023in}}%
\pgfpathlineto{\pgfqpoint{1.853004in}{1.488703in}}%
\pgfpathlineto{\pgfqpoint{1.847789in}{1.491379in}}%
\pgfpathlineto{\pgfqpoint{1.832132in}{1.501251in}}%
\pgfpathlineto{\pgfqpoint{1.830684in}{1.502314in}}%
\pgfpathlineto{\pgfqpoint{1.816476in}{1.512264in}}%
\pgfpathlineto{\pgfqpoint{1.811778in}{1.515925in}}%
\pgfpathlineto{\pgfqpoint{1.800819in}{1.524181in}}%
\pgfpathlineto{\pgfqpoint{1.794228in}{1.529536in}}%
\pgfpathlineto{\pgfqpoint{1.785162in}{1.536759in}}%
\pgfpathlineto{\pgfqpoint{1.777541in}{1.543148in}}%
\pgfpathlineto{\pgfqpoint{1.769506in}{1.549851in}}%
\pgfpathlineto{\pgfqpoint{1.761468in}{1.556759in}}%
\pgfpathlineto{\pgfqpoint{1.753849in}{1.563382in}}%
\pgfpathlineto{\pgfqpoint{1.745903in}{1.570370in}}%
\pgfpathlineto{\pgfqpoint{1.738193in}{1.577355in}}%
\pgfpathlineto{\pgfqpoint{1.730844in}{1.583981in}}%
\pgfpathlineto{\pgfqpoint{1.722536in}{1.591862in}}%
\pgfpathlineto{\pgfqpoint{1.716376in}{1.597592in}}%
\pgfpathlineto{\pgfqpoint{1.706880in}{1.607119in}}%
\pgfpathlineto{\pgfqpoint{1.702669in}{1.611203in}}%
\pgfpathlineto{\pgfqpoint{1.691223in}{1.623555in}}%
\pgfpathlineto{\pgfqpoint{1.690000in}{1.624814in}}%
\pgfpathlineto{\pgfqpoint{1.678644in}{1.638425in}}%
\pgfpathlineto{\pgfqpoint{1.675567in}{1.642959in}}%
\pgfpathlineto{\pgfqpoint{1.669032in}{1.652036in}}%
\pgfpathlineto{\pgfqpoint{1.661768in}{1.665648in}}%
\pgfpathlineto{\pgfqpoint{1.659910in}{1.671310in}}%
\pgfpathlineto{\pgfqpoint{1.657115in}{1.679259in}}%
\pgfpathlineto{\pgfqpoint{1.655500in}{1.692870in}}%
\pgfpathlineto{\pgfqpoint{1.644253in}{1.692870in}}%
\pgfpathlineto{\pgfqpoint{1.628597in}{1.692870in}}%
\pgfpathlineto{\pgfqpoint{1.612940in}{1.692870in}}%
\pgfpathlineto{\pgfqpoint{1.597284in}{1.692870in}}%
\pgfpathlineto{\pgfqpoint{1.581627in}{1.692870in}}%
\pgfpathlineto{\pgfqpoint{1.565971in}{1.692870in}}%
\pgfpathlineto{\pgfqpoint{1.550314in}{1.692870in}}%
\pgfpathlineto{\pgfqpoint{1.534657in}{1.692870in}}%
\pgfpathlineto{\pgfqpoint{1.519001in}{1.692870in}}%
\pgfpathlineto{\pgfqpoint{1.503344in}{1.692870in}}%
\pgfpathlineto{\pgfqpoint{1.487688in}{1.692870in}}%
\pgfpathlineto{\pgfqpoint{1.472031in}{1.692870in}}%
\pgfpathlineto{\pgfqpoint{1.456375in}{1.692870in}}%
\pgfpathlineto{\pgfqpoint{1.440718in}{1.692870in}}%
\pgfpathlineto{\pgfqpoint{1.425061in}{1.692870in}}%
\pgfpathlineto{\pgfqpoint{1.409405in}{1.692870in}}%
\pgfpathlineto{\pgfqpoint{1.393748in}{1.692870in}}%
\pgfpathlineto{\pgfqpoint{1.390305in}{1.692870in}}%
\pgfpathlineto{\pgfqpoint{1.388745in}{1.679259in}}%
\pgfpathlineto{\pgfqpoint{1.384130in}{1.665648in}}%
\pgfpathlineto{\pgfqpoint{1.378092in}{1.654678in}}%
\pgfpathlineto{\pgfqpoint{1.376728in}{1.652036in}}%
\pgfpathlineto{\pgfqpoint{1.367210in}{1.638425in}}%
\pgfpathlineto{\pgfqpoint{1.362435in}{1.632874in}}%
\pgfpathlineto{\pgfqpoint{1.355863in}{1.624814in}}%
\pgfpathlineto{\pgfqpoint{1.346779in}{1.615197in}}%
\pgfpathlineto{\pgfqpoint{1.343152in}{1.611203in}}%
\pgfpathlineto{\pgfqpoint{1.331122in}{1.599291in}}%
\pgfpathlineto{\pgfqpoint{1.329450in}{1.597592in}}%
\pgfpathlineto{\pgfqpoint{1.315466in}{1.584415in}}%
\pgfpathlineto{\pgfqpoint{1.315010in}{1.583981in}}%
\pgfpathlineto{\pgfqpoint{1.299986in}{1.570370in}}%
\pgfpathlineto{\pgfqpoint{1.299809in}{1.570214in}}%
\pgfpathlineto{\pgfqpoint{1.284452in}{1.556759in}}%
\pgfpathlineto{\pgfqpoint{1.284152in}{1.556499in}}%
\pgfpathlineto{\pgfqpoint{1.268496in}{1.543209in}}%
\pgfpathlineto{\pgfqpoint{1.268421in}{1.543148in}}%
\pgfpathlineto{\pgfqpoint{1.252839in}{1.530391in}}%
\pgfpathlineto{\pgfqpoint{1.251728in}{1.529536in}}%
\pgfpathlineto{\pgfqpoint{1.237183in}{1.518137in}}%
\pgfpathlineto{\pgfqpoint{1.234095in}{1.515925in}}%
\pgfpathlineto{\pgfqpoint{1.221526in}{1.506623in}}%
\pgfpathlineto{\pgfqpoint{1.214930in}{1.502314in}}%
\pgfpathlineto{\pgfqpoint{1.205870in}{1.496117in}}%
\pgfpathlineto{\pgfqpoint{1.192985in}{1.488703in}}%
\pgfpathlineto{\pgfqpoint{1.190213in}{1.487009in}}%
\pgfpathlineto{\pgfqpoint{1.174556in}{1.479577in}}%
\pgfpathlineto{\pgfqpoint{1.160862in}{1.475092in}}%
\pgfpathlineto{\pgfqpoint{1.158900in}{1.474402in}}%
\pgfpathlineto{\pgfqpoint{1.143243in}{1.471610in}}%
\pgfpathlineto{\pgfqpoint{1.127587in}{1.471610in}}%
\pgfpathlineto{\pgfqpoint{1.111930in}{1.474402in}}%
\pgfpathlineto{\pgfqpoint{1.109968in}{1.475092in}}%
\pgfpathlineto{\pgfqpoint{1.096274in}{1.479577in}}%
\pgfpathlineto{\pgfqpoint{1.080617in}{1.487009in}}%
\pgfpathlineto{\pgfqpoint{1.077845in}{1.488703in}}%
\pgfpathlineto{\pgfqpoint{1.064960in}{1.496117in}}%
\pgfpathlineto{\pgfqpoint{1.055900in}{1.502314in}}%
\pgfpathlineto{\pgfqpoint{1.049304in}{1.506623in}}%
\pgfpathlineto{\pgfqpoint{1.036735in}{1.515925in}}%
\pgfpathlineto{\pgfqpoint{1.033647in}{1.518137in}}%
\pgfpathlineto{\pgfqpoint{1.019102in}{1.529536in}}%
\pgfpathlineto{\pgfqpoint{1.017991in}{1.530391in}}%
\pgfpathlineto{\pgfqpoint{1.002409in}{1.543148in}}%
\pgfpathlineto{\pgfqpoint{1.002334in}{1.543209in}}%
\pgfpathlineto{\pgfqpoint{0.986678in}{1.556499in}}%
\pgfpathlineto{\pgfqpoint{0.986378in}{1.556759in}}%
\pgfpathlineto{\pgfqpoint{0.971021in}{1.570214in}}%
\pgfpathlineto{\pgfqpoint{0.970844in}{1.570370in}}%
\pgfpathlineto{\pgfqpoint{0.955820in}{1.583981in}}%
\pgfpathlineto{\pgfqpoint{0.955364in}{1.584415in}}%
\pgfpathlineto{\pgfqpoint{0.941380in}{1.597592in}}%
\pgfpathlineto{\pgfqpoint{0.939708in}{1.599291in}}%
\pgfpathlineto{\pgfqpoint{0.927678in}{1.611203in}}%
\pgfpathlineto{\pgfqpoint{0.924051in}{1.615197in}}%
\pgfpathlineto{\pgfqpoint{0.914967in}{1.624814in}}%
\pgfpathlineto{\pgfqpoint{0.908395in}{1.632874in}}%
\pgfpathlineto{\pgfqpoint{0.903620in}{1.638425in}}%
\pgfpathlineto{\pgfqpoint{0.894102in}{1.652036in}}%
\pgfpathlineto{\pgfqpoint{0.892738in}{1.654678in}}%
\pgfpathlineto{\pgfqpoint{0.886700in}{1.665648in}}%
\pgfpathlineto{\pgfqpoint{0.882085in}{1.679259in}}%
\pgfpathlineto{\pgfqpoint{0.880525in}{1.692870in}}%
\pgfpathlineto{\pgfqpoint{0.877082in}{1.692870in}}%
\pgfpathlineto{\pgfqpoint{0.861425in}{1.692870in}}%
\pgfpathlineto{\pgfqpoint{0.845769in}{1.692870in}}%
\pgfpathlineto{\pgfqpoint{0.830112in}{1.692870in}}%
\pgfpathlineto{\pgfqpoint{0.814455in}{1.692870in}}%
\pgfpathlineto{\pgfqpoint{0.798799in}{1.692870in}}%
\pgfpathlineto{\pgfqpoint{0.783142in}{1.692870in}}%
\pgfpathlineto{\pgfqpoint{0.767486in}{1.692870in}}%
\pgfpathlineto{\pgfqpoint{0.751829in}{1.692870in}}%
\pgfpathlineto{\pgfqpoint{0.736173in}{1.692870in}}%
\pgfpathlineto{\pgfqpoint{0.720516in}{1.692870in}}%
\pgfpathlineto{\pgfqpoint{0.704859in}{1.692870in}}%
\pgfpathlineto{\pgfqpoint{0.689203in}{1.692870in}}%
\pgfpathlineto{\pgfqpoint{0.673546in}{1.692870in}}%
\pgfpathlineto{\pgfqpoint{0.657890in}{1.692870in}}%
\pgfpathlineto{\pgfqpoint{0.642233in}{1.692870in}}%
\pgfpathlineto{\pgfqpoint{0.626577in}{1.692870in}}%
\pgfpathlineto{\pgfqpoint{0.615330in}{1.692870in}}%
\pgfpathlineto{\pgfqpoint{0.613715in}{1.679259in}}%
\pgfpathlineto{\pgfqpoint{0.610920in}{1.671310in}}%
\pgfpathlineto{\pgfqpoint{0.609062in}{1.665648in}}%
\pgfpathlineto{\pgfqpoint{0.601798in}{1.652036in}}%
\pgfpathlineto{\pgfqpoint{0.595263in}{1.642959in}}%
\pgfpathlineto{\pgfqpoint{0.592186in}{1.638425in}}%
\pgfpathlineto{\pgfqpoint{0.580830in}{1.624814in}}%
\pgfpathlineto{\pgfqpoint{0.579607in}{1.623555in}}%
\pgfpathlineto{\pgfqpoint{0.568161in}{1.611203in}}%
\pgfpathlineto{\pgfqpoint{0.563950in}{1.607119in}}%
\pgfpathlineto{\pgfqpoint{0.554454in}{1.597592in}}%
\pgfpathlineto{\pgfqpoint{0.548294in}{1.591862in}}%
\pgfpathlineto{\pgfqpoint{0.539986in}{1.583981in}}%
\pgfpathlineto{\pgfqpoint{0.532637in}{1.577355in}}%
\pgfpathlineto{\pgfqpoint{0.524927in}{1.570370in}}%
\pgfpathlineto{\pgfqpoint{0.516981in}{1.563382in}}%
\pgfpathlineto{\pgfqpoint{0.509362in}{1.556759in}}%
\pgfpathlineto{\pgfqpoint{0.501324in}{1.549851in}}%
\pgfpathlineto{\pgfqpoint{0.493289in}{1.543148in}}%
\pgfpathlineto{\pgfqpoint{0.485668in}{1.536759in}}%
\pgfpathlineto{\pgfqpoint{0.476602in}{1.529536in}}%
\pgfpathlineto{\pgfqpoint{0.470011in}{1.524181in}}%
\pgfpathlineto{\pgfqpoint{0.459052in}{1.515925in}}%
\pgfpathlineto{\pgfqpoint{0.454354in}{1.512264in}}%
\pgfpathlineto{\pgfqpoint{0.440146in}{1.502314in}}%
\pgfpathlineto{\pgfqpoint{0.438698in}{1.501251in}}%
\pgfpathlineto{\pgfqpoint{0.423041in}{1.491379in}}%
\pgfpathlineto{\pgfqpoint{0.417826in}{1.488703in}}%
\pgfpathlineto{\pgfqpoint{0.407385in}{1.483023in}}%
\pgfpathlineto{\pgfqpoint{0.391728in}{1.476708in}}%
\pgfpathlineto{\pgfqpoint{0.385214in}{1.475092in}}%
\pgfpathlineto{\pgfqpoint{0.376072in}{1.472662in}}%
\pgfpathlineto{\pgfqpoint{0.360415in}{1.471258in}}%
\pgfpathlineto{\pgfqpoint{0.360415in}{1.461481in}}%
\pgfpathlineto{\pgfqpoint{0.360415in}{1.447870in}}%
\pgfpathlineto{\pgfqpoint{0.360415in}{1.434259in}}%
\pgfpathlineto{\pgfqpoint{0.360415in}{1.420648in}}%
\pgfpathlineto{\pgfqpoint{0.360415in}{1.407036in}}%
\pgfpathlineto{\pgfqpoint{0.360415in}{1.393425in}}%
\pgfpathlineto{\pgfqpoint{0.360415in}{1.379814in}}%
\pgfpathlineto{\pgfqpoint{0.360415in}{1.366203in}}%
\pgfpathlineto{\pgfqpoint{0.360415in}{1.352592in}}%
\pgfpathlineto{\pgfqpoint{0.360415in}{1.338981in}}%
\pgfpathlineto{\pgfqpoint{0.360415in}{1.325370in}}%
\pgfpathlineto{\pgfqpoint{0.360415in}{1.311759in}}%
\pgfpathlineto{\pgfqpoint{0.360415in}{1.298148in}}%
\pgfpathlineto{\pgfqpoint{0.360415in}{1.284536in}}%
\pgfpathlineto{\pgfqpoint{0.360415in}{1.270925in}}%
\pgfpathlineto{\pgfqpoint{0.360415in}{1.257314in}}%
\pgfpathlineto{\pgfqpoint{0.360415in}{1.243703in}}%
\pgfpathlineto{\pgfqpoint{0.360415in}{1.240710in}}%
\pgfpathlineto{\pgfqpoint{0.376072in}{1.239353in}}%
\pgfpathlineto{\pgfqpoint{0.391728in}{1.235341in}}%
\pgfpathlineto{\pgfqpoint{0.404347in}{1.230092in}}%
\pgfpathlineto{\pgfqpoint{0.407385in}{1.228906in}}%
\pgfpathlineto{\pgfqpoint{0.423041in}{1.220632in}}%
\pgfpathlineto{\pgfqpoint{0.429427in}{1.216481in}}%
\pgfpathlineto{\pgfqpoint{0.438698in}{1.210767in}}%
\pgfpathlineto{\pgfqpoint{0.449760in}{1.202870in}}%
\pgfpathlineto{\pgfqpoint{0.454354in}{1.199716in}}%
\pgfpathlineto{\pgfqpoint{0.468056in}{1.189259in}}%
\pgfpathlineto{\pgfqpoint{0.470011in}{1.187805in}}%
\pgfpathlineto{\pgfqpoint{0.485168in}{1.175647in}}%
\pgfpathlineto{\pgfqpoint{0.485668in}{1.175252in}}%
\pgfpathlineto{\pgfqpoint{0.501324in}{1.162190in}}%
\pgfpathlineto{\pgfqpoint{0.501503in}{1.162036in}}%
\pgfpathlineto{\pgfqpoint{0.516981in}{1.148686in}}%
\pgfpathlineto{\pgfqpoint{0.517279in}{1.148425in}}%
\pgfpathlineto{\pgfqpoint{0.532567in}{1.134814in}}%
\pgfpathlineto{\pgfqpoint{0.532637in}{1.134749in}}%
\pgfpathlineto{\pgfqpoint{0.547310in}{1.121203in}}%
\pgfpathlineto{\pgfqpoint{0.548294in}{1.120237in}}%
\pgfpathlineto{\pgfqpoint{0.561406in}{1.107592in}}%
\pgfpathlineto{\pgfqpoint{0.563950in}{1.104908in}}%
\pgfpathlineto{\pgfqpoint{0.574651in}{1.093981in}}%
\pgfpathlineto{\pgfqpoint{0.579607in}{1.088246in}}%
\pgfpathlineto{\pgfqpoint{0.586736in}{1.080370in}}%
\pgfpathlineto{\pgfqpoint{0.595263in}{1.069168in}}%
\pgfpathlineto{\pgfqpoint{0.597212in}{1.066759in}}%
\pgfpathlineto{\pgfqpoint{0.605761in}{1.053148in}}%
\pgfpathlineto{\pgfqpoint{0.610920in}{1.041242in}}%
\pgfpathlineto{\pgfqpoint{0.611714in}{1.039536in}}%
\pgfpathlineto{\pgfqpoint{0.614925in}{1.025925in}}%
\pgfpathlineto{\pgfqpoint{0.614925in}{1.012314in}}%
\pgfpathlineto{\pgfqpoint{0.611714in}{0.998703in}}%
\pgfpathlineto{\pgfqpoint{0.610920in}{0.996997in}}%
\pgfpathlineto{\pgfqpoint{0.605761in}{0.985092in}}%
\pgfpathlineto{\pgfqpoint{0.597212in}{0.971481in}}%
\pgfpathlineto{\pgfqpoint{0.595263in}{0.969071in}}%
\pgfpathlineto{\pgfqpoint{0.586736in}{0.957870in}}%
\pgfpathlineto{\pgfqpoint{0.579607in}{0.949993in}}%
\pgfpathlineto{\pgfqpoint{0.574651in}{0.944259in}}%
\pgfpathlineto{\pgfqpoint{0.563950in}{0.933332in}}%
\pgfpathlineto{\pgfqpoint{0.561406in}{0.930648in}}%
\pgfpathlineto{\pgfqpoint{0.548294in}{0.918003in}}%
\pgfpathlineto{\pgfqpoint{0.547310in}{0.917036in}}%
\pgfpathlineto{\pgfqpoint{0.532637in}{0.903490in}}%
\pgfpathlineto{\pgfqpoint{0.532567in}{0.903425in}}%
\pgfpathlineto{\pgfqpoint{0.517279in}{0.889814in}}%
\pgfpathlineto{\pgfqpoint{0.516981in}{0.889554in}}%
\pgfpathlineto{\pgfqpoint{0.501503in}{0.876203in}}%
\pgfpathlineto{\pgfqpoint{0.501324in}{0.876049in}}%
\pgfpathlineto{\pgfqpoint{0.485668in}{0.862988in}}%
\pgfpathlineto{\pgfqpoint{0.485168in}{0.862592in}}%
\pgfpathlineto{\pgfqpoint{0.470011in}{0.850435in}}%
\pgfpathlineto{\pgfqpoint{0.468056in}{0.848981in}}%
\pgfpathlineto{\pgfqpoint{0.454354in}{0.838523in}}%
\pgfpathlineto{\pgfqpoint{0.449760in}{0.835370in}}%
\pgfpathlineto{\pgfqpoint{0.438698in}{0.827473in}}%
\pgfpathlineto{\pgfqpoint{0.429427in}{0.821759in}}%
\pgfpathlineto{\pgfqpoint{0.423041in}{0.817608in}}%
\pgfpathlineto{\pgfqpoint{0.407385in}{0.809333in}}%
\pgfpathlineto{\pgfqpoint{0.404347in}{0.808148in}}%
\pgfpathlineto{\pgfqpoint{0.391728in}{0.802899in}}%
\pgfpathlineto{\pgfqpoint{0.376072in}{0.798886in}}%
\pgfpathlineto{\pgfqpoint{0.360415in}{0.797530in}}%
\pgfpathlineto{\pgfqpoint{0.360415in}{0.794536in}}%
\pgfpathlineto{\pgfqpoint{0.360415in}{0.780925in}}%
\pgfpathlineto{\pgfqpoint{0.360415in}{0.767314in}}%
\pgfpathlineto{\pgfqpoint{0.360415in}{0.753703in}}%
\pgfpathlineto{\pgfqpoint{0.360415in}{0.740092in}}%
\pgfpathlineto{\pgfqpoint{0.360415in}{0.726481in}}%
\pgfpathlineto{\pgfqpoint{0.360415in}{0.712870in}}%
\pgfpathlineto{\pgfqpoint{0.360415in}{0.699259in}}%
\pgfpathlineto{\pgfqpoint{0.360415in}{0.685648in}}%
\pgfpathlineto{\pgfqpoint{0.360415in}{0.672036in}}%
\pgfpathlineto{\pgfqpoint{0.360415in}{0.658425in}}%
\pgfpathlineto{\pgfqpoint{0.360415in}{0.644814in}}%
\pgfpathlineto{\pgfqpoint{0.360415in}{0.631203in}}%
\pgfpathlineto{\pgfqpoint{0.360415in}{0.617592in}}%
\pgfpathlineto{\pgfqpoint{0.360415in}{0.603981in}}%
\pgfpathlineto{\pgfqpoint{0.360415in}{0.590370in}}%
\pgfpathlineto{\pgfqpoint{0.360415in}{0.576759in}}%
\pgfpathlineto{\pgfqpoint{0.360415in}{0.566981in}}%
\pgfpathlineto{\pgfqpoint{0.376072in}{0.565578in}}%
\pgfpathlineto{\pgfqpoint{0.385214in}{0.563148in}}%
\pgfpathlineto{\pgfqpoint{0.391728in}{0.561532in}}%
\pgfpathlineto{\pgfqpoint{0.407385in}{0.555217in}}%
\pgfpathlineto{\pgfqpoint{0.417826in}{0.549536in}}%
\pgfpathlineto{\pgfqpoint{0.423041in}{0.546861in}}%
\pgfpathlineto{\pgfqpoint{0.438698in}{0.536989in}}%
\pgfpathlineto{\pgfqpoint{0.440146in}{0.535925in}}%
\pgfpathlineto{\pgfqpoint{0.454354in}{0.525975in}}%
\pgfpathlineto{\pgfqpoint{0.459052in}{0.522314in}}%
\pgfpathlineto{\pgfqpoint{0.470011in}{0.514059in}}%
\pgfpathlineto{\pgfqpoint{0.476602in}{0.508703in}}%
\pgfpathlineto{\pgfqpoint{0.485668in}{0.501480in}}%
\pgfpathlineto{\pgfqpoint{0.493289in}{0.495092in}}%
\pgfpathlineto{\pgfqpoint{0.501324in}{0.488389in}}%
\pgfpathlineto{\pgfqpoint{0.509362in}{0.481481in}}%
\pgfpathlineto{\pgfqpoint{0.516981in}{0.474858in}}%
\pgfpathlineto{\pgfqpoint{0.524927in}{0.467870in}}%
\pgfpathlineto{\pgfqpoint{0.532637in}{0.460885in}}%
\pgfpathlineto{\pgfqpoint{0.539986in}{0.454259in}}%
\pgfpathlineto{\pgfqpoint{0.548294in}{0.446378in}}%
\pgfpathlineto{\pgfqpoint{0.554454in}{0.440648in}}%
\pgfpathlineto{\pgfqpoint{0.563950in}{0.431121in}}%
\pgfpathlineto{\pgfqpoint{0.568161in}{0.427036in}}%
\pgfpathlineto{\pgfqpoint{0.579607in}{0.414684in}}%
\pgfpathlineto{\pgfqpoint{0.580830in}{0.413425in}}%
\pgfpathlineto{\pgfqpoint{0.592186in}{0.399814in}}%
\pgfpathlineto{\pgfqpoint{0.595263in}{0.395281in}}%
\pgfpathlineto{\pgfqpoint{0.601798in}{0.386203in}}%
\pgfpathlineto{\pgfqpoint{0.609062in}{0.372592in}}%
\pgfpathlineto{\pgfqpoint{0.610920in}{0.366929in}}%
\pgfpathlineto{\pgfqpoint{0.613715in}{0.358981in}}%
\pgfpathlineto{\pgfqpoint{0.615330in}{0.345370in}}%
\pgfpathlineto{\pgfqpoint{0.626577in}{0.345370in}}%
\pgfpathclose%
\pgfpathmoveto{\pgfqpoint{0.724678in}{0.386203in}}%
\pgfpathlineto{\pgfqpoint{0.720516in}{0.387078in}}%
\pgfpathlineto{\pgfqpoint{0.704859in}{0.392993in}}%
\pgfpathlineto{\pgfqpoint{0.692348in}{0.399814in}}%
\pgfpathlineto{\pgfqpoint{0.689203in}{0.401190in}}%
\pgfpathlineto{\pgfqpoint{0.673546in}{0.410086in}}%
\pgfpathlineto{\pgfqpoint{0.668742in}{0.413425in}}%
\pgfpathlineto{\pgfqpoint{0.657890in}{0.419863in}}%
\pgfpathlineto{\pgfqpoint{0.647622in}{0.427036in}}%
\pgfpathlineto{\pgfqpoint{0.642233in}{0.430388in}}%
\pgfpathlineto{\pgfqpoint{0.627805in}{0.440648in}}%
\pgfpathlineto{\pgfqpoint{0.626577in}{0.441451in}}%
\pgfpathlineto{\pgfqpoint{0.610920in}{0.452852in}}%
\pgfpathlineto{\pgfqpoint{0.609150in}{0.454259in}}%
\pgfpathlineto{\pgfqpoint{0.595263in}{0.464697in}}%
\pgfpathlineto{\pgfqpoint{0.591333in}{0.467870in}}%
\pgfpathlineto{\pgfqpoint{0.579607in}{0.477028in}}%
\pgfpathlineto{\pgfqpoint{0.574208in}{0.481481in}}%
\pgfpathlineto{\pgfqpoint{0.563950in}{0.489833in}}%
\pgfpathlineto{\pgfqpoint{0.557734in}{0.495092in}}%
\pgfpathlineto{\pgfqpoint{0.548294in}{0.503120in}}%
\pgfpathlineto{\pgfqpoint{0.541871in}{0.508703in}}%
\pgfpathlineto{\pgfqpoint{0.532637in}{0.516910in}}%
\pgfpathlineto{\pgfqpoint{0.526588in}{0.522314in}}%
\pgfpathlineto{\pgfqpoint{0.516981in}{0.531232in}}%
\pgfpathlineto{\pgfqpoint{0.511859in}{0.535925in}}%
\pgfpathlineto{\pgfqpoint{0.501324in}{0.546119in}}%
\pgfpathlineto{\pgfqpoint{0.497675in}{0.549536in}}%
\pgfpathlineto{\pgfqpoint{0.485668in}{0.561608in}}%
\pgfpathlineto{\pgfqpoint{0.484049in}{0.563148in}}%
\pgfpathlineto{\pgfqpoint{0.470935in}{0.576759in}}%
\pgfpathlineto{\pgfqpoint{0.470011in}{0.577827in}}%
\pgfpathlineto{\pgfqpoint{0.458209in}{0.590370in}}%
\pgfpathlineto{\pgfqpoint{0.454354in}{0.595055in}}%
\pgfpathlineto{\pgfqpoint{0.446103in}{0.603981in}}%
\pgfpathlineto{\pgfqpoint{0.438698in}{0.613415in}}%
\pgfpathlineto{\pgfqpoint{0.434857in}{0.617592in}}%
\pgfpathlineto{\pgfqpoint{0.424624in}{0.631203in}}%
\pgfpathlineto{\pgfqpoint{0.423041in}{0.633938in}}%
\pgfpathlineto{\pgfqpoint{0.415195in}{0.644814in}}%
\pgfpathlineto{\pgfqpoint{0.408392in}{0.658425in}}%
\pgfpathlineto{\pgfqpoint{0.407385in}{0.662043in}}%
\pgfpathlineto{\pgfqpoint{0.403568in}{0.672036in}}%
\pgfpathlineto{\pgfqpoint{0.402527in}{0.685648in}}%
\pgfpathlineto{\pgfqpoint{0.405649in}{0.699259in}}%
\pgfpathlineto{\pgfqpoint{0.407385in}{0.702513in}}%
\pgfpathlineto{\pgfqpoint{0.411415in}{0.712870in}}%
\pgfpathlineto{\pgfqpoint{0.419734in}{0.726481in}}%
\pgfpathlineto{\pgfqpoint{0.423041in}{0.730475in}}%
\pgfpathlineto{\pgfqpoint{0.429438in}{0.740092in}}%
\pgfpathlineto{\pgfqpoint{0.438698in}{0.751136in}}%
\pgfpathlineto{\pgfqpoint{0.440532in}{0.753703in}}%
\pgfpathlineto{\pgfqpoint{0.452180in}{0.767314in}}%
\pgfpathlineto{\pgfqpoint{0.454354in}{0.769514in}}%
\pgfpathlineto{\pgfqpoint{0.464405in}{0.780925in}}%
\pgfpathlineto{\pgfqpoint{0.470011in}{0.786567in}}%
\pgfpathlineto{\pgfqpoint{0.477301in}{0.794536in}}%
\pgfpathlineto{\pgfqpoint{0.485668in}{0.802820in}}%
\pgfpathlineto{\pgfqpoint{0.490754in}{0.808148in}}%
\pgfpathlineto{\pgfqpoint{0.501324in}{0.818363in}}%
\pgfpathlineto{\pgfqpoint{0.504720in}{0.821759in}}%
\pgfpathlineto{\pgfqpoint{0.516981in}{0.833265in}}%
\pgfpathlineto{\pgfqpoint{0.519193in}{0.835370in}}%
\pgfpathlineto{\pgfqpoint{0.532637in}{0.847578in}}%
\pgfpathlineto{\pgfqpoint{0.534189in}{0.848981in}}%
\pgfpathlineto{\pgfqpoint{0.548294in}{0.861351in}}%
\pgfpathlineto{\pgfqpoint{0.549740in}{0.862592in}}%
\pgfpathlineto{\pgfqpoint{0.563950in}{0.874625in}}%
\pgfpathlineto{\pgfqpoint{0.565887in}{0.876203in}}%
\pgfpathlineto{\pgfqpoint{0.579607in}{0.887432in}}%
\pgfpathlineto{\pgfqpoint{0.582682in}{0.889814in}}%
\pgfpathlineto{\pgfqpoint{0.595263in}{0.899793in}}%
\pgfpathlineto{\pgfqpoint{0.600184in}{0.903425in}}%
\pgfpathlineto{\pgfqpoint{0.610920in}{0.911714in}}%
\pgfpathlineto{\pgfqpoint{0.618460in}{0.917036in}}%
\pgfpathlineto{\pgfqpoint{0.626577in}{0.923175in}}%
\pgfpathlineto{\pgfqpoint{0.637600in}{0.930648in}}%
\pgfpathlineto{\pgfqpoint{0.642233in}{0.934112in}}%
\pgfpathlineto{\pgfqpoint{0.657731in}{0.944259in}}%
\pgfpathlineto{\pgfqpoint{0.657890in}{0.944378in}}%
\pgfpathlineto{\pgfqpoint{0.673546in}{0.954403in}}%
\pgfpathlineto{\pgfqpoint{0.680192in}{0.957870in}}%
\pgfpathlineto{\pgfqpoint{0.689203in}{0.963534in}}%
\pgfpathlineto{\pgfqpoint{0.704859in}{0.971093in}}%
\pgfpathlineto{\pgfqpoint{0.706015in}{0.971481in}}%
\pgfpathlineto{\pgfqpoint{0.720516in}{0.977814in}}%
\pgfpathlineto{\pgfqpoint{0.736173in}{0.981612in}}%
\pgfpathlineto{\pgfqpoint{0.751829in}{0.982371in}}%
\pgfpathlineto{\pgfqpoint{0.767486in}{0.980093in}}%
\pgfpathlineto{\pgfqpoint{0.783142in}{0.974774in}}%
\pgfpathlineto{\pgfqpoint{0.789337in}{0.971481in}}%
\pgfpathlineto{\pgfqpoint{0.798799in}{0.967605in}}%
\pgfpathlineto{\pgfqpoint{0.814455in}{0.958879in}}%
\pgfpathlineto{\pgfqpoint{0.815902in}{0.957870in}}%
\pgfpathlineto{\pgfqpoint{0.830112in}{0.949629in}}%
\pgfpathlineto{\pgfqpoint{0.837833in}{0.944259in}}%
\pgfpathlineto{\pgfqpoint{0.845769in}{0.939441in}}%
\pgfpathlineto{\pgfqpoint{0.858260in}{0.930648in}}%
\pgfpathlineto{\pgfqpoint{0.861425in}{0.928630in}}%
\pgfpathlineto{\pgfqpoint{0.877082in}{0.917365in}}%
\pgfpathlineto{\pgfqpoint{0.877498in}{0.917036in}}%
\pgfpathlineto{\pgfqpoint{0.892738in}{0.905777in}}%
\pgfpathlineto{\pgfqpoint{0.895678in}{0.903425in}}%
\pgfpathlineto{\pgfqpoint{0.908395in}{0.893686in}}%
\pgfpathlineto{\pgfqpoint{0.913143in}{0.889814in}}%
\pgfpathlineto{\pgfqpoint{0.924051in}{0.881119in}}%
\pgfpathlineto{\pgfqpoint{0.929938in}{0.876203in}}%
\pgfpathlineto{\pgfqpoint{0.939708in}{0.868075in}}%
\pgfpathlineto{\pgfqpoint{0.946102in}{0.862592in}}%
\pgfpathlineto{\pgfqpoint{0.955364in}{0.854539in}}%
\pgfpathlineto{\pgfqpoint{0.961671in}{0.848981in}}%
\pgfpathlineto{\pgfqpoint{0.971021in}{0.840487in}}%
\pgfpathlineto{\pgfqpoint{0.976675in}{0.835370in}}%
\pgfpathlineto{\pgfqpoint{0.986678in}{0.825887in}}%
\pgfpathlineto{\pgfqpoint{0.991132in}{0.821759in}}%
\pgfpathlineto{\pgfqpoint{1.002334in}{0.810703in}}%
\pgfpathlineto{\pgfqpoint{1.005040in}{0.808148in}}%
\pgfpathlineto{\pgfqpoint{1.017991in}{0.794898in}}%
\pgfpathlineto{\pgfqpoint{1.018369in}{0.794536in}}%
\pgfpathlineto{\pgfqpoint{1.031327in}{0.780925in}}%
\pgfpathlineto{\pgfqpoint{1.033647in}{0.778173in}}%
\pgfpathlineto{\pgfqpoint{1.043762in}{0.767314in}}%
\pgfpathlineto{\pgfqpoint{1.049304in}{0.760416in}}%
\pgfpathlineto{\pgfqpoint{1.055482in}{0.753703in}}%
\pgfpathlineto{\pgfqpoint{1.064960in}{0.741349in}}%
\pgfpathlineto{\pgfqpoint{1.066122in}{0.740092in}}%
\pgfpathlineto{\pgfqpoint{1.076159in}{0.726481in}}%
\pgfpathlineto{\pgfqpoint{1.080617in}{0.718255in}}%
\pgfpathlineto{\pgfqpoint{1.084405in}{0.712870in}}%
\pgfpathlineto{\pgfqpoint{1.090524in}{0.699259in}}%
\pgfpathlineto{\pgfqpoint{1.093144in}{0.685648in}}%
\pgfpathlineto{\pgfqpoint{1.092271in}{0.672036in}}%
\pgfpathlineto{\pgfqpoint{1.087902in}{0.658425in}}%
\pgfpathlineto{\pgfqpoint{1.080617in}{0.645818in}}%
\pgfpathlineto{\pgfqpoint{1.080171in}{0.644814in}}%
\pgfpathlineto{\pgfqpoint{1.071476in}{0.631203in}}%
\pgfpathlineto{\pgfqpoint{1.064960in}{0.623370in}}%
\pgfpathlineto{\pgfqpoint{1.060972in}{0.617592in}}%
\pgfpathlineto{\pgfqpoint{1.049441in}{0.603981in}}%
\pgfpathlineto{\pgfqpoint{1.049304in}{0.603843in}}%
\pgfpathlineto{\pgfqpoint{1.037633in}{0.590370in}}%
\pgfpathlineto{\pgfqpoint{1.033647in}{0.586342in}}%
\pgfpathlineto{\pgfqpoint{1.025052in}{0.576759in}}%
\pgfpathlineto{\pgfqpoint{1.017991in}{0.569703in}}%
\pgfpathlineto{\pgfqpoint{1.011868in}{0.563148in}}%
\pgfpathlineto{\pgfqpoint{1.002334in}{0.553814in}}%
\pgfpathlineto{\pgfqpoint{0.998156in}{0.549536in}}%
\pgfpathlineto{\pgfqpoint{0.986678in}{0.538599in}}%
\pgfpathlineto{\pgfqpoint{0.983937in}{0.535925in}}%
\pgfpathlineto{\pgfqpoint{0.971021in}{0.523998in}}%
\pgfpathlineto{\pgfqpoint{0.969206in}{0.522314in}}%
\pgfpathlineto{\pgfqpoint{0.955364in}{0.509960in}}%
\pgfpathlineto{\pgfqpoint{0.953937in}{0.508703in}}%
\pgfpathlineto{\pgfqpoint{0.939708in}{0.496441in}}%
\pgfpathlineto{\pgfqpoint{0.938094in}{0.495092in}}%
\pgfpathlineto{\pgfqpoint{0.924051in}{0.483405in}}%
\pgfpathlineto{\pgfqpoint{0.921630in}{0.481481in}}%
\pgfpathlineto{\pgfqpoint{0.908395in}{0.470822in}}%
\pgfpathlineto{\pgfqpoint{0.904489in}{0.467870in}}%
\pgfpathlineto{\pgfqpoint{0.892738in}{0.458681in}}%
\pgfpathlineto{\pgfqpoint{0.886610in}{0.454259in}}%
\pgfpathlineto{\pgfqpoint{0.877082in}{0.446985in}}%
\pgfpathlineto{\pgfqpoint{0.867914in}{0.440648in}}%
\pgfpathlineto{\pgfqpoint{0.861425in}{0.435774in}}%
\pgfpathlineto{\pgfqpoint{0.848299in}{0.427036in}}%
\pgfpathlineto{\pgfqpoint{0.845769in}{0.425146in}}%
\pgfpathlineto{\pgfqpoint{0.830112in}{0.415020in}}%
\pgfpathlineto{\pgfqpoint{0.827160in}{0.413425in}}%
\pgfpathlineto{\pgfqpoint{0.814455in}{0.405375in}}%
\pgfpathlineto{\pgfqpoint{0.803393in}{0.399814in}}%
\pgfpathlineto{\pgfqpoint{0.798799in}{0.396939in}}%
\pgfpathlineto{\pgfqpoint{0.783142in}{0.389707in}}%
\pgfpathlineto{\pgfqpoint{0.771229in}{0.386203in}}%
\pgfpathlineto{\pgfqpoint{0.767486in}{0.384694in}}%
\pgfpathlineto{\pgfqpoint{0.751829in}{0.381980in}}%
\pgfpathlineto{\pgfqpoint{0.736173in}{0.382885in}}%
\pgfpathlineto{\pgfqpoint{0.724678in}{0.386203in}}%
\pgfpathclose%
\pgfpathmoveto{\pgfqpoint{1.499601in}{0.386203in}}%
\pgfpathlineto{\pgfqpoint{1.487688in}{0.389707in}}%
\pgfpathlineto{\pgfqpoint{1.472031in}{0.396939in}}%
\pgfpathlineto{\pgfqpoint{1.467437in}{0.399814in}}%
\pgfpathlineto{\pgfqpoint{1.456375in}{0.405375in}}%
\pgfpathlineto{\pgfqpoint{1.443670in}{0.413425in}}%
\pgfpathlineto{\pgfqpoint{1.440718in}{0.415020in}}%
\pgfpathlineto{\pgfqpoint{1.425061in}{0.425146in}}%
\pgfpathlineto{\pgfqpoint{1.422531in}{0.427036in}}%
\pgfpathlineto{\pgfqpoint{1.409405in}{0.435774in}}%
\pgfpathlineto{\pgfqpoint{1.402916in}{0.440648in}}%
\pgfpathlineto{\pgfqpoint{1.393748in}{0.446985in}}%
\pgfpathlineto{\pgfqpoint{1.384220in}{0.454259in}}%
\pgfpathlineto{\pgfqpoint{1.378092in}{0.458681in}}%
\pgfpathlineto{\pgfqpoint{1.366341in}{0.467870in}}%
\pgfpathlineto{\pgfqpoint{1.362435in}{0.470822in}}%
\pgfpathlineto{\pgfqpoint{1.349200in}{0.481481in}}%
\pgfpathlineto{\pgfqpoint{1.346779in}{0.483405in}}%
\pgfpathlineto{\pgfqpoint{1.332736in}{0.495092in}}%
\pgfpathlineto{\pgfqpoint{1.331122in}{0.496441in}}%
\pgfpathlineto{\pgfqpoint{1.316893in}{0.508703in}}%
\pgfpathlineto{\pgfqpoint{1.315466in}{0.509960in}}%
\pgfpathlineto{\pgfqpoint{1.301624in}{0.522314in}}%
\pgfpathlineto{\pgfqpoint{1.299809in}{0.523998in}}%
\pgfpathlineto{\pgfqpoint{1.286893in}{0.535925in}}%
\pgfpathlineto{\pgfqpoint{1.284152in}{0.538599in}}%
\pgfpathlineto{\pgfqpoint{1.272674in}{0.549536in}}%
\pgfpathlineto{\pgfqpoint{1.268496in}{0.553814in}}%
\pgfpathlineto{\pgfqpoint{1.258962in}{0.563148in}}%
\pgfpathlineto{\pgfqpoint{1.252839in}{0.569703in}}%
\pgfpathlineto{\pgfqpoint{1.245778in}{0.576759in}}%
\pgfpathlineto{\pgfqpoint{1.237183in}{0.586342in}}%
\pgfpathlineto{\pgfqpoint{1.233197in}{0.590370in}}%
\pgfpathlineto{\pgfqpoint{1.221526in}{0.603843in}}%
\pgfpathlineto{\pgfqpoint{1.221389in}{0.603981in}}%
\pgfpathlineto{\pgfqpoint{1.209858in}{0.617592in}}%
\pgfpathlineto{\pgfqpoint{1.205870in}{0.623370in}}%
\pgfpathlineto{\pgfqpoint{1.199354in}{0.631203in}}%
\pgfpathlineto{\pgfqpoint{1.190659in}{0.644814in}}%
\pgfpathlineto{\pgfqpoint{1.190213in}{0.645818in}}%
\pgfpathlineto{\pgfqpoint{1.182928in}{0.658425in}}%
\pgfpathlineto{\pgfqpoint{1.178559in}{0.672036in}}%
\pgfpathlineto{\pgfqpoint{1.177686in}{0.685648in}}%
\pgfpathlineto{\pgfqpoint{1.180306in}{0.699259in}}%
\pgfpathlineto{\pgfqpoint{1.186425in}{0.712870in}}%
\pgfpathlineto{\pgfqpoint{1.190213in}{0.718255in}}%
\pgfpathlineto{\pgfqpoint{1.194671in}{0.726481in}}%
\pgfpathlineto{\pgfqpoint{1.204708in}{0.740092in}}%
\pgfpathlineto{\pgfqpoint{1.205870in}{0.741349in}}%
\pgfpathlineto{\pgfqpoint{1.215348in}{0.753703in}}%
\pgfpathlineto{\pgfqpoint{1.221526in}{0.760416in}}%
\pgfpathlineto{\pgfqpoint{1.227068in}{0.767314in}}%
\pgfpathlineto{\pgfqpoint{1.237183in}{0.778173in}}%
\pgfpathlineto{\pgfqpoint{1.239503in}{0.780925in}}%
\pgfpathlineto{\pgfqpoint{1.252461in}{0.794536in}}%
\pgfpathlineto{\pgfqpoint{1.252839in}{0.794898in}}%
\pgfpathlineto{\pgfqpoint{1.265790in}{0.808148in}}%
\pgfpathlineto{\pgfqpoint{1.268496in}{0.810703in}}%
\pgfpathlineto{\pgfqpoint{1.279698in}{0.821759in}}%
\pgfpathlineto{\pgfqpoint{1.284152in}{0.825887in}}%
\pgfpathlineto{\pgfqpoint{1.294155in}{0.835370in}}%
\pgfpathlineto{\pgfqpoint{1.299809in}{0.840487in}}%
\pgfpathlineto{\pgfqpoint{1.309159in}{0.848981in}}%
\pgfpathlineto{\pgfqpoint{1.315466in}{0.854539in}}%
\pgfpathlineto{\pgfqpoint{1.324728in}{0.862592in}}%
\pgfpathlineto{\pgfqpoint{1.331122in}{0.868075in}}%
\pgfpathlineto{\pgfqpoint{1.340892in}{0.876203in}}%
\pgfpathlineto{\pgfqpoint{1.346779in}{0.881119in}}%
\pgfpathlineto{\pgfqpoint{1.357687in}{0.889814in}}%
\pgfpathlineto{\pgfqpoint{1.362435in}{0.893686in}}%
\pgfpathlineto{\pgfqpoint{1.375152in}{0.903425in}}%
\pgfpathlineto{\pgfqpoint{1.378092in}{0.905777in}}%
\pgfpathlineto{\pgfqpoint{1.393332in}{0.917036in}}%
\pgfpathlineto{\pgfqpoint{1.393748in}{0.917365in}}%
\pgfpathlineto{\pgfqpoint{1.409405in}{0.928630in}}%
\pgfpathlineto{\pgfqpoint{1.412570in}{0.930648in}}%
\pgfpathlineto{\pgfqpoint{1.425061in}{0.939441in}}%
\pgfpathlineto{\pgfqpoint{1.432997in}{0.944259in}}%
\pgfpathlineto{\pgfqpoint{1.440718in}{0.949629in}}%
\pgfpathlineto{\pgfqpoint{1.454928in}{0.957870in}}%
\pgfpathlineto{\pgfqpoint{1.456375in}{0.958879in}}%
\pgfpathlineto{\pgfqpoint{1.472031in}{0.967605in}}%
\pgfpathlineto{\pgfqpoint{1.481493in}{0.971481in}}%
\pgfpathlineto{\pgfqpoint{1.487688in}{0.974774in}}%
\pgfpathlineto{\pgfqpoint{1.503344in}{0.980093in}}%
\pgfpathlineto{\pgfqpoint{1.519001in}{0.982371in}}%
\pgfpathlineto{\pgfqpoint{1.534657in}{0.981612in}}%
\pgfpathlineto{\pgfqpoint{1.550314in}{0.977814in}}%
\pgfpathlineto{\pgfqpoint{1.564815in}{0.971481in}}%
\pgfpathlineto{\pgfqpoint{1.565971in}{0.971093in}}%
\pgfpathlineto{\pgfqpoint{1.581627in}{0.963534in}}%
\pgfpathlineto{\pgfqpoint{1.590638in}{0.957870in}}%
\pgfpathlineto{\pgfqpoint{1.597284in}{0.954403in}}%
\pgfpathlineto{\pgfqpoint{1.612940in}{0.944378in}}%
\pgfpathlineto{\pgfqpoint{1.613099in}{0.944259in}}%
\pgfpathlineto{\pgfqpoint{1.628597in}{0.934112in}}%
\pgfpathlineto{\pgfqpoint{1.633230in}{0.930648in}}%
\pgfpathlineto{\pgfqpoint{1.644253in}{0.923175in}}%
\pgfpathlineto{\pgfqpoint{1.652370in}{0.917036in}}%
\pgfpathlineto{\pgfqpoint{1.659910in}{0.911714in}}%
\pgfpathlineto{\pgfqpoint{1.670646in}{0.903425in}}%
\pgfpathlineto{\pgfqpoint{1.675567in}{0.899793in}}%
\pgfpathlineto{\pgfqpoint{1.688148in}{0.889814in}}%
\pgfpathlineto{\pgfqpoint{1.691223in}{0.887432in}}%
\pgfpathlineto{\pgfqpoint{1.704943in}{0.876203in}}%
\pgfpathlineto{\pgfqpoint{1.706880in}{0.874625in}}%
\pgfpathlineto{\pgfqpoint{1.721090in}{0.862592in}}%
\pgfpathlineto{\pgfqpoint{1.722536in}{0.861351in}}%
\pgfpathlineto{\pgfqpoint{1.736641in}{0.848981in}}%
\pgfpathlineto{\pgfqpoint{1.738193in}{0.847578in}}%
\pgfpathlineto{\pgfqpoint{1.751637in}{0.835370in}}%
\pgfpathlineto{\pgfqpoint{1.753849in}{0.833265in}}%
\pgfpathlineto{\pgfqpoint{1.766110in}{0.821759in}}%
\pgfpathlineto{\pgfqpoint{1.769506in}{0.818363in}}%
\pgfpathlineto{\pgfqpoint{1.780076in}{0.808148in}}%
\pgfpathlineto{\pgfqpoint{1.785162in}{0.802820in}}%
\pgfpathlineto{\pgfqpoint{1.793529in}{0.794536in}}%
\pgfpathlineto{\pgfqpoint{1.800819in}{0.786567in}}%
\pgfpathlineto{\pgfqpoint{1.806425in}{0.780925in}}%
\pgfpathlineto{\pgfqpoint{1.816476in}{0.769514in}}%
\pgfpathlineto{\pgfqpoint{1.818650in}{0.767314in}}%
\pgfpathlineto{\pgfqpoint{1.830298in}{0.753703in}}%
\pgfpathlineto{\pgfqpoint{1.832132in}{0.751136in}}%
\pgfpathlineto{\pgfqpoint{1.841392in}{0.740092in}}%
\pgfpathlineto{\pgfqpoint{1.847789in}{0.730475in}}%
\pgfpathlineto{\pgfqpoint{1.851096in}{0.726481in}}%
\pgfpathlineto{\pgfqpoint{1.859415in}{0.712870in}}%
\pgfpathlineto{\pgfqpoint{1.863445in}{0.702513in}}%
\pgfpathlineto{\pgfqpoint{1.865181in}{0.699259in}}%
\pgfpathlineto{\pgfqpoint{1.868303in}{0.685648in}}%
\pgfpathlineto{\pgfqpoint{1.867262in}{0.672036in}}%
\pgfpathlineto{\pgfqpoint{1.863445in}{0.662043in}}%
\pgfpathlineto{\pgfqpoint{1.862438in}{0.658425in}}%
\pgfpathlineto{\pgfqpoint{1.855635in}{0.644814in}}%
\pgfpathlineto{\pgfqpoint{1.847789in}{0.633938in}}%
\pgfpathlineto{\pgfqpoint{1.846206in}{0.631203in}}%
\pgfpathlineto{\pgfqpoint{1.835973in}{0.617592in}}%
\pgfpathlineto{\pgfqpoint{1.832132in}{0.613415in}}%
\pgfpathlineto{\pgfqpoint{1.824727in}{0.603981in}}%
\pgfpathlineto{\pgfqpoint{1.816476in}{0.595055in}}%
\pgfpathlineto{\pgfqpoint{1.812621in}{0.590370in}}%
\pgfpathlineto{\pgfqpoint{1.800819in}{0.577827in}}%
\pgfpathlineto{\pgfqpoint{1.799895in}{0.576759in}}%
\pgfpathlineto{\pgfqpoint{1.786781in}{0.563148in}}%
\pgfpathlineto{\pgfqpoint{1.785162in}{0.561608in}}%
\pgfpathlineto{\pgfqpoint{1.773155in}{0.549536in}}%
\pgfpathlineto{\pgfqpoint{1.769506in}{0.546119in}}%
\pgfpathlineto{\pgfqpoint{1.758971in}{0.535925in}}%
\pgfpathlineto{\pgfqpoint{1.753849in}{0.531232in}}%
\pgfpathlineto{\pgfqpoint{1.744242in}{0.522314in}}%
\pgfpathlineto{\pgfqpoint{1.738193in}{0.516910in}}%
\pgfpathlineto{\pgfqpoint{1.728959in}{0.508703in}}%
\pgfpathlineto{\pgfqpoint{1.722536in}{0.503120in}}%
\pgfpathlineto{\pgfqpoint{1.713096in}{0.495092in}}%
\pgfpathlineto{\pgfqpoint{1.706880in}{0.489833in}}%
\pgfpathlineto{\pgfqpoint{1.696622in}{0.481481in}}%
\pgfpathlineto{\pgfqpoint{1.691223in}{0.477028in}}%
\pgfpathlineto{\pgfqpoint{1.679497in}{0.467870in}}%
\pgfpathlineto{\pgfqpoint{1.675567in}{0.464697in}}%
\pgfpathlineto{\pgfqpoint{1.661680in}{0.454259in}}%
\pgfpathlineto{\pgfqpoint{1.659910in}{0.452852in}}%
\pgfpathlineto{\pgfqpoint{1.644253in}{0.441451in}}%
\pgfpathlineto{\pgfqpoint{1.643025in}{0.440648in}}%
\pgfpathlineto{\pgfqpoint{1.628597in}{0.430388in}}%
\pgfpathlineto{\pgfqpoint{1.623208in}{0.427036in}}%
\pgfpathlineto{\pgfqpoint{1.612940in}{0.419863in}}%
\pgfpathlineto{\pgfqpoint{1.602088in}{0.413425in}}%
\pgfpathlineto{\pgfqpoint{1.597284in}{0.410086in}}%
\pgfpathlineto{\pgfqpoint{1.581627in}{0.401190in}}%
\pgfpathlineto{\pgfqpoint{1.578482in}{0.399814in}}%
\pgfpathlineto{\pgfqpoint{1.565971in}{0.392993in}}%
\pgfpathlineto{\pgfqpoint{1.550314in}{0.387078in}}%
\pgfpathlineto{\pgfqpoint{1.546152in}{0.386203in}}%
\pgfpathlineto{\pgfqpoint{1.534657in}{0.382885in}}%
\pgfpathlineto{\pgfqpoint{1.519001in}{0.381980in}}%
\pgfpathlineto{\pgfqpoint{1.503344in}{0.384694in}}%
\pgfpathlineto{\pgfqpoint{1.499601in}{0.386203in}}%
\pgfpathclose%
\pgfpathmoveto{\pgfqpoint{1.091617in}{0.808148in}}%
\pgfpathlineto{\pgfqpoint{1.080617in}{0.813225in}}%
\pgfpathlineto{\pgfqpoint{1.066134in}{0.821759in}}%
\pgfpathlineto{\pgfqpoint{1.064960in}{0.822415in}}%
\pgfpathlineto{\pgfqpoint{1.049304in}{0.832876in}}%
\pgfpathlineto{\pgfqpoint{1.046016in}{0.835370in}}%
\pgfpathlineto{\pgfqpoint{1.033647in}{0.844377in}}%
\pgfpathlineto{\pgfqpoint{1.027871in}{0.848981in}}%
\pgfpathlineto{\pgfqpoint{1.017991in}{0.856648in}}%
\pgfpathlineto{\pgfqpoint{1.010796in}{0.862592in}}%
\pgfpathlineto{\pgfqpoint{1.002334in}{0.869497in}}%
\pgfpathlineto{\pgfqpoint{0.994437in}{0.876203in}}%
\pgfpathlineto{\pgfqpoint{0.986678in}{0.882813in}}%
\pgfpathlineto{\pgfqpoint{0.978624in}{0.889814in}}%
\pgfpathlineto{\pgfqpoint{0.971021in}{0.896560in}}%
\pgfpathlineto{\pgfqpoint{0.963307in}{0.903425in}}%
\pgfpathlineto{\pgfqpoint{0.955364in}{0.910782in}}%
\pgfpathlineto{\pgfqpoint{0.948528in}{0.917036in}}%
\pgfpathlineto{\pgfqpoint{0.939708in}{0.925626in}}%
\pgfpathlineto{\pgfqpoint{0.934412in}{0.930648in}}%
\pgfpathlineto{\pgfqpoint{0.924051in}{0.941400in}}%
\pgfpathlineto{\pgfqpoint{0.921182in}{0.944259in}}%
\pgfpathlineto{\pgfqpoint{0.909150in}{0.957870in}}%
\pgfpathlineto{\pgfqpoint{0.908395in}{0.958890in}}%
\pgfpathlineto{\pgfqpoint{0.898579in}{0.971481in}}%
\pgfpathlineto{\pgfqpoint{0.892738in}{0.981044in}}%
\pgfpathlineto{\pgfqpoint{0.890097in}{0.985092in}}%
\pgfpathlineto{\pgfqpoint{0.884020in}{0.998703in}}%
\pgfpathlineto{\pgfqpoint{0.880916in}{1.012314in}}%
\pgfpathlineto{\pgfqpoint{0.880916in}{1.025925in}}%
\pgfpathlineto{\pgfqpoint{0.884020in}{1.039536in}}%
\pgfpathlineto{\pgfqpoint{0.890097in}{1.053148in}}%
\pgfpathlineto{\pgfqpoint{0.892738in}{1.057196in}}%
\pgfpathlineto{\pgfqpoint{0.898579in}{1.066759in}}%
\pgfpathlineto{\pgfqpoint{0.908395in}{1.079350in}}%
\pgfpathlineto{\pgfqpoint{0.909150in}{1.080370in}}%
\pgfpathlineto{\pgfqpoint{0.921182in}{1.093981in}}%
\pgfpathlineto{\pgfqpoint{0.924051in}{1.096839in}}%
\pgfpathlineto{\pgfqpoint{0.934412in}{1.107592in}}%
\pgfpathlineto{\pgfqpoint{0.939708in}{1.112613in}}%
\pgfpathlineto{\pgfqpoint{0.948528in}{1.121203in}}%
\pgfpathlineto{\pgfqpoint{0.955364in}{1.127458in}}%
\pgfpathlineto{\pgfqpoint{0.963307in}{1.134814in}}%
\pgfpathlineto{\pgfqpoint{0.971021in}{1.141680in}}%
\pgfpathlineto{\pgfqpoint{0.978624in}{1.148425in}}%
\pgfpathlineto{\pgfqpoint{0.986678in}{1.155427in}}%
\pgfpathlineto{\pgfqpoint{0.994437in}{1.162036in}}%
\pgfpathlineto{\pgfqpoint{1.002334in}{1.168742in}}%
\pgfpathlineto{\pgfqpoint{1.010796in}{1.175647in}}%
\pgfpathlineto{\pgfqpoint{1.017991in}{1.181591in}}%
\pgfpathlineto{\pgfqpoint{1.027871in}{1.189259in}}%
\pgfpathlineto{\pgfqpoint{1.033647in}{1.193862in}}%
\pgfpathlineto{\pgfqpoint{1.046016in}{1.202870in}}%
\pgfpathlineto{\pgfqpoint{1.049304in}{1.205364in}}%
\pgfpathlineto{\pgfqpoint{1.064960in}{1.215824in}}%
\pgfpathlineto{\pgfqpoint{1.066134in}{1.216481in}}%
\pgfpathlineto{\pgfqpoint{1.080617in}{1.225014in}}%
\pgfpathlineto{\pgfqpoint{1.091617in}{1.230092in}}%
\pgfpathlineto{\pgfqpoint{1.096274in}{1.232388in}}%
\pgfpathlineto{\pgfqpoint{1.111930in}{1.237671in}}%
\pgfpathlineto{\pgfqpoint{1.127587in}{1.240370in}}%
\pgfpathlineto{\pgfqpoint{1.143243in}{1.240370in}}%
\pgfpathlineto{\pgfqpoint{1.158900in}{1.237671in}}%
\pgfpathlineto{\pgfqpoint{1.174556in}{1.232388in}}%
\pgfpathlineto{\pgfqpoint{1.179213in}{1.230092in}}%
\pgfpathlineto{\pgfqpoint{1.190213in}{1.225014in}}%
\pgfpathlineto{\pgfqpoint{1.204696in}{1.216481in}}%
\pgfpathlineto{\pgfqpoint{1.205870in}{1.215824in}}%
\pgfpathlineto{\pgfqpoint{1.221526in}{1.205364in}}%
\pgfpathlineto{\pgfqpoint{1.224814in}{1.202870in}}%
\pgfpathlineto{\pgfqpoint{1.237183in}{1.193862in}}%
\pgfpathlineto{\pgfqpoint{1.242959in}{1.189259in}}%
\pgfpathlineto{\pgfqpoint{1.252839in}{1.181591in}}%
\pgfpathlineto{\pgfqpoint{1.260034in}{1.175647in}}%
\pgfpathlineto{\pgfqpoint{1.268496in}{1.168742in}}%
\pgfpathlineto{\pgfqpoint{1.276393in}{1.162036in}}%
\pgfpathlineto{\pgfqpoint{1.284152in}{1.155427in}}%
\pgfpathlineto{\pgfqpoint{1.292206in}{1.148425in}}%
\pgfpathlineto{\pgfqpoint{1.299809in}{1.141680in}}%
\pgfpathlineto{\pgfqpoint{1.307523in}{1.134814in}}%
\pgfpathlineto{\pgfqpoint{1.315466in}{1.127458in}}%
\pgfpathlineto{\pgfqpoint{1.322302in}{1.121203in}}%
\pgfpathlineto{\pgfqpoint{1.331122in}{1.112613in}}%
\pgfpathlineto{\pgfqpoint{1.336418in}{1.107592in}}%
\pgfpathlineto{\pgfqpoint{1.346779in}{1.096839in}}%
\pgfpathlineto{\pgfqpoint{1.349648in}{1.093981in}}%
\pgfpathlineto{\pgfqpoint{1.361680in}{1.080370in}}%
\pgfpathlineto{\pgfqpoint{1.362435in}{1.079350in}}%
\pgfpathlineto{\pgfqpoint{1.372251in}{1.066759in}}%
\pgfpathlineto{\pgfqpoint{1.378092in}{1.057196in}}%
\pgfpathlineto{\pgfqpoint{1.380733in}{1.053148in}}%
\pgfpathlineto{\pgfqpoint{1.386810in}{1.039536in}}%
\pgfpathlineto{\pgfqpoint{1.389914in}{1.025925in}}%
\pgfpathlineto{\pgfqpoint{1.389914in}{1.012314in}}%
\pgfpathlineto{\pgfqpoint{1.386810in}{0.998703in}}%
\pgfpathlineto{\pgfqpoint{1.380733in}{0.985092in}}%
\pgfpathlineto{\pgfqpoint{1.378092in}{0.981044in}}%
\pgfpathlineto{\pgfqpoint{1.372251in}{0.971481in}}%
\pgfpathlineto{\pgfqpoint{1.362435in}{0.958890in}}%
\pgfpathlineto{\pgfqpoint{1.361680in}{0.957870in}}%
\pgfpathlineto{\pgfqpoint{1.349648in}{0.944259in}}%
\pgfpathlineto{\pgfqpoint{1.346779in}{0.941400in}}%
\pgfpathlineto{\pgfqpoint{1.336418in}{0.930648in}}%
\pgfpathlineto{\pgfqpoint{1.331122in}{0.925626in}}%
\pgfpathlineto{\pgfqpoint{1.322302in}{0.917036in}}%
\pgfpathlineto{\pgfqpoint{1.315466in}{0.910782in}}%
\pgfpathlineto{\pgfqpoint{1.307523in}{0.903425in}}%
\pgfpathlineto{\pgfqpoint{1.299809in}{0.896560in}}%
\pgfpathlineto{\pgfqpoint{1.292206in}{0.889814in}}%
\pgfpathlineto{\pgfqpoint{1.284152in}{0.882813in}}%
\pgfpathlineto{\pgfqpoint{1.276393in}{0.876203in}}%
\pgfpathlineto{\pgfqpoint{1.268496in}{0.869497in}}%
\pgfpathlineto{\pgfqpoint{1.260034in}{0.862592in}}%
\pgfpathlineto{\pgfqpoint{1.252839in}{0.856648in}}%
\pgfpathlineto{\pgfqpoint{1.242959in}{0.848981in}}%
\pgfpathlineto{\pgfqpoint{1.237183in}{0.844377in}}%
\pgfpathlineto{\pgfqpoint{1.224814in}{0.835370in}}%
\pgfpathlineto{\pgfqpoint{1.221526in}{0.832876in}}%
\pgfpathlineto{\pgfqpoint{1.205870in}{0.822415in}}%
\pgfpathlineto{\pgfqpoint{1.204696in}{0.821759in}}%
\pgfpathlineto{\pgfqpoint{1.190213in}{0.813225in}}%
\pgfpathlineto{\pgfqpoint{1.179213in}{0.808148in}}%
\pgfpathlineto{\pgfqpoint{1.174556in}{0.805852in}}%
\pgfpathlineto{\pgfqpoint{1.158900in}{0.800569in}}%
\pgfpathlineto{\pgfqpoint{1.143243in}{0.797870in}}%
\pgfpathlineto{\pgfqpoint{1.127587in}{0.797870in}}%
\pgfpathlineto{\pgfqpoint{1.111930in}{0.800569in}}%
\pgfpathlineto{\pgfqpoint{1.096274in}{0.805852in}}%
\pgfpathlineto{\pgfqpoint{1.091617in}{0.808148in}}%
\pgfpathclose%
\pgfpathmoveto{\pgfqpoint{0.706015in}{1.066759in}}%
\pgfpathlineto{\pgfqpoint{0.704859in}{1.067147in}}%
\pgfpathlineto{\pgfqpoint{0.689203in}{1.074705in}}%
\pgfpathlineto{\pgfqpoint{0.680192in}{1.080370in}}%
\pgfpathlineto{\pgfqpoint{0.673546in}{1.083837in}}%
\pgfpathlineto{\pgfqpoint{0.657890in}{1.093862in}}%
\pgfpathlineto{\pgfqpoint{0.657731in}{1.093981in}}%
\pgfpathlineto{\pgfqpoint{0.642233in}{1.104127in}}%
\pgfpathlineto{\pgfqpoint{0.637600in}{1.107592in}}%
\pgfpathlineto{\pgfqpoint{0.626577in}{1.115064in}}%
\pgfpathlineto{\pgfqpoint{0.618460in}{1.121203in}}%
\pgfpathlineto{\pgfqpoint{0.610920in}{1.126525in}}%
\pgfpathlineto{\pgfqpoint{0.600184in}{1.134814in}}%
\pgfpathlineto{\pgfqpoint{0.595263in}{1.138447in}}%
\pgfpathlineto{\pgfqpoint{0.582682in}{1.148425in}}%
\pgfpathlineto{\pgfqpoint{0.579607in}{1.150808in}}%
\pgfpathlineto{\pgfqpoint{0.565887in}{1.162036in}}%
\pgfpathlineto{\pgfqpoint{0.563950in}{1.163615in}}%
\pgfpathlineto{\pgfqpoint{0.549740in}{1.175647in}}%
\pgfpathlineto{\pgfqpoint{0.548294in}{1.176888in}}%
\pgfpathlineto{\pgfqpoint{0.534189in}{1.189259in}}%
\pgfpathlineto{\pgfqpoint{0.532637in}{1.190661in}}%
\pgfpathlineto{\pgfqpoint{0.519193in}{1.202870in}}%
\pgfpathlineto{\pgfqpoint{0.516981in}{1.204975in}}%
\pgfpathlineto{\pgfqpoint{0.504720in}{1.216481in}}%
\pgfpathlineto{\pgfqpoint{0.501324in}{1.219876in}}%
\pgfpathlineto{\pgfqpoint{0.490754in}{1.230092in}}%
\pgfpathlineto{\pgfqpoint{0.485668in}{1.235420in}}%
\pgfpathlineto{\pgfqpoint{0.477301in}{1.243703in}}%
\pgfpathlineto{\pgfqpoint{0.470011in}{1.251673in}}%
\pgfpathlineto{\pgfqpoint{0.464405in}{1.257314in}}%
\pgfpathlineto{\pgfqpoint{0.454354in}{1.268725in}}%
\pgfpathlineto{\pgfqpoint{0.452180in}{1.270925in}}%
\pgfpathlineto{\pgfqpoint{0.440532in}{1.284536in}}%
\pgfpathlineto{\pgfqpoint{0.438698in}{1.287103in}}%
\pgfpathlineto{\pgfqpoint{0.429438in}{1.298148in}}%
\pgfpathlineto{\pgfqpoint{0.423041in}{1.307765in}}%
\pgfpathlineto{\pgfqpoint{0.419734in}{1.311759in}}%
\pgfpathlineto{\pgfqpoint{0.411415in}{1.325370in}}%
\pgfpathlineto{\pgfqpoint{0.407385in}{1.335727in}}%
\pgfpathlineto{\pgfqpoint{0.405649in}{1.338981in}}%
\pgfpathlineto{\pgfqpoint{0.402527in}{1.352592in}}%
\pgfpathlineto{\pgfqpoint{0.403568in}{1.366203in}}%
\pgfpathlineto{\pgfqpoint{0.407385in}{1.376196in}}%
\pgfpathlineto{\pgfqpoint{0.408392in}{1.379814in}}%
\pgfpathlineto{\pgfqpoint{0.415195in}{1.393425in}}%
\pgfpathlineto{\pgfqpoint{0.423041in}{1.404302in}}%
\pgfpathlineto{\pgfqpoint{0.424624in}{1.407036in}}%
\pgfpathlineto{\pgfqpoint{0.434857in}{1.420648in}}%
\pgfpathlineto{\pgfqpoint{0.438698in}{1.424824in}}%
\pgfpathlineto{\pgfqpoint{0.446103in}{1.434259in}}%
\pgfpathlineto{\pgfqpoint{0.454354in}{1.443185in}}%
\pgfpathlineto{\pgfqpoint{0.458209in}{1.447870in}}%
\pgfpathlineto{\pgfqpoint{0.470011in}{1.460413in}}%
\pgfpathlineto{\pgfqpoint{0.470935in}{1.461481in}}%
\pgfpathlineto{\pgfqpoint{0.484049in}{1.475092in}}%
\pgfpathlineto{\pgfqpoint{0.485668in}{1.476631in}}%
\pgfpathlineto{\pgfqpoint{0.497675in}{1.488703in}}%
\pgfpathlineto{\pgfqpoint{0.501324in}{1.492120in}}%
\pgfpathlineto{\pgfqpoint{0.511859in}{1.502314in}}%
\pgfpathlineto{\pgfqpoint{0.516981in}{1.507007in}}%
\pgfpathlineto{\pgfqpoint{0.526588in}{1.515925in}}%
\pgfpathlineto{\pgfqpoint{0.532637in}{1.521329in}}%
\pgfpathlineto{\pgfqpoint{0.541871in}{1.529536in}}%
\pgfpathlineto{\pgfqpoint{0.548294in}{1.535120in}}%
\pgfpathlineto{\pgfqpoint{0.557734in}{1.543148in}}%
\pgfpathlineto{\pgfqpoint{0.563950in}{1.548407in}}%
\pgfpathlineto{\pgfqpoint{0.574208in}{1.556759in}}%
\pgfpathlineto{\pgfqpoint{0.579607in}{1.561211in}}%
\pgfpathlineto{\pgfqpoint{0.591333in}{1.570370in}}%
\pgfpathlineto{\pgfqpoint{0.595263in}{1.573542in}}%
\pgfpathlineto{\pgfqpoint{0.609150in}{1.583981in}}%
\pgfpathlineto{\pgfqpoint{0.610920in}{1.585387in}}%
\pgfpathlineto{\pgfqpoint{0.626577in}{1.596789in}}%
\pgfpathlineto{\pgfqpoint{0.627805in}{1.597592in}}%
\pgfpathlineto{\pgfqpoint{0.642233in}{1.607852in}}%
\pgfpathlineto{\pgfqpoint{0.647622in}{1.611203in}}%
\pgfpathlineto{\pgfqpoint{0.657890in}{1.618377in}}%
\pgfpathlineto{\pgfqpoint{0.668742in}{1.624814in}}%
\pgfpathlineto{\pgfqpoint{0.673546in}{1.628154in}}%
\pgfpathlineto{\pgfqpoint{0.689203in}{1.637050in}}%
\pgfpathlineto{\pgfqpoint{0.692348in}{1.638425in}}%
\pgfpathlineto{\pgfqpoint{0.704859in}{1.645246in}}%
\pgfpathlineto{\pgfqpoint{0.720516in}{1.651161in}}%
\pgfpathlineto{\pgfqpoint{0.724678in}{1.652036in}}%
\pgfpathlineto{\pgfqpoint{0.736173in}{1.655355in}}%
\pgfpathlineto{\pgfqpoint{0.751829in}{1.656259in}}%
\pgfpathlineto{\pgfqpoint{0.767486in}{1.653545in}}%
\pgfpathlineto{\pgfqpoint{0.771229in}{1.652036in}}%
\pgfpathlineto{\pgfqpoint{0.783142in}{1.648533in}}%
\pgfpathlineto{\pgfqpoint{0.798799in}{1.641300in}}%
\pgfpathlineto{\pgfqpoint{0.803393in}{1.638425in}}%
\pgfpathlineto{\pgfqpoint{0.814455in}{1.632864in}}%
\pgfpathlineto{\pgfqpoint{0.827160in}{1.624814in}}%
\pgfpathlineto{\pgfqpoint{0.830112in}{1.623220in}}%
\pgfpathlineto{\pgfqpoint{0.845769in}{1.613093in}}%
\pgfpathlineto{\pgfqpoint{0.848299in}{1.611203in}}%
\pgfpathlineto{\pgfqpoint{0.861425in}{1.602466in}}%
\pgfpathlineto{\pgfqpoint{0.867914in}{1.597592in}}%
\pgfpathlineto{\pgfqpoint{0.877082in}{1.591255in}}%
\pgfpathlineto{\pgfqpoint{0.886610in}{1.583981in}}%
\pgfpathlineto{\pgfqpoint{0.892738in}{1.579559in}}%
\pgfpathlineto{\pgfqpoint{0.904489in}{1.570370in}}%
\pgfpathlineto{\pgfqpoint{0.908395in}{1.567417in}}%
\pgfpathlineto{\pgfqpoint{0.921630in}{1.556759in}}%
\pgfpathlineto{\pgfqpoint{0.924051in}{1.554835in}}%
\pgfpathlineto{\pgfqpoint{0.938094in}{1.543148in}}%
\pgfpathlineto{\pgfqpoint{0.939708in}{1.541798in}}%
\pgfpathlineto{\pgfqpoint{0.953937in}{1.529536in}}%
\pgfpathlineto{\pgfqpoint{0.955364in}{1.528279in}}%
\pgfpathlineto{\pgfqpoint{0.969206in}{1.515925in}}%
\pgfpathlineto{\pgfqpoint{0.971021in}{1.514242in}}%
\pgfpathlineto{\pgfqpoint{0.983937in}{1.502314in}}%
\pgfpathlineto{\pgfqpoint{0.986678in}{1.499641in}}%
\pgfpathlineto{\pgfqpoint{0.998156in}{1.488703in}}%
\pgfpathlineto{\pgfqpoint{1.002334in}{1.484426in}}%
\pgfpathlineto{\pgfqpoint{1.011868in}{1.475092in}}%
\pgfpathlineto{\pgfqpoint{1.017991in}{1.468537in}}%
\pgfpathlineto{\pgfqpoint{1.025052in}{1.461481in}}%
\pgfpathlineto{\pgfqpoint{1.033647in}{1.451898in}}%
\pgfpathlineto{\pgfqpoint{1.037633in}{1.447870in}}%
\pgfpathlineto{\pgfqpoint{1.049304in}{1.434397in}}%
\pgfpathlineto{\pgfqpoint{1.049441in}{1.434259in}}%
\pgfpathlineto{\pgfqpoint{1.060972in}{1.420648in}}%
\pgfpathlineto{\pgfqpoint{1.064960in}{1.414870in}}%
\pgfpathlineto{\pgfqpoint{1.071476in}{1.407036in}}%
\pgfpathlineto{\pgfqpoint{1.080171in}{1.393425in}}%
\pgfpathlineto{\pgfqpoint{1.080617in}{1.392421in}}%
\pgfpathlineto{\pgfqpoint{1.087902in}{1.379814in}}%
\pgfpathlineto{\pgfqpoint{1.092271in}{1.366203in}}%
\pgfpathlineto{\pgfqpoint{1.093144in}{1.352592in}}%
\pgfpathlineto{\pgfqpoint{1.090524in}{1.338981in}}%
\pgfpathlineto{\pgfqpoint{1.084405in}{1.325370in}}%
\pgfpathlineto{\pgfqpoint{1.080617in}{1.319985in}}%
\pgfpathlineto{\pgfqpoint{1.076159in}{1.311759in}}%
\pgfpathlineto{\pgfqpoint{1.066122in}{1.298148in}}%
\pgfpathlineto{\pgfqpoint{1.064960in}{1.296890in}}%
\pgfpathlineto{\pgfqpoint{1.055482in}{1.284536in}}%
\pgfpathlineto{\pgfqpoint{1.049304in}{1.277824in}}%
\pgfpathlineto{\pgfqpoint{1.043762in}{1.270925in}}%
\pgfpathlineto{\pgfqpoint{1.033647in}{1.260066in}}%
\pgfpathlineto{\pgfqpoint{1.031327in}{1.257314in}}%
\pgfpathlineto{\pgfqpoint{1.018369in}{1.243703in}}%
\pgfpathlineto{\pgfqpoint{1.017991in}{1.243341in}}%
\pgfpathlineto{\pgfqpoint{1.005040in}{1.230092in}}%
\pgfpathlineto{\pgfqpoint{1.002334in}{1.227536in}}%
\pgfpathlineto{\pgfqpoint{0.991132in}{1.216481in}}%
\pgfpathlineto{\pgfqpoint{0.986678in}{1.212353in}}%
\pgfpathlineto{\pgfqpoint{0.976675in}{1.202870in}}%
\pgfpathlineto{\pgfqpoint{0.971021in}{1.197753in}}%
\pgfpathlineto{\pgfqpoint{0.961671in}{1.189259in}}%
\pgfpathlineto{\pgfqpoint{0.955364in}{1.183700in}}%
\pgfpathlineto{\pgfqpoint{0.946102in}{1.175647in}}%
\pgfpathlineto{\pgfqpoint{0.939708in}{1.170165in}}%
\pgfpathlineto{\pgfqpoint{0.929938in}{1.162036in}}%
\pgfpathlineto{\pgfqpoint{0.924051in}{1.157121in}}%
\pgfpathlineto{\pgfqpoint{0.913143in}{1.148425in}}%
\pgfpathlineto{\pgfqpoint{0.908395in}{1.144553in}}%
\pgfpathlineto{\pgfqpoint{0.895678in}{1.134814in}}%
\pgfpathlineto{\pgfqpoint{0.892738in}{1.132462in}}%
\pgfpathlineto{\pgfqpoint{0.877498in}{1.121203in}}%
\pgfpathlineto{\pgfqpoint{0.877082in}{1.120874in}}%
\pgfpathlineto{\pgfqpoint{0.861425in}{1.109610in}}%
\pgfpathlineto{\pgfqpoint{0.858260in}{1.107592in}}%
\pgfpathlineto{\pgfqpoint{0.845769in}{1.098799in}}%
\pgfpathlineto{\pgfqpoint{0.837833in}{1.093981in}}%
\pgfpathlineto{\pgfqpoint{0.830112in}{1.088610in}}%
\pgfpathlineto{\pgfqpoint{0.815902in}{1.080370in}}%
\pgfpathlineto{\pgfqpoint{0.814455in}{1.079360in}}%
\pgfpathlineto{\pgfqpoint{0.798799in}{1.070634in}}%
\pgfpathlineto{\pgfqpoint{0.789337in}{1.066759in}}%
\pgfpathlineto{\pgfqpoint{0.783142in}{1.063465in}}%
\pgfpathlineto{\pgfqpoint{0.767486in}{1.058146in}}%
\pgfpathlineto{\pgfqpoint{0.751829in}{1.055868in}}%
\pgfpathlineto{\pgfqpoint{0.736173in}{1.056627in}}%
\pgfpathlineto{\pgfqpoint{0.720516in}{1.060425in}}%
\pgfpathlineto{\pgfqpoint{0.706015in}{1.066759in}}%
\pgfpathclose%
\pgfpathmoveto{\pgfqpoint{1.481493in}{1.066759in}}%
\pgfpathlineto{\pgfqpoint{1.472031in}{1.070634in}}%
\pgfpathlineto{\pgfqpoint{1.456375in}{1.079360in}}%
\pgfpathlineto{\pgfqpoint{1.454928in}{1.080370in}}%
\pgfpathlineto{\pgfqpoint{1.440718in}{1.088610in}}%
\pgfpathlineto{\pgfqpoint{1.432997in}{1.093981in}}%
\pgfpathlineto{\pgfqpoint{1.425061in}{1.098799in}}%
\pgfpathlineto{\pgfqpoint{1.412570in}{1.107592in}}%
\pgfpathlineto{\pgfqpoint{1.409405in}{1.109610in}}%
\pgfpathlineto{\pgfqpoint{1.393748in}{1.120874in}}%
\pgfpathlineto{\pgfqpoint{1.393332in}{1.121203in}}%
\pgfpathlineto{\pgfqpoint{1.378092in}{1.132462in}}%
\pgfpathlineto{\pgfqpoint{1.375152in}{1.134814in}}%
\pgfpathlineto{\pgfqpoint{1.362435in}{1.144553in}}%
\pgfpathlineto{\pgfqpoint{1.357687in}{1.148425in}}%
\pgfpathlineto{\pgfqpoint{1.346779in}{1.157121in}}%
\pgfpathlineto{\pgfqpoint{1.340892in}{1.162036in}}%
\pgfpathlineto{\pgfqpoint{1.331122in}{1.170165in}}%
\pgfpathlineto{\pgfqpoint{1.324728in}{1.175647in}}%
\pgfpathlineto{\pgfqpoint{1.315466in}{1.183700in}}%
\pgfpathlineto{\pgfqpoint{1.309159in}{1.189259in}}%
\pgfpathlineto{\pgfqpoint{1.299809in}{1.197753in}}%
\pgfpathlineto{\pgfqpoint{1.294155in}{1.202870in}}%
\pgfpathlineto{\pgfqpoint{1.284152in}{1.212353in}}%
\pgfpathlineto{\pgfqpoint{1.279698in}{1.216481in}}%
\pgfpathlineto{\pgfqpoint{1.268496in}{1.227536in}}%
\pgfpathlineto{\pgfqpoint{1.265790in}{1.230092in}}%
\pgfpathlineto{\pgfqpoint{1.252839in}{1.243341in}}%
\pgfpathlineto{\pgfqpoint{1.252461in}{1.243703in}}%
\pgfpathlineto{\pgfqpoint{1.239503in}{1.257314in}}%
\pgfpathlineto{\pgfqpoint{1.237183in}{1.260066in}}%
\pgfpathlineto{\pgfqpoint{1.227068in}{1.270925in}}%
\pgfpathlineto{\pgfqpoint{1.221526in}{1.277824in}}%
\pgfpathlineto{\pgfqpoint{1.215348in}{1.284536in}}%
\pgfpathlineto{\pgfqpoint{1.205870in}{1.296890in}}%
\pgfpathlineto{\pgfqpoint{1.204708in}{1.298148in}}%
\pgfpathlineto{\pgfqpoint{1.194671in}{1.311759in}}%
\pgfpathlineto{\pgfqpoint{1.190213in}{1.319985in}}%
\pgfpathlineto{\pgfqpoint{1.186425in}{1.325370in}}%
\pgfpathlineto{\pgfqpoint{1.180306in}{1.338981in}}%
\pgfpathlineto{\pgfqpoint{1.177686in}{1.352592in}}%
\pgfpathlineto{\pgfqpoint{1.178559in}{1.366203in}}%
\pgfpathlineto{\pgfqpoint{1.182928in}{1.379814in}}%
\pgfpathlineto{\pgfqpoint{1.190213in}{1.392421in}}%
\pgfpathlineto{\pgfqpoint{1.190659in}{1.393425in}}%
\pgfpathlineto{\pgfqpoint{1.199354in}{1.407036in}}%
\pgfpathlineto{\pgfqpoint{1.205870in}{1.414870in}}%
\pgfpathlineto{\pgfqpoint{1.209858in}{1.420648in}}%
\pgfpathlineto{\pgfqpoint{1.221389in}{1.434259in}}%
\pgfpathlineto{\pgfqpoint{1.221526in}{1.434397in}}%
\pgfpathlineto{\pgfqpoint{1.233197in}{1.447870in}}%
\pgfpathlineto{\pgfqpoint{1.237183in}{1.451898in}}%
\pgfpathlineto{\pgfqpoint{1.245778in}{1.461481in}}%
\pgfpathlineto{\pgfqpoint{1.252839in}{1.468537in}}%
\pgfpathlineto{\pgfqpoint{1.258962in}{1.475092in}}%
\pgfpathlineto{\pgfqpoint{1.268496in}{1.484426in}}%
\pgfpathlineto{\pgfqpoint{1.272674in}{1.488703in}}%
\pgfpathlineto{\pgfqpoint{1.284152in}{1.499641in}}%
\pgfpathlineto{\pgfqpoint{1.286893in}{1.502314in}}%
\pgfpathlineto{\pgfqpoint{1.299809in}{1.514242in}}%
\pgfpathlineto{\pgfqpoint{1.301624in}{1.515925in}}%
\pgfpathlineto{\pgfqpoint{1.315466in}{1.528279in}}%
\pgfpathlineto{\pgfqpoint{1.316893in}{1.529536in}}%
\pgfpathlineto{\pgfqpoint{1.331122in}{1.541798in}}%
\pgfpathlineto{\pgfqpoint{1.332736in}{1.543148in}}%
\pgfpathlineto{\pgfqpoint{1.346779in}{1.554835in}}%
\pgfpathlineto{\pgfqpoint{1.349200in}{1.556759in}}%
\pgfpathlineto{\pgfqpoint{1.362435in}{1.567417in}}%
\pgfpathlineto{\pgfqpoint{1.366341in}{1.570370in}}%
\pgfpathlineto{\pgfqpoint{1.378092in}{1.579559in}}%
\pgfpathlineto{\pgfqpoint{1.384220in}{1.583981in}}%
\pgfpathlineto{\pgfqpoint{1.393748in}{1.591255in}}%
\pgfpathlineto{\pgfqpoint{1.402916in}{1.597592in}}%
\pgfpathlineto{\pgfqpoint{1.409405in}{1.602466in}}%
\pgfpathlineto{\pgfqpoint{1.422531in}{1.611203in}}%
\pgfpathlineto{\pgfqpoint{1.425061in}{1.613093in}}%
\pgfpathlineto{\pgfqpoint{1.440718in}{1.623220in}}%
\pgfpathlineto{\pgfqpoint{1.443670in}{1.624814in}}%
\pgfpathlineto{\pgfqpoint{1.456375in}{1.632864in}}%
\pgfpathlineto{\pgfqpoint{1.467437in}{1.638425in}}%
\pgfpathlineto{\pgfqpoint{1.472031in}{1.641300in}}%
\pgfpathlineto{\pgfqpoint{1.487688in}{1.648533in}}%
\pgfpathlineto{\pgfqpoint{1.499601in}{1.652036in}}%
\pgfpathlineto{\pgfqpoint{1.503344in}{1.653545in}}%
\pgfpathlineto{\pgfqpoint{1.519001in}{1.656259in}}%
\pgfpathlineto{\pgfqpoint{1.534657in}{1.655355in}}%
\pgfpathlineto{\pgfqpoint{1.546152in}{1.652036in}}%
\pgfpathlineto{\pgfqpoint{1.550314in}{1.651161in}}%
\pgfpathlineto{\pgfqpoint{1.565971in}{1.645246in}}%
\pgfpathlineto{\pgfqpoint{1.578482in}{1.638425in}}%
\pgfpathlineto{\pgfqpoint{1.581627in}{1.637050in}}%
\pgfpathlineto{\pgfqpoint{1.597284in}{1.628154in}}%
\pgfpathlineto{\pgfqpoint{1.602088in}{1.624814in}}%
\pgfpathlineto{\pgfqpoint{1.612940in}{1.618377in}}%
\pgfpathlineto{\pgfqpoint{1.623208in}{1.611203in}}%
\pgfpathlineto{\pgfqpoint{1.628597in}{1.607852in}}%
\pgfpathlineto{\pgfqpoint{1.643025in}{1.597592in}}%
\pgfpathlineto{\pgfqpoint{1.644253in}{1.596789in}}%
\pgfpathlineto{\pgfqpoint{1.659910in}{1.585387in}}%
\pgfpathlineto{\pgfqpoint{1.661680in}{1.583981in}}%
\pgfpathlineto{\pgfqpoint{1.675567in}{1.573542in}}%
\pgfpathlineto{\pgfqpoint{1.679497in}{1.570370in}}%
\pgfpathlineto{\pgfqpoint{1.691223in}{1.561211in}}%
\pgfpathlineto{\pgfqpoint{1.696622in}{1.556759in}}%
\pgfpathlineto{\pgfqpoint{1.706880in}{1.548407in}}%
\pgfpathlineto{\pgfqpoint{1.713096in}{1.543148in}}%
\pgfpathlineto{\pgfqpoint{1.722536in}{1.535120in}}%
\pgfpathlineto{\pgfqpoint{1.728959in}{1.529536in}}%
\pgfpathlineto{\pgfqpoint{1.738193in}{1.521329in}}%
\pgfpathlineto{\pgfqpoint{1.744242in}{1.515925in}}%
\pgfpathlineto{\pgfqpoint{1.753849in}{1.507007in}}%
\pgfpathlineto{\pgfqpoint{1.758971in}{1.502314in}}%
\pgfpathlineto{\pgfqpoint{1.769506in}{1.492120in}}%
\pgfpathlineto{\pgfqpoint{1.773155in}{1.488703in}}%
\pgfpathlineto{\pgfqpoint{1.785162in}{1.476631in}}%
\pgfpathlineto{\pgfqpoint{1.786781in}{1.475092in}}%
\pgfpathlineto{\pgfqpoint{1.799895in}{1.461481in}}%
\pgfpathlineto{\pgfqpoint{1.800819in}{1.460413in}}%
\pgfpathlineto{\pgfqpoint{1.812621in}{1.447870in}}%
\pgfpathlineto{\pgfqpoint{1.816476in}{1.443185in}}%
\pgfpathlineto{\pgfqpoint{1.824727in}{1.434259in}}%
\pgfpathlineto{\pgfqpoint{1.832132in}{1.424824in}}%
\pgfpathlineto{\pgfqpoint{1.835973in}{1.420648in}}%
\pgfpathlineto{\pgfqpoint{1.846206in}{1.407036in}}%
\pgfpathlineto{\pgfqpoint{1.847789in}{1.404302in}}%
\pgfpathlineto{\pgfqpoint{1.855635in}{1.393425in}}%
\pgfpathlineto{\pgfqpoint{1.862438in}{1.379814in}}%
\pgfpathlineto{\pgfqpoint{1.863445in}{1.376196in}}%
\pgfpathlineto{\pgfqpoint{1.867262in}{1.366203in}}%
\pgfpathlineto{\pgfqpoint{1.868303in}{1.352592in}}%
\pgfpathlineto{\pgfqpoint{1.865181in}{1.338981in}}%
\pgfpathlineto{\pgfqpoint{1.863445in}{1.335727in}}%
\pgfpathlineto{\pgfqpoint{1.859415in}{1.325370in}}%
\pgfpathlineto{\pgfqpoint{1.851096in}{1.311759in}}%
\pgfpathlineto{\pgfqpoint{1.847789in}{1.307765in}}%
\pgfpathlineto{\pgfqpoint{1.841392in}{1.298148in}}%
\pgfpathlineto{\pgfqpoint{1.832132in}{1.287103in}}%
\pgfpathlineto{\pgfqpoint{1.830298in}{1.284536in}}%
\pgfpathlineto{\pgfqpoint{1.818650in}{1.270925in}}%
\pgfpathlineto{\pgfqpoint{1.816476in}{1.268725in}}%
\pgfpathlineto{\pgfqpoint{1.806425in}{1.257314in}}%
\pgfpathlineto{\pgfqpoint{1.800819in}{1.251673in}}%
\pgfpathlineto{\pgfqpoint{1.793529in}{1.243703in}}%
\pgfpathlineto{\pgfqpoint{1.785162in}{1.235420in}}%
\pgfpathlineto{\pgfqpoint{1.780076in}{1.230092in}}%
\pgfpathlineto{\pgfqpoint{1.769506in}{1.219876in}}%
\pgfpathlineto{\pgfqpoint{1.766110in}{1.216481in}}%
\pgfpathlineto{\pgfqpoint{1.753849in}{1.204975in}}%
\pgfpathlineto{\pgfqpoint{1.751637in}{1.202870in}}%
\pgfpathlineto{\pgfqpoint{1.738193in}{1.190661in}}%
\pgfpathlineto{\pgfqpoint{1.736641in}{1.189259in}}%
\pgfpathlineto{\pgfqpoint{1.722536in}{1.176888in}}%
\pgfpathlineto{\pgfqpoint{1.721090in}{1.175647in}}%
\pgfpathlineto{\pgfqpoint{1.706880in}{1.163615in}}%
\pgfpathlineto{\pgfqpoint{1.704943in}{1.162036in}}%
\pgfpathlineto{\pgfqpoint{1.691223in}{1.150808in}}%
\pgfpathlineto{\pgfqpoint{1.688148in}{1.148425in}}%
\pgfpathlineto{\pgfqpoint{1.675567in}{1.138447in}}%
\pgfpathlineto{\pgfqpoint{1.670646in}{1.134814in}}%
\pgfpathlineto{\pgfqpoint{1.659910in}{1.126525in}}%
\pgfpathlineto{\pgfqpoint{1.652370in}{1.121203in}}%
\pgfpathlineto{\pgfqpoint{1.644253in}{1.115064in}}%
\pgfpathlineto{\pgfqpoint{1.633230in}{1.107592in}}%
\pgfpathlineto{\pgfqpoint{1.628597in}{1.104127in}}%
\pgfpathlineto{\pgfqpoint{1.613099in}{1.093981in}}%
\pgfpathlineto{\pgfqpoint{1.612940in}{1.093862in}}%
\pgfpathlineto{\pgfqpoint{1.597284in}{1.083837in}}%
\pgfpathlineto{\pgfqpoint{1.590638in}{1.080370in}}%
\pgfpathlineto{\pgfqpoint{1.581627in}{1.074705in}}%
\pgfpathlineto{\pgfqpoint{1.565971in}{1.067147in}}%
\pgfpathlineto{\pgfqpoint{1.564815in}{1.066759in}}%
\pgfpathlineto{\pgfqpoint{1.550314in}{1.060425in}}%
\pgfpathlineto{\pgfqpoint{1.534657in}{1.056627in}}%
\pgfpathlineto{\pgfqpoint{1.519001in}{1.055868in}}%
\pgfpathlineto{\pgfqpoint{1.503344in}{1.058146in}}%
\pgfpathlineto{\pgfqpoint{1.487688in}{1.063465in}}%
\pgfpathlineto{\pgfqpoint{1.481493in}{1.066759in}}%
\pgfpathclose%
\pgfusepath{fill}%
\end{pgfscope}%
\begin{pgfscope}%
\pgfpathrectangle{\pgfqpoint{0.360415in}{0.345370in}}{\pgfqpoint{1.550000in}{1.347500in}}%
\pgfusepath{clip}%
\pgfsetbuttcap%
\pgfsetroundjoin%
\definecolor{currentfill}{rgb}{0.534952,0.031217,0.650165}%
\pgfsetfillcolor{currentfill}%
\pgfsetlinewidth{0.000000pt}%
\definecolor{currentstroke}{rgb}{0.000000,0.000000,0.000000}%
\pgfsetstrokecolor{currentstroke}%
\pgfsetdash{}{0pt}%
\pgfpathmoveto{\pgfqpoint{0.548294in}{0.345370in}}%
\pgfpathlineto{\pgfqpoint{0.563950in}{0.345370in}}%
\pgfpathlineto{\pgfqpoint{0.579607in}{0.345370in}}%
\pgfpathlineto{\pgfqpoint{0.595263in}{0.345370in}}%
\pgfpathlineto{\pgfqpoint{0.610920in}{0.345370in}}%
\pgfpathlineto{\pgfqpoint{0.615330in}{0.345370in}}%
\pgfpathlineto{\pgfqpoint{0.613715in}{0.358981in}}%
\pgfpathlineto{\pgfqpoint{0.610920in}{0.366929in}}%
\pgfpathlineto{\pgfqpoint{0.609062in}{0.372592in}}%
\pgfpathlineto{\pgfqpoint{0.601798in}{0.386203in}}%
\pgfpathlineto{\pgfqpoint{0.595263in}{0.395281in}}%
\pgfpathlineto{\pgfqpoint{0.592186in}{0.399814in}}%
\pgfpathlineto{\pgfqpoint{0.580830in}{0.413425in}}%
\pgfpathlineto{\pgfqpoint{0.579607in}{0.414684in}}%
\pgfpathlineto{\pgfqpoint{0.568161in}{0.427036in}}%
\pgfpathlineto{\pgfqpoint{0.563950in}{0.431121in}}%
\pgfpathlineto{\pgfqpoint{0.554454in}{0.440648in}}%
\pgfpathlineto{\pgfqpoint{0.548294in}{0.446378in}}%
\pgfpathlineto{\pgfqpoint{0.539986in}{0.454259in}}%
\pgfpathlineto{\pgfqpoint{0.532637in}{0.460885in}}%
\pgfpathlineto{\pgfqpoint{0.524927in}{0.467870in}}%
\pgfpathlineto{\pgfqpoint{0.516981in}{0.474858in}}%
\pgfpathlineto{\pgfqpoint{0.509362in}{0.481481in}}%
\pgfpathlineto{\pgfqpoint{0.501324in}{0.488389in}}%
\pgfpathlineto{\pgfqpoint{0.493289in}{0.495092in}}%
\pgfpathlineto{\pgfqpoint{0.485668in}{0.501480in}}%
\pgfpathlineto{\pgfqpoint{0.476602in}{0.508703in}}%
\pgfpathlineto{\pgfqpoint{0.470011in}{0.514059in}}%
\pgfpathlineto{\pgfqpoint{0.459052in}{0.522314in}}%
\pgfpathlineto{\pgfqpoint{0.454354in}{0.525975in}}%
\pgfpathlineto{\pgfqpoint{0.440146in}{0.535925in}}%
\pgfpathlineto{\pgfqpoint{0.438698in}{0.536989in}}%
\pgfpathlineto{\pgfqpoint{0.423041in}{0.546861in}}%
\pgfpathlineto{\pgfqpoint{0.417826in}{0.549536in}}%
\pgfpathlineto{\pgfqpoint{0.407385in}{0.555217in}}%
\pgfpathlineto{\pgfqpoint{0.391728in}{0.561532in}}%
\pgfpathlineto{\pgfqpoint{0.385214in}{0.563148in}}%
\pgfpathlineto{\pgfqpoint{0.376072in}{0.565578in}}%
\pgfpathlineto{\pgfqpoint{0.360415in}{0.566981in}}%
\pgfpathlineto{\pgfqpoint{0.360415in}{0.563148in}}%
\pgfpathlineto{\pgfqpoint{0.360415in}{0.549536in}}%
\pgfpathlineto{\pgfqpoint{0.360415in}{0.535925in}}%
\pgfpathlineto{\pgfqpoint{0.360415in}{0.522314in}}%
\pgfpathlineto{\pgfqpoint{0.360415in}{0.508703in}}%
\pgfpathlineto{\pgfqpoint{0.360415in}{0.498296in}}%
\pgfpathlineto{\pgfqpoint{0.376072in}{0.497222in}}%
\pgfpathlineto{\pgfqpoint{0.386554in}{0.495092in}}%
\pgfpathlineto{\pgfqpoint{0.391728in}{0.494023in}}%
\pgfpathlineto{\pgfqpoint{0.407385in}{0.488761in}}%
\pgfpathlineto{\pgfqpoint{0.423041in}{0.481698in}}%
\pgfpathlineto{\pgfqpoint{0.423430in}{0.481481in}}%
\pgfpathlineto{\pgfqpoint{0.438698in}{0.472764in}}%
\pgfpathlineto{\pgfqpoint{0.446114in}{0.467870in}}%
\pgfpathlineto{\pgfqpoint{0.454354in}{0.462228in}}%
\pgfpathlineto{\pgfqpoint{0.464821in}{0.454259in}}%
\pgfpathlineto{\pgfqpoint{0.470011in}{0.450083in}}%
\pgfpathlineto{\pgfqpoint{0.480865in}{0.440648in}}%
\pgfpathlineto{\pgfqpoint{0.485668in}{0.436136in}}%
\pgfpathlineto{\pgfqpoint{0.494834in}{0.427036in}}%
\pgfpathlineto{\pgfqpoint{0.501324in}{0.419873in}}%
\pgfpathlineto{\pgfqpoint{0.506954in}{0.413425in}}%
\pgfpathlineto{\pgfqpoint{0.516981in}{0.400152in}}%
\pgfpathlineto{\pgfqpoint{0.517231in}{0.399814in}}%
\pgfpathlineto{\pgfqpoint{0.525354in}{0.386203in}}%
\pgfpathlineto{\pgfqpoint{0.531408in}{0.372592in}}%
\pgfpathlineto{\pgfqpoint{0.532637in}{0.368093in}}%
\pgfpathlineto{\pgfqpoint{0.535088in}{0.358981in}}%
\pgfpathlineto{\pgfqpoint{0.536323in}{0.345370in}}%
\pgfpathlineto{\pgfqpoint{0.548294in}{0.345370in}}%
\pgfpathclose%
\pgfpathmoveto{\pgfqpoint{0.892738in}{0.345370in}}%
\pgfpathlineto{\pgfqpoint{0.908395in}{0.345370in}}%
\pgfpathlineto{\pgfqpoint{0.924051in}{0.345370in}}%
\pgfpathlineto{\pgfqpoint{0.939708in}{0.345370in}}%
\pgfpathlineto{\pgfqpoint{0.955364in}{0.345370in}}%
\pgfpathlineto{\pgfqpoint{0.959517in}{0.345370in}}%
\pgfpathlineto{\pgfqpoint{0.960764in}{0.358981in}}%
\pgfpathlineto{\pgfqpoint{0.964451in}{0.372592in}}%
\pgfpathlineto{\pgfqpoint{0.970422in}{0.386203in}}%
\pgfpathlineto{\pgfqpoint{0.971021in}{0.387206in}}%
\pgfpathlineto{\pgfqpoint{0.978674in}{0.399814in}}%
\pgfpathlineto{\pgfqpoint{0.986678in}{0.410602in}}%
\pgfpathlineto{\pgfqpoint{0.988838in}{0.413425in}}%
\pgfpathlineto{\pgfqpoint{1.000907in}{0.427036in}}%
\pgfpathlineto{\pgfqpoint{1.002334in}{0.428478in}}%
\pgfpathlineto{\pgfqpoint{1.014921in}{0.440648in}}%
\pgfpathlineto{\pgfqpoint{1.017991in}{0.443397in}}%
\pgfpathlineto{\pgfqpoint{1.030933in}{0.454259in}}%
\pgfpathlineto{\pgfqpoint{1.033647in}{0.456417in}}%
\pgfpathlineto{\pgfqpoint{1.049304in}{0.467790in}}%
\pgfpathlineto{\pgfqpoint{1.049429in}{0.467870in}}%
\pgfpathlineto{\pgfqpoint{1.064960in}{0.477415in}}%
\pgfpathlineto{\pgfqpoint{1.072804in}{0.481481in}}%
\pgfpathlineto{\pgfqpoint{1.080617in}{0.485438in}}%
\pgfpathlineto{\pgfqpoint{1.096274in}{0.491632in}}%
\pgfpathlineto{\pgfqpoint{1.108959in}{0.495092in}}%
\pgfpathlineto{\pgfqpoint{1.111930in}{0.495891in}}%
\pgfpathlineto{\pgfqpoint{1.127587in}{0.498027in}}%
\pgfpathlineto{\pgfqpoint{1.143243in}{0.498027in}}%
\pgfpathlineto{\pgfqpoint{1.158900in}{0.495891in}}%
\pgfpathlineto{\pgfqpoint{1.161871in}{0.495092in}}%
\pgfpathlineto{\pgfqpoint{1.174556in}{0.491632in}}%
\pgfpathlineto{\pgfqpoint{1.190213in}{0.485438in}}%
\pgfpathlineto{\pgfqpoint{1.198026in}{0.481481in}}%
\pgfpathlineto{\pgfqpoint{1.205870in}{0.477415in}}%
\pgfpathlineto{\pgfqpoint{1.221401in}{0.467870in}}%
\pgfpathlineto{\pgfqpoint{1.221526in}{0.467790in}}%
\pgfpathlineto{\pgfqpoint{1.237183in}{0.456417in}}%
\pgfpathlineto{\pgfqpoint{1.239897in}{0.454259in}}%
\pgfpathlineto{\pgfqpoint{1.252839in}{0.443397in}}%
\pgfpathlineto{\pgfqpoint{1.255909in}{0.440648in}}%
\pgfpathlineto{\pgfqpoint{1.268496in}{0.428478in}}%
\pgfpathlineto{\pgfqpoint{1.269923in}{0.427036in}}%
\pgfpathlineto{\pgfqpoint{1.281992in}{0.413425in}}%
\pgfpathlineto{\pgfqpoint{1.284152in}{0.410602in}}%
\pgfpathlineto{\pgfqpoint{1.292156in}{0.399814in}}%
\pgfpathlineto{\pgfqpoint{1.299809in}{0.387206in}}%
\pgfpathlineto{\pgfqpoint{1.300408in}{0.386203in}}%
\pgfpathlineto{\pgfqpoint{1.306379in}{0.372592in}}%
\pgfpathlineto{\pgfqpoint{1.310066in}{0.358981in}}%
\pgfpathlineto{\pgfqpoint{1.311313in}{0.345370in}}%
\pgfpathlineto{\pgfqpoint{1.315466in}{0.345370in}}%
\pgfpathlineto{\pgfqpoint{1.331122in}{0.345370in}}%
\pgfpathlineto{\pgfqpoint{1.346779in}{0.345370in}}%
\pgfpathlineto{\pgfqpoint{1.362435in}{0.345370in}}%
\pgfpathlineto{\pgfqpoint{1.378092in}{0.345370in}}%
\pgfpathlineto{\pgfqpoint{1.390305in}{0.345370in}}%
\pgfpathlineto{\pgfqpoint{1.388745in}{0.358981in}}%
\pgfpathlineto{\pgfqpoint{1.384130in}{0.372592in}}%
\pgfpathlineto{\pgfqpoint{1.378092in}{0.383562in}}%
\pgfpathlineto{\pgfqpoint{1.376728in}{0.386203in}}%
\pgfpathlineto{\pgfqpoint{1.367210in}{0.399814in}}%
\pgfpathlineto{\pgfqpoint{1.362435in}{0.405366in}}%
\pgfpathlineto{\pgfqpoint{1.355863in}{0.413425in}}%
\pgfpathlineto{\pgfqpoint{1.346779in}{0.423042in}}%
\pgfpathlineto{\pgfqpoint{1.343152in}{0.427036in}}%
\pgfpathlineto{\pgfqpoint{1.331122in}{0.438948in}}%
\pgfpathlineto{\pgfqpoint{1.329450in}{0.440648in}}%
\pgfpathlineto{\pgfqpoint{1.315466in}{0.453825in}}%
\pgfpathlineto{\pgfqpoint{1.315010in}{0.454259in}}%
\pgfpathlineto{\pgfqpoint{1.299986in}{0.467870in}}%
\pgfpathlineto{\pgfqpoint{1.299809in}{0.468025in}}%
\pgfpathlineto{\pgfqpoint{1.284452in}{0.481481in}}%
\pgfpathlineto{\pgfqpoint{1.284152in}{0.481740in}}%
\pgfpathlineto{\pgfqpoint{1.268496in}{0.495031in}}%
\pgfpathlineto{\pgfqpoint{1.268421in}{0.495092in}}%
\pgfpathlineto{\pgfqpoint{1.252839in}{0.507848in}}%
\pgfpathlineto{\pgfqpoint{1.251728in}{0.508703in}}%
\pgfpathlineto{\pgfqpoint{1.237183in}{0.520102in}}%
\pgfpathlineto{\pgfqpoint{1.234095in}{0.522314in}}%
\pgfpathlineto{\pgfqpoint{1.221526in}{0.531617in}}%
\pgfpathlineto{\pgfqpoint{1.214930in}{0.535925in}}%
\pgfpathlineto{\pgfqpoint{1.205870in}{0.542123in}}%
\pgfpathlineto{\pgfqpoint{1.192985in}{0.549536in}}%
\pgfpathlineto{\pgfqpoint{1.190213in}{0.551230in}}%
\pgfpathlineto{\pgfqpoint{1.174556in}{0.558662in}}%
\pgfpathlineto{\pgfqpoint{1.160862in}{0.563148in}}%
\pgfpathlineto{\pgfqpoint{1.158900in}{0.563838in}}%
\pgfpathlineto{\pgfqpoint{1.143243in}{0.566629in}}%
\pgfpathlineto{\pgfqpoint{1.127587in}{0.566629in}}%
\pgfpathlineto{\pgfqpoint{1.111930in}{0.563838in}}%
\pgfpathlineto{\pgfqpoint{1.109968in}{0.563148in}}%
\pgfpathlineto{\pgfqpoint{1.096274in}{0.558662in}}%
\pgfpathlineto{\pgfqpoint{1.080617in}{0.551230in}}%
\pgfpathlineto{\pgfqpoint{1.077845in}{0.549536in}}%
\pgfpathlineto{\pgfqpoint{1.064960in}{0.542123in}}%
\pgfpathlineto{\pgfqpoint{1.055900in}{0.535925in}}%
\pgfpathlineto{\pgfqpoint{1.049304in}{0.531617in}}%
\pgfpathlineto{\pgfqpoint{1.036735in}{0.522314in}}%
\pgfpathlineto{\pgfqpoint{1.033647in}{0.520102in}}%
\pgfpathlineto{\pgfqpoint{1.019102in}{0.508703in}}%
\pgfpathlineto{\pgfqpoint{1.017991in}{0.507848in}}%
\pgfpathlineto{\pgfqpoint{1.002409in}{0.495092in}}%
\pgfpathlineto{\pgfqpoint{1.002334in}{0.495031in}}%
\pgfpathlineto{\pgfqpoint{0.986678in}{0.481740in}}%
\pgfpathlineto{\pgfqpoint{0.986378in}{0.481481in}}%
\pgfpathlineto{\pgfqpoint{0.971021in}{0.468025in}}%
\pgfpathlineto{\pgfqpoint{0.970844in}{0.467870in}}%
\pgfpathlineto{\pgfqpoint{0.955820in}{0.454259in}}%
\pgfpathlineto{\pgfqpoint{0.955364in}{0.453825in}}%
\pgfpathlineto{\pgfqpoint{0.941380in}{0.440648in}}%
\pgfpathlineto{\pgfqpoint{0.939708in}{0.438948in}}%
\pgfpathlineto{\pgfqpoint{0.927678in}{0.427036in}}%
\pgfpathlineto{\pgfqpoint{0.924051in}{0.423042in}}%
\pgfpathlineto{\pgfqpoint{0.914967in}{0.413425in}}%
\pgfpathlineto{\pgfqpoint{0.908395in}{0.405366in}}%
\pgfpathlineto{\pgfqpoint{0.903620in}{0.399814in}}%
\pgfpathlineto{\pgfqpoint{0.894102in}{0.386203in}}%
\pgfpathlineto{\pgfqpoint{0.892738in}{0.383562in}}%
\pgfpathlineto{\pgfqpoint{0.886700in}{0.372592in}}%
\pgfpathlineto{\pgfqpoint{0.882085in}{0.358981in}}%
\pgfpathlineto{\pgfqpoint{0.880525in}{0.345370in}}%
\pgfpathlineto{\pgfqpoint{0.892738in}{0.345370in}}%
\pgfpathclose%
\pgfpathmoveto{\pgfqpoint{1.659910in}{0.345370in}}%
\pgfpathlineto{\pgfqpoint{1.675567in}{0.345370in}}%
\pgfpathlineto{\pgfqpoint{1.691223in}{0.345370in}}%
\pgfpathlineto{\pgfqpoint{1.706880in}{0.345370in}}%
\pgfpathlineto{\pgfqpoint{1.722536in}{0.345370in}}%
\pgfpathlineto{\pgfqpoint{1.734507in}{0.345370in}}%
\pgfpathlineto{\pgfqpoint{1.735742in}{0.358981in}}%
\pgfpathlineto{\pgfqpoint{1.738193in}{0.368093in}}%
\pgfpathlineto{\pgfqpoint{1.739422in}{0.372592in}}%
\pgfpathlineto{\pgfqpoint{1.745476in}{0.386203in}}%
\pgfpathlineto{\pgfqpoint{1.753599in}{0.399814in}}%
\pgfpathlineto{\pgfqpoint{1.753849in}{0.400152in}}%
\pgfpathlineto{\pgfqpoint{1.763876in}{0.413425in}}%
\pgfpathlineto{\pgfqpoint{1.769506in}{0.419873in}}%
\pgfpathlineto{\pgfqpoint{1.775996in}{0.427036in}}%
\pgfpathlineto{\pgfqpoint{1.785162in}{0.436136in}}%
\pgfpathlineto{\pgfqpoint{1.789965in}{0.440648in}}%
\pgfpathlineto{\pgfqpoint{1.800819in}{0.450083in}}%
\pgfpathlineto{\pgfqpoint{1.806009in}{0.454259in}}%
\pgfpathlineto{\pgfqpoint{1.816476in}{0.462228in}}%
\pgfpathlineto{\pgfqpoint{1.824716in}{0.467870in}}%
\pgfpathlineto{\pgfqpoint{1.832132in}{0.472764in}}%
\pgfpathlineto{\pgfqpoint{1.847400in}{0.481481in}}%
\pgfpathlineto{\pgfqpoint{1.847789in}{0.481698in}}%
\pgfpathlineto{\pgfqpoint{1.863445in}{0.488761in}}%
\pgfpathlineto{\pgfqpoint{1.879102in}{0.494023in}}%
\pgfpathlineto{\pgfqpoint{1.884276in}{0.495092in}}%
\pgfpathlineto{\pgfqpoint{1.894758in}{0.497222in}}%
\pgfpathlineto{\pgfqpoint{1.910415in}{0.498296in}}%
\pgfpathlineto{\pgfqpoint{1.910415in}{0.508703in}}%
\pgfpathlineto{\pgfqpoint{1.910415in}{0.522314in}}%
\pgfpathlineto{\pgfqpoint{1.910415in}{0.535925in}}%
\pgfpathlineto{\pgfqpoint{1.910415in}{0.549536in}}%
\pgfpathlineto{\pgfqpoint{1.910415in}{0.563148in}}%
\pgfpathlineto{\pgfqpoint{1.910415in}{0.566981in}}%
\pgfpathlineto{\pgfqpoint{1.894758in}{0.565578in}}%
\pgfpathlineto{\pgfqpoint{1.885616in}{0.563148in}}%
\pgfpathlineto{\pgfqpoint{1.879102in}{0.561532in}}%
\pgfpathlineto{\pgfqpoint{1.863445in}{0.555217in}}%
\pgfpathlineto{\pgfqpoint{1.853004in}{0.549536in}}%
\pgfpathlineto{\pgfqpoint{1.847789in}{0.546861in}}%
\pgfpathlineto{\pgfqpoint{1.832132in}{0.536989in}}%
\pgfpathlineto{\pgfqpoint{1.830684in}{0.535925in}}%
\pgfpathlineto{\pgfqpoint{1.816476in}{0.525975in}}%
\pgfpathlineto{\pgfqpoint{1.811778in}{0.522314in}}%
\pgfpathlineto{\pgfqpoint{1.800819in}{0.514059in}}%
\pgfpathlineto{\pgfqpoint{1.794228in}{0.508703in}}%
\pgfpathlineto{\pgfqpoint{1.785162in}{0.501480in}}%
\pgfpathlineto{\pgfqpoint{1.777541in}{0.495092in}}%
\pgfpathlineto{\pgfqpoint{1.769506in}{0.488389in}}%
\pgfpathlineto{\pgfqpoint{1.761468in}{0.481481in}}%
\pgfpathlineto{\pgfqpoint{1.753849in}{0.474858in}}%
\pgfpathlineto{\pgfqpoint{1.745903in}{0.467870in}}%
\pgfpathlineto{\pgfqpoint{1.738193in}{0.460885in}}%
\pgfpathlineto{\pgfqpoint{1.730844in}{0.454259in}}%
\pgfpathlineto{\pgfqpoint{1.722536in}{0.446378in}}%
\pgfpathlineto{\pgfqpoint{1.716376in}{0.440648in}}%
\pgfpathlineto{\pgfqpoint{1.706880in}{0.431121in}}%
\pgfpathlineto{\pgfqpoint{1.702669in}{0.427036in}}%
\pgfpathlineto{\pgfqpoint{1.691223in}{0.414684in}}%
\pgfpathlineto{\pgfqpoint{1.690000in}{0.413425in}}%
\pgfpathlineto{\pgfqpoint{1.678644in}{0.399814in}}%
\pgfpathlineto{\pgfqpoint{1.675567in}{0.395281in}}%
\pgfpathlineto{\pgfqpoint{1.669032in}{0.386203in}}%
\pgfpathlineto{\pgfqpoint{1.661768in}{0.372592in}}%
\pgfpathlineto{\pgfqpoint{1.659910in}{0.366929in}}%
\pgfpathlineto{\pgfqpoint{1.657115in}{0.358981in}}%
\pgfpathlineto{\pgfqpoint{1.655500in}{0.345370in}}%
\pgfpathlineto{\pgfqpoint{1.659910in}{0.345370in}}%
\pgfpathclose%
\pgfpathmoveto{\pgfqpoint{0.376072in}{0.798886in}}%
\pgfpathlineto{\pgfqpoint{0.391728in}{0.802899in}}%
\pgfpathlineto{\pgfqpoint{0.404347in}{0.808148in}}%
\pgfpathlineto{\pgfqpoint{0.407385in}{0.809333in}}%
\pgfpathlineto{\pgfqpoint{0.423041in}{0.817608in}}%
\pgfpathlineto{\pgfqpoint{0.429427in}{0.821759in}}%
\pgfpathlineto{\pgfqpoint{0.438698in}{0.827473in}}%
\pgfpathlineto{\pgfqpoint{0.449760in}{0.835370in}}%
\pgfpathlineto{\pgfqpoint{0.454354in}{0.838523in}}%
\pgfpathlineto{\pgfqpoint{0.468056in}{0.848981in}}%
\pgfpathlineto{\pgfqpoint{0.470011in}{0.850435in}}%
\pgfpathlineto{\pgfqpoint{0.485168in}{0.862592in}}%
\pgfpathlineto{\pgfqpoint{0.485668in}{0.862988in}}%
\pgfpathlineto{\pgfqpoint{0.501324in}{0.876049in}}%
\pgfpathlineto{\pgfqpoint{0.501503in}{0.876203in}}%
\pgfpathlineto{\pgfqpoint{0.516981in}{0.889554in}}%
\pgfpathlineto{\pgfqpoint{0.517279in}{0.889814in}}%
\pgfpathlineto{\pgfqpoint{0.532567in}{0.903425in}}%
\pgfpathlineto{\pgfqpoint{0.532637in}{0.903490in}}%
\pgfpathlineto{\pgfqpoint{0.547310in}{0.917036in}}%
\pgfpathlineto{\pgfqpoint{0.548294in}{0.918003in}}%
\pgfpathlineto{\pgfqpoint{0.561406in}{0.930648in}}%
\pgfpathlineto{\pgfqpoint{0.563950in}{0.933332in}}%
\pgfpathlineto{\pgfqpoint{0.574651in}{0.944259in}}%
\pgfpathlineto{\pgfqpoint{0.579607in}{0.949993in}}%
\pgfpathlineto{\pgfqpoint{0.586736in}{0.957870in}}%
\pgfpathlineto{\pgfqpoint{0.595263in}{0.969071in}}%
\pgfpathlineto{\pgfqpoint{0.597212in}{0.971481in}}%
\pgfpathlineto{\pgfqpoint{0.605761in}{0.985092in}}%
\pgfpathlineto{\pgfqpoint{0.610920in}{0.996997in}}%
\pgfpathlineto{\pgfqpoint{0.611714in}{0.998703in}}%
\pgfpathlineto{\pgfqpoint{0.614925in}{1.012314in}}%
\pgfpathlineto{\pgfqpoint{0.614925in}{1.025925in}}%
\pgfpathlineto{\pgfqpoint{0.611714in}{1.039536in}}%
\pgfpathlineto{\pgfqpoint{0.610920in}{1.041242in}}%
\pgfpathlineto{\pgfqpoint{0.605761in}{1.053148in}}%
\pgfpathlineto{\pgfqpoint{0.597212in}{1.066759in}}%
\pgfpathlineto{\pgfqpoint{0.595263in}{1.069168in}}%
\pgfpathlineto{\pgfqpoint{0.586736in}{1.080370in}}%
\pgfpathlineto{\pgfqpoint{0.579607in}{1.088246in}}%
\pgfpathlineto{\pgfqpoint{0.574651in}{1.093981in}}%
\pgfpathlineto{\pgfqpoint{0.563950in}{1.104908in}}%
\pgfpathlineto{\pgfqpoint{0.561406in}{1.107592in}}%
\pgfpathlineto{\pgfqpoint{0.548294in}{1.120237in}}%
\pgfpathlineto{\pgfqpoint{0.547310in}{1.121203in}}%
\pgfpathlineto{\pgfqpoint{0.532637in}{1.134749in}}%
\pgfpathlineto{\pgfqpoint{0.532567in}{1.134814in}}%
\pgfpathlineto{\pgfqpoint{0.517279in}{1.148425in}}%
\pgfpathlineto{\pgfqpoint{0.516981in}{1.148686in}}%
\pgfpathlineto{\pgfqpoint{0.501503in}{1.162036in}}%
\pgfpathlineto{\pgfqpoint{0.501324in}{1.162190in}}%
\pgfpathlineto{\pgfqpoint{0.485668in}{1.175252in}}%
\pgfpathlineto{\pgfqpoint{0.485168in}{1.175647in}}%
\pgfpathlineto{\pgfqpoint{0.470011in}{1.187805in}}%
\pgfpathlineto{\pgfqpoint{0.468056in}{1.189259in}}%
\pgfpathlineto{\pgfqpoint{0.454354in}{1.199716in}}%
\pgfpathlineto{\pgfqpoint{0.449760in}{1.202870in}}%
\pgfpathlineto{\pgfqpoint{0.438698in}{1.210767in}}%
\pgfpathlineto{\pgfqpoint{0.429427in}{1.216481in}}%
\pgfpathlineto{\pgfqpoint{0.423041in}{1.220632in}}%
\pgfpathlineto{\pgfqpoint{0.407385in}{1.228906in}}%
\pgfpathlineto{\pgfqpoint{0.404347in}{1.230092in}}%
\pgfpathlineto{\pgfqpoint{0.391728in}{1.235341in}}%
\pgfpathlineto{\pgfqpoint{0.376072in}{1.239353in}}%
\pgfpathlineto{\pgfqpoint{0.360415in}{1.240710in}}%
\pgfpathlineto{\pgfqpoint{0.360415in}{1.230092in}}%
\pgfpathlineto{\pgfqpoint{0.360415in}{1.216481in}}%
\pgfpathlineto{\pgfqpoint{0.360415in}{1.202870in}}%
\pgfpathlineto{\pgfqpoint{0.360415in}{1.189259in}}%
\pgfpathlineto{\pgfqpoint{0.360415in}{1.175647in}}%
\pgfpathlineto{\pgfqpoint{0.360415in}{1.172037in}}%
\pgfpathlineto{\pgfqpoint{0.376072in}{1.170953in}}%
\pgfpathlineto{\pgfqpoint{0.391728in}{1.167748in}}%
\pgfpathlineto{\pgfqpoint{0.407385in}{1.162557in}}%
\pgfpathlineto{\pgfqpoint{0.408538in}{1.162036in}}%
\pgfpathlineto{\pgfqpoint{0.423041in}{1.155384in}}%
\pgfpathlineto{\pgfqpoint{0.435450in}{1.148425in}}%
\pgfpathlineto{\pgfqpoint{0.438698in}{1.146548in}}%
\pgfpathlineto{\pgfqpoint{0.454354in}{1.136054in}}%
\pgfpathlineto{\pgfqpoint{0.456013in}{1.134814in}}%
\pgfpathlineto{\pgfqpoint{0.470011in}{1.123872in}}%
\pgfpathlineto{\pgfqpoint{0.473173in}{1.121203in}}%
\pgfpathlineto{\pgfqpoint{0.485668in}{1.109952in}}%
\pgfpathlineto{\pgfqpoint{0.488150in}{1.107592in}}%
\pgfpathlineto{\pgfqpoint{0.501233in}{1.093981in}}%
\pgfpathlineto{\pgfqpoint{0.501324in}{1.093872in}}%
\pgfpathlineto{\pgfqpoint{0.512304in}{1.080370in}}%
\pgfpathlineto{\pgfqpoint{0.516981in}{1.073551in}}%
\pgfpathlineto{\pgfqpoint{0.521533in}{1.066759in}}%
\pgfpathlineto{\pgfqpoint{0.528657in}{1.053148in}}%
\pgfpathlineto{\pgfqpoint{0.532637in}{1.042119in}}%
\pgfpathlineto{\pgfqpoint{0.533556in}{1.039536in}}%
\pgfpathlineto{\pgfqpoint{0.536013in}{1.025925in}}%
\pgfpathlineto{\pgfqpoint{0.536013in}{1.012314in}}%
\pgfpathlineto{\pgfqpoint{0.533556in}{0.998703in}}%
\pgfpathlineto{\pgfqpoint{0.532637in}{0.996120in}}%
\pgfpathlineto{\pgfqpoint{0.528657in}{0.985092in}}%
\pgfpathlineto{\pgfqpoint{0.521533in}{0.971481in}}%
\pgfpathlineto{\pgfqpoint{0.516981in}{0.964688in}}%
\pgfpathlineto{\pgfqpoint{0.512304in}{0.957870in}}%
\pgfpathlineto{\pgfqpoint{0.501324in}{0.944367in}}%
\pgfpathlineto{\pgfqpoint{0.501233in}{0.944259in}}%
\pgfpathlineto{\pgfqpoint{0.488150in}{0.930648in}}%
\pgfpathlineto{\pgfqpoint{0.485668in}{0.928288in}}%
\pgfpathlineto{\pgfqpoint{0.473173in}{0.917036in}}%
\pgfpathlineto{\pgfqpoint{0.470011in}{0.914367in}}%
\pgfpathlineto{\pgfqpoint{0.456013in}{0.903425in}}%
\pgfpathlineto{\pgfqpoint{0.454354in}{0.902185in}}%
\pgfpathlineto{\pgfqpoint{0.438698in}{0.891692in}}%
\pgfpathlineto{\pgfqpoint{0.435450in}{0.889814in}}%
\pgfpathlineto{\pgfqpoint{0.423041in}{0.882856in}}%
\pgfpathlineto{\pgfqpoint{0.408538in}{0.876203in}}%
\pgfpathlineto{\pgfqpoint{0.407385in}{0.875682in}}%
\pgfpathlineto{\pgfqpoint{0.391728in}{0.870491in}}%
\pgfpathlineto{\pgfqpoint{0.376072in}{0.867286in}}%
\pgfpathlineto{\pgfqpoint{0.360415in}{0.866202in}}%
\pgfpathlineto{\pgfqpoint{0.360415in}{0.862592in}}%
\pgfpathlineto{\pgfqpoint{0.360415in}{0.848981in}}%
\pgfpathlineto{\pgfqpoint{0.360415in}{0.835370in}}%
\pgfpathlineto{\pgfqpoint{0.360415in}{0.821759in}}%
\pgfpathlineto{\pgfqpoint{0.360415in}{0.808148in}}%
\pgfpathlineto{\pgfqpoint{0.360415in}{0.797530in}}%
\pgfpathlineto{\pgfqpoint{0.376072in}{0.798886in}}%
\pgfpathclose%
\pgfpathmoveto{\pgfqpoint{1.096274in}{0.805852in}}%
\pgfpathlineto{\pgfqpoint{1.111930in}{0.800569in}}%
\pgfpathlineto{\pgfqpoint{1.127587in}{0.797870in}}%
\pgfpathlineto{\pgfqpoint{1.143243in}{0.797870in}}%
\pgfpathlineto{\pgfqpoint{1.158900in}{0.800569in}}%
\pgfpathlineto{\pgfqpoint{1.174556in}{0.805852in}}%
\pgfpathlineto{\pgfqpoint{1.179213in}{0.808148in}}%
\pgfpathlineto{\pgfqpoint{1.190213in}{0.813225in}}%
\pgfpathlineto{\pgfqpoint{1.204696in}{0.821759in}}%
\pgfpathlineto{\pgfqpoint{1.205870in}{0.822415in}}%
\pgfpathlineto{\pgfqpoint{1.221526in}{0.832876in}}%
\pgfpathlineto{\pgfqpoint{1.224814in}{0.835370in}}%
\pgfpathlineto{\pgfqpoint{1.237183in}{0.844377in}}%
\pgfpathlineto{\pgfqpoint{1.242959in}{0.848981in}}%
\pgfpathlineto{\pgfqpoint{1.252839in}{0.856648in}}%
\pgfpathlineto{\pgfqpoint{1.260034in}{0.862592in}}%
\pgfpathlineto{\pgfqpoint{1.268496in}{0.869497in}}%
\pgfpathlineto{\pgfqpoint{1.276393in}{0.876203in}}%
\pgfpathlineto{\pgfqpoint{1.284152in}{0.882813in}}%
\pgfpathlineto{\pgfqpoint{1.292206in}{0.889814in}}%
\pgfpathlineto{\pgfqpoint{1.299809in}{0.896560in}}%
\pgfpathlineto{\pgfqpoint{1.307523in}{0.903425in}}%
\pgfpathlineto{\pgfqpoint{1.315466in}{0.910782in}}%
\pgfpathlineto{\pgfqpoint{1.322302in}{0.917036in}}%
\pgfpathlineto{\pgfqpoint{1.331122in}{0.925626in}}%
\pgfpathlineto{\pgfqpoint{1.336418in}{0.930648in}}%
\pgfpathlineto{\pgfqpoint{1.346779in}{0.941400in}}%
\pgfpathlineto{\pgfqpoint{1.349648in}{0.944259in}}%
\pgfpathlineto{\pgfqpoint{1.361680in}{0.957870in}}%
\pgfpathlineto{\pgfqpoint{1.362435in}{0.958890in}}%
\pgfpathlineto{\pgfqpoint{1.372251in}{0.971481in}}%
\pgfpathlineto{\pgfqpoint{1.378092in}{0.981044in}}%
\pgfpathlineto{\pgfqpoint{1.380733in}{0.985092in}}%
\pgfpathlineto{\pgfqpoint{1.386810in}{0.998703in}}%
\pgfpathlineto{\pgfqpoint{1.389914in}{1.012314in}}%
\pgfpathlineto{\pgfqpoint{1.389914in}{1.025925in}}%
\pgfpathlineto{\pgfqpoint{1.386810in}{1.039536in}}%
\pgfpathlineto{\pgfqpoint{1.380733in}{1.053148in}}%
\pgfpathlineto{\pgfqpoint{1.378092in}{1.057196in}}%
\pgfpathlineto{\pgfqpoint{1.372251in}{1.066759in}}%
\pgfpathlineto{\pgfqpoint{1.362435in}{1.079350in}}%
\pgfpathlineto{\pgfqpoint{1.361680in}{1.080370in}}%
\pgfpathlineto{\pgfqpoint{1.349648in}{1.093981in}}%
\pgfpathlineto{\pgfqpoint{1.346779in}{1.096839in}}%
\pgfpathlineto{\pgfqpoint{1.336418in}{1.107592in}}%
\pgfpathlineto{\pgfqpoint{1.331122in}{1.112613in}}%
\pgfpathlineto{\pgfqpoint{1.322302in}{1.121203in}}%
\pgfpathlineto{\pgfqpoint{1.315466in}{1.127458in}}%
\pgfpathlineto{\pgfqpoint{1.307523in}{1.134814in}}%
\pgfpathlineto{\pgfqpoint{1.299809in}{1.141680in}}%
\pgfpathlineto{\pgfqpoint{1.292206in}{1.148425in}}%
\pgfpathlineto{\pgfqpoint{1.284152in}{1.155427in}}%
\pgfpathlineto{\pgfqpoint{1.276393in}{1.162036in}}%
\pgfpathlineto{\pgfqpoint{1.268496in}{1.168742in}}%
\pgfpathlineto{\pgfqpoint{1.260034in}{1.175647in}}%
\pgfpathlineto{\pgfqpoint{1.252839in}{1.181591in}}%
\pgfpathlineto{\pgfqpoint{1.242959in}{1.189259in}}%
\pgfpathlineto{\pgfqpoint{1.237183in}{1.193862in}}%
\pgfpathlineto{\pgfqpoint{1.224814in}{1.202870in}}%
\pgfpathlineto{\pgfqpoint{1.221526in}{1.205364in}}%
\pgfpathlineto{\pgfqpoint{1.205870in}{1.215824in}}%
\pgfpathlineto{\pgfqpoint{1.204696in}{1.216481in}}%
\pgfpathlineto{\pgfqpoint{1.190213in}{1.225014in}}%
\pgfpathlineto{\pgfqpoint{1.179213in}{1.230092in}}%
\pgfpathlineto{\pgfqpoint{1.174556in}{1.232388in}}%
\pgfpathlineto{\pgfqpoint{1.158900in}{1.237671in}}%
\pgfpathlineto{\pgfqpoint{1.143243in}{1.240370in}}%
\pgfpathlineto{\pgfqpoint{1.127587in}{1.240370in}}%
\pgfpathlineto{\pgfqpoint{1.111930in}{1.237671in}}%
\pgfpathlineto{\pgfqpoint{1.096274in}{1.232388in}}%
\pgfpathlineto{\pgfqpoint{1.091617in}{1.230092in}}%
\pgfpathlineto{\pgfqpoint{1.080617in}{1.225014in}}%
\pgfpathlineto{\pgfqpoint{1.066134in}{1.216481in}}%
\pgfpathlineto{\pgfqpoint{1.064960in}{1.215824in}}%
\pgfpathlineto{\pgfqpoint{1.049304in}{1.205364in}}%
\pgfpathlineto{\pgfqpoint{1.046016in}{1.202870in}}%
\pgfpathlineto{\pgfqpoint{1.033647in}{1.193862in}}%
\pgfpathlineto{\pgfqpoint{1.027871in}{1.189259in}}%
\pgfpathlineto{\pgfqpoint{1.017991in}{1.181591in}}%
\pgfpathlineto{\pgfqpoint{1.010796in}{1.175647in}}%
\pgfpathlineto{\pgfqpoint{1.002334in}{1.168742in}}%
\pgfpathlineto{\pgfqpoint{0.994437in}{1.162036in}}%
\pgfpathlineto{\pgfqpoint{0.986678in}{1.155427in}}%
\pgfpathlineto{\pgfqpoint{0.978624in}{1.148425in}}%
\pgfpathlineto{\pgfqpoint{0.971021in}{1.141680in}}%
\pgfpathlineto{\pgfqpoint{0.963307in}{1.134814in}}%
\pgfpathlineto{\pgfqpoint{0.955364in}{1.127458in}}%
\pgfpathlineto{\pgfqpoint{0.948528in}{1.121203in}}%
\pgfpathlineto{\pgfqpoint{0.939708in}{1.112613in}}%
\pgfpathlineto{\pgfqpoint{0.934412in}{1.107592in}}%
\pgfpathlineto{\pgfqpoint{0.924051in}{1.096839in}}%
\pgfpathlineto{\pgfqpoint{0.921182in}{1.093981in}}%
\pgfpathlineto{\pgfqpoint{0.909150in}{1.080370in}}%
\pgfpathlineto{\pgfqpoint{0.908395in}{1.079350in}}%
\pgfpathlineto{\pgfqpoint{0.898579in}{1.066759in}}%
\pgfpathlineto{\pgfqpoint{0.892738in}{1.057196in}}%
\pgfpathlineto{\pgfqpoint{0.890097in}{1.053148in}}%
\pgfpathlineto{\pgfqpoint{0.884020in}{1.039536in}}%
\pgfpathlineto{\pgfqpoint{0.880916in}{1.025925in}}%
\pgfpathlineto{\pgfqpoint{0.880916in}{1.012314in}}%
\pgfpathlineto{\pgfqpoint{0.884020in}{0.998703in}}%
\pgfpathlineto{\pgfqpoint{0.890097in}{0.985092in}}%
\pgfpathlineto{\pgfqpoint{0.892738in}{0.981044in}}%
\pgfpathlineto{\pgfqpoint{0.898579in}{0.971481in}}%
\pgfpathlineto{\pgfqpoint{0.908395in}{0.958890in}}%
\pgfpathlineto{\pgfqpoint{0.909150in}{0.957870in}}%
\pgfpathlineto{\pgfqpoint{0.921182in}{0.944259in}}%
\pgfpathlineto{\pgfqpoint{0.924051in}{0.941400in}}%
\pgfpathlineto{\pgfqpoint{0.934412in}{0.930648in}}%
\pgfpathlineto{\pgfqpoint{0.939708in}{0.925626in}}%
\pgfpathlineto{\pgfqpoint{0.948528in}{0.917036in}}%
\pgfpathlineto{\pgfqpoint{0.955364in}{0.910782in}}%
\pgfpathlineto{\pgfqpoint{0.963307in}{0.903425in}}%
\pgfpathlineto{\pgfqpoint{0.971021in}{0.896560in}}%
\pgfpathlineto{\pgfqpoint{0.978624in}{0.889814in}}%
\pgfpathlineto{\pgfqpoint{0.986678in}{0.882813in}}%
\pgfpathlineto{\pgfqpoint{0.994437in}{0.876203in}}%
\pgfpathlineto{\pgfqpoint{1.002334in}{0.869497in}}%
\pgfpathlineto{\pgfqpoint{1.010796in}{0.862592in}}%
\pgfpathlineto{\pgfqpoint{1.017991in}{0.856648in}}%
\pgfpathlineto{\pgfqpoint{1.027871in}{0.848981in}}%
\pgfpathlineto{\pgfqpoint{1.033647in}{0.844377in}}%
\pgfpathlineto{\pgfqpoint{1.046016in}{0.835370in}}%
\pgfpathlineto{\pgfqpoint{1.049304in}{0.832876in}}%
\pgfpathlineto{\pgfqpoint{1.064960in}{0.822415in}}%
\pgfpathlineto{\pgfqpoint{1.066134in}{0.821759in}}%
\pgfpathlineto{\pgfqpoint{1.080617in}{0.813225in}}%
\pgfpathlineto{\pgfqpoint{1.091617in}{0.808148in}}%
\pgfpathlineto{\pgfqpoint{1.096274in}{0.805852in}}%
\pgfpathclose%
\pgfpathmoveto{\pgfqpoint{1.087732in}{0.876203in}}%
\pgfpathlineto{\pgfqpoint{1.080617in}{0.879048in}}%
\pgfpathlineto{\pgfqpoint{1.064960in}{0.887053in}}%
\pgfpathlineto{\pgfqpoint{1.060431in}{0.889814in}}%
\pgfpathlineto{\pgfqpoint{1.049304in}{0.896797in}}%
\pgfpathlineto{\pgfqpoint{1.039971in}{0.903425in}}%
\pgfpathlineto{\pgfqpoint{1.033647in}{0.908126in}}%
\pgfpathlineto{\pgfqpoint{1.022702in}{0.917036in}}%
\pgfpathlineto{\pgfqpoint{1.017991in}{0.921132in}}%
\pgfpathlineto{\pgfqpoint{1.007742in}{0.930648in}}%
\pgfpathlineto{\pgfqpoint{1.002334in}{0.936145in}}%
\pgfpathlineto{\pgfqpoint{0.994709in}{0.944259in}}%
\pgfpathlineto{\pgfqpoint{0.986678in}{0.953932in}}%
\pgfpathlineto{\pgfqpoint{0.983502in}{0.957870in}}%
\pgfpathlineto{\pgfqpoint{0.974293in}{0.971481in}}%
\pgfpathlineto{\pgfqpoint{0.971021in}{0.977667in}}%
\pgfpathlineto{\pgfqpoint{0.967164in}{0.985092in}}%
\pgfpathlineto{\pgfqpoint{0.962310in}{0.998703in}}%
\pgfpathlineto{\pgfqpoint{0.959830in}{1.012314in}}%
\pgfpathlineto{\pgfqpoint{0.959830in}{1.025925in}}%
\pgfpathlineto{\pgfqpoint{0.962310in}{1.039536in}}%
\pgfpathlineto{\pgfqpoint{0.967164in}{1.053148in}}%
\pgfpathlineto{\pgfqpoint{0.971021in}{1.060573in}}%
\pgfpathlineto{\pgfqpoint{0.974293in}{1.066759in}}%
\pgfpathlineto{\pgfqpoint{0.983502in}{1.080370in}}%
\pgfpathlineto{\pgfqpoint{0.986678in}{1.084307in}}%
\pgfpathlineto{\pgfqpoint{0.994709in}{1.093981in}}%
\pgfpathlineto{\pgfqpoint{1.002334in}{1.102094in}}%
\pgfpathlineto{\pgfqpoint{1.007742in}{1.107592in}}%
\pgfpathlineto{\pgfqpoint{1.017991in}{1.117107in}}%
\pgfpathlineto{\pgfqpoint{1.022702in}{1.121203in}}%
\pgfpathlineto{\pgfqpoint{1.033647in}{1.130113in}}%
\pgfpathlineto{\pgfqpoint{1.039971in}{1.134814in}}%
\pgfpathlineto{\pgfqpoint{1.049304in}{1.141443in}}%
\pgfpathlineto{\pgfqpoint{1.060431in}{1.148425in}}%
\pgfpathlineto{\pgfqpoint{1.064960in}{1.151186in}}%
\pgfpathlineto{\pgfqpoint{1.080617in}{1.159192in}}%
\pgfpathlineto{\pgfqpoint{1.087732in}{1.162036in}}%
\pgfpathlineto{\pgfqpoint{1.096274in}{1.165389in}}%
\pgfpathlineto{\pgfqpoint{1.111930in}{1.169609in}}%
\pgfpathlineto{\pgfqpoint{1.127587in}{1.171765in}}%
\pgfpathlineto{\pgfqpoint{1.143243in}{1.171765in}}%
\pgfpathlineto{\pgfqpoint{1.158900in}{1.169609in}}%
\pgfpathlineto{\pgfqpoint{1.174556in}{1.165389in}}%
\pgfpathlineto{\pgfqpoint{1.183098in}{1.162036in}}%
\pgfpathlineto{\pgfqpoint{1.190213in}{1.159192in}}%
\pgfpathlineto{\pgfqpoint{1.205870in}{1.151186in}}%
\pgfpathlineto{\pgfqpoint{1.210399in}{1.148425in}}%
\pgfpathlineto{\pgfqpoint{1.221526in}{1.141443in}}%
\pgfpathlineto{\pgfqpoint{1.230859in}{1.134814in}}%
\pgfpathlineto{\pgfqpoint{1.237183in}{1.130113in}}%
\pgfpathlineto{\pgfqpoint{1.248128in}{1.121203in}}%
\pgfpathlineto{\pgfqpoint{1.252839in}{1.117107in}}%
\pgfpathlineto{\pgfqpoint{1.263088in}{1.107592in}}%
\pgfpathlineto{\pgfqpoint{1.268496in}{1.102094in}}%
\pgfpathlineto{\pgfqpoint{1.276121in}{1.093981in}}%
\pgfpathlineto{\pgfqpoint{1.284152in}{1.084307in}}%
\pgfpathlineto{\pgfqpoint{1.287328in}{1.080370in}}%
\pgfpathlineto{\pgfqpoint{1.296537in}{1.066759in}}%
\pgfpathlineto{\pgfqpoint{1.299809in}{1.060573in}}%
\pgfpathlineto{\pgfqpoint{1.303666in}{1.053148in}}%
\pgfpathlineto{\pgfqpoint{1.308520in}{1.039536in}}%
\pgfpathlineto{\pgfqpoint{1.311000in}{1.025925in}}%
\pgfpathlineto{\pgfqpoint{1.311000in}{1.012314in}}%
\pgfpathlineto{\pgfqpoint{1.308520in}{0.998703in}}%
\pgfpathlineto{\pgfqpoint{1.303666in}{0.985092in}}%
\pgfpathlineto{\pgfqpoint{1.299809in}{0.977667in}}%
\pgfpathlineto{\pgfqpoint{1.296537in}{0.971481in}}%
\pgfpathlineto{\pgfqpoint{1.287328in}{0.957870in}}%
\pgfpathlineto{\pgfqpoint{1.284152in}{0.953932in}}%
\pgfpathlineto{\pgfqpoint{1.276121in}{0.944259in}}%
\pgfpathlineto{\pgfqpoint{1.268496in}{0.936145in}}%
\pgfpathlineto{\pgfqpoint{1.263088in}{0.930648in}}%
\pgfpathlineto{\pgfqpoint{1.252839in}{0.921132in}}%
\pgfpathlineto{\pgfqpoint{1.248128in}{0.917036in}}%
\pgfpathlineto{\pgfqpoint{1.237183in}{0.908126in}}%
\pgfpathlineto{\pgfqpoint{1.230859in}{0.903425in}}%
\pgfpathlineto{\pgfqpoint{1.221526in}{0.896797in}}%
\pgfpathlineto{\pgfqpoint{1.210399in}{0.889814in}}%
\pgfpathlineto{\pgfqpoint{1.205870in}{0.887053in}}%
\pgfpathlineto{\pgfqpoint{1.190213in}{0.879048in}}%
\pgfpathlineto{\pgfqpoint{1.183098in}{0.876203in}}%
\pgfpathlineto{\pgfqpoint{1.174556in}{0.872850in}}%
\pgfpathlineto{\pgfqpoint{1.158900in}{0.868630in}}%
\pgfpathlineto{\pgfqpoint{1.143243in}{0.866474in}}%
\pgfpathlineto{\pgfqpoint{1.127587in}{0.866474in}}%
\pgfpathlineto{\pgfqpoint{1.111930in}{0.868630in}}%
\pgfpathlineto{\pgfqpoint{1.096274in}{0.872850in}}%
\pgfpathlineto{\pgfqpoint{1.087732in}{0.876203in}}%
\pgfpathclose%
\pgfpathmoveto{\pgfqpoint{1.879102in}{0.802899in}}%
\pgfpathlineto{\pgfqpoint{1.894758in}{0.798886in}}%
\pgfpathlineto{\pgfqpoint{1.910415in}{0.797530in}}%
\pgfpathlineto{\pgfqpoint{1.910415in}{0.808148in}}%
\pgfpathlineto{\pgfqpoint{1.910415in}{0.821759in}}%
\pgfpathlineto{\pgfqpoint{1.910415in}{0.835370in}}%
\pgfpathlineto{\pgfqpoint{1.910415in}{0.848981in}}%
\pgfpathlineto{\pgfqpoint{1.910415in}{0.862592in}}%
\pgfpathlineto{\pgfqpoint{1.910415in}{0.866202in}}%
\pgfpathlineto{\pgfqpoint{1.894758in}{0.867286in}}%
\pgfpathlineto{\pgfqpoint{1.879102in}{0.870491in}}%
\pgfpathlineto{\pgfqpoint{1.863445in}{0.875682in}}%
\pgfpathlineto{\pgfqpoint{1.862292in}{0.876203in}}%
\pgfpathlineto{\pgfqpoint{1.847789in}{0.882856in}}%
\pgfpathlineto{\pgfqpoint{1.835380in}{0.889814in}}%
\pgfpathlineto{\pgfqpoint{1.832132in}{0.891692in}}%
\pgfpathlineto{\pgfqpoint{1.816476in}{0.902185in}}%
\pgfpathlineto{\pgfqpoint{1.814817in}{0.903425in}}%
\pgfpathlineto{\pgfqpoint{1.800819in}{0.914367in}}%
\pgfpathlineto{\pgfqpoint{1.797657in}{0.917036in}}%
\pgfpathlineto{\pgfqpoint{1.785162in}{0.928288in}}%
\pgfpathlineto{\pgfqpoint{1.782680in}{0.930648in}}%
\pgfpathlineto{\pgfqpoint{1.769597in}{0.944259in}}%
\pgfpathlineto{\pgfqpoint{1.769506in}{0.944367in}}%
\pgfpathlineto{\pgfqpoint{1.758526in}{0.957870in}}%
\pgfpathlineto{\pgfqpoint{1.753849in}{0.964688in}}%
\pgfpathlineto{\pgfqpoint{1.749297in}{0.971481in}}%
\pgfpathlineto{\pgfqpoint{1.742173in}{0.985092in}}%
\pgfpathlineto{\pgfqpoint{1.738193in}{0.996120in}}%
\pgfpathlineto{\pgfqpoint{1.737274in}{0.998703in}}%
\pgfpathlineto{\pgfqpoint{1.734817in}{1.012314in}}%
\pgfpathlineto{\pgfqpoint{1.734817in}{1.025925in}}%
\pgfpathlineto{\pgfqpoint{1.737274in}{1.039536in}}%
\pgfpathlineto{\pgfqpoint{1.738193in}{1.042119in}}%
\pgfpathlineto{\pgfqpoint{1.742173in}{1.053148in}}%
\pgfpathlineto{\pgfqpoint{1.749297in}{1.066759in}}%
\pgfpathlineto{\pgfqpoint{1.753849in}{1.073551in}}%
\pgfpathlineto{\pgfqpoint{1.758526in}{1.080370in}}%
\pgfpathlineto{\pgfqpoint{1.769506in}{1.093872in}}%
\pgfpathlineto{\pgfqpoint{1.769597in}{1.093981in}}%
\pgfpathlineto{\pgfqpoint{1.782680in}{1.107592in}}%
\pgfpathlineto{\pgfqpoint{1.785162in}{1.109952in}}%
\pgfpathlineto{\pgfqpoint{1.797657in}{1.121203in}}%
\pgfpathlineto{\pgfqpoint{1.800819in}{1.123872in}}%
\pgfpathlineto{\pgfqpoint{1.814817in}{1.134814in}}%
\pgfpathlineto{\pgfqpoint{1.816476in}{1.136054in}}%
\pgfpathlineto{\pgfqpoint{1.832132in}{1.146548in}}%
\pgfpathlineto{\pgfqpoint{1.835380in}{1.148425in}}%
\pgfpathlineto{\pgfqpoint{1.847789in}{1.155384in}}%
\pgfpathlineto{\pgfqpoint{1.862292in}{1.162036in}}%
\pgfpathlineto{\pgfqpoint{1.863445in}{1.162557in}}%
\pgfpathlineto{\pgfqpoint{1.879102in}{1.167748in}}%
\pgfpathlineto{\pgfqpoint{1.894758in}{1.170953in}}%
\pgfpathlineto{\pgfqpoint{1.910415in}{1.172037in}}%
\pgfpathlineto{\pgfqpoint{1.910415in}{1.175647in}}%
\pgfpathlineto{\pgfqpoint{1.910415in}{1.189259in}}%
\pgfpathlineto{\pgfqpoint{1.910415in}{1.202870in}}%
\pgfpathlineto{\pgfqpoint{1.910415in}{1.216481in}}%
\pgfpathlineto{\pgfqpoint{1.910415in}{1.230092in}}%
\pgfpathlineto{\pgfqpoint{1.910415in}{1.240710in}}%
\pgfpathlineto{\pgfqpoint{1.894758in}{1.239353in}}%
\pgfpathlineto{\pgfqpoint{1.879102in}{1.235341in}}%
\pgfpathlineto{\pgfqpoint{1.866483in}{1.230092in}}%
\pgfpathlineto{\pgfqpoint{1.863445in}{1.228906in}}%
\pgfpathlineto{\pgfqpoint{1.847789in}{1.220632in}}%
\pgfpathlineto{\pgfqpoint{1.841403in}{1.216481in}}%
\pgfpathlineto{\pgfqpoint{1.832132in}{1.210767in}}%
\pgfpathlineto{\pgfqpoint{1.821070in}{1.202870in}}%
\pgfpathlineto{\pgfqpoint{1.816476in}{1.199716in}}%
\pgfpathlineto{\pgfqpoint{1.802774in}{1.189259in}}%
\pgfpathlineto{\pgfqpoint{1.800819in}{1.187805in}}%
\pgfpathlineto{\pgfqpoint{1.785662in}{1.175647in}}%
\pgfpathlineto{\pgfqpoint{1.785162in}{1.175252in}}%
\pgfpathlineto{\pgfqpoint{1.769506in}{1.162190in}}%
\pgfpathlineto{\pgfqpoint{1.769327in}{1.162036in}}%
\pgfpathlineto{\pgfqpoint{1.753849in}{1.148686in}}%
\pgfpathlineto{\pgfqpoint{1.753551in}{1.148425in}}%
\pgfpathlineto{\pgfqpoint{1.738263in}{1.134814in}}%
\pgfpathlineto{\pgfqpoint{1.738193in}{1.134749in}}%
\pgfpathlineto{\pgfqpoint{1.723520in}{1.121203in}}%
\pgfpathlineto{\pgfqpoint{1.722536in}{1.120237in}}%
\pgfpathlineto{\pgfqpoint{1.709424in}{1.107592in}}%
\pgfpathlineto{\pgfqpoint{1.706880in}{1.104908in}}%
\pgfpathlineto{\pgfqpoint{1.696179in}{1.093981in}}%
\pgfpathlineto{\pgfqpoint{1.691223in}{1.088246in}}%
\pgfpathlineto{\pgfqpoint{1.684094in}{1.080370in}}%
\pgfpathlineto{\pgfqpoint{1.675567in}{1.069168in}}%
\pgfpathlineto{\pgfqpoint{1.673618in}{1.066759in}}%
\pgfpathlineto{\pgfqpoint{1.665069in}{1.053148in}}%
\pgfpathlineto{\pgfqpoint{1.659910in}{1.041242in}}%
\pgfpathlineto{\pgfqpoint{1.659116in}{1.039536in}}%
\pgfpathlineto{\pgfqpoint{1.655905in}{1.025925in}}%
\pgfpathlineto{\pgfqpoint{1.655905in}{1.012314in}}%
\pgfpathlineto{\pgfqpoint{1.659116in}{0.998703in}}%
\pgfpathlineto{\pgfqpoint{1.659910in}{0.996997in}}%
\pgfpathlineto{\pgfqpoint{1.665069in}{0.985092in}}%
\pgfpathlineto{\pgfqpoint{1.673618in}{0.971481in}}%
\pgfpathlineto{\pgfqpoint{1.675567in}{0.969071in}}%
\pgfpathlineto{\pgfqpoint{1.684094in}{0.957870in}}%
\pgfpathlineto{\pgfqpoint{1.691223in}{0.949993in}}%
\pgfpathlineto{\pgfqpoint{1.696179in}{0.944259in}}%
\pgfpathlineto{\pgfqpoint{1.706880in}{0.933332in}}%
\pgfpathlineto{\pgfqpoint{1.709424in}{0.930648in}}%
\pgfpathlineto{\pgfqpoint{1.722536in}{0.918003in}}%
\pgfpathlineto{\pgfqpoint{1.723520in}{0.917036in}}%
\pgfpathlineto{\pgfqpoint{1.738193in}{0.903490in}}%
\pgfpathlineto{\pgfqpoint{1.738263in}{0.903425in}}%
\pgfpathlineto{\pgfqpoint{1.753551in}{0.889814in}}%
\pgfpathlineto{\pgfqpoint{1.753849in}{0.889554in}}%
\pgfpathlineto{\pgfqpoint{1.769327in}{0.876203in}}%
\pgfpathlineto{\pgfqpoint{1.769506in}{0.876049in}}%
\pgfpathlineto{\pgfqpoint{1.785162in}{0.862988in}}%
\pgfpathlineto{\pgfqpoint{1.785662in}{0.862592in}}%
\pgfpathlineto{\pgfqpoint{1.800819in}{0.850435in}}%
\pgfpathlineto{\pgfqpoint{1.802774in}{0.848981in}}%
\pgfpathlineto{\pgfqpoint{1.816476in}{0.838523in}}%
\pgfpathlineto{\pgfqpoint{1.821070in}{0.835370in}}%
\pgfpathlineto{\pgfqpoint{1.832132in}{0.827473in}}%
\pgfpathlineto{\pgfqpoint{1.841403in}{0.821759in}}%
\pgfpathlineto{\pgfqpoint{1.847789in}{0.817608in}}%
\pgfpathlineto{\pgfqpoint{1.863445in}{0.809333in}}%
\pgfpathlineto{\pgfqpoint{1.866483in}{0.808148in}}%
\pgfpathlineto{\pgfqpoint{1.879102in}{0.802899in}}%
\pgfpathclose%
\pgfpathmoveto{\pgfqpoint{0.376072in}{1.472662in}}%
\pgfpathlineto{\pgfqpoint{0.385214in}{1.475092in}}%
\pgfpathlineto{\pgfqpoint{0.391728in}{1.476708in}}%
\pgfpathlineto{\pgfqpoint{0.407385in}{1.483023in}}%
\pgfpathlineto{\pgfqpoint{0.417826in}{1.488703in}}%
\pgfpathlineto{\pgfqpoint{0.423041in}{1.491379in}}%
\pgfpathlineto{\pgfqpoint{0.438698in}{1.501251in}}%
\pgfpathlineto{\pgfqpoint{0.440146in}{1.502314in}}%
\pgfpathlineto{\pgfqpoint{0.454354in}{1.512264in}}%
\pgfpathlineto{\pgfqpoint{0.459052in}{1.515925in}}%
\pgfpathlineto{\pgfqpoint{0.470011in}{1.524181in}}%
\pgfpathlineto{\pgfqpoint{0.476602in}{1.529536in}}%
\pgfpathlineto{\pgfqpoint{0.485668in}{1.536759in}}%
\pgfpathlineto{\pgfqpoint{0.493289in}{1.543148in}}%
\pgfpathlineto{\pgfqpoint{0.501324in}{1.549851in}}%
\pgfpathlineto{\pgfqpoint{0.509362in}{1.556759in}}%
\pgfpathlineto{\pgfqpoint{0.516981in}{1.563382in}}%
\pgfpathlineto{\pgfqpoint{0.524927in}{1.570370in}}%
\pgfpathlineto{\pgfqpoint{0.532637in}{1.577355in}}%
\pgfpathlineto{\pgfqpoint{0.539986in}{1.583981in}}%
\pgfpathlineto{\pgfqpoint{0.548294in}{1.591862in}}%
\pgfpathlineto{\pgfqpoint{0.554454in}{1.597592in}}%
\pgfpathlineto{\pgfqpoint{0.563950in}{1.607119in}}%
\pgfpathlineto{\pgfqpoint{0.568161in}{1.611203in}}%
\pgfpathlineto{\pgfqpoint{0.579607in}{1.623555in}}%
\pgfpathlineto{\pgfqpoint{0.580830in}{1.624814in}}%
\pgfpathlineto{\pgfqpoint{0.592186in}{1.638425in}}%
\pgfpathlineto{\pgfqpoint{0.595263in}{1.642959in}}%
\pgfpathlineto{\pgfqpoint{0.601798in}{1.652036in}}%
\pgfpathlineto{\pgfqpoint{0.609062in}{1.665648in}}%
\pgfpathlineto{\pgfqpoint{0.610920in}{1.671310in}}%
\pgfpathlineto{\pgfqpoint{0.613715in}{1.679259in}}%
\pgfpathlineto{\pgfqpoint{0.615330in}{1.692870in}}%
\pgfpathlineto{\pgfqpoint{0.610920in}{1.692870in}}%
\pgfpathlineto{\pgfqpoint{0.595263in}{1.692870in}}%
\pgfpathlineto{\pgfqpoint{0.579607in}{1.692870in}}%
\pgfpathlineto{\pgfqpoint{0.563950in}{1.692870in}}%
\pgfpathlineto{\pgfqpoint{0.548294in}{1.692870in}}%
\pgfpathlineto{\pgfqpoint{0.536323in}{1.692870in}}%
\pgfpathlineto{\pgfqpoint{0.535088in}{1.679259in}}%
\pgfpathlineto{\pgfqpoint{0.532637in}{1.670146in}}%
\pgfpathlineto{\pgfqpoint{0.531408in}{1.665648in}}%
\pgfpathlineto{\pgfqpoint{0.525354in}{1.652036in}}%
\pgfpathlineto{\pgfqpoint{0.517231in}{1.638425in}}%
\pgfpathlineto{\pgfqpoint{0.516981in}{1.638087in}}%
\pgfpathlineto{\pgfqpoint{0.506954in}{1.624814in}}%
\pgfpathlineto{\pgfqpoint{0.501324in}{1.618367in}}%
\pgfpathlineto{\pgfqpoint{0.494834in}{1.611203in}}%
\pgfpathlineto{\pgfqpoint{0.485668in}{1.602104in}}%
\pgfpathlineto{\pgfqpoint{0.480865in}{1.597592in}}%
\pgfpathlineto{\pgfqpoint{0.470011in}{1.588156in}}%
\pgfpathlineto{\pgfqpoint{0.464821in}{1.583981in}}%
\pgfpathlineto{\pgfqpoint{0.454354in}{1.576012in}}%
\pgfpathlineto{\pgfqpoint{0.446114in}{1.570370in}}%
\pgfpathlineto{\pgfqpoint{0.438698in}{1.565476in}}%
\pgfpathlineto{\pgfqpoint{0.423430in}{1.556759in}}%
\pgfpathlineto{\pgfqpoint{0.423041in}{1.556541in}}%
\pgfpathlineto{\pgfqpoint{0.407385in}{1.549479in}}%
\pgfpathlineto{\pgfqpoint{0.391728in}{1.544216in}}%
\pgfpathlineto{\pgfqpoint{0.386554in}{1.543148in}}%
\pgfpathlineto{\pgfqpoint{0.376072in}{1.541017in}}%
\pgfpathlineto{\pgfqpoint{0.360415in}{1.539944in}}%
\pgfpathlineto{\pgfqpoint{0.360415in}{1.529536in}}%
\pgfpathlineto{\pgfqpoint{0.360415in}{1.515925in}}%
\pgfpathlineto{\pgfqpoint{0.360415in}{1.502314in}}%
\pgfpathlineto{\pgfqpoint{0.360415in}{1.488703in}}%
\pgfpathlineto{\pgfqpoint{0.360415in}{1.475092in}}%
\pgfpathlineto{\pgfqpoint{0.360415in}{1.471258in}}%
\pgfpathlineto{\pgfqpoint{0.376072in}{1.472662in}}%
\pgfpathclose%
\pgfpathmoveto{\pgfqpoint{1.111930in}{1.474402in}}%
\pgfpathlineto{\pgfqpoint{1.127587in}{1.471610in}}%
\pgfpathlineto{\pgfqpoint{1.143243in}{1.471610in}}%
\pgfpathlineto{\pgfqpoint{1.158900in}{1.474402in}}%
\pgfpathlineto{\pgfqpoint{1.160862in}{1.475092in}}%
\pgfpathlineto{\pgfqpoint{1.174556in}{1.479577in}}%
\pgfpathlineto{\pgfqpoint{1.190213in}{1.487009in}}%
\pgfpathlineto{\pgfqpoint{1.192985in}{1.488703in}}%
\pgfpathlineto{\pgfqpoint{1.205870in}{1.496117in}}%
\pgfpathlineto{\pgfqpoint{1.214930in}{1.502314in}}%
\pgfpathlineto{\pgfqpoint{1.221526in}{1.506623in}}%
\pgfpathlineto{\pgfqpoint{1.234095in}{1.515925in}}%
\pgfpathlineto{\pgfqpoint{1.237183in}{1.518137in}}%
\pgfpathlineto{\pgfqpoint{1.251728in}{1.529536in}}%
\pgfpathlineto{\pgfqpoint{1.252839in}{1.530391in}}%
\pgfpathlineto{\pgfqpoint{1.268421in}{1.543148in}}%
\pgfpathlineto{\pgfqpoint{1.268496in}{1.543209in}}%
\pgfpathlineto{\pgfqpoint{1.284152in}{1.556499in}}%
\pgfpathlineto{\pgfqpoint{1.284452in}{1.556759in}}%
\pgfpathlineto{\pgfqpoint{1.299809in}{1.570214in}}%
\pgfpathlineto{\pgfqpoint{1.299986in}{1.570370in}}%
\pgfpathlineto{\pgfqpoint{1.315010in}{1.583981in}}%
\pgfpathlineto{\pgfqpoint{1.315466in}{1.584415in}}%
\pgfpathlineto{\pgfqpoint{1.329450in}{1.597592in}}%
\pgfpathlineto{\pgfqpoint{1.331122in}{1.599291in}}%
\pgfpathlineto{\pgfqpoint{1.343152in}{1.611203in}}%
\pgfpathlineto{\pgfqpoint{1.346779in}{1.615197in}}%
\pgfpathlineto{\pgfqpoint{1.355863in}{1.624814in}}%
\pgfpathlineto{\pgfqpoint{1.362435in}{1.632874in}}%
\pgfpathlineto{\pgfqpoint{1.367210in}{1.638425in}}%
\pgfpathlineto{\pgfqpoint{1.376728in}{1.652036in}}%
\pgfpathlineto{\pgfqpoint{1.378092in}{1.654678in}}%
\pgfpathlineto{\pgfqpoint{1.384130in}{1.665648in}}%
\pgfpathlineto{\pgfqpoint{1.388745in}{1.679259in}}%
\pgfpathlineto{\pgfqpoint{1.390305in}{1.692870in}}%
\pgfpathlineto{\pgfqpoint{1.378092in}{1.692870in}}%
\pgfpathlineto{\pgfqpoint{1.362435in}{1.692870in}}%
\pgfpathlineto{\pgfqpoint{1.346779in}{1.692870in}}%
\pgfpathlineto{\pgfqpoint{1.331122in}{1.692870in}}%
\pgfpathlineto{\pgfqpoint{1.315466in}{1.692870in}}%
\pgfpathlineto{\pgfqpoint{1.311313in}{1.692870in}}%
\pgfpathlineto{\pgfqpoint{1.310066in}{1.679259in}}%
\pgfpathlineto{\pgfqpoint{1.306379in}{1.665648in}}%
\pgfpathlineto{\pgfqpoint{1.300408in}{1.652036in}}%
\pgfpathlineto{\pgfqpoint{1.299809in}{1.651034in}}%
\pgfpathlineto{\pgfqpoint{1.292156in}{1.638425in}}%
\pgfpathlineto{\pgfqpoint{1.284152in}{1.627638in}}%
\pgfpathlineto{\pgfqpoint{1.281992in}{1.624814in}}%
\pgfpathlineto{\pgfqpoint{1.269923in}{1.611203in}}%
\pgfpathlineto{\pgfqpoint{1.268496in}{1.609761in}}%
\pgfpathlineto{\pgfqpoint{1.255909in}{1.597592in}}%
\pgfpathlineto{\pgfqpoint{1.252839in}{1.594843in}}%
\pgfpathlineto{\pgfqpoint{1.239897in}{1.583981in}}%
\pgfpathlineto{\pgfqpoint{1.237183in}{1.581822in}}%
\pgfpathlineto{\pgfqpoint{1.221526in}{1.570449in}}%
\pgfpathlineto{\pgfqpoint{1.221401in}{1.570370in}}%
\pgfpathlineto{\pgfqpoint{1.205870in}{1.560825in}}%
\pgfpathlineto{\pgfqpoint{1.198026in}{1.556759in}}%
\pgfpathlineto{\pgfqpoint{1.190213in}{1.552801in}}%
\pgfpathlineto{\pgfqpoint{1.174556in}{1.546608in}}%
\pgfpathlineto{\pgfqpoint{1.161871in}{1.543148in}}%
\pgfpathlineto{\pgfqpoint{1.158900in}{1.542348in}}%
\pgfpathlineto{\pgfqpoint{1.143243in}{1.540213in}}%
\pgfpathlineto{\pgfqpoint{1.127587in}{1.540213in}}%
\pgfpathlineto{\pgfqpoint{1.111930in}{1.542348in}}%
\pgfpathlineto{\pgfqpoint{1.108959in}{1.543148in}}%
\pgfpathlineto{\pgfqpoint{1.096274in}{1.546608in}}%
\pgfpathlineto{\pgfqpoint{1.080617in}{1.552801in}}%
\pgfpathlineto{\pgfqpoint{1.072804in}{1.556759in}}%
\pgfpathlineto{\pgfqpoint{1.064960in}{1.560825in}}%
\pgfpathlineto{\pgfqpoint{1.049429in}{1.570370in}}%
\pgfpathlineto{\pgfqpoint{1.049304in}{1.570449in}}%
\pgfpathlineto{\pgfqpoint{1.033647in}{1.581822in}}%
\pgfpathlineto{\pgfqpoint{1.030933in}{1.583981in}}%
\pgfpathlineto{\pgfqpoint{1.017991in}{1.594843in}}%
\pgfpathlineto{\pgfqpoint{1.014921in}{1.597592in}}%
\pgfpathlineto{\pgfqpoint{1.002334in}{1.609761in}}%
\pgfpathlineto{\pgfqpoint{1.000907in}{1.611203in}}%
\pgfpathlineto{\pgfqpoint{0.988838in}{1.624814in}}%
\pgfpathlineto{\pgfqpoint{0.986678in}{1.627638in}}%
\pgfpathlineto{\pgfqpoint{0.978674in}{1.638425in}}%
\pgfpathlineto{\pgfqpoint{0.971021in}{1.651034in}}%
\pgfpathlineto{\pgfqpoint{0.970422in}{1.652036in}}%
\pgfpathlineto{\pgfqpoint{0.964451in}{1.665648in}}%
\pgfpathlineto{\pgfqpoint{0.960764in}{1.679259in}}%
\pgfpathlineto{\pgfqpoint{0.959517in}{1.692870in}}%
\pgfpathlineto{\pgfqpoint{0.955364in}{1.692870in}}%
\pgfpathlineto{\pgfqpoint{0.939708in}{1.692870in}}%
\pgfpathlineto{\pgfqpoint{0.924051in}{1.692870in}}%
\pgfpathlineto{\pgfqpoint{0.908395in}{1.692870in}}%
\pgfpathlineto{\pgfqpoint{0.892738in}{1.692870in}}%
\pgfpathlineto{\pgfqpoint{0.880525in}{1.692870in}}%
\pgfpathlineto{\pgfqpoint{0.882085in}{1.679259in}}%
\pgfpathlineto{\pgfqpoint{0.886700in}{1.665648in}}%
\pgfpathlineto{\pgfqpoint{0.892738in}{1.654678in}}%
\pgfpathlineto{\pgfqpoint{0.894102in}{1.652036in}}%
\pgfpathlineto{\pgfqpoint{0.903620in}{1.638425in}}%
\pgfpathlineto{\pgfqpoint{0.908395in}{1.632874in}}%
\pgfpathlineto{\pgfqpoint{0.914967in}{1.624814in}}%
\pgfpathlineto{\pgfqpoint{0.924051in}{1.615197in}}%
\pgfpathlineto{\pgfqpoint{0.927678in}{1.611203in}}%
\pgfpathlineto{\pgfqpoint{0.939708in}{1.599291in}}%
\pgfpathlineto{\pgfqpoint{0.941380in}{1.597592in}}%
\pgfpathlineto{\pgfqpoint{0.955364in}{1.584415in}}%
\pgfpathlineto{\pgfqpoint{0.955820in}{1.583981in}}%
\pgfpathlineto{\pgfqpoint{0.970844in}{1.570370in}}%
\pgfpathlineto{\pgfqpoint{0.971021in}{1.570214in}}%
\pgfpathlineto{\pgfqpoint{0.986378in}{1.556759in}}%
\pgfpathlineto{\pgfqpoint{0.986678in}{1.556499in}}%
\pgfpathlineto{\pgfqpoint{1.002334in}{1.543209in}}%
\pgfpathlineto{\pgfqpoint{1.002409in}{1.543148in}}%
\pgfpathlineto{\pgfqpoint{1.017991in}{1.530391in}}%
\pgfpathlineto{\pgfqpoint{1.019102in}{1.529536in}}%
\pgfpathlineto{\pgfqpoint{1.033647in}{1.518137in}}%
\pgfpathlineto{\pgfqpoint{1.036735in}{1.515925in}}%
\pgfpathlineto{\pgfqpoint{1.049304in}{1.506623in}}%
\pgfpathlineto{\pgfqpoint{1.055900in}{1.502314in}}%
\pgfpathlineto{\pgfqpoint{1.064960in}{1.496117in}}%
\pgfpathlineto{\pgfqpoint{1.077845in}{1.488703in}}%
\pgfpathlineto{\pgfqpoint{1.080617in}{1.487009in}}%
\pgfpathlineto{\pgfqpoint{1.096274in}{1.479577in}}%
\pgfpathlineto{\pgfqpoint{1.109968in}{1.475092in}}%
\pgfpathlineto{\pgfqpoint{1.111930in}{1.474402in}}%
\pgfpathclose%
\pgfpathmoveto{\pgfqpoint{1.894758in}{1.472662in}}%
\pgfpathlineto{\pgfqpoint{1.910415in}{1.471258in}}%
\pgfpathlineto{\pgfqpoint{1.910415in}{1.475092in}}%
\pgfpathlineto{\pgfqpoint{1.910415in}{1.488703in}}%
\pgfpathlineto{\pgfqpoint{1.910415in}{1.502314in}}%
\pgfpathlineto{\pgfqpoint{1.910415in}{1.515925in}}%
\pgfpathlineto{\pgfqpoint{1.910415in}{1.529536in}}%
\pgfpathlineto{\pgfqpoint{1.910415in}{1.539944in}}%
\pgfpathlineto{\pgfqpoint{1.894758in}{1.541017in}}%
\pgfpathlineto{\pgfqpoint{1.884276in}{1.543148in}}%
\pgfpathlineto{\pgfqpoint{1.879102in}{1.544216in}}%
\pgfpathlineto{\pgfqpoint{1.863445in}{1.549479in}}%
\pgfpathlineto{\pgfqpoint{1.847789in}{1.556541in}}%
\pgfpathlineto{\pgfqpoint{1.847400in}{1.556759in}}%
\pgfpathlineto{\pgfqpoint{1.832132in}{1.565476in}}%
\pgfpathlineto{\pgfqpoint{1.824716in}{1.570370in}}%
\pgfpathlineto{\pgfqpoint{1.816476in}{1.576012in}}%
\pgfpathlineto{\pgfqpoint{1.806009in}{1.583981in}}%
\pgfpathlineto{\pgfqpoint{1.800819in}{1.588156in}}%
\pgfpathlineto{\pgfqpoint{1.789965in}{1.597592in}}%
\pgfpathlineto{\pgfqpoint{1.785162in}{1.602104in}}%
\pgfpathlineto{\pgfqpoint{1.775996in}{1.611203in}}%
\pgfpathlineto{\pgfqpoint{1.769506in}{1.618367in}}%
\pgfpathlineto{\pgfqpoint{1.763876in}{1.624814in}}%
\pgfpathlineto{\pgfqpoint{1.753849in}{1.638087in}}%
\pgfpathlineto{\pgfqpoint{1.753599in}{1.638425in}}%
\pgfpathlineto{\pgfqpoint{1.745476in}{1.652036in}}%
\pgfpathlineto{\pgfqpoint{1.739422in}{1.665648in}}%
\pgfpathlineto{\pgfqpoint{1.738193in}{1.670146in}}%
\pgfpathlineto{\pgfqpoint{1.735742in}{1.679259in}}%
\pgfpathlineto{\pgfqpoint{1.734507in}{1.692870in}}%
\pgfpathlineto{\pgfqpoint{1.722536in}{1.692870in}}%
\pgfpathlineto{\pgfqpoint{1.706880in}{1.692870in}}%
\pgfpathlineto{\pgfqpoint{1.691223in}{1.692870in}}%
\pgfpathlineto{\pgfqpoint{1.675567in}{1.692870in}}%
\pgfpathlineto{\pgfqpoint{1.659910in}{1.692870in}}%
\pgfpathlineto{\pgfqpoint{1.655500in}{1.692870in}}%
\pgfpathlineto{\pgfqpoint{1.657115in}{1.679259in}}%
\pgfpathlineto{\pgfqpoint{1.659910in}{1.671310in}}%
\pgfpathlineto{\pgfqpoint{1.661768in}{1.665648in}}%
\pgfpathlineto{\pgfqpoint{1.669032in}{1.652036in}}%
\pgfpathlineto{\pgfqpoint{1.675567in}{1.642959in}}%
\pgfpathlineto{\pgfqpoint{1.678644in}{1.638425in}}%
\pgfpathlineto{\pgfqpoint{1.690000in}{1.624814in}}%
\pgfpathlineto{\pgfqpoint{1.691223in}{1.623555in}}%
\pgfpathlineto{\pgfqpoint{1.702669in}{1.611203in}}%
\pgfpathlineto{\pgfqpoint{1.706880in}{1.607119in}}%
\pgfpathlineto{\pgfqpoint{1.716376in}{1.597592in}}%
\pgfpathlineto{\pgfqpoint{1.722536in}{1.591862in}}%
\pgfpathlineto{\pgfqpoint{1.730844in}{1.583981in}}%
\pgfpathlineto{\pgfqpoint{1.738193in}{1.577355in}}%
\pgfpathlineto{\pgfqpoint{1.745903in}{1.570370in}}%
\pgfpathlineto{\pgfqpoint{1.753849in}{1.563382in}}%
\pgfpathlineto{\pgfqpoint{1.761468in}{1.556759in}}%
\pgfpathlineto{\pgfqpoint{1.769506in}{1.549851in}}%
\pgfpathlineto{\pgfqpoint{1.777541in}{1.543148in}}%
\pgfpathlineto{\pgfqpoint{1.785162in}{1.536759in}}%
\pgfpathlineto{\pgfqpoint{1.794228in}{1.529536in}}%
\pgfpathlineto{\pgfqpoint{1.800819in}{1.524181in}}%
\pgfpathlineto{\pgfqpoint{1.811778in}{1.515925in}}%
\pgfpathlineto{\pgfqpoint{1.816476in}{1.512264in}}%
\pgfpathlineto{\pgfqpoint{1.830684in}{1.502314in}}%
\pgfpathlineto{\pgfqpoint{1.832132in}{1.501251in}}%
\pgfpathlineto{\pgfqpoint{1.847789in}{1.491379in}}%
\pgfpathlineto{\pgfqpoint{1.853004in}{1.488703in}}%
\pgfpathlineto{\pgfqpoint{1.863445in}{1.483023in}}%
\pgfpathlineto{\pgfqpoint{1.879102in}{1.476708in}}%
\pgfpathlineto{\pgfqpoint{1.885616in}{1.475092in}}%
\pgfpathlineto{\pgfqpoint{1.894758in}{1.472662in}}%
\pgfpathclose%
\pgfusepath{fill}%
\end{pgfscope}%
\begin{pgfscope}%
\pgfpathrectangle{\pgfqpoint{0.360415in}{0.345370in}}{\pgfqpoint{1.550000in}{1.347500in}}%
\pgfusepath{clip}%
\pgfsetbuttcap%
\pgfsetroundjoin%
\definecolor{currentfill}{rgb}{0.362553,0.003243,0.649245}%
\pgfsetfillcolor{currentfill}%
\pgfsetlinewidth{0.000000pt}%
\definecolor{currentstroke}{rgb}{0.000000,0.000000,0.000000}%
\pgfsetstrokecolor{currentstroke}%
\pgfsetdash{}{0pt}%
\pgfpathmoveto{\pgfqpoint{0.454354in}{0.345370in}}%
\pgfpathlineto{\pgfqpoint{0.470011in}{0.345370in}}%
\pgfpathlineto{\pgfqpoint{0.485668in}{0.345370in}}%
\pgfpathlineto{\pgfqpoint{0.501324in}{0.345370in}}%
\pgfpathlineto{\pgfqpoint{0.516981in}{0.345370in}}%
\pgfpathlineto{\pgfqpoint{0.532637in}{0.345370in}}%
\pgfpathlineto{\pgfqpoint{0.536323in}{0.345370in}}%
\pgfpathlineto{\pgfqpoint{0.535088in}{0.358981in}}%
\pgfpathlineto{\pgfqpoint{0.532637in}{0.368093in}}%
\pgfpathlineto{\pgfqpoint{0.531408in}{0.372592in}}%
\pgfpathlineto{\pgfqpoint{0.525354in}{0.386203in}}%
\pgfpathlineto{\pgfqpoint{0.517231in}{0.399814in}}%
\pgfpathlineto{\pgfqpoint{0.516981in}{0.400152in}}%
\pgfpathlineto{\pgfqpoint{0.506954in}{0.413425in}}%
\pgfpathlineto{\pgfqpoint{0.501324in}{0.419873in}}%
\pgfpathlineto{\pgfqpoint{0.494834in}{0.427036in}}%
\pgfpathlineto{\pgfqpoint{0.485668in}{0.436136in}}%
\pgfpathlineto{\pgfqpoint{0.480865in}{0.440648in}}%
\pgfpathlineto{\pgfqpoint{0.470011in}{0.450083in}}%
\pgfpathlineto{\pgfqpoint{0.464821in}{0.454259in}}%
\pgfpathlineto{\pgfqpoint{0.454354in}{0.462228in}}%
\pgfpathlineto{\pgfqpoint{0.446114in}{0.467870in}}%
\pgfpathlineto{\pgfqpoint{0.438698in}{0.472764in}}%
\pgfpathlineto{\pgfqpoint{0.423430in}{0.481481in}}%
\pgfpathlineto{\pgfqpoint{0.423041in}{0.481698in}}%
\pgfpathlineto{\pgfqpoint{0.407385in}{0.488761in}}%
\pgfpathlineto{\pgfqpoint{0.391728in}{0.494023in}}%
\pgfpathlineto{\pgfqpoint{0.386554in}{0.495092in}}%
\pgfpathlineto{\pgfqpoint{0.376072in}{0.497222in}}%
\pgfpathlineto{\pgfqpoint{0.360415in}{0.498296in}}%
\pgfpathlineto{\pgfqpoint{0.360415in}{0.495092in}}%
\pgfpathlineto{\pgfqpoint{0.360415in}{0.481481in}}%
\pgfpathlineto{\pgfqpoint{0.360415in}{0.467870in}}%
\pgfpathlineto{\pgfqpoint{0.360415in}{0.454259in}}%
\pgfpathlineto{\pgfqpoint{0.360415in}{0.440648in}}%
\pgfpathlineto{\pgfqpoint{0.360415in}{0.427036in}}%
\pgfpathlineto{\pgfqpoint{0.360415in}{0.420133in}}%
\pgfpathlineto{\pgfqpoint{0.376072in}{0.418705in}}%
\pgfpathlineto{\pgfqpoint{0.391728in}{0.414483in}}%
\pgfpathlineto{\pgfqpoint{0.394126in}{0.413425in}}%
\pgfpathlineto{\pgfqpoint{0.407385in}{0.406548in}}%
\pgfpathlineto{\pgfqpoint{0.416984in}{0.399814in}}%
\pgfpathlineto{\pgfqpoint{0.423041in}{0.394549in}}%
\pgfpathlineto{\pgfqpoint{0.430787in}{0.386203in}}%
\pgfpathlineto{\pgfqpoint{0.438698in}{0.374676in}}%
\pgfpathlineto{\pgfqpoint{0.439914in}{0.372592in}}%
\pgfpathlineto{\pgfqpoint{0.444772in}{0.358981in}}%
\pgfpathlineto{\pgfqpoint{0.446414in}{0.345370in}}%
\pgfpathlineto{\pgfqpoint{0.454354in}{0.345370in}}%
\pgfpathclose%
\pgfpathmoveto{\pgfqpoint{0.971021in}{0.345370in}}%
\pgfpathlineto{\pgfqpoint{0.986678in}{0.345370in}}%
\pgfpathlineto{\pgfqpoint{1.002334in}{0.345370in}}%
\pgfpathlineto{\pgfqpoint{1.017991in}{0.345370in}}%
\pgfpathlineto{\pgfqpoint{1.033647in}{0.345370in}}%
\pgfpathlineto{\pgfqpoint{1.049132in}{0.345370in}}%
\pgfpathlineto{\pgfqpoint{1.049304in}{0.346900in}}%
\pgfpathlineto{\pgfqpoint{1.050884in}{0.358981in}}%
\pgfpathlineto{\pgfqpoint{1.056152in}{0.372592in}}%
\pgfpathlineto{\pgfqpoint{1.064683in}{0.386203in}}%
\pgfpathlineto{\pgfqpoint{1.064960in}{0.386528in}}%
\pgfpathlineto{\pgfqpoint{1.078589in}{0.399814in}}%
\pgfpathlineto{\pgfqpoint{1.080617in}{0.401412in}}%
\pgfpathlineto{\pgfqpoint{1.096274in}{0.410986in}}%
\pgfpathlineto{\pgfqpoint{1.102081in}{0.413425in}}%
\pgfpathlineto{\pgfqpoint{1.111930in}{0.416935in}}%
\pgfpathlineto{\pgfqpoint{1.127587in}{0.419775in}}%
\pgfpathlineto{\pgfqpoint{1.143243in}{0.419775in}}%
\pgfpathlineto{\pgfqpoint{1.158900in}{0.416935in}}%
\pgfpathlineto{\pgfqpoint{1.168749in}{0.413425in}}%
\pgfpathlineto{\pgfqpoint{1.174556in}{0.410986in}}%
\pgfpathlineto{\pgfqpoint{1.190213in}{0.401412in}}%
\pgfpathlineto{\pgfqpoint{1.192241in}{0.399814in}}%
\pgfpathlineto{\pgfqpoint{1.205870in}{0.386528in}}%
\pgfpathlineto{\pgfqpoint{1.206147in}{0.386203in}}%
\pgfpathlineto{\pgfqpoint{1.214678in}{0.372592in}}%
\pgfpathlineto{\pgfqpoint{1.219946in}{0.358981in}}%
\pgfpathlineto{\pgfqpoint{1.221526in}{0.346900in}}%
\pgfpathlineto{\pgfqpoint{1.221698in}{0.345370in}}%
\pgfpathlineto{\pgfqpoint{1.237183in}{0.345370in}}%
\pgfpathlineto{\pgfqpoint{1.252839in}{0.345370in}}%
\pgfpathlineto{\pgfqpoint{1.268496in}{0.345370in}}%
\pgfpathlineto{\pgfqpoint{1.284152in}{0.345370in}}%
\pgfpathlineto{\pgfqpoint{1.299809in}{0.345370in}}%
\pgfpathlineto{\pgfqpoint{1.311313in}{0.345370in}}%
\pgfpathlineto{\pgfqpoint{1.310066in}{0.358981in}}%
\pgfpathlineto{\pgfqpoint{1.306379in}{0.372592in}}%
\pgfpathlineto{\pgfqpoint{1.300408in}{0.386203in}}%
\pgfpathlineto{\pgfqpoint{1.299809in}{0.387206in}}%
\pgfpathlineto{\pgfqpoint{1.292156in}{0.399814in}}%
\pgfpathlineto{\pgfqpoint{1.284152in}{0.410602in}}%
\pgfpathlineto{\pgfqpoint{1.281992in}{0.413425in}}%
\pgfpathlineto{\pgfqpoint{1.269923in}{0.427036in}}%
\pgfpathlineto{\pgfqpoint{1.268496in}{0.428478in}}%
\pgfpathlineto{\pgfqpoint{1.255909in}{0.440648in}}%
\pgfpathlineto{\pgfqpoint{1.252839in}{0.443397in}}%
\pgfpathlineto{\pgfqpoint{1.239897in}{0.454259in}}%
\pgfpathlineto{\pgfqpoint{1.237183in}{0.456417in}}%
\pgfpathlineto{\pgfqpoint{1.221526in}{0.467790in}}%
\pgfpathlineto{\pgfqpoint{1.221401in}{0.467870in}}%
\pgfpathlineto{\pgfqpoint{1.205870in}{0.477415in}}%
\pgfpathlineto{\pgfqpoint{1.198026in}{0.481481in}}%
\pgfpathlineto{\pgfqpoint{1.190213in}{0.485438in}}%
\pgfpathlineto{\pgfqpoint{1.174556in}{0.491632in}}%
\pgfpathlineto{\pgfqpoint{1.161871in}{0.495092in}}%
\pgfpathlineto{\pgfqpoint{1.158900in}{0.495891in}}%
\pgfpathlineto{\pgfqpoint{1.143243in}{0.498027in}}%
\pgfpathlineto{\pgfqpoint{1.127587in}{0.498027in}}%
\pgfpathlineto{\pgfqpoint{1.111930in}{0.495891in}}%
\pgfpathlineto{\pgfqpoint{1.108959in}{0.495092in}}%
\pgfpathlineto{\pgfqpoint{1.096274in}{0.491632in}}%
\pgfpathlineto{\pgfqpoint{1.080617in}{0.485438in}}%
\pgfpathlineto{\pgfqpoint{1.072804in}{0.481481in}}%
\pgfpathlineto{\pgfqpoint{1.064960in}{0.477415in}}%
\pgfpathlineto{\pgfqpoint{1.049429in}{0.467870in}}%
\pgfpathlineto{\pgfqpoint{1.049304in}{0.467790in}}%
\pgfpathlineto{\pgfqpoint{1.033647in}{0.456417in}}%
\pgfpathlineto{\pgfqpoint{1.030933in}{0.454259in}}%
\pgfpathlineto{\pgfqpoint{1.017991in}{0.443397in}}%
\pgfpathlineto{\pgfqpoint{1.014921in}{0.440648in}}%
\pgfpathlineto{\pgfqpoint{1.002334in}{0.428478in}}%
\pgfpathlineto{\pgfqpoint{1.000907in}{0.427036in}}%
\pgfpathlineto{\pgfqpoint{0.988838in}{0.413425in}}%
\pgfpathlineto{\pgfqpoint{0.986678in}{0.410602in}}%
\pgfpathlineto{\pgfqpoint{0.978674in}{0.399814in}}%
\pgfpathlineto{\pgfqpoint{0.971021in}{0.387206in}}%
\pgfpathlineto{\pgfqpoint{0.970422in}{0.386203in}}%
\pgfpathlineto{\pgfqpoint{0.964451in}{0.372592in}}%
\pgfpathlineto{\pgfqpoint{0.960764in}{0.358981in}}%
\pgfpathlineto{\pgfqpoint{0.959517in}{0.345370in}}%
\pgfpathlineto{\pgfqpoint{0.971021in}{0.345370in}}%
\pgfpathclose%
\pgfpathmoveto{\pgfqpoint{1.738193in}{0.345370in}}%
\pgfpathlineto{\pgfqpoint{1.753849in}{0.345370in}}%
\pgfpathlineto{\pgfqpoint{1.769506in}{0.345370in}}%
\pgfpathlineto{\pgfqpoint{1.785162in}{0.345370in}}%
\pgfpathlineto{\pgfqpoint{1.800819in}{0.345370in}}%
\pgfpathlineto{\pgfqpoint{1.816476in}{0.345370in}}%
\pgfpathlineto{\pgfqpoint{1.824416in}{0.345370in}}%
\pgfpathlineto{\pgfqpoint{1.826058in}{0.358981in}}%
\pgfpathlineto{\pgfqpoint{1.830916in}{0.372592in}}%
\pgfpathlineto{\pgfqpoint{1.832132in}{0.374676in}}%
\pgfpathlineto{\pgfqpoint{1.840043in}{0.386203in}}%
\pgfpathlineto{\pgfqpoint{1.847789in}{0.394549in}}%
\pgfpathlineto{\pgfqpoint{1.853846in}{0.399814in}}%
\pgfpathlineto{\pgfqpoint{1.863445in}{0.406548in}}%
\pgfpathlineto{\pgfqpoint{1.876704in}{0.413425in}}%
\pgfpathlineto{\pgfqpoint{1.879102in}{0.414483in}}%
\pgfpathlineto{\pgfqpoint{1.894758in}{0.418705in}}%
\pgfpathlineto{\pgfqpoint{1.910415in}{0.420133in}}%
\pgfpathlineto{\pgfqpoint{1.910415in}{0.427036in}}%
\pgfpathlineto{\pgfqpoint{1.910415in}{0.440648in}}%
\pgfpathlineto{\pgfqpoint{1.910415in}{0.454259in}}%
\pgfpathlineto{\pgfqpoint{1.910415in}{0.467870in}}%
\pgfpathlineto{\pgfqpoint{1.910415in}{0.481481in}}%
\pgfpathlineto{\pgfqpoint{1.910415in}{0.495092in}}%
\pgfpathlineto{\pgfqpoint{1.910415in}{0.498296in}}%
\pgfpathlineto{\pgfqpoint{1.894758in}{0.497222in}}%
\pgfpathlineto{\pgfqpoint{1.884276in}{0.495092in}}%
\pgfpathlineto{\pgfqpoint{1.879102in}{0.494023in}}%
\pgfpathlineto{\pgfqpoint{1.863445in}{0.488761in}}%
\pgfpathlineto{\pgfqpoint{1.847789in}{0.481698in}}%
\pgfpathlineto{\pgfqpoint{1.847400in}{0.481481in}}%
\pgfpathlineto{\pgfqpoint{1.832132in}{0.472764in}}%
\pgfpathlineto{\pgfqpoint{1.824716in}{0.467870in}}%
\pgfpathlineto{\pgfqpoint{1.816476in}{0.462228in}}%
\pgfpathlineto{\pgfqpoint{1.806009in}{0.454259in}}%
\pgfpathlineto{\pgfqpoint{1.800819in}{0.450083in}}%
\pgfpathlineto{\pgfqpoint{1.789965in}{0.440648in}}%
\pgfpathlineto{\pgfqpoint{1.785162in}{0.436136in}}%
\pgfpathlineto{\pgfqpoint{1.775996in}{0.427036in}}%
\pgfpathlineto{\pgfqpoint{1.769506in}{0.419873in}}%
\pgfpathlineto{\pgfqpoint{1.763876in}{0.413425in}}%
\pgfpathlineto{\pgfqpoint{1.753849in}{0.400152in}}%
\pgfpathlineto{\pgfqpoint{1.753599in}{0.399814in}}%
\pgfpathlineto{\pgfqpoint{1.745476in}{0.386203in}}%
\pgfpathlineto{\pgfqpoint{1.739422in}{0.372592in}}%
\pgfpathlineto{\pgfqpoint{1.738193in}{0.368093in}}%
\pgfpathlineto{\pgfqpoint{1.735742in}{0.358981in}}%
\pgfpathlineto{\pgfqpoint{1.734507in}{0.345370in}}%
\pgfpathlineto{\pgfqpoint{1.738193in}{0.345370in}}%
\pgfpathclose%
\pgfpathmoveto{\pgfqpoint{0.376072in}{0.867286in}}%
\pgfpathlineto{\pgfqpoint{0.391728in}{0.870491in}}%
\pgfpathlineto{\pgfqpoint{0.407385in}{0.875682in}}%
\pgfpathlineto{\pgfqpoint{0.408538in}{0.876203in}}%
\pgfpathlineto{\pgfqpoint{0.423041in}{0.882856in}}%
\pgfpathlineto{\pgfqpoint{0.435450in}{0.889814in}}%
\pgfpathlineto{\pgfqpoint{0.438698in}{0.891692in}}%
\pgfpathlineto{\pgfqpoint{0.454354in}{0.902185in}}%
\pgfpathlineto{\pgfqpoint{0.456013in}{0.903425in}}%
\pgfpathlineto{\pgfqpoint{0.470011in}{0.914367in}}%
\pgfpathlineto{\pgfqpoint{0.473173in}{0.917036in}}%
\pgfpathlineto{\pgfqpoint{0.485668in}{0.928288in}}%
\pgfpathlineto{\pgfqpoint{0.488150in}{0.930648in}}%
\pgfpathlineto{\pgfqpoint{0.501233in}{0.944259in}}%
\pgfpathlineto{\pgfqpoint{0.501324in}{0.944367in}}%
\pgfpathlineto{\pgfqpoint{0.512304in}{0.957870in}}%
\pgfpathlineto{\pgfqpoint{0.516981in}{0.964688in}}%
\pgfpathlineto{\pgfqpoint{0.521533in}{0.971481in}}%
\pgfpathlineto{\pgfqpoint{0.528657in}{0.985092in}}%
\pgfpathlineto{\pgfqpoint{0.532637in}{0.996120in}}%
\pgfpathlineto{\pgfqpoint{0.533556in}{0.998703in}}%
\pgfpathlineto{\pgfqpoint{0.536013in}{1.012314in}}%
\pgfpathlineto{\pgfqpoint{0.536013in}{1.025925in}}%
\pgfpathlineto{\pgfqpoint{0.533556in}{1.039536in}}%
\pgfpathlineto{\pgfqpoint{0.532637in}{1.042119in}}%
\pgfpathlineto{\pgfqpoint{0.528657in}{1.053148in}}%
\pgfpathlineto{\pgfqpoint{0.521533in}{1.066759in}}%
\pgfpathlineto{\pgfqpoint{0.516981in}{1.073551in}}%
\pgfpathlineto{\pgfqpoint{0.512304in}{1.080370in}}%
\pgfpathlineto{\pgfqpoint{0.501324in}{1.093872in}}%
\pgfpathlineto{\pgfqpoint{0.501233in}{1.093981in}}%
\pgfpathlineto{\pgfqpoint{0.488150in}{1.107592in}}%
\pgfpathlineto{\pgfqpoint{0.485668in}{1.109952in}}%
\pgfpathlineto{\pgfqpoint{0.473173in}{1.121203in}}%
\pgfpathlineto{\pgfqpoint{0.470011in}{1.123872in}}%
\pgfpathlineto{\pgfqpoint{0.456013in}{1.134814in}}%
\pgfpathlineto{\pgfqpoint{0.454354in}{1.136054in}}%
\pgfpathlineto{\pgfqpoint{0.438698in}{1.146548in}}%
\pgfpathlineto{\pgfqpoint{0.435450in}{1.148425in}}%
\pgfpathlineto{\pgfqpoint{0.423041in}{1.155384in}}%
\pgfpathlineto{\pgfqpoint{0.408538in}{1.162036in}}%
\pgfpathlineto{\pgfqpoint{0.407385in}{1.162557in}}%
\pgfpathlineto{\pgfqpoint{0.391728in}{1.167748in}}%
\pgfpathlineto{\pgfqpoint{0.376072in}{1.170953in}}%
\pgfpathlineto{\pgfqpoint{0.360415in}{1.172037in}}%
\pgfpathlineto{\pgfqpoint{0.360415in}{1.162036in}}%
\pgfpathlineto{\pgfqpoint{0.360415in}{1.148425in}}%
\pgfpathlineto{\pgfqpoint{0.360415in}{1.134814in}}%
\pgfpathlineto{\pgfqpoint{0.360415in}{1.121203in}}%
\pgfpathlineto{\pgfqpoint{0.360415in}{1.107592in}}%
\pgfpathlineto{\pgfqpoint{0.360415in}{1.094131in}}%
\pgfpathlineto{\pgfqpoint{0.362175in}{1.093981in}}%
\pgfpathlineto{\pgfqpoint{0.376072in}{1.092607in}}%
\pgfpathlineto{\pgfqpoint{0.391728in}{1.088028in}}%
\pgfpathlineto{\pgfqpoint{0.407385in}{1.080611in}}%
\pgfpathlineto{\pgfqpoint{0.407759in}{1.080370in}}%
\pgfpathlineto{\pgfqpoint{0.423041in}{1.068521in}}%
\pgfpathlineto{\pgfqpoint{0.424879in}{1.066759in}}%
\pgfpathlineto{\pgfqpoint{0.435892in}{1.053148in}}%
\pgfpathlineto{\pgfqpoint{0.438698in}{1.048099in}}%
\pgfpathlineto{\pgfqpoint{0.442735in}{1.039536in}}%
\pgfpathlineto{\pgfqpoint{0.446002in}{1.025925in}}%
\pgfpathlineto{\pgfqpoint{0.446002in}{1.012314in}}%
\pgfpathlineto{\pgfqpoint{0.442735in}{0.998703in}}%
\pgfpathlineto{\pgfqpoint{0.438698in}{0.990140in}}%
\pgfpathlineto{\pgfqpoint{0.435892in}{0.985092in}}%
\pgfpathlineto{\pgfqpoint{0.424879in}{0.971481in}}%
\pgfpathlineto{\pgfqpoint{0.423041in}{0.969718in}}%
\pgfpathlineto{\pgfqpoint{0.407759in}{0.957870in}}%
\pgfpathlineto{\pgfqpoint{0.407385in}{0.957628in}}%
\pgfpathlineto{\pgfqpoint{0.391728in}{0.950212in}}%
\pgfpathlineto{\pgfqpoint{0.376072in}{0.945633in}}%
\pgfpathlineto{\pgfqpoint{0.362175in}{0.944259in}}%
\pgfpathlineto{\pgfqpoint{0.360415in}{0.944109in}}%
\pgfpathlineto{\pgfqpoint{0.360415in}{0.930648in}}%
\pgfpathlineto{\pgfqpoint{0.360415in}{0.917036in}}%
\pgfpathlineto{\pgfqpoint{0.360415in}{0.903425in}}%
\pgfpathlineto{\pgfqpoint{0.360415in}{0.889814in}}%
\pgfpathlineto{\pgfqpoint{0.360415in}{0.876203in}}%
\pgfpathlineto{\pgfqpoint{0.360415in}{0.866202in}}%
\pgfpathlineto{\pgfqpoint{0.376072in}{0.867286in}}%
\pgfpathclose%
\pgfpathmoveto{\pgfqpoint{1.096274in}{0.872850in}}%
\pgfpathlineto{\pgfqpoint{1.111930in}{0.868630in}}%
\pgfpathlineto{\pgfqpoint{1.127587in}{0.866474in}}%
\pgfpathlineto{\pgfqpoint{1.143243in}{0.866474in}}%
\pgfpathlineto{\pgfqpoint{1.158900in}{0.868630in}}%
\pgfpathlineto{\pgfqpoint{1.174556in}{0.872850in}}%
\pgfpathlineto{\pgfqpoint{1.183098in}{0.876203in}}%
\pgfpathlineto{\pgfqpoint{1.190213in}{0.879048in}}%
\pgfpathlineto{\pgfqpoint{1.205870in}{0.887053in}}%
\pgfpathlineto{\pgfqpoint{1.210399in}{0.889814in}}%
\pgfpathlineto{\pgfqpoint{1.221526in}{0.896797in}}%
\pgfpathlineto{\pgfqpoint{1.230859in}{0.903425in}}%
\pgfpathlineto{\pgfqpoint{1.237183in}{0.908126in}}%
\pgfpathlineto{\pgfqpoint{1.248128in}{0.917036in}}%
\pgfpathlineto{\pgfqpoint{1.252839in}{0.921132in}}%
\pgfpathlineto{\pgfqpoint{1.263088in}{0.930648in}}%
\pgfpathlineto{\pgfqpoint{1.268496in}{0.936145in}}%
\pgfpathlineto{\pgfqpoint{1.276121in}{0.944259in}}%
\pgfpathlineto{\pgfqpoint{1.284152in}{0.953932in}}%
\pgfpathlineto{\pgfqpoint{1.287328in}{0.957870in}}%
\pgfpathlineto{\pgfqpoint{1.296537in}{0.971481in}}%
\pgfpathlineto{\pgfqpoint{1.299809in}{0.977667in}}%
\pgfpathlineto{\pgfqpoint{1.303666in}{0.985092in}}%
\pgfpathlineto{\pgfqpoint{1.308520in}{0.998703in}}%
\pgfpathlineto{\pgfqpoint{1.311000in}{1.012314in}}%
\pgfpathlineto{\pgfqpoint{1.311000in}{1.025925in}}%
\pgfpathlineto{\pgfqpoint{1.308520in}{1.039536in}}%
\pgfpathlineto{\pgfqpoint{1.303666in}{1.053148in}}%
\pgfpathlineto{\pgfqpoint{1.299809in}{1.060573in}}%
\pgfpathlineto{\pgfqpoint{1.296537in}{1.066759in}}%
\pgfpathlineto{\pgfqpoint{1.287328in}{1.080370in}}%
\pgfpathlineto{\pgfqpoint{1.284152in}{1.084307in}}%
\pgfpathlineto{\pgfqpoint{1.276121in}{1.093981in}}%
\pgfpathlineto{\pgfqpoint{1.268496in}{1.102094in}}%
\pgfpathlineto{\pgfqpoint{1.263088in}{1.107592in}}%
\pgfpathlineto{\pgfqpoint{1.252839in}{1.117107in}}%
\pgfpathlineto{\pgfqpoint{1.248128in}{1.121203in}}%
\pgfpathlineto{\pgfqpoint{1.237183in}{1.130113in}}%
\pgfpathlineto{\pgfqpoint{1.230859in}{1.134814in}}%
\pgfpathlineto{\pgfqpoint{1.221526in}{1.141443in}}%
\pgfpathlineto{\pgfqpoint{1.210399in}{1.148425in}}%
\pgfpathlineto{\pgfqpoint{1.205870in}{1.151186in}}%
\pgfpathlineto{\pgfqpoint{1.190213in}{1.159192in}}%
\pgfpathlineto{\pgfqpoint{1.183098in}{1.162036in}}%
\pgfpathlineto{\pgfqpoint{1.174556in}{1.165389in}}%
\pgfpathlineto{\pgfqpoint{1.158900in}{1.169609in}}%
\pgfpathlineto{\pgfqpoint{1.143243in}{1.171765in}}%
\pgfpathlineto{\pgfqpoint{1.127587in}{1.171765in}}%
\pgfpathlineto{\pgfqpoint{1.111930in}{1.169609in}}%
\pgfpathlineto{\pgfqpoint{1.096274in}{1.165389in}}%
\pgfpathlineto{\pgfqpoint{1.087732in}{1.162036in}}%
\pgfpathlineto{\pgfqpoint{1.080617in}{1.159192in}}%
\pgfpathlineto{\pgfqpoint{1.064960in}{1.151186in}}%
\pgfpathlineto{\pgfqpoint{1.060431in}{1.148425in}}%
\pgfpathlineto{\pgfqpoint{1.049304in}{1.141443in}}%
\pgfpathlineto{\pgfqpoint{1.039971in}{1.134814in}}%
\pgfpathlineto{\pgfqpoint{1.033647in}{1.130113in}}%
\pgfpathlineto{\pgfqpoint{1.022702in}{1.121203in}}%
\pgfpathlineto{\pgfqpoint{1.017991in}{1.117107in}}%
\pgfpathlineto{\pgfqpoint{1.007742in}{1.107592in}}%
\pgfpathlineto{\pgfqpoint{1.002334in}{1.102094in}}%
\pgfpathlineto{\pgfqpoint{0.994709in}{1.093981in}}%
\pgfpathlineto{\pgfqpoint{0.986678in}{1.084307in}}%
\pgfpathlineto{\pgfqpoint{0.983502in}{1.080370in}}%
\pgfpathlineto{\pgfqpoint{0.974293in}{1.066759in}}%
\pgfpathlineto{\pgfqpoint{0.971021in}{1.060573in}}%
\pgfpathlineto{\pgfqpoint{0.967164in}{1.053147in}}%
\pgfpathlineto{\pgfqpoint{0.962310in}{1.039536in}}%
\pgfpathlineto{\pgfqpoint{0.959830in}{1.025925in}}%
\pgfpathlineto{\pgfqpoint{0.959830in}{1.012314in}}%
\pgfpathlineto{\pgfqpoint{0.962310in}{0.998703in}}%
\pgfpathlineto{\pgfqpoint{0.967164in}{0.985092in}}%
\pgfpathlineto{\pgfqpoint{0.971021in}{0.977667in}}%
\pgfpathlineto{\pgfqpoint{0.974293in}{0.971481in}}%
\pgfpathlineto{\pgfqpoint{0.983502in}{0.957870in}}%
\pgfpathlineto{\pgfqpoint{0.986678in}{0.953932in}}%
\pgfpathlineto{\pgfqpoint{0.994709in}{0.944259in}}%
\pgfpathlineto{\pgfqpoint{1.002334in}{0.936145in}}%
\pgfpathlineto{\pgfqpoint{1.007742in}{0.930648in}}%
\pgfpathlineto{\pgfqpoint{1.017991in}{0.921132in}}%
\pgfpathlineto{\pgfqpoint{1.022702in}{0.917036in}}%
\pgfpathlineto{\pgfqpoint{1.033647in}{0.908126in}}%
\pgfpathlineto{\pgfqpoint{1.039971in}{0.903425in}}%
\pgfpathlineto{\pgfqpoint{1.049304in}{0.896797in}}%
\pgfpathlineto{\pgfqpoint{1.060431in}{0.889814in}}%
\pgfpathlineto{\pgfqpoint{1.064960in}{0.887053in}}%
\pgfpathlineto{\pgfqpoint{1.080617in}{0.879048in}}%
\pgfpathlineto{\pgfqpoint{1.087732in}{0.876203in}}%
\pgfpathlineto{\pgfqpoint{1.096274in}{0.872850in}}%
\pgfpathclose%
\pgfpathmoveto{\pgfqpoint{1.088634in}{0.957870in}}%
\pgfpathlineto{\pgfqpoint{1.080617in}{0.963287in}}%
\pgfpathlineto{\pgfqpoint{1.071192in}{0.971481in}}%
\pgfpathlineto{\pgfqpoint{1.064960in}{0.978451in}}%
\pgfpathlineto{\pgfqpoint{1.060028in}{0.985092in}}%
\pgfpathlineto{\pgfqpoint{1.053093in}{0.998703in}}%
\pgfpathlineto{\pgfqpoint{1.049550in}{1.012314in}}%
\pgfpathlineto{\pgfqpoint{1.049550in}{1.025925in}}%
\pgfpathlineto{\pgfqpoint{1.053093in}{1.039536in}}%
\pgfpathlineto{\pgfqpoint{1.060028in}{1.053148in}}%
\pgfpathlineto{\pgfqpoint{1.064960in}{1.059789in}}%
\pgfpathlineto{\pgfqpoint{1.071192in}{1.066759in}}%
\pgfpathlineto{\pgfqpoint{1.080617in}{1.074952in}}%
\pgfpathlineto{\pgfqpoint{1.088634in}{1.080370in}}%
\pgfpathlineto{\pgfqpoint{1.096274in}{1.084658in}}%
\pgfpathlineto{\pgfqpoint{1.111930in}{1.090687in}}%
\pgfpathlineto{\pgfqpoint{1.127587in}{1.093767in}}%
\pgfpathlineto{\pgfqpoint{1.143243in}{1.093767in}}%
\pgfpathlineto{\pgfqpoint{1.158900in}{1.090687in}}%
\pgfpathlineto{\pgfqpoint{1.174556in}{1.084658in}}%
\pgfpathlineto{\pgfqpoint{1.182196in}{1.080370in}}%
\pgfpathlineto{\pgfqpoint{1.190213in}{1.074952in}}%
\pgfpathlineto{\pgfqpoint{1.199638in}{1.066759in}}%
\pgfpathlineto{\pgfqpoint{1.205870in}{1.059789in}}%
\pgfpathlineto{\pgfqpoint{1.210802in}{1.053148in}}%
\pgfpathlineto{\pgfqpoint{1.217737in}{1.039536in}}%
\pgfpathlineto{\pgfqpoint{1.221280in}{1.025925in}}%
\pgfpathlineto{\pgfqpoint{1.221280in}{1.012314in}}%
\pgfpathlineto{\pgfqpoint{1.217737in}{0.998703in}}%
\pgfpathlineto{\pgfqpoint{1.210802in}{0.985092in}}%
\pgfpathlineto{\pgfqpoint{1.205870in}{0.978451in}}%
\pgfpathlineto{\pgfqpoint{1.199638in}{0.971481in}}%
\pgfpathlineto{\pgfqpoint{1.190213in}{0.963287in}}%
\pgfpathlineto{\pgfqpoint{1.182196in}{0.957870in}}%
\pgfpathlineto{\pgfqpoint{1.174556in}{0.953582in}}%
\pgfpathlineto{\pgfqpoint{1.158900in}{0.947553in}}%
\pgfpathlineto{\pgfqpoint{1.143243in}{0.944472in}}%
\pgfpathlineto{\pgfqpoint{1.127587in}{0.944472in}}%
\pgfpathlineto{\pgfqpoint{1.111930in}{0.947553in}}%
\pgfpathlineto{\pgfqpoint{1.096274in}{0.953582in}}%
\pgfpathlineto{\pgfqpoint{1.088634in}{0.957870in}}%
\pgfpathclose%
\pgfpathmoveto{\pgfqpoint{1.863445in}{0.875682in}}%
\pgfpathlineto{\pgfqpoint{1.879102in}{0.870491in}}%
\pgfpathlineto{\pgfqpoint{1.894758in}{0.867286in}}%
\pgfpathlineto{\pgfqpoint{1.910415in}{0.866202in}}%
\pgfpathlineto{\pgfqpoint{1.910415in}{0.876203in}}%
\pgfpathlineto{\pgfqpoint{1.910415in}{0.889814in}}%
\pgfpathlineto{\pgfqpoint{1.910415in}{0.903425in}}%
\pgfpathlineto{\pgfqpoint{1.910415in}{0.917036in}}%
\pgfpathlineto{\pgfqpoint{1.910415in}{0.930648in}}%
\pgfpathlineto{\pgfqpoint{1.910415in}{0.944109in}}%
\pgfpathlineto{\pgfqpoint{1.908655in}{0.944259in}}%
\pgfpathlineto{\pgfqpoint{1.894758in}{0.945633in}}%
\pgfpathlineto{\pgfqpoint{1.879102in}{0.950212in}}%
\pgfpathlineto{\pgfqpoint{1.863445in}{0.957628in}}%
\pgfpathlineto{\pgfqpoint{1.863071in}{0.957870in}}%
\pgfpathlineto{\pgfqpoint{1.847789in}{0.969718in}}%
\pgfpathlineto{\pgfqpoint{1.845951in}{0.971481in}}%
\pgfpathlineto{\pgfqpoint{1.834938in}{0.985092in}}%
\pgfpathlineto{\pgfqpoint{1.832132in}{0.990140in}}%
\pgfpathlineto{\pgfqpoint{1.828095in}{0.998703in}}%
\pgfpathlineto{\pgfqpoint{1.824828in}{1.012314in}}%
\pgfpathlineto{\pgfqpoint{1.824828in}{1.025925in}}%
\pgfpathlineto{\pgfqpoint{1.828095in}{1.039536in}}%
\pgfpathlineto{\pgfqpoint{1.832132in}{1.048099in}}%
\pgfpathlineto{\pgfqpoint{1.834938in}{1.053148in}}%
\pgfpathlineto{\pgfqpoint{1.845951in}{1.066759in}}%
\pgfpathlineto{\pgfqpoint{1.847789in}{1.068521in}}%
\pgfpathlineto{\pgfqpoint{1.863071in}{1.080370in}}%
\pgfpathlineto{\pgfqpoint{1.863445in}{1.080611in}}%
\pgfpathlineto{\pgfqpoint{1.879102in}{1.088028in}}%
\pgfpathlineto{\pgfqpoint{1.894758in}{1.092607in}}%
\pgfpathlineto{\pgfqpoint{1.908655in}{1.093981in}}%
\pgfpathlineto{\pgfqpoint{1.910415in}{1.094131in}}%
\pgfpathlineto{\pgfqpoint{1.910415in}{1.107592in}}%
\pgfpathlineto{\pgfqpoint{1.910415in}{1.121203in}}%
\pgfpathlineto{\pgfqpoint{1.910415in}{1.134814in}}%
\pgfpathlineto{\pgfqpoint{1.910415in}{1.148425in}}%
\pgfpathlineto{\pgfqpoint{1.910415in}{1.162036in}}%
\pgfpathlineto{\pgfqpoint{1.910415in}{1.172037in}}%
\pgfpathlineto{\pgfqpoint{1.894758in}{1.170953in}}%
\pgfpathlineto{\pgfqpoint{1.879102in}{1.167748in}}%
\pgfpathlineto{\pgfqpoint{1.863445in}{1.162557in}}%
\pgfpathlineto{\pgfqpoint{1.862292in}{1.162036in}}%
\pgfpathlineto{\pgfqpoint{1.847789in}{1.155384in}}%
\pgfpathlineto{\pgfqpoint{1.835380in}{1.148425in}}%
\pgfpathlineto{\pgfqpoint{1.832132in}{1.146548in}}%
\pgfpathlineto{\pgfqpoint{1.816476in}{1.136054in}}%
\pgfpathlineto{\pgfqpoint{1.814817in}{1.134814in}}%
\pgfpathlineto{\pgfqpoint{1.800819in}{1.123872in}}%
\pgfpathlineto{\pgfqpoint{1.797657in}{1.121203in}}%
\pgfpathlineto{\pgfqpoint{1.785162in}{1.109952in}}%
\pgfpathlineto{\pgfqpoint{1.782680in}{1.107592in}}%
\pgfpathlineto{\pgfqpoint{1.769597in}{1.093981in}}%
\pgfpathlineto{\pgfqpoint{1.769506in}{1.093872in}}%
\pgfpathlineto{\pgfqpoint{1.758526in}{1.080370in}}%
\pgfpathlineto{\pgfqpoint{1.753849in}{1.073551in}}%
\pgfpathlineto{\pgfqpoint{1.749297in}{1.066759in}}%
\pgfpathlineto{\pgfqpoint{1.742173in}{1.053148in}}%
\pgfpathlineto{\pgfqpoint{1.738193in}{1.042119in}}%
\pgfpathlineto{\pgfqpoint{1.737274in}{1.039536in}}%
\pgfpathlineto{\pgfqpoint{1.734817in}{1.025925in}}%
\pgfpathlineto{\pgfqpoint{1.734817in}{1.012314in}}%
\pgfpathlineto{\pgfqpoint{1.737274in}{0.998703in}}%
\pgfpathlineto{\pgfqpoint{1.738193in}{0.996120in}}%
\pgfpathlineto{\pgfqpoint{1.742173in}{0.985092in}}%
\pgfpathlineto{\pgfqpoint{1.749297in}{0.971481in}}%
\pgfpathlineto{\pgfqpoint{1.753849in}{0.964688in}}%
\pgfpathlineto{\pgfqpoint{1.758526in}{0.957870in}}%
\pgfpathlineto{\pgfqpoint{1.769506in}{0.944367in}}%
\pgfpathlineto{\pgfqpoint{1.769597in}{0.944259in}}%
\pgfpathlineto{\pgfqpoint{1.782680in}{0.930647in}}%
\pgfpathlineto{\pgfqpoint{1.785162in}{0.928288in}}%
\pgfpathlineto{\pgfqpoint{1.797657in}{0.917036in}}%
\pgfpathlineto{\pgfqpoint{1.800819in}{0.914367in}}%
\pgfpathlineto{\pgfqpoint{1.814817in}{0.903425in}}%
\pgfpathlineto{\pgfqpoint{1.816476in}{0.902185in}}%
\pgfpathlineto{\pgfqpoint{1.832132in}{0.891692in}}%
\pgfpathlineto{\pgfqpoint{1.835380in}{0.889814in}}%
\pgfpathlineto{\pgfqpoint{1.847789in}{0.882856in}}%
\pgfpathlineto{\pgfqpoint{1.862292in}{0.876203in}}%
\pgfpathlineto{\pgfqpoint{1.863445in}{0.875682in}}%
\pgfpathclose%
\pgfpathmoveto{\pgfqpoint{0.376072in}{1.541017in}}%
\pgfpathlineto{\pgfqpoint{0.386554in}{1.543148in}}%
\pgfpathlineto{\pgfqpoint{0.391728in}{1.544216in}}%
\pgfpathlineto{\pgfqpoint{0.407385in}{1.549479in}}%
\pgfpathlineto{\pgfqpoint{0.423041in}{1.556541in}}%
\pgfpathlineto{\pgfqpoint{0.423430in}{1.556759in}}%
\pgfpathlineto{\pgfqpoint{0.438698in}{1.565476in}}%
\pgfpathlineto{\pgfqpoint{0.446114in}{1.570370in}}%
\pgfpathlineto{\pgfqpoint{0.454354in}{1.576012in}}%
\pgfpathlineto{\pgfqpoint{0.464821in}{1.583981in}}%
\pgfpathlineto{\pgfqpoint{0.470011in}{1.588156in}}%
\pgfpathlineto{\pgfqpoint{0.480865in}{1.597592in}}%
\pgfpathlineto{\pgfqpoint{0.485668in}{1.602104in}}%
\pgfpathlineto{\pgfqpoint{0.494834in}{1.611203in}}%
\pgfpathlineto{\pgfqpoint{0.501324in}{1.618367in}}%
\pgfpathlineto{\pgfqpoint{0.506954in}{1.624814in}}%
\pgfpathlineto{\pgfqpoint{0.516981in}{1.638087in}}%
\pgfpathlineto{\pgfqpoint{0.517231in}{1.638425in}}%
\pgfpathlineto{\pgfqpoint{0.525354in}{1.652036in}}%
\pgfpathlineto{\pgfqpoint{0.531408in}{1.665648in}}%
\pgfpathlineto{\pgfqpoint{0.532637in}{1.670146in}}%
\pgfpathlineto{\pgfqpoint{0.535088in}{1.679259in}}%
\pgfpathlineto{\pgfqpoint{0.536323in}{1.692870in}}%
\pgfpathlineto{\pgfqpoint{0.532637in}{1.692870in}}%
\pgfpathlineto{\pgfqpoint{0.516981in}{1.692870in}}%
\pgfpathlineto{\pgfqpoint{0.501324in}{1.692870in}}%
\pgfpathlineto{\pgfqpoint{0.485668in}{1.692870in}}%
\pgfpathlineto{\pgfqpoint{0.470011in}{1.692870in}}%
\pgfpathlineto{\pgfqpoint{0.454354in}{1.692870in}}%
\pgfpathlineto{\pgfqpoint{0.446414in}{1.692870in}}%
\pgfpathlineto{\pgfqpoint{0.444772in}{1.679259in}}%
\pgfpathlineto{\pgfqpoint{0.439914in}{1.665648in}}%
\pgfpathlineto{\pgfqpoint{0.438698in}{1.663563in}}%
\pgfpathlineto{\pgfqpoint{0.430787in}{1.652036in}}%
\pgfpathlineto{\pgfqpoint{0.423041in}{1.643691in}}%
\pgfpathlineto{\pgfqpoint{0.416984in}{1.638425in}}%
\pgfpathlineto{\pgfqpoint{0.407385in}{1.631692in}}%
\pgfpathlineto{\pgfqpoint{0.394126in}{1.624814in}}%
\pgfpathlineto{\pgfqpoint{0.391728in}{1.623757in}}%
\pgfpathlineto{\pgfqpoint{0.376072in}{1.619534in}}%
\pgfpathlineto{\pgfqpoint{0.360415in}{1.618106in}}%
\pgfpathlineto{\pgfqpoint{0.360415in}{1.611203in}}%
\pgfpathlineto{\pgfqpoint{0.360415in}{1.597592in}}%
\pgfpathlineto{\pgfqpoint{0.360415in}{1.583981in}}%
\pgfpathlineto{\pgfqpoint{0.360415in}{1.570370in}}%
\pgfpathlineto{\pgfqpoint{0.360415in}{1.556759in}}%
\pgfpathlineto{\pgfqpoint{0.360415in}{1.543148in}}%
\pgfpathlineto{\pgfqpoint{0.360415in}{1.539944in}}%
\pgfpathlineto{\pgfqpoint{0.376072in}{1.541017in}}%
\pgfpathclose%
\pgfpathmoveto{\pgfqpoint{1.111930in}{1.542348in}}%
\pgfpathlineto{\pgfqpoint{1.127587in}{1.540213in}}%
\pgfpathlineto{\pgfqpoint{1.143243in}{1.540213in}}%
\pgfpathlineto{\pgfqpoint{1.158900in}{1.542348in}}%
\pgfpathlineto{\pgfqpoint{1.161871in}{1.543148in}}%
\pgfpathlineto{\pgfqpoint{1.174556in}{1.546608in}}%
\pgfpathlineto{\pgfqpoint{1.190213in}{1.552801in}}%
\pgfpathlineto{\pgfqpoint{1.198026in}{1.556759in}}%
\pgfpathlineto{\pgfqpoint{1.205870in}{1.560825in}}%
\pgfpathlineto{\pgfqpoint{1.221401in}{1.570370in}}%
\pgfpathlineto{\pgfqpoint{1.221526in}{1.570449in}}%
\pgfpathlineto{\pgfqpoint{1.237183in}{1.581822in}}%
\pgfpathlineto{\pgfqpoint{1.239897in}{1.583981in}}%
\pgfpathlineto{\pgfqpoint{1.252839in}{1.594843in}}%
\pgfpathlineto{\pgfqpoint{1.255909in}{1.597592in}}%
\pgfpathlineto{\pgfqpoint{1.268496in}{1.609761in}}%
\pgfpathlineto{\pgfqpoint{1.269923in}{1.611203in}}%
\pgfpathlineto{\pgfqpoint{1.281992in}{1.624814in}}%
\pgfpathlineto{\pgfqpoint{1.284152in}{1.627638in}}%
\pgfpathlineto{\pgfqpoint{1.292156in}{1.638425in}}%
\pgfpathlineto{\pgfqpoint{1.299809in}{1.651034in}}%
\pgfpathlineto{\pgfqpoint{1.300408in}{1.652036in}}%
\pgfpathlineto{\pgfqpoint{1.306379in}{1.665648in}}%
\pgfpathlineto{\pgfqpoint{1.310066in}{1.679259in}}%
\pgfpathlineto{\pgfqpoint{1.311313in}{1.692870in}}%
\pgfpathlineto{\pgfqpoint{1.299809in}{1.692870in}}%
\pgfpathlineto{\pgfqpoint{1.284152in}{1.692870in}}%
\pgfpathlineto{\pgfqpoint{1.268496in}{1.692870in}}%
\pgfpathlineto{\pgfqpoint{1.252839in}{1.692870in}}%
\pgfpathlineto{\pgfqpoint{1.237183in}{1.692870in}}%
\pgfpathlineto{\pgfqpoint{1.221698in}{1.692870in}}%
\pgfpathlineto{\pgfqpoint{1.221526in}{1.691340in}}%
\pgfpathlineto{\pgfqpoint{1.219946in}{1.679259in}}%
\pgfpathlineto{\pgfqpoint{1.214678in}{1.665648in}}%
\pgfpathlineto{\pgfqpoint{1.206147in}{1.652036in}}%
\pgfpathlineto{\pgfqpoint{1.205870in}{1.651711in}}%
\pgfpathlineto{\pgfqpoint{1.192241in}{1.638425in}}%
\pgfpathlineto{\pgfqpoint{1.190213in}{1.636828in}}%
\pgfpathlineto{\pgfqpoint{1.174556in}{1.627253in}}%
\pgfpathlineto{\pgfqpoint{1.168749in}{1.624814in}}%
\pgfpathlineto{\pgfqpoint{1.158900in}{1.621305in}}%
\pgfpathlineto{\pgfqpoint{1.143243in}{1.618464in}}%
\pgfpathlineto{\pgfqpoint{1.127587in}{1.618464in}}%
\pgfpathlineto{\pgfqpoint{1.111930in}{1.621305in}}%
\pgfpathlineto{\pgfqpoint{1.102081in}{1.624814in}}%
\pgfpathlineto{\pgfqpoint{1.096274in}{1.627253in}}%
\pgfpathlineto{\pgfqpoint{1.080617in}{1.636828in}}%
\pgfpathlineto{\pgfqpoint{1.078589in}{1.638425in}}%
\pgfpathlineto{\pgfqpoint{1.064960in}{1.651711in}}%
\pgfpathlineto{\pgfqpoint{1.064683in}{1.652036in}}%
\pgfpathlineto{\pgfqpoint{1.056152in}{1.665648in}}%
\pgfpathlineto{\pgfqpoint{1.050884in}{1.679259in}}%
\pgfpathlineto{\pgfqpoint{1.049304in}{1.691340in}}%
\pgfpathlineto{\pgfqpoint{1.049132in}{1.692870in}}%
\pgfpathlineto{\pgfqpoint{1.033647in}{1.692870in}}%
\pgfpathlineto{\pgfqpoint{1.017991in}{1.692870in}}%
\pgfpathlineto{\pgfqpoint{1.002334in}{1.692870in}}%
\pgfpathlineto{\pgfqpoint{0.986678in}{1.692870in}}%
\pgfpathlineto{\pgfqpoint{0.971021in}{1.692870in}}%
\pgfpathlineto{\pgfqpoint{0.959517in}{1.692870in}}%
\pgfpathlineto{\pgfqpoint{0.960764in}{1.679259in}}%
\pgfpathlineto{\pgfqpoint{0.964451in}{1.665648in}}%
\pgfpathlineto{\pgfqpoint{0.970422in}{1.652036in}}%
\pgfpathlineto{\pgfqpoint{0.971021in}{1.651034in}}%
\pgfpathlineto{\pgfqpoint{0.978674in}{1.638425in}}%
\pgfpathlineto{\pgfqpoint{0.986678in}{1.627638in}}%
\pgfpathlineto{\pgfqpoint{0.988838in}{1.624814in}}%
\pgfpathlineto{\pgfqpoint{1.000907in}{1.611203in}}%
\pgfpathlineto{\pgfqpoint{1.002334in}{1.609761in}}%
\pgfpathlineto{\pgfqpoint{1.014921in}{1.597592in}}%
\pgfpathlineto{\pgfqpoint{1.017991in}{1.594843in}}%
\pgfpathlineto{\pgfqpoint{1.030933in}{1.583981in}}%
\pgfpathlineto{\pgfqpoint{1.033647in}{1.581822in}}%
\pgfpathlineto{\pgfqpoint{1.049304in}{1.570449in}}%
\pgfpathlineto{\pgfqpoint{1.049429in}{1.570370in}}%
\pgfpathlineto{\pgfqpoint{1.064960in}{1.560825in}}%
\pgfpathlineto{\pgfqpoint{1.072804in}{1.556759in}}%
\pgfpathlineto{\pgfqpoint{1.080617in}{1.552801in}}%
\pgfpathlineto{\pgfqpoint{1.096274in}{1.546608in}}%
\pgfpathlineto{\pgfqpoint{1.108959in}{1.543148in}}%
\pgfpathlineto{\pgfqpoint{1.111930in}{1.542348in}}%
\pgfpathclose%
\pgfpathmoveto{\pgfqpoint{1.894758in}{1.541017in}}%
\pgfpathlineto{\pgfqpoint{1.910415in}{1.539944in}}%
\pgfpathlineto{\pgfqpoint{1.910415in}{1.543148in}}%
\pgfpathlineto{\pgfqpoint{1.910415in}{1.556759in}}%
\pgfpathlineto{\pgfqpoint{1.910415in}{1.570370in}}%
\pgfpathlineto{\pgfqpoint{1.910415in}{1.583981in}}%
\pgfpathlineto{\pgfqpoint{1.910415in}{1.597592in}}%
\pgfpathlineto{\pgfqpoint{1.910415in}{1.611203in}}%
\pgfpathlineto{\pgfqpoint{1.910415in}{1.618106in}}%
\pgfpathlineto{\pgfqpoint{1.894758in}{1.619534in}}%
\pgfpathlineto{\pgfqpoint{1.879102in}{1.623757in}}%
\pgfpathlineto{\pgfqpoint{1.876704in}{1.624814in}}%
\pgfpathlineto{\pgfqpoint{1.863445in}{1.631692in}}%
\pgfpathlineto{\pgfqpoint{1.853846in}{1.638425in}}%
\pgfpathlineto{\pgfqpoint{1.847789in}{1.643691in}}%
\pgfpathlineto{\pgfqpoint{1.840043in}{1.652036in}}%
\pgfpathlineto{\pgfqpoint{1.832132in}{1.663563in}}%
\pgfpathlineto{\pgfqpoint{1.830916in}{1.665648in}}%
\pgfpathlineto{\pgfqpoint{1.826058in}{1.679259in}}%
\pgfpathlineto{\pgfqpoint{1.824416in}{1.692870in}}%
\pgfpathlineto{\pgfqpoint{1.816476in}{1.692870in}}%
\pgfpathlineto{\pgfqpoint{1.800819in}{1.692870in}}%
\pgfpathlineto{\pgfqpoint{1.785162in}{1.692870in}}%
\pgfpathlineto{\pgfqpoint{1.769506in}{1.692870in}}%
\pgfpathlineto{\pgfqpoint{1.753849in}{1.692870in}}%
\pgfpathlineto{\pgfqpoint{1.738193in}{1.692870in}}%
\pgfpathlineto{\pgfqpoint{1.734507in}{1.692870in}}%
\pgfpathlineto{\pgfqpoint{1.735742in}{1.679259in}}%
\pgfpathlineto{\pgfqpoint{1.738193in}{1.670146in}}%
\pgfpathlineto{\pgfqpoint{1.739422in}{1.665648in}}%
\pgfpathlineto{\pgfqpoint{1.745476in}{1.652036in}}%
\pgfpathlineto{\pgfqpoint{1.753599in}{1.638425in}}%
\pgfpathlineto{\pgfqpoint{1.753849in}{1.638087in}}%
\pgfpathlineto{\pgfqpoint{1.763876in}{1.624814in}}%
\pgfpathlineto{\pgfqpoint{1.769506in}{1.618367in}}%
\pgfpathlineto{\pgfqpoint{1.775996in}{1.611203in}}%
\pgfpathlineto{\pgfqpoint{1.785162in}{1.602104in}}%
\pgfpathlineto{\pgfqpoint{1.789965in}{1.597592in}}%
\pgfpathlineto{\pgfqpoint{1.800819in}{1.588156in}}%
\pgfpathlineto{\pgfqpoint{1.806009in}{1.583981in}}%
\pgfpathlineto{\pgfqpoint{1.816476in}{1.576012in}}%
\pgfpathlineto{\pgfqpoint{1.824716in}{1.570370in}}%
\pgfpathlineto{\pgfqpoint{1.832132in}{1.565476in}}%
\pgfpathlineto{\pgfqpoint{1.847400in}{1.556759in}}%
\pgfpathlineto{\pgfqpoint{1.847789in}{1.556541in}}%
\pgfpathlineto{\pgfqpoint{1.863445in}{1.549479in}}%
\pgfpathlineto{\pgfqpoint{1.879102in}{1.544216in}}%
\pgfpathlineto{\pgfqpoint{1.884276in}{1.543148in}}%
\pgfpathlineto{\pgfqpoint{1.894758in}{1.541017in}}%
\pgfpathclose%
\pgfusepath{fill}%
\end{pgfscope}%
\begin{pgfscope}%
\pgfpathrectangle{\pgfqpoint{0.360415in}{0.345370in}}{\pgfqpoint{1.550000in}{1.347500in}}%
\pgfusepath{clip}%
\pgfsetbuttcap%
\pgfsetroundjoin%
\definecolor{currentfill}{rgb}{0.178950,0.019252,0.584054}%
\pgfsetfillcolor{currentfill}%
\pgfsetlinewidth{0.000000pt}%
\definecolor{currentstroke}{rgb}{0.000000,0.000000,0.000000}%
\pgfsetstrokecolor{currentstroke}%
\pgfsetdash{}{0pt}%
\pgfpathmoveto{\pgfqpoint{0.376072in}{0.345370in}}%
\pgfpathlineto{\pgfqpoint{0.391728in}{0.345370in}}%
\pgfpathlineto{\pgfqpoint{0.407385in}{0.345370in}}%
\pgfpathlineto{\pgfqpoint{0.423041in}{0.345370in}}%
\pgfpathlineto{\pgfqpoint{0.438698in}{0.345370in}}%
\pgfpathlineto{\pgfqpoint{0.446414in}{0.345370in}}%
\pgfpathlineto{\pgfqpoint{0.444772in}{0.358981in}}%
\pgfpathlineto{\pgfqpoint{0.439914in}{0.372592in}}%
\pgfpathlineto{\pgfqpoint{0.438698in}{0.374676in}}%
\pgfpathlineto{\pgfqpoint{0.430787in}{0.386203in}}%
\pgfpathlineto{\pgfqpoint{0.423041in}{0.394549in}}%
\pgfpathlineto{\pgfqpoint{0.416984in}{0.399814in}}%
\pgfpathlineto{\pgfqpoint{0.407385in}{0.406548in}}%
\pgfpathlineto{\pgfqpoint{0.394126in}{0.413425in}}%
\pgfpathlineto{\pgfqpoint{0.391728in}{0.414483in}}%
\pgfpathlineto{\pgfqpoint{0.376072in}{0.418705in}}%
\pgfpathlineto{\pgfqpoint{0.360415in}{0.420133in}}%
\pgfpathlineto{\pgfqpoint{0.360415in}{0.413425in}}%
\pgfpathlineto{\pgfqpoint{0.360415in}{0.399814in}}%
\pgfpathlineto{\pgfqpoint{0.360415in}{0.386203in}}%
\pgfpathlineto{\pgfqpoint{0.360415in}{0.372592in}}%
\pgfpathlineto{\pgfqpoint{0.360415in}{0.358981in}}%
\pgfpathlineto{\pgfqpoint{0.360415in}{0.345370in}}%
\pgfpathlineto{\pgfqpoint{0.376072in}{0.345370in}}%
\pgfpathclose%
\pgfpathmoveto{\pgfqpoint{1.049304in}{0.345370in}}%
\pgfpathlineto{\pgfqpoint{1.064960in}{0.345370in}}%
\pgfpathlineto{\pgfqpoint{1.080617in}{0.345370in}}%
\pgfpathlineto{\pgfqpoint{1.096274in}{0.345370in}}%
\pgfpathlineto{\pgfqpoint{1.111930in}{0.345370in}}%
\pgfpathlineto{\pgfqpoint{1.127587in}{0.345370in}}%
\pgfpathlineto{\pgfqpoint{1.143243in}{0.345370in}}%
\pgfpathlineto{\pgfqpoint{1.158900in}{0.345370in}}%
\pgfpathlineto{\pgfqpoint{1.174556in}{0.345370in}}%
\pgfpathlineto{\pgfqpoint{1.190213in}{0.345370in}}%
\pgfpathlineto{\pgfqpoint{1.205870in}{0.345370in}}%
\pgfpathlineto{\pgfqpoint{1.221526in}{0.345370in}}%
\pgfpathlineto{\pgfqpoint{1.221698in}{0.345370in}}%
\pgfpathlineto{\pgfqpoint{1.221526in}{0.346900in}}%
\pgfpathlineto{\pgfqpoint{1.219946in}{0.358981in}}%
\pgfpathlineto{\pgfqpoint{1.214678in}{0.372592in}}%
\pgfpathlineto{\pgfqpoint{1.206147in}{0.386203in}}%
\pgfpathlineto{\pgfqpoint{1.205870in}{0.386528in}}%
\pgfpathlineto{\pgfqpoint{1.192241in}{0.399814in}}%
\pgfpathlineto{\pgfqpoint{1.190213in}{0.401412in}}%
\pgfpathlineto{\pgfqpoint{1.174556in}{0.410986in}}%
\pgfpathlineto{\pgfqpoint{1.168749in}{0.413425in}}%
\pgfpathlineto{\pgfqpoint{1.158900in}{0.416935in}}%
\pgfpathlineto{\pgfqpoint{1.143243in}{0.419775in}}%
\pgfpathlineto{\pgfqpoint{1.127587in}{0.419775in}}%
\pgfpathlineto{\pgfqpoint{1.111930in}{0.416935in}}%
\pgfpathlineto{\pgfqpoint{1.102081in}{0.413425in}}%
\pgfpathlineto{\pgfqpoint{1.096274in}{0.410986in}}%
\pgfpathlineto{\pgfqpoint{1.080617in}{0.401412in}}%
\pgfpathlineto{\pgfqpoint{1.078589in}{0.399814in}}%
\pgfpathlineto{\pgfqpoint{1.064960in}{0.386528in}}%
\pgfpathlineto{\pgfqpoint{1.064683in}{0.386203in}}%
\pgfpathlineto{\pgfqpoint{1.056152in}{0.372592in}}%
\pgfpathlineto{\pgfqpoint{1.050884in}{0.358981in}}%
\pgfpathlineto{\pgfqpoint{1.049304in}{0.346900in}}%
\pgfpathlineto{\pgfqpoint{1.049132in}{0.345370in}}%
\pgfpathlineto{\pgfqpoint{1.049304in}{0.345370in}}%
\pgfpathclose%
\pgfpathmoveto{\pgfqpoint{1.832132in}{0.345370in}}%
\pgfpathlineto{\pgfqpoint{1.847789in}{0.345370in}}%
\pgfpathlineto{\pgfqpoint{1.863445in}{0.345370in}}%
\pgfpathlineto{\pgfqpoint{1.879102in}{0.345370in}}%
\pgfpathlineto{\pgfqpoint{1.894758in}{0.345370in}}%
\pgfpathlineto{\pgfqpoint{1.910415in}{0.345370in}}%
\pgfpathlineto{\pgfqpoint{1.910415in}{0.358981in}}%
\pgfpathlineto{\pgfqpoint{1.910415in}{0.372592in}}%
\pgfpathlineto{\pgfqpoint{1.910415in}{0.386203in}}%
\pgfpathlineto{\pgfqpoint{1.910415in}{0.399814in}}%
\pgfpathlineto{\pgfqpoint{1.910415in}{0.413425in}}%
\pgfpathlineto{\pgfqpoint{1.910415in}{0.420133in}}%
\pgfpathlineto{\pgfqpoint{1.894758in}{0.418705in}}%
\pgfpathlineto{\pgfqpoint{1.879102in}{0.414483in}}%
\pgfpathlineto{\pgfqpoint{1.876704in}{0.413425in}}%
\pgfpathlineto{\pgfqpoint{1.863445in}{0.406548in}}%
\pgfpathlineto{\pgfqpoint{1.853846in}{0.399814in}}%
\pgfpathlineto{\pgfqpoint{1.847789in}{0.394549in}}%
\pgfpathlineto{\pgfqpoint{1.840043in}{0.386203in}}%
\pgfpathlineto{\pgfqpoint{1.832132in}{0.374676in}}%
\pgfpathlineto{\pgfqpoint{1.830916in}{0.372592in}}%
\pgfpathlineto{\pgfqpoint{1.826058in}{0.358981in}}%
\pgfpathlineto{\pgfqpoint{1.824416in}{0.345370in}}%
\pgfpathlineto{\pgfqpoint{1.832132in}{0.345370in}}%
\pgfpathclose%
\pgfpathmoveto{\pgfqpoint{0.362175in}{0.944259in}}%
\pgfpathlineto{\pgfqpoint{0.376072in}{0.945633in}}%
\pgfpathlineto{\pgfqpoint{0.391728in}{0.950212in}}%
\pgfpathlineto{\pgfqpoint{0.407385in}{0.957628in}}%
\pgfpathlineto{\pgfqpoint{0.407759in}{0.957870in}}%
\pgfpathlineto{\pgfqpoint{0.423041in}{0.969718in}}%
\pgfpathlineto{\pgfqpoint{0.424879in}{0.971481in}}%
\pgfpathlineto{\pgfqpoint{0.435892in}{0.985092in}}%
\pgfpathlineto{\pgfqpoint{0.438698in}{0.990140in}}%
\pgfpathlineto{\pgfqpoint{0.442735in}{0.998703in}}%
\pgfpathlineto{\pgfqpoint{0.446002in}{1.012314in}}%
\pgfpathlineto{\pgfqpoint{0.446002in}{1.025925in}}%
\pgfpathlineto{\pgfqpoint{0.442735in}{1.039536in}}%
\pgfpathlineto{\pgfqpoint{0.438698in}{1.048099in}}%
\pgfpathlineto{\pgfqpoint{0.435892in}{1.053148in}}%
\pgfpathlineto{\pgfqpoint{0.424879in}{1.066759in}}%
\pgfpathlineto{\pgfqpoint{0.423041in}{1.068521in}}%
\pgfpathlineto{\pgfqpoint{0.407759in}{1.080370in}}%
\pgfpathlineto{\pgfqpoint{0.407385in}{1.080611in}}%
\pgfpathlineto{\pgfqpoint{0.391728in}{1.088028in}}%
\pgfpathlineto{\pgfqpoint{0.376072in}{1.092607in}}%
\pgfpathlineto{\pgfqpoint{0.362175in}{1.093981in}}%
\pgfpathlineto{\pgfqpoint{0.360415in}{1.094131in}}%
\pgfpathlineto{\pgfqpoint{0.360415in}{1.093981in}}%
\pgfpathlineto{\pgfqpoint{0.360415in}{1.080370in}}%
\pgfpathlineto{\pgfqpoint{0.360415in}{1.066759in}}%
\pgfpathlineto{\pgfqpoint{0.360415in}{1.053148in}}%
\pgfpathlineto{\pgfqpoint{0.360415in}{1.039536in}}%
\pgfpathlineto{\pgfqpoint{0.360415in}{1.025925in}}%
\pgfpathlineto{\pgfqpoint{0.360415in}{1.012314in}}%
\pgfpathlineto{\pgfqpoint{0.360415in}{0.998703in}}%
\pgfpathlineto{\pgfqpoint{0.360415in}{0.985092in}}%
\pgfpathlineto{\pgfqpoint{0.360415in}{0.971481in}}%
\pgfpathlineto{\pgfqpoint{0.360415in}{0.957870in}}%
\pgfpathlineto{\pgfqpoint{0.360415in}{0.944259in}}%
\pgfpathlineto{\pgfqpoint{0.360415in}{0.944109in}}%
\pgfpathlineto{\pgfqpoint{0.362175in}{0.944259in}}%
\pgfpathclose%
\pgfpathmoveto{\pgfqpoint{1.910415in}{0.944109in}}%
\pgfpathlineto{\pgfqpoint{1.910415in}{0.944259in}}%
\pgfpathlineto{\pgfqpoint{1.910415in}{0.957870in}}%
\pgfpathlineto{\pgfqpoint{1.910415in}{0.971481in}}%
\pgfpathlineto{\pgfqpoint{1.910415in}{0.985092in}}%
\pgfpathlineto{\pgfqpoint{1.910415in}{0.998703in}}%
\pgfpathlineto{\pgfqpoint{1.910415in}{1.012314in}}%
\pgfpathlineto{\pgfqpoint{1.910415in}{1.025925in}}%
\pgfpathlineto{\pgfqpoint{1.910415in}{1.039536in}}%
\pgfpathlineto{\pgfqpoint{1.910415in}{1.053148in}}%
\pgfpathlineto{\pgfqpoint{1.910415in}{1.066759in}}%
\pgfpathlineto{\pgfqpoint{1.910415in}{1.080370in}}%
\pgfpathlineto{\pgfqpoint{1.910415in}{1.093981in}}%
\pgfpathlineto{\pgfqpoint{1.910415in}{1.094131in}}%
\pgfpathlineto{\pgfqpoint{1.908655in}{1.093981in}}%
\pgfpathlineto{\pgfqpoint{1.894758in}{1.092607in}}%
\pgfpathlineto{\pgfqpoint{1.879102in}{1.088028in}}%
\pgfpathlineto{\pgfqpoint{1.863445in}{1.080611in}}%
\pgfpathlineto{\pgfqpoint{1.863071in}{1.080370in}}%
\pgfpathlineto{\pgfqpoint{1.847789in}{1.068521in}}%
\pgfpathlineto{\pgfqpoint{1.845951in}{1.066759in}}%
\pgfpathlineto{\pgfqpoint{1.834938in}{1.053148in}}%
\pgfpathlineto{\pgfqpoint{1.832132in}{1.048099in}}%
\pgfpathlineto{\pgfqpoint{1.828095in}{1.039536in}}%
\pgfpathlineto{\pgfqpoint{1.824828in}{1.025925in}}%
\pgfpathlineto{\pgfqpoint{1.824828in}{1.012314in}}%
\pgfpathlineto{\pgfqpoint{1.828095in}{0.998703in}}%
\pgfpathlineto{\pgfqpoint{1.832132in}{0.990140in}}%
\pgfpathlineto{\pgfqpoint{1.834938in}{0.985092in}}%
\pgfpathlineto{\pgfqpoint{1.845951in}{0.971481in}}%
\pgfpathlineto{\pgfqpoint{1.847789in}{0.969718in}}%
\pgfpathlineto{\pgfqpoint{1.863071in}{0.957870in}}%
\pgfpathlineto{\pgfqpoint{1.863445in}{0.957628in}}%
\pgfpathlineto{\pgfqpoint{1.879102in}{0.950212in}}%
\pgfpathlineto{\pgfqpoint{1.894758in}{0.945633in}}%
\pgfpathlineto{\pgfqpoint{1.908655in}{0.944259in}}%
\pgfpathlineto{\pgfqpoint{1.910415in}{0.944109in}}%
\pgfpathclose%
\pgfpathmoveto{\pgfqpoint{1.096274in}{0.953582in}}%
\pgfpathlineto{\pgfqpoint{1.111930in}{0.947553in}}%
\pgfpathlineto{\pgfqpoint{1.127587in}{0.944472in}}%
\pgfpathlineto{\pgfqpoint{1.143243in}{0.944472in}}%
\pgfpathlineto{\pgfqpoint{1.158900in}{0.947553in}}%
\pgfpathlineto{\pgfqpoint{1.174556in}{0.953582in}}%
\pgfpathlineto{\pgfqpoint{1.182196in}{0.957870in}}%
\pgfpathlineto{\pgfqpoint{1.190213in}{0.963287in}}%
\pgfpathlineto{\pgfqpoint{1.199638in}{0.971481in}}%
\pgfpathlineto{\pgfqpoint{1.205870in}{0.978451in}}%
\pgfpathlineto{\pgfqpoint{1.210802in}{0.985092in}}%
\pgfpathlineto{\pgfqpoint{1.217737in}{0.998703in}}%
\pgfpathlineto{\pgfqpoint{1.221280in}{1.012314in}}%
\pgfpathlineto{\pgfqpoint{1.221280in}{1.025925in}}%
\pgfpathlineto{\pgfqpoint{1.217737in}{1.039536in}}%
\pgfpathlineto{\pgfqpoint{1.210802in}{1.053148in}}%
\pgfpathlineto{\pgfqpoint{1.205870in}{1.059789in}}%
\pgfpathlineto{\pgfqpoint{1.199638in}{1.066759in}}%
\pgfpathlineto{\pgfqpoint{1.190213in}{1.074952in}}%
\pgfpathlineto{\pgfqpoint{1.182196in}{1.080370in}}%
\pgfpathlineto{\pgfqpoint{1.174556in}{1.084658in}}%
\pgfpathlineto{\pgfqpoint{1.158900in}{1.090687in}}%
\pgfpathlineto{\pgfqpoint{1.143243in}{1.093767in}}%
\pgfpathlineto{\pgfqpoint{1.127587in}{1.093767in}}%
\pgfpathlineto{\pgfqpoint{1.111930in}{1.090687in}}%
\pgfpathlineto{\pgfqpoint{1.096274in}{1.084658in}}%
\pgfpathlineto{\pgfqpoint{1.088634in}{1.080370in}}%
\pgfpathlineto{\pgfqpoint{1.080617in}{1.074952in}}%
\pgfpathlineto{\pgfqpoint{1.071192in}{1.066759in}}%
\pgfpathlineto{\pgfqpoint{1.064960in}{1.059789in}}%
\pgfpathlineto{\pgfqpoint{1.060028in}{1.053148in}}%
\pgfpathlineto{\pgfqpoint{1.053093in}{1.039536in}}%
\pgfpathlineto{\pgfqpoint{1.049550in}{1.025925in}}%
\pgfpathlineto{\pgfqpoint{1.049550in}{1.012314in}}%
\pgfpathlineto{\pgfqpoint{1.053093in}{0.998703in}}%
\pgfpathlineto{\pgfqpoint{1.060028in}{0.985092in}}%
\pgfpathlineto{\pgfqpoint{1.064960in}{0.978451in}}%
\pgfpathlineto{\pgfqpoint{1.071192in}{0.971481in}}%
\pgfpathlineto{\pgfqpoint{1.080617in}{0.963287in}}%
\pgfpathlineto{\pgfqpoint{1.088634in}{0.957870in}}%
\pgfpathlineto{\pgfqpoint{1.096274in}{0.953582in}}%
\pgfpathclose%
\pgfpathmoveto{\pgfqpoint{0.376072in}{1.619534in}}%
\pgfpathlineto{\pgfqpoint{0.391728in}{1.623757in}}%
\pgfpathlineto{\pgfqpoint{0.394126in}{1.624814in}}%
\pgfpathlineto{\pgfqpoint{0.407385in}{1.631692in}}%
\pgfpathlineto{\pgfqpoint{0.416984in}{1.638425in}}%
\pgfpathlineto{\pgfqpoint{0.423041in}{1.643691in}}%
\pgfpathlineto{\pgfqpoint{0.430787in}{1.652036in}}%
\pgfpathlineto{\pgfqpoint{0.438698in}{1.663563in}}%
\pgfpathlineto{\pgfqpoint{0.439914in}{1.665648in}}%
\pgfpathlineto{\pgfqpoint{0.444772in}{1.679259in}}%
\pgfpathlineto{\pgfqpoint{0.446414in}{1.692870in}}%
\pgfpathlineto{\pgfqpoint{0.438698in}{1.692870in}}%
\pgfpathlineto{\pgfqpoint{0.423041in}{1.692870in}}%
\pgfpathlineto{\pgfqpoint{0.407385in}{1.692870in}}%
\pgfpathlineto{\pgfqpoint{0.391728in}{1.692870in}}%
\pgfpathlineto{\pgfqpoint{0.376072in}{1.692870in}}%
\pgfpathlineto{\pgfqpoint{0.360415in}{1.692870in}}%
\pgfpathlineto{\pgfqpoint{0.360415in}{1.679259in}}%
\pgfpathlineto{\pgfqpoint{0.360415in}{1.665648in}}%
\pgfpathlineto{\pgfqpoint{0.360415in}{1.652036in}}%
\pgfpathlineto{\pgfqpoint{0.360415in}{1.638425in}}%
\pgfpathlineto{\pgfqpoint{0.360415in}{1.624814in}}%
\pgfpathlineto{\pgfqpoint{0.360415in}{1.618106in}}%
\pgfpathlineto{\pgfqpoint{0.376072in}{1.619534in}}%
\pgfpathclose%
\pgfpathmoveto{\pgfqpoint{1.111930in}{1.621305in}}%
\pgfpathlineto{\pgfqpoint{1.127587in}{1.618464in}}%
\pgfpathlineto{\pgfqpoint{1.143243in}{1.618464in}}%
\pgfpathlineto{\pgfqpoint{1.158900in}{1.621305in}}%
\pgfpathlineto{\pgfqpoint{1.168749in}{1.624814in}}%
\pgfpathlineto{\pgfqpoint{1.174556in}{1.627253in}}%
\pgfpathlineto{\pgfqpoint{1.190213in}{1.636828in}}%
\pgfpathlineto{\pgfqpoint{1.192241in}{1.638425in}}%
\pgfpathlineto{\pgfqpoint{1.205870in}{1.651711in}}%
\pgfpathlineto{\pgfqpoint{1.206147in}{1.652036in}}%
\pgfpathlineto{\pgfqpoint{1.214678in}{1.665648in}}%
\pgfpathlineto{\pgfqpoint{1.219946in}{1.679259in}}%
\pgfpathlineto{\pgfqpoint{1.221526in}{1.691340in}}%
\pgfpathlineto{\pgfqpoint{1.221698in}{1.692870in}}%
\pgfpathlineto{\pgfqpoint{1.221526in}{1.692870in}}%
\pgfpathlineto{\pgfqpoint{1.205870in}{1.692870in}}%
\pgfpathlineto{\pgfqpoint{1.190213in}{1.692870in}}%
\pgfpathlineto{\pgfqpoint{1.174556in}{1.692870in}}%
\pgfpathlineto{\pgfqpoint{1.158900in}{1.692870in}}%
\pgfpathlineto{\pgfqpoint{1.143243in}{1.692870in}}%
\pgfpathlineto{\pgfqpoint{1.127587in}{1.692870in}}%
\pgfpathlineto{\pgfqpoint{1.111930in}{1.692870in}}%
\pgfpathlineto{\pgfqpoint{1.096274in}{1.692870in}}%
\pgfpathlineto{\pgfqpoint{1.080617in}{1.692870in}}%
\pgfpathlineto{\pgfqpoint{1.064960in}{1.692870in}}%
\pgfpathlineto{\pgfqpoint{1.049304in}{1.692870in}}%
\pgfpathlineto{\pgfqpoint{1.049132in}{1.692870in}}%
\pgfpathlineto{\pgfqpoint{1.049304in}{1.691340in}}%
\pgfpathlineto{\pgfqpoint{1.050884in}{1.679259in}}%
\pgfpathlineto{\pgfqpoint{1.056152in}{1.665648in}}%
\pgfpathlineto{\pgfqpoint{1.064683in}{1.652036in}}%
\pgfpathlineto{\pgfqpoint{1.064960in}{1.651711in}}%
\pgfpathlineto{\pgfqpoint{1.078589in}{1.638425in}}%
\pgfpathlineto{\pgfqpoint{1.080617in}{1.636828in}}%
\pgfpathlineto{\pgfqpoint{1.096274in}{1.627253in}}%
\pgfpathlineto{\pgfqpoint{1.102081in}{1.624814in}}%
\pgfpathlineto{\pgfqpoint{1.111930in}{1.621305in}}%
\pgfpathclose%
\pgfpathmoveto{\pgfqpoint{1.879102in}{1.623757in}}%
\pgfpathlineto{\pgfqpoint{1.894758in}{1.619534in}}%
\pgfpathlineto{\pgfqpoint{1.910415in}{1.618106in}}%
\pgfpathlineto{\pgfqpoint{1.910415in}{1.624814in}}%
\pgfpathlineto{\pgfqpoint{1.910415in}{1.638425in}}%
\pgfpathlineto{\pgfqpoint{1.910415in}{1.652036in}}%
\pgfpathlineto{\pgfqpoint{1.910415in}{1.665648in}}%
\pgfpathlineto{\pgfqpoint{1.910415in}{1.679259in}}%
\pgfpathlineto{\pgfqpoint{1.910415in}{1.692870in}}%
\pgfpathlineto{\pgfqpoint{1.894758in}{1.692870in}}%
\pgfpathlineto{\pgfqpoint{1.879102in}{1.692870in}}%
\pgfpathlineto{\pgfqpoint{1.863445in}{1.692870in}}%
\pgfpathlineto{\pgfqpoint{1.847789in}{1.692870in}}%
\pgfpathlineto{\pgfqpoint{1.832132in}{1.692870in}}%
\pgfpathlineto{\pgfqpoint{1.824416in}{1.692870in}}%
\pgfpathlineto{\pgfqpoint{1.826058in}{1.679259in}}%
\pgfpathlineto{\pgfqpoint{1.830916in}{1.665648in}}%
\pgfpathlineto{\pgfqpoint{1.832132in}{1.663563in}}%
\pgfpathlineto{\pgfqpoint{1.840043in}{1.652036in}}%
\pgfpathlineto{\pgfqpoint{1.847789in}{1.643691in}}%
\pgfpathlineto{\pgfqpoint{1.853846in}{1.638425in}}%
\pgfpathlineto{\pgfqpoint{1.863445in}{1.631692in}}%
\pgfpathlineto{\pgfqpoint{1.876704in}{1.624814in}}%
\pgfpathlineto{\pgfqpoint{1.879102in}{1.623757in}}%
\pgfpathclose%
\pgfusepath{fill}%
\end{pgfscope}%
\begin{pgfscope}%
\pgfsetbuttcap%
\pgfsetroundjoin%
\definecolor{currentfill}{rgb}{0.000000,0.000000,0.000000}%
\pgfsetfillcolor{currentfill}%
\pgfsetlinewidth{0.803000pt}%
\definecolor{currentstroke}{rgb}{0.000000,0.000000,0.000000}%
\pgfsetstrokecolor{currentstroke}%
\pgfsetdash{}{0pt}%
\pgfsys@defobject{currentmarker}{\pgfqpoint{0.000000in}{-0.048611in}}{\pgfqpoint{0.000000in}{0.000000in}}{%
\pgfpathmoveto{\pgfqpoint{0.000000in}{0.000000in}}%
\pgfpathlineto{\pgfqpoint{0.000000in}{-0.048611in}}%
\pgfusepath{stroke,fill}%
}%
\begin{pgfscope}%
\pgfsys@transformshift{0.360415in}{0.345370in}%
\pgfsys@useobject{currentmarker}{}%
\end{pgfscope}%
\end{pgfscope}%
\begin{pgfscope}%
\definecolor{textcolor}{rgb}{0.000000,0.000000,0.000000}%
\pgfsetstrokecolor{textcolor}%
\pgfsetfillcolor{textcolor}%
\pgftext[x=0.360415in,y=0.248148in,,top]{\color{textcolor}{\rmfamily\fontsize{12.000000}{14.400000}\selectfont\catcode`\^=\active\def^{\ifmmode\sp\else\^{}\fi}\catcode`\%=\active\def%{\%}$\mathdefault{0}$}}%
\end{pgfscope}%
\begin{pgfscope}%
\pgfsetbuttcap%
\pgfsetroundjoin%
\definecolor{currentfill}{rgb}{0.000000,0.000000,0.000000}%
\pgfsetfillcolor{currentfill}%
\pgfsetlinewidth{0.803000pt}%
\definecolor{currentstroke}{rgb}{0.000000,0.000000,0.000000}%
\pgfsetstrokecolor{currentstroke}%
\pgfsetdash{}{0pt}%
\pgfsys@defobject{currentmarker}{\pgfqpoint{0.000000in}{-0.048611in}}{\pgfqpoint{0.000000in}{0.000000in}}{%
\pgfpathmoveto{\pgfqpoint{0.000000in}{0.000000in}}%
\pgfpathlineto{\pgfqpoint{0.000000in}{-0.048611in}}%
\pgfusepath{stroke,fill}%
}%
\begin{pgfscope}%
\pgfsys@transformshift{1.006248in}{0.345370in}%
\pgfsys@useobject{currentmarker}{}%
\end{pgfscope}%
\end{pgfscope}%
\begin{pgfscope}%
\definecolor{textcolor}{rgb}{0.000000,0.000000,0.000000}%
\pgfsetstrokecolor{textcolor}%
\pgfsetfillcolor{textcolor}%
\pgftext[x=1.006248in,y=0.248148in,,top]{\color{textcolor}{\rmfamily\fontsize{12.000000}{14.400000}\selectfont\catcode`\^=\active\def^{\ifmmode\sp\else\^{}\fi}\catcode`\%=\active\def%{\%}$\mathdefault{5}$}}%
\end{pgfscope}%
\begin{pgfscope}%
\pgfsetbuttcap%
\pgfsetroundjoin%
\definecolor{currentfill}{rgb}{0.000000,0.000000,0.000000}%
\pgfsetfillcolor{currentfill}%
\pgfsetlinewidth{0.803000pt}%
\definecolor{currentstroke}{rgb}{0.000000,0.000000,0.000000}%
\pgfsetstrokecolor{currentstroke}%
\pgfsetdash{}{0pt}%
\pgfsys@defobject{currentmarker}{\pgfqpoint{0.000000in}{-0.048611in}}{\pgfqpoint{0.000000in}{0.000000in}}{%
\pgfpathmoveto{\pgfqpoint{0.000000in}{0.000000in}}%
\pgfpathlineto{\pgfqpoint{0.000000in}{-0.048611in}}%
\pgfusepath{stroke,fill}%
}%
\begin{pgfscope}%
\pgfsys@transformshift{1.652082in}{0.345370in}%
\pgfsys@useobject{currentmarker}{}%
\end{pgfscope}%
\end{pgfscope}%
\begin{pgfscope}%
\definecolor{textcolor}{rgb}{0.000000,0.000000,0.000000}%
\pgfsetstrokecolor{textcolor}%
\pgfsetfillcolor{textcolor}%
\pgftext[x=1.652082in,y=0.248148in,,top]{\color{textcolor}{\rmfamily\fontsize{12.000000}{14.400000}\selectfont\catcode`\^=\active\def^{\ifmmode\sp\else\^{}\fi}\catcode`\%=\active\def%{\%}$\mathdefault{10}$}}%
\end{pgfscope}%
\begin{pgfscope}%
\pgfsetbuttcap%
\pgfsetroundjoin%
\definecolor{currentfill}{rgb}{0.000000,0.000000,0.000000}%
\pgfsetfillcolor{currentfill}%
\pgfsetlinewidth{0.803000pt}%
\definecolor{currentstroke}{rgb}{0.000000,0.000000,0.000000}%
\pgfsetstrokecolor{currentstroke}%
\pgfsetdash{}{0pt}%
\pgfsys@defobject{currentmarker}{\pgfqpoint{-0.048611in}{0.000000in}}{\pgfqpoint{-0.000000in}{0.000000in}}{%
\pgfpathmoveto{\pgfqpoint{-0.000000in}{0.000000in}}%
\pgfpathlineto{\pgfqpoint{-0.048611in}{0.000000in}}%
\pgfusepath{stroke,fill}%
}%
\begin{pgfscope}%
\pgfsys@transformshift{0.360415in}{0.345370in}%
\pgfsys@useobject{currentmarker}{}%
\end{pgfscope}%
\end{pgfscope}%
\begin{pgfscope}%
\definecolor{textcolor}{rgb}{0.000000,0.000000,0.000000}%
\pgfsetstrokecolor{textcolor}%
\pgfsetfillcolor{textcolor}%
\pgftext[x=0.181596in, y=0.287500in, left, base]{\color{textcolor}{\rmfamily\fontsize{12.000000}{14.400000}\selectfont\catcode`\^=\active\def^{\ifmmode\sp\else\^{}\fi}\catcode`\%=\active\def%{\%}$\mathdefault{0}$}}%
\end{pgfscope}%
\begin{pgfscope}%
\pgfsetbuttcap%
\pgfsetroundjoin%
\definecolor{currentfill}{rgb}{0.000000,0.000000,0.000000}%
\pgfsetfillcolor{currentfill}%
\pgfsetlinewidth{0.803000pt}%
\definecolor{currentstroke}{rgb}{0.000000,0.000000,0.000000}%
\pgfsetstrokecolor{currentstroke}%
\pgfsetdash{}{0pt}%
\pgfsys@defobject{currentmarker}{\pgfqpoint{-0.048611in}{0.000000in}}{\pgfqpoint{-0.000000in}{0.000000in}}{%
\pgfpathmoveto{\pgfqpoint{-0.000000in}{0.000000in}}%
\pgfpathlineto{\pgfqpoint{-0.048611in}{0.000000in}}%
\pgfusepath{stroke,fill}%
}%
\begin{pgfscope}%
\pgfsys@transformshift{0.360415in}{0.906828in}%
\pgfsys@useobject{currentmarker}{}%
\end{pgfscope}%
\end{pgfscope}%
\begin{pgfscope}%
\definecolor{textcolor}{rgb}{0.000000,0.000000,0.000000}%
\pgfsetstrokecolor{textcolor}%
\pgfsetfillcolor{textcolor}%
\pgftext[x=0.181596in, y=0.848958in, left, base]{\color{textcolor}{\rmfamily\fontsize{12.000000}{14.400000}\selectfont\catcode`\^=\active\def^{\ifmmode\sp\else\^{}\fi}\catcode`\%=\active\def%{\%}$\mathdefault{5}$}}%
\end{pgfscope}%
\begin{pgfscope}%
\pgfsetbuttcap%
\pgfsetroundjoin%
\definecolor{currentfill}{rgb}{0.000000,0.000000,0.000000}%
\pgfsetfillcolor{currentfill}%
\pgfsetlinewidth{0.803000pt}%
\definecolor{currentstroke}{rgb}{0.000000,0.000000,0.000000}%
\pgfsetstrokecolor{currentstroke}%
\pgfsetdash{}{0pt}%
\pgfsys@defobject{currentmarker}{\pgfqpoint{-0.048611in}{0.000000in}}{\pgfqpoint{-0.000000in}{0.000000in}}{%
\pgfpathmoveto{\pgfqpoint{-0.000000in}{0.000000in}}%
\pgfpathlineto{\pgfqpoint{-0.048611in}{0.000000in}}%
\pgfusepath{stroke,fill}%
}%
\begin{pgfscope}%
\pgfsys@transformshift{0.360415in}{1.468286in}%
\pgfsys@useobject{currentmarker}{}%
\end{pgfscope}%
\end{pgfscope}%
\begin{pgfscope}%
\definecolor{textcolor}{rgb}{0.000000,0.000000,0.000000}%
\pgfsetstrokecolor{textcolor}%
\pgfsetfillcolor{textcolor}%
\pgftext[x=0.100000in, y=1.410416in, left, base]{\color{textcolor}{\rmfamily\fontsize{12.000000}{14.400000}\selectfont\catcode`\^=\active\def^{\ifmmode\sp\else\^{}\fi}\catcode`\%=\active\def%{\%}$\mathdefault{10}$}}%
\end{pgfscope}%
\begin{pgfscope}%
\pgfsetrectcap%
\pgfsetmiterjoin%
\pgfsetlinewidth{0.803000pt}%
\definecolor{currentstroke}{rgb}{0.000000,0.000000,0.000000}%
\pgfsetstrokecolor{currentstroke}%
\pgfsetdash{}{0pt}%
\pgfpathmoveto{\pgfqpoint{0.360415in}{0.345370in}}%
\pgfpathlineto{\pgfqpoint{0.360415in}{1.692870in}}%
\pgfusepath{stroke}%
\end{pgfscope}%
\begin{pgfscope}%
\pgfsetrectcap%
\pgfsetmiterjoin%
\pgfsetlinewidth{0.803000pt}%
\definecolor{currentstroke}{rgb}{0.000000,0.000000,0.000000}%
\pgfsetstrokecolor{currentstroke}%
\pgfsetdash{}{0pt}%
\pgfpathmoveto{\pgfqpoint{1.910415in}{0.345370in}}%
\pgfpathlineto{\pgfqpoint{1.910415in}{1.692870in}}%
\pgfusepath{stroke}%
\end{pgfscope}%
\begin{pgfscope}%
\pgfsetrectcap%
\pgfsetmiterjoin%
\pgfsetlinewidth{0.803000pt}%
\definecolor{currentstroke}{rgb}{0.000000,0.000000,0.000000}%
\pgfsetstrokecolor{currentstroke}%
\pgfsetdash{}{0pt}%
\pgfpathmoveto{\pgfqpoint{0.360415in}{0.345370in}}%
\pgfpathlineto{\pgfqpoint{1.910415in}{0.345370in}}%
\pgfusepath{stroke}%
\end{pgfscope}%
\begin{pgfscope}%
\pgfsetrectcap%
\pgfsetmiterjoin%
\pgfsetlinewidth{0.803000pt}%
\definecolor{currentstroke}{rgb}{0.000000,0.000000,0.000000}%
\pgfsetstrokecolor{currentstroke}%
\pgfsetdash{}{0pt}%
\pgfpathmoveto{\pgfqpoint{0.360415in}{1.692870in}}%
\pgfpathlineto{\pgfqpoint{1.910415in}{1.692870in}}%
\pgfusepath{stroke}%
\end{pgfscope}%
\end{pgfpicture}%
\makeatother%
\endgroup%

        \caption{$n_c=2$}
        \label{fig:gaussian-well-2}
    \end{subfigure}
    \begin{subfigure}[b]{0.32\columnwidth}
        %% Creator: Matplotlib, PGF backend
%%
%% To include the figure in your LaTeX document, write
%%   \input{<filename>.pgf}
%%
%% Make sure the required packages are loaded in your preamble
%%   \usepackage{pgf}
%%
%% Also ensure that all the required font packages are loaded; for instance,
%% the lmodern package is sometimes necessary when using math font.
%%   \usepackage{lmodern}
%%
%% Figures using additional raster images can only be included by \input if
%% they are in the same directory as the main LaTeX file. For loading figures
%% from other directories you can use the `import` package
%%   \usepackage{import}
%%
%% and then include the figures with
%%   \import{<path to file>}{<filename>.pgf}
%%
%% Matplotlib used the following preamble
%%   \def\mathdefault#1{#1}
%%   \everymath=\expandafter{\the\everymath\displaystyle}
%%   \IfFileExists{scrextend.sty}{
%%     \usepackage[fontsize=12.000000pt]{scrextend}
%%   }{
%%     \renewcommand{\normalsize}{\fontsize{12.000000}{14.400000}\selectfont}
%%     \normalsize
%%   }
%%   
%%   \ifdefined\pdftexversion\else  % non-pdftex case.
%%     \usepackage{fontspec}
%%     \setmainfont{DejaVuSerif.ttf}[Path=\detokenize{/opt/hostedtoolcache/Python/3.12.9/x64/lib/python3.12/site-packages/matplotlib/mpl-data/fonts/ttf/}]
%%     \setsansfont{DejaVuSans.ttf}[Path=\detokenize{/opt/hostedtoolcache/Python/3.12.9/x64/lib/python3.12/site-packages/matplotlib/mpl-data/fonts/ttf/}]
%%     \setmonofont{DejaVuSansMono.ttf}[Path=\detokenize{/opt/hostedtoolcache/Python/3.12.9/x64/lib/python3.12/site-packages/matplotlib/mpl-data/fonts/ttf/}]
%%   \fi
%%   \makeatletter\@ifpackageloaded{underscore}{}{\usepackage[strings]{underscore}}\makeatother
%%
\begingroup%
\makeatletter%
\begin{pgfpicture}%
\pgfpathrectangle{\pgfpointorigin}{\pgfqpoint{2.092011in}{1.850740in}}%
\pgfusepath{use as bounding box, clip}%
\begin{pgfscope}%
\pgfsetbuttcap%
\pgfsetmiterjoin%
\definecolor{currentfill}{rgb}{1.000000,1.000000,1.000000}%
\pgfsetfillcolor{currentfill}%
\pgfsetlinewidth{0.000000pt}%
\definecolor{currentstroke}{rgb}{1.000000,1.000000,1.000000}%
\pgfsetstrokecolor{currentstroke}%
\pgfsetdash{}{0pt}%
\pgfpathmoveto{\pgfqpoint{0.000000in}{0.000000in}}%
\pgfpathlineto{\pgfqpoint{2.092011in}{0.000000in}}%
\pgfpathlineto{\pgfqpoint{2.092011in}{1.850740in}}%
\pgfpathlineto{\pgfqpoint{0.000000in}{1.850740in}}%
\pgfpathlineto{\pgfqpoint{0.000000in}{0.000000in}}%
\pgfpathclose%
\pgfusepath{fill}%
\end{pgfscope}%
\begin{pgfscope}%
\pgfsetbuttcap%
\pgfsetmiterjoin%
\definecolor{currentfill}{rgb}{1.000000,1.000000,1.000000}%
\pgfsetfillcolor{currentfill}%
\pgfsetlinewidth{0.000000pt}%
\definecolor{currentstroke}{rgb}{0.000000,0.000000,0.000000}%
\pgfsetstrokecolor{currentstroke}%
\pgfsetstrokeopacity{0.000000}%
\pgfsetdash{}{0pt}%
\pgfpathmoveto{\pgfqpoint{0.360415in}{0.345370in}}%
\pgfpathlineto{\pgfqpoint{1.910415in}{0.345370in}}%
\pgfpathlineto{\pgfqpoint{1.910415in}{1.692870in}}%
\pgfpathlineto{\pgfqpoint{0.360415in}{1.692870in}}%
\pgfpathlineto{\pgfqpoint{0.360415in}{0.345370in}}%
\pgfpathclose%
\pgfusepath{fill}%
\end{pgfscope}%
\begin{pgfscope}%
\pgfpathrectangle{\pgfqpoint{0.360415in}{0.345370in}}{\pgfqpoint{1.550000in}{1.347500in}}%
\pgfusepath{clip}%
\pgfsetbuttcap%
\pgfsetroundjoin%
\definecolor{currentfill}{rgb}{0.972530,0.881250,0.144923}%
\pgfsetfillcolor{currentfill}%
\pgfsetlinewidth{0.000000pt}%
\definecolor{currentstroke}{rgb}{0.000000,0.000000,0.000000}%
\pgfsetstrokecolor{currentstroke}%
\pgfsetdash{}{0pt}%
\pgfpathmoveto{\pgfqpoint{0.485668in}{0.453085in}}%
\pgfpathlineto{\pgfqpoint{0.501324in}{0.444842in}}%
\pgfpathlineto{\pgfqpoint{0.516981in}{0.442491in}}%
\pgfpathlineto{\pgfqpoint{0.532637in}{0.446018in}}%
\pgfpathlineto{\pgfqpoint{0.546376in}{0.454259in}}%
\pgfpathlineto{\pgfqpoint{0.548294in}{0.456158in}}%
\pgfpathlineto{\pgfqpoint{0.555933in}{0.467870in}}%
\pgfpathlineto{\pgfqpoint{0.558474in}{0.481481in}}%
\pgfpathlineto{\pgfqpoint{0.554661in}{0.495092in}}%
\pgfpathlineto{\pgfqpoint{0.548294in}{0.503691in}}%
\pgfpathlineto{\pgfqpoint{0.542529in}{0.508703in}}%
\pgfpathlineto{\pgfqpoint{0.532637in}{0.514239in}}%
\pgfpathlineto{\pgfqpoint{0.516981in}{0.517553in}}%
\pgfpathlineto{\pgfqpoint{0.501324in}{0.515344in}}%
\pgfpathlineto{\pgfqpoint{0.487853in}{0.508703in}}%
\pgfpathlineto{\pgfqpoint{0.485668in}{0.507035in}}%
\pgfpathlineto{\pgfqpoint{0.476188in}{0.495092in}}%
\pgfpathlineto{\pgfqpoint{0.472131in}{0.481481in}}%
\pgfpathlineto{\pgfqpoint{0.474835in}{0.467870in}}%
\pgfpathlineto{\pgfqpoint{0.484318in}{0.454259in}}%
\pgfpathlineto{\pgfqpoint{0.485668in}{0.453085in}}%
\pgfpathclose%
\pgfpathmoveto{\pgfqpoint{0.798799in}{0.450964in}}%
\pgfpathlineto{\pgfqpoint{0.814455in}{0.443901in}}%
\pgfpathlineto{\pgfqpoint{0.830112in}{0.442726in}}%
\pgfpathlineto{\pgfqpoint{0.845769in}{0.447430in}}%
\pgfpathlineto{\pgfqpoint{0.855963in}{0.454259in}}%
\pgfpathlineto{\pgfqpoint{0.861425in}{0.460320in}}%
\pgfpathlineto{\pgfqpoint{0.866051in}{0.467870in}}%
\pgfpathlineto{\pgfqpoint{0.868457in}{0.481481in}}%
\pgfpathlineto{\pgfqpoint{0.864847in}{0.495092in}}%
\pgfpathlineto{\pgfqpoint{0.861425in}{0.500012in}}%
\pgfpathlineto{\pgfqpoint{0.852503in}{0.508703in}}%
\pgfpathlineto{\pgfqpoint{0.845769in}{0.512912in}}%
\pgfpathlineto{\pgfqpoint{0.830112in}{0.517332in}}%
\pgfpathlineto{\pgfqpoint{0.814455in}{0.516228in}}%
\pgfpathlineto{\pgfqpoint{0.798799in}{0.509592in}}%
\pgfpathlineto{\pgfqpoint{0.797615in}{0.508703in}}%
\pgfpathlineto{\pgfqpoint{0.785979in}{0.495092in}}%
\pgfpathlineto{\pgfqpoint{0.783142in}{0.486259in}}%
\pgfpathlineto{\pgfqpoint{0.781993in}{0.481481in}}%
\pgfpathlineto{\pgfqpoint{0.783142in}{0.474338in}}%
\pgfpathlineto{\pgfqpoint{0.784527in}{0.467870in}}%
\pgfpathlineto{\pgfqpoint{0.794704in}{0.454259in}}%
\pgfpathlineto{\pgfqpoint{0.798799in}{0.450964in}}%
\pgfpathclose%
\pgfpathmoveto{\pgfqpoint{1.111930in}{0.449079in}}%
\pgfpathlineto{\pgfqpoint{1.127587in}{0.443196in}}%
\pgfpathlineto{\pgfqpoint{1.143243in}{0.443196in}}%
\pgfpathlineto{\pgfqpoint{1.158900in}{0.449079in}}%
\pgfpathlineto{\pgfqpoint{1.165917in}{0.454259in}}%
\pgfpathlineto{\pgfqpoint{1.174556in}{0.464859in}}%
\pgfpathlineto{\pgfqpoint{1.176300in}{0.467870in}}%
\pgfpathlineto{\pgfqpoint{1.178592in}{0.481481in}}%
\pgfpathlineto{\pgfqpoint{1.175152in}{0.495092in}}%
\pgfpathlineto{\pgfqpoint{1.174556in}{0.495999in}}%
\pgfpathlineto{\pgfqpoint{1.162762in}{0.508703in}}%
\pgfpathlineto{\pgfqpoint{1.158900in}{0.511363in}}%
\pgfpathlineto{\pgfqpoint{1.143243in}{0.516890in}}%
\pgfpathlineto{\pgfqpoint{1.127587in}{0.516890in}}%
\pgfpathlineto{\pgfqpoint{1.111930in}{0.511363in}}%
\pgfpathlineto{\pgfqpoint{1.108068in}{0.508703in}}%
\pgfpathlineto{\pgfqpoint{1.096274in}{0.495999in}}%
\pgfpathlineto{\pgfqpoint{1.095678in}{0.495092in}}%
\pgfpathlineto{\pgfqpoint{1.092238in}{0.481481in}}%
\pgfpathlineto{\pgfqpoint{1.094530in}{0.467870in}}%
\pgfpathlineto{\pgfqpoint{1.096274in}{0.464859in}}%
\pgfpathlineto{\pgfqpoint{1.104913in}{0.454259in}}%
\pgfpathlineto{\pgfqpoint{1.111930in}{0.449079in}}%
\pgfpathclose%
\pgfpathmoveto{\pgfqpoint{1.425061in}{0.447430in}}%
\pgfpathlineto{\pgfqpoint{1.440718in}{0.442726in}}%
\pgfpathlineto{\pgfqpoint{1.456375in}{0.443901in}}%
\pgfpathlineto{\pgfqpoint{1.472031in}{0.450964in}}%
\pgfpathlineto{\pgfqpoint{1.476126in}{0.454259in}}%
\pgfpathlineto{\pgfqpoint{1.486303in}{0.467870in}}%
\pgfpathlineto{\pgfqpoint{1.487688in}{0.474338in}}%
\pgfpathlineto{\pgfqpoint{1.488837in}{0.481481in}}%
\pgfpathlineto{\pgfqpoint{1.487688in}{0.486259in}}%
\pgfpathlineto{\pgfqpoint{1.484851in}{0.495092in}}%
\pgfpathlineto{\pgfqpoint{1.473215in}{0.508703in}}%
\pgfpathlineto{\pgfqpoint{1.472031in}{0.509592in}}%
\pgfpathlineto{\pgfqpoint{1.456375in}{0.516228in}}%
\pgfpathlineto{\pgfqpoint{1.440718in}{0.517332in}}%
\pgfpathlineto{\pgfqpoint{1.425061in}{0.512912in}}%
\pgfpathlineto{\pgfqpoint{1.418327in}{0.508703in}}%
\pgfpathlineto{\pgfqpoint{1.409405in}{0.500012in}}%
\pgfpathlineto{\pgfqpoint{1.405983in}{0.495092in}}%
\pgfpathlineto{\pgfqpoint{1.402373in}{0.481481in}}%
\pgfpathlineto{\pgfqpoint{1.404779in}{0.467870in}}%
\pgfpathlineto{\pgfqpoint{1.409405in}{0.460320in}}%
\pgfpathlineto{\pgfqpoint{1.414867in}{0.454259in}}%
\pgfpathlineto{\pgfqpoint{1.425061in}{0.447430in}}%
\pgfpathclose%
\pgfpathmoveto{\pgfqpoint{1.738193in}{0.446018in}}%
\pgfpathlineto{\pgfqpoint{1.753849in}{0.442491in}}%
\pgfpathlineto{\pgfqpoint{1.769506in}{0.444842in}}%
\pgfpathlineto{\pgfqpoint{1.785162in}{0.453085in}}%
\pgfpathlineto{\pgfqpoint{1.786512in}{0.454259in}}%
\pgfpathlineto{\pgfqpoint{1.795995in}{0.467870in}}%
\pgfpathlineto{\pgfqpoint{1.798699in}{0.481481in}}%
\pgfpathlineto{\pgfqpoint{1.794642in}{0.495092in}}%
\pgfpathlineto{\pgfqpoint{1.785162in}{0.507035in}}%
\pgfpathlineto{\pgfqpoint{1.782977in}{0.508703in}}%
\pgfpathlineto{\pgfqpoint{1.769506in}{0.515344in}}%
\pgfpathlineto{\pgfqpoint{1.753849in}{0.517553in}}%
\pgfpathlineto{\pgfqpoint{1.738193in}{0.514239in}}%
\pgfpathlineto{\pgfqpoint{1.728301in}{0.508703in}}%
\pgfpathlineto{\pgfqpoint{1.722536in}{0.503691in}}%
\pgfpathlineto{\pgfqpoint{1.716169in}{0.495092in}}%
\pgfpathlineto{\pgfqpoint{1.712356in}{0.481481in}}%
\pgfpathlineto{\pgfqpoint{1.714897in}{0.467870in}}%
\pgfpathlineto{\pgfqpoint{1.722536in}{0.456158in}}%
\pgfpathlineto{\pgfqpoint{1.724454in}{0.454259in}}%
\pgfpathlineto{\pgfqpoint{1.738193in}{0.446018in}}%
\pgfpathclose%
\pgfpathmoveto{\pgfqpoint{0.516981in}{0.711870in}}%
\pgfpathlineto{\pgfqpoint{0.522477in}{0.712870in}}%
\pgfpathlineto{\pgfqpoint{0.532637in}{0.715336in}}%
\pgfpathlineto{\pgfqpoint{0.548294in}{0.725452in}}%
\pgfpathlineto{\pgfqpoint{0.549316in}{0.726481in}}%
\pgfpathlineto{\pgfqpoint{0.556949in}{0.740092in}}%
\pgfpathlineto{\pgfqpoint{0.558220in}{0.753703in}}%
\pgfpathlineto{\pgfqpoint{0.553135in}{0.767314in}}%
\pgfpathlineto{\pgfqpoint{0.548294in}{0.773169in}}%
\pgfpathlineto{\pgfqpoint{0.538297in}{0.780925in}}%
\pgfpathlineto{\pgfqpoint{0.532637in}{0.783900in}}%
\pgfpathlineto{\pgfqpoint{0.516981in}{0.787038in}}%
\pgfpathlineto{\pgfqpoint{0.501324in}{0.784947in}}%
\pgfpathlineto{\pgfqpoint{0.492639in}{0.780925in}}%
\pgfpathlineto{\pgfqpoint{0.485668in}{0.776177in}}%
\pgfpathlineto{\pgfqpoint{0.477813in}{0.767314in}}%
\pgfpathlineto{\pgfqpoint{0.472401in}{0.753703in}}%
\pgfpathlineto{\pgfqpoint{0.473753in}{0.740092in}}%
\pgfpathlineto{\pgfqpoint{0.481877in}{0.726481in}}%
\pgfpathlineto{\pgfqpoint{0.485668in}{0.722921in}}%
\pgfpathlineto{\pgfqpoint{0.501324in}{0.714073in}}%
\pgfpathlineto{\pgfqpoint{0.508764in}{0.712870in}}%
\pgfpathlineto{\pgfqpoint{0.516981in}{0.711870in}}%
\pgfpathclose%
\pgfpathmoveto{\pgfqpoint{0.830112in}{0.712061in}}%
\pgfpathlineto{\pgfqpoint{0.833460in}{0.712870in}}%
\pgfpathlineto{\pgfqpoint{0.845769in}{0.716852in}}%
\pgfpathlineto{\pgfqpoint{0.859076in}{0.726481in}}%
\pgfpathlineto{\pgfqpoint{0.861425in}{0.729515in}}%
\pgfpathlineto{\pgfqpoint{0.867014in}{0.740092in}}%
\pgfpathlineto{\pgfqpoint{0.868216in}{0.753703in}}%
\pgfpathlineto{\pgfqpoint{0.863402in}{0.767314in}}%
\pgfpathlineto{\pgfqpoint{0.861425in}{0.769860in}}%
\pgfpathlineto{\pgfqpoint{0.848697in}{0.780925in}}%
\pgfpathlineto{\pgfqpoint{0.845769in}{0.782644in}}%
\pgfpathlineto{\pgfqpoint{0.830112in}{0.786829in}}%
\pgfpathlineto{\pgfqpoint{0.814455in}{0.785784in}}%
\pgfpathlineto{\pgfqpoint{0.802289in}{0.780925in}}%
\pgfpathlineto{\pgfqpoint{0.798799in}{0.778883in}}%
\pgfpathlineto{\pgfqpoint{0.787723in}{0.767314in}}%
\pgfpathlineto{\pgfqpoint{0.783142in}{0.756614in}}%
\pgfpathlineto{\pgfqpoint{0.782212in}{0.753703in}}%
\pgfpathlineto{\pgfqpoint{0.783142in}{0.742179in}}%
\pgfpathlineto{\pgfqpoint{0.783366in}{0.740092in}}%
\pgfpathlineto{\pgfqpoint{0.792085in}{0.726481in}}%
\pgfpathlineto{\pgfqpoint{0.798799in}{0.720644in}}%
\pgfpathlineto{\pgfqpoint{0.814455in}{0.713064in}}%
\pgfpathlineto{\pgfqpoint{0.816857in}{0.712870in}}%
\pgfpathlineto{\pgfqpoint{0.830112in}{0.712061in}}%
\pgfpathclose%
\pgfpathmoveto{\pgfqpoint{1.127587in}{0.712443in}}%
\pgfpathlineto{\pgfqpoint{1.143243in}{0.712443in}}%
\pgfpathlineto{\pgfqpoint{1.144662in}{0.712870in}}%
\pgfpathlineto{\pgfqpoint{1.158900in}{0.718621in}}%
\pgfpathlineto{\pgfqpoint{1.168756in}{0.726481in}}%
\pgfpathlineto{\pgfqpoint{1.174556in}{0.734764in}}%
\pgfpathlineto{\pgfqpoint{1.177217in}{0.740092in}}%
\pgfpathlineto{\pgfqpoint{1.178363in}{0.753703in}}%
\pgfpathlineto{\pgfqpoint{1.174556in}{0.765024in}}%
\pgfpathlineto{\pgfqpoint{1.173484in}{0.767314in}}%
\pgfpathlineto{\pgfqpoint{1.159290in}{0.780925in}}%
\pgfpathlineto{\pgfqpoint{1.158900in}{0.781178in}}%
\pgfpathlineto{\pgfqpoint{1.143243in}{0.786411in}}%
\pgfpathlineto{\pgfqpoint{1.127587in}{0.786411in}}%
\pgfpathlineto{\pgfqpoint{1.111930in}{0.781178in}}%
\pgfpathlineto{\pgfqpoint{1.111540in}{0.780925in}}%
\pgfpathlineto{\pgfqpoint{1.097346in}{0.767314in}}%
\pgfpathlineto{\pgfqpoint{1.096274in}{0.765024in}}%
\pgfpathlineto{\pgfqpoint{1.092467in}{0.753703in}}%
\pgfpathlineto{\pgfqpoint{1.093613in}{0.740092in}}%
\pgfpathlineto{\pgfqpoint{1.096274in}{0.734764in}}%
\pgfpathlineto{\pgfqpoint{1.102074in}{0.726481in}}%
\pgfpathlineto{\pgfqpoint{1.111930in}{0.718621in}}%
\pgfpathlineto{\pgfqpoint{1.126168in}{0.712870in}}%
\pgfpathlineto{\pgfqpoint{1.127587in}{0.712443in}}%
\pgfpathclose%
\pgfpathmoveto{\pgfqpoint{1.440718in}{0.712061in}}%
\pgfpathlineto{\pgfqpoint{1.453973in}{0.712870in}}%
\pgfpathlineto{\pgfqpoint{1.456375in}{0.713064in}}%
\pgfpathlineto{\pgfqpoint{1.472031in}{0.720644in}}%
\pgfpathlineto{\pgfqpoint{1.478745in}{0.726481in}}%
\pgfpathlineto{\pgfqpoint{1.487464in}{0.740092in}}%
\pgfpathlineto{\pgfqpoint{1.487688in}{0.742179in}}%
\pgfpathlineto{\pgfqpoint{1.488618in}{0.753703in}}%
\pgfpathlineto{\pgfqpoint{1.487688in}{0.756614in}}%
\pgfpathlineto{\pgfqpoint{1.483107in}{0.767314in}}%
\pgfpathlineto{\pgfqpoint{1.472031in}{0.778883in}}%
\pgfpathlineto{\pgfqpoint{1.468541in}{0.780925in}}%
\pgfpathlineto{\pgfqpoint{1.456375in}{0.785784in}}%
\pgfpathlineto{\pgfqpoint{1.440718in}{0.786829in}}%
\pgfpathlineto{\pgfqpoint{1.425061in}{0.782644in}}%
\pgfpathlineto{\pgfqpoint{1.422133in}{0.780925in}}%
\pgfpathlineto{\pgfqpoint{1.409405in}{0.769860in}}%
\pgfpathlineto{\pgfqpoint{1.407428in}{0.767314in}}%
\pgfpathlineto{\pgfqpoint{1.402614in}{0.753703in}}%
\pgfpathlineto{\pgfqpoint{1.403816in}{0.740092in}}%
\pgfpathlineto{\pgfqpoint{1.409405in}{0.729515in}}%
\pgfpathlineto{\pgfqpoint{1.411754in}{0.726481in}}%
\pgfpathlineto{\pgfqpoint{1.425061in}{0.716852in}}%
\pgfpathlineto{\pgfqpoint{1.437370in}{0.712870in}}%
\pgfpathlineto{\pgfqpoint{1.440718in}{0.712061in}}%
\pgfpathclose%
\pgfpathmoveto{\pgfqpoint{1.753849in}{0.711870in}}%
\pgfpathlineto{\pgfqpoint{1.762066in}{0.712870in}}%
\pgfpathlineto{\pgfqpoint{1.769506in}{0.714073in}}%
\pgfpathlineto{\pgfqpoint{1.785162in}{0.722921in}}%
\pgfpathlineto{\pgfqpoint{1.788953in}{0.726481in}}%
\pgfpathlineto{\pgfqpoint{1.797077in}{0.740092in}}%
\pgfpathlineto{\pgfqpoint{1.798429in}{0.753703in}}%
\pgfpathlineto{\pgfqpoint{1.793017in}{0.767314in}}%
\pgfpathlineto{\pgfqpoint{1.785162in}{0.776177in}}%
\pgfpathlineto{\pgfqpoint{1.778191in}{0.780925in}}%
\pgfpathlineto{\pgfqpoint{1.769506in}{0.784947in}}%
\pgfpathlineto{\pgfqpoint{1.753849in}{0.787038in}}%
\pgfpathlineto{\pgfqpoint{1.738193in}{0.783900in}}%
\pgfpathlineto{\pgfqpoint{1.732533in}{0.780925in}}%
\pgfpathlineto{\pgfqpoint{1.722536in}{0.773169in}}%
\pgfpathlineto{\pgfqpoint{1.717695in}{0.767314in}}%
\pgfpathlineto{\pgfqpoint{1.712610in}{0.753703in}}%
\pgfpathlineto{\pgfqpoint{1.713881in}{0.740092in}}%
\pgfpathlineto{\pgfqpoint{1.721514in}{0.726481in}}%
\pgfpathlineto{\pgfqpoint{1.722536in}{0.725452in}}%
\pgfpathlineto{\pgfqpoint{1.738193in}{0.715336in}}%
\pgfpathlineto{\pgfqpoint{1.748353in}{0.712870in}}%
\pgfpathlineto{\pgfqpoint{1.753849in}{0.711870in}}%
\pgfpathclose%
\pgfpathmoveto{\pgfqpoint{0.501324in}{0.983576in}}%
\pgfpathlineto{\pgfqpoint{0.516981in}{0.981583in}}%
\pgfpathlineto{\pgfqpoint{0.532637in}{0.984574in}}%
\pgfpathlineto{\pgfqpoint{0.533680in}{0.985092in}}%
\pgfpathlineto{\pgfqpoint{0.548294in}{0.995346in}}%
\pgfpathlineto{\pgfqpoint{0.551353in}{0.998703in}}%
\pgfpathlineto{\pgfqpoint{0.557712in}{1.012314in}}%
\pgfpathlineto{\pgfqpoint{0.557712in}{1.025925in}}%
\pgfpathlineto{\pgfqpoint{0.551353in}{1.039536in}}%
\pgfpathlineto{\pgfqpoint{0.548294in}{1.042894in}}%
\pgfpathlineto{\pgfqpoint{0.533680in}{1.053148in}}%
\pgfpathlineto{\pgfqpoint{0.532637in}{1.053666in}}%
\pgfpathlineto{\pgfqpoint{0.516981in}{1.056656in}}%
\pgfpathlineto{\pgfqpoint{0.501324in}{1.054663in}}%
\pgfpathlineto{\pgfqpoint{0.497861in}{1.053148in}}%
\pgfpathlineto{\pgfqpoint{0.485668in}{1.045637in}}%
\pgfpathlineto{\pgfqpoint{0.479709in}{1.039536in}}%
\pgfpathlineto{\pgfqpoint{0.472942in}{1.025925in}}%
\pgfpathlineto{\pgfqpoint{0.472942in}{1.012314in}}%
\pgfpathlineto{\pgfqpoint{0.479709in}{0.998703in}}%
\pgfpathlineto{\pgfqpoint{0.485668in}{0.992603in}}%
\pgfpathlineto{\pgfqpoint{0.497861in}{0.985092in}}%
\pgfpathlineto{\pgfqpoint{0.501324in}{0.983576in}}%
\pgfpathclose%
\pgfpathmoveto{\pgfqpoint{0.814455in}{0.982779in}}%
\pgfpathlineto{\pgfqpoint{0.830112in}{0.981783in}}%
\pgfpathlineto{\pgfqpoint{0.843135in}{0.985092in}}%
\pgfpathlineto{\pgfqpoint{0.845769in}{0.986024in}}%
\pgfpathlineto{\pgfqpoint{0.861425in}{0.998364in}}%
\pgfpathlineto{\pgfqpoint{0.861715in}{0.998703in}}%
\pgfpathlineto{\pgfqpoint{0.867735in}{1.012314in}}%
\pgfpathlineto{\pgfqpoint{0.867735in}{1.025925in}}%
\pgfpathlineto{\pgfqpoint{0.861715in}{1.039536in}}%
\pgfpathlineto{\pgfqpoint{0.861425in}{1.039876in}}%
\pgfpathlineto{\pgfqpoint{0.845769in}{1.052216in}}%
\pgfpathlineto{\pgfqpoint{0.843135in}{1.053148in}}%
\pgfpathlineto{\pgfqpoint{0.830112in}{1.056457in}}%
\pgfpathlineto{\pgfqpoint{0.814455in}{1.055460in}}%
\pgfpathlineto{\pgfqpoint{0.808326in}{1.053148in}}%
\pgfpathlineto{\pgfqpoint{0.798799in}{1.048105in}}%
\pgfpathlineto{\pgfqpoint{0.789758in}{1.039536in}}%
\pgfpathlineto{\pgfqpoint{0.783142in}{1.027159in}}%
\pgfpathlineto{\pgfqpoint{0.782652in}{1.025925in}}%
\pgfpathlineto{\pgfqpoint{0.782652in}{1.012314in}}%
\pgfpathlineto{\pgfqpoint{0.783142in}{1.011080in}}%
\pgfpathlineto{\pgfqpoint{0.789758in}{0.998703in}}%
\pgfpathlineto{\pgfqpoint{0.798799in}{0.990135in}}%
\pgfpathlineto{\pgfqpoint{0.808326in}{0.985092in}}%
\pgfpathlineto{\pgfqpoint{0.814455in}{0.982779in}}%
\pgfpathclose%
\pgfpathmoveto{\pgfqpoint{1.127587in}{0.982181in}}%
\pgfpathlineto{\pgfqpoint{1.143243in}{0.982181in}}%
\pgfpathlineto{\pgfqpoint{1.152450in}{0.985092in}}%
\pgfpathlineto{\pgfqpoint{1.158900in}{0.987942in}}%
\pgfpathlineto{\pgfqpoint{1.171278in}{0.998703in}}%
\pgfpathlineto{\pgfqpoint{1.174556in}{1.004310in}}%
\pgfpathlineto{\pgfqpoint{1.177905in}{1.012314in}}%
\pgfpathlineto{\pgfqpoint{1.177905in}{1.025925in}}%
\pgfpathlineto{\pgfqpoint{1.174556in}{1.033929in}}%
\pgfpathlineto{\pgfqpoint{1.171278in}{1.039536in}}%
\pgfpathlineto{\pgfqpoint{1.158900in}{1.050298in}}%
\pgfpathlineto{\pgfqpoint{1.152450in}{1.053148in}}%
\pgfpathlineto{\pgfqpoint{1.143243in}{1.056058in}}%
\pgfpathlineto{\pgfqpoint{1.127587in}{1.056058in}}%
\pgfpathlineto{\pgfqpoint{1.118380in}{1.053148in}}%
\pgfpathlineto{\pgfqpoint{1.111930in}{1.050298in}}%
\pgfpathlineto{\pgfqpoint{1.099552in}{1.039536in}}%
\pgfpathlineto{\pgfqpoint{1.096274in}{1.033929in}}%
\pgfpathlineto{\pgfqpoint{1.092925in}{1.025925in}}%
\pgfpathlineto{\pgfqpoint{1.092925in}{1.012314in}}%
\pgfpathlineto{\pgfqpoint{1.096274in}{1.004310in}}%
\pgfpathlineto{\pgfqpoint{1.099552in}{0.998703in}}%
\pgfpathlineto{\pgfqpoint{1.111930in}{0.987942in}}%
\pgfpathlineto{\pgfqpoint{1.118380in}{0.985092in}}%
\pgfpathlineto{\pgfqpoint{1.127587in}{0.982181in}}%
\pgfpathclose%
\pgfpathmoveto{\pgfqpoint{1.440718in}{0.981783in}}%
\pgfpathlineto{\pgfqpoint{1.456375in}{0.982779in}}%
\pgfpathlineto{\pgfqpoint{1.462504in}{0.985092in}}%
\pgfpathlineto{\pgfqpoint{1.472031in}{0.990135in}}%
\pgfpathlineto{\pgfqpoint{1.481072in}{0.998703in}}%
\pgfpathlineto{\pgfqpoint{1.487688in}{1.011080in}}%
\pgfpathlineto{\pgfqpoint{1.488178in}{1.012314in}}%
\pgfpathlineto{\pgfqpoint{1.488178in}{1.025925in}}%
\pgfpathlineto{\pgfqpoint{1.487688in}{1.027159in}}%
\pgfpathlineto{\pgfqpoint{1.481072in}{1.039536in}}%
\pgfpathlineto{\pgfqpoint{1.472031in}{1.048105in}}%
\pgfpathlineto{\pgfqpoint{1.462504in}{1.053148in}}%
\pgfpathlineto{\pgfqpoint{1.456375in}{1.055460in}}%
\pgfpathlineto{\pgfqpoint{1.440718in}{1.056457in}}%
\pgfpathlineto{\pgfqpoint{1.427695in}{1.053148in}}%
\pgfpathlineto{\pgfqpoint{1.425061in}{1.052216in}}%
\pgfpathlineto{\pgfqpoint{1.409405in}{1.039876in}}%
\pgfpathlineto{\pgfqpoint{1.409115in}{1.039536in}}%
\pgfpathlineto{\pgfqpoint{1.403095in}{1.025925in}}%
\pgfpathlineto{\pgfqpoint{1.403095in}{1.012314in}}%
\pgfpathlineto{\pgfqpoint{1.409115in}{0.998703in}}%
\pgfpathlineto{\pgfqpoint{1.409405in}{0.998364in}}%
\pgfpathlineto{\pgfqpoint{1.425061in}{0.986024in}}%
\pgfpathlineto{\pgfqpoint{1.427695in}{0.985092in}}%
\pgfpathlineto{\pgfqpoint{1.440718in}{0.981783in}}%
\pgfpathclose%
\pgfpathmoveto{\pgfqpoint{1.738193in}{0.984574in}}%
\pgfpathlineto{\pgfqpoint{1.753849in}{0.981583in}}%
\pgfpathlineto{\pgfqpoint{1.769506in}{0.983576in}}%
\pgfpathlineto{\pgfqpoint{1.772969in}{0.985092in}}%
\pgfpathlineto{\pgfqpoint{1.785162in}{0.992603in}}%
\pgfpathlineto{\pgfqpoint{1.791121in}{0.998703in}}%
\pgfpathlineto{\pgfqpoint{1.797888in}{1.012314in}}%
\pgfpathlineto{\pgfqpoint{1.797888in}{1.025925in}}%
\pgfpathlineto{\pgfqpoint{1.791121in}{1.039536in}}%
\pgfpathlineto{\pgfqpoint{1.785162in}{1.045637in}}%
\pgfpathlineto{\pgfqpoint{1.772969in}{1.053148in}}%
\pgfpathlineto{\pgfqpoint{1.769506in}{1.054663in}}%
\pgfpathlineto{\pgfqpoint{1.753849in}{1.056656in}}%
\pgfpathlineto{\pgfqpoint{1.738193in}{1.053666in}}%
\pgfpathlineto{\pgfqpoint{1.737150in}{1.053148in}}%
\pgfpathlineto{\pgfqpoint{1.722536in}{1.042894in}}%
\pgfpathlineto{\pgfqpoint{1.719477in}{1.039536in}}%
\pgfpathlineto{\pgfqpoint{1.713118in}{1.025925in}}%
\pgfpathlineto{\pgfqpoint{1.713118in}{1.012314in}}%
\pgfpathlineto{\pgfqpoint{1.719477in}{0.998703in}}%
\pgfpathlineto{\pgfqpoint{1.722536in}{0.995346in}}%
\pgfpathlineto{\pgfqpoint{1.737150in}{0.985092in}}%
\pgfpathlineto{\pgfqpoint{1.738193in}{0.984574in}}%
\pgfpathclose%
\pgfpathmoveto{\pgfqpoint{0.501324in}{1.253293in}}%
\pgfpathlineto{\pgfqpoint{0.516981in}{1.251201in}}%
\pgfpathlineto{\pgfqpoint{0.532637in}{1.254339in}}%
\pgfpathlineto{\pgfqpoint{0.538297in}{1.257314in}}%
\pgfpathlineto{\pgfqpoint{0.548294in}{1.265070in}}%
\pgfpathlineto{\pgfqpoint{0.553135in}{1.270925in}}%
\pgfpathlineto{\pgfqpoint{0.558220in}{1.284536in}}%
\pgfpathlineto{\pgfqpoint{0.556949in}{1.298147in}}%
\pgfpathlineto{\pgfqpoint{0.549316in}{1.311759in}}%
\pgfpathlineto{\pgfqpoint{0.548294in}{1.312787in}}%
\pgfpathlineto{\pgfqpoint{0.532637in}{1.322903in}}%
\pgfpathlineto{\pgfqpoint{0.522477in}{1.325370in}}%
\pgfpathlineto{\pgfqpoint{0.516981in}{1.326369in}}%
\pgfpathlineto{\pgfqpoint{0.508764in}{1.325370in}}%
\pgfpathlineto{\pgfqpoint{0.501324in}{1.324166in}}%
\pgfpathlineto{\pgfqpoint{0.485668in}{1.315318in}}%
\pgfpathlineto{\pgfqpoint{0.481877in}{1.311759in}}%
\pgfpathlineto{\pgfqpoint{0.473753in}{1.298147in}}%
\pgfpathlineto{\pgfqpoint{0.472401in}{1.284536in}}%
\pgfpathlineto{\pgfqpoint{0.477813in}{1.270925in}}%
\pgfpathlineto{\pgfqpoint{0.485668in}{1.262063in}}%
\pgfpathlineto{\pgfqpoint{0.492639in}{1.257314in}}%
\pgfpathlineto{\pgfqpoint{0.501324in}{1.253293in}}%
\pgfpathclose%
\pgfpathmoveto{\pgfqpoint{0.814455in}{1.252456in}}%
\pgfpathlineto{\pgfqpoint{0.830112in}{1.251410in}}%
\pgfpathlineto{\pgfqpoint{0.845769in}{1.255595in}}%
\pgfpathlineto{\pgfqpoint{0.848697in}{1.257314in}}%
\pgfpathlineto{\pgfqpoint{0.861425in}{1.268380in}}%
\pgfpathlineto{\pgfqpoint{0.863402in}{1.270925in}}%
\pgfpathlineto{\pgfqpoint{0.868216in}{1.284536in}}%
\pgfpathlineto{\pgfqpoint{0.867014in}{1.298147in}}%
\pgfpathlineto{\pgfqpoint{0.861425in}{1.308725in}}%
\pgfpathlineto{\pgfqpoint{0.859076in}{1.311759in}}%
\pgfpathlineto{\pgfqpoint{0.845769in}{1.321388in}}%
\pgfpathlineto{\pgfqpoint{0.833460in}{1.325370in}}%
\pgfpathlineto{\pgfqpoint{0.830112in}{1.326178in}}%
\pgfpathlineto{\pgfqpoint{0.816857in}{1.325370in}}%
\pgfpathlineto{\pgfqpoint{0.814455in}{1.325176in}}%
\pgfpathlineto{\pgfqpoint{0.798799in}{1.317595in}}%
\pgfpathlineto{\pgfqpoint{0.792085in}{1.311759in}}%
\pgfpathlineto{\pgfqpoint{0.783366in}{1.298147in}}%
\pgfpathlineto{\pgfqpoint{0.783142in}{1.296060in}}%
\pgfpathlineto{\pgfqpoint{0.782212in}{1.284536in}}%
\pgfpathlineto{\pgfqpoint{0.783142in}{1.281626in}}%
\pgfpathlineto{\pgfqpoint{0.787723in}{1.270925in}}%
\pgfpathlineto{\pgfqpoint{0.798799in}{1.259357in}}%
\pgfpathlineto{\pgfqpoint{0.802289in}{1.257314in}}%
\pgfpathlineto{\pgfqpoint{0.814455in}{1.252456in}}%
\pgfpathclose%
\pgfpathmoveto{\pgfqpoint{1.111930in}{1.257062in}}%
\pgfpathlineto{\pgfqpoint{1.127587in}{1.251828in}}%
\pgfpathlineto{\pgfqpoint{1.143243in}{1.251828in}}%
\pgfpathlineto{\pgfqpoint{1.158900in}{1.257062in}}%
\pgfpathlineto{\pgfqpoint{1.159290in}{1.257314in}}%
\pgfpathlineto{\pgfqpoint{1.173484in}{1.270925in}}%
\pgfpathlineto{\pgfqpoint{1.174556in}{1.273215in}}%
\pgfpathlineto{\pgfqpoint{1.178363in}{1.284536in}}%
\pgfpathlineto{\pgfqpoint{1.177217in}{1.298148in}}%
\pgfpathlineto{\pgfqpoint{1.174556in}{1.303476in}}%
\pgfpathlineto{\pgfqpoint{1.168756in}{1.311759in}}%
\pgfpathlineto{\pgfqpoint{1.158900in}{1.319618in}}%
\pgfpathlineto{\pgfqpoint{1.144662in}{1.325370in}}%
\pgfpathlineto{\pgfqpoint{1.143243in}{1.325796in}}%
\pgfpathlineto{\pgfqpoint{1.127587in}{1.325796in}}%
\pgfpathlineto{\pgfqpoint{1.126168in}{1.325370in}}%
\pgfpathlineto{\pgfqpoint{1.111930in}{1.319618in}}%
\pgfpathlineto{\pgfqpoint{1.102074in}{1.311759in}}%
\pgfpathlineto{\pgfqpoint{1.096274in}{1.303476in}}%
\pgfpathlineto{\pgfqpoint{1.093613in}{1.298147in}}%
\pgfpathlineto{\pgfqpoint{1.092467in}{1.284536in}}%
\pgfpathlineto{\pgfqpoint{1.096274in}{1.273215in}}%
\pgfpathlineto{\pgfqpoint{1.097346in}{1.270925in}}%
\pgfpathlineto{\pgfqpoint{1.111540in}{1.257314in}}%
\pgfpathlineto{\pgfqpoint{1.111930in}{1.257062in}}%
\pgfpathclose%
\pgfpathmoveto{\pgfqpoint{1.425061in}{1.255595in}}%
\pgfpathlineto{\pgfqpoint{1.440718in}{1.251410in}}%
\pgfpathlineto{\pgfqpoint{1.456375in}{1.252456in}}%
\pgfpathlineto{\pgfqpoint{1.468541in}{1.257314in}}%
\pgfpathlineto{\pgfqpoint{1.472031in}{1.259357in}}%
\pgfpathlineto{\pgfqpoint{1.483107in}{1.270925in}}%
\pgfpathlineto{\pgfqpoint{1.487688in}{1.281626in}}%
\pgfpathlineto{\pgfqpoint{1.488618in}{1.284536in}}%
\pgfpathlineto{\pgfqpoint{1.487688in}{1.296060in}}%
\pgfpathlineto{\pgfqpoint{1.487464in}{1.298147in}}%
\pgfpathlineto{\pgfqpoint{1.478745in}{1.311759in}}%
\pgfpathlineto{\pgfqpoint{1.472031in}{1.317595in}}%
\pgfpathlineto{\pgfqpoint{1.456375in}{1.325176in}}%
\pgfpathlineto{\pgfqpoint{1.453973in}{1.325370in}}%
\pgfpathlineto{\pgfqpoint{1.440718in}{1.326178in}}%
\pgfpathlineto{\pgfqpoint{1.437370in}{1.325370in}}%
\pgfpathlineto{\pgfqpoint{1.425061in}{1.321388in}}%
\pgfpathlineto{\pgfqpoint{1.411754in}{1.311759in}}%
\pgfpathlineto{\pgfqpoint{1.409405in}{1.308725in}}%
\pgfpathlineto{\pgfqpoint{1.403816in}{1.298147in}}%
\pgfpathlineto{\pgfqpoint{1.402614in}{1.284536in}}%
\pgfpathlineto{\pgfqpoint{1.407428in}{1.270925in}}%
\pgfpathlineto{\pgfqpoint{1.409405in}{1.268380in}}%
\pgfpathlineto{\pgfqpoint{1.422133in}{1.257314in}}%
\pgfpathlineto{\pgfqpoint{1.425061in}{1.255595in}}%
\pgfpathclose%
\pgfpathmoveto{\pgfqpoint{1.738193in}{1.254339in}}%
\pgfpathlineto{\pgfqpoint{1.753849in}{1.251201in}}%
\pgfpathlineto{\pgfqpoint{1.769506in}{1.253293in}}%
\pgfpathlineto{\pgfqpoint{1.778191in}{1.257314in}}%
\pgfpathlineto{\pgfqpoint{1.785162in}{1.262063in}}%
\pgfpathlineto{\pgfqpoint{1.793017in}{1.270925in}}%
\pgfpathlineto{\pgfqpoint{1.798429in}{1.284536in}}%
\pgfpathlineto{\pgfqpoint{1.797077in}{1.298147in}}%
\pgfpathlineto{\pgfqpoint{1.788953in}{1.311759in}}%
\pgfpathlineto{\pgfqpoint{1.785162in}{1.315318in}}%
\pgfpathlineto{\pgfqpoint{1.769506in}{1.324166in}}%
\pgfpathlineto{\pgfqpoint{1.762066in}{1.325370in}}%
\pgfpathlineto{\pgfqpoint{1.753849in}{1.326369in}}%
\pgfpathlineto{\pgfqpoint{1.748353in}{1.325370in}}%
\pgfpathlineto{\pgfqpoint{1.738193in}{1.322903in}}%
\pgfpathlineto{\pgfqpoint{1.722536in}{1.312787in}}%
\pgfpathlineto{\pgfqpoint{1.721514in}{1.311759in}}%
\pgfpathlineto{\pgfqpoint{1.713881in}{1.298147in}}%
\pgfpathlineto{\pgfqpoint{1.712610in}{1.284536in}}%
\pgfpathlineto{\pgfqpoint{1.717695in}{1.270925in}}%
\pgfpathlineto{\pgfqpoint{1.722536in}{1.265070in}}%
\pgfpathlineto{\pgfqpoint{1.732533in}{1.257314in}}%
\pgfpathlineto{\pgfqpoint{1.738193in}{1.254339in}}%
\pgfpathclose%
\pgfpathmoveto{\pgfqpoint{0.501324in}{1.522896in}}%
\pgfpathlineto{\pgfqpoint{0.516981in}{1.520687in}}%
\pgfpathlineto{\pgfqpoint{0.532637in}{1.524001in}}%
\pgfpathlineto{\pgfqpoint{0.542529in}{1.529536in}}%
\pgfpathlineto{\pgfqpoint{0.548294in}{1.534548in}}%
\pgfpathlineto{\pgfqpoint{0.554661in}{1.543148in}}%
\pgfpathlineto{\pgfqpoint{0.558474in}{1.556759in}}%
\pgfpathlineto{\pgfqpoint{0.555933in}{1.570370in}}%
\pgfpathlineto{\pgfqpoint{0.548294in}{1.582081in}}%
\pgfpathlineto{\pgfqpoint{0.546376in}{1.583981in}}%
\pgfpathlineto{\pgfqpoint{0.532637in}{1.592221in}}%
\pgfpathlineto{\pgfqpoint{0.516981in}{1.595749in}}%
\pgfpathlineto{\pgfqpoint{0.501324in}{1.593398in}}%
\pgfpathlineto{\pgfqpoint{0.485668in}{1.585154in}}%
\pgfpathlineto{\pgfqpoint{0.484318in}{1.583981in}}%
\pgfpathlineto{\pgfqpoint{0.474835in}{1.570370in}}%
\pgfpathlineto{\pgfqpoint{0.472131in}{1.556759in}}%
\pgfpathlineto{\pgfqpoint{0.476188in}{1.543148in}}%
\pgfpathlineto{\pgfqpoint{0.485668in}{1.531204in}}%
\pgfpathlineto{\pgfqpoint{0.487853in}{1.529536in}}%
\pgfpathlineto{\pgfqpoint{0.501324in}{1.522896in}}%
\pgfpathclose%
\pgfpathmoveto{\pgfqpoint{0.798799in}{1.528648in}}%
\pgfpathlineto{\pgfqpoint{0.814455in}{1.522012in}}%
\pgfpathlineto{\pgfqpoint{0.830112in}{1.520907in}}%
\pgfpathlineto{\pgfqpoint{0.845769in}{1.525328in}}%
\pgfpathlineto{\pgfqpoint{0.852503in}{1.529536in}}%
\pgfpathlineto{\pgfqpoint{0.861425in}{1.538227in}}%
\pgfpathlineto{\pgfqpoint{0.864847in}{1.543148in}}%
\pgfpathlineto{\pgfqpoint{0.868457in}{1.556759in}}%
\pgfpathlineto{\pgfqpoint{0.866051in}{1.570370in}}%
\pgfpathlineto{\pgfqpoint{0.861425in}{1.577920in}}%
\pgfpathlineto{\pgfqpoint{0.855963in}{1.583981in}}%
\pgfpathlineto{\pgfqpoint{0.845769in}{1.590809in}}%
\pgfpathlineto{\pgfqpoint{0.830112in}{1.595514in}}%
\pgfpathlineto{\pgfqpoint{0.814455in}{1.594338in}}%
\pgfpathlineto{\pgfqpoint{0.798799in}{1.587276in}}%
\pgfpathlineto{\pgfqpoint{0.794704in}{1.583981in}}%
\pgfpathlineto{\pgfqpoint{0.784527in}{1.570370in}}%
\pgfpathlineto{\pgfqpoint{0.783142in}{1.563902in}}%
\pgfpathlineto{\pgfqpoint{0.781993in}{1.556759in}}%
\pgfpathlineto{\pgfqpoint{0.783142in}{1.551980in}}%
\pgfpathlineto{\pgfqpoint{0.785979in}{1.543148in}}%
\pgfpathlineto{\pgfqpoint{0.797615in}{1.529536in}}%
\pgfpathlineto{\pgfqpoint{0.798799in}{1.528648in}}%
\pgfpathclose%
\pgfpathmoveto{\pgfqpoint{1.111930in}{1.526877in}}%
\pgfpathlineto{\pgfqpoint{1.127587in}{1.521349in}}%
\pgfpathlineto{\pgfqpoint{1.143243in}{1.521349in}}%
\pgfpathlineto{\pgfqpoint{1.158900in}{1.526877in}}%
\pgfpathlineto{\pgfqpoint{1.162762in}{1.529536in}}%
\pgfpathlineto{\pgfqpoint{1.174556in}{1.542241in}}%
\pgfpathlineto{\pgfqpoint{1.175152in}{1.543148in}}%
\pgfpathlineto{\pgfqpoint{1.178592in}{1.556759in}}%
\pgfpathlineto{\pgfqpoint{1.176300in}{1.570370in}}%
\pgfpathlineto{\pgfqpoint{1.174556in}{1.573380in}}%
\pgfpathlineto{\pgfqpoint{1.165917in}{1.583981in}}%
\pgfpathlineto{\pgfqpoint{1.158900in}{1.589161in}}%
\pgfpathlineto{\pgfqpoint{1.143243in}{1.595044in}}%
\pgfpathlineto{\pgfqpoint{1.127587in}{1.595044in}}%
\pgfpathlineto{\pgfqpoint{1.111930in}{1.589161in}}%
\pgfpathlineto{\pgfqpoint{1.104913in}{1.583981in}}%
\pgfpathlineto{\pgfqpoint{1.096274in}{1.573380in}}%
\pgfpathlineto{\pgfqpoint{1.094530in}{1.570370in}}%
\pgfpathlineto{\pgfqpoint{1.092238in}{1.556759in}}%
\pgfpathlineto{\pgfqpoint{1.095678in}{1.543148in}}%
\pgfpathlineto{\pgfqpoint{1.096274in}{1.542241in}}%
\pgfpathlineto{\pgfqpoint{1.108068in}{1.529536in}}%
\pgfpathlineto{\pgfqpoint{1.111930in}{1.526877in}}%
\pgfpathclose%
\pgfpathmoveto{\pgfqpoint{1.425061in}{1.525328in}}%
\pgfpathlineto{\pgfqpoint{1.440718in}{1.520907in}}%
\pgfpathlineto{\pgfqpoint{1.456375in}{1.522012in}}%
\pgfpathlineto{\pgfqpoint{1.472031in}{1.528648in}}%
\pgfpathlineto{\pgfqpoint{1.473215in}{1.529536in}}%
\pgfpathlineto{\pgfqpoint{1.484851in}{1.543148in}}%
\pgfpathlineto{\pgfqpoint{1.487688in}{1.551980in}}%
\pgfpathlineto{\pgfqpoint{1.488837in}{1.556759in}}%
\pgfpathlineto{\pgfqpoint{1.487688in}{1.563902in}}%
\pgfpathlineto{\pgfqpoint{1.486303in}{1.570370in}}%
\pgfpathlineto{\pgfqpoint{1.476126in}{1.583981in}}%
\pgfpathlineto{\pgfqpoint{1.472031in}{1.587276in}}%
\pgfpathlineto{\pgfqpoint{1.456375in}{1.594338in}}%
\pgfpathlineto{\pgfqpoint{1.440718in}{1.595514in}}%
\pgfpathlineto{\pgfqpoint{1.425061in}{1.590809in}}%
\pgfpathlineto{\pgfqpoint{1.414867in}{1.583981in}}%
\pgfpathlineto{\pgfqpoint{1.409405in}{1.577920in}}%
\pgfpathlineto{\pgfqpoint{1.404779in}{1.570370in}}%
\pgfpathlineto{\pgfqpoint{1.402373in}{1.556759in}}%
\pgfpathlineto{\pgfqpoint{1.405983in}{1.543148in}}%
\pgfpathlineto{\pgfqpoint{1.409405in}{1.538227in}}%
\pgfpathlineto{\pgfqpoint{1.418327in}{1.529536in}}%
\pgfpathlineto{\pgfqpoint{1.425061in}{1.525328in}}%
\pgfpathclose%
\pgfpathmoveto{\pgfqpoint{1.738193in}{1.524001in}}%
\pgfpathlineto{\pgfqpoint{1.753849in}{1.520687in}}%
\pgfpathlineto{\pgfqpoint{1.769506in}{1.522896in}}%
\pgfpathlineto{\pgfqpoint{1.782977in}{1.529536in}}%
\pgfpathlineto{\pgfqpoint{1.785162in}{1.531204in}}%
\pgfpathlineto{\pgfqpoint{1.794642in}{1.543147in}}%
\pgfpathlineto{\pgfqpoint{1.798699in}{1.556759in}}%
\pgfpathlineto{\pgfqpoint{1.795995in}{1.570370in}}%
\pgfpathlineto{\pgfqpoint{1.786512in}{1.583981in}}%
\pgfpathlineto{\pgfqpoint{1.785162in}{1.585154in}}%
\pgfpathlineto{\pgfqpoint{1.769506in}{1.593398in}}%
\pgfpathlineto{\pgfqpoint{1.753849in}{1.595749in}}%
\pgfpathlineto{\pgfqpoint{1.738193in}{1.592221in}}%
\pgfpathlineto{\pgfqpoint{1.724454in}{1.583981in}}%
\pgfpathlineto{\pgfqpoint{1.722536in}{1.582081in}}%
\pgfpathlineto{\pgfqpoint{1.714897in}{1.570370in}}%
\pgfpathlineto{\pgfqpoint{1.712356in}{1.556759in}}%
\pgfpathlineto{\pgfqpoint{1.716169in}{1.543148in}}%
\pgfpathlineto{\pgfqpoint{1.722536in}{1.534548in}}%
\pgfpathlineto{\pgfqpoint{1.728301in}{1.529536in}}%
\pgfpathlineto{\pgfqpoint{1.738193in}{1.524001in}}%
\pgfpathclose%
\pgfusepath{fill}%
\end{pgfscope}%
\begin{pgfscope}%
\pgfpathrectangle{\pgfqpoint{0.360415in}{0.345370in}}{\pgfqpoint{1.550000in}{1.347500in}}%
\pgfusepath{clip}%
\pgfsetbuttcap%
\pgfsetroundjoin%
\definecolor{currentfill}{rgb}{0.993814,0.704741,0.183043}%
\pgfsetfillcolor{currentfill}%
\pgfsetlinewidth{0.000000pt}%
\definecolor{currentstroke}{rgb}{0.000000,0.000000,0.000000}%
\pgfsetstrokecolor{currentstroke}%
\pgfsetdash{}{0pt}%
\pgfpathmoveto{\pgfqpoint{0.501324in}{0.424860in}}%
\pgfpathlineto{\pgfqpoint{0.516981in}{0.423335in}}%
\pgfpathlineto{\pgfqpoint{0.532637in}{0.425623in}}%
\pgfpathlineto{\pgfqpoint{0.536336in}{0.427036in}}%
\pgfpathlineto{\pgfqpoint{0.548294in}{0.432281in}}%
\pgfpathlineto{\pgfqpoint{0.560239in}{0.440648in}}%
\pgfpathlineto{\pgfqpoint{0.563950in}{0.444116in}}%
\pgfpathlineto{\pgfqpoint{0.572364in}{0.454259in}}%
\pgfpathlineto{\pgfqpoint{0.579015in}{0.467870in}}%
\pgfpathlineto{\pgfqpoint{0.579607in}{0.472096in}}%
\pgfpathlineto{\pgfqpoint{0.580803in}{0.481481in}}%
\pgfpathlineto{\pgfqpoint{0.579607in}{0.487759in}}%
\pgfpathlineto{\pgfqpoint{0.578066in}{0.495092in}}%
\pgfpathlineto{\pgfqpoint{0.570461in}{0.508703in}}%
\pgfpathlineto{\pgfqpoint{0.563950in}{0.516026in}}%
\pgfpathlineto{\pgfqpoint{0.556717in}{0.522314in}}%
\pgfpathlineto{\pgfqpoint{0.548294in}{0.527974in}}%
\pgfpathlineto{\pgfqpoint{0.532637in}{0.534585in}}%
\pgfpathlineto{\pgfqpoint{0.524202in}{0.535925in}}%
\pgfpathlineto{\pgfqpoint{0.516981in}{0.536965in}}%
\pgfpathlineto{\pgfqpoint{0.506186in}{0.535925in}}%
\pgfpathlineto{\pgfqpoint{0.501324in}{0.535411in}}%
\pgfpathlineto{\pgfqpoint{0.485668in}{0.529628in}}%
\pgfpathlineto{\pgfqpoint{0.474001in}{0.522314in}}%
\pgfpathlineto{\pgfqpoint{0.470011in}{0.519088in}}%
\pgfpathlineto{\pgfqpoint{0.460387in}{0.508703in}}%
\pgfpathlineto{\pgfqpoint{0.454354in}{0.498307in}}%
\pgfpathlineto{\pgfqpoint{0.452729in}{0.495092in}}%
\pgfpathlineto{\pgfqpoint{0.450097in}{0.481481in}}%
\pgfpathlineto{\pgfqpoint{0.451851in}{0.467870in}}%
\pgfpathlineto{\pgfqpoint{0.454354in}{0.462248in}}%
\pgfpathlineto{\pgfqpoint{0.458424in}{0.454259in}}%
\pgfpathlineto{\pgfqpoint{0.470011in}{0.440857in}}%
\pgfpathlineto{\pgfqpoint{0.470252in}{0.440648in}}%
\pgfpathlineto{\pgfqpoint{0.485668in}{0.430575in}}%
\pgfpathlineto{\pgfqpoint{0.494857in}{0.427036in}}%
\pgfpathlineto{\pgfqpoint{0.501324in}{0.424860in}}%
\pgfpathclose%
\pgfpathmoveto{\pgfqpoint{0.484318in}{0.454259in}}%
\pgfpathlineto{\pgfqpoint{0.474835in}{0.467870in}}%
\pgfpathlineto{\pgfqpoint{0.472131in}{0.481481in}}%
\pgfpathlineto{\pgfqpoint{0.476188in}{0.495092in}}%
\pgfpathlineto{\pgfqpoint{0.485668in}{0.507035in}}%
\pgfpathlineto{\pgfqpoint{0.487853in}{0.508703in}}%
\pgfpathlineto{\pgfqpoint{0.501324in}{0.515344in}}%
\pgfpathlineto{\pgfqpoint{0.516981in}{0.517553in}}%
\pgfpathlineto{\pgfqpoint{0.532637in}{0.514239in}}%
\pgfpathlineto{\pgfqpoint{0.542529in}{0.508703in}}%
\pgfpathlineto{\pgfqpoint{0.548294in}{0.503691in}}%
\pgfpathlineto{\pgfqpoint{0.554661in}{0.495092in}}%
\pgfpathlineto{\pgfqpoint{0.558474in}{0.481481in}}%
\pgfpathlineto{\pgfqpoint{0.555933in}{0.467870in}}%
\pgfpathlineto{\pgfqpoint{0.548294in}{0.456158in}}%
\pgfpathlineto{\pgfqpoint{0.546376in}{0.454259in}}%
\pgfpathlineto{\pgfqpoint{0.532637in}{0.446018in}}%
\pgfpathlineto{\pgfqpoint{0.516981in}{0.442491in}}%
\pgfpathlineto{\pgfqpoint{0.501324in}{0.444842in}}%
\pgfpathlineto{\pgfqpoint{0.485668in}{0.453085in}}%
\pgfpathlineto{\pgfqpoint{0.484318in}{0.454259in}}%
\pgfpathclose%
\pgfpathmoveto{\pgfqpoint{0.814455in}{0.424250in}}%
\pgfpathlineto{\pgfqpoint{0.830112in}{0.423488in}}%
\pgfpathlineto{\pgfqpoint{0.845769in}{0.426540in}}%
\pgfpathlineto{\pgfqpoint{0.846933in}{0.427036in}}%
\pgfpathlineto{\pgfqpoint{0.861425in}{0.434158in}}%
\pgfpathlineto{\pgfqpoint{0.870128in}{0.440648in}}%
\pgfpathlineto{\pgfqpoint{0.877082in}{0.447587in}}%
\pgfpathlineto{\pgfqpoint{0.882414in}{0.454259in}}%
\pgfpathlineto{\pgfqpoint{0.888893in}{0.467870in}}%
\pgfpathlineto{\pgfqpoint{0.890741in}{0.481481in}}%
\pgfpathlineto{\pgfqpoint{0.887968in}{0.495092in}}%
\pgfpathlineto{\pgfqpoint{0.880561in}{0.508703in}}%
\pgfpathlineto{\pgfqpoint{0.877082in}{0.512764in}}%
\pgfpathlineto{\pgfqpoint{0.866793in}{0.522314in}}%
\pgfpathlineto{\pgfqpoint{0.861425in}{0.526154in}}%
\pgfpathlineto{\pgfqpoint{0.845769in}{0.533595in}}%
\pgfpathlineto{\pgfqpoint{0.834760in}{0.535925in}}%
\pgfpathlineto{\pgfqpoint{0.830112in}{0.536814in}}%
\pgfpathlineto{\pgfqpoint{0.814455in}{0.536058in}}%
\pgfpathlineto{\pgfqpoint{0.813987in}{0.535925in}}%
\pgfpathlineto{\pgfqpoint{0.798799in}{0.531116in}}%
\pgfpathlineto{\pgfqpoint{0.783631in}{0.522314in}}%
\pgfpathlineto{\pgfqpoint{0.783142in}{0.521949in}}%
\pgfpathlineto{\pgfqpoint{0.770289in}{0.508703in}}%
\pgfpathlineto{\pgfqpoint{0.767486in}{0.504102in}}%
\pgfpathlineto{\pgfqpoint{0.762815in}{0.495092in}}%
\pgfpathlineto{\pgfqpoint{0.760149in}{0.481481in}}%
\pgfpathlineto{\pgfqpoint{0.761926in}{0.467870in}}%
\pgfpathlineto{\pgfqpoint{0.767486in}{0.455693in}}%
\pgfpathlineto{\pgfqpoint{0.768253in}{0.454259in}}%
\pgfpathlineto{\pgfqpoint{0.780465in}{0.440648in}}%
\pgfpathlineto{\pgfqpoint{0.783142in}{0.438424in}}%
\pgfpathlineto{\pgfqpoint{0.798799in}{0.429039in}}%
\pgfpathlineto{\pgfqpoint{0.804853in}{0.427036in}}%
\pgfpathlineto{\pgfqpoint{0.814455in}{0.424250in}}%
\pgfpathclose%
\pgfpathmoveto{\pgfqpoint{0.794704in}{0.454259in}}%
\pgfpathlineto{\pgfqpoint{0.784527in}{0.467870in}}%
\pgfpathlineto{\pgfqpoint{0.783142in}{0.474338in}}%
\pgfpathlineto{\pgfqpoint{0.781993in}{0.481481in}}%
\pgfpathlineto{\pgfqpoint{0.783142in}{0.486259in}}%
\pgfpathlineto{\pgfqpoint{0.785979in}{0.495092in}}%
\pgfpathlineto{\pgfqpoint{0.797615in}{0.508703in}}%
\pgfpathlineto{\pgfqpoint{0.798799in}{0.509592in}}%
\pgfpathlineto{\pgfqpoint{0.814455in}{0.516228in}}%
\pgfpathlineto{\pgfqpoint{0.830112in}{0.517332in}}%
\pgfpathlineto{\pgfqpoint{0.845769in}{0.512912in}}%
\pgfpathlineto{\pgfqpoint{0.852503in}{0.508703in}}%
\pgfpathlineto{\pgfqpoint{0.861425in}{0.500012in}}%
\pgfpathlineto{\pgfqpoint{0.864847in}{0.495092in}}%
\pgfpathlineto{\pgfqpoint{0.868457in}{0.481481in}}%
\pgfpathlineto{\pgfqpoint{0.866051in}{0.467870in}}%
\pgfpathlineto{\pgfqpoint{0.861425in}{0.460320in}}%
\pgfpathlineto{\pgfqpoint{0.855963in}{0.454259in}}%
\pgfpathlineto{\pgfqpoint{0.845769in}{0.447430in}}%
\pgfpathlineto{\pgfqpoint{0.830112in}{0.442726in}}%
\pgfpathlineto{\pgfqpoint{0.814455in}{0.443901in}}%
\pgfpathlineto{\pgfqpoint{0.798799in}{0.450964in}}%
\pgfpathlineto{\pgfqpoint{0.794704in}{0.454259in}}%
\pgfpathclose%
\pgfpathmoveto{\pgfqpoint{1.127587in}{0.423793in}}%
\pgfpathlineto{\pgfqpoint{1.143243in}{0.423793in}}%
\pgfpathlineto{\pgfqpoint{1.156586in}{0.427036in}}%
\pgfpathlineto{\pgfqpoint{1.158900in}{0.427675in}}%
\pgfpathlineto{\pgfqpoint{1.174556in}{0.436206in}}%
\pgfpathlineto{\pgfqpoint{1.180186in}{0.440648in}}%
\pgfpathlineto{\pgfqpoint{1.190213in}{0.451268in}}%
\pgfpathlineto{\pgfqpoint{1.192526in}{0.454259in}}%
\pgfpathlineto{\pgfqpoint{1.198866in}{0.467870in}}%
\pgfpathlineto{\pgfqpoint{1.200674in}{0.481481in}}%
\pgfpathlineto{\pgfqpoint{1.197962in}{0.495092in}}%
\pgfpathlineto{\pgfqpoint{1.190713in}{0.508703in}}%
\pgfpathlineto{\pgfqpoint{1.190213in}{0.509306in}}%
\pgfpathlineto{\pgfqpoint{1.177007in}{0.522314in}}%
\pgfpathlineto{\pgfqpoint{1.174556in}{0.524169in}}%
\pgfpathlineto{\pgfqpoint{1.158900in}{0.532439in}}%
\pgfpathlineto{\pgfqpoint{1.145709in}{0.535925in}}%
\pgfpathlineto{\pgfqpoint{1.143243in}{0.536512in}}%
\pgfpathlineto{\pgfqpoint{1.127587in}{0.536512in}}%
\pgfpathlineto{\pgfqpoint{1.125121in}{0.535925in}}%
\pgfpathlineto{\pgfqpoint{1.111930in}{0.532439in}}%
\pgfpathlineto{\pgfqpoint{1.096274in}{0.524169in}}%
\pgfpathlineto{\pgfqpoint{1.093823in}{0.522314in}}%
\pgfpathlineto{\pgfqpoint{1.080617in}{0.509306in}}%
\pgfpathlineto{\pgfqpoint{1.080117in}{0.508703in}}%
\pgfpathlineto{\pgfqpoint{1.072868in}{0.495092in}}%
\pgfpathlineto{\pgfqpoint{1.070156in}{0.481481in}}%
\pgfpathlineto{\pgfqpoint{1.071964in}{0.467870in}}%
\pgfpathlineto{\pgfqpoint{1.078304in}{0.454259in}}%
\pgfpathlineto{\pgfqpoint{1.080617in}{0.451268in}}%
\pgfpathlineto{\pgfqpoint{1.090644in}{0.440648in}}%
\pgfpathlineto{\pgfqpoint{1.096274in}{0.436206in}}%
\pgfpathlineto{\pgfqpoint{1.111930in}{0.427675in}}%
\pgfpathlineto{\pgfqpoint{1.114244in}{0.427036in}}%
\pgfpathlineto{\pgfqpoint{1.127587in}{0.423793in}}%
\pgfpathclose%
\pgfpathmoveto{\pgfqpoint{1.104913in}{0.454259in}}%
\pgfpathlineto{\pgfqpoint{1.096274in}{0.464859in}}%
\pgfpathlineto{\pgfqpoint{1.094530in}{0.467870in}}%
\pgfpathlineto{\pgfqpoint{1.092238in}{0.481481in}}%
\pgfpathlineto{\pgfqpoint{1.095678in}{0.495092in}}%
\pgfpathlineto{\pgfqpoint{1.096274in}{0.495999in}}%
\pgfpathlineto{\pgfqpoint{1.108068in}{0.508703in}}%
\pgfpathlineto{\pgfqpoint{1.111930in}{0.511363in}}%
\pgfpathlineto{\pgfqpoint{1.127587in}{0.516890in}}%
\pgfpathlineto{\pgfqpoint{1.143243in}{0.516890in}}%
\pgfpathlineto{\pgfqpoint{1.158900in}{0.511363in}}%
\pgfpathlineto{\pgfqpoint{1.162762in}{0.508703in}}%
\pgfpathlineto{\pgfqpoint{1.174556in}{0.495999in}}%
\pgfpathlineto{\pgfqpoint{1.175152in}{0.495092in}}%
\pgfpathlineto{\pgfqpoint{1.178592in}{0.481481in}}%
\pgfpathlineto{\pgfqpoint{1.176300in}{0.467870in}}%
\pgfpathlineto{\pgfqpoint{1.174556in}{0.464859in}}%
\pgfpathlineto{\pgfqpoint{1.165917in}{0.454259in}}%
\pgfpathlineto{\pgfqpoint{1.158900in}{0.449079in}}%
\pgfpathlineto{\pgfqpoint{1.143243in}{0.443196in}}%
\pgfpathlineto{\pgfqpoint{1.127587in}{0.443196in}}%
\pgfpathlineto{\pgfqpoint{1.111930in}{0.449079in}}%
\pgfpathlineto{\pgfqpoint{1.104913in}{0.454259in}}%
\pgfpathclose%
\pgfpathmoveto{\pgfqpoint{1.425061in}{0.426540in}}%
\pgfpathlineto{\pgfqpoint{1.440718in}{0.423488in}}%
\pgfpathlineto{\pgfqpoint{1.456375in}{0.424250in}}%
\pgfpathlineto{\pgfqpoint{1.465977in}{0.427036in}}%
\pgfpathlineto{\pgfqpoint{1.472031in}{0.429039in}}%
\pgfpathlineto{\pgfqpoint{1.487688in}{0.438424in}}%
\pgfpathlineto{\pgfqpoint{1.490365in}{0.440648in}}%
\pgfpathlineto{\pgfqpoint{1.502577in}{0.454259in}}%
\pgfpathlineto{\pgfqpoint{1.503344in}{0.455693in}}%
\pgfpathlineto{\pgfqpoint{1.508904in}{0.467870in}}%
\pgfpathlineto{\pgfqpoint{1.510681in}{0.481481in}}%
\pgfpathlineto{\pgfqpoint{1.508015in}{0.495092in}}%
\pgfpathlineto{\pgfqpoint{1.503344in}{0.504102in}}%
\pgfpathlineto{\pgfqpoint{1.500541in}{0.508703in}}%
\pgfpathlineto{\pgfqpoint{1.487688in}{0.521949in}}%
\pgfpathlineto{\pgfqpoint{1.487199in}{0.522314in}}%
\pgfpathlineto{\pgfqpoint{1.472031in}{0.531116in}}%
\pgfpathlineto{\pgfqpoint{1.456843in}{0.535925in}}%
\pgfpathlineto{\pgfqpoint{1.456375in}{0.536058in}}%
\pgfpathlineto{\pgfqpoint{1.440718in}{0.536814in}}%
\pgfpathlineto{\pgfqpoint{1.436070in}{0.535925in}}%
\pgfpathlineto{\pgfqpoint{1.425061in}{0.533595in}}%
\pgfpathlineto{\pgfqpoint{1.409405in}{0.526154in}}%
\pgfpathlineto{\pgfqpoint{1.404037in}{0.522314in}}%
\pgfpathlineto{\pgfqpoint{1.393748in}{0.512764in}}%
\pgfpathlineto{\pgfqpoint{1.390269in}{0.508703in}}%
\pgfpathlineto{\pgfqpoint{1.382862in}{0.495092in}}%
\pgfpathlineto{\pgfqpoint{1.380089in}{0.481481in}}%
\pgfpathlineto{\pgfqpoint{1.381937in}{0.467870in}}%
\pgfpathlineto{\pgfqpoint{1.388416in}{0.454259in}}%
\pgfpathlineto{\pgfqpoint{1.393748in}{0.447587in}}%
\pgfpathlineto{\pgfqpoint{1.400702in}{0.440648in}}%
\pgfpathlineto{\pgfqpoint{1.409405in}{0.434158in}}%
\pgfpathlineto{\pgfqpoint{1.423897in}{0.427036in}}%
\pgfpathlineto{\pgfqpoint{1.425061in}{0.426540in}}%
\pgfpathclose%
\pgfpathmoveto{\pgfqpoint{1.414867in}{0.454259in}}%
\pgfpathlineto{\pgfqpoint{1.409405in}{0.460320in}}%
\pgfpathlineto{\pgfqpoint{1.404779in}{0.467870in}}%
\pgfpathlineto{\pgfqpoint{1.402373in}{0.481481in}}%
\pgfpathlineto{\pgfqpoint{1.405983in}{0.495092in}}%
\pgfpathlineto{\pgfqpoint{1.409405in}{0.500012in}}%
\pgfpathlineto{\pgfqpoint{1.418327in}{0.508703in}}%
\pgfpathlineto{\pgfqpoint{1.425061in}{0.512912in}}%
\pgfpathlineto{\pgfqpoint{1.440718in}{0.517332in}}%
\pgfpathlineto{\pgfqpoint{1.456375in}{0.516228in}}%
\pgfpathlineto{\pgfqpoint{1.472031in}{0.509592in}}%
\pgfpathlineto{\pgfqpoint{1.473215in}{0.508703in}}%
\pgfpathlineto{\pgfqpoint{1.484851in}{0.495092in}}%
\pgfpathlineto{\pgfqpoint{1.487688in}{0.486259in}}%
\pgfpathlineto{\pgfqpoint{1.488837in}{0.481481in}}%
\pgfpathlineto{\pgfqpoint{1.487688in}{0.474338in}}%
\pgfpathlineto{\pgfqpoint{1.486303in}{0.467870in}}%
\pgfpathlineto{\pgfqpoint{1.476126in}{0.454259in}}%
\pgfpathlineto{\pgfqpoint{1.472031in}{0.450964in}}%
\pgfpathlineto{\pgfqpoint{1.456375in}{0.443901in}}%
\pgfpathlineto{\pgfqpoint{1.440718in}{0.442726in}}%
\pgfpathlineto{\pgfqpoint{1.425061in}{0.447430in}}%
\pgfpathlineto{\pgfqpoint{1.414867in}{0.454259in}}%
\pgfpathclose%
\pgfpathmoveto{\pgfqpoint{1.738193in}{0.425623in}}%
\pgfpathlineto{\pgfqpoint{1.753849in}{0.423335in}}%
\pgfpathlineto{\pgfqpoint{1.769506in}{0.424860in}}%
\pgfpathlineto{\pgfqpoint{1.775973in}{0.427036in}}%
\pgfpathlineto{\pgfqpoint{1.785162in}{0.430575in}}%
\pgfpathlineto{\pgfqpoint{1.800578in}{0.440648in}}%
\pgfpathlineto{\pgfqpoint{1.800819in}{0.440857in}}%
\pgfpathlineto{\pgfqpoint{1.812406in}{0.454259in}}%
\pgfpathlineto{\pgfqpoint{1.816476in}{0.462248in}}%
\pgfpathlineto{\pgfqpoint{1.818979in}{0.467870in}}%
\pgfpathlineto{\pgfqpoint{1.820733in}{0.481481in}}%
\pgfpathlineto{\pgfqpoint{1.818101in}{0.495092in}}%
\pgfpathlineto{\pgfqpoint{1.816476in}{0.498307in}}%
\pgfpathlineto{\pgfqpoint{1.810443in}{0.508703in}}%
\pgfpathlineto{\pgfqpoint{1.800819in}{0.519088in}}%
\pgfpathlineto{\pgfqpoint{1.796829in}{0.522314in}}%
\pgfpathlineto{\pgfqpoint{1.785162in}{0.529628in}}%
\pgfpathlineto{\pgfqpoint{1.769506in}{0.535411in}}%
\pgfpathlineto{\pgfqpoint{1.764644in}{0.535925in}}%
\pgfpathlineto{\pgfqpoint{1.753849in}{0.536965in}}%
\pgfpathlineto{\pgfqpoint{1.746628in}{0.535925in}}%
\pgfpathlineto{\pgfqpoint{1.738193in}{0.534585in}}%
\pgfpathlineto{\pgfqpoint{1.722536in}{0.527974in}}%
\pgfpathlineto{\pgfqpoint{1.714113in}{0.522314in}}%
\pgfpathlineto{\pgfqpoint{1.706880in}{0.516026in}}%
\pgfpathlineto{\pgfqpoint{1.700369in}{0.508703in}}%
\pgfpathlineto{\pgfqpoint{1.692764in}{0.495092in}}%
\pgfpathlineto{\pgfqpoint{1.691223in}{0.487759in}}%
\pgfpathlineto{\pgfqpoint{1.690027in}{0.481481in}}%
\pgfpathlineto{\pgfqpoint{1.691223in}{0.472096in}}%
\pgfpathlineto{\pgfqpoint{1.691815in}{0.467870in}}%
\pgfpathlineto{\pgfqpoint{1.698466in}{0.454259in}}%
\pgfpathlineto{\pgfqpoint{1.706880in}{0.444116in}}%
\pgfpathlineto{\pgfqpoint{1.710591in}{0.440648in}}%
\pgfpathlineto{\pgfqpoint{1.722536in}{0.432281in}}%
\pgfpathlineto{\pgfqpoint{1.734494in}{0.427036in}}%
\pgfpathlineto{\pgfqpoint{1.738193in}{0.425623in}}%
\pgfpathclose%
\pgfpathmoveto{\pgfqpoint{1.724454in}{0.454259in}}%
\pgfpathlineto{\pgfqpoint{1.722536in}{0.456158in}}%
\pgfpathlineto{\pgfqpoint{1.714897in}{0.467870in}}%
\pgfpathlineto{\pgfqpoint{1.712356in}{0.481481in}}%
\pgfpathlineto{\pgfqpoint{1.716169in}{0.495092in}}%
\pgfpathlineto{\pgfqpoint{1.722536in}{0.503691in}}%
\pgfpathlineto{\pgfqpoint{1.728301in}{0.508703in}}%
\pgfpathlineto{\pgfqpoint{1.738193in}{0.514239in}}%
\pgfpathlineto{\pgfqpoint{1.753849in}{0.517553in}}%
\pgfpathlineto{\pgfqpoint{1.769506in}{0.515344in}}%
\pgfpathlineto{\pgfqpoint{1.782977in}{0.508703in}}%
\pgfpathlineto{\pgfqpoint{1.785162in}{0.507035in}}%
\pgfpathlineto{\pgfqpoint{1.794642in}{0.495092in}}%
\pgfpathlineto{\pgfqpoint{1.798699in}{0.481481in}}%
\pgfpathlineto{\pgfqpoint{1.795995in}{0.467870in}}%
\pgfpathlineto{\pgfqpoint{1.786512in}{0.454259in}}%
\pgfpathlineto{\pgfqpoint{1.785162in}{0.453085in}}%
\pgfpathlineto{\pgfqpoint{1.769506in}{0.444842in}}%
\pgfpathlineto{\pgfqpoint{1.753849in}{0.442491in}}%
\pgfpathlineto{\pgfqpoint{1.738193in}{0.446018in}}%
\pgfpathlineto{\pgfqpoint{1.724454in}{0.454259in}}%
\pgfpathclose%
\pgfpathmoveto{\pgfqpoint{0.501324in}{0.694425in}}%
\pgfpathlineto{\pgfqpoint{0.516981in}{0.692880in}}%
\pgfpathlineto{\pgfqpoint{0.532637in}{0.695198in}}%
\pgfpathlineto{\pgfqpoint{0.543002in}{0.699259in}}%
\pgfpathlineto{\pgfqpoint{0.548294in}{0.701695in}}%
\pgfpathlineto{\pgfqpoint{0.563531in}{0.712870in}}%
\pgfpathlineto{\pgfqpoint{0.563950in}{0.713295in}}%
\pgfpathlineto{\pgfqpoint{0.574075in}{0.726481in}}%
\pgfpathlineto{\pgfqpoint{0.579607in}{0.739685in}}%
\pgfpathlineto{\pgfqpoint{0.579760in}{0.740092in}}%
\pgfpathlineto{\pgfqpoint{0.580629in}{0.753703in}}%
\pgfpathlineto{\pgfqpoint{0.579607in}{0.757744in}}%
\pgfpathlineto{\pgfqpoint{0.576926in}{0.767314in}}%
\pgfpathlineto{\pgfqpoint{0.568368in}{0.780925in}}%
\pgfpathlineto{\pgfqpoint{0.563950in}{0.785592in}}%
\pgfpathlineto{\pgfqpoint{0.552965in}{0.794536in}}%
\pgfpathlineto{\pgfqpoint{0.548294in}{0.797561in}}%
\pgfpathlineto{\pgfqpoint{0.532637in}{0.804001in}}%
\pgfpathlineto{\pgfqpoint{0.516981in}{0.806411in}}%
\pgfpathlineto{\pgfqpoint{0.501324in}{0.804805in}}%
\pgfpathlineto{\pgfqpoint{0.485668in}{0.799172in}}%
\pgfpathlineto{\pgfqpoint{0.477994in}{0.794536in}}%
\pgfpathlineto{\pgfqpoint{0.470011in}{0.788492in}}%
\pgfpathlineto{\pgfqpoint{0.462547in}{0.780925in}}%
\pgfpathlineto{\pgfqpoint{0.454354in}{0.768326in}}%
\pgfpathlineto{\pgfqpoint{0.453783in}{0.767314in}}%
\pgfpathlineto{\pgfqpoint{0.450272in}{0.753703in}}%
\pgfpathlineto{\pgfqpoint{0.451149in}{0.740092in}}%
\pgfpathlineto{\pgfqpoint{0.454354in}{0.731744in}}%
\pgfpathlineto{\pgfqpoint{0.456658in}{0.726481in}}%
\pgfpathlineto{\pgfqpoint{0.467454in}{0.712870in}}%
\pgfpathlineto{\pgfqpoint{0.470011in}{0.710542in}}%
\pgfpathlineto{\pgfqpoint{0.485668in}{0.699926in}}%
\pgfpathlineto{\pgfqpoint{0.487318in}{0.699259in}}%
\pgfpathlineto{\pgfqpoint{0.501324in}{0.694425in}}%
\pgfpathclose%
\pgfpathmoveto{\pgfqpoint{0.508764in}{0.712870in}}%
\pgfpathlineto{\pgfqpoint{0.501324in}{0.714073in}}%
\pgfpathlineto{\pgfqpoint{0.485668in}{0.722921in}}%
\pgfpathlineto{\pgfqpoint{0.481877in}{0.726481in}}%
\pgfpathlineto{\pgfqpoint{0.473753in}{0.740092in}}%
\pgfpathlineto{\pgfqpoint{0.472401in}{0.753703in}}%
\pgfpathlineto{\pgfqpoint{0.477813in}{0.767314in}}%
\pgfpathlineto{\pgfqpoint{0.485668in}{0.776177in}}%
\pgfpathlineto{\pgfqpoint{0.492639in}{0.780925in}}%
\pgfpathlineto{\pgfqpoint{0.501324in}{0.784947in}}%
\pgfpathlineto{\pgfqpoint{0.516981in}{0.787038in}}%
\pgfpathlineto{\pgfqpoint{0.532637in}{0.783900in}}%
\pgfpathlineto{\pgfqpoint{0.538297in}{0.780925in}}%
\pgfpathlineto{\pgfqpoint{0.548294in}{0.773169in}}%
\pgfpathlineto{\pgfqpoint{0.553135in}{0.767314in}}%
\pgfpathlineto{\pgfqpoint{0.558220in}{0.753703in}}%
\pgfpathlineto{\pgfqpoint{0.556949in}{0.740092in}}%
\pgfpathlineto{\pgfqpoint{0.549316in}{0.726481in}}%
\pgfpathlineto{\pgfqpoint{0.548294in}{0.725452in}}%
\pgfpathlineto{\pgfqpoint{0.532637in}{0.715336in}}%
\pgfpathlineto{\pgfqpoint{0.522477in}{0.712870in}}%
\pgfpathlineto{\pgfqpoint{0.516981in}{0.711870in}}%
\pgfpathlineto{\pgfqpoint{0.508764in}{0.712870in}}%
\pgfpathclose%
\pgfpathmoveto{\pgfqpoint{0.798799in}{0.698449in}}%
\pgfpathlineto{\pgfqpoint{0.814455in}{0.693807in}}%
\pgfpathlineto{\pgfqpoint{0.830112in}{0.693034in}}%
\pgfpathlineto{\pgfqpoint{0.845769in}{0.696126in}}%
\pgfpathlineto{\pgfqpoint{0.852929in}{0.699259in}}%
\pgfpathlineto{\pgfqpoint{0.861425in}{0.703642in}}%
\pgfpathlineto{\pgfqpoint{0.873244in}{0.712870in}}%
\pgfpathlineto{\pgfqpoint{0.877082in}{0.717020in}}%
\pgfpathlineto{\pgfqpoint{0.884081in}{0.726481in}}%
\pgfpathlineto{\pgfqpoint{0.889632in}{0.740092in}}%
\pgfpathlineto{\pgfqpoint{0.890556in}{0.753703in}}%
\pgfpathlineto{\pgfqpoint{0.886858in}{0.767314in}}%
\pgfpathlineto{\pgfqpoint{0.878522in}{0.780925in}}%
\pgfpathlineto{\pgfqpoint{0.877082in}{0.782504in}}%
\pgfpathlineto{\pgfqpoint{0.863241in}{0.794536in}}%
\pgfpathlineto{\pgfqpoint{0.861425in}{0.795788in}}%
\pgfpathlineto{\pgfqpoint{0.845769in}{0.803036in}}%
\pgfpathlineto{\pgfqpoint{0.830112in}{0.806250in}}%
\pgfpathlineto{\pgfqpoint{0.814455in}{0.805447in}}%
\pgfpathlineto{\pgfqpoint{0.798799in}{0.800622in}}%
\pgfpathlineto{\pgfqpoint{0.787917in}{0.794536in}}%
\pgfpathlineto{\pgfqpoint{0.783142in}{0.791200in}}%
\pgfpathlineto{\pgfqpoint{0.772528in}{0.780925in}}%
\pgfpathlineto{\pgfqpoint{0.767486in}{0.773539in}}%
\pgfpathlineto{\pgfqpoint{0.763883in}{0.767314in}}%
\pgfpathlineto{\pgfqpoint{0.760326in}{0.753703in}}%
\pgfpathlineto{\pgfqpoint{0.761215in}{0.740092in}}%
\pgfpathlineto{\pgfqpoint{0.766554in}{0.726481in}}%
\pgfpathlineto{\pgfqpoint{0.767486in}{0.725141in}}%
\pgfpathlineto{\pgfqpoint{0.777617in}{0.712870in}}%
\pgfpathlineto{\pgfqpoint{0.783142in}{0.708066in}}%
\pgfpathlineto{\pgfqpoint{0.797258in}{0.699259in}}%
\pgfpathlineto{\pgfqpoint{0.798799in}{0.698449in}}%
\pgfpathclose%
\pgfpathmoveto{\pgfqpoint{0.816857in}{0.712870in}}%
\pgfpathlineto{\pgfqpoint{0.814455in}{0.713064in}}%
\pgfpathlineto{\pgfqpoint{0.798799in}{0.720644in}}%
\pgfpathlineto{\pgfqpoint{0.792085in}{0.726481in}}%
\pgfpathlineto{\pgfqpoint{0.783366in}{0.740092in}}%
\pgfpathlineto{\pgfqpoint{0.783142in}{0.742179in}}%
\pgfpathlineto{\pgfqpoint{0.782212in}{0.753703in}}%
\pgfpathlineto{\pgfqpoint{0.783142in}{0.756614in}}%
\pgfpathlineto{\pgfqpoint{0.787723in}{0.767314in}}%
\pgfpathlineto{\pgfqpoint{0.798799in}{0.778883in}}%
\pgfpathlineto{\pgfqpoint{0.802289in}{0.780925in}}%
\pgfpathlineto{\pgfqpoint{0.814455in}{0.785784in}}%
\pgfpathlineto{\pgfqpoint{0.830112in}{0.786829in}}%
\pgfpathlineto{\pgfqpoint{0.845769in}{0.782644in}}%
\pgfpathlineto{\pgfqpoint{0.848697in}{0.780925in}}%
\pgfpathlineto{\pgfqpoint{0.861425in}{0.769860in}}%
\pgfpathlineto{\pgfqpoint{0.863402in}{0.767314in}}%
\pgfpathlineto{\pgfqpoint{0.868216in}{0.753703in}}%
\pgfpathlineto{\pgfqpoint{0.867014in}{0.740092in}}%
\pgfpathlineto{\pgfqpoint{0.861425in}{0.729515in}}%
\pgfpathlineto{\pgfqpoint{0.859076in}{0.726481in}}%
\pgfpathlineto{\pgfqpoint{0.845769in}{0.716852in}}%
\pgfpathlineto{\pgfqpoint{0.833460in}{0.712870in}}%
\pgfpathlineto{\pgfqpoint{0.830112in}{0.712061in}}%
\pgfpathlineto{\pgfqpoint{0.816857in}{0.712870in}}%
\pgfpathclose%
\pgfpathmoveto{\pgfqpoint{1.111930in}{0.697210in}}%
\pgfpathlineto{\pgfqpoint{1.127587in}{0.693343in}}%
\pgfpathlineto{\pgfqpoint{1.143243in}{0.693343in}}%
\pgfpathlineto{\pgfqpoint{1.158900in}{0.697210in}}%
\pgfpathlineto{\pgfqpoint{1.163150in}{0.699259in}}%
\pgfpathlineto{\pgfqpoint{1.174556in}{0.705766in}}%
\pgfpathlineto{\pgfqpoint{1.183155in}{0.712870in}}%
\pgfpathlineto{\pgfqpoint{1.190213in}{0.720971in}}%
\pgfpathlineto{\pgfqpoint{1.194158in}{0.726481in}}%
\pgfpathlineto{\pgfqpoint{1.199590in}{0.740092in}}%
\pgfpathlineto{\pgfqpoint{1.200494in}{0.753703in}}%
\pgfpathlineto{\pgfqpoint{1.196875in}{0.767314in}}%
\pgfpathlineto{\pgfqpoint{1.190213in}{0.778495in}}%
\pgfpathlineto{\pgfqpoint{1.188462in}{0.780925in}}%
\pgfpathlineto{\pgfqpoint{1.174556in}{0.793717in}}%
\pgfpathlineto{\pgfqpoint{1.173274in}{0.794536in}}%
\pgfpathlineto{\pgfqpoint{1.158900in}{0.801910in}}%
\pgfpathlineto{\pgfqpoint{1.143243in}{0.805929in}}%
\pgfpathlineto{\pgfqpoint{1.127587in}{0.805929in}}%
\pgfpathlineto{\pgfqpoint{1.111930in}{0.801910in}}%
\pgfpathlineto{\pgfqpoint{1.097556in}{0.794536in}}%
\pgfpathlineto{\pgfqpoint{1.096274in}{0.793717in}}%
\pgfpathlineto{\pgfqpoint{1.082368in}{0.780925in}}%
\pgfpathlineto{\pgfqpoint{1.080617in}{0.778495in}}%
\pgfpathlineto{\pgfqpoint{1.073955in}{0.767314in}}%
\pgfpathlineto{\pgfqpoint{1.070336in}{0.753703in}}%
\pgfpathlineto{\pgfqpoint{1.071240in}{0.740092in}}%
\pgfpathlineto{\pgfqpoint{1.076672in}{0.726481in}}%
\pgfpathlineto{\pgfqpoint{1.080617in}{0.720971in}}%
\pgfpathlineto{\pgfqpoint{1.087675in}{0.712870in}}%
\pgfpathlineto{\pgfqpoint{1.096274in}{0.705766in}}%
\pgfpathlineto{\pgfqpoint{1.107680in}{0.699259in}}%
\pgfpathlineto{\pgfqpoint{1.111930in}{0.697210in}}%
\pgfpathclose%
\pgfpathmoveto{\pgfqpoint{1.126168in}{0.712870in}}%
\pgfpathlineto{\pgfqpoint{1.111930in}{0.718621in}}%
\pgfpathlineto{\pgfqpoint{1.102074in}{0.726481in}}%
\pgfpathlineto{\pgfqpoint{1.096274in}{0.734764in}}%
\pgfpathlineto{\pgfqpoint{1.093613in}{0.740092in}}%
\pgfpathlineto{\pgfqpoint{1.092467in}{0.753703in}}%
\pgfpathlineto{\pgfqpoint{1.096274in}{0.765024in}}%
\pgfpathlineto{\pgfqpoint{1.097346in}{0.767314in}}%
\pgfpathlineto{\pgfqpoint{1.111540in}{0.780925in}}%
\pgfpathlineto{\pgfqpoint{1.111930in}{0.781178in}}%
\pgfpathlineto{\pgfqpoint{1.127587in}{0.786411in}}%
\pgfpathlineto{\pgfqpoint{1.143243in}{0.786411in}}%
\pgfpathlineto{\pgfqpoint{1.158900in}{0.781178in}}%
\pgfpathlineto{\pgfqpoint{1.159290in}{0.780925in}}%
\pgfpathlineto{\pgfqpoint{1.173484in}{0.767314in}}%
\pgfpathlineto{\pgfqpoint{1.174556in}{0.765024in}}%
\pgfpathlineto{\pgfqpoint{1.178363in}{0.753703in}}%
\pgfpathlineto{\pgfqpoint{1.177217in}{0.740092in}}%
\pgfpathlineto{\pgfqpoint{1.174556in}{0.734764in}}%
\pgfpathlineto{\pgfqpoint{1.168756in}{0.726481in}}%
\pgfpathlineto{\pgfqpoint{1.158900in}{0.718621in}}%
\pgfpathlineto{\pgfqpoint{1.144662in}{0.712870in}}%
\pgfpathlineto{\pgfqpoint{1.143243in}{0.712443in}}%
\pgfpathlineto{\pgfqpoint{1.127587in}{0.712443in}}%
\pgfpathlineto{\pgfqpoint{1.126168in}{0.712870in}}%
\pgfpathclose%
\pgfpathmoveto{\pgfqpoint{1.425061in}{0.696126in}}%
\pgfpathlineto{\pgfqpoint{1.440718in}{0.693034in}}%
\pgfpathlineto{\pgfqpoint{1.456375in}{0.693807in}}%
\pgfpathlineto{\pgfqpoint{1.472031in}{0.698449in}}%
\pgfpathlineto{\pgfqpoint{1.473572in}{0.699259in}}%
\pgfpathlineto{\pgfqpoint{1.487688in}{0.708066in}}%
\pgfpathlineto{\pgfqpoint{1.493213in}{0.712870in}}%
\pgfpathlineto{\pgfqpoint{1.503344in}{0.725141in}}%
\pgfpathlineto{\pgfqpoint{1.504276in}{0.726481in}}%
\pgfpathlineto{\pgfqpoint{1.509615in}{0.740092in}}%
\pgfpathlineto{\pgfqpoint{1.510504in}{0.753703in}}%
\pgfpathlineto{\pgfqpoint{1.506947in}{0.767314in}}%
\pgfpathlineto{\pgfqpoint{1.503344in}{0.773539in}}%
\pgfpathlineto{\pgfqpoint{1.498302in}{0.780925in}}%
\pgfpathlineto{\pgfqpoint{1.487688in}{0.791200in}}%
\pgfpathlineto{\pgfqpoint{1.482913in}{0.794536in}}%
\pgfpathlineto{\pgfqpoint{1.472031in}{0.800622in}}%
\pgfpathlineto{\pgfqpoint{1.456375in}{0.805447in}}%
\pgfpathlineto{\pgfqpoint{1.440718in}{0.806250in}}%
\pgfpathlineto{\pgfqpoint{1.425061in}{0.803036in}}%
\pgfpathlineto{\pgfqpoint{1.409405in}{0.795788in}}%
\pgfpathlineto{\pgfqpoint{1.407589in}{0.794536in}}%
\pgfpathlineto{\pgfqpoint{1.393748in}{0.782504in}}%
\pgfpathlineto{\pgfqpoint{1.392308in}{0.780925in}}%
\pgfpathlineto{\pgfqpoint{1.383972in}{0.767314in}}%
\pgfpathlineto{\pgfqpoint{1.380274in}{0.753703in}}%
\pgfpathlineto{\pgfqpoint{1.381198in}{0.740092in}}%
\pgfpathlineto{\pgfqpoint{1.386749in}{0.726481in}}%
\pgfpathlineto{\pgfqpoint{1.393748in}{0.717020in}}%
\pgfpathlineto{\pgfqpoint{1.397586in}{0.712870in}}%
\pgfpathlineto{\pgfqpoint{1.409405in}{0.703642in}}%
\pgfpathlineto{\pgfqpoint{1.417901in}{0.699259in}}%
\pgfpathlineto{\pgfqpoint{1.425061in}{0.696126in}}%
\pgfpathclose%
\pgfpathmoveto{\pgfqpoint{1.437370in}{0.712870in}}%
\pgfpathlineto{\pgfqpoint{1.425061in}{0.716852in}}%
\pgfpathlineto{\pgfqpoint{1.411754in}{0.726481in}}%
\pgfpathlineto{\pgfqpoint{1.409405in}{0.729515in}}%
\pgfpathlineto{\pgfqpoint{1.403816in}{0.740092in}}%
\pgfpathlineto{\pgfqpoint{1.402614in}{0.753703in}}%
\pgfpathlineto{\pgfqpoint{1.407428in}{0.767314in}}%
\pgfpathlineto{\pgfqpoint{1.409405in}{0.769860in}}%
\pgfpathlineto{\pgfqpoint{1.422133in}{0.780925in}}%
\pgfpathlineto{\pgfqpoint{1.425061in}{0.782644in}}%
\pgfpathlineto{\pgfqpoint{1.440718in}{0.786829in}}%
\pgfpathlineto{\pgfqpoint{1.456375in}{0.785784in}}%
\pgfpathlineto{\pgfqpoint{1.468541in}{0.780925in}}%
\pgfpathlineto{\pgfqpoint{1.472031in}{0.778883in}}%
\pgfpathlineto{\pgfqpoint{1.483107in}{0.767314in}}%
\pgfpathlineto{\pgfqpoint{1.487688in}{0.756614in}}%
\pgfpathlineto{\pgfqpoint{1.488618in}{0.753703in}}%
\pgfpathlineto{\pgfqpoint{1.487688in}{0.742179in}}%
\pgfpathlineto{\pgfqpoint{1.487464in}{0.740092in}}%
\pgfpathlineto{\pgfqpoint{1.478745in}{0.726481in}}%
\pgfpathlineto{\pgfqpoint{1.472031in}{0.720644in}}%
\pgfpathlineto{\pgfqpoint{1.456375in}{0.713064in}}%
\pgfpathlineto{\pgfqpoint{1.453973in}{0.712870in}}%
\pgfpathlineto{\pgfqpoint{1.440718in}{0.712061in}}%
\pgfpathlineto{\pgfqpoint{1.437370in}{0.712870in}}%
\pgfpathclose%
\pgfpathmoveto{\pgfqpoint{1.738193in}{0.695198in}}%
\pgfpathlineto{\pgfqpoint{1.753849in}{0.692880in}}%
\pgfpathlineto{\pgfqpoint{1.769506in}{0.694425in}}%
\pgfpathlineto{\pgfqpoint{1.783512in}{0.699259in}}%
\pgfpathlineto{\pgfqpoint{1.785162in}{0.699926in}}%
\pgfpathlineto{\pgfqpoint{1.800819in}{0.710542in}}%
\pgfpathlineto{\pgfqpoint{1.803376in}{0.712870in}}%
\pgfpathlineto{\pgfqpoint{1.814172in}{0.726481in}}%
\pgfpathlineto{\pgfqpoint{1.816476in}{0.731744in}}%
\pgfpathlineto{\pgfqpoint{1.819681in}{0.740092in}}%
\pgfpathlineto{\pgfqpoint{1.820558in}{0.753703in}}%
\pgfpathlineto{\pgfqpoint{1.817047in}{0.767314in}}%
\pgfpathlineto{\pgfqpoint{1.816476in}{0.768326in}}%
\pgfpathlineto{\pgfqpoint{1.808283in}{0.780925in}}%
\pgfpathlineto{\pgfqpoint{1.800819in}{0.788492in}}%
\pgfpathlineto{\pgfqpoint{1.792836in}{0.794536in}}%
\pgfpathlineto{\pgfqpoint{1.785162in}{0.799172in}}%
\pgfpathlineto{\pgfqpoint{1.769506in}{0.804805in}}%
\pgfpathlineto{\pgfqpoint{1.753849in}{0.806411in}}%
\pgfpathlineto{\pgfqpoint{1.738193in}{0.804001in}}%
\pgfpathlineto{\pgfqpoint{1.722536in}{0.797561in}}%
\pgfpathlineto{\pgfqpoint{1.717865in}{0.794536in}}%
\pgfpathlineto{\pgfqpoint{1.706880in}{0.785592in}}%
\pgfpathlineto{\pgfqpoint{1.702462in}{0.780925in}}%
\pgfpathlineto{\pgfqpoint{1.693904in}{0.767314in}}%
\pgfpathlineto{\pgfqpoint{1.691223in}{0.757744in}}%
\pgfpathlineto{\pgfqpoint{1.690201in}{0.753703in}}%
\pgfpathlineto{\pgfqpoint{1.691070in}{0.740092in}}%
\pgfpathlineto{\pgfqpoint{1.691223in}{0.739685in}}%
\pgfpathlineto{\pgfqpoint{1.696755in}{0.726481in}}%
\pgfpathlineto{\pgfqpoint{1.706880in}{0.713295in}}%
\pgfpathlineto{\pgfqpoint{1.707299in}{0.712870in}}%
\pgfpathlineto{\pgfqpoint{1.722536in}{0.701695in}}%
\pgfpathlineto{\pgfqpoint{1.727828in}{0.699259in}}%
\pgfpathlineto{\pgfqpoint{1.738193in}{0.695198in}}%
\pgfpathclose%
\pgfpathmoveto{\pgfqpoint{1.748353in}{0.712870in}}%
\pgfpathlineto{\pgfqpoint{1.738193in}{0.715336in}}%
\pgfpathlineto{\pgfqpoint{1.722536in}{0.725452in}}%
\pgfpathlineto{\pgfqpoint{1.721514in}{0.726481in}}%
\pgfpathlineto{\pgfqpoint{1.713881in}{0.740092in}}%
\pgfpathlineto{\pgfqpoint{1.712610in}{0.753703in}}%
\pgfpathlineto{\pgfqpoint{1.717695in}{0.767314in}}%
\pgfpathlineto{\pgfqpoint{1.722536in}{0.773169in}}%
\pgfpathlineto{\pgfqpoint{1.732533in}{0.780925in}}%
\pgfpathlineto{\pgfqpoint{1.738193in}{0.783900in}}%
\pgfpathlineto{\pgfqpoint{1.753849in}{0.787038in}}%
\pgfpathlineto{\pgfqpoint{1.769506in}{0.784947in}}%
\pgfpathlineto{\pgfqpoint{1.778191in}{0.780925in}}%
\pgfpathlineto{\pgfqpoint{1.785162in}{0.776177in}}%
\pgfpathlineto{\pgfqpoint{1.793017in}{0.767314in}}%
\pgfpathlineto{\pgfqpoint{1.798429in}{0.753703in}}%
\pgfpathlineto{\pgfqpoint{1.797077in}{0.740092in}}%
\pgfpathlineto{\pgfqpoint{1.788953in}{0.726481in}}%
\pgfpathlineto{\pgfqpoint{1.785162in}{0.722921in}}%
\pgfpathlineto{\pgfqpoint{1.769506in}{0.714073in}}%
\pgfpathlineto{\pgfqpoint{1.762066in}{0.712870in}}%
\pgfpathlineto{\pgfqpoint{1.753849in}{0.711870in}}%
\pgfpathlineto{\pgfqpoint{1.748353in}{0.712870in}}%
\pgfpathclose%
\pgfpathmoveto{\pgfqpoint{0.485668in}{0.969470in}}%
\pgfpathlineto{\pgfqpoint{0.501324in}{0.963958in}}%
\pgfpathlineto{\pgfqpoint{0.516981in}{0.962386in}}%
\pgfpathlineto{\pgfqpoint{0.532637in}{0.964745in}}%
\pgfpathlineto{\pgfqpoint{0.548294in}{0.971046in}}%
\pgfpathlineto{\pgfqpoint{0.548987in}{0.971481in}}%
\pgfpathlineto{\pgfqpoint{0.563950in}{0.982961in}}%
\pgfpathlineto{\pgfqpoint{0.566084in}{0.985092in}}%
\pgfpathlineto{\pgfqpoint{0.575596in}{0.998703in}}%
\pgfpathlineto{\pgfqpoint{0.579607in}{1.010171in}}%
\pgfpathlineto{\pgfqpoint{0.580282in}{1.012314in}}%
\pgfpathlineto{\pgfqpoint{0.580282in}{1.025925in}}%
\pgfpathlineto{\pgfqpoint{0.579607in}{1.028069in}}%
\pgfpathlineto{\pgfqpoint{0.575596in}{1.039536in}}%
\pgfpathlineto{\pgfqpoint{0.566084in}{1.053148in}}%
\pgfpathlineto{\pgfqpoint{0.563950in}{1.055278in}}%
\pgfpathlineto{\pgfqpoint{0.548987in}{1.066759in}}%
\pgfpathlineto{\pgfqpoint{0.548294in}{1.067193in}}%
\pgfpathlineto{\pgfqpoint{0.532637in}{1.073495in}}%
\pgfpathlineto{\pgfqpoint{0.516981in}{1.075853in}}%
\pgfpathlineto{\pgfqpoint{0.501324in}{1.074281in}}%
\pgfpathlineto{\pgfqpoint{0.485668in}{1.068770in}}%
\pgfpathlineto{\pgfqpoint{0.482227in}{1.066759in}}%
\pgfpathlineto{\pgfqpoint{0.470011in}{1.058041in}}%
\pgfpathlineto{\pgfqpoint{0.464902in}{1.053148in}}%
\pgfpathlineto{\pgfqpoint{0.455089in}{1.039536in}}%
\pgfpathlineto{\pgfqpoint{0.454354in}{1.037525in}}%
\pgfpathlineto{\pgfqpoint{0.450623in}{1.025925in}}%
\pgfpathlineto{\pgfqpoint{0.450623in}{1.012314in}}%
\pgfpathlineto{\pgfqpoint{0.454354in}{1.000714in}}%
\pgfpathlineto{\pgfqpoint{0.455089in}{0.998703in}}%
\pgfpathlineto{\pgfqpoint{0.464902in}{0.985092in}}%
\pgfpathlineto{\pgfqpoint{0.470011in}{0.980198in}}%
\pgfpathlineto{\pgfqpoint{0.482227in}{0.971481in}}%
\pgfpathlineto{\pgfqpoint{0.485668in}{0.969470in}}%
\pgfpathclose%
\pgfpathmoveto{\pgfqpoint{0.497861in}{0.985092in}}%
\pgfpathlineto{\pgfqpoint{0.485668in}{0.992603in}}%
\pgfpathlineto{\pgfqpoint{0.479709in}{0.998703in}}%
\pgfpathlineto{\pgfqpoint{0.472942in}{1.012314in}}%
\pgfpathlineto{\pgfqpoint{0.472942in}{1.025925in}}%
\pgfpathlineto{\pgfqpoint{0.479709in}{1.039536in}}%
\pgfpathlineto{\pgfqpoint{0.485668in}{1.045637in}}%
\pgfpathlineto{\pgfqpoint{0.497861in}{1.053148in}}%
\pgfpathlineto{\pgfqpoint{0.501324in}{1.054663in}}%
\pgfpathlineto{\pgfqpoint{0.516981in}{1.056656in}}%
\pgfpathlineto{\pgfqpoint{0.532637in}{1.053666in}}%
\pgfpathlineto{\pgfqpoint{0.533680in}{1.053148in}}%
\pgfpathlineto{\pgfqpoint{0.548294in}{1.042894in}}%
\pgfpathlineto{\pgfqpoint{0.551353in}{1.039536in}}%
\pgfpathlineto{\pgfqpoint{0.557712in}{1.025925in}}%
\pgfpathlineto{\pgfqpoint{0.557712in}{1.012314in}}%
\pgfpathlineto{\pgfqpoint{0.551353in}{0.998703in}}%
\pgfpathlineto{\pgfqpoint{0.548294in}{0.995346in}}%
\pgfpathlineto{\pgfqpoint{0.533680in}{0.985092in}}%
\pgfpathlineto{\pgfqpoint{0.532637in}{0.984574in}}%
\pgfpathlineto{\pgfqpoint{0.516981in}{0.981583in}}%
\pgfpathlineto{\pgfqpoint{0.501324in}{0.983576in}}%
\pgfpathlineto{\pgfqpoint{0.497861in}{0.985092in}}%
\pgfpathclose%
\pgfpathmoveto{\pgfqpoint{0.798799in}{0.968051in}}%
\pgfpathlineto{\pgfqpoint{0.814455in}{0.963329in}}%
\pgfpathlineto{\pgfqpoint{0.830112in}{0.962543in}}%
\pgfpathlineto{\pgfqpoint{0.845769in}{0.965689in}}%
\pgfpathlineto{\pgfqpoint{0.858630in}{0.971481in}}%
\pgfpathlineto{\pgfqpoint{0.861425in}{0.973003in}}%
\pgfpathlineto{\pgfqpoint{0.876139in}{0.985092in}}%
\pgfpathlineto{\pgfqpoint{0.877082in}{0.986207in}}%
\pgfpathlineto{\pgfqpoint{0.885563in}{0.998703in}}%
\pgfpathlineto{\pgfqpoint{0.890187in}{1.012314in}}%
\pgfpathlineto{\pgfqpoint{0.890187in}{1.025925in}}%
\pgfpathlineto{\pgfqpoint{0.885563in}{1.039536in}}%
\pgfpathlineto{\pgfqpoint{0.877082in}{1.052033in}}%
\pgfpathlineto{\pgfqpoint{0.876139in}{1.053148in}}%
\pgfpathlineto{\pgfqpoint{0.861425in}{1.065237in}}%
\pgfpathlineto{\pgfqpoint{0.858630in}{1.066759in}}%
\pgfpathlineto{\pgfqpoint{0.845769in}{1.072551in}}%
\pgfpathlineto{\pgfqpoint{0.830112in}{1.075696in}}%
\pgfpathlineto{\pgfqpoint{0.814455in}{1.074910in}}%
\pgfpathlineto{\pgfqpoint{0.798799in}{1.070188in}}%
\pgfpathlineto{\pgfqpoint{0.792460in}{1.066759in}}%
\pgfpathlineto{\pgfqpoint{0.783142in}{1.060623in}}%
\pgfpathlineto{\pgfqpoint{0.774971in}{1.053148in}}%
\pgfpathlineto{\pgfqpoint{0.767486in}{1.043231in}}%
\pgfpathlineto{\pgfqpoint{0.765129in}{1.039536in}}%
\pgfpathlineto{\pgfqpoint{0.760682in}{1.025925in}}%
\pgfpathlineto{\pgfqpoint{0.760682in}{1.012314in}}%
\pgfpathlineto{\pgfqpoint{0.765129in}{0.998703in}}%
\pgfpathlineto{\pgfqpoint{0.767486in}{0.995009in}}%
\pgfpathlineto{\pgfqpoint{0.774971in}{0.985092in}}%
\pgfpathlineto{\pgfqpoint{0.783142in}{0.977616in}}%
\pgfpathlineto{\pgfqpoint{0.792460in}{0.971481in}}%
\pgfpathlineto{\pgfqpoint{0.798799in}{0.968051in}}%
\pgfpathclose%
\pgfpathmoveto{\pgfqpoint{0.808326in}{0.985092in}}%
\pgfpathlineto{\pgfqpoint{0.798799in}{0.990135in}}%
\pgfpathlineto{\pgfqpoint{0.789758in}{0.998703in}}%
\pgfpathlineto{\pgfqpoint{0.783142in}{1.011080in}}%
\pgfpathlineto{\pgfqpoint{0.782652in}{1.012314in}}%
\pgfpathlineto{\pgfqpoint{0.782652in}{1.025925in}}%
\pgfpathlineto{\pgfqpoint{0.783142in}{1.027159in}}%
\pgfpathlineto{\pgfqpoint{0.789758in}{1.039536in}}%
\pgfpathlineto{\pgfqpoint{0.798799in}{1.048105in}}%
\pgfpathlineto{\pgfqpoint{0.808326in}{1.053148in}}%
\pgfpathlineto{\pgfqpoint{0.814455in}{1.055460in}}%
\pgfpathlineto{\pgfqpoint{0.830112in}{1.056457in}}%
\pgfpathlineto{\pgfqpoint{0.843135in}{1.053148in}}%
\pgfpathlineto{\pgfqpoint{0.845769in}{1.052216in}}%
\pgfpathlineto{\pgfqpoint{0.861425in}{1.039876in}}%
\pgfpathlineto{\pgfqpoint{0.861715in}{1.039536in}}%
\pgfpathlineto{\pgfqpoint{0.867735in}{1.025925in}}%
\pgfpathlineto{\pgfqpoint{0.867735in}{1.012314in}}%
\pgfpathlineto{\pgfqpoint{0.861715in}{0.998703in}}%
\pgfpathlineto{\pgfqpoint{0.861425in}{0.998364in}}%
\pgfpathlineto{\pgfqpoint{0.845769in}{0.986024in}}%
\pgfpathlineto{\pgfqpoint{0.843135in}{0.985092in}}%
\pgfpathlineto{\pgfqpoint{0.830112in}{0.981783in}}%
\pgfpathlineto{\pgfqpoint{0.814455in}{0.982779in}}%
\pgfpathlineto{\pgfqpoint{0.808326in}{0.985092in}}%
\pgfpathclose%
\pgfpathmoveto{\pgfqpoint{1.111930in}{0.966791in}}%
\pgfpathlineto{\pgfqpoint{1.127587in}{0.962857in}}%
\pgfpathlineto{\pgfqpoint{1.143243in}{0.962857in}}%
\pgfpathlineto{\pgfqpoint{1.158900in}{0.966791in}}%
\pgfpathlineto{\pgfqpoint{1.168349in}{0.971481in}}%
\pgfpathlineto{\pgfqpoint{1.174556in}{0.975218in}}%
\pgfpathlineto{\pgfqpoint{1.185915in}{0.985092in}}%
\pgfpathlineto{\pgfqpoint{1.190213in}{0.990488in}}%
\pgfpathlineto{\pgfqpoint{1.195608in}{0.998703in}}%
\pgfpathlineto{\pgfqpoint{1.200132in}{1.012314in}}%
\pgfpathlineto{\pgfqpoint{1.200132in}{1.025925in}}%
\pgfpathlineto{\pgfqpoint{1.195608in}{1.039536in}}%
\pgfpathlineto{\pgfqpoint{1.190213in}{1.047751in}}%
\pgfpathlineto{\pgfqpoint{1.185915in}{1.053148in}}%
\pgfpathlineto{\pgfqpoint{1.174556in}{1.063022in}}%
\pgfpathlineto{\pgfqpoint{1.168349in}{1.066759in}}%
\pgfpathlineto{\pgfqpoint{1.158900in}{1.071448in}}%
\pgfpathlineto{\pgfqpoint{1.143243in}{1.075382in}}%
\pgfpathlineto{\pgfqpoint{1.127587in}{1.075382in}}%
\pgfpathlineto{\pgfqpoint{1.111930in}{1.071448in}}%
\pgfpathlineto{\pgfqpoint{1.102481in}{1.066759in}}%
\pgfpathlineto{\pgfqpoint{1.096274in}{1.063022in}}%
\pgfpathlineto{\pgfqpoint{1.084915in}{1.053148in}}%
\pgfpathlineto{\pgfqpoint{1.080617in}{1.047751in}}%
\pgfpathlineto{\pgfqpoint{1.075222in}{1.039536in}}%
\pgfpathlineto{\pgfqpoint{1.070698in}{1.025925in}}%
\pgfpathlineto{\pgfqpoint{1.070698in}{1.012314in}}%
\pgfpathlineto{\pgfqpoint{1.075222in}{0.998703in}}%
\pgfpathlineto{\pgfqpoint{1.080617in}{0.990488in}}%
\pgfpathlineto{\pgfqpoint{1.084915in}{0.985092in}}%
\pgfpathlineto{\pgfqpoint{1.096274in}{0.975218in}}%
\pgfpathlineto{\pgfqpoint{1.102481in}{0.971481in}}%
\pgfpathlineto{\pgfqpoint{1.111930in}{0.966791in}}%
\pgfpathclose%
\pgfpathmoveto{\pgfqpoint{1.118380in}{0.985092in}}%
\pgfpathlineto{\pgfqpoint{1.111930in}{0.987942in}}%
\pgfpathlineto{\pgfqpoint{1.099552in}{0.998703in}}%
\pgfpathlineto{\pgfqpoint{1.096274in}{1.004310in}}%
\pgfpathlineto{\pgfqpoint{1.092925in}{1.012314in}}%
\pgfpathlineto{\pgfqpoint{1.092925in}{1.025925in}}%
\pgfpathlineto{\pgfqpoint{1.096274in}{1.033929in}}%
\pgfpathlineto{\pgfqpoint{1.099552in}{1.039536in}}%
\pgfpathlineto{\pgfqpoint{1.111930in}{1.050298in}}%
\pgfpathlineto{\pgfqpoint{1.118380in}{1.053148in}}%
\pgfpathlineto{\pgfqpoint{1.127587in}{1.056058in}}%
\pgfpathlineto{\pgfqpoint{1.143243in}{1.056058in}}%
\pgfpathlineto{\pgfqpoint{1.152450in}{1.053148in}}%
\pgfpathlineto{\pgfqpoint{1.158900in}{1.050298in}}%
\pgfpathlineto{\pgfqpoint{1.171278in}{1.039536in}}%
\pgfpathlineto{\pgfqpoint{1.174556in}{1.033929in}}%
\pgfpathlineto{\pgfqpoint{1.177905in}{1.025925in}}%
\pgfpathlineto{\pgfqpoint{1.177905in}{1.012314in}}%
\pgfpathlineto{\pgfqpoint{1.174556in}{1.004310in}}%
\pgfpathlineto{\pgfqpoint{1.171278in}{0.998703in}}%
\pgfpathlineto{\pgfqpoint{1.158900in}{0.987942in}}%
\pgfpathlineto{\pgfqpoint{1.152450in}{0.985092in}}%
\pgfpathlineto{\pgfqpoint{1.143243in}{0.982181in}}%
\pgfpathlineto{\pgfqpoint{1.127587in}{0.982181in}}%
\pgfpathlineto{\pgfqpoint{1.118380in}{0.985092in}}%
\pgfpathclose%
\pgfpathmoveto{\pgfqpoint{1.425061in}{0.965689in}}%
\pgfpathlineto{\pgfqpoint{1.440718in}{0.962543in}}%
\pgfpathlineto{\pgfqpoint{1.456375in}{0.963329in}}%
\pgfpathlineto{\pgfqpoint{1.472031in}{0.968051in}}%
\pgfpathlineto{\pgfqpoint{1.478370in}{0.971481in}}%
\pgfpathlineto{\pgfqpoint{1.487688in}{0.977616in}}%
\pgfpathlineto{\pgfqpoint{1.495859in}{0.985092in}}%
\pgfpathlineto{\pgfqpoint{1.503344in}{0.995009in}}%
\pgfpathlineto{\pgfqpoint{1.505701in}{0.998703in}}%
\pgfpathlineto{\pgfqpoint{1.510148in}{1.012314in}}%
\pgfpathlineto{\pgfqpoint{1.510148in}{1.025925in}}%
\pgfpathlineto{\pgfqpoint{1.505701in}{1.039536in}}%
\pgfpathlineto{\pgfqpoint{1.503344in}{1.043231in}}%
\pgfpathlineto{\pgfqpoint{1.495859in}{1.053148in}}%
\pgfpathlineto{\pgfqpoint{1.487688in}{1.060623in}}%
\pgfpathlineto{\pgfqpoint{1.478370in}{1.066759in}}%
\pgfpathlineto{\pgfqpoint{1.472031in}{1.070188in}}%
\pgfpathlineto{\pgfqpoint{1.456375in}{1.074910in}}%
\pgfpathlineto{\pgfqpoint{1.440718in}{1.075696in}}%
\pgfpathlineto{\pgfqpoint{1.425061in}{1.072551in}}%
\pgfpathlineto{\pgfqpoint{1.412200in}{1.066759in}}%
\pgfpathlineto{\pgfqpoint{1.409405in}{1.065237in}}%
\pgfpathlineto{\pgfqpoint{1.394691in}{1.053148in}}%
\pgfpathlineto{\pgfqpoint{1.393748in}{1.052033in}}%
\pgfpathlineto{\pgfqpoint{1.385267in}{1.039536in}}%
\pgfpathlineto{\pgfqpoint{1.380643in}{1.025925in}}%
\pgfpathlineto{\pgfqpoint{1.380643in}{1.012314in}}%
\pgfpathlineto{\pgfqpoint{1.385267in}{0.998703in}}%
\pgfpathlineto{\pgfqpoint{1.393748in}{0.986207in}}%
\pgfpathlineto{\pgfqpoint{1.394691in}{0.985092in}}%
\pgfpathlineto{\pgfqpoint{1.409405in}{0.973003in}}%
\pgfpathlineto{\pgfqpoint{1.412200in}{0.971481in}}%
\pgfpathlineto{\pgfqpoint{1.425061in}{0.965689in}}%
\pgfpathclose%
\pgfpathmoveto{\pgfqpoint{1.427695in}{0.985092in}}%
\pgfpathlineto{\pgfqpoint{1.425061in}{0.986024in}}%
\pgfpathlineto{\pgfqpoint{1.409405in}{0.998364in}}%
\pgfpathlineto{\pgfqpoint{1.409115in}{0.998703in}}%
\pgfpathlineto{\pgfqpoint{1.403095in}{1.012314in}}%
\pgfpathlineto{\pgfqpoint{1.403095in}{1.025925in}}%
\pgfpathlineto{\pgfqpoint{1.409115in}{1.039536in}}%
\pgfpathlineto{\pgfqpoint{1.409405in}{1.039876in}}%
\pgfpathlineto{\pgfqpoint{1.425061in}{1.052216in}}%
\pgfpathlineto{\pgfqpoint{1.427695in}{1.053148in}}%
\pgfpathlineto{\pgfqpoint{1.440718in}{1.056457in}}%
\pgfpathlineto{\pgfqpoint{1.456375in}{1.055460in}}%
\pgfpathlineto{\pgfqpoint{1.462504in}{1.053148in}}%
\pgfpathlineto{\pgfqpoint{1.472031in}{1.048105in}}%
\pgfpathlineto{\pgfqpoint{1.481072in}{1.039536in}}%
\pgfpathlineto{\pgfqpoint{1.487688in}{1.027159in}}%
\pgfpathlineto{\pgfqpoint{1.488178in}{1.025925in}}%
\pgfpathlineto{\pgfqpoint{1.488178in}{1.012314in}}%
\pgfpathlineto{\pgfqpoint{1.487688in}{1.011080in}}%
\pgfpathlineto{\pgfqpoint{1.481072in}{0.998703in}}%
\pgfpathlineto{\pgfqpoint{1.472031in}{0.990135in}}%
\pgfpathlineto{\pgfqpoint{1.462504in}{0.985092in}}%
\pgfpathlineto{\pgfqpoint{1.456375in}{0.982779in}}%
\pgfpathlineto{\pgfqpoint{1.440718in}{0.981783in}}%
\pgfpathlineto{\pgfqpoint{1.427695in}{0.985092in}}%
\pgfpathclose%
\pgfpathmoveto{\pgfqpoint{1.722536in}{0.971046in}}%
\pgfpathlineto{\pgfqpoint{1.738193in}{0.964745in}}%
\pgfpathlineto{\pgfqpoint{1.753849in}{0.962386in}}%
\pgfpathlineto{\pgfqpoint{1.769506in}{0.963958in}}%
\pgfpathlineto{\pgfqpoint{1.785162in}{0.969470in}}%
\pgfpathlineto{\pgfqpoint{1.788603in}{0.971481in}}%
\pgfpathlineto{\pgfqpoint{1.800819in}{0.980198in}}%
\pgfpathlineto{\pgfqpoint{1.805928in}{0.985092in}}%
\pgfpathlineto{\pgfqpoint{1.815741in}{0.998703in}}%
\pgfpathlineto{\pgfqpoint{1.816476in}{1.000714in}}%
\pgfpathlineto{\pgfqpoint{1.820207in}{1.012314in}}%
\pgfpathlineto{\pgfqpoint{1.820207in}{1.025925in}}%
\pgfpathlineto{\pgfqpoint{1.816476in}{1.037525in}}%
\pgfpathlineto{\pgfqpoint{1.815741in}{1.039536in}}%
\pgfpathlineto{\pgfqpoint{1.805928in}{1.053148in}}%
\pgfpathlineto{\pgfqpoint{1.800819in}{1.058041in}}%
\pgfpathlineto{\pgfqpoint{1.788603in}{1.066759in}}%
\pgfpathlineto{\pgfqpoint{1.785162in}{1.068770in}}%
\pgfpathlineto{\pgfqpoint{1.769506in}{1.074281in}}%
\pgfpathlineto{\pgfqpoint{1.753849in}{1.075853in}}%
\pgfpathlineto{\pgfqpoint{1.738193in}{1.073495in}}%
\pgfpathlineto{\pgfqpoint{1.722536in}{1.067193in}}%
\pgfpathlineto{\pgfqpoint{1.721843in}{1.066759in}}%
\pgfpathlineto{\pgfqpoint{1.706880in}{1.055278in}}%
\pgfpathlineto{\pgfqpoint{1.704746in}{1.053148in}}%
\pgfpathlineto{\pgfqpoint{1.695234in}{1.039536in}}%
\pgfpathlineto{\pgfqpoint{1.691223in}{1.028069in}}%
\pgfpathlineto{\pgfqpoint{1.690548in}{1.025925in}}%
\pgfpathlineto{\pgfqpoint{1.690548in}{1.012314in}}%
\pgfpathlineto{\pgfqpoint{1.691223in}{1.010171in}}%
\pgfpathlineto{\pgfqpoint{1.695234in}{0.998703in}}%
\pgfpathlineto{\pgfqpoint{1.704746in}{0.985092in}}%
\pgfpathlineto{\pgfqpoint{1.706880in}{0.982961in}}%
\pgfpathlineto{\pgfqpoint{1.721843in}{0.971481in}}%
\pgfpathlineto{\pgfqpoint{1.722536in}{0.971046in}}%
\pgfpathclose%
\pgfpathmoveto{\pgfqpoint{1.737150in}{0.985092in}}%
\pgfpathlineto{\pgfqpoint{1.722536in}{0.995346in}}%
\pgfpathlineto{\pgfqpoint{1.719477in}{0.998703in}}%
\pgfpathlineto{\pgfqpoint{1.713118in}{1.012314in}}%
\pgfpathlineto{\pgfqpoint{1.713118in}{1.025925in}}%
\pgfpathlineto{\pgfqpoint{1.719477in}{1.039536in}}%
\pgfpathlineto{\pgfqpoint{1.722536in}{1.042894in}}%
\pgfpathlineto{\pgfqpoint{1.737150in}{1.053148in}}%
\pgfpathlineto{\pgfqpoint{1.738193in}{1.053666in}}%
\pgfpathlineto{\pgfqpoint{1.753849in}{1.056656in}}%
\pgfpathlineto{\pgfqpoint{1.769506in}{1.054663in}}%
\pgfpathlineto{\pgfqpoint{1.772969in}{1.053148in}}%
\pgfpathlineto{\pgfqpoint{1.785162in}{1.045637in}}%
\pgfpathlineto{\pgfqpoint{1.791121in}{1.039536in}}%
\pgfpathlineto{\pgfqpoint{1.797888in}{1.025925in}}%
\pgfpathlineto{\pgfqpoint{1.797888in}{1.012314in}}%
\pgfpathlineto{\pgfqpoint{1.791121in}{0.998703in}}%
\pgfpathlineto{\pgfqpoint{1.785162in}{0.992603in}}%
\pgfpathlineto{\pgfqpoint{1.772969in}{0.985092in}}%
\pgfpathlineto{\pgfqpoint{1.769506in}{0.983576in}}%
\pgfpathlineto{\pgfqpoint{1.753849in}{0.981583in}}%
\pgfpathlineto{\pgfqpoint{1.738193in}{0.984574in}}%
\pgfpathlineto{\pgfqpoint{1.737150in}{0.985092in}}%
\pgfpathclose%
\pgfpathmoveto{\pgfqpoint{0.485668in}{1.239067in}}%
\pgfpathlineto{\pgfqpoint{0.501324in}{1.233435in}}%
\pgfpathlineto{\pgfqpoint{0.516981in}{1.231829in}}%
\pgfpathlineto{\pgfqpoint{0.532637in}{1.234239in}}%
\pgfpathlineto{\pgfqpoint{0.548294in}{1.240678in}}%
\pgfpathlineto{\pgfqpoint{0.552965in}{1.243703in}}%
\pgfpathlineto{\pgfqpoint{0.563950in}{1.252647in}}%
\pgfpathlineto{\pgfqpoint{0.568368in}{1.257314in}}%
\pgfpathlineto{\pgfqpoint{0.576926in}{1.270925in}}%
\pgfpathlineto{\pgfqpoint{0.579607in}{1.280496in}}%
\pgfpathlineto{\pgfqpoint{0.580629in}{1.284536in}}%
\pgfpathlineto{\pgfqpoint{0.579760in}{1.298147in}}%
\pgfpathlineto{\pgfqpoint{0.579607in}{1.298555in}}%
\pgfpathlineto{\pgfqpoint{0.574075in}{1.311759in}}%
\pgfpathlineto{\pgfqpoint{0.563950in}{1.324945in}}%
\pgfpathlineto{\pgfqpoint{0.563531in}{1.325370in}}%
\pgfpathlineto{\pgfqpoint{0.548294in}{1.336544in}}%
\pgfpathlineto{\pgfqpoint{0.543002in}{1.338981in}}%
\pgfpathlineto{\pgfqpoint{0.532637in}{1.343041in}}%
\pgfpathlineto{\pgfqpoint{0.516981in}{1.345359in}}%
\pgfpathlineto{\pgfqpoint{0.501324in}{1.343814in}}%
\pgfpathlineto{\pgfqpoint{0.487318in}{1.338981in}}%
\pgfpathlineto{\pgfqpoint{0.485668in}{1.338314in}}%
\pgfpathlineto{\pgfqpoint{0.470011in}{1.327698in}}%
\pgfpathlineto{\pgfqpoint{0.467454in}{1.325370in}}%
\pgfpathlineto{\pgfqpoint{0.456658in}{1.311759in}}%
\pgfpathlineto{\pgfqpoint{0.454354in}{1.306495in}}%
\pgfpathlineto{\pgfqpoint{0.451149in}{1.298147in}}%
\pgfpathlineto{\pgfqpoint{0.450272in}{1.284536in}}%
\pgfpathlineto{\pgfqpoint{0.453783in}{1.270925in}}%
\pgfpathlineto{\pgfqpoint{0.454354in}{1.269913in}}%
\pgfpathlineto{\pgfqpoint{0.462547in}{1.257314in}}%
\pgfpathlineto{\pgfqpoint{0.470011in}{1.249748in}}%
\pgfpathlineto{\pgfqpoint{0.477994in}{1.243703in}}%
\pgfpathlineto{\pgfqpoint{0.485668in}{1.239067in}}%
\pgfpathclose%
\pgfpathmoveto{\pgfqpoint{0.492639in}{1.257314in}}%
\pgfpathlineto{\pgfqpoint{0.485668in}{1.262063in}}%
\pgfpathlineto{\pgfqpoint{0.477813in}{1.270925in}}%
\pgfpathlineto{\pgfqpoint{0.472401in}{1.284536in}}%
\pgfpathlineto{\pgfqpoint{0.473753in}{1.298148in}}%
\pgfpathlineto{\pgfqpoint{0.481877in}{1.311759in}}%
\pgfpathlineto{\pgfqpoint{0.485668in}{1.315318in}}%
\pgfpathlineto{\pgfqpoint{0.501324in}{1.324166in}}%
\pgfpathlineto{\pgfqpoint{0.508764in}{1.325370in}}%
\pgfpathlineto{\pgfqpoint{0.516981in}{1.326369in}}%
\pgfpathlineto{\pgfqpoint{0.522477in}{1.325370in}}%
\pgfpathlineto{\pgfqpoint{0.532637in}{1.322903in}}%
\pgfpathlineto{\pgfqpoint{0.548294in}{1.312787in}}%
\pgfpathlineto{\pgfqpoint{0.549316in}{1.311759in}}%
\pgfpathlineto{\pgfqpoint{0.556949in}{1.298147in}}%
\pgfpathlineto{\pgfqpoint{0.558220in}{1.284536in}}%
\pgfpathlineto{\pgfqpoint{0.553135in}{1.270925in}}%
\pgfpathlineto{\pgfqpoint{0.548294in}{1.265070in}}%
\pgfpathlineto{\pgfqpoint{0.538297in}{1.257314in}}%
\pgfpathlineto{\pgfqpoint{0.532637in}{1.254339in}}%
\pgfpathlineto{\pgfqpoint{0.516981in}{1.251201in}}%
\pgfpathlineto{\pgfqpoint{0.501324in}{1.253293in}}%
\pgfpathlineto{\pgfqpoint{0.492639in}{1.257314in}}%
\pgfpathclose%
\pgfpathmoveto{\pgfqpoint{0.798799in}{1.237618in}}%
\pgfpathlineto{\pgfqpoint{0.814455in}{1.232792in}}%
\pgfpathlineto{\pgfqpoint{0.830112in}{1.231989in}}%
\pgfpathlineto{\pgfqpoint{0.845769in}{1.235204in}}%
\pgfpathlineto{\pgfqpoint{0.861425in}{1.242451in}}%
\pgfpathlineto{\pgfqpoint{0.863241in}{1.243703in}}%
\pgfpathlineto{\pgfqpoint{0.877082in}{1.255735in}}%
\pgfpathlineto{\pgfqpoint{0.878522in}{1.257314in}}%
\pgfpathlineto{\pgfqpoint{0.886858in}{1.270925in}}%
\pgfpathlineto{\pgfqpoint{0.890556in}{1.284536in}}%
\pgfpathlineto{\pgfqpoint{0.889632in}{1.298147in}}%
\pgfpathlineto{\pgfqpoint{0.884081in}{1.311759in}}%
\pgfpathlineto{\pgfqpoint{0.877082in}{1.321219in}}%
\pgfpathlineto{\pgfqpoint{0.873244in}{1.325370in}}%
\pgfpathlineto{\pgfqpoint{0.861425in}{1.334597in}}%
\pgfpathlineto{\pgfqpoint{0.852929in}{1.338981in}}%
\pgfpathlineto{\pgfqpoint{0.845769in}{1.342113in}}%
\pgfpathlineto{\pgfqpoint{0.830112in}{1.345205in}}%
\pgfpathlineto{\pgfqpoint{0.814455in}{1.344433in}}%
\pgfpathlineto{\pgfqpoint{0.798799in}{1.339791in}}%
\pgfpathlineto{\pgfqpoint{0.797258in}{1.338981in}}%
\pgfpathlineto{\pgfqpoint{0.783142in}{1.330173in}}%
\pgfpathlineto{\pgfqpoint{0.777617in}{1.325370in}}%
\pgfpathlineto{\pgfqpoint{0.767486in}{1.313098in}}%
\pgfpathlineto{\pgfqpoint{0.766554in}{1.311759in}}%
\pgfpathlineto{\pgfqpoint{0.761215in}{1.298147in}}%
\pgfpathlineto{\pgfqpoint{0.760326in}{1.284536in}}%
\pgfpathlineto{\pgfqpoint{0.763883in}{1.270925in}}%
\pgfpathlineto{\pgfqpoint{0.767486in}{1.264701in}}%
\pgfpathlineto{\pgfqpoint{0.772528in}{1.257314in}}%
\pgfpathlineto{\pgfqpoint{0.783142in}{1.247039in}}%
\pgfpathlineto{\pgfqpoint{0.787917in}{1.243703in}}%
\pgfpathlineto{\pgfqpoint{0.798799in}{1.237618in}}%
\pgfpathclose%
\pgfpathmoveto{\pgfqpoint{0.802289in}{1.257314in}}%
\pgfpathlineto{\pgfqpoint{0.798799in}{1.259357in}}%
\pgfpathlineto{\pgfqpoint{0.787723in}{1.270925in}}%
\pgfpathlineto{\pgfqpoint{0.783142in}{1.281626in}}%
\pgfpathlineto{\pgfqpoint{0.782212in}{1.284536in}}%
\pgfpathlineto{\pgfqpoint{0.783142in}{1.296060in}}%
\pgfpathlineto{\pgfqpoint{0.783366in}{1.298147in}}%
\pgfpathlineto{\pgfqpoint{0.792085in}{1.311759in}}%
\pgfpathlineto{\pgfqpoint{0.798799in}{1.317595in}}%
\pgfpathlineto{\pgfqpoint{0.814455in}{1.325176in}}%
\pgfpathlineto{\pgfqpoint{0.816857in}{1.325370in}}%
\pgfpathlineto{\pgfqpoint{0.830112in}{1.326178in}}%
\pgfpathlineto{\pgfqpoint{0.833460in}{1.325370in}}%
\pgfpathlineto{\pgfqpoint{0.845769in}{1.321388in}}%
\pgfpathlineto{\pgfqpoint{0.859076in}{1.311759in}}%
\pgfpathlineto{\pgfqpoint{0.861425in}{1.308725in}}%
\pgfpathlineto{\pgfqpoint{0.867014in}{1.298147in}}%
\pgfpathlineto{\pgfqpoint{0.868216in}{1.284536in}}%
\pgfpathlineto{\pgfqpoint{0.863402in}{1.270925in}}%
\pgfpathlineto{\pgfqpoint{0.861425in}{1.268380in}}%
\pgfpathlineto{\pgfqpoint{0.848697in}{1.257314in}}%
\pgfpathlineto{\pgfqpoint{0.845769in}{1.255595in}}%
\pgfpathlineto{\pgfqpoint{0.830112in}{1.251410in}}%
\pgfpathlineto{\pgfqpoint{0.814455in}{1.252456in}}%
\pgfpathlineto{\pgfqpoint{0.802289in}{1.257314in}}%
\pgfpathclose%
\pgfpathmoveto{\pgfqpoint{1.111930in}{1.236330in}}%
\pgfpathlineto{\pgfqpoint{1.127587in}{1.232310in}}%
\pgfpathlineto{\pgfqpoint{1.143243in}{1.232310in}}%
\pgfpathlineto{\pgfqpoint{1.158900in}{1.236330in}}%
\pgfpathlineto{\pgfqpoint{1.173274in}{1.243703in}}%
\pgfpathlineto{\pgfqpoint{1.174556in}{1.244522in}}%
\pgfpathlineto{\pgfqpoint{1.188462in}{1.257314in}}%
\pgfpathlineto{\pgfqpoint{1.190213in}{1.259744in}}%
\pgfpathlineto{\pgfqpoint{1.196875in}{1.270925in}}%
\pgfpathlineto{\pgfqpoint{1.200494in}{1.284536in}}%
\pgfpathlineto{\pgfqpoint{1.199590in}{1.298147in}}%
\pgfpathlineto{\pgfqpoint{1.194158in}{1.311759in}}%
\pgfpathlineto{\pgfqpoint{1.190213in}{1.317269in}}%
\pgfpathlineto{\pgfqpoint{1.183155in}{1.325370in}}%
\pgfpathlineto{\pgfqpoint{1.174556in}{1.332473in}}%
\pgfpathlineto{\pgfqpoint{1.163150in}{1.338981in}}%
\pgfpathlineto{\pgfqpoint{1.158900in}{1.341030in}}%
\pgfpathlineto{\pgfqpoint{1.143243in}{1.344896in}}%
\pgfpathlineto{\pgfqpoint{1.127587in}{1.344896in}}%
\pgfpathlineto{\pgfqpoint{1.111930in}{1.341030in}}%
\pgfpathlineto{\pgfqpoint{1.107680in}{1.338981in}}%
\pgfpathlineto{\pgfqpoint{1.096274in}{1.332473in}}%
\pgfpathlineto{\pgfqpoint{1.087675in}{1.325370in}}%
\pgfpathlineto{\pgfqpoint{1.080617in}{1.317269in}}%
\pgfpathlineto{\pgfqpoint{1.076672in}{1.311759in}}%
\pgfpathlineto{\pgfqpoint{1.071240in}{1.298147in}}%
\pgfpathlineto{\pgfqpoint{1.070336in}{1.284536in}}%
\pgfpathlineto{\pgfqpoint{1.073955in}{1.270925in}}%
\pgfpathlineto{\pgfqpoint{1.080617in}{1.259744in}}%
\pgfpathlineto{\pgfqpoint{1.082368in}{1.257314in}}%
\pgfpathlineto{\pgfqpoint{1.096274in}{1.244522in}}%
\pgfpathlineto{\pgfqpoint{1.097556in}{1.243703in}}%
\pgfpathlineto{\pgfqpoint{1.111930in}{1.236330in}}%
\pgfpathclose%
\pgfpathmoveto{\pgfqpoint{1.111540in}{1.257314in}}%
\pgfpathlineto{\pgfqpoint{1.097346in}{1.270925in}}%
\pgfpathlineto{\pgfqpoint{1.096274in}{1.273215in}}%
\pgfpathlineto{\pgfqpoint{1.092467in}{1.284536in}}%
\pgfpathlineto{\pgfqpoint{1.093613in}{1.298147in}}%
\pgfpathlineto{\pgfqpoint{1.096274in}{1.303476in}}%
\pgfpathlineto{\pgfqpoint{1.102074in}{1.311759in}}%
\pgfpathlineto{\pgfqpoint{1.111930in}{1.319618in}}%
\pgfpathlineto{\pgfqpoint{1.126168in}{1.325370in}}%
\pgfpathlineto{\pgfqpoint{1.127587in}{1.325796in}}%
\pgfpathlineto{\pgfqpoint{1.143243in}{1.325796in}}%
\pgfpathlineto{\pgfqpoint{1.144662in}{1.325370in}}%
\pgfpathlineto{\pgfqpoint{1.158900in}{1.319618in}}%
\pgfpathlineto{\pgfqpoint{1.168756in}{1.311759in}}%
\pgfpathlineto{\pgfqpoint{1.174556in}{1.303476in}}%
\pgfpathlineto{\pgfqpoint{1.177217in}{1.298147in}}%
\pgfpathlineto{\pgfqpoint{1.178363in}{1.284536in}}%
\pgfpathlineto{\pgfqpoint{1.174556in}{1.273215in}}%
\pgfpathlineto{\pgfqpoint{1.173484in}{1.270925in}}%
\pgfpathlineto{\pgfqpoint{1.159290in}{1.257314in}}%
\pgfpathlineto{\pgfqpoint{1.158900in}{1.257062in}}%
\pgfpathlineto{\pgfqpoint{1.143243in}{1.251828in}}%
\pgfpathlineto{\pgfqpoint{1.127587in}{1.251828in}}%
\pgfpathlineto{\pgfqpoint{1.111930in}{1.257062in}}%
\pgfpathlineto{\pgfqpoint{1.111540in}{1.257314in}}%
\pgfpathclose%
\pgfpathmoveto{\pgfqpoint{1.409405in}{1.242451in}}%
\pgfpathlineto{\pgfqpoint{1.425061in}{1.235204in}}%
\pgfpathlineto{\pgfqpoint{1.440718in}{1.231989in}}%
\pgfpathlineto{\pgfqpoint{1.456375in}{1.232792in}}%
\pgfpathlineto{\pgfqpoint{1.472031in}{1.237618in}}%
\pgfpathlineto{\pgfqpoint{1.482913in}{1.243703in}}%
\pgfpathlineto{\pgfqpoint{1.487688in}{1.247039in}}%
\pgfpathlineto{\pgfqpoint{1.498302in}{1.257314in}}%
\pgfpathlineto{\pgfqpoint{1.503344in}{1.264701in}}%
\pgfpathlineto{\pgfqpoint{1.506947in}{1.270925in}}%
\pgfpathlineto{\pgfqpoint{1.510504in}{1.284536in}}%
\pgfpathlineto{\pgfqpoint{1.509615in}{1.298147in}}%
\pgfpathlineto{\pgfqpoint{1.504276in}{1.311759in}}%
\pgfpathlineto{\pgfqpoint{1.503344in}{1.313098in}}%
\pgfpathlineto{\pgfqpoint{1.493213in}{1.325370in}}%
\pgfpathlineto{\pgfqpoint{1.487688in}{1.330173in}}%
\pgfpathlineto{\pgfqpoint{1.473572in}{1.338981in}}%
\pgfpathlineto{\pgfqpoint{1.472031in}{1.339791in}}%
\pgfpathlineto{\pgfqpoint{1.456375in}{1.344433in}}%
\pgfpathlineto{\pgfqpoint{1.440718in}{1.345205in}}%
\pgfpathlineto{\pgfqpoint{1.425061in}{1.342113in}}%
\pgfpathlineto{\pgfqpoint{1.417901in}{1.338981in}}%
\pgfpathlineto{\pgfqpoint{1.409405in}{1.334597in}}%
\pgfpathlineto{\pgfqpoint{1.397586in}{1.325370in}}%
\pgfpathlineto{\pgfqpoint{1.393748in}{1.321219in}}%
\pgfpathlineto{\pgfqpoint{1.386749in}{1.311759in}}%
\pgfpathlineto{\pgfqpoint{1.381198in}{1.298147in}}%
\pgfpathlineto{\pgfqpoint{1.380274in}{1.284536in}}%
\pgfpathlineto{\pgfqpoint{1.383972in}{1.270925in}}%
\pgfpathlineto{\pgfqpoint{1.392308in}{1.257314in}}%
\pgfpathlineto{\pgfqpoint{1.393748in}{1.255735in}}%
\pgfpathlineto{\pgfqpoint{1.407589in}{1.243703in}}%
\pgfpathlineto{\pgfqpoint{1.409405in}{1.242451in}}%
\pgfpathclose%
\pgfpathmoveto{\pgfqpoint{1.422133in}{1.257314in}}%
\pgfpathlineto{\pgfqpoint{1.409405in}{1.268380in}}%
\pgfpathlineto{\pgfqpoint{1.407428in}{1.270925in}}%
\pgfpathlineto{\pgfqpoint{1.402614in}{1.284536in}}%
\pgfpathlineto{\pgfqpoint{1.403816in}{1.298147in}}%
\pgfpathlineto{\pgfqpoint{1.409405in}{1.308725in}}%
\pgfpathlineto{\pgfqpoint{1.411754in}{1.311759in}}%
\pgfpathlineto{\pgfqpoint{1.425061in}{1.321388in}}%
\pgfpathlineto{\pgfqpoint{1.437370in}{1.325370in}}%
\pgfpathlineto{\pgfqpoint{1.440718in}{1.326178in}}%
\pgfpathlineto{\pgfqpoint{1.453973in}{1.325370in}}%
\pgfpathlineto{\pgfqpoint{1.456375in}{1.325176in}}%
\pgfpathlineto{\pgfqpoint{1.472031in}{1.317595in}}%
\pgfpathlineto{\pgfqpoint{1.478745in}{1.311759in}}%
\pgfpathlineto{\pgfqpoint{1.487464in}{1.298147in}}%
\pgfpathlineto{\pgfqpoint{1.487688in}{1.296060in}}%
\pgfpathlineto{\pgfqpoint{1.488618in}{1.284536in}}%
\pgfpathlineto{\pgfqpoint{1.487688in}{1.281626in}}%
\pgfpathlineto{\pgfqpoint{1.483107in}{1.270925in}}%
\pgfpathlineto{\pgfqpoint{1.472031in}{1.259357in}}%
\pgfpathlineto{\pgfqpoint{1.468541in}{1.257314in}}%
\pgfpathlineto{\pgfqpoint{1.456375in}{1.252456in}}%
\pgfpathlineto{\pgfqpoint{1.440718in}{1.251410in}}%
\pgfpathlineto{\pgfqpoint{1.425061in}{1.255595in}}%
\pgfpathlineto{\pgfqpoint{1.422133in}{1.257314in}}%
\pgfpathclose%
\pgfpathmoveto{\pgfqpoint{1.722536in}{1.240678in}}%
\pgfpathlineto{\pgfqpoint{1.738193in}{1.234239in}}%
\pgfpathlineto{\pgfqpoint{1.753849in}{1.231829in}}%
\pgfpathlineto{\pgfqpoint{1.769506in}{1.233435in}}%
\pgfpathlineto{\pgfqpoint{1.785162in}{1.239067in}}%
\pgfpathlineto{\pgfqpoint{1.792836in}{1.243703in}}%
\pgfpathlineto{\pgfqpoint{1.800819in}{1.249748in}}%
\pgfpathlineto{\pgfqpoint{1.808283in}{1.257314in}}%
\pgfpathlineto{\pgfqpoint{1.816476in}{1.269913in}}%
\pgfpathlineto{\pgfqpoint{1.817047in}{1.270925in}}%
\pgfpathlineto{\pgfqpoint{1.820558in}{1.284536in}}%
\pgfpathlineto{\pgfqpoint{1.819681in}{1.298147in}}%
\pgfpathlineto{\pgfqpoint{1.816476in}{1.306495in}}%
\pgfpathlineto{\pgfqpoint{1.814172in}{1.311759in}}%
\pgfpathlineto{\pgfqpoint{1.803376in}{1.325370in}}%
\pgfpathlineto{\pgfqpoint{1.800819in}{1.327698in}}%
\pgfpathlineto{\pgfqpoint{1.785162in}{1.338314in}}%
\pgfpathlineto{\pgfqpoint{1.783512in}{1.338981in}}%
\pgfpathlineto{\pgfqpoint{1.769506in}{1.343814in}}%
\pgfpathlineto{\pgfqpoint{1.753849in}{1.345359in}}%
\pgfpathlineto{\pgfqpoint{1.738193in}{1.343041in}}%
\pgfpathlineto{\pgfqpoint{1.727828in}{1.338981in}}%
\pgfpathlineto{\pgfqpoint{1.722536in}{1.336544in}}%
\pgfpathlineto{\pgfqpoint{1.707299in}{1.325370in}}%
\pgfpathlineto{\pgfqpoint{1.706880in}{1.324945in}}%
\pgfpathlineto{\pgfqpoint{1.696755in}{1.311759in}}%
\pgfpathlineto{\pgfqpoint{1.691223in}{1.298555in}}%
\pgfpathlineto{\pgfqpoint{1.691070in}{1.298148in}}%
\pgfpathlineto{\pgfqpoint{1.690201in}{1.284536in}}%
\pgfpathlineto{\pgfqpoint{1.691223in}{1.280496in}}%
\pgfpathlineto{\pgfqpoint{1.693904in}{1.270925in}}%
\pgfpathlineto{\pgfqpoint{1.702462in}{1.257314in}}%
\pgfpathlineto{\pgfqpoint{1.706880in}{1.252647in}}%
\pgfpathlineto{\pgfqpoint{1.717865in}{1.243703in}}%
\pgfpathlineto{\pgfqpoint{1.722536in}{1.240678in}}%
\pgfpathclose%
\pgfpathmoveto{\pgfqpoint{1.732533in}{1.257314in}}%
\pgfpathlineto{\pgfqpoint{1.722536in}{1.265070in}}%
\pgfpathlineto{\pgfqpoint{1.717695in}{1.270925in}}%
\pgfpathlineto{\pgfqpoint{1.712610in}{1.284536in}}%
\pgfpathlineto{\pgfqpoint{1.713881in}{1.298147in}}%
\pgfpathlineto{\pgfqpoint{1.721514in}{1.311759in}}%
\pgfpathlineto{\pgfqpoint{1.722536in}{1.312787in}}%
\pgfpathlineto{\pgfqpoint{1.738193in}{1.322903in}}%
\pgfpathlineto{\pgfqpoint{1.748353in}{1.325370in}}%
\pgfpathlineto{\pgfqpoint{1.753849in}{1.326369in}}%
\pgfpathlineto{\pgfqpoint{1.762066in}{1.325370in}}%
\pgfpathlineto{\pgfqpoint{1.769506in}{1.324166in}}%
\pgfpathlineto{\pgfqpoint{1.785162in}{1.315318in}}%
\pgfpathlineto{\pgfqpoint{1.788953in}{1.311759in}}%
\pgfpathlineto{\pgfqpoint{1.797077in}{1.298147in}}%
\pgfpathlineto{\pgfqpoint{1.798429in}{1.284536in}}%
\pgfpathlineto{\pgfqpoint{1.793017in}{1.270925in}}%
\pgfpathlineto{\pgfqpoint{1.785162in}{1.262063in}}%
\pgfpathlineto{\pgfqpoint{1.778191in}{1.257314in}}%
\pgfpathlineto{\pgfqpoint{1.769506in}{1.253293in}}%
\pgfpathlineto{\pgfqpoint{1.753849in}{1.251201in}}%
\pgfpathlineto{\pgfqpoint{1.738193in}{1.254339in}}%
\pgfpathlineto{\pgfqpoint{1.732533in}{1.257314in}}%
\pgfpathclose%
\pgfpathmoveto{\pgfqpoint{0.516981in}{1.501274in}}%
\pgfpathlineto{\pgfqpoint{0.524202in}{1.502314in}}%
\pgfpathlineto{\pgfqpoint{0.532637in}{1.503654in}}%
\pgfpathlineto{\pgfqpoint{0.548294in}{1.510265in}}%
\pgfpathlineto{\pgfqpoint{0.556717in}{1.515925in}}%
\pgfpathlineto{\pgfqpoint{0.563950in}{1.522214in}}%
\pgfpathlineto{\pgfqpoint{0.570461in}{1.529536in}}%
\pgfpathlineto{\pgfqpoint{0.578066in}{1.543148in}}%
\pgfpathlineto{\pgfqpoint{0.579607in}{1.550481in}}%
\pgfpathlineto{\pgfqpoint{0.580803in}{1.556759in}}%
\pgfpathlineto{\pgfqpoint{0.579607in}{1.566143in}}%
\pgfpathlineto{\pgfqpoint{0.579015in}{1.570370in}}%
\pgfpathlineto{\pgfqpoint{0.572364in}{1.583981in}}%
\pgfpathlineto{\pgfqpoint{0.563950in}{1.594123in}}%
\pgfpathlineto{\pgfqpoint{0.560239in}{1.597592in}}%
\pgfpathlineto{\pgfqpoint{0.548294in}{1.605959in}}%
\pgfpathlineto{\pgfqpoint{0.536336in}{1.611203in}}%
\pgfpathlineto{\pgfqpoint{0.532637in}{1.612616in}}%
\pgfpathlineto{\pgfqpoint{0.516981in}{1.614904in}}%
\pgfpathlineto{\pgfqpoint{0.501324in}{1.613379in}}%
\pgfpathlineto{\pgfqpoint{0.494857in}{1.611203in}}%
\pgfpathlineto{\pgfqpoint{0.485668in}{1.607665in}}%
\pgfpathlineto{\pgfqpoint{0.470252in}{1.597592in}}%
\pgfpathlineto{\pgfqpoint{0.470011in}{1.597383in}}%
\pgfpathlineto{\pgfqpoint{0.458424in}{1.583981in}}%
\pgfpathlineto{\pgfqpoint{0.454354in}{1.575992in}}%
\pgfpathlineto{\pgfqpoint{0.451851in}{1.570370in}}%
\pgfpathlineto{\pgfqpoint{0.450097in}{1.556759in}}%
\pgfpathlineto{\pgfqpoint{0.452729in}{1.543148in}}%
\pgfpathlineto{\pgfqpoint{0.454354in}{1.539932in}}%
\pgfpathlineto{\pgfqpoint{0.460387in}{1.529536in}}%
\pgfpathlineto{\pgfqpoint{0.470011in}{1.519151in}}%
\pgfpathlineto{\pgfqpoint{0.474001in}{1.515925in}}%
\pgfpathlineto{\pgfqpoint{0.485668in}{1.508611in}}%
\pgfpathlineto{\pgfqpoint{0.501324in}{1.502829in}}%
\pgfpathlineto{\pgfqpoint{0.506186in}{1.502314in}}%
\pgfpathlineto{\pgfqpoint{0.516981in}{1.501274in}}%
\pgfpathclose%
\pgfpathmoveto{\pgfqpoint{0.487853in}{1.529536in}}%
\pgfpathlineto{\pgfqpoint{0.485668in}{1.531204in}}%
\pgfpathlineto{\pgfqpoint{0.476188in}{1.543148in}}%
\pgfpathlineto{\pgfqpoint{0.472131in}{1.556759in}}%
\pgfpathlineto{\pgfqpoint{0.474835in}{1.570370in}}%
\pgfpathlineto{\pgfqpoint{0.484318in}{1.583981in}}%
\pgfpathlineto{\pgfqpoint{0.485668in}{1.585154in}}%
\pgfpathlineto{\pgfqpoint{0.501324in}{1.593398in}}%
\pgfpathlineto{\pgfqpoint{0.516981in}{1.595749in}}%
\pgfpathlineto{\pgfqpoint{0.532637in}{1.592221in}}%
\pgfpathlineto{\pgfqpoint{0.546376in}{1.583981in}}%
\pgfpathlineto{\pgfqpoint{0.548294in}{1.582081in}}%
\pgfpathlineto{\pgfqpoint{0.555933in}{1.570370in}}%
\pgfpathlineto{\pgfqpoint{0.558474in}{1.556759in}}%
\pgfpathlineto{\pgfqpoint{0.554661in}{1.543148in}}%
\pgfpathlineto{\pgfqpoint{0.548294in}{1.534548in}}%
\pgfpathlineto{\pgfqpoint{0.542529in}{1.529536in}}%
\pgfpathlineto{\pgfqpoint{0.532637in}{1.524001in}}%
\pgfpathlineto{\pgfqpoint{0.516981in}{1.520687in}}%
\pgfpathlineto{\pgfqpoint{0.501324in}{1.522896in}}%
\pgfpathlineto{\pgfqpoint{0.487853in}{1.529536in}}%
\pgfpathclose%
\pgfpathmoveto{\pgfqpoint{0.814455in}{1.502181in}}%
\pgfpathlineto{\pgfqpoint{0.830112in}{1.501425in}}%
\pgfpathlineto{\pgfqpoint{0.834760in}{1.502314in}}%
\pgfpathlineto{\pgfqpoint{0.845769in}{1.504645in}}%
\pgfpathlineto{\pgfqpoint{0.861425in}{1.512085in}}%
\pgfpathlineto{\pgfqpoint{0.866793in}{1.515925in}}%
\pgfpathlineto{\pgfqpoint{0.877082in}{1.525476in}}%
\pgfpathlineto{\pgfqpoint{0.880561in}{1.529536in}}%
\pgfpathlineto{\pgfqpoint{0.887968in}{1.543147in}}%
\pgfpathlineto{\pgfqpoint{0.890741in}{1.556759in}}%
\pgfpathlineto{\pgfqpoint{0.888893in}{1.570370in}}%
\pgfpathlineto{\pgfqpoint{0.882414in}{1.583981in}}%
\pgfpathlineto{\pgfqpoint{0.877082in}{1.590652in}}%
\pgfpathlineto{\pgfqpoint{0.870128in}{1.597592in}}%
\pgfpathlineto{\pgfqpoint{0.861425in}{1.604081in}}%
\pgfpathlineto{\pgfqpoint{0.846933in}{1.611203in}}%
\pgfpathlineto{\pgfqpoint{0.845769in}{1.611700in}}%
\pgfpathlineto{\pgfqpoint{0.830112in}{1.614752in}}%
\pgfpathlineto{\pgfqpoint{0.814455in}{1.613989in}}%
\pgfpathlineto{\pgfqpoint{0.804853in}{1.611203in}}%
\pgfpathlineto{\pgfqpoint{0.798799in}{1.609200in}}%
\pgfpathlineto{\pgfqpoint{0.783142in}{1.599815in}}%
\pgfpathlineto{\pgfqpoint{0.780465in}{1.597592in}}%
\pgfpathlineto{\pgfqpoint{0.768253in}{1.583981in}}%
\pgfpathlineto{\pgfqpoint{0.767486in}{1.582546in}}%
\pgfpathlineto{\pgfqpoint{0.761926in}{1.570370in}}%
\pgfpathlineto{\pgfqpoint{0.760149in}{1.556759in}}%
\pgfpathlineto{\pgfqpoint{0.762815in}{1.543148in}}%
\pgfpathlineto{\pgfqpoint{0.767486in}{1.534137in}}%
\pgfpathlineto{\pgfqpoint{0.770289in}{1.529536in}}%
\pgfpathlineto{\pgfqpoint{0.783142in}{1.516290in}}%
\pgfpathlineto{\pgfqpoint{0.783631in}{1.515925in}}%
\pgfpathlineto{\pgfqpoint{0.798799in}{1.507123in}}%
\pgfpathlineto{\pgfqpoint{0.813987in}{1.502314in}}%
\pgfpathlineto{\pgfqpoint{0.814455in}{1.502181in}}%
\pgfpathclose%
\pgfpathmoveto{\pgfqpoint{0.797615in}{1.529536in}}%
\pgfpathlineto{\pgfqpoint{0.785979in}{1.543148in}}%
\pgfpathlineto{\pgfqpoint{0.783142in}{1.551980in}}%
\pgfpathlineto{\pgfqpoint{0.781993in}{1.556759in}}%
\pgfpathlineto{\pgfqpoint{0.783142in}{1.563902in}}%
\pgfpathlineto{\pgfqpoint{0.784527in}{1.570370in}}%
\pgfpathlineto{\pgfqpoint{0.794704in}{1.583981in}}%
\pgfpathlineto{\pgfqpoint{0.798799in}{1.587276in}}%
\pgfpathlineto{\pgfqpoint{0.814455in}{1.594338in}}%
\pgfpathlineto{\pgfqpoint{0.830112in}{1.595514in}}%
\pgfpathlineto{\pgfqpoint{0.845769in}{1.590809in}}%
\pgfpathlineto{\pgfqpoint{0.855963in}{1.583981in}}%
\pgfpathlineto{\pgfqpoint{0.861425in}{1.577920in}}%
\pgfpathlineto{\pgfqpoint{0.866051in}{1.570370in}}%
\pgfpathlineto{\pgfqpoint{0.868457in}{1.556759in}}%
\pgfpathlineto{\pgfqpoint{0.864847in}{1.543148in}}%
\pgfpathlineto{\pgfqpoint{0.861425in}{1.538227in}}%
\pgfpathlineto{\pgfqpoint{0.852503in}{1.529536in}}%
\pgfpathlineto{\pgfqpoint{0.845769in}{1.525328in}}%
\pgfpathlineto{\pgfqpoint{0.830112in}{1.520907in}}%
\pgfpathlineto{\pgfqpoint{0.814455in}{1.522012in}}%
\pgfpathlineto{\pgfqpoint{0.798799in}{1.528648in}}%
\pgfpathlineto{\pgfqpoint{0.797615in}{1.529536in}}%
\pgfpathclose%
\pgfpathmoveto{\pgfqpoint{1.127587in}{1.501728in}}%
\pgfpathlineto{\pgfqpoint{1.143243in}{1.501728in}}%
\pgfpathlineto{\pgfqpoint{1.145709in}{1.502314in}}%
\pgfpathlineto{\pgfqpoint{1.158900in}{1.505801in}}%
\pgfpathlineto{\pgfqpoint{1.174556in}{1.514070in}}%
\pgfpathlineto{\pgfqpoint{1.177007in}{1.515925in}}%
\pgfpathlineto{\pgfqpoint{1.190213in}{1.528934in}}%
\pgfpathlineto{\pgfqpoint{1.190713in}{1.529536in}}%
\pgfpathlineto{\pgfqpoint{1.197962in}{1.543147in}}%
\pgfpathlineto{\pgfqpoint{1.200674in}{1.556759in}}%
\pgfpathlineto{\pgfqpoint{1.198866in}{1.570370in}}%
\pgfpathlineto{\pgfqpoint{1.192526in}{1.583981in}}%
\pgfpathlineto{\pgfqpoint{1.190213in}{1.586972in}}%
\pgfpathlineto{\pgfqpoint{1.180186in}{1.597592in}}%
\pgfpathlineto{\pgfqpoint{1.174556in}{1.602033in}}%
\pgfpathlineto{\pgfqpoint{1.158900in}{1.610564in}}%
\pgfpathlineto{\pgfqpoint{1.156586in}{1.611203in}}%
\pgfpathlineto{\pgfqpoint{1.143243in}{1.614447in}}%
\pgfpathlineto{\pgfqpoint{1.127587in}{1.614447in}}%
\pgfpathlineto{\pgfqpoint{1.114244in}{1.611203in}}%
\pgfpathlineto{\pgfqpoint{1.111930in}{1.610564in}}%
\pgfpathlineto{\pgfqpoint{1.096274in}{1.602033in}}%
\pgfpathlineto{\pgfqpoint{1.090644in}{1.597592in}}%
\pgfpathlineto{\pgfqpoint{1.080617in}{1.586972in}}%
\pgfpathlineto{\pgfqpoint{1.078304in}{1.583981in}}%
\pgfpathlineto{\pgfqpoint{1.071964in}{1.570370in}}%
\pgfpathlineto{\pgfqpoint{1.070156in}{1.556759in}}%
\pgfpathlineto{\pgfqpoint{1.072868in}{1.543147in}}%
\pgfpathlineto{\pgfqpoint{1.080117in}{1.529536in}}%
\pgfpathlineto{\pgfqpoint{1.080617in}{1.528934in}}%
\pgfpathlineto{\pgfqpoint{1.093823in}{1.515925in}}%
\pgfpathlineto{\pgfqpoint{1.096274in}{1.514070in}}%
\pgfpathlineto{\pgfqpoint{1.111930in}{1.505801in}}%
\pgfpathlineto{\pgfqpoint{1.125121in}{1.502314in}}%
\pgfpathlineto{\pgfqpoint{1.127587in}{1.501728in}}%
\pgfpathclose%
\pgfpathmoveto{\pgfqpoint{1.108068in}{1.529536in}}%
\pgfpathlineto{\pgfqpoint{1.096274in}{1.542241in}}%
\pgfpathlineto{\pgfqpoint{1.095678in}{1.543148in}}%
\pgfpathlineto{\pgfqpoint{1.092238in}{1.556759in}}%
\pgfpathlineto{\pgfqpoint{1.094530in}{1.570370in}}%
\pgfpathlineto{\pgfqpoint{1.096274in}{1.573380in}}%
\pgfpathlineto{\pgfqpoint{1.104913in}{1.583981in}}%
\pgfpathlineto{\pgfqpoint{1.111930in}{1.589161in}}%
\pgfpathlineto{\pgfqpoint{1.127587in}{1.595044in}}%
\pgfpathlineto{\pgfqpoint{1.143243in}{1.595044in}}%
\pgfpathlineto{\pgfqpoint{1.158900in}{1.589161in}}%
\pgfpathlineto{\pgfqpoint{1.165917in}{1.583981in}}%
\pgfpathlineto{\pgfqpoint{1.174556in}{1.573380in}}%
\pgfpathlineto{\pgfqpoint{1.176300in}{1.570370in}}%
\pgfpathlineto{\pgfqpoint{1.178592in}{1.556759in}}%
\pgfpathlineto{\pgfqpoint{1.175152in}{1.543148in}}%
\pgfpathlineto{\pgfqpoint{1.174556in}{1.542241in}}%
\pgfpathlineto{\pgfqpoint{1.162762in}{1.529536in}}%
\pgfpathlineto{\pgfqpoint{1.158900in}{1.526877in}}%
\pgfpathlineto{\pgfqpoint{1.143243in}{1.521349in}}%
\pgfpathlineto{\pgfqpoint{1.127587in}{1.521349in}}%
\pgfpathlineto{\pgfqpoint{1.111930in}{1.526877in}}%
\pgfpathlineto{\pgfqpoint{1.108068in}{1.529536in}}%
\pgfpathclose%
\pgfpathmoveto{\pgfqpoint{1.440718in}{1.501425in}}%
\pgfpathlineto{\pgfqpoint{1.456375in}{1.502181in}}%
\pgfpathlineto{\pgfqpoint{1.456843in}{1.502314in}}%
\pgfpathlineto{\pgfqpoint{1.472031in}{1.507123in}}%
\pgfpathlineto{\pgfqpoint{1.487199in}{1.515925in}}%
\pgfpathlineto{\pgfqpoint{1.487688in}{1.516290in}}%
\pgfpathlineto{\pgfqpoint{1.500541in}{1.529536in}}%
\pgfpathlineto{\pgfqpoint{1.503344in}{1.534137in}}%
\pgfpathlineto{\pgfqpoint{1.508015in}{1.543148in}}%
\pgfpathlineto{\pgfqpoint{1.510681in}{1.556759in}}%
\pgfpathlineto{\pgfqpoint{1.508904in}{1.570370in}}%
\pgfpathlineto{\pgfqpoint{1.503344in}{1.582546in}}%
\pgfpathlineto{\pgfqpoint{1.502577in}{1.583981in}}%
\pgfpathlineto{\pgfqpoint{1.490365in}{1.597592in}}%
\pgfpathlineto{\pgfqpoint{1.487688in}{1.599815in}}%
\pgfpathlineto{\pgfqpoint{1.472031in}{1.609200in}}%
\pgfpathlineto{\pgfqpoint{1.465977in}{1.611203in}}%
\pgfpathlineto{\pgfqpoint{1.456375in}{1.613989in}}%
\pgfpathlineto{\pgfqpoint{1.440718in}{1.614752in}}%
\pgfpathlineto{\pgfqpoint{1.425061in}{1.611700in}}%
\pgfpathlineto{\pgfqpoint{1.423897in}{1.611203in}}%
\pgfpathlineto{\pgfqpoint{1.409405in}{1.604081in}}%
\pgfpathlineto{\pgfqpoint{1.400702in}{1.597592in}}%
\pgfpathlineto{\pgfqpoint{1.393748in}{1.590652in}}%
\pgfpathlineto{\pgfqpoint{1.388416in}{1.583981in}}%
\pgfpathlineto{\pgfqpoint{1.381937in}{1.570370in}}%
\pgfpathlineto{\pgfqpoint{1.380089in}{1.556759in}}%
\pgfpathlineto{\pgfqpoint{1.382862in}{1.543148in}}%
\pgfpathlineto{\pgfqpoint{1.390269in}{1.529536in}}%
\pgfpathlineto{\pgfqpoint{1.393748in}{1.525476in}}%
\pgfpathlineto{\pgfqpoint{1.404037in}{1.515925in}}%
\pgfpathlineto{\pgfqpoint{1.409405in}{1.512085in}}%
\pgfpathlineto{\pgfqpoint{1.425061in}{1.504645in}}%
\pgfpathlineto{\pgfqpoint{1.436070in}{1.502314in}}%
\pgfpathlineto{\pgfqpoint{1.440718in}{1.501425in}}%
\pgfpathclose%
\pgfpathmoveto{\pgfqpoint{1.418327in}{1.529536in}}%
\pgfpathlineto{\pgfqpoint{1.409405in}{1.538227in}}%
\pgfpathlineto{\pgfqpoint{1.405983in}{1.543148in}}%
\pgfpathlineto{\pgfqpoint{1.402373in}{1.556759in}}%
\pgfpathlineto{\pgfqpoint{1.404779in}{1.570370in}}%
\pgfpathlineto{\pgfqpoint{1.409405in}{1.577920in}}%
\pgfpathlineto{\pgfqpoint{1.414867in}{1.583981in}}%
\pgfpathlineto{\pgfqpoint{1.425061in}{1.590809in}}%
\pgfpathlineto{\pgfqpoint{1.440718in}{1.595514in}}%
\pgfpathlineto{\pgfqpoint{1.456375in}{1.594338in}}%
\pgfpathlineto{\pgfqpoint{1.472031in}{1.587276in}}%
\pgfpathlineto{\pgfqpoint{1.476126in}{1.583981in}}%
\pgfpathlineto{\pgfqpoint{1.486303in}{1.570370in}}%
\pgfpathlineto{\pgfqpoint{1.487688in}{1.563902in}}%
\pgfpathlineto{\pgfqpoint{1.488837in}{1.556759in}}%
\pgfpathlineto{\pgfqpoint{1.487688in}{1.551980in}}%
\pgfpathlineto{\pgfqpoint{1.484851in}{1.543147in}}%
\pgfpathlineto{\pgfqpoint{1.473215in}{1.529536in}}%
\pgfpathlineto{\pgfqpoint{1.472031in}{1.528648in}}%
\pgfpathlineto{\pgfqpoint{1.456375in}{1.522012in}}%
\pgfpathlineto{\pgfqpoint{1.440718in}{1.520907in}}%
\pgfpathlineto{\pgfqpoint{1.425061in}{1.525328in}}%
\pgfpathlineto{\pgfqpoint{1.418327in}{1.529536in}}%
\pgfpathclose%
\pgfpathmoveto{\pgfqpoint{1.753849in}{1.501274in}}%
\pgfpathlineto{\pgfqpoint{1.764644in}{1.502314in}}%
\pgfpathlineto{\pgfqpoint{1.769506in}{1.502829in}}%
\pgfpathlineto{\pgfqpoint{1.785162in}{1.508611in}}%
\pgfpathlineto{\pgfqpoint{1.796829in}{1.515925in}}%
\pgfpathlineto{\pgfqpoint{1.800819in}{1.519151in}}%
\pgfpathlineto{\pgfqpoint{1.810443in}{1.529536in}}%
\pgfpathlineto{\pgfqpoint{1.816476in}{1.539932in}}%
\pgfpathlineto{\pgfqpoint{1.818101in}{1.543148in}}%
\pgfpathlineto{\pgfqpoint{1.820733in}{1.556759in}}%
\pgfpathlineto{\pgfqpoint{1.818979in}{1.570370in}}%
\pgfpathlineto{\pgfqpoint{1.816476in}{1.575992in}}%
\pgfpathlineto{\pgfqpoint{1.812406in}{1.583981in}}%
\pgfpathlineto{\pgfqpoint{1.800819in}{1.597383in}}%
\pgfpathlineto{\pgfqpoint{1.800578in}{1.597592in}}%
\pgfpathlineto{\pgfqpoint{1.785162in}{1.607665in}}%
\pgfpathlineto{\pgfqpoint{1.775973in}{1.611203in}}%
\pgfpathlineto{\pgfqpoint{1.769506in}{1.613379in}}%
\pgfpathlineto{\pgfqpoint{1.753849in}{1.614904in}}%
\pgfpathlineto{\pgfqpoint{1.738193in}{1.612616in}}%
\pgfpathlineto{\pgfqpoint{1.734494in}{1.611203in}}%
\pgfpathlineto{\pgfqpoint{1.722536in}{1.605959in}}%
\pgfpathlineto{\pgfqpoint{1.710591in}{1.597592in}}%
\pgfpathlineto{\pgfqpoint{1.706880in}{1.594123in}}%
\pgfpathlineto{\pgfqpoint{1.698466in}{1.583981in}}%
\pgfpathlineto{\pgfqpoint{1.691815in}{1.570370in}}%
\pgfpathlineto{\pgfqpoint{1.691223in}{1.566143in}}%
\pgfpathlineto{\pgfqpoint{1.690027in}{1.556759in}}%
\pgfpathlineto{\pgfqpoint{1.691223in}{1.550481in}}%
\pgfpathlineto{\pgfqpoint{1.692764in}{1.543148in}}%
\pgfpathlineto{\pgfqpoint{1.700369in}{1.529536in}}%
\pgfpathlineto{\pgfqpoint{1.706880in}{1.522214in}}%
\pgfpathlineto{\pgfqpoint{1.714113in}{1.515925in}}%
\pgfpathlineto{\pgfqpoint{1.722536in}{1.510265in}}%
\pgfpathlineto{\pgfqpoint{1.738193in}{1.503654in}}%
\pgfpathlineto{\pgfqpoint{1.746628in}{1.502314in}}%
\pgfpathlineto{\pgfqpoint{1.753849in}{1.501274in}}%
\pgfpathclose%
\pgfpathmoveto{\pgfqpoint{1.728301in}{1.529536in}}%
\pgfpathlineto{\pgfqpoint{1.722536in}{1.534548in}}%
\pgfpathlineto{\pgfqpoint{1.716169in}{1.543148in}}%
\pgfpathlineto{\pgfqpoint{1.712356in}{1.556759in}}%
\pgfpathlineto{\pgfqpoint{1.714897in}{1.570370in}}%
\pgfpathlineto{\pgfqpoint{1.722536in}{1.582081in}}%
\pgfpathlineto{\pgfqpoint{1.724454in}{1.583981in}}%
\pgfpathlineto{\pgfqpoint{1.738193in}{1.592221in}}%
\pgfpathlineto{\pgfqpoint{1.753849in}{1.595749in}}%
\pgfpathlineto{\pgfqpoint{1.769506in}{1.593398in}}%
\pgfpathlineto{\pgfqpoint{1.785162in}{1.585154in}}%
\pgfpathlineto{\pgfqpoint{1.786512in}{1.583981in}}%
\pgfpathlineto{\pgfqpoint{1.795995in}{1.570370in}}%
\pgfpathlineto{\pgfqpoint{1.798699in}{1.556759in}}%
\pgfpathlineto{\pgfqpoint{1.794642in}{1.543148in}}%
\pgfpathlineto{\pgfqpoint{1.785162in}{1.531204in}}%
\pgfpathlineto{\pgfqpoint{1.782977in}{1.529536in}}%
\pgfpathlineto{\pgfqpoint{1.769506in}{1.522896in}}%
\pgfpathlineto{\pgfqpoint{1.753849in}{1.520687in}}%
\pgfpathlineto{\pgfqpoint{1.738193in}{1.524001in}}%
\pgfpathlineto{\pgfqpoint{1.728301in}{1.529536in}}%
\pgfpathclose%
\pgfusepath{fill}%
\end{pgfscope}%
\begin{pgfscope}%
\pgfpathrectangle{\pgfqpoint{0.360415in}{0.345370in}}{\pgfqpoint{1.550000in}{1.347500in}}%
\pgfusepath{clip}%
\pgfsetbuttcap%
\pgfsetroundjoin%
\definecolor{currentfill}{rgb}{0.959424,0.543431,0.278701}%
\pgfsetfillcolor{currentfill}%
\pgfsetlinewidth{0.000000pt}%
\definecolor{currentstroke}{rgb}{0.000000,0.000000,0.000000}%
\pgfsetstrokecolor{currentstroke}%
\pgfsetdash{}{0pt}%
\pgfpathmoveto{\pgfqpoint{0.485668in}{0.412440in}}%
\pgfpathlineto{\pgfqpoint{0.501324in}{0.407554in}}%
\pgfpathlineto{\pgfqpoint{0.516981in}{0.406160in}}%
\pgfpathlineto{\pgfqpoint{0.532637in}{0.408251in}}%
\pgfpathlineto{\pgfqpoint{0.547170in}{0.413425in}}%
\pgfpathlineto{\pgfqpoint{0.548294in}{0.413839in}}%
\pgfpathlineto{\pgfqpoint{0.563950in}{0.422949in}}%
\pgfpathlineto{\pgfqpoint{0.569274in}{0.427036in}}%
\pgfpathlineto{\pgfqpoint{0.579607in}{0.436422in}}%
\pgfpathlineto{\pgfqpoint{0.583638in}{0.440648in}}%
\pgfpathlineto{\pgfqpoint{0.593230in}{0.454259in}}%
\pgfpathlineto{\pgfqpoint{0.595263in}{0.459130in}}%
\pgfpathlineto{\pgfqpoint{0.598881in}{0.467870in}}%
\pgfpathlineto{\pgfqpoint{0.600502in}{0.481481in}}%
\pgfpathlineto{\pgfqpoint{0.598070in}{0.495092in}}%
\pgfpathlineto{\pgfqpoint{0.595263in}{0.501064in}}%
\pgfpathlineto{\pgfqpoint{0.591631in}{0.508703in}}%
\pgfpathlineto{\pgfqpoint{0.581243in}{0.522314in}}%
\pgfpathlineto{\pgfqpoint{0.579607in}{0.523960in}}%
\pgfpathlineto{\pgfqpoint{0.565844in}{0.535925in}}%
\pgfpathlineto{\pgfqpoint{0.563950in}{0.537348in}}%
\pgfpathlineto{\pgfqpoint{0.548294in}{0.546379in}}%
\pgfpathlineto{\pgfqpoint{0.539507in}{0.549536in}}%
\pgfpathlineto{\pgfqpoint{0.532637in}{0.551976in}}%
\pgfpathlineto{\pgfqpoint{0.516981in}{0.554091in}}%
\pgfpathlineto{\pgfqpoint{0.501324in}{0.552681in}}%
\pgfpathlineto{\pgfqpoint{0.491271in}{0.549536in}}%
\pgfpathlineto{\pgfqpoint{0.485668in}{0.547769in}}%
\pgfpathlineto{\pgfqpoint{0.470011in}{0.539429in}}%
\pgfpathlineto{\pgfqpoint{0.465150in}{0.535925in}}%
\pgfpathlineto{\pgfqpoint{0.454354in}{0.526942in}}%
\pgfpathlineto{\pgfqpoint{0.449653in}{0.522314in}}%
\pgfpathlineto{\pgfqpoint{0.439173in}{0.508703in}}%
\pgfpathlineto{\pgfqpoint{0.438698in}{0.507726in}}%
\pgfpathlineto{\pgfqpoint{0.432746in}{0.495092in}}%
\pgfpathlineto{\pgfqpoint{0.430341in}{0.481481in}}%
\pgfpathlineto{\pgfqpoint{0.431944in}{0.467870in}}%
\pgfpathlineto{\pgfqpoint{0.437564in}{0.454259in}}%
\pgfpathlineto{\pgfqpoint{0.438698in}{0.452598in}}%
\pgfpathlineto{\pgfqpoint{0.447237in}{0.440648in}}%
\pgfpathlineto{\pgfqpoint{0.454354in}{0.433345in}}%
\pgfpathlineto{\pgfqpoint{0.461611in}{0.427036in}}%
\pgfpathlineto{\pgfqpoint{0.470011in}{0.420849in}}%
\pgfpathlineto{\pgfqpoint{0.483757in}{0.413425in}}%
\pgfpathlineto{\pgfqpoint{0.485668in}{0.412440in}}%
\pgfpathclose%
\pgfpathmoveto{\pgfqpoint{0.494857in}{0.427036in}}%
\pgfpathlineto{\pgfqpoint{0.485668in}{0.430575in}}%
\pgfpathlineto{\pgfqpoint{0.470252in}{0.440648in}}%
\pgfpathlineto{\pgfqpoint{0.470011in}{0.440857in}}%
\pgfpathlineto{\pgfqpoint{0.458424in}{0.454259in}}%
\pgfpathlineto{\pgfqpoint{0.454354in}{0.462248in}}%
\pgfpathlineto{\pgfqpoint{0.451851in}{0.467870in}}%
\pgfpathlineto{\pgfqpoint{0.450097in}{0.481481in}}%
\pgfpathlineto{\pgfqpoint{0.452729in}{0.495092in}}%
\pgfpathlineto{\pgfqpoint{0.454354in}{0.498307in}}%
\pgfpathlineto{\pgfqpoint{0.460387in}{0.508703in}}%
\pgfpathlineto{\pgfqpoint{0.470011in}{0.519088in}}%
\pgfpathlineto{\pgfqpoint{0.474001in}{0.522314in}}%
\pgfpathlineto{\pgfqpoint{0.485668in}{0.529628in}}%
\pgfpathlineto{\pgfqpoint{0.501324in}{0.535411in}}%
\pgfpathlineto{\pgfqpoint{0.506186in}{0.535925in}}%
\pgfpathlineto{\pgfqpoint{0.516981in}{0.536965in}}%
\pgfpathlineto{\pgfqpoint{0.524202in}{0.535925in}}%
\pgfpathlineto{\pgfqpoint{0.532637in}{0.534585in}}%
\pgfpathlineto{\pgfqpoint{0.548294in}{0.527974in}}%
\pgfpathlineto{\pgfqpoint{0.556717in}{0.522314in}}%
\pgfpathlineto{\pgfqpoint{0.563950in}{0.516026in}}%
\pgfpathlineto{\pgfqpoint{0.570461in}{0.508703in}}%
\pgfpathlineto{\pgfqpoint{0.578066in}{0.495092in}}%
\pgfpathlineto{\pgfqpoint{0.579607in}{0.487759in}}%
\pgfpathlineto{\pgfqpoint{0.580803in}{0.481481in}}%
\pgfpathlineto{\pgfqpoint{0.579607in}{0.472096in}}%
\pgfpathlineto{\pgfqpoint{0.579015in}{0.467870in}}%
\pgfpathlineto{\pgfqpoint{0.572364in}{0.454259in}}%
\pgfpathlineto{\pgfqpoint{0.563950in}{0.444116in}}%
\pgfpathlineto{\pgfqpoint{0.560239in}{0.440648in}}%
\pgfpathlineto{\pgfqpoint{0.548294in}{0.432281in}}%
\pgfpathlineto{\pgfqpoint{0.536336in}{0.427036in}}%
\pgfpathlineto{\pgfqpoint{0.532637in}{0.425623in}}%
\pgfpathlineto{\pgfqpoint{0.516981in}{0.423335in}}%
\pgfpathlineto{\pgfqpoint{0.501324in}{0.424860in}}%
\pgfpathlineto{\pgfqpoint{0.494857in}{0.427036in}}%
\pgfpathclose%
\pgfpathmoveto{\pgfqpoint{0.798799in}{0.411182in}}%
\pgfpathlineto{\pgfqpoint{0.814455in}{0.406996in}}%
\pgfpathlineto{\pgfqpoint{0.830112in}{0.406299in}}%
\pgfpathlineto{\pgfqpoint{0.845769in}{0.409088in}}%
\pgfpathlineto{\pgfqpoint{0.856678in}{0.413425in}}%
\pgfpathlineto{\pgfqpoint{0.861425in}{0.415381in}}%
\pgfpathlineto{\pgfqpoint{0.877082in}{0.425185in}}%
\pgfpathlineto{\pgfqpoint{0.879405in}{0.427036in}}%
\pgfpathlineto{\pgfqpoint{0.892738in}{0.439634in}}%
\pgfpathlineto{\pgfqpoint{0.893688in}{0.440648in}}%
\pgfpathlineto{\pgfqpoint{0.903236in}{0.454259in}}%
\pgfpathlineto{\pgfqpoint{0.908395in}{0.466857in}}%
\pgfpathlineto{\pgfqpoint{0.908815in}{0.467870in}}%
\pgfpathlineto{\pgfqpoint{0.910462in}{0.481481in}}%
\pgfpathlineto{\pgfqpoint{0.908395in}{0.492886in}}%
\pgfpathlineto{\pgfqpoint{0.908005in}{0.495092in}}%
\pgfpathlineto{\pgfqpoint{0.901644in}{0.508703in}}%
\pgfpathlineto{\pgfqpoint{0.892738in}{0.520500in}}%
\pgfpathlineto{\pgfqpoint{0.891208in}{0.522314in}}%
\pgfpathlineto{\pgfqpoint{0.877082in}{0.535059in}}%
\pgfpathlineto{\pgfqpoint{0.875858in}{0.535925in}}%
\pgfpathlineto{\pgfqpoint{0.861425in}{0.544849in}}%
\pgfpathlineto{\pgfqpoint{0.849785in}{0.549536in}}%
\pgfpathlineto{\pgfqpoint{0.845769in}{0.551129in}}%
\pgfpathlineto{\pgfqpoint{0.830112in}{0.553950in}}%
\pgfpathlineto{\pgfqpoint{0.814455in}{0.553245in}}%
\pgfpathlineto{\pgfqpoint{0.800706in}{0.549536in}}%
\pgfpathlineto{\pgfqpoint{0.798799in}{0.549019in}}%
\pgfpathlineto{\pgfqpoint{0.783142in}{0.541374in}}%
\pgfpathlineto{\pgfqpoint{0.775228in}{0.535925in}}%
\pgfpathlineto{\pgfqpoint{0.767486in}{0.529789in}}%
\pgfpathlineto{\pgfqpoint{0.759699in}{0.522314in}}%
\pgfpathlineto{\pgfqpoint{0.751829in}{0.512344in}}%
\pgfpathlineto{\pgfqpoint{0.749147in}{0.508703in}}%
\pgfpathlineto{\pgfqpoint{0.742769in}{0.495092in}}%
\pgfpathlineto{\pgfqpoint{0.740382in}{0.481481in}}%
\pgfpathlineto{\pgfqpoint{0.741973in}{0.467870in}}%
\pgfpathlineto{\pgfqpoint{0.747551in}{0.454259in}}%
\pgfpathlineto{\pgfqpoint{0.751829in}{0.448034in}}%
\pgfpathlineto{\pgfqpoint{0.757252in}{0.440648in}}%
\pgfpathlineto{\pgfqpoint{0.767486in}{0.430409in}}%
\pgfpathlineto{\pgfqpoint{0.771558in}{0.427036in}}%
\pgfpathlineto{\pgfqpoint{0.783142in}{0.418887in}}%
\pgfpathlineto{\pgfqpoint{0.794103in}{0.413425in}}%
\pgfpathlineto{\pgfqpoint{0.798799in}{0.411182in}}%
\pgfpathclose%
\pgfpathmoveto{\pgfqpoint{0.804853in}{0.427036in}}%
\pgfpathlineto{\pgfqpoint{0.798799in}{0.429039in}}%
\pgfpathlineto{\pgfqpoint{0.783142in}{0.438424in}}%
\pgfpathlineto{\pgfqpoint{0.780465in}{0.440648in}}%
\pgfpathlineto{\pgfqpoint{0.768253in}{0.454259in}}%
\pgfpathlineto{\pgfqpoint{0.767486in}{0.455693in}}%
\pgfpathlineto{\pgfqpoint{0.761926in}{0.467870in}}%
\pgfpathlineto{\pgfqpoint{0.760149in}{0.481481in}}%
\pgfpathlineto{\pgfqpoint{0.762815in}{0.495092in}}%
\pgfpathlineto{\pgfqpoint{0.767486in}{0.504102in}}%
\pgfpathlineto{\pgfqpoint{0.770289in}{0.508703in}}%
\pgfpathlineto{\pgfqpoint{0.783142in}{0.521949in}}%
\pgfpathlineto{\pgfqpoint{0.783631in}{0.522314in}}%
\pgfpathlineto{\pgfqpoint{0.798799in}{0.531116in}}%
\pgfpathlineto{\pgfqpoint{0.813987in}{0.535925in}}%
\pgfpathlineto{\pgfqpoint{0.814455in}{0.536058in}}%
\pgfpathlineto{\pgfqpoint{0.830112in}{0.536814in}}%
\pgfpathlineto{\pgfqpoint{0.834760in}{0.535925in}}%
\pgfpathlineto{\pgfqpoint{0.845769in}{0.533595in}}%
\pgfpathlineto{\pgfqpoint{0.861425in}{0.526154in}}%
\pgfpathlineto{\pgfqpoint{0.866793in}{0.522314in}}%
\pgfpathlineto{\pgfqpoint{0.877082in}{0.512764in}}%
\pgfpathlineto{\pgfqpoint{0.880561in}{0.508703in}}%
\pgfpathlineto{\pgfqpoint{0.887968in}{0.495092in}}%
\pgfpathlineto{\pgfqpoint{0.890741in}{0.481481in}}%
\pgfpathlineto{\pgfqpoint{0.888893in}{0.467870in}}%
\pgfpathlineto{\pgfqpoint{0.882414in}{0.454259in}}%
\pgfpathlineto{\pgfqpoint{0.877082in}{0.447587in}}%
\pgfpathlineto{\pgfqpoint{0.870128in}{0.440648in}}%
\pgfpathlineto{\pgfqpoint{0.861425in}{0.434158in}}%
\pgfpathlineto{\pgfqpoint{0.846933in}{0.427036in}}%
\pgfpathlineto{\pgfqpoint{0.845769in}{0.426540in}}%
\pgfpathlineto{\pgfqpoint{0.830112in}{0.423488in}}%
\pgfpathlineto{\pgfqpoint{0.814455in}{0.424250in}}%
\pgfpathlineto{\pgfqpoint{0.804853in}{0.427036in}}%
\pgfpathclose%
\pgfpathmoveto{\pgfqpoint{1.111930in}{0.410065in}}%
\pgfpathlineto{\pgfqpoint{1.127587in}{0.406578in}}%
\pgfpathlineto{\pgfqpoint{1.143243in}{0.406578in}}%
\pgfpathlineto{\pgfqpoint{1.158900in}{0.410065in}}%
\pgfpathlineto{\pgfqpoint{1.166569in}{0.413425in}}%
\pgfpathlineto{\pgfqpoint{1.174556in}{0.417064in}}%
\pgfpathlineto{\pgfqpoint{1.189474in}{0.427036in}}%
\pgfpathlineto{\pgfqpoint{1.190213in}{0.427617in}}%
\pgfpathlineto{\pgfqpoint{1.203622in}{0.440648in}}%
\pgfpathlineto{\pgfqpoint{1.205870in}{0.443616in}}%
\pgfpathlineto{\pgfqpoint{1.213260in}{0.454259in}}%
\pgfpathlineto{\pgfqpoint{1.218820in}{0.467870in}}%
\pgfpathlineto{\pgfqpoint{1.220406in}{0.481481in}}%
\pgfpathlineto{\pgfqpoint{1.218027in}{0.495092in}}%
\pgfpathlineto{\pgfqpoint{1.211670in}{0.508703in}}%
\pgfpathlineto{\pgfqpoint{1.205870in}{0.516495in}}%
\pgfpathlineto{\pgfqpoint{1.201132in}{0.522314in}}%
\pgfpathlineto{\pgfqpoint{1.190213in}{0.532496in}}%
\pgfpathlineto{\pgfqpoint{1.185646in}{0.535925in}}%
\pgfpathlineto{\pgfqpoint{1.174556in}{0.543181in}}%
\pgfpathlineto{\pgfqpoint{1.160283in}{0.549536in}}%
\pgfpathlineto{\pgfqpoint{1.158900in}{0.550141in}}%
\pgfpathlineto{\pgfqpoint{1.143243in}{0.553668in}}%
\pgfpathlineto{\pgfqpoint{1.127587in}{0.553668in}}%
\pgfpathlineto{\pgfqpoint{1.111930in}{0.550141in}}%
\pgfpathlineto{\pgfqpoint{1.110547in}{0.549536in}}%
\pgfpathlineto{\pgfqpoint{1.096274in}{0.543181in}}%
\pgfpathlineto{\pgfqpoint{1.085184in}{0.535925in}}%
\pgfpathlineto{\pgfqpoint{1.080617in}{0.532496in}}%
\pgfpathlineto{\pgfqpoint{1.069698in}{0.522314in}}%
\pgfpathlineto{\pgfqpoint{1.064960in}{0.516495in}}%
\pgfpathlineto{\pgfqpoint{1.059160in}{0.508703in}}%
\pgfpathlineto{\pgfqpoint{1.052803in}{0.495092in}}%
\pgfpathlineto{\pgfqpoint{1.050424in}{0.481481in}}%
\pgfpathlineto{\pgfqpoint{1.052010in}{0.467870in}}%
\pgfpathlineto{\pgfqpoint{1.057570in}{0.454259in}}%
\pgfpathlineto{\pgfqpoint{1.064960in}{0.443616in}}%
\pgfpathlineto{\pgfqpoint{1.067208in}{0.440648in}}%
\pgfpathlineto{\pgfqpoint{1.080617in}{0.427617in}}%
\pgfpathlineto{\pgfqpoint{1.081356in}{0.427036in}}%
\pgfpathlineto{\pgfqpoint{1.096274in}{0.417064in}}%
\pgfpathlineto{\pgfqpoint{1.104261in}{0.413425in}}%
\pgfpathlineto{\pgfqpoint{1.111930in}{0.410065in}}%
\pgfpathclose%
\pgfpathmoveto{\pgfqpoint{1.114244in}{0.427036in}}%
\pgfpathlineto{\pgfqpoint{1.111930in}{0.427675in}}%
\pgfpathlineto{\pgfqpoint{1.096274in}{0.436206in}}%
\pgfpathlineto{\pgfqpoint{1.090644in}{0.440648in}}%
\pgfpathlineto{\pgfqpoint{1.080617in}{0.451268in}}%
\pgfpathlineto{\pgfqpoint{1.078304in}{0.454259in}}%
\pgfpathlineto{\pgfqpoint{1.071964in}{0.467870in}}%
\pgfpathlineto{\pgfqpoint{1.070156in}{0.481481in}}%
\pgfpathlineto{\pgfqpoint{1.072868in}{0.495092in}}%
\pgfpathlineto{\pgfqpoint{1.080117in}{0.508703in}}%
\pgfpathlineto{\pgfqpoint{1.080617in}{0.509306in}}%
\pgfpathlineto{\pgfqpoint{1.093823in}{0.522314in}}%
\pgfpathlineto{\pgfqpoint{1.096274in}{0.524169in}}%
\pgfpathlineto{\pgfqpoint{1.111930in}{0.532439in}}%
\pgfpathlineto{\pgfqpoint{1.125121in}{0.535925in}}%
\pgfpathlineto{\pgfqpoint{1.127587in}{0.536512in}}%
\pgfpathlineto{\pgfqpoint{1.143243in}{0.536512in}}%
\pgfpathlineto{\pgfqpoint{1.145709in}{0.535925in}}%
\pgfpathlineto{\pgfqpoint{1.158900in}{0.532439in}}%
\pgfpathlineto{\pgfqpoint{1.174556in}{0.524169in}}%
\pgfpathlineto{\pgfqpoint{1.177007in}{0.522314in}}%
\pgfpathlineto{\pgfqpoint{1.190213in}{0.509306in}}%
\pgfpathlineto{\pgfqpoint{1.190713in}{0.508703in}}%
\pgfpathlineto{\pgfqpoint{1.197962in}{0.495092in}}%
\pgfpathlineto{\pgfqpoint{1.200674in}{0.481481in}}%
\pgfpathlineto{\pgfqpoint{1.198866in}{0.467870in}}%
\pgfpathlineto{\pgfqpoint{1.192526in}{0.454259in}}%
\pgfpathlineto{\pgfqpoint{1.190213in}{0.451268in}}%
\pgfpathlineto{\pgfqpoint{1.180186in}{0.440648in}}%
\pgfpathlineto{\pgfqpoint{1.174556in}{0.436206in}}%
\pgfpathlineto{\pgfqpoint{1.158900in}{0.427675in}}%
\pgfpathlineto{\pgfqpoint{1.156586in}{0.427036in}}%
\pgfpathlineto{\pgfqpoint{1.143243in}{0.423793in}}%
\pgfpathlineto{\pgfqpoint{1.127587in}{0.423793in}}%
\pgfpathlineto{\pgfqpoint{1.114244in}{0.427036in}}%
\pgfpathclose%
\pgfpathmoveto{\pgfqpoint{1.425061in}{0.409088in}}%
\pgfpathlineto{\pgfqpoint{1.440718in}{0.406299in}}%
\pgfpathlineto{\pgfqpoint{1.456375in}{0.406996in}}%
\pgfpathlineto{\pgfqpoint{1.472031in}{0.411182in}}%
\pgfpathlineto{\pgfqpoint{1.476727in}{0.413425in}}%
\pgfpathlineto{\pgfqpoint{1.487688in}{0.418887in}}%
\pgfpathlineto{\pgfqpoint{1.499272in}{0.427036in}}%
\pgfpathlineto{\pgfqpoint{1.503344in}{0.430409in}}%
\pgfpathlineto{\pgfqpoint{1.513578in}{0.440648in}}%
\pgfpathlineto{\pgfqpoint{1.519001in}{0.448034in}}%
\pgfpathlineto{\pgfqpoint{1.523279in}{0.454259in}}%
\pgfpathlineto{\pgfqpoint{1.528857in}{0.467870in}}%
\pgfpathlineto{\pgfqpoint{1.530448in}{0.481481in}}%
\pgfpathlineto{\pgfqpoint{1.528061in}{0.495092in}}%
\pgfpathlineto{\pgfqpoint{1.521683in}{0.508703in}}%
\pgfpathlineto{\pgfqpoint{1.519001in}{0.512344in}}%
\pgfpathlineto{\pgfqpoint{1.511131in}{0.522314in}}%
\pgfpathlineto{\pgfqpoint{1.503344in}{0.529789in}}%
\pgfpathlineto{\pgfqpoint{1.495602in}{0.535925in}}%
\pgfpathlineto{\pgfqpoint{1.487688in}{0.541374in}}%
\pgfpathlineto{\pgfqpoint{1.472031in}{0.549019in}}%
\pgfpathlineto{\pgfqpoint{1.470124in}{0.549536in}}%
\pgfpathlineto{\pgfqpoint{1.456375in}{0.553245in}}%
\pgfpathlineto{\pgfqpoint{1.440718in}{0.553950in}}%
\pgfpathlineto{\pgfqpoint{1.425061in}{0.551129in}}%
\pgfpathlineto{\pgfqpoint{1.421045in}{0.549536in}}%
\pgfpathlineto{\pgfqpoint{1.409405in}{0.544849in}}%
\pgfpathlineto{\pgfqpoint{1.394972in}{0.535925in}}%
\pgfpathlineto{\pgfqpoint{1.393748in}{0.535059in}}%
\pgfpathlineto{\pgfqpoint{1.379622in}{0.522314in}}%
\pgfpathlineto{\pgfqpoint{1.378092in}{0.520500in}}%
\pgfpathlineto{\pgfqpoint{1.369186in}{0.508703in}}%
\pgfpathlineto{\pgfqpoint{1.362825in}{0.495092in}}%
\pgfpathlineto{\pgfqpoint{1.362435in}{0.492886in}}%
\pgfpathlineto{\pgfqpoint{1.360368in}{0.481481in}}%
\pgfpathlineto{\pgfqpoint{1.362015in}{0.467870in}}%
\pgfpathlineto{\pgfqpoint{1.362435in}{0.466857in}}%
\pgfpathlineto{\pgfqpoint{1.367594in}{0.454259in}}%
\pgfpathlineto{\pgfqpoint{1.377142in}{0.440648in}}%
\pgfpathlineto{\pgfqpoint{1.378092in}{0.439634in}}%
\pgfpathlineto{\pgfqpoint{1.391425in}{0.427036in}}%
\pgfpathlineto{\pgfqpoint{1.393748in}{0.425185in}}%
\pgfpathlineto{\pgfqpoint{1.409405in}{0.415381in}}%
\pgfpathlineto{\pgfqpoint{1.414152in}{0.413425in}}%
\pgfpathlineto{\pgfqpoint{1.425061in}{0.409088in}}%
\pgfpathclose%
\pgfpathmoveto{\pgfqpoint{1.423897in}{0.427036in}}%
\pgfpathlineto{\pgfqpoint{1.409405in}{0.434158in}}%
\pgfpathlineto{\pgfqpoint{1.400702in}{0.440648in}}%
\pgfpathlineto{\pgfqpoint{1.393748in}{0.447587in}}%
\pgfpathlineto{\pgfqpoint{1.388416in}{0.454259in}}%
\pgfpathlineto{\pgfqpoint{1.381937in}{0.467870in}}%
\pgfpathlineto{\pgfqpoint{1.380089in}{0.481481in}}%
\pgfpathlineto{\pgfqpoint{1.382862in}{0.495092in}}%
\pgfpathlineto{\pgfqpoint{1.390269in}{0.508703in}}%
\pgfpathlineto{\pgfqpoint{1.393748in}{0.512764in}}%
\pgfpathlineto{\pgfqpoint{1.404037in}{0.522314in}}%
\pgfpathlineto{\pgfqpoint{1.409405in}{0.526154in}}%
\pgfpathlineto{\pgfqpoint{1.425061in}{0.533595in}}%
\pgfpathlineto{\pgfqpoint{1.436070in}{0.535925in}}%
\pgfpathlineto{\pgfqpoint{1.440718in}{0.536814in}}%
\pgfpathlineto{\pgfqpoint{1.456375in}{0.536058in}}%
\pgfpathlineto{\pgfqpoint{1.456843in}{0.535925in}}%
\pgfpathlineto{\pgfqpoint{1.472031in}{0.531116in}}%
\pgfpathlineto{\pgfqpoint{1.487199in}{0.522314in}}%
\pgfpathlineto{\pgfqpoint{1.487688in}{0.521949in}}%
\pgfpathlineto{\pgfqpoint{1.500541in}{0.508703in}}%
\pgfpathlineto{\pgfqpoint{1.503344in}{0.504102in}}%
\pgfpathlineto{\pgfqpoint{1.508015in}{0.495092in}}%
\pgfpathlineto{\pgfqpoint{1.510681in}{0.481481in}}%
\pgfpathlineto{\pgfqpoint{1.508904in}{0.467870in}}%
\pgfpathlineto{\pgfqpoint{1.503344in}{0.455693in}}%
\pgfpathlineto{\pgfqpoint{1.502577in}{0.454259in}}%
\pgfpathlineto{\pgfqpoint{1.490365in}{0.440648in}}%
\pgfpathlineto{\pgfqpoint{1.487688in}{0.438424in}}%
\pgfpathlineto{\pgfqpoint{1.472031in}{0.429039in}}%
\pgfpathlineto{\pgfqpoint{1.465977in}{0.427036in}}%
\pgfpathlineto{\pgfqpoint{1.456375in}{0.424250in}}%
\pgfpathlineto{\pgfqpoint{1.440718in}{0.423488in}}%
\pgfpathlineto{\pgfqpoint{1.425061in}{0.426540in}}%
\pgfpathlineto{\pgfqpoint{1.423897in}{0.427036in}}%
\pgfpathclose%
\pgfpathmoveto{\pgfqpoint{1.738193in}{0.408251in}}%
\pgfpathlineto{\pgfqpoint{1.753849in}{0.406160in}}%
\pgfpathlineto{\pgfqpoint{1.769506in}{0.407554in}}%
\pgfpathlineto{\pgfqpoint{1.785162in}{0.412440in}}%
\pgfpathlineto{\pgfqpoint{1.787073in}{0.413425in}}%
\pgfpathlineto{\pgfqpoint{1.800819in}{0.420849in}}%
\pgfpathlineto{\pgfqpoint{1.809219in}{0.427036in}}%
\pgfpathlineto{\pgfqpoint{1.816476in}{0.433345in}}%
\pgfpathlineto{\pgfqpoint{1.823593in}{0.440648in}}%
\pgfpathlineto{\pgfqpoint{1.832132in}{0.452598in}}%
\pgfpathlineto{\pgfqpoint{1.833266in}{0.454259in}}%
\pgfpathlineto{\pgfqpoint{1.838886in}{0.467870in}}%
\pgfpathlineto{\pgfqpoint{1.840489in}{0.481481in}}%
\pgfpathlineto{\pgfqpoint{1.838084in}{0.495092in}}%
\pgfpathlineto{\pgfqpoint{1.832132in}{0.507726in}}%
\pgfpathlineto{\pgfqpoint{1.831657in}{0.508703in}}%
\pgfpathlineto{\pgfqpoint{1.821177in}{0.522314in}}%
\pgfpathlineto{\pgfqpoint{1.816476in}{0.526942in}}%
\pgfpathlineto{\pgfqpoint{1.805680in}{0.535925in}}%
\pgfpathlineto{\pgfqpoint{1.800819in}{0.539429in}}%
\pgfpathlineto{\pgfqpoint{1.785162in}{0.547769in}}%
\pgfpathlineto{\pgfqpoint{1.779559in}{0.549536in}}%
\pgfpathlineto{\pgfqpoint{1.769506in}{0.552681in}}%
\pgfpathlineto{\pgfqpoint{1.753849in}{0.554091in}}%
\pgfpathlineto{\pgfqpoint{1.738193in}{0.551976in}}%
\pgfpathlineto{\pgfqpoint{1.731323in}{0.549536in}}%
\pgfpathlineto{\pgfqpoint{1.722536in}{0.546379in}}%
\pgfpathlineto{\pgfqpoint{1.706880in}{0.537348in}}%
\pgfpathlineto{\pgfqpoint{1.704986in}{0.535925in}}%
\pgfpathlineto{\pgfqpoint{1.691223in}{0.523960in}}%
\pgfpathlineto{\pgfqpoint{1.689587in}{0.522314in}}%
\pgfpathlineto{\pgfqpoint{1.679199in}{0.508703in}}%
\pgfpathlineto{\pgfqpoint{1.675567in}{0.501064in}}%
\pgfpathlineto{\pgfqpoint{1.672760in}{0.495092in}}%
\pgfpathlineto{\pgfqpoint{1.670328in}{0.481481in}}%
\pgfpathlineto{\pgfqpoint{1.671949in}{0.467870in}}%
\pgfpathlineto{\pgfqpoint{1.675567in}{0.459130in}}%
\pgfpathlineto{\pgfqpoint{1.677600in}{0.454259in}}%
\pgfpathlineto{\pgfqpoint{1.687192in}{0.440648in}}%
\pgfpathlineto{\pgfqpoint{1.691223in}{0.436422in}}%
\pgfpathlineto{\pgfqpoint{1.701556in}{0.427036in}}%
\pgfpathlineto{\pgfqpoint{1.706880in}{0.422949in}}%
\pgfpathlineto{\pgfqpoint{1.722536in}{0.413839in}}%
\pgfpathlineto{\pgfqpoint{1.723660in}{0.413425in}}%
\pgfpathlineto{\pgfqpoint{1.738193in}{0.408251in}}%
\pgfpathclose%
\pgfpathmoveto{\pgfqpoint{1.734494in}{0.427036in}}%
\pgfpathlineto{\pgfqpoint{1.722536in}{0.432281in}}%
\pgfpathlineto{\pgfqpoint{1.710591in}{0.440648in}}%
\pgfpathlineto{\pgfqpoint{1.706880in}{0.444116in}}%
\pgfpathlineto{\pgfqpoint{1.698466in}{0.454259in}}%
\pgfpathlineto{\pgfqpoint{1.691815in}{0.467870in}}%
\pgfpathlineto{\pgfqpoint{1.691223in}{0.472096in}}%
\pgfpathlineto{\pgfqpoint{1.690027in}{0.481481in}}%
\pgfpathlineto{\pgfqpoint{1.691223in}{0.487759in}}%
\pgfpathlineto{\pgfqpoint{1.692764in}{0.495092in}}%
\pgfpathlineto{\pgfqpoint{1.700369in}{0.508703in}}%
\pgfpathlineto{\pgfqpoint{1.706880in}{0.516026in}}%
\pgfpathlineto{\pgfqpoint{1.714113in}{0.522314in}}%
\pgfpathlineto{\pgfqpoint{1.722536in}{0.527974in}}%
\pgfpathlineto{\pgfqpoint{1.738193in}{0.534585in}}%
\pgfpathlineto{\pgfqpoint{1.746628in}{0.535925in}}%
\pgfpathlineto{\pgfqpoint{1.753849in}{0.536965in}}%
\pgfpathlineto{\pgfqpoint{1.764644in}{0.535925in}}%
\pgfpathlineto{\pgfqpoint{1.769506in}{0.535411in}}%
\pgfpathlineto{\pgfqpoint{1.785162in}{0.529628in}}%
\pgfpathlineto{\pgfqpoint{1.796829in}{0.522314in}}%
\pgfpathlineto{\pgfqpoint{1.800819in}{0.519088in}}%
\pgfpathlineto{\pgfqpoint{1.810443in}{0.508703in}}%
\pgfpathlineto{\pgfqpoint{1.816476in}{0.498307in}}%
\pgfpathlineto{\pgfqpoint{1.818101in}{0.495092in}}%
\pgfpathlineto{\pgfqpoint{1.820733in}{0.481481in}}%
\pgfpathlineto{\pgfqpoint{1.818979in}{0.467870in}}%
\pgfpathlineto{\pgfqpoint{1.816476in}{0.462248in}}%
\pgfpathlineto{\pgfqpoint{1.812406in}{0.454259in}}%
\pgfpathlineto{\pgfqpoint{1.800819in}{0.440857in}}%
\pgfpathlineto{\pgfqpoint{1.800578in}{0.440648in}}%
\pgfpathlineto{\pgfqpoint{1.785162in}{0.430575in}}%
\pgfpathlineto{\pgfqpoint{1.775973in}{0.427036in}}%
\pgfpathlineto{\pgfqpoint{1.769506in}{0.424860in}}%
\pgfpathlineto{\pgfqpoint{1.753849in}{0.423335in}}%
\pgfpathlineto{\pgfqpoint{1.738193in}{0.425623in}}%
\pgfpathlineto{\pgfqpoint{1.734494in}{0.427036in}}%
\pgfpathclose%
\pgfpathmoveto{\pgfqpoint{0.485668in}{0.681928in}}%
\pgfpathlineto{\pgfqpoint{0.501324in}{0.677079in}}%
\pgfpathlineto{\pgfqpoint{0.516981in}{0.675695in}}%
\pgfpathlineto{\pgfqpoint{0.532637in}{0.677771in}}%
\pgfpathlineto{\pgfqpoint{0.548294in}{0.683316in}}%
\pgfpathlineto{\pgfqpoint{0.552482in}{0.685648in}}%
\pgfpathlineto{\pgfqpoint{0.563950in}{0.692489in}}%
\pgfpathlineto{\pgfqpoint{0.572548in}{0.699259in}}%
\pgfpathlineto{\pgfqpoint{0.579607in}{0.705990in}}%
\pgfpathlineto{\pgfqpoint{0.585875in}{0.712870in}}%
\pgfpathlineto{\pgfqpoint{0.594669in}{0.726481in}}%
\pgfpathlineto{\pgfqpoint{0.595263in}{0.728139in}}%
\pgfpathlineto{\pgfqpoint{0.599530in}{0.740092in}}%
\pgfpathlineto{\pgfqpoint{0.600340in}{0.753703in}}%
\pgfpathlineto{\pgfqpoint{0.597096in}{0.767314in}}%
\pgfpathlineto{\pgfqpoint{0.595263in}{0.770806in}}%
\pgfpathlineto{\pgfqpoint{0.589872in}{0.780925in}}%
\pgfpathlineto{\pgfqpoint{0.579607in}{0.793472in}}%
\pgfpathlineto{\pgfqpoint{0.578611in}{0.794536in}}%
\pgfpathlineto{\pgfqpoint{0.563950in}{0.806818in}}%
\pgfpathlineto{\pgfqpoint{0.561863in}{0.808148in}}%
\pgfpathlineto{\pgfqpoint{0.548294in}{0.815890in}}%
\pgfpathlineto{\pgfqpoint{0.532637in}{0.821420in}}%
\pgfpathlineto{\pgfqpoint{0.530100in}{0.821759in}}%
\pgfpathlineto{\pgfqpoint{0.516981in}{0.823556in}}%
\pgfpathlineto{\pgfqpoint{0.501324in}{0.822124in}}%
\pgfpathlineto{\pgfqpoint{0.500159in}{0.821759in}}%
\pgfpathlineto{\pgfqpoint{0.485668in}{0.817274in}}%
\pgfpathlineto{\pgfqpoint{0.470011in}{0.808973in}}%
\pgfpathlineto{\pgfqpoint{0.468845in}{0.808148in}}%
\pgfpathlineto{\pgfqpoint{0.454354in}{0.796556in}}%
\pgfpathlineto{\pgfqpoint{0.452225in}{0.794536in}}%
\pgfpathlineto{\pgfqpoint{0.440948in}{0.780925in}}%
\pgfpathlineto{\pgfqpoint{0.438698in}{0.776798in}}%
\pgfpathlineto{\pgfqpoint{0.433709in}{0.767314in}}%
\pgfpathlineto{\pgfqpoint{0.430501in}{0.753703in}}%
\pgfpathlineto{\pgfqpoint{0.431302in}{0.740092in}}%
\pgfpathlineto{\pgfqpoint{0.436118in}{0.726481in}}%
\pgfpathlineto{\pgfqpoint{0.438698in}{0.722399in}}%
\pgfpathlineto{\pgfqpoint{0.444981in}{0.712870in}}%
\pgfpathlineto{\pgfqpoint{0.454354in}{0.702799in}}%
\pgfpathlineto{\pgfqpoint{0.458234in}{0.699259in}}%
\pgfpathlineto{\pgfqpoint{0.470011in}{0.690362in}}%
\pgfpathlineto{\pgfqpoint{0.478508in}{0.685648in}}%
\pgfpathlineto{\pgfqpoint{0.485668in}{0.681928in}}%
\pgfpathclose%
\pgfpathmoveto{\pgfqpoint{0.487318in}{0.699259in}}%
\pgfpathlineto{\pgfqpoint{0.485668in}{0.699926in}}%
\pgfpathlineto{\pgfqpoint{0.470011in}{0.710542in}}%
\pgfpathlineto{\pgfqpoint{0.467454in}{0.712870in}}%
\pgfpathlineto{\pgfqpoint{0.456658in}{0.726481in}}%
\pgfpathlineto{\pgfqpoint{0.454354in}{0.731744in}}%
\pgfpathlineto{\pgfqpoint{0.451149in}{0.740092in}}%
\pgfpathlineto{\pgfqpoint{0.450272in}{0.753703in}}%
\pgfpathlineto{\pgfqpoint{0.453783in}{0.767314in}}%
\pgfpathlineto{\pgfqpoint{0.454354in}{0.768326in}}%
\pgfpathlineto{\pgfqpoint{0.462547in}{0.780925in}}%
\pgfpathlineto{\pgfqpoint{0.470011in}{0.788492in}}%
\pgfpathlineto{\pgfqpoint{0.477994in}{0.794536in}}%
\pgfpathlineto{\pgfqpoint{0.485668in}{0.799172in}}%
\pgfpathlineto{\pgfqpoint{0.501324in}{0.804805in}}%
\pgfpathlineto{\pgfqpoint{0.516981in}{0.806411in}}%
\pgfpathlineto{\pgfqpoint{0.532637in}{0.804001in}}%
\pgfpathlineto{\pgfqpoint{0.548294in}{0.797561in}}%
\pgfpathlineto{\pgfqpoint{0.552965in}{0.794536in}}%
\pgfpathlineto{\pgfqpoint{0.563950in}{0.785592in}}%
\pgfpathlineto{\pgfqpoint{0.568368in}{0.780925in}}%
\pgfpathlineto{\pgfqpoint{0.576926in}{0.767314in}}%
\pgfpathlineto{\pgfqpoint{0.579607in}{0.757744in}}%
\pgfpathlineto{\pgfqpoint{0.580629in}{0.753703in}}%
\pgfpathlineto{\pgfqpoint{0.579760in}{0.740092in}}%
\pgfpathlineto{\pgfqpoint{0.579607in}{0.739685in}}%
\pgfpathlineto{\pgfqpoint{0.574075in}{0.726481in}}%
\pgfpathlineto{\pgfqpoint{0.563950in}{0.713295in}}%
\pgfpathlineto{\pgfqpoint{0.563531in}{0.712870in}}%
\pgfpathlineto{\pgfqpoint{0.548294in}{0.701695in}}%
\pgfpathlineto{\pgfqpoint{0.543002in}{0.699259in}}%
\pgfpathlineto{\pgfqpoint{0.532637in}{0.695198in}}%
\pgfpathlineto{\pgfqpoint{0.516981in}{0.692880in}}%
\pgfpathlineto{\pgfqpoint{0.501324in}{0.694425in}}%
\pgfpathlineto{\pgfqpoint{0.487318in}{0.699259in}}%
\pgfpathclose%
\pgfpathmoveto{\pgfqpoint{0.798799in}{0.680680in}}%
\pgfpathlineto{\pgfqpoint{0.814455in}{0.676525in}}%
\pgfpathlineto{\pgfqpoint{0.830112in}{0.675834in}}%
\pgfpathlineto{\pgfqpoint{0.845769in}{0.678602in}}%
\pgfpathlineto{\pgfqpoint{0.861425in}{0.684842in}}%
\pgfpathlineto{\pgfqpoint{0.862784in}{0.685648in}}%
\pgfpathlineto{\pgfqpoint{0.877082in}{0.694754in}}%
\pgfpathlineto{\pgfqpoint{0.882594in}{0.699259in}}%
\pgfpathlineto{\pgfqpoint{0.892738in}{0.709320in}}%
\pgfpathlineto{\pgfqpoint{0.895915in}{0.712870in}}%
\pgfpathlineto{\pgfqpoint{0.904668in}{0.726481in}}%
\pgfpathlineto{\pgfqpoint{0.908395in}{0.737073in}}%
\pgfpathlineto{\pgfqpoint{0.909474in}{0.740092in}}%
\pgfpathlineto{\pgfqpoint{0.910297in}{0.753703in}}%
\pgfpathlineto{\pgfqpoint{0.908395in}{0.761608in}}%
\pgfpathlineto{\pgfqpoint{0.907052in}{0.767314in}}%
\pgfpathlineto{\pgfqpoint{0.899893in}{0.780925in}}%
\pgfpathlineto{\pgfqpoint{0.892738in}{0.789828in}}%
\pgfpathlineto{\pgfqpoint{0.888499in}{0.794536in}}%
\pgfpathlineto{\pgfqpoint{0.877082in}{0.804462in}}%
\pgfpathlineto{\pgfqpoint{0.871666in}{0.808148in}}%
\pgfpathlineto{\pgfqpoint{0.861425in}{0.814368in}}%
\pgfpathlineto{\pgfqpoint{0.845769in}{0.820591in}}%
\pgfpathlineto{\pgfqpoint{0.839204in}{0.821759in}}%
\pgfpathlineto{\pgfqpoint{0.830112in}{0.823412in}}%
\pgfpathlineto{\pgfqpoint{0.814455in}{0.822697in}}%
\pgfpathlineto{\pgfqpoint{0.810983in}{0.821759in}}%
\pgfpathlineto{\pgfqpoint{0.798799in}{0.818518in}}%
\pgfpathlineto{\pgfqpoint{0.783142in}{0.810909in}}%
\pgfpathlineto{\pgfqpoint{0.779060in}{0.808148in}}%
\pgfpathlineto{\pgfqpoint{0.767486in}{0.799329in}}%
\pgfpathlineto{\pgfqpoint{0.762305in}{0.794536in}}%
\pgfpathlineto{\pgfqpoint{0.751829in}{0.782107in}}%
\pgfpathlineto{\pgfqpoint{0.750903in}{0.780925in}}%
\pgfpathlineto{\pgfqpoint{0.743724in}{0.767314in}}%
\pgfpathlineto{\pgfqpoint{0.740541in}{0.753703in}}%
\pgfpathlineto{\pgfqpoint{0.741336in}{0.740092in}}%
\pgfpathlineto{\pgfqpoint{0.746116in}{0.726481in}}%
\pgfpathlineto{\pgfqpoint{0.751829in}{0.717500in}}%
\pgfpathlineto{\pgfqpoint{0.754966in}{0.712870in}}%
\pgfpathlineto{\pgfqpoint{0.767486in}{0.699754in}}%
\pgfpathlineto{\pgfqpoint{0.768055in}{0.699259in}}%
\pgfpathlineto{\pgfqpoint{0.783142in}{0.688374in}}%
\pgfpathlineto{\pgfqpoint{0.788469in}{0.685648in}}%
\pgfpathlineto{\pgfqpoint{0.798799in}{0.680680in}}%
\pgfpathclose%
\pgfpathmoveto{\pgfqpoint{0.797258in}{0.699259in}}%
\pgfpathlineto{\pgfqpoint{0.783142in}{0.708066in}}%
\pgfpathlineto{\pgfqpoint{0.777617in}{0.712870in}}%
\pgfpathlineto{\pgfqpoint{0.767486in}{0.725141in}}%
\pgfpathlineto{\pgfqpoint{0.766554in}{0.726481in}}%
\pgfpathlineto{\pgfqpoint{0.761215in}{0.740092in}}%
\pgfpathlineto{\pgfqpoint{0.760326in}{0.753703in}}%
\pgfpathlineto{\pgfqpoint{0.763883in}{0.767314in}}%
\pgfpathlineto{\pgfqpoint{0.767486in}{0.773539in}}%
\pgfpathlineto{\pgfqpoint{0.772528in}{0.780925in}}%
\pgfpathlineto{\pgfqpoint{0.783142in}{0.791200in}}%
\pgfpathlineto{\pgfqpoint{0.787917in}{0.794536in}}%
\pgfpathlineto{\pgfqpoint{0.798799in}{0.800622in}}%
\pgfpathlineto{\pgfqpoint{0.814455in}{0.805447in}}%
\pgfpathlineto{\pgfqpoint{0.830112in}{0.806250in}}%
\pgfpathlineto{\pgfqpoint{0.845769in}{0.803036in}}%
\pgfpathlineto{\pgfqpoint{0.861425in}{0.795788in}}%
\pgfpathlineto{\pgfqpoint{0.863241in}{0.794536in}}%
\pgfpathlineto{\pgfqpoint{0.877082in}{0.782504in}}%
\pgfpathlineto{\pgfqpoint{0.878522in}{0.780925in}}%
\pgfpathlineto{\pgfqpoint{0.886858in}{0.767314in}}%
\pgfpathlineto{\pgfqpoint{0.890556in}{0.753703in}}%
\pgfpathlineto{\pgfqpoint{0.889632in}{0.740092in}}%
\pgfpathlineto{\pgfqpoint{0.884081in}{0.726481in}}%
\pgfpathlineto{\pgfqpoint{0.877082in}{0.717020in}}%
\pgfpathlineto{\pgfqpoint{0.873244in}{0.712870in}}%
\pgfpathlineto{\pgfqpoint{0.861425in}{0.703642in}}%
\pgfpathlineto{\pgfqpoint{0.852929in}{0.699259in}}%
\pgfpathlineto{\pgfqpoint{0.845769in}{0.696126in}}%
\pgfpathlineto{\pgfqpoint{0.830112in}{0.693034in}}%
\pgfpathlineto{\pgfqpoint{0.814455in}{0.693807in}}%
\pgfpathlineto{\pgfqpoint{0.798799in}{0.698449in}}%
\pgfpathlineto{\pgfqpoint{0.797258in}{0.699259in}}%
\pgfpathclose%
\pgfpathmoveto{\pgfqpoint{1.111930in}{0.679571in}}%
\pgfpathlineto{\pgfqpoint{1.127587in}{0.676110in}}%
\pgfpathlineto{\pgfqpoint{1.143243in}{0.676110in}}%
\pgfpathlineto{\pgfqpoint{1.158900in}{0.679571in}}%
\pgfpathlineto{\pgfqpoint{1.172676in}{0.685648in}}%
\pgfpathlineto{\pgfqpoint{1.174556in}{0.686528in}}%
\pgfpathlineto{\pgfqpoint{1.190213in}{0.697156in}}%
\pgfpathlineto{\pgfqpoint{1.192702in}{0.699259in}}%
\pgfpathlineto{\pgfqpoint{1.205870in}{0.712785in}}%
\pgfpathlineto{\pgfqpoint{1.205944in}{0.712870in}}%
\pgfpathlineto{\pgfqpoint{1.214691in}{0.726481in}}%
\pgfpathlineto{\pgfqpoint{1.219455in}{0.740092in}}%
\pgfpathlineto{\pgfqpoint{1.220247in}{0.753703in}}%
\pgfpathlineto{\pgfqpoint{1.217074in}{0.767314in}}%
\pgfpathlineto{\pgfqpoint{1.209920in}{0.780925in}}%
\pgfpathlineto{\pgfqpoint{1.205870in}{0.786037in}}%
\pgfpathlineto{\pgfqpoint{1.198481in}{0.794536in}}%
\pgfpathlineto{\pgfqpoint{1.190213in}{0.801965in}}%
\pgfpathlineto{\pgfqpoint{1.181651in}{0.808148in}}%
\pgfpathlineto{\pgfqpoint{1.174556in}{0.812707in}}%
\pgfpathlineto{\pgfqpoint{1.158900in}{0.819624in}}%
\pgfpathlineto{\pgfqpoint{1.149286in}{0.821759in}}%
\pgfpathlineto{\pgfqpoint{1.143243in}{0.823126in}}%
\pgfpathlineto{\pgfqpoint{1.127587in}{0.823126in}}%
\pgfpathlineto{\pgfqpoint{1.121544in}{0.821759in}}%
\pgfpathlineto{\pgfqpoint{1.111930in}{0.819624in}}%
\pgfpathlineto{\pgfqpoint{1.096274in}{0.812707in}}%
\pgfpathlineto{\pgfqpoint{1.089179in}{0.808147in}}%
\pgfpathlineto{\pgfqpoint{1.080617in}{0.801965in}}%
\pgfpathlineto{\pgfqpoint{1.072349in}{0.794536in}}%
\pgfpathlineto{\pgfqpoint{1.064960in}{0.786037in}}%
\pgfpathlineto{\pgfqpoint{1.060910in}{0.780925in}}%
\pgfpathlineto{\pgfqpoint{1.053756in}{0.767314in}}%
\pgfpathlineto{\pgfqpoint{1.050583in}{0.753703in}}%
\pgfpathlineto{\pgfqpoint{1.051375in}{0.740092in}}%
\pgfpathlineto{\pgfqpoint{1.056139in}{0.726481in}}%
\pgfpathlineto{\pgfqpoint{1.064886in}{0.712870in}}%
\pgfpathlineto{\pgfqpoint{1.064960in}{0.712785in}}%
\pgfpathlineto{\pgfqpoint{1.078128in}{0.699259in}}%
\pgfpathlineto{\pgfqpoint{1.080617in}{0.697156in}}%
\pgfpathlineto{\pgfqpoint{1.096274in}{0.686528in}}%
\pgfpathlineto{\pgfqpoint{1.098154in}{0.685648in}}%
\pgfpathlineto{\pgfqpoint{1.111930in}{0.679571in}}%
\pgfpathclose%
\pgfpathmoveto{\pgfqpoint{1.107680in}{0.699259in}}%
\pgfpathlineto{\pgfqpoint{1.096274in}{0.705766in}}%
\pgfpathlineto{\pgfqpoint{1.087675in}{0.712870in}}%
\pgfpathlineto{\pgfqpoint{1.080617in}{0.720971in}}%
\pgfpathlineto{\pgfqpoint{1.076672in}{0.726481in}}%
\pgfpathlineto{\pgfqpoint{1.071240in}{0.740092in}}%
\pgfpathlineto{\pgfqpoint{1.070336in}{0.753703in}}%
\pgfpathlineto{\pgfqpoint{1.073955in}{0.767314in}}%
\pgfpathlineto{\pgfqpoint{1.080617in}{0.778495in}}%
\pgfpathlineto{\pgfqpoint{1.082368in}{0.780925in}}%
\pgfpathlineto{\pgfqpoint{1.096274in}{0.793717in}}%
\pgfpathlineto{\pgfqpoint{1.097556in}{0.794536in}}%
\pgfpathlineto{\pgfqpoint{1.111930in}{0.801910in}}%
\pgfpathlineto{\pgfqpoint{1.127587in}{0.805929in}}%
\pgfpathlineto{\pgfqpoint{1.143243in}{0.805929in}}%
\pgfpathlineto{\pgfqpoint{1.158900in}{0.801910in}}%
\pgfpathlineto{\pgfqpoint{1.173274in}{0.794536in}}%
\pgfpathlineto{\pgfqpoint{1.174556in}{0.793717in}}%
\pgfpathlineto{\pgfqpoint{1.188462in}{0.780925in}}%
\pgfpathlineto{\pgfqpoint{1.190213in}{0.778495in}}%
\pgfpathlineto{\pgfqpoint{1.196875in}{0.767314in}}%
\pgfpathlineto{\pgfqpoint{1.200494in}{0.753703in}}%
\pgfpathlineto{\pgfqpoint{1.199590in}{0.740092in}}%
\pgfpathlineto{\pgfqpoint{1.194158in}{0.726481in}}%
\pgfpathlineto{\pgfqpoint{1.190213in}{0.720971in}}%
\pgfpathlineto{\pgfqpoint{1.183155in}{0.712870in}}%
\pgfpathlineto{\pgfqpoint{1.174556in}{0.705766in}}%
\pgfpathlineto{\pgfqpoint{1.163150in}{0.699259in}}%
\pgfpathlineto{\pgfqpoint{1.158900in}{0.697210in}}%
\pgfpathlineto{\pgfqpoint{1.143243in}{0.693343in}}%
\pgfpathlineto{\pgfqpoint{1.127587in}{0.693343in}}%
\pgfpathlineto{\pgfqpoint{1.111930in}{0.697210in}}%
\pgfpathlineto{\pgfqpoint{1.107680in}{0.699259in}}%
\pgfpathclose%
\pgfpathmoveto{\pgfqpoint{1.409405in}{0.684842in}}%
\pgfpathlineto{\pgfqpoint{1.425061in}{0.678602in}}%
\pgfpathlineto{\pgfqpoint{1.440718in}{0.675834in}}%
\pgfpathlineto{\pgfqpoint{1.456375in}{0.676525in}}%
\pgfpathlineto{\pgfqpoint{1.472031in}{0.680680in}}%
\pgfpathlineto{\pgfqpoint{1.482361in}{0.685648in}}%
\pgfpathlineto{\pgfqpoint{1.487688in}{0.688374in}}%
\pgfpathlineto{\pgfqpoint{1.502775in}{0.699259in}}%
\pgfpathlineto{\pgfqpoint{1.503344in}{0.699754in}}%
\pgfpathlineto{\pgfqpoint{1.515864in}{0.712870in}}%
\pgfpathlineto{\pgfqpoint{1.519001in}{0.717500in}}%
\pgfpathlineto{\pgfqpoint{1.524714in}{0.726481in}}%
\pgfpathlineto{\pgfqpoint{1.529494in}{0.740092in}}%
\pgfpathlineto{\pgfqpoint{1.530289in}{0.753703in}}%
\pgfpathlineto{\pgfqpoint{1.527106in}{0.767314in}}%
\pgfpathlineto{\pgfqpoint{1.519927in}{0.780925in}}%
\pgfpathlineto{\pgfqpoint{1.519001in}{0.782107in}}%
\pgfpathlineto{\pgfqpoint{1.508525in}{0.794536in}}%
\pgfpathlineto{\pgfqpoint{1.503344in}{0.799329in}}%
\pgfpathlineto{\pgfqpoint{1.491770in}{0.808148in}}%
\pgfpathlineto{\pgfqpoint{1.487688in}{0.810909in}}%
\pgfpathlineto{\pgfqpoint{1.472031in}{0.818518in}}%
\pgfpathlineto{\pgfqpoint{1.459847in}{0.821759in}}%
\pgfpathlineto{\pgfqpoint{1.456375in}{0.822697in}}%
\pgfpathlineto{\pgfqpoint{1.440718in}{0.823412in}}%
\pgfpathlineto{\pgfqpoint{1.431626in}{0.821759in}}%
\pgfpathlineto{\pgfqpoint{1.425061in}{0.820591in}}%
\pgfpathlineto{\pgfqpoint{1.409405in}{0.814368in}}%
\pgfpathlineto{\pgfqpoint{1.399164in}{0.808148in}}%
\pgfpathlineto{\pgfqpoint{1.393748in}{0.804462in}}%
\pgfpathlineto{\pgfqpoint{1.382331in}{0.794536in}}%
\pgfpathlineto{\pgfqpoint{1.378092in}{0.789828in}}%
\pgfpathlineto{\pgfqpoint{1.370937in}{0.780925in}}%
\pgfpathlineto{\pgfqpoint{1.363778in}{0.767314in}}%
\pgfpathlineto{\pgfqpoint{1.362435in}{0.761608in}}%
\pgfpathlineto{\pgfqpoint{1.360533in}{0.753703in}}%
\pgfpathlineto{\pgfqpoint{1.361356in}{0.740092in}}%
\pgfpathlineto{\pgfqpoint{1.362435in}{0.737073in}}%
\pgfpathlineto{\pgfqpoint{1.366162in}{0.726481in}}%
\pgfpathlineto{\pgfqpoint{1.374915in}{0.712870in}}%
\pgfpathlineto{\pgfqpoint{1.378092in}{0.709320in}}%
\pgfpathlineto{\pgfqpoint{1.388236in}{0.699259in}}%
\pgfpathlineto{\pgfqpoint{1.393748in}{0.694754in}}%
\pgfpathlineto{\pgfqpoint{1.408046in}{0.685648in}}%
\pgfpathlineto{\pgfqpoint{1.409405in}{0.684842in}}%
\pgfpathclose%
\pgfpathmoveto{\pgfqpoint{1.417901in}{0.699259in}}%
\pgfpathlineto{\pgfqpoint{1.409405in}{0.703642in}}%
\pgfpathlineto{\pgfqpoint{1.397586in}{0.712870in}}%
\pgfpathlineto{\pgfqpoint{1.393748in}{0.717020in}}%
\pgfpathlineto{\pgfqpoint{1.386749in}{0.726481in}}%
\pgfpathlineto{\pgfqpoint{1.381198in}{0.740092in}}%
\pgfpathlineto{\pgfqpoint{1.380274in}{0.753703in}}%
\pgfpathlineto{\pgfqpoint{1.383972in}{0.767314in}}%
\pgfpathlineto{\pgfqpoint{1.392308in}{0.780925in}}%
\pgfpathlineto{\pgfqpoint{1.393748in}{0.782504in}}%
\pgfpathlineto{\pgfqpoint{1.407589in}{0.794536in}}%
\pgfpathlineto{\pgfqpoint{1.409405in}{0.795788in}}%
\pgfpathlineto{\pgfqpoint{1.425061in}{0.803036in}}%
\pgfpathlineto{\pgfqpoint{1.440718in}{0.806250in}}%
\pgfpathlineto{\pgfqpoint{1.456375in}{0.805447in}}%
\pgfpathlineto{\pgfqpoint{1.472031in}{0.800622in}}%
\pgfpathlineto{\pgfqpoint{1.482913in}{0.794536in}}%
\pgfpathlineto{\pgfqpoint{1.487688in}{0.791200in}}%
\pgfpathlineto{\pgfqpoint{1.498302in}{0.780925in}}%
\pgfpathlineto{\pgfqpoint{1.503344in}{0.773539in}}%
\pgfpathlineto{\pgfqpoint{1.506947in}{0.767314in}}%
\pgfpathlineto{\pgfqpoint{1.510504in}{0.753703in}}%
\pgfpathlineto{\pgfqpoint{1.509615in}{0.740092in}}%
\pgfpathlineto{\pgfqpoint{1.504276in}{0.726481in}}%
\pgfpathlineto{\pgfqpoint{1.503344in}{0.725141in}}%
\pgfpathlineto{\pgfqpoint{1.493213in}{0.712870in}}%
\pgfpathlineto{\pgfqpoint{1.487688in}{0.708066in}}%
\pgfpathlineto{\pgfqpoint{1.473572in}{0.699259in}}%
\pgfpathlineto{\pgfqpoint{1.472031in}{0.698449in}}%
\pgfpathlineto{\pgfqpoint{1.456375in}{0.693807in}}%
\pgfpathlineto{\pgfqpoint{1.440718in}{0.693034in}}%
\pgfpathlineto{\pgfqpoint{1.425061in}{0.696126in}}%
\pgfpathlineto{\pgfqpoint{1.417901in}{0.699259in}}%
\pgfpathclose%
\pgfpathmoveto{\pgfqpoint{1.722536in}{0.683316in}}%
\pgfpathlineto{\pgfqpoint{1.738193in}{0.677771in}}%
\pgfpathlineto{\pgfqpoint{1.753849in}{0.675695in}}%
\pgfpathlineto{\pgfqpoint{1.769506in}{0.677079in}}%
\pgfpathlineto{\pgfqpoint{1.785162in}{0.681928in}}%
\pgfpathlineto{\pgfqpoint{1.792322in}{0.685648in}}%
\pgfpathlineto{\pgfqpoint{1.800819in}{0.690362in}}%
\pgfpathlineto{\pgfqpoint{1.812596in}{0.699259in}}%
\pgfpathlineto{\pgfqpoint{1.816476in}{0.702799in}}%
\pgfpathlineto{\pgfqpoint{1.825849in}{0.712870in}}%
\pgfpathlineto{\pgfqpoint{1.832132in}{0.722399in}}%
\pgfpathlineto{\pgfqpoint{1.834712in}{0.726481in}}%
\pgfpathlineto{\pgfqpoint{1.839528in}{0.740092in}}%
\pgfpathlineto{\pgfqpoint{1.840329in}{0.753703in}}%
\pgfpathlineto{\pgfqpoint{1.837121in}{0.767314in}}%
\pgfpathlineto{\pgfqpoint{1.832132in}{0.776798in}}%
\pgfpathlineto{\pgfqpoint{1.829882in}{0.780925in}}%
\pgfpathlineto{\pgfqpoint{1.818605in}{0.794536in}}%
\pgfpathlineto{\pgfqpoint{1.816476in}{0.796556in}}%
\pgfpathlineto{\pgfqpoint{1.801985in}{0.808147in}}%
\pgfpathlineto{\pgfqpoint{1.800819in}{0.808973in}}%
\pgfpathlineto{\pgfqpoint{1.785162in}{0.817274in}}%
\pgfpathlineto{\pgfqpoint{1.770671in}{0.821759in}}%
\pgfpathlineto{\pgfqpoint{1.769506in}{0.822124in}}%
\pgfpathlineto{\pgfqpoint{1.753849in}{0.823556in}}%
\pgfpathlineto{\pgfqpoint{1.740730in}{0.821759in}}%
\pgfpathlineto{\pgfqpoint{1.738193in}{0.821420in}}%
\pgfpathlineto{\pgfqpoint{1.722536in}{0.815890in}}%
\pgfpathlineto{\pgfqpoint{1.708967in}{0.808148in}}%
\pgfpathlineto{\pgfqpoint{1.706880in}{0.806818in}}%
\pgfpathlineto{\pgfqpoint{1.692219in}{0.794536in}}%
\pgfpathlineto{\pgfqpoint{1.691223in}{0.793472in}}%
\pgfpathlineto{\pgfqpoint{1.680958in}{0.780925in}}%
\pgfpathlineto{\pgfqpoint{1.675567in}{0.770806in}}%
\pgfpathlineto{\pgfqpoint{1.673734in}{0.767314in}}%
\pgfpathlineto{\pgfqpoint{1.670490in}{0.753703in}}%
\pgfpathlineto{\pgfqpoint{1.671300in}{0.740092in}}%
\pgfpathlineto{\pgfqpoint{1.675567in}{0.728139in}}%
\pgfpathlineto{\pgfqpoint{1.676161in}{0.726481in}}%
\pgfpathlineto{\pgfqpoint{1.684955in}{0.712870in}}%
\pgfpathlineto{\pgfqpoint{1.691223in}{0.705990in}}%
\pgfpathlineto{\pgfqpoint{1.698282in}{0.699259in}}%
\pgfpathlineto{\pgfqpoint{1.706880in}{0.692489in}}%
\pgfpathlineto{\pgfqpoint{1.718348in}{0.685648in}}%
\pgfpathlineto{\pgfqpoint{1.722536in}{0.683316in}}%
\pgfpathclose%
\pgfpathmoveto{\pgfqpoint{1.727828in}{0.699259in}}%
\pgfpathlineto{\pgfqpoint{1.722536in}{0.701695in}}%
\pgfpathlineto{\pgfqpoint{1.707299in}{0.712870in}}%
\pgfpathlineto{\pgfqpoint{1.706880in}{0.713295in}}%
\pgfpathlineto{\pgfqpoint{1.696755in}{0.726481in}}%
\pgfpathlineto{\pgfqpoint{1.691223in}{0.739685in}}%
\pgfpathlineto{\pgfqpoint{1.691070in}{0.740092in}}%
\pgfpathlineto{\pgfqpoint{1.690201in}{0.753703in}}%
\pgfpathlineto{\pgfqpoint{1.691223in}{0.757744in}}%
\pgfpathlineto{\pgfqpoint{1.693904in}{0.767314in}}%
\pgfpathlineto{\pgfqpoint{1.702462in}{0.780925in}}%
\pgfpathlineto{\pgfqpoint{1.706880in}{0.785592in}}%
\pgfpathlineto{\pgfqpoint{1.717865in}{0.794536in}}%
\pgfpathlineto{\pgfqpoint{1.722536in}{0.797561in}}%
\pgfpathlineto{\pgfqpoint{1.738193in}{0.804001in}}%
\pgfpathlineto{\pgfqpoint{1.753849in}{0.806411in}}%
\pgfpathlineto{\pgfqpoint{1.769506in}{0.804805in}}%
\pgfpathlineto{\pgfqpoint{1.785162in}{0.799172in}}%
\pgfpathlineto{\pgfqpoint{1.792836in}{0.794536in}}%
\pgfpathlineto{\pgfqpoint{1.800819in}{0.788492in}}%
\pgfpathlineto{\pgfqpoint{1.808283in}{0.780925in}}%
\pgfpathlineto{\pgfqpoint{1.816476in}{0.768326in}}%
\pgfpathlineto{\pgfqpoint{1.817047in}{0.767314in}}%
\pgfpathlineto{\pgfqpoint{1.820558in}{0.753703in}}%
\pgfpathlineto{\pgfqpoint{1.819681in}{0.740092in}}%
\pgfpathlineto{\pgfqpoint{1.816476in}{0.731744in}}%
\pgfpathlineto{\pgfqpoint{1.814172in}{0.726481in}}%
\pgfpathlineto{\pgfqpoint{1.803376in}{0.712870in}}%
\pgfpathlineto{\pgfqpoint{1.800819in}{0.710542in}}%
\pgfpathlineto{\pgfqpoint{1.785162in}{0.699926in}}%
\pgfpathlineto{\pgfqpoint{1.783512in}{0.699259in}}%
\pgfpathlineto{\pgfqpoint{1.769506in}{0.694425in}}%
\pgfpathlineto{\pgfqpoint{1.753849in}{0.692880in}}%
\pgfpathlineto{\pgfqpoint{1.738193in}{0.695198in}}%
\pgfpathlineto{\pgfqpoint{1.727828in}{0.699259in}}%
\pgfpathclose%
\pgfpathmoveto{\pgfqpoint{0.485668in}{0.951445in}}%
\pgfpathlineto{\pgfqpoint{0.501324in}{0.946611in}}%
\pgfpathlineto{\pgfqpoint{0.516981in}{0.945233in}}%
\pgfpathlineto{\pgfqpoint{0.532637in}{0.947301in}}%
\pgfpathlineto{\pgfqpoint{0.548294in}{0.952827in}}%
\pgfpathlineto{\pgfqpoint{0.557257in}{0.957870in}}%
\pgfpathlineto{\pgfqpoint{0.563950in}{0.961988in}}%
\pgfpathlineto{\pgfqpoint{0.575662in}{0.971481in}}%
\pgfpathlineto{\pgfqpoint{0.579607in}{0.975451in}}%
\pgfpathlineto{\pgfqpoint{0.587953in}{0.985092in}}%
\pgfpathlineto{\pgfqpoint{0.595263in}{0.997501in}}%
\pgfpathlineto{\pgfqpoint{0.595959in}{0.998703in}}%
\pgfpathlineto{\pgfqpoint{0.600016in}{1.012314in}}%
\pgfpathlineto{\pgfqpoint{0.600016in}{1.025925in}}%
\pgfpathlineto{\pgfqpoint{0.595959in}{1.039536in}}%
\pgfpathlineto{\pgfqpoint{0.595263in}{1.040739in}}%
\pgfpathlineto{\pgfqpoint{0.587953in}{1.053148in}}%
\pgfpathlineto{\pgfqpoint{0.579607in}{1.062789in}}%
\pgfpathlineto{\pgfqpoint{0.575662in}{1.066759in}}%
\pgfpathlineto{\pgfqpoint{0.563950in}{1.076251in}}%
\pgfpathlineto{\pgfqpoint{0.557257in}{1.080370in}}%
\pgfpathlineto{\pgfqpoint{0.548294in}{1.085412in}}%
\pgfpathlineto{\pgfqpoint{0.532637in}{1.090939in}}%
\pgfpathlineto{\pgfqpoint{0.516981in}{1.093007in}}%
\pgfpathlineto{\pgfqpoint{0.501324in}{1.091628in}}%
\pgfpathlineto{\pgfqpoint{0.485668in}{1.086795in}}%
\pgfpathlineto{\pgfqpoint{0.473426in}{1.080370in}}%
\pgfpathlineto{\pgfqpoint{0.470011in}{1.078415in}}%
\pgfpathlineto{\pgfqpoint{0.455022in}{1.066759in}}%
\pgfpathlineto{\pgfqpoint{0.454354in}{1.066116in}}%
\pgfpathlineto{\pgfqpoint{0.442884in}{1.053148in}}%
\pgfpathlineto{\pgfqpoint{0.438698in}{1.046203in}}%
\pgfpathlineto{\pgfqpoint{0.434833in}{1.039536in}}%
\pgfpathlineto{\pgfqpoint{0.430822in}{1.025925in}}%
\pgfpathlineto{\pgfqpoint{0.430822in}{1.012314in}}%
\pgfpathlineto{\pgfqpoint{0.434833in}{0.998703in}}%
\pgfpathlineto{\pgfqpoint{0.438698in}{0.992036in}}%
\pgfpathlineto{\pgfqpoint{0.442884in}{0.985092in}}%
\pgfpathlineto{\pgfqpoint{0.454354in}{0.972124in}}%
\pgfpathlineto{\pgfqpoint{0.455022in}{0.971481in}}%
\pgfpathlineto{\pgfqpoint{0.470011in}{0.959824in}}%
\pgfpathlineto{\pgfqpoint{0.473426in}{0.957870in}}%
\pgfpathlineto{\pgfqpoint{0.485668in}{0.951445in}}%
\pgfpathclose%
\pgfpathmoveto{\pgfqpoint{0.482227in}{0.971481in}}%
\pgfpathlineto{\pgfqpoint{0.470011in}{0.980198in}}%
\pgfpathlineto{\pgfqpoint{0.464902in}{0.985092in}}%
\pgfpathlineto{\pgfqpoint{0.455089in}{0.998703in}}%
\pgfpathlineto{\pgfqpoint{0.454354in}{1.000714in}}%
\pgfpathlineto{\pgfqpoint{0.450623in}{1.012314in}}%
\pgfpathlineto{\pgfqpoint{0.450623in}{1.025925in}}%
\pgfpathlineto{\pgfqpoint{0.454354in}{1.037525in}}%
\pgfpathlineto{\pgfqpoint{0.455089in}{1.039536in}}%
\pgfpathlineto{\pgfqpoint{0.464902in}{1.053148in}}%
\pgfpathlineto{\pgfqpoint{0.470011in}{1.058041in}}%
\pgfpathlineto{\pgfqpoint{0.482227in}{1.066759in}}%
\pgfpathlineto{\pgfqpoint{0.485668in}{1.068770in}}%
\pgfpathlineto{\pgfqpoint{0.501324in}{1.074281in}}%
\pgfpathlineto{\pgfqpoint{0.516981in}{1.075853in}}%
\pgfpathlineto{\pgfqpoint{0.532637in}{1.073495in}}%
\pgfpathlineto{\pgfqpoint{0.548294in}{1.067193in}}%
\pgfpathlineto{\pgfqpoint{0.548987in}{1.066759in}}%
\pgfpathlineto{\pgfqpoint{0.563950in}{1.055278in}}%
\pgfpathlineto{\pgfqpoint{0.566084in}{1.053148in}}%
\pgfpathlineto{\pgfqpoint{0.575596in}{1.039536in}}%
\pgfpathlineto{\pgfqpoint{0.579607in}{1.028069in}}%
\pgfpathlineto{\pgfqpoint{0.580282in}{1.025925in}}%
\pgfpathlineto{\pgfqpoint{0.580282in}{1.012314in}}%
\pgfpathlineto{\pgfqpoint{0.579607in}{1.010171in}}%
\pgfpathlineto{\pgfqpoint{0.575596in}{0.998703in}}%
\pgfpathlineto{\pgfqpoint{0.566084in}{0.985092in}}%
\pgfpathlineto{\pgfqpoint{0.563950in}{0.982961in}}%
\pgfpathlineto{\pgfqpoint{0.548987in}{0.971481in}}%
\pgfpathlineto{\pgfqpoint{0.548294in}{0.971046in}}%
\pgfpathlineto{\pgfqpoint{0.532637in}{0.964745in}}%
\pgfpathlineto{\pgfqpoint{0.516981in}{0.962386in}}%
\pgfpathlineto{\pgfqpoint{0.501324in}{0.963958in}}%
\pgfpathlineto{\pgfqpoint{0.485668in}{0.969470in}}%
\pgfpathlineto{\pgfqpoint{0.482227in}{0.971481in}}%
\pgfpathclose%
\pgfpathmoveto{\pgfqpoint{0.783142in}{0.957805in}}%
\pgfpathlineto{\pgfqpoint{0.798799in}{0.950201in}}%
\pgfpathlineto{\pgfqpoint{0.814455in}{0.946059in}}%
\pgfpathlineto{\pgfqpoint{0.830112in}{0.945370in}}%
\pgfpathlineto{\pgfqpoint{0.845769in}{0.948129in}}%
\pgfpathlineto{\pgfqpoint{0.861425in}{0.954348in}}%
\pgfpathlineto{\pgfqpoint{0.867305in}{0.957870in}}%
\pgfpathlineto{\pgfqpoint{0.877082in}{0.964293in}}%
\pgfpathlineto{\pgfqpoint{0.885627in}{0.971481in}}%
\pgfpathlineto{\pgfqpoint{0.892738in}{0.978924in}}%
\pgfpathlineto{\pgfqpoint{0.897983in}{0.985092in}}%
\pgfpathlineto{\pgfqpoint{0.905940in}{0.998703in}}%
\pgfpathlineto{\pgfqpoint{0.908395in}{1.007061in}}%
\pgfpathlineto{\pgfqpoint{0.909968in}{1.012314in}}%
\pgfpathlineto{\pgfqpoint{0.909968in}{1.025925in}}%
\pgfpathlineto{\pgfqpoint{0.908395in}{1.031179in}}%
\pgfpathlineto{\pgfqpoint{0.905940in}{1.039536in}}%
\pgfpathlineto{\pgfqpoint{0.897983in}{1.053148in}}%
\pgfpathlineto{\pgfqpoint{0.892738in}{1.059315in}}%
\pgfpathlineto{\pgfqpoint{0.885627in}{1.066759in}}%
\pgfpathlineto{\pgfqpoint{0.877082in}{1.073946in}}%
\pgfpathlineto{\pgfqpoint{0.867305in}{1.080370in}}%
\pgfpathlineto{\pgfqpoint{0.861425in}{1.083891in}}%
\pgfpathlineto{\pgfqpoint{0.845769in}{1.090111in}}%
\pgfpathlineto{\pgfqpoint{0.830112in}{1.092869in}}%
\pgfpathlineto{\pgfqpoint{0.814455in}{1.092180in}}%
\pgfpathlineto{\pgfqpoint{0.798799in}{1.088039in}}%
\pgfpathlineto{\pgfqpoint{0.783142in}{1.080434in}}%
\pgfpathlineto{\pgfqpoint{0.783045in}{1.080370in}}%
\pgfpathlineto{\pgfqpoint{0.767486in}{1.068923in}}%
\pgfpathlineto{\pgfqpoint{0.765068in}{1.066759in}}%
\pgfpathlineto{\pgfqpoint{0.752842in}{1.053148in}}%
\pgfpathlineto{\pgfqpoint{0.751829in}{1.051513in}}%
\pgfpathlineto{\pgfqpoint{0.744840in}{1.039536in}}%
\pgfpathlineto{\pgfqpoint{0.740859in}{1.025925in}}%
\pgfpathlineto{\pgfqpoint{0.740859in}{1.012314in}}%
\pgfpathlineto{\pgfqpoint{0.744840in}{0.998703in}}%
\pgfpathlineto{\pgfqpoint{0.751829in}{0.986727in}}%
\pgfpathlineto{\pgfqpoint{0.752842in}{0.985092in}}%
\pgfpathlineto{\pgfqpoint{0.765068in}{0.971481in}}%
\pgfpathlineto{\pgfqpoint{0.767486in}{0.969317in}}%
\pgfpathlineto{\pgfqpoint{0.783045in}{0.957870in}}%
\pgfpathlineto{\pgfqpoint{0.783142in}{0.957805in}}%
\pgfpathclose%
\pgfpathmoveto{\pgfqpoint{0.792460in}{0.971481in}}%
\pgfpathlineto{\pgfqpoint{0.783142in}{0.977616in}}%
\pgfpathlineto{\pgfqpoint{0.774971in}{0.985092in}}%
\pgfpathlineto{\pgfqpoint{0.767486in}{0.995009in}}%
\pgfpathlineto{\pgfqpoint{0.765129in}{0.998703in}}%
\pgfpathlineto{\pgfqpoint{0.760682in}{1.012314in}}%
\pgfpathlineto{\pgfqpoint{0.760682in}{1.025925in}}%
\pgfpathlineto{\pgfqpoint{0.765129in}{1.039536in}}%
\pgfpathlineto{\pgfqpoint{0.767486in}{1.043231in}}%
\pgfpathlineto{\pgfqpoint{0.774971in}{1.053148in}}%
\pgfpathlineto{\pgfqpoint{0.783142in}{1.060623in}}%
\pgfpathlineto{\pgfqpoint{0.792460in}{1.066759in}}%
\pgfpathlineto{\pgfqpoint{0.798799in}{1.070188in}}%
\pgfpathlineto{\pgfqpoint{0.814455in}{1.074910in}}%
\pgfpathlineto{\pgfqpoint{0.830112in}{1.075696in}}%
\pgfpathlineto{\pgfqpoint{0.845769in}{1.072551in}}%
\pgfpathlineto{\pgfqpoint{0.858630in}{1.066759in}}%
\pgfpathlineto{\pgfqpoint{0.861425in}{1.065237in}}%
\pgfpathlineto{\pgfqpoint{0.876139in}{1.053148in}}%
\pgfpathlineto{\pgfqpoint{0.877082in}{1.052033in}}%
\pgfpathlineto{\pgfqpoint{0.885563in}{1.039536in}}%
\pgfpathlineto{\pgfqpoint{0.890187in}{1.025925in}}%
\pgfpathlineto{\pgfqpoint{0.890187in}{1.012314in}}%
\pgfpathlineto{\pgfqpoint{0.885563in}{0.998703in}}%
\pgfpathlineto{\pgfqpoint{0.877082in}{0.986207in}}%
\pgfpathlineto{\pgfqpoint{0.876139in}{0.985092in}}%
\pgfpathlineto{\pgfqpoint{0.861425in}{0.973003in}}%
\pgfpathlineto{\pgfqpoint{0.858630in}{0.971481in}}%
\pgfpathlineto{\pgfqpoint{0.845769in}{0.965689in}}%
\pgfpathlineto{\pgfqpoint{0.830112in}{0.962543in}}%
\pgfpathlineto{\pgfqpoint{0.814455in}{0.963329in}}%
\pgfpathlineto{\pgfqpoint{0.798799in}{0.968051in}}%
\pgfpathlineto{\pgfqpoint{0.792460in}{0.971481in}}%
\pgfpathclose%
\pgfpathmoveto{\pgfqpoint{1.096274in}{0.956008in}}%
\pgfpathlineto{\pgfqpoint{1.111930in}{0.949095in}}%
\pgfpathlineto{\pgfqpoint{1.127587in}{0.945646in}}%
\pgfpathlineto{\pgfqpoint{1.143243in}{0.945646in}}%
\pgfpathlineto{\pgfqpoint{1.158900in}{0.949095in}}%
\pgfpathlineto{\pgfqpoint{1.174556in}{0.956008in}}%
\pgfpathlineto{\pgfqpoint{1.177494in}{0.957870in}}%
\pgfpathlineto{\pgfqpoint{1.190213in}{0.966737in}}%
\pgfpathlineto{\pgfqpoint{1.195670in}{0.971481in}}%
\pgfpathlineto{\pgfqpoint{1.205870in}{0.982538in}}%
\pgfpathlineto{\pgfqpoint{1.208011in}{0.985092in}}%
\pgfpathlineto{\pgfqpoint{1.215962in}{0.998703in}}%
\pgfpathlineto{\pgfqpoint{1.219930in}{1.012314in}}%
\pgfpathlineto{\pgfqpoint{1.219930in}{1.025925in}}%
\pgfpathlineto{\pgfqpoint{1.215962in}{1.039536in}}%
\pgfpathlineto{\pgfqpoint{1.208011in}{1.053148in}}%
\pgfpathlineto{\pgfqpoint{1.205870in}{1.055702in}}%
\pgfpathlineto{\pgfqpoint{1.195670in}{1.066759in}}%
\pgfpathlineto{\pgfqpoint{1.190213in}{1.071503in}}%
\pgfpathlineto{\pgfqpoint{1.177494in}{1.080370in}}%
\pgfpathlineto{\pgfqpoint{1.174556in}{1.082232in}}%
\pgfpathlineto{\pgfqpoint{1.158900in}{1.089144in}}%
\pgfpathlineto{\pgfqpoint{1.143243in}{1.092593in}}%
\pgfpathlineto{\pgfqpoint{1.127587in}{1.092593in}}%
\pgfpathlineto{\pgfqpoint{1.111930in}{1.089144in}}%
\pgfpathlineto{\pgfqpoint{1.096274in}{1.082232in}}%
\pgfpathlineto{\pgfqpoint{1.093336in}{1.080370in}}%
\pgfpathlineto{\pgfqpoint{1.080617in}{1.071503in}}%
\pgfpathlineto{\pgfqpoint{1.075160in}{1.066759in}}%
\pgfpathlineto{\pgfqpoint{1.064960in}{1.055702in}}%
\pgfpathlineto{\pgfqpoint{1.062819in}{1.053148in}}%
\pgfpathlineto{\pgfqpoint{1.054868in}{1.039536in}}%
\pgfpathlineto{\pgfqpoint{1.050900in}{1.025925in}}%
\pgfpathlineto{\pgfqpoint{1.050900in}{1.012314in}}%
\pgfpathlineto{\pgfqpoint{1.054868in}{0.998703in}}%
\pgfpathlineto{\pgfqpoint{1.062819in}{0.985092in}}%
\pgfpathlineto{\pgfqpoint{1.064960in}{0.982538in}}%
\pgfpathlineto{\pgfqpoint{1.075160in}{0.971481in}}%
\pgfpathlineto{\pgfqpoint{1.080617in}{0.966737in}}%
\pgfpathlineto{\pgfqpoint{1.093336in}{0.957870in}}%
\pgfpathlineto{\pgfqpoint{1.096274in}{0.956008in}}%
\pgfpathclose%
\pgfpathmoveto{\pgfqpoint{1.102481in}{0.971481in}}%
\pgfpathlineto{\pgfqpoint{1.096274in}{0.975218in}}%
\pgfpathlineto{\pgfqpoint{1.084915in}{0.985092in}}%
\pgfpathlineto{\pgfqpoint{1.080617in}{0.990488in}}%
\pgfpathlineto{\pgfqpoint{1.075222in}{0.998703in}}%
\pgfpathlineto{\pgfqpoint{1.070698in}{1.012314in}}%
\pgfpathlineto{\pgfqpoint{1.070698in}{1.025925in}}%
\pgfpathlineto{\pgfqpoint{1.075222in}{1.039536in}}%
\pgfpathlineto{\pgfqpoint{1.080617in}{1.047751in}}%
\pgfpathlineto{\pgfqpoint{1.084915in}{1.053148in}}%
\pgfpathlineto{\pgfqpoint{1.096274in}{1.063022in}}%
\pgfpathlineto{\pgfqpoint{1.102481in}{1.066759in}}%
\pgfpathlineto{\pgfqpoint{1.111930in}{1.071448in}}%
\pgfpathlineto{\pgfqpoint{1.127587in}{1.075382in}}%
\pgfpathlineto{\pgfqpoint{1.143243in}{1.075382in}}%
\pgfpathlineto{\pgfqpoint{1.158900in}{1.071448in}}%
\pgfpathlineto{\pgfqpoint{1.168349in}{1.066759in}}%
\pgfpathlineto{\pgfqpoint{1.174556in}{1.063022in}}%
\pgfpathlineto{\pgfqpoint{1.185915in}{1.053148in}}%
\pgfpathlineto{\pgfqpoint{1.190213in}{1.047751in}}%
\pgfpathlineto{\pgfqpoint{1.195608in}{1.039536in}}%
\pgfpathlineto{\pgfqpoint{1.200132in}{1.025925in}}%
\pgfpathlineto{\pgfqpoint{1.200132in}{1.012314in}}%
\pgfpathlineto{\pgfqpoint{1.195608in}{0.998703in}}%
\pgfpathlineto{\pgfqpoint{1.190213in}{0.990488in}}%
\pgfpathlineto{\pgfqpoint{1.185915in}{0.985092in}}%
\pgfpathlineto{\pgfqpoint{1.174556in}{0.975218in}}%
\pgfpathlineto{\pgfqpoint{1.168349in}{0.971481in}}%
\pgfpathlineto{\pgfqpoint{1.158900in}{0.966791in}}%
\pgfpathlineto{\pgfqpoint{1.143243in}{0.962857in}}%
\pgfpathlineto{\pgfqpoint{1.127587in}{0.962857in}}%
\pgfpathlineto{\pgfqpoint{1.111930in}{0.966791in}}%
\pgfpathlineto{\pgfqpoint{1.102481in}{0.971481in}}%
\pgfpathclose%
\pgfpathmoveto{\pgfqpoint{1.409405in}{0.954348in}}%
\pgfpathlineto{\pgfqpoint{1.425061in}{0.948129in}}%
\pgfpathlineto{\pgfqpoint{1.440718in}{0.945370in}}%
\pgfpathlineto{\pgfqpoint{1.456375in}{0.946059in}}%
\pgfpathlineto{\pgfqpoint{1.472031in}{0.950201in}}%
\pgfpathlineto{\pgfqpoint{1.487688in}{0.957805in}}%
\pgfpathlineto{\pgfqpoint{1.487785in}{0.957870in}}%
\pgfpathlineto{\pgfqpoint{1.503344in}{0.969317in}}%
\pgfpathlineto{\pgfqpoint{1.505762in}{0.971481in}}%
\pgfpathlineto{\pgfqpoint{1.517988in}{0.985092in}}%
\pgfpathlineto{\pgfqpoint{1.519001in}{0.986727in}}%
\pgfpathlineto{\pgfqpoint{1.525990in}{0.998703in}}%
\pgfpathlineto{\pgfqpoint{1.529971in}{1.012314in}}%
\pgfpathlineto{\pgfqpoint{1.529971in}{1.025925in}}%
\pgfpathlineto{\pgfqpoint{1.525990in}{1.039536in}}%
\pgfpathlineto{\pgfqpoint{1.519001in}{1.051513in}}%
\pgfpathlineto{\pgfqpoint{1.517988in}{1.053148in}}%
\pgfpathlineto{\pgfqpoint{1.505762in}{1.066759in}}%
\pgfpathlineto{\pgfqpoint{1.503344in}{1.068923in}}%
\pgfpathlineto{\pgfqpoint{1.487785in}{1.080370in}}%
\pgfpathlineto{\pgfqpoint{1.487688in}{1.080434in}}%
\pgfpathlineto{\pgfqpoint{1.472031in}{1.088039in}}%
\pgfpathlineto{\pgfqpoint{1.456375in}{1.092180in}}%
\pgfpathlineto{\pgfqpoint{1.440718in}{1.092869in}}%
\pgfpathlineto{\pgfqpoint{1.425061in}{1.090111in}}%
\pgfpathlineto{\pgfqpoint{1.409405in}{1.083891in}}%
\pgfpathlineto{\pgfqpoint{1.403525in}{1.080370in}}%
\pgfpathlineto{\pgfqpoint{1.393748in}{1.073946in}}%
\pgfpathlineto{\pgfqpoint{1.385203in}{1.066759in}}%
\pgfpathlineto{\pgfqpoint{1.378092in}{1.059315in}}%
\pgfpathlineto{\pgfqpoint{1.372847in}{1.053148in}}%
\pgfpathlineto{\pgfqpoint{1.364890in}{1.039536in}}%
\pgfpathlineto{\pgfqpoint{1.362435in}{1.031179in}}%
\pgfpathlineto{\pgfqpoint{1.360862in}{1.025925in}}%
\pgfpathlineto{\pgfqpoint{1.360862in}{1.012314in}}%
\pgfpathlineto{\pgfqpoint{1.362435in}{1.007061in}}%
\pgfpathlineto{\pgfqpoint{1.364890in}{0.998703in}}%
\pgfpathlineto{\pgfqpoint{1.372847in}{0.985092in}}%
\pgfpathlineto{\pgfqpoint{1.378092in}{0.978924in}}%
\pgfpathlineto{\pgfqpoint{1.385203in}{0.971481in}}%
\pgfpathlineto{\pgfqpoint{1.393748in}{0.964293in}}%
\pgfpathlineto{\pgfqpoint{1.403525in}{0.957870in}}%
\pgfpathlineto{\pgfqpoint{1.409405in}{0.954348in}}%
\pgfpathclose%
\pgfpathmoveto{\pgfqpoint{1.412200in}{0.971481in}}%
\pgfpathlineto{\pgfqpoint{1.409405in}{0.973003in}}%
\pgfpathlineto{\pgfqpoint{1.394691in}{0.985092in}}%
\pgfpathlineto{\pgfqpoint{1.393748in}{0.986207in}}%
\pgfpathlineto{\pgfqpoint{1.385267in}{0.998703in}}%
\pgfpathlineto{\pgfqpoint{1.380643in}{1.012314in}}%
\pgfpathlineto{\pgfqpoint{1.380643in}{1.025925in}}%
\pgfpathlineto{\pgfqpoint{1.385267in}{1.039536in}}%
\pgfpathlineto{\pgfqpoint{1.393748in}{1.052033in}}%
\pgfpathlineto{\pgfqpoint{1.394691in}{1.053148in}}%
\pgfpathlineto{\pgfqpoint{1.409405in}{1.065237in}}%
\pgfpathlineto{\pgfqpoint{1.412200in}{1.066759in}}%
\pgfpathlineto{\pgfqpoint{1.425061in}{1.072551in}}%
\pgfpathlineto{\pgfqpoint{1.440718in}{1.075696in}}%
\pgfpathlineto{\pgfqpoint{1.456375in}{1.074910in}}%
\pgfpathlineto{\pgfqpoint{1.472031in}{1.070188in}}%
\pgfpathlineto{\pgfqpoint{1.478370in}{1.066759in}}%
\pgfpathlineto{\pgfqpoint{1.487688in}{1.060623in}}%
\pgfpathlineto{\pgfqpoint{1.495859in}{1.053148in}}%
\pgfpathlineto{\pgfqpoint{1.503344in}{1.043231in}}%
\pgfpathlineto{\pgfqpoint{1.505701in}{1.039536in}}%
\pgfpathlineto{\pgfqpoint{1.510148in}{1.025925in}}%
\pgfpathlineto{\pgfqpoint{1.510148in}{1.012314in}}%
\pgfpathlineto{\pgfqpoint{1.505701in}{0.998703in}}%
\pgfpathlineto{\pgfqpoint{1.503344in}{0.995009in}}%
\pgfpathlineto{\pgfqpoint{1.495859in}{0.985092in}}%
\pgfpathlineto{\pgfqpoint{1.487688in}{0.977616in}}%
\pgfpathlineto{\pgfqpoint{1.478370in}{0.971481in}}%
\pgfpathlineto{\pgfqpoint{1.472031in}{0.968051in}}%
\pgfpathlineto{\pgfqpoint{1.456375in}{0.963329in}}%
\pgfpathlineto{\pgfqpoint{1.440718in}{0.962543in}}%
\pgfpathlineto{\pgfqpoint{1.425061in}{0.965689in}}%
\pgfpathlineto{\pgfqpoint{1.412200in}{0.971481in}}%
\pgfpathclose%
\pgfpathmoveto{\pgfqpoint{1.722536in}{0.952827in}}%
\pgfpathlineto{\pgfqpoint{1.738193in}{0.947301in}}%
\pgfpathlineto{\pgfqpoint{1.753849in}{0.945233in}}%
\pgfpathlineto{\pgfqpoint{1.769506in}{0.946611in}}%
\pgfpathlineto{\pgfqpoint{1.785162in}{0.951445in}}%
\pgfpathlineto{\pgfqpoint{1.797404in}{0.957870in}}%
\pgfpathlineto{\pgfqpoint{1.800819in}{0.959824in}}%
\pgfpathlineto{\pgfqpoint{1.815808in}{0.971481in}}%
\pgfpathlineto{\pgfqpoint{1.816476in}{0.972124in}}%
\pgfpathlineto{\pgfqpoint{1.827946in}{0.985092in}}%
\pgfpathlineto{\pgfqpoint{1.832132in}{0.992036in}}%
\pgfpathlineto{\pgfqpoint{1.835997in}{0.998703in}}%
\pgfpathlineto{\pgfqpoint{1.840008in}{1.012314in}}%
\pgfpathlineto{\pgfqpoint{1.840008in}{1.025925in}}%
\pgfpathlineto{\pgfqpoint{1.835997in}{1.039536in}}%
\pgfpathlineto{\pgfqpoint{1.832132in}{1.046203in}}%
\pgfpathlineto{\pgfqpoint{1.827946in}{1.053148in}}%
\pgfpathlineto{\pgfqpoint{1.816476in}{1.066116in}}%
\pgfpathlineto{\pgfqpoint{1.815808in}{1.066759in}}%
\pgfpathlineto{\pgfqpoint{1.800819in}{1.078415in}}%
\pgfpathlineto{\pgfqpoint{1.797404in}{1.080370in}}%
\pgfpathlineto{\pgfqpoint{1.785162in}{1.086795in}}%
\pgfpathlineto{\pgfqpoint{1.769506in}{1.091628in}}%
\pgfpathlineto{\pgfqpoint{1.753849in}{1.093007in}}%
\pgfpathlineto{\pgfqpoint{1.738193in}{1.090939in}}%
\pgfpathlineto{\pgfqpoint{1.722536in}{1.085412in}}%
\pgfpathlineto{\pgfqpoint{1.713573in}{1.080370in}}%
\pgfpathlineto{\pgfqpoint{1.706880in}{1.076251in}}%
\pgfpathlineto{\pgfqpoint{1.695168in}{1.066759in}}%
\pgfpathlineto{\pgfqpoint{1.691223in}{1.062789in}}%
\pgfpathlineto{\pgfqpoint{1.682877in}{1.053148in}}%
\pgfpathlineto{\pgfqpoint{1.675567in}{1.040739in}}%
\pgfpathlineto{\pgfqpoint{1.674871in}{1.039536in}}%
\pgfpathlineto{\pgfqpoint{1.670814in}{1.025925in}}%
\pgfpathlineto{\pgfqpoint{1.670814in}{1.012314in}}%
\pgfpathlineto{\pgfqpoint{1.674871in}{0.998703in}}%
\pgfpathlineto{\pgfqpoint{1.675567in}{0.997501in}}%
\pgfpathlineto{\pgfqpoint{1.682877in}{0.985092in}}%
\pgfpathlineto{\pgfqpoint{1.691223in}{0.975451in}}%
\pgfpathlineto{\pgfqpoint{1.695168in}{0.971481in}}%
\pgfpathlineto{\pgfqpoint{1.706880in}{0.961988in}}%
\pgfpathlineto{\pgfqpoint{1.713573in}{0.957870in}}%
\pgfpathlineto{\pgfqpoint{1.722536in}{0.952827in}}%
\pgfpathclose%
\pgfpathmoveto{\pgfqpoint{1.721843in}{0.971481in}}%
\pgfpathlineto{\pgfqpoint{1.706880in}{0.982961in}}%
\pgfpathlineto{\pgfqpoint{1.704746in}{0.985092in}}%
\pgfpathlineto{\pgfqpoint{1.695234in}{0.998703in}}%
\pgfpathlineto{\pgfqpoint{1.691223in}{1.010171in}}%
\pgfpathlineto{\pgfqpoint{1.690548in}{1.012314in}}%
\pgfpathlineto{\pgfqpoint{1.690548in}{1.025925in}}%
\pgfpathlineto{\pgfqpoint{1.691223in}{1.028069in}}%
\pgfpathlineto{\pgfqpoint{1.695234in}{1.039536in}}%
\pgfpathlineto{\pgfqpoint{1.704746in}{1.053148in}}%
\pgfpathlineto{\pgfqpoint{1.706880in}{1.055278in}}%
\pgfpathlineto{\pgfqpoint{1.721843in}{1.066759in}}%
\pgfpathlineto{\pgfqpoint{1.722536in}{1.067193in}}%
\pgfpathlineto{\pgfqpoint{1.738193in}{1.073495in}}%
\pgfpathlineto{\pgfqpoint{1.753849in}{1.075853in}}%
\pgfpathlineto{\pgfqpoint{1.769506in}{1.074281in}}%
\pgfpathlineto{\pgfqpoint{1.785162in}{1.068770in}}%
\pgfpathlineto{\pgfqpoint{1.788603in}{1.066759in}}%
\pgfpathlineto{\pgfqpoint{1.800819in}{1.058041in}}%
\pgfpathlineto{\pgfqpoint{1.805928in}{1.053148in}}%
\pgfpathlineto{\pgfqpoint{1.815741in}{1.039536in}}%
\pgfpathlineto{\pgfqpoint{1.816476in}{1.037525in}}%
\pgfpathlineto{\pgfqpoint{1.820207in}{1.025925in}}%
\pgfpathlineto{\pgfqpoint{1.820207in}{1.012314in}}%
\pgfpathlineto{\pgfqpoint{1.816476in}{1.000714in}}%
\pgfpathlineto{\pgfqpoint{1.815741in}{0.998703in}}%
\pgfpathlineto{\pgfqpoint{1.805928in}{0.985092in}}%
\pgfpathlineto{\pgfqpoint{1.800819in}{0.980198in}}%
\pgfpathlineto{\pgfqpoint{1.788603in}{0.971481in}}%
\pgfpathlineto{\pgfqpoint{1.785162in}{0.969470in}}%
\pgfpathlineto{\pgfqpoint{1.769506in}{0.963958in}}%
\pgfpathlineto{\pgfqpoint{1.753849in}{0.962386in}}%
\pgfpathlineto{\pgfqpoint{1.738193in}{0.964745in}}%
\pgfpathlineto{\pgfqpoint{1.722536in}{0.971046in}}%
\pgfpathlineto{\pgfqpoint{1.721843in}{0.971481in}}%
\pgfpathclose%
\pgfpathmoveto{\pgfqpoint{0.501324in}{1.216116in}}%
\pgfpathlineto{\pgfqpoint{0.516981in}{1.214684in}}%
\pgfpathlineto{\pgfqpoint{0.530100in}{1.216481in}}%
\pgfpathlineto{\pgfqpoint{0.532637in}{1.216819in}}%
\pgfpathlineto{\pgfqpoint{0.548294in}{1.222349in}}%
\pgfpathlineto{\pgfqpoint{0.561863in}{1.230092in}}%
\pgfpathlineto{\pgfqpoint{0.563950in}{1.231422in}}%
\pgfpathlineto{\pgfqpoint{0.578611in}{1.243703in}}%
\pgfpathlineto{\pgfqpoint{0.579607in}{1.244767in}}%
\pgfpathlineto{\pgfqpoint{0.589872in}{1.257314in}}%
\pgfpathlineto{\pgfqpoint{0.595263in}{1.267433in}}%
\pgfpathlineto{\pgfqpoint{0.597096in}{1.270925in}}%
\pgfpathlineto{\pgfqpoint{0.600340in}{1.284536in}}%
\pgfpathlineto{\pgfqpoint{0.599530in}{1.298147in}}%
\pgfpathlineto{\pgfqpoint{0.595263in}{1.310101in}}%
\pgfpathlineto{\pgfqpoint{0.594669in}{1.311759in}}%
\pgfpathlineto{\pgfqpoint{0.585875in}{1.325370in}}%
\pgfpathlineto{\pgfqpoint{0.579607in}{1.332250in}}%
\pgfpathlineto{\pgfqpoint{0.572548in}{1.338981in}}%
\pgfpathlineto{\pgfqpoint{0.563950in}{1.345751in}}%
\pgfpathlineto{\pgfqpoint{0.552482in}{1.352592in}}%
\pgfpathlineto{\pgfqpoint{0.548294in}{1.354924in}}%
\pgfpathlineto{\pgfqpoint{0.532637in}{1.360469in}}%
\pgfpathlineto{\pgfqpoint{0.516981in}{1.362544in}}%
\pgfpathlineto{\pgfqpoint{0.501324in}{1.361161in}}%
\pgfpathlineto{\pgfqpoint{0.485668in}{1.356311in}}%
\pgfpathlineto{\pgfqpoint{0.478508in}{1.352592in}}%
\pgfpathlineto{\pgfqpoint{0.470011in}{1.347878in}}%
\pgfpathlineto{\pgfqpoint{0.458234in}{1.338981in}}%
\pgfpathlineto{\pgfqpoint{0.454354in}{1.335440in}}%
\pgfpathlineto{\pgfqpoint{0.444981in}{1.325370in}}%
\pgfpathlineto{\pgfqpoint{0.438698in}{1.315841in}}%
\pgfpathlineto{\pgfqpoint{0.436118in}{1.311759in}}%
\pgfpathlineto{\pgfqpoint{0.431302in}{1.298147in}}%
\pgfpathlineto{\pgfqpoint{0.430501in}{1.284536in}}%
\pgfpathlineto{\pgfqpoint{0.433709in}{1.270925in}}%
\pgfpathlineto{\pgfqpoint{0.438698in}{1.261441in}}%
\pgfpathlineto{\pgfqpoint{0.440948in}{1.257314in}}%
\pgfpathlineto{\pgfqpoint{0.452225in}{1.243703in}}%
\pgfpathlineto{\pgfqpoint{0.454354in}{1.241683in}}%
\pgfpathlineto{\pgfqpoint{0.468845in}{1.230092in}}%
\pgfpathlineto{\pgfqpoint{0.470011in}{1.229266in}}%
\pgfpathlineto{\pgfqpoint{0.485668in}{1.220966in}}%
\pgfpathlineto{\pgfqpoint{0.500159in}{1.216481in}}%
\pgfpathlineto{\pgfqpoint{0.501324in}{1.216116in}}%
\pgfpathclose%
\pgfpathmoveto{\pgfqpoint{0.477994in}{1.243703in}}%
\pgfpathlineto{\pgfqpoint{0.470011in}{1.249748in}}%
\pgfpathlineto{\pgfqpoint{0.462547in}{1.257314in}}%
\pgfpathlineto{\pgfqpoint{0.454354in}{1.269913in}}%
\pgfpathlineto{\pgfqpoint{0.453783in}{1.270925in}}%
\pgfpathlineto{\pgfqpoint{0.450272in}{1.284536in}}%
\pgfpathlineto{\pgfqpoint{0.451149in}{1.298147in}}%
\pgfpathlineto{\pgfqpoint{0.454354in}{1.306495in}}%
\pgfpathlineto{\pgfqpoint{0.456658in}{1.311759in}}%
\pgfpathlineto{\pgfqpoint{0.467454in}{1.325370in}}%
\pgfpathlineto{\pgfqpoint{0.470011in}{1.327698in}}%
\pgfpathlineto{\pgfqpoint{0.485668in}{1.338314in}}%
\pgfpathlineto{\pgfqpoint{0.487318in}{1.338981in}}%
\pgfpathlineto{\pgfqpoint{0.501324in}{1.343814in}}%
\pgfpathlineto{\pgfqpoint{0.516981in}{1.345359in}}%
\pgfpathlineto{\pgfqpoint{0.532637in}{1.343041in}}%
\pgfpathlineto{\pgfqpoint{0.543002in}{1.338981in}}%
\pgfpathlineto{\pgfqpoint{0.548294in}{1.336544in}}%
\pgfpathlineto{\pgfqpoint{0.563531in}{1.325370in}}%
\pgfpathlineto{\pgfqpoint{0.563950in}{1.324945in}}%
\pgfpathlineto{\pgfqpoint{0.574075in}{1.311759in}}%
\pgfpathlineto{\pgfqpoint{0.579607in}{1.298555in}}%
\pgfpathlineto{\pgfqpoint{0.579760in}{1.298147in}}%
\pgfpathlineto{\pgfqpoint{0.580629in}{1.284536in}}%
\pgfpathlineto{\pgfqpoint{0.579607in}{1.280496in}}%
\pgfpathlineto{\pgfqpoint{0.576926in}{1.270925in}}%
\pgfpathlineto{\pgfqpoint{0.568368in}{1.257314in}}%
\pgfpathlineto{\pgfqpoint{0.563950in}{1.252647in}}%
\pgfpathlineto{\pgfqpoint{0.552965in}{1.243703in}}%
\pgfpathlineto{\pgfqpoint{0.548294in}{1.240678in}}%
\pgfpathlineto{\pgfqpoint{0.532637in}{1.234239in}}%
\pgfpathlineto{\pgfqpoint{0.516981in}{1.231829in}}%
\pgfpathlineto{\pgfqpoint{0.501324in}{1.233435in}}%
\pgfpathlineto{\pgfqpoint{0.485668in}{1.239067in}}%
\pgfpathlineto{\pgfqpoint{0.477994in}{1.243703in}}%
\pgfpathclose%
\pgfpathmoveto{\pgfqpoint{0.814455in}{1.215543in}}%
\pgfpathlineto{\pgfqpoint{0.830112in}{1.214827in}}%
\pgfpathlineto{\pgfqpoint{0.839204in}{1.216481in}}%
\pgfpathlineto{\pgfqpoint{0.845769in}{1.217648in}}%
\pgfpathlineto{\pgfqpoint{0.861425in}{1.223872in}}%
\pgfpathlineto{\pgfqpoint{0.871666in}{1.230092in}}%
\pgfpathlineto{\pgfqpoint{0.877082in}{1.233777in}}%
\pgfpathlineto{\pgfqpoint{0.888499in}{1.243703in}}%
\pgfpathlineto{\pgfqpoint{0.892738in}{1.248411in}}%
\pgfpathlineto{\pgfqpoint{0.899893in}{1.257314in}}%
\pgfpathlineto{\pgfqpoint{0.907052in}{1.270925in}}%
\pgfpathlineto{\pgfqpoint{0.908395in}{1.276632in}}%
\pgfpathlineto{\pgfqpoint{0.910297in}{1.284536in}}%
\pgfpathlineto{\pgfqpoint{0.909474in}{1.298147in}}%
\pgfpathlineto{\pgfqpoint{0.908395in}{1.301166in}}%
\pgfpathlineto{\pgfqpoint{0.904668in}{1.311759in}}%
\pgfpathlineto{\pgfqpoint{0.895915in}{1.325370in}}%
\pgfpathlineto{\pgfqpoint{0.892738in}{1.328919in}}%
\pgfpathlineto{\pgfqpoint{0.882594in}{1.338981in}}%
\pgfpathlineto{\pgfqpoint{0.877082in}{1.343485in}}%
\pgfpathlineto{\pgfqpoint{0.862784in}{1.352592in}}%
\pgfpathlineto{\pgfqpoint{0.861425in}{1.353397in}}%
\pgfpathlineto{\pgfqpoint{0.845769in}{1.359638in}}%
\pgfpathlineto{\pgfqpoint{0.830112in}{1.362406in}}%
\pgfpathlineto{\pgfqpoint{0.814455in}{1.361714in}}%
\pgfpathlineto{\pgfqpoint{0.798799in}{1.357559in}}%
\pgfpathlineto{\pgfqpoint{0.788469in}{1.352592in}}%
\pgfpathlineto{\pgfqpoint{0.783142in}{1.349865in}}%
\pgfpathlineto{\pgfqpoint{0.768055in}{1.338981in}}%
\pgfpathlineto{\pgfqpoint{0.767486in}{1.338486in}}%
\pgfpathlineto{\pgfqpoint{0.754966in}{1.325370in}}%
\pgfpathlineto{\pgfqpoint{0.751829in}{1.320739in}}%
\pgfpathlineto{\pgfqpoint{0.746116in}{1.311759in}}%
\pgfpathlineto{\pgfqpoint{0.741336in}{1.298147in}}%
\pgfpathlineto{\pgfqpoint{0.740541in}{1.284536in}}%
\pgfpathlineto{\pgfqpoint{0.743724in}{1.270925in}}%
\pgfpathlineto{\pgfqpoint{0.750903in}{1.257314in}}%
\pgfpathlineto{\pgfqpoint{0.751829in}{1.256133in}}%
\pgfpathlineto{\pgfqpoint{0.762305in}{1.243703in}}%
\pgfpathlineto{\pgfqpoint{0.767486in}{1.238911in}}%
\pgfpathlineto{\pgfqpoint{0.779060in}{1.230092in}}%
\pgfpathlineto{\pgfqpoint{0.783142in}{1.227331in}}%
\pgfpathlineto{\pgfqpoint{0.798799in}{1.219721in}}%
\pgfpathlineto{\pgfqpoint{0.810983in}{1.216481in}}%
\pgfpathlineto{\pgfqpoint{0.814455in}{1.215543in}}%
\pgfpathclose%
\pgfpathmoveto{\pgfqpoint{0.787917in}{1.243703in}}%
\pgfpathlineto{\pgfqpoint{0.783142in}{1.247039in}}%
\pgfpathlineto{\pgfqpoint{0.772528in}{1.257314in}}%
\pgfpathlineto{\pgfqpoint{0.767486in}{1.264701in}}%
\pgfpathlineto{\pgfqpoint{0.763883in}{1.270925in}}%
\pgfpathlineto{\pgfqpoint{0.760326in}{1.284536in}}%
\pgfpathlineto{\pgfqpoint{0.761215in}{1.298147in}}%
\pgfpathlineto{\pgfqpoint{0.766554in}{1.311759in}}%
\pgfpathlineto{\pgfqpoint{0.767486in}{1.313098in}}%
\pgfpathlineto{\pgfqpoint{0.777617in}{1.325370in}}%
\pgfpathlineto{\pgfqpoint{0.783142in}{1.330173in}}%
\pgfpathlineto{\pgfqpoint{0.797258in}{1.338981in}}%
\pgfpathlineto{\pgfqpoint{0.798799in}{1.339791in}}%
\pgfpathlineto{\pgfqpoint{0.814455in}{1.344433in}}%
\pgfpathlineto{\pgfqpoint{0.830112in}{1.345205in}}%
\pgfpathlineto{\pgfqpoint{0.845769in}{1.342113in}}%
\pgfpathlineto{\pgfqpoint{0.852929in}{1.338981in}}%
\pgfpathlineto{\pgfqpoint{0.861425in}{1.334597in}}%
\pgfpathlineto{\pgfqpoint{0.873244in}{1.325370in}}%
\pgfpathlineto{\pgfqpoint{0.877082in}{1.321219in}}%
\pgfpathlineto{\pgfqpoint{0.884081in}{1.311759in}}%
\pgfpathlineto{\pgfqpoint{0.889632in}{1.298147in}}%
\pgfpathlineto{\pgfqpoint{0.890556in}{1.284536in}}%
\pgfpathlineto{\pgfqpoint{0.886858in}{1.270925in}}%
\pgfpathlineto{\pgfqpoint{0.878522in}{1.257314in}}%
\pgfpathlineto{\pgfqpoint{0.877082in}{1.255735in}}%
\pgfpathlineto{\pgfqpoint{0.863241in}{1.243703in}}%
\pgfpathlineto{\pgfqpoint{0.861425in}{1.242451in}}%
\pgfpathlineto{\pgfqpoint{0.845769in}{1.235204in}}%
\pgfpathlineto{\pgfqpoint{0.830112in}{1.231989in}}%
\pgfpathlineto{\pgfqpoint{0.814455in}{1.232792in}}%
\pgfpathlineto{\pgfqpoint{0.798799in}{1.237618in}}%
\pgfpathlineto{\pgfqpoint{0.787917in}{1.243703in}}%
\pgfpathclose%
\pgfpathmoveto{\pgfqpoint{1.127587in}{1.215113in}}%
\pgfpathlineto{\pgfqpoint{1.143243in}{1.215113in}}%
\pgfpathlineto{\pgfqpoint{1.149286in}{1.216481in}}%
\pgfpathlineto{\pgfqpoint{1.158900in}{1.218615in}}%
\pgfpathlineto{\pgfqpoint{1.174556in}{1.225532in}}%
\pgfpathlineto{\pgfqpoint{1.181651in}{1.230092in}}%
\pgfpathlineto{\pgfqpoint{1.190213in}{1.236274in}}%
\pgfpathlineto{\pgfqpoint{1.198481in}{1.243703in}}%
\pgfpathlineto{\pgfqpoint{1.205870in}{1.252203in}}%
\pgfpathlineto{\pgfqpoint{1.209920in}{1.257314in}}%
\pgfpathlineto{\pgfqpoint{1.217074in}{1.270925in}}%
\pgfpathlineto{\pgfqpoint{1.220247in}{1.284536in}}%
\pgfpathlineto{\pgfqpoint{1.219455in}{1.298147in}}%
\pgfpathlineto{\pgfqpoint{1.214691in}{1.311759in}}%
\pgfpathlineto{\pgfqpoint{1.205944in}{1.325370in}}%
\pgfpathlineto{\pgfqpoint{1.205870in}{1.325454in}}%
\pgfpathlineto{\pgfqpoint{1.192702in}{1.338981in}}%
\pgfpathlineto{\pgfqpoint{1.190213in}{1.341083in}}%
\pgfpathlineto{\pgfqpoint{1.174556in}{1.351712in}}%
\pgfpathlineto{\pgfqpoint{1.172676in}{1.352592in}}%
\pgfpathlineto{\pgfqpoint{1.158900in}{1.358668in}}%
\pgfpathlineto{\pgfqpoint{1.143243in}{1.362129in}}%
\pgfpathlineto{\pgfqpoint{1.127587in}{1.362129in}}%
\pgfpathlineto{\pgfqpoint{1.111930in}{1.358668in}}%
\pgfpathlineto{\pgfqpoint{1.098154in}{1.352592in}}%
\pgfpathlineto{\pgfqpoint{1.096274in}{1.351712in}}%
\pgfpathlineto{\pgfqpoint{1.080617in}{1.341083in}}%
\pgfpathlineto{\pgfqpoint{1.078128in}{1.338981in}}%
\pgfpathlineto{\pgfqpoint{1.064960in}{1.325454in}}%
\pgfpathlineto{\pgfqpoint{1.064886in}{1.325370in}}%
\pgfpathlineto{\pgfqpoint{1.056139in}{1.311759in}}%
\pgfpathlineto{\pgfqpoint{1.051375in}{1.298147in}}%
\pgfpathlineto{\pgfqpoint{1.050583in}{1.284536in}}%
\pgfpathlineto{\pgfqpoint{1.053756in}{1.270925in}}%
\pgfpathlineto{\pgfqpoint{1.060910in}{1.257314in}}%
\pgfpathlineto{\pgfqpoint{1.064960in}{1.252203in}}%
\pgfpathlineto{\pgfqpoint{1.072349in}{1.243703in}}%
\pgfpathlineto{\pgfqpoint{1.080617in}{1.236274in}}%
\pgfpathlineto{\pgfqpoint{1.089179in}{1.230092in}}%
\pgfpathlineto{\pgfqpoint{1.096274in}{1.225532in}}%
\pgfpathlineto{\pgfqpoint{1.111930in}{1.218615in}}%
\pgfpathlineto{\pgfqpoint{1.121544in}{1.216481in}}%
\pgfpathlineto{\pgfqpoint{1.127587in}{1.215113in}}%
\pgfpathclose%
\pgfpathmoveto{\pgfqpoint{1.097556in}{1.243703in}}%
\pgfpathlineto{\pgfqpoint{1.096274in}{1.244522in}}%
\pgfpathlineto{\pgfqpoint{1.082368in}{1.257314in}}%
\pgfpathlineto{\pgfqpoint{1.080617in}{1.259744in}}%
\pgfpathlineto{\pgfqpoint{1.073955in}{1.270925in}}%
\pgfpathlineto{\pgfqpoint{1.070336in}{1.284536in}}%
\pgfpathlineto{\pgfqpoint{1.071240in}{1.298147in}}%
\pgfpathlineto{\pgfqpoint{1.076672in}{1.311759in}}%
\pgfpathlineto{\pgfqpoint{1.080617in}{1.317269in}}%
\pgfpathlineto{\pgfqpoint{1.087675in}{1.325370in}}%
\pgfpathlineto{\pgfqpoint{1.096274in}{1.332473in}}%
\pgfpathlineto{\pgfqpoint{1.107680in}{1.338981in}}%
\pgfpathlineto{\pgfqpoint{1.111930in}{1.341030in}}%
\pgfpathlineto{\pgfqpoint{1.127587in}{1.344896in}}%
\pgfpathlineto{\pgfqpoint{1.143243in}{1.344896in}}%
\pgfpathlineto{\pgfqpoint{1.158900in}{1.341030in}}%
\pgfpathlineto{\pgfqpoint{1.163150in}{1.338981in}}%
\pgfpathlineto{\pgfqpoint{1.174556in}{1.332473in}}%
\pgfpathlineto{\pgfqpoint{1.183155in}{1.325370in}}%
\pgfpathlineto{\pgfqpoint{1.190213in}{1.317269in}}%
\pgfpathlineto{\pgfqpoint{1.194158in}{1.311759in}}%
\pgfpathlineto{\pgfqpoint{1.199590in}{1.298147in}}%
\pgfpathlineto{\pgfqpoint{1.200494in}{1.284536in}}%
\pgfpathlineto{\pgfqpoint{1.196875in}{1.270925in}}%
\pgfpathlineto{\pgfqpoint{1.190213in}{1.259744in}}%
\pgfpathlineto{\pgfqpoint{1.188462in}{1.257314in}}%
\pgfpathlineto{\pgfqpoint{1.174556in}{1.244522in}}%
\pgfpathlineto{\pgfqpoint{1.173274in}{1.243703in}}%
\pgfpathlineto{\pgfqpoint{1.158900in}{1.236330in}}%
\pgfpathlineto{\pgfqpoint{1.143243in}{1.232310in}}%
\pgfpathlineto{\pgfqpoint{1.127587in}{1.232310in}}%
\pgfpathlineto{\pgfqpoint{1.111930in}{1.236330in}}%
\pgfpathlineto{\pgfqpoint{1.097556in}{1.243703in}}%
\pgfpathclose%
\pgfpathmoveto{\pgfqpoint{1.440718in}{1.214827in}}%
\pgfpathlineto{\pgfqpoint{1.456375in}{1.215543in}}%
\pgfpathlineto{\pgfqpoint{1.459847in}{1.216481in}}%
\pgfpathlineto{\pgfqpoint{1.472031in}{1.219721in}}%
\pgfpathlineto{\pgfqpoint{1.487688in}{1.227331in}}%
\pgfpathlineto{\pgfqpoint{1.491770in}{1.230092in}}%
\pgfpathlineto{\pgfqpoint{1.503344in}{1.238911in}}%
\pgfpathlineto{\pgfqpoint{1.508525in}{1.243703in}}%
\pgfpathlineto{\pgfqpoint{1.519001in}{1.256133in}}%
\pgfpathlineto{\pgfqpoint{1.519927in}{1.257314in}}%
\pgfpathlineto{\pgfqpoint{1.527106in}{1.270925in}}%
\pgfpathlineto{\pgfqpoint{1.530289in}{1.284536in}}%
\pgfpathlineto{\pgfqpoint{1.529494in}{1.298147in}}%
\pgfpathlineto{\pgfqpoint{1.524714in}{1.311759in}}%
\pgfpathlineto{\pgfqpoint{1.519001in}{1.320739in}}%
\pgfpathlineto{\pgfqpoint{1.515864in}{1.325370in}}%
\pgfpathlineto{\pgfqpoint{1.503344in}{1.338486in}}%
\pgfpathlineto{\pgfqpoint{1.502775in}{1.338981in}}%
\pgfpathlineto{\pgfqpoint{1.487688in}{1.349865in}}%
\pgfpathlineto{\pgfqpoint{1.482361in}{1.352592in}}%
\pgfpathlineto{\pgfqpoint{1.472031in}{1.357559in}}%
\pgfpathlineto{\pgfqpoint{1.456375in}{1.361714in}}%
\pgfpathlineto{\pgfqpoint{1.440718in}{1.362406in}}%
\pgfpathlineto{\pgfqpoint{1.425061in}{1.359638in}}%
\pgfpathlineto{\pgfqpoint{1.409405in}{1.353397in}}%
\pgfpathlineto{\pgfqpoint{1.408046in}{1.352592in}}%
\pgfpathlineto{\pgfqpoint{1.393748in}{1.343485in}}%
\pgfpathlineto{\pgfqpoint{1.388236in}{1.338981in}}%
\pgfpathlineto{\pgfqpoint{1.378092in}{1.328919in}}%
\pgfpathlineto{\pgfqpoint{1.374915in}{1.325370in}}%
\pgfpathlineto{\pgfqpoint{1.366162in}{1.311759in}}%
\pgfpathlineto{\pgfqpoint{1.362435in}{1.301166in}}%
\pgfpathlineto{\pgfqpoint{1.361356in}{1.298147in}}%
\pgfpathlineto{\pgfqpoint{1.360533in}{1.284536in}}%
\pgfpathlineto{\pgfqpoint{1.362435in}{1.276632in}}%
\pgfpathlineto{\pgfqpoint{1.363778in}{1.270925in}}%
\pgfpathlineto{\pgfqpoint{1.370937in}{1.257314in}}%
\pgfpathlineto{\pgfqpoint{1.378092in}{1.248411in}}%
\pgfpathlineto{\pgfqpoint{1.382331in}{1.243703in}}%
\pgfpathlineto{\pgfqpoint{1.393748in}{1.233777in}}%
\pgfpathlineto{\pgfqpoint{1.399164in}{1.230092in}}%
\pgfpathlineto{\pgfqpoint{1.409405in}{1.223872in}}%
\pgfpathlineto{\pgfqpoint{1.425061in}{1.217648in}}%
\pgfpathlineto{\pgfqpoint{1.431626in}{1.216481in}}%
\pgfpathlineto{\pgfqpoint{1.440718in}{1.214827in}}%
\pgfpathclose%
\pgfpathmoveto{\pgfqpoint{1.407589in}{1.243703in}}%
\pgfpathlineto{\pgfqpoint{1.393748in}{1.255735in}}%
\pgfpathlineto{\pgfqpoint{1.392308in}{1.257314in}}%
\pgfpathlineto{\pgfqpoint{1.383972in}{1.270925in}}%
\pgfpathlineto{\pgfqpoint{1.380274in}{1.284536in}}%
\pgfpathlineto{\pgfqpoint{1.381198in}{1.298147in}}%
\pgfpathlineto{\pgfqpoint{1.386749in}{1.311759in}}%
\pgfpathlineto{\pgfqpoint{1.393748in}{1.321219in}}%
\pgfpathlineto{\pgfqpoint{1.397586in}{1.325370in}}%
\pgfpathlineto{\pgfqpoint{1.409405in}{1.334597in}}%
\pgfpathlineto{\pgfqpoint{1.417901in}{1.338981in}}%
\pgfpathlineto{\pgfqpoint{1.425061in}{1.342113in}}%
\pgfpathlineto{\pgfqpoint{1.440718in}{1.345205in}}%
\pgfpathlineto{\pgfqpoint{1.456375in}{1.344433in}}%
\pgfpathlineto{\pgfqpoint{1.472031in}{1.339791in}}%
\pgfpathlineto{\pgfqpoint{1.473572in}{1.338981in}}%
\pgfpathlineto{\pgfqpoint{1.487688in}{1.330173in}}%
\pgfpathlineto{\pgfqpoint{1.493213in}{1.325370in}}%
\pgfpathlineto{\pgfqpoint{1.503344in}{1.313098in}}%
\pgfpathlineto{\pgfqpoint{1.504276in}{1.311759in}}%
\pgfpathlineto{\pgfqpoint{1.509615in}{1.298147in}}%
\pgfpathlineto{\pgfqpoint{1.510504in}{1.284536in}}%
\pgfpathlineto{\pgfqpoint{1.506947in}{1.270925in}}%
\pgfpathlineto{\pgfqpoint{1.503344in}{1.264701in}}%
\pgfpathlineto{\pgfqpoint{1.498302in}{1.257314in}}%
\pgfpathlineto{\pgfqpoint{1.487688in}{1.247039in}}%
\pgfpathlineto{\pgfqpoint{1.482913in}{1.243703in}}%
\pgfpathlineto{\pgfqpoint{1.472031in}{1.237618in}}%
\pgfpathlineto{\pgfqpoint{1.456375in}{1.232792in}}%
\pgfpathlineto{\pgfqpoint{1.440718in}{1.231989in}}%
\pgfpathlineto{\pgfqpoint{1.425061in}{1.235204in}}%
\pgfpathlineto{\pgfqpoint{1.409405in}{1.242451in}}%
\pgfpathlineto{\pgfqpoint{1.407589in}{1.243703in}}%
\pgfpathclose%
\pgfpathmoveto{\pgfqpoint{1.753849in}{1.214684in}}%
\pgfpathlineto{\pgfqpoint{1.769506in}{1.216116in}}%
\pgfpathlineto{\pgfqpoint{1.770671in}{1.216481in}}%
\pgfpathlineto{\pgfqpoint{1.785162in}{1.220966in}}%
\pgfpathlineto{\pgfqpoint{1.800819in}{1.229266in}}%
\pgfpathlineto{\pgfqpoint{1.801985in}{1.230092in}}%
\pgfpathlineto{\pgfqpoint{1.816476in}{1.241683in}}%
\pgfpathlineto{\pgfqpoint{1.818605in}{1.243703in}}%
\pgfpathlineto{\pgfqpoint{1.829882in}{1.257314in}}%
\pgfpathlineto{\pgfqpoint{1.832132in}{1.261441in}}%
\pgfpathlineto{\pgfqpoint{1.837121in}{1.270925in}}%
\pgfpathlineto{\pgfqpoint{1.840329in}{1.284536in}}%
\pgfpathlineto{\pgfqpoint{1.839528in}{1.298147in}}%
\pgfpathlineto{\pgfqpoint{1.834712in}{1.311759in}}%
\pgfpathlineto{\pgfqpoint{1.832132in}{1.315841in}}%
\pgfpathlineto{\pgfqpoint{1.825849in}{1.325370in}}%
\pgfpathlineto{\pgfqpoint{1.816476in}{1.335440in}}%
\pgfpathlineto{\pgfqpoint{1.812596in}{1.338981in}}%
\pgfpathlineto{\pgfqpoint{1.800819in}{1.347878in}}%
\pgfpathlineto{\pgfqpoint{1.792322in}{1.352592in}}%
\pgfpathlineto{\pgfqpoint{1.785162in}{1.356311in}}%
\pgfpathlineto{\pgfqpoint{1.769506in}{1.361161in}}%
\pgfpathlineto{\pgfqpoint{1.753849in}{1.362544in}}%
\pgfpathlineto{\pgfqpoint{1.738193in}{1.360469in}}%
\pgfpathlineto{\pgfqpoint{1.722536in}{1.354924in}}%
\pgfpathlineto{\pgfqpoint{1.718348in}{1.352592in}}%
\pgfpathlineto{\pgfqpoint{1.706880in}{1.345751in}}%
\pgfpathlineto{\pgfqpoint{1.698282in}{1.338981in}}%
\pgfpathlineto{\pgfqpoint{1.691223in}{1.332250in}}%
\pgfpathlineto{\pgfqpoint{1.684955in}{1.325370in}}%
\pgfpathlineto{\pgfqpoint{1.676161in}{1.311759in}}%
\pgfpathlineto{\pgfqpoint{1.675567in}{1.310101in}}%
\pgfpathlineto{\pgfqpoint{1.671300in}{1.298147in}}%
\pgfpathlineto{\pgfqpoint{1.670490in}{1.284536in}}%
\pgfpathlineto{\pgfqpoint{1.673734in}{1.270925in}}%
\pgfpathlineto{\pgfqpoint{1.675567in}{1.267433in}}%
\pgfpathlineto{\pgfqpoint{1.680958in}{1.257314in}}%
\pgfpathlineto{\pgfqpoint{1.691223in}{1.244767in}}%
\pgfpathlineto{\pgfqpoint{1.692219in}{1.243703in}}%
\pgfpathlineto{\pgfqpoint{1.706880in}{1.231422in}}%
\pgfpathlineto{\pgfqpoint{1.708967in}{1.230092in}}%
\pgfpathlineto{\pgfqpoint{1.722536in}{1.222349in}}%
\pgfpathlineto{\pgfqpoint{1.738193in}{1.216819in}}%
\pgfpathlineto{\pgfqpoint{1.740730in}{1.216481in}}%
\pgfpathlineto{\pgfqpoint{1.753849in}{1.214684in}}%
\pgfpathclose%
\pgfpathmoveto{\pgfqpoint{1.717865in}{1.243703in}}%
\pgfpathlineto{\pgfqpoint{1.706880in}{1.252647in}}%
\pgfpathlineto{\pgfqpoint{1.702462in}{1.257314in}}%
\pgfpathlineto{\pgfqpoint{1.693904in}{1.270925in}}%
\pgfpathlineto{\pgfqpoint{1.691223in}{1.280496in}}%
\pgfpathlineto{\pgfqpoint{1.690201in}{1.284536in}}%
\pgfpathlineto{\pgfqpoint{1.691070in}{1.298147in}}%
\pgfpathlineto{\pgfqpoint{1.691223in}{1.298555in}}%
\pgfpathlineto{\pgfqpoint{1.696755in}{1.311759in}}%
\pgfpathlineto{\pgfqpoint{1.706880in}{1.324945in}}%
\pgfpathlineto{\pgfqpoint{1.707299in}{1.325370in}}%
\pgfpathlineto{\pgfqpoint{1.722536in}{1.336544in}}%
\pgfpathlineto{\pgfqpoint{1.727828in}{1.338981in}}%
\pgfpathlineto{\pgfqpoint{1.738193in}{1.343041in}}%
\pgfpathlineto{\pgfqpoint{1.753849in}{1.345359in}}%
\pgfpathlineto{\pgfqpoint{1.769506in}{1.343814in}}%
\pgfpathlineto{\pgfqpoint{1.783512in}{1.338981in}}%
\pgfpathlineto{\pgfqpoint{1.785162in}{1.338314in}}%
\pgfpathlineto{\pgfqpoint{1.800819in}{1.327698in}}%
\pgfpathlineto{\pgfqpoint{1.803376in}{1.325370in}}%
\pgfpathlineto{\pgfqpoint{1.814172in}{1.311759in}}%
\pgfpathlineto{\pgfqpoint{1.816476in}{1.306495in}}%
\pgfpathlineto{\pgfqpoint{1.819681in}{1.298147in}}%
\pgfpathlineto{\pgfqpoint{1.820558in}{1.284536in}}%
\pgfpathlineto{\pgfqpoint{1.817047in}{1.270925in}}%
\pgfpathlineto{\pgfqpoint{1.816476in}{1.269913in}}%
\pgfpathlineto{\pgfqpoint{1.808283in}{1.257314in}}%
\pgfpathlineto{\pgfqpoint{1.800819in}{1.249748in}}%
\pgfpathlineto{\pgfqpoint{1.792836in}{1.243703in}}%
\pgfpathlineto{\pgfqpoint{1.785162in}{1.239067in}}%
\pgfpathlineto{\pgfqpoint{1.769506in}{1.233435in}}%
\pgfpathlineto{\pgfqpoint{1.753849in}{1.231829in}}%
\pgfpathlineto{\pgfqpoint{1.738193in}{1.234239in}}%
\pgfpathlineto{\pgfqpoint{1.722536in}{1.240678in}}%
\pgfpathlineto{\pgfqpoint{1.717865in}{1.243703in}}%
\pgfpathclose%
\pgfpathmoveto{\pgfqpoint{0.501324in}{1.485558in}}%
\pgfpathlineto{\pgfqpoint{0.516981in}{1.484149in}}%
\pgfpathlineto{\pgfqpoint{0.532637in}{1.486263in}}%
\pgfpathlineto{\pgfqpoint{0.539507in}{1.488703in}}%
\pgfpathlineto{\pgfqpoint{0.548294in}{1.491861in}}%
\pgfpathlineto{\pgfqpoint{0.563950in}{1.500891in}}%
\pgfpathlineto{\pgfqpoint{0.565844in}{1.502314in}}%
\pgfpathlineto{\pgfqpoint{0.579607in}{1.514279in}}%
\pgfpathlineto{\pgfqpoint{0.581243in}{1.515925in}}%
\pgfpathlineto{\pgfqpoint{0.591631in}{1.529536in}}%
\pgfpathlineto{\pgfqpoint{0.595263in}{1.537175in}}%
\pgfpathlineto{\pgfqpoint{0.598070in}{1.543148in}}%
\pgfpathlineto{\pgfqpoint{0.600502in}{1.556759in}}%
\pgfpathlineto{\pgfqpoint{0.598881in}{1.570370in}}%
\pgfpathlineto{\pgfqpoint{0.595263in}{1.579110in}}%
\pgfpathlineto{\pgfqpoint{0.593230in}{1.583981in}}%
\pgfpathlineto{\pgfqpoint{0.583638in}{1.597592in}}%
\pgfpathlineto{\pgfqpoint{0.579607in}{1.601817in}}%
\pgfpathlineto{\pgfqpoint{0.569274in}{1.611203in}}%
\pgfpathlineto{\pgfqpoint{0.563950in}{1.615290in}}%
\pgfpathlineto{\pgfqpoint{0.548294in}{1.624401in}}%
\pgfpathlineto{\pgfqpoint{0.547170in}{1.624814in}}%
\pgfpathlineto{\pgfqpoint{0.532637in}{1.629989in}}%
\pgfpathlineto{\pgfqpoint{0.516981in}{1.632079in}}%
\pgfpathlineto{\pgfqpoint{0.501324in}{1.630686in}}%
\pgfpathlineto{\pgfqpoint{0.485668in}{1.625800in}}%
\pgfpathlineto{\pgfqpoint{0.483757in}{1.624814in}}%
\pgfpathlineto{\pgfqpoint{0.470011in}{1.617390in}}%
\pgfpathlineto{\pgfqpoint{0.461611in}{1.611203in}}%
\pgfpathlineto{\pgfqpoint{0.454354in}{1.604894in}}%
\pgfpathlineto{\pgfqpoint{0.447237in}{1.597592in}}%
\pgfpathlineto{\pgfqpoint{0.438698in}{1.585641in}}%
\pgfpathlineto{\pgfqpoint{0.437564in}{1.583981in}}%
\pgfpathlineto{\pgfqpoint{0.431944in}{1.570370in}}%
\pgfpathlineto{\pgfqpoint{0.430341in}{1.556759in}}%
\pgfpathlineto{\pgfqpoint{0.432746in}{1.543147in}}%
\pgfpathlineto{\pgfqpoint{0.438698in}{1.530513in}}%
\pgfpathlineto{\pgfqpoint{0.439173in}{1.529536in}}%
\pgfpathlineto{\pgfqpoint{0.449653in}{1.515925in}}%
\pgfpathlineto{\pgfqpoint{0.454354in}{1.511297in}}%
\pgfpathlineto{\pgfqpoint{0.465150in}{1.502314in}}%
\pgfpathlineto{\pgfqpoint{0.470011in}{1.498810in}}%
\pgfpathlineto{\pgfqpoint{0.485668in}{1.490471in}}%
\pgfpathlineto{\pgfqpoint{0.491271in}{1.488703in}}%
\pgfpathlineto{\pgfqpoint{0.501324in}{1.485558in}}%
\pgfpathclose%
\pgfpathmoveto{\pgfqpoint{0.506186in}{1.502314in}}%
\pgfpathlineto{\pgfqpoint{0.501324in}{1.502829in}}%
\pgfpathlineto{\pgfqpoint{0.485668in}{1.508611in}}%
\pgfpathlineto{\pgfqpoint{0.474001in}{1.515925in}}%
\pgfpathlineto{\pgfqpoint{0.470011in}{1.519151in}}%
\pgfpathlineto{\pgfqpoint{0.460387in}{1.529536in}}%
\pgfpathlineto{\pgfqpoint{0.454354in}{1.539932in}}%
\pgfpathlineto{\pgfqpoint{0.452729in}{1.543148in}}%
\pgfpathlineto{\pgfqpoint{0.450097in}{1.556759in}}%
\pgfpathlineto{\pgfqpoint{0.451851in}{1.570370in}}%
\pgfpathlineto{\pgfqpoint{0.454354in}{1.575992in}}%
\pgfpathlineto{\pgfqpoint{0.458424in}{1.583981in}}%
\pgfpathlineto{\pgfqpoint{0.470011in}{1.597383in}}%
\pgfpathlineto{\pgfqpoint{0.470252in}{1.597592in}}%
\pgfpathlineto{\pgfqpoint{0.485668in}{1.607665in}}%
\pgfpathlineto{\pgfqpoint{0.494857in}{1.611203in}}%
\pgfpathlineto{\pgfqpoint{0.501324in}{1.613379in}}%
\pgfpathlineto{\pgfqpoint{0.516981in}{1.614904in}}%
\pgfpathlineto{\pgfqpoint{0.532637in}{1.612616in}}%
\pgfpathlineto{\pgfqpoint{0.536336in}{1.611203in}}%
\pgfpathlineto{\pgfqpoint{0.548294in}{1.605959in}}%
\pgfpathlineto{\pgfqpoint{0.560239in}{1.597592in}}%
\pgfpathlineto{\pgfqpoint{0.563950in}{1.594123in}}%
\pgfpathlineto{\pgfqpoint{0.572364in}{1.583981in}}%
\pgfpathlineto{\pgfqpoint{0.579015in}{1.570370in}}%
\pgfpathlineto{\pgfqpoint{0.579607in}{1.566143in}}%
\pgfpathlineto{\pgfqpoint{0.580803in}{1.556759in}}%
\pgfpathlineto{\pgfqpoint{0.579607in}{1.550481in}}%
\pgfpathlineto{\pgfqpoint{0.578066in}{1.543148in}}%
\pgfpathlineto{\pgfqpoint{0.570461in}{1.529536in}}%
\pgfpathlineto{\pgfqpoint{0.563950in}{1.522214in}}%
\pgfpathlineto{\pgfqpoint{0.556717in}{1.515925in}}%
\pgfpathlineto{\pgfqpoint{0.548294in}{1.510265in}}%
\pgfpathlineto{\pgfqpoint{0.532637in}{1.503654in}}%
\pgfpathlineto{\pgfqpoint{0.524202in}{1.502314in}}%
\pgfpathlineto{\pgfqpoint{0.516981in}{1.501274in}}%
\pgfpathlineto{\pgfqpoint{0.506186in}{1.502314in}}%
\pgfpathclose%
\pgfpathmoveto{\pgfqpoint{0.814455in}{1.484994in}}%
\pgfpathlineto{\pgfqpoint{0.830112in}{1.484289in}}%
\pgfpathlineto{\pgfqpoint{0.845769in}{1.487110in}}%
\pgfpathlineto{\pgfqpoint{0.849785in}{1.488703in}}%
\pgfpathlineto{\pgfqpoint{0.861425in}{1.493390in}}%
\pgfpathlineto{\pgfqpoint{0.875858in}{1.502314in}}%
\pgfpathlineto{\pgfqpoint{0.877082in}{1.503180in}}%
\pgfpathlineto{\pgfqpoint{0.891208in}{1.515925in}}%
\pgfpathlineto{\pgfqpoint{0.892738in}{1.517740in}}%
\pgfpathlineto{\pgfqpoint{0.901644in}{1.529536in}}%
\pgfpathlineto{\pgfqpoint{0.908005in}{1.543148in}}%
\pgfpathlineto{\pgfqpoint{0.908395in}{1.545354in}}%
\pgfpathlineto{\pgfqpoint{0.910462in}{1.556759in}}%
\pgfpathlineto{\pgfqpoint{0.908815in}{1.570370in}}%
\pgfpathlineto{\pgfqpoint{0.908395in}{1.571383in}}%
\pgfpathlineto{\pgfqpoint{0.903236in}{1.583981in}}%
\pgfpathlineto{\pgfqpoint{0.893688in}{1.597592in}}%
\pgfpathlineto{\pgfqpoint{0.892738in}{1.598606in}}%
\pgfpathlineto{\pgfqpoint{0.879405in}{1.611203in}}%
\pgfpathlineto{\pgfqpoint{0.877082in}{1.613054in}}%
\pgfpathlineto{\pgfqpoint{0.861425in}{1.622858in}}%
\pgfpathlineto{\pgfqpoint{0.856678in}{1.624814in}}%
\pgfpathlineto{\pgfqpoint{0.845769in}{1.629152in}}%
\pgfpathlineto{\pgfqpoint{0.830112in}{1.631940in}}%
\pgfpathlineto{\pgfqpoint{0.814455in}{1.631243in}}%
\pgfpathlineto{\pgfqpoint{0.798799in}{1.627057in}}%
\pgfpathlineto{\pgfqpoint{0.794103in}{1.624814in}}%
\pgfpathlineto{\pgfqpoint{0.783142in}{1.619352in}}%
\pgfpathlineto{\pgfqpoint{0.771558in}{1.611203in}}%
\pgfpathlineto{\pgfqpoint{0.767486in}{1.607831in}}%
\pgfpathlineto{\pgfqpoint{0.757252in}{1.597592in}}%
\pgfpathlineto{\pgfqpoint{0.751829in}{1.590205in}}%
\pgfpathlineto{\pgfqpoint{0.747551in}{1.583981in}}%
\pgfpathlineto{\pgfqpoint{0.741973in}{1.570370in}}%
\pgfpathlineto{\pgfqpoint{0.740382in}{1.556759in}}%
\pgfpathlineto{\pgfqpoint{0.742769in}{1.543148in}}%
\pgfpathlineto{\pgfqpoint{0.749147in}{1.529536in}}%
\pgfpathlineto{\pgfqpoint{0.751829in}{1.525896in}}%
\pgfpathlineto{\pgfqpoint{0.759699in}{1.515925in}}%
\pgfpathlineto{\pgfqpoint{0.767486in}{1.508451in}}%
\pgfpathlineto{\pgfqpoint{0.775228in}{1.502314in}}%
\pgfpathlineto{\pgfqpoint{0.783142in}{1.496865in}}%
\pgfpathlineto{\pgfqpoint{0.798799in}{1.489220in}}%
\pgfpathlineto{\pgfqpoint{0.800706in}{1.488703in}}%
\pgfpathlineto{\pgfqpoint{0.814455in}{1.484994in}}%
\pgfpathclose%
\pgfpathmoveto{\pgfqpoint{0.813987in}{1.502314in}}%
\pgfpathlineto{\pgfqpoint{0.798799in}{1.507123in}}%
\pgfpathlineto{\pgfqpoint{0.783631in}{1.515925in}}%
\pgfpathlineto{\pgfqpoint{0.783142in}{1.516290in}}%
\pgfpathlineto{\pgfqpoint{0.770289in}{1.529536in}}%
\pgfpathlineto{\pgfqpoint{0.767486in}{1.534137in}}%
\pgfpathlineto{\pgfqpoint{0.762815in}{1.543148in}}%
\pgfpathlineto{\pgfqpoint{0.760149in}{1.556759in}}%
\pgfpathlineto{\pgfqpoint{0.761926in}{1.570370in}}%
\pgfpathlineto{\pgfqpoint{0.767486in}{1.582546in}}%
\pgfpathlineto{\pgfqpoint{0.768253in}{1.583981in}}%
\pgfpathlineto{\pgfqpoint{0.780465in}{1.597592in}}%
\pgfpathlineto{\pgfqpoint{0.783142in}{1.599815in}}%
\pgfpathlineto{\pgfqpoint{0.798799in}{1.609200in}}%
\pgfpathlineto{\pgfqpoint{0.804853in}{1.611203in}}%
\pgfpathlineto{\pgfqpoint{0.814455in}{1.613989in}}%
\pgfpathlineto{\pgfqpoint{0.830112in}{1.614752in}}%
\pgfpathlineto{\pgfqpoint{0.845769in}{1.611700in}}%
\pgfpathlineto{\pgfqpoint{0.846933in}{1.611203in}}%
\pgfpathlineto{\pgfqpoint{0.861425in}{1.604081in}}%
\pgfpathlineto{\pgfqpoint{0.870128in}{1.597592in}}%
\pgfpathlineto{\pgfqpoint{0.877082in}{1.590652in}}%
\pgfpathlineto{\pgfqpoint{0.882414in}{1.583981in}}%
\pgfpathlineto{\pgfqpoint{0.888893in}{1.570370in}}%
\pgfpathlineto{\pgfqpoint{0.890741in}{1.556759in}}%
\pgfpathlineto{\pgfqpoint{0.887968in}{1.543147in}}%
\pgfpathlineto{\pgfqpoint{0.880561in}{1.529536in}}%
\pgfpathlineto{\pgfqpoint{0.877082in}{1.525476in}}%
\pgfpathlineto{\pgfqpoint{0.866793in}{1.515925in}}%
\pgfpathlineto{\pgfqpoint{0.861425in}{1.512085in}}%
\pgfpathlineto{\pgfqpoint{0.845769in}{1.504645in}}%
\pgfpathlineto{\pgfqpoint{0.834760in}{1.502314in}}%
\pgfpathlineto{\pgfqpoint{0.830112in}{1.501425in}}%
\pgfpathlineto{\pgfqpoint{0.814455in}{1.502181in}}%
\pgfpathlineto{\pgfqpoint{0.813987in}{1.502314in}}%
\pgfpathclose%
\pgfpathmoveto{\pgfqpoint{1.111930in}{1.488099in}}%
\pgfpathlineto{\pgfqpoint{1.127587in}{1.484571in}}%
\pgfpathlineto{\pgfqpoint{1.143243in}{1.484571in}}%
\pgfpathlineto{\pgfqpoint{1.158900in}{1.488099in}}%
\pgfpathlineto{\pgfqpoint{1.160283in}{1.488703in}}%
\pgfpathlineto{\pgfqpoint{1.174556in}{1.495058in}}%
\pgfpathlineto{\pgfqpoint{1.185646in}{1.502314in}}%
\pgfpathlineto{\pgfqpoint{1.190213in}{1.505744in}}%
\pgfpathlineto{\pgfqpoint{1.201132in}{1.515925in}}%
\pgfpathlineto{\pgfqpoint{1.205870in}{1.521744in}}%
\pgfpathlineto{\pgfqpoint{1.211670in}{1.529536in}}%
\pgfpathlineto{\pgfqpoint{1.218027in}{1.543148in}}%
\pgfpathlineto{\pgfqpoint{1.220406in}{1.556759in}}%
\pgfpathlineto{\pgfqpoint{1.218820in}{1.570370in}}%
\pgfpathlineto{\pgfqpoint{1.213260in}{1.583981in}}%
\pgfpathlineto{\pgfqpoint{1.205870in}{1.594623in}}%
\pgfpathlineto{\pgfqpoint{1.203622in}{1.597592in}}%
\pgfpathlineto{\pgfqpoint{1.190213in}{1.610623in}}%
\pgfpathlineto{\pgfqpoint{1.189474in}{1.611203in}}%
\pgfpathlineto{\pgfqpoint{1.174556in}{1.621175in}}%
\pgfpathlineto{\pgfqpoint{1.166569in}{1.624814in}}%
\pgfpathlineto{\pgfqpoint{1.158900in}{1.628174in}}%
\pgfpathlineto{\pgfqpoint{1.143243in}{1.631661in}}%
\pgfpathlineto{\pgfqpoint{1.127587in}{1.631661in}}%
\pgfpathlineto{\pgfqpoint{1.111930in}{1.628174in}}%
\pgfpathlineto{\pgfqpoint{1.104261in}{1.624814in}}%
\pgfpathlineto{\pgfqpoint{1.096274in}{1.621175in}}%
\pgfpathlineto{\pgfqpoint{1.081356in}{1.611203in}}%
\pgfpathlineto{\pgfqpoint{1.080617in}{1.610623in}}%
\pgfpathlineto{\pgfqpoint{1.067208in}{1.597592in}}%
\pgfpathlineto{\pgfqpoint{1.064960in}{1.594623in}}%
\pgfpathlineto{\pgfqpoint{1.057570in}{1.583981in}}%
\pgfpathlineto{\pgfqpoint{1.052010in}{1.570370in}}%
\pgfpathlineto{\pgfqpoint{1.050424in}{1.556759in}}%
\pgfpathlineto{\pgfqpoint{1.052803in}{1.543148in}}%
\pgfpathlineto{\pgfqpoint{1.059160in}{1.529536in}}%
\pgfpathlineto{\pgfqpoint{1.064960in}{1.521744in}}%
\pgfpathlineto{\pgfqpoint{1.069698in}{1.515925in}}%
\pgfpathlineto{\pgfqpoint{1.080617in}{1.505744in}}%
\pgfpathlineto{\pgfqpoint{1.085184in}{1.502314in}}%
\pgfpathlineto{\pgfqpoint{1.096274in}{1.495058in}}%
\pgfpathlineto{\pgfqpoint{1.110547in}{1.488703in}}%
\pgfpathlineto{\pgfqpoint{1.111930in}{1.488099in}}%
\pgfpathclose%
\pgfpathmoveto{\pgfqpoint{1.125121in}{1.502314in}}%
\pgfpathlineto{\pgfqpoint{1.111930in}{1.505801in}}%
\pgfpathlineto{\pgfqpoint{1.096274in}{1.514070in}}%
\pgfpathlineto{\pgfqpoint{1.093823in}{1.515925in}}%
\pgfpathlineto{\pgfqpoint{1.080617in}{1.528934in}}%
\pgfpathlineto{\pgfqpoint{1.080117in}{1.529536in}}%
\pgfpathlineto{\pgfqpoint{1.072868in}{1.543148in}}%
\pgfpathlineto{\pgfqpoint{1.070156in}{1.556759in}}%
\pgfpathlineto{\pgfqpoint{1.071964in}{1.570370in}}%
\pgfpathlineto{\pgfqpoint{1.078304in}{1.583981in}}%
\pgfpathlineto{\pgfqpoint{1.080617in}{1.586972in}}%
\pgfpathlineto{\pgfqpoint{1.090644in}{1.597592in}}%
\pgfpathlineto{\pgfqpoint{1.096274in}{1.602033in}}%
\pgfpathlineto{\pgfqpoint{1.111930in}{1.610564in}}%
\pgfpathlineto{\pgfqpoint{1.114244in}{1.611203in}}%
\pgfpathlineto{\pgfqpoint{1.127587in}{1.614447in}}%
\pgfpathlineto{\pgfqpoint{1.143243in}{1.614447in}}%
\pgfpathlineto{\pgfqpoint{1.156586in}{1.611203in}}%
\pgfpathlineto{\pgfqpoint{1.158900in}{1.610564in}}%
\pgfpathlineto{\pgfqpoint{1.174556in}{1.602033in}}%
\pgfpathlineto{\pgfqpoint{1.180186in}{1.597592in}}%
\pgfpathlineto{\pgfqpoint{1.190213in}{1.586972in}}%
\pgfpathlineto{\pgfqpoint{1.192526in}{1.583981in}}%
\pgfpathlineto{\pgfqpoint{1.198866in}{1.570370in}}%
\pgfpathlineto{\pgfqpoint{1.200674in}{1.556759in}}%
\pgfpathlineto{\pgfqpoint{1.197962in}{1.543148in}}%
\pgfpathlineto{\pgfqpoint{1.190713in}{1.529536in}}%
\pgfpathlineto{\pgfqpoint{1.190213in}{1.528934in}}%
\pgfpathlineto{\pgfqpoint{1.177007in}{1.515925in}}%
\pgfpathlineto{\pgfqpoint{1.174556in}{1.514070in}}%
\pgfpathlineto{\pgfqpoint{1.158900in}{1.505801in}}%
\pgfpathlineto{\pgfqpoint{1.145709in}{1.502314in}}%
\pgfpathlineto{\pgfqpoint{1.143243in}{1.501728in}}%
\pgfpathlineto{\pgfqpoint{1.127587in}{1.501728in}}%
\pgfpathlineto{\pgfqpoint{1.125121in}{1.502314in}}%
\pgfpathclose%
\pgfpathmoveto{\pgfqpoint{1.425061in}{1.487110in}}%
\pgfpathlineto{\pgfqpoint{1.440718in}{1.484289in}}%
\pgfpathlineto{\pgfqpoint{1.456375in}{1.484994in}}%
\pgfpathlineto{\pgfqpoint{1.470124in}{1.488703in}}%
\pgfpathlineto{\pgfqpoint{1.472031in}{1.489220in}}%
\pgfpathlineto{\pgfqpoint{1.487688in}{1.496865in}}%
\pgfpathlineto{\pgfqpoint{1.495602in}{1.502314in}}%
\pgfpathlineto{\pgfqpoint{1.503344in}{1.508451in}}%
\pgfpathlineto{\pgfqpoint{1.511131in}{1.515925in}}%
\pgfpathlineto{\pgfqpoint{1.519001in}{1.525896in}}%
\pgfpathlineto{\pgfqpoint{1.521683in}{1.529536in}}%
\pgfpathlineto{\pgfqpoint{1.528061in}{1.543148in}}%
\pgfpathlineto{\pgfqpoint{1.530448in}{1.556759in}}%
\pgfpathlineto{\pgfqpoint{1.528857in}{1.570370in}}%
\pgfpathlineto{\pgfqpoint{1.523279in}{1.583981in}}%
\pgfpathlineto{\pgfqpoint{1.519001in}{1.590205in}}%
\pgfpathlineto{\pgfqpoint{1.513578in}{1.597592in}}%
\pgfpathlineto{\pgfqpoint{1.503344in}{1.607831in}}%
\pgfpathlineto{\pgfqpoint{1.499272in}{1.611203in}}%
\pgfpathlineto{\pgfqpoint{1.487688in}{1.619352in}}%
\pgfpathlineto{\pgfqpoint{1.476727in}{1.624814in}}%
\pgfpathlineto{\pgfqpoint{1.472031in}{1.627057in}}%
\pgfpathlineto{\pgfqpoint{1.456375in}{1.631243in}}%
\pgfpathlineto{\pgfqpoint{1.440718in}{1.631940in}}%
\pgfpathlineto{\pgfqpoint{1.425061in}{1.629152in}}%
\pgfpathlineto{\pgfqpoint{1.414152in}{1.624814in}}%
\pgfpathlineto{\pgfqpoint{1.409405in}{1.622858in}}%
\pgfpathlineto{\pgfqpoint{1.393748in}{1.613054in}}%
\pgfpathlineto{\pgfqpoint{1.391425in}{1.611203in}}%
\pgfpathlineto{\pgfqpoint{1.378092in}{1.598606in}}%
\pgfpathlineto{\pgfqpoint{1.377142in}{1.597592in}}%
\pgfpathlineto{\pgfqpoint{1.367594in}{1.583981in}}%
\pgfpathlineto{\pgfqpoint{1.362435in}{1.571383in}}%
\pgfpathlineto{\pgfqpoint{1.362015in}{1.570370in}}%
\pgfpathlineto{\pgfqpoint{1.360368in}{1.556759in}}%
\pgfpathlineto{\pgfqpoint{1.362435in}{1.545354in}}%
\pgfpathlineto{\pgfqpoint{1.362825in}{1.543148in}}%
\pgfpathlineto{\pgfqpoint{1.369186in}{1.529536in}}%
\pgfpathlineto{\pgfqpoint{1.378092in}{1.517740in}}%
\pgfpathlineto{\pgfqpoint{1.379622in}{1.515925in}}%
\pgfpathlineto{\pgfqpoint{1.393748in}{1.503180in}}%
\pgfpathlineto{\pgfqpoint{1.394972in}{1.502314in}}%
\pgfpathlineto{\pgfqpoint{1.409405in}{1.493390in}}%
\pgfpathlineto{\pgfqpoint{1.421045in}{1.488703in}}%
\pgfpathlineto{\pgfqpoint{1.425061in}{1.487110in}}%
\pgfpathclose%
\pgfpathmoveto{\pgfqpoint{1.436070in}{1.502314in}}%
\pgfpathlineto{\pgfqpoint{1.425061in}{1.504645in}}%
\pgfpathlineto{\pgfqpoint{1.409405in}{1.512085in}}%
\pgfpathlineto{\pgfqpoint{1.404037in}{1.515925in}}%
\pgfpathlineto{\pgfqpoint{1.393748in}{1.525476in}}%
\pgfpathlineto{\pgfqpoint{1.390269in}{1.529536in}}%
\pgfpathlineto{\pgfqpoint{1.382862in}{1.543148in}}%
\pgfpathlineto{\pgfqpoint{1.380089in}{1.556759in}}%
\pgfpathlineto{\pgfqpoint{1.381937in}{1.570370in}}%
\pgfpathlineto{\pgfqpoint{1.388416in}{1.583981in}}%
\pgfpathlineto{\pgfqpoint{1.393748in}{1.590652in}}%
\pgfpathlineto{\pgfqpoint{1.400702in}{1.597592in}}%
\pgfpathlineto{\pgfqpoint{1.409405in}{1.604081in}}%
\pgfpathlineto{\pgfqpoint{1.423897in}{1.611203in}}%
\pgfpathlineto{\pgfqpoint{1.425061in}{1.611700in}}%
\pgfpathlineto{\pgfqpoint{1.440718in}{1.614752in}}%
\pgfpathlineto{\pgfqpoint{1.456375in}{1.613989in}}%
\pgfpathlineto{\pgfqpoint{1.465977in}{1.611203in}}%
\pgfpathlineto{\pgfqpoint{1.472031in}{1.609200in}}%
\pgfpathlineto{\pgfqpoint{1.487688in}{1.599815in}}%
\pgfpathlineto{\pgfqpoint{1.490365in}{1.597592in}}%
\pgfpathlineto{\pgfqpoint{1.502577in}{1.583981in}}%
\pgfpathlineto{\pgfqpoint{1.503344in}{1.582546in}}%
\pgfpathlineto{\pgfqpoint{1.508904in}{1.570370in}}%
\pgfpathlineto{\pgfqpoint{1.510681in}{1.556759in}}%
\pgfpathlineto{\pgfqpoint{1.508015in}{1.543148in}}%
\pgfpathlineto{\pgfqpoint{1.503344in}{1.534137in}}%
\pgfpathlineto{\pgfqpoint{1.500541in}{1.529536in}}%
\pgfpathlineto{\pgfqpoint{1.487688in}{1.516290in}}%
\pgfpathlineto{\pgfqpoint{1.487199in}{1.515925in}}%
\pgfpathlineto{\pgfqpoint{1.472031in}{1.507123in}}%
\pgfpathlineto{\pgfqpoint{1.456843in}{1.502314in}}%
\pgfpathlineto{\pgfqpoint{1.456375in}{1.502181in}}%
\pgfpathlineto{\pgfqpoint{1.440718in}{1.501425in}}%
\pgfpathlineto{\pgfqpoint{1.436070in}{1.502314in}}%
\pgfpathclose%
\pgfpathmoveto{\pgfqpoint{1.738193in}{1.486263in}}%
\pgfpathlineto{\pgfqpoint{1.753849in}{1.484149in}}%
\pgfpathlineto{\pgfqpoint{1.769506in}{1.485558in}}%
\pgfpathlineto{\pgfqpoint{1.779559in}{1.488703in}}%
\pgfpathlineto{\pgfqpoint{1.785162in}{1.490471in}}%
\pgfpathlineto{\pgfqpoint{1.800819in}{1.498810in}}%
\pgfpathlineto{\pgfqpoint{1.805680in}{1.502314in}}%
\pgfpathlineto{\pgfqpoint{1.816476in}{1.511297in}}%
\pgfpathlineto{\pgfqpoint{1.821177in}{1.515925in}}%
\pgfpathlineto{\pgfqpoint{1.831657in}{1.529536in}}%
\pgfpathlineto{\pgfqpoint{1.832132in}{1.530513in}}%
\pgfpathlineto{\pgfqpoint{1.838084in}{1.543148in}}%
\pgfpathlineto{\pgfqpoint{1.840489in}{1.556759in}}%
\pgfpathlineto{\pgfqpoint{1.838886in}{1.570370in}}%
\pgfpathlineto{\pgfqpoint{1.833266in}{1.583981in}}%
\pgfpathlineto{\pgfqpoint{1.832132in}{1.585641in}}%
\pgfpathlineto{\pgfqpoint{1.823593in}{1.597592in}}%
\pgfpathlineto{\pgfqpoint{1.816476in}{1.604894in}}%
\pgfpathlineto{\pgfqpoint{1.809219in}{1.611203in}}%
\pgfpathlineto{\pgfqpoint{1.800819in}{1.617390in}}%
\pgfpathlineto{\pgfqpoint{1.787073in}{1.624814in}}%
\pgfpathlineto{\pgfqpoint{1.785162in}{1.625800in}}%
\pgfpathlineto{\pgfqpoint{1.769506in}{1.630686in}}%
\pgfpathlineto{\pgfqpoint{1.753849in}{1.632079in}}%
\pgfpathlineto{\pgfqpoint{1.738193in}{1.629989in}}%
\pgfpathlineto{\pgfqpoint{1.723660in}{1.624814in}}%
\pgfpathlineto{\pgfqpoint{1.722536in}{1.624401in}}%
\pgfpathlineto{\pgfqpoint{1.706880in}{1.615290in}}%
\pgfpathlineto{\pgfqpoint{1.701556in}{1.611203in}}%
\pgfpathlineto{\pgfqpoint{1.691223in}{1.601817in}}%
\pgfpathlineto{\pgfqpoint{1.687192in}{1.597592in}}%
\pgfpathlineto{\pgfqpoint{1.677600in}{1.583981in}}%
\pgfpathlineto{\pgfqpoint{1.675567in}{1.579110in}}%
\pgfpathlineto{\pgfqpoint{1.671949in}{1.570370in}}%
\pgfpathlineto{\pgfqpoint{1.670328in}{1.556759in}}%
\pgfpathlineto{\pgfqpoint{1.672760in}{1.543148in}}%
\pgfpathlineto{\pgfqpoint{1.675567in}{1.537175in}}%
\pgfpathlineto{\pgfqpoint{1.679199in}{1.529536in}}%
\pgfpathlineto{\pgfqpoint{1.689587in}{1.515925in}}%
\pgfpathlineto{\pgfqpoint{1.691223in}{1.514279in}}%
\pgfpathlineto{\pgfqpoint{1.704986in}{1.502314in}}%
\pgfpathlineto{\pgfqpoint{1.706880in}{1.500891in}}%
\pgfpathlineto{\pgfqpoint{1.722536in}{1.491861in}}%
\pgfpathlineto{\pgfqpoint{1.731323in}{1.488703in}}%
\pgfpathlineto{\pgfqpoint{1.738193in}{1.486263in}}%
\pgfpathclose%
\pgfpathmoveto{\pgfqpoint{1.746628in}{1.502314in}}%
\pgfpathlineto{\pgfqpoint{1.738193in}{1.503654in}}%
\pgfpathlineto{\pgfqpoint{1.722536in}{1.510265in}}%
\pgfpathlineto{\pgfqpoint{1.714113in}{1.515925in}}%
\pgfpathlineto{\pgfqpoint{1.706880in}{1.522214in}}%
\pgfpathlineto{\pgfqpoint{1.700369in}{1.529536in}}%
\pgfpathlineto{\pgfqpoint{1.692764in}{1.543148in}}%
\pgfpathlineto{\pgfqpoint{1.691223in}{1.550481in}}%
\pgfpathlineto{\pgfqpoint{1.690027in}{1.556759in}}%
\pgfpathlineto{\pgfqpoint{1.691223in}{1.566143in}}%
\pgfpathlineto{\pgfqpoint{1.691815in}{1.570370in}}%
\pgfpathlineto{\pgfqpoint{1.698466in}{1.583981in}}%
\pgfpathlineto{\pgfqpoint{1.706880in}{1.594123in}}%
\pgfpathlineto{\pgfqpoint{1.710591in}{1.597592in}}%
\pgfpathlineto{\pgfqpoint{1.722536in}{1.605959in}}%
\pgfpathlineto{\pgfqpoint{1.734494in}{1.611203in}}%
\pgfpathlineto{\pgfqpoint{1.738193in}{1.612616in}}%
\pgfpathlineto{\pgfqpoint{1.753849in}{1.614904in}}%
\pgfpathlineto{\pgfqpoint{1.769506in}{1.613379in}}%
\pgfpathlineto{\pgfqpoint{1.775973in}{1.611203in}}%
\pgfpathlineto{\pgfqpoint{1.785162in}{1.607665in}}%
\pgfpathlineto{\pgfqpoint{1.800578in}{1.597592in}}%
\pgfpathlineto{\pgfqpoint{1.800819in}{1.597383in}}%
\pgfpathlineto{\pgfqpoint{1.812406in}{1.583981in}}%
\pgfpathlineto{\pgfqpoint{1.816476in}{1.575992in}}%
\pgfpathlineto{\pgfqpoint{1.818979in}{1.570370in}}%
\pgfpathlineto{\pgfqpoint{1.820733in}{1.556759in}}%
\pgfpathlineto{\pgfqpoint{1.818101in}{1.543148in}}%
\pgfpathlineto{\pgfqpoint{1.816476in}{1.539932in}}%
\pgfpathlineto{\pgfqpoint{1.810443in}{1.529536in}}%
\pgfpathlineto{\pgfqpoint{1.800819in}{1.519151in}}%
\pgfpathlineto{\pgfqpoint{1.796829in}{1.515925in}}%
\pgfpathlineto{\pgfqpoint{1.785162in}{1.508611in}}%
\pgfpathlineto{\pgfqpoint{1.769506in}{1.502829in}}%
\pgfpathlineto{\pgfqpoint{1.764644in}{1.502314in}}%
\pgfpathlineto{\pgfqpoint{1.753849in}{1.501274in}}%
\pgfpathlineto{\pgfqpoint{1.746628in}{1.502314in}}%
\pgfpathclose%
\pgfusepath{fill}%
\end{pgfscope}%
\begin{pgfscope}%
\pgfpathrectangle{\pgfqpoint{0.360415in}{0.345370in}}{\pgfqpoint{1.550000in}{1.347500in}}%
\pgfusepath{clip}%
\pgfsetbuttcap%
\pgfsetroundjoin%
\definecolor{currentfill}{rgb}{0.890340,0.406398,0.373130}%
\pgfsetfillcolor{currentfill}%
\pgfsetlinewidth{0.000000pt}%
\definecolor{currentstroke}{rgb}{0.000000,0.000000,0.000000}%
\pgfsetstrokecolor{currentstroke}%
\pgfsetdash{}{0pt}%
\pgfpathmoveto{\pgfqpoint{0.485668in}{0.394185in}}%
\pgfpathlineto{\pgfqpoint{0.501324in}{0.389270in}}%
\pgfpathlineto{\pgfqpoint{0.516981in}{0.387868in}}%
\pgfpathlineto{\pgfqpoint{0.532637in}{0.389971in}}%
\pgfpathlineto{\pgfqpoint{0.548294in}{0.395591in}}%
\pgfpathlineto{\pgfqpoint{0.555709in}{0.399814in}}%
\pgfpathlineto{\pgfqpoint{0.563950in}{0.404252in}}%
\pgfpathlineto{\pgfqpoint{0.576769in}{0.413425in}}%
\pgfpathlineto{\pgfqpoint{0.579607in}{0.415586in}}%
\pgfpathlineto{\pgfqpoint{0.592498in}{0.427036in}}%
\pgfpathlineto{\pgfqpoint{0.595263in}{0.429971in}}%
\pgfpathlineto{\pgfqpoint{0.604937in}{0.440648in}}%
\pgfpathlineto{\pgfqpoint{0.610920in}{0.449702in}}%
\pgfpathlineto{\pgfqpoint{0.614188in}{0.454259in}}%
\pgfpathlineto{\pgfqpoint{0.620039in}{0.467870in}}%
\pgfpathlineto{\pgfqpoint{0.621708in}{0.481481in}}%
\pgfpathlineto{\pgfqpoint{0.619204in}{0.495092in}}%
\pgfpathlineto{\pgfqpoint{0.612514in}{0.508703in}}%
\pgfpathlineto{\pgfqpoint{0.610920in}{0.510777in}}%
\pgfpathlineto{\pgfqpoint{0.602723in}{0.522314in}}%
\pgfpathlineto{\pgfqpoint{0.595263in}{0.530213in}}%
\pgfpathlineto{\pgfqpoint{0.589638in}{0.535925in}}%
\pgfpathlineto{\pgfqpoint{0.579607in}{0.544646in}}%
\pgfpathlineto{\pgfqpoint{0.573037in}{0.549536in}}%
\pgfpathlineto{\pgfqpoint{0.563950in}{0.556021in}}%
\pgfpathlineto{\pgfqpoint{0.550679in}{0.563148in}}%
\pgfpathlineto{\pgfqpoint{0.548294in}{0.564533in}}%
\pgfpathlineto{\pgfqpoint{0.532637in}{0.570349in}}%
\pgfpathlineto{\pgfqpoint{0.516981in}{0.572526in}}%
\pgfpathlineto{\pgfqpoint{0.501324in}{0.571075in}}%
\pgfpathlineto{\pgfqpoint{0.485668in}{0.565988in}}%
\pgfpathlineto{\pgfqpoint{0.480427in}{0.563148in}}%
\pgfpathlineto{\pgfqpoint{0.470011in}{0.557946in}}%
\pgfpathlineto{\pgfqpoint{0.457730in}{0.549536in}}%
\pgfpathlineto{\pgfqpoint{0.454354in}{0.547132in}}%
\pgfpathlineto{\pgfqpoint{0.441184in}{0.535925in}}%
\pgfpathlineto{\pgfqpoint{0.438698in}{0.533458in}}%
\pgfpathlineto{\pgfqpoint{0.428146in}{0.522314in}}%
\pgfpathlineto{\pgfqpoint{0.423041in}{0.515150in}}%
\pgfpathlineto{\pgfqpoint{0.418183in}{0.508703in}}%
\pgfpathlineto{\pgfqpoint{0.411719in}{0.495092in}}%
\pgfpathlineto{\pgfqpoint{0.409300in}{0.481481in}}%
\pgfpathlineto{\pgfqpoint{0.410912in}{0.467870in}}%
\pgfpathlineto{\pgfqpoint{0.416566in}{0.454259in}}%
\pgfpathlineto{\pgfqpoint{0.423041in}{0.445048in}}%
\pgfpathlineto{\pgfqpoint{0.425958in}{0.440648in}}%
\pgfpathlineto{\pgfqpoint{0.438301in}{0.427036in}}%
\pgfpathlineto{\pgfqpoint{0.438698in}{0.426666in}}%
\pgfpathlineto{\pgfqpoint{0.453928in}{0.413425in}}%
\pgfpathlineto{\pgfqpoint{0.454354in}{0.413080in}}%
\pgfpathlineto{\pgfqpoint{0.470011in}{0.402349in}}%
\pgfpathlineto{\pgfqpoint{0.475073in}{0.399814in}}%
\pgfpathlineto{\pgfqpoint{0.485668in}{0.394185in}}%
\pgfpathclose%
\pgfpathmoveto{\pgfqpoint{0.483757in}{0.413425in}}%
\pgfpathlineto{\pgfqpoint{0.470011in}{0.420849in}}%
\pgfpathlineto{\pgfqpoint{0.461611in}{0.427036in}}%
\pgfpathlineto{\pgfqpoint{0.454354in}{0.433345in}}%
\pgfpathlineto{\pgfqpoint{0.447237in}{0.440648in}}%
\pgfpathlineto{\pgfqpoint{0.438698in}{0.452598in}}%
\pgfpathlineto{\pgfqpoint{0.437564in}{0.454259in}}%
\pgfpathlineto{\pgfqpoint{0.431944in}{0.467870in}}%
\pgfpathlineto{\pgfqpoint{0.430341in}{0.481481in}}%
\pgfpathlineto{\pgfqpoint{0.432746in}{0.495092in}}%
\pgfpathlineto{\pgfqpoint{0.438698in}{0.507726in}}%
\pgfpathlineto{\pgfqpoint{0.439173in}{0.508703in}}%
\pgfpathlineto{\pgfqpoint{0.449653in}{0.522314in}}%
\pgfpathlineto{\pgfqpoint{0.454354in}{0.526942in}}%
\pgfpathlineto{\pgfqpoint{0.465150in}{0.535925in}}%
\pgfpathlineto{\pgfqpoint{0.470011in}{0.539429in}}%
\pgfpathlineto{\pgfqpoint{0.485668in}{0.547769in}}%
\pgfpathlineto{\pgfqpoint{0.491271in}{0.549536in}}%
\pgfpathlineto{\pgfqpoint{0.501324in}{0.552681in}}%
\pgfpathlineto{\pgfqpoint{0.516981in}{0.554091in}}%
\pgfpathlineto{\pgfqpoint{0.532637in}{0.551976in}}%
\pgfpathlineto{\pgfqpoint{0.539507in}{0.549536in}}%
\pgfpathlineto{\pgfqpoint{0.548294in}{0.546379in}}%
\pgfpathlineto{\pgfqpoint{0.563950in}{0.537348in}}%
\pgfpathlineto{\pgfqpoint{0.565844in}{0.535925in}}%
\pgfpathlineto{\pgfqpoint{0.579607in}{0.523960in}}%
\pgfpathlineto{\pgfqpoint{0.581243in}{0.522314in}}%
\pgfpathlineto{\pgfqpoint{0.591631in}{0.508703in}}%
\pgfpathlineto{\pgfqpoint{0.595263in}{0.501064in}}%
\pgfpathlineto{\pgfqpoint{0.598070in}{0.495092in}}%
\pgfpathlineto{\pgfqpoint{0.600502in}{0.481481in}}%
\pgfpathlineto{\pgfqpoint{0.598881in}{0.467870in}}%
\pgfpathlineto{\pgfqpoint{0.595263in}{0.459130in}}%
\pgfpathlineto{\pgfqpoint{0.593230in}{0.454259in}}%
\pgfpathlineto{\pgfqpoint{0.583638in}{0.440648in}}%
\pgfpathlineto{\pgfqpoint{0.579607in}{0.436422in}}%
\pgfpathlineto{\pgfqpoint{0.569274in}{0.427036in}}%
\pgfpathlineto{\pgfqpoint{0.563950in}{0.422949in}}%
\pgfpathlineto{\pgfqpoint{0.548294in}{0.413839in}}%
\pgfpathlineto{\pgfqpoint{0.547170in}{0.413425in}}%
\pgfpathlineto{\pgfqpoint{0.532637in}{0.408251in}}%
\pgfpathlineto{\pgfqpoint{0.516981in}{0.406160in}}%
\pgfpathlineto{\pgfqpoint{0.501324in}{0.407554in}}%
\pgfpathlineto{\pgfqpoint{0.485668in}{0.412440in}}%
\pgfpathlineto{\pgfqpoint{0.483757in}{0.413425in}}%
\pgfpathclose%
\pgfpathmoveto{\pgfqpoint{0.798799in}{0.392920in}}%
\pgfpathlineto{\pgfqpoint{0.814455in}{0.388709in}}%
\pgfpathlineto{\pgfqpoint{0.830112in}{0.388008in}}%
\pgfpathlineto{\pgfqpoint{0.845769in}{0.390813in}}%
\pgfpathlineto{\pgfqpoint{0.861425in}{0.397138in}}%
\pgfpathlineto{\pgfqpoint{0.865840in}{0.399814in}}%
\pgfpathlineto{\pgfqpoint{0.877082in}{0.406278in}}%
\pgfpathlineto{\pgfqpoint{0.886706in}{0.413425in}}%
\pgfpathlineto{\pgfqpoint{0.892738in}{0.418204in}}%
\pgfpathlineto{\pgfqpoint{0.902506in}{0.427036in}}%
\pgfpathlineto{\pgfqpoint{0.908395in}{0.433404in}}%
\pgfpathlineto{\pgfqpoint{0.914966in}{0.440648in}}%
\pgfpathlineto{\pgfqpoint{0.923977in}{0.454259in}}%
\pgfpathlineto{\pgfqpoint{0.924051in}{0.454447in}}%
\pgfpathlineto{\pgfqpoint{0.930053in}{0.467870in}}%
\pgfpathlineto{\pgfqpoint{0.931790in}{0.481481in}}%
\pgfpathlineto{\pgfqpoint{0.929185in}{0.495092in}}%
\pgfpathlineto{\pgfqpoint{0.924051in}{0.505205in}}%
\pgfpathlineto{\pgfqpoint{0.922475in}{0.508703in}}%
\pgfpathlineto{\pgfqpoint{0.912717in}{0.522314in}}%
\pgfpathlineto{\pgfqpoint{0.908395in}{0.526885in}}%
\pgfpathlineto{\pgfqpoint{0.899661in}{0.535925in}}%
\pgfpathlineto{\pgfqpoint{0.892738in}{0.542051in}}%
\pgfpathlineto{\pgfqpoint{0.883070in}{0.549536in}}%
\pgfpathlineto{\pgfqpoint{0.877082in}{0.553971in}}%
\pgfpathlineto{\pgfqpoint{0.861425in}{0.562957in}}%
\pgfpathlineto{\pgfqpoint{0.860926in}{0.563148in}}%
\pgfpathlineto{\pgfqpoint{0.845769in}{0.569478in}}%
\pgfpathlineto{\pgfqpoint{0.830112in}{0.572381in}}%
\pgfpathlineto{\pgfqpoint{0.814455in}{0.571655in}}%
\pgfpathlineto{\pgfqpoint{0.798799in}{0.567297in}}%
\pgfpathlineto{\pgfqpoint{0.790528in}{0.563148in}}%
\pgfpathlineto{\pgfqpoint{0.783142in}{0.559744in}}%
\pgfpathlineto{\pgfqpoint{0.767533in}{0.549536in}}%
\pgfpathlineto{\pgfqpoint{0.767486in}{0.549504in}}%
\pgfpathlineto{\pgfqpoint{0.751829in}{0.536554in}}%
\pgfpathlineto{\pgfqpoint{0.751136in}{0.535925in}}%
\pgfpathlineto{\pgfqpoint{0.738203in}{0.522314in}}%
\pgfpathlineto{\pgfqpoint{0.736173in}{0.519485in}}%
\pgfpathlineto{\pgfqpoint{0.728167in}{0.508703in}}%
\pgfpathlineto{\pgfqpoint{0.721890in}{0.495092in}}%
\pgfpathlineto{\pgfqpoint{0.720516in}{0.487169in}}%
\pgfpathlineto{\pgfqpoint{0.719324in}{0.481481in}}%
\pgfpathlineto{\pgfqpoint{0.720516in}{0.472978in}}%
\pgfpathlineto{\pgfqpoint{0.721106in}{0.467870in}}%
\pgfpathlineto{\pgfqpoint{0.726597in}{0.454259in}}%
\pgfpathlineto{\pgfqpoint{0.736019in}{0.440648in}}%
\pgfpathlineto{\pgfqpoint{0.736173in}{0.440481in}}%
\pgfpathlineto{\pgfqpoint{0.748282in}{0.427036in}}%
\pgfpathlineto{\pgfqpoint{0.751829in}{0.423750in}}%
\pgfpathlineto{\pgfqpoint{0.764030in}{0.413425in}}%
\pgfpathlineto{\pgfqpoint{0.767486in}{0.410695in}}%
\pgfpathlineto{\pgfqpoint{0.783142in}{0.400572in}}%
\pgfpathlineto{\pgfqpoint{0.784782in}{0.399814in}}%
\pgfpathlineto{\pgfqpoint{0.798799in}{0.392920in}}%
\pgfpathclose%
\pgfpathmoveto{\pgfqpoint{0.794103in}{0.413425in}}%
\pgfpathlineto{\pgfqpoint{0.783142in}{0.418887in}}%
\pgfpathlineto{\pgfqpoint{0.771558in}{0.427036in}}%
\pgfpathlineto{\pgfqpoint{0.767486in}{0.430409in}}%
\pgfpathlineto{\pgfqpoint{0.757252in}{0.440648in}}%
\pgfpathlineto{\pgfqpoint{0.751829in}{0.448034in}}%
\pgfpathlineto{\pgfqpoint{0.747551in}{0.454259in}}%
\pgfpathlineto{\pgfqpoint{0.741973in}{0.467870in}}%
\pgfpathlineto{\pgfqpoint{0.740382in}{0.481481in}}%
\pgfpathlineto{\pgfqpoint{0.742769in}{0.495092in}}%
\pgfpathlineto{\pgfqpoint{0.749147in}{0.508703in}}%
\pgfpathlineto{\pgfqpoint{0.751829in}{0.512344in}}%
\pgfpathlineto{\pgfqpoint{0.759699in}{0.522314in}}%
\pgfpathlineto{\pgfqpoint{0.767486in}{0.529789in}}%
\pgfpathlineto{\pgfqpoint{0.775228in}{0.535925in}}%
\pgfpathlineto{\pgfqpoint{0.783142in}{0.541374in}}%
\pgfpathlineto{\pgfqpoint{0.798799in}{0.549019in}}%
\pgfpathlineto{\pgfqpoint{0.800706in}{0.549536in}}%
\pgfpathlineto{\pgfqpoint{0.814455in}{0.553245in}}%
\pgfpathlineto{\pgfqpoint{0.830112in}{0.553950in}}%
\pgfpathlineto{\pgfqpoint{0.845769in}{0.551129in}}%
\pgfpathlineto{\pgfqpoint{0.849785in}{0.549536in}}%
\pgfpathlineto{\pgfqpoint{0.861425in}{0.544849in}}%
\pgfpathlineto{\pgfqpoint{0.875858in}{0.535925in}}%
\pgfpathlineto{\pgfqpoint{0.877082in}{0.535059in}}%
\pgfpathlineto{\pgfqpoint{0.891208in}{0.522314in}}%
\pgfpathlineto{\pgfqpoint{0.892738in}{0.520500in}}%
\pgfpathlineto{\pgfqpoint{0.901644in}{0.508703in}}%
\pgfpathlineto{\pgfqpoint{0.908005in}{0.495092in}}%
\pgfpathlineto{\pgfqpoint{0.908395in}{0.492886in}}%
\pgfpathlineto{\pgfqpoint{0.910462in}{0.481481in}}%
\pgfpathlineto{\pgfqpoint{0.908815in}{0.467870in}}%
\pgfpathlineto{\pgfqpoint{0.908395in}{0.466857in}}%
\pgfpathlineto{\pgfqpoint{0.903236in}{0.454259in}}%
\pgfpathlineto{\pgfqpoint{0.893688in}{0.440648in}}%
\pgfpathlineto{\pgfqpoint{0.892738in}{0.439634in}}%
\pgfpathlineto{\pgfqpoint{0.879405in}{0.427036in}}%
\pgfpathlineto{\pgfqpoint{0.877082in}{0.425185in}}%
\pgfpathlineto{\pgfqpoint{0.861425in}{0.415381in}}%
\pgfpathlineto{\pgfqpoint{0.856678in}{0.413425in}}%
\pgfpathlineto{\pgfqpoint{0.845769in}{0.409088in}}%
\pgfpathlineto{\pgfqpoint{0.830112in}{0.406299in}}%
\pgfpathlineto{\pgfqpoint{0.814455in}{0.406996in}}%
\pgfpathlineto{\pgfqpoint{0.798799in}{0.411182in}}%
\pgfpathlineto{\pgfqpoint{0.794103in}{0.413425in}}%
\pgfpathclose%
\pgfpathmoveto{\pgfqpoint{1.096274in}{0.398825in}}%
\pgfpathlineto{\pgfqpoint{1.111930in}{0.391796in}}%
\pgfpathlineto{\pgfqpoint{1.127587in}{0.388288in}}%
\pgfpathlineto{\pgfqpoint{1.143243in}{0.388288in}}%
\pgfpathlineto{\pgfqpoint{1.158900in}{0.391796in}}%
\pgfpathlineto{\pgfqpoint{1.174556in}{0.398825in}}%
\pgfpathlineto{\pgfqpoint{1.176098in}{0.399814in}}%
\pgfpathlineto{\pgfqpoint{1.190213in}{0.408427in}}%
\pgfpathlineto{\pgfqpoint{1.196726in}{0.413425in}}%
\pgfpathlineto{\pgfqpoint{1.205870in}{0.420927in}}%
\pgfpathlineto{\pgfqpoint{1.212532in}{0.427036in}}%
\pgfpathlineto{\pgfqpoint{1.221526in}{0.436912in}}%
\pgfpathlineto{\pgfqpoint{1.224935in}{0.440648in}}%
\pgfpathlineto{\pgfqpoint{1.234128in}{0.454259in}}%
\pgfpathlineto{\pgfqpoint{1.237183in}{0.461935in}}%
\pgfpathlineto{\pgfqpoint{1.239921in}{0.467870in}}%
\pgfpathlineto{\pgfqpoint{1.241738in}{0.481481in}}%
\pgfpathlineto{\pgfqpoint{1.239012in}{0.495092in}}%
\pgfpathlineto{\pgfqpoint{1.237183in}{0.498584in}}%
\pgfpathlineto{\pgfqpoint{1.232596in}{0.508703in}}%
\pgfpathlineto{\pgfqpoint{1.222640in}{0.522314in}}%
\pgfpathlineto{\pgfqpoint{1.221526in}{0.523486in}}%
\pgfpathlineto{\pgfqpoint{1.209688in}{0.535925in}}%
\pgfpathlineto{\pgfqpoint{1.205870in}{0.539352in}}%
\pgfpathlineto{\pgfqpoint{1.193168in}{0.549536in}}%
\pgfpathlineto{\pgfqpoint{1.190213in}{0.551798in}}%
\pgfpathlineto{\pgfqpoint{1.174556in}{0.561415in}}%
\pgfpathlineto{\pgfqpoint{1.170444in}{0.563148in}}%
\pgfpathlineto{\pgfqpoint{1.158900in}{0.568460in}}%
\pgfpathlineto{\pgfqpoint{1.143243in}{0.572091in}}%
\pgfpathlineto{\pgfqpoint{1.127587in}{0.572091in}}%
\pgfpathlineto{\pgfqpoint{1.111930in}{0.568460in}}%
\pgfpathlineto{\pgfqpoint{1.100386in}{0.563148in}}%
\pgfpathlineto{\pgfqpoint{1.096274in}{0.561415in}}%
\pgfpathlineto{\pgfqpoint{1.080617in}{0.551798in}}%
\pgfpathlineto{\pgfqpoint{1.077662in}{0.549536in}}%
\pgfpathlineto{\pgfqpoint{1.064960in}{0.539352in}}%
\pgfpathlineto{\pgfqpoint{1.061142in}{0.535925in}}%
\pgfpathlineto{\pgfqpoint{1.049304in}{0.523486in}}%
\pgfpathlineto{\pgfqpoint{1.048190in}{0.522314in}}%
\pgfpathlineto{\pgfqpoint{1.038234in}{0.508703in}}%
\pgfpathlineto{\pgfqpoint{1.033647in}{0.498584in}}%
\pgfpathlineto{\pgfqpoint{1.031818in}{0.495092in}}%
\pgfpathlineto{\pgfqpoint{1.029092in}{0.481481in}}%
\pgfpathlineto{\pgfqpoint{1.030909in}{0.467870in}}%
\pgfpathlineto{\pgfqpoint{1.033647in}{0.461935in}}%
\pgfpathlineto{\pgfqpoint{1.036702in}{0.454259in}}%
\pgfpathlineto{\pgfqpoint{1.045895in}{0.440648in}}%
\pgfpathlineto{\pgfqpoint{1.049304in}{0.436912in}}%
\pgfpathlineto{\pgfqpoint{1.058298in}{0.427036in}}%
\pgfpathlineto{\pgfqpoint{1.064960in}{0.420927in}}%
\pgfpathlineto{\pgfqpoint{1.074104in}{0.413425in}}%
\pgfpathlineto{\pgfqpoint{1.080617in}{0.408427in}}%
\pgfpathlineto{\pgfqpoint{1.094732in}{0.399814in}}%
\pgfpathlineto{\pgfqpoint{1.096274in}{0.398825in}}%
\pgfpathclose%
\pgfpathmoveto{\pgfqpoint{1.104261in}{0.413425in}}%
\pgfpathlineto{\pgfqpoint{1.096274in}{0.417064in}}%
\pgfpathlineto{\pgfqpoint{1.081356in}{0.427036in}}%
\pgfpathlineto{\pgfqpoint{1.080617in}{0.427617in}}%
\pgfpathlineto{\pgfqpoint{1.067208in}{0.440648in}}%
\pgfpathlineto{\pgfqpoint{1.064960in}{0.443616in}}%
\pgfpathlineto{\pgfqpoint{1.057570in}{0.454259in}}%
\pgfpathlineto{\pgfqpoint{1.052010in}{0.467870in}}%
\pgfpathlineto{\pgfqpoint{1.050424in}{0.481481in}}%
\pgfpathlineto{\pgfqpoint{1.052803in}{0.495092in}}%
\pgfpathlineto{\pgfqpoint{1.059160in}{0.508703in}}%
\pgfpathlineto{\pgfqpoint{1.064960in}{0.516495in}}%
\pgfpathlineto{\pgfqpoint{1.069698in}{0.522314in}}%
\pgfpathlineto{\pgfqpoint{1.080617in}{0.532496in}}%
\pgfpathlineto{\pgfqpoint{1.085184in}{0.535925in}}%
\pgfpathlineto{\pgfqpoint{1.096274in}{0.543181in}}%
\pgfpathlineto{\pgfqpoint{1.110547in}{0.549536in}}%
\pgfpathlineto{\pgfqpoint{1.111930in}{0.550141in}}%
\pgfpathlineto{\pgfqpoint{1.127587in}{0.553668in}}%
\pgfpathlineto{\pgfqpoint{1.143243in}{0.553668in}}%
\pgfpathlineto{\pgfqpoint{1.158900in}{0.550141in}}%
\pgfpathlineto{\pgfqpoint{1.160283in}{0.549536in}}%
\pgfpathlineto{\pgfqpoint{1.174556in}{0.543181in}}%
\pgfpathlineto{\pgfqpoint{1.185646in}{0.535925in}}%
\pgfpathlineto{\pgfqpoint{1.190213in}{0.532496in}}%
\pgfpathlineto{\pgfqpoint{1.201132in}{0.522314in}}%
\pgfpathlineto{\pgfqpoint{1.205870in}{0.516495in}}%
\pgfpathlineto{\pgfqpoint{1.211670in}{0.508703in}}%
\pgfpathlineto{\pgfqpoint{1.218027in}{0.495092in}}%
\pgfpathlineto{\pgfqpoint{1.220406in}{0.481481in}}%
\pgfpathlineto{\pgfqpoint{1.218820in}{0.467870in}}%
\pgfpathlineto{\pgfqpoint{1.213260in}{0.454259in}}%
\pgfpathlineto{\pgfqpoint{1.205870in}{0.443616in}}%
\pgfpathlineto{\pgfqpoint{1.203622in}{0.440648in}}%
\pgfpathlineto{\pgfqpoint{1.190213in}{0.427617in}}%
\pgfpathlineto{\pgfqpoint{1.189474in}{0.427036in}}%
\pgfpathlineto{\pgfqpoint{1.174556in}{0.417064in}}%
\pgfpathlineto{\pgfqpoint{1.166569in}{0.413425in}}%
\pgfpathlineto{\pgfqpoint{1.158900in}{0.410065in}}%
\pgfpathlineto{\pgfqpoint{1.143243in}{0.406578in}}%
\pgfpathlineto{\pgfqpoint{1.127587in}{0.406578in}}%
\pgfpathlineto{\pgfqpoint{1.111930in}{0.410065in}}%
\pgfpathlineto{\pgfqpoint{1.104261in}{0.413425in}}%
\pgfpathclose%
\pgfpathmoveto{\pgfqpoint{1.409405in}{0.397138in}}%
\pgfpathlineto{\pgfqpoint{1.425061in}{0.390813in}}%
\pgfpathlineto{\pgfqpoint{1.440718in}{0.388008in}}%
\pgfpathlineto{\pgfqpoint{1.456375in}{0.388709in}}%
\pgfpathlineto{\pgfqpoint{1.472031in}{0.392920in}}%
\pgfpathlineto{\pgfqpoint{1.486048in}{0.399814in}}%
\pgfpathlineto{\pgfqpoint{1.487688in}{0.400572in}}%
\pgfpathlineto{\pgfqpoint{1.503344in}{0.410695in}}%
\pgfpathlineto{\pgfqpoint{1.506800in}{0.413425in}}%
\pgfpathlineto{\pgfqpoint{1.519001in}{0.423750in}}%
\pgfpathlineto{\pgfqpoint{1.522548in}{0.427036in}}%
\pgfpathlineto{\pgfqpoint{1.534657in}{0.440481in}}%
\pgfpathlineto{\pgfqpoint{1.534811in}{0.440648in}}%
\pgfpathlineto{\pgfqpoint{1.544233in}{0.454259in}}%
\pgfpathlineto{\pgfqpoint{1.549724in}{0.467870in}}%
\pgfpathlineto{\pgfqpoint{1.550314in}{0.472978in}}%
\pgfpathlineto{\pgfqpoint{1.551506in}{0.481481in}}%
\pgfpathlineto{\pgfqpoint{1.550314in}{0.487169in}}%
\pgfpathlineto{\pgfqpoint{1.548940in}{0.495092in}}%
\pgfpathlineto{\pgfqpoint{1.542663in}{0.508703in}}%
\pgfpathlineto{\pgfqpoint{1.534657in}{0.519485in}}%
\pgfpathlineto{\pgfqpoint{1.532627in}{0.522314in}}%
\pgfpathlineto{\pgfqpoint{1.519694in}{0.535925in}}%
\pgfpathlineto{\pgfqpoint{1.519001in}{0.536554in}}%
\pgfpathlineto{\pgfqpoint{1.503344in}{0.549504in}}%
\pgfpathlineto{\pgfqpoint{1.503297in}{0.549536in}}%
\pgfpathlineto{\pgfqpoint{1.487688in}{0.559744in}}%
\pgfpathlineto{\pgfqpoint{1.480302in}{0.563148in}}%
\pgfpathlineto{\pgfqpoint{1.472031in}{0.567297in}}%
\pgfpathlineto{\pgfqpoint{1.456375in}{0.571655in}}%
\pgfpathlineto{\pgfqpoint{1.440718in}{0.572381in}}%
\pgfpathlineto{\pgfqpoint{1.425061in}{0.569478in}}%
\pgfpathlineto{\pgfqpoint{1.409904in}{0.563148in}}%
\pgfpathlineto{\pgfqpoint{1.409405in}{0.562957in}}%
\pgfpathlineto{\pgfqpoint{1.393748in}{0.553971in}}%
\pgfpathlineto{\pgfqpoint{1.387760in}{0.549536in}}%
\pgfpathlineto{\pgfqpoint{1.378092in}{0.542051in}}%
\pgfpathlineto{\pgfqpoint{1.371169in}{0.535925in}}%
\pgfpathlineto{\pgfqpoint{1.362435in}{0.526885in}}%
\pgfpathlineto{\pgfqpoint{1.358113in}{0.522314in}}%
\pgfpathlineto{\pgfqpoint{1.348355in}{0.508703in}}%
\pgfpathlineto{\pgfqpoint{1.346779in}{0.505205in}}%
\pgfpathlineto{\pgfqpoint{1.341645in}{0.495092in}}%
\pgfpathlineto{\pgfqpoint{1.339040in}{0.481481in}}%
\pgfpathlineto{\pgfqpoint{1.340777in}{0.467870in}}%
\pgfpathlineto{\pgfqpoint{1.346779in}{0.454447in}}%
\pgfpathlineto{\pgfqpoint{1.346853in}{0.454259in}}%
\pgfpathlineto{\pgfqpoint{1.355864in}{0.440648in}}%
\pgfpathlineto{\pgfqpoint{1.362435in}{0.433404in}}%
\pgfpathlineto{\pgfqpoint{1.368324in}{0.427036in}}%
\pgfpathlineto{\pgfqpoint{1.378092in}{0.418204in}}%
\pgfpathlineto{\pgfqpoint{1.384124in}{0.413425in}}%
\pgfpathlineto{\pgfqpoint{1.393748in}{0.406278in}}%
\pgfpathlineto{\pgfqpoint{1.404990in}{0.399814in}}%
\pgfpathlineto{\pgfqpoint{1.409405in}{0.397138in}}%
\pgfpathclose%
\pgfpathmoveto{\pgfqpoint{1.414152in}{0.413425in}}%
\pgfpathlineto{\pgfqpoint{1.409405in}{0.415381in}}%
\pgfpathlineto{\pgfqpoint{1.393748in}{0.425185in}}%
\pgfpathlineto{\pgfqpoint{1.391425in}{0.427036in}}%
\pgfpathlineto{\pgfqpoint{1.378092in}{0.439634in}}%
\pgfpathlineto{\pgfqpoint{1.377142in}{0.440648in}}%
\pgfpathlineto{\pgfqpoint{1.367594in}{0.454259in}}%
\pgfpathlineto{\pgfqpoint{1.362435in}{0.466857in}}%
\pgfpathlineto{\pgfqpoint{1.362015in}{0.467870in}}%
\pgfpathlineto{\pgfqpoint{1.360368in}{0.481481in}}%
\pgfpathlineto{\pgfqpoint{1.362435in}{0.492886in}}%
\pgfpathlineto{\pgfqpoint{1.362825in}{0.495092in}}%
\pgfpathlineto{\pgfqpoint{1.369186in}{0.508703in}}%
\pgfpathlineto{\pgfqpoint{1.378092in}{0.520500in}}%
\pgfpathlineto{\pgfqpoint{1.379622in}{0.522314in}}%
\pgfpathlineto{\pgfqpoint{1.393748in}{0.535059in}}%
\pgfpathlineto{\pgfqpoint{1.394972in}{0.535925in}}%
\pgfpathlineto{\pgfqpoint{1.409405in}{0.544849in}}%
\pgfpathlineto{\pgfqpoint{1.421045in}{0.549536in}}%
\pgfpathlineto{\pgfqpoint{1.425061in}{0.551129in}}%
\pgfpathlineto{\pgfqpoint{1.440718in}{0.553950in}}%
\pgfpathlineto{\pgfqpoint{1.456375in}{0.553245in}}%
\pgfpathlineto{\pgfqpoint{1.470124in}{0.549536in}}%
\pgfpathlineto{\pgfqpoint{1.472031in}{0.549019in}}%
\pgfpathlineto{\pgfqpoint{1.487688in}{0.541374in}}%
\pgfpathlineto{\pgfqpoint{1.495602in}{0.535925in}}%
\pgfpathlineto{\pgfqpoint{1.503344in}{0.529789in}}%
\pgfpathlineto{\pgfqpoint{1.511131in}{0.522314in}}%
\pgfpathlineto{\pgfqpoint{1.519001in}{0.512344in}}%
\pgfpathlineto{\pgfqpoint{1.521683in}{0.508703in}}%
\pgfpathlineto{\pgfqpoint{1.528061in}{0.495092in}}%
\pgfpathlineto{\pgfqpoint{1.530448in}{0.481481in}}%
\pgfpathlineto{\pgfqpoint{1.528857in}{0.467870in}}%
\pgfpathlineto{\pgfqpoint{1.523279in}{0.454259in}}%
\pgfpathlineto{\pgfqpoint{1.519001in}{0.448034in}}%
\pgfpathlineto{\pgfqpoint{1.513578in}{0.440648in}}%
\pgfpathlineto{\pgfqpoint{1.503344in}{0.430409in}}%
\pgfpathlineto{\pgfqpoint{1.499272in}{0.427036in}}%
\pgfpathlineto{\pgfqpoint{1.487688in}{0.418887in}}%
\pgfpathlineto{\pgfqpoint{1.476727in}{0.413425in}}%
\pgfpathlineto{\pgfqpoint{1.472031in}{0.411182in}}%
\pgfpathlineto{\pgfqpoint{1.456375in}{0.406996in}}%
\pgfpathlineto{\pgfqpoint{1.440718in}{0.406299in}}%
\pgfpathlineto{\pgfqpoint{1.425061in}{0.409088in}}%
\pgfpathlineto{\pgfqpoint{1.414152in}{0.413425in}}%
\pgfpathclose%
\pgfpathmoveto{\pgfqpoint{1.722536in}{0.395591in}}%
\pgfpathlineto{\pgfqpoint{1.738193in}{0.389971in}}%
\pgfpathlineto{\pgfqpoint{1.753849in}{0.387868in}}%
\pgfpathlineto{\pgfqpoint{1.769506in}{0.389270in}}%
\pgfpathlineto{\pgfqpoint{1.785162in}{0.394185in}}%
\pgfpathlineto{\pgfqpoint{1.795757in}{0.399814in}}%
\pgfpathlineto{\pgfqpoint{1.800819in}{0.402349in}}%
\pgfpathlineto{\pgfqpoint{1.816476in}{0.413080in}}%
\pgfpathlineto{\pgfqpoint{1.816902in}{0.413425in}}%
\pgfpathlineto{\pgfqpoint{1.832132in}{0.426666in}}%
\pgfpathlineto{\pgfqpoint{1.832529in}{0.427036in}}%
\pgfpathlineto{\pgfqpoint{1.844872in}{0.440648in}}%
\pgfpathlineto{\pgfqpoint{1.847789in}{0.445048in}}%
\pgfpathlineto{\pgfqpoint{1.854264in}{0.454259in}}%
\pgfpathlineto{\pgfqpoint{1.859918in}{0.467870in}}%
\pgfpathlineto{\pgfqpoint{1.861530in}{0.481481in}}%
\pgfpathlineto{\pgfqpoint{1.859111in}{0.495092in}}%
\pgfpathlineto{\pgfqpoint{1.852647in}{0.508703in}}%
\pgfpathlineto{\pgfqpoint{1.847789in}{0.515150in}}%
\pgfpathlineto{\pgfqpoint{1.842684in}{0.522314in}}%
\pgfpathlineto{\pgfqpoint{1.832132in}{0.533458in}}%
\pgfpathlineto{\pgfqpoint{1.829646in}{0.535925in}}%
\pgfpathlineto{\pgfqpoint{1.816476in}{0.547132in}}%
\pgfpathlineto{\pgfqpoint{1.813100in}{0.549536in}}%
\pgfpathlineto{\pgfqpoint{1.800819in}{0.557946in}}%
\pgfpathlineto{\pgfqpoint{1.790403in}{0.563148in}}%
\pgfpathlineto{\pgfqpoint{1.785162in}{0.565988in}}%
\pgfpathlineto{\pgfqpoint{1.769506in}{0.571075in}}%
\pgfpathlineto{\pgfqpoint{1.753849in}{0.572526in}}%
\pgfpathlineto{\pgfqpoint{1.738193in}{0.570349in}}%
\pgfpathlineto{\pgfqpoint{1.722536in}{0.564533in}}%
\pgfpathlineto{\pgfqpoint{1.720151in}{0.563148in}}%
\pgfpathlineto{\pgfqpoint{1.706880in}{0.556021in}}%
\pgfpathlineto{\pgfqpoint{1.697793in}{0.549536in}}%
\pgfpathlineto{\pgfqpoint{1.691223in}{0.544646in}}%
\pgfpathlineto{\pgfqpoint{1.681192in}{0.535925in}}%
\pgfpathlineto{\pgfqpoint{1.675567in}{0.530213in}}%
\pgfpathlineto{\pgfqpoint{1.668107in}{0.522314in}}%
\pgfpathlineto{\pgfqpoint{1.659910in}{0.510777in}}%
\pgfpathlineto{\pgfqpoint{1.658316in}{0.508703in}}%
\pgfpathlineto{\pgfqpoint{1.651626in}{0.495092in}}%
\pgfpathlineto{\pgfqpoint{1.649122in}{0.481481in}}%
\pgfpathlineto{\pgfqpoint{1.650791in}{0.467870in}}%
\pgfpathlineto{\pgfqpoint{1.656642in}{0.454259in}}%
\pgfpathlineto{\pgfqpoint{1.659910in}{0.449702in}}%
\pgfpathlineto{\pgfqpoint{1.665893in}{0.440648in}}%
\pgfpathlineto{\pgfqpoint{1.675567in}{0.429971in}}%
\pgfpathlineto{\pgfqpoint{1.678332in}{0.427036in}}%
\pgfpathlineto{\pgfqpoint{1.691223in}{0.415586in}}%
\pgfpathlineto{\pgfqpoint{1.694061in}{0.413425in}}%
\pgfpathlineto{\pgfqpoint{1.706880in}{0.404252in}}%
\pgfpathlineto{\pgfqpoint{1.715121in}{0.399814in}}%
\pgfpathlineto{\pgfqpoint{1.722536in}{0.395591in}}%
\pgfpathclose%
\pgfpathmoveto{\pgfqpoint{1.723660in}{0.413425in}}%
\pgfpathlineto{\pgfqpoint{1.722536in}{0.413839in}}%
\pgfpathlineto{\pgfqpoint{1.706880in}{0.422949in}}%
\pgfpathlineto{\pgfqpoint{1.701556in}{0.427036in}}%
\pgfpathlineto{\pgfqpoint{1.691223in}{0.436422in}}%
\pgfpathlineto{\pgfqpoint{1.687192in}{0.440648in}}%
\pgfpathlineto{\pgfqpoint{1.677600in}{0.454259in}}%
\pgfpathlineto{\pgfqpoint{1.675567in}{0.459130in}}%
\pgfpathlineto{\pgfqpoint{1.671949in}{0.467870in}}%
\pgfpathlineto{\pgfqpoint{1.670328in}{0.481481in}}%
\pgfpathlineto{\pgfqpoint{1.672760in}{0.495092in}}%
\pgfpathlineto{\pgfqpoint{1.675567in}{0.501064in}}%
\pgfpathlineto{\pgfqpoint{1.679199in}{0.508703in}}%
\pgfpathlineto{\pgfqpoint{1.689587in}{0.522314in}}%
\pgfpathlineto{\pgfqpoint{1.691223in}{0.523960in}}%
\pgfpathlineto{\pgfqpoint{1.704986in}{0.535925in}}%
\pgfpathlineto{\pgfqpoint{1.706880in}{0.537348in}}%
\pgfpathlineto{\pgfqpoint{1.722536in}{0.546379in}}%
\pgfpathlineto{\pgfqpoint{1.731323in}{0.549536in}}%
\pgfpathlineto{\pgfqpoint{1.738193in}{0.551976in}}%
\pgfpathlineto{\pgfqpoint{1.753849in}{0.554091in}}%
\pgfpathlineto{\pgfqpoint{1.769506in}{0.552681in}}%
\pgfpathlineto{\pgfqpoint{1.779559in}{0.549536in}}%
\pgfpathlineto{\pgfqpoint{1.785162in}{0.547769in}}%
\pgfpathlineto{\pgfqpoint{1.800819in}{0.539429in}}%
\pgfpathlineto{\pgfqpoint{1.805680in}{0.535925in}}%
\pgfpathlineto{\pgfqpoint{1.816476in}{0.526942in}}%
\pgfpathlineto{\pgfqpoint{1.821177in}{0.522314in}}%
\pgfpathlineto{\pgfqpoint{1.831657in}{0.508703in}}%
\pgfpathlineto{\pgfqpoint{1.832132in}{0.507726in}}%
\pgfpathlineto{\pgfqpoint{1.838084in}{0.495092in}}%
\pgfpathlineto{\pgfqpoint{1.840489in}{0.481481in}}%
\pgfpathlineto{\pgfqpoint{1.838886in}{0.467870in}}%
\pgfpathlineto{\pgfqpoint{1.833266in}{0.454259in}}%
\pgfpathlineto{\pgfqpoint{1.832132in}{0.452598in}}%
\pgfpathlineto{\pgfqpoint{1.823593in}{0.440648in}}%
\pgfpathlineto{\pgfqpoint{1.816476in}{0.433345in}}%
\pgfpathlineto{\pgfqpoint{1.809219in}{0.427036in}}%
\pgfpathlineto{\pgfqpoint{1.800819in}{0.420849in}}%
\pgfpathlineto{\pgfqpoint{1.787073in}{0.413425in}}%
\pgfpathlineto{\pgfqpoint{1.785162in}{0.412440in}}%
\pgfpathlineto{\pgfqpoint{1.769506in}{0.407554in}}%
\pgfpathlineto{\pgfqpoint{1.753849in}{0.406160in}}%
\pgfpathlineto{\pgfqpoint{1.738193in}{0.408251in}}%
\pgfpathlineto{\pgfqpoint{1.723660in}{0.413425in}}%
\pgfpathclose%
\pgfpathmoveto{\pgfqpoint{0.516981in}{0.657389in}}%
\pgfpathlineto{\pgfqpoint{0.523523in}{0.658425in}}%
\pgfpathlineto{\pgfqpoint{0.532637in}{0.659620in}}%
\pgfpathlineto{\pgfqpoint{0.548294in}{0.665077in}}%
\pgfpathlineto{\pgfqpoint{0.560696in}{0.672036in}}%
\pgfpathlineto{\pgfqpoint{0.563950in}{0.673801in}}%
\pgfpathlineto{\pgfqpoint{0.579607in}{0.685045in}}%
\pgfpathlineto{\pgfqpoint{0.580330in}{0.685648in}}%
\pgfpathlineto{\pgfqpoint{0.595227in}{0.699259in}}%
\pgfpathlineto{\pgfqpoint{0.595263in}{0.699300in}}%
\pgfpathlineto{\pgfqpoint{0.607005in}{0.712870in}}%
\pgfpathlineto{\pgfqpoint{0.610920in}{0.719291in}}%
\pgfpathlineto{\pgfqpoint{0.615693in}{0.726481in}}%
\pgfpathlineto{\pgfqpoint{0.620707in}{0.740092in}}%
\pgfpathlineto{\pgfqpoint{0.621541in}{0.753703in}}%
\pgfpathlineto{\pgfqpoint{0.618201in}{0.767314in}}%
\pgfpathlineto{\pgfqpoint{0.610920in}{0.780492in}}%
\pgfpathlineto{\pgfqpoint{0.610701in}{0.780925in}}%
\pgfpathlineto{\pgfqpoint{0.600365in}{0.794536in}}%
\pgfpathlineto{\pgfqpoint{0.595263in}{0.799742in}}%
\pgfpathlineto{\pgfqpoint{0.586654in}{0.808148in}}%
\pgfpathlineto{\pgfqpoint{0.579607in}{0.814166in}}%
\pgfpathlineto{\pgfqpoint{0.569209in}{0.821759in}}%
\pgfpathlineto{\pgfqpoint{0.563950in}{0.825517in}}%
\pgfpathlineto{\pgfqpoint{0.548294in}{0.833999in}}%
\pgfpathlineto{\pgfqpoint{0.544270in}{0.835370in}}%
\pgfpathlineto{\pgfqpoint{0.532637in}{0.839832in}}%
\pgfpathlineto{\pgfqpoint{0.516981in}{0.842097in}}%
\pgfpathlineto{\pgfqpoint{0.501324in}{0.840588in}}%
\pgfpathlineto{\pgfqpoint{0.485884in}{0.835370in}}%
\pgfpathlineto{\pgfqpoint{0.485668in}{0.835305in}}%
\pgfpathlineto{\pgfqpoint{0.470011in}{0.827472in}}%
\pgfpathlineto{\pgfqpoint{0.461679in}{0.821759in}}%
\pgfpathlineto{\pgfqpoint{0.454354in}{0.816640in}}%
\pgfpathlineto{\pgfqpoint{0.444195in}{0.808148in}}%
\pgfpathlineto{\pgfqpoint{0.438698in}{0.802903in}}%
\pgfpathlineto{\pgfqpoint{0.430477in}{0.794536in}}%
\pgfpathlineto{\pgfqpoint{0.423041in}{0.784763in}}%
\pgfpathlineto{\pgfqpoint{0.419962in}{0.780925in}}%
\pgfpathlineto{\pgfqpoint{0.412688in}{0.767314in}}%
\pgfpathlineto{\pgfqpoint{0.409461in}{0.753703in}}%
\pgfpathlineto{\pgfqpoint{0.410267in}{0.740092in}}%
\pgfpathlineto{\pgfqpoint{0.415111in}{0.726481in}}%
\pgfpathlineto{\pgfqpoint{0.423041in}{0.714295in}}%
\pgfpathlineto{\pgfqpoint{0.423913in}{0.712870in}}%
\pgfpathlineto{\pgfqpoint{0.435557in}{0.699259in}}%
\pgfpathlineto{\pgfqpoint{0.438698in}{0.696254in}}%
\pgfpathlineto{\pgfqpoint{0.450574in}{0.685648in}}%
\pgfpathlineto{\pgfqpoint{0.454354in}{0.682564in}}%
\pgfpathlineto{\pgfqpoint{0.469819in}{0.672036in}}%
\pgfpathlineto{\pgfqpoint{0.470011in}{0.671903in}}%
\pgfpathlineto{\pgfqpoint{0.485668in}{0.663712in}}%
\pgfpathlineto{\pgfqpoint{0.501324in}{0.658939in}}%
\pgfpathlineto{\pgfqpoint{0.507200in}{0.658425in}}%
\pgfpathlineto{\pgfqpoint{0.516981in}{0.657389in}}%
\pgfpathclose%
\pgfpathmoveto{\pgfqpoint{0.478508in}{0.685648in}}%
\pgfpathlineto{\pgfqpoint{0.470011in}{0.690362in}}%
\pgfpathlineto{\pgfqpoint{0.458234in}{0.699259in}}%
\pgfpathlineto{\pgfqpoint{0.454354in}{0.702799in}}%
\pgfpathlineto{\pgfqpoint{0.444981in}{0.712870in}}%
\pgfpathlineto{\pgfqpoint{0.438698in}{0.722399in}}%
\pgfpathlineto{\pgfqpoint{0.436118in}{0.726481in}}%
\pgfpathlineto{\pgfqpoint{0.431302in}{0.740092in}}%
\pgfpathlineto{\pgfqpoint{0.430501in}{0.753703in}}%
\pgfpathlineto{\pgfqpoint{0.433709in}{0.767314in}}%
\pgfpathlineto{\pgfqpoint{0.438698in}{0.776798in}}%
\pgfpathlineto{\pgfqpoint{0.440948in}{0.780925in}}%
\pgfpathlineto{\pgfqpoint{0.452225in}{0.794536in}}%
\pgfpathlineto{\pgfqpoint{0.454354in}{0.796556in}}%
\pgfpathlineto{\pgfqpoint{0.468845in}{0.808148in}}%
\pgfpathlineto{\pgfqpoint{0.470011in}{0.808973in}}%
\pgfpathlineto{\pgfqpoint{0.485668in}{0.817274in}}%
\pgfpathlineto{\pgfqpoint{0.500159in}{0.821759in}}%
\pgfpathlineto{\pgfqpoint{0.501324in}{0.822124in}}%
\pgfpathlineto{\pgfqpoint{0.516981in}{0.823556in}}%
\pgfpathlineto{\pgfqpoint{0.530100in}{0.821759in}}%
\pgfpathlineto{\pgfqpoint{0.532637in}{0.821420in}}%
\pgfpathlineto{\pgfqpoint{0.548294in}{0.815890in}}%
\pgfpathlineto{\pgfqpoint{0.561863in}{0.808148in}}%
\pgfpathlineto{\pgfqpoint{0.563950in}{0.806818in}}%
\pgfpathlineto{\pgfqpoint{0.578611in}{0.794536in}}%
\pgfpathlineto{\pgfqpoint{0.579607in}{0.793472in}}%
\pgfpathlineto{\pgfqpoint{0.589872in}{0.780925in}}%
\pgfpathlineto{\pgfqpoint{0.595263in}{0.770806in}}%
\pgfpathlineto{\pgfqpoint{0.597096in}{0.767314in}}%
\pgfpathlineto{\pgfqpoint{0.600340in}{0.753703in}}%
\pgfpathlineto{\pgfqpoint{0.599530in}{0.740092in}}%
\pgfpathlineto{\pgfqpoint{0.595263in}{0.728139in}}%
\pgfpathlineto{\pgfqpoint{0.594669in}{0.726481in}}%
\pgfpathlineto{\pgfqpoint{0.585875in}{0.712870in}}%
\pgfpathlineto{\pgfqpoint{0.579607in}{0.705990in}}%
\pgfpathlineto{\pgfqpoint{0.572548in}{0.699259in}}%
\pgfpathlineto{\pgfqpoint{0.563950in}{0.692489in}}%
\pgfpathlineto{\pgfqpoint{0.552482in}{0.685648in}}%
\pgfpathlineto{\pgfqpoint{0.548294in}{0.683316in}}%
\pgfpathlineto{\pgfqpoint{0.532637in}{0.677771in}}%
\pgfpathlineto{\pgfqpoint{0.516981in}{0.675695in}}%
\pgfpathlineto{\pgfqpoint{0.501324in}{0.677079in}}%
\pgfpathlineto{\pgfqpoint{0.485668in}{0.681928in}}%
\pgfpathlineto{\pgfqpoint{0.478508in}{0.685648in}}%
\pgfpathclose%
\pgfpathmoveto{\pgfqpoint{0.814455in}{0.658387in}}%
\pgfpathlineto{\pgfqpoint{0.830112in}{0.657555in}}%
\pgfpathlineto{\pgfqpoint{0.834248in}{0.658425in}}%
\pgfpathlineto{\pgfqpoint{0.845769in}{0.660437in}}%
\pgfpathlineto{\pgfqpoint{0.861425in}{0.666579in}}%
\pgfpathlineto{\pgfqpoint{0.870560in}{0.672036in}}%
\pgfpathlineto{\pgfqpoint{0.877082in}{0.675813in}}%
\pgfpathlineto{\pgfqpoint{0.890238in}{0.685648in}}%
\pgfpathlineto{\pgfqpoint{0.892738in}{0.687682in}}%
\pgfpathlineto{\pgfqpoint{0.905223in}{0.699259in}}%
\pgfpathlineto{\pgfqpoint{0.908395in}{0.702860in}}%
\pgfpathlineto{\pgfqpoint{0.917068in}{0.712870in}}%
\pgfpathlineto{\pgfqpoint{0.924051in}{0.724307in}}%
\pgfpathlineto{\pgfqpoint{0.925532in}{0.726481in}}%
\pgfpathlineto{\pgfqpoint{0.930748in}{0.740092in}}%
\pgfpathlineto{\pgfqpoint{0.931616in}{0.753703in}}%
\pgfpathlineto{\pgfqpoint{0.928142in}{0.767314in}}%
\pgfpathlineto{\pgfqpoint{0.924051in}{0.774530in}}%
\pgfpathlineto{\pgfqpoint{0.920823in}{0.780925in}}%
\pgfpathlineto{\pgfqpoint{0.910322in}{0.794536in}}%
\pgfpathlineto{\pgfqpoint{0.908395in}{0.796500in}}%
\pgfpathlineto{\pgfqpoint{0.896690in}{0.808148in}}%
\pgfpathlineto{\pgfqpoint{0.892738in}{0.811583in}}%
\pgfpathlineto{\pgfqpoint{0.879341in}{0.821759in}}%
\pgfpathlineto{\pgfqpoint{0.877082in}{0.823434in}}%
\pgfpathlineto{\pgfqpoint{0.861425in}{0.832563in}}%
\pgfpathlineto{\pgfqpoint{0.854069in}{0.835370in}}%
\pgfpathlineto{\pgfqpoint{0.845769in}{0.838926in}}%
\pgfpathlineto{\pgfqpoint{0.830112in}{0.841946in}}%
\pgfpathlineto{\pgfqpoint{0.814455in}{0.841191in}}%
\pgfpathlineto{\pgfqpoint{0.798799in}{0.836657in}}%
\pgfpathlineto{\pgfqpoint{0.796298in}{0.835370in}}%
\pgfpathlineto{\pgfqpoint{0.783142in}{0.829298in}}%
\pgfpathlineto{\pgfqpoint{0.771628in}{0.821759in}}%
\pgfpathlineto{\pgfqpoint{0.767486in}{0.819001in}}%
\pgfpathlineto{\pgfqpoint{0.754170in}{0.808147in}}%
\pgfpathlineto{\pgfqpoint{0.751829in}{0.805974in}}%
\pgfpathlineto{\pgfqpoint{0.740516in}{0.794536in}}%
\pgfpathlineto{\pgfqpoint{0.736173in}{0.788867in}}%
\pgfpathlineto{\pgfqpoint{0.729895in}{0.780925in}}%
\pgfpathlineto{\pgfqpoint{0.722830in}{0.767314in}}%
\pgfpathlineto{\pgfqpoint{0.720516in}{0.757299in}}%
\pgfpathlineto{\pgfqpoint{0.719515in}{0.753703in}}%
\pgfpathlineto{\pgfqpoint{0.720472in}{0.740092in}}%
\pgfpathlineto{\pgfqpoint{0.720516in}{0.739985in}}%
\pgfpathlineto{\pgfqpoint{0.725184in}{0.726481in}}%
\pgfpathlineto{\pgfqpoint{0.733822in}{0.712870in}}%
\pgfpathlineto{\pgfqpoint{0.736173in}{0.710199in}}%
\pgfpathlineto{\pgfqpoint{0.745559in}{0.699259in}}%
\pgfpathlineto{\pgfqpoint{0.751829in}{0.693300in}}%
\pgfpathlineto{\pgfqpoint{0.760632in}{0.685648in}}%
\pgfpathlineto{\pgfqpoint{0.767486in}{0.680196in}}%
\pgfpathlineto{\pgfqpoint{0.780070in}{0.672036in}}%
\pgfpathlineto{\pgfqpoint{0.783142in}{0.669993in}}%
\pgfpathlineto{\pgfqpoint{0.798799in}{0.662483in}}%
\pgfpathlineto{\pgfqpoint{0.814332in}{0.658425in}}%
\pgfpathlineto{\pgfqpoint{0.814455in}{0.658387in}}%
\pgfpathclose%
\pgfpathmoveto{\pgfqpoint{0.788469in}{0.685648in}}%
\pgfpathlineto{\pgfqpoint{0.783142in}{0.688374in}}%
\pgfpathlineto{\pgfqpoint{0.768055in}{0.699259in}}%
\pgfpathlineto{\pgfqpoint{0.767486in}{0.699754in}}%
\pgfpathlineto{\pgfqpoint{0.754966in}{0.712870in}}%
\pgfpathlineto{\pgfqpoint{0.751829in}{0.717500in}}%
\pgfpathlineto{\pgfqpoint{0.746116in}{0.726481in}}%
\pgfpathlineto{\pgfqpoint{0.741336in}{0.740092in}}%
\pgfpathlineto{\pgfqpoint{0.740541in}{0.753703in}}%
\pgfpathlineto{\pgfqpoint{0.743724in}{0.767314in}}%
\pgfpathlineto{\pgfqpoint{0.750903in}{0.780925in}}%
\pgfpathlineto{\pgfqpoint{0.751829in}{0.782107in}}%
\pgfpathlineto{\pgfqpoint{0.762305in}{0.794536in}}%
\pgfpathlineto{\pgfqpoint{0.767486in}{0.799329in}}%
\pgfpathlineto{\pgfqpoint{0.779060in}{0.808148in}}%
\pgfpathlineto{\pgfqpoint{0.783142in}{0.810909in}}%
\pgfpathlineto{\pgfqpoint{0.798799in}{0.818518in}}%
\pgfpathlineto{\pgfqpoint{0.810983in}{0.821759in}}%
\pgfpathlineto{\pgfqpoint{0.814455in}{0.822697in}}%
\pgfpathlineto{\pgfqpoint{0.830112in}{0.823412in}}%
\pgfpathlineto{\pgfqpoint{0.839204in}{0.821759in}}%
\pgfpathlineto{\pgfqpoint{0.845769in}{0.820591in}}%
\pgfpathlineto{\pgfqpoint{0.861425in}{0.814368in}}%
\pgfpathlineto{\pgfqpoint{0.871666in}{0.808148in}}%
\pgfpathlineto{\pgfqpoint{0.877082in}{0.804462in}}%
\pgfpathlineto{\pgfqpoint{0.888499in}{0.794536in}}%
\pgfpathlineto{\pgfqpoint{0.892738in}{0.789828in}}%
\pgfpathlineto{\pgfqpoint{0.899893in}{0.780925in}}%
\pgfpathlineto{\pgfqpoint{0.907052in}{0.767314in}}%
\pgfpathlineto{\pgfqpoint{0.908395in}{0.761608in}}%
\pgfpathlineto{\pgfqpoint{0.910297in}{0.753703in}}%
\pgfpathlineto{\pgfqpoint{0.909474in}{0.740092in}}%
\pgfpathlineto{\pgfqpoint{0.908395in}{0.737073in}}%
\pgfpathlineto{\pgfqpoint{0.904668in}{0.726481in}}%
\pgfpathlineto{\pgfqpoint{0.895915in}{0.712870in}}%
\pgfpathlineto{\pgfqpoint{0.892738in}{0.709320in}}%
\pgfpathlineto{\pgfqpoint{0.882594in}{0.699259in}}%
\pgfpathlineto{\pgfqpoint{0.877082in}{0.694754in}}%
\pgfpathlineto{\pgfqpoint{0.862784in}{0.685648in}}%
\pgfpathlineto{\pgfqpoint{0.861425in}{0.684842in}}%
\pgfpathlineto{\pgfqpoint{0.845769in}{0.678602in}}%
\pgfpathlineto{\pgfqpoint{0.830112in}{0.675834in}}%
\pgfpathlineto{\pgfqpoint{0.814455in}{0.676525in}}%
\pgfpathlineto{\pgfqpoint{0.798799in}{0.680680in}}%
\pgfpathlineto{\pgfqpoint{0.788469in}{0.685648in}}%
\pgfpathclose%
\pgfpathmoveto{\pgfqpoint{1.127587in}{0.657888in}}%
\pgfpathlineto{\pgfqpoint{1.143243in}{0.657888in}}%
\pgfpathlineto{\pgfqpoint{1.145297in}{0.658425in}}%
\pgfpathlineto{\pgfqpoint{1.158900in}{0.661392in}}%
\pgfpathlineto{\pgfqpoint{1.174556in}{0.668218in}}%
\pgfpathlineto{\pgfqpoint{1.180597in}{0.672036in}}%
\pgfpathlineto{\pgfqpoint{1.190213in}{0.677945in}}%
\pgfpathlineto{\pgfqpoint{1.200183in}{0.685648in}}%
\pgfpathlineto{\pgfqpoint{1.205870in}{0.690441in}}%
\pgfpathlineto{\pgfqpoint{1.215246in}{0.699259in}}%
\pgfpathlineto{\pgfqpoint{1.221526in}{0.706497in}}%
\pgfpathlineto{\pgfqpoint{1.227078in}{0.712870in}}%
\pgfpathlineto{\pgfqpoint{1.235507in}{0.726481in}}%
\pgfpathlineto{\pgfqpoint{1.237183in}{0.731383in}}%
\pgfpathlineto{\pgfqpoint{1.240648in}{0.740092in}}%
\pgfpathlineto{\pgfqpoint{1.241556in}{0.753703in}}%
\pgfpathlineto{\pgfqpoint{1.237920in}{0.767314in}}%
\pgfpathlineto{\pgfqpoint{1.237183in}{0.768575in}}%
\pgfpathlineto{\pgfqpoint{1.230910in}{0.780925in}}%
\pgfpathlineto{\pgfqpoint{1.221526in}{0.792917in}}%
\pgfpathlineto{\pgfqpoint{1.220271in}{0.794536in}}%
\pgfpathlineto{\pgfqpoint{1.206719in}{0.808148in}}%
\pgfpathlineto{\pgfqpoint{1.205870in}{0.808896in}}%
\pgfpathlineto{\pgfqpoint{1.190213in}{0.821246in}}%
\pgfpathlineto{\pgfqpoint{1.189401in}{0.821759in}}%
\pgfpathlineto{\pgfqpoint{1.174556in}{0.830996in}}%
\pgfpathlineto{\pgfqpoint{1.164190in}{0.835370in}}%
\pgfpathlineto{\pgfqpoint{1.158900in}{0.837867in}}%
\pgfpathlineto{\pgfqpoint{1.143243in}{0.841644in}}%
\pgfpathlineto{\pgfqpoint{1.127587in}{0.841644in}}%
\pgfpathlineto{\pgfqpoint{1.111930in}{0.837867in}}%
\pgfpathlineto{\pgfqpoint{1.106640in}{0.835370in}}%
\pgfpathlineto{\pgfqpoint{1.096274in}{0.830996in}}%
\pgfpathlineto{\pgfqpoint{1.081429in}{0.821759in}}%
\pgfpathlineto{\pgfqpoint{1.080617in}{0.821246in}}%
\pgfpathlineto{\pgfqpoint{1.064960in}{0.808896in}}%
\pgfpathlineto{\pgfqpoint{1.064111in}{0.808148in}}%
\pgfpathlineto{\pgfqpoint{1.050559in}{0.794536in}}%
\pgfpathlineto{\pgfqpoint{1.049304in}{0.792917in}}%
\pgfpathlineto{\pgfqpoint{1.039920in}{0.780925in}}%
\pgfpathlineto{\pgfqpoint{1.033647in}{0.768575in}}%
\pgfpathlineto{\pgfqpoint{1.032910in}{0.767314in}}%
\pgfpathlineto{\pgfqpoint{1.029274in}{0.753703in}}%
\pgfpathlineto{\pgfqpoint{1.030182in}{0.740092in}}%
\pgfpathlineto{\pgfqpoint{1.033647in}{0.731383in}}%
\pgfpathlineto{\pgfqpoint{1.035323in}{0.726481in}}%
\pgfpathlineto{\pgfqpoint{1.043752in}{0.712870in}}%
\pgfpathlineto{\pgfqpoint{1.049304in}{0.706497in}}%
\pgfpathlineto{\pgfqpoint{1.055584in}{0.699259in}}%
\pgfpathlineto{\pgfqpoint{1.064960in}{0.690441in}}%
\pgfpathlineto{\pgfqpoint{1.070647in}{0.685648in}}%
\pgfpathlineto{\pgfqpoint{1.080617in}{0.677945in}}%
\pgfpathlineto{\pgfqpoint{1.090233in}{0.672036in}}%
\pgfpathlineto{\pgfqpoint{1.096274in}{0.668218in}}%
\pgfpathlineto{\pgfqpoint{1.111930in}{0.661392in}}%
\pgfpathlineto{\pgfqpoint{1.125533in}{0.658425in}}%
\pgfpathlineto{\pgfqpoint{1.127587in}{0.657888in}}%
\pgfpathclose%
\pgfpathmoveto{\pgfqpoint{1.098154in}{0.685648in}}%
\pgfpathlineto{\pgfqpoint{1.096274in}{0.686528in}}%
\pgfpathlineto{\pgfqpoint{1.080617in}{0.697156in}}%
\pgfpathlineto{\pgfqpoint{1.078128in}{0.699259in}}%
\pgfpathlineto{\pgfqpoint{1.064960in}{0.712785in}}%
\pgfpathlineto{\pgfqpoint{1.064886in}{0.712870in}}%
\pgfpathlineto{\pgfqpoint{1.056139in}{0.726481in}}%
\pgfpathlineto{\pgfqpoint{1.051375in}{0.740092in}}%
\pgfpathlineto{\pgfqpoint{1.050583in}{0.753703in}}%
\pgfpathlineto{\pgfqpoint{1.053756in}{0.767314in}}%
\pgfpathlineto{\pgfqpoint{1.060910in}{0.780925in}}%
\pgfpathlineto{\pgfqpoint{1.064960in}{0.786037in}}%
\pgfpathlineto{\pgfqpoint{1.072349in}{0.794536in}}%
\pgfpathlineto{\pgfqpoint{1.080617in}{0.801965in}}%
\pgfpathlineto{\pgfqpoint{1.089179in}{0.808148in}}%
\pgfpathlineto{\pgfqpoint{1.096274in}{0.812707in}}%
\pgfpathlineto{\pgfqpoint{1.111930in}{0.819624in}}%
\pgfpathlineto{\pgfqpoint{1.121544in}{0.821759in}}%
\pgfpathlineto{\pgfqpoint{1.127587in}{0.823126in}}%
\pgfpathlineto{\pgfqpoint{1.143243in}{0.823126in}}%
\pgfpathlineto{\pgfqpoint{1.149286in}{0.821759in}}%
\pgfpathlineto{\pgfqpoint{1.158900in}{0.819624in}}%
\pgfpathlineto{\pgfqpoint{1.174556in}{0.812707in}}%
\pgfpathlineto{\pgfqpoint{1.181651in}{0.808148in}}%
\pgfpathlineto{\pgfqpoint{1.190213in}{0.801965in}}%
\pgfpathlineto{\pgfqpoint{1.198481in}{0.794536in}}%
\pgfpathlineto{\pgfqpoint{1.205870in}{0.786037in}}%
\pgfpathlineto{\pgfqpoint{1.209920in}{0.780925in}}%
\pgfpathlineto{\pgfqpoint{1.217074in}{0.767314in}}%
\pgfpathlineto{\pgfqpoint{1.220247in}{0.753703in}}%
\pgfpathlineto{\pgfqpoint{1.219455in}{0.740092in}}%
\pgfpathlineto{\pgfqpoint{1.214691in}{0.726481in}}%
\pgfpathlineto{\pgfqpoint{1.205944in}{0.712870in}}%
\pgfpathlineto{\pgfqpoint{1.205870in}{0.712785in}}%
\pgfpathlineto{\pgfqpoint{1.192702in}{0.699259in}}%
\pgfpathlineto{\pgfqpoint{1.190213in}{0.697156in}}%
\pgfpathlineto{\pgfqpoint{1.174556in}{0.686528in}}%
\pgfpathlineto{\pgfqpoint{1.172676in}{0.685648in}}%
\pgfpathlineto{\pgfqpoint{1.158900in}{0.679571in}}%
\pgfpathlineto{\pgfqpoint{1.143243in}{0.676110in}}%
\pgfpathlineto{\pgfqpoint{1.127587in}{0.676110in}}%
\pgfpathlineto{\pgfqpoint{1.111930in}{0.679571in}}%
\pgfpathlineto{\pgfqpoint{1.098154in}{0.685648in}}%
\pgfpathclose%
\pgfpathmoveto{\pgfqpoint{1.440718in}{0.657555in}}%
\pgfpathlineto{\pgfqpoint{1.456375in}{0.658387in}}%
\pgfpathlineto{\pgfqpoint{1.456498in}{0.658425in}}%
\pgfpathlineto{\pgfqpoint{1.472031in}{0.662483in}}%
\pgfpathlineto{\pgfqpoint{1.487688in}{0.669993in}}%
\pgfpathlineto{\pgfqpoint{1.490760in}{0.672036in}}%
\pgfpathlineto{\pgfqpoint{1.503344in}{0.680196in}}%
\pgfpathlineto{\pgfqpoint{1.510198in}{0.685648in}}%
\pgfpathlineto{\pgfqpoint{1.519001in}{0.693300in}}%
\pgfpathlineto{\pgfqpoint{1.525271in}{0.699259in}}%
\pgfpathlineto{\pgfqpoint{1.534657in}{0.710199in}}%
\pgfpathlineto{\pgfqpoint{1.537008in}{0.712870in}}%
\pgfpathlineto{\pgfqpoint{1.545646in}{0.726481in}}%
\pgfpathlineto{\pgfqpoint{1.550314in}{0.739985in}}%
\pgfpathlineto{\pgfqpoint{1.550358in}{0.740092in}}%
\pgfpathlineto{\pgfqpoint{1.551315in}{0.753703in}}%
\pgfpathlineto{\pgfqpoint{1.550314in}{0.757299in}}%
\pgfpathlineto{\pgfqpoint{1.548000in}{0.767314in}}%
\pgfpathlineto{\pgfqpoint{1.540935in}{0.780925in}}%
\pgfpathlineto{\pgfqpoint{1.534657in}{0.788867in}}%
\pgfpathlineto{\pgfqpoint{1.530314in}{0.794536in}}%
\pgfpathlineto{\pgfqpoint{1.519001in}{0.805974in}}%
\pgfpathlineto{\pgfqpoint{1.516660in}{0.808147in}}%
\pgfpathlineto{\pgfqpoint{1.503344in}{0.819001in}}%
\pgfpathlineto{\pgfqpoint{1.499202in}{0.821759in}}%
\pgfpathlineto{\pgfqpoint{1.487688in}{0.829298in}}%
\pgfpathlineto{\pgfqpoint{1.474532in}{0.835370in}}%
\pgfpathlineto{\pgfqpoint{1.472031in}{0.836657in}}%
\pgfpathlineto{\pgfqpoint{1.456375in}{0.841191in}}%
\pgfpathlineto{\pgfqpoint{1.440718in}{0.841946in}}%
\pgfpathlineto{\pgfqpoint{1.425061in}{0.838926in}}%
\pgfpathlineto{\pgfqpoint{1.416761in}{0.835370in}}%
\pgfpathlineto{\pgfqpoint{1.409405in}{0.832563in}}%
\pgfpathlineto{\pgfqpoint{1.393748in}{0.823434in}}%
\pgfpathlineto{\pgfqpoint{1.391489in}{0.821759in}}%
\pgfpathlineto{\pgfqpoint{1.378092in}{0.811583in}}%
\pgfpathlineto{\pgfqpoint{1.374140in}{0.808148in}}%
\pgfpathlineto{\pgfqpoint{1.362435in}{0.796500in}}%
\pgfpathlineto{\pgfqpoint{1.360508in}{0.794536in}}%
\pgfpathlineto{\pgfqpoint{1.350007in}{0.780925in}}%
\pgfpathlineto{\pgfqpoint{1.346779in}{0.774530in}}%
\pgfpathlineto{\pgfqpoint{1.342688in}{0.767314in}}%
\pgfpathlineto{\pgfqpoint{1.339214in}{0.753703in}}%
\pgfpathlineto{\pgfqpoint{1.340082in}{0.740092in}}%
\pgfpathlineto{\pgfqpoint{1.345298in}{0.726481in}}%
\pgfpathlineto{\pgfqpoint{1.346779in}{0.724307in}}%
\pgfpathlineto{\pgfqpoint{1.353762in}{0.712870in}}%
\pgfpathlineto{\pgfqpoint{1.362435in}{0.702860in}}%
\pgfpathlineto{\pgfqpoint{1.365607in}{0.699259in}}%
\pgfpathlineto{\pgfqpoint{1.378092in}{0.687682in}}%
\pgfpathlineto{\pgfqpoint{1.380592in}{0.685648in}}%
\pgfpathlineto{\pgfqpoint{1.393748in}{0.675813in}}%
\pgfpathlineto{\pgfqpoint{1.400270in}{0.672036in}}%
\pgfpathlineto{\pgfqpoint{1.409405in}{0.666579in}}%
\pgfpathlineto{\pgfqpoint{1.425061in}{0.660437in}}%
\pgfpathlineto{\pgfqpoint{1.436582in}{0.658425in}}%
\pgfpathlineto{\pgfqpoint{1.440718in}{0.657555in}}%
\pgfpathclose%
\pgfpathmoveto{\pgfqpoint{1.408046in}{0.685648in}}%
\pgfpathlineto{\pgfqpoint{1.393748in}{0.694754in}}%
\pgfpathlineto{\pgfqpoint{1.388236in}{0.699259in}}%
\pgfpathlineto{\pgfqpoint{1.378092in}{0.709320in}}%
\pgfpathlineto{\pgfqpoint{1.374915in}{0.712870in}}%
\pgfpathlineto{\pgfqpoint{1.366162in}{0.726481in}}%
\pgfpathlineto{\pgfqpoint{1.362435in}{0.737073in}}%
\pgfpathlineto{\pgfqpoint{1.361356in}{0.740092in}}%
\pgfpathlineto{\pgfqpoint{1.360533in}{0.753703in}}%
\pgfpathlineto{\pgfqpoint{1.362435in}{0.761608in}}%
\pgfpathlineto{\pgfqpoint{1.363778in}{0.767314in}}%
\pgfpathlineto{\pgfqpoint{1.370937in}{0.780925in}}%
\pgfpathlineto{\pgfqpoint{1.378092in}{0.789828in}}%
\pgfpathlineto{\pgfqpoint{1.382331in}{0.794536in}}%
\pgfpathlineto{\pgfqpoint{1.393748in}{0.804462in}}%
\pgfpathlineto{\pgfqpoint{1.399164in}{0.808148in}}%
\pgfpathlineto{\pgfqpoint{1.409405in}{0.814368in}}%
\pgfpathlineto{\pgfqpoint{1.425061in}{0.820591in}}%
\pgfpathlineto{\pgfqpoint{1.431626in}{0.821759in}}%
\pgfpathlineto{\pgfqpoint{1.440718in}{0.823412in}}%
\pgfpathlineto{\pgfqpoint{1.456375in}{0.822697in}}%
\pgfpathlineto{\pgfqpoint{1.459847in}{0.821759in}}%
\pgfpathlineto{\pgfqpoint{1.472031in}{0.818518in}}%
\pgfpathlineto{\pgfqpoint{1.487688in}{0.810909in}}%
\pgfpathlineto{\pgfqpoint{1.491770in}{0.808148in}}%
\pgfpathlineto{\pgfqpoint{1.503344in}{0.799329in}}%
\pgfpathlineto{\pgfqpoint{1.508525in}{0.794536in}}%
\pgfpathlineto{\pgfqpoint{1.519001in}{0.782107in}}%
\pgfpathlineto{\pgfqpoint{1.519927in}{0.780925in}}%
\pgfpathlineto{\pgfqpoint{1.527106in}{0.767314in}}%
\pgfpathlineto{\pgfqpoint{1.530289in}{0.753703in}}%
\pgfpathlineto{\pgfqpoint{1.529494in}{0.740092in}}%
\pgfpathlineto{\pgfqpoint{1.524714in}{0.726481in}}%
\pgfpathlineto{\pgfqpoint{1.519001in}{0.717500in}}%
\pgfpathlineto{\pgfqpoint{1.515864in}{0.712870in}}%
\pgfpathlineto{\pgfqpoint{1.503344in}{0.699754in}}%
\pgfpathlineto{\pgfqpoint{1.502775in}{0.699259in}}%
\pgfpathlineto{\pgfqpoint{1.487688in}{0.688374in}}%
\pgfpathlineto{\pgfqpoint{1.482361in}{0.685648in}}%
\pgfpathlineto{\pgfqpoint{1.472031in}{0.680680in}}%
\pgfpathlineto{\pgfqpoint{1.456375in}{0.676525in}}%
\pgfpathlineto{\pgfqpoint{1.440718in}{0.675834in}}%
\pgfpathlineto{\pgfqpoint{1.425061in}{0.678602in}}%
\pgfpathlineto{\pgfqpoint{1.409405in}{0.684842in}}%
\pgfpathlineto{\pgfqpoint{1.408046in}{0.685648in}}%
\pgfpathclose%
\pgfpathmoveto{\pgfqpoint{1.753849in}{0.657389in}}%
\pgfpathlineto{\pgfqpoint{1.763630in}{0.658425in}}%
\pgfpathlineto{\pgfqpoint{1.769506in}{0.658939in}}%
\pgfpathlineto{\pgfqpoint{1.785162in}{0.663712in}}%
\pgfpathlineto{\pgfqpoint{1.800819in}{0.671903in}}%
\pgfpathlineto{\pgfqpoint{1.801011in}{0.672036in}}%
\pgfpathlineto{\pgfqpoint{1.816476in}{0.682564in}}%
\pgfpathlineto{\pgfqpoint{1.820256in}{0.685648in}}%
\pgfpathlineto{\pgfqpoint{1.832132in}{0.696254in}}%
\pgfpathlineto{\pgfqpoint{1.835273in}{0.699259in}}%
\pgfpathlineto{\pgfqpoint{1.846917in}{0.712870in}}%
\pgfpathlineto{\pgfqpoint{1.847789in}{0.714295in}}%
\pgfpathlineto{\pgfqpoint{1.855719in}{0.726481in}}%
\pgfpathlineto{\pgfqpoint{1.860563in}{0.740092in}}%
\pgfpathlineto{\pgfqpoint{1.861369in}{0.753703in}}%
\pgfpathlineto{\pgfqpoint{1.858142in}{0.767314in}}%
\pgfpathlineto{\pgfqpoint{1.850868in}{0.780925in}}%
\pgfpathlineto{\pgfqpoint{1.847789in}{0.784763in}}%
\pgfpathlineto{\pgfqpoint{1.840353in}{0.794536in}}%
\pgfpathlineto{\pgfqpoint{1.832132in}{0.802903in}}%
\pgfpathlineto{\pgfqpoint{1.826635in}{0.808148in}}%
\pgfpathlineto{\pgfqpoint{1.816476in}{0.816640in}}%
\pgfpathlineto{\pgfqpoint{1.809151in}{0.821759in}}%
\pgfpathlineto{\pgfqpoint{1.800819in}{0.827472in}}%
\pgfpathlineto{\pgfqpoint{1.785162in}{0.835305in}}%
\pgfpathlineto{\pgfqpoint{1.784946in}{0.835370in}}%
\pgfpathlineto{\pgfqpoint{1.769506in}{0.840588in}}%
\pgfpathlineto{\pgfqpoint{1.753849in}{0.842097in}}%
\pgfpathlineto{\pgfqpoint{1.738193in}{0.839832in}}%
\pgfpathlineto{\pgfqpoint{1.726560in}{0.835370in}}%
\pgfpathlineto{\pgfqpoint{1.722536in}{0.833999in}}%
\pgfpathlineto{\pgfqpoint{1.706880in}{0.825517in}}%
\pgfpathlineto{\pgfqpoint{1.701621in}{0.821759in}}%
\pgfpathlineto{\pgfqpoint{1.691223in}{0.814166in}}%
\pgfpathlineto{\pgfqpoint{1.684176in}{0.808148in}}%
\pgfpathlineto{\pgfqpoint{1.675567in}{0.799742in}}%
\pgfpathlineto{\pgfqpoint{1.670465in}{0.794536in}}%
\pgfpathlineto{\pgfqpoint{1.660129in}{0.780925in}}%
\pgfpathlineto{\pgfqpoint{1.659910in}{0.780492in}}%
\pgfpathlineto{\pgfqpoint{1.652629in}{0.767314in}}%
\pgfpathlineto{\pgfqpoint{1.649289in}{0.753703in}}%
\pgfpathlineto{\pgfqpoint{1.650123in}{0.740092in}}%
\pgfpathlineto{\pgfqpoint{1.655137in}{0.726481in}}%
\pgfpathlineto{\pgfqpoint{1.659910in}{0.719291in}}%
\pgfpathlineto{\pgfqpoint{1.663825in}{0.712870in}}%
\pgfpathlineto{\pgfqpoint{1.675567in}{0.699300in}}%
\pgfpathlineto{\pgfqpoint{1.675603in}{0.699259in}}%
\pgfpathlineto{\pgfqpoint{1.690500in}{0.685648in}}%
\pgfpathlineto{\pgfqpoint{1.691223in}{0.685045in}}%
\pgfpathlineto{\pgfqpoint{1.706880in}{0.673801in}}%
\pgfpathlineto{\pgfqpoint{1.710134in}{0.672036in}}%
\pgfpathlineto{\pgfqpoint{1.722536in}{0.665077in}}%
\pgfpathlineto{\pgfqpoint{1.738193in}{0.659620in}}%
\pgfpathlineto{\pgfqpoint{1.747307in}{0.658425in}}%
\pgfpathlineto{\pgfqpoint{1.753849in}{0.657389in}}%
\pgfpathclose%
\pgfpathmoveto{\pgfqpoint{1.718348in}{0.685648in}}%
\pgfpathlineto{\pgfqpoint{1.706880in}{0.692489in}}%
\pgfpathlineto{\pgfqpoint{1.698282in}{0.699259in}}%
\pgfpathlineto{\pgfqpoint{1.691223in}{0.705990in}}%
\pgfpathlineto{\pgfqpoint{1.684955in}{0.712870in}}%
\pgfpathlineto{\pgfqpoint{1.676161in}{0.726481in}}%
\pgfpathlineto{\pgfqpoint{1.675567in}{0.728139in}}%
\pgfpathlineto{\pgfqpoint{1.671300in}{0.740092in}}%
\pgfpathlineto{\pgfqpoint{1.670490in}{0.753703in}}%
\pgfpathlineto{\pgfqpoint{1.673734in}{0.767314in}}%
\pgfpathlineto{\pgfqpoint{1.675567in}{0.770806in}}%
\pgfpathlineto{\pgfqpoint{1.680958in}{0.780925in}}%
\pgfpathlineto{\pgfqpoint{1.691223in}{0.793472in}}%
\pgfpathlineto{\pgfqpoint{1.692219in}{0.794536in}}%
\pgfpathlineto{\pgfqpoint{1.706880in}{0.806818in}}%
\pgfpathlineto{\pgfqpoint{1.708967in}{0.808148in}}%
\pgfpathlineto{\pgfqpoint{1.722536in}{0.815890in}}%
\pgfpathlineto{\pgfqpoint{1.738193in}{0.821420in}}%
\pgfpathlineto{\pgfqpoint{1.740730in}{0.821759in}}%
\pgfpathlineto{\pgfqpoint{1.753849in}{0.823556in}}%
\pgfpathlineto{\pgfqpoint{1.769506in}{0.822124in}}%
\pgfpathlineto{\pgfqpoint{1.770671in}{0.821759in}}%
\pgfpathlineto{\pgfqpoint{1.785162in}{0.817274in}}%
\pgfpathlineto{\pgfqpoint{1.800819in}{0.808973in}}%
\pgfpathlineto{\pgfqpoint{1.801985in}{0.808147in}}%
\pgfpathlineto{\pgfqpoint{1.816476in}{0.796556in}}%
\pgfpathlineto{\pgfqpoint{1.818605in}{0.794536in}}%
\pgfpathlineto{\pgfqpoint{1.829882in}{0.780925in}}%
\pgfpathlineto{\pgfqpoint{1.832132in}{0.776798in}}%
\pgfpathlineto{\pgfqpoint{1.837121in}{0.767314in}}%
\pgfpathlineto{\pgfqpoint{1.840329in}{0.753703in}}%
\pgfpathlineto{\pgfqpoint{1.839528in}{0.740092in}}%
\pgfpathlineto{\pgfqpoint{1.834712in}{0.726481in}}%
\pgfpathlineto{\pgfqpoint{1.832132in}{0.722399in}}%
\pgfpathlineto{\pgfqpoint{1.825849in}{0.712870in}}%
\pgfpathlineto{\pgfqpoint{1.816476in}{0.702799in}}%
\pgfpathlineto{\pgfqpoint{1.812596in}{0.699259in}}%
\pgfpathlineto{\pgfqpoint{1.800819in}{0.690362in}}%
\pgfpathlineto{\pgfqpoint{1.792322in}{0.685648in}}%
\pgfpathlineto{\pgfqpoint{1.785162in}{0.681928in}}%
\pgfpathlineto{\pgfqpoint{1.769506in}{0.677079in}}%
\pgfpathlineto{\pgfqpoint{1.753849in}{0.675695in}}%
\pgfpathlineto{\pgfqpoint{1.738193in}{0.677771in}}%
\pgfpathlineto{\pgfqpoint{1.722536in}{0.683316in}}%
\pgfpathlineto{\pgfqpoint{1.718348in}{0.685648in}}%
\pgfpathclose%
\pgfpathmoveto{\pgfqpoint{0.501324in}{0.928267in}}%
\pgfpathlineto{\pgfqpoint{0.516981in}{0.926687in}}%
\pgfpathlineto{\pgfqpoint{0.532637in}{0.929057in}}%
\pgfpathlineto{\pgfqpoint{0.536654in}{0.930648in}}%
\pgfpathlineto{\pgfqpoint{0.548294in}{0.934635in}}%
\pgfpathlineto{\pgfqpoint{0.563950in}{0.943290in}}%
\pgfpathlineto{\pgfqpoint{0.565298in}{0.944259in}}%
\pgfpathlineto{\pgfqpoint{0.579607in}{0.954550in}}%
\pgfpathlineto{\pgfqpoint{0.583549in}{0.957870in}}%
\pgfpathlineto{\pgfqpoint{0.595263in}{0.968912in}}%
\pgfpathlineto{\pgfqpoint{0.597865in}{0.971481in}}%
\pgfpathlineto{\pgfqpoint{0.608927in}{0.985092in}}%
\pgfpathlineto{\pgfqpoint{0.610920in}{0.988667in}}%
\pgfpathlineto{\pgfqpoint{0.617031in}{0.998703in}}%
\pgfpathlineto{\pgfqpoint{0.621207in}{1.012314in}}%
\pgfpathlineto{\pgfqpoint{0.621207in}{1.025925in}}%
\pgfpathlineto{\pgfqpoint{0.617031in}{1.039536in}}%
\pgfpathlineto{\pgfqpoint{0.610920in}{1.049572in}}%
\pgfpathlineto{\pgfqpoint{0.608927in}{1.053148in}}%
\pgfpathlineto{\pgfqpoint{0.597865in}{1.066759in}}%
\pgfpathlineto{\pgfqpoint{0.595263in}{1.069327in}}%
\pgfpathlineto{\pgfqpoint{0.583549in}{1.080370in}}%
\pgfpathlineto{\pgfqpoint{0.579607in}{1.083689in}}%
\pgfpathlineto{\pgfqpoint{0.565298in}{1.093981in}}%
\pgfpathlineto{\pgfqpoint{0.563950in}{1.094949in}}%
\pgfpathlineto{\pgfqpoint{0.548294in}{1.103604in}}%
\pgfpathlineto{\pgfqpoint{0.536654in}{1.107592in}}%
\pgfpathlineto{\pgfqpoint{0.532637in}{1.109182in}}%
\pgfpathlineto{\pgfqpoint{0.516981in}{1.111552in}}%
\pgfpathlineto{\pgfqpoint{0.501324in}{1.109973in}}%
\pgfpathlineto{\pgfqpoint{0.494498in}{1.107592in}}%
\pgfpathlineto{\pgfqpoint{0.485668in}{1.104936in}}%
\pgfpathlineto{\pgfqpoint{0.470011in}{1.096944in}}%
\pgfpathlineto{\pgfqpoint{0.465714in}{1.093981in}}%
\pgfpathlineto{\pgfqpoint{0.454354in}{1.086161in}}%
\pgfpathlineto{\pgfqpoint{0.447327in}{1.080370in}}%
\pgfpathlineto{\pgfqpoint{0.438698in}{1.072421in}}%
\pgfpathlineto{\pgfqpoint{0.432948in}{1.066759in}}%
\pgfpathlineto{\pgfqpoint{0.423041in}{1.054488in}}%
\pgfpathlineto{\pgfqpoint{0.421903in}{1.053148in}}%
\pgfpathlineto{\pgfqpoint{0.413818in}{1.039536in}}%
\pgfpathlineto{\pgfqpoint{0.409783in}{1.025925in}}%
\pgfpathlineto{\pgfqpoint{0.409783in}{1.012314in}}%
\pgfpathlineto{\pgfqpoint{0.413818in}{0.998703in}}%
\pgfpathlineto{\pgfqpoint{0.421903in}{0.985092in}}%
\pgfpathlineto{\pgfqpoint{0.423041in}{0.983752in}}%
\pgfpathlineto{\pgfqpoint{0.432948in}{0.971481in}}%
\pgfpathlineto{\pgfqpoint{0.438698in}{0.965819in}}%
\pgfpathlineto{\pgfqpoint{0.447327in}{0.957870in}}%
\pgfpathlineto{\pgfqpoint{0.454354in}{0.952078in}}%
\pgfpathlineto{\pgfqpoint{0.465714in}{0.944259in}}%
\pgfpathlineto{\pgfqpoint{0.470011in}{0.941295in}}%
\pgfpathlineto{\pgfqpoint{0.485668in}{0.933303in}}%
\pgfpathlineto{\pgfqpoint{0.494498in}{0.930648in}}%
\pgfpathlineto{\pgfqpoint{0.501324in}{0.928267in}}%
\pgfpathclose%
\pgfpathmoveto{\pgfqpoint{0.473426in}{0.957870in}}%
\pgfpathlineto{\pgfqpoint{0.470011in}{0.959824in}}%
\pgfpathlineto{\pgfqpoint{0.455022in}{0.971481in}}%
\pgfpathlineto{\pgfqpoint{0.454354in}{0.972124in}}%
\pgfpathlineto{\pgfqpoint{0.442884in}{0.985092in}}%
\pgfpathlineto{\pgfqpoint{0.438698in}{0.992036in}}%
\pgfpathlineto{\pgfqpoint{0.434833in}{0.998703in}}%
\pgfpathlineto{\pgfqpoint{0.430822in}{1.012314in}}%
\pgfpathlineto{\pgfqpoint{0.430822in}{1.025925in}}%
\pgfpathlineto{\pgfqpoint{0.434833in}{1.039536in}}%
\pgfpathlineto{\pgfqpoint{0.438698in}{1.046203in}}%
\pgfpathlineto{\pgfqpoint{0.442884in}{1.053148in}}%
\pgfpathlineto{\pgfqpoint{0.454354in}{1.066116in}}%
\pgfpathlineto{\pgfqpoint{0.455022in}{1.066759in}}%
\pgfpathlineto{\pgfqpoint{0.470011in}{1.078415in}}%
\pgfpathlineto{\pgfqpoint{0.473426in}{1.080370in}}%
\pgfpathlineto{\pgfqpoint{0.485668in}{1.086795in}}%
\pgfpathlineto{\pgfqpoint{0.501324in}{1.091628in}}%
\pgfpathlineto{\pgfqpoint{0.516981in}{1.093007in}}%
\pgfpathlineto{\pgfqpoint{0.532637in}{1.090939in}}%
\pgfpathlineto{\pgfqpoint{0.548294in}{1.085412in}}%
\pgfpathlineto{\pgfqpoint{0.557257in}{1.080370in}}%
\pgfpathlineto{\pgfqpoint{0.563950in}{1.076251in}}%
\pgfpathlineto{\pgfqpoint{0.575662in}{1.066759in}}%
\pgfpathlineto{\pgfqpoint{0.579607in}{1.062789in}}%
\pgfpathlineto{\pgfqpoint{0.587953in}{1.053147in}}%
\pgfpathlineto{\pgfqpoint{0.595263in}{1.040739in}}%
\pgfpathlineto{\pgfqpoint{0.595959in}{1.039536in}}%
\pgfpathlineto{\pgfqpoint{0.600016in}{1.025925in}}%
\pgfpathlineto{\pgfqpoint{0.600016in}{1.012314in}}%
\pgfpathlineto{\pgfqpoint{0.595959in}{0.998703in}}%
\pgfpathlineto{\pgfqpoint{0.595263in}{0.997501in}}%
\pgfpathlineto{\pgfqpoint{0.587953in}{0.985092in}}%
\pgfpathlineto{\pgfqpoint{0.579607in}{0.975451in}}%
\pgfpathlineto{\pgfqpoint{0.575662in}{0.971481in}}%
\pgfpathlineto{\pgfqpoint{0.563950in}{0.961988in}}%
\pgfpathlineto{\pgfqpoint{0.557257in}{0.957870in}}%
\pgfpathlineto{\pgfqpoint{0.548294in}{0.952827in}}%
\pgfpathlineto{\pgfqpoint{0.532637in}{0.947301in}}%
\pgfpathlineto{\pgfqpoint{0.516981in}{0.945233in}}%
\pgfpathlineto{\pgfqpoint{0.501324in}{0.946611in}}%
\pgfpathlineto{\pgfqpoint{0.485668in}{0.951445in}}%
\pgfpathlineto{\pgfqpoint{0.473426in}{0.957870in}}%
\pgfpathclose%
\pgfpathmoveto{\pgfqpoint{0.814455in}{0.927635in}}%
\pgfpathlineto{\pgfqpoint{0.830112in}{0.926845in}}%
\pgfpathlineto{\pgfqpoint{0.845769in}{0.930006in}}%
\pgfpathlineto{\pgfqpoint{0.847219in}{0.930648in}}%
\pgfpathlineto{\pgfqpoint{0.861425in}{0.936101in}}%
\pgfpathlineto{\pgfqpoint{0.875219in}{0.944259in}}%
\pgfpathlineto{\pgfqpoint{0.877082in}{0.945349in}}%
\pgfpathlineto{\pgfqpoint{0.892738in}{0.957131in}}%
\pgfpathlineto{\pgfqpoint{0.893600in}{0.957870in}}%
\pgfpathlineto{\pgfqpoint{0.907805in}{0.971481in}}%
\pgfpathlineto{\pgfqpoint{0.908395in}{0.972187in}}%
\pgfpathlineto{\pgfqpoint{0.919020in}{0.985092in}}%
\pgfpathlineto{\pgfqpoint{0.924051in}{0.994104in}}%
\pgfpathlineto{\pgfqpoint{0.926924in}{0.998703in}}%
\pgfpathlineto{\pgfqpoint{0.931269in}{1.012314in}}%
\pgfpathlineto{\pgfqpoint{0.931269in}{1.025925in}}%
\pgfpathlineto{\pgfqpoint{0.926924in}{1.039536in}}%
\pgfpathlineto{\pgfqpoint{0.924051in}{1.044135in}}%
\pgfpathlineto{\pgfqpoint{0.919020in}{1.053148in}}%
\pgfpathlineto{\pgfqpoint{0.908395in}{1.066052in}}%
\pgfpathlineto{\pgfqpoint{0.907805in}{1.066759in}}%
\pgfpathlineto{\pgfqpoint{0.893600in}{1.080370in}}%
\pgfpathlineto{\pgfqpoint{0.892738in}{1.081108in}}%
\pgfpathlineto{\pgfqpoint{0.877082in}{1.092890in}}%
\pgfpathlineto{\pgfqpoint{0.875219in}{1.093981in}}%
\pgfpathlineto{\pgfqpoint{0.861425in}{1.102138in}}%
\pgfpathlineto{\pgfqpoint{0.847219in}{1.107592in}}%
\pgfpathlineto{\pgfqpoint{0.845769in}{1.108233in}}%
\pgfpathlineto{\pgfqpoint{0.830112in}{1.111394in}}%
\pgfpathlineto{\pgfqpoint{0.814455in}{1.110605in}}%
\pgfpathlineto{\pgfqpoint{0.804437in}{1.107592in}}%
\pgfpathlineto{\pgfqpoint{0.798799in}{1.106135in}}%
\pgfpathlineto{\pgfqpoint{0.783142in}{1.098808in}}%
\pgfpathlineto{\pgfqpoint{0.775812in}{1.093981in}}%
\pgfpathlineto{\pgfqpoint{0.767486in}{1.088521in}}%
\pgfpathlineto{\pgfqpoint{0.757343in}{1.080370in}}%
\pgfpathlineto{\pgfqpoint{0.751829in}{1.075426in}}%
\pgfpathlineto{\pgfqpoint{0.742969in}{1.066759in}}%
\pgfpathlineto{\pgfqpoint{0.736173in}{1.058399in}}%
\pgfpathlineto{\pgfqpoint{0.731780in}{1.053148in}}%
\pgfpathlineto{\pgfqpoint{0.723928in}{1.039536in}}%
\pgfpathlineto{\pgfqpoint{0.720516in}{1.027711in}}%
\pgfpathlineto{\pgfqpoint{0.719898in}{1.025925in}}%
\pgfpathlineto{\pgfqpoint{0.719898in}{1.012314in}}%
\pgfpathlineto{\pgfqpoint{0.720516in}{1.010529in}}%
\pgfpathlineto{\pgfqpoint{0.723928in}{0.998703in}}%
\pgfpathlineto{\pgfqpoint{0.731780in}{0.985092in}}%
\pgfpathlineto{\pgfqpoint{0.736173in}{0.979840in}}%
\pgfpathlineto{\pgfqpoint{0.742969in}{0.971481in}}%
\pgfpathlineto{\pgfqpoint{0.751829in}{0.962814in}}%
\pgfpathlineto{\pgfqpoint{0.757343in}{0.957870in}}%
\pgfpathlineto{\pgfqpoint{0.767486in}{0.949718in}}%
\pgfpathlineto{\pgfqpoint{0.775812in}{0.944259in}}%
\pgfpathlineto{\pgfqpoint{0.783142in}{0.939432in}}%
\pgfpathlineto{\pgfqpoint{0.798799in}{0.932105in}}%
\pgfpathlineto{\pgfqpoint{0.804437in}{0.930648in}}%
\pgfpathlineto{\pgfqpoint{0.814455in}{0.927635in}}%
\pgfpathclose%
\pgfpathmoveto{\pgfqpoint{0.783045in}{0.957870in}}%
\pgfpathlineto{\pgfqpoint{0.767486in}{0.969317in}}%
\pgfpathlineto{\pgfqpoint{0.765068in}{0.971481in}}%
\pgfpathlineto{\pgfqpoint{0.752842in}{0.985092in}}%
\pgfpathlineto{\pgfqpoint{0.751829in}{0.986727in}}%
\pgfpathlineto{\pgfqpoint{0.744840in}{0.998703in}}%
\pgfpathlineto{\pgfqpoint{0.740859in}{1.012314in}}%
\pgfpathlineto{\pgfqpoint{0.740859in}{1.025925in}}%
\pgfpathlineto{\pgfqpoint{0.744840in}{1.039536in}}%
\pgfpathlineto{\pgfqpoint{0.751829in}{1.051513in}}%
\pgfpathlineto{\pgfqpoint{0.752842in}{1.053148in}}%
\pgfpathlineto{\pgfqpoint{0.765068in}{1.066759in}}%
\pgfpathlineto{\pgfqpoint{0.767486in}{1.068923in}}%
\pgfpathlineto{\pgfqpoint{0.783045in}{1.080370in}}%
\pgfpathlineto{\pgfqpoint{0.783142in}{1.080434in}}%
\pgfpathlineto{\pgfqpoint{0.798799in}{1.088039in}}%
\pgfpathlineto{\pgfqpoint{0.814455in}{1.092180in}}%
\pgfpathlineto{\pgfqpoint{0.830112in}{1.092869in}}%
\pgfpathlineto{\pgfqpoint{0.845769in}{1.090111in}}%
\pgfpathlineto{\pgfqpoint{0.861425in}{1.083891in}}%
\pgfpathlineto{\pgfqpoint{0.867305in}{1.080370in}}%
\pgfpathlineto{\pgfqpoint{0.877082in}{1.073946in}}%
\pgfpathlineto{\pgfqpoint{0.885627in}{1.066759in}}%
\pgfpathlineto{\pgfqpoint{0.892738in}{1.059315in}}%
\pgfpathlineto{\pgfqpoint{0.897983in}{1.053148in}}%
\pgfpathlineto{\pgfqpoint{0.905940in}{1.039536in}}%
\pgfpathlineto{\pgfqpoint{0.908395in}{1.031179in}}%
\pgfpathlineto{\pgfqpoint{0.909968in}{1.025925in}}%
\pgfpathlineto{\pgfqpoint{0.909968in}{1.012314in}}%
\pgfpathlineto{\pgfqpoint{0.908395in}{1.007061in}}%
\pgfpathlineto{\pgfqpoint{0.905940in}{0.998703in}}%
\pgfpathlineto{\pgfqpoint{0.897983in}{0.985092in}}%
\pgfpathlineto{\pgfqpoint{0.892738in}{0.978924in}}%
\pgfpathlineto{\pgfqpoint{0.885627in}{0.971481in}}%
\pgfpathlineto{\pgfqpoint{0.877082in}{0.964293in}}%
\pgfpathlineto{\pgfqpoint{0.867305in}{0.957870in}}%
\pgfpathlineto{\pgfqpoint{0.861425in}{0.954348in}}%
\pgfpathlineto{\pgfqpoint{0.845769in}{0.948129in}}%
\pgfpathlineto{\pgfqpoint{0.830112in}{0.945370in}}%
\pgfpathlineto{\pgfqpoint{0.814455in}{0.946059in}}%
\pgfpathlineto{\pgfqpoint{0.798799in}{0.950201in}}%
\pgfpathlineto{\pgfqpoint{0.783142in}{0.957805in}}%
\pgfpathlineto{\pgfqpoint{0.783045in}{0.957870in}}%
\pgfpathclose%
\pgfpathmoveto{\pgfqpoint{1.127587in}{0.927161in}}%
\pgfpathlineto{\pgfqpoint{1.143243in}{0.927161in}}%
\pgfpathlineto{\pgfqpoint{1.157082in}{0.930648in}}%
\pgfpathlineto{\pgfqpoint{1.158900in}{0.931040in}}%
\pgfpathlineto{\pgfqpoint{1.174556in}{0.937700in}}%
\pgfpathlineto{\pgfqpoint{1.185037in}{0.944259in}}%
\pgfpathlineto{\pgfqpoint{1.190213in}{0.947475in}}%
\pgfpathlineto{\pgfqpoint{1.203529in}{0.957870in}}%
\pgfpathlineto{\pgfqpoint{1.205870in}{0.959904in}}%
\pgfpathlineto{\pgfqpoint{1.217827in}{0.971481in}}%
\pgfpathlineto{\pgfqpoint{1.221526in}{0.975980in}}%
\pgfpathlineto{\pgfqpoint{1.229070in}{0.985092in}}%
\pgfpathlineto{\pgfqpoint{1.236732in}{0.998703in}}%
\pgfpathlineto{\pgfqpoint{1.237183in}{1.000284in}}%
\pgfpathlineto{\pgfqpoint{1.241193in}{1.012314in}}%
\pgfpathlineto{\pgfqpoint{1.241193in}{1.025925in}}%
\pgfpathlineto{\pgfqpoint{1.237183in}{1.037956in}}%
\pgfpathlineto{\pgfqpoint{1.236732in}{1.039536in}}%
\pgfpathlineto{\pgfqpoint{1.229070in}{1.053148in}}%
\pgfpathlineto{\pgfqpoint{1.221526in}{1.062259in}}%
\pgfpathlineto{\pgfqpoint{1.217827in}{1.066759in}}%
\pgfpathlineto{\pgfqpoint{1.205870in}{1.078335in}}%
\pgfpathlineto{\pgfqpoint{1.203529in}{1.080370in}}%
\pgfpathlineto{\pgfqpoint{1.190213in}{1.090765in}}%
\pgfpathlineto{\pgfqpoint{1.185037in}{1.093981in}}%
\pgfpathlineto{\pgfqpoint{1.174556in}{1.100540in}}%
\pgfpathlineto{\pgfqpoint{1.158900in}{1.107200in}}%
\pgfpathlineto{\pgfqpoint{1.157082in}{1.107592in}}%
\pgfpathlineto{\pgfqpoint{1.143243in}{1.111078in}}%
\pgfpathlineto{\pgfqpoint{1.127587in}{1.111078in}}%
\pgfpathlineto{\pgfqpoint{1.113748in}{1.107592in}}%
\pgfpathlineto{\pgfqpoint{1.111930in}{1.107200in}}%
\pgfpathlineto{\pgfqpoint{1.096274in}{1.100540in}}%
\pgfpathlineto{\pgfqpoint{1.085793in}{1.093981in}}%
\pgfpathlineto{\pgfqpoint{1.080617in}{1.090765in}}%
\pgfpathlineto{\pgfqpoint{1.067301in}{1.080370in}}%
\pgfpathlineto{\pgfqpoint{1.064960in}{1.078335in}}%
\pgfpathlineto{\pgfqpoint{1.053003in}{1.066759in}}%
\pgfpathlineto{\pgfqpoint{1.049304in}{1.062259in}}%
\pgfpathlineto{\pgfqpoint{1.041760in}{1.053147in}}%
\pgfpathlineto{\pgfqpoint{1.034098in}{1.039536in}}%
\pgfpathlineto{\pgfqpoint{1.033647in}{1.037956in}}%
\pgfpathlineto{\pgfqpoint{1.029637in}{1.025925in}}%
\pgfpathlineto{\pgfqpoint{1.029637in}{1.012314in}}%
\pgfpathlineto{\pgfqpoint{1.033647in}{1.000284in}}%
\pgfpathlineto{\pgfqpoint{1.034098in}{0.998703in}}%
\pgfpathlineto{\pgfqpoint{1.041760in}{0.985092in}}%
\pgfpathlineto{\pgfqpoint{1.049304in}{0.975980in}}%
\pgfpathlineto{\pgfqpoint{1.053003in}{0.971481in}}%
\pgfpathlineto{\pgfqpoint{1.064960in}{0.959904in}}%
\pgfpathlineto{\pgfqpoint{1.067301in}{0.957870in}}%
\pgfpathlineto{\pgfqpoint{1.080617in}{0.947475in}}%
\pgfpathlineto{\pgfqpoint{1.085793in}{0.944259in}}%
\pgfpathlineto{\pgfqpoint{1.096274in}{0.937700in}}%
\pgfpathlineto{\pgfqpoint{1.111930in}{0.931040in}}%
\pgfpathlineto{\pgfqpoint{1.113748in}{0.930648in}}%
\pgfpathlineto{\pgfqpoint{1.127587in}{0.927161in}}%
\pgfpathclose%
\pgfpathmoveto{\pgfqpoint{1.093336in}{0.957870in}}%
\pgfpathlineto{\pgfqpoint{1.080617in}{0.966737in}}%
\pgfpathlineto{\pgfqpoint{1.075160in}{0.971481in}}%
\pgfpathlineto{\pgfqpoint{1.064960in}{0.982538in}}%
\pgfpathlineto{\pgfqpoint{1.062819in}{0.985092in}}%
\pgfpathlineto{\pgfqpoint{1.054868in}{0.998703in}}%
\pgfpathlineto{\pgfqpoint{1.050900in}{1.012314in}}%
\pgfpathlineto{\pgfqpoint{1.050900in}{1.025925in}}%
\pgfpathlineto{\pgfqpoint{1.054868in}{1.039536in}}%
\pgfpathlineto{\pgfqpoint{1.062819in}{1.053148in}}%
\pgfpathlineto{\pgfqpoint{1.064960in}{1.055702in}}%
\pgfpathlineto{\pgfqpoint{1.075160in}{1.066759in}}%
\pgfpathlineto{\pgfqpoint{1.080617in}{1.071503in}}%
\pgfpathlineto{\pgfqpoint{1.093336in}{1.080370in}}%
\pgfpathlineto{\pgfqpoint{1.096274in}{1.082232in}}%
\pgfpathlineto{\pgfqpoint{1.111930in}{1.089144in}}%
\pgfpathlineto{\pgfqpoint{1.127587in}{1.092593in}}%
\pgfpathlineto{\pgfqpoint{1.143243in}{1.092593in}}%
\pgfpathlineto{\pgfqpoint{1.158900in}{1.089144in}}%
\pgfpathlineto{\pgfqpoint{1.174556in}{1.082232in}}%
\pgfpathlineto{\pgfqpoint{1.177494in}{1.080370in}}%
\pgfpathlineto{\pgfqpoint{1.190213in}{1.071503in}}%
\pgfpathlineto{\pgfqpoint{1.195670in}{1.066759in}}%
\pgfpathlineto{\pgfqpoint{1.205870in}{1.055702in}}%
\pgfpathlineto{\pgfqpoint{1.208011in}{1.053148in}}%
\pgfpathlineto{\pgfqpoint{1.215962in}{1.039536in}}%
\pgfpathlineto{\pgfqpoint{1.219930in}{1.025925in}}%
\pgfpathlineto{\pgfqpoint{1.219930in}{1.012314in}}%
\pgfpathlineto{\pgfqpoint{1.215962in}{0.998703in}}%
\pgfpathlineto{\pgfqpoint{1.208011in}{0.985092in}}%
\pgfpathlineto{\pgfqpoint{1.205870in}{0.982538in}}%
\pgfpathlineto{\pgfqpoint{1.195670in}{0.971481in}}%
\pgfpathlineto{\pgfqpoint{1.190213in}{0.966737in}}%
\pgfpathlineto{\pgfqpoint{1.177494in}{0.957870in}}%
\pgfpathlineto{\pgfqpoint{1.174556in}{0.956008in}}%
\pgfpathlineto{\pgfqpoint{1.158900in}{0.949095in}}%
\pgfpathlineto{\pgfqpoint{1.143243in}{0.945646in}}%
\pgfpathlineto{\pgfqpoint{1.127587in}{0.945646in}}%
\pgfpathlineto{\pgfqpoint{1.111930in}{0.949095in}}%
\pgfpathlineto{\pgfqpoint{1.096274in}{0.956008in}}%
\pgfpathlineto{\pgfqpoint{1.093336in}{0.957870in}}%
\pgfpathclose%
\pgfpathmoveto{\pgfqpoint{1.425061in}{0.930006in}}%
\pgfpathlineto{\pgfqpoint{1.440718in}{0.926845in}}%
\pgfpathlineto{\pgfqpoint{1.456375in}{0.927635in}}%
\pgfpathlineto{\pgfqpoint{1.466393in}{0.930648in}}%
\pgfpathlineto{\pgfqpoint{1.472031in}{0.932105in}}%
\pgfpathlineto{\pgfqpoint{1.487688in}{0.939432in}}%
\pgfpathlineto{\pgfqpoint{1.495018in}{0.944259in}}%
\pgfpathlineto{\pgfqpoint{1.503344in}{0.949718in}}%
\pgfpathlineto{\pgfqpoint{1.513487in}{0.957870in}}%
\pgfpathlineto{\pgfqpoint{1.519001in}{0.962814in}}%
\pgfpathlineto{\pgfqpoint{1.527861in}{0.971481in}}%
\pgfpathlineto{\pgfqpoint{1.534657in}{0.979840in}}%
\pgfpathlineto{\pgfqpoint{1.539050in}{0.985092in}}%
\pgfpathlineto{\pgfqpoint{1.546902in}{0.998703in}}%
\pgfpathlineto{\pgfqpoint{1.550314in}{1.010529in}}%
\pgfpathlineto{\pgfqpoint{1.550932in}{1.012314in}}%
\pgfpathlineto{\pgfqpoint{1.550932in}{1.025925in}}%
\pgfpathlineto{\pgfqpoint{1.550314in}{1.027711in}}%
\pgfpathlineto{\pgfqpoint{1.546902in}{1.039536in}}%
\pgfpathlineto{\pgfqpoint{1.539050in}{1.053148in}}%
\pgfpathlineto{\pgfqpoint{1.534657in}{1.058399in}}%
\pgfpathlineto{\pgfqpoint{1.527861in}{1.066759in}}%
\pgfpathlineto{\pgfqpoint{1.519001in}{1.075426in}}%
\pgfpathlineto{\pgfqpoint{1.513487in}{1.080370in}}%
\pgfpathlineto{\pgfqpoint{1.503344in}{1.088521in}}%
\pgfpathlineto{\pgfqpoint{1.495018in}{1.093981in}}%
\pgfpathlineto{\pgfqpoint{1.487688in}{1.098808in}}%
\pgfpathlineto{\pgfqpoint{1.472031in}{1.106135in}}%
\pgfpathlineto{\pgfqpoint{1.466393in}{1.107592in}}%
\pgfpathlineto{\pgfqpoint{1.456375in}{1.110605in}}%
\pgfpathlineto{\pgfqpoint{1.440718in}{1.111394in}}%
\pgfpathlineto{\pgfqpoint{1.425061in}{1.108233in}}%
\pgfpathlineto{\pgfqpoint{1.423611in}{1.107592in}}%
\pgfpathlineto{\pgfqpoint{1.409405in}{1.102138in}}%
\pgfpathlineto{\pgfqpoint{1.395611in}{1.093981in}}%
\pgfpathlineto{\pgfqpoint{1.393748in}{1.092890in}}%
\pgfpathlineto{\pgfqpoint{1.378092in}{1.081108in}}%
\pgfpathlineto{\pgfqpoint{1.377230in}{1.080370in}}%
\pgfpathlineto{\pgfqpoint{1.363025in}{1.066759in}}%
\pgfpathlineto{\pgfqpoint{1.362435in}{1.066052in}}%
\pgfpathlineto{\pgfqpoint{1.351810in}{1.053148in}}%
\pgfpathlineto{\pgfqpoint{1.346779in}{1.044135in}}%
\pgfpathlineto{\pgfqpoint{1.343906in}{1.039536in}}%
\pgfpathlineto{\pgfqpoint{1.339561in}{1.025925in}}%
\pgfpathlineto{\pgfqpoint{1.339561in}{1.012314in}}%
\pgfpathlineto{\pgfqpoint{1.343906in}{0.998703in}}%
\pgfpathlineto{\pgfqpoint{1.346779in}{0.994104in}}%
\pgfpathlineto{\pgfqpoint{1.351810in}{0.985092in}}%
\pgfpathlineto{\pgfqpoint{1.362435in}{0.972187in}}%
\pgfpathlineto{\pgfqpoint{1.363025in}{0.971481in}}%
\pgfpathlineto{\pgfqpoint{1.377230in}{0.957870in}}%
\pgfpathlineto{\pgfqpoint{1.378092in}{0.957131in}}%
\pgfpathlineto{\pgfqpoint{1.393748in}{0.945349in}}%
\pgfpathlineto{\pgfqpoint{1.395611in}{0.944259in}}%
\pgfpathlineto{\pgfqpoint{1.409405in}{0.936101in}}%
\pgfpathlineto{\pgfqpoint{1.423611in}{0.930648in}}%
\pgfpathlineto{\pgfqpoint{1.425061in}{0.930006in}}%
\pgfpathclose%
\pgfpathmoveto{\pgfqpoint{1.403525in}{0.957870in}}%
\pgfpathlineto{\pgfqpoint{1.393748in}{0.964293in}}%
\pgfpathlineto{\pgfqpoint{1.385203in}{0.971481in}}%
\pgfpathlineto{\pgfqpoint{1.378092in}{0.978924in}}%
\pgfpathlineto{\pgfqpoint{1.372847in}{0.985092in}}%
\pgfpathlineto{\pgfqpoint{1.364890in}{0.998703in}}%
\pgfpathlineto{\pgfqpoint{1.362435in}{1.007061in}}%
\pgfpathlineto{\pgfqpoint{1.360862in}{1.012314in}}%
\pgfpathlineto{\pgfqpoint{1.360862in}{1.025925in}}%
\pgfpathlineto{\pgfqpoint{1.362435in}{1.031179in}}%
\pgfpathlineto{\pgfqpoint{1.364890in}{1.039536in}}%
\pgfpathlineto{\pgfqpoint{1.372847in}{1.053148in}}%
\pgfpathlineto{\pgfqpoint{1.378092in}{1.059315in}}%
\pgfpathlineto{\pgfqpoint{1.385203in}{1.066759in}}%
\pgfpathlineto{\pgfqpoint{1.393748in}{1.073946in}}%
\pgfpathlineto{\pgfqpoint{1.403525in}{1.080370in}}%
\pgfpathlineto{\pgfqpoint{1.409405in}{1.083891in}}%
\pgfpathlineto{\pgfqpoint{1.425061in}{1.090111in}}%
\pgfpathlineto{\pgfqpoint{1.440718in}{1.092869in}}%
\pgfpathlineto{\pgfqpoint{1.456375in}{1.092180in}}%
\pgfpathlineto{\pgfqpoint{1.472031in}{1.088039in}}%
\pgfpathlineto{\pgfqpoint{1.487688in}{1.080434in}}%
\pgfpathlineto{\pgfqpoint{1.487785in}{1.080370in}}%
\pgfpathlineto{\pgfqpoint{1.503344in}{1.068923in}}%
\pgfpathlineto{\pgfqpoint{1.505762in}{1.066759in}}%
\pgfpathlineto{\pgfqpoint{1.517988in}{1.053147in}}%
\pgfpathlineto{\pgfqpoint{1.519001in}{1.051513in}}%
\pgfpathlineto{\pgfqpoint{1.525990in}{1.039536in}}%
\pgfpathlineto{\pgfqpoint{1.529971in}{1.025925in}}%
\pgfpathlineto{\pgfqpoint{1.529971in}{1.012314in}}%
\pgfpathlineto{\pgfqpoint{1.525990in}{0.998703in}}%
\pgfpathlineto{\pgfqpoint{1.519001in}{0.986727in}}%
\pgfpathlineto{\pgfqpoint{1.517988in}{0.985092in}}%
\pgfpathlineto{\pgfqpoint{1.505762in}{0.971481in}}%
\pgfpathlineto{\pgfqpoint{1.503344in}{0.969317in}}%
\pgfpathlineto{\pgfqpoint{1.487785in}{0.957870in}}%
\pgfpathlineto{\pgfqpoint{1.487688in}{0.957805in}}%
\pgfpathlineto{\pgfqpoint{1.472031in}{0.950201in}}%
\pgfpathlineto{\pgfqpoint{1.456375in}{0.946059in}}%
\pgfpathlineto{\pgfqpoint{1.440718in}{0.945370in}}%
\pgfpathlineto{\pgfqpoint{1.425061in}{0.948129in}}%
\pgfpathlineto{\pgfqpoint{1.409405in}{0.954348in}}%
\pgfpathlineto{\pgfqpoint{1.403525in}{0.957870in}}%
\pgfpathclose%
\pgfpathmoveto{\pgfqpoint{1.738193in}{0.929057in}}%
\pgfpathlineto{\pgfqpoint{1.753849in}{0.926687in}}%
\pgfpathlineto{\pgfqpoint{1.769506in}{0.928267in}}%
\pgfpathlineto{\pgfqpoint{1.776332in}{0.930648in}}%
\pgfpathlineto{\pgfqpoint{1.785162in}{0.933303in}}%
\pgfpathlineto{\pgfqpoint{1.800819in}{0.941295in}}%
\pgfpathlineto{\pgfqpoint{1.805116in}{0.944259in}}%
\pgfpathlineto{\pgfqpoint{1.816476in}{0.952078in}}%
\pgfpathlineto{\pgfqpoint{1.823503in}{0.957870in}}%
\pgfpathlineto{\pgfqpoint{1.832132in}{0.965819in}}%
\pgfpathlineto{\pgfqpoint{1.837882in}{0.971481in}}%
\pgfpathlineto{\pgfqpoint{1.847789in}{0.983752in}}%
\pgfpathlineto{\pgfqpoint{1.848927in}{0.985092in}}%
\pgfpathlineto{\pgfqpoint{1.857012in}{0.998703in}}%
\pgfpathlineto{\pgfqpoint{1.861047in}{1.012314in}}%
\pgfpathlineto{\pgfqpoint{1.861047in}{1.025925in}}%
\pgfpathlineto{\pgfqpoint{1.857012in}{1.039536in}}%
\pgfpathlineto{\pgfqpoint{1.848927in}{1.053148in}}%
\pgfpathlineto{\pgfqpoint{1.847789in}{1.054488in}}%
\pgfpathlineto{\pgfqpoint{1.837882in}{1.066759in}}%
\pgfpathlineto{\pgfqpoint{1.832132in}{1.072421in}}%
\pgfpathlineto{\pgfqpoint{1.823503in}{1.080370in}}%
\pgfpathlineto{\pgfqpoint{1.816476in}{1.086161in}}%
\pgfpathlineto{\pgfqpoint{1.805116in}{1.093981in}}%
\pgfpathlineto{\pgfqpoint{1.800819in}{1.096944in}}%
\pgfpathlineto{\pgfqpoint{1.785162in}{1.104936in}}%
\pgfpathlineto{\pgfqpoint{1.776332in}{1.107592in}}%
\pgfpathlineto{\pgfqpoint{1.769506in}{1.109973in}}%
\pgfpathlineto{\pgfqpoint{1.753849in}{1.111552in}}%
\pgfpathlineto{\pgfqpoint{1.738193in}{1.109182in}}%
\pgfpathlineto{\pgfqpoint{1.734176in}{1.107592in}}%
\pgfpathlineto{\pgfqpoint{1.722536in}{1.103604in}}%
\pgfpathlineto{\pgfqpoint{1.706880in}{1.094949in}}%
\pgfpathlineto{\pgfqpoint{1.705532in}{1.093981in}}%
\pgfpathlineto{\pgfqpoint{1.691223in}{1.083689in}}%
\pgfpathlineto{\pgfqpoint{1.687281in}{1.080370in}}%
\pgfpathlineto{\pgfqpoint{1.675567in}{1.069327in}}%
\pgfpathlineto{\pgfqpoint{1.672965in}{1.066759in}}%
\pgfpathlineto{\pgfqpoint{1.661903in}{1.053148in}}%
\pgfpathlineto{\pgfqpoint{1.659910in}{1.049572in}}%
\pgfpathlineto{\pgfqpoint{1.653799in}{1.039536in}}%
\pgfpathlineto{\pgfqpoint{1.649623in}{1.025925in}}%
\pgfpathlineto{\pgfqpoint{1.649623in}{1.012314in}}%
\pgfpathlineto{\pgfqpoint{1.653799in}{0.998703in}}%
\pgfpathlineto{\pgfqpoint{1.659910in}{0.988667in}}%
\pgfpathlineto{\pgfqpoint{1.661903in}{0.985092in}}%
\pgfpathlineto{\pgfqpoint{1.672965in}{0.971481in}}%
\pgfpathlineto{\pgfqpoint{1.675567in}{0.968912in}}%
\pgfpathlineto{\pgfqpoint{1.687281in}{0.957870in}}%
\pgfpathlineto{\pgfqpoint{1.691223in}{0.954550in}}%
\pgfpathlineto{\pgfqpoint{1.705532in}{0.944259in}}%
\pgfpathlineto{\pgfqpoint{1.706880in}{0.943290in}}%
\pgfpathlineto{\pgfqpoint{1.722536in}{0.934635in}}%
\pgfpathlineto{\pgfqpoint{1.734176in}{0.930648in}}%
\pgfpathlineto{\pgfqpoint{1.738193in}{0.929057in}}%
\pgfpathclose%
\pgfpathmoveto{\pgfqpoint{1.713573in}{0.957870in}}%
\pgfpathlineto{\pgfqpoint{1.706880in}{0.961988in}}%
\pgfpathlineto{\pgfqpoint{1.695168in}{0.971481in}}%
\pgfpathlineto{\pgfqpoint{1.691223in}{0.975451in}}%
\pgfpathlineto{\pgfqpoint{1.682877in}{0.985092in}}%
\pgfpathlineto{\pgfqpoint{1.675567in}{0.997501in}}%
\pgfpathlineto{\pgfqpoint{1.674871in}{0.998703in}}%
\pgfpathlineto{\pgfqpoint{1.670814in}{1.012314in}}%
\pgfpathlineto{\pgfqpoint{1.670814in}{1.025925in}}%
\pgfpathlineto{\pgfqpoint{1.674871in}{1.039536in}}%
\pgfpathlineto{\pgfqpoint{1.675567in}{1.040739in}}%
\pgfpathlineto{\pgfqpoint{1.682877in}{1.053148in}}%
\pgfpathlineto{\pgfqpoint{1.691223in}{1.062789in}}%
\pgfpathlineto{\pgfqpoint{1.695168in}{1.066759in}}%
\pgfpathlineto{\pgfqpoint{1.706880in}{1.076251in}}%
\pgfpathlineto{\pgfqpoint{1.713573in}{1.080370in}}%
\pgfpathlineto{\pgfqpoint{1.722536in}{1.085412in}}%
\pgfpathlineto{\pgfqpoint{1.738193in}{1.090939in}}%
\pgfpathlineto{\pgfqpoint{1.753849in}{1.093007in}}%
\pgfpathlineto{\pgfqpoint{1.769506in}{1.091628in}}%
\pgfpathlineto{\pgfqpoint{1.785162in}{1.086795in}}%
\pgfpathlineto{\pgfqpoint{1.797404in}{1.080370in}}%
\pgfpathlineto{\pgfqpoint{1.800819in}{1.078415in}}%
\pgfpathlineto{\pgfqpoint{1.815808in}{1.066759in}}%
\pgfpathlineto{\pgfqpoint{1.816476in}{1.066116in}}%
\pgfpathlineto{\pgfqpoint{1.827946in}{1.053148in}}%
\pgfpathlineto{\pgfqpoint{1.832132in}{1.046203in}}%
\pgfpathlineto{\pgfqpoint{1.835997in}{1.039536in}}%
\pgfpathlineto{\pgfqpoint{1.840008in}{1.025925in}}%
\pgfpathlineto{\pgfqpoint{1.840008in}{1.012314in}}%
\pgfpathlineto{\pgfqpoint{1.835997in}{0.998703in}}%
\pgfpathlineto{\pgfqpoint{1.832132in}{0.992036in}}%
\pgfpathlineto{\pgfqpoint{1.827946in}{0.985092in}}%
\pgfpathlineto{\pgfqpoint{1.816476in}{0.972124in}}%
\pgfpathlineto{\pgfqpoint{1.815808in}{0.971481in}}%
\pgfpathlineto{\pgfqpoint{1.800819in}{0.959824in}}%
\pgfpathlineto{\pgfqpoint{1.797404in}{0.957870in}}%
\pgfpathlineto{\pgfqpoint{1.785162in}{0.951445in}}%
\pgfpathlineto{\pgfqpoint{1.769506in}{0.946611in}}%
\pgfpathlineto{\pgfqpoint{1.753849in}{0.945233in}}%
\pgfpathlineto{\pgfqpoint{1.738193in}{0.947301in}}%
\pgfpathlineto{\pgfqpoint{1.722536in}{0.952827in}}%
\pgfpathlineto{\pgfqpoint{1.713573in}{0.957870in}}%
\pgfpathclose%
\pgfpathmoveto{\pgfqpoint{0.501324in}{1.197652in}}%
\pgfpathlineto{\pgfqpoint{0.516981in}{1.196143in}}%
\pgfpathlineto{\pgfqpoint{0.532637in}{1.198407in}}%
\pgfpathlineto{\pgfqpoint{0.544270in}{1.202870in}}%
\pgfpathlineto{\pgfqpoint{0.548294in}{1.204240in}}%
\pgfpathlineto{\pgfqpoint{0.563950in}{1.212723in}}%
\pgfpathlineto{\pgfqpoint{0.569209in}{1.216481in}}%
\pgfpathlineto{\pgfqpoint{0.579607in}{1.224074in}}%
\pgfpathlineto{\pgfqpoint{0.586654in}{1.230092in}}%
\pgfpathlineto{\pgfqpoint{0.595263in}{1.238497in}}%
\pgfpathlineto{\pgfqpoint{0.600365in}{1.243703in}}%
\pgfpathlineto{\pgfqpoint{0.610701in}{1.257314in}}%
\pgfpathlineto{\pgfqpoint{0.610920in}{1.257748in}}%
\pgfpathlineto{\pgfqpoint{0.618201in}{1.270925in}}%
\pgfpathlineto{\pgfqpoint{0.621541in}{1.284536in}}%
\pgfpathlineto{\pgfqpoint{0.620707in}{1.298147in}}%
\pgfpathlineto{\pgfqpoint{0.615693in}{1.311759in}}%
\pgfpathlineto{\pgfqpoint{0.610920in}{1.318949in}}%
\pgfpathlineto{\pgfqpoint{0.607005in}{1.325370in}}%
\pgfpathlineto{\pgfqpoint{0.595263in}{1.338940in}}%
\pgfpathlineto{\pgfqpoint{0.595227in}{1.338981in}}%
\pgfpathlineto{\pgfqpoint{0.580330in}{1.352592in}}%
\pgfpathlineto{\pgfqpoint{0.579607in}{1.353195in}}%
\pgfpathlineto{\pgfqpoint{0.563950in}{1.364438in}}%
\pgfpathlineto{\pgfqpoint{0.560696in}{1.366203in}}%
\pgfpathlineto{\pgfqpoint{0.548294in}{1.373162in}}%
\pgfpathlineto{\pgfqpoint{0.532637in}{1.378620in}}%
\pgfpathlineto{\pgfqpoint{0.523523in}{1.379814in}}%
\pgfpathlineto{\pgfqpoint{0.516981in}{1.380851in}}%
\pgfpathlineto{\pgfqpoint{0.507200in}{1.379814in}}%
\pgfpathlineto{\pgfqpoint{0.501324in}{1.379301in}}%
\pgfpathlineto{\pgfqpoint{0.485668in}{1.374528in}}%
\pgfpathlineto{\pgfqpoint{0.470011in}{1.366337in}}%
\pgfpathlineto{\pgfqpoint{0.469819in}{1.366203in}}%
\pgfpathlineto{\pgfqpoint{0.454354in}{1.355675in}}%
\pgfpathlineto{\pgfqpoint{0.450574in}{1.352592in}}%
\pgfpathlineto{\pgfqpoint{0.438698in}{1.341985in}}%
\pgfpathlineto{\pgfqpoint{0.435557in}{1.338981in}}%
\pgfpathlineto{\pgfqpoint{0.423913in}{1.325370in}}%
\pgfpathlineto{\pgfqpoint{0.423041in}{1.323944in}}%
\pgfpathlineto{\pgfqpoint{0.415111in}{1.311759in}}%
\pgfpathlineto{\pgfqpoint{0.410267in}{1.298147in}}%
\pgfpathlineto{\pgfqpoint{0.409461in}{1.284536in}}%
\pgfpathlineto{\pgfqpoint{0.412688in}{1.270925in}}%
\pgfpathlineto{\pgfqpoint{0.419962in}{1.257314in}}%
\pgfpathlineto{\pgfqpoint{0.423041in}{1.253476in}}%
\pgfpathlineto{\pgfqpoint{0.430477in}{1.243703in}}%
\pgfpathlineto{\pgfqpoint{0.438698in}{1.235336in}}%
\pgfpathlineto{\pgfqpoint{0.444195in}{1.230092in}}%
\pgfpathlineto{\pgfqpoint{0.454354in}{1.221600in}}%
\pgfpathlineto{\pgfqpoint{0.461679in}{1.216481in}}%
\pgfpathlineto{\pgfqpoint{0.470011in}{1.210768in}}%
\pgfpathlineto{\pgfqpoint{0.485668in}{1.202934in}}%
\pgfpathlineto{\pgfqpoint{0.485884in}{1.202870in}}%
\pgfpathlineto{\pgfqpoint{0.501324in}{1.197652in}}%
\pgfpathclose%
\pgfpathmoveto{\pgfqpoint{0.500159in}{1.216481in}}%
\pgfpathlineto{\pgfqpoint{0.485668in}{1.220966in}}%
\pgfpathlineto{\pgfqpoint{0.470011in}{1.229266in}}%
\pgfpathlineto{\pgfqpoint{0.468845in}{1.230092in}}%
\pgfpathlineto{\pgfqpoint{0.454354in}{1.241683in}}%
\pgfpathlineto{\pgfqpoint{0.452225in}{1.243703in}}%
\pgfpathlineto{\pgfqpoint{0.440948in}{1.257314in}}%
\pgfpathlineto{\pgfqpoint{0.438698in}{1.261441in}}%
\pgfpathlineto{\pgfqpoint{0.433709in}{1.270925in}}%
\pgfpathlineto{\pgfqpoint{0.430501in}{1.284536in}}%
\pgfpathlineto{\pgfqpoint{0.431302in}{1.298147in}}%
\pgfpathlineto{\pgfqpoint{0.436118in}{1.311759in}}%
\pgfpathlineto{\pgfqpoint{0.438698in}{1.315841in}}%
\pgfpathlineto{\pgfqpoint{0.444981in}{1.325370in}}%
\pgfpathlineto{\pgfqpoint{0.454354in}{1.335440in}}%
\pgfpathlineto{\pgfqpoint{0.458234in}{1.338981in}}%
\pgfpathlineto{\pgfqpoint{0.470011in}{1.347878in}}%
\pgfpathlineto{\pgfqpoint{0.478508in}{1.352592in}}%
\pgfpathlineto{\pgfqpoint{0.485668in}{1.356311in}}%
\pgfpathlineto{\pgfqpoint{0.501324in}{1.361161in}}%
\pgfpathlineto{\pgfqpoint{0.516981in}{1.362544in}}%
\pgfpathlineto{\pgfqpoint{0.532637in}{1.360469in}}%
\pgfpathlineto{\pgfqpoint{0.548294in}{1.354924in}}%
\pgfpathlineto{\pgfqpoint{0.552482in}{1.352592in}}%
\pgfpathlineto{\pgfqpoint{0.563950in}{1.345751in}}%
\pgfpathlineto{\pgfqpoint{0.572548in}{1.338981in}}%
\pgfpathlineto{\pgfqpoint{0.579607in}{1.332250in}}%
\pgfpathlineto{\pgfqpoint{0.585875in}{1.325370in}}%
\pgfpathlineto{\pgfqpoint{0.594669in}{1.311759in}}%
\pgfpathlineto{\pgfqpoint{0.595263in}{1.310101in}}%
\pgfpathlineto{\pgfqpoint{0.599530in}{1.298147in}}%
\pgfpathlineto{\pgfqpoint{0.600340in}{1.284536in}}%
\pgfpathlineto{\pgfqpoint{0.597096in}{1.270925in}}%
\pgfpathlineto{\pgfqpoint{0.595263in}{1.267433in}}%
\pgfpathlineto{\pgfqpoint{0.589872in}{1.257314in}}%
\pgfpathlineto{\pgfqpoint{0.579607in}{1.244767in}}%
\pgfpathlineto{\pgfqpoint{0.578611in}{1.243703in}}%
\pgfpathlineto{\pgfqpoint{0.563950in}{1.231422in}}%
\pgfpathlineto{\pgfqpoint{0.561863in}{1.230092in}}%
\pgfpathlineto{\pgfqpoint{0.548294in}{1.222349in}}%
\pgfpathlineto{\pgfqpoint{0.532637in}{1.216819in}}%
\pgfpathlineto{\pgfqpoint{0.530100in}{1.216481in}}%
\pgfpathlineto{\pgfqpoint{0.516981in}{1.214684in}}%
\pgfpathlineto{\pgfqpoint{0.501324in}{1.216116in}}%
\pgfpathlineto{\pgfqpoint{0.500159in}{1.216481in}}%
\pgfpathclose%
\pgfpathmoveto{\pgfqpoint{0.798799in}{1.201583in}}%
\pgfpathlineto{\pgfqpoint{0.814455in}{1.197048in}}%
\pgfpathlineto{\pgfqpoint{0.830112in}{1.196293in}}%
\pgfpathlineto{\pgfqpoint{0.845769in}{1.199314in}}%
\pgfpathlineto{\pgfqpoint{0.854069in}{1.202870in}}%
\pgfpathlineto{\pgfqpoint{0.861425in}{1.205677in}}%
\pgfpathlineto{\pgfqpoint{0.877082in}{1.214805in}}%
\pgfpathlineto{\pgfqpoint{0.879341in}{1.216481in}}%
\pgfpathlineto{\pgfqpoint{0.892738in}{1.226656in}}%
\pgfpathlineto{\pgfqpoint{0.896690in}{1.230092in}}%
\pgfpathlineto{\pgfqpoint{0.908395in}{1.241739in}}%
\pgfpathlineto{\pgfqpoint{0.910322in}{1.243703in}}%
\pgfpathlineto{\pgfqpoint{0.920823in}{1.257314in}}%
\pgfpathlineto{\pgfqpoint{0.924051in}{1.263709in}}%
\pgfpathlineto{\pgfqpoint{0.928142in}{1.270925in}}%
\pgfpathlineto{\pgfqpoint{0.931616in}{1.284536in}}%
\pgfpathlineto{\pgfqpoint{0.930748in}{1.298147in}}%
\pgfpathlineto{\pgfqpoint{0.925532in}{1.311759in}}%
\pgfpathlineto{\pgfqpoint{0.924051in}{1.313933in}}%
\pgfpathlineto{\pgfqpoint{0.917068in}{1.325370in}}%
\pgfpathlineto{\pgfqpoint{0.908395in}{1.335379in}}%
\pgfpathlineto{\pgfqpoint{0.905223in}{1.338981in}}%
\pgfpathlineto{\pgfqpoint{0.892738in}{1.350557in}}%
\pgfpathlineto{\pgfqpoint{0.890238in}{1.352592in}}%
\pgfpathlineto{\pgfqpoint{0.877082in}{1.362427in}}%
\pgfpathlineto{\pgfqpoint{0.870560in}{1.366203in}}%
\pgfpathlineto{\pgfqpoint{0.861425in}{1.371660in}}%
\pgfpathlineto{\pgfqpoint{0.845769in}{1.377802in}}%
\pgfpathlineto{\pgfqpoint{0.834248in}{1.379814in}}%
\pgfpathlineto{\pgfqpoint{0.830112in}{1.380684in}}%
\pgfpathlineto{\pgfqpoint{0.814455in}{1.379853in}}%
\pgfpathlineto{\pgfqpoint{0.814332in}{1.379814in}}%
\pgfpathlineto{\pgfqpoint{0.798799in}{1.375756in}}%
\pgfpathlineto{\pgfqpoint{0.783142in}{1.368247in}}%
\pgfpathlineto{\pgfqpoint{0.780070in}{1.366203in}}%
\pgfpathlineto{\pgfqpoint{0.767486in}{1.358043in}}%
\pgfpathlineto{\pgfqpoint{0.760632in}{1.352592in}}%
\pgfpathlineto{\pgfqpoint{0.751829in}{1.344939in}}%
\pgfpathlineto{\pgfqpoint{0.745559in}{1.338981in}}%
\pgfpathlineto{\pgfqpoint{0.736173in}{1.328041in}}%
\pgfpathlineto{\pgfqpoint{0.733822in}{1.325370in}}%
\pgfpathlineto{\pgfqpoint{0.725184in}{1.311759in}}%
\pgfpathlineto{\pgfqpoint{0.720516in}{1.298254in}}%
\pgfpathlineto{\pgfqpoint{0.720472in}{1.298147in}}%
\pgfpathlineto{\pgfqpoint{0.719515in}{1.284536in}}%
\pgfpathlineto{\pgfqpoint{0.720516in}{1.280940in}}%
\pgfpathlineto{\pgfqpoint{0.722830in}{1.270925in}}%
\pgfpathlineto{\pgfqpoint{0.729895in}{1.257314in}}%
\pgfpathlineto{\pgfqpoint{0.736173in}{1.249372in}}%
\pgfpathlineto{\pgfqpoint{0.740516in}{1.243703in}}%
\pgfpathlineto{\pgfqpoint{0.751829in}{1.232266in}}%
\pgfpathlineto{\pgfqpoint{0.754170in}{1.230092in}}%
\pgfpathlineto{\pgfqpoint{0.767486in}{1.219239in}}%
\pgfpathlineto{\pgfqpoint{0.771628in}{1.216481in}}%
\pgfpathlineto{\pgfqpoint{0.783142in}{1.208941in}}%
\pgfpathlineto{\pgfqpoint{0.796298in}{1.202870in}}%
\pgfpathlineto{\pgfqpoint{0.798799in}{1.201583in}}%
\pgfpathclose%
\pgfpathmoveto{\pgfqpoint{0.810983in}{1.216481in}}%
\pgfpathlineto{\pgfqpoint{0.798799in}{1.219721in}}%
\pgfpathlineto{\pgfqpoint{0.783142in}{1.227331in}}%
\pgfpathlineto{\pgfqpoint{0.779060in}{1.230092in}}%
\pgfpathlineto{\pgfqpoint{0.767486in}{1.238911in}}%
\pgfpathlineto{\pgfqpoint{0.762305in}{1.243703in}}%
\pgfpathlineto{\pgfqpoint{0.751829in}{1.256133in}}%
\pgfpathlineto{\pgfqpoint{0.750903in}{1.257314in}}%
\pgfpathlineto{\pgfqpoint{0.743724in}{1.270925in}}%
\pgfpathlineto{\pgfqpoint{0.740541in}{1.284536in}}%
\pgfpathlineto{\pgfqpoint{0.741336in}{1.298147in}}%
\pgfpathlineto{\pgfqpoint{0.746116in}{1.311759in}}%
\pgfpathlineto{\pgfqpoint{0.751829in}{1.320739in}}%
\pgfpathlineto{\pgfqpoint{0.754966in}{1.325370in}}%
\pgfpathlineto{\pgfqpoint{0.767486in}{1.338486in}}%
\pgfpathlineto{\pgfqpoint{0.768055in}{1.338981in}}%
\pgfpathlineto{\pgfqpoint{0.783142in}{1.349865in}}%
\pgfpathlineto{\pgfqpoint{0.788469in}{1.352592in}}%
\pgfpathlineto{\pgfqpoint{0.798799in}{1.357559in}}%
\pgfpathlineto{\pgfqpoint{0.814455in}{1.361714in}}%
\pgfpathlineto{\pgfqpoint{0.830112in}{1.362406in}}%
\pgfpathlineto{\pgfqpoint{0.845769in}{1.359638in}}%
\pgfpathlineto{\pgfqpoint{0.861425in}{1.353397in}}%
\pgfpathlineto{\pgfqpoint{0.862784in}{1.352592in}}%
\pgfpathlineto{\pgfqpoint{0.877082in}{1.343485in}}%
\pgfpathlineto{\pgfqpoint{0.882594in}{1.338981in}}%
\pgfpathlineto{\pgfqpoint{0.892738in}{1.328919in}}%
\pgfpathlineto{\pgfqpoint{0.895915in}{1.325370in}}%
\pgfpathlineto{\pgfqpoint{0.904668in}{1.311759in}}%
\pgfpathlineto{\pgfqpoint{0.908395in}{1.301166in}}%
\pgfpathlineto{\pgfqpoint{0.909474in}{1.298147in}}%
\pgfpathlineto{\pgfqpoint{0.910297in}{1.284536in}}%
\pgfpathlineto{\pgfqpoint{0.908395in}{1.276632in}}%
\pgfpathlineto{\pgfqpoint{0.907052in}{1.270925in}}%
\pgfpathlineto{\pgfqpoint{0.899893in}{1.257314in}}%
\pgfpathlineto{\pgfqpoint{0.892738in}{1.248411in}}%
\pgfpathlineto{\pgfqpoint{0.888499in}{1.243703in}}%
\pgfpathlineto{\pgfqpoint{0.877082in}{1.233777in}}%
\pgfpathlineto{\pgfqpoint{0.871666in}{1.230092in}}%
\pgfpathlineto{\pgfqpoint{0.861425in}{1.223872in}}%
\pgfpathlineto{\pgfqpoint{0.845769in}{1.217648in}}%
\pgfpathlineto{\pgfqpoint{0.839204in}{1.216481in}}%
\pgfpathlineto{\pgfqpoint{0.830112in}{1.214827in}}%
\pgfpathlineto{\pgfqpoint{0.814455in}{1.215543in}}%
\pgfpathlineto{\pgfqpoint{0.810983in}{1.216481in}}%
\pgfpathclose%
\pgfpathmoveto{\pgfqpoint{1.111930in}{1.200372in}}%
\pgfpathlineto{\pgfqpoint{1.127587in}{1.196595in}}%
\pgfpathlineto{\pgfqpoint{1.143243in}{1.196595in}}%
\pgfpathlineto{\pgfqpoint{1.158900in}{1.200372in}}%
\pgfpathlineto{\pgfqpoint{1.164190in}{1.202870in}}%
\pgfpathlineto{\pgfqpoint{1.174556in}{1.207244in}}%
\pgfpathlineto{\pgfqpoint{1.189401in}{1.216481in}}%
\pgfpathlineto{\pgfqpoint{1.190213in}{1.216993in}}%
\pgfpathlineto{\pgfqpoint{1.205870in}{1.229343in}}%
\pgfpathlineto{\pgfqpoint{1.206719in}{1.230092in}}%
\pgfpathlineto{\pgfqpoint{1.220271in}{1.243703in}}%
\pgfpathlineto{\pgfqpoint{1.221526in}{1.245323in}}%
\pgfpathlineto{\pgfqpoint{1.230910in}{1.257314in}}%
\pgfpathlineto{\pgfqpoint{1.237183in}{1.269664in}}%
\pgfpathlineto{\pgfqpoint{1.237920in}{1.270925in}}%
\pgfpathlineto{\pgfqpoint{1.241556in}{1.284536in}}%
\pgfpathlineto{\pgfqpoint{1.240648in}{1.298147in}}%
\pgfpathlineto{\pgfqpoint{1.237183in}{1.306857in}}%
\pgfpathlineto{\pgfqpoint{1.235507in}{1.311759in}}%
\pgfpathlineto{\pgfqpoint{1.227078in}{1.325370in}}%
\pgfpathlineto{\pgfqpoint{1.221526in}{1.331742in}}%
\pgfpathlineto{\pgfqpoint{1.215246in}{1.338981in}}%
\pgfpathlineto{\pgfqpoint{1.205870in}{1.347799in}}%
\pgfpathlineto{\pgfqpoint{1.200183in}{1.352592in}}%
\pgfpathlineto{\pgfqpoint{1.190213in}{1.360294in}}%
\pgfpathlineto{\pgfqpoint{1.180597in}{1.366203in}}%
\pgfpathlineto{\pgfqpoint{1.174556in}{1.370022in}}%
\pgfpathlineto{\pgfqpoint{1.158900in}{1.376848in}}%
\pgfpathlineto{\pgfqpoint{1.145297in}{1.379814in}}%
\pgfpathlineto{\pgfqpoint{1.143243in}{1.380352in}}%
\pgfpathlineto{\pgfqpoint{1.127587in}{1.380352in}}%
\pgfpathlineto{\pgfqpoint{1.125533in}{1.379814in}}%
\pgfpathlineto{\pgfqpoint{1.111930in}{1.376848in}}%
\pgfpathlineto{\pgfqpoint{1.096274in}{1.370022in}}%
\pgfpathlineto{\pgfqpoint{1.090233in}{1.366203in}}%
\pgfpathlineto{\pgfqpoint{1.080617in}{1.360294in}}%
\pgfpathlineto{\pgfqpoint{1.070647in}{1.352592in}}%
\pgfpathlineto{\pgfqpoint{1.064960in}{1.347799in}}%
\pgfpathlineto{\pgfqpoint{1.055584in}{1.338981in}}%
\pgfpathlineto{\pgfqpoint{1.049304in}{1.331742in}}%
\pgfpathlineto{\pgfqpoint{1.043752in}{1.325370in}}%
\pgfpathlineto{\pgfqpoint{1.035323in}{1.311759in}}%
\pgfpathlineto{\pgfqpoint{1.033647in}{1.306857in}}%
\pgfpathlineto{\pgfqpoint{1.030182in}{1.298147in}}%
\pgfpathlineto{\pgfqpoint{1.029274in}{1.284536in}}%
\pgfpathlineto{\pgfqpoint{1.032910in}{1.270925in}}%
\pgfpathlineto{\pgfqpoint{1.033647in}{1.269664in}}%
\pgfpathlineto{\pgfqpoint{1.039920in}{1.257314in}}%
\pgfpathlineto{\pgfqpoint{1.049304in}{1.245323in}}%
\pgfpathlineto{\pgfqpoint{1.050559in}{1.243703in}}%
\pgfpathlineto{\pgfqpoint{1.064111in}{1.230092in}}%
\pgfpathlineto{\pgfqpoint{1.064960in}{1.229343in}}%
\pgfpathlineto{\pgfqpoint{1.080617in}{1.216993in}}%
\pgfpathlineto{\pgfqpoint{1.081429in}{1.216481in}}%
\pgfpathlineto{\pgfqpoint{1.096274in}{1.207244in}}%
\pgfpathlineto{\pgfqpoint{1.106640in}{1.202870in}}%
\pgfpathlineto{\pgfqpoint{1.111930in}{1.200372in}}%
\pgfpathclose%
\pgfpathmoveto{\pgfqpoint{1.121544in}{1.216481in}}%
\pgfpathlineto{\pgfqpoint{1.111930in}{1.218615in}}%
\pgfpathlineto{\pgfqpoint{1.096274in}{1.225532in}}%
\pgfpathlineto{\pgfqpoint{1.089179in}{1.230092in}}%
\pgfpathlineto{\pgfqpoint{1.080617in}{1.236274in}}%
\pgfpathlineto{\pgfqpoint{1.072349in}{1.243703in}}%
\pgfpathlineto{\pgfqpoint{1.064960in}{1.252203in}}%
\pgfpathlineto{\pgfqpoint{1.060910in}{1.257314in}}%
\pgfpathlineto{\pgfqpoint{1.053756in}{1.270925in}}%
\pgfpathlineto{\pgfqpoint{1.050583in}{1.284536in}}%
\pgfpathlineto{\pgfqpoint{1.051375in}{1.298147in}}%
\pgfpathlineto{\pgfqpoint{1.056139in}{1.311759in}}%
\pgfpathlineto{\pgfqpoint{1.064886in}{1.325370in}}%
\pgfpathlineto{\pgfqpoint{1.064960in}{1.325454in}}%
\pgfpathlineto{\pgfqpoint{1.078128in}{1.338981in}}%
\pgfpathlineto{\pgfqpoint{1.080617in}{1.341083in}}%
\pgfpathlineto{\pgfqpoint{1.096274in}{1.351712in}}%
\pgfpathlineto{\pgfqpoint{1.098154in}{1.352592in}}%
\pgfpathlineto{\pgfqpoint{1.111930in}{1.358668in}}%
\pgfpathlineto{\pgfqpoint{1.127587in}{1.362129in}}%
\pgfpathlineto{\pgfqpoint{1.143243in}{1.362129in}}%
\pgfpathlineto{\pgfqpoint{1.158900in}{1.358668in}}%
\pgfpathlineto{\pgfqpoint{1.172676in}{1.352592in}}%
\pgfpathlineto{\pgfqpoint{1.174556in}{1.351712in}}%
\pgfpathlineto{\pgfqpoint{1.190213in}{1.341083in}}%
\pgfpathlineto{\pgfqpoint{1.192702in}{1.338981in}}%
\pgfpathlineto{\pgfqpoint{1.205870in}{1.325454in}}%
\pgfpathlineto{\pgfqpoint{1.205944in}{1.325370in}}%
\pgfpathlineto{\pgfqpoint{1.214691in}{1.311759in}}%
\pgfpathlineto{\pgfqpoint{1.219455in}{1.298147in}}%
\pgfpathlineto{\pgfqpoint{1.220247in}{1.284536in}}%
\pgfpathlineto{\pgfqpoint{1.217074in}{1.270925in}}%
\pgfpathlineto{\pgfqpoint{1.209920in}{1.257314in}}%
\pgfpathlineto{\pgfqpoint{1.205870in}{1.252203in}}%
\pgfpathlineto{\pgfqpoint{1.198481in}{1.243703in}}%
\pgfpathlineto{\pgfqpoint{1.190213in}{1.236274in}}%
\pgfpathlineto{\pgfqpoint{1.181651in}{1.230092in}}%
\pgfpathlineto{\pgfqpoint{1.174556in}{1.225532in}}%
\pgfpathlineto{\pgfqpoint{1.158900in}{1.218615in}}%
\pgfpathlineto{\pgfqpoint{1.149286in}{1.216481in}}%
\pgfpathlineto{\pgfqpoint{1.143243in}{1.215113in}}%
\pgfpathlineto{\pgfqpoint{1.127587in}{1.215113in}}%
\pgfpathlineto{\pgfqpoint{1.121544in}{1.216481in}}%
\pgfpathclose%
\pgfpathmoveto{\pgfqpoint{1.425061in}{1.199314in}}%
\pgfpathlineto{\pgfqpoint{1.440718in}{1.196293in}}%
\pgfpathlineto{\pgfqpoint{1.456375in}{1.197048in}}%
\pgfpathlineto{\pgfqpoint{1.472031in}{1.201583in}}%
\pgfpathlineto{\pgfqpoint{1.474532in}{1.202870in}}%
\pgfpathlineto{\pgfqpoint{1.487688in}{1.208941in}}%
\pgfpathlineto{\pgfqpoint{1.499202in}{1.216481in}}%
\pgfpathlineto{\pgfqpoint{1.503344in}{1.219239in}}%
\pgfpathlineto{\pgfqpoint{1.516660in}{1.230092in}}%
\pgfpathlineto{\pgfqpoint{1.519001in}{1.232266in}}%
\pgfpathlineto{\pgfqpoint{1.530314in}{1.243703in}}%
\pgfpathlineto{\pgfqpoint{1.534657in}{1.249372in}}%
\pgfpathlineto{\pgfqpoint{1.540935in}{1.257314in}}%
\pgfpathlineto{\pgfqpoint{1.548000in}{1.270925in}}%
\pgfpathlineto{\pgfqpoint{1.550314in}{1.280940in}}%
\pgfpathlineto{\pgfqpoint{1.551315in}{1.284536in}}%
\pgfpathlineto{\pgfqpoint{1.550358in}{1.298147in}}%
\pgfpathlineto{\pgfqpoint{1.550314in}{1.298254in}}%
\pgfpathlineto{\pgfqpoint{1.545646in}{1.311759in}}%
\pgfpathlineto{\pgfqpoint{1.537008in}{1.325370in}}%
\pgfpathlineto{\pgfqpoint{1.534657in}{1.328041in}}%
\pgfpathlineto{\pgfqpoint{1.525271in}{1.338981in}}%
\pgfpathlineto{\pgfqpoint{1.519001in}{1.344939in}}%
\pgfpathlineto{\pgfqpoint{1.510198in}{1.352592in}}%
\pgfpathlineto{\pgfqpoint{1.503344in}{1.358043in}}%
\pgfpathlineto{\pgfqpoint{1.490760in}{1.366203in}}%
\pgfpathlineto{\pgfqpoint{1.487688in}{1.368247in}}%
\pgfpathlineto{\pgfqpoint{1.472031in}{1.375756in}}%
\pgfpathlineto{\pgfqpoint{1.456498in}{1.379814in}}%
\pgfpathlineto{\pgfqpoint{1.456375in}{1.379853in}}%
\pgfpathlineto{\pgfqpoint{1.440718in}{1.380684in}}%
\pgfpathlineto{\pgfqpoint{1.436582in}{1.379814in}}%
\pgfpathlineto{\pgfqpoint{1.425061in}{1.377802in}}%
\pgfpathlineto{\pgfqpoint{1.409405in}{1.371660in}}%
\pgfpathlineto{\pgfqpoint{1.400270in}{1.366203in}}%
\pgfpathlineto{\pgfqpoint{1.393748in}{1.362427in}}%
\pgfpathlineto{\pgfqpoint{1.380592in}{1.352592in}}%
\pgfpathlineto{\pgfqpoint{1.378092in}{1.350557in}}%
\pgfpathlineto{\pgfqpoint{1.365607in}{1.338981in}}%
\pgfpathlineto{\pgfqpoint{1.362435in}{1.335379in}}%
\pgfpathlineto{\pgfqpoint{1.353762in}{1.325370in}}%
\pgfpathlineto{\pgfqpoint{1.346779in}{1.313933in}}%
\pgfpathlineto{\pgfqpoint{1.345298in}{1.311759in}}%
\pgfpathlineto{\pgfqpoint{1.340082in}{1.298147in}}%
\pgfpathlineto{\pgfqpoint{1.339214in}{1.284536in}}%
\pgfpathlineto{\pgfqpoint{1.342688in}{1.270925in}}%
\pgfpathlineto{\pgfqpoint{1.346779in}{1.263709in}}%
\pgfpathlineto{\pgfqpoint{1.350007in}{1.257314in}}%
\pgfpathlineto{\pgfqpoint{1.360508in}{1.243703in}}%
\pgfpathlineto{\pgfqpoint{1.362435in}{1.241739in}}%
\pgfpathlineto{\pgfqpoint{1.374140in}{1.230092in}}%
\pgfpathlineto{\pgfqpoint{1.378092in}{1.226656in}}%
\pgfpathlineto{\pgfqpoint{1.391489in}{1.216481in}}%
\pgfpathlineto{\pgfqpoint{1.393748in}{1.214805in}}%
\pgfpathlineto{\pgfqpoint{1.409405in}{1.205677in}}%
\pgfpathlineto{\pgfqpoint{1.416761in}{1.202870in}}%
\pgfpathlineto{\pgfqpoint{1.425061in}{1.199314in}}%
\pgfpathclose%
\pgfpathmoveto{\pgfqpoint{1.431626in}{1.216481in}}%
\pgfpathlineto{\pgfqpoint{1.425061in}{1.217648in}}%
\pgfpathlineto{\pgfqpoint{1.409405in}{1.223872in}}%
\pgfpathlineto{\pgfqpoint{1.399164in}{1.230092in}}%
\pgfpathlineto{\pgfqpoint{1.393748in}{1.233777in}}%
\pgfpathlineto{\pgfqpoint{1.382331in}{1.243703in}}%
\pgfpathlineto{\pgfqpoint{1.378092in}{1.248411in}}%
\pgfpathlineto{\pgfqpoint{1.370937in}{1.257314in}}%
\pgfpathlineto{\pgfqpoint{1.363778in}{1.270925in}}%
\pgfpathlineto{\pgfqpoint{1.362435in}{1.276632in}}%
\pgfpathlineto{\pgfqpoint{1.360533in}{1.284536in}}%
\pgfpathlineto{\pgfqpoint{1.361356in}{1.298147in}}%
\pgfpathlineto{\pgfqpoint{1.362435in}{1.301166in}}%
\pgfpathlineto{\pgfqpoint{1.366162in}{1.311759in}}%
\pgfpathlineto{\pgfqpoint{1.374915in}{1.325370in}}%
\pgfpathlineto{\pgfqpoint{1.378092in}{1.328919in}}%
\pgfpathlineto{\pgfqpoint{1.388236in}{1.338981in}}%
\pgfpathlineto{\pgfqpoint{1.393748in}{1.343485in}}%
\pgfpathlineto{\pgfqpoint{1.408046in}{1.352592in}}%
\pgfpathlineto{\pgfqpoint{1.409405in}{1.353397in}}%
\pgfpathlineto{\pgfqpoint{1.425061in}{1.359638in}}%
\pgfpathlineto{\pgfqpoint{1.440718in}{1.362406in}}%
\pgfpathlineto{\pgfqpoint{1.456375in}{1.361714in}}%
\pgfpathlineto{\pgfqpoint{1.472031in}{1.357559in}}%
\pgfpathlineto{\pgfqpoint{1.482361in}{1.352592in}}%
\pgfpathlineto{\pgfqpoint{1.487688in}{1.349865in}}%
\pgfpathlineto{\pgfqpoint{1.502775in}{1.338981in}}%
\pgfpathlineto{\pgfqpoint{1.503344in}{1.338486in}}%
\pgfpathlineto{\pgfqpoint{1.515864in}{1.325370in}}%
\pgfpathlineto{\pgfqpoint{1.519001in}{1.320739in}}%
\pgfpathlineto{\pgfqpoint{1.524714in}{1.311759in}}%
\pgfpathlineto{\pgfqpoint{1.529494in}{1.298147in}}%
\pgfpathlineto{\pgfqpoint{1.530289in}{1.284536in}}%
\pgfpathlineto{\pgfqpoint{1.527106in}{1.270925in}}%
\pgfpathlineto{\pgfqpoint{1.519927in}{1.257314in}}%
\pgfpathlineto{\pgfqpoint{1.519001in}{1.256133in}}%
\pgfpathlineto{\pgfqpoint{1.508525in}{1.243703in}}%
\pgfpathlineto{\pgfqpoint{1.503344in}{1.238911in}}%
\pgfpathlineto{\pgfqpoint{1.491770in}{1.230092in}}%
\pgfpathlineto{\pgfqpoint{1.487688in}{1.227331in}}%
\pgfpathlineto{\pgfqpoint{1.472031in}{1.219721in}}%
\pgfpathlineto{\pgfqpoint{1.459847in}{1.216481in}}%
\pgfpathlineto{\pgfqpoint{1.456375in}{1.215543in}}%
\pgfpathlineto{\pgfqpoint{1.440718in}{1.214827in}}%
\pgfpathlineto{\pgfqpoint{1.431626in}{1.216481in}}%
\pgfpathclose%
\pgfpathmoveto{\pgfqpoint{1.738193in}{1.198407in}}%
\pgfpathlineto{\pgfqpoint{1.753849in}{1.196143in}}%
\pgfpathlineto{\pgfqpoint{1.769506in}{1.197652in}}%
\pgfpathlineto{\pgfqpoint{1.784946in}{1.202870in}}%
\pgfpathlineto{\pgfqpoint{1.785162in}{1.202934in}}%
\pgfpathlineto{\pgfqpoint{1.800819in}{1.210768in}}%
\pgfpathlineto{\pgfqpoint{1.809151in}{1.216481in}}%
\pgfpathlineto{\pgfqpoint{1.816476in}{1.221600in}}%
\pgfpathlineto{\pgfqpoint{1.826635in}{1.230092in}}%
\pgfpathlineto{\pgfqpoint{1.832132in}{1.235336in}}%
\pgfpathlineto{\pgfqpoint{1.840353in}{1.243703in}}%
\pgfpathlineto{\pgfqpoint{1.847789in}{1.253476in}}%
\pgfpathlineto{\pgfqpoint{1.850868in}{1.257314in}}%
\pgfpathlineto{\pgfqpoint{1.858142in}{1.270925in}}%
\pgfpathlineto{\pgfqpoint{1.861369in}{1.284536in}}%
\pgfpathlineto{\pgfqpoint{1.860563in}{1.298147in}}%
\pgfpathlineto{\pgfqpoint{1.855719in}{1.311759in}}%
\pgfpathlineto{\pgfqpoint{1.847789in}{1.323944in}}%
\pgfpathlineto{\pgfqpoint{1.846917in}{1.325370in}}%
\pgfpathlineto{\pgfqpoint{1.835273in}{1.338981in}}%
\pgfpathlineto{\pgfqpoint{1.832132in}{1.341985in}}%
\pgfpathlineto{\pgfqpoint{1.820256in}{1.352592in}}%
\pgfpathlineto{\pgfqpoint{1.816476in}{1.355675in}}%
\pgfpathlineto{\pgfqpoint{1.801011in}{1.366203in}}%
\pgfpathlineto{\pgfqpoint{1.800819in}{1.366337in}}%
\pgfpathlineto{\pgfqpoint{1.785162in}{1.374528in}}%
\pgfpathlineto{\pgfqpoint{1.769506in}{1.379301in}}%
\pgfpathlineto{\pgfqpoint{1.763630in}{1.379814in}}%
\pgfpathlineto{\pgfqpoint{1.753849in}{1.380851in}}%
\pgfpathlineto{\pgfqpoint{1.747307in}{1.379814in}}%
\pgfpathlineto{\pgfqpoint{1.738193in}{1.378620in}}%
\pgfpathlineto{\pgfqpoint{1.722536in}{1.373162in}}%
\pgfpathlineto{\pgfqpoint{1.710134in}{1.366203in}}%
\pgfpathlineto{\pgfqpoint{1.706880in}{1.364438in}}%
\pgfpathlineto{\pgfqpoint{1.691223in}{1.353195in}}%
\pgfpathlineto{\pgfqpoint{1.690500in}{1.352592in}}%
\pgfpathlineto{\pgfqpoint{1.675603in}{1.338981in}}%
\pgfpathlineto{\pgfqpoint{1.675567in}{1.338940in}}%
\pgfpathlineto{\pgfqpoint{1.663825in}{1.325370in}}%
\pgfpathlineto{\pgfqpoint{1.659910in}{1.318949in}}%
\pgfpathlineto{\pgfqpoint{1.655137in}{1.311759in}}%
\pgfpathlineto{\pgfqpoint{1.650123in}{1.298147in}}%
\pgfpathlineto{\pgfqpoint{1.649289in}{1.284536in}}%
\pgfpathlineto{\pgfqpoint{1.652629in}{1.270925in}}%
\pgfpathlineto{\pgfqpoint{1.659910in}{1.257748in}}%
\pgfpathlineto{\pgfqpoint{1.660129in}{1.257314in}}%
\pgfpathlineto{\pgfqpoint{1.670465in}{1.243703in}}%
\pgfpathlineto{\pgfqpoint{1.675567in}{1.238497in}}%
\pgfpathlineto{\pgfqpoint{1.684176in}{1.230092in}}%
\pgfpathlineto{\pgfqpoint{1.691223in}{1.224074in}}%
\pgfpathlineto{\pgfqpoint{1.701621in}{1.216481in}}%
\pgfpathlineto{\pgfqpoint{1.706880in}{1.212723in}}%
\pgfpathlineto{\pgfqpoint{1.722536in}{1.204240in}}%
\pgfpathlineto{\pgfqpoint{1.726560in}{1.202870in}}%
\pgfpathlineto{\pgfqpoint{1.738193in}{1.198407in}}%
\pgfpathclose%
\pgfpathmoveto{\pgfqpoint{1.740730in}{1.216481in}}%
\pgfpathlineto{\pgfqpoint{1.738193in}{1.216819in}}%
\pgfpathlineto{\pgfqpoint{1.722536in}{1.222349in}}%
\pgfpathlineto{\pgfqpoint{1.708967in}{1.230092in}}%
\pgfpathlineto{\pgfqpoint{1.706880in}{1.231422in}}%
\pgfpathlineto{\pgfqpoint{1.692219in}{1.243703in}}%
\pgfpathlineto{\pgfqpoint{1.691223in}{1.244767in}}%
\pgfpathlineto{\pgfqpoint{1.680958in}{1.257314in}}%
\pgfpathlineto{\pgfqpoint{1.675567in}{1.267433in}}%
\pgfpathlineto{\pgfqpoint{1.673734in}{1.270925in}}%
\pgfpathlineto{\pgfqpoint{1.670490in}{1.284536in}}%
\pgfpathlineto{\pgfqpoint{1.671300in}{1.298147in}}%
\pgfpathlineto{\pgfqpoint{1.675567in}{1.310101in}}%
\pgfpathlineto{\pgfqpoint{1.676161in}{1.311759in}}%
\pgfpathlineto{\pgfqpoint{1.684955in}{1.325370in}}%
\pgfpathlineto{\pgfqpoint{1.691223in}{1.332250in}}%
\pgfpathlineto{\pgfqpoint{1.698282in}{1.338981in}}%
\pgfpathlineto{\pgfqpoint{1.706880in}{1.345751in}}%
\pgfpathlineto{\pgfqpoint{1.718348in}{1.352592in}}%
\pgfpathlineto{\pgfqpoint{1.722536in}{1.354924in}}%
\pgfpathlineto{\pgfqpoint{1.738193in}{1.360469in}}%
\pgfpathlineto{\pgfqpoint{1.753849in}{1.362544in}}%
\pgfpathlineto{\pgfqpoint{1.769506in}{1.361161in}}%
\pgfpathlineto{\pgfqpoint{1.785162in}{1.356311in}}%
\pgfpathlineto{\pgfqpoint{1.792322in}{1.352592in}}%
\pgfpathlineto{\pgfqpoint{1.800819in}{1.347878in}}%
\pgfpathlineto{\pgfqpoint{1.812596in}{1.338981in}}%
\pgfpathlineto{\pgfqpoint{1.816476in}{1.335440in}}%
\pgfpathlineto{\pgfqpoint{1.825849in}{1.325370in}}%
\pgfpathlineto{\pgfqpoint{1.832132in}{1.315841in}}%
\pgfpathlineto{\pgfqpoint{1.834712in}{1.311759in}}%
\pgfpathlineto{\pgfqpoint{1.839528in}{1.298147in}}%
\pgfpathlineto{\pgfqpoint{1.840329in}{1.284536in}}%
\pgfpathlineto{\pgfqpoint{1.837121in}{1.270925in}}%
\pgfpathlineto{\pgfqpoint{1.832132in}{1.261441in}}%
\pgfpathlineto{\pgfqpoint{1.829882in}{1.257314in}}%
\pgfpathlineto{\pgfqpoint{1.818605in}{1.243703in}}%
\pgfpathlineto{\pgfqpoint{1.816476in}{1.241683in}}%
\pgfpathlineto{\pgfqpoint{1.801985in}{1.230092in}}%
\pgfpathlineto{\pgfqpoint{1.800819in}{1.229266in}}%
\pgfpathlineto{\pgfqpoint{1.785162in}{1.220966in}}%
\pgfpathlineto{\pgfqpoint{1.770671in}{1.216481in}}%
\pgfpathlineto{\pgfqpoint{1.769506in}{1.216116in}}%
\pgfpathlineto{\pgfqpoint{1.753849in}{1.214684in}}%
\pgfpathlineto{\pgfqpoint{1.740730in}{1.216481in}}%
\pgfpathclose%
\pgfpathmoveto{\pgfqpoint{0.485668in}{1.472251in}}%
\pgfpathlineto{\pgfqpoint{0.501324in}{1.467164in}}%
\pgfpathlineto{\pgfqpoint{0.516981in}{1.465714in}}%
\pgfpathlineto{\pgfqpoint{0.532637in}{1.467890in}}%
\pgfpathlineto{\pgfqpoint{0.548294in}{1.473706in}}%
\pgfpathlineto{\pgfqpoint{0.550679in}{1.475092in}}%
\pgfpathlineto{\pgfqpoint{0.563950in}{1.482218in}}%
\pgfpathlineto{\pgfqpoint{0.573037in}{1.488703in}}%
\pgfpathlineto{\pgfqpoint{0.579607in}{1.493593in}}%
\pgfpathlineto{\pgfqpoint{0.589638in}{1.502314in}}%
\pgfpathlineto{\pgfqpoint{0.595263in}{1.508026in}}%
\pgfpathlineto{\pgfqpoint{0.602723in}{1.515925in}}%
\pgfpathlineto{\pgfqpoint{0.610920in}{1.527463in}}%
\pgfpathlineto{\pgfqpoint{0.612514in}{1.529536in}}%
\pgfpathlineto{\pgfqpoint{0.619204in}{1.543148in}}%
\pgfpathlineto{\pgfqpoint{0.621708in}{1.556759in}}%
\pgfpathlineto{\pgfqpoint{0.620039in}{1.570370in}}%
\pgfpathlineto{\pgfqpoint{0.614188in}{1.583981in}}%
\pgfpathlineto{\pgfqpoint{0.610920in}{1.588537in}}%
\pgfpathlineto{\pgfqpoint{0.604937in}{1.597592in}}%
\pgfpathlineto{\pgfqpoint{0.595263in}{1.608268in}}%
\pgfpathlineto{\pgfqpoint{0.592498in}{1.611203in}}%
\pgfpathlineto{\pgfqpoint{0.579607in}{1.622653in}}%
\pgfpathlineto{\pgfqpoint{0.576769in}{1.624814in}}%
\pgfpathlineto{\pgfqpoint{0.563950in}{1.633987in}}%
\pgfpathlineto{\pgfqpoint{0.555709in}{1.638425in}}%
\pgfpathlineto{\pgfqpoint{0.548294in}{1.642649in}}%
\pgfpathlineto{\pgfqpoint{0.532637in}{1.648268in}}%
\pgfpathlineto{\pgfqpoint{0.516981in}{1.650371in}}%
\pgfpathlineto{\pgfqpoint{0.501324in}{1.648970in}}%
\pgfpathlineto{\pgfqpoint{0.485668in}{1.644055in}}%
\pgfpathlineto{\pgfqpoint{0.475073in}{1.638425in}}%
\pgfpathlineto{\pgfqpoint{0.470011in}{1.635890in}}%
\pgfpathlineto{\pgfqpoint{0.454354in}{1.625159in}}%
\pgfpathlineto{\pgfqpoint{0.453928in}{1.624814in}}%
\pgfpathlineto{\pgfqpoint{0.438698in}{1.611573in}}%
\pgfpathlineto{\pgfqpoint{0.438301in}{1.611203in}}%
\pgfpathlineto{\pgfqpoint{0.425958in}{1.597592in}}%
\pgfpathlineto{\pgfqpoint{0.423041in}{1.593191in}}%
\pgfpathlineto{\pgfqpoint{0.416566in}{1.583981in}}%
\pgfpathlineto{\pgfqpoint{0.410912in}{1.570370in}}%
\pgfpathlineto{\pgfqpoint{0.409300in}{1.556759in}}%
\pgfpathlineto{\pgfqpoint{0.411719in}{1.543148in}}%
\pgfpathlineto{\pgfqpoint{0.418183in}{1.529536in}}%
\pgfpathlineto{\pgfqpoint{0.423041in}{1.523090in}}%
\pgfpathlineto{\pgfqpoint{0.428146in}{1.515925in}}%
\pgfpathlineto{\pgfqpoint{0.438698in}{1.504781in}}%
\pgfpathlineto{\pgfqpoint{0.441184in}{1.502314in}}%
\pgfpathlineto{\pgfqpoint{0.454354in}{1.491108in}}%
\pgfpathlineto{\pgfqpoint{0.457730in}{1.488703in}}%
\pgfpathlineto{\pgfqpoint{0.470011in}{1.480294in}}%
\pgfpathlineto{\pgfqpoint{0.480427in}{1.475092in}}%
\pgfpathlineto{\pgfqpoint{0.485668in}{1.472251in}}%
\pgfpathclose%
\pgfpathmoveto{\pgfqpoint{0.491271in}{1.488703in}}%
\pgfpathlineto{\pgfqpoint{0.485668in}{1.490471in}}%
\pgfpathlineto{\pgfqpoint{0.470011in}{1.498810in}}%
\pgfpathlineto{\pgfqpoint{0.465150in}{1.502314in}}%
\pgfpathlineto{\pgfqpoint{0.454354in}{1.511297in}}%
\pgfpathlineto{\pgfqpoint{0.449653in}{1.515925in}}%
\pgfpathlineto{\pgfqpoint{0.439173in}{1.529536in}}%
\pgfpathlineto{\pgfqpoint{0.438698in}{1.530513in}}%
\pgfpathlineto{\pgfqpoint{0.432746in}{1.543147in}}%
\pgfpathlineto{\pgfqpoint{0.430341in}{1.556759in}}%
\pgfpathlineto{\pgfqpoint{0.431944in}{1.570370in}}%
\pgfpathlineto{\pgfqpoint{0.437564in}{1.583981in}}%
\pgfpathlineto{\pgfqpoint{0.438698in}{1.585641in}}%
\pgfpathlineto{\pgfqpoint{0.447237in}{1.597592in}}%
\pgfpathlineto{\pgfqpoint{0.454354in}{1.604894in}}%
\pgfpathlineto{\pgfqpoint{0.461611in}{1.611203in}}%
\pgfpathlineto{\pgfqpoint{0.470011in}{1.617390in}}%
\pgfpathlineto{\pgfqpoint{0.483757in}{1.624814in}}%
\pgfpathlineto{\pgfqpoint{0.485668in}{1.625800in}}%
\pgfpathlineto{\pgfqpoint{0.501324in}{1.630686in}}%
\pgfpathlineto{\pgfqpoint{0.516981in}{1.632079in}}%
\pgfpathlineto{\pgfqpoint{0.532637in}{1.629989in}}%
\pgfpathlineto{\pgfqpoint{0.547170in}{1.624814in}}%
\pgfpathlineto{\pgfqpoint{0.548294in}{1.624401in}}%
\pgfpathlineto{\pgfqpoint{0.563950in}{1.615290in}}%
\pgfpathlineto{\pgfqpoint{0.569274in}{1.611203in}}%
\pgfpathlineto{\pgfqpoint{0.579607in}{1.601817in}}%
\pgfpathlineto{\pgfqpoint{0.583638in}{1.597592in}}%
\pgfpathlineto{\pgfqpoint{0.593230in}{1.583981in}}%
\pgfpathlineto{\pgfqpoint{0.595263in}{1.579110in}}%
\pgfpathlineto{\pgfqpoint{0.598881in}{1.570370in}}%
\pgfpathlineto{\pgfqpoint{0.600502in}{1.556759in}}%
\pgfpathlineto{\pgfqpoint{0.598070in}{1.543148in}}%
\pgfpathlineto{\pgfqpoint{0.595263in}{1.537175in}}%
\pgfpathlineto{\pgfqpoint{0.591631in}{1.529536in}}%
\pgfpathlineto{\pgfqpoint{0.581243in}{1.515925in}}%
\pgfpathlineto{\pgfqpoint{0.579607in}{1.514279in}}%
\pgfpathlineto{\pgfqpoint{0.565844in}{1.502314in}}%
\pgfpathlineto{\pgfqpoint{0.563950in}{1.500891in}}%
\pgfpathlineto{\pgfqpoint{0.548294in}{1.491861in}}%
\pgfpathlineto{\pgfqpoint{0.539507in}{1.488703in}}%
\pgfpathlineto{\pgfqpoint{0.532637in}{1.486263in}}%
\pgfpathlineto{\pgfqpoint{0.516981in}{1.484149in}}%
\pgfpathlineto{\pgfqpoint{0.501324in}{1.485558in}}%
\pgfpathlineto{\pgfqpoint{0.491271in}{1.488703in}}%
\pgfpathclose%
\pgfpathmoveto{\pgfqpoint{0.798799in}{1.470942in}}%
\pgfpathlineto{\pgfqpoint{0.814455in}{1.466584in}}%
\pgfpathlineto{\pgfqpoint{0.830112in}{1.465859in}}%
\pgfpathlineto{\pgfqpoint{0.845769in}{1.468762in}}%
\pgfpathlineto{\pgfqpoint{0.860926in}{1.475092in}}%
\pgfpathlineto{\pgfqpoint{0.861425in}{1.475282in}}%
\pgfpathlineto{\pgfqpoint{0.877082in}{1.484268in}}%
\pgfpathlineto{\pgfqpoint{0.883070in}{1.488703in}}%
\pgfpathlineto{\pgfqpoint{0.892738in}{1.496188in}}%
\pgfpathlineto{\pgfqpoint{0.899661in}{1.502314in}}%
\pgfpathlineto{\pgfqpoint{0.908395in}{1.511354in}}%
\pgfpathlineto{\pgfqpoint{0.912717in}{1.515925in}}%
\pgfpathlineto{\pgfqpoint{0.922475in}{1.529536in}}%
\pgfpathlineto{\pgfqpoint{0.924051in}{1.533035in}}%
\pgfpathlineto{\pgfqpoint{0.929185in}{1.543148in}}%
\pgfpathlineto{\pgfqpoint{0.931790in}{1.556759in}}%
\pgfpathlineto{\pgfqpoint{0.930053in}{1.570370in}}%
\pgfpathlineto{\pgfqpoint{0.924051in}{1.583793in}}%
\pgfpathlineto{\pgfqpoint{0.923977in}{1.583981in}}%
\pgfpathlineto{\pgfqpoint{0.914966in}{1.597592in}}%
\pgfpathlineto{\pgfqpoint{0.908395in}{1.604835in}}%
\pgfpathlineto{\pgfqpoint{0.902506in}{1.611203in}}%
\pgfpathlineto{\pgfqpoint{0.892738in}{1.620035in}}%
\pgfpathlineto{\pgfqpoint{0.886706in}{1.624814in}}%
\pgfpathlineto{\pgfqpoint{0.877082in}{1.631961in}}%
\pgfpathlineto{\pgfqpoint{0.865840in}{1.638425in}}%
\pgfpathlineto{\pgfqpoint{0.861425in}{1.641102in}}%
\pgfpathlineto{\pgfqpoint{0.845769in}{1.647426in}}%
\pgfpathlineto{\pgfqpoint{0.830112in}{1.650231in}}%
\pgfpathlineto{\pgfqpoint{0.814455in}{1.649531in}}%
\pgfpathlineto{\pgfqpoint{0.798799in}{1.645320in}}%
\pgfpathlineto{\pgfqpoint{0.784782in}{1.638425in}}%
\pgfpathlineto{\pgfqpoint{0.783142in}{1.637668in}}%
\pgfpathlineto{\pgfqpoint{0.767486in}{1.627545in}}%
\pgfpathlineto{\pgfqpoint{0.764030in}{1.624814in}}%
\pgfpathlineto{\pgfqpoint{0.751829in}{1.614489in}}%
\pgfpathlineto{\pgfqpoint{0.748282in}{1.611203in}}%
\pgfpathlineto{\pgfqpoint{0.736173in}{1.597759in}}%
\pgfpathlineto{\pgfqpoint{0.736019in}{1.597592in}}%
\pgfpathlineto{\pgfqpoint{0.726597in}{1.583981in}}%
\pgfpathlineto{\pgfqpoint{0.721106in}{1.570370in}}%
\pgfpathlineto{\pgfqpoint{0.720516in}{1.565261in}}%
\pgfpathlineto{\pgfqpoint{0.719324in}{1.556759in}}%
\pgfpathlineto{\pgfqpoint{0.720516in}{1.551071in}}%
\pgfpathlineto{\pgfqpoint{0.721890in}{1.543148in}}%
\pgfpathlineto{\pgfqpoint{0.728167in}{1.529536in}}%
\pgfpathlineto{\pgfqpoint{0.736173in}{1.518755in}}%
\pgfpathlineto{\pgfqpoint{0.738203in}{1.515925in}}%
\pgfpathlineto{\pgfqpoint{0.751136in}{1.502314in}}%
\pgfpathlineto{\pgfqpoint{0.751829in}{1.501685in}}%
\pgfpathlineto{\pgfqpoint{0.767486in}{1.488735in}}%
\pgfpathlineto{\pgfqpoint{0.767533in}{1.488703in}}%
\pgfpathlineto{\pgfqpoint{0.783142in}{1.478495in}}%
\pgfpathlineto{\pgfqpoint{0.790528in}{1.475092in}}%
\pgfpathlineto{\pgfqpoint{0.798799in}{1.470942in}}%
\pgfpathclose%
\pgfpathmoveto{\pgfqpoint{0.800706in}{1.488703in}}%
\pgfpathlineto{\pgfqpoint{0.798799in}{1.489220in}}%
\pgfpathlineto{\pgfqpoint{0.783142in}{1.496865in}}%
\pgfpathlineto{\pgfqpoint{0.775228in}{1.502314in}}%
\pgfpathlineto{\pgfqpoint{0.767486in}{1.508451in}}%
\pgfpathlineto{\pgfqpoint{0.759699in}{1.515925in}}%
\pgfpathlineto{\pgfqpoint{0.751829in}{1.525896in}}%
\pgfpathlineto{\pgfqpoint{0.749147in}{1.529536in}}%
\pgfpathlineto{\pgfqpoint{0.742769in}{1.543147in}}%
\pgfpathlineto{\pgfqpoint{0.740382in}{1.556759in}}%
\pgfpathlineto{\pgfqpoint{0.741973in}{1.570370in}}%
\pgfpathlineto{\pgfqpoint{0.747551in}{1.583981in}}%
\pgfpathlineto{\pgfqpoint{0.751829in}{1.590205in}}%
\pgfpathlineto{\pgfqpoint{0.757252in}{1.597592in}}%
\pgfpathlineto{\pgfqpoint{0.767486in}{1.607831in}}%
\pgfpathlineto{\pgfqpoint{0.771558in}{1.611203in}}%
\pgfpathlineto{\pgfqpoint{0.783142in}{1.619352in}}%
\pgfpathlineto{\pgfqpoint{0.794103in}{1.624814in}}%
\pgfpathlineto{\pgfqpoint{0.798799in}{1.627057in}}%
\pgfpathlineto{\pgfqpoint{0.814455in}{1.631243in}}%
\pgfpathlineto{\pgfqpoint{0.830112in}{1.631940in}}%
\pgfpathlineto{\pgfqpoint{0.845769in}{1.629152in}}%
\pgfpathlineto{\pgfqpoint{0.856678in}{1.624814in}}%
\pgfpathlineto{\pgfqpoint{0.861425in}{1.622858in}}%
\pgfpathlineto{\pgfqpoint{0.877082in}{1.613054in}}%
\pgfpathlineto{\pgfqpoint{0.879405in}{1.611203in}}%
\pgfpathlineto{\pgfqpoint{0.892738in}{1.598606in}}%
\pgfpathlineto{\pgfqpoint{0.893688in}{1.597592in}}%
\pgfpathlineto{\pgfqpoint{0.903236in}{1.583981in}}%
\pgfpathlineto{\pgfqpoint{0.908395in}{1.571383in}}%
\pgfpathlineto{\pgfqpoint{0.908815in}{1.570370in}}%
\pgfpathlineto{\pgfqpoint{0.910462in}{1.556759in}}%
\pgfpathlineto{\pgfqpoint{0.908395in}{1.545354in}}%
\pgfpathlineto{\pgfqpoint{0.908005in}{1.543148in}}%
\pgfpathlineto{\pgfqpoint{0.901644in}{1.529536in}}%
\pgfpathlineto{\pgfqpoint{0.892738in}{1.517740in}}%
\pgfpathlineto{\pgfqpoint{0.891208in}{1.515925in}}%
\pgfpathlineto{\pgfqpoint{0.877082in}{1.503180in}}%
\pgfpathlineto{\pgfqpoint{0.875858in}{1.502314in}}%
\pgfpathlineto{\pgfqpoint{0.861425in}{1.493390in}}%
\pgfpathlineto{\pgfqpoint{0.849785in}{1.488703in}}%
\pgfpathlineto{\pgfqpoint{0.845769in}{1.487110in}}%
\pgfpathlineto{\pgfqpoint{0.830112in}{1.484289in}}%
\pgfpathlineto{\pgfqpoint{0.814455in}{1.484994in}}%
\pgfpathlineto{\pgfqpoint{0.800706in}{1.488703in}}%
\pgfpathclose%
\pgfpathmoveto{\pgfqpoint{1.111930in}{1.469779in}}%
\pgfpathlineto{\pgfqpoint{1.127587in}{1.466149in}}%
\pgfpathlineto{\pgfqpoint{1.143243in}{1.466149in}}%
\pgfpathlineto{\pgfqpoint{1.158900in}{1.469779in}}%
\pgfpathlineto{\pgfqpoint{1.170444in}{1.475092in}}%
\pgfpathlineto{\pgfqpoint{1.174556in}{1.476825in}}%
\pgfpathlineto{\pgfqpoint{1.190213in}{1.486441in}}%
\pgfpathlineto{\pgfqpoint{1.193168in}{1.488703in}}%
\pgfpathlineto{\pgfqpoint{1.205870in}{1.498887in}}%
\pgfpathlineto{\pgfqpoint{1.209688in}{1.502314in}}%
\pgfpathlineto{\pgfqpoint{1.221526in}{1.514754in}}%
\pgfpathlineto{\pgfqpoint{1.222640in}{1.515925in}}%
\pgfpathlineto{\pgfqpoint{1.232596in}{1.529536in}}%
\pgfpathlineto{\pgfqpoint{1.237183in}{1.539656in}}%
\pgfpathlineto{\pgfqpoint{1.239012in}{1.543148in}}%
\pgfpathlineto{\pgfqpoint{1.241738in}{1.556759in}}%
\pgfpathlineto{\pgfqpoint{1.239921in}{1.570370in}}%
\pgfpathlineto{\pgfqpoint{1.237183in}{1.576304in}}%
\pgfpathlineto{\pgfqpoint{1.234128in}{1.583981in}}%
\pgfpathlineto{\pgfqpoint{1.224935in}{1.597592in}}%
\pgfpathlineto{\pgfqpoint{1.221526in}{1.601328in}}%
\pgfpathlineto{\pgfqpoint{1.212532in}{1.611203in}}%
\pgfpathlineto{\pgfqpoint{1.205870in}{1.617312in}}%
\pgfpathlineto{\pgfqpoint{1.196726in}{1.624814in}}%
\pgfpathlineto{\pgfqpoint{1.190213in}{1.629813in}}%
\pgfpathlineto{\pgfqpoint{1.176098in}{1.638425in}}%
\pgfpathlineto{\pgfqpoint{1.174556in}{1.639414in}}%
\pgfpathlineto{\pgfqpoint{1.158900in}{1.646443in}}%
\pgfpathlineto{\pgfqpoint{1.143243in}{1.649951in}}%
\pgfpathlineto{\pgfqpoint{1.127587in}{1.649951in}}%
\pgfpathlineto{\pgfqpoint{1.111930in}{1.646443in}}%
\pgfpathlineto{\pgfqpoint{1.096274in}{1.639414in}}%
\pgfpathlineto{\pgfqpoint{1.094732in}{1.638425in}}%
\pgfpathlineto{\pgfqpoint{1.080617in}{1.629813in}}%
\pgfpathlineto{\pgfqpoint{1.074104in}{1.624814in}}%
\pgfpathlineto{\pgfqpoint{1.064960in}{1.617312in}}%
\pgfpathlineto{\pgfqpoint{1.058298in}{1.611203in}}%
\pgfpathlineto{\pgfqpoint{1.049304in}{1.601328in}}%
\pgfpathlineto{\pgfqpoint{1.045895in}{1.597592in}}%
\pgfpathlineto{\pgfqpoint{1.036702in}{1.583981in}}%
\pgfpathlineto{\pgfqpoint{1.033647in}{1.576304in}}%
\pgfpathlineto{\pgfqpoint{1.030909in}{1.570370in}}%
\pgfpathlineto{\pgfqpoint{1.029092in}{1.556759in}}%
\pgfpathlineto{\pgfqpoint{1.031818in}{1.543147in}}%
\pgfpathlineto{\pgfqpoint{1.033647in}{1.539656in}}%
\pgfpathlineto{\pgfqpoint{1.038234in}{1.529536in}}%
\pgfpathlineto{\pgfqpoint{1.048190in}{1.515925in}}%
\pgfpathlineto{\pgfqpoint{1.049304in}{1.514754in}}%
\pgfpathlineto{\pgfqpoint{1.061142in}{1.502314in}}%
\pgfpathlineto{\pgfqpoint{1.064960in}{1.498887in}}%
\pgfpathlineto{\pgfqpoint{1.077662in}{1.488703in}}%
\pgfpathlineto{\pgfqpoint{1.080617in}{1.486441in}}%
\pgfpathlineto{\pgfqpoint{1.096274in}{1.476825in}}%
\pgfpathlineto{\pgfqpoint{1.100386in}{1.475092in}}%
\pgfpathlineto{\pgfqpoint{1.111930in}{1.469779in}}%
\pgfpathclose%
\pgfpathmoveto{\pgfqpoint{1.110547in}{1.488703in}}%
\pgfpathlineto{\pgfqpoint{1.096274in}{1.495058in}}%
\pgfpathlineto{\pgfqpoint{1.085184in}{1.502314in}}%
\pgfpathlineto{\pgfqpoint{1.080617in}{1.505744in}}%
\pgfpathlineto{\pgfqpoint{1.069698in}{1.515925in}}%
\pgfpathlineto{\pgfqpoint{1.064960in}{1.521744in}}%
\pgfpathlineto{\pgfqpoint{1.059160in}{1.529536in}}%
\pgfpathlineto{\pgfqpoint{1.052803in}{1.543147in}}%
\pgfpathlineto{\pgfqpoint{1.050424in}{1.556759in}}%
\pgfpathlineto{\pgfqpoint{1.052010in}{1.570370in}}%
\pgfpathlineto{\pgfqpoint{1.057570in}{1.583981in}}%
\pgfpathlineto{\pgfqpoint{1.064960in}{1.594623in}}%
\pgfpathlineto{\pgfqpoint{1.067208in}{1.597592in}}%
\pgfpathlineto{\pgfqpoint{1.080617in}{1.610623in}}%
\pgfpathlineto{\pgfqpoint{1.081356in}{1.611203in}}%
\pgfpathlineto{\pgfqpoint{1.096274in}{1.621175in}}%
\pgfpathlineto{\pgfqpoint{1.104261in}{1.624814in}}%
\pgfpathlineto{\pgfqpoint{1.111930in}{1.628174in}}%
\pgfpathlineto{\pgfqpoint{1.127587in}{1.631661in}}%
\pgfpathlineto{\pgfqpoint{1.143243in}{1.631661in}}%
\pgfpathlineto{\pgfqpoint{1.158900in}{1.628174in}}%
\pgfpathlineto{\pgfqpoint{1.166569in}{1.624814in}}%
\pgfpathlineto{\pgfqpoint{1.174556in}{1.621175in}}%
\pgfpathlineto{\pgfqpoint{1.189474in}{1.611203in}}%
\pgfpathlineto{\pgfqpoint{1.190213in}{1.610623in}}%
\pgfpathlineto{\pgfqpoint{1.203622in}{1.597592in}}%
\pgfpathlineto{\pgfqpoint{1.205870in}{1.594623in}}%
\pgfpathlineto{\pgfqpoint{1.213260in}{1.583981in}}%
\pgfpathlineto{\pgfqpoint{1.218820in}{1.570370in}}%
\pgfpathlineto{\pgfqpoint{1.220406in}{1.556759in}}%
\pgfpathlineto{\pgfqpoint{1.218027in}{1.543148in}}%
\pgfpathlineto{\pgfqpoint{1.211670in}{1.529536in}}%
\pgfpathlineto{\pgfqpoint{1.205870in}{1.521744in}}%
\pgfpathlineto{\pgfqpoint{1.201132in}{1.515925in}}%
\pgfpathlineto{\pgfqpoint{1.190213in}{1.505744in}}%
\pgfpathlineto{\pgfqpoint{1.185646in}{1.502314in}}%
\pgfpathlineto{\pgfqpoint{1.174556in}{1.495058in}}%
\pgfpathlineto{\pgfqpoint{1.160283in}{1.488703in}}%
\pgfpathlineto{\pgfqpoint{1.158900in}{1.488099in}}%
\pgfpathlineto{\pgfqpoint{1.143243in}{1.484571in}}%
\pgfpathlineto{\pgfqpoint{1.127587in}{1.484571in}}%
\pgfpathlineto{\pgfqpoint{1.111930in}{1.488099in}}%
\pgfpathlineto{\pgfqpoint{1.110547in}{1.488703in}}%
\pgfpathclose%
\pgfpathmoveto{\pgfqpoint{1.425061in}{1.468762in}}%
\pgfpathlineto{\pgfqpoint{1.440718in}{1.465859in}}%
\pgfpathlineto{\pgfqpoint{1.456375in}{1.466584in}}%
\pgfpathlineto{\pgfqpoint{1.472031in}{1.470942in}}%
\pgfpathlineto{\pgfqpoint{1.480302in}{1.475092in}}%
\pgfpathlineto{\pgfqpoint{1.487688in}{1.478495in}}%
\pgfpathlineto{\pgfqpoint{1.503297in}{1.488703in}}%
\pgfpathlineto{\pgfqpoint{1.503344in}{1.488735in}}%
\pgfpathlineto{\pgfqpoint{1.519001in}{1.501685in}}%
\pgfpathlineto{\pgfqpoint{1.519694in}{1.502314in}}%
\pgfpathlineto{\pgfqpoint{1.532627in}{1.515925in}}%
\pgfpathlineto{\pgfqpoint{1.534657in}{1.518755in}}%
\pgfpathlineto{\pgfqpoint{1.542663in}{1.529536in}}%
\pgfpathlineto{\pgfqpoint{1.548940in}{1.543148in}}%
\pgfpathlineto{\pgfqpoint{1.550314in}{1.551071in}}%
\pgfpathlineto{\pgfqpoint{1.551506in}{1.556759in}}%
\pgfpathlineto{\pgfqpoint{1.550314in}{1.565261in}}%
\pgfpathlineto{\pgfqpoint{1.549724in}{1.570370in}}%
\pgfpathlineto{\pgfqpoint{1.544233in}{1.583981in}}%
\pgfpathlineto{\pgfqpoint{1.534811in}{1.597592in}}%
\pgfpathlineto{\pgfqpoint{1.534657in}{1.597759in}}%
\pgfpathlineto{\pgfqpoint{1.522548in}{1.611203in}}%
\pgfpathlineto{\pgfqpoint{1.519001in}{1.614489in}}%
\pgfpathlineto{\pgfqpoint{1.506800in}{1.624814in}}%
\pgfpathlineto{\pgfqpoint{1.503344in}{1.627545in}}%
\pgfpathlineto{\pgfqpoint{1.487688in}{1.637668in}}%
\pgfpathlineto{\pgfqpoint{1.486048in}{1.638425in}}%
\pgfpathlineto{\pgfqpoint{1.472031in}{1.645320in}}%
\pgfpathlineto{\pgfqpoint{1.456375in}{1.649531in}}%
\pgfpathlineto{\pgfqpoint{1.440718in}{1.650231in}}%
\pgfpathlineto{\pgfqpoint{1.425061in}{1.647426in}}%
\pgfpathlineto{\pgfqpoint{1.409405in}{1.641102in}}%
\pgfpathlineto{\pgfqpoint{1.404990in}{1.638425in}}%
\pgfpathlineto{\pgfqpoint{1.393748in}{1.631961in}}%
\pgfpathlineto{\pgfqpoint{1.384124in}{1.624814in}}%
\pgfpathlineto{\pgfqpoint{1.378092in}{1.620035in}}%
\pgfpathlineto{\pgfqpoint{1.368324in}{1.611203in}}%
\pgfpathlineto{\pgfqpoint{1.362435in}{1.604835in}}%
\pgfpathlineto{\pgfqpoint{1.355864in}{1.597592in}}%
\pgfpathlineto{\pgfqpoint{1.346853in}{1.583981in}}%
\pgfpathlineto{\pgfqpoint{1.346779in}{1.583793in}}%
\pgfpathlineto{\pgfqpoint{1.340777in}{1.570370in}}%
\pgfpathlineto{\pgfqpoint{1.339040in}{1.556759in}}%
\pgfpathlineto{\pgfqpoint{1.341645in}{1.543148in}}%
\pgfpathlineto{\pgfqpoint{1.346779in}{1.533035in}}%
\pgfpathlineto{\pgfqpoint{1.348355in}{1.529536in}}%
\pgfpathlineto{\pgfqpoint{1.358113in}{1.515925in}}%
\pgfpathlineto{\pgfqpoint{1.362435in}{1.511354in}}%
\pgfpathlineto{\pgfqpoint{1.371169in}{1.502314in}}%
\pgfpathlineto{\pgfqpoint{1.378092in}{1.496188in}}%
\pgfpathlineto{\pgfqpoint{1.387760in}{1.488703in}}%
\pgfpathlineto{\pgfqpoint{1.393748in}{1.484268in}}%
\pgfpathlineto{\pgfqpoint{1.409405in}{1.475282in}}%
\pgfpathlineto{\pgfqpoint{1.409904in}{1.475092in}}%
\pgfpathlineto{\pgfqpoint{1.425061in}{1.468762in}}%
\pgfpathclose%
\pgfpathmoveto{\pgfqpoint{1.421045in}{1.488703in}}%
\pgfpathlineto{\pgfqpoint{1.409405in}{1.493390in}}%
\pgfpathlineto{\pgfqpoint{1.394972in}{1.502314in}}%
\pgfpathlineto{\pgfqpoint{1.393748in}{1.503180in}}%
\pgfpathlineto{\pgfqpoint{1.379622in}{1.515925in}}%
\pgfpathlineto{\pgfqpoint{1.378092in}{1.517740in}}%
\pgfpathlineto{\pgfqpoint{1.369186in}{1.529536in}}%
\pgfpathlineto{\pgfqpoint{1.362825in}{1.543147in}}%
\pgfpathlineto{\pgfqpoint{1.362435in}{1.545354in}}%
\pgfpathlineto{\pgfqpoint{1.360368in}{1.556759in}}%
\pgfpathlineto{\pgfqpoint{1.362015in}{1.570370in}}%
\pgfpathlineto{\pgfqpoint{1.362435in}{1.571383in}}%
\pgfpathlineto{\pgfqpoint{1.367594in}{1.583981in}}%
\pgfpathlineto{\pgfqpoint{1.377142in}{1.597592in}}%
\pgfpathlineto{\pgfqpoint{1.378092in}{1.598606in}}%
\pgfpathlineto{\pgfqpoint{1.391425in}{1.611203in}}%
\pgfpathlineto{\pgfqpoint{1.393748in}{1.613054in}}%
\pgfpathlineto{\pgfqpoint{1.409405in}{1.622858in}}%
\pgfpathlineto{\pgfqpoint{1.414152in}{1.624814in}}%
\pgfpathlineto{\pgfqpoint{1.425061in}{1.629152in}}%
\pgfpathlineto{\pgfqpoint{1.440718in}{1.631940in}}%
\pgfpathlineto{\pgfqpoint{1.456375in}{1.631243in}}%
\pgfpathlineto{\pgfqpoint{1.472031in}{1.627057in}}%
\pgfpathlineto{\pgfqpoint{1.476727in}{1.624814in}}%
\pgfpathlineto{\pgfqpoint{1.487688in}{1.619352in}}%
\pgfpathlineto{\pgfqpoint{1.499272in}{1.611203in}}%
\pgfpathlineto{\pgfqpoint{1.503344in}{1.607831in}}%
\pgfpathlineto{\pgfqpoint{1.513578in}{1.597592in}}%
\pgfpathlineto{\pgfqpoint{1.519001in}{1.590205in}}%
\pgfpathlineto{\pgfqpoint{1.523279in}{1.583981in}}%
\pgfpathlineto{\pgfqpoint{1.528857in}{1.570370in}}%
\pgfpathlineto{\pgfqpoint{1.530448in}{1.556759in}}%
\pgfpathlineto{\pgfqpoint{1.528061in}{1.543148in}}%
\pgfpathlineto{\pgfqpoint{1.521683in}{1.529536in}}%
\pgfpathlineto{\pgfqpoint{1.519001in}{1.525896in}}%
\pgfpathlineto{\pgfqpoint{1.511131in}{1.515925in}}%
\pgfpathlineto{\pgfqpoint{1.503344in}{1.508451in}}%
\pgfpathlineto{\pgfqpoint{1.495602in}{1.502314in}}%
\pgfpathlineto{\pgfqpoint{1.487688in}{1.496865in}}%
\pgfpathlineto{\pgfqpoint{1.472031in}{1.489220in}}%
\pgfpathlineto{\pgfqpoint{1.470124in}{1.488703in}}%
\pgfpathlineto{\pgfqpoint{1.456375in}{1.484994in}}%
\pgfpathlineto{\pgfqpoint{1.440718in}{1.484289in}}%
\pgfpathlineto{\pgfqpoint{1.425061in}{1.487110in}}%
\pgfpathlineto{\pgfqpoint{1.421045in}{1.488703in}}%
\pgfpathclose%
\pgfpathmoveto{\pgfqpoint{1.722536in}{1.473706in}}%
\pgfpathlineto{\pgfqpoint{1.738193in}{1.467890in}}%
\pgfpathlineto{\pgfqpoint{1.753849in}{1.465714in}}%
\pgfpathlineto{\pgfqpoint{1.769506in}{1.467164in}}%
\pgfpathlineto{\pgfqpoint{1.785162in}{1.472251in}}%
\pgfpathlineto{\pgfqpoint{1.790403in}{1.475092in}}%
\pgfpathlineto{\pgfqpoint{1.800819in}{1.480294in}}%
\pgfpathlineto{\pgfqpoint{1.813100in}{1.488703in}}%
\pgfpathlineto{\pgfqpoint{1.816476in}{1.491108in}}%
\pgfpathlineto{\pgfqpoint{1.829646in}{1.502314in}}%
\pgfpathlineto{\pgfqpoint{1.832132in}{1.504781in}}%
\pgfpathlineto{\pgfqpoint{1.842684in}{1.515925in}}%
\pgfpathlineto{\pgfqpoint{1.847789in}{1.523090in}}%
\pgfpathlineto{\pgfqpoint{1.852647in}{1.529536in}}%
\pgfpathlineto{\pgfqpoint{1.859111in}{1.543148in}}%
\pgfpathlineto{\pgfqpoint{1.861530in}{1.556759in}}%
\pgfpathlineto{\pgfqpoint{1.859918in}{1.570370in}}%
\pgfpathlineto{\pgfqpoint{1.854264in}{1.583981in}}%
\pgfpathlineto{\pgfqpoint{1.847789in}{1.593191in}}%
\pgfpathlineto{\pgfqpoint{1.844872in}{1.597592in}}%
\pgfpathlineto{\pgfqpoint{1.832529in}{1.611203in}}%
\pgfpathlineto{\pgfqpoint{1.832132in}{1.611573in}}%
\pgfpathlineto{\pgfqpoint{1.816902in}{1.624814in}}%
\pgfpathlineto{\pgfqpoint{1.816476in}{1.625159in}}%
\pgfpathlineto{\pgfqpoint{1.800819in}{1.635890in}}%
\pgfpathlineto{\pgfqpoint{1.795757in}{1.638425in}}%
\pgfpathlineto{\pgfqpoint{1.785162in}{1.644055in}}%
\pgfpathlineto{\pgfqpoint{1.769506in}{1.648970in}}%
\pgfpathlineto{\pgfqpoint{1.753849in}{1.650371in}}%
\pgfpathlineto{\pgfqpoint{1.738193in}{1.648268in}}%
\pgfpathlineto{\pgfqpoint{1.722536in}{1.642649in}}%
\pgfpathlineto{\pgfqpoint{1.715121in}{1.638425in}}%
\pgfpathlineto{\pgfqpoint{1.706880in}{1.633987in}}%
\pgfpathlineto{\pgfqpoint{1.694061in}{1.624814in}}%
\pgfpathlineto{\pgfqpoint{1.691223in}{1.622653in}}%
\pgfpathlineto{\pgfqpoint{1.678332in}{1.611203in}}%
\pgfpathlineto{\pgfqpoint{1.675567in}{1.608268in}}%
\pgfpathlineto{\pgfqpoint{1.665893in}{1.597592in}}%
\pgfpathlineto{\pgfqpoint{1.659910in}{1.588537in}}%
\pgfpathlineto{\pgfqpoint{1.656642in}{1.583981in}}%
\pgfpathlineto{\pgfqpoint{1.650791in}{1.570370in}}%
\pgfpathlineto{\pgfqpoint{1.649122in}{1.556759in}}%
\pgfpathlineto{\pgfqpoint{1.651626in}{1.543148in}}%
\pgfpathlineto{\pgfqpoint{1.658316in}{1.529536in}}%
\pgfpathlineto{\pgfqpoint{1.659910in}{1.527463in}}%
\pgfpathlineto{\pgfqpoint{1.668107in}{1.515925in}}%
\pgfpathlineto{\pgfqpoint{1.675567in}{1.508026in}}%
\pgfpathlineto{\pgfqpoint{1.681192in}{1.502314in}}%
\pgfpathlineto{\pgfqpoint{1.691223in}{1.493593in}}%
\pgfpathlineto{\pgfqpoint{1.697793in}{1.488703in}}%
\pgfpathlineto{\pgfqpoint{1.706880in}{1.482218in}}%
\pgfpathlineto{\pgfqpoint{1.720151in}{1.475092in}}%
\pgfpathlineto{\pgfqpoint{1.722536in}{1.473706in}}%
\pgfpathclose%
\pgfpathmoveto{\pgfqpoint{1.731323in}{1.488703in}}%
\pgfpathlineto{\pgfqpoint{1.722536in}{1.491861in}}%
\pgfpathlineto{\pgfqpoint{1.706880in}{1.500891in}}%
\pgfpathlineto{\pgfqpoint{1.704986in}{1.502314in}}%
\pgfpathlineto{\pgfqpoint{1.691223in}{1.514279in}}%
\pgfpathlineto{\pgfqpoint{1.689587in}{1.515925in}}%
\pgfpathlineto{\pgfqpoint{1.679199in}{1.529536in}}%
\pgfpathlineto{\pgfqpoint{1.675567in}{1.537175in}}%
\pgfpathlineto{\pgfqpoint{1.672760in}{1.543148in}}%
\pgfpathlineto{\pgfqpoint{1.670328in}{1.556759in}}%
\pgfpathlineto{\pgfqpoint{1.671949in}{1.570370in}}%
\pgfpathlineto{\pgfqpoint{1.675567in}{1.579110in}}%
\pgfpathlineto{\pgfqpoint{1.677600in}{1.583981in}}%
\pgfpathlineto{\pgfqpoint{1.687192in}{1.597592in}}%
\pgfpathlineto{\pgfqpoint{1.691223in}{1.601817in}}%
\pgfpathlineto{\pgfqpoint{1.701556in}{1.611203in}}%
\pgfpathlineto{\pgfqpoint{1.706880in}{1.615290in}}%
\pgfpathlineto{\pgfqpoint{1.722536in}{1.624401in}}%
\pgfpathlineto{\pgfqpoint{1.723660in}{1.624814in}}%
\pgfpathlineto{\pgfqpoint{1.738193in}{1.629989in}}%
\pgfpathlineto{\pgfqpoint{1.753849in}{1.632079in}}%
\pgfpathlineto{\pgfqpoint{1.769506in}{1.630686in}}%
\pgfpathlineto{\pgfqpoint{1.785162in}{1.625800in}}%
\pgfpathlineto{\pgfqpoint{1.787073in}{1.624814in}}%
\pgfpathlineto{\pgfqpoint{1.800819in}{1.617390in}}%
\pgfpathlineto{\pgfqpoint{1.809219in}{1.611203in}}%
\pgfpathlineto{\pgfqpoint{1.816476in}{1.604894in}}%
\pgfpathlineto{\pgfqpoint{1.823593in}{1.597592in}}%
\pgfpathlineto{\pgfqpoint{1.832132in}{1.585641in}}%
\pgfpathlineto{\pgfqpoint{1.833266in}{1.583981in}}%
\pgfpathlineto{\pgfqpoint{1.838886in}{1.570370in}}%
\pgfpathlineto{\pgfqpoint{1.840489in}{1.556759in}}%
\pgfpathlineto{\pgfqpoint{1.838084in}{1.543148in}}%
\pgfpathlineto{\pgfqpoint{1.832132in}{1.530513in}}%
\pgfpathlineto{\pgfqpoint{1.831657in}{1.529536in}}%
\pgfpathlineto{\pgfqpoint{1.821177in}{1.515925in}}%
\pgfpathlineto{\pgfqpoint{1.816476in}{1.511297in}}%
\pgfpathlineto{\pgfqpoint{1.805680in}{1.502314in}}%
\pgfpathlineto{\pgfqpoint{1.800819in}{1.498810in}}%
\pgfpathlineto{\pgfqpoint{1.785162in}{1.490471in}}%
\pgfpathlineto{\pgfqpoint{1.779559in}{1.488703in}}%
\pgfpathlineto{\pgfqpoint{1.769506in}{1.485558in}}%
\pgfpathlineto{\pgfqpoint{1.753849in}{1.484149in}}%
\pgfpathlineto{\pgfqpoint{1.738193in}{1.486263in}}%
\pgfpathlineto{\pgfqpoint{1.731323in}{1.488703in}}%
\pgfpathclose%
\pgfusepath{fill}%
\end{pgfscope}%
\begin{pgfscope}%
\pgfpathrectangle{\pgfqpoint{0.360415in}{0.345370in}}{\pgfqpoint{1.550000in}{1.347500in}}%
\pgfusepath{clip}%
\pgfsetbuttcap%
\pgfsetroundjoin%
\definecolor{currentfill}{rgb}{0.794549,0.275770,0.473117}%
\pgfsetfillcolor{currentfill}%
\pgfsetlinewidth{0.000000pt}%
\definecolor{currentstroke}{rgb}{0.000000,0.000000,0.000000}%
\pgfsetstrokecolor{currentstroke}%
\pgfsetdash{}{0pt}%
\pgfpathmoveto{\pgfqpoint{0.485668in}{0.371010in}}%
\pgfpathlineto{\pgfqpoint{0.501324in}{0.362339in}}%
\pgfpathlineto{\pgfqpoint{0.516981in}{0.359866in}}%
\pgfpathlineto{\pgfqpoint{0.532637in}{0.363576in}}%
\pgfpathlineto{\pgfqpoint{0.546912in}{0.372592in}}%
\pgfpathlineto{\pgfqpoint{0.548294in}{0.373169in}}%
\pgfpathlineto{\pgfqpoint{0.563950in}{0.383519in}}%
\pgfpathlineto{\pgfqpoint{0.567054in}{0.386203in}}%
\pgfpathlineto{\pgfqpoint{0.579607in}{0.395351in}}%
\pgfpathlineto{\pgfqpoint{0.584842in}{0.399814in}}%
\pgfpathlineto{\pgfqpoint{0.595263in}{0.408412in}}%
\pgfpathlineto{\pgfqpoint{0.601015in}{0.413425in}}%
\pgfpathlineto{\pgfqpoint{0.610920in}{0.422737in}}%
\pgfpathlineto{\pgfqpoint{0.615762in}{0.427036in}}%
\pgfpathlineto{\pgfqpoint{0.626577in}{0.438507in}}%
\pgfpathlineto{\pgfqpoint{0.629057in}{0.440648in}}%
\pgfpathlineto{\pgfqpoint{0.640810in}{0.454259in}}%
\pgfpathlineto{\pgfqpoint{0.642233in}{0.457030in}}%
\pgfpathlineto{\pgfqpoint{0.651265in}{0.467870in}}%
\pgfpathlineto{\pgfqpoint{0.654517in}{0.481481in}}%
\pgfpathlineto{\pgfqpoint{0.649639in}{0.495092in}}%
\pgfpathlineto{\pgfqpoint{0.642233in}{0.502921in}}%
\pgfpathlineto{\pgfqpoint{0.638851in}{0.508703in}}%
\pgfpathlineto{\pgfqpoint{0.626577in}{0.521850in}}%
\pgfpathlineto{\pgfqpoint{0.626228in}{0.522314in}}%
\pgfpathlineto{\pgfqpoint{0.612799in}{0.535925in}}%
\pgfpathlineto{\pgfqpoint{0.610920in}{0.537558in}}%
\pgfpathlineto{\pgfqpoint{0.597882in}{0.549536in}}%
\pgfpathlineto{\pgfqpoint{0.595263in}{0.551813in}}%
\pgfpathlineto{\pgfqpoint{0.581485in}{0.563148in}}%
\pgfpathlineto{\pgfqpoint{0.579607in}{0.564781in}}%
\pgfpathlineto{\pgfqpoint{0.563950in}{0.576455in}}%
\pgfpathlineto{\pgfqpoint{0.563416in}{0.576759in}}%
\pgfpathlineto{\pgfqpoint{0.548294in}{0.587429in}}%
\pgfpathlineto{\pgfqpoint{0.541642in}{0.590370in}}%
\pgfpathlineto{\pgfqpoint{0.532637in}{0.596808in}}%
\pgfpathlineto{\pgfqpoint{0.516981in}{0.601049in}}%
\pgfpathlineto{\pgfqpoint{0.501324in}{0.598222in}}%
\pgfpathlineto{\pgfqpoint{0.488855in}{0.590370in}}%
\pgfpathlineto{\pgfqpoint{0.485668in}{0.589133in}}%
\pgfpathlineto{\pgfqpoint{0.470011in}{0.578915in}}%
\pgfpathlineto{\pgfqpoint{0.467549in}{0.576759in}}%
\pgfpathlineto{\pgfqpoint{0.454354in}{0.567357in}}%
\pgfpathlineto{\pgfqpoint{0.449409in}{0.563148in}}%
\pgfpathlineto{\pgfqpoint{0.438698in}{0.554537in}}%
\pgfpathlineto{\pgfqpoint{0.432931in}{0.549536in}}%
\pgfpathlineto{\pgfqpoint{0.423041in}{0.540476in}}%
\pgfpathlineto{\pgfqpoint{0.417908in}{0.535925in}}%
\pgfpathlineto{\pgfqpoint{0.407385in}{0.525012in}}%
\pgfpathlineto{\pgfqpoint{0.404297in}{0.522314in}}%
\pgfpathlineto{\pgfqpoint{0.392392in}{0.508703in}}%
\pgfpathlineto{\pgfqpoint{0.391728in}{0.507502in}}%
\pgfpathlineto{\pgfqpoint{0.381358in}{0.495092in}}%
\pgfpathlineto{\pgfqpoint{0.377090in}{0.481481in}}%
\pgfpathlineto{\pgfqpoint{0.379935in}{0.467870in}}%
\pgfpathlineto{\pgfqpoint{0.389908in}{0.454259in}}%
\pgfpathlineto{\pgfqpoint{0.391728in}{0.452756in}}%
\pgfpathlineto{\pgfqpoint{0.401553in}{0.440648in}}%
\pgfpathlineto{\pgfqpoint{0.407385in}{0.435337in}}%
\pgfpathlineto{\pgfqpoint{0.415045in}{0.427036in}}%
\pgfpathlineto{\pgfqpoint{0.423041in}{0.419793in}}%
\pgfpathlineto{\pgfqpoint{0.429834in}{0.413425in}}%
\pgfpathlineto{\pgfqpoint{0.438698in}{0.405720in}}%
\pgfpathlineto{\pgfqpoint{0.446023in}{0.399814in}}%
\pgfpathlineto{\pgfqpoint{0.454354in}{0.392863in}}%
\pgfpathlineto{\pgfqpoint{0.463902in}{0.386203in}}%
\pgfpathlineto{\pgfqpoint{0.470011in}{0.381133in}}%
\pgfpathlineto{\pgfqpoint{0.483939in}{0.372592in}}%
\pgfpathlineto{\pgfqpoint{0.485668in}{0.371010in}}%
\pgfpathclose%
\pgfpathmoveto{\pgfqpoint{0.475073in}{0.399814in}}%
\pgfpathlineto{\pgfqpoint{0.470011in}{0.402349in}}%
\pgfpathlineto{\pgfqpoint{0.454354in}{0.413080in}}%
\pgfpathlineto{\pgfqpoint{0.453928in}{0.413425in}}%
\pgfpathlineto{\pgfqpoint{0.438698in}{0.426666in}}%
\pgfpathlineto{\pgfqpoint{0.438301in}{0.427036in}}%
\pgfpathlineto{\pgfqpoint{0.425958in}{0.440648in}}%
\pgfpathlineto{\pgfqpoint{0.423041in}{0.445048in}}%
\pgfpathlineto{\pgfqpoint{0.416566in}{0.454259in}}%
\pgfpathlineto{\pgfqpoint{0.410912in}{0.467870in}}%
\pgfpathlineto{\pgfqpoint{0.409300in}{0.481481in}}%
\pgfpathlineto{\pgfqpoint{0.411719in}{0.495092in}}%
\pgfpathlineto{\pgfqpoint{0.418183in}{0.508703in}}%
\pgfpathlineto{\pgfqpoint{0.423041in}{0.515150in}}%
\pgfpathlineto{\pgfqpoint{0.428146in}{0.522314in}}%
\pgfpathlineto{\pgfqpoint{0.438698in}{0.533458in}}%
\pgfpathlineto{\pgfqpoint{0.441184in}{0.535925in}}%
\pgfpathlineto{\pgfqpoint{0.454354in}{0.547132in}}%
\pgfpathlineto{\pgfqpoint{0.457730in}{0.549536in}}%
\pgfpathlineto{\pgfqpoint{0.470011in}{0.557946in}}%
\pgfpathlineto{\pgfqpoint{0.480427in}{0.563148in}}%
\pgfpathlineto{\pgfqpoint{0.485668in}{0.565988in}}%
\pgfpathlineto{\pgfqpoint{0.501324in}{0.571075in}}%
\pgfpathlineto{\pgfqpoint{0.516981in}{0.572526in}}%
\pgfpathlineto{\pgfqpoint{0.532637in}{0.570349in}}%
\pgfpathlineto{\pgfqpoint{0.548294in}{0.564533in}}%
\pgfpathlineto{\pgfqpoint{0.550679in}{0.563148in}}%
\pgfpathlineto{\pgfqpoint{0.563950in}{0.556021in}}%
\pgfpathlineto{\pgfqpoint{0.573037in}{0.549536in}}%
\pgfpathlineto{\pgfqpoint{0.579607in}{0.544646in}}%
\pgfpathlineto{\pgfqpoint{0.589638in}{0.535925in}}%
\pgfpathlineto{\pgfqpoint{0.595263in}{0.530213in}}%
\pgfpathlineto{\pgfqpoint{0.602723in}{0.522314in}}%
\pgfpathlineto{\pgfqpoint{0.610920in}{0.510777in}}%
\pgfpathlineto{\pgfqpoint{0.612514in}{0.508703in}}%
\pgfpathlineto{\pgfqpoint{0.619204in}{0.495092in}}%
\pgfpathlineto{\pgfqpoint{0.621708in}{0.481481in}}%
\pgfpathlineto{\pgfqpoint{0.620039in}{0.467870in}}%
\pgfpathlineto{\pgfqpoint{0.614188in}{0.454259in}}%
\pgfpathlineto{\pgfqpoint{0.610920in}{0.449702in}}%
\pgfpathlineto{\pgfqpoint{0.604937in}{0.440648in}}%
\pgfpathlineto{\pgfqpoint{0.595263in}{0.429971in}}%
\pgfpathlineto{\pgfqpoint{0.592498in}{0.427036in}}%
\pgfpathlineto{\pgfqpoint{0.579607in}{0.415586in}}%
\pgfpathlineto{\pgfqpoint{0.576769in}{0.413425in}}%
\pgfpathlineto{\pgfqpoint{0.563950in}{0.404252in}}%
\pgfpathlineto{\pgfqpoint{0.555709in}{0.399814in}}%
\pgfpathlineto{\pgfqpoint{0.548294in}{0.395591in}}%
\pgfpathlineto{\pgfqpoint{0.532637in}{0.389971in}}%
\pgfpathlineto{\pgfqpoint{0.516981in}{0.387868in}}%
\pgfpathlineto{\pgfqpoint{0.501324in}{0.389270in}}%
\pgfpathlineto{\pgfqpoint{0.485668in}{0.394185in}}%
\pgfpathlineto{\pgfqpoint{0.475073in}{0.399814in}}%
\pgfpathclose%
\pgfpathmoveto{\pgfqpoint{0.798799in}{0.368779in}}%
\pgfpathlineto{\pgfqpoint{0.814455in}{0.361350in}}%
\pgfpathlineto{\pgfqpoint{0.830112in}{0.360113in}}%
\pgfpathlineto{\pgfqpoint{0.845769in}{0.365062in}}%
\pgfpathlineto{\pgfqpoint{0.856446in}{0.372592in}}%
\pgfpathlineto{\pgfqpoint{0.861425in}{0.374922in}}%
\pgfpathlineto{\pgfqpoint{0.877082in}{0.386059in}}%
\pgfpathlineto{\pgfqpoint{0.877242in}{0.386203in}}%
\pgfpathlineto{\pgfqpoint{0.892738in}{0.397950in}}%
\pgfpathlineto{\pgfqpoint{0.894887in}{0.399814in}}%
\pgfpathlineto{\pgfqpoint{0.908395in}{0.411173in}}%
\pgfpathlineto{\pgfqpoint{0.910983in}{0.413425in}}%
\pgfpathlineto{\pgfqpoint{0.924051in}{0.425693in}}%
\pgfpathlineto{\pgfqpoint{0.925603in}{0.427036in}}%
\pgfpathlineto{\pgfqpoint{0.938840in}{0.440648in}}%
\pgfpathlineto{\pgfqpoint{0.939708in}{0.441857in}}%
\pgfpathlineto{\pgfqpoint{0.951218in}{0.454259in}}%
\pgfpathlineto{\pgfqpoint{0.955364in}{0.461811in}}%
\pgfpathlineto{\pgfqpoint{0.961230in}{0.467870in}}%
\pgfpathlineto{\pgfqpoint{0.965042in}{0.481481in}}%
\pgfpathlineto{\pgfqpoint{0.959323in}{0.495092in}}%
\pgfpathlineto{\pgfqpoint{0.955364in}{0.498693in}}%
\pgfpathlineto{\pgfqpoint{0.949104in}{0.508703in}}%
\pgfpathlineto{\pgfqpoint{0.939708in}{0.518149in}}%
\pgfpathlineto{\pgfqpoint{0.936493in}{0.522314in}}%
\pgfpathlineto{\pgfqpoint{0.924051in}{0.534510in}}%
\pgfpathlineto{\pgfqpoint{0.922731in}{0.535925in}}%
\pgfpathlineto{\pgfqpoint{0.908395in}{0.549029in}}%
\pgfpathlineto{\pgfqpoint{0.907822in}{0.549536in}}%
\pgfpathlineto{\pgfqpoint{0.892738in}{0.562215in}}%
\pgfpathlineto{\pgfqpoint{0.891465in}{0.563148in}}%
\pgfpathlineto{\pgfqpoint{0.877082in}{0.574367in}}%
\pgfpathlineto{\pgfqpoint{0.873136in}{0.576759in}}%
\pgfpathlineto{\pgfqpoint{0.861425in}{0.585556in}}%
\pgfpathlineto{\pgfqpoint{0.851706in}{0.590370in}}%
\pgfpathlineto{\pgfqpoint{0.845769in}{0.595110in}}%
\pgfpathlineto{\pgfqpoint{0.830112in}{0.600766in}}%
\pgfpathlineto{\pgfqpoint{0.814455in}{0.599353in}}%
\pgfpathlineto{\pgfqpoint{0.798799in}{0.590861in}}%
\pgfpathlineto{\pgfqpoint{0.798286in}{0.590370in}}%
\pgfpathlineto{\pgfqpoint{0.783142in}{0.581297in}}%
\pgfpathlineto{\pgfqpoint{0.777716in}{0.576759in}}%
\pgfpathlineto{\pgfqpoint{0.767486in}{0.569815in}}%
\pgfpathlineto{\pgfqpoint{0.759452in}{0.563148in}}%
\pgfpathlineto{\pgfqpoint{0.751829in}{0.557183in}}%
\pgfpathlineto{\pgfqpoint{0.742952in}{0.549536in}}%
\pgfpathlineto{\pgfqpoint{0.736173in}{0.543369in}}%
\pgfpathlineto{\pgfqpoint{0.727900in}{0.535925in}}%
\pgfpathlineto{\pgfqpoint{0.720516in}{0.528145in}}%
\pgfpathlineto{\pgfqpoint{0.714139in}{0.522314in}}%
\pgfpathlineto{\pgfqpoint{0.704859in}{0.511150in}}%
\pgfpathlineto{\pgfqpoint{0.701996in}{0.508703in}}%
\pgfpathlineto{\pgfqpoint{0.691825in}{0.495092in}}%
\pgfpathlineto{\pgfqpoint{0.689203in}{0.485746in}}%
\pgfpathlineto{\pgfqpoint{0.686441in}{0.481481in}}%
\pgfpathlineto{\pgfqpoint{0.689203in}{0.475105in}}%
\pgfpathlineto{\pgfqpoint{0.690556in}{0.467870in}}%
\pgfpathlineto{\pgfqpoint{0.699452in}{0.454259in}}%
\pgfpathlineto{\pgfqpoint{0.704859in}{0.449305in}}%
\pgfpathlineto{\pgfqpoint{0.711553in}{0.440648in}}%
\pgfpathlineto{\pgfqpoint{0.720516in}{0.432105in}}%
\pgfpathlineto{\pgfqpoint{0.725120in}{0.427036in}}%
\pgfpathlineto{\pgfqpoint{0.736173in}{0.416875in}}%
\pgfpathlineto{\pgfqpoint{0.739878in}{0.413425in}}%
\pgfpathlineto{\pgfqpoint{0.751829in}{0.403104in}}%
\pgfpathlineto{\pgfqpoint{0.756021in}{0.399814in}}%
\pgfpathlineto{\pgfqpoint{0.767486in}{0.390487in}}%
\pgfpathlineto{\pgfqpoint{0.773934in}{0.386203in}}%
\pgfpathlineto{\pgfqpoint{0.783142in}{0.378904in}}%
\pgfpathlineto{\pgfqpoint{0.794298in}{0.372592in}}%
\pgfpathlineto{\pgfqpoint{0.798799in}{0.368779in}}%
\pgfpathclose%
\pgfpathmoveto{\pgfqpoint{0.784782in}{0.399814in}}%
\pgfpathlineto{\pgfqpoint{0.783142in}{0.400572in}}%
\pgfpathlineto{\pgfqpoint{0.767486in}{0.410695in}}%
\pgfpathlineto{\pgfqpoint{0.764030in}{0.413425in}}%
\pgfpathlineto{\pgfqpoint{0.751829in}{0.423750in}}%
\pgfpathlineto{\pgfqpoint{0.748282in}{0.427036in}}%
\pgfpathlineto{\pgfqpoint{0.736173in}{0.440481in}}%
\pgfpathlineto{\pgfqpoint{0.736019in}{0.440648in}}%
\pgfpathlineto{\pgfqpoint{0.726597in}{0.454259in}}%
\pgfpathlineto{\pgfqpoint{0.721106in}{0.467870in}}%
\pgfpathlineto{\pgfqpoint{0.720516in}{0.472978in}}%
\pgfpathlineto{\pgfqpoint{0.719324in}{0.481481in}}%
\pgfpathlineto{\pgfqpoint{0.720516in}{0.487169in}}%
\pgfpathlineto{\pgfqpoint{0.721890in}{0.495092in}}%
\pgfpathlineto{\pgfqpoint{0.728167in}{0.508703in}}%
\pgfpathlineto{\pgfqpoint{0.736173in}{0.519485in}}%
\pgfpathlineto{\pgfqpoint{0.738203in}{0.522314in}}%
\pgfpathlineto{\pgfqpoint{0.751136in}{0.535925in}}%
\pgfpathlineto{\pgfqpoint{0.751829in}{0.536554in}}%
\pgfpathlineto{\pgfqpoint{0.767486in}{0.549504in}}%
\pgfpathlineto{\pgfqpoint{0.767533in}{0.549536in}}%
\pgfpathlineto{\pgfqpoint{0.783142in}{0.559744in}}%
\pgfpathlineto{\pgfqpoint{0.790528in}{0.563148in}}%
\pgfpathlineto{\pgfqpoint{0.798799in}{0.567297in}}%
\pgfpathlineto{\pgfqpoint{0.814455in}{0.571655in}}%
\pgfpathlineto{\pgfqpoint{0.830112in}{0.572381in}}%
\pgfpathlineto{\pgfqpoint{0.845769in}{0.569478in}}%
\pgfpathlineto{\pgfqpoint{0.860926in}{0.563148in}}%
\pgfpathlineto{\pgfqpoint{0.861425in}{0.562957in}}%
\pgfpathlineto{\pgfqpoint{0.877082in}{0.553971in}}%
\pgfpathlineto{\pgfqpoint{0.883070in}{0.549536in}}%
\pgfpathlineto{\pgfqpoint{0.892738in}{0.542051in}}%
\pgfpathlineto{\pgfqpoint{0.899661in}{0.535925in}}%
\pgfpathlineto{\pgfqpoint{0.908395in}{0.526885in}}%
\pgfpathlineto{\pgfqpoint{0.912717in}{0.522314in}}%
\pgfpathlineto{\pgfqpoint{0.922475in}{0.508703in}}%
\pgfpathlineto{\pgfqpoint{0.924051in}{0.505205in}}%
\pgfpathlineto{\pgfqpoint{0.929185in}{0.495092in}}%
\pgfpathlineto{\pgfqpoint{0.931790in}{0.481481in}}%
\pgfpathlineto{\pgfqpoint{0.930053in}{0.467870in}}%
\pgfpathlineto{\pgfqpoint{0.924051in}{0.454447in}}%
\pgfpathlineto{\pgfqpoint{0.923977in}{0.454259in}}%
\pgfpathlineto{\pgfqpoint{0.914966in}{0.440648in}}%
\pgfpathlineto{\pgfqpoint{0.908395in}{0.433404in}}%
\pgfpathlineto{\pgfqpoint{0.902506in}{0.427036in}}%
\pgfpathlineto{\pgfqpoint{0.892738in}{0.418204in}}%
\pgfpathlineto{\pgfqpoint{0.886706in}{0.413425in}}%
\pgfpathlineto{\pgfqpoint{0.877082in}{0.406278in}}%
\pgfpathlineto{\pgfqpoint{0.865840in}{0.399814in}}%
\pgfpathlineto{\pgfqpoint{0.861425in}{0.397138in}}%
\pgfpathlineto{\pgfqpoint{0.845769in}{0.390813in}}%
\pgfpathlineto{\pgfqpoint{0.830112in}{0.388008in}}%
\pgfpathlineto{\pgfqpoint{0.814455in}{0.388709in}}%
\pgfpathlineto{\pgfqpoint{0.798799in}{0.392920in}}%
\pgfpathlineto{\pgfqpoint{0.784782in}{0.399814in}}%
\pgfpathclose%
\pgfpathmoveto{\pgfqpoint{1.111930in}{0.366796in}}%
\pgfpathlineto{\pgfqpoint{1.127587in}{0.360608in}}%
\pgfpathlineto{\pgfqpoint{1.143243in}{0.360608in}}%
\pgfpathlineto{\pgfqpoint{1.158900in}{0.366796in}}%
\pgfpathlineto{\pgfqpoint{1.166357in}{0.372592in}}%
\pgfpathlineto{\pgfqpoint{1.174556in}{0.376834in}}%
\pgfpathlineto{\pgfqpoint{1.186996in}{0.386203in}}%
\pgfpathlineto{\pgfqpoint{1.190213in}{0.388228in}}%
\pgfpathlineto{\pgfqpoint{1.204874in}{0.399814in}}%
\pgfpathlineto{\pgfqpoint{1.205870in}{0.400572in}}%
\pgfpathlineto{\pgfqpoint{1.220907in}{0.413425in}}%
\pgfpathlineto{\pgfqpoint{1.221526in}{0.413995in}}%
\pgfpathlineto{\pgfqpoint{1.235569in}{0.427036in}}%
\pgfpathlineto{\pgfqpoint{1.237183in}{0.428832in}}%
\pgfpathlineto{\pgfqpoint{1.249117in}{0.440648in}}%
\pgfpathlineto{\pgfqpoint{1.252839in}{0.445662in}}%
\pgfpathlineto{\pgfqpoint{1.261435in}{0.454259in}}%
\pgfpathlineto{\pgfqpoint{1.268496in}{0.466149in}}%
\pgfpathlineto{\pgfqpoint{1.270502in}{0.467870in}}%
\pgfpathlineto{\pgfqpoint{1.275130in}{0.481481in}}%
\pgfpathlineto{\pgfqpoint{1.268496in}{0.494492in}}%
\pgfpathlineto{\pgfqpoint{1.268342in}{0.495092in}}%
\pgfpathlineto{\pgfqpoint{1.259131in}{0.508703in}}%
\pgfpathlineto{\pgfqpoint{1.252839in}{0.514573in}}%
\pgfpathlineto{\pgfqpoint{1.246660in}{0.522314in}}%
\pgfpathlineto{\pgfqpoint{1.237183in}{0.531317in}}%
\pgfpathlineto{\pgfqpoint{1.232856in}{0.535925in}}%
\pgfpathlineto{\pgfqpoint{1.221526in}{0.546224in}}%
\pgfpathlineto{\pgfqpoint{1.217844in}{0.549536in}}%
\pgfpathlineto{\pgfqpoint{1.205870in}{0.559744in}}%
\pgfpathlineto{\pgfqpoint{1.201383in}{0.563148in}}%
\pgfpathlineto{\pgfqpoint{1.190213in}{0.572153in}}%
\pgfpathlineto{\pgfqpoint{1.183052in}{0.576759in}}%
\pgfpathlineto{\pgfqpoint{1.174556in}{0.583511in}}%
\pgfpathlineto{\pgfqpoint{1.162034in}{0.590370in}}%
\pgfpathlineto{\pgfqpoint{1.158900in}{0.593128in}}%
\pgfpathlineto{\pgfqpoint{1.143243in}{0.600201in}}%
\pgfpathlineto{\pgfqpoint{1.127587in}{0.600201in}}%
\pgfpathlineto{\pgfqpoint{1.111930in}{0.593128in}}%
\pgfpathlineto{\pgfqpoint{1.108796in}{0.590370in}}%
\pgfpathlineto{\pgfqpoint{1.096274in}{0.583511in}}%
\pgfpathlineto{\pgfqpoint{1.087778in}{0.576759in}}%
\pgfpathlineto{\pgfqpoint{1.080617in}{0.572153in}}%
\pgfpathlineto{\pgfqpoint{1.069447in}{0.563148in}}%
\pgfpathlineto{\pgfqpoint{1.064960in}{0.559744in}}%
\pgfpathlineto{\pgfqpoint{1.052986in}{0.549536in}}%
\pgfpathlineto{\pgfqpoint{1.049304in}{0.546224in}}%
\pgfpathlineto{\pgfqpoint{1.037974in}{0.535925in}}%
\pgfpathlineto{\pgfqpoint{1.033647in}{0.531317in}}%
\pgfpathlineto{\pgfqpoint{1.024170in}{0.522314in}}%
\pgfpathlineto{\pgfqpoint{1.017991in}{0.514573in}}%
\pgfpathlineto{\pgfqpoint{1.011699in}{0.508703in}}%
\pgfpathlineto{\pgfqpoint{1.002488in}{0.495092in}}%
\pgfpathlineto{\pgfqpoint{1.002334in}{0.494492in}}%
\pgfpathlineto{\pgfqpoint{0.995700in}{0.481481in}}%
\pgfpathlineto{\pgfqpoint{1.000328in}{0.467870in}}%
\pgfpathlineto{\pgfqpoint{1.002334in}{0.466149in}}%
\pgfpathlineto{\pgfqpoint{1.009395in}{0.454259in}}%
\pgfpathlineto{\pgfqpoint{1.017991in}{0.445662in}}%
\pgfpathlineto{\pgfqpoint{1.021713in}{0.440648in}}%
\pgfpathlineto{\pgfqpoint{1.033647in}{0.428832in}}%
\pgfpathlineto{\pgfqpoint{1.035261in}{0.427036in}}%
\pgfpathlineto{\pgfqpoint{1.049304in}{0.413995in}}%
\pgfpathlineto{\pgfqpoint{1.049923in}{0.413425in}}%
\pgfpathlineto{\pgfqpoint{1.064960in}{0.400572in}}%
\pgfpathlineto{\pgfqpoint{1.065956in}{0.399814in}}%
\pgfpathlineto{\pgfqpoint{1.080617in}{0.388228in}}%
\pgfpathlineto{\pgfqpoint{1.083834in}{0.386203in}}%
\pgfpathlineto{\pgfqpoint{1.096274in}{0.376834in}}%
\pgfpathlineto{\pgfqpoint{1.104473in}{0.372592in}}%
\pgfpathlineto{\pgfqpoint{1.111930in}{0.366796in}}%
\pgfpathclose%
\pgfpathmoveto{\pgfqpoint{1.094732in}{0.399814in}}%
\pgfpathlineto{\pgfqpoint{1.080617in}{0.408427in}}%
\pgfpathlineto{\pgfqpoint{1.074104in}{0.413425in}}%
\pgfpathlineto{\pgfqpoint{1.064960in}{0.420927in}}%
\pgfpathlineto{\pgfqpoint{1.058298in}{0.427036in}}%
\pgfpathlineto{\pgfqpoint{1.049304in}{0.436912in}}%
\pgfpathlineto{\pgfqpoint{1.045895in}{0.440648in}}%
\pgfpathlineto{\pgfqpoint{1.036702in}{0.454259in}}%
\pgfpathlineto{\pgfqpoint{1.033647in}{0.461935in}}%
\pgfpathlineto{\pgfqpoint{1.030909in}{0.467870in}}%
\pgfpathlineto{\pgfqpoint{1.029092in}{0.481481in}}%
\pgfpathlineto{\pgfqpoint{1.031818in}{0.495092in}}%
\pgfpathlineto{\pgfqpoint{1.033647in}{0.498584in}}%
\pgfpathlineto{\pgfqpoint{1.038234in}{0.508703in}}%
\pgfpathlineto{\pgfqpoint{1.048190in}{0.522314in}}%
\pgfpathlineto{\pgfqpoint{1.049304in}{0.523486in}}%
\pgfpathlineto{\pgfqpoint{1.061142in}{0.535925in}}%
\pgfpathlineto{\pgfqpoint{1.064960in}{0.539352in}}%
\pgfpathlineto{\pgfqpoint{1.077662in}{0.549536in}}%
\pgfpathlineto{\pgfqpoint{1.080617in}{0.551798in}}%
\pgfpathlineto{\pgfqpoint{1.096274in}{0.561415in}}%
\pgfpathlineto{\pgfqpoint{1.100386in}{0.563148in}}%
\pgfpathlineto{\pgfqpoint{1.111930in}{0.568460in}}%
\pgfpathlineto{\pgfqpoint{1.127587in}{0.572091in}}%
\pgfpathlineto{\pgfqpoint{1.143243in}{0.572091in}}%
\pgfpathlineto{\pgfqpoint{1.158900in}{0.568460in}}%
\pgfpathlineto{\pgfqpoint{1.170444in}{0.563148in}}%
\pgfpathlineto{\pgfqpoint{1.174556in}{0.561415in}}%
\pgfpathlineto{\pgfqpoint{1.190213in}{0.551798in}}%
\pgfpathlineto{\pgfqpoint{1.193168in}{0.549536in}}%
\pgfpathlineto{\pgfqpoint{1.205870in}{0.539352in}}%
\pgfpathlineto{\pgfqpoint{1.209688in}{0.535925in}}%
\pgfpathlineto{\pgfqpoint{1.221526in}{0.523486in}}%
\pgfpathlineto{\pgfqpoint{1.222640in}{0.522314in}}%
\pgfpathlineto{\pgfqpoint{1.232596in}{0.508703in}}%
\pgfpathlineto{\pgfqpoint{1.237183in}{0.498584in}}%
\pgfpathlineto{\pgfqpoint{1.239012in}{0.495092in}}%
\pgfpathlineto{\pgfqpoint{1.241738in}{0.481481in}}%
\pgfpathlineto{\pgfqpoint{1.239921in}{0.467870in}}%
\pgfpathlineto{\pgfqpoint{1.237183in}{0.461935in}}%
\pgfpathlineto{\pgfqpoint{1.234128in}{0.454259in}}%
\pgfpathlineto{\pgfqpoint{1.224935in}{0.440648in}}%
\pgfpathlineto{\pgfqpoint{1.221526in}{0.436912in}}%
\pgfpathlineto{\pgfqpoint{1.212532in}{0.427036in}}%
\pgfpathlineto{\pgfqpoint{1.205870in}{0.420927in}}%
\pgfpathlineto{\pgfqpoint{1.196726in}{0.413425in}}%
\pgfpathlineto{\pgfqpoint{1.190213in}{0.408427in}}%
\pgfpathlineto{\pgfqpoint{1.176098in}{0.399814in}}%
\pgfpathlineto{\pgfqpoint{1.174556in}{0.398825in}}%
\pgfpathlineto{\pgfqpoint{1.158900in}{0.391796in}}%
\pgfpathlineto{\pgfqpoint{1.143243in}{0.388288in}}%
\pgfpathlineto{\pgfqpoint{1.127587in}{0.388288in}}%
\pgfpathlineto{\pgfqpoint{1.111930in}{0.391796in}}%
\pgfpathlineto{\pgfqpoint{1.096274in}{0.398825in}}%
\pgfpathlineto{\pgfqpoint{1.094732in}{0.399814in}}%
\pgfpathclose%
\pgfpathmoveto{\pgfqpoint{1.425061in}{0.365062in}}%
\pgfpathlineto{\pgfqpoint{1.440718in}{0.360113in}}%
\pgfpathlineto{\pgfqpoint{1.456375in}{0.361350in}}%
\pgfpathlineto{\pgfqpoint{1.472031in}{0.368779in}}%
\pgfpathlineto{\pgfqpoint{1.476532in}{0.372592in}}%
\pgfpathlineto{\pgfqpoint{1.487688in}{0.378904in}}%
\pgfpathlineto{\pgfqpoint{1.496896in}{0.386203in}}%
\pgfpathlineto{\pgfqpoint{1.503344in}{0.390487in}}%
\pgfpathlineto{\pgfqpoint{1.514809in}{0.399814in}}%
\pgfpathlineto{\pgfqpoint{1.519001in}{0.403104in}}%
\pgfpathlineto{\pgfqpoint{1.530952in}{0.413425in}}%
\pgfpathlineto{\pgfqpoint{1.534657in}{0.416875in}}%
\pgfpathlineto{\pgfqpoint{1.545710in}{0.427036in}}%
\pgfpathlineto{\pgfqpoint{1.550314in}{0.432105in}}%
\pgfpathlineto{\pgfqpoint{1.559277in}{0.440648in}}%
\pgfpathlineto{\pgfqpoint{1.565971in}{0.449305in}}%
\pgfpathlineto{\pgfqpoint{1.571378in}{0.454259in}}%
\pgfpathlineto{\pgfqpoint{1.580274in}{0.467870in}}%
\pgfpathlineto{\pgfqpoint{1.581627in}{0.475105in}}%
\pgfpathlineto{\pgfqpoint{1.584389in}{0.481481in}}%
\pgfpathlineto{\pgfqpoint{1.581627in}{0.485746in}}%
\pgfpathlineto{\pgfqpoint{1.579005in}{0.495092in}}%
\pgfpathlineto{\pgfqpoint{1.568834in}{0.508703in}}%
\pgfpathlineto{\pgfqpoint{1.565971in}{0.511150in}}%
\pgfpathlineto{\pgfqpoint{1.556691in}{0.522314in}}%
\pgfpathlineto{\pgfqpoint{1.550314in}{0.528145in}}%
\pgfpathlineto{\pgfqpoint{1.542930in}{0.535925in}}%
\pgfpathlineto{\pgfqpoint{1.534657in}{0.543369in}}%
\pgfpathlineto{\pgfqpoint{1.527878in}{0.549536in}}%
\pgfpathlineto{\pgfqpoint{1.519001in}{0.557183in}}%
\pgfpathlineto{\pgfqpoint{1.511378in}{0.563148in}}%
\pgfpathlineto{\pgfqpoint{1.503344in}{0.569815in}}%
\pgfpathlineto{\pgfqpoint{1.493114in}{0.576759in}}%
\pgfpathlineto{\pgfqpoint{1.487688in}{0.581297in}}%
\pgfpathlineto{\pgfqpoint{1.472544in}{0.590370in}}%
\pgfpathlineto{\pgfqpoint{1.472031in}{0.590861in}}%
\pgfpathlineto{\pgfqpoint{1.456375in}{0.599353in}}%
\pgfpathlineto{\pgfqpoint{1.440718in}{0.600766in}}%
\pgfpathlineto{\pgfqpoint{1.425061in}{0.595110in}}%
\pgfpathlineto{\pgfqpoint{1.419124in}{0.590370in}}%
\pgfpathlineto{\pgfqpoint{1.409405in}{0.585556in}}%
\pgfpathlineto{\pgfqpoint{1.397694in}{0.576759in}}%
\pgfpathlineto{\pgfqpoint{1.393748in}{0.574367in}}%
\pgfpathlineto{\pgfqpoint{1.379365in}{0.563148in}}%
\pgfpathlineto{\pgfqpoint{1.378092in}{0.562215in}}%
\pgfpathlineto{\pgfqpoint{1.363008in}{0.549536in}}%
\pgfpathlineto{\pgfqpoint{1.362435in}{0.549029in}}%
\pgfpathlineto{\pgfqpoint{1.348099in}{0.535925in}}%
\pgfpathlineto{\pgfqpoint{1.346779in}{0.534510in}}%
\pgfpathlineto{\pgfqpoint{1.334337in}{0.522314in}}%
\pgfpathlineto{\pgfqpoint{1.331122in}{0.518149in}}%
\pgfpathlineto{\pgfqpoint{1.321726in}{0.508703in}}%
\pgfpathlineto{\pgfqpoint{1.315466in}{0.498693in}}%
\pgfpathlineto{\pgfqpoint{1.311507in}{0.495092in}}%
\pgfpathlineto{\pgfqpoint{1.305788in}{0.481481in}}%
\pgfpathlineto{\pgfqpoint{1.309600in}{0.467870in}}%
\pgfpathlineto{\pgfqpoint{1.315466in}{0.461811in}}%
\pgfpathlineto{\pgfqpoint{1.319612in}{0.454259in}}%
\pgfpathlineto{\pgfqpoint{1.331122in}{0.441857in}}%
\pgfpathlineto{\pgfqpoint{1.331990in}{0.440648in}}%
\pgfpathlineto{\pgfqpoint{1.345227in}{0.427036in}}%
\pgfpathlineto{\pgfqpoint{1.346779in}{0.425693in}}%
\pgfpathlineto{\pgfqpoint{1.359847in}{0.413425in}}%
\pgfpathlineto{\pgfqpoint{1.362435in}{0.411173in}}%
\pgfpathlineto{\pgfqpoint{1.375943in}{0.399814in}}%
\pgfpathlineto{\pgfqpoint{1.378092in}{0.397950in}}%
\pgfpathlineto{\pgfqpoint{1.393588in}{0.386203in}}%
\pgfpathlineto{\pgfqpoint{1.393748in}{0.386059in}}%
\pgfpathlineto{\pgfqpoint{1.409405in}{0.374922in}}%
\pgfpathlineto{\pgfqpoint{1.414384in}{0.372592in}}%
\pgfpathlineto{\pgfqpoint{1.425061in}{0.365062in}}%
\pgfpathclose%
\pgfpathmoveto{\pgfqpoint{1.404990in}{0.399814in}}%
\pgfpathlineto{\pgfqpoint{1.393748in}{0.406278in}}%
\pgfpathlineto{\pgfqpoint{1.384124in}{0.413425in}}%
\pgfpathlineto{\pgfqpoint{1.378092in}{0.418204in}}%
\pgfpathlineto{\pgfqpoint{1.368324in}{0.427036in}}%
\pgfpathlineto{\pgfqpoint{1.362435in}{0.433404in}}%
\pgfpathlineto{\pgfqpoint{1.355864in}{0.440648in}}%
\pgfpathlineto{\pgfqpoint{1.346853in}{0.454259in}}%
\pgfpathlineto{\pgfqpoint{1.346779in}{0.454447in}}%
\pgfpathlineto{\pgfqpoint{1.340777in}{0.467870in}}%
\pgfpathlineto{\pgfqpoint{1.339040in}{0.481481in}}%
\pgfpathlineto{\pgfqpoint{1.341645in}{0.495092in}}%
\pgfpathlineto{\pgfqpoint{1.346779in}{0.505205in}}%
\pgfpathlineto{\pgfqpoint{1.348355in}{0.508703in}}%
\pgfpathlineto{\pgfqpoint{1.358113in}{0.522314in}}%
\pgfpathlineto{\pgfqpoint{1.362435in}{0.526885in}}%
\pgfpathlineto{\pgfqpoint{1.371169in}{0.535925in}}%
\pgfpathlineto{\pgfqpoint{1.378092in}{0.542051in}}%
\pgfpathlineto{\pgfqpoint{1.387760in}{0.549536in}}%
\pgfpathlineto{\pgfqpoint{1.393748in}{0.553971in}}%
\pgfpathlineto{\pgfqpoint{1.409405in}{0.562957in}}%
\pgfpathlineto{\pgfqpoint{1.409904in}{0.563148in}}%
\pgfpathlineto{\pgfqpoint{1.425061in}{0.569478in}}%
\pgfpathlineto{\pgfqpoint{1.440718in}{0.572381in}}%
\pgfpathlineto{\pgfqpoint{1.456375in}{0.571655in}}%
\pgfpathlineto{\pgfqpoint{1.472031in}{0.567297in}}%
\pgfpathlineto{\pgfqpoint{1.480302in}{0.563148in}}%
\pgfpathlineto{\pgfqpoint{1.487688in}{0.559744in}}%
\pgfpathlineto{\pgfqpoint{1.503297in}{0.549536in}}%
\pgfpathlineto{\pgfqpoint{1.503344in}{0.549504in}}%
\pgfpathlineto{\pgfqpoint{1.519001in}{0.536554in}}%
\pgfpathlineto{\pgfqpoint{1.519694in}{0.535925in}}%
\pgfpathlineto{\pgfqpoint{1.532627in}{0.522314in}}%
\pgfpathlineto{\pgfqpoint{1.534657in}{0.519485in}}%
\pgfpathlineto{\pgfqpoint{1.542663in}{0.508703in}}%
\pgfpathlineto{\pgfqpoint{1.548940in}{0.495092in}}%
\pgfpathlineto{\pgfqpoint{1.550314in}{0.487169in}}%
\pgfpathlineto{\pgfqpoint{1.551506in}{0.481481in}}%
\pgfpathlineto{\pgfqpoint{1.550314in}{0.472978in}}%
\pgfpathlineto{\pgfqpoint{1.549724in}{0.467870in}}%
\pgfpathlineto{\pgfqpoint{1.544233in}{0.454259in}}%
\pgfpathlineto{\pgfqpoint{1.534811in}{0.440648in}}%
\pgfpathlineto{\pgfqpoint{1.534657in}{0.440481in}}%
\pgfpathlineto{\pgfqpoint{1.522548in}{0.427036in}}%
\pgfpathlineto{\pgfqpoint{1.519001in}{0.423750in}}%
\pgfpathlineto{\pgfqpoint{1.506800in}{0.413425in}}%
\pgfpathlineto{\pgfqpoint{1.503344in}{0.410695in}}%
\pgfpathlineto{\pgfqpoint{1.487688in}{0.400572in}}%
\pgfpathlineto{\pgfqpoint{1.486048in}{0.399814in}}%
\pgfpathlineto{\pgfqpoint{1.472031in}{0.392920in}}%
\pgfpathlineto{\pgfqpoint{1.456375in}{0.388709in}}%
\pgfpathlineto{\pgfqpoint{1.440718in}{0.388008in}}%
\pgfpathlineto{\pgfqpoint{1.425061in}{0.390813in}}%
\pgfpathlineto{\pgfqpoint{1.409405in}{0.397138in}}%
\pgfpathlineto{\pgfqpoint{1.404990in}{0.399814in}}%
\pgfpathclose%
\pgfpathmoveto{\pgfqpoint{1.738193in}{0.363576in}}%
\pgfpathlineto{\pgfqpoint{1.753849in}{0.359866in}}%
\pgfpathlineto{\pgfqpoint{1.769506in}{0.362339in}}%
\pgfpathlineto{\pgfqpoint{1.785162in}{0.371010in}}%
\pgfpathlineto{\pgfqpoint{1.786891in}{0.372592in}}%
\pgfpathlineto{\pgfqpoint{1.800819in}{0.381133in}}%
\pgfpathlineto{\pgfqpoint{1.806928in}{0.386203in}}%
\pgfpathlineto{\pgfqpoint{1.816476in}{0.392863in}}%
\pgfpathlineto{\pgfqpoint{1.824807in}{0.399814in}}%
\pgfpathlineto{\pgfqpoint{1.832132in}{0.405720in}}%
\pgfpathlineto{\pgfqpoint{1.840996in}{0.413425in}}%
\pgfpathlineto{\pgfqpoint{1.847789in}{0.419793in}}%
\pgfpathlineto{\pgfqpoint{1.855785in}{0.427036in}}%
\pgfpathlineto{\pgfqpoint{1.863445in}{0.435337in}}%
\pgfpathlineto{\pgfqpoint{1.869277in}{0.440648in}}%
\pgfpathlineto{\pgfqpoint{1.879102in}{0.452756in}}%
\pgfpathlineto{\pgfqpoint{1.880922in}{0.454259in}}%
\pgfpathlineto{\pgfqpoint{1.890895in}{0.467870in}}%
\pgfpathlineto{\pgfqpoint{1.893740in}{0.481481in}}%
\pgfpathlineto{\pgfqpoint{1.889472in}{0.495092in}}%
\pgfpathlineto{\pgfqpoint{1.879102in}{0.507502in}}%
\pgfpathlineto{\pgfqpoint{1.878438in}{0.508703in}}%
\pgfpathlineto{\pgfqpoint{1.866533in}{0.522314in}}%
\pgfpathlineto{\pgfqpoint{1.863445in}{0.525012in}}%
\pgfpathlineto{\pgfqpoint{1.852922in}{0.535925in}}%
\pgfpathlineto{\pgfqpoint{1.847789in}{0.540476in}}%
\pgfpathlineto{\pgfqpoint{1.837899in}{0.549536in}}%
\pgfpathlineto{\pgfqpoint{1.832132in}{0.554537in}}%
\pgfpathlineto{\pgfqpoint{1.821421in}{0.563148in}}%
\pgfpathlineto{\pgfqpoint{1.816476in}{0.567357in}}%
\pgfpathlineto{\pgfqpoint{1.803281in}{0.576759in}}%
\pgfpathlineto{\pgfqpoint{1.800819in}{0.578915in}}%
\pgfpathlineto{\pgfqpoint{1.785162in}{0.589133in}}%
\pgfpathlineto{\pgfqpoint{1.781975in}{0.590370in}}%
\pgfpathlineto{\pgfqpoint{1.769506in}{0.598222in}}%
\pgfpathlineto{\pgfqpoint{1.753849in}{0.601049in}}%
\pgfpathlineto{\pgfqpoint{1.738193in}{0.596808in}}%
\pgfpathlineto{\pgfqpoint{1.729188in}{0.590370in}}%
\pgfpathlineto{\pgfqpoint{1.722536in}{0.587429in}}%
\pgfpathlineto{\pgfqpoint{1.707414in}{0.576759in}}%
\pgfpathlineto{\pgfqpoint{1.706880in}{0.576455in}}%
\pgfpathlineto{\pgfqpoint{1.691223in}{0.564781in}}%
\pgfpathlineto{\pgfqpoint{1.689345in}{0.563148in}}%
\pgfpathlineto{\pgfqpoint{1.675567in}{0.551813in}}%
\pgfpathlineto{\pgfqpoint{1.672948in}{0.549536in}}%
\pgfpathlineto{\pgfqpoint{1.659910in}{0.537558in}}%
\pgfpathlineto{\pgfqpoint{1.658031in}{0.535925in}}%
\pgfpathlineto{\pgfqpoint{1.644602in}{0.522314in}}%
\pgfpathlineto{\pgfqpoint{1.644253in}{0.521850in}}%
\pgfpathlineto{\pgfqpoint{1.631979in}{0.508703in}}%
\pgfpathlineto{\pgfqpoint{1.628597in}{0.502921in}}%
\pgfpathlineto{\pgfqpoint{1.621191in}{0.495092in}}%
\pgfpathlineto{\pgfqpoint{1.616313in}{0.481481in}}%
\pgfpathlineto{\pgfqpoint{1.619565in}{0.467870in}}%
\pgfpathlineto{\pgfqpoint{1.628597in}{0.457030in}}%
\pgfpathlineto{\pgfqpoint{1.630020in}{0.454259in}}%
\pgfpathlineto{\pgfqpoint{1.641773in}{0.440648in}}%
\pgfpathlineto{\pgfqpoint{1.644253in}{0.438507in}}%
\pgfpathlineto{\pgfqpoint{1.655068in}{0.427036in}}%
\pgfpathlineto{\pgfqpoint{1.659910in}{0.422737in}}%
\pgfpathlineto{\pgfqpoint{1.669815in}{0.413425in}}%
\pgfpathlineto{\pgfqpoint{1.675567in}{0.408412in}}%
\pgfpathlineto{\pgfqpoint{1.685988in}{0.399814in}}%
\pgfpathlineto{\pgfqpoint{1.691223in}{0.395351in}}%
\pgfpathlineto{\pgfqpoint{1.703776in}{0.386203in}}%
\pgfpathlineto{\pgfqpoint{1.706880in}{0.383519in}}%
\pgfpathlineto{\pgfqpoint{1.722536in}{0.373169in}}%
\pgfpathlineto{\pgfqpoint{1.723918in}{0.372592in}}%
\pgfpathlineto{\pgfqpoint{1.738193in}{0.363576in}}%
\pgfpathclose%
\pgfpathmoveto{\pgfqpoint{1.715121in}{0.399814in}}%
\pgfpathlineto{\pgfqpoint{1.706880in}{0.404252in}}%
\pgfpathlineto{\pgfqpoint{1.694061in}{0.413425in}}%
\pgfpathlineto{\pgfqpoint{1.691223in}{0.415586in}}%
\pgfpathlineto{\pgfqpoint{1.678332in}{0.427036in}}%
\pgfpathlineto{\pgfqpoint{1.675567in}{0.429971in}}%
\pgfpathlineto{\pgfqpoint{1.665893in}{0.440648in}}%
\pgfpathlineto{\pgfqpoint{1.659910in}{0.449702in}}%
\pgfpathlineto{\pgfqpoint{1.656642in}{0.454259in}}%
\pgfpathlineto{\pgfqpoint{1.650791in}{0.467870in}}%
\pgfpathlineto{\pgfqpoint{1.649122in}{0.481481in}}%
\pgfpathlineto{\pgfqpoint{1.651626in}{0.495092in}}%
\pgfpathlineto{\pgfqpoint{1.658316in}{0.508703in}}%
\pgfpathlineto{\pgfqpoint{1.659910in}{0.510777in}}%
\pgfpathlineto{\pgfqpoint{1.668107in}{0.522314in}}%
\pgfpathlineto{\pgfqpoint{1.675567in}{0.530213in}}%
\pgfpathlineto{\pgfqpoint{1.681192in}{0.535925in}}%
\pgfpathlineto{\pgfqpoint{1.691223in}{0.544646in}}%
\pgfpathlineto{\pgfqpoint{1.697793in}{0.549536in}}%
\pgfpathlineto{\pgfqpoint{1.706880in}{0.556021in}}%
\pgfpathlineto{\pgfqpoint{1.720151in}{0.563148in}}%
\pgfpathlineto{\pgfqpoint{1.722536in}{0.564533in}}%
\pgfpathlineto{\pgfqpoint{1.738193in}{0.570349in}}%
\pgfpathlineto{\pgfqpoint{1.753849in}{0.572526in}}%
\pgfpathlineto{\pgfqpoint{1.769506in}{0.571075in}}%
\pgfpathlineto{\pgfqpoint{1.785162in}{0.565988in}}%
\pgfpathlineto{\pgfqpoint{1.790403in}{0.563148in}}%
\pgfpathlineto{\pgfqpoint{1.800819in}{0.557946in}}%
\pgfpathlineto{\pgfqpoint{1.813100in}{0.549536in}}%
\pgfpathlineto{\pgfqpoint{1.816476in}{0.547132in}}%
\pgfpathlineto{\pgfqpoint{1.829646in}{0.535925in}}%
\pgfpathlineto{\pgfqpoint{1.832132in}{0.533458in}}%
\pgfpathlineto{\pgfqpoint{1.842684in}{0.522314in}}%
\pgfpathlineto{\pgfqpoint{1.847789in}{0.515150in}}%
\pgfpathlineto{\pgfqpoint{1.852647in}{0.508703in}}%
\pgfpathlineto{\pgfqpoint{1.859111in}{0.495092in}}%
\pgfpathlineto{\pgfqpoint{1.861530in}{0.481481in}}%
\pgfpathlineto{\pgfqpoint{1.859918in}{0.467870in}}%
\pgfpathlineto{\pgfqpoint{1.854264in}{0.454259in}}%
\pgfpathlineto{\pgfqpoint{1.847789in}{0.445048in}}%
\pgfpathlineto{\pgfqpoint{1.844872in}{0.440648in}}%
\pgfpathlineto{\pgfqpoint{1.832529in}{0.427036in}}%
\pgfpathlineto{\pgfqpoint{1.832132in}{0.426666in}}%
\pgfpathlineto{\pgfqpoint{1.816902in}{0.413425in}}%
\pgfpathlineto{\pgfqpoint{1.816476in}{0.413080in}}%
\pgfpathlineto{\pgfqpoint{1.800819in}{0.402349in}}%
\pgfpathlineto{\pgfqpoint{1.795757in}{0.399814in}}%
\pgfpathlineto{\pgfqpoint{1.785162in}{0.394185in}}%
\pgfpathlineto{\pgfqpoint{1.769506in}{0.389270in}}%
\pgfpathlineto{\pgfqpoint{1.753849in}{0.387868in}}%
\pgfpathlineto{\pgfqpoint{1.738193in}{0.389971in}}%
\pgfpathlineto{\pgfqpoint{1.722536in}{0.395591in}}%
\pgfpathlineto{\pgfqpoint{1.715121in}{0.399814in}}%
\pgfpathclose%
\pgfpathmoveto{\pgfqpoint{0.516981in}{0.628802in}}%
\pgfpathlineto{\pgfqpoint{0.521887in}{0.631203in}}%
\pgfpathlineto{\pgfqpoint{0.532637in}{0.633483in}}%
\pgfpathlineto{\pgfqpoint{0.548294in}{0.642325in}}%
\pgfpathlineto{\pgfqpoint{0.551108in}{0.644814in}}%
\pgfpathlineto{\pgfqpoint{0.563950in}{0.652882in}}%
\pgfpathlineto{\pgfqpoint{0.570657in}{0.658425in}}%
\pgfpathlineto{\pgfqpoint{0.579607in}{0.664845in}}%
\pgfpathlineto{\pgfqpoint{0.588169in}{0.672036in}}%
\pgfpathlineto{\pgfqpoint{0.595263in}{0.677930in}}%
\pgfpathlineto{\pgfqpoint{0.604059in}{0.685648in}}%
\pgfpathlineto{\pgfqpoint{0.610920in}{0.692274in}}%
\pgfpathlineto{\pgfqpoint{0.618590in}{0.699259in}}%
\pgfpathlineto{\pgfqpoint{0.626577in}{0.708152in}}%
\pgfpathlineto{\pgfqpoint{0.631798in}{0.712870in}}%
\pgfpathlineto{\pgfqpoint{0.642233in}{0.726035in}}%
\pgfpathlineto{\pgfqpoint{0.642799in}{0.726481in}}%
\pgfpathlineto{\pgfqpoint{0.652566in}{0.740092in}}%
\pgfpathlineto{\pgfqpoint{0.654192in}{0.753703in}}%
\pgfpathlineto{\pgfqpoint{0.647685in}{0.767314in}}%
\pgfpathlineto{\pgfqpoint{0.642233in}{0.772476in}}%
\pgfpathlineto{\pgfqpoint{0.636696in}{0.780925in}}%
\pgfpathlineto{\pgfqpoint{0.626577in}{0.791106in}}%
\pgfpathlineto{\pgfqpoint{0.623826in}{0.794536in}}%
\pgfpathlineto{\pgfqpoint{0.610920in}{0.807041in}}%
\pgfpathlineto{\pgfqpoint{0.609847in}{0.808148in}}%
\pgfpathlineto{\pgfqpoint{0.595263in}{0.821261in}}%
\pgfpathlineto{\pgfqpoint{0.594680in}{0.821759in}}%
\pgfpathlineto{\pgfqpoint{0.579607in}{0.834222in}}%
\pgfpathlineto{\pgfqpoint{0.577979in}{0.835370in}}%
\pgfpathlineto{\pgfqpoint{0.563950in}{0.846186in}}%
\pgfpathlineto{\pgfqpoint{0.559159in}{0.848981in}}%
\pgfpathlineto{\pgfqpoint{0.548294in}{0.857150in}}%
\pgfpathlineto{\pgfqpoint{0.536780in}{0.862592in}}%
\pgfpathlineto{\pgfqpoint{0.532637in}{0.866033in}}%
\pgfpathlineto{\pgfqpoint{0.516981in}{0.871005in}}%
\pgfpathlineto{\pgfqpoint{0.501324in}{0.867691in}}%
\pgfpathlineto{\pgfqpoint{0.494355in}{0.862592in}}%
\pgfpathlineto{\pgfqpoint{0.485668in}{0.858987in}}%
\pgfpathlineto{\pgfqpoint{0.471402in}{0.848981in}}%
\pgfpathlineto{\pgfqpoint{0.470011in}{0.848226in}}%
\pgfpathlineto{\pgfqpoint{0.454354in}{0.836719in}}%
\pgfpathlineto{\pgfqpoint{0.452809in}{0.835370in}}%
\pgfpathlineto{\pgfqpoint{0.438698in}{0.824008in}}%
\pgfpathlineto{\pgfqpoint{0.436107in}{0.821759in}}%
\pgfpathlineto{\pgfqpoint{0.423041in}{0.810015in}}%
\pgfpathlineto{\pgfqpoint{0.420897in}{0.808148in}}%
\pgfpathlineto{\pgfqpoint{0.407385in}{0.794676in}}%
\pgfpathlineto{\pgfqpoint{0.407219in}{0.794536in}}%
\pgfpathlineto{\pgfqpoint{0.394408in}{0.780925in}}%
\pgfpathlineto{\pgfqpoint{0.391728in}{0.776596in}}%
\pgfpathlineto{\pgfqpoint{0.383067in}{0.767314in}}%
\pgfpathlineto{\pgfqpoint{0.377374in}{0.753703in}}%
\pgfpathlineto{\pgfqpoint{0.378796in}{0.740092in}}%
\pgfpathlineto{\pgfqpoint{0.387342in}{0.726481in}}%
\pgfpathlineto{\pgfqpoint{0.391728in}{0.722568in}}%
\pgfpathlineto{\pgfqpoint{0.398989in}{0.712870in}}%
\pgfpathlineto{\pgfqpoint{0.407385in}{0.704864in}}%
\pgfpathlineto{\pgfqpoint{0.412313in}{0.699259in}}%
\pgfpathlineto{\pgfqpoint{0.423041in}{0.689292in}}%
\pgfpathlineto{\pgfqpoint{0.426825in}{0.685648in}}%
\pgfpathlineto{\pgfqpoint{0.438698in}{0.675258in}}%
\pgfpathlineto{\pgfqpoint{0.442666in}{0.672036in}}%
\pgfpathlineto{\pgfqpoint{0.454354in}{0.662428in}}%
\pgfpathlineto{\pgfqpoint{0.460185in}{0.658425in}}%
\pgfpathlineto{\pgfqpoint{0.470011in}{0.650633in}}%
\pgfpathlineto{\pgfqpoint{0.479970in}{0.644814in}}%
\pgfpathlineto{\pgfqpoint{0.485668in}{0.640113in}}%
\pgfpathlineto{\pgfqpoint{0.501324in}{0.632379in}}%
\pgfpathlineto{\pgfqpoint{0.509646in}{0.631203in}}%
\pgfpathlineto{\pgfqpoint{0.516981in}{0.628802in}}%
\pgfpathclose%
\pgfpathmoveto{\pgfqpoint{0.507200in}{0.658425in}}%
\pgfpathlineto{\pgfqpoint{0.501324in}{0.658939in}}%
\pgfpathlineto{\pgfqpoint{0.485668in}{0.663712in}}%
\pgfpathlineto{\pgfqpoint{0.470011in}{0.671903in}}%
\pgfpathlineto{\pgfqpoint{0.469819in}{0.672036in}}%
\pgfpathlineto{\pgfqpoint{0.454354in}{0.682564in}}%
\pgfpathlineto{\pgfqpoint{0.450574in}{0.685648in}}%
\pgfpathlineto{\pgfqpoint{0.438698in}{0.696254in}}%
\pgfpathlineto{\pgfqpoint{0.435557in}{0.699259in}}%
\pgfpathlineto{\pgfqpoint{0.423913in}{0.712870in}}%
\pgfpathlineto{\pgfqpoint{0.423041in}{0.714295in}}%
\pgfpathlineto{\pgfqpoint{0.415111in}{0.726481in}}%
\pgfpathlineto{\pgfqpoint{0.410267in}{0.740092in}}%
\pgfpathlineto{\pgfqpoint{0.409461in}{0.753703in}}%
\pgfpathlineto{\pgfqpoint{0.412688in}{0.767314in}}%
\pgfpathlineto{\pgfqpoint{0.419962in}{0.780925in}}%
\pgfpathlineto{\pgfqpoint{0.423041in}{0.784763in}}%
\pgfpathlineto{\pgfqpoint{0.430477in}{0.794536in}}%
\pgfpathlineto{\pgfqpoint{0.438698in}{0.802903in}}%
\pgfpathlineto{\pgfqpoint{0.444195in}{0.808148in}}%
\pgfpathlineto{\pgfqpoint{0.454354in}{0.816640in}}%
\pgfpathlineto{\pgfqpoint{0.461679in}{0.821759in}}%
\pgfpathlineto{\pgfqpoint{0.470011in}{0.827472in}}%
\pgfpathlineto{\pgfqpoint{0.485668in}{0.835305in}}%
\pgfpathlineto{\pgfqpoint{0.485884in}{0.835370in}}%
\pgfpathlineto{\pgfqpoint{0.501324in}{0.840588in}}%
\pgfpathlineto{\pgfqpoint{0.516981in}{0.842097in}}%
\pgfpathlineto{\pgfqpoint{0.532637in}{0.839832in}}%
\pgfpathlineto{\pgfqpoint{0.544270in}{0.835370in}}%
\pgfpathlineto{\pgfqpoint{0.548294in}{0.833999in}}%
\pgfpathlineto{\pgfqpoint{0.563950in}{0.825517in}}%
\pgfpathlineto{\pgfqpoint{0.569209in}{0.821759in}}%
\pgfpathlineto{\pgfqpoint{0.579607in}{0.814166in}}%
\pgfpathlineto{\pgfqpoint{0.586654in}{0.808148in}}%
\pgfpathlineto{\pgfqpoint{0.595263in}{0.799742in}}%
\pgfpathlineto{\pgfqpoint{0.600365in}{0.794536in}}%
\pgfpathlineto{\pgfqpoint{0.610701in}{0.780925in}}%
\pgfpathlineto{\pgfqpoint{0.610920in}{0.780492in}}%
\pgfpathlineto{\pgfqpoint{0.618201in}{0.767314in}}%
\pgfpathlineto{\pgfqpoint{0.621541in}{0.753703in}}%
\pgfpathlineto{\pgfqpoint{0.620707in}{0.740092in}}%
\pgfpathlineto{\pgfqpoint{0.615693in}{0.726481in}}%
\pgfpathlineto{\pgfqpoint{0.610920in}{0.719291in}}%
\pgfpathlineto{\pgfqpoint{0.607005in}{0.712870in}}%
\pgfpathlineto{\pgfqpoint{0.595263in}{0.699300in}}%
\pgfpathlineto{\pgfqpoint{0.595227in}{0.699259in}}%
\pgfpathlineto{\pgfqpoint{0.580330in}{0.685648in}}%
\pgfpathlineto{\pgfqpoint{0.579607in}{0.685045in}}%
\pgfpathlineto{\pgfqpoint{0.563950in}{0.673801in}}%
\pgfpathlineto{\pgfqpoint{0.560696in}{0.672036in}}%
\pgfpathlineto{\pgfqpoint{0.548294in}{0.665077in}}%
\pgfpathlineto{\pgfqpoint{0.532637in}{0.659620in}}%
\pgfpathlineto{\pgfqpoint{0.523523in}{0.658425in}}%
\pgfpathlineto{\pgfqpoint{0.516981in}{0.657389in}}%
\pgfpathlineto{\pgfqpoint{0.507200in}{0.658425in}}%
\pgfpathclose%
\pgfpathmoveto{\pgfqpoint{0.830112in}{0.629316in}}%
\pgfpathlineto{\pgfqpoint{0.833015in}{0.631203in}}%
\pgfpathlineto{\pgfqpoint{0.845769in}{0.634808in}}%
\pgfpathlineto{\pgfqpoint{0.861425in}{0.644759in}}%
\pgfpathlineto{\pgfqpoint{0.861484in}{0.644814in}}%
\pgfpathlineto{\pgfqpoint{0.877082in}{0.655277in}}%
\pgfpathlineto{\pgfqpoint{0.880752in}{0.658425in}}%
\pgfpathlineto{\pgfqpoint{0.892738in}{0.667368in}}%
\pgfpathlineto{\pgfqpoint{0.898199in}{0.672036in}}%
\pgfpathlineto{\pgfqpoint{0.908395in}{0.680671in}}%
\pgfpathlineto{\pgfqpoint{0.914075in}{0.685648in}}%
\pgfpathlineto{\pgfqpoint{0.924051in}{0.695269in}}%
\pgfpathlineto{\pgfqpoint{0.928545in}{0.699259in}}%
\pgfpathlineto{\pgfqpoint{0.939708in}{0.711355in}}%
\pgfpathlineto{\pgfqpoint{0.941494in}{0.712870in}}%
\pgfpathlineto{\pgfqpoint{0.953120in}{0.726481in}}%
\pgfpathlineto{\pgfqpoint{0.955364in}{0.731240in}}%
\pgfpathlineto{\pgfqpoint{0.962755in}{0.740092in}}%
\pgfpathlineto{\pgfqpoint{0.964661in}{0.753703in}}%
\pgfpathlineto{\pgfqpoint{0.957033in}{0.767314in}}%
\pgfpathlineto{\pgfqpoint{0.955364in}{0.768673in}}%
\pgfpathlineto{\pgfqpoint{0.946779in}{0.780925in}}%
\pgfpathlineto{\pgfqpoint{0.939708in}{0.787602in}}%
\pgfpathlineto{\pgfqpoint{0.933993in}{0.794536in}}%
\pgfpathlineto{\pgfqpoint{0.924051in}{0.803927in}}%
\pgfpathlineto{\pgfqpoint{0.919955in}{0.808148in}}%
\pgfpathlineto{\pgfqpoint{0.908395in}{0.818528in}}%
\pgfpathlineto{\pgfqpoint{0.904678in}{0.821759in}}%
\pgfpathlineto{\pgfqpoint{0.892738in}{0.831809in}}%
\pgfpathlineto{\pgfqpoint{0.887884in}{0.835370in}}%
\pgfpathlineto{\pgfqpoint{0.877082in}{0.844013in}}%
\pgfpathlineto{\pgfqpoint{0.869105in}{0.848981in}}%
\pgfpathlineto{\pgfqpoint{0.861425in}{0.855128in}}%
\pgfpathlineto{\pgfqpoint{0.847332in}{0.862592in}}%
\pgfpathlineto{\pgfqpoint{0.845769in}{0.864042in}}%
\pgfpathlineto{\pgfqpoint{0.830112in}{0.870674in}}%
\pgfpathlineto{\pgfqpoint{0.814455in}{0.869017in}}%
\pgfpathlineto{\pgfqpoint{0.804273in}{0.862592in}}%
\pgfpathlineto{\pgfqpoint{0.798799in}{0.860640in}}%
\pgfpathlineto{\pgfqpoint{0.783142in}{0.850534in}}%
\pgfpathlineto{\pgfqpoint{0.781400in}{0.848981in}}%
\pgfpathlineto{\pgfqpoint{0.767486in}{0.839277in}}%
\pgfpathlineto{\pgfqpoint{0.762896in}{0.835370in}}%
\pgfpathlineto{\pgfqpoint{0.751829in}{0.826696in}}%
\pgfpathlineto{\pgfqpoint{0.746105in}{0.821759in}}%
\pgfpathlineto{\pgfqpoint{0.736173in}{0.812894in}}%
\pgfpathlineto{\pgfqpoint{0.730802in}{0.808148in}}%
\pgfpathlineto{\pgfqpoint{0.720516in}{0.797727in}}%
\pgfpathlineto{\pgfqpoint{0.716895in}{0.794536in}}%
\pgfpathlineto{\pgfqpoint{0.704859in}{0.780976in}}%
\pgfpathlineto{\pgfqpoint{0.704796in}{0.780925in}}%
\pgfpathlineto{\pgfqpoint{0.693349in}{0.767314in}}%
\pgfpathlineto{\pgfqpoint{0.689203in}{0.756227in}}%
\pgfpathlineto{\pgfqpoint{0.687033in}{0.753703in}}%
\pgfpathlineto{\pgfqpoint{0.689203in}{0.743710in}}%
\pgfpathlineto{\pgfqpoint{0.689541in}{0.740092in}}%
\pgfpathlineto{\pgfqpoint{0.697162in}{0.726481in}}%
\pgfpathlineto{\pgfqpoint{0.704859in}{0.718864in}}%
\pgfpathlineto{\pgfqpoint{0.709136in}{0.712870in}}%
\pgfpathlineto{\pgfqpoint{0.720516in}{0.701513in}}%
\pgfpathlineto{\pgfqpoint{0.722466in}{0.699259in}}%
\pgfpathlineto{\pgfqpoint{0.736173in}{0.686336in}}%
\pgfpathlineto{\pgfqpoint{0.736892in}{0.685648in}}%
\pgfpathlineto{\pgfqpoint{0.751829in}{0.672662in}}%
\pgfpathlineto{\pgfqpoint{0.752621in}{0.672036in}}%
\pgfpathlineto{\pgfqpoint{0.767486in}{0.660121in}}%
\pgfpathlineto{\pgfqpoint{0.770079in}{0.658425in}}%
\pgfpathlineto{\pgfqpoint{0.783142in}{0.648532in}}%
\pgfpathlineto{\pgfqpoint{0.790038in}{0.644814in}}%
\pgfpathlineto{\pgfqpoint{0.798799in}{0.638123in}}%
\pgfpathlineto{\pgfqpoint{0.814455in}{0.631497in}}%
\pgfpathlineto{\pgfqpoint{0.818617in}{0.631203in}}%
\pgfpathlineto{\pgfqpoint{0.830112in}{0.629316in}}%
\pgfpathclose%
\pgfpathmoveto{\pgfqpoint{0.814332in}{0.658425in}}%
\pgfpathlineto{\pgfqpoint{0.798799in}{0.662483in}}%
\pgfpathlineto{\pgfqpoint{0.783142in}{0.669993in}}%
\pgfpathlineto{\pgfqpoint{0.780070in}{0.672036in}}%
\pgfpathlineto{\pgfqpoint{0.767486in}{0.680196in}}%
\pgfpathlineto{\pgfqpoint{0.760632in}{0.685648in}}%
\pgfpathlineto{\pgfqpoint{0.751829in}{0.693300in}}%
\pgfpathlineto{\pgfqpoint{0.745559in}{0.699259in}}%
\pgfpathlineto{\pgfqpoint{0.736173in}{0.710199in}}%
\pgfpathlineto{\pgfqpoint{0.733822in}{0.712870in}}%
\pgfpathlineto{\pgfqpoint{0.725184in}{0.726481in}}%
\pgfpathlineto{\pgfqpoint{0.720516in}{0.739985in}}%
\pgfpathlineto{\pgfqpoint{0.720472in}{0.740092in}}%
\pgfpathlineto{\pgfqpoint{0.719515in}{0.753703in}}%
\pgfpathlineto{\pgfqpoint{0.720516in}{0.757299in}}%
\pgfpathlineto{\pgfqpoint{0.722830in}{0.767314in}}%
\pgfpathlineto{\pgfqpoint{0.729895in}{0.780925in}}%
\pgfpathlineto{\pgfqpoint{0.736173in}{0.788867in}}%
\pgfpathlineto{\pgfqpoint{0.740516in}{0.794536in}}%
\pgfpathlineto{\pgfqpoint{0.751829in}{0.805974in}}%
\pgfpathlineto{\pgfqpoint{0.754170in}{0.808147in}}%
\pgfpathlineto{\pgfqpoint{0.767486in}{0.819001in}}%
\pgfpathlineto{\pgfqpoint{0.771628in}{0.821759in}}%
\pgfpathlineto{\pgfqpoint{0.783142in}{0.829298in}}%
\pgfpathlineto{\pgfqpoint{0.796298in}{0.835370in}}%
\pgfpathlineto{\pgfqpoint{0.798799in}{0.836657in}}%
\pgfpathlineto{\pgfqpoint{0.814455in}{0.841191in}}%
\pgfpathlineto{\pgfqpoint{0.830112in}{0.841946in}}%
\pgfpathlineto{\pgfqpoint{0.845769in}{0.838926in}}%
\pgfpathlineto{\pgfqpoint{0.854069in}{0.835370in}}%
\pgfpathlineto{\pgfqpoint{0.861425in}{0.832563in}}%
\pgfpathlineto{\pgfqpoint{0.877082in}{0.823434in}}%
\pgfpathlineto{\pgfqpoint{0.879341in}{0.821759in}}%
\pgfpathlineto{\pgfqpoint{0.892738in}{0.811583in}}%
\pgfpathlineto{\pgfqpoint{0.896690in}{0.808148in}}%
\pgfpathlineto{\pgfqpoint{0.908395in}{0.796500in}}%
\pgfpathlineto{\pgfqpoint{0.910322in}{0.794536in}}%
\pgfpathlineto{\pgfqpoint{0.920823in}{0.780925in}}%
\pgfpathlineto{\pgfqpoint{0.924051in}{0.774530in}}%
\pgfpathlineto{\pgfqpoint{0.928142in}{0.767314in}}%
\pgfpathlineto{\pgfqpoint{0.931616in}{0.753703in}}%
\pgfpathlineto{\pgfqpoint{0.930748in}{0.740092in}}%
\pgfpathlineto{\pgfqpoint{0.925532in}{0.726481in}}%
\pgfpathlineto{\pgfqpoint{0.924051in}{0.724307in}}%
\pgfpathlineto{\pgfqpoint{0.917068in}{0.712870in}}%
\pgfpathlineto{\pgfqpoint{0.908395in}{0.702860in}}%
\pgfpathlineto{\pgfqpoint{0.905223in}{0.699259in}}%
\pgfpathlineto{\pgfqpoint{0.892738in}{0.687682in}}%
\pgfpathlineto{\pgfqpoint{0.890238in}{0.685648in}}%
\pgfpathlineto{\pgfqpoint{0.877082in}{0.675813in}}%
\pgfpathlineto{\pgfqpoint{0.870560in}{0.672036in}}%
\pgfpathlineto{\pgfqpoint{0.861425in}{0.666579in}}%
\pgfpathlineto{\pgfqpoint{0.845769in}{0.660437in}}%
\pgfpathlineto{\pgfqpoint{0.834248in}{0.658425in}}%
\pgfpathlineto{\pgfqpoint{0.830112in}{0.657555in}}%
\pgfpathlineto{\pgfqpoint{0.814455in}{0.658387in}}%
\pgfpathlineto{\pgfqpoint{0.814332in}{0.658425in}}%
\pgfpathclose%
\pgfpathmoveto{\pgfqpoint{1.127587in}{0.630345in}}%
\pgfpathlineto{\pgfqpoint{1.143243in}{0.630345in}}%
\pgfpathlineto{\pgfqpoint{1.144304in}{0.631203in}}%
\pgfpathlineto{\pgfqpoint{1.158900in}{0.636354in}}%
\pgfpathlineto{\pgfqpoint{1.170975in}{0.644814in}}%
\pgfpathlineto{\pgfqpoint{1.174556in}{0.646579in}}%
\pgfpathlineto{\pgfqpoint{1.190213in}{0.657816in}}%
\pgfpathlineto{\pgfqpoint{1.190900in}{0.658425in}}%
\pgfpathlineto{\pgfqpoint{1.205870in}{0.669993in}}%
\pgfpathlineto{\pgfqpoint{1.208226in}{0.672036in}}%
\pgfpathlineto{\pgfqpoint{1.221526in}{0.683470in}}%
\pgfpathlineto{\pgfqpoint{1.224025in}{0.685648in}}%
\pgfpathlineto{\pgfqpoint{1.237183in}{0.698260in}}%
\pgfpathlineto{\pgfqpoint{1.238343in}{0.699259in}}%
\pgfpathlineto{\pgfqpoint{1.251411in}{0.712870in}}%
\pgfpathlineto{\pgfqpoint{1.252839in}{0.714954in}}%
\pgfpathlineto{\pgfqpoint{1.263509in}{0.726481in}}%
\pgfpathlineto{\pgfqpoint{1.268496in}{0.736255in}}%
\pgfpathlineto{\pgfqpoint{1.272354in}{0.740092in}}%
\pgfpathlineto{\pgfqpoint{1.274668in}{0.753703in}}%
\pgfpathlineto{\pgfqpoint{1.268496in}{0.762818in}}%
\pgfpathlineto{\pgfqpoint{1.266962in}{0.767314in}}%
\pgfpathlineto{\pgfqpoint{1.256595in}{0.780925in}}%
\pgfpathlineto{\pgfqpoint{1.252839in}{0.784217in}}%
\pgfpathlineto{\pgfqpoint{1.244044in}{0.794536in}}%
\pgfpathlineto{\pgfqpoint{1.237183in}{0.800817in}}%
\pgfpathlineto{\pgfqpoint{1.230024in}{0.808148in}}%
\pgfpathlineto{\pgfqpoint{1.221526in}{0.815736in}}%
\pgfpathlineto{\pgfqpoint{1.214702in}{0.821759in}}%
\pgfpathlineto{\pgfqpoint{1.205870in}{0.829298in}}%
\pgfpathlineto{\pgfqpoint{1.197879in}{0.835370in}}%
\pgfpathlineto{\pgfqpoint{1.190213in}{0.841709in}}%
\pgfpathlineto{\pgfqpoint{1.179211in}{0.848981in}}%
\pgfpathlineto{\pgfqpoint{1.174556in}{0.852922in}}%
\pgfpathlineto{\pgfqpoint{1.158900in}{0.862109in}}%
\pgfpathlineto{\pgfqpoint{1.157278in}{0.862592in}}%
\pgfpathlineto{\pgfqpoint{1.143243in}{0.870011in}}%
\pgfpathlineto{\pgfqpoint{1.127587in}{0.870011in}}%
\pgfpathlineto{\pgfqpoint{1.113552in}{0.862592in}}%
\pgfpathlineto{\pgfqpoint{1.111930in}{0.862109in}}%
\pgfpathlineto{\pgfqpoint{1.096274in}{0.852922in}}%
\pgfpathlineto{\pgfqpoint{1.091619in}{0.848981in}}%
\pgfpathlineto{\pgfqpoint{1.080617in}{0.841709in}}%
\pgfpathlineto{\pgfqpoint{1.072951in}{0.835370in}}%
\pgfpathlineto{\pgfqpoint{1.064960in}{0.829298in}}%
\pgfpathlineto{\pgfqpoint{1.056128in}{0.821759in}}%
\pgfpathlineto{\pgfqpoint{1.049304in}{0.815736in}}%
\pgfpathlineto{\pgfqpoint{1.040806in}{0.808148in}}%
\pgfpathlineto{\pgfqpoint{1.033647in}{0.800817in}}%
\pgfpathlineto{\pgfqpoint{1.026786in}{0.794536in}}%
\pgfpathlineto{\pgfqpoint{1.017991in}{0.784217in}}%
\pgfpathlineto{\pgfqpoint{1.014235in}{0.780925in}}%
\pgfpathlineto{\pgfqpoint{1.003868in}{0.767314in}}%
\pgfpathlineto{\pgfqpoint{1.002334in}{0.762818in}}%
\pgfpathlineto{\pgfqpoint{0.996162in}{0.753703in}}%
\pgfpathlineto{\pgfqpoint{0.998476in}{0.740092in}}%
\pgfpathlineto{\pgfqpoint{1.002334in}{0.736255in}}%
\pgfpathlineto{\pgfqpoint{1.007321in}{0.726481in}}%
\pgfpathlineto{\pgfqpoint{1.017991in}{0.714954in}}%
\pgfpathlineto{\pgfqpoint{1.019419in}{0.712870in}}%
\pgfpathlineto{\pgfqpoint{1.032487in}{0.699259in}}%
\pgfpathlineto{\pgfqpoint{1.033647in}{0.698260in}}%
\pgfpathlineto{\pgfqpoint{1.046805in}{0.685648in}}%
\pgfpathlineto{\pgfqpoint{1.049304in}{0.683470in}}%
\pgfpathlineto{\pgfqpoint{1.062604in}{0.672036in}}%
\pgfpathlineto{\pgfqpoint{1.064960in}{0.669993in}}%
\pgfpathlineto{\pgfqpoint{1.079930in}{0.658425in}}%
\pgfpathlineto{\pgfqpoint{1.080617in}{0.657816in}}%
\pgfpathlineto{\pgfqpoint{1.096274in}{0.646579in}}%
\pgfpathlineto{\pgfqpoint{1.099855in}{0.644814in}}%
\pgfpathlineto{\pgfqpoint{1.111930in}{0.636354in}}%
\pgfpathlineto{\pgfqpoint{1.126526in}{0.631203in}}%
\pgfpathlineto{\pgfqpoint{1.127587in}{0.630345in}}%
\pgfpathclose%
\pgfpathmoveto{\pgfqpoint{1.125533in}{0.658425in}}%
\pgfpathlineto{\pgfqpoint{1.111930in}{0.661392in}}%
\pgfpathlineto{\pgfqpoint{1.096274in}{0.668218in}}%
\pgfpathlineto{\pgfqpoint{1.090233in}{0.672036in}}%
\pgfpathlineto{\pgfqpoint{1.080617in}{0.677945in}}%
\pgfpathlineto{\pgfqpoint{1.070647in}{0.685648in}}%
\pgfpathlineto{\pgfqpoint{1.064960in}{0.690441in}}%
\pgfpathlineto{\pgfqpoint{1.055584in}{0.699259in}}%
\pgfpathlineto{\pgfqpoint{1.049304in}{0.706497in}}%
\pgfpathlineto{\pgfqpoint{1.043752in}{0.712870in}}%
\pgfpathlineto{\pgfqpoint{1.035323in}{0.726481in}}%
\pgfpathlineto{\pgfqpoint{1.033647in}{0.731383in}}%
\pgfpathlineto{\pgfqpoint{1.030182in}{0.740092in}}%
\pgfpathlineto{\pgfqpoint{1.029274in}{0.753703in}}%
\pgfpathlineto{\pgfqpoint{1.032910in}{0.767314in}}%
\pgfpathlineto{\pgfqpoint{1.033647in}{0.768575in}}%
\pgfpathlineto{\pgfqpoint{1.039920in}{0.780925in}}%
\pgfpathlineto{\pgfqpoint{1.049304in}{0.792917in}}%
\pgfpathlineto{\pgfqpoint{1.050559in}{0.794536in}}%
\pgfpathlineto{\pgfqpoint{1.064111in}{0.808148in}}%
\pgfpathlineto{\pgfqpoint{1.064960in}{0.808896in}}%
\pgfpathlineto{\pgfqpoint{1.080617in}{0.821246in}}%
\pgfpathlineto{\pgfqpoint{1.081429in}{0.821759in}}%
\pgfpathlineto{\pgfqpoint{1.096274in}{0.830996in}}%
\pgfpathlineto{\pgfqpoint{1.106640in}{0.835370in}}%
\pgfpathlineto{\pgfqpoint{1.111930in}{0.837867in}}%
\pgfpathlineto{\pgfqpoint{1.127587in}{0.841644in}}%
\pgfpathlineto{\pgfqpoint{1.143243in}{0.841644in}}%
\pgfpathlineto{\pgfqpoint{1.158900in}{0.837867in}}%
\pgfpathlineto{\pgfqpoint{1.164190in}{0.835370in}}%
\pgfpathlineto{\pgfqpoint{1.174556in}{0.830996in}}%
\pgfpathlineto{\pgfqpoint{1.189401in}{0.821759in}}%
\pgfpathlineto{\pgfqpoint{1.190213in}{0.821246in}}%
\pgfpathlineto{\pgfqpoint{1.205870in}{0.808896in}}%
\pgfpathlineto{\pgfqpoint{1.206719in}{0.808148in}}%
\pgfpathlineto{\pgfqpoint{1.220271in}{0.794536in}}%
\pgfpathlineto{\pgfqpoint{1.221526in}{0.792917in}}%
\pgfpathlineto{\pgfqpoint{1.230910in}{0.780925in}}%
\pgfpathlineto{\pgfqpoint{1.237183in}{0.768575in}}%
\pgfpathlineto{\pgfqpoint{1.237920in}{0.767314in}}%
\pgfpathlineto{\pgfqpoint{1.241556in}{0.753703in}}%
\pgfpathlineto{\pgfqpoint{1.240648in}{0.740092in}}%
\pgfpathlineto{\pgfqpoint{1.237183in}{0.731383in}}%
\pgfpathlineto{\pgfqpoint{1.235507in}{0.726481in}}%
\pgfpathlineto{\pgfqpoint{1.227078in}{0.712870in}}%
\pgfpathlineto{\pgfqpoint{1.221526in}{0.706497in}}%
\pgfpathlineto{\pgfqpoint{1.215246in}{0.699259in}}%
\pgfpathlineto{\pgfqpoint{1.205870in}{0.690441in}}%
\pgfpathlineto{\pgfqpoint{1.200183in}{0.685648in}}%
\pgfpathlineto{\pgfqpoint{1.190213in}{0.677945in}}%
\pgfpathlineto{\pgfqpoint{1.180597in}{0.672036in}}%
\pgfpathlineto{\pgfqpoint{1.174556in}{0.668218in}}%
\pgfpathlineto{\pgfqpoint{1.158900in}{0.661392in}}%
\pgfpathlineto{\pgfqpoint{1.145297in}{0.658425in}}%
\pgfpathlineto{\pgfqpoint{1.143243in}{0.657888in}}%
\pgfpathlineto{\pgfqpoint{1.127587in}{0.657888in}}%
\pgfpathlineto{\pgfqpoint{1.125533in}{0.658425in}}%
\pgfpathclose%
\pgfpathmoveto{\pgfqpoint{1.440718in}{0.629316in}}%
\pgfpathlineto{\pgfqpoint{1.452213in}{0.631203in}}%
\pgfpathlineto{\pgfqpoint{1.456375in}{0.631497in}}%
\pgfpathlineto{\pgfqpoint{1.472031in}{0.638123in}}%
\pgfpathlineto{\pgfqpoint{1.480792in}{0.644814in}}%
\pgfpathlineto{\pgfqpoint{1.487688in}{0.648532in}}%
\pgfpathlineto{\pgfqpoint{1.500751in}{0.658425in}}%
\pgfpathlineto{\pgfqpoint{1.503344in}{0.660121in}}%
\pgfpathlineto{\pgfqpoint{1.518209in}{0.672036in}}%
\pgfpathlineto{\pgfqpoint{1.519001in}{0.672662in}}%
\pgfpathlineto{\pgfqpoint{1.533938in}{0.685648in}}%
\pgfpathlineto{\pgfqpoint{1.534657in}{0.686336in}}%
\pgfpathlineto{\pgfqpoint{1.548364in}{0.699259in}}%
\pgfpathlineto{\pgfqpoint{1.550314in}{0.701513in}}%
\pgfpathlineto{\pgfqpoint{1.561694in}{0.712870in}}%
\pgfpathlineto{\pgfqpoint{1.565971in}{0.718864in}}%
\pgfpathlineto{\pgfqpoint{1.573668in}{0.726481in}}%
\pgfpathlineto{\pgfqpoint{1.581289in}{0.740092in}}%
\pgfpathlineto{\pgfqpoint{1.581627in}{0.743710in}}%
\pgfpathlineto{\pgfqpoint{1.583797in}{0.753703in}}%
\pgfpathlineto{\pgfqpoint{1.581627in}{0.756227in}}%
\pgfpathlineto{\pgfqpoint{1.577481in}{0.767314in}}%
\pgfpathlineto{\pgfqpoint{1.566034in}{0.780925in}}%
\pgfpathlineto{\pgfqpoint{1.565971in}{0.780976in}}%
\pgfpathlineto{\pgfqpoint{1.553935in}{0.794536in}}%
\pgfpathlineto{\pgfqpoint{1.550314in}{0.797727in}}%
\pgfpathlineto{\pgfqpoint{1.540028in}{0.808148in}}%
\pgfpathlineto{\pgfqpoint{1.534657in}{0.812894in}}%
\pgfpathlineto{\pgfqpoint{1.524725in}{0.821759in}}%
\pgfpathlineto{\pgfqpoint{1.519001in}{0.826696in}}%
\pgfpathlineto{\pgfqpoint{1.507934in}{0.835370in}}%
\pgfpathlineto{\pgfqpoint{1.503344in}{0.839277in}}%
\pgfpathlineto{\pgfqpoint{1.489430in}{0.848981in}}%
\pgfpathlineto{\pgfqpoint{1.487688in}{0.850534in}}%
\pgfpathlineto{\pgfqpoint{1.472031in}{0.860640in}}%
\pgfpathlineto{\pgfqpoint{1.466557in}{0.862592in}}%
\pgfpathlineto{\pgfqpoint{1.456375in}{0.869017in}}%
\pgfpathlineto{\pgfqpoint{1.440718in}{0.870674in}}%
\pgfpathlineto{\pgfqpoint{1.425061in}{0.864042in}}%
\pgfpathlineto{\pgfqpoint{1.423498in}{0.862592in}}%
\pgfpathlineto{\pgfqpoint{1.409405in}{0.855128in}}%
\pgfpathlineto{\pgfqpoint{1.401725in}{0.848981in}}%
\pgfpathlineto{\pgfqpoint{1.393748in}{0.844013in}}%
\pgfpathlineto{\pgfqpoint{1.382946in}{0.835370in}}%
\pgfpathlineto{\pgfqpoint{1.378092in}{0.831809in}}%
\pgfpathlineto{\pgfqpoint{1.366152in}{0.821759in}}%
\pgfpathlineto{\pgfqpoint{1.362435in}{0.818528in}}%
\pgfpathlineto{\pgfqpoint{1.350875in}{0.808148in}}%
\pgfpathlineto{\pgfqpoint{1.346779in}{0.803927in}}%
\pgfpathlineto{\pgfqpoint{1.336837in}{0.794536in}}%
\pgfpathlineto{\pgfqpoint{1.331122in}{0.787602in}}%
\pgfpathlineto{\pgfqpoint{1.324051in}{0.780925in}}%
\pgfpathlineto{\pgfqpoint{1.315466in}{0.768673in}}%
\pgfpathlineto{\pgfqpoint{1.313797in}{0.767314in}}%
\pgfpathlineto{\pgfqpoint{1.306169in}{0.753703in}}%
\pgfpathlineto{\pgfqpoint{1.308075in}{0.740092in}}%
\pgfpathlineto{\pgfqpoint{1.315466in}{0.731240in}}%
\pgfpathlineto{\pgfqpoint{1.317710in}{0.726481in}}%
\pgfpathlineto{\pgfqpoint{1.329336in}{0.712870in}}%
\pgfpathlineto{\pgfqpoint{1.331122in}{0.711355in}}%
\pgfpathlineto{\pgfqpoint{1.342285in}{0.699259in}}%
\pgfpathlineto{\pgfqpoint{1.346779in}{0.695269in}}%
\pgfpathlineto{\pgfqpoint{1.356755in}{0.685648in}}%
\pgfpathlineto{\pgfqpoint{1.362435in}{0.680671in}}%
\pgfpathlineto{\pgfqpoint{1.372631in}{0.672036in}}%
\pgfpathlineto{\pgfqpoint{1.378092in}{0.667368in}}%
\pgfpathlineto{\pgfqpoint{1.390078in}{0.658425in}}%
\pgfpathlineto{\pgfqpoint{1.393748in}{0.655277in}}%
\pgfpathlineto{\pgfqpoint{1.409346in}{0.644814in}}%
\pgfpathlineto{\pgfqpoint{1.409405in}{0.644759in}}%
\pgfpathlineto{\pgfqpoint{1.425061in}{0.634808in}}%
\pgfpathlineto{\pgfqpoint{1.437815in}{0.631203in}}%
\pgfpathlineto{\pgfqpoint{1.440718in}{0.629316in}}%
\pgfpathclose%
\pgfpathmoveto{\pgfqpoint{1.436582in}{0.658425in}}%
\pgfpathlineto{\pgfqpoint{1.425061in}{0.660437in}}%
\pgfpathlineto{\pgfqpoint{1.409405in}{0.666579in}}%
\pgfpathlineto{\pgfqpoint{1.400270in}{0.672036in}}%
\pgfpathlineto{\pgfqpoint{1.393748in}{0.675813in}}%
\pgfpathlineto{\pgfqpoint{1.380592in}{0.685648in}}%
\pgfpathlineto{\pgfqpoint{1.378092in}{0.687682in}}%
\pgfpathlineto{\pgfqpoint{1.365607in}{0.699259in}}%
\pgfpathlineto{\pgfqpoint{1.362435in}{0.702860in}}%
\pgfpathlineto{\pgfqpoint{1.353762in}{0.712870in}}%
\pgfpathlineto{\pgfqpoint{1.346779in}{0.724307in}}%
\pgfpathlineto{\pgfqpoint{1.345298in}{0.726481in}}%
\pgfpathlineto{\pgfqpoint{1.340082in}{0.740092in}}%
\pgfpathlineto{\pgfqpoint{1.339214in}{0.753703in}}%
\pgfpathlineto{\pgfqpoint{1.342688in}{0.767314in}}%
\pgfpathlineto{\pgfqpoint{1.346779in}{0.774530in}}%
\pgfpathlineto{\pgfqpoint{1.350007in}{0.780925in}}%
\pgfpathlineto{\pgfqpoint{1.360508in}{0.794536in}}%
\pgfpathlineto{\pgfqpoint{1.362435in}{0.796500in}}%
\pgfpathlineto{\pgfqpoint{1.374140in}{0.808148in}}%
\pgfpathlineto{\pgfqpoint{1.378092in}{0.811583in}}%
\pgfpathlineto{\pgfqpoint{1.391489in}{0.821759in}}%
\pgfpathlineto{\pgfqpoint{1.393748in}{0.823434in}}%
\pgfpathlineto{\pgfqpoint{1.409405in}{0.832563in}}%
\pgfpathlineto{\pgfqpoint{1.416761in}{0.835370in}}%
\pgfpathlineto{\pgfqpoint{1.425061in}{0.838926in}}%
\pgfpathlineto{\pgfqpoint{1.440718in}{0.841946in}}%
\pgfpathlineto{\pgfqpoint{1.456375in}{0.841191in}}%
\pgfpathlineto{\pgfqpoint{1.472031in}{0.836657in}}%
\pgfpathlineto{\pgfqpoint{1.474532in}{0.835370in}}%
\pgfpathlineto{\pgfqpoint{1.487688in}{0.829298in}}%
\pgfpathlineto{\pgfqpoint{1.499202in}{0.821759in}}%
\pgfpathlineto{\pgfqpoint{1.503344in}{0.819001in}}%
\pgfpathlineto{\pgfqpoint{1.516660in}{0.808147in}}%
\pgfpathlineto{\pgfqpoint{1.519001in}{0.805974in}}%
\pgfpathlineto{\pgfqpoint{1.530314in}{0.794536in}}%
\pgfpathlineto{\pgfqpoint{1.534657in}{0.788867in}}%
\pgfpathlineto{\pgfqpoint{1.540935in}{0.780925in}}%
\pgfpathlineto{\pgfqpoint{1.548000in}{0.767314in}}%
\pgfpathlineto{\pgfqpoint{1.550314in}{0.757299in}}%
\pgfpathlineto{\pgfqpoint{1.551315in}{0.753703in}}%
\pgfpathlineto{\pgfqpoint{1.550358in}{0.740092in}}%
\pgfpathlineto{\pgfqpoint{1.550314in}{0.739985in}}%
\pgfpathlineto{\pgfqpoint{1.545646in}{0.726481in}}%
\pgfpathlineto{\pgfqpoint{1.537008in}{0.712870in}}%
\pgfpathlineto{\pgfqpoint{1.534657in}{0.710199in}}%
\pgfpathlineto{\pgfqpoint{1.525271in}{0.699259in}}%
\pgfpathlineto{\pgfqpoint{1.519001in}{0.693300in}}%
\pgfpathlineto{\pgfqpoint{1.510198in}{0.685648in}}%
\pgfpathlineto{\pgfqpoint{1.503344in}{0.680196in}}%
\pgfpathlineto{\pgfqpoint{1.490760in}{0.672036in}}%
\pgfpathlineto{\pgfqpoint{1.487688in}{0.669993in}}%
\pgfpathlineto{\pgfqpoint{1.472031in}{0.662483in}}%
\pgfpathlineto{\pgfqpoint{1.456498in}{0.658425in}}%
\pgfpathlineto{\pgfqpoint{1.456375in}{0.658387in}}%
\pgfpathlineto{\pgfqpoint{1.440718in}{0.657555in}}%
\pgfpathlineto{\pgfqpoint{1.436582in}{0.658425in}}%
\pgfpathclose%
\pgfpathmoveto{\pgfqpoint{1.753849in}{0.628802in}}%
\pgfpathlineto{\pgfqpoint{1.761184in}{0.631203in}}%
\pgfpathlineto{\pgfqpoint{1.769506in}{0.632379in}}%
\pgfpathlineto{\pgfqpoint{1.785162in}{0.640113in}}%
\pgfpathlineto{\pgfqpoint{1.790860in}{0.644814in}}%
\pgfpathlineto{\pgfqpoint{1.800819in}{0.650633in}}%
\pgfpathlineto{\pgfqpoint{1.810645in}{0.658425in}}%
\pgfpathlineto{\pgfqpoint{1.816476in}{0.662428in}}%
\pgfpathlineto{\pgfqpoint{1.828164in}{0.672036in}}%
\pgfpathlineto{\pgfqpoint{1.832132in}{0.675258in}}%
\pgfpathlineto{\pgfqpoint{1.844005in}{0.685648in}}%
\pgfpathlineto{\pgfqpoint{1.847789in}{0.689292in}}%
\pgfpathlineto{\pgfqpoint{1.858517in}{0.699259in}}%
\pgfpathlineto{\pgfqpoint{1.863445in}{0.704864in}}%
\pgfpathlineto{\pgfqpoint{1.871841in}{0.712870in}}%
\pgfpathlineto{\pgfqpoint{1.879102in}{0.722568in}}%
\pgfpathlineto{\pgfqpoint{1.883488in}{0.726481in}}%
\pgfpathlineto{\pgfqpoint{1.892034in}{0.740092in}}%
\pgfpathlineto{\pgfqpoint{1.893456in}{0.753703in}}%
\pgfpathlineto{\pgfqpoint{1.887763in}{0.767314in}}%
\pgfpathlineto{\pgfqpoint{1.879102in}{0.776596in}}%
\pgfpathlineto{\pgfqpoint{1.876422in}{0.780925in}}%
\pgfpathlineto{\pgfqpoint{1.863611in}{0.794536in}}%
\pgfpathlineto{\pgfqpoint{1.863445in}{0.794676in}}%
\pgfpathlineto{\pgfqpoint{1.849933in}{0.808148in}}%
\pgfpathlineto{\pgfqpoint{1.847789in}{0.810015in}}%
\pgfpathlineto{\pgfqpoint{1.834723in}{0.821759in}}%
\pgfpathlineto{\pgfqpoint{1.832132in}{0.824008in}}%
\pgfpathlineto{\pgfqpoint{1.818021in}{0.835370in}}%
\pgfpathlineto{\pgfqpoint{1.816476in}{0.836719in}}%
\pgfpathlineto{\pgfqpoint{1.800819in}{0.848226in}}%
\pgfpathlineto{\pgfqpoint{1.799428in}{0.848981in}}%
\pgfpathlineto{\pgfqpoint{1.785162in}{0.858987in}}%
\pgfpathlineto{\pgfqpoint{1.776475in}{0.862592in}}%
\pgfpathlineto{\pgfqpoint{1.769506in}{0.867691in}}%
\pgfpathlineto{\pgfqpoint{1.753849in}{0.871005in}}%
\pgfpathlineto{\pgfqpoint{1.738193in}{0.866033in}}%
\pgfpathlineto{\pgfqpoint{1.734050in}{0.862592in}}%
\pgfpathlineto{\pgfqpoint{1.722536in}{0.857150in}}%
\pgfpathlineto{\pgfqpoint{1.711671in}{0.848981in}}%
\pgfpathlineto{\pgfqpoint{1.706880in}{0.846186in}}%
\pgfpathlineto{\pgfqpoint{1.692851in}{0.835370in}}%
\pgfpathlineto{\pgfqpoint{1.691223in}{0.834222in}}%
\pgfpathlineto{\pgfqpoint{1.676150in}{0.821759in}}%
\pgfpathlineto{\pgfqpoint{1.675567in}{0.821261in}}%
\pgfpathlineto{\pgfqpoint{1.660983in}{0.808148in}}%
\pgfpathlineto{\pgfqpoint{1.659910in}{0.807041in}}%
\pgfpathlineto{\pgfqpoint{1.647004in}{0.794536in}}%
\pgfpathlineto{\pgfqpoint{1.644253in}{0.791106in}}%
\pgfpathlineto{\pgfqpoint{1.634134in}{0.780925in}}%
\pgfpathlineto{\pgfqpoint{1.628597in}{0.772476in}}%
\pgfpathlineto{\pgfqpoint{1.623145in}{0.767314in}}%
\pgfpathlineto{\pgfqpoint{1.616638in}{0.753703in}}%
\pgfpathlineto{\pgfqpoint{1.618264in}{0.740092in}}%
\pgfpathlineto{\pgfqpoint{1.628031in}{0.726481in}}%
\pgfpathlineto{\pgfqpoint{1.628597in}{0.726035in}}%
\pgfpathlineto{\pgfqpoint{1.639032in}{0.712870in}}%
\pgfpathlineto{\pgfqpoint{1.644253in}{0.708152in}}%
\pgfpathlineto{\pgfqpoint{1.652240in}{0.699259in}}%
\pgfpathlineto{\pgfqpoint{1.659910in}{0.692274in}}%
\pgfpathlineto{\pgfqpoint{1.666771in}{0.685648in}}%
\pgfpathlineto{\pgfqpoint{1.675567in}{0.677930in}}%
\pgfpathlineto{\pgfqpoint{1.682661in}{0.672036in}}%
\pgfpathlineto{\pgfqpoint{1.691223in}{0.664845in}}%
\pgfpathlineto{\pgfqpoint{1.700173in}{0.658425in}}%
\pgfpathlineto{\pgfqpoint{1.706880in}{0.652882in}}%
\pgfpathlineto{\pgfqpoint{1.719722in}{0.644814in}}%
\pgfpathlineto{\pgfqpoint{1.722536in}{0.642325in}}%
\pgfpathlineto{\pgfqpoint{1.738193in}{0.633483in}}%
\pgfpathlineto{\pgfqpoint{1.748943in}{0.631203in}}%
\pgfpathlineto{\pgfqpoint{1.753849in}{0.628802in}}%
\pgfpathclose%
\pgfpathmoveto{\pgfqpoint{1.747307in}{0.658425in}}%
\pgfpathlineto{\pgfqpoint{1.738193in}{0.659620in}}%
\pgfpathlineto{\pgfqpoint{1.722536in}{0.665077in}}%
\pgfpathlineto{\pgfqpoint{1.710134in}{0.672036in}}%
\pgfpathlineto{\pgfqpoint{1.706880in}{0.673801in}}%
\pgfpathlineto{\pgfqpoint{1.691223in}{0.685045in}}%
\pgfpathlineto{\pgfqpoint{1.690500in}{0.685648in}}%
\pgfpathlineto{\pgfqpoint{1.675603in}{0.699259in}}%
\pgfpathlineto{\pgfqpoint{1.675567in}{0.699300in}}%
\pgfpathlineto{\pgfqpoint{1.663825in}{0.712870in}}%
\pgfpathlineto{\pgfqpoint{1.659910in}{0.719291in}}%
\pgfpathlineto{\pgfqpoint{1.655137in}{0.726481in}}%
\pgfpathlineto{\pgfqpoint{1.650123in}{0.740092in}}%
\pgfpathlineto{\pgfqpoint{1.649289in}{0.753703in}}%
\pgfpathlineto{\pgfqpoint{1.652629in}{0.767314in}}%
\pgfpathlineto{\pgfqpoint{1.659910in}{0.780492in}}%
\pgfpathlineto{\pgfqpoint{1.660129in}{0.780925in}}%
\pgfpathlineto{\pgfqpoint{1.670465in}{0.794536in}}%
\pgfpathlineto{\pgfqpoint{1.675567in}{0.799742in}}%
\pgfpathlineto{\pgfqpoint{1.684176in}{0.808148in}}%
\pgfpathlineto{\pgfqpoint{1.691223in}{0.814166in}}%
\pgfpathlineto{\pgfqpoint{1.701621in}{0.821759in}}%
\pgfpathlineto{\pgfqpoint{1.706880in}{0.825517in}}%
\pgfpathlineto{\pgfqpoint{1.722536in}{0.833999in}}%
\pgfpathlineto{\pgfqpoint{1.726560in}{0.835370in}}%
\pgfpathlineto{\pgfqpoint{1.738193in}{0.839832in}}%
\pgfpathlineto{\pgfqpoint{1.753849in}{0.842097in}}%
\pgfpathlineto{\pgfqpoint{1.769506in}{0.840588in}}%
\pgfpathlineto{\pgfqpoint{1.784946in}{0.835370in}}%
\pgfpathlineto{\pgfqpoint{1.785162in}{0.835305in}}%
\pgfpathlineto{\pgfqpoint{1.800819in}{0.827472in}}%
\pgfpathlineto{\pgfqpoint{1.809151in}{0.821759in}}%
\pgfpathlineto{\pgfqpoint{1.816476in}{0.816640in}}%
\pgfpathlineto{\pgfqpoint{1.826635in}{0.808148in}}%
\pgfpathlineto{\pgfqpoint{1.832132in}{0.802903in}}%
\pgfpathlineto{\pgfqpoint{1.840353in}{0.794536in}}%
\pgfpathlineto{\pgfqpoint{1.847789in}{0.784763in}}%
\pgfpathlineto{\pgfqpoint{1.850868in}{0.780925in}}%
\pgfpathlineto{\pgfqpoint{1.858142in}{0.767314in}}%
\pgfpathlineto{\pgfqpoint{1.861369in}{0.753703in}}%
\pgfpathlineto{\pgfqpoint{1.860563in}{0.740092in}}%
\pgfpathlineto{\pgfqpoint{1.855719in}{0.726481in}}%
\pgfpathlineto{\pgfqpoint{1.847789in}{0.714295in}}%
\pgfpathlineto{\pgfqpoint{1.846917in}{0.712870in}}%
\pgfpathlineto{\pgfqpoint{1.835273in}{0.699259in}}%
\pgfpathlineto{\pgfqpoint{1.832132in}{0.696254in}}%
\pgfpathlineto{\pgfqpoint{1.820256in}{0.685648in}}%
\pgfpathlineto{\pgfqpoint{1.816476in}{0.682564in}}%
\pgfpathlineto{\pgfqpoint{1.801011in}{0.672036in}}%
\pgfpathlineto{\pgfqpoint{1.800819in}{0.671903in}}%
\pgfpathlineto{\pgfqpoint{1.785162in}{0.663712in}}%
\pgfpathlineto{\pgfqpoint{1.769506in}{0.658939in}}%
\pgfpathlineto{\pgfqpoint{1.763630in}{0.658425in}}%
\pgfpathlineto{\pgfqpoint{1.753849in}{0.657389in}}%
\pgfpathlineto{\pgfqpoint{1.747307in}{0.658425in}}%
\pgfpathclose%
\pgfpathmoveto{\pgfqpoint{0.501324in}{0.901681in}}%
\pgfpathlineto{\pgfqpoint{0.516981in}{0.897657in}}%
\pgfpathlineto{\pgfqpoint{0.531947in}{0.903425in}}%
\pgfpathlineto{\pgfqpoint{0.532637in}{0.903559in}}%
\pgfpathlineto{\pgfqpoint{0.548294in}{0.911567in}}%
\pgfpathlineto{\pgfqpoint{0.555046in}{0.917036in}}%
\pgfpathlineto{\pgfqpoint{0.563950in}{0.922408in}}%
\pgfpathlineto{\pgfqpoint{0.574306in}{0.930648in}}%
\pgfpathlineto{\pgfqpoint{0.579607in}{0.934409in}}%
\pgfpathlineto{\pgfqpoint{0.591453in}{0.944259in}}%
\pgfpathlineto{\pgfqpoint{0.595263in}{0.947460in}}%
\pgfpathlineto{\pgfqpoint{0.607005in}{0.957870in}}%
\pgfpathlineto{\pgfqpoint{0.610920in}{0.961770in}}%
\pgfpathlineto{\pgfqpoint{0.621279in}{0.971481in}}%
\pgfpathlineto{\pgfqpoint{0.626577in}{0.977706in}}%
\pgfpathlineto{\pgfqpoint{0.634344in}{0.985092in}}%
\pgfpathlineto{\pgfqpoint{0.642233in}{0.995978in}}%
\pgfpathlineto{\pgfqpoint{0.645405in}{0.998703in}}%
\pgfpathlineto{\pgfqpoint{0.653542in}{1.012314in}}%
\pgfpathlineto{\pgfqpoint{0.653542in}{1.025925in}}%
\pgfpathlineto{\pgfqpoint{0.645405in}{1.039536in}}%
\pgfpathlineto{\pgfqpoint{0.642233in}{1.042262in}}%
\pgfpathlineto{\pgfqpoint{0.634344in}{1.053148in}}%
\pgfpathlineto{\pgfqpoint{0.626577in}{1.060533in}}%
\pgfpathlineto{\pgfqpoint{0.621279in}{1.066759in}}%
\pgfpathlineto{\pgfqpoint{0.610920in}{1.076470in}}%
\pgfpathlineto{\pgfqpoint{0.607005in}{1.080370in}}%
\pgfpathlineto{\pgfqpoint{0.595263in}{1.090780in}}%
\pgfpathlineto{\pgfqpoint{0.591453in}{1.093981in}}%
\pgfpathlineto{\pgfqpoint{0.579607in}{1.103831in}}%
\pgfpathlineto{\pgfqpoint{0.574306in}{1.107592in}}%
\pgfpathlineto{\pgfqpoint{0.563950in}{1.115831in}}%
\pgfpathlineto{\pgfqpoint{0.555046in}{1.121203in}}%
\pgfpathlineto{\pgfqpoint{0.548294in}{1.126672in}}%
\pgfpathlineto{\pgfqpoint{0.532637in}{1.134681in}}%
\pgfpathlineto{\pgfqpoint{0.531947in}{1.134814in}}%
\pgfpathlineto{\pgfqpoint{0.516981in}{1.140582in}}%
\pgfpathlineto{\pgfqpoint{0.501324in}{1.136558in}}%
\pgfpathlineto{\pgfqpoint{0.499345in}{1.134814in}}%
\pgfpathlineto{\pgfqpoint{0.485668in}{1.128676in}}%
\pgfpathlineto{\pgfqpoint{0.475779in}{1.121203in}}%
\pgfpathlineto{\pgfqpoint{0.470011in}{1.117967in}}%
\pgfpathlineto{\pgfqpoint{0.456420in}{1.107592in}}%
\pgfpathlineto{\pgfqpoint{0.454354in}{1.106189in}}%
\pgfpathlineto{\pgfqpoint{0.439353in}{1.093981in}}%
\pgfpathlineto{\pgfqpoint{0.438698in}{1.093443in}}%
\pgfpathlineto{\pgfqpoint{0.423913in}{1.080370in}}%
\pgfpathlineto{\pgfqpoint{0.423041in}{1.079504in}}%
\pgfpathlineto{\pgfqpoint{0.409714in}{1.066759in}}%
\pgfpathlineto{\pgfqpoint{0.407385in}{1.063962in}}%
\pgfpathlineto{\pgfqpoint{0.396607in}{1.053148in}}%
\pgfpathlineto{\pgfqpoint{0.391728in}{1.046019in}}%
\pgfpathlineto{\pgfqpoint{0.385061in}{1.039536in}}%
\pgfpathlineto{\pgfqpoint{0.377943in}{1.025925in}}%
\pgfpathlineto{\pgfqpoint{0.377943in}{1.012314in}}%
\pgfpathlineto{\pgfqpoint{0.385061in}{0.998703in}}%
\pgfpathlineto{\pgfqpoint{0.391728in}{0.992220in}}%
\pgfpathlineto{\pgfqpoint{0.396607in}{0.985092in}}%
\pgfpathlineto{\pgfqpoint{0.407385in}{0.974277in}}%
\pgfpathlineto{\pgfqpoint{0.409714in}{0.971481in}}%
\pgfpathlineto{\pgfqpoint{0.423041in}{0.958736in}}%
\pgfpathlineto{\pgfqpoint{0.423913in}{0.957870in}}%
\pgfpathlineto{\pgfqpoint{0.438698in}{0.944797in}}%
\pgfpathlineto{\pgfqpoint{0.439353in}{0.944259in}}%
\pgfpathlineto{\pgfqpoint{0.454354in}{0.932050in}}%
\pgfpathlineto{\pgfqpoint{0.456420in}{0.930648in}}%
\pgfpathlineto{\pgfqpoint{0.470011in}{0.920273in}}%
\pgfpathlineto{\pgfqpoint{0.475779in}{0.917036in}}%
\pgfpathlineto{\pgfqpoint{0.485668in}{0.909564in}}%
\pgfpathlineto{\pgfqpoint{0.499345in}{0.903425in}}%
\pgfpathlineto{\pgfqpoint{0.501324in}{0.901681in}}%
\pgfpathclose%
\pgfpathmoveto{\pgfqpoint{0.494498in}{0.930648in}}%
\pgfpathlineto{\pgfqpoint{0.485668in}{0.933303in}}%
\pgfpathlineto{\pgfqpoint{0.470011in}{0.941295in}}%
\pgfpathlineto{\pgfqpoint{0.465714in}{0.944259in}}%
\pgfpathlineto{\pgfqpoint{0.454354in}{0.952078in}}%
\pgfpathlineto{\pgfqpoint{0.447327in}{0.957870in}}%
\pgfpathlineto{\pgfqpoint{0.438698in}{0.965819in}}%
\pgfpathlineto{\pgfqpoint{0.432948in}{0.971481in}}%
\pgfpathlineto{\pgfqpoint{0.423041in}{0.983752in}}%
\pgfpathlineto{\pgfqpoint{0.421903in}{0.985092in}}%
\pgfpathlineto{\pgfqpoint{0.413818in}{0.998703in}}%
\pgfpathlineto{\pgfqpoint{0.409783in}{1.012314in}}%
\pgfpathlineto{\pgfqpoint{0.409783in}{1.025925in}}%
\pgfpathlineto{\pgfqpoint{0.413818in}{1.039536in}}%
\pgfpathlineto{\pgfqpoint{0.421903in}{1.053148in}}%
\pgfpathlineto{\pgfqpoint{0.423041in}{1.054488in}}%
\pgfpathlineto{\pgfqpoint{0.432948in}{1.066759in}}%
\pgfpathlineto{\pgfqpoint{0.438698in}{1.072421in}}%
\pgfpathlineto{\pgfqpoint{0.447327in}{1.080370in}}%
\pgfpathlineto{\pgfqpoint{0.454354in}{1.086161in}}%
\pgfpathlineto{\pgfqpoint{0.465714in}{1.093981in}}%
\pgfpathlineto{\pgfqpoint{0.470011in}{1.096944in}}%
\pgfpathlineto{\pgfqpoint{0.485668in}{1.104936in}}%
\pgfpathlineto{\pgfqpoint{0.494498in}{1.107592in}}%
\pgfpathlineto{\pgfqpoint{0.501324in}{1.109973in}}%
\pgfpathlineto{\pgfqpoint{0.516981in}{1.111552in}}%
\pgfpathlineto{\pgfqpoint{0.532637in}{1.109182in}}%
\pgfpathlineto{\pgfqpoint{0.536654in}{1.107592in}}%
\pgfpathlineto{\pgfqpoint{0.548294in}{1.103604in}}%
\pgfpathlineto{\pgfqpoint{0.563950in}{1.094949in}}%
\pgfpathlineto{\pgfqpoint{0.565298in}{1.093981in}}%
\pgfpathlineto{\pgfqpoint{0.579607in}{1.083689in}}%
\pgfpathlineto{\pgfqpoint{0.583549in}{1.080370in}}%
\pgfpathlineto{\pgfqpoint{0.595263in}{1.069327in}}%
\pgfpathlineto{\pgfqpoint{0.597865in}{1.066759in}}%
\pgfpathlineto{\pgfqpoint{0.608927in}{1.053148in}}%
\pgfpathlineto{\pgfqpoint{0.610920in}{1.049572in}}%
\pgfpathlineto{\pgfqpoint{0.617031in}{1.039536in}}%
\pgfpathlineto{\pgfqpoint{0.621207in}{1.025925in}}%
\pgfpathlineto{\pgfqpoint{0.621207in}{1.012314in}}%
\pgfpathlineto{\pgfqpoint{0.617031in}{0.998703in}}%
\pgfpathlineto{\pgfqpoint{0.610920in}{0.988667in}}%
\pgfpathlineto{\pgfqpoint{0.608927in}{0.985092in}}%
\pgfpathlineto{\pgfqpoint{0.597865in}{0.971481in}}%
\pgfpathlineto{\pgfqpoint{0.595263in}{0.968912in}}%
\pgfpathlineto{\pgfqpoint{0.583549in}{0.957870in}}%
\pgfpathlineto{\pgfqpoint{0.579607in}{0.954550in}}%
\pgfpathlineto{\pgfqpoint{0.565298in}{0.944259in}}%
\pgfpathlineto{\pgfqpoint{0.563950in}{0.943290in}}%
\pgfpathlineto{\pgfqpoint{0.548294in}{0.934635in}}%
\pgfpathlineto{\pgfqpoint{0.536654in}{0.930648in}}%
\pgfpathlineto{\pgfqpoint{0.532637in}{0.929057in}}%
\pgfpathlineto{\pgfqpoint{0.516981in}{0.926687in}}%
\pgfpathlineto{\pgfqpoint{0.501324in}{0.928267in}}%
\pgfpathlineto{\pgfqpoint{0.494498in}{0.930648in}}%
\pgfpathclose%
\pgfpathmoveto{\pgfqpoint{0.814455in}{0.900071in}}%
\pgfpathlineto{\pgfqpoint{0.830112in}{0.898060in}}%
\pgfpathlineto{\pgfqpoint{0.840596in}{0.903425in}}%
\pgfpathlineto{\pgfqpoint{0.845769in}{0.904759in}}%
\pgfpathlineto{\pgfqpoint{0.861425in}{0.913771in}}%
\pgfpathlineto{\pgfqpoint{0.865211in}{0.917036in}}%
\pgfpathlineto{\pgfqpoint{0.877082in}{0.924682in}}%
\pgfpathlineto{\pgfqpoint{0.884306in}{0.930648in}}%
\pgfpathlineto{\pgfqpoint{0.892738in}{0.936871in}}%
\pgfpathlineto{\pgfqpoint{0.901467in}{0.944259in}}%
\pgfpathlineto{\pgfqpoint{0.908395in}{0.950191in}}%
\pgfpathlineto{\pgfqpoint{0.917068in}{0.957870in}}%
\pgfpathlineto{\pgfqpoint{0.924051in}{0.964816in}}%
\pgfpathlineto{\pgfqpoint{0.931343in}{0.971481in}}%
\pgfpathlineto{\pgfqpoint{0.939708in}{0.981046in}}%
\pgfpathlineto{\pgfqpoint{0.944242in}{0.985092in}}%
\pgfpathlineto{\pgfqpoint{0.954809in}{0.998703in}}%
\pgfpathlineto{\pgfqpoint{0.955364in}{1.000113in}}%
\pgfpathlineto{\pgfqpoint{0.963899in}{1.012314in}}%
\pgfpathlineto{\pgfqpoint{0.963899in}{1.025925in}}%
\pgfpathlineto{\pgfqpoint{0.955364in}{1.038126in}}%
\pgfpathlineto{\pgfqpoint{0.954809in}{1.039536in}}%
\pgfpathlineto{\pgfqpoint{0.944242in}{1.053148in}}%
\pgfpathlineto{\pgfqpoint{0.939708in}{1.057194in}}%
\pgfpathlineto{\pgfqpoint{0.931343in}{1.066759in}}%
\pgfpathlineto{\pgfqpoint{0.924051in}{1.073423in}}%
\pgfpathlineto{\pgfqpoint{0.917068in}{1.080370in}}%
\pgfpathlineto{\pgfqpoint{0.908395in}{1.088048in}}%
\pgfpathlineto{\pgfqpoint{0.901467in}{1.093981in}}%
\pgfpathlineto{\pgfqpoint{0.892738in}{1.101369in}}%
\pgfpathlineto{\pgfqpoint{0.884306in}{1.107592in}}%
\pgfpathlineto{\pgfqpoint{0.877082in}{1.113557in}}%
\pgfpathlineto{\pgfqpoint{0.865211in}{1.121203in}}%
\pgfpathlineto{\pgfqpoint{0.861425in}{1.124468in}}%
\pgfpathlineto{\pgfqpoint{0.845769in}{1.133481in}}%
\pgfpathlineto{\pgfqpoint{0.840596in}{1.134814in}}%
\pgfpathlineto{\pgfqpoint{0.830112in}{1.140180in}}%
\pgfpathlineto{\pgfqpoint{0.814455in}{1.138168in}}%
\pgfpathlineto{\pgfqpoint{0.810042in}{1.134814in}}%
\pgfpathlineto{\pgfqpoint{0.798799in}{1.130478in}}%
\pgfpathlineto{\pgfqpoint{0.785540in}{1.121203in}}%
\pgfpathlineto{\pgfqpoint{0.783142in}{1.119962in}}%
\pgfpathlineto{\pgfqpoint{0.767486in}{1.108601in}}%
\pgfpathlineto{\pgfqpoint{0.766338in}{1.107592in}}%
\pgfpathlineto{\pgfqpoint{0.751829in}{1.096153in}}%
\pgfpathlineto{\pgfqpoint{0.749325in}{1.093981in}}%
\pgfpathlineto{\pgfqpoint{0.736173in}{1.082419in}}%
\pgfpathlineto{\pgfqpoint{0.733822in}{1.080370in}}%
\pgfpathlineto{\pgfqpoint{0.720516in}{1.067356in}}%
\pgfpathlineto{\pgfqpoint{0.719816in}{1.066759in}}%
\pgfpathlineto{\pgfqpoint{0.706890in}{1.053148in}}%
\pgfpathlineto{\pgfqpoint{0.704859in}{1.050034in}}%
\pgfpathlineto{\pgfqpoint{0.695128in}{1.039536in}}%
\pgfpathlineto{\pgfqpoint{0.689203in}{1.026848in}}%
\pgfpathlineto{\pgfqpoint{0.688216in}{1.025925in}}%
\pgfpathlineto{\pgfqpoint{0.688216in}{1.012314in}}%
\pgfpathlineto{\pgfqpoint{0.689203in}{1.011392in}}%
\pgfpathlineto{\pgfqpoint{0.695128in}{0.998703in}}%
\pgfpathlineto{\pgfqpoint{0.704859in}{0.988205in}}%
\pgfpathlineto{\pgfqpoint{0.706890in}{0.985092in}}%
\pgfpathlineto{\pgfqpoint{0.719816in}{0.971481in}}%
\pgfpathlineto{\pgfqpoint{0.720516in}{0.970884in}}%
\pgfpathlineto{\pgfqpoint{0.733822in}{0.957870in}}%
\pgfpathlineto{\pgfqpoint{0.736173in}{0.955821in}}%
\pgfpathlineto{\pgfqpoint{0.749325in}{0.944259in}}%
\pgfpathlineto{\pgfqpoint{0.751829in}{0.942087in}}%
\pgfpathlineto{\pgfqpoint{0.766338in}{0.930648in}}%
\pgfpathlineto{\pgfqpoint{0.767486in}{0.929639in}}%
\pgfpathlineto{\pgfqpoint{0.783142in}{0.918278in}}%
\pgfpathlineto{\pgfqpoint{0.785540in}{0.917036in}}%
\pgfpathlineto{\pgfqpoint{0.798799in}{0.907761in}}%
\pgfpathlineto{\pgfqpoint{0.810042in}{0.903425in}}%
\pgfpathlineto{\pgfqpoint{0.814455in}{0.900071in}}%
\pgfpathclose%
\pgfpathmoveto{\pgfqpoint{0.804437in}{0.930648in}}%
\pgfpathlineto{\pgfqpoint{0.798799in}{0.932105in}}%
\pgfpathlineto{\pgfqpoint{0.783142in}{0.939432in}}%
\pgfpathlineto{\pgfqpoint{0.775812in}{0.944259in}}%
\pgfpathlineto{\pgfqpoint{0.767486in}{0.949718in}}%
\pgfpathlineto{\pgfqpoint{0.757343in}{0.957870in}}%
\pgfpathlineto{\pgfqpoint{0.751829in}{0.962814in}}%
\pgfpathlineto{\pgfqpoint{0.742969in}{0.971481in}}%
\pgfpathlineto{\pgfqpoint{0.736173in}{0.979840in}}%
\pgfpathlineto{\pgfqpoint{0.731780in}{0.985092in}}%
\pgfpathlineto{\pgfqpoint{0.723928in}{0.998703in}}%
\pgfpathlineto{\pgfqpoint{0.720516in}{1.010529in}}%
\pgfpathlineto{\pgfqpoint{0.719898in}{1.012314in}}%
\pgfpathlineto{\pgfqpoint{0.719898in}{1.025925in}}%
\pgfpathlineto{\pgfqpoint{0.720516in}{1.027711in}}%
\pgfpathlineto{\pgfqpoint{0.723928in}{1.039536in}}%
\pgfpathlineto{\pgfqpoint{0.731780in}{1.053148in}}%
\pgfpathlineto{\pgfqpoint{0.736173in}{1.058399in}}%
\pgfpathlineto{\pgfqpoint{0.742969in}{1.066759in}}%
\pgfpathlineto{\pgfqpoint{0.751829in}{1.075426in}}%
\pgfpathlineto{\pgfqpoint{0.757343in}{1.080370in}}%
\pgfpathlineto{\pgfqpoint{0.767486in}{1.088521in}}%
\pgfpathlineto{\pgfqpoint{0.775812in}{1.093981in}}%
\pgfpathlineto{\pgfqpoint{0.783142in}{1.098808in}}%
\pgfpathlineto{\pgfqpoint{0.798799in}{1.106135in}}%
\pgfpathlineto{\pgfqpoint{0.804437in}{1.107592in}}%
\pgfpathlineto{\pgfqpoint{0.814455in}{1.110605in}}%
\pgfpathlineto{\pgfqpoint{0.830112in}{1.111394in}}%
\pgfpathlineto{\pgfqpoint{0.845769in}{1.108233in}}%
\pgfpathlineto{\pgfqpoint{0.847219in}{1.107592in}}%
\pgfpathlineto{\pgfqpoint{0.861425in}{1.102138in}}%
\pgfpathlineto{\pgfqpoint{0.875219in}{1.093981in}}%
\pgfpathlineto{\pgfqpoint{0.877082in}{1.092890in}}%
\pgfpathlineto{\pgfqpoint{0.892738in}{1.081108in}}%
\pgfpathlineto{\pgfqpoint{0.893600in}{1.080370in}}%
\pgfpathlineto{\pgfqpoint{0.907805in}{1.066759in}}%
\pgfpathlineto{\pgfqpoint{0.908395in}{1.066052in}}%
\pgfpathlineto{\pgfqpoint{0.919020in}{1.053148in}}%
\pgfpathlineto{\pgfqpoint{0.924051in}{1.044135in}}%
\pgfpathlineto{\pgfqpoint{0.926924in}{1.039536in}}%
\pgfpathlineto{\pgfqpoint{0.931269in}{1.025925in}}%
\pgfpathlineto{\pgfqpoint{0.931269in}{1.012314in}}%
\pgfpathlineto{\pgfqpoint{0.926924in}{0.998703in}}%
\pgfpathlineto{\pgfqpoint{0.924051in}{0.994104in}}%
\pgfpathlineto{\pgfqpoint{0.919020in}{0.985092in}}%
\pgfpathlineto{\pgfqpoint{0.908395in}{0.972187in}}%
\pgfpathlineto{\pgfqpoint{0.907805in}{0.971481in}}%
\pgfpathlineto{\pgfqpoint{0.893600in}{0.957870in}}%
\pgfpathlineto{\pgfqpoint{0.892738in}{0.957131in}}%
\pgfpathlineto{\pgfqpoint{0.877082in}{0.945349in}}%
\pgfpathlineto{\pgfqpoint{0.875219in}{0.944259in}}%
\pgfpathlineto{\pgfqpoint{0.861425in}{0.936101in}}%
\pgfpathlineto{\pgfqpoint{0.847219in}{0.930648in}}%
\pgfpathlineto{\pgfqpoint{0.845769in}{0.930006in}}%
\pgfpathlineto{\pgfqpoint{0.830112in}{0.926845in}}%
\pgfpathlineto{\pgfqpoint{0.814455in}{0.927635in}}%
\pgfpathlineto{\pgfqpoint{0.804437in}{0.930648in}}%
\pgfpathclose%
\pgfpathmoveto{\pgfqpoint{1.127587in}{0.898864in}}%
\pgfpathlineto{\pgfqpoint{1.143243in}{0.898864in}}%
\pgfpathlineto{\pgfqpoint{1.150407in}{0.903425in}}%
\pgfpathlineto{\pgfqpoint{1.158900in}{0.906159in}}%
\pgfpathlineto{\pgfqpoint{1.174556in}{0.916176in}}%
\pgfpathlineto{\pgfqpoint{1.175499in}{0.917036in}}%
\pgfpathlineto{\pgfqpoint{1.190213in}{0.927093in}}%
\pgfpathlineto{\pgfqpoint{1.194378in}{0.930648in}}%
\pgfpathlineto{\pgfqpoint{1.205870in}{0.939432in}}%
\pgfpathlineto{\pgfqpoint{1.211492in}{0.944259in}}%
\pgfpathlineto{\pgfqpoint{1.221526in}{0.952981in}}%
\pgfpathlineto{\pgfqpoint{1.227078in}{0.957870in}}%
\pgfpathlineto{\pgfqpoint{1.237183in}{0.967860in}}%
\pgfpathlineto{\pgfqpoint{1.241271in}{0.971481in}}%
\pgfpathlineto{\pgfqpoint{1.252839in}{0.984272in}}%
\pgfpathlineto{\pgfqpoint{1.253829in}{0.985092in}}%
\pgfpathlineto{\pgfqpoint{1.265351in}{0.998703in}}%
\pgfpathlineto{\pgfqpoint{1.268496in}{1.006086in}}%
\pgfpathlineto{\pgfqpoint{1.273742in}{1.012314in}}%
\pgfpathlineto{\pgfqpoint{1.273742in}{1.025925in}}%
\pgfpathlineto{\pgfqpoint{1.268496in}{1.032153in}}%
\pgfpathlineto{\pgfqpoint{1.265351in}{1.039536in}}%
\pgfpathlineto{\pgfqpoint{1.253829in}{1.053148in}}%
\pgfpathlineto{\pgfqpoint{1.252839in}{1.053967in}}%
\pgfpathlineto{\pgfqpoint{1.241271in}{1.066759in}}%
\pgfpathlineto{\pgfqpoint{1.237183in}{1.070379in}}%
\pgfpathlineto{\pgfqpoint{1.227078in}{1.080370in}}%
\pgfpathlineto{\pgfqpoint{1.221526in}{1.085258in}}%
\pgfpathlineto{\pgfqpoint{1.211492in}{1.093981in}}%
\pgfpathlineto{\pgfqpoint{1.205870in}{1.098808in}}%
\pgfpathlineto{\pgfqpoint{1.194378in}{1.107592in}}%
\pgfpathlineto{\pgfqpoint{1.190213in}{1.111146in}}%
\pgfpathlineto{\pgfqpoint{1.175499in}{1.121203in}}%
\pgfpathlineto{\pgfqpoint{1.174556in}{1.122063in}}%
\pgfpathlineto{\pgfqpoint{1.158900in}{1.132080in}}%
\pgfpathlineto{\pgfqpoint{1.150407in}{1.134814in}}%
\pgfpathlineto{\pgfqpoint{1.143243in}{1.139375in}}%
\pgfpathlineto{\pgfqpoint{1.127587in}{1.139375in}}%
\pgfpathlineto{\pgfqpoint{1.120423in}{1.134814in}}%
\pgfpathlineto{\pgfqpoint{1.111930in}{1.132080in}}%
\pgfpathlineto{\pgfqpoint{1.096274in}{1.122063in}}%
\pgfpathlineto{\pgfqpoint{1.095331in}{1.121203in}}%
\pgfpathlineto{\pgfqpoint{1.080617in}{1.111146in}}%
\pgfpathlineto{\pgfqpoint{1.076452in}{1.107592in}}%
\pgfpathlineto{\pgfqpoint{1.064960in}{1.098808in}}%
\pgfpathlineto{\pgfqpoint{1.059338in}{1.093981in}}%
\pgfpathlineto{\pgfqpoint{1.049304in}{1.085258in}}%
\pgfpathlineto{\pgfqpoint{1.043752in}{1.080370in}}%
\pgfpathlineto{\pgfqpoint{1.033647in}{1.070379in}}%
\pgfpathlineto{\pgfqpoint{1.029559in}{1.066759in}}%
\pgfpathlineto{\pgfqpoint{1.017991in}{1.053967in}}%
\pgfpathlineto{\pgfqpoint{1.017001in}{1.053148in}}%
\pgfpathlineto{\pgfqpoint{1.005479in}{1.039536in}}%
\pgfpathlineto{\pgfqpoint{1.002334in}{1.032153in}}%
\pgfpathlineto{\pgfqpoint{0.997088in}{1.025925in}}%
\pgfpathlineto{\pgfqpoint{0.997088in}{1.012314in}}%
\pgfpathlineto{\pgfqpoint{1.002334in}{1.006086in}}%
\pgfpathlineto{\pgfqpoint{1.005479in}{0.998703in}}%
\pgfpathlineto{\pgfqpoint{1.017001in}{0.985092in}}%
\pgfpathlineto{\pgfqpoint{1.017991in}{0.984272in}}%
\pgfpathlineto{\pgfqpoint{1.029559in}{0.971481in}}%
\pgfpathlineto{\pgfqpoint{1.033647in}{0.967860in}}%
\pgfpathlineto{\pgfqpoint{1.043752in}{0.957870in}}%
\pgfpathlineto{\pgfqpoint{1.049304in}{0.952981in}}%
\pgfpathlineto{\pgfqpoint{1.059338in}{0.944259in}}%
\pgfpathlineto{\pgfqpoint{1.064960in}{0.939432in}}%
\pgfpathlineto{\pgfqpoint{1.076452in}{0.930648in}}%
\pgfpathlineto{\pgfqpoint{1.080617in}{0.927093in}}%
\pgfpathlineto{\pgfqpoint{1.095331in}{0.917036in}}%
\pgfpathlineto{\pgfqpoint{1.096274in}{0.916176in}}%
\pgfpathlineto{\pgfqpoint{1.111930in}{0.906159in}}%
\pgfpathlineto{\pgfqpoint{1.120423in}{0.903425in}}%
\pgfpathlineto{\pgfqpoint{1.127587in}{0.898864in}}%
\pgfpathclose%
\pgfpathmoveto{\pgfqpoint{1.113748in}{0.930648in}}%
\pgfpathlineto{\pgfqpoint{1.111930in}{0.931040in}}%
\pgfpathlineto{\pgfqpoint{1.096274in}{0.937700in}}%
\pgfpathlineto{\pgfqpoint{1.085793in}{0.944259in}}%
\pgfpathlineto{\pgfqpoint{1.080617in}{0.947475in}}%
\pgfpathlineto{\pgfqpoint{1.067301in}{0.957870in}}%
\pgfpathlineto{\pgfqpoint{1.064960in}{0.959904in}}%
\pgfpathlineto{\pgfqpoint{1.053003in}{0.971481in}}%
\pgfpathlineto{\pgfqpoint{1.049304in}{0.975980in}}%
\pgfpathlineto{\pgfqpoint{1.041760in}{0.985092in}}%
\pgfpathlineto{\pgfqpoint{1.034098in}{0.998703in}}%
\pgfpathlineto{\pgfqpoint{1.033647in}{1.000284in}}%
\pgfpathlineto{\pgfqpoint{1.029637in}{1.012314in}}%
\pgfpathlineto{\pgfqpoint{1.029637in}{1.025925in}}%
\pgfpathlineto{\pgfqpoint{1.033647in}{1.037956in}}%
\pgfpathlineto{\pgfqpoint{1.034098in}{1.039536in}}%
\pgfpathlineto{\pgfqpoint{1.041760in}{1.053148in}}%
\pgfpathlineto{\pgfqpoint{1.049304in}{1.062259in}}%
\pgfpathlineto{\pgfqpoint{1.053003in}{1.066759in}}%
\pgfpathlineto{\pgfqpoint{1.064960in}{1.078335in}}%
\pgfpathlineto{\pgfqpoint{1.067301in}{1.080370in}}%
\pgfpathlineto{\pgfqpoint{1.080617in}{1.090765in}}%
\pgfpathlineto{\pgfqpoint{1.085793in}{1.093981in}}%
\pgfpathlineto{\pgfqpoint{1.096274in}{1.100540in}}%
\pgfpathlineto{\pgfqpoint{1.111930in}{1.107200in}}%
\pgfpathlineto{\pgfqpoint{1.113748in}{1.107592in}}%
\pgfpathlineto{\pgfqpoint{1.127587in}{1.111078in}}%
\pgfpathlineto{\pgfqpoint{1.143243in}{1.111078in}}%
\pgfpathlineto{\pgfqpoint{1.157082in}{1.107592in}}%
\pgfpathlineto{\pgfqpoint{1.158900in}{1.107200in}}%
\pgfpathlineto{\pgfqpoint{1.174556in}{1.100540in}}%
\pgfpathlineto{\pgfqpoint{1.185037in}{1.093981in}}%
\pgfpathlineto{\pgfqpoint{1.190213in}{1.090765in}}%
\pgfpathlineto{\pgfqpoint{1.203529in}{1.080370in}}%
\pgfpathlineto{\pgfqpoint{1.205870in}{1.078335in}}%
\pgfpathlineto{\pgfqpoint{1.217827in}{1.066759in}}%
\pgfpathlineto{\pgfqpoint{1.221526in}{1.062259in}}%
\pgfpathlineto{\pgfqpoint{1.229070in}{1.053148in}}%
\pgfpathlineto{\pgfqpoint{1.236732in}{1.039536in}}%
\pgfpathlineto{\pgfqpoint{1.237183in}{1.037956in}}%
\pgfpathlineto{\pgfqpoint{1.241193in}{1.025925in}}%
\pgfpathlineto{\pgfqpoint{1.241193in}{1.012314in}}%
\pgfpathlineto{\pgfqpoint{1.237183in}{1.000284in}}%
\pgfpathlineto{\pgfqpoint{1.236732in}{0.998703in}}%
\pgfpathlineto{\pgfqpoint{1.229070in}{0.985092in}}%
\pgfpathlineto{\pgfqpoint{1.221526in}{0.975980in}}%
\pgfpathlineto{\pgfqpoint{1.217827in}{0.971481in}}%
\pgfpathlineto{\pgfqpoint{1.205870in}{0.959904in}}%
\pgfpathlineto{\pgfqpoint{1.203529in}{0.957870in}}%
\pgfpathlineto{\pgfqpoint{1.190213in}{0.947475in}}%
\pgfpathlineto{\pgfqpoint{1.185037in}{0.944259in}}%
\pgfpathlineto{\pgfqpoint{1.174556in}{0.937700in}}%
\pgfpathlineto{\pgfqpoint{1.158900in}{0.931040in}}%
\pgfpathlineto{\pgfqpoint{1.157082in}{0.930648in}}%
\pgfpathlineto{\pgfqpoint{1.143243in}{0.927161in}}%
\pgfpathlineto{\pgfqpoint{1.127587in}{0.927161in}}%
\pgfpathlineto{\pgfqpoint{1.113748in}{0.930648in}}%
\pgfpathclose%
\pgfpathmoveto{\pgfqpoint{1.440718in}{0.898060in}}%
\pgfpathlineto{\pgfqpoint{1.456375in}{0.900071in}}%
\pgfpathlineto{\pgfqpoint{1.460788in}{0.903425in}}%
\pgfpathlineto{\pgfqpoint{1.472031in}{0.907761in}}%
\pgfpathlineto{\pgfqpoint{1.485290in}{0.917036in}}%
\pgfpathlineto{\pgfqpoint{1.487688in}{0.918278in}}%
\pgfpathlineto{\pgfqpoint{1.503344in}{0.929639in}}%
\pgfpathlineto{\pgfqpoint{1.504492in}{0.930648in}}%
\pgfpathlineto{\pgfqpoint{1.519001in}{0.942087in}}%
\pgfpathlineto{\pgfqpoint{1.521505in}{0.944259in}}%
\pgfpathlineto{\pgfqpoint{1.534657in}{0.955821in}}%
\pgfpathlineto{\pgfqpoint{1.537008in}{0.957870in}}%
\pgfpathlineto{\pgfqpoint{1.550314in}{0.970884in}}%
\pgfpathlineto{\pgfqpoint{1.551014in}{0.971481in}}%
\pgfpathlineto{\pgfqpoint{1.563940in}{0.985092in}}%
\pgfpathlineto{\pgfqpoint{1.565971in}{0.988205in}}%
\pgfpathlineto{\pgfqpoint{1.575702in}{0.998703in}}%
\pgfpathlineto{\pgfqpoint{1.581627in}{1.011392in}}%
\pgfpathlineto{\pgfqpoint{1.582614in}{1.012314in}}%
\pgfpathlineto{\pgfqpoint{1.582614in}{1.025925in}}%
\pgfpathlineto{\pgfqpoint{1.581627in}{1.026848in}}%
\pgfpathlineto{\pgfqpoint{1.575702in}{1.039536in}}%
\pgfpathlineto{\pgfqpoint{1.565971in}{1.050034in}}%
\pgfpathlineto{\pgfqpoint{1.563940in}{1.053148in}}%
\pgfpathlineto{\pgfqpoint{1.551014in}{1.066759in}}%
\pgfpathlineto{\pgfqpoint{1.550314in}{1.067356in}}%
\pgfpathlineto{\pgfqpoint{1.537008in}{1.080370in}}%
\pgfpathlineto{\pgfqpoint{1.534657in}{1.082419in}}%
\pgfpathlineto{\pgfqpoint{1.521505in}{1.093981in}}%
\pgfpathlineto{\pgfqpoint{1.519001in}{1.096153in}}%
\pgfpathlineto{\pgfqpoint{1.504492in}{1.107592in}}%
\pgfpathlineto{\pgfqpoint{1.503344in}{1.108601in}}%
\pgfpathlineto{\pgfqpoint{1.487688in}{1.119962in}}%
\pgfpathlineto{\pgfqpoint{1.485290in}{1.121203in}}%
\pgfpathlineto{\pgfqpoint{1.472031in}{1.130478in}}%
\pgfpathlineto{\pgfqpoint{1.460788in}{1.134814in}}%
\pgfpathlineto{\pgfqpoint{1.456375in}{1.138168in}}%
\pgfpathlineto{\pgfqpoint{1.440718in}{1.140180in}}%
\pgfpathlineto{\pgfqpoint{1.430234in}{1.134814in}}%
\pgfpathlineto{\pgfqpoint{1.425061in}{1.133481in}}%
\pgfpathlineto{\pgfqpoint{1.409405in}{1.124468in}}%
\pgfpathlineto{\pgfqpoint{1.405619in}{1.121203in}}%
\pgfpathlineto{\pgfqpoint{1.393748in}{1.113557in}}%
\pgfpathlineto{\pgfqpoint{1.386524in}{1.107592in}}%
\pgfpathlineto{\pgfqpoint{1.378092in}{1.101369in}}%
\pgfpathlineto{\pgfqpoint{1.369363in}{1.093981in}}%
\pgfpathlineto{\pgfqpoint{1.362435in}{1.088048in}}%
\pgfpathlineto{\pgfqpoint{1.353762in}{1.080370in}}%
\pgfpathlineto{\pgfqpoint{1.346779in}{1.073423in}}%
\pgfpathlineto{\pgfqpoint{1.339487in}{1.066759in}}%
\pgfpathlineto{\pgfqpoint{1.331122in}{1.057194in}}%
\pgfpathlineto{\pgfqpoint{1.326588in}{1.053148in}}%
\pgfpathlineto{\pgfqpoint{1.316021in}{1.039536in}}%
\pgfpathlineto{\pgfqpoint{1.315466in}{1.038126in}}%
\pgfpathlineto{\pgfqpoint{1.306931in}{1.025925in}}%
\pgfpathlineto{\pgfqpoint{1.306931in}{1.012314in}}%
\pgfpathlineto{\pgfqpoint{1.315466in}{1.000113in}}%
\pgfpathlineto{\pgfqpoint{1.316021in}{0.998703in}}%
\pgfpathlineto{\pgfqpoint{1.326588in}{0.985092in}}%
\pgfpathlineto{\pgfqpoint{1.331122in}{0.981046in}}%
\pgfpathlineto{\pgfqpoint{1.339487in}{0.971481in}}%
\pgfpathlineto{\pgfqpoint{1.346779in}{0.964816in}}%
\pgfpathlineto{\pgfqpoint{1.353762in}{0.957870in}}%
\pgfpathlineto{\pgfqpoint{1.362435in}{0.950191in}}%
\pgfpathlineto{\pgfqpoint{1.369363in}{0.944259in}}%
\pgfpathlineto{\pgfqpoint{1.378092in}{0.936871in}}%
\pgfpathlineto{\pgfqpoint{1.386524in}{0.930648in}}%
\pgfpathlineto{\pgfqpoint{1.393748in}{0.924682in}}%
\pgfpathlineto{\pgfqpoint{1.405619in}{0.917036in}}%
\pgfpathlineto{\pgfqpoint{1.409405in}{0.913771in}}%
\pgfpathlineto{\pgfqpoint{1.425061in}{0.904759in}}%
\pgfpathlineto{\pgfqpoint{1.430234in}{0.903425in}}%
\pgfpathlineto{\pgfqpoint{1.440718in}{0.898060in}}%
\pgfpathclose%
\pgfpathmoveto{\pgfqpoint{1.423611in}{0.930648in}}%
\pgfpathlineto{\pgfqpoint{1.409405in}{0.936101in}}%
\pgfpathlineto{\pgfqpoint{1.395611in}{0.944259in}}%
\pgfpathlineto{\pgfqpoint{1.393748in}{0.945349in}}%
\pgfpathlineto{\pgfqpoint{1.378092in}{0.957131in}}%
\pgfpathlineto{\pgfqpoint{1.377230in}{0.957870in}}%
\pgfpathlineto{\pgfqpoint{1.363025in}{0.971481in}}%
\pgfpathlineto{\pgfqpoint{1.362435in}{0.972187in}}%
\pgfpathlineto{\pgfqpoint{1.351810in}{0.985092in}}%
\pgfpathlineto{\pgfqpoint{1.346779in}{0.994104in}}%
\pgfpathlineto{\pgfqpoint{1.343906in}{0.998703in}}%
\pgfpathlineto{\pgfqpoint{1.339561in}{1.012314in}}%
\pgfpathlineto{\pgfqpoint{1.339561in}{1.025925in}}%
\pgfpathlineto{\pgfqpoint{1.343906in}{1.039536in}}%
\pgfpathlineto{\pgfqpoint{1.346779in}{1.044135in}}%
\pgfpathlineto{\pgfqpoint{1.351810in}{1.053148in}}%
\pgfpathlineto{\pgfqpoint{1.362435in}{1.066052in}}%
\pgfpathlineto{\pgfqpoint{1.363025in}{1.066759in}}%
\pgfpathlineto{\pgfqpoint{1.377230in}{1.080370in}}%
\pgfpathlineto{\pgfqpoint{1.378092in}{1.081108in}}%
\pgfpathlineto{\pgfqpoint{1.393748in}{1.092890in}}%
\pgfpathlineto{\pgfqpoint{1.395611in}{1.093981in}}%
\pgfpathlineto{\pgfqpoint{1.409405in}{1.102138in}}%
\pgfpathlineto{\pgfqpoint{1.423611in}{1.107592in}}%
\pgfpathlineto{\pgfqpoint{1.425061in}{1.108233in}}%
\pgfpathlineto{\pgfqpoint{1.440718in}{1.111394in}}%
\pgfpathlineto{\pgfqpoint{1.456375in}{1.110605in}}%
\pgfpathlineto{\pgfqpoint{1.466393in}{1.107592in}}%
\pgfpathlineto{\pgfqpoint{1.472031in}{1.106135in}}%
\pgfpathlineto{\pgfqpoint{1.487688in}{1.098808in}}%
\pgfpathlineto{\pgfqpoint{1.495018in}{1.093981in}}%
\pgfpathlineto{\pgfqpoint{1.503344in}{1.088521in}}%
\pgfpathlineto{\pgfqpoint{1.513487in}{1.080370in}}%
\pgfpathlineto{\pgfqpoint{1.519001in}{1.075426in}}%
\pgfpathlineto{\pgfqpoint{1.527861in}{1.066759in}}%
\pgfpathlineto{\pgfqpoint{1.534657in}{1.058399in}}%
\pgfpathlineto{\pgfqpoint{1.539050in}{1.053148in}}%
\pgfpathlineto{\pgfqpoint{1.546902in}{1.039536in}}%
\pgfpathlineto{\pgfqpoint{1.550314in}{1.027711in}}%
\pgfpathlineto{\pgfqpoint{1.550932in}{1.025925in}}%
\pgfpathlineto{\pgfqpoint{1.550932in}{1.012314in}}%
\pgfpathlineto{\pgfqpoint{1.550314in}{1.010529in}}%
\pgfpathlineto{\pgfqpoint{1.546902in}{0.998703in}}%
\pgfpathlineto{\pgfqpoint{1.539050in}{0.985092in}}%
\pgfpathlineto{\pgfqpoint{1.534657in}{0.979840in}}%
\pgfpathlineto{\pgfqpoint{1.527861in}{0.971481in}}%
\pgfpathlineto{\pgfqpoint{1.519001in}{0.962814in}}%
\pgfpathlineto{\pgfqpoint{1.513487in}{0.957870in}}%
\pgfpathlineto{\pgfqpoint{1.503344in}{0.949718in}}%
\pgfpathlineto{\pgfqpoint{1.495018in}{0.944259in}}%
\pgfpathlineto{\pgfqpoint{1.487688in}{0.939432in}}%
\pgfpathlineto{\pgfqpoint{1.472031in}{0.932105in}}%
\pgfpathlineto{\pgfqpoint{1.466393in}{0.930648in}}%
\pgfpathlineto{\pgfqpoint{1.456375in}{0.927635in}}%
\pgfpathlineto{\pgfqpoint{1.440718in}{0.926845in}}%
\pgfpathlineto{\pgfqpoint{1.425061in}{0.930006in}}%
\pgfpathlineto{\pgfqpoint{1.423611in}{0.930648in}}%
\pgfpathclose%
\pgfpathmoveto{\pgfqpoint{1.753849in}{0.897657in}}%
\pgfpathlineto{\pgfqpoint{1.769506in}{0.901681in}}%
\pgfpathlineto{\pgfqpoint{1.771485in}{0.903425in}}%
\pgfpathlineto{\pgfqpoint{1.785162in}{0.909564in}}%
\pgfpathlineto{\pgfqpoint{1.795051in}{0.917036in}}%
\pgfpathlineto{\pgfqpoint{1.800819in}{0.920273in}}%
\pgfpathlineto{\pgfqpoint{1.814410in}{0.930648in}}%
\pgfpathlineto{\pgfqpoint{1.816476in}{0.932050in}}%
\pgfpathlineto{\pgfqpoint{1.831477in}{0.944259in}}%
\pgfpathlineto{\pgfqpoint{1.832132in}{0.944797in}}%
\pgfpathlineto{\pgfqpoint{1.846917in}{0.957870in}}%
\pgfpathlineto{\pgfqpoint{1.847789in}{0.958736in}}%
\pgfpathlineto{\pgfqpoint{1.861116in}{0.971481in}}%
\pgfpathlineto{\pgfqpoint{1.863445in}{0.974277in}}%
\pgfpathlineto{\pgfqpoint{1.874223in}{0.985092in}}%
\pgfpathlineto{\pgfqpoint{1.879102in}{0.992220in}}%
\pgfpathlineto{\pgfqpoint{1.885769in}{0.998703in}}%
\pgfpathlineto{\pgfqpoint{1.892887in}{1.012314in}}%
\pgfpathlineto{\pgfqpoint{1.892887in}{1.025925in}}%
\pgfpathlineto{\pgfqpoint{1.885769in}{1.039536in}}%
\pgfpathlineto{\pgfqpoint{1.879102in}{1.046019in}}%
\pgfpathlineto{\pgfqpoint{1.874223in}{1.053148in}}%
\pgfpathlineto{\pgfqpoint{1.863445in}{1.063962in}}%
\pgfpathlineto{\pgfqpoint{1.861116in}{1.066759in}}%
\pgfpathlineto{\pgfqpoint{1.847789in}{1.079504in}}%
\pgfpathlineto{\pgfqpoint{1.846917in}{1.080370in}}%
\pgfpathlineto{\pgfqpoint{1.832132in}{1.093443in}}%
\pgfpathlineto{\pgfqpoint{1.831477in}{1.093981in}}%
\pgfpathlineto{\pgfqpoint{1.816476in}{1.106189in}}%
\pgfpathlineto{\pgfqpoint{1.814410in}{1.107592in}}%
\pgfpathlineto{\pgfqpoint{1.800819in}{1.117967in}}%
\pgfpathlineto{\pgfqpoint{1.795051in}{1.121203in}}%
\pgfpathlineto{\pgfqpoint{1.785162in}{1.128676in}}%
\pgfpathlineto{\pgfqpoint{1.771485in}{1.134814in}}%
\pgfpathlineto{\pgfqpoint{1.769506in}{1.136558in}}%
\pgfpathlineto{\pgfqpoint{1.753849in}{1.140582in}}%
\pgfpathlineto{\pgfqpoint{1.738883in}{1.134814in}}%
\pgfpathlineto{\pgfqpoint{1.738193in}{1.134681in}}%
\pgfpathlineto{\pgfqpoint{1.722536in}{1.126672in}}%
\pgfpathlineto{\pgfqpoint{1.715784in}{1.121203in}}%
\pgfpathlineto{\pgfqpoint{1.706880in}{1.115831in}}%
\pgfpathlineto{\pgfqpoint{1.696524in}{1.107592in}}%
\pgfpathlineto{\pgfqpoint{1.691223in}{1.103831in}}%
\pgfpathlineto{\pgfqpoint{1.679377in}{1.093981in}}%
\pgfpathlineto{\pgfqpoint{1.675567in}{1.090780in}}%
\pgfpathlineto{\pgfqpoint{1.663825in}{1.080370in}}%
\pgfpathlineto{\pgfqpoint{1.659910in}{1.076470in}}%
\pgfpathlineto{\pgfqpoint{1.649551in}{1.066759in}}%
\pgfpathlineto{\pgfqpoint{1.644253in}{1.060533in}}%
\pgfpathlineto{\pgfqpoint{1.636486in}{1.053148in}}%
\pgfpathlineto{\pgfqpoint{1.628597in}{1.042262in}}%
\pgfpathlineto{\pgfqpoint{1.625425in}{1.039536in}}%
\pgfpathlineto{\pgfqpoint{1.617288in}{1.025925in}}%
\pgfpathlineto{\pgfqpoint{1.617288in}{1.012314in}}%
\pgfpathlineto{\pgfqpoint{1.625425in}{0.998703in}}%
\pgfpathlineto{\pgfqpoint{1.628597in}{0.995978in}}%
\pgfpathlineto{\pgfqpoint{1.636486in}{0.985092in}}%
\pgfpathlineto{\pgfqpoint{1.644253in}{0.977706in}}%
\pgfpathlineto{\pgfqpoint{1.649551in}{0.971481in}}%
\pgfpathlineto{\pgfqpoint{1.659910in}{0.961770in}}%
\pgfpathlineto{\pgfqpoint{1.663825in}{0.957870in}}%
\pgfpathlineto{\pgfqpoint{1.675567in}{0.947460in}}%
\pgfpathlineto{\pgfqpoint{1.679377in}{0.944259in}}%
\pgfpathlineto{\pgfqpoint{1.691223in}{0.934409in}}%
\pgfpathlineto{\pgfqpoint{1.696524in}{0.930648in}}%
\pgfpathlineto{\pgfqpoint{1.706880in}{0.922408in}}%
\pgfpathlineto{\pgfqpoint{1.715784in}{0.917036in}}%
\pgfpathlineto{\pgfqpoint{1.722536in}{0.911567in}}%
\pgfpathlineto{\pgfqpoint{1.738193in}{0.903559in}}%
\pgfpathlineto{\pgfqpoint{1.738883in}{0.903425in}}%
\pgfpathlineto{\pgfqpoint{1.753849in}{0.897657in}}%
\pgfpathclose%
\pgfpathmoveto{\pgfqpoint{1.734176in}{0.930648in}}%
\pgfpathlineto{\pgfqpoint{1.722536in}{0.934635in}}%
\pgfpathlineto{\pgfqpoint{1.706880in}{0.943290in}}%
\pgfpathlineto{\pgfqpoint{1.705532in}{0.944259in}}%
\pgfpathlineto{\pgfqpoint{1.691223in}{0.954550in}}%
\pgfpathlineto{\pgfqpoint{1.687281in}{0.957870in}}%
\pgfpathlineto{\pgfqpoint{1.675567in}{0.968912in}}%
\pgfpathlineto{\pgfqpoint{1.672965in}{0.971481in}}%
\pgfpathlineto{\pgfqpoint{1.661903in}{0.985092in}}%
\pgfpathlineto{\pgfqpoint{1.659910in}{0.988667in}}%
\pgfpathlineto{\pgfqpoint{1.653799in}{0.998703in}}%
\pgfpathlineto{\pgfqpoint{1.649623in}{1.012314in}}%
\pgfpathlineto{\pgfqpoint{1.649623in}{1.025925in}}%
\pgfpathlineto{\pgfqpoint{1.653799in}{1.039536in}}%
\pgfpathlineto{\pgfqpoint{1.659910in}{1.049572in}}%
\pgfpathlineto{\pgfqpoint{1.661903in}{1.053147in}}%
\pgfpathlineto{\pgfqpoint{1.672965in}{1.066759in}}%
\pgfpathlineto{\pgfqpoint{1.675567in}{1.069327in}}%
\pgfpathlineto{\pgfqpoint{1.687281in}{1.080370in}}%
\pgfpathlineto{\pgfqpoint{1.691223in}{1.083689in}}%
\pgfpathlineto{\pgfqpoint{1.705532in}{1.093981in}}%
\pgfpathlineto{\pgfqpoint{1.706880in}{1.094949in}}%
\pgfpathlineto{\pgfqpoint{1.722536in}{1.103604in}}%
\pgfpathlineto{\pgfqpoint{1.734176in}{1.107592in}}%
\pgfpathlineto{\pgfqpoint{1.738193in}{1.109182in}}%
\pgfpathlineto{\pgfqpoint{1.753849in}{1.111552in}}%
\pgfpathlineto{\pgfqpoint{1.769506in}{1.109973in}}%
\pgfpathlineto{\pgfqpoint{1.776332in}{1.107592in}}%
\pgfpathlineto{\pgfqpoint{1.785162in}{1.104936in}}%
\pgfpathlineto{\pgfqpoint{1.800819in}{1.096944in}}%
\pgfpathlineto{\pgfqpoint{1.805116in}{1.093981in}}%
\pgfpathlineto{\pgfqpoint{1.816476in}{1.086161in}}%
\pgfpathlineto{\pgfqpoint{1.823503in}{1.080370in}}%
\pgfpathlineto{\pgfqpoint{1.832132in}{1.072421in}}%
\pgfpathlineto{\pgfqpoint{1.837882in}{1.066759in}}%
\pgfpathlineto{\pgfqpoint{1.847789in}{1.054488in}}%
\pgfpathlineto{\pgfqpoint{1.848927in}{1.053148in}}%
\pgfpathlineto{\pgfqpoint{1.857012in}{1.039536in}}%
\pgfpathlineto{\pgfqpoint{1.861047in}{1.025925in}}%
\pgfpathlineto{\pgfqpoint{1.861047in}{1.012314in}}%
\pgfpathlineto{\pgfqpoint{1.857012in}{0.998703in}}%
\pgfpathlineto{\pgfqpoint{1.848927in}{0.985092in}}%
\pgfpathlineto{\pgfqpoint{1.847789in}{0.983752in}}%
\pgfpathlineto{\pgfqpoint{1.837882in}{0.971481in}}%
\pgfpathlineto{\pgfqpoint{1.832132in}{0.965819in}}%
\pgfpathlineto{\pgfqpoint{1.823503in}{0.957870in}}%
\pgfpathlineto{\pgfqpoint{1.816476in}{0.952078in}}%
\pgfpathlineto{\pgfqpoint{1.805116in}{0.944259in}}%
\pgfpathlineto{\pgfqpoint{1.800819in}{0.941295in}}%
\pgfpathlineto{\pgfqpoint{1.785162in}{0.933303in}}%
\pgfpathlineto{\pgfqpoint{1.776332in}{0.930648in}}%
\pgfpathlineto{\pgfqpoint{1.769506in}{0.928267in}}%
\pgfpathlineto{\pgfqpoint{1.753849in}{0.926687in}}%
\pgfpathlineto{\pgfqpoint{1.738193in}{0.929057in}}%
\pgfpathlineto{\pgfqpoint{1.734176in}{0.930648in}}%
\pgfpathclose%
\pgfpathmoveto{\pgfqpoint{0.501324in}{1.170548in}}%
\pgfpathlineto{\pgfqpoint{0.516981in}{1.167234in}}%
\pgfpathlineto{\pgfqpoint{0.532637in}{1.172206in}}%
\pgfpathlineto{\pgfqpoint{0.536780in}{1.175647in}}%
\pgfpathlineto{\pgfqpoint{0.548294in}{1.181090in}}%
\pgfpathlineto{\pgfqpoint{0.559159in}{1.189259in}}%
\pgfpathlineto{\pgfqpoint{0.563950in}{1.192054in}}%
\pgfpathlineto{\pgfqpoint{0.577979in}{1.202870in}}%
\pgfpathlineto{\pgfqpoint{0.579607in}{1.204018in}}%
\pgfpathlineto{\pgfqpoint{0.594680in}{1.216481in}}%
\pgfpathlineto{\pgfqpoint{0.595263in}{1.216979in}}%
\pgfpathlineto{\pgfqpoint{0.609847in}{1.230092in}}%
\pgfpathlineto{\pgfqpoint{0.610920in}{1.231199in}}%
\pgfpathlineto{\pgfqpoint{0.623826in}{1.243703in}}%
\pgfpathlineto{\pgfqpoint{0.626577in}{1.247133in}}%
\pgfpathlineto{\pgfqpoint{0.636696in}{1.257314in}}%
\pgfpathlineto{\pgfqpoint{0.642233in}{1.265764in}}%
\pgfpathlineto{\pgfqpoint{0.647685in}{1.270925in}}%
\pgfpathlineto{\pgfqpoint{0.654192in}{1.284536in}}%
\pgfpathlineto{\pgfqpoint{0.652566in}{1.298147in}}%
\pgfpathlineto{\pgfqpoint{0.642799in}{1.311759in}}%
\pgfpathlineto{\pgfqpoint{0.642233in}{1.312204in}}%
\pgfpathlineto{\pgfqpoint{0.631798in}{1.325370in}}%
\pgfpathlineto{\pgfqpoint{0.626577in}{1.330087in}}%
\pgfpathlineto{\pgfqpoint{0.618590in}{1.338981in}}%
\pgfpathlineto{\pgfqpoint{0.610920in}{1.345965in}}%
\pgfpathlineto{\pgfqpoint{0.604059in}{1.352592in}}%
\pgfpathlineto{\pgfqpoint{0.595263in}{1.360309in}}%
\pgfpathlineto{\pgfqpoint{0.588169in}{1.366203in}}%
\pgfpathlineto{\pgfqpoint{0.579607in}{1.373395in}}%
\pgfpathlineto{\pgfqpoint{0.570657in}{1.379814in}}%
\pgfpathlineto{\pgfqpoint{0.563950in}{1.385358in}}%
\pgfpathlineto{\pgfqpoint{0.551108in}{1.393425in}}%
\pgfpathlineto{\pgfqpoint{0.548294in}{1.395914in}}%
\pgfpathlineto{\pgfqpoint{0.532637in}{1.404756in}}%
\pgfpathlineto{\pgfqpoint{0.521887in}{1.407036in}}%
\pgfpathlineto{\pgfqpoint{0.516981in}{1.409438in}}%
\pgfpathlineto{\pgfqpoint{0.509646in}{1.407036in}}%
\pgfpathlineto{\pgfqpoint{0.501324in}{1.405860in}}%
\pgfpathlineto{\pgfqpoint{0.485668in}{1.398126in}}%
\pgfpathlineto{\pgfqpoint{0.479970in}{1.393425in}}%
\pgfpathlineto{\pgfqpoint{0.470011in}{1.387607in}}%
\pgfpathlineto{\pgfqpoint{0.460185in}{1.379814in}}%
\pgfpathlineto{\pgfqpoint{0.454354in}{1.375812in}}%
\pgfpathlineto{\pgfqpoint{0.442666in}{1.366203in}}%
\pgfpathlineto{\pgfqpoint{0.438698in}{1.362981in}}%
\pgfpathlineto{\pgfqpoint{0.426825in}{1.352592in}}%
\pgfpathlineto{\pgfqpoint{0.423041in}{1.348948in}}%
\pgfpathlineto{\pgfqpoint{0.412313in}{1.338981in}}%
\pgfpathlineto{\pgfqpoint{0.407385in}{1.333375in}}%
\pgfpathlineto{\pgfqpoint{0.398989in}{1.325370in}}%
\pgfpathlineto{\pgfqpoint{0.391728in}{1.315671in}}%
\pgfpathlineto{\pgfqpoint{0.387342in}{1.311759in}}%
\pgfpathlineto{\pgfqpoint{0.378796in}{1.298147in}}%
\pgfpathlineto{\pgfqpoint{0.377374in}{1.284536in}}%
\pgfpathlineto{\pgfqpoint{0.383067in}{1.270925in}}%
\pgfpathlineto{\pgfqpoint{0.391728in}{1.261643in}}%
\pgfpathlineto{\pgfqpoint{0.394408in}{1.257314in}}%
\pgfpathlineto{\pgfqpoint{0.407219in}{1.243703in}}%
\pgfpathlineto{\pgfqpoint{0.407385in}{1.243564in}}%
\pgfpathlineto{\pgfqpoint{0.420897in}{1.230092in}}%
\pgfpathlineto{\pgfqpoint{0.423041in}{1.228224in}}%
\pgfpathlineto{\pgfqpoint{0.436107in}{1.216481in}}%
\pgfpathlineto{\pgfqpoint{0.438698in}{1.214231in}}%
\pgfpathlineto{\pgfqpoint{0.452809in}{1.202870in}}%
\pgfpathlineto{\pgfqpoint{0.454354in}{1.201521in}}%
\pgfpathlineto{\pgfqpoint{0.470011in}{1.190013in}}%
\pgfpathlineto{\pgfqpoint{0.471402in}{1.189259in}}%
\pgfpathlineto{\pgfqpoint{0.485668in}{1.179252in}}%
\pgfpathlineto{\pgfqpoint{0.494355in}{1.175647in}}%
\pgfpathlineto{\pgfqpoint{0.501324in}{1.170548in}}%
\pgfpathclose%
\pgfpathmoveto{\pgfqpoint{0.485884in}{1.202870in}}%
\pgfpathlineto{\pgfqpoint{0.485668in}{1.202934in}}%
\pgfpathlineto{\pgfqpoint{0.470011in}{1.210768in}}%
\pgfpathlineto{\pgfqpoint{0.461679in}{1.216481in}}%
\pgfpathlineto{\pgfqpoint{0.454354in}{1.221600in}}%
\pgfpathlineto{\pgfqpoint{0.444195in}{1.230092in}}%
\pgfpathlineto{\pgfqpoint{0.438698in}{1.235336in}}%
\pgfpathlineto{\pgfqpoint{0.430477in}{1.243703in}}%
\pgfpathlineto{\pgfqpoint{0.423041in}{1.253476in}}%
\pgfpathlineto{\pgfqpoint{0.419962in}{1.257314in}}%
\pgfpathlineto{\pgfqpoint{0.412688in}{1.270925in}}%
\pgfpathlineto{\pgfqpoint{0.409461in}{1.284536in}}%
\pgfpathlineto{\pgfqpoint{0.410267in}{1.298147in}}%
\pgfpathlineto{\pgfqpoint{0.415111in}{1.311759in}}%
\pgfpathlineto{\pgfqpoint{0.423041in}{1.323944in}}%
\pgfpathlineto{\pgfqpoint{0.423913in}{1.325370in}}%
\pgfpathlineto{\pgfqpoint{0.435557in}{1.338981in}}%
\pgfpathlineto{\pgfqpoint{0.438698in}{1.341985in}}%
\pgfpathlineto{\pgfqpoint{0.450574in}{1.352592in}}%
\pgfpathlineto{\pgfqpoint{0.454354in}{1.355675in}}%
\pgfpathlineto{\pgfqpoint{0.469819in}{1.366203in}}%
\pgfpathlineto{\pgfqpoint{0.470011in}{1.366337in}}%
\pgfpathlineto{\pgfqpoint{0.485668in}{1.374528in}}%
\pgfpathlineto{\pgfqpoint{0.501324in}{1.379301in}}%
\pgfpathlineto{\pgfqpoint{0.507200in}{1.379814in}}%
\pgfpathlineto{\pgfqpoint{0.516981in}{1.380851in}}%
\pgfpathlineto{\pgfqpoint{0.523523in}{1.379814in}}%
\pgfpathlineto{\pgfqpoint{0.532637in}{1.378620in}}%
\pgfpathlineto{\pgfqpoint{0.548294in}{1.373162in}}%
\pgfpathlineto{\pgfqpoint{0.560696in}{1.366203in}}%
\pgfpathlineto{\pgfqpoint{0.563950in}{1.364438in}}%
\pgfpathlineto{\pgfqpoint{0.579607in}{1.353195in}}%
\pgfpathlineto{\pgfqpoint{0.580330in}{1.352592in}}%
\pgfpathlineto{\pgfqpoint{0.595227in}{1.338981in}}%
\pgfpathlineto{\pgfqpoint{0.595263in}{1.338940in}}%
\pgfpathlineto{\pgfqpoint{0.607005in}{1.325370in}}%
\pgfpathlineto{\pgfqpoint{0.610920in}{1.318949in}}%
\pgfpathlineto{\pgfqpoint{0.615693in}{1.311759in}}%
\pgfpathlineto{\pgfqpoint{0.620707in}{1.298147in}}%
\pgfpathlineto{\pgfqpoint{0.621541in}{1.284536in}}%
\pgfpathlineto{\pgfqpoint{0.618201in}{1.270925in}}%
\pgfpathlineto{\pgfqpoint{0.610920in}{1.257748in}}%
\pgfpathlineto{\pgfqpoint{0.610701in}{1.257314in}}%
\pgfpathlineto{\pgfqpoint{0.600365in}{1.243703in}}%
\pgfpathlineto{\pgfqpoint{0.595263in}{1.238497in}}%
\pgfpathlineto{\pgfqpoint{0.586654in}{1.230092in}}%
\pgfpathlineto{\pgfqpoint{0.579607in}{1.224074in}}%
\pgfpathlineto{\pgfqpoint{0.569209in}{1.216481in}}%
\pgfpathlineto{\pgfqpoint{0.563950in}{1.212723in}}%
\pgfpathlineto{\pgfqpoint{0.548294in}{1.204240in}}%
\pgfpathlineto{\pgfqpoint{0.544270in}{1.202870in}}%
\pgfpathlineto{\pgfqpoint{0.532637in}{1.198407in}}%
\pgfpathlineto{\pgfqpoint{0.516981in}{1.196143in}}%
\pgfpathlineto{\pgfqpoint{0.501324in}{1.197652in}}%
\pgfpathlineto{\pgfqpoint{0.485884in}{1.202870in}}%
\pgfpathclose%
\pgfpathmoveto{\pgfqpoint{0.814455in}{1.169222in}}%
\pgfpathlineto{\pgfqpoint{0.830112in}{1.167565in}}%
\pgfpathlineto{\pgfqpoint{0.845769in}{1.174197in}}%
\pgfpathlineto{\pgfqpoint{0.847332in}{1.175647in}}%
\pgfpathlineto{\pgfqpoint{0.861425in}{1.183112in}}%
\pgfpathlineto{\pgfqpoint{0.869105in}{1.189259in}}%
\pgfpathlineto{\pgfqpoint{0.877082in}{1.194227in}}%
\pgfpathlineto{\pgfqpoint{0.887884in}{1.202870in}}%
\pgfpathlineto{\pgfqpoint{0.892738in}{1.206431in}}%
\pgfpathlineto{\pgfqpoint{0.904678in}{1.216481in}}%
\pgfpathlineto{\pgfqpoint{0.908395in}{1.219712in}}%
\pgfpathlineto{\pgfqpoint{0.919955in}{1.230092in}}%
\pgfpathlineto{\pgfqpoint{0.924051in}{1.234312in}}%
\pgfpathlineto{\pgfqpoint{0.933993in}{1.243703in}}%
\pgfpathlineto{\pgfqpoint{0.939708in}{1.250637in}}%
\pgfpathlineto{\pgfqpoint{0.946779in}{1.257314in}}%
\pgfpathlineto{\pgfqpoint{0.955364in}{1.269566in}}%
\pgfpathlineto{\pgfqpoint{0.957033in}{1.270925in}}%
\pgfpathlineto{\pgfqpoint{0.964661in}{1.284536in}}%
\pgfpathlineto{\pgfqpoint{0.962755in}{1.298147in}}%
\pgfpathlineto{\pgfqpoint{0.955364in}{1.307000in}}%
\pgfpathlineto{\pgfqpoint{0.953120in}{1.311759in}}%
\pgfpathlineto{\pgfqpoint{0.941494in}{1.325370in}}%
\pgfpathlineto{\pgfqpoint{0.939708in}{1.326885in}}%
\pgfpathlineto{\pgfqpoint{0.928545in}{1.338981in}}%
\pgfpathlineto{\pgfqpoint{0.924051in}{1.342971in}}%
\pgfpathlineto{\pgfqpoint{0.914075in}{1.352592in}}%
\pgfpathlineto{\pgfqpoint{0.908395in}{1.357569in}}%
\pgfpathlineto{\pgfqpoint{0.898199in}{1.366203in}}%
\pgfpathlineto{\pgfqpoint{0.892738in}{1.370872in}}%
\pgfpathlineto{\pgfqpoint{0.880752in}{1.379814in}}%
\pgfpathlineto{\pgfqpoint{0.877082in}{1.382962in}}%
\pgfpathlineto{\pgfqpoint{0.861484in}{1.393425in}}%
\pgfpathlineto{\pgfqpoint{0.861425in}{1.393481in}}%
\pgfpathlineto{\pgfqpoint{0.845769in}{1.403432in}}%
\pgfpathlineto{\pgfqpoint{0.833015in}{1.407036in}}%
\pgfpathlineto{\pgfqpoint{0.830112in}{1.408923in}}%
\pgfpathlineto{\pgfqpoint{0.818617in}{1.407036in}}%
\pgfpathlineto{\pgfqpoint{0.814455in}{1.406743in}}%
\pgfpathlineto{\pgfqpoint{0.798799in}{1.400117in}}%
\pgfpathlineto{\pgfqpoint{0.790038in}{1.393425in}}%
\pgfpathlineto{\pgfqpoint{0.783142in}{1.389708in}}%
\pgfpathlineto{\pgfqpoint{0.770079in}{1.379814in}}%
\pgfpathlineto{\pgfqpoint{0.767486in}{1.378119in}}%
\pgfpathlineto{\pgfqpoint{0.752621in}{1.366203in}}%
\pgfpathlineto{\pgfqpoint{0.751829in}{1.365578in}}%
\pgfpathlineto{\pgfqpoint{0.736892in}{1.352592in}}%
\pgfpathlineto{\pgfqpoint{0.736173in}{1.351904in}}%
\pgfpathlineto{\pgfqpoint{0.722466in}{1.338981in}}%
\pgfpathlineto{\pgfqpoint{0.720516in}{1.336727in}}%
\pgfpathlineto{\pgfqpoint{0.709136in}{1.325370in}}%
\pgfpathlineto{\pgfqpoint{0.704859in}{1.319375in}}%
\pgfpathlineto{\pgfqpoint{0.697162in}{1.311759in}}%
\pgfpathlineto{\pgfqpoint{0.689541in}{1.298147in}}%
\pgfpathlineto{\pgfqpoint{0.689203in}{1.294529in}}%
\pgfpathlineto{\pgfqpoint{0.687033in}{1.284536in}}%
\pgfpathlineto{\pgfqpoint{0.689203in}{1.282012in}}%
\pgfpathlineto{\pgfqpoint{0.693349in}{1.270925in}}%
\pgfpathlineto{\pgfqpoint{0.704796in}{1.257314in}}%
\pgfpathlineto{\pgfqpoint{0.704859in}{1.257263in}}%
\pgfpathlineto{\pgfqpoint{0.716895in}{1.243703in}}%
\pgfpathlineto{\pgfqpoint{0.720516in}{1.240512in}}%
\pgfpathlineto{\pgfqpoint{0.730802in}{1.230092in}}%
\pgfpathlineto{\pgfqpoint{0.736173in}{1.225345in}}%
\pgfpathlineto{\pgfqpoint{0.746105in}{1.216481in}}%
\pgfpathlineto{\pgfqpoint{0.751829in}{1.211543in}}%
\pgfpathlineto{\pgfqpoint{0.762896in}{1.202870in}}%
\pgfpathlineto{\pgfqpoint{0.767486in}{1.198963in}}%
\pgfpathlineto{\pgfqpoint{0.781400in}{1.189259in}}%
\pgfpathlineto{\pgfqpoint{0.783142in}{1.187706in}}%
\pgfpathlineto{\pgfqpoint{0.798799in}{1.177599in}}%
\pgfpathlineto{\pgfqpoint{0.804273in}{1.175647in}}%
\pgfpathlineto{\pgfqpoint{0.814455in}{1.169222in}}%
\pgfpathclose%
\pgfpathmoveto{\pgfqpoint{0.796298in}{1.202870in}}%
\pgfpathlineto{\pgfqpoint{0.783142in}{1.208941in}}%
\pgfpathlineto{\pgfqpoint{0.771628in}{1.216481in}}%
\pgfpathlineto{\pgfqpoint{0.767486in}{1.219239in}}%
\pgfpathlineto{\pgfqpoint{0.754170in}{1.230092in}}%
\pgfpathlineto{\pgfqpoint{0.751829in}{1.232266in}}%
\pgfpathlineto{\pgfqpoint{0.740516in}{1.243703in}}%
\pgfpathlineto{\pgfqpoint{0.736173in}{1.249372in}}%
\pgfpathlineto{\pgfqpoint{0.729895in}{1.257314in}}%
\pgfpathlineto{\pgfqpoint{0.722830in}{1.270925in}}%
\pgfpathlineto{\pgfqpoint{0.720516in}{1.280940in}}%
\pgfpathlineto{\pgfqpoint{0.719515in}{1.284536in}}%
\pgfpathlineto{\pgfqpoint{0.720472in}{1.298147in}}%
\pgfpathlineto{\pgfqpoint{0.720516in}{1.298254in}}%
\pgfpathlineto{\pgfqpoint{0.725184in}{1.311759in}}%
\pgfpathlineto{\pgfqpoint{0.733822in}{1.325370in}}%
\pgfpathlineto{\pgfqpoint{0.736173in}{1.328041in}}%
\pgfpathlineto{\pgfqpoint{0.745559in}{1.338981in}}%
\pgfpathlineto{\pgfqpoint{0.751829in}{1.344939in}}%
\pgfpathlineto{\pgfqpoint{0.760632in}{1.352592in}}%
\pgfpathlineto{\pgfqpoint{0.767486in}{1.358043in}}%
\pgfpathlineto{\pgfqpoint{0.780070in}{1.366203in}}%
\pgfpathlineto{\pgfqpoint{0.783142in}{1.368247in}}%
\pgfpathlineto{\pgfqpoint{0.798799in}{1.375756in}}%
\pgfpathlineto{\pgfqpoint{0.814332in}{1.379814in}}%
\pgfpathlineto{\pgfqpoint{0.814455in}{1.379853in}}%
\pgfpathlineto{\pgfqpoint{0.830112in}{1.380684in}}%
\pgfpathlineto{\pgfqpoint{0.834248in}{1.379814in}}%
\pgfpathlineto{\pgfqpoint{0.845769in}{1.377802in}}%
\pgfpathlineto{\pgfqpoint{0.861425in}{1.371660in}}%
\pgfpathlineto{\pgfqpoint{0.870560in}{1.366203in}}%
\pgfpathlineto{\pgfqpoint{0.877082in}{1.362427in}}%
\pgfpathlineto{\pgfqpoint{0.890238in}{1.352592in}}%
\pgfpathlineto{\pgfqpoint{0.892738in}{1.350557in}}%
\pgfpathlineto{\pgfqpoint{0.905223in}{1.338981in}}%
\pgfpathlineto{\pgfqpoint{0.908395in}{1.335379in}}%
\pgfpathlineto{\pgfqpoint{0.917068in}{1.325370in}}%
\pgfpathlineto{\pgfqpoint{0.924051in}{1.313933in}}%
\pgfpathlineto{\pgfqpoint{0.925532in}{1.311759in}}%
\pgfpathlineto{\pgfqpoint{0.930748in}{1.298147in}}%
\pgfpathlineto{\pgfqpoint{0.931616in}{1.284536in}}%
\pgfpathlineto{\pgfqpoint{0.928142in}{1.270925in}}%
\pgfpathlineto{\pgfqpoint{0.924051in}{1.263709in}}%
\pgfpathlineto{\pgfqpoint{0.920823in}{1.257314in}}%
\pgfpathlineto{\pgfqpoint{0.910322in}{1.243703in}}%
\pgfpathlineto{\pgfqpoint{0.908395in}{1.241739in}}%
\pgfpathlineto{\pgfqpoint{0.896690in}{1.230092in}}%
\pgfpathlineto{\pgfqpoint{0.892738in}{1.226656in}}%
\pgfpathlineto{\pgfqpoint{0.879341in}{1.216481in}}%
\pgfpathlineto{\pgfqpoint{0.877082in}{1.214805in}}%
\pgfpathlineto{\pgfqpoint{0.861425in}{1.205677in}}%
\pgfpathlineto{\pgfqpoint{0.854069in}{1.202870in}}%
\pgfpathlineto{\pgfqpoint{0.845769in}{1.199314in}}%
\pgfpathlineto{\pgfqpoint{0.830112in}{1.196293in}}%
\pgfpathlineto{\pgfqpoint{0.814455in}{1.197048in}}%
\pgfpathlineto{\pgfqpoint{0.798799in}{1.201583in}}%
\pgfpathlineto{\pgfqpoint{0.796298in}{1.202870in}}%
\pgfpathclose%
\pgfpathmoveto{\pgfqpoint{1.127587in}{1.168228in}}%
\pgfpathlineto{\pgfqpoint{1.143243in}{1.168228in}}%
\pgfpathlineto{\pgfqpoint{1.157278in}{1.175647in}}%
\pgfpathlineto{\pgfqpoint{1.158900in}{1.176130in}}%
\pgfpathlineto{\pgfqpoint{1.174556in}{1.185317in}}%
\pgfpathlineto{\pgfqpoint{1.179211in}{1.189259in}}%
\pgfpathlineto{\pgfqpoint{1.190213in}{1.196530in}}%
\pgfpathlineto{\pgfqpoint{1.197879in}{1.202870in}}%
\pgfpathlineto{\pgfqpoint{1.205870in}{1.208941in}}%
\pgfpathlineto{\pgfqpoint{1.214702in}{1.216481in}}%
\pgfpathlineto{\pgfqpoint{1.221526in}{1.222504in}}%
\pgfpathlineto{\pgfqpoint{1.230024in}{1.230092in}}%
\pgfpathlineto{\pgfqpoint{1.237183in}{1.237422in}}%
\pgfpathlineto{\pgfqpoint{1.244044in}{1.243703in}}%
\pgfpathlineto{\pgfqpoint{1.252839in}{1.254023in}}%
\pgfpathlineto{\pgfqpoint{1.256595in}{1.257314in}}%
\pgfpathlineto{\pgfqpoint{1.266962in}{1.270925in}}%
\pgfpathlineto{\pgfqpoint{1.268496in}{1.275422in}}%
\pgfpathlineto{\pgfqpoint{1.274668in}{1.284536in}}%
\pgfpathlineto{\pgfqpoint{1.272354in}{1.298147in}}%
\pgfpathlineto{\pgfqpoint{1.268496in}{1.301984in}}%
\pgfpathlineto{\pgfqpoint{1.263509in}{1.311759in}}%
\pgfpathlineto{\pgfqpoint{1.252839in}{1.323285in}}%
\pgfpathlineto{\pgfqpoint{1.251411in}{1.325370in}}%
\pgfpathlineto{\pgfqpoint{1.238343in}{1.338981in}}%
\pgfpathlineto{\pgfqpoint{1.237183in}{1.339979in}}%
\pgfpathlineto{\pgfqpoint{1.224025in}{1.352592in}}%
\pgfpathlineto{\pgfqpoint{1.221526in}{1.354769in}}%
\pgfpathlineto{\pgfqpoint{1.208226in}{1.366203in}}%
\pgfpathlineto{\pgfqpoint{1.205870in}{1.368247in}}%
\pgfpathlineto{\pgfqpoint{1.190900in}{1.379814in}}%
\pgfpathlineto{\pgfqpoint{1.190213in}{1.380423in}}%
\pgfpathlineto{\pgfqpoint{1.174556in}{1.391660in}}%
\pgfpathlineto{\pgfqpoint{1.170975in}{1.393425in}}%
\pgfpathlineto{\pgfqpoint{1.158900in}{1.401885in}}%
\pgfpathlineto{\pgfqpoint{1.144304in}{1.407036in}}%
\pgfpathlineto{\pgfqpoint{1.143243in}{1.407894in}}%
\pgfpathlineto{\pgfqpoint{1.127587in}{1.407894in}}%
\pgfpathlineto{\pgfqpoint{1.126526in}{1.407036in}}%
\pgfpathlineto{\pgfqpoint{1.111930in}{1.401885in}}%
\pgfpathlineto{\pgfqpoint{1.099855in}{1.393425in}}%
\pgfpathlineto{\pgfqpoint{1.096274in}{1.391660in}}%
\pgfpathlineto{\pgfqpoint{1.080617in}{1.380423in}}%
\pgfpathlineto{\pgfqpoint{1.079930in}{1.379814in}}%
\pgfpathlineto{\pgfqpoint{1.064960in}{1.368247in}}%
\pgfpathlineto{\pgfqpoint{1.062604in}{1.366203in}}%
\pgfpathlineto{\pgfqpoint{1.049304in}{1.354769in}}%
\pgfpathlineto{\pgfqpoint{1.046805in}{1.352592in}}%
\pgfpathlineto{\pgfqpoint{1.033647in}{1.339979in}}%
\pgfpathlineto{\pgfqpoint{1.032487in}{1.338981in}}%
\pgfpathlineto{\pgfqpoint{1.019419in}{1.325370in}}%
\pgfpathlineto{\pgfqpoint{1.017991in}{1.323285in}}%
\pgfpathlineto{\pgfqpoint{1.007321in}{1.311759in}}%
\pgfpathlineto{\pgfqpoint{1.002334in}{1.301984in}}%
\pgfpathlineto{\pgfqpoint{0.998476in}{1.298147in}}%
\pgfpathlineto{\pgfqpoint{0.996162in}{1.284536in}}%
\pgfpathlineto{\pgfqpoint{1.002334in}{1.275422in}}%
\pgfpathlineto{\pgfqpoint{1.003868in}{1.270925in}}%
\pgfpathlineto{\pgfqpoint{1.014235in}{1.257314in}}%
\pgfpathlineto{\pgfqpoint{1.017991in}{1.254023in}}%
\pgfpathlineto{\pgfqpoint{1.026786in}{1.243703in}}%
\pgfpathlineto{\pgfqpoint{1.033647in}{1.237422in}}%
\pgfpathlineto{\pgfqpoint{1.040806in}{1.230092in}}%
\pgfpathlineto{\pgfqpoint{1.049304in}{1.222504in}}%
\pgfpathlineto{\pgfqpoint{1.056128in}{1.216481in}}%
\pgfpathlineto{\pgfqpoint{1.064960in}{1.208941in}}%
\pgfpathlineto{\pgfqpoint{1.072951in}{1.202870in}}%
\pgfpathlineto{\pgfqpoint{1.080617in}{1.196530in}}%
\pgfpathlineto{\pgfqpoint{1.091619in}{1.189259in}}%
\pgfpathlineto{\pgfqpoint{1.096274in}{1.185317in}}%
\pgfpathlineto{\pgfqpoint{1.111930in}{1.176130in}}%
\pgfpathlineto{\pgfqpoint{1.113552in}{1.175647in}}%
\pgfpathlineto{\pgfqpoint{1.127587in}{1.168228in}}%
\pgfpathclose%
\pgfpathmoveto{\pgfqpoint{1.106640in}{1.202870in}}%
\pgfpathlineto{\pgfqpoint{1.096274in}{1.207244in}}%
\pgfpathlineto{\pgfqpoint{1.081429in}{1.216481in}}%
\pgfpathlineto{\pgfqpoint{1.080617in}{1.216993in}}%
\pgfpathlineto{\pgfqpoint{1.064960in}{1.229343in}}%
\pgfpathlineto{\pgfqpoint{1.064111in}{1.230092in}}%
\pgfpathlineto{\pgfqpoint{1.050559in}{1.243703in}}%
\pgfpathlineto{\pgfqpoint{1.049304in}{1.245323in}}%
\pgfpathlineto{\pgfqpoint{1.039920in}{1.257314in}}%
\pgfpathlineto{\pgfqpoint{1.033647in}{1.269664in}}%
\pgfpathlineto{\pgfqpoint{1.032910in}{1.270925in}}%
\pgfpathlineto{\pgfqpoint{1.029274in}{1.284536in}}%
\pgfpathlineto{\pgfqpoint{1.030182in}{1.298147in}}%
\pgfpathlineto{\pgfqpoint{1.033647in}{1.306857in}}%
\pgfpathlineto{\pgfqpoint{1.035323in}{1.311759in}}%
\pgfpathlineto{\pgfqpoint{1.043752in}{1.325370in}}%
\pgfpathlineto{\pgfqpoint{1.049304in}{1.331742in}}%
\pgfpathlineto{\pgfqpoint{1.055584in}{1.338981in}}%
\pgfpathlineto{\pgfqpoint{1.064960in}{1.347799in}}%
\pgfpathlineto{\pgfqpoint{1.070647in}{1.352592in}}%
\pgfpathlineto{\pgfqpoint{1.080617in}{1.360294in}}%
\pgfpathlineto{\pgfqpoint{1.090233in}{1.366203in}}%
\pgfpathlineto{\pgfqpoint{1.096274in}{1.370022in}}%
\pgfpathlineto{\pgfqpoint{1.111930in}{1.376848in}}%
\pgfpathlineto{\pgfqpoint{1.125533in}{1.379814in}}%
\pgfpathlineto{\pgfqpoint{1.127587in}{1.380352in}}%
\pgfpathlineto{\pgfqpoint{1.143243in}{1.380352in}}%
\pgfpathlineto{\pgfqpoint{1.145297in}{1.379814in}}%
\pgfpathlineto{\pgfqpoint{1.158900in}{1.376848in}}%
\pgfpathlineto{\pgfqpoint{1.174556in}{1.370022in}}%
\pgfpathlineto{\pgfqpoint{1.180597in}{1.366203in}}%
\pgfpathlineto{\pgfqpoint{1.190213in}{1.360294in}}%
\pgfpathlineto{\pgfqpoint{1.200183in}{1.352592in}}%
\pgfpathlineto{\pgfqpoint{1.205870in}{1.347799in}}%
\pgfpathlineto{\pgfqpoint{1.215246in}{1.338981in}}%
\pgfpathlineto{\pgfqpoint{1.221526in}{1.331742in}}%
\pgfpathlineto{\pgfqpoint{1.227078in}{1.325370in}}%
\pgfpathlineto{\pgfqpoint{1.235507in}{1.311759in}}%
\pgfpathlineto{\pgfqpoint{1.237183in}{1.306857in}}%
\pgfpathlineto{\pgfqpoint{1.240648in}{1.298147in}}%
\pgfpathlineto{\pgfqpoint{1.241556in}{1.284536in}}%
\pgfpathlineto{\pgfqpoint{1.237920in}{1.270925in}}%
\pgfpathlineto{\pgfqpoint{1.237183in}{1.269664in}}%
\pgfpathlineto{\pgfqpoint{1.230910in}{1.257314in}}%
\pgfpathlineto{\pgfqpoint{1.221526in}{1.245323in}}%
\pgfpathlineto{\pgfqpoint{1.220271in}{1.243703in}}%
\pgfpathlineto{\pgfqpoint{1.206719in}{1.230092in}}%
\pgfpathlineto{\pgfqpoint{1.205870in}{1.229343in}}%
\pgfpathlineto{\pgfqpoint{1.190213in}{1.216993in}}%
\pgfpathlineto{\pgfqpoint{1.189401in}{1.216481in}}%
\pgfpathlineto{\pgfqpoint{1.174556in}{1.207244in}}%
\pgfpathlineto{\pgfqpoint{1.164190in}{1.202870in}}%
\pgfpathlineto{\pgfqpoint{1.158900in}{1.200372in}}%
\pgfpathlineto{\pgfqpoint{1.143243in}{1.196595in}}%
\pgfpathlineto{\pgfqpoint{1.127587in}{1.196595in}}%
\pgfpathlineto{\pgfqpoint{1.111930in}{1.200372in}}%
\pgfpathlineto{\pgfqpoint{1.106640in}{1.202870in}}%
\pgfpathclose%
\pgfpathmoveto{\pgfqpoint{1.425061in}{1.174197in}}%
\pgfpathlineto{\pgfqpoint{1.440718in}{1.167565in}}%
\pgfpathlineto{\pgfqpoint{1.456375in}{1.169222in}}%
\pgfpathlineto{\pgfqpoint{1.466557in}{1.175647in}}%
\pgfpathlineto{\pgfqpoint{1.472031in}{1.177599in}}%
\pgfpathlineto{\pgfqpoint{1.487688in}{1.187706in}}%
\pgfpathlineto{\pgfqpoint{1.489430in}{1.189259in}}%
\pgfpathlineto{\pgfqpoint{1.503344in}{1.198963in}}%
\pgfpathlineto{\pgfqpoint{1.507934in}{1.202870in}}%
\pgfpathlineto{\pgfqpoint{1.519001in}{1.211543in}}%
\pgfpathlineto{\pgfqpoint{1.524725in}{1.216481in}}%
\pgfpathlineto{\pgfqpoint{1.534657in}{1.225345in}}%
\pgfpathlineto{\pgfqpoint{1.540028in}{1.230092in}}%
\pgfpathlineto{\pgfqpoint{1.550314in}{1.240512in}}%
\pgfpathlineto{\pgfqpoint{1.553935in}{1.243703in}}%
\pgfpathlineto{\pgfqpoint{1.565971in}{1.257263in}}%
\pgfpathlineto{\pgfqpoint{1.566034in}{1.257314in}}%
\pgfpathlineto{\pgfqpoint{1.577481in}{1.270925in}}%
\pgfpathlineto{\pgfqpoint{1.581627in}{1.282012in}}%
\pgfpathlineto{\pgfqpoint{1.583797in}{1.284536in}}%
\pgfpathlineto{\pgfqpoint{1.581627in}{1.294529in}}%
\pgfpathlineto{\pgfqpoint{1.581289in}{1.298147in}}%
\pgfpathlineto{\pgfqpoint{1.573668in}{1.311759in}}%
\pgfpathlineto{\pgfqpoint{1.565971in}{1.319375in}}%
\pgfpathlineto{\pgfqpoint{1.561694in}{1.325370in}}%
\pgfpathlineto{\pgfqpoint{1.550314in}{1.336727in}}%
\pgfpathlineto{\pgfqpoint{1.548364in}{1.338981in}}%
\pgfpathlineto{\pgfqpoint{1.534657in}{1.351904in}}%
\pgfpathlineto{\pgfqpoint{1.533938in}{1.352592in}}%
\pgfpathlineto{\pgfqpoint{1.519001in}{1.365578in}}%
\pgfpathlineto{\pgfqpoint{1.518209in}{1.366203in}}%
\pgfpathlineto{\pgfqpoint{1.503344in}{1.378119in}}%
\pgfpathlineto{\pgfqpoint{1.500751in}{1.379814in}}%
\pgfpathlineto{\pgfqpoint{1.487688in}{1.389708in}}%
\pgfpathlineto{\pgfqpoint{1.480792in}{1.393425in}}%
\pgfpathlineto{\pgfqpoint{1.472031in}{1.400117in}}%
\pgfpathlineto{\pgfqpoint{1.456375in}{1.406743in}}%
\pgfpathlineto{\pgfqpoint{1.452213in}{1.407036in}}%
\pgfpathlineto{\pgfqpoint{1.440718in}{1.408923in}}%
\pgfpathlineto{\pgfqpoint{1.437815in}{1.407036in}}%
\pgfpathlineto{\pgfqpoint{1.425061in}{1.403432in}}%
\pgfpathlineto{\pgfqpoint{1.409405in}{1.393481in}}%
\pgfpathlineto{\pgfqpoint{1.409346in}{1.393425in}}%
\pgfpathlineto{\pgfqpoint{1.393748in}{1.382962in}}%
\pgfpathlineto{\pgfqpoint{1.390078in}{1.379814in}}%
\pgfpathlineto{\pgfqpoint{1.378092in}{1.370872in}}%
\pgfpathlineto{\pgfqpoint{1.372631in}{1.366203in}}%
\pgfpathlineto{\pgfqpoint{1.362435in}{1.357569in}}%
\pgfpathlineto{\pgfqpoint{1.356755in}{1.352592in}}%
\pgfpathlineto{\pgfqpoint{1.346779in}{1.342971in}}%
\pgfpathlineto{\pgfqpoint{1.342285in}{1.338981in}}%
\pgfpathlineto{\pgfqpoint{1.331122in}{1.326885in}}%
\pgfpathlineto{\pgfqpoint{1.329336in}{1.325370in}}%
\pgfpathlineto{\pgfqpoint{1.317710in}{1.311759in}}%
\pgfpathlineto{\pgfqpoint{1.315466in}{1.307000in}}%
\pgfpathlineto{\pgfqpoint{1.308075in}{1.298147in}}%
\pgfpathlineto{\pgfqpoint{1.306169in}{1.284536in}}%
\pgfpathlineto{\pgfqpoint{1.313797in}{1.270925in}}%
\pgfpathlineto{\pgfqpoint{1.315466in}{1.269566in}}%
\pgfpathlineto{\pgfqpoint{1.324051in}{1.257314in}}%
\pgfpathlineto{\pgfqpoint{1.331122in}{1.250637in}}%
\pgfpathlineto{\pgfqpoint{1.336837in}{1.243703in}}%
\pgfpathlineto{\pgfqpoint{1.346779in}{1.234312in}}%
\pgfpathlineto{\pgfqpoint{1.350875in}{1.230092in}}%
\pgfpathlineto{\pgfqpoint{1.362435in}{1.219712in}}%
\pgfpathlineto{\pgfqpoint{1.366152in}{1.216481in}}%
\pgfpathlineto{\pgfqpoint{1.378092in}{1.206431in}}%
\pgfpathlineto{\pgfqpoint{1.382946in}{1.202870in}}%
\pgfpathlineto{\pgfqpoint{1.393748in}{1.194227in}}%
\pgfpathlineto{\pgfqpoint{1.401725in}{1.189259in}}%
\pgfpathlineto{\pgfqpoint{1.409405in}{1.183112in}}%
\pgfpathlineto{\pgfqpoint{1.423498in}{1.175647in}}%
\pgfpathlineto{\pgfqpoint{1.425061in}{1.174197in}}%
\pgfpathclose%
\pgfpathmoveto{\pgfqpoint{1.416761in}{1.202870in}}%
\pgfpathlineto{\pgfqpoint{1.409405in}{1.205677in}}%
\pgfpathlineto{\pgfqpoint{1.393748in}{1.214805in}}%
\pgfpathlineto{\pgfqpoint{1.391489in}{1.216481in}}%
\pgfpathlineto{\pgfqpoint{1.378092in}{1.226656in}}%
\pgfpathlineto{\pgfqpoint{1.374140in}{1.230092in}}%
\pgfpathlineto{\pgfqpoint{1.362435in}{1.241739in}}%
\pgfpathlineto{\pgfqpoint{1.360508in}{1.243703in}}%
\pgfpathlineto{\pgfqpoint{1.350007in}{1.257314in}}%
\pgfpathlineto{\pgfqpoint{1.346779in}{1.263709in}}%
\pgfpathlineto{\pgfqpoint{1.342688in}{1.270925in}}%
\pgfpathlineto{\pgfqpoint{1.339214in}{1.284536in}}%
\pgfpathlineto{\pgfqpoint{1.340082in}{1.298147in}}%
\pgfpathlineto{\pgfqpoint{1.345298in}{1.311759in}}%
\pgfpathlineto{\pgfqpoint{1.346779in}{1.313933in}}%
\pgfpathlineto{\pgfqpoint{1.353762in}{1.325370in}}%
\pgfpathlineto{\pgfqpoint{1.362435in}{1.335379in}}%
\pgfpathlineto{\pgfqpoint{1.365607in}{1.338981in}}%
\pgfpathlineto{\pgfqpoint{1.378092in}{1.350557in}}%
\pgfpathlineto{\pgfqpoint{1.380592in}{1.352592in}}%
\pgfpathlineto{\pgfqpoint{1.393748in}{1.362427in}}%
\pgfpathlineto{\pgfqpoint{1.400270in}{1.366203in}}%
\pgfpathlineto{\pgfqpoint{1.409405in}{1.371660in}}%
\pgfpathlineto{\pgfqpoint{1.425061in}{1.377802in}}%
\pgfpathlineto{\pgfqpoint{1.436582in}{1.379814in}}%
\pgfpathlineto{\pgfqpoint{1.440718in}{1.380684in}}%
\pgfpathlineto{\pgfqpoint{1.456375in}{1.379853in}}%
\pgfpathlineto{\pgfqpoint{1.456498in}{1.379814in}}%
\pgfpathlineto{\pgfqpoint{1.472031in}{1.375756in}}%
\pgfpathlineto{\pgfqpoint{1.487688in}{1.368247in}}%
\pgfpathlineto{\pgfqpoint{1.490760in}{1.366203in}}%
\pgfpathlineto{\pgfqpoint{1.503344in}{1.358043in}}%
\pgfpathlineto{\pgfqpoint{1.510198in}{1.352592in}}%
\pgfpathlineto{\pgfqpoint{1.519001in}{1.344939in}}%
\pgfpathlineto{\pgfqpoint{1.525271in}{1.338981in}}%
\pgfpathlineto{\pgfqpoint{1.534657in}{1.328041in}}%
\pgfpathlineto{\pgfqpoint{1.537008in}{1.325370in}}%
\pgfpathlineto{\pgfqpoint{1.545646in}{1.311759in}}%
\pgfpathlineto{\pgfqpoint{1.550314in}{1.298254in}}%
\pgfpathlineto{\pgfqpoint{1.550358in}{1.298147in}}%
\pgfpathlineto{\pgfqpoint{1.551315in}{1.284536in}}%
\pgfpathlineto{\pgfqpoint{1.550314in}{1.280940in}}%
\pgfpathlineto{\pgfqpoint{1.548000in}{1.270925in}}%
\pgfpathlineto{\pgfqpoint{1.540935in}{1.257314in}}%
\pgfpathlineto{\pgfqpoint{1.534657in}{1.249372in}}%
\pgfpathlineto{\pgfqpoint{1.530314in}{1.243703in}}%
\pgfpathlineto{\pgfqpoint{1.519001in}{1.232266in}}%
\pgfpathlineto{\pgfqpoint{1.516660in}{1.230092in}}%
\pgfpathlineto{\pgfqpoint{1.503344in}{1.219239in}}%
\pgfpathlineto{\pgfqpoint{1.499202in}{1.216481in}}%
\pgfpathlineto{\pgfqpoint{1.487688in}{1.208941in}}%
\pgfpathlineto{\pgfqpoint{1.474532in}{1.202870in}}%
\pgfpathlineto{\pgfqpoint{1.472031in}{1.201583in}}%
\pgfpathlineto{\pgfqpoint{1.456375in}{1.197048in}}%
\pgfpathlineto{\pgfqpoint{1.440718in}{1.196293in}}%
\pgfpathlineto{\pgfqpoint{1.425061in}{1.199314in}}%
\pgfpathlineto{\pgfqpoint{1.416761in}{1.202870in}}%
\pgfpathclose%
\pgfpathmoveto{\pgfqpoint{1.738193in}{1.172206in}}%
\pgfpathlineto{\pgfqpoint{1.753849in}{1.167234in}}%
\pgfpathlineto{\pgfqpoint{1.769506in}{1.170548in}}%
\pgfpathlineto{\pgfqpoint{1.776475in}{1.175647in}}%
\pgfpathlineto{\pgfqpoint{1.785162in}{1.179252in}}%
\pgfpathlineto{\pgfqpoint{1.799428in}{1.189259in}}%
\pgfpathlineto{\pgfqpoint{1.800819in}{1.190013in}}%
\pgfpathlineto{\pgfqpoint{1.816476in}{1.201521in}}%
\pgfpathlineto{\pgfqpoint{1.818021in}{1.202870in}}%
\pgfpathlineto{\pgfqpoint{1.832132in}{1.214231in}}%
\pgfpathlineto{\pgfqpoint{1.834723in}{1.216481in}}%
\pgfpathlineto{\pgfqpoint{1.847789in}{1.228224in}}%
\pgfpathlineto{\pgfqpoint{1.849933in}{1.230092in}}%
\pgfpathlineto{\pgfqpoint{1.863445in}{1.243564in}}%
\pgfpathlineto{\pgfqpoint{1.863611in}{1.243703in}}%
\pgfpathlineto{\pgfqpoint{1.876422in}{1.257314in}}%
\pgfpathlineto{\pgfqpoint{1.879102in}{1.261643in}}%
\pgfpathlineto{\pgfqpoint{1.887763in}{1.270925in}}%
\pgfpathlineto{\pgfqpoint{1.893456in}{1.284536in}}%
\pgfpathlineto{\pgfqpoint{1.892034in}{1.298147in}}%
\pgfpathlineto{\pgfqpoint{1.883488in}{1.311759in}}%
\pgfpathlineto{\pgfqpoint{1.879102in}{1.315671in}}%
\pgfpathlineto{\pgfqpoint{1.871841in}{1.325370in}}%
\pgfpathlineto{\pgfqpoint{1.863445in}{1.333375in}}%
\pgfpathlineto{\pgfqpoint{1.858517in}{1.338981in}}%
\pgfpathlineto{\pgfqpoint{1.847789in}{1.348948in}}%
\pgfpathlineto{\pgfqpoint{1.844005in}{1.352592in}}%
\pgfpathlineto{\pgfqpoint{1.832132in}{1.362981in}}%
\pgfpathlineto{\pgfqpoint{1.828164in}{1.366203in}}%
\pgfpathlineto{\pgfqpoint{1.816476in}{1.375812in}}%
\pgfpathlineto{\pgfqpoint{1.810645in}{1.379814in}}%
\pgfpathlineto{\pgfqpoint{1.800819in}{1.387607in}}%
\pgfpathlineto{\pgfqpoint{1.790860in}{1.393425in}}%
\pgfpathlineto{\pgfqpoint{1.785162in}{1.398126in}}%
\pgfpathlineto{\pgfqpoint{1.769506in}{1.405860in}}%
\pgfpathlineto{\pgfqpoint{1.761184in}{1.407036in}}%
\pgfpathlineto{\pgfqpoint{1.753849in}{1.409438in}}%
\pgfpathlineto{\pgfqpoint{1.748943in}{1.407036in}}%
\pgfpathlineto{\pgfqpoint{1.738193in}{1.404756in}}%
\pgfpathlineto{\pgfqpoint{1.722536in}{1.395914in}}%
\pgfpathlineto{\pgfqpoint{1.719722in}{1.393425in}}%
\pgfpathlineto{\pgfqpoint{1.706880in}{1.385358in}}%
\pgfpathlineto{\pgfqpoint{1.700173in}{1.379814in}}%
\pgfpathlineto{\pgfqpoint{1.691223in}{1.373395in}}%
\pgfpathlineto{\pgfqpoint{1.682661in}{1.366203in}}%
\pgfpathlineto{\pgfqpoint{1.675567in}{1.360309in}}%
\pgfpathlineto{\pgfqpoint{1.666771in}{1.352592in}}%
\pgfpathlineto{\pgfqpoint{1.659910in}{1.345965in}}%
\pgfpathlineto{\pgfqpoint{1.652240in}{1.338981in}}%
\pgfpathlineto{\pgfqpoint{1.644253in}{1.330087in}}%
\pgfpathlineto{\pgfqpoint{1.639032in}{1.325370in}}%
\pgfpathlineto{\pgfqpoint{1.628597in}{1.312204in}}%
\pgfpathlineto{\pgfqpoint{1.628031in}{1.311759in}}%
\pgfpathlineto{\pgfqpoint{1.618264in}{1.298147in}}%
\pgfpathlineto{\pgfqpoint{1.616638in}{1.284536in}}%
\pgfpathlineto{\pgfqpoint{1.623145in}{1.270925in}}%
\pgfpathlineto{\pgfqpoint{1.628597in}{1.265764in}}%
\pgfpathlineto{\pgfqpoint{1.634134in}{1.257314in}}%
\pgfpathlineto{\pgfqpoint{1.644253in}{1.247133in}}%
\pgfpathlineto{\pgfqpoint{1.647004in}{1.243703in}}%
\pgfpathlineto{\pgfqpoint{1.659910in}{1.231199in}}%
\pgfpathlineto{\pgfqpoint{1.660983in}{1.230092in}}%
\pgfpathlineto{\pgfqpoint{1.675567in}{1.216979in}}%
\pgfpathlineto{\pgfqpoint{1.676150in}{1.216481in}}%
\pgfpathlineto{\pgfqpoint{1.691223in}{1.204018in}}%
\pgfpathlineto{\pgfqpoint{1.692851in}{1.202870in}}%
\pgfpathlineto{\pgfqpoint{1.706880in}{1.192054in}}%
\pgfpathlineto{\pgfqpoint{1.711671in}{1.189259in}}%
\pgfpathlineto{\pgfqpoint{1.722536in}{1.181090in}}%
\pgfpathlineto{\pgfqpoint{1.734050in}{1.175647in}}%
\pgfpathlineto{\pgfqpoint{1.738193in}{1.172206in}}%
\pgfpathclose%
\pgfpathmoveto{\pgfqpoint{1.726560in}{1.202870in}}%
\pgfpathlineto{\pgfqpoint{1.722536in}{1.204240in}}%
\pgfpathlineto{\pgfqpoint{1.706880in}{1.212723in}}%
\pgfpathlineto{\pgfqpoint{1.701621in}{1.216481in}}%
\pgfpathlineto{\pgfqpoint{1.691223in}{1.224074in}}%
\pgfpathlineto{\pgfqpoint{1.684176in}{1.230092in}}%
\pgfpathlineto{\pgfqpoint{1.675567in}{1.238497in}}%
\pgfpathlineto{\pgfqpoint{1.670465in}{1.243703in}}%
\pgfpathlineto{\pgfqpoint{1.660129in}{1.257314in}}%
\pgfpathlineto{\pgfqpoint{1.659910in}{1.257748in}}%
\pgfpathlineto{\pgfqpoint{1.652629in}{1.270925in}}%
\pgfpathlineto{\pgfqpoint{1.649289in}{1.284536in}}%
\pgfpathlineto{\pgfqpoint{1.650123in}{1.298147in}}%
\pgfpathlineto{\pgfqpoint{1.655137in}{1.311759in}}%
\pgfpathlineto{\pgfqpoint{1.659910in}{1.318949in}}%
\pgfpathlineto{\pgfqpoint{1.663825in}{1.325370in}}%
\pgfpathlineto{\pgfqpoint{1.675567in}{1.338940in}}%
\pgfpathlineto{\pgfqpoint{1.675603in}{1.338981in}}%
\pgfpathlineto{\pgfqpoint{1.690500in}{1.352592in}}%
\pgfpathlineto{\pgfqpoint{1.691223in}{1.353195in}}%
\pgfpathlineto{\pgfqpoint{1.706880in}{1.364438in}}%
\pgfpathlineto{\pgfqpoint{1.710134in}{1.366203in}}%
\pgfpathlineto{\pgfqpoint{1.722536in}{1.373162in}}%
\pgfpathlineto{\pgfqpoint{1.738193in}{1.378620in}}%
\pgfpathlineto{\pgfqpoint{1.747307in}{1.379814in}}%
\pgfpathlineto{\pgfqpoint{1.753849in}{1.380851in}}%
\pgfpathlineto{\pgfqpoint{1.763630in}{1.379814in}}%
\pgfpathlineto{\pgfqpoint{1.769506in}{1.379301in}}%
\pgfpathlineto{\pgfqpoint{1.785162in}{1.374528in}}%
\pgfpathlineto{\pgfqpoint{1.800819in}{1.366337in}}%
\pgfpathlineto{\pgfqpoint{1.801011in}{1.366203in}}%
\pgfpathlineto{\pgfqpoint{1.816476in}{1.355675in}}%
\pgfpathlineto{\pgfqpoint{1.820256in}{1.352592in}}%
\pgfpathlineto{\pgfqpoint{1.832132in}{1.341985in}}%
\pgfpathlineto{\pgfqpoint{1.835273in}{1.338981in}}%
\pgfpathlineto{\pgfqpoint{1.846917in}{1.325370in}}%
\pgfpathlineto{\pgfqpoint{1.847789in}{1.323944in}}%
\pgfpathlineto{\pgfqpoint{1.855719in}{1.311759in}}%
\pgfpathlineto{\pgfqpoint{1.860563in}{1.298147in}}%
\pgfpathlineto{\pgfqpoint{1.861369in}{1.284536in}}%
\pgfpathlineto{\pgfqpoint{1.858142in}{1.270925in}}%
\pgfpathlineto{\pgfqpoint{1.850868in}{1.257314in}}%
\pgfpathlineto{\pgfqpoint{1.847789in}{1.253476in}}%
\pgfpathlineto{\pgfqpoint{1.840353in}{1.243703in}}%
\pgfpathlineto{\pgfqpoint{1.832132in}{1.235336in}}%
\pgfpathlineto{\pgfqpoint{1.826635in}{1.230092in}}%
\pgfpathlineto{\pgfqpoint{1.816476in}{1.221600in}}%
\pgfpathlineto{\pgfqpoint{1.809151in}{1.216481in}}%
\pgfpathlineto{\pgfqpoint{1.800819in}{1.210768in}}%
\pgfpathlineto{\pgfqpoint{1.785162in}{1.202934in}}%
\pgfpathlineto{\pgfqpoint{1.784946in}{1.202870in}}%
\pgfpathlineto{\pgfqpoint{1.769506in}{1.197652in}}%
\pgfpathlineto{\pgfqpoint{1.753849in}{1.196143in}}%
\pgfpathlineto{\pgfqpoint{1.738193in}{1.198407in}}%
\pgfpathlineto{\pgfqpoint{1.726560in}{1.202870in}}%
\pgfpathclose%
\pgfpathmoveto{\pgfqpoint{0.501324in}{1.440017in}}%
\pgfpathlineto{\pgfqpoint{0.516981in}{1.437191in}}%
\pgfpathlineto{\pgfqpoint{0.532637in}{1.441432in}}%
\pgfpathlineto{\pgfqpoint{0.541642in}{1.447870in}}%
\pgfpathlineto{\pgfqpoint{0.548294in}{1.450810in}}%
\pgfpathlineto{\pgfqpoint{0.563416in}{1.461481in}}%
\pgfpathlineto{\pgfqpoint{0.563950in}{1.461784in}}%
\pgfpathlineto{\pgfqpoint{0.579607in}{1.473459in}}%
\pgfpathlineto{\pgfqpoint{0.581485in}{1.475092in}}%
\pgfpathlineto{\pgfqpoint{0.595263in}{1.486426in}}%
\pgfpathlineto{\pgfqpoint{0.597882in}{1.488703in}}%
\pgfpathlineto{\pgfqpoint{0.610920in}{1.500681in}}%
\pgfpathlineto{\pgfqpoint{0.612799in}{1.502314in}}%
\pgfpathlineto{\pgfqpoint{0.626228in}{1.515925in}}%
\pgfpathlineto{\pgfqpoint{0.626577in}{1.516390in}}%
\pgfpathlineto{\pgfqpoint{0.638851in}{1.529536in}}%
\pgfpathlineto{\pgfqpoint{0.642233in}{1.535319in}}%
\pgfpathlineto{\pgfqpoint{0.649639in}{1.543147in}}%
\pgfpathlineto{\pgfqpoint{0.654517in}{1.556759in}}%
\pgfpathlineto{\pgfqpoint{0.651265in}{1.570370in}}%
\pgfpathlineto{\pgfqpoint{0.642233in}{1.581209in}}%
\pgfpathlineto{\pgfqpoint{0.640810in}{1.583981in}}%
\pgfpathlineto{\pgfqpoint{0.629057in}{1.597592in}}%
\pgfpathlineto{\pgfqpoint{0.626577in}{1.599732in}}%
\pgfpathlineto{\pgfqpoint{0.615762in}{1.611203in}}%
\pgfpathlineto{\pgfqpoint{0.610920in}{1.615502in}}%
\pgfpathlineto{\pgfqpoint{0.601015in}{1.624814in}}%
\pgfpathlineto{\pgfqpoint{0.595263in}{1.629828in}}%
\pgfpathlineto{\pgfqpoint{0.584842in}{1.638425in}}%
\pgfpathlineto{\pgfqpoint{0.579607in}{1.642888in}}%
\pgfpathlineto{\pgfqpoint{0.567054in}{1.652036in}}%
\pgfpathlineto{\pgfqpoint{0.563950in}{1.654721in}}%
\pgfpathlineto{\pgfqpoint{0.548294in}{1.665071in}}%
\pgfpathlineto{\pgfqpoint{0.546912in}{1.665648in}}%
\pgfpathlineto{\pgfqpoint{0.532637in}{1.674663in}}%
\pgfpathlineto{\pgfqpoint{0.516981in}{1.678373in}}%
\pgfpathlineto{\pgfqpoint{0.501324in}{1.675900in}}%
\pgfpathlineto{\pgfqpoint{0.485668in}{1.667229in}}%
\pgfpathlineto{\pgfqpoint{0.483939in}{1.665648in}}%
\pgfpathlineto{\pgfqpoint{0.470011in}{1.657106in}}%
\pgfpathlineto{\pgfqpoint{0.463902in}{1.652036in}}%
\pgfpathlineto{\pgfqpoint{0.454354in}{1.645377in}}%
\pgfpathlineto{\pgfqpoint{0.446023in}{1.638425in}}%
\pgfpathlineto{\pgfqpoint{0.438698in}{1.632520in}}%
\pgfpathlineto{\pgfqpoint{0.429834in}{1.624814in}}%
\pgfpathlineto{\pgfqpoint{0.423041in}{1.618446in}}%
\pgfpathlineto{\pgfqpoint{0.415045in}{1.611203in}}%
\pgfpathlineto{\pgfqpoint{0.407385in}{1.602903in}}%
\pgfpathlineto{\pgfqpoint{0.401553in}{1.597592in}}%
\pgfpathlineto{\pgfqpoint{0.391728in}{1.585483in}}%
\pgfpathlineto{\pgfqpoint{0.389908in}{1.583981in}}%
\pgfpathlineto{\pgfqpoint{0.379935in}{1.570370in}}%
\pgfpathlineto{\pgfqpoint{0.377090in}{1.556759in}}%
\pgfpathlineto{\pgfqpoint{0.381358in}{1.543148in}}%
\pgfpathlineto{\pgfqpoint{0.391728in}{1.530738in}}%
\pgfpathlineto{\pgfqpoint{0.392392in}{1.529536in}}%
\pgfpathlineto{\pgfqpoint{0.404297in}{1.515925in}}%
\pgfpathlineto{\pgfqpoint{0.407385in}{1.513227in}}%
\pgfpathlineto{\pgfqpoint{0.417908in}{1.502314in}}%
\pgfpathlineto{\pgfqpoint{0.423041in}{1.497763in}}%
\pgfpathlineto{\pgfqpoint{0.432931in}{1.488703in}}%
\pgfpathlineto{\pgfqpoint{0.438698in}{1.483703in}}%
\pgfpathlineto{\pgfqpoint{0.449409in}{1.475092in}}%
\pgfpathlineto{\pgfqpoint{0.454354in}{1.470883in}}%
\pgfpathlineto{\pgfqpoint{0.467549in}{1.461481in}}%
\pgfpathlineto{\pgfqpoint{0.470011in}{1.459325in}}%
\pgfpathlineto{\pgfqpoint{0.485668in}{1.449107in}}%
\pgfpathlineto{\pgfqpoint{0.488855in}{1.447870in}}%
\pgfpathlineto{\pgfqpoint{0.501324in}{1.440017in}}%
\pgfpathclose%
\pgfpathmoveto{\pgfqpoint{0.480427in}{1.475092in}}%
\pgfpathlineto{\pgfqpoint{0.470011in}{1.480294in}}%
\pgfpathlineto{\pgfqpoint{0.457730in}{1.488703in}}%
\pgfpathlineto{\pgfqpoint{0.454354in}{1.491108in}}%
\pgfpathlineto{\pgfqpoint{0.441184in}{1.502314in}}%
\pgfpathlineto{\pgfqpoint{0.438698in}{1.504781in}}%
\pgfpathlineto{\pgfqpoint{0.428146in}{1.515925in}}%
\pgfpathlineto{\pgfqpoint{0.423041in}{1.523090in}}%
\pgfpathlineto{\pgfqpoint{0.418183in}{1.529536in}}%
\pgfpathlineto{\pgfqpoint{0.411719in}{1.543148in}}%
\pgfpathlineto{\pgfqpoint{0.409300in}{1.556759in}}%
\pgfpathlineto{\pgfqpoint{0.410912in}{1.570370in}}%
\pgfpathlineto{\pgfqpoint{0.416566in}{1.583981in}}%
\pgfpathlineto{\pgfqpoint{0.423041in}{1.593191in}}%
\pgfpathlineto{\pgfqpoint{0.425958in}{1.597592in}}%
\pgfpathlineto{\pgfqpoint{0.438301in}{1.611203in}}%
\pgfpathlineto{\pgfqpoint{0.438698in}{1.611573in}}%
\pgfpathlineto{\pgfqpoint{0.453928in}{1.624814in}}%
\pgfpathlineto{\pgfqpoint{0.454354in}{1.625159in}}%
\pgfpathlineto{\pgfqpoint{0.470011in}{1.635890in}}%
\pgfpathlineto{\pgfqpoint{0.475073in}{1.638425in}}%
\pgfpathlineto{\pgfqpoint{0.485668in}{1.644055in}}%
\pgfpathlineto{\pgfqpoint{0.501324in}{1.648970in}}%
\pgfpathlineto{\pgfqpoint{0.516981in}{1.650371in}}%
\pgfpathlineto{\pgfqpoint{0.532637in}{1.648268in}}%
\pgfpathlineto{\pgfqpoint{0.548294in}{1.642649in}}%
\pgfpathlineto{\pgfqpoint{0.555709in}{1.638425in}}%
\pgfpathlineto{\pgfqpoint{0.563950in}{1.633987in}}%
\pgfpathlineto{\pgfqpoint{0.576769in}{1.624814in}}%
\pgfpathlineto{\pgfqpoint{0.579607in}{1.622653in}}%
\pgfpathlineto{\pgfqpoint{0.592498in}{1.611203in}}%
\pgfpathlineto{\pgfqpoint{0.595263in}{1.608268in}}%
\pgfpathlineto{\pgfqpoint{0.604937in}{1.597592in}}%
\pgfpathlineto{\pgfqpoint{0.610920in}{1.588537in}}%
\pgfpathlineto{\pgfqpoint{0.614188in}{1.583981in}}%
\pgfpathlineto{\pgfqpoint{0.620039in}{1.570370in}}%
\pgfpathlineto{\pgfqpoint{0.621708in}{1.556759in}}%
\pgfpathlineto{\pgfqpoint{0.619204in}{1.543148in}}%
\pgfpathlineto{\pgfqpoint{0.612514in}{1.529536in}}%
\pgfpathlineto{\pgfqpoint{0.610920in}{1.527463in}}%
\pgfpathlineto{\pgfqpoint{0.602723in}{1.515925in}}%
\pgfpathlineto{\pgfqpoint{0.595263in}{1.508026in}}%
\pgfpathlineto{\pgfqpoint{0.589638in}{1.502314in}}%
\pgfpathlineto{\pgfqpoint{0.579607in}{1.493593in}}%
\pgfpathlineto{\pgfqpoint{0.573037in}{1.488703in}}%
\pgfpathlineto{\pgfqpoint{0.563950in}{1.482218in}}%
\pgfpathlineto{\pgfqpoint{0.550679in}{1.475092in}}%
\pgfpathlineto{\pgfqpoint{0.548294in}{1.473706in}}%
\pgfpathlineto{\pgfqpoint{0.532637in}{1.467890in}}%
\pgfpathlineto{\pgfqpoint{0.516981in}{1.465714in}}%
\pgfpathlineto{\pgfqpoint{0.501324in}{1.467164in}}%
\pgfpathlineto{\pgfqpoint{0.485668in}{1.472251in}}%
\pgfpathlineto{\pgfqpoint{0.480427in}{1.475092in}}%
\pgfpathclose%
\pgfpathmoveto{\pgfqpoint{0.798799in}{1.447378in}}%
\pgfpathlineto{\pgfqpoint{0.814455in}{1.438886in}}%
\pgfpathlineto{\pgfqpoint{0.830112in}{1.437473in}}%
\pgfpathlineto{\pgfqpoint{0.845769in}{1.443130in}}%
\pgfpathlineto{\pgfqpoint{0.851706in}{1.447870in}}%
\pgfpathlineto{\pgfqpoint{0.861425in}{1.452684in}}%
\pgfpathlineto{\pgfqpoint{0.873136in}{1.461481in}}%
\pgfpathlineto{\pgfqpoint{0.877082in}{1.463872in}}%
\pgfpathlineto{\pgfqpoint{0.891465in}{1.475092in}}%
\pgfpathlineto{\pgfqpoint{0.892738in}{1.476025in}}%
\pgfpathlineto{\pgfqpoint{0.907822in}{1.488703in}}%
\pgfpathlineto{\pgfqpoint{0.908395in}{1.489210in}}%
\pgfpathlineto{\pgfqpoint{0.922731in}{1.502314in}}%
\pgfpathlineto{\pgfqpoint{0.924051in}{1.503729in}}%
\pgfpathlineto{\pgfqpoint{0.936493in}{1.515925in}}%
\pgfpathlineto{\pgfqpoint{0.939708in}{1.520091in}}%
\pgfpathlineto{\pgfqpoint{0.949104in}{1.529536in}}%
\pgfpathlineto{\pgfqpoint{0.955364in}{1.539546in}}%
\pgfpathlineto{\pgfqpoint{0.959323in}{1.543148in}}%
\pgfpathlineto{\pgfqpoint{0.965042in}{1.556759in}}%
\pgfpathlineto{\pgfqpoint{0.961230in}{1.570370in}}%
\pgfpathlineto{\pgfqpoint{0.955364in}{1.576428in}}%
\pgfpathlineto{\pgfqpoint{0.951218in}{1.583981in}}%
\pgfpathlineto{\pgfqpoint{0.939708in}{1.596383in}}%
\pgfpathlineto{\pgfqpoint{0.938840in}{1.597592in}}%
\pgfpathlineto{\pgfqpoint{0.925603in}{1.611203in}}%
\pgfpathlineto{\pgfqpoint{0.924051in}{1.612546in}}%
\pgfpathlineto{\pgfqpoint{0.910983in}{1.624814in}}%
\pgfpathlineto{\pgfqpoint{0.908395in}{1.627067in}}%
\pgfpathlineto{\pgfqpoint{0.894887in}{1.638425in}}%
\pgfpathlineto{\pgfqpoint{0.892738in}{1.640290in}}%
\pgfpathlineto{\pgfqpoint{0.877242in}{1.652036in}}%
\pgfpathlineto{\pgfqpoint{0.877082in}{1.652180in}}%
\pgfpathlineto{\pgfqpoint{0.861425in}{1.663318in}}%
\pgfpathlineto{\pgfqpoint{0.856446in}{1.665648in}}%
\pgfpathlineto{\pgfqpoint{0.845769in}{1.673177in}}%
\pgfpathlineto{\pgfqpoint{0.830112in}{1.678126in}}%
\pgfpathlineto{\pgfqpoint{0.814455in}{1.676890in}}%
\pgfpathlineto{\pgfqpoint{0.798799in}{1.669461in}}%
\pgfpathlineto{\pgfqpoint{0.794298in}{1.665648in}}%
\pgfpathlineto{\pgfqpoint{0.783142in}{1.659335in}}%
\pgfpathlineto{\pgfqpoint{0.773934in}{1.652036in}}%
\pgfpathlineto{\pgfqpoint{0.767486in}{1.647752in}}%
\pgfpathlineto{\pgfqpoint{0.756021in}{1.638425in}}%
\pgfpathlineto{\pgfqpoint{0.751829in}{1.635135in}}%
\pgfpathlineto{\pgfqpoint{0.739878in}{1.624814in}}%
\pgfpathlineto{\pgfqpoint{0.736173in}{1.621365in}}%
\pgfpathlineto{\pgfqpoint{0.725120in}{1.611203in}}%
\pgfpathlineto{\pgfqpoint{0.720516in}{1.606135in}}%
\pgfpathlineto{\pgfqpoint{0.711553in}{1.597592in}}%
\pgfpathlineto{\pgfqpoint{0.704859in}{1.588934in}}%
\pgfpathlineto{\pgfqpoint{0.699452in}{1.583981in}}%
\pgfpathlineto{\pgfqpoint{0.690556in}{1.570370in}}%
\pgfpathlineto{\pgfqpoint{0.689203in}{1.563135in}}%
\pgfpathlineto{\pgfqpoint{0.686441in}{1.556759in}}%
\pgfpathlineto{\pgfqpoint{0.689203in}{1.552493in}}%
\pgfpathlineto{\pgfqpoint{0.691825in}{1.543148in}}%
\pgfpathlineto{\pgfqpoint{0.701996in}{1.529536in}}%
\pgfpathlineto{\pgfqpoint{0.704859in}{1.527090in}}%
\pgfpathlineto{\pgfqpoint{0.714139in}{1.515925in}}%
\pgfpathlineto{\pgfqpoint{0.720516in}{1.510095in}}%
\pgfpathlineto{\pgfqpoint{0.727900in}{1.502314in}}%
\pgfpathlineto{\pgfqpoint{0.736173in}{1.494870in}}%
\pgfpathlineto{\pgfqpoint{0.742952in}{1.488703in}}%
\pgfpathlineto{\pgfqpoint{0.751829in}{1.481057in}}%
\pgfpathlineto{\pgfqpoint{0.759452in}{1.475092in}}%
\pgfpathlineto{\pgfqpoint{0.767486in}{1.468424in}}%
\pgfpathlineto{\pgfqpoint{0.777716in}{1.461481in}}%
\pgfpathlineto{\pgfqpoint{0.783142in}{1.456942in}}%
\pgfpathlineto{\pgfqpoint{0.798286in}{1.447870in}}%
\pgfpathlineto{\pgfqpoint{0.798799in}{1.447378in}}%
\pgfpathclose%
\pgfpathmoveto{\pgfqpoint{0.790528in}{1.475092in}}%
\pgfpathlineto{\pgfqpoint{0.783142in}{1.478495in}}%
\pgfpathlineto{\pgfqpoint{0.767533in}{1.488703in}}%
\pgfpathlineto{\pgfqpoint{0.767486in}{1.488735in}}%
\pgfpathlineto{\pgfqpoint{0.751829in}{1.501685in}}%
\pgfpathlineto{\pgfqpoint{0.751136in}{1.502314in}}%
\pgfpathlineto{\pgfqpoint{0.738203in}{1.515925in}}%
\pgfpathlineto{\pgfqpoint{0.736173in}{1.518755in}}%
\pgfpathlineto{\pgfqpoint{0.728167in}{1.529536in}}%
\pgfpathlineto{\pgfqpoint{0.721890in}{1.543148in}}%
\pgfpathlineto{\pgfqpoint{0.720516in}{1.551071in}}%
\pgfpathlineto{\pgfqpoint{0.719324in}{1.556759in}}%
\pgfpathlineto{\pgfqpoint{0.720516in}{1.565261in}}%
\pgfpathlineto{\pgfqpoint{0.721106in}{1.570370in}}%
\pgfpathlineto{\pgfqpoint{0.726597in}{1.583981in}}%
\pgfpathlineto{\pgfqpoint{0.736019in}{1.597592in}}%
\pgfpathlineto{\pgfqpoint{0.736173in}{1.597759in}}%
\pgfpathlineto{\pgfqpoint{0.748282in}{1.611203in}}%
\pgfpathlineto{\pgfqpoint{0.751829in}{1.614489in}}%
\pgfpathlineto{\pgfqpoint{0.764030in}{1.624814in}}%
\pgfpathlineto{\pgfqpoint{0.767486in}{1.627545in}}%
\pgfpathlineto{\pgfqpoint{0.783142in}{1.637668in}}%
\pgfpathlineto{\pgfqpoint{0.784782in}{1.638425in}}%
\pgfpathlineto{\pgfqpoint{0.798799in}{1.645320in}}%
\pgfpathlineto{\pgfqpoint{0.814455in}{1.649531in}}%
\pgfpathlineto{\pgfqpoint{0.830112in}{1.650231in}}%
\pgfpathlineto{\pgfqpoint{0.845769in}{1.647426in}}%
\pgfpathlineto{\pgfqpoint{0.861425in}{1.641102in}}%
\pgfpathlineto{\pgfqpoint{0.865840in}{1.638425in}}%
\pgfpathlineto{\pgfqpoint{0.877082in}{1.631961in}}%
\pgfpathlineto{\pgfqpoint{0.886706in}{1.624814in}}%
\pgfpathlineto{\pgfqpoint{0.892738in}{1.620035in}}%
\pgfpathlineto{\pgfqpoint{0.902506in}{1.611203in}}%
\pgfpathlineto{\pgfqpoint{0.908395in}{1.604835in}}%
\pgfpathlineto{\pgfqpoint{0.914966in}{1.597592in}}%
\pgfpathlineto{\pgfqpoint{0.923977in}{1.583981in}}%
\pgfpathlineto{\pgfqpoint{0.924051in}{1.583793in}}%
\pgfpathlineto{\pgfqpoint{0.930053in}{1.570370in}}%
\pgfpathlineto{\pgfqpoint{0.931790in}{1.556759in}}%
\pgfpathlineto{\pgfqpoint{0.929185in}{1.543148in}}%
\pgfpathlineto{\pgfqpoint{0.924051in}{1.533035in}}%
\pgfpathlineto{\pgfqpoint{0.922475in}{1.529536in}}%
\pgfpathlineto{\pgfqpoint{0.912717in}{1.515925in}}%
\pgfpathlineto{\pgfqpoint{0.908395in}{1.511354in}}%
\pgfpathlineto{\pgfqpoint{0.899661in}{1.502314in}}%
\pgfpathlineto{\pgfqpoint{0.892738in}{1.496188in}}%
\pgfpathlineto{\pgfqpoint{0.883070in}{1.488703in}}%
\pgfpathlineto{\pgfqpoint{0.877082in}{1.484268in}}%
\pgfpathlineto{\pgfqpoint{0.861425in}{1.475282in}}%
\pgfpathlineto{\pgfqpoint{0.860926in}{1.475092in}}%
\pgfpathlineto{\pgfqpoint{0.845769in}{1.468762in}}%
\pgfpathlineto{\pgfqpoint{0.830112in}{1.465859in}}%
\pgfpathlineto{\pgfqpoint{0.814455in}{1.466584in}}%
\pgfpathlineto{\pgfqpoint{0.798799in}{1.470942in}}%
\pgfpathlineto{\pgfqpoint{0.790528in}{1.475092in}}%
\pgfpathclose%
\pgfpathmoveto{\pgfqpoint{1.111930in}{1.445112in}}%
\pgfpathlineto{\pgfqpoint{1.127587in}{1.438039in}}%
\pgfpathlineto{\pgfqpoint{1.143243in}{1.438039in}}%
\pgfpathlineto{\pgfqpoint{1.158900in}{1.445112in}}%
\pgfpathlineto{\pgfqpoint{1.162034in}{1.447870in}}%
\pgfpathlineto{\pgfqpoint{1.174556in}{1.454728in}}%
\pgfpathlineto{\pgfqpoint{1.183052in}{1.461481in}}%
\pgfpathlineto{\pgfqpoint{1.190213in}{1.466087in}}%
\pgfpathlineto{\pgfqpoint{1.201383in}{1.475092in}}%
\pgfpathlineto{\pgfqpoint{1.205870in}{1.478495in}}%
\pgfpathlineto{\pgfqpoint{1.217844in}{1.488703in}}%
\pgfpathlineto{\pgfqpoint{1.221526in}{1.492016in}}%
\pgfpathlineto{\pgfqpoint{1.232856in}{1.502314in}}%
\pgfpathlineto{\pgfqpoint{1.237183in}{1.506922in}}%
\pgfpathlineto{\pgfqpoint{1.246660in}{1.515925in}}%
\pgfpathlineto{\pgfqpoint{1.252839in}{1.523667in}}%
\pgfpathlineto{\pgfqpoint{1.259131in}{1.529536in}}%
\pgfpathlineto{\pgfqpoint{1.268342in}{1.543148in}}%
\pgfpathlineto{\pgfqpoint{1.268496in}{1.543748in}}%
\pgfpathlineto{\pgfqpoint{1.275130in}{1.556759in}}%
\pgfpathlineto{\pgfqpoint{1.270502in}{1.570370in}}%
\pgfpathlineto{\pgfqpoint{1.268496in}{1.572090in}}%
\pgfpathlineto{\pgfqpoint{1.261435in}{1.583981in}}%
\pgfpathlineto{\pgfqpoint{1.252839in}{1.592577in}}%
\pgfpathlineto{\pgfqpoint{1.249117in}{1.597592in}}%
\pgfpathlineto{\pgfqpoint{1.237183in}{1.609407in}}%
\pgfpathlineto{\pgfqpoint{1.235569in}{1.611203in}}%
\pgfpathlineto{\pgfqpoint{1.221526in}{1.624244in}}%
\pgfpathlineto{\pgfqpoint{1.220907in}{1.624814in}}%
\pgfpathlineto{\pgfqpoint{1.205870in}{1.637668in}}%
\pgfpathlineto{\pgfqpoint{1.204874in}{1.638425in}}%
\pgfpathlineto{\pgfqpoint{1.190213in}{1.650011in}}%
\pgfpathlineto{\pgfqpoint{1.186996in}{1.652036in}}%
\pgfpathlineto{\pgfqpoint{1.174556in}{1.661406in}}%
\pgfpathlineto{\pgfqpoint{1.166357in}{1.665648in}}%
\pgfpathlineto{\pgfqpoint{1.158900in}{1.671444in}}%
\pgfpathlineto{\pgfqpoint{1.143243in}{1.677632in}}%
\pgfpathlineto{\pgfqpoint{1.127587in}{1.677632in}}%
\pgfpathlineto{\pgfqpoint{1.111930in}{1.671444in}}%
\pgfpathlineto{\pgfqpoint{1.104473in}{1.665648in}}%
\pgfpathlineto{\pgfqpoint{1.096274in}{1.661406in}}%
\pgfpathlineto{\pgfqpoint{1.083834in}{1.652036in}}%
\pgfpathlineto{\pgfqpoint{1.080617in}{1.650011in}}%
\pgfpathlineto{\pgfqpoint{1.065956in}{1.638425in}}%
\pgfpathlineto{\pgfqpoint{1.064960in}{1.637668in}}%
\pgfpathlineto{\pgfqpoint{1.049923in}{1.624814in}}%
\pgfpathlineto{\pgfqpoint{1.049304in}{1.624244in}}%
\pgfpathlineto{\pgfqpoint{1.035261in}{1.611203in}}%
\pgfpathlineto{\pgfqpoint{1.033647in}{1.609407in}}%
\pgfpathlineto{\pgfqpoint{1.021713in}{1.597592in}}%
\pgfpathlineto{\pgfqpoint{1.017991in}{1.592577in}}%
\pgfpathlineto{\pgfqpoint{1.009395in}{1.583981in}}%
\pgfpathlineto{\pgfqpoint{1.002334in}{1.572090in}}%
\pgfpathlineto{\pgfqpoint{1.000328in}{1.570370in}}%
\pgfpathlineto{\pgfqpoint{0.995700in}{1.556759in}}%
\pgfpathlineto{\pgfqpoint{1.002334in}{1.543748in}}%
\pgfpathlineto{\pgfqpoint{1.002488in}{1.543148in}}%
\pgfpathlineto{\pgfqpoint{1.011699in}{1.529536in}}%
\pgfpathlineto{\pgfqpoint{1.017991in}{1.523667in}}%
\pgfpathlineto{\pgfqpoint{1.024170in}{1.515925in}}%
\pgfpathlineto{\pgfqpoint{1.033647in}{1.506922in}}%
\pgfpathlineto{\pgfqpoint{1.037974in}{1.502314in}}%
\pgfpathlineto{\pgfqpoint{1.049304in}{1.492016in}}%
\pgfpathlineto{\pgfqpoint{1.052986in}{1.488703in}}%
\pgfpathlineto{\pgfqpoint{1.064960in}{1.478495in}}%
\pgfpathlineto{\pgfqpoint{1.069447in}{1.475092in}}%
\pgfpathlineto{\pgfqpoint{1.080617in}{1.466087in}}%
\pgfpathlineto{\pgfqpoint{1.087778in}{1.461481in}}%
\pgfpathlineto{\pgfqpoint{1.096274in}{1.454728in}}%
\pgfpathlineto{\pgfqpoint{1.108796in}{1.447870in}}%
\pgfpathlineto{\pgfqpoint{1.111930in}{1.445112in}}%
\pgfpathclose%
\pgfpathmoveto{\pgfqpoint{1.100386in}{1.475092in}}%
\pgfpathlineto{\pgfqpoint{1.096274in}{1.476825in}}%
\pgfpathlineto{\pgfqpoint{1.080617in}{1.486441in}}%
\pgfpathlineto{\pgfqpoint{1.077662in}{1.488703in}}%
\pgfpathlineto{\pgfqpoint{1.064960in}{1.498887in}}%
\pgfpathlineto{\pgfqpoint{1.061142in}{1.502314in}}%
\pgfpathlineto{\pgfqpoint{1.049304in}{1.514754in}}%
\pgfpathlineto{\pgfqpoint{1.048190in}{1.515925in}}%
\pgfpathlineto{\pgfqpoint{1.038234in}{1.529536in}}%
\pgfpathlineto{\pgfqpoint{1.033647in}{1.539656in}}%
\pgfpathlineto{\pgfqpoint{1.031818in}{1.543148in}}%
\pgfpathlineto{\pgfqpoint{1.029092in}{1.556759in}}%
\pgfpathlineto{\pgfqpoint{1.030909in}{1.570370in}}%
\pgfpathlineto{\pgfqpoint{1.033647in}{1.576304in}}%
\pgfpathlineto{\pgfqpoint{1.036702in}{1.583981in}}%
\pgfpathlineto{\pgfqpoint{1.045895in}{1.597592in}}%
\pgfpathlineto{\pgfqpoint{1.049304in}{1.601328in}}%
\pgfpathlineto{\pgfqpoint{1.058298in}{1.611203in}}%
\pgfpathlineto{\pgfqpoint{1.064960in}{1.617312in}}%
\pgfpathlineto{\pgfqpoint{1.074104in}{1.624814in}}%
\pgfpathlineto{\pgfqpoint{1.080617in}{1.629813in}}%
\pgfpathlineto{\pgfqpoint{1.094732in}{1.638425in}}%
\pgfpathlineto{\pgfqpoint{1.096274in}{1.639414in}}%
\pgfpathlineto{\pgfqpoint{1.111930in}{1.646443in}}%
\pgfpathlineto{\pgfqpoint{1.127587in}{1.649951in}}%
\pgfpathlineto{\pgfqpoint{1.143243in}{1.649951in}}%
\pgfpathlineto{\pgfqpoint{1.158900in}{1.646443in}}%
\pgfpathlineto{\pgfqpoint{1.174556in}{1.639414in}}%
\pgfpathlineto{\pgfqpoint{1.176098in}{1.638425in}}%
\pgfpathlineto{\pgfqpoint{1.190213in}{1.629813in}}%
\pgfpathlineto{\pgfqpoint{1.196726in}{1.624814in}}%
\pgfpathlineto{\pgfqpoint{1.205870in}{1.617312in}}%
\pgfpathlineto{\pgfqpoint{1.212532in}{1.611203in}}%
\pgfpathlineto{\pgfqpoint{1.221526in}{1.601328in}}%
\pgfpathlineto{\pgfqpoint{1.224935in}{1.597592in}}%
\pgfpathlineto{\pgfqpoint{1.234128in}{1.583981in}}%
\pgfpathlineto{\pgfqpoint{1.237183in}{1.576304in}}%
\pgfpathlineto{\pgfqpoint{1.239921in}{1.570370in}}%
\pgfpathlineto{\pgfqpoint{1.241738in}{1.556759in}}%
\pgfpathlineto{\pgfqpoint{1.239012in}{1.543148in}}%
\pgfpathlineto{\pgfqpoint{1.237183in}{1.539656in}}%
\pgfpathlineto{\pgfqpoint{1.232596in}{1.529536in}}%
\pgfpathlineto{\pgfqpoint{1.222640in}{1.515925in}}%
\pgfpathlineto{\pgfqpoint{1.221526in}{1.514754in}}%
\pgfpathlineto{\pgfqpoint{1.209688in}{1.502314in}}%
\pgfpathlineto{\pgfqpoint{1.205870in}{1.498887in}}%
\pgfpathlineto{\pgfqpoint{1.193168in}{1.488703in}}%
\pgfpathlineto{\pgfqpoint{1.190213in}{1.486441in}}%
\pgfpathlineto{\pgfqpoint{1.174556in}{1.476825in}}%
\pgfpathlineto{\pgfqpoint{1.170444in}{1.475092in}}%
\pgfpathlineto{\pgfqpoint{1.158900in}{1.469779in}}%
\pgfpathlineto{\pgfqpoint{1.143243in}{1.466149in}}%
\pgfpathlineto{\pgfqpoint{1.127587in}{1.466149in}}%
\pgfpathlineto{\pgfqpoint{1.111930in}{1.469779in}}%
\pgfpathlineto{\pgfqpoint{1.100386in}{1.475092in}}%
\pgfpathclose%
\pgfpathmoveto{\pgfqpoint{1.425061in}{1.443130in}}%
\pgfpathlineto{\pgfqpoint{1.440718in}{1.437473in}}%
\pgfpathlineto{\pgfqpoint{1.456375in}{1.438886in}}%
\pgfpathlineto{\pgfqpoint{1.472031in}{1.447378in}}%
\pgfpathlineto{\pgfqpoint{1.472544in}{1.447870in}}%
\pgfpathlineto{\pgfqpoint{1.487688in}{1.456942in}}%
\pgfpathlineto{\pgfqpoint{1.493114in}{1.461481in}}%
\pgfpathlineto{\pgfqpoint{1.503344in}{1.468424in}}%
\pgfpathlineto{\pgfqpoint{1.511378in}{1.475092in}}%
\pgfpathlineto{\pgfqpoint{1.519001in}{1.481057in}}%
\pgfpathlineto{\pgfqpoint{1.527878in}{1.488703in}}%
\pgfpathlineto{\pgfqpoint{1.534657in}{1.494870in}}%
\pgfpathlineto{\pgfqpoint{1.542930in}{1.502314in}}%
\pgfpathlineto{\pgfqpoint{1.550314in}{1.510095in}}%
\pgfpathlineto{\pgfqpoint{1.556691in}{1.515925in}}%
\pgfpathlineto{\pgfqpoint{1.565971in}{1.527090in}}%
\pgfpathlineto{\pgfqpoint{1.568834in}{1.529536in}}%
\pgfpathlineto{\pgfqpoint{1.579005in}{1.543148in}}%
\pgfpathlineto{\pgfqpoint{1.581627in}{1.552493in}}%
\pgfpathlineto{\pgfqpoint{1.584389in}{1.556759in}}%
\pgfpathlineto{\pgfqpoint{1.581627in}{1.563135in}}%
\pgfpathlineto{\pgfqpoint{1.580274in}{1.570370in}}%
\pgfpathlineto{\pgfqpoint{1.571378in}{1.583981in}}%
\pgfpathlineto{\pgfqpoint{1.565971in}{1.588934in}}%
\pgfpathlineto{\pgfqpoint{1.559277in}{1.597592in}}%
\pgfpathlineto{\pgfqpoint{1.550314in}{1.606135in}}%
\pgfpathlineto{\pgfqpoint{1.545710in}{1.611203in}}%
\pgfpathlineto{\pgfqpoint{1.534657in}{1.621365in}}%
\pgfpathlineto{\pgfqpoint{1.530952in}{1.624814in}}%
\pgfpathlineto{\pgfqpoint{1.519001in}{1.635135in}}%
\pgfpathlineto{\pgfqpoint{1.514809in}{1.638425in}}%
\pgfpathlineto{\pgfqpoint{1.503344in}{1.647752in}}%
\pgfpathlineto{\pgfqpoint{1.496896in}{1.652036in}}%
\pgfpathlineto{\pgfqpoint{1.487688in}{1.659335in}}%
\pgfpathlineto{\pgfqpoint{1.476532in}{1.665648in}}%
\pgfpathlineto{\pgfqpoint{1.472031in}{1.669461in}}%
\pgfpathlineto{\pgfqpoint{1.456375in}{1.676890in}}%
\pgfpathlineto{\pgfqpoint{1.440718in}{1.678126in}}%
\pgfpathlineto{\pgfqpoint{1.425061in}{1.673177in}}%
\pgfpathlineto{\pgfqpoint{1.414384in}{1.665648in}}%
\pgfpathlineto{\pgfqpoint{1.409405in}{1.663318in}}%
\pgfpathlineto{\pgfqpoint{1.393748in}{1.652180in}}%
\pgfpathlineto{\pgfqpoint{1.393588in}{1.652036in}}%
\pgfpathlineto{\pgfqpoint{1.378092in}{1.640290in}}%
\pgfpathlineto{\pgfqpoint{1.375943in}{1.638425in}}%
\pgfpathlineto{\pgfqpoint{1.362435in}{1.627067in}}%
\pgfpathlineto{\pgfqpoint{1.359847in}{1.624814in}}%
\pgfpathlineto{\pgfqpoint{1.346779in}{1.612546in}}%
\pgfpathlineto{\pgfqpoint{1.345227in}{1.611203in}}%
\pgfpathlineto{\pgfqpoint{1.331990in}{1.597592in}}%
\pgfpathlineto{\pgfqpoint{1.331122in}{1.596383in}}%
\pgfpathlineto{\pgfqpoint{1.319612in}{1.583981in}}%
\pgfpathlineto{\pgfqpoint{1.315466in}{1.576428in}}%
\pgfpathlineto{\pgfqpoint{1.309600in}{1.570370in}}%
\pgfpathlineto{\pgfqpoint{1.305788in}{1.556759in}}%
\pgfpathlineto{\pgfqpoint{1.311507in}{1.543148in}}%
\pgfpathlineto{\pgfqpoint{1.315466in}{1.539546in}}%
\pgfpathlineto{\pgfqpoint{1.321726in}{1.529536in}}%
\pgfpathlineto{\pgfqpoint{1.331122in}{1.520091in}}%
\pgfpathlineto{\pgfqpoint{1.334337in}{1.515925in}}%
\pgfpathlineto{\pgfqpoint{1.346779in}{1.503729in}}%
\pgfpathlineto{\pgfqpoint{1.348099in}{1.502314in}}%
\pgfpathlineto{\pgfqpoint{1.362435in}{1.489210in}}%
\pgfpathlineto{\pgfqpoint{1.363008in}{1.488703in}}%
\pgfpathlineto{\pgfqpoint{1.378092in}{1.476025in}}%
\pgfpathlineto{\pgfqpoint{1.379365in}{1.475092in}}%
\pgfpathlineto{\pgfqpoint{1.393748in}{1.463872in}}%
\pgfpathlineto{\pgfqpoint{1.397694in}{1.461481in}}%
\pgfpathlineto{\pgfqpoint{1.409405in}{1.452684in}}%
\pgfpathlineto{\pgfqpoint{1.419124in}{1.447870in}}%
\pgfpathlineto{\pgfqpoint{1.425061in}{1.443130in}}%
\pgfpathclose%
\pgfpathmoveto{\pgfqpoint{1.409904in}{1.475092in}}%
\pgfpathlineto{\pgfqpoint{1.409405in}{1.475282in}}%
\pgfpathlineto{\pgfqpoint{1.393748in}{1.484268in}}%
\pgfpathlineto{\pgfqpoint{1.387760in}{1.488703in}}%
\pgfpathlineto{\pgfqpoint{1.378092in}{1.496188in}}%
\pgfpathlineto{\pgfqpoint{1.371169in}{1.502314in}}%
\pgfpathlineto{\pgfqpoint{1.362435in}{1.511354in}}%
\pgfpathlineto{\pgfqpoint{1.358113in}{1.515925in}}%
\pgfpathlineto{\pgfqpoint{1.348355in}{1.529536in}}%
\pgfpathlineto{\pgfqpoint{1.346779in}{1.533035in}}%
\pgfpathlineto{\pgfqpoint{1.341645in}{1.543147in}}%
\pgfpathlineto{\pgfqpoint{1.339040in}{1.556759in}}%
\pgfpathlineto{\pgfqpoint{1.340777in}{1.570370in}}%
\pgfpathlineto{\pgfqpoint{1.346779in}{1.583793in}}%
\pgfpathlineto{\pgfqpoint{1.346853in}{1.583981in}}%
\pgfpathlineto{\pgfqpoint{1.355864in}{1.597592in}}%
\pgfpathlineto{\pgfqpoint{1.362435in}{1.604835in}}%
\pgfpathlineto{\pgfqpoint{1.368324in}{1.611203in}}%
\pgfpathlineto{\pgfqpoint{1.378092in}{1.620035in}}%
\pgfpathlineto{\pgfqpoint{1.384124in}{1.624814in}}%
\pgfpathlineto{\pgfqpoint{1.393748in}{1.631961in}}%
\pgfpathlineto{\pgfqpoint{1.404990in}{1.638425in}}%
\pgfpathlineto{\pgfqpoint{1.409405in}{1.641102in}}%
\pgfpathlineto{\pgfqpoint{1.425061in}{1.647426in}}%
\pgfpathlineto{\pgfqpoint{1.440718in}{1.650231in}}%
\pgfpathlineto{\pgfqpoint{1.456375in}{1.649531in}}%
\pgfpathlineto{\pgfqpoint{1.472031in}{1.645320in}}%
\pgfpathlineto{\pgfqpoint{1.486048in}{1.638425in}}%
\pgfpathlineto{\pgfqpoint{1.487688in}{1.637668in}}%
\pgfpathlineto{\pgfqpoint{1.503344in}{1.627545in}}%
\pgfpathlineto{\pgfqpoint{1.506800in}{1.624814in}}%
\pgfpathlineto{\pgfqpoint{1.519001in}{1.614489in}}%
\pgfpathlineto{\pgfqpoint{1.522548in}{1.611203in}}%
\pgfpathlineto{\pgfqpoint{1.534657in}{1.597759in}}%
\pgfpathlineto{\pgfqpoint{1.534811in}{1.597592in}}%
\pgfpathlineto{\pgfqpoint{1.544233in}{1.583981in}}%
\pgfpathlineto{\pgfqpoint{1.549724in}{1.570370in}}%
\pgfpathlineto{\pgfqpoint{1.550314in}{1.565261in}}%
\pgfpathlineto{\pgfqpoint{1.551506in}{1.556759in}}%
\pgfpathlineto{\pgfqpoint{1.550314in}{1.551071in}}%
\pgfpathlineto{\pgfqpoint{1.548940in}{1.543148in}}%
\pgfpathlineto{\pgfqpoint{1.542663in}{1.529536in}}%
\pgfpathlineto{\pgfqpoint{1.534657in}{1.518755in}}%
\pgfpathlineto{\pgfqpoint{1.532627in}{1.515925in}}%
\pgfpathlineto{\pgfqpoint{1.519694in}{1.502314in}}%
\pgfpathlineto{\pgfqpoint{1.519001in}{1.501685in}}%
\pgfpathlineto{\pgfqpoint{1.503344in}{1.488735in}}%
\pgfpathlineto{\pgfqpoint{1.503297in}{1.488703in}}%
\pgfpathlineto{\pgfqpoint{1.487688in}{1.478495in}}%
\pgfpathlineto{\pgfqpoint{1.480302in}{1.475092in}}%
\pgfpathlineto{\pgfqpoint{1.472031in}{1.470942in}}%
\pgfpathlineto{\pgfqpoint{1.456375in}{1.466584in}}%
\pgfpathlineto{\pgfqpoint{1.440718in}{1.465859in}}%
\pgfpathlineto{\pgfqpoint{1.425061in}{1.468762in}}%
\pgfpathlineto{\pgfqpoint{1.409904in}{1.475092in}}%
\pgfpathclose%
\pgfpathmoveto{\pgfqpoint{1.738193in}{1.441432in}}%
\pgfpathlineto{\pgfqpoint{1.753849in}{1.437191in}}%
\pgfpathlineto{\pgfqpoint{1.769506in}{1.440017in}}%
\pgfpathlineto{\pgfqpoint{1.781975in}{1.447870in}}%
\pgfpathlineto{\pgfqpoint{1.785162in}{1.449107in}}%
\pgfpathlineto{\pgfqpoint{1.800819in}{1.459325in}}%
\pgfpathlineto{\pgfqpoint{1.803281in}{1.461481in}}%
\pgfpathlineto{\pgfqpoint{1.816476in}{1.470883in}}%
\pgfpathlineto{\pgfqpoint{1.821421in}{1.475092in}}%
\pgfpathlineto{\pgfqpoint{1.832132in}{1.483703in}}%
\pgfpathlineto{\pgfqpoint{1.837899in}{1.488703in}}%
\pgfpathlineto{\pgfqpoint{1.847789in}{1.497763in}}%
\pgfpathlineto{\pgfqpoint{1.852922in}{1.502314in}}%
\pgfpathlineto{\pgfqpoint{1.863445in}{1.513227in}}%
\pgfpathlineto{\pgfqpoint{1.866533in}{1.515925in}}%
\pgfpathlineto{\pgfqpoint{1.878438in}{1.529536in}}%
\pgfpathlineto{\pgfqpoint{1.879102in}{1.530738in}}%
\pgfpathlineto{\pgfqpoint{1.889472in}{1.543148in}}%
\pgfpathlineto{\pgfqpoint{1.893740in}{1.556759in}}%
\pgfpathlineto{\pgfqpoint{1.890895in}{1.570370in}}%
\pgfpathlineto{\pgfqpoint{1.880922in}{1.583981in}}%
\pgfpathlineto{\pgfqpoint{1.879102in}{1.585483in}}%
\pgfpathlineto{\pgfqpoint{1.869277in}{1.597592in}}%
\pgfpathlineto{\pgfqpoint{1.863445in}{1.602903in}}%
\pgfpathlineto{\pgfqpoint{1.855785in}{1.611203in}}%
\pgfpathlineto{\pgfqpoint{1.847789in}{1.618446in}}%
\pgfpathlineto{\pgfqpoint{1.840996in}{1.624814in}}%
\pgfpathlineto{\pgfqpoint{1.832132in}{1.632520in}}%
\pgfpathlineto{\pgfqpoint{1.824807in}{1.638425in}}%
\pgfpathlineto{\pgfqpoint{1.816476in}{1.645377in}}%
\pgfpathlineto{\pgfqpoint{1.806928in}{1.652036in}}%
\pgfpathlineto{\pgfqpoint{1.800819in}{1.657106in}}%
\pgfpathlineto{\pgfqpoint{1.786891in}{1.665648in}}%
\pgfpathlineto{\pgfqpoint{1.785162in}{1.667229in}}%
\pgfpathlineto{\pgfqpoint{1.769506in}{1.675900in}}%
\pgfpathlineto{\pgfqpoint{1.753849in}{1.678373in}}%
\pgfpathlineto{\pgfqpoint{1.738193in}{1.674663in}}%
\pgfpathlineto{\pgfqpoint{1.723918in}{1.665647in}}%
\pgfpathlineto{\pgfqpoint{1.722536in}{1.665071in}}%
\pgfpathlineto{\pgfqpoint{1.706880in}{1.654721in}}%
\pgfpathlineto{\pgfqpoint{1.703776in}{1.652036in}}%
\pgfpathlineto{\pgfqpoint{1.691223in}{1.642888in}}%
\pgfpathlineto{\pgfqpoint{1.685988in}{1.638425in}}%
\pgfpathlineto{\pgfqpoint{1.675567in}{1.629828in}}%
\pgfpathlineto{\pgfqpoint{1.669815in}{1.624814in}}%
\pgfpathlineto{\pgfqpoint{1.659910in}{1.615502in}}%
\pgfpathlineto{\pgfqpoint{1.655068in}{1.611203in}}%
\pgfpathlineto{\pgfqpoint{1.644253in}{1.599732in}}%
\pgfpathlineto{\pgfqpoint{1.641773in}{1.597592in}}%
\pgfpathlineto{\pgfqpoint{1.630020in}{1.583981in}}%
\pgfpathlineto{\pgfqpoint{1.628597in}{1.581209in}}%
\pgfpathlineto{\pgfqpoint{1.619565in}{1.570370in}}%
\pgfpathlineto{\pgfqpoint{1.616313in}{1.556759in}}%
\pgfpathlineto{\pgfqpoint{1.621191in}{1.543148in}}%
\pgfpathlineto{\pgfqpoint{1.628597in}{1.535319in}}%
\pgfpathlineto{\pgfqpoint{1.631979in}{1.529536in}}%
\pgfpathlineto{\pgfqpoint{1.644253in}{1.516390in}}%
\pgfpathlineto{\pgfqpoint{1.644602in}{1.515925in}}%
\pgfpathlineto{\pgfqpoint{1.658031in}{1.502314in}}%
\pgfpathlineto{\pgfqpoint{1.659910in}{1.500681in}}%
\pgfpathlineto{\pgfqpoint{1.672948in}{1.488703in}}%
\pgfpathlineto{\pgfqpoint{1.675567in}{1.486426in}}%
\pgfpathlineto{\pgfqpoint{1.689345in}{1.475092in}}%
\pgfpathlineto{\pgfqpoint{1.691223in}{1.473459in}}%
\pgfpathlineto{\pgfqpoint{1.706880in}{1.461784in}}%
\pgfpathlineto{\pgfqpoint{1.707414in}{1.461481in}}%
\pgfpathlineto{\pgfqpoint{1.722536in}{1.450810in}}%
\pgfpathlineto{\pgfqpoint{1.729188in}{1.447870in}}%
\pgfpathlineto{\pgfqpoint{1.738193in}{1.441432in}}%
\pgfpathclose%
\pgfpathmoveto{\pgfqpoint{1.720151in}{1.475092in}}%
\pgfpathlineto{\pgfqpoint{1.706880in}{1.482218in}}%
\pgfpathlineto{\pgfqpoint{1.697793in}{1.488703in}}%
\pgfpathlineto{\pgfqpoint{1.691223in}{1.493593in}}%
\pgfpathlineto{\pgfqpoint{1.681192in}{1.502314in}}%
\pgfpathlineto{\pgfqpoint{1.675567in}{1.508026in}}%
\pgfpathlineto{\pgfqpoint{1.668107in}{1.515925in}}%
\pgfpathlineto{\pgfqpoint{1.659910in}{1.527463in}}%
\pgfpathlineto{\pgfqpoint{1.658316in}{1.529536in}}%
\pgfpathlineto{\pgfqpoint{1.651626in}{1.543148in}}%
\pgfpathlineto{\pgfqpoint{1.649122in}{1.556759in}}%
\pgfpathlineto{\pgfqpoint{1.650791in}{1.570370in}}%
\pgfpathlineto{\pgfqpoint{1.656642in}{1.583981in}}%
\pgfpathlineto{\pgfqpoint{1.659910in}{1.588537in}}%
\pgfpathlineto{\pgfqpoint{1.665893in}{1.597592in}}%
\pgfpathlineto{\pgfqpoint{1.675567in}{1.608268in}}%
\pgfpathlineto{\pgfqpoint{1.678332in}{1.611203in}}%
\pgfpathlineto{\pgfqpoint{1.691223in}{1.622653in}}%
\pgfpathlineto{\pgfqpoint{1.694061in}{1.624814in}}%
\pgfpathlineto{\pgfqpoint{1.706880in}{1.633987in}}%
\pgfpathlineto{\pgfqpoint{1.715121in}{1.638425in}}%
\pgfpathlineto{\pgfqpoint{1.722536in}{1.642649in}}%
\pgfpathlineto{\pgfqpoint{1.738193in}{1.648268in}}%
\pgfpathlineto{\pgfqpoint{1.753849in}{1.650371in}}%
\pgfpathlineto{\pgfqpoint{1.769506in}{1.648970in}}%
\pgfpathlineto{\pgfqpoint{1.785162in}{1.644055in}}%
\pgfpathlineto{\pgfqpoint{1.795757in}{1.638425in}}%
\pgfpathlineto{\pgfqpoint{1.800819in}{1.635890in}}%
\pgfpathlineto{\pgfqpoint{1.816476in}{1.625159in}}%
\pgfpathlineto{\pgfqpoint{1.816902in}{1.624814in}}%
\pgfpathlineto{\pgfqpoint{1.832132in}{1.611573in}}%
\pgfpathlineto{\pgfqpoint{1.832529in}{1.611203in}}%
\pgfpathlineto{\pgfqpoint{1.844872in}{1.597592in}}%
\pgfpathlineto{\pgfqpoint{1.847789in}{1.593191in}}%
\pgfpathlineto{\pgfqpoint{1.854264in}{1.583981in}}%
\pgfpathlineto{\pgfqpoint{1.859918in}{1.570370in}}%
\pgfpathlineto{\pgfqpoint{1.861530in}{1.556759in}}%
\pgfpathlineto{\pgfqpoint{1.859111in}{1.543148in}}%
\pgfpathlineto{\pgfqpoint{1.852647in}{1.529536in}}%
\pgfpathlineto{\pgfqpoint{1.847789in}{1.523090in}}%
\pgfpathlineto{\pgfqpoint{1.842684in}{1.515925in}}%
\pgfpathlineto{\pgfqpoint{1.832132in}{1.504781in}}%
\pgfpathlineto{\pgfqpoint{1.829646in}{1.502314in}}%
\pgfpathlineto{\pgfqpoint{1.816476in}{1.491108in}}%
\pgfpathlineto{\pgfqpoint{1.813100in}{1.488703in}}%
\pgfpathlineto{\pgfqpoint{1.800819in}{1.480294in}}%
\pgfpathlineto{\pgfqpoint{1.790403in}{1.475092in}}%
\pgfpathlineto{\pgfqpoint{1.785162in}{1.472251in}}%
\pgfpathlineto{\pgfqpoint{1.769506in}{1.467164in}}%
\pgfpathlineto{\pgfqpoint{1.753849in}{1.465714in}}%
\pgfpathlineto{\pgfqpoint{1.738193in}{1.467890in}}%
\pgfpathlineto{\pgfqpoint{1.722536in}{1.473706in}}%
\pgfpathlineto{\pgfqpoint{1.720151in}{1.475092in}}%
\pgfpathclose%
\pgfusepath{fill}%
\end{pgfscope}%
\begin{pgfscope}%
\pgfpathrectangle{\pgfqpoint{0.360415in}{0.345370in}}{\pgfqpoint{1.550000in}{1.347500in}}%
\pgfusepath{clip}%
\pgfsetbuttcap%
\pgfsetroundjoin%
\definecolor{currentfill}{rgb}{0.679160,0.151848,0.575189}%
\pgfsetfillcolor{currentfill}%
\pgfsetlinewidth{0.000000pt}%
\definecolor{currentstroke}{rgb}{0.000000,0.000000,0.000000}%
\pgfsetstrokecolor{currentstroke}%
\pgfsetdash{}{0pt}%
\pgfpathmoveto{\pgfqpoint{0.470011in}{0.345370in}}%
\pgfpathlineto{\pgfqpoint{0.485668in}{0.345370in}}%
\pgfpathlineto{\pgfqpoint{0.501324in}{0.345370in}}%
\pgfpathlineto{\pgfqpoint{0.516981in}{0.345370in}}%
\pgfpathlineto{\pgfqpoint{0.532637in}{0.345370in}}%
\pgfpathlineto{\pgfqpoint{0.548294in}{0.345370in}}%
\pgfpathlineto{\pgfqpoint{0.563950in}{0.345370in}}%
\pgfpathlineto{\pgfqpoint{0.568251in}{0.345370in}}%
\pgfpathlineto{\pgfqpoint{0.572036in}{0.358981in}}%
\pgfpathlineto{\pgfqpoint{0.579607in}{0.368774in}}%
\pgfpathlineto{\pgfqpoint{0.582207in}{0.372592in}}%
\pgfpathlineto{\pgfqpoint{0.595263in}{0.385780in}}%
\pgfpathlineto{\pgfqpoint{0.595665in}{0.386203in}}%
\pgfpathlineto{\pgfqpoint{0.610803in}{0.399814in}}%
\pgfpathlineto{\pgfqpoint{0.610920in}{0.399916in}}%
\pgfpathlineto{\pgfqpoint{0.626577in}{0.412836in}}%
\pgfpathlineto{\pgfqpoint{0.627441in}{0.413425in}}%
\pgfpathlineto{\pgfqpoint{0.642233in}{0.424122in}}%
\pgfpathlineto{\pgfqpoint{0.648506in}{0.427036in}}%
\pgfpathlineto{\pgfqpoint{0.657890in}{0.432073in}}%
\pgfpathlineto{\pgfqpoint{0.673546in}{0.434124in}}%
\pgfpathlineto{\pgfqpoint{0.689203in}{0.429421in}}%
\pgfpathlineto{\pgfqpoint{0.692709in}{0.427036in}}%
\pgfpathlineto{\pgfqpoint{0.704859in}{0.419948in}}%
\pgfpathlineto{\pgfqpoint{0.712977in}{0.413425in}}%
\pgfpathlineto{\pgfqpoint{0.720516in}{0.407751in}}%
\pgfpathlineto{\pgfqpoint{0.729787in}{0.399814in}}%
\pgfpathlineto{\pgfqpoint{0.736173in}{0.394180in}}%
\pgfpathlineto{\pgfqpoint{0.745128in}{0.386203in}}%
\pgfpathlineto{\pgfqpoint{0.751829in}{0.379208in}}%
\pgfpathlineto{\pgfqpoint{0.758714in}{0.372592in}}%
\pgfpathlineto{\pgfqpoint{0.767486in}{0.360299in}}%
\pgfpathlineto{\pgfqpoint{0.768603in}{0.358981in}}%
\pgfpathlineto{\pgfqpoint{0.772653in}{0.345370in}}%
\pgfpathlineto{\pgfqpoint{0.783142in}{0.345370in}}%
\pgfpathlineto{\pgfqpoint{0.798799in}{0.345370in}}%
\pgfpathlineto{\pgfqpoint{0.814455in}{0.345370in}}%
\pgfpathlineto{\pgfqpoint{0.830112in}{0.345370in}}%
\pgfpathlineto{\pgfqpoint{0.845769in}{0.345370in}}%
\pgfpathlineto{\pgfqpoint{0.861425in}{0.345370in}}%
\pgfpathlineto{\pgfqpoint{0.877082in}{0.345370in}}%
\pgfpathlineto{\pgfqpoint{0.878408in}{0.345370in}}%
\pgfpathlineto{\pgfqpoint{0.882095in}{0.358981in}}%
\pgfpathlineto{\pgfqpoint{0.892232in}{0.372592in}}%
\pgfpathlineto{\pgfqpoint{0.892738in}{0.373047in}}%
\pgfpathlineto{\pgfqpoint{0.905652in}{0.386203in}}%
\pgfpathlineto{\pgfqpoint{0.908395in}{0.388583in}}%
\pgfpathlineto{\pgfqpoint{0.920926in}{0.399814in}}%
\pgfpathlineto{\pgfqpoint{0.924051in}{0.402535in}}%
\pgfpathlineto{\pgfqpoint{0.937548in}{0.413425in}}%
\pgfpathlineto{\pgfqpoint{0.939708in}{0.415295in}}%
\pgfpathlineto{\pgfqpoint{0.955364in}{0.425985in}}%
\pgfpathlineto{\pgfqpoint{0.957995in}{0.427036in}}%
\pgfpathlineto{\pgfqpoint{0.971021in}{0.433023in}}%
\pgfpathlineto{\pgfqpoint{0.986678in}{0.433710in}}%
\pgfpathlineto{\pgfqpoint{1.002334in}{0.427748in}}%
\pgfpathlineto{\pgfqpoint{1.003288in}{0.427036in}}%
\pgfpathlineto{\pgfqpoint{1.017991in}{0.417672in}}%
\pgfpathlineto{\pgfqpoint{1.023066in}{0.413425in}}%
\pgfpathlineto{\pgfqpoint{1.033647in}{0.405152in}}%
\pgfpathlineto{\pgfqpoint{1.039815in}{0.399814in}}%
\pgfpathlineto{\pgfqpoint{1.049304in}{0.391357in}}%
\pgfpathlineto{\pgfqpoint{1.055155in}{0.386203in}}%
\pgfpathlineto{\pgfqpoint{1.064960in}{0.376072in}}%
\pgfpathlineto{\pgfqpoint{1.068696in}{0.372592in}}%
\pgfpathlineto{\pgfqpoint{1.078616in}{0.358981in}}%
\pgfpathlineto{\pgfqpoint{1.080617in}{0.351384in}}%
\pgfpathlineto{\pgfqpoint{1.082498in}{0.345370in}}%
\pgfpathlineto{\pgfqpoint{1.096274in}{0.345370in}}%
\pgfpathlineto{\pgfqpoint{1.111930in}{0.345370in}}%
\pgfpathlineto{\pgfqpoint{1.127587in}{0.345370in}}%
\pgfpathlineto{\pgfqpoint{1.143243in}{0.345370in}}%
\pgfpathlineto{\pgfqpoint{1.158900in}{0.345370in}}%
\pgfpathlineto{\pgfqpoint{1.174556in}{0.345370in}}%
\pgfpathlineto{\pgfqpoint{1.188332in}{0.345370in}}%
\pgfpathlineto{\pgfqpoint{1.190213in}{0.351384in}}%
\pgfpathlineto{\pgfqpoint{1.192214in}{0.358981in}}%
\pgfpathlineto{\pgfqpoint{1.202134in}{0.372592in}}%
\pgfpathlineto{\pgfqpoint{1.205870in}{0.376072in}}%
\pgfpathlineto{\pgfqpoint{1.215675in}{0.386203in}}%
\pgfpathlineto{\pgfqpoint{1.221526in}{0.391357in}}%
\pgfpathlineto{\pgfqpoint{1.231015in}{0.399814in}}%
\pgfpathlineto{\pgfqpoint{1.237183in}{0.405152in}}%
\pgfpathlineto{\pgfqpoint{1.247764in}{0.413425in}}%
\pgfpathlineto{\pgfqpoint{1.252839in}{0.417672in}}%
\pgfpathlineto{\pgfqpoint{1.267542in}{0.427036in}}%
\pgfpathlineto{\pgfqpoint{1.268496in}{0.427748in}}%
\pgfpathlineto{\pgfqpoint{1.284152in}{0.433710in}}%
\pgfpathlineto{\pgfqpoint{1.299809in}{0.433023in}}%
\pgfpathlineto{\pgfqpoint{1.312835in}{0.427036in}}%
\pgfpathlineto{\pgfqpoint{1.315466in}{0.425985in}}%
\pgfpathlineto{\pgfqpoint{1.331122in}{0.415295in}}%
\pgfpathlineto{\pgfqpoint{1.333282in}{0.413425in}}%
\pgfpathlineto{\pgfqpoint{1.346779in}{0.402535in}}%
\pgfpathlineto{\pgfqpoint{1.349904in}{0.399814in}}%
\pgfpathlineto{\pgfqpoint{1.362435in}{0.388583in}}%
\pgfpathlineto{\pgfqpoint{1.365178in}{0.386203in}}%
\pgfpathlineto{\pgfqpoint{1.378092in}{0.373047in}}%
\pgfpathlineto{\pgfqpoint{1.378598in}{0.372592in}}%
\pgfpathlineto{\pgfqpoint{1.388735in}{0.358981in}}%
\pgfpathlineto{\pgfqpoint{1.392422in}{0.345370in}}%
\pgfpathlineto{\pgfqpoint{1.393748in}{0.345370in}}%
\pgfpathlineto{\pgfqpoint{1.409405in}{0.345370in}}%
\pgfpathlineto{\pgfqpoint{1.425061in}{0.345370in}}%
\pgfpathlineto{\pgfqpoint{1.440718in}{0.345370in}}%
\pgfpathlineto{\pgfqpoint{1.456375in}{0.345370in}}%
\pgfpathlineto{\pgfqpoint{1.472031in}{0.345370in}}%
\pgfpathlineto{\pgfqpoint{1.487688in}{0.345370in}}%
\pgfpathlineto{\pgfqpoint{1.498177in}{0.345370in}}%
\pgfpathlineto{\pgfqpoint{1.502227in}{0.358981in}}%
\pgfpathlineto{\pgfqpoint{1.503344in}{0.360299in}}%
\pgfpathlineto{\pgfqpoint{1.512116in}{0.372592in}}%
\pgfpathlineto{\pgfqpoint{1.519001in}{0.379208in}}%
\pgfpathlineto{\pgfqpoint{1.525702in}{0.386203in}}%
\pgfpathlineto{\pgfqpoint{1.534657in}{0.394180in}}%
\pgfpathlineto{\pgfqpoint{1.541043in}{0.399814in}}%
\pgfpathlineto{\pgfqpoint{1.550314in}{0.407751in}}%
\pgfpathlineto{\pgfqpoint{1.557853in}{0.413425in}}%
\pgfpathlineto{\pgfqpoint{1.565971in}{0.419948in}}%
\pgfpathlineto{\pgfqpoint{1.578121in}{0.427036in}}%
\pgfpathlineto{\pgfqpoint{1.581627in}{0.429421in}}%
\pgfpathlineto{\pgfqpoint{1.597284in}{0.434124in}}%
\pgfpathlineto{\pgfqpoint{1.612940in}{0.432073in}}%
\pgfpathlineto{\pgfqpoint{1.622324in}{0.427036in}}%
\pgfpathlineto{\pgfqpoint{1.628597in}{0.424122in}}%
\pgfpathlineto{\pgfqpoint{1.643389in}{0.413425in}}%
\pgfpathlineto{\pgfqpoint{1.644253in}{0.412836in}}%
\pgfpathlineto{\pgfqpoint{1.659910in}{0.399916in}}%
\pgfpathlineto{\pgfqpoint{1.660027in}{0.399814in}}%
\pgfpathlineto{\pgfqpoint{1.675165in}{0.386203in}}%
\pgfpathlineto{\pgfqpoint{1.675567in}{0.385780in}}%
\pgfpathlineto{\pgfqpoint{1.688623in}{0.372592in}}%
\pgfpathlineto{\pgfqpoint{1.691223in}{0.368774in}}%
\pgfpathlineto{\pgfqpoint{1.698794in}{0.358981in}}%
\pgfpathlineto{\pgfqpoint{1.702579in}{0.345370in}}%
\pgfpathlineto{\pgfqpoint{1.706880in}{0.345370in}}%
\pgfpathlineto{\pgfqpoint{1.722536in}{0.345370in}}%
\pgfpathlineto{\pgfqpoint{1.738193in}{0.345370in}}%
\pgfpathlineto{\pgfqpoint{1.753849in}{0.345370in}}%
\pgfpathlineto{\pgfqpoint{1.769506in}{0.345370in}}%
\pgfpathlineto{\pgfqpoint{1.785162in}{0.345370in}}%
\pgfpathlineto{\pgfqpoint{1.800819in}{0.345370in}}%
\pgfpathlineto{\pgfqpoint{1.808163in}{0.345370in}}%
\pgfpathlineto{\pgfqpoint{1.812068in}{0.358981in}}%
\pgfpathlineto{\pgfqpoint{1.816476in}{0.364438in}}%
\pgfpathlineto{\pgfqpoint{1.822149in}{0.372592in}}%
\pgfpathlineto{\pgfqpoint{1.832132in}{0.382447in}}%
\pgfpathlineto{\pgfqpoint{1.835707in}{0.386203in}}%
\pgfpathlineto{\pgfqpoint{1.847789in}{0.397041in}}%
\pgfpathlineto{\pgfqpoint{1.850978in}{0.399814in}}%
\pgfpathlineto{\pgfqpoint{1.863445in}{0.410318in}}%
\pgfpathlineto{\pgfqpoint{1.867766in}{0.413425in}}%
\pgfpathlineto{\pgfqpoint{1.879102in}{0.422104in}}%
\pgfpathlineto{\pgfqpoint{1.888482in}{0.427036in}}%
\pgfpathlineto{\pgfqpoint{1.894758in}{0.430868in}}%
\pgfpathlineto{\pgfqpoint{1.910415in}{0.434263in}}%
\pgfpathlineto{\pgfqpoint{1.910415in}{0.440648in}}%
\pgfpathlineto{\pgfqpoint{1.910415in}{0.454259in}}%
\pgfpathlineto{\pgfqpoint{1.910415in}{0.467870in}}%
\pgfpathlineto{\pgfqpoint{1.910415in}{0.481481in}}%
\pgfpathlineto{\pgfqpoint{1.910415in}{0.495092in}}%
\pgfpathlineto{\pgfqpoint{1.910415in}{0.508703in}}%
\pgfpathlineto{\pgfqpoint{1.910415in}{0.522314in}}%
\pgfpathlineto{\pgfqpoint{1.910415in}{0.526053in}}%
\pgfpathlineto{\pgfqpoint{1.894758in}{0.529344in}}%
\pgfpathlineto{\pgfqpoint{1.883494in}{0.535925in}}%
\pgfpathlineto{\pgfqpoint{1.879102in}{0.538185in}}%
\pgfpathlineto{\pgfqpoint{1.863931in}{0.549536in}}%
\pgfpathlineto{\pgfqpoint{1.863445in}{0.549885in}}%
\pgfpathlineto{\pgfqpoint{1.847789in}{0.563046in}}%
\pgfpathlineto{\pgfqpoint{1.847671in}{0.563148in}}%
\pgfpathlineto{\pgfqpoint{1.832810in}{0.576759in}}%
\pgfpathlineto{\pgfqpoint{1.832132in}{0.577510in}}%
\pgfpathlineto{\pgfqpoint{1.819828in}{0.590370in}}%
\pgfpathlineto{\pgfqpoint{1.816476in}{0.595823in}}%
\pgfpathlineto{\pgfqpoint{1.810682in}{0.603981in}}%
\pgfpathlineto{\pgfqpoint{1.808322in}{0.617592in}}%
\pgfpathlineto{\pgfqpoint{1.813732in}{0.631203in}}%
\pgfpathlineto{\pgfqpoint{1.816476in}{0.634251in}}%
\pgfpathlineto{\pgfqpoint{1.824629in}{0.644814in}}%
\pgfpathlineto{\pgfqpoint{1.832132in}{0.651871in}}%
\pgfpathlineto{\pgfqpoint{1.838659in}{0.658425in}}%
\pgfpathlineto{\pgfqpoint{1.847789in}{0.666485in}}%
\pgfpathlineto{\pgfqpoint{1.854269in}{0.672036in}}%
\pgfpathlineto{\pgfqpoint{1.863445in}{0.679822in}}%
\pgfpathlineto{\pgfqpoint{1.871492in}{0.685648in}}%
\pgfpathlineto{\pgfqpoint{1.879102in}{0.691633in}}%
\pgfpathlineto{\pgfqpoint{1.893243in}{0.699259in}}%
\pgfpathlineto{\pgfqpoint{1.894758in}{0.700230in}}%
\pgfpathlineto{\pgfqpoint{1.910415in}{0.703751in}}%
\pgfpathlineto{\pgfqpoint{1.910415in}{0.712870in}}%
\pgfpathlineto{\pgfqpoint{1.910415in}{0.726481in}}%
\pgfpathlineto{\pgfqpoint{1.910415in}{0.740092in}}%
\pgfpathlineto{\pgfqpoint{1.910415in}{0.753703in}}%
\pgfpathlineto{\pgfqpoint{1.910415in}{0.767314in}}%
\pgfpathlineto{\pgfqpoint{1.910415in}{0.780925in}}%
\pgfpathlineto{\pgfqpoint{1.910415in}{0.794536in}}%
\pgfpathlineto{\pgfqpoint{1.910415in}{0.795689in}}%
\pgfpathlineto{\pgfqpoint{1.894758in}{0.798895in}}%
\pgfpathlineto{\pgfqpoint{1.879102in}{0.807707in}}%
\pgfpathlineto{\pgfqpoint{1.878578in}{0.808148in}}%
\pgfpathlineto{\pgfqpoint{1.863445in}{0.819374in}}%
\pgfpathlineto{\pgfqpoint{1.860708in}{0.821759in}}%
\pgfpathlineto{\pgfqpoint{1.847789in}{0.832652in}}%
\pgfpathlineto{\pgfqpoint{1.844659in}{0.835370in}}%
\pgfpathlineto{\pgfqpoint{1.832132in}{0.847103in}}%
\pgfpathlineto{\pgfqpoint{1.829982in}{0.848981in}}%
\pgfpathlineto{\pgfqpoint{1.817685in}{0.862592in}}%
\pgfpathlineto{\pgfqpoint{1.816476in}{0.864879in}}%
\pgfpathlineto{\pgfqpoint{1.809589in}{0.876203in}}%
\pgfpathlineto{\pgfqpoint{1.808799in}{0.889814in}}%
\pgfpathlineto{\pgfqpoint{1.815657in}{0.903425in}}%
\pgfpathlineto{\pgfqpoint{1.816476in}{0.904254in}}%
\pgfpathlineto{\pgfqpoint{1.827247in}{0.917036in}}%
\pgfpathlineto{\pgfqpoint{1.832132in}{0.921448in}}%
\pgfpathlineto{\pgfqpoint{1.841649in}{0.930648in}}%
\pgfpathlineto{\pgfqpoint{1.847789in}{0.936010in}}%
\pgfpathlineto{\pgfqpoint{1.857517in}{0.944259in}}%
\pgfpathlineto{\pgfqpoint{1.863445in}{0.949345in}}%
\pgfpathlineto{\pgfqpoint{1.875099in}{0.957870in}}%
\pgfpathlineto{\pgfqpoint{1.879102in}{0.961118in}}%
\pgfpathlineto{\pgfqpoint{1.894758in}{0.969741in}}%
\pgfpathlineto{\pgfqpoint{1.903497in}{0.971481in}}%
\pgfpathlineto{\pgfqpoint{1.910415in}{0.973116in}}%
\pgfpathlineto{\pgfqpoint{1.910415in}{0.985092in}}%
\pgfpathlineto{\pgfqpoint{1.910415in}{0.998703in}}%
\pgfpathlineto{\pgfqpoint{1.910415in}{1.012314in}}%
\pgfpathlineto{\pgfqpoint{1.910415in}{1.025925in}}%
\pgfpathlineto{\pgfqpoint{1.910415in}{1.039536in}}%
\pgfpathlineto{\pgfqpoint{1.910415in}{1.053148in}}%
\pgfpathlineto{\pgfqpoint{1.910415in}{1.065123in}}%
\pgfpathlineto{\pgfqpoint{1.903497in}{1.066759in}}%
\pgfpathlineto{\pgfqpoint{1.894758in}{1.068499in}}%
\pgfpathlineto{\pgfqpoint{1.879102in}{1.077122in}}%
\pgfpathlineto{\pgfqpoint{1.875099in}{1.080370in}}%
\pgfpathlineto{\pgfqpoint{1.863445in}{1.088894in}}%
\pgfpathlineto{\pgfqpoint{1.857517in}{1.093981in}}%
\pgfpathlineto{\pgfqpoint{1.847789in}{1.102230in}}%
\pgfpathlineto{\pgfqpoint{1.841649in}{1.107592in}}%
\pgfpathlineto{\pgfqpoint{1.832132in}{1.116791in}}%
\pgfpathlineto{\pgfqpoint{1.827247in}{1.121203in}}%
\pgfpathlineto{\pgfqpoint{1.816476in}{1.133985in}}%
\pgfpathlineto{\pgfqpoint{1.815657in}{1.134814in}}%
\pgfpathlineto{\pgfqpoint{1.808799in}{1.148425in}}%
\pgfpathlineto{\pgfqpoint{1.809589in}{1.162036in}}%
\pgfpathlineto{\pgfqpoint{1.816476in}{1.173360in}}%
\pgfpathlineto{\pgfqpoint{1.817685in}{1.175647in}}%
\pgfpathlineto{\pgfqpoint{1.829982in}{1.189259in}}%
\pgfpathlineto{\pgfqpoint{1.832132in}{1.191137in}}%
\pgfpathlineto{\pgfqpoint{1.844659in}{1.202870in}}%
\pgfpathlineto{\pgfqpoint{1.847789in}{1.205587in}}%
\pgfpathlineto{\pgfqpoint{1.860708in}{1.216481in}}%
\pgfpathlineto{\pgfqpoint{1.863445in}{1.218865in}}%
\pgfpathlineto{\pgfqpoint{1.878578in}{1.230092in}}%
\pgfpathlineto{\pgfqpoint{1.879102in}{1.230532in}}%
\pgfpathlineto{\pgfqpoint{1.894758in}{1.239344in}}%
\pgfpathlineto{\pgfqpoint{1.910415in}{1.242550in}}%
\pgfpathlineto{\pgfqpoint{1.910415in}{1.243703in}}%
\pgfpathlineto{\pgfqpoint{1.910415in}{1.257314in}}%
\pgfpathlineto{\pgfqpoint{1.910415in}{1.270925in}}%
\pgfpathlineto{\pgfqpoint{1.910415in}{1.284536in}}%
\pgfpathlineto{\pgfqpoint{1.910415in}{1.298147in}}%
\pgfpathlineto{\pgfqpoint{1.910415in}{1.311759in}}%
\pgfpathlineto{\pgfqpoint{1.910415in}{1.325370in}}%
\pgfpathlineto{\pgfqpoint{1.910415in}{1.334488in}}%
\pgfpathlineto{\pgfqpoint{1.894758in}{1.338009in}}%
\pgfpathlineto{\pgfqpoint{1.893243in}{1.338981in}}%
\pgfpathlineto{\pgfqpoint{1.879102in}{1.346606in}}%
\pgfpathlineto{\pgfqpoint{1.871492in}{1.352592in}}%
\pgfpathlineto{\pgfqpoint{1.863445in}{1.358417in}}%
\pgfpathlineto{\pgfqpoint{1.854269in}{1.366203in}}%
\pgfpathlineto{\pgfqpoint{1.847789in}{1.371754in}}%
\pgfpathlineto{\pgfqpoint{1.838659in}{1.379814in}}%
\pgfpathlineto{\pgfqpoint{1.832132in}{1.386368in}}%
\pgfpathlineto{\pgfqpoint{1.824629in}{1.393425in}}%
\pgfpathlineto{\pgfqpoint{1.816476in}{1.403988in}}%
\pgfpathlineto{\pgfqpoint{1.813732in}{1.407036in}}%
\pgfpathlineto{\pgfqpoint{1.808322in}{1.420648in}}%
\pgfpathlineto{\pgfqpoint{1.810682in}{1.434259in}}%
\pgfpathlineto{\pgfqpoint{1.816476in}{1.442416in}}%
\pgfpathlineto{\pgfqpoint{1.819828in}{1.447870in}}%
\pgfpathlineto{\pgfqpoint{1.832132in}{1.460729in}}%
\pgfpathlineto{\pgfqpoint{1.832810in}{1.461481in}}%
\pgfpathlineto{\pgfqpoint{1.847671in}{1.475092in}}%
\pgfpathlineto{\pgfqpoint{1.847789in}{1.475194in}}%
\pgfpathlineto{\pgfqpoint{1.863445in}{1.488354in}}%
\pgfpathlineto{\pgfqpoint{1.863931in}{1.488703in}}%
\pgfpathlineto{\pgfqpoint{1.879102in}{1.500054in}}%
\pgfpathlineto{\pgfqpoint{1.883494in}{1.502314in}}%
\pgfpathlineto{\pgfqpoint{1.894758in}{1.508896in}}%
\pgfpathlineto{\pgfqpoint{1.910415in}{1.512187in}}%
\pgfpathlineto{\pgfqpoint{1.910415in}{1.515925in}}%
\pgfpathlineto{\pgfqpoint{1.910415in}{1.529536in}}%
\pgfpathlineto{\pgfqpoint{1.910415in}{1.543148in}}%
\pgfpathlineto{\pgfqpoint{1.910415in}{1.556759in}}%
\pgfpathlineto{\pgfqpoint{1.910415in}{1.570370in}}%
\pgfpathlineto{\pgfqpoint{1.910415in}{1.583981in}}%
\pgfpathlineto{\pgfqpoint{1.910415in}{1.597592in}}%
\pgfpathlineto{\pgfqpoint{1.910415in}{1.603976in}}%
\pgfpathlineto{\pgfqpoint{1.894758in}{1.607372in}}%
\pgfpathlineto{\pgfqpoint{1.888482in}{1.611203in}}%
\pgfpathlineto{\pgfqpoint{1.879102in}{1.616135in}}%
\pgfpathlineto{\pgfqpoint{1.867766in}{1.624814in}}%
\pgfpathlineto{\pgfqpoint{1.863445in}{1.627922in}}%
\pgfpathlineto{\pgfqpoint{1.850978in}{1.638425in}}%
\pgfpathlineto{\pgfqpoint{1.847789in}{1.641198in}}%
\pgfpathlineto{\pgfqpoint{1.835707in}{1.652036in}}%
\pgfpathlineto{\pgfqpoint{1.832132in}{1.655793in}}%
\pgfpathlineto{\pgfqpoint{1.822149in}{1.665648in}}%
\pgfpathlineto{\pgfqpoint{1.816476in}{1.673802in}}%
\pgfpathlineto{\pgfqpoint{1.812068in}{1.679259in}}%
\pgfpathlineto{\pgfqpoint{1.808163in}{1.692870in}}%
\pgfpathlineto{\pgfqpoint{1.800819in}{1.692870in}}%
\pgfpathlineto{\pgfqpoint{1.785162in}{1.692870in}}%
\pgfpathlineto{\pgfqpoint{1.769506in}{1.692870in}}%
\pgfpathlineto{\pgfqpoint{1.753849in}{1.692870in}}%
\pgfpathlineto{\pgfqpoint{1.738193in}{1.692870in}}%
\pgfpathlineto{\pgfqpoint{1.722536in}{1.692870in}}%
\pgfpathlineto{\pgfqpoint{1.706880in}{1.692870in}}%
\pgfpathlineto{\pgfqpoint{1.702579in}{1.692870in}}%
\pgfpathlineto{\pgfqpoint{1.698794in}{1.679259in}}%
\pgfpathlineto{\pgfqpoint{1.691223in}{1.669465in}}%
\pgfpathlineto{\pgfqpoint{1.688623in}{1.665648in}}%
\pgfpathlineto{\pgfqpoint{1.675567in}{1.652459in}}%
\pgfpathlineto{\pgfqpoint{1.675165in}{1.652036in}}%
\pgfpathlineto{\pgfqpoint{1.660027in}{1.638425in}}%
\pgfpathlineto{\pgfqpoint{1.659910in}{1.638323in}}%
\pgfpathlineto{\pgfqpoint{1.644253in}{1.625403in}}%
\pgfpathlineto{\pgfqpoint{1.643389in}{1.624814in}}%
\pgfpathlineto{\pgfqpoint{1.628597in}{1.614117in}}%
\pgfpathlineto{\pgfqpoint{1.622324in}{1.611203in}}%
\pgfpathlineto{\pgfqpoint{1.612940in}{1.606167in}}%
\pgfpathlineto{\pgfqpoint{1.597284in}{1.604115in}}%
\pgfpathlineto{\pgfqpoint{1.581627in}{1.608818in}}%
\pgfpathlineto{\pgfqpoint{1.578121in}{1.611203in}}%
\pgfpathlineto{\pgfqpoint{1.565971in}{1.618291in}}%
\pgfpathlineto{\pgfqpoint{1.557853in}{1.624814in}}%
\pgfpathlineto{\pgfqpoint{1.550314in}{1.630489in}}%
\pgfpathlineto{\pgfqpoint{1.541043in}{1.638425in}}%
\pgfpathlineto{\pgfqpoint{1.534657in}{1.644059in}}%
\pgfpathlineto{\pgfqpoint{1.525702in}{1.652036in}}%
\pgfpathlineto{\pgfqpoint{1.519001in}{1.659032in}}%
\pgfpathlineto{\pgfqpoint{1.512116in}{1.665648in}}%
\pgfpathlineto{\pgfqpoint{1.503344in}{1.677941in}}%
\pgfpathlineto{\pgfqpoint{1.502227in}{1.679259in}}%
\pgfpathlineto{\pgfqpoint{1.498177in}{1.692870in}}%
\pgfpathlineto{\pgfqpoint{1.487688in}{1.692870in}}%
\pgfpathlineto{\pgfqpoint{1.472031in}{1.692870in}}%
\pgfpathlineto{\pgfqpoint{1.456375in}{1.692870in}}%
\pgfpathlineto{\pgfqpoint{1.440718in}{1.692870in}}%
\pgfpathlineto{\pgfqpoint{1.425061in}{1.692870in}}%
\pgfpathlineto{\pgfqpoint{1.409405in}{1.692870in}}%
\pgfpathlineto{\pgfqpoint{1.393748in}{1.692870in}}%
\pgfpathlineto{\pgfqpoint{1.392422in}{1.692870in}}%
\pgfpathlineto{\pgfqpoint{1.388735in}{1.679259in}}%
\pgfpathlineto{\pgfqpoint{1.378598in}{1.665648in}}%
\pgfpathlineto{\pgfqpoint{1.378092in}{1.665192in}}%
\pgfpathlineto{\pgfqpoint{1.365178in}{1.652036in}}%
\pgfpathlineto{\pgfqpoint{1.362435in}{1.649656in}}%
\pgfpathlineto{\pgfqpoint{1.349904in}{1.638425in}}%
\pgfpathlineto{\pgfqpoint{1.346779in}{1.635704in}}%
\pgfpathlineto{\pgfqpoint{1.333282in}{1.624814in}}%
\pgfpathlineto{\pgfqpoint{1.331122in}{1.622944in}}%
\pgfpathlineto{\pgfqpoint{1.315466in}{1.612255in}}%
\pgfpathlineto{\pgfqpoint{1.312835in}{1.611203in}}%
\pgfpathlineto{\pgfqpoint{1.299809in}{1.605216in}}%
\pgfpathlineto{\pgfqpoint{1.284152in}{1.604530in}}%
\pgfpathlineto{\pgfqpoint{1.268496in}{1.610492in}}%
\pgfpathlineto{\pgfqpoint{1.267542in}{1.611203in}}%
\pgfpathlineto{\pgfqpoint{1.252839in}{1.620567in}}%
\pgfpathlineto{\pgfqpoint{1.247764in}{1.624814in}}%
\pgfpathlineto{\pgfqpoint{1.237183in}{1.633088in}}%
\pgfpathlineto{\pgfqpoint{1.231015in}{1.638425in}}%
\pgfpathlineto{\pgfqpoint{1.221526in}{1.646882in}}%
\pgfpathlineto{\pgfqpoint{1.215675in}{1.652036in}}%
\pgfpathlineto{\pgfqpoint{1.205870in}{1.662167in}}%
\pgfpathlineto{\pgfqpoint{1.202134in}{1.665647in}}%
\pgfpathlineto{\pgfqpoint{1.192214in}{1.679259in}}%
\pgfpathlineto{\pgfqpoint{1.190213in}{1.686855in}}%
\pgfpathlineto{\pgfqpoint{1.188332in}{1.692870in}}%
\pgfpathlineto{\pgfqpoint{1.174556in}{1.692870in}}%
\pgfpathlineto{\pgfqpoint{1.158900in}{1.692870in}}%
\pgfpathlineto{\pgfqpoint{1.143243in}{1.692870in}}%
\pgfpathlineto{\pgfqpoint{1.127587in}{1.692870in}}%
\pgfpathlineto{\pgfqpoint{1.111930in}{1.692870in}}%
\pgfpathlineto{\pgfqpoint{1.096274in}{1.692870in}}%
\pgfpathlineto{\pgfqpoint{1.082498in}{1.692870in}}%
\pgfpathlineto{\pgfqpoint{1.080617in}{1.686855in}}%
\pgfpathlineto{\pgfqpoint{1.078616in}{1.679259in}}%
\pgfpathlineto{\pgfqpoint{1.068696in}{1.665648in}}%
\pgfpathlineto{\pgfqpoint{1.064960in}{1.662167in}}%
\pgfpathlineto{\pgfqpoint{1.055155in}{1.652036in}}%
\pgfpathlineto{\pgfqpoint{1.049304in}{1.646882in}}%
\pgfpathlineto{\pgfqpoint{1.039815in}{1.638425in}}%
\pgfpathlineto{\pgfqpoint{1.033647in}{1.633088in}}%
\pgfpathlineto{\pgfqpoint{1.023066in}{1.624814in}}%
\pgfpathlineto{\pgfqpoint{1.017991in}{1.620567in}}%
\pgfpathlineto{\pgfqpoint{1.003288in}{1.611203in}}%
\pgfpathlineto{\pgfqpoint{1.002334in}{1.610492in}}%
\pgfpathlineto{\pgfqpoint{0.986678in}{1.604530in}}%
\pgfpathlineto{\pgfqpoint{0.971021in}{1.605216in}}%
\pgfpathlineto{\pgfqpoint{0.957995in}{1.611203in}}%
\pgfpathlineto{\pgfqpoint{0.955364in}{1.612255in}}%
\pgfpathlineto{\pgfqpoint{0.939708in}{1.622944in}}%
\pgfpathlineto{\pgfqpoint{0.937548in}{1.624814in}}%
\pgfpathlineto{\pgfqpoint{0.924051in}{1.635704in}}%
\pgfpathlineto{\pgfqpoint{0.920926in}{1.638425in}}%
\pgfpathlineto{\pgfqpoint{0.908395in}{1.649656in}}%
\pgfpathlineto{\pgfqpoint{0.905652in}{1.652036in}}%
\pgfpathlineto{\pgfqpoint{0.892738in}{1.665192in}}%
\pgfpathlineto{\pgfqpoint{0.892232in}{1.665648in}}%
\pgfpathlineto{\pgfqpoint{0.882095in}{1.679259in}}%
\pgfpathlineto{\pgfqpoint{0.878408in}{1.692870in}}%
\pgfpathlineto{\pgfqpoint{0.877082in}{1.692870in}}%
\pgfpathlineto{\pgfqpoint{0.861425in}{1.692870in}}%
\pgfpathlineto{\pgfqpoint{0.845769in}{1.692870in}}%
\pgfpathlineto{\pgfqpoint{0.830112in}{1.692870in}}%
\pgfpathlineto{\pgfqpoint{0.814455in}{1.692870in}}%
\pgfpathlineto{\pgfqpoint{0.798799in}{1.692870in}}%
\pgfpathlineto{\pgfqpoint{0.783142in}{1.692870in}}%
\pgfpathlineto{\pgfqpoint{0.772653in}{1.692870in}}%
\pgfpathlineto{\pgfqpoint{0.768603in}{1.679259in}}%
\pgfpathlineto{\pgfqpoint{0.767486in}{1.677941in}}%
\pgfpathlineto{\pgfqpoint{0.758714in}{1.665648in}}%
\pgfpathlineto{\pgfqpoint{0.751829in}{1.659032in}}%
\pgfpathlineto{\pgfqpoint{0.745128in}{1.652036in}}%
\pgfpathlineto{\pgfqpoint{0.736173in}{1.644059in}}%
\pgfpathlineto{\pgfqpoint{0.729787in}{1.638425in}}%
\pgfpathlineto{\pgfqpoint{0.720516in}{1.630489in}}%
\pgfpathlineto{\pgfqpoint{0.712977in}{1.624814in}}%
\pgfpathlineto{\pgfqpoint{0.704859in}{1.618291in}}%
\pgfpathlineto{\pgfqpoint{0.692709in}{1.611203in}}%
\pgfpathlineto{\pgfqpoint{0.689203in}{1.608818in}}%
\pgfpathlineto{\pgfqpoint{0.673546in}{1.604115in}}%
\pgfpathlineto{\pgfqpoint{0.657890in}{1.606167in}}%
\pgfpathlineto{\pgfqpoint{0.648506in}{1.611203in}}%
\pgfpathlineto{\pgfqpoint{0.642233in}{1.614117in}}%
\pgfpathlineto{\pgfqpoint{0.627441in}{1.624814in}}%
\pgfpathlineto{\pgfqpoint{0.626577in}{1.625403in}}%
\pgfpathlineto{\pgfqpoint{0.610920in}{1.638323in}}%
\pgfpathlineto{\pgfqpoint{0.610803in}{1.638425in}}%
\pgfpathlineto{\pgfqpoint{0.595665in}{1.652036in}}%
\pgfpathlineto{\pgfqpoint{0.595263in}{1.652459in}}%
\pgfpathlineto{\pgfqpoint{0.582207in}{1.665648in}}%
\pgfpathlineto{\pgfqpoint{0.579607in}{1.669465in}}%
\pgfpathlineto{\pgfqpoint{0.572036in}{1.679259in}}%
\pgfpathlineto{\pgfqpoint{0.568251in}{1.692870in}}%
\pgfpathlineto{\pgfqpoint{0.563950in}{1.692870in}}%
\pgfpathlineto{\pgfqpoint{0.548294in}{1.692870in}}%
\pgfpathlineto{\pgfqpoint{0.532637in}{1.692870in}}%
\pgfpathlineto{\pgfqpoint{0.516981in}{1.692870in}}%
\pgfpathlineto{\pgfqpoint{0.501324in}{1.692870in}}%
\pgfpathlineto{\pgfqpoint{0.485668in}{1.692870in}}%
\pgfpathlineto{\pgfqpoint{0.470011in}{1.692870in}}%
\pgfpathlineto{\pgfqpoint{0.462667in}{1.692870in}}%
\pgfpathlineto{\pgfqpoint{0.458762in}{1.679259in}}%
\pgfpathlineto{\pgfqpoint{0.454354in}{1.673802in}}%
\pgfpathlineto{\pgfqpoint{0.448681in}{1.665647in}}%
\pgfpathlineto{\pgfqpoint{0.438698in}{1.655793in}}%
\pgfpathlineto{\pgfqpoint{0.435123in}{1.652036in}}%
\pgfpathlineto{\pgfqpoint{0.423041in}{1.641198in}}%
\pgfpathlineto{\pgfqpoint{0.419852in}{1.638425in}}%
\pgfpathlineto{\pgfqpoint{0.407385in}{1.627922in}}%
\pgfpathlineto{\pgfqpoint{0.403064in}{1.624814in}}%
\pgfpathlineto{\pgfqpoint{0.391728in}{1.616135in}}%
\pgfpathlineto{\pgfqpoint{0.382348in}{1.611203in}}%
\pgfpathlineto{\pgfqpoint{0.376072in}{1.607372in}}%
\pgfpathlineto{\pgfqpoint{0.360415in}{1.603976in}}%
\pgfpathlineto{\pgfqpoint{0.360415in}{1.597592in}}%
\pgfpathlineto{\pgfqpoint{0.360415in}{1.583981in}}%
\pgfpathlineto{\pgfqpoint{0.360415in}{1.570370in}}%
\pgfpathlineto{\pgfqpoint{0.360415in}{1.556759in}}%
\pgfpathlineto{\pgfqpoint{0.360415in}{1.543148in}}%
\pgfpathlineto{\pgfqpoint{0.360415in}{1.529536in}}%
\pgfpathlineto{\pgfqpoint{0.360415in}{1.515925in}}%
\pgfpathlineto{\pgfqpoint{0.360415in}{1.512187in}}%
\pgfpathlineto{\pgfqpoint{0.376072in}{1.508896in}}%
\pgfpathlineto{\pgfqpoint{0.387336in}{1.502314in}}%
\pgfpathlineto{\pgfqpoint{0.391728in}{1.500054in}}%
\pgfpathlineto{\pgfqpoint{0.406899in}{1.488703in}}%
\pgfpathlineto{\pgfqpoint{0.407385in}{1.488354in}}%
\pgfpathlineto{\pgfqpoint{0.423041in}{1.475194in}}%
\pgfpathlineto{\pgfqpoint{0.423159in}{1.475092in}}%
\pgfpathlineto{\pgfqpoint{0.438020in}{1.461481in}}%
\pgfpathlineto{\pgfqpoint{0.438698in}{1.460729in}}%
\pgfpathlineto{\pgfqpoint{0.451002in}{1.447870in}}%
\pgfpathlineto{\pgfqpoint{0.454354in}{1.442416in}}%
\pgfpathlineto{\pgfqpoint{0.460148in}{1.434259in}}%
\pgfpathlineto{\pgfqpoint{0.462508in}{1.420648in}}%
\pgfpathlineto{\pgfqpoint{0.457098in}{1.407036in}}%
\pgfpathlineto{\pgfqpoint{0.454354in}{1.403988in}}%
\pgfpathlineto{\pgfqpoint{0.446201in}{1.393425in}}%
\pgfpathlineto{\pgfqpoint{0.438698in}{1.386368in}}%
\pgfpathlineto{\pgfqpoint{0.432171in}{1.379814in}}%
\pgfpathlineto{\pgfqpoint{0.423041in}{1.371754in}}%
\pgfpathlineto{\pgfqpoint{0.416561in}{1.366203in}}%
\pgfpathlineto{\pgfqpoint{0.407385in}{1.358417in}}%
\pgfpathlineto{\pgfqpoint{0.399338in}{1.352592in}}%
\pgfpathlineto{\pgfqpoint{0.391728in}{1.346606in}}%
\pgfpathlineto{\pgfqpoint{0.377587in}{1.338981in}}%
\pgfpathlineto{\pgfqpoint{0.376072in}{1.338009in}}%
\pgfpathlineto{\pgfqpoint{0.360415in}{1.334488in}}%
\pgfpathlineto{\pgfqpoint{0.360415in}{1.325370in}}%
\pgfpathlineto{\pgfqpoint{0.360415in}{1.311759in}}%
\pgfpathlineto{\pgfqpoint{0.360415in}{1.298147in}}%
\pgfpathlineto{\pgfqpoint{0.360415in}{1.284536in}}%
\pgfpathlineto{\pgfqpoint{0.360415in}{1.270925in}}%
\pgfpathlineto{\pgfqpoint{0.360415in}{1.257314in}}%
\pgfpathlineto{\pgfqpoint{0.360415in}{1.243703in}}%
\pgfpathlineto{\pgfqpoint{0.360415in}{1.242550in}}%
\pgfpathlineto{\pgfqpoint{0.376072in}{1.239344in}}%
\pgfpathlineto{\pgfqpoint{0.391728in}{1.230532in}}%
\pgfpathlineto{\pgfqpoint{0.392252in}{1.230092in}}%
\pgfpathlineto{\pgfqpoint{0.407385in}{1.218865in}}%
\pgfpathlineto{\pgfqpoint{0.410122in}{1.216481in}}%
\pgfpathlineto{\pgfqpoint{0.423041in}{1.205587in}}%
\pgfpathlineto{\pgfqpoint{0.426171in}{1.202870in}}%
\pgfpathlineto{\pgfqpoint{0.438698in}{1.191137in}}%
\pgfpathlineto{\pgfqpoint{0.440848in}{1.189259in}}%
\pgfpathlineto{\pgfqpoint{0.453145in}{1.175647in}}%
\pgfpathlineto{\pgfqpoint{0.454354in}{1.173360in}}%
\pgfpathlineto{\pgfqpoint{0.461241in}{1.162036in}}%
\pgfpathlineto{\pgfqpoint{0.462031in}{1.148425in}}%
\pgfpathlineto{\pgfqpoint{0.455173in}{1.134814in}}%
\pgfpathlineto{\pgfqpoint{0.454354in}{1.133985in}}%
\pgfpathlineto{\pgfqpoint{0.443583in}{1.121203in}}%
\pgfpathlineto{\pgfqpoint{0.438698in}{1.116791in}}%
\pgfpathlineto{\pgfqpoint{0.429181in}{1.107592in}}%
\pgfpathlineto{\pgfqpoint{0.423041in}{1.102230in}}%
\pgfpathlineto{\pgfqpoint{0.413313in}{1.093981in}}%
\pgfpathlineto{\pgfqpoint{0.407385in}{1.088894in}}%
\pgfpathlineto{\pgfqpoint{0.395731in}{1.080370in}}%
\pgfpathlineto{\pgfqpoint{0.391728in}{1.077122in}}%
\pgfpathlineto{\pgfqpoint{0.376072in}{1.068499in}}%
\pgfpathlineto{\pgfqpoint{0.367333in}{1.066759in}}%
\pgfpathlineto{\pgfqpoint{0.360415in}{1.065123in}}%
\pgfpathlineto{\pgfqpoint{0.360415in}{1.053148in}}%
\pgfpathlineto{\pgfqpoint{0.360415in}{1.039536in}}%
\pgfpathlineto{\pgfqpoint{0.360415in}{1.025925in}}%
\pgfpathlineto{\pgfqpoint{0.360415in}{1.012314in}}%
\pgfpathlineto{\pgfqpoint{0.360415in}{0.998703in}}%
\pgfpathlineto{\pgfqpoint{0.360415in}{0.985092in}}%
\pgfpathlineto{\pgfqpoint{0.360415in}{0.973116in}}%
\pgfpathlineto{\pgfqpoint{0.367333in}{0.971481in}}%
\pgfpathlineto{\pgfqpoint{0.376072in}{0.969741in}}%
\pgfpathlineto{\pgfqpoint{0.391728in}{0.961118in}}%
\pgfpathlineto{\pgfqpoint{0.395731in}{0.957870in}}%
\pgfpathlineto{\pgfqpoint{0.407385in}{0.949345in}}%
\pgfpathlineto{\pgfqpoint{0.413313in}{0.944259in}}%
\pgfpathlineto{\pgfqpoint{0.423041in}{0.936010in}}%
\pgfpathlineto{\pgfqpoint{0.429181in}{0.930648in}}%
\pgfpathlineto{\pgfqpoint{0.438698in}{0.921448in}}%
\pgfpathlineto{\pgfqpoint{0.443583in}{0.917036in}}%
\pgfpathlineto{\pgfqpoint{0.454354in}{0.904254in}}%
\pgfpathlineto{\pgfqpoint{0.455173in}{0.903425in}}%
\pgfpathlineto{\pgfqpoint{0.462031in}{0.889814in}}%
\pgfpathlineto{\pgfqpoint{0.461241in}{0.876203in}}%
\pgfpathlineto{\pgfqpoint{0.454354in}{0.864879in}}%
\pgfpathlineto{\pgfqpoint{0.453145in}{0.862592in}}%
\pgfpathlineto{\pgfqpoint{0.440848in}{0.848981in}}%
\pgfpathlineto{\pgfqpoint{0.438698in}{0.847103in}}%
\pgfpathlineto{\pgfqpoint{0.426171in}{0.835370in}}%
\pgfpathlineto{\pgfqpoint{0.423041in}{0.832652in}}%
\pgfpathlineto{\pgfqpoint{0.410122in}{0.821759in}}%
\pgfpathlineto{\pgfqpoint{0.407385in}{0.819374in}}%
\pgfpathlineto{\pgfqpoint{0.392252in}{0.808148in}}%
\pgfpathlineto{\pgfqpoint{0.391728in}{0.807707in}}%
\pgfpathlineto{\pgfqpoint{0.376072in}{0.798895in}}%
\pgfpathlineto{\pgfqpoint{0.360415in}{0.795689in}}%
\pgfpathlineto{\pgfqpoint{0.360415in}{0.794536in}}%
\pgfpathlineto{\pgfqpoint{0.360415in}{0.780925in}}%
\pgfpathlineto{\pgfqpoint{0.360415in}{0.767314in}}%
\pgfpathlineto{\pgfqpoint{0.360415in}{0.753703in}}%
\pgfpathlineto{\pgfqpoint{0.360415in}{0.740092in}}%
\pgfpathlineto{\pgfqpoint{0.360415in}{0.726481in}}%
\pgfpathlineto{\pgfqpoint{0.360415in}{0.712870in}}%
\pgfpathlineto{\pgfqpoint{0.360415in}{0.703751in}}%
\pgfpathlineto{\pgfqpoint{0.376072in}{0.700230in}}%
\pgfpathlineto{\pgfqpoint{0.377587in}{0.699259in}}%
\pgfpathlineto{\pgfqpoint{0.391728in}{0.691633in}}%
\pgfpathlineto{\pgfqpoint{0.399338in}{0.685648in}}%
\pgfpathlineto{\pgfqpoint{0.407385in}{0.679822in}}%
\pgfpathlineto{\pgfqpoint{0.416561in}{0.672036in}}%
\pgfpathlineto{\pgfqpoint{0.423041in}{0.666485in}}%
\pgfpathlineto{\pgfqpoint{0.432171in}{0.658425in}}%
\pgfpathlineto{\pgfqpoint{0.438698in}{0.651871in}}%
\pgfpathlineto{\pgfqpoint{0.446201in}{0.644814in}}%
\pgfpathlineto{\pgfqpoint{0.454354in}{0.634251in}}%
\pgfpathlineto{\pgfqpoint{0.457098in}{0.631203in}}%
\pgfpathlineto{\pgfqpoint{0.462508in}{0.617592in}}%
\pgfpathlineto{\pgfqpoint{0.460148in}{0.603981in}}%
\pgfpathlineto{\pgfqpoint{0.454354in}{0.595823in}}%
\pgfpathlineto{\pgfqpoint{0.451002in}{0.590370in}}%
\pgfpathlineto{\pgfqpoint{0.438698in}{0.577510in}}%
\pgfpathlineto{\pgfqpoint{0.438020in}{0.576759in}}%
\pgfpathlineto{\pgfqpoint{0.423159in}{0.563148in}}%
\pgfpathlineto{\pgfqpoint{0.423041in}{0.563046in}}%
\pgfpathlineto{\pgfqpoint{0.407385in}{0.549885in}}%
\pgfpathlineto{\pgfqpoint{0.406899in}{0.549536in}}%
\pgfpathlineto{\pgfqpoint{0.391728in}{0.538185in}}%
\pgfpathlineto{\pgfqpoint{0.387336in}{0.535925in}}%
\pgfpathlineto{\pgfqpoint{0.376072in}{0.529344in}}%
\pgfpathlineto{\pgfqpoint{0.360415in}{0.526053in}}%
\pgfpathlineto{\pgfqpoint{0.360415in}{0.522314in}}%
\pgfpathlineto{\pgfqpoint{0.360415in}{0.508703in}}%
\pgfpathlineto{\pgfqpoint{0.360415in}{0.495092in}}%
\pgfpathlineto{\pgfqpoint{0.360415in}{0.481481in}}%
\pgfpathlineto{\pgfqpoint{0.360415in}{0.467870in}}%
\pgfpathlineto{\pgfqpoint{0.360415in}{0.454259in}}%
\pgfpathlineto{\pgfqpoint{0.360415in}{0.440648in}}%
\pgfpathlineto{\pgfqpoint{0.360415in}{0.434263in}}%
\pgfpathlineto{\pgfqpoint{0.376072in}{0.430868in}}%
\pgfpathlineto{\pgfqpoint{0.382348in}{0.427036in}}%
\pgfpathlineto{\pgfqpoint{0.391728in}{0.422104in}}%
\pgfpathlineto{\pgfqpoint{0.403064in}{0.413425in}}%
\pgfpathlineto{\pgfqpoint{0.407385in}{0.410318in}}%
\pgfpathlineto{\pgfqpoint{0.419852in}{0.399814in}}%
\pgfpathlineto{\pgfqpoint{0.423041in}{0.397041in}}%
\pgfpathlineto{\pgfqpoint{0.435123in}{0.386203in}}%
\pgfpathlineto{\pgfqpoint{0.438698in}{0.382447in}}%
\pgfpathlineto{\pgfqpoint{0.448681in}{0.372592in}}%
\pgfpathlineto{\pgfqpoint{0.454354in}{0.364438in}}%
\pgfpathlineto{\pgfqpoint{0.458762in}{0.358981in}}%
\pgfpathlineto{\pgfqpoint{0.462667in}{0.345370in}}%
\pgfpathlineto{\pgfqpoint{0.470011in}{0.345370in}}%
\pgfpathclose%
\pgfpathmoveto{\pgfqpoint{0.483939in}{0.372592in}}%
\pgfpathlineto{\pgfqpoint{0.470011in}{0.381133in}}%
\pgfpathlineto{\pgfqpoint{0.463902in}{0.386203in}}%
\pgfpathlineto{\pgfqpoint{0.454354in}{0.392863in}}%
\pgfpathlineto{\pgfqpoint{0.446023in}{0.399814in}}%
\pgfpathlineto{\pgfqpoint{0.438698in}{0.405720in}}%
\pgfpathlineto{\pgfqpoint{0.429834in}{0.413425in}}%
\pgfpathlineto{\pgfqpoint{0.423041in}{0.419793in}}%
\pgfpathlineto{\pgfqpoint{0.415045in}{0.427036in}}%
\pgfpathlineto{\pgfqpoint{0.407385in}{0.435337in}}%
\pgfpathlineto{\pgfqpoint{0.401553in}{0.440648in}}%
\pgfpathlineto{\pgfqpoint{0.391728in}{0.452756in}}%
\pgfpathlineto{\pgfqpoint{0.389908in}{0.454259in}}%
\pgfpathlineto{\pgfqpoint{0.379935in}{0.467870in}}%
\pgfpathlineto{\pgfqpoint{0.377090in}{0.481481in}}%
\pgfpathlineto{\pgfqpoint{0.381358in}{0.495092in}}%
\pgfpathlineto{\pgfqpoint{0.391728in}{0.507502in}}%
\pgfpathlineto{\pgfqpoint{0.392392in}{0.508703in}}%
\pgfpathlineto{\pgfqpoint{0.404297in}{0.522314in}}%
\pgfpathlineto{\pgfqpoint{0.407385in}{0.525012in}}%
\pgfpathlineto{\pgfqpoint{0.417908in}{0.535925in}}%
\pgfpathlineto{\pgfqpoint{0.423041in}{0.540476in}}%
\pgfpathlineto{\pgfqpoint{0.432931in}{0.549536in}}%
\pgfpathlineto{\pgfqpoint{0.438698in}{0.554537in}}%
\pgfpathlineto{\pgfqpoint{0.449409in}{0.563148in}}%
\pgfpathlineto{\pgfqpoint{0.454354in}{0.567357in}}%
\pgfpathlineto{\pgfqpoint{0.467549in}{0.576759in}}%
\pgfpathlineto{\pgfqpoint{0.470011in}{0.578915in}}%
\pgfpathlineto{\pgfqpoint{0.485668in}{0.589133in}}%
\pgfpathlineto{\pgfqpoint{0.488855in}{0.590370in}}%
\pgfpathlineto{\pgfqpoint{0.501324in}{0.598222in}}%
\pgfpathlineto{\pgfqpoint{0.516981in}{0.601049in}}%
\pgfpathlineto{\pgfqpoint{0.532637in}{0.596808in}}%
\pgfpathlineto{\pgfqpoint{0.541642in}{0.590370in}}%
\pgfpathlineto{\pgfqpoint{0.548294in}{0.587429in}}%
\pgfpathlineto{\pgfqpoint{0.563416in}{0.576759in}}%
\pgfpathlineto{\pgfqpoint{0.563950in}{0.576455in}}%
\pgfpathlineto{\pgfqpoint{0.579607in}{0.564781in}}%
\pgfpathlineto{\pgfqpoint{0.581485in}{0.563148in}}%
\pgfpathlineto{\pgfqpoint{0.595263in}{0.551813in}}%
\pgfpathlineto{\pgfqpoint{0.597882in}{0.549536in}}%
\pgfpathlineto{\pgfqpoint{0.610920in}{0.537558in}}%
\pgfpathlineto{\pgfqpoint{0.612799in}{0.535925in}}%
\pgfpathlineto{\pgfqpoint{0.626228in}{0.522314in}}%
\pgfpathlineto{\pgfqpoint{0.626577in}{0.521850in}}%
\pgfpathlineto{\pgfqpoint{0.638851in}{0.508703in}}%
\pgfpathlineto{\pgfqpoint{0.642233in}{0.502921in}}%
\pgfpathlineto{\pgfqpoint{0.649639in}{0.495092in}}%
\pgfpathlineto{\pgfqpoint{0.654517in}{0.481481in}}%
\pgfpathlineto{\pgfqpoint{0.651265in}{0.467870in}}%
\pgfpathlineto{\pgfqpoint{0.642233in}{0.457030in}}%
\pgfpathlineto{\pgfqpoint{0.640810in}{0.454259in}}%
\pgfpathlineto{\pgfqpoint{0.629057in}{0.440648in}}%
\pgfpathlineto{\pgfqpoint{0.626577in}{0.438507in}}%
\pgfpathlineto{\pgfqpoint{0.615762in}{0.427036in}}%
\pgfpathlineto{\pgfqpoint{0.610920in}{0.422737in}}%
\pgfpathlineto{\pgfqpoint{0.601015in}{0.413425in}}%
\pgfpathlineto{\pgfqpoint{0.595263in}{0.408412in}}%
\pgfpathlineto{\pgfqpoint{0.584842in}{0.399814in}}%
\pgfpathlineto{\pgfqpoint{0.579607in}{0.395351in}}%
\pgfpathlineto{\pgfqpoint{0.567054in}{0.386203in}}%
\pgfpathlineto{\pgfqpoint{0.563950in}{0.383519in}}%
\pgfpathlineto{\pgfqpoint{0.548294in}{0.373169in}}%
\pgfpathlineto{\pgfqpoint{0.546912in}{0.372592in}}%
\pgfpathlineto{\pgfqpoint{0.532637in}{0.363576in}}%
\pgfpathlineto{\pgfqpoint{0.516981in}{0.359866in}}%
\pgfpathlineto{\pgfqpoint{0.501324in}{0.362339in}}%
\pgfpathlineto{\pgfqpoint{0.485668in}{0.371010in}}%
\pgfpathlineto{\pgfqpoint{0.483939in}{0.372592in}}%
\pgfpathclose%
\pgfpathmoveto{\pgfqpoint{0.794298in}{0.372592in}}%
\pgfpathlineto{\pgfqpoint{0.783142in}{0.378904in}}%
\pgfpathlineto{\pgfqpoint{0.773934in}{0.386203in}}%
\pgfpathlineto{\pgfqpoint{0.767486in}{0.390487in}}%
\pgfpathlineto{\pgfqpoint{0.756021in}{0.399814in}}%
\pgfpathlineto{\pgfqpoint{0.751829in}{0.403104in}}%
\pgfpathlineto{\pgfqpoint{0.739878in}{0.413425in}}%
\pgfpathlineto{\pgfqpoint{0.736173in}{0.416875in}}%
\pgfpathlineto{\pgfqpoint{0.725120in}{0.427036in}}%
\pgfpathlineto{\pgfqpoint{0.720516in}{0.432105in}}%
\pgfpathlineto{\pgfqpoint{0.711553in}{0.440648in}}%
\pgfpathlineto{\pgfqpoint{0.704859in}{0.449305in}}%
\pgfpathlineto{\pgfqpoint{0.699452in}{0.454259in}}%
\pgfpathlineto{\pgfqpoint{0.690556in}{0.467870in}}%
\pgfpathlineto{\pgfqpoint{0.689203in}{0.475105in}}%
\pgfpathlineto{\pgfqpoint{0.686441in}{0.481481in}}%
\pgfpathlineto{\pgfqpoint{0.689203in}{0.485746in}}%
\pgfpathlineto{\pgfqpoint{0.691825in}{0.495092in}}%
\pgfpathlineto{\pgfqpoint{0.701996in}{0.508703in}}%
\pgfpathlineto{\pgfqpoint{0.704859in}{0.511150in}}%
\pgfpathlineto{\pgfqpoint{0.714139in}{0.522314in}}%
\pgfpathlineto{\pgfqpoint{0.720516in}{0.528145in}}%
\pgfpathlineto{\pgfqpoint{0.727900in}{0.535925in}}%
\pgfpathlineto{\pgfqpoint{0.736173in}{0.543369in}}%
\pgfpathlineto{\pgfqpoint{0.742952in}{0.549536in}}%
\pgfpathlineto{\pgfqpoint{0.751829in}{0.557183in}}%
\pgfpathlineto{\pgfqpoint{0.759452in}{0.563148in}}%
\pgfpathlineto{\pgfqpoint{0.767486in}{0.569815in}}%
\pgfpathlineto{\pgfqpoint{0.777716in}{0.576759in}}%
\pgfpathlineto{\pgfqpoint{0.783142in}{0.581297in}}%
\pgfpathlineto{\pgfqpoint{0.798286in}{0.590370in}}%
\pgfpathlineto{\pgfqpoint{0.798799in}{0.590861in}}%
\pgfpathlineto{\pgfqpoint{0.814455in}{0.599353in}}%
\pgfpathlineto{\pgfqpoint{0.830112in}{0.600766in}}%
\pgfpathlineto{\pgfqpoint{0.845769in}{0.595110in}}%
\pgfpathlineto{\pgfqpoint{0.851706in}{0.590370in}}%
\pgfpathlineto{\pgfqpoint{0.861425in}{0.585556in}}%
\pgfpathlineto{\pgfqpoint{0.873136in}{0.576759in}}%
\pgfpathlineto{\pgfqpoint{0.877082in}{0.574367in}}%
\pgfpathlineto{\pgfqpoint{0.891465in}{0.563148in}}%
\pgfpathlineto{\pgfqpoint{0.892738in}{0.562215in}}%
\pgfpathlineto{\pgfqpoint{0.907822in}{0.549536in}}%
\pgfpathlineto{\pgfqpoint{0.908395in}{0.549029in}}%
\pgfpathlineto{\pgfqpoint{0.922731in}{0.535925in}}%
\pgfpathlineto{\pgfqpoint{0.924051in}{0.534510in}}%
\pgfpathlineto{\pgfqpoint{0.936493in}{0.522314in}}%
\pgfpathlineto{\pgfqpoint{0.939708in}{0.518149in}}%
\pgfpathlineto{\pgfqpoint{0.949104in}{0.508703in}}%
\pgfpathlineto{\pgfqpoint{0.955364in}{0.498693in}}%
\pgfpathlineto{\pgfqpoint{0.959323in}{0.495092in}}%
\pgfpathlineto{\pgfqpoint{0.965042in}{0.481481in}}%
\pgfpathlineto{\pgfqpoint{0.961230in}{0.467870in}}%
\pgfpathlineto{\pgfqpoint{0.955364in}{0.461811in}}%
\pgfpathlineto{\pgfqpoint{0.951218in}{0.454259in}}%
\pgfpathlineto{\pgfqpoint{0.939708in}{0.441857in}}%
\pgfpathlineto{\pgfqpoint{0.938840in}{0.440648in}}%
\pgfpathlineto{\pgfqpoint{0.925603in}{0.427036in}}%
\pgfpathlineto{\pgfqpoint{0.924051in}{0.425693in}}%
\pgfpathlineto{\pgfqpoint{0.910983in}{0.413425in}}%
\pgfpathlineto{\pgfqpoint{0.908395in}{0.411173in}}%
\pgfpathlineto{\pgfqpoint{0.894887in}{0.399814in}}%
\pgfpathlineto{\pgfqpoint{0.892738in}{0.397950in}}%
\pgfpathlineto{\pgfqpoint{0.877242in}{0.386203in}}%
\pgfpathlineto{\pgfqpoint{0.877082in}{0.386059in}}%
\pgfpathlineto{\pgfqpoint{0.861425in}{0.374922in}}%
\pgfpathlineto{\pgfqpoint{0.856446in}{0.372592in}}%
\pgfpathlineto{\pgfqpoint{0.845769in}{0.365062in}}%
\pgfpathlineto{\pgfqpoint{0.830112in}{0.360113in}}%
\pgfpathlineto{\pgfqpoint{0.814455in}{0.361350in}}%
\pgfpathlineto{\pgfqpoint{0.798799in}{0.368779in}}%
\pgfpathlineto{\pgfqpoint{0.794298in}{0.372592in}}%
\pgfpathclose%
\pgfpathmoveto{\pgfqpoint{1.104473in}{0.372592in}}%
\pgfpathlineto{\pgfqpoint{1.096274in}{0.376834in}}%
\pgfpathlineto{\pgfqpoint{1.083834in}{0.386203in}}%
\pgfpathlineto{\pgfqpoint{1.080617in}{0.388228in}}%
\pgfpathlineto{\pgfqpoint{1.065956in}{0.399814in}}%
\pgfpathlineto{\pgfqpoint{1.064960in}{0.400572in}}%
\pgfpathlineto{\pgfqpoint{1.049923in}{0.413425in}}%
\pgfpathlineto{\pgfqpoint{1.049304in}{0.413995in}}%
\pgfpathlineto{\pgfqpoint{1.035261in}{0.427036in}}%
\pgfpathlineto{\pgfqpoint{1.033647in}{0.428832in}}%
\pgfpathlineto{\pgfqpoint{1.021713in}{0.440648in}}%
\pgfpathlineto{\pgfqpoint{1.017991in}{0.445662in}}%
\pgfpathlineto{\pgfqpoint{1.009395in}{0.454259in}}%
\pgfpathlineto{\pgfqpoint{1.002334in}{0.466149in}}%
\pgfpathlineto{\pgfqpoint{1.000328in}{0.467870in}}%
\pgfpathlineto{\pgfqpoint{0.995700in}{0.481481in}}%
\pgfpathlineto{\pgfqpoint{1.002334in}{0.494492in}}%
\pgfpathlineto{\pgfqpoint{1.002488in}{0.495092in}}%
\pgfpathlineto{\pgfqpoint{1.011699in}{0.508703in}}%
\pgfpathlineto{\pgfqpoint{1.017991in}{0.514573in}}%
\pgfpathlineto{\pgfqpoint{1.024170in}{0.522314in}}%
\pgfpathlineto{\pgfqpoint{1.033647in}{0.531317in}}%
\pgfpathlineto{\pgfqpoint{1.037974in}{0.535925in}}%
\pgfpathlineto{\pgfqpoint{1.049304in}{0.546224in}}%
\pgfpathlineto{\pgfqpoint{1.052986in}{0.549536in}}%
\pgfpathlineto{\pgfqpoint{1.064960in}{0.559744in}}%
\pgfpathlineto{\pgfqpoint{1.069447in}{0.563148in}}%
\pgfpathlineto{\pgfqpoint{1.080617in}{0.572153in}}%
\pgfpathlineto{\pgfqpoint{1.087778in}{0.576759in}}%
\pgfpathlineto{\pgfqpoint{1.096274in}{0.583511in}}%
\pgfpathlineto{\pgfqpoint{1.108796in}{0.590370in}}%
\pgfpathlineto{\pgfqpoint{1.111930in}{0.593128in}}%
\pgfpathlineto{\pgfqpoint{1.127587in}{0.600201in}}%
\pgfpathlineto{\pgfqpoint{1.143243in}{0.600201in}}%
\pgfpathlineto{\pgfqpoint{1.158900in}{0.593128in}}%
\pgfpathlineto{\pgfqpoint{1.162034in}{0.590370in}}%
\pgfpathlineto{\pgfqpoint{1.174556in}{0.583511in}}%
\pgfpathlineto{\pgfqpoint{1.183052in}{0.576759in}}%
\pgfpathlineto{\pgfqpoint{1.190213in}{0.572153in}}%
\pgfpathlineto{\pgfqpoint{1.201383in}{0.563148in}}%
\pgfpathlineto{\pgfqpoint{1.205870in}{0.559744in}}%
\pgfpathlineto{\pgfqpoint{1.217844in}{0.549536in}}%
\pgfpathlineto{\pgfqpoint{1.221526in}{0.546224in}}%
\pgfpathlineto{\pgfqpoint{1.232856in}{0.535925in}}%
\pgfpathlineto{\pgfqpoint{1.237183in}{0.531317in}}%
\pgfpathlineto{\pgfqpoint{1.246660in}{0.522314in}}%
\pgfpathlineto{\pgfqpoint{1.252839in}{0.514573in}}%
\pgfpathlineto{\pgfqpoint{1.259131in}{0.508703in}}%
\pgfpathlineto{\pgfqpoint{1.268342in}{0.495092in}}%
\pgfpathlineto{\pgfqpoint{1.268496in}{0.494492in}}%
\pgfpathlineto{\pgfqpoint{1.275130in}{0.481481in}}%
\pgfpathlineto{\pgfqpoint{1.270502in}{0.467870in}}%
\pgfpathlineto{\pgfqpoint{1.268496in}{0.466149in}}%
\pgfpathlineto{\pgfqpoint{1.261435in}{0.454259in}}%
\pgfpathlineto{\pgfqpoint{1.252839in}{0.445662in}}%
\pgfpathlineto{\pgfqpoint{1.249117in}{0.440648in}}%
\pgfpathlineto{\pgfqpoint{1.237183in}{0.428832in}}%
\pgfpathlineto{\pgfqpoint{1.235569in}{0.427036in}}%
\pgfpathlineto{\pgfqpoint{1.221526in}{0.413995in}}%
\pgfpathlineto{\pgfqpoint{1.220907in}{0.413425in}}%
\pgfpathlineto{\pgfqpoint{1.205870in}{0.400572in}}%
\pgfpathlineto{\pgfqpoint{1.204874in}{0.399814in}}%
\pgfpathlineto{\pgfqpoint{1.190213in}{0.388228in}}%
\pgfpathlineto{\pgfqpoint{1.186996in}{0.386203in}}%
\pgfpathlineto{\pgfqpoint{1.174556in}{0.376834in}}%
\pgfpathlineto{\pgfqpoint{1.166357in}{0.372592in}}%
\pgfpathlineto{\pgfqpoint{1.158900in}{0.366796in}}%
\pgfpathlineto{\pgfqpoint{1.143243in}{0.360608in}}%
\pgfpathlineto{\pgfqpoint{1.127587in}{0.360608in}}%
\pgfpathlineto{\pgfqpoint{1.111930in}{0.366796in}}%
\pgfpathlineto{\pgfqpoint{1.104473in}{0.372592in}}%
\pgfpathclose%
\pgfpathmoveto{\pgfqpoint{1.414384in}{0.372592in}}%
\pgfpathlineto{\pgfqpoint{1.409405in}{0.374922in}}%
\pgfpathlineto{\pgfqpoint{1.393748in}{0.386059in}}%
\pgfpathlineto{\pgfqpoint{1.393588in}{0.386203in}}%
\pgfpathlineto{\pgfqpoint{1.378092in}{0.397950in}}%
\pgfpathlineto{\pgfqpoint{1.375943in}{0.399814in}}%
\pgfpathlineto{\pgfqpoint{1.362435in}{0.411173in}}%
\pgfpathlineto{\pgfqpoint{1.359847in}{0.413425in}}%
\pgfpathlineto{\pgfqpoint{1.346779in}{0.425693in}}%
\pgfpathlineto{\pgfqpoint{1.345227in}{0.427036in}}%
\pgfpathlineto{\pgfqpoint{1.331990in}{0.440648in}}%
\pgfpathlineto{\pgfqpoint{1.331122in}{0.441857in}}%
\pgfpathlineto{\pgfqpoint{1.319612in}{0.454259in}}%
\pgfpathlineto{\pgfqpoint{1.315466in}{0.461811in}}%
\pgfpathlineto{\pgfqpoint{1.309600in}{0.467870in}}%
\pgfpathlineto{\pgfqpoint{1.305788in}{0.481481in}}%
\pgfpathlineto{\pgfqpoint{1.311507in}{0.495092in}}%
\pgfpathlineto{\pgfqpoint{1.315466in}{0.498693in}}%
\pgfpathlineto{\pgfqpoint{1.321726in}{0.508703in}}%
\pgfpathlineto{\pgfqpoint{1.331122in}{0.518149in}}%
\pgfpathlineto{\pgfqpoint{1.334337in}{0.522314in}}%
\pgfpathlineto{\pgfqpoint{1.346779in}{0.534510in}}%
\pgfpathlineto{\pgfqpoint{1.348099in}{0.535925in}}%
\pgfpathlineto{\pgfqpoint{1.362435in}{0.549029in}}%
\pgfpathlineto{\pgfqpoint{1.363008in}{0.549536in}}%
\pgfpathlineto{\pgfqpoint{1.378092in}{0.562215in}}%
\pgfpathlineto{\pgfqpoint{1.379365in}{0.563148in}}%
\pgfpathlineto{\pgfqpoint{1.393748in}{0.574367in}}%
\pgfpathlineto{\pgfqpoint{1.397694in}{0.576759in}}%
\pgfpathlineto{\pgfqpoint{1.409405in}{0.585556in}}%
\pgfpathlineto{\pgfqpoint{1.419124in}{0.590370in}}%
\pgfpathlineto{\pgfqpoint{1.425061in}{0.595110in}}%
\pgfpathlineto{\pgfqpoint{1.440718in}{0.600766in}}%
\pgfpathlineto{\pgfqpoint{1.456375in}{0.599353in}}%
\pgfpathlineto{\pgfqpoint{1.472031in}{0.590861in}}%
\pgfpathlineto{\pgfqpoint{1.472544in}{0.590370in}}%
\pgfpathlineto{\pgfqpoint{1.487688in}{0.581297in}}%
\pgfpathlineto{\pgfqpoint{1.493114in}{0.576759in}}%
\pgfpathlineto{\pgfqpoint{1.503344in}{0.569815in}}%
\pgfpathlineto{\pgfqpoint{1.511378in}{0.563148in}}%
\pgfpathlineto{\pgfqpoint{1.519001in}{0.557183in}}%
\pgfpathlineto{\pgfqpoint{1.527878in}{0.549536in}}%
\pgfpathlineto{\pgfqpoint{1.534657in}{0.543369in}}%
\pgfpathlineto{\pgfqpoint{1.542930in}{0.535925in}}%
\pgfpathlineto{\pgfqpoint{1.550314in}{0.528145in}}%
\pgfpathlineto{\pgfqpoint{1.556691in}{0.522314in}}%
\pgfpathlineto{\pgfqpoint{1.565971in}{0.511150in}}%
\pgfpathlineto{\pgfqpoint{1.568834in}{0.508703in}}%
\pgfpathlineto{\pgfqpoint{1.579005in}{0.495092in}}%
\pgfpathlineto{\pgfqpoint{1.581627in}{0.485746in}}%
\pgfpathlineto{\pgfqpoint{1.584389in}{0.481481in}}%
\pgfpathlineto{\pgfqpoint{1.581627in}{0.475105in}}%
\pgfpathlineto{\pgfqpoint{1.580274in}{0.467870in}}%
\pgfpathlineto{\pgfqpoint{1.571378in}{0.454259in}}%
\pgfpathlineto{\pgfqpoint{1.565971in}{0.449305in}}%
\pgfpathlineto{\pgfqpoint{1.559277in}{0.440648in}}%
\pgfpathlineto{\pgfqpoint{1.550314in}{0.432105in}}%
\pgfpathlineto{\pgfqpoint{1.545710in}{0.427036in}}%
\pgfpathlineto{\pgfqpoint{1.534657in}{0.416875in}}%
\pgfpathlineto{\pgfqpoint{1.530952in}{0.413425in}}%
\pgfpathlineto{\pgfqpoint{1.519001in}{0.403104in}}%
\pgfpathlineto{\pgfqpoint{1.514809in}{0.399814in}}%
\pgfpathlineto{\pgfqpoint{1.503344in}{0.390487in}}%
\pgfpathlineto{\pgfqpoint{1.496896in}{0.386203in}}%
\pgfpathlineto{\pgfqpoint{1.487688in}{0.378904in}}%
\pgfpathlineto{\pgfqpoint{1.476532in}{0.372592in}}%
\pgfpathlineto{\pgfqpoint{1.472031in}{0.368779in}}%
\pgfpathlineto{\pgfqpoint{1.456375in}{0.361350in}}%
\pgfpathlineto{\pgfqpoint{1.440718in}{0.360113in}}%
\pgfpathlineto{\pgfqpoint{1.425061in}{0.365062in}}%
\pgfpathlineto{\pgfqpoint{1.414384in}{0.372592in}}%
\pgfpathclose%
\pgfpathmoveto{\pgfqpoint{1.723918in}{0.372592in}}%
\pgfpathlineto{\pgfqpoint{1.722536in}{0.373169in}}%
\pgfpathlineto{\pgfqpoint{1.706880in}{0.383519in}}%
\pgfpathlineto{\pgfqpoint{1.703776in}{0.386203in}}%
\pgfpathlineto{\pgfqpoint{1.691223in}{0.395351in}}%
\pgfpathlineto{\pgfqpoint{1.685988in}{0.399814in}}%
\pgfpathlineto{\pgfqpoint{1.675567in}{0.408412in}}%
\pgfpathlineto{\pgfqpoint{1.669815in}{0.413425in}}%
\pgfpathlineto{\pgfqpoint{1.659910in}{0.422737in}}%
\pgfpathlineto{\pgfqpoint{1.655068in}{0.427036in}}%
\pgfpathlineto{\pgfqpoint{1.644253in}{0.438507in}}%
\pgfpathlineto{\pgfqpoint{1.641773in}{0.440648in}}%
\pgfpathlineto{\pgfqpoint{1.630020in}{0.454259in}}%
\pgfpathlineto{\pgfqpoint{1.628597in}{0.457030in}}%
\pgfpathlineto{\pgfqpoint{1.619565in}{0.467870in}}%
\pgfpathlineto{\pgfqpoint{1.616313in}{0.481481in}}%
\pgfpathlineto{\pgfqpoint{1.621191in}{0.495092in}}%
\pgfpathlineto{\pgfqpoint{1.628597in}{0.502921in}}%
\pgfpathlineto{\pgfqpoint{1.631979in}{0.508703in}}%
\pgfpathlineto{\pgfqpoint{1.644253in}{0.521850in}}%
\pgfpathlineto{\pgfqpoint{1.644602in}{0.522314in}}%
\pgfpathlineto{\pgfqpoint{1.658031in}{0.535925in}}%
\pgfpathlineto{\pgfqpoint{1.659910in}{0.537558in}}%
\pgfpathlineto{\pgfqpoint{1.672948in}{0.549536in}}%
\pgfpathlineto{\pgfqpoint{1.675567in}{0.551813in}}%
\pgfpathlineto{\pgfqpoint{1.689345in}{0.563148in}}%
\pgfpathlineto{\pgfqpoint{1.691223in}{0.564781in}}%
\pgfpathlineto{\pgfqpoint{1.706880in}{0.576455in}}%
\pgfpathlineto{\pgfqpoint{1.707414in}{0.576759in}}%
\pgfpathlineto{\pgfqpoint{1.722536in}{0.587429in}}%
\pgfpathlineto{\pgfqpoint{1.729188in}{0.590370in}}%
\pgfpathlineto{\pgfqpoint{1.738193in}{0.596808in}}%
\pgfpathlineto{\pgfqpoint{1.753849in}{0.601049in}}%
\pgfpathlineto{\pgfqpoint{1.769506in}{0.598222in}}%
\pgfpathlineto{\pgfqpoint{1.781975in}{0.590370in}}%
\pgfpathlineto{\pgfqpoint{1.785162in}{0.589133in}}%
\pgfpathlineto{\pgfqpoint{1.800819in}{0.578915in}}%
\pgfpathlineto{\pgfqpoint{1.803281in}{0.576759in}}%
\pgfpathlineto{\pgfqpoint{1.816476in}{0.567357in}}%
\pgfpathlineto{\pgfqpoint{1.821421in}{0.563148in}}%
\pgfpathlineto{\pgfqpoint{1.832132in}{0.554537in}}%
\pgfpathlineto{\pgfqpoint{1.837899in}{0.549536in}}%
\pgfpathlineto{\pgfqpoint{1.847789in}{0.540476in}}%
\pgfpathlineto{\pgfqpoint{1.852922in}{0.535925in}}%
\pgfpathlineto{\pgfqpoint{1.863445in}{0.525012in}}%
\pgfpathlineto{\pgfqpoint{1.866533in}{0.522314in}}%
\pgfpathlineto{\pgfqpoint{1.878438in}{0.508703in}}%
\pgfpathlineto{\pgfqpoint{1.879102in}{0.507502in}}%
\pgfpathlineto{\pgfqpoint{1.889472in}{0.495092in}}%
\pgfpathlineto{\pgfqpoint{1.893740in}{0.481481in}}%
\pgfpathlineto{\pgfqpoint{1.890895in}{0.467870in}}%
\pgfpathlineto{\pgfqpoint{1.880922in}{0.454259in}}%
\pgfpathlineto{\pgfqpoint{1.879102in}{0.452756in}}%
\pgfpathlineto{\pgfqpoint{1.869277in}{0.440648in}}%
\pgfpathlineto{\pgfqpoint{1.863445in}{0.435337in}}%
\pgfpathlineto{\pgfqpoint{1.855785in}{0.427036in}}%
\pgfpathlineto{\pgfqpoint{1.847789in}{0.419793in}}%
\pgfpathlineto{\pgfqpoint{1.840996in}{0.413425in}}%
\pgfpathlineto{\pgfqpoint{1.832132in}{0.405720in}}%
\pgfpathlineto{\pgfqpoint{1.824807in}{0.399814in}}%
\pgfpathlineto{\pgfqpoint{1.816476in}{0.392863in}}%
\pgfpathlineto{\pgfqpoint{1.806928in}{0.386203in}}%
\pgfpathlineto{\pgfqpoint{1.800819in}{0.381133in}}%
\pgfpathlineto{\pgfqpoint{1.786891in}{0.372592in}}%
\pgfpathlineto{\pgfqpoint{1.785162in}{0.371010in}}%
\pgfpathlineto{\pgfqpoint{1.769506in}{0.362339in}}%
\pgfpathlineto{\pgfqpoint{1.753849in}{0.359866in}}%
\pgfpathlineto{\pgfqpoint{1.738193in}{0.363576in}}%
\pgfpathlineto{\pgfqpoint{1.723918in}{0.372592in}}%
\pgfpathclose%
\pgfpathmoveto{\pgfqpoint{0.642804in}{0.535925in}}%
\pgfpathlineto{\pgfqpoint{0.642233in}{0.536185in}}%
\pgfpathlineto{\pgfqpoint{0.626577in}{0.547375in}}%
\pgfpathlineto{\pgfqpoint{0.624089in}{0.549536in}}%
\pgfpathlineto{\pgfqpoint{0.610920in}{0.560407in}}%
\pgfpathlineto{\pgfqpoint{0.607768in}{0.563148in}}%
\pgfpathlineto{\pgfqpoint{0.595263in}{0.574596in}}%
\pgfpathlineto{\pgfqpoint{0.592777in}{0.576759in}}%
\pgfpathlineto{\pgfqpoint{0.579906in}{0.590370in}}%
\pgfpathlineto{\pgfqpoint{0.579607in}{0.590866in}}%
\pgfpathlineto{\pgfqpoint{0.570693in}{0.603981in}}%
\pgfpathlineto{\pgfqpoint{0.568405in}{0.617592in}}%
\pgfpathlineto{\pgfqpoint{0.573649in}{0.631203in}}%
\pgfpathlineto{\pgfqpoint{0.579607in}{0.638119in}}%
\pgfpathlineto{\pgfqpoint{0.584665in}{0.644814in}}%
\pgfpathlineto{\pgfqpoint{0.595263in}{0.655014in}}%
\pgfpathlineto{\pgfqpoint{0.598651in}{0.658425in}}%
\pgfpathlineto{\pgfqpoint{0.610920in}{0.669288in}}%
\pgfpathlineto{\pgfqpoint{0.614193in}{0.672036in}}%
\pgfpathlineto{\pgfqpoint{0.626577in}{0.682322in}}%
\pgfpathlineto{\pgfqpoint{0.631425in}{0.685648in}}%
\pgfpathlineto{\pgfqpoint{0.642233in}{0.693677in}}%
\pgfpathlineto{\pgfqpoint{0.653949in}{0.699259in}}%
\pgfpathlineto{\pgfqpoint{0.657890in}{0.701479in}}%
\pgfpathlineto{\pgfqpoint{0.673546in}{0.703607in}}%
\pgfpathlineto{\pgfqpoint{0.687476in}{0.699259in}}%
\pgfpathlineto{\pgfqpoint{0.689203in}{0.698796in}}%
\pgfpathlineto{\pgfqpoint{0.704859in}{0.689449in}}%
\pgfpathlineto{\pgfqpoint{0.709464in}{0.685648in}}%
\pgfpathlineto{\pgfqpoint{0.720516in}{0.677274in}}%
\pgfpathlineto{\pgfqpoint{0.726592in}{0.672036in}}%
\pgfpathlineto{\pgfqpoint{0.736173in}{0.663707in}}%
\pgfpathlineto{\pgfqpoint{0.742198in}{0.658425in}}%
\pgfpathlineto{\pgfqpoint{0.751829in}{0.648818in}}%
\pgfpathlineto{\pgfqpoint{0.756202in}{0.644814in}}%
\pgfpathlineto{\pgfqpoint{0.766953in}{0.631203in}}%
\pgfpathlineto{\pgfqpoint{0.767486in}{0.629701in}}%
\pgfpathlineto{\pgfqpoint{0.772488in}{0.617592in}}%
\pgfpathlineto{\pgfqpoint{0.770040in}{0.603981in}}%
\pgfpathlineto{\pgfqpoint{0.767486in}{0.600555in}}%
\pgfpathlineto{\pgfqpoint{0.761066in}{0.590370in}}%
\pgfpathlineto{\pgfqpoint{0.751829in}{0.580973in}}%
\pgfpathlineto{\pgfqpoint{0.748004in}{0.576759in}}%
\pgfpathlineto{\pgfqpoint{0.736173in}{0.565993in}}%
\pgfpathlineto{\pgfqpoint{0.733011in}{0.563148in}}%
\pgfpathlineto{\pgfqpoint{0.720516in}{0.552482in}}%
\pgfpathlineto{\pgfqpoint{0.716592in}{0.549536in}}%
\pgfpathlineto{\pgfqpoint{0.704859in}{0.540323in}}%
\pgfpathlineto{\pgfqpoint{0.697158in}{0.535925in}}%
\pgfpathlineto{\pgfqpoint{0.689203in}{0.530746in}}%
\pgfpathlineto{\pgfqpoint{0.673546in}{0.526187in}}%
\pgfpathlineto{\pgfqpoint{0.657890in}{0.528176in}}%
\pgfpathlineto{\pgfqpoint{0.642804in}{0.535925in}}%
\pgfpathclose%
\pgfpathmoveto{\pgfqpoint{0.953123in}{0.535925in}}%
\pgfpathlineto{\pgfqpoint{0.939708in}{0.544935in}}%
\pgfpathlineto{\pgfqpoint{0.934268in}{0.549536in}}%
\pgfpathlineto{\pgfqpoint{0.924051in}{0.557758in}}%
\pgfpathlineto{\pgfqpoint{0.917843in}{0.563148in}}%
\pgfpathlineto{\pgfqpoint{0.908395in}{0.571786in}}%
\pgfpathlineto{\pgfqpoint{0.902784in}{0.576759in}}%
\pgfpathlineto{\pgfqpoint{0.892738in}{0.587559in}}%
\pgfpathlineto{\pgfqpoint{0.889787in}{0.590370in}}%
\pgfpathlineto{\pgfqpoint{0.880787in}{0.603981in}}%
\pgfpathlineto{\pgfqpoint{0.878559in}{0.617592in}}%
\pgfpathlineto{\pgfqpoint{0.883667in}{0.631203in}}%
\pgfpathlineto{\pgfqpoint{0.892738in}{0.642156in}}%
\pgfpathlineto{\pgfqpoint{0.894711in}{0.644814in}}%
\pgfpathlineto{\pgfqpoint{0.908395in}{0.658237in}}%
\pgfpathlineto{\pgfqpoint{0.908582in}{0.658425in}}%
\pgfpathlineto{\pgfqpoint{0.923982in}{0.672036in}}%
\pgfpathlineto{\pgfqpoint{0.924051in}{0.672097in}}%
\pgfpathlineto{\pgfqpoint{0.939708in}{0.684756in}}%
\pgfpathlineto{\pgfqpoint{0.941092in}{0.685648in}}%
\pgfpathlineto{\pgfqpoint{0.955364in}{0.695564in}}%
\pgfpathlineto{\pgfqpoint{0.964376in}{0.699259in}}%
\pgfpathlineto{\pgfqpoint{0.971021in}{0.702465in}}%
\pgfpathlineto{\pgfqpoint{0.986678in}{0.703177in}}%
\pgfpathlineto{\pgfqpoint{0.996509in}{0.699259in}}%
\pgfpathlineto{\pgfqpoint{1.002334in}{0.697275in}}%
\pgfpathlineto{\pgfqpoint{1.017991in}{0.687144in}}%
\pgfpathlineto{\pgfqpoint{1.019731in}{0.685648in}}%
\pgfpathlineto{\pgfqpoint{1.033647in}{0.674694in}}%
\pgfpathlineto{\pgfqpoint{1.036697in}{0.672036in}}%
\pgfpathlineto{\pgfqpoint{1.049304in}{0.660966in}}%
\pgfpathlineto{\pgfqpoint{1.052234in}{0.658425in}}%
\pgfpathlineto{\pgfqpoint{1.064960in}{0.645861in}}%
\pgfpathlineto{\pgfqpoint{1.066140in}{0.644814in}}%
\pgfpathlineto{\pgfqpoint{1.077078in}{0.631203in}}%
\pgfpathlineto{\pgfqpoint{1.080617in}{0.621512in}}%
\pgfpathlineto{\pgfqpoint{1.082326in}{0.617592in}}%
\pgfpathlineto{\pgfqpoint{1.080617in}{0.608506in}}%
\pgfpathlineto{\pgfqpoint{1.079896in}{0.603981in}}%
\pgfpathlineto{\pgfqpoint{1.071089in}{0.590370in}}%
\pgfpathlineto{\pgfqpoint{1.064960in}{0.584325in}}%
\pgfpathlineto{\pgfqpoint{1.058021in}{0.576759in}}%
\pgfpathlineto{\pgfqpoint{1.049304in}{0.568914in}}%
\pgfpathlineto{\pgfqpoint{1.042961in}{0.563148in}}%
\pgfpathlineto{\pgfqpoint{1.033647in}{0.555111in}}%
\pgfpathlineto{\pgfqpoint{1.026499in}{0.549536in}}%
\pgfpathlineto{\pgfqpoint{1.017991in}{0.542579in}}%
\pgfpathlineto{\pgfqpoint{1.007317in}{0.535925in}}%
\pgfpathlineto{\pgfqpoint{1.002334in}{0.532368in}}%
\pgfpathlineto{\pgfqpoint{0.986678in}{0.526589in}}%
\pgfpathlineto{\pgfqpoint{0.971021in}{0.527255in}}%
\pgfpathlineto{\pgfqpoint{0.955364in}{0.534195in}}%
\pgfpathlineto{\pgfqpoint{0.953123in}{0.535925in}}%
\pgfpathclose%
\pgfpathmoveto{\pgfqpoint{1.263513in}{0.535925in}}%
\pgfpathlineto{\pgfqpoint{1.252839in}{0.542579in}}%
\pgfpathlineto{\pgfqpoint{1.244331in}{0.549536in}}%
\pgfpathlineto{\pgfqpoint{1.237183in}{0.555111in}}%
\pgfpathlineto{\pgfqpoint{1.227869in}{0.563148in}}%
\pgfpathlineto{\pgfqpoint{1.221526in}{0.568914in}}%
\pgfpathlineto{\pgfqpoint{1.212809in}{0.576759in}}%
\pgfpathlineto{\pgfqpoint{1.205870in}{0.584325in}}%
\pgfpathlineto{\pgfqpoint{1.199741in}{0.590370in}}%
\pgfpathlineto{\pgfqpoint{1.190934in}{0.603981in}}%
\pgfpathlineto{\pgfqpoint{1.190213in}{0.608506in}}%
\pgfpathlineto{\pgfqpoint{1.188504in}{0.617592in}}%
\pgfpathlineto{\pgfqpoint{1.190213in}{0.621512in}}%
\pgfpathlineto{\pgfqpoint{1.193752in}{0.631203in}}%
\pgfpathlineto{\pgfqpoint{1.204690in}{0.644814in}}%
\pgfpathlineto{\pgfqpoint{1.205870in}{0.645861in}}%
\pgfpathlineto{\pgfqpoint{1.218596in}{0.658425in}}%
\pgfpathlineto{\pgfqpoint{1.221526in}{0.660966in}}%
\pgfpathlineto{\pgfqpoint{1.234133in}{0.672036in}}%
\pgfpathlineto{\pgfqpoint{1.237183in}{0.674694in}}%
\pgfpathlineto{\pgfqpoint{1.251099in}{0.685648in}}%
\pgfpathlineto{\pgfqpoint{1.252839in}{0.687144in}}%
\pgfpathlineto{\pgfqpoint{1.268496in}{0.697275in}}%
\pgfpathlineto{\pgfqpoint{1.274321in}{0.699259in}}%
\pgfpathlineto{\pgfqpoint{1.284152in}{0.703177in}}%
\pgfpathlineto{\pgfqpoint{1.299809in}{0.702465in}}%
\pgfpathlineto{\pgfqpoint{1.306454in}{0.699259in}}%
\pgfpathlineto{\pgfqpoint{1.315466in}{0.695564in}}%
\pgfpathlineto{\pgfqpoint{1.329738in}{0.685648in}}%
\pgfpathlineto{\pgfqpoint{1.331122in}{0.684756in}}%
\pgfpathlineto{\pgfqpoint{1.346779in}{0.672097in}}%
\pgfpathlineto{\pgfqpoint{1.346848in}{0.672036in}}%
\pgfpathlineto{\pgfqpoint{1.362248in}{0.658425in}}%
\pgfpathlineto{\pgfqpoint{1.362435in}{0.658237in}}%
\pgfpathlineto{\pgfqpoint{1.376119in}{0.644814in}}%
\pgfpathlineto{\pgfqpoint{1.378092in}{0.642156in}}%
\pgfpathlineto{\pgfqpoint{1.387163in}{0.631203in}}%
\pgfpathlineto{\pgfqpoint{1.392271in}{0.617592in}}%
\pgfpathlineto{\pgfqpoint{1.390043in}{0.603981in}}%
\pgfpathlineto{\pgfqpoint{1.381043in}{0.590370in}}%
\pgfpathlineto{\pgfqpoint{1.378092in}{0.587559in}}%
\pgfpathlineto{\pgfqpoint{1.368046in}{0.576759in}}%
\pgfpathlineto{\pgfqpoint{1.362435in}{0.571786in}}%
\pgfpathlineto{\pgfqpoint{1.352987in}{0.563148in}}%
\pgfpathlineto{\pgfqpoint{1.346779in}{0.557758in}}%
\pgfpathlineto{\pgfqpoint{1.336562in}{0.549536in}}%
\pgfpathlineto{\pgfqpoint{1.331122in}{0.544935in}}%
\pgfpathlineto{\pgfqpoint{1.317707in}{0.535925in}}%
\pgfpathlineto{\pgfqpoint{1.315466in}{0.534195in}}%
\pgfpathlineto{\pgfqpoint{1.299809in}{0.527255in}}%
\pgfpathlineto{\pgfqpoint{1.284152in}{0.526589in}}%
\pgfpathlineto{\pgfqpoint{1.268496in}{0.532368in}}%
\pgfpathlineto{\pgfqpoint{1.263513in}{0.535925in}}%
\pgfpathclose%
\pgfpathmoveto{\pgfqpoint{1.573672in}{0.535925in}}%
\pgfpathlineto{\pgfqpoint{1.565971in}{0.540323in}}%
\pgfpathlineto{\pgfqpoint{1.554238in}{0.549536in}}%
\pgfpathlineto{\pgfqpoint{1.550314in}{0.552482in}}%
\pgfpathlineto{\pgfqpoint{1.537819in}{0.563148in}}%
\pgfpathlineto{\pgfqpoint{1.534657in}{0.565993in}}%
\pgfpathlineto{\pgfqpoint{1.522826in}{0.576759in}}%
\pgfpathlineto{\pgfqpoint{1.519001in}{0.580973in}}%
\pgfpathlineto{\pgfqpoint{1.509764in}{0.590370in}}%
\pgfpathlineto{\pgfqpoint{1.503344in}{0.600555in}}%
\pgfpathlineto{\pgfqpoint{1.500790in}{0.603981in}}%
\pgfpathlineto{\pgfqpoint{1.498342in}{0.617592in}}%
\pgfpathlineto{\pgfqpoint{1.503344in}{0.629701in}}%
\pgfpathlineto{\pgfqpoint{1.503877in}{0.631203in}}%
\pgfpathlineto{\pgfqpoint{1.514628in}{0.644814in}}%
\pgfpathlineto{\pgfqpoint{1.519001in}{0.648818in}}%
\pgfpathlineto{\pgfqpoint{1.528632in}{0.658425in}}%
\pgfpathlineto{\pgfqpoint{1.534657in}{0.663707in}}%
\pgfpathlineto{\pgfqpoint{1.544238in}{0.672036in}}%
\pgfpathlineto{\pgfqpoint{1.550314in}{0.677274in}}%
\pgfpathlineto{\pgfqpoint{1.561366in}{0.685648in}}%
\pgfpathlineto{\pgfqpoint{1.565971in}{0.689449in}}%
\pgfpathlineto{\pgfqpoint{1.581627in}{0.698796in}}%
\pgfpathlineto{\pgfqpoint{1.583354in}{0.699259in}}%
\pgfpathlineto{\pgfqpoint{1.597284in}{0.703607in}}%
\pgfpathlineto{\pgfqpoint{1.612940in}{0.701479in}}%
\pgfpathlineto{\pgfqpoint{1.616881in}{0.699259in}}%
\pgfpathlineto{\pgfqpoint{1.628597in}{0.693677in}}%
\pgfpathlineto{\pgfqpoint{1.639405in}{0.685648in}}%
\pgfpathlineto{\pgfqpoint{1.644253in}{0.682322in}}%
\pgfpathlineto{\pgfqpoint{1.656637in}{0.672036in}}%
\pgfpathlineto{\pgfqpoint{1.659910in}{0.669288in}}%
\pgfpathlineto{\pgfqpoint{1.672179in}{0.658425in}}%
\pgfpathlineto{\pgfqpoint{1.675567in}{0.655014in}}%
\pgfpathlineto{\pgfqpoint{1.686165in}{0.644814in}}%
\pgfpathlineto{\pgfqpoint{1.691223in}{0.638119in}}%
\pgfpathlineto{\pgfqpoint{1.697181in}{0.631203in}}%
\pgfpathlineto{\pgfqpoint{1.702425in}{0.617592in}}%
\pgfpathlineto{\pgfqpoint{1.700137in}{0.603981in}}%
\pgfpathlineto{\pgfqpoint{1.691223in}{0.590866in}}%
\pgfpathlineto{\pgfqpoint{1.690924in}{0.590370in}}%
\pgfpathlineto{\pgfqpoint{1.678053in}{0.576759in}}%
\pgfpathlineto{\pgfqpoint{1.675567in}{0.574596in}}%
\pgfpathlineto{\pgfqpoint{1.663062in}{0.563148in}}%
\pgfpathlineto{\pgfqpoint{1.659910in}{0.560407in}}%
\pgfpathlineto{\pgfqpoint{1.646741in}{0.549536in}}%
\pgfpathlineto{\pgfqpoint{1.644253in}{0.547375in}}%
\pgfpathlineto{\pgfqpoint{1.628597in}{0.536185in}}%
\pgfpathlineto{\pgfqpoint{1.628026in}{0.535925in}}%
\pgfpathlineto{\pgfqpoint{1.612940in}{0.528176in}}%
\pgfpathlineto{\pgfqpoint{1.597284in}{0.526187in}}%
\pgfpathlineto{\pgfqpoint{1.581627in}{0.530746in}}%
\pgfpathlineto{\pgfqpoint{1.573672in}{0.535925in}}%
\pgfpathclose%
\pgfpathmoveto{\pgfqpoint{0.509646in}{0.631203in}}%
\pgfpathlineto{\pgfqpoint{0.501324in}{0.632379in}}%
\pgfpathlineto{\pgfqpoint{0.485668in}{0.640113in}}%
\pgfpathlineto{\pgfqpoint{0.479970in}{0.644814in}}%
\pgfpathlineto{\pgfqpoint{0.470011in}{0.650633in}}%
\pgfpathlineto{\pgfqpoint{0.460185in}{0.658425in}}%
\pgfpathlineto{\pgfqpoint{0.454354in}{0.662428in}}%
\pgfpathlineto{\pgfqpoint{0.442666in}{0.672036in}}%
\pgfpathlineto{\pgfqpoint{0.438698in}{0.675258in}}%
\pgfpathlineto{\pgfqpoint{0.426825in}{0.685648in}}%
\pgfpathlineto{\pgfqpoint{0.423041in}{0.689292in}}%
\pgfpathlineto{\pgfqpoint{0.412313in}{0.699259in}}%
\pgfpathlineto{\pgfqpoint{0.407385in}{0.704864in}}%
\pgfpathlineto{\pgfqpoint{0.398989in}{0.712870in}}%
\pgfpathlineto{\pgfqpoint{0.391728in}{0.722568in}}%
\pgfpathlineto{\pgfqpoint{0.387342in}{0.726481in}}%
\pgfpathlineto{\pgfqpoint{0.378796in}{0.740092in}}%
\pgfpathlineto{\pgfqpoint{0.377374in}{0.753703in}}%
\pgfpathlineto{\pgfqpoint{0.383067in}{0.767314in}}%
\pgfpathlineto{\pgfqpoint{0.391728in}{0.776596in}}%
\pgfpathlineto{\pgfqpoint{0.394408in}{0.780925in}}%
\pgfpathlineto{\pgfqpoint{0.407219in}{0.794536in}}%
\pgfpathlineto{\pgfqpoint{0.407385in}{0.794676in}}%
\pgfpathlineto{\pgfqpoint{0.420897in}{0.808148in}}%
\pgfpathlineto{\pgfqpoint{0.423041in}{0.810015in}}%
\pgfpathlineto{\pgfqpoint{0.436107in}{0.821759in}}%
\pgfpathlineto{\pgfqpoint{0.438698in}{0.824008in}}%
\pgfpathlineto{\pgfqpoint{0.452809in}{0.835370in}}%
\pgfpathlineto{\pgfqpoint{0.454354in}{0.836719in}}%
\pgfpathlineto{\pgfqpoint{0.470011in}{0.848226in}}%
\pgfpathlineto{\pgfqpoint{0.471402in}{0.848981in}}%
\pgfpathlineto{\pgfqpoint{0.485668in}{0.858987in}}%
\pgfpathlineto{\pgfqpoint{0.494355in}{0.862592in}}%
\pgfpathlineto{\pgfqpoint{0.501324in}{0.867691in}}%
\pgfpathlineto{\pgfqpoint{0.516981in}{0.871005in}}%
\pgfpathlineto{\pgfqpoint{0.532637in}{0.866033in}}%
\pgfpathlineto{\pgfqpoint{0.536780in}{0.862592in}}%
\pgfpathlineto{\pgfqpoint{0.548294in}{0.857150in}}%
\pgfpathlineto{\pgfqpoint{0.559159in}{0.848981in}}%
\pgfpathlineto{\pgfqpoint{0.563950in}{0.846186in}}%
\pgfpathlineto{\pgfqpoint{0.577979in}{0.835370in}}%
\pgfpathlineto{\pgfqpoint{0.579607in}{0.834222in}}%
\pgfpathlineto{\pgfqpoint{0.594680in}{0.821759in}}%
\pgfpathlineto{\pgfqpoint{0.595263in}{0.821261in}}%
\pgfpathlineto{\pgfqpoint{0.609847in}{0.808148in}}%
\pgfpathlineto{\pgfqpoint{0.610920in}{0.807041in}}%
\pgfpathlineto{\pgfqpoint{0.623826in}{0.794536in}}%
\pgfpathlineto{\pgfqpoint{0.626577in}{0.791106in}}%
\pgfpathlineto{\pgfqpoint{0.636696in}{0.780925in}}%
\pgfpathlineto{\pgfqpoint{0.642233in}{0.772476in}}%
\pgfpathlineto{\pgfqpoint{0.647685in}{0.767314in}}%
\pgfpathlineto{\pgfqpoint{0.654192in}{0.753703in}}%
\pgfpathlineto{\pgfqpoint{0.652566in}{0.740092in}}%
\pgfpathlineto{\pgfqpoint{0.642799in}{0.726481in}}%
\pgfpathlineto{\pgfqpoint{0.642233in}{0.726035in}}%
\pgfpathlineto{\pgfqpoint{0.631798in}{0.712870in}}%
\pgfpathlineto{\pgfqpoint{0.626577in}{0.708152in}}%
\pgfpathlineto{\pgfqpoint{0.618590in}{0.699259in}}%
\pgfpathlineto{\pgfqpoint{0.610920in}{0.692274in}}%
\pgfpathlineto{\pgfqpoint{0.604059in}{0.685648in}}%
\pgfpathlineto{\pgfqpoint{0.595263in}{0.677930in}}%
\pgfpathlineto{\pgfqpoint{0.588169in}{0.672036in}}%
\pgfpathlineto{\pgfqpoint{0.579607in}{0.664845in}}%
\pgfpathlineto{\pgfqpoint{0.570657in}{0.658425in}}%
\pgfpathlineto{\pgfqpoint{0.563950in}{0.652882in}}%
\pgfpathlineto{\pgfqpoint{0.551108in}{0.644814in}}%
\pgfpathlineto{\pgfqpoint{0.548294in}{0.642325in}}%
\pgfpathlineto{\pgfqpoint{0.532637in}{0.633483in}}%
\pgfpathlineto{\pgfqpoint{0.521887in}{0.631203in}}%
\pgfpathlineto{\pgfqpoint{0.516981in}{0.628802in}}%
\pgfpathlineto{\pgfqpoint{0.509646in}{0.631203in}}%
\pgfpathclose%
\pgfpathmoveto{\pgfqpoint{0.818617in}{0.631203in}}%
\pgfpathlineto{\pgfqpoint{0.814455in}{0.631497in}}%
\pgfpathlineto{\pgfqpoint{0.798799in}{0.638123in}}%
\pgfpathlineto{\pgfqpoint{0.790038in}{0.644814in}}%
\pgfpathlineto{\pgfqpoint{0.783142in}{0.648532in}}%
\pgfpathlineto{\pgfqpoint{0.770079in}{0.658425in}}%
\pgfpathlineto{\pgfqpoint{0.767486in}{0.660121in}}%
\pgfpathlineto{\pgfqpoint{0.752621in}{0.672036in}}%
\pgfpathlineto{\pgfqpoint{0.751829in}{0.672662in}}%
\pgfpathlineto{\pgfqpoint{0.736892in}{0.685648in}}%
\pgfpathlineto{\pgfqpoint{0.736173in}{0.686336in}}%
\pgfpathlineto{\pgfqpoint{0.722466in}{0.699259in}}%
\pgfpathlineto{\pgfqpoint{0.720516in}{0.701513in}}%
\pgfpathlineto{\pgfqpoint{0.709136in}{0.712870in}}%
\pgfpathlineto{\pgfqpoint{0.704859in}{0.718864in}}%
\pgfpathlineto{\pgfqpoint{0.697162in}{0.726481in}}%
\pgfpathlineto{\pgfqpoint{0.689541in}{0.740092in}}%
\pgfpathlineto{\pgfqpoint{0.689203in}{0.743710in}}%
\pgfpathlineto{\pgfqpoint{0.687033in}{0.753703in}}%
\pgfpathlineto{\pgfqpoint{0.689203in}{0.756227in}}%
\pgfpathlineto{\pgfqpoint{0.693349in}{0.767314in}}%
\pgfpathlineto{\pgfqpoint{0.704796in}{0.780925in}}%
\pgfpathlineto{\pgfqpoint{0.704859in}{0.780976in}}%
\pgfpathlineto{\pgfqpoint{0.716895in}{0.794536in}}%
\pgfpathlineto{\pgfqpoint{0.720516in}{0.797727in}}%
\pgfpathlineto{\pgfqpoint{0.730802in}{0.808148in}}%
\pgfpathlineto{\pgfqpoint{0.736173in}{0.812894in}}%
\pgfpathlineto{\pgfqpoint{0.746105in}{0.821759in}}%
\pgfpathlineto{\pgfqpoint{0.751829in}{0.826696in}}%
\pgfpathlineto{\pgfqpoint{0.762896in}{0.835370in}}%
\pgfpathlineto{\pgfqpoint{0.767486in}{0.839277in}}%
\pgfpathlineto{\pgfqpoint{0.781400in}{0.848981in}}%
\pgfpathlineto{\pgfqpoint{0.783142in}{0.850534in}}%
\pgfpathlineto{\pgfqpoint{0.798799in}{0.860640in}}%
\pgfpathlineto{\pgfqpoint{0.804273in}{0.862592in}}%
\pgfpathlineto{\pgfqpoint{0.814455in}{0.869017in}}%
\pgfpathlineto{\pgfqpoint{0.830112in}{0.870674in}}%
\pgfpathlineto{\pgfqpoint{0.845769in}{0.864042in}}%
\pgfpathlineto{\pgfqpoint{0.847332in}{0.862592in}}%
\pgfpathlineto{\pgfqpoint{0.861425in}{0.855128in}}%
\pgfpathlineto{\pgfqpoint{0.869105in}{0.848981in}}%
\pgfpathlineto{\pgfqpoint{0.877082in}{0.844013in}}%
\pgfpathlineto{\pgfqpoint{0.887884in}{0.835370in}}%
\pgfpathlineto{\pgfqpoint{0.892738in}{0.831809in}}%
\pgfpathlineto{\pgfqpoint{0.904678in}{0.821759in}}%
\pgfpathlineto{\pgfqpoint{0.908395in}{0.818528in}}%
\pgfpathlineto{\pgfqpoint{0.919955in}{0.808148in}}%
\pgfpathlineto{\pgfqpoint{0.924051in}{0.803927in}}%
\pgfpathlineto{\pgfqpoint{0.933993in}{0.794536in}}%
\pgfpathlineto{\pgfqpoint{0.939708in}{0.787602in}}%
\pgfpathlineto{\pgfqpoint{0.946779in}{0.780925in}}%
\pgfpathlineto{\pgfqpoint{0.955364in}{0.768673in}}%
\pgfpathlineto{\pgfqpoint{0.957033in}{0.767314in}}%
\pgfpathlineto{\pgfqpoint{0.964661in}{0.753703in}}%
\pgfpathlineto{\pgfqpoint{0.962755in}{0.740092in}}%
\pgfpathlineto{\pgfqpoint{0.955364in}{0.731240in}}%
\pgfpathlineto{\pgfqpoint{0.953120in}{0.726481in}}%
\pgfpathlineto{\pgfqpoint{0.941494in}{0.712870in}}%
\pgfpathlineto{\pgfqpoint{0.939708in}{0.711355in}}%
\pgfpathlineto{\pgfqpoint{0.928545in}{0.699259in}}%
\pgfpathlineto{\pgfqpoint{0.924051in}{0.695269in}}%
\pgfpathlineto{\pgfqpoint{0.914075in}{0.685648in}}%
\pgfpathlineto{\pgfqpoint{0.908395in}{0.680671in}}%
\pgfpathlineto{\pgfqpoint{0.898199in}{0.672036in}}%
\pgfpathlineto{\pgfqpoint{0.892738in}{0.667368in}}%
\pgfpathlineto{\pgfqpoint{0.880752in}{0.658425in}}%
\pgfpathlineto{\pgfqpoint{0.877082in}{0.655277in}}%
\pgfpathlineto{\pgfqpoint{0.861484in}{0.644814in}}%
\pgfpathlineto{\pgfqpoint{0.861425in}{0.644759in}}%
\pgfpathlineto{\pgfqpoint{0.845769in}{0.634808in}}%
\pgfpathlineto{\pgfqpoint{0.833015in}{0.631203in}}%
\pgfpathlineto{\pgfqpoint{0.830112in}{0.629316in}}%
\pgfpathlineto{\pgfqpoint{0.818617in}{0.631203in}}%
\pgfpathclose%
\pgfpathmoveto{\pgfqpoint{1.126526in}{0.631203in}}%
\pgfpathlineto{\pgfqpoint{1.111930in}{0.636354in}}%
\pgfpathlineto{\pgfqpoint{1.099855in}{0.644814in}}%
\pgfpathlineto{\pgfqpoint{1.096274in}{0.646579in}}%
\pgfpathlineto{\pgfqpoint{1.080617in}{0.657816in}}%
\pgfpathlineto{\pgfqpoint{1.079930in}{0.658425in}}%
\pgfpathlineto{\pgfqpoint{1.064960in}{0.669993in}}%
\pgfpathlineto{\pgfqpoint{1.062604in}{0.672036in}}%
\pgfpathlineto{\pgfqpoint{1.049304in}{0.683470in}}%
\pgfpathlineto{\pgfqpoint{1.046805in}{0.685648in}}%
\pgfpathlineto{\pgfqpoint{1.033647in}{0.698260in}}%
\pgfpathlineto{\pgfqpoint{1.032487in}{0.699259in}}%
\pgfpathlineto{\pgfqpoint{1.019419in}{0.712870in}}%
\pgfpathlineto{\pgfqpoint{1.017991in}{0.714954in}}%
\pgfpathlineto{\pgfqpoint{1.007321in}{0.726481in}}%
\pgfpathlineto{\pgfqpoint{1.002334in}{0.736255in}}%
\pgfpathlineto{\pgfqpoint{0.998476in}{0.740092in}}%
\pgfpathlineto{\pgfqpoint{0.996162in}{0.753703in}}%
\pgfpathlineto{\pgfqpoint{1.002334in}{0.762818in}}%
\pgfpathlineto{\pgfqpoint{1.003868in}{0.767314in}}%
\pgfpathlineto{\pgfqpoint{1.014235in}{0.780925in}}%
\pgfpathlineto{\pgfqpoint{1.017991in}{0.784217in}}%
\pgfpathlineto{\pgfqpoint{1.026786in}{0.794536in}}%
\pgfpathlineto{\pgfqpoint{1.033647in}{0.800817in}}%
\pgfpathlineto{\pgfqpoint{1.040806in}{0.808148in}}%
\pgfpathlineto{\pgfqpoint{1.049304in}{0.815736in}}%
\pgfpathlineto{\pgfqpoint{1.056128in}{0.821759in}}%
\pgfpathlineto{\pgfqpoint{1.064960in}{0.829298in}}%
\pgfpathlineto{\pgfqpoint{1.072951in}{0.835370in}}%
\pgfpathlineto{\pgfqpoint{1.080617in}{0.841709in}}%
\pgfpathlineto{\pgfqpoint{1.091619in}{0.848981in}}%
\pgfpathlineto{\pgfqpoint{1.096274in}{0.852922in}}%
\pgfpathlineto{\pgfqpoint{1.111930in}{0.862109in}}%
\pgfpathlineto{\pgfqpoint{1.113552in}{0.862592in}}%
\pgfpathlineto{\pgfqpoint{1.127587in}{0.870011in}}%
\pgfpathlineto{\pgfqpoint{1.143243in}{0.870011in}}%
\pgfpathlineto{\pgfqpoint{1.157278in}{0.862592in}}%
\pgfpathlineto{\pgfqpoint{1.158900in}{0.862109in}}%
\pgfpathlineto{\pgfqpoint{1.174556in}{0.852922in}}%
\pgfpathlineto{\pgfqpoint{1.179211in}{0.848981in}}%
\pgfpathlineto{\pgfqpoint{1.190213in}{0.841709in}}%
\pgfpathlineto{\pgfqpoint{1.197879in}{0.835370in}}%
\pgfpathlineto{\pgfqpoint{1.205870in}{0.829298in}}%
\pgfpathlineto{\pgfqpoint{1.214702in}{0.821759in}}%
\pgfpathlineto{\pgfqpoint{1.221526in}{0.815736in}}%
\pgfpathlineto{\pgfqpoint{1.230024in}{0.808148in}}%
\pgfpathlineto{\pgfqpoint{1.237183in}{0.800817in}}%
\pgfpathlineto{\pgfqpoint{1.244044in}{0.794536in}}%
\pgfpathlineto{\pgfqpoint{1.252839in}{0.784217in}}%
\pgfpathlineto{\pgfqpoint{1.256595in}{0.780925in}}%
\pgfpathlineto{\pgfqpoint{1.266962in}{0.767314in}}%
\pgfpathlineto{\pgfqpoint{1.268496in}{0.762818in}}%
\pgfpathlineto{\pgfqpoint{1.274668in}{0.753703in}}%
\pgfpathlineto{\pgfqpoint{1.272354in}{0.740092in}}%
\pgfpathlineto{\pgfqpoint{1.268496in}{0.736255in}}%
\pgfpathlineto{\pgfqpoint{1.263509in}{0.726481in}}%
\pgfpathlineto{\pgfqpoint{1.252839in}{0.714954in}}%
\pgfpathlineto{\pgfqpoint{1.251411in}{0.712870in}}%
\pgfpathlineto{\pgfqpoint{1.238343in}{0.699259in}}%
\pgfpathlineto{\pgfqpoint{1.237183in}{0.698260in}}%
\pgfpathlineto{\pgfqpoint{1.224025in}{0.685648in}}%
\pgfpathlineto{\pgfqpoint{1.221526in}{0.683470in}}%
\pgfpathlineto{\pgfqpoint{1.208226in}{0.672036in}}%
\pgfpathlineto{\pgfqpoint{1.205870in}{0.669993in}}%
\pgfpathlineto{\pgfqpoint{1.190900in}{0.658425in}}%
\pgfpathlineto{\pgfqpoint{1.190213in}{0.657816in}}%
\pgfpathlineto{\pgfqpoint{1.174556in}{0.646579in}}%
\pgfpathlineto{\pgfqpoint{1.170975in}{0.644814in}}%
\pgfpathlineto{\pgfqpoint{1.158900in}{0.636354in}}%
\pgfpathlineto{\pgfqpoint{1.144304in}{0.631203in}}%
\pgfpathlineto{\pgfqpoint{1.143243in}{0.630345in}}%
\pgfpathlineto{\pgfqpoint{1.127587in}{0.630345in}}%
\pgfpathlineto{\pgfqpoint{1.126526in}{0.631203in}}%
\pgfpathclose%
\pgfpathmoveto{\pgfqpoint{1.437815in}{0.631203in}}%
\pgfpathlineto{\pgfqpoint{1.425061in}{0.634808in}}%
\pgfpathlineto{\pgfqpoint{1.409405in}{0.644759in}}%
\pgfpathlineto{\pgfqpoint{1.409346in}{0.644814in}}%
\pgfpathlineto{\pgfqpoint{1.393748in}{0.655277in}}%
\pgfpathlineto{\pgfqpoint{1.390078in}{0.658425in}}%
\pgfpathlineto{\pgfqpoint{1.378092in}{0.667368in}}%
\pgfpathlineto{\pgfqpoint{1.372631in}{0.672036in}}%
\pgfpathlineto{\pgfqpoint{1.362435in}{0.680671in}}%
\pgfpathlineto{\pgfqpoint{1.356755in}{0.685648in}}%
\pgfpathlineto{\pgfqpoint{1.346779in}{0.695269in}}%
\pgfpathlineto{\pgfqpoint{1.342285in}{0.699259in}}%
\pgfpathlineto{\pgfqpoint{1.331122in}{0.711355in}}%
\pgfpathlineto{\pgfqpoint{1.329336in}{0.712870in}}%
\pgfpathlineto{\pgfqpoint{1.317710in}{0.726481in}}%
\pgfpathlineto{\pgfqpoint{1.315466in}{0.731240in}}%
\pgfpathlineto{\pgfqpoint{1.308075in}{0.740092in}}%
\pgfpathlineto{\pgfqpoint{1.306169in}{0.753703in}}%
\pgfpathlineto{\pgfqpoint{1.313797in}{0.767314in}}%
\pgfpathlineto{\pgfqpoint{1.315466in}{0.768673in}}%
\pgfpathlineto{\pgfqpoint{1.324051in}{0.780925in}}%
\pgfpathlineto{\pgfqpoint{1.331122in}{0.787602in}}%
\pgfpathlineto{\pgfqpoint{1.336837in}{0.794536in}}%
\pgfpathlineto{\pgfqpoint{1.346779in}{0.803927in}}%
\pgfpathlineto{\pgfqpoint{1.350875in}{0.808148in}}%
\pgfpathlineto{\pgfqpoint{1.362435in}{0.818528in}}%
\pgfpathlineto{\pgfqpoint{1.366152in}{0.821759in}}%
\pgfpathlineto{\pgfqpoint{1.378092in}{0.831809in}}%
\pgfpathlineto{\pgfqpoint{1.382946in}{0.835370in}}%
\pgfpathlineto{\pgfqpoint{1.393748in}{0.844013in}}%
\pgfpathlineto{\pgfqpoint{1.401725in}{0.848981in}}%
\pgfpathlineto{\pgfqpoint{1.409405in}{0.855128in}}%
\pgfpathlineto{\pgfqpoint{1.423498in}{0.862592in}}%
\pgfpathlineto{\pgfqpoint{1.425061in}{0.864042in}}%
\pgfpathlineto{\pgfqpoint{1.440718in}{0.870674in}}%
\pgfpathlineto{\pgfqpoint{1.456375in}{0.869017in}}%
\pgfpathlineto{\pgfqpoint{1.466557in}{0.862592in}}%
\pgfpathlineto{\pgfqpoint{1.472031in}{0.860640in}}%
\pgfpathlineto{\pgfqpoint{1.487688in}{0.850534in}}%
\pgfpathlineto{\pgfqpoint{1.489430in}{0.848981in}}%
\pgfpathlineto{\pgfqpoint{1.503344in}{0.839277in}}%
\pgfpathlineto{\pgfqpoint{1.507934in}{0.835370in}}%
\pgfpathlineto{\pgfqpoint{1.519001in}{0.826696in}}%
\pgfpathlineto{\pgfqpoint{1.524725in}{0.821759in}}%
\pgfpathlineto{\pgfqpoint{1.534657in}{0.812894in}}%
\pgfpathlineto{\pgfqpoint{1.540028in}{0.808148in}}%
\pgfpathlineto{\pgfqpoint{1.550314in}{0.797727in}}%
\pgfpathlineto{\pgfqpoint{1.553935in}{0.794536in}}%
\pgfpathlineto{\pgfqpoint{1.565971in}{0.780976in}}%
\pgfpathlineto{\pgfqpoint{1.566034in}{0.780925in}}%
\pgfpathlineto{\pgfqpoint{1.577481in}{0.767314in}}%
\pgfpathlineto{\pgfqpoint{1.581627in}{0.756227in}}%
\pgfpathlineto{\pgfqpoint{1.583797in}{0.753703in}}%
\pgfpathlineto{\pgfqpoint{1.581627in}{0.743710in}}%
\pgfpathlineto{\pgfqpoint{1.581289in}{0.740092in}}%
\pgfpathlineto{\pgfqpoint{1.573668in}{0.726481in}}%
\pgfpathlineto{\pgfqpoint{1.565971in}{0.718864in}}%
\pgfpathlineto{\pgfqpoint{1.561694in}{0.712870in}}%
\pgfpathlineto{\pgfqpoint{1.550314in}{0.701513in}}%
\pgfpathlineto{\pgfqpoint{1.548364in}{0.699259in}}%
\pgfpathlineto{\pgfqpoint{1.534657in}{0.686336in}}%
\pgfpathlineto{\pgfqpoint{1.533938in}{0.685648in}}%
\pgfpathlineto{\pgfqpoint{1.519001in}{0.672662in}}%
\pgfpathlineto{\pgfqpoint{1.518209in}{0.672036in}}%
\pgfpathlineto{\pgfqpoint{1.503344in}{0.660121in}}%
\pgfpathlineto{\pgfqpoint{1.500751in}{0.658425in}}%
\pgfpathlineto{\pgfqpoint{1.487688in}{0.648532in}}%
\pgfpathlineto{\pgfqpoint{1.480792in}{0.644814in}}%
\pgfpathlineto{\pgfqpoint{1.472031in}{0.638123in}}%
\pgfpathlineto{\pgfqpoint{1.456375in}{0.631497in}}%
\pgfpathlineto{\pgfqpoint{1.452213in}{0.631203in}}%
\pgfpathlineto{\pgfqpoint{1.440718in}{0.629316in}}%
\pgfpathlineto{\pgfqpoint{1.437815in}{0.631203in}}%
\pgfpathclose%
\pgfpathmoveto{\pgfqpoint{1.748943in}{0.631203in}}%
\pgfpathlineto{\pgfqpoint{1.738193in}{0.633483in}}%
\pgfpathlineto{\pgfqpoint{1.722536in}{0.642325in}}%
\pgfpathlineto{\pgfqpoint{1.719722in}{0.644814in}}%
\pgfpathlineto{\pgfqpoint{1.706880in}{0.652882in}}%
\pgfpathlineto{\pgfqpoint{1.700173in}{0.658425in}}%
\pgfpathlineto{\pgfqpoint{1.691223in}{0.664845in}}%
\pgfpathlineto{\pgfqpoint{1.682661in}{0.672036in}}%
\pgfpathlineto{\pgfqpoint{1.675567in}{0.677930in}}%
\pgfpathlineto{\pgfqpoint{1.666771in}{0.685648in}}%
\pgfpathlineto{\pgfqpoint{1.659910in}{0.692274in}}%
\pgfpathlineto{\pgfqpoint{1.652240in}{0.699259in}}%
\pgfpathlineto{\pgfqpoint{1.644253in}{0.708152in}}%
\pgfpathlineto{\pgfqpoint{1.639032in}{0.712870in}}%
\pgfpathlineto{\pgfqpoint{1.628597in}{0.726035in}}%
\pgfpathlineto{\pgfqpoint{1.628031in}{0.726481in}}%
\pgfpathlineto{\pgfqpoint{1.618264in}{0.740092in}}%
\pgfpathlineto{\pgfqpoint{1.616638in}{0.753703in}}%
\pgfpathlineto{\pgfqpoint{1.623145in}{0.767314in}}%
\pgfpathlineto{\pgfqpoint{1.628597in}{0.772476in}}%
\pgfpathlineto{\pgfqpoint{1.634134in}{0.780925in}}%
\pgfpathlineto{\pgfqpoint{1.644253in}{0.791106in}}%
\pgfpathlineto{\pgfqpoint{1.647004in}{0.794536in}}%
\pgfpathlineto{\pgfqpoint{1.659910in}{0.807041in}}%
\pgfpathlineto{\pgfqpoint{1.660983in}{0.808148in}}%
\pgfpathlineto{\pgfqpoint{1.675567in}{0.821261in}}%
\pgfpathlineto{\pgfqpoint{1.676150in}{0.821759in}}%
\pgfpathlineto{\pgfqpoint{1.691223in}{0.834222in}}%
\pgfpathlineto{\pgfqpoint{1.692851in}{0.835370in}}%
\pgfpathlineto{\pgfqpoint{1.706880in}{0.846186in}}%
\pgfpathlineto{\pgfqpoint{1.711671in}{0.848981in}}%
\pgfpathlineto{\pgfqpoint{1.722536in}{0.857150in}}%
\pgfpathlineto{\pgfqpoint{1.734050in}{0.862592in}}%
\pgfpathlineto{\pgfqpoint{1.738193in}{0.866033in}}%
\pgfpathlineto{\pgfqpoint{1.753849in}{0.871005in}}%
\pgfpathlineto{\pgfqpoint{1.769506in}{0.867691in}}%
\pgfpathlineto{\pgfqpoint{1.776475in}{0.862592in}}%
\pgfpathlineto{\pgfqpoint{1.785162in}{0.858987in}}%
\pgfpathlineto{\pgfqpoint{1.799428in}{0.848981in}}%
\pgfpathlineto{\pgfqpoint{1.800819in}{0.848226in}}%
\pgfpathlineto{\pgfqpoint{1.816476in}{0.836719in}}%
\pgfpathlineto{\pgfqpoint{1.818021in}{0.835370in}}%
\pgfpathlineto{\pgfqpoint{1.832132in}{0.824008in}}%
\pgfpathlineto{\pgfqpoint{1.834723in}{0.821759in}}%
\pgfpathlineto{\pgfqpoint{1.847789in}{0.810015in}}%
\pgfpathlineto{\pgfqpoint{1.849933in}{0.808148in}}%
\pgfpathlineto{\pgfqpoint{1.863445in}{0.794676in}}%
\pgfpathlineto{\pgfqpoint{1.863611in}{0.794536in}}%
\pgfpathlineto{\pgfqpoint{1.876422in}{0.780925in}}%
\pgfpathlineto{\pgfqpoint{1.879102in}{0.776596in}}%
\pgfpathlineto{\pgfqpoint{1.887763in}{0.767314in}}%
\pgfpathlineto{\pgfqpoint{1.893456in}{0.753703in}}%
\pgfpathlineto{\pgfqpoint{1.892034in}{0.740092in}}%
\pgfpathlineto{\pgfqpoint{1.883488in}{0.726481in}}%
\pgfpathlineto{\pgfqpoint{1.879102in}{0.722568in}}%
\pgfpathlineto{\pgfqpoint{1.871841in}{0.712870in}}%
\pgfpathlineto{\pgfqpoint{1.863445in}{0.704864in}}%
\pgfpathlineto{\pgfqpoint{1.858517in}{0.699259in}}%
\pgfpathlineto{\pgfqpoint{1.847789in}{0.689292in}}%
\pgfpathlineto{\pgfqpoint{1.844005in}{0.685648in}}%
\pgfpathlineto{\pgfqpoint{1.832132in}{0.675258in}}%
\pgfpathlineto{\pgfqpoint{1.828164in}{0.672036in}}%
\pgfpathlineto{\pgfqpoint{1.816476in}{0.662428in}}%
\pgfpathlineto{\pgfqpoint{1.810645in}{0.658425in}}%
\pgfpathlineto{\pgfqpoint{1.800819in}{0.650633in}}%
\pgfpathlineto{\pgfqpoint{1.790860in}{0.644814in}}%
\pgfpathlineto{\pgfqpoint{1.785162in}{0.640113in}}%
\pgfpathlineto{\pgfqpoint{1.769506in}{0.632379in}}%
\pgfpathlineto{\pgfqpoint{1.761184in}{0.631203in}}%
\pgfpathlineto{\pgfqpoint{1.753849in}{0.628802in}}%
\pgfpathlineto{\pgfqpoint{1.748943in}{0.631203in}}%
\pgfpathclose%
\pgfpathmoveto{\pgfqpoint{0.639000in}{0.808148in}}%
\pgfpathlineto{\pgfqpoint{0.626577in}{0.816881in}}%
\pgfpathlineto{\pgfqpoint{0.620856in}{0.821759in}}%
\pgfpathlineto{\pgfqpoint{0.610920in}{0.829972in}}%
\pgfpathlineto{\pgfqpoint{0.604720in}{0.835370in}}%
\pgfpathlineto{\pgfqpoint{0.595263in}{0.844251in}}%
\pgfpathlineto{\pgfqpoint{0.589971in}{0.848981in}}%
\pgfpathlineto{\pgfqpoint{0.579607in}{0.860644in}}%
\pgfpathlineto{\pgfqpoint{0.577617in}{0.862592in}}%
\pgfpathlineto{\pgfqpoint{0.569633in}{0.876203in}}%
\pgfpathlineto{\pgfqpoint{0.568868in}{0.889814in}}%
\pgfpathlineto{\pgfqpoint{0.575516in}{0.903425in}}%
\pgfpathlineto{\pgfqpoint{0.579607in}{0.907757in}}%
\pgfpathlineto{\pgfqpoint{0.587260in}{0.917036in}}%
\pgfpathlineto{\pgfqpoint{0.595263in}{0.924433in}}%
\pgfpathlineto{\pgfqpoint{0.601676in}{0.930648in}}%
\pgfpathlineto{\pgfqpoint{0.610920in}{0.938744in}}%
\pgfpathlineto{\pgfqpoint{0.617554in}{0.944259in}}%
\pgfpathlineto{\pgfqpoint{0.626577in}{0.951837in}}%
\pgfpathlineto{\pgfqpoint{0.635281in}{0.957870in}}%
\pgfpathlineto{\pgfqpoint{0.642233in}{0.963197in}}%
\pgfpathlineto{\pgfqpoint{0.657890in}{0.970854in}}%
\pgfpathlineto{\pgfqpoint{0.663095in}{0.971481in}}%
\pgfpathlineto{\pgfqpoint{0.673546in}{0.972966in}}%
\pgfpathlineto{\pgfqpoint{0.678055in}{0.971481in}}%
\pgfpathlineto{\pgfqpoint{0.689203in}{0.968404in}}%
\pgfpathlineto{\pgfqpoint{0.704859in}{0.958895in}}%
\pgfpathlineto{\pgfqpoint{0.706064in}{0.957870in}}%
\pgfpathlineto{\pgfqpoint{0.720516in}{0.946806in}}%
\pgfpathlineto{\pgfqpoint{0.723438in}{0.944259in}}%
\pgfpathlineto{\pgfqpoint{0.736173in}{0.933299in}}%
\pgfpathlineto{\pgfqpoint{0.739230in}{0.930648in}}%
\pgfpathlineto{\pgfqpoint{0.751829in}{0.918549in}}%
\pgfpathlineto{\pgfqpoint{0.753550in}{0.917036in}}%
\pgfpathlineto{\pgfqpoint{0.765205in}{0.903425in}}%
\pgfpathlineto{\pgfqpoint{0.767486in}{0.898361in}}%
\pgfpathlineto{\pgfqpoint{0.771993in}{0.889814in}}%
\pgfpathlineto{\pgfqpoint{0.771174in}{0.876203in}}%
\pgfpathlineto{\pgfqpoint{0.767486in}{0.870426in}}%
\pgfpathlineto{\pgfqpoint{0.763236in}{0.862592in}}%
\pgfpathlineto{\pgfqpoint{0.751829in}{0.850184in}}%
\pgfpathlineto{\pgfqpoint{0.750804in}{0.848981in}}%
\pgfpathlineto{\pgfqpoint{0.736243in}{0.835370in}}%
\pgfpathlineto{\pgfqpoint{0.736173in}{0.835309in}}%
\pgfpathlineto{\pgfqpoint{0.720516in}{0.821921in}}%
\pgfpathlineto{\pgfqpoint{0.720300in}{0.821759in}}%
\pgfpathlineto{\pgfqpoint{0.704859in}{0.809862in}}%
\pgfpathlineto{\pgfqpoint{0.701802in}{0.808148in}}%
\pgfpathlineto{\pgfqpoint{0.689203in}{0.800261in}}%
\pgfpathlineto{\pgfqpoint{0.673546in}{0.795820in}}%
\pgfpathlineto{\pgfqpoint{0.657890in}{0.797758in}}%
\pgfpathlineto{\pgfqpoint{0.642233in}{0.805582in}}%
\pgfpathlineto{\pgfqpoint{0.639000in}{0.808148in}}%
\pgfpathclose%
\pgfpathmoveto{\pgfqpoint{0.949265in}{0.808148in}}%
\pgfpathlineto{\pgfqpoint{0.939708in}{0.814453in}}%
\pgfpathlineto{\pgfqpoint{0.930904in}{0.821759in}}%
\pgfpathlineto{\pgfqpoint{0.924051in}{0.827281in}}%
\pgfpathlineto{\pgfqpoint{0.914747in}{0.835370in}}%
\pgfpathlineto{\pgfqpoint{0.908395in}{0.841327in}}%
\pgfpathlineto{\pgfqpoint{0.899991in}{0.848981in}}%
\pgfpathlineto{\pgfqpoint{0.892738in}{0.857290in}}%
\pgfpathlineto{\pgfqpoint{0.887531in}{0.862592in}}%
\pgfpathlineto{\pgfqpoint{0.879755in}{0.876203in}}%
\pgfpathlineto{\pgfqpoint{0.879009in}{0.889814in}}%
\pgfpathlineto{\pgfqpoint{0.885484in}{0.903425in}}%
\pgfpathlineto{\pgfqpoint{0.892738in}{0.911414in}}%
\pgfpathlineto{\pgfqpoint{0.897293in}{0.917036in}}%
\pgfpathlineto{\pgfqpoint{0.908395in}{0.927493in}}%
\pgfpathlineto{\pgfqpoint{0.911654in}{0.930648in}}%
\pgfpathlineto{\pgfqpoint{0.924051in}{0.941490in}}%
\pgfpathlineto{\pgfqpoint{0.927467in}{0.944259in}}%
\pgfpathlineto{\pgfqpoint{0.939708in}{0.954263in}}%
\pgfpathlineto{\pgfqpoint{0.945252in}{0.957870in}}%
\pgfpathlineto{\pgfqpoint{0.955364in}{0.965117in}}%
\pgfpathlineto{\pgfqpoint{0.970443in}{0.971481in}}%
\pgfpathlineto{\pgfqpoint{0.971021in}{0.971775in}}%
\pgfpathlineto{\pgfqpoint{0.986678in}{0.972518in}}%
\pgfpathlineto{\pgfqpoint{0.989143in}{0.971481in}}%
\pgfpathlineto{\pgfqpoint{1.002334in}{0.966858in}}%
\pgfpathlineto{\pgfqpoint{1.015900in}{0.957870in}}%
\pgfpathlineto{\pgfqpoint{1.017991in}{0.956607in}}%
\pgfpathlineto{\pgfqpoint{1.033615in}{0.944259in}}%
\pgfpathlineto{\pgfqpoint{1.033647in}{0.944233in}}%
\pgfpathlineto{\pgfqpoint{1.049275in}{0.930648in}}%
\pgfpathlineto{\pgfqpoint{1.049304in}{0.930620in}}%
\pgfpathlineto{\pgfqpoint{1.063508in}{0.917036in}}%
\pgfpathlineto{\pgfqpoint{1.064960in}{0.915218in}}%
\pgfpathlineto{\pgfqpoint{1.075299in}{0.903425in}}%
\pgfpathlineto{\pgfqpoint{1.080617in}{0.891957in}}%
\pgfpathlineto{\pgfqpoint{1.081810in}{0.889814in}}%
\pgfpathlineto{\pgfqpoint{1.080956in}{0.876203in}}%
\pgfpathlineto{\pgfqpoint{1.080617in}{0.875700in}}%
\pgfpathlineto{\pgfqpoint{1.073296in}{0.862592in}}%
\pgfpathlineto{\pgfqpoint{1.064960in}{0.853801in}}%
\pgfpathlineto{\pgfqpoint{1.060812in}{0.848981in}}%
\pgfpathlineto{\pgfqpoint{1.049304in}{0.838340in}}%
\pgfpathlineto{\pgfqpoint{1.046119in}{0.835370in}}%
\pgfpathlineto{\pgfqpoint{1.033647in}{0.824592in}}%
\pgfpathlineto{\pgfqpoint{1.030019in}{0.821759in}}%
\pgfpathlineto{\pgfqpoint{1.017991in}{0.812108in}}%
\pgfpathlineto{\pgfqpoint{1.011524in}{0.808147in}}%
\pgfpathlineto{\pgfqpoint{1.002334in}{0.801841in}}%
\pgfpathlineto{\pgfqpoint{0.986678in}{0.796212in}}%
\pgfpathlineto{\pgfqpoint{0.971021in}{0.796860in}}%
\pgfpathlineto{\pgfqpoint{0.955364in}{0.803621in}}%
\pgfpathlineto{\pgfqpoint{0.949265in}{0.808148in}}%
\pgfpathclose%
\pgfpathmoveto{\pgfqpoint{1.259306in}{0.808148in}}%
\pgfpathlineto{\pgfqpoint{1.252839in}{0.812108in}}%
\pgfpathlineto{\pgfqpoint{1.240811in}{0.821759in}}%
\pgfpathlineto{\pgfqpoint{1.237183in}{0.824592in}}%
\pgfpathlineto{\pgfqpoint{1.224711in}{0.835370in}}%
\pgfpathlineto{\pgfqpoint{1.221526in}{0.838340in}}%
\pgfpathlineto{\pgfqpoint{1.210018in}{0.848981in}}%
\pgfpathlineto{\pgfqpoint{1.205870in}{0.853801in}}%
\pgfpathlineto{\pgfqpoint{1.197534in}{0.862592in}}%
\pgfpathlineto{\pgfqpoint{1.190213in}{0.875700in}}%
\pgfpathlineto{\pgfqpoint{1.189874in}{0.876203in}}%
\pgfpathlineto{\pgfqpoint{1.189020in}{0.889814in}}%
\pgfpathlineto{\pgfqpoint{1.190213in}{0.891957in}}%
\pgfpathlineto{\pgfqpoint{1.195531in}{0.903425in}}%
\pgfpathlineto{\pgfqpoint{1.205870in}{0.915218in}}%
\pgfpathlineto{\pgfqpoint{1.207322in}{0.917036in}}%
\pgfpathlineto{\pgfqpoint{1.221526in}{0.930620in}}%
\pgfpathlineto{\pgfqpoint{1.221555in}{0.930648in}}%
\pgfpathlineto{\pgfqpoint{1.237183in}{0.944233in}}%
\pgfpathlineto{\pgfqpoint{1.237215in}{0.944259in}}%
\pgfpathlineto{\pgfqpoint{1.252839in}{0.956607in}}%
\pgfpathlineto{\pgfqpoint{1.254930in}{0.957870in}}%
\pgfpathlineto{\pgfqpoint{1.268496in}{0.966858in}}%
\pgfpathlineto{\pgfqpoint{1.281687in}{0.971481in}}%
\pgfpathlineto{\pgfqpoint{1.284152in}{0.972518in}}%
\pgfpathlineto{\pgfqpoint{1.299809in}{0.971775in}}%
\pgfpathlineto{\pgfqpoint{1.300387in}{0.971481in}}%
\pgfpathlineto{\pgfqpoint{1.315466in}{0.965117in}}%
\pgfpathlineto{\pgfqpoint{1.325578in}{0.957870in}}%
\pgfpathlineto{\pgfqpoint{1.331122in}{0.954263in}}%
\pgfpathlineto{\pgfqpoint{1.343363in}{0.944259in}}%
\pgfpathlineto{\pgfqpoint{1.346779in}{0.941490in}}%
\pgfpathlineto{\pgfqpoint{1.359176in}{0.930648in}}%
\pgfpathlineto{\pgfqpoint{1.362435in}{0.927493in}}%
\pgfpathlineto{\pgfqpoint{1.373537in}{0.917036in}}%
\pgfpathlineto{\pgfqpoint{1.378092in}{0.911414in}}%
\pgfpathlineto{\pgfqpoint{1.385346in}{0.903425in}}%
\pgfpathlineto{\pgfqpoint{1.391821in}{0.889814in}}%
\pgfpathlineto{\pgfqpoint{1.391075in}{0.876203in}}%
\pgfpathlineto{\pgfqpoint{1.383299in}{0.862592in}}%
\pgfpathlineto{\pgfqpoint{1.378092in}{0.857290in}}%
\pgfpathlineto{\pgfqpoint{1.370839in}{0.848981in}}%
\pgfpathlineto{\pgfqpoint{1.362435in}{0.841327in}}%
\pgfpathlineto{\pgfqpoint{1.356083in}{0.835370in}}%
\pgfpathlineto{\pgfqpoint{1.346779in}{0.827281in}}%
\pgfpathlineto{\pgfqpoint{1.339926in}{0.821759in}}%
\pgfpathlineto{\pgfqpoint{1.331122in}{0.814453in}}%
\pgfpathlineto{\pgfqpoint{1.321565in}{0.808148in}}%
\pgfpathlineto{\pgfqpoint{1.315466in}{0.803621in}}%
\pgfpathlineto{\pgfqpoint{1.299809in}{0.796860in}}%
\pgfpathlineto{\pgfqpoint{1.284152in}{0.796212in}}%
\pgfpathlineto{\pgfqpoint{1.268496in}{0.801841in}}%
\pgfpathlineto{\pgfqpoint{1.259306in}{0.808148in}}%
\pgfpathclose%
\pgfpathmoveto{\pgfqpoint{1.569028in}{0.808148in}}%
\pgfpathlineto{\pgfqpoint{1.565971in}{0.809862in}}%
\pgfpathlineto{\pgfqpoint{1.550530in}{0.821759in}}%
\pgfpathlineto{\pgfqpoint{1.550314in}{0.821921in}}%
\pgfpathlineto{\pgfqpoint{1.534657in}{0.835309in}}%
\pgfpathlineto{\pgfqpoint{1.534587in}{0.835370in}}%
\pgfpathlineto{\pgfqpoint{1.520026in}{0.848981in}}%
\pgfpathlineto{\pgfqpoint{1.519001in}{0.850184in}}%
\pgfpathlineto{\pgfqpoint{1.507594in}{0.862592in}}%
\pgfpathlineto{\pgfqpoint{1.503344in}{0.870426in}}%
\pgfpathlineto{\pgfqpoint{1.499656in}{0.876203in}}%
\pgfpathlineto{\pgfqpoint{1.498837in}{0.889814in}}%
\pgfpathlineto{\pgfqpoint{1.503344in}{0.898361in}}%
\pgfpathlineto{\pgfqpoint{1.505625in}{0.903425in}}%
\pgfpathlineto{\pgfqpoint{1.517280in}{0.917036in}}%
\pgfpathlineto{\pgfqpoint{1.519001in}{0.918549in}}%
\pgfpathlineto{\pgfqpoint{1.531600in}{0.930648in}}%
\pgfpathlineto{\pgfqpoint{1.534657in}{0.933299in}}%
\pgfpathlineto{\pgfqpoint{1.547392in}{0.944259in}}%
\pgfpathlineto{\pgfqpoint{1.550314in}{0.946806in}}%
\pgfpathlineto{\pgfqpoint{1.564766in}{0.957870in}}%
\pgfpathlineto{\pgfqpoint{1.565971in}{0.958895in}}%
\pgfpathlineto{\pgfqpoint{1.581627in}{0.968404in}}%
\pgfpathlineto{\pgfqpoint{1.592775in}{0.971481in}}%
\pgfpathlineto{\pgfqpoint{1.597284in}{0.972966in}}%
\pgfpathlineto{\pgfqpoint{1.607735in}{0.971481in}}%
\pgfpathlineto{\pgfqpoint{1.612940in}{0.970854in}}%
\pgfpathlineto{\pgfqpoint{1.628597in}{0.963197in}}%
\pgfpathlineto{\pgfqpoint{1.635549in}{0.957870in}}%
\pgfpathlineto{\pgfqpoint{1.644253in}{0.951837in}}%
\pgfpathlineto{\pgfqpoint{1.653276in}{0.944259in}}%
\pgfpathlineto{\pgfqpoint{1.659910in}{0.938744in}}%
\pgfpathlineto{\pgfqpoint{1.669154in}{0.930648in}}%
\pgfpathlineto{\pgfqpoint{1.675567in}{0.924433in}}%
\pgfpathlineto{\pgfqpoint{1.683570in}{0.917036in}}%
\pgfpathlineto{\pgfqpoint{1.691223in}{0.907757in}}%
\pgfpathlineto{\pgfqpoint{1.695314in}{0.903425in}}%
\pgfpathlineto{\pgfqpoint{1.701962in}{0.889814in}}%
\pgfpathlineto{\pgfqpoint{1.701197in}{0.876203in}}%
\pgfpathlineto{\pgfqpoint{1.693213in}{0.862592in}}%
\pgfpathlineto{\pgfqpoint{1.691223in}{0.860644in}}%
\pgfpathlineto{\pgfqpoint{1.680859in}{0.848981in}}%
\pgfpathlineto{\pgfqpoint{1.675567in}{0.844251in}}%
\pgfpathlineto{\pgfqpoint{1.666110in}{0.835370in}}%
\pgfpathlineto{\pgfqpoint{1.659910in}{0.829972in}}%
\pgfpathlineto{\pgfqpoint{1.649974in}{0.821759in}}%
\pgfpathlineto{\pgfqpoint{1.644253in}{0.816881in}}%
\pgfpathlineto{\pgfqpoint{1.631830in}{0.808148in}}%
\pgfpathlineto{\pgfqpoint{1.628597in}{0.805582in}}%
\pgfpathlineto{\pgfqpoint{1.612940in}{0.797758in}}%
\pgfpathlineto{\pgfqpoint{1.597284in}{0.795820in}}%
\pgfpathlineto{\pgfqpoint{1.581627in}{0.800261in}}%
\pgfpathlineto{\pgfqpoint{1.569028in}{0.808148in}}%
\pgfpathclose%
\pgfpathmoveto{\pgfqpoint{0.499345in}{0.903425in}}%
\pgfpathlineto{\pgfqpoint{0.485668in}{0.909564in}}%
\pgfpathlineto{\pgfqpoint{0.475779in}{0.917036in}}%
\pgfpathlineto{\pgfqpoint{0.470011in}{0.920273in}}%
\pgfpathlineto{\pgfqpoint{0.456420in}{0.930648in}}%
\pgfpathlineto{\pgfqpoint{0.454354in}{0.932050in}}%
\pgfpathlineto{\pgfqpoint{0.439353in}{0.944259in}}%
\pgfpathlineto{\pgfqpoint{0.438698in}{0.944797in}}%
\pgfpathlineto{\pgfqpoint{0.423913in}{0.957870in}}%
\pgfpathlineto{\pgfqpoint{0.423041in}{0.958736in}}%
\pgfpathlineto{\pgfqpoint{0.409714in}{0.971481in}}%
\pgfpathlineto{\pgfqpoint{0.407385in}{0.974277in}}%
\pgfpathlineto{\pgfqpoint{0.396607in}{0.985092in}}%
\pgfpathlineto{\pgfqpoint{0.391728in}{0.992220in}}%
\pgfpathlineto{\pgfqpoint{0.385061in}{0.998703in}}%
\pgfpathlineto{\pgfqpoint{0.377943in}{1.012314in}}%
\pgfpathlineto{\pgfqpoint{0.377943in}{1.025925in}}%
\pgfpathlineto{\pgfqpoint{0.385061in}{1.039536in}}%
\pgfpathlineto{\pgfqpoint{0.391728in}{1.046019in}}%
\pgfpathlineto{\pgfqpoint{0.396607in}{1.053148in}}%
\pgfpathlineto{\pgfqpoint{0.407385in}{1.063962in}}%
\pgfpathlineto{\pgfqpoint{0.409714in}{1.066759in}}%
\pgfpathlineto{\pgfqpoint{0.423041in}{1.079504in}}%
\pgfpathlineto{\pgfqpoint{0.423913in}{1.080370in}}%
\pgfpathlineto{\pgfqpoint{0.438698in}{1.093443in}}%
\pgfpathlineto{\pgfqpoint{0.439353in}{1.093981in}}%
\pgfpathlineto{\pgfqpoint{0.454354in}{1.106189in}}%
\pgfpathlineto{\pgfqpoint{0.456420in}{1.107592in}}%
\pgfpathlineto{\pgfqpoint{0.470011in}{1.117967in}}%
\pgfpathlineto{\pgfqpoint{0.475779in}{1.121203in}}%
\pgfpathlineto{\pgfqpoint{0.485668in}{1.128676in}}%
\pgfpathlineto{\pgfqpoint{0.499345in}{1.134814in}}%
\pgfpathlineto{\pgfqpoint{0.501324in}{1.136558in}}%
\pgfpathlineto{\pgfqpoint{0.516981in}{1.140582in}}%
\pgfpathlineto{\pgfqpoint{0.531947in}{1.134814in}}%
\pgfpathlineto{\pgfqpoint{0.532637in}{1.134681in}}%
\pgfpathlineto{\pgfqpoint{0.548294in}{1.126672in}}%
\pgfpathlineto{\pgfqpoint{0.555046in}{1.121203in}}%
\pgfpathlineto{\pgfqpoint{0.563950in}{1.115831in}}%
\pgfpathlineto{\pgfqpoint{0.574306in}{1.107592in}}%
\pgfpathlineto{\pgfqpoint{0.579607in}{1.103831in}}%
\pgfpathlineto{\pgfqpoint{0.591453in}{1.093981in}}%
\pgfpathlineto{\pgfqpoint{0.595263in}{1.090780in}}%
\pgfpathlineto{\pgfqpoint{0.607005in}{1.080370in}}%
\pgfpathlineto{\pgfqpoint{0.610920in}{1.076470in}}%
\pgfpathlineto{\pgfqpoint{0.621279in}{1.066759in}}%
\pgfpathlineto{\pgfqpoint{0.626577in}{1.060533in}}%
\pgfpathlineto{\pgfqpoint{0.634344in}{1.053148in}}%
\pgfpathlineto{\pgfqpoint{0.642233in}{1.042262in}}%
\pgfpathlineto{\pgfqpoint{0.645405in}{1.039536in}}%
\pgfpathlineto{\pgfqpoint{0.653542in}{1.025925in}}%
\pgfpathlineto{\pgfqpoint{0.653542in}{1.012314in}}%
\pgfpathlineto{\pgfqpoint{0.645405in}{0.998703in}}%
\pgfpathlineto{\pgfqpoint{0.642233in}{0.995978in}}%
\pgfpathlineto{\pgfqpoint{0.634344in}{0.985092in}}%
\pgfpathlineto{\pgfqpoint{0.626577in}{0.977706in}}%
\pgfpathlineto{\pgfqpoint{0.621279in}{0.971481in}}%
\pgfpathlineto{\pgfqpoint{0.610920in}{0.961770in}}%
\pgfpathlineto{\pgfqpoint{0.607005in}{0.957870in}}%
\pgfpathlineto{\pgfqpoint{0.595263in}{0.947460in}}%
\pgfpathlineto{\pgfqpoint{0.591453in}{0.944259in}}%
\pgfpathlineto{\pgfqpoint{0.579607in}{0.934409in}}%
\pgfpathlineto{\pgfqpoint{0.574306in}{0.930648in}}%
\pgfpathlineto{\pgfqpoint{0.563950in}{0.922408in}}%
\pgfpathlineto{\pgfqpoint{0.555046in}{0.917036in}}%
\pgfpathlineto{\pgfqpoint{0.548294in}{0.911567in}}%
\pgfpathlineto{\pgfqpoint{0.532637in}{0.903559in}}%
\pgfpathlineto{\pgfqpoint{0.531947in}{0.903425in}}%
\pgfpathlineto{\pgfqpoint{0.516981in}{0.897657in}}%
\pgfpathlineto{\pgfqpoint{0.501324in}{0.901681in}}%
\pgfpathlineto{\pgfqpoint{0.499345in}{0.903425in}}%
\pgfpathclose%
\pgfpathmoveto{\pgfqpoint{0.810042in}{0.903425in}}%
\pgfpathlineto{\pgfqpoint{0.798799in}{0.907761in}}%
\pgfpathlineto{\pgfqpoint{0.785540in}{0.917036in}}%
\pgfpathlineto{\pgfqpoint{0.783142in}{0.918278in}}%
\pgfpathlineto{\pgfqpoint{0.767486in}{0.929639in}}%
\pgfpathlineto{\pgfqpoint{0.766338in}{0.930648in}}%
\pgfpathlineto{\pgfqpoint{0.751829in}{0.942087in}}%
\pgfpathlineto{\pgfqpoint{0.749325in}{0.944259in}}%
\pgfpathlineto{\pgfqpoint{0.736173in}{0.955821in}}%
\pgfpathlineto{\pgfqpoint{0.733822in}{0.957870in}}%
\pgfpathlineto{\pgfqpoint{0.720516in}{0.970884in}}%
\pgfpathlineto{\pgfqpoint{0.719816in}{0.971481in}}%
\pgfpathlineto{\pgfqpoint{0.706890in}{0.985092in}}%
\pgfpathlineto{\pgfqpoint{0.704859in}{0.988205in}}%
\pgfpathlineto{\pgfqpoint{0.695128in}{0.998703in}}%
\pgfpathlineto{\pgfqpoint{0.689203in}{1.011392in}}%
\pgfpathlineto{\pgfqpoint{0.688216in}{1.012314in}}%
\pgfpathlineto{\pgfqpoint{0.688216in}{1.025925in}}%
\pgfpathlineto{\pgfqpoint{0.689203in}{1.026848in}}%
\pgfpathlineto{\pgfqpoint{0.695128in}{1.039536in}}%
\pgfpathlineto{\pgfqpoint{0.704859in}{1.050034in}}%
\pgfpathlineto{\pgfqpoint{0.706890in}{1.053148in}}%
\pgfpathlineto{\pgfqpoint{0.719816in}{1.066759in}}%
\pgfpathlineto{\pgfqpoint{0.720516in}{1.067356in}}%
\pgfpathlineto{\pgfqpoint{0.733822in}{1.080370in}}%
\pgfpathlineto{\pgfqpoint{0.736173in}{1.082419in}}%
\pgfpathlineto{\pgfqpoint{0.749325in}{1.093981in}}%
\pgfpathlineto{\pgfqpoint{0.751829in}{1.096153in}}%
\pgfpathlineto{\pgfqpoint{0.766338in}{1.107592in}}%
\pgfpathlineto{\pgfqpoint{0.767486in}{1.108601in}}%
\pgfpathlineto{\pgfqpoint{0.783142in}{1.119962in}}%
\pgfpathlineto{\pgfqpoint{0.785540in}{1.121203in}}%
\pgfpathlineto{\pgfqpoint{0.798799in}{1.130478in}}%
\pgfpathlineto{\pgfqpoint{0.810042in}{1.134814in}}%
\pgfpathlineto{\pgfqpoint{0.814455in}{1.138168in}}%
\pgfpathlineto{\pgfqpoint{0.830112in}{1.140180in}}%
\pgfpathlineto{\pgfqpoint{0.840596in}{1.134814in}}%
\pgfpathlineto{\pgfqpoint{0.845769in}{1.133481in}}%
\pgfpathlineto{\pgfqpoint{0.861425in}{1.124468in}}%
\pgfpathlineto{\pgfqpoint{0.865211in}{1.121203in}}%
\pgfpathlineto{\pgfqpoint{0.877082in}{1.113557in}}%
\pgfpathlineto{\pgfqpoint{0.884306in}{1.107592in}}%
\pgfpathlineto{\pgfqpoint{0.892738in}{1.101369in}}%
\pgfpathlineto{\pgfqpoint{0.901467in}{1.093981in}}%
\pgfpathlineto{\pgfqpoint{0.908395in}{1.088048in}}%
\pgfpathlineto{\pgfqpoint{0.917068in}{1.080370in}}%
\pgfpathlineto{\pgfqpoint{0.924051in}{1.073423in}}%
\pgfpathlineto{\pgfqpoint{0.931343in}{1.066759in}}%
\pgfpathlineto{\pgfqpoint{0.939708in}{1.057194in}}%
\pgfpathlineto{\pgfqpoint{0.944242in}{1.053148in}}%
\pgfpathlineto{\pgfqpoint{0.954809in}{1.039536in}}%
\pgfpathlineto{\pgfqpoint{0.955364in}{1.038126in}}%
\pgfpathlineto{\pgfqpoint{0.963899in}{1.025925in}}%
\pgfpathlineto{\pgfqpoint{0.963899in}{1.012314in}}%
\pgfpathlineto{\pgfqpoint{0.955364in}{1.000113in}}%
\pgfpathlineto{\pgfqpoint{0.954809in}{0.998703in}}%
\pgfpathlineto{\pgfqpoint{0.944242in}{0.985092in}}%
\pgfpathlineto{\pgfqpoint{0.939708in}{0.981046in}}%
\pgfpathlineto{\pgfqpoint{0.931343in}{0.971481in}}%
\pgfpathlineto{\pgfqpoint{0.924051in}{0.964816in}}%
\pgfpathlineto{\pgfqpoint{0.917068in}{0.957870in}}%
\pgfpathlineto{\pgfqpoint{0.908395in}{0.950191in}}%
\pgfpathlineto{\pgfqpoint{0.901467in}{0.944259in}}%
\pgfpathlineto{\pgfqpoint{0.892738in}{0.936871in}}%
\pgfpathlineto{\pgfqpoint{0.884306in}{0.930648in}}%
\pgfpathlineto{\pgfqpoint{0.877082in}{0.924682in}}%
\pgfpathlineto{\pgfqpoint{0.865211in}{0.917036in}}%
\pgfpathlineto{\pgfqpoint{0.861425in}{0.913771in}}%
\pgfpathlineto{\pgfqpoint{0.845769in}{0.904759in}}%
\pgfpathlineto{\pgfqpoint{0.840596in}{0.903425in}}%
\pgfpathlineto{\pgfqpoint{0.830112in}{0.898060in}}%
\pgfpathlineto{\pgfqpoint{0.814455in}{0.900071in}}%
\pgfpathlineto{\pgfqpoint{0.810042in}{0.903425in}}%
\pgfpathclose%
\pgfpathmoveto{\pgfqpoint{1.120423in}{0.903425in}}%
\pgfpathlineto{\pgfqpoint{1.111930in}{0.906159in}}%
\pgfpathlineto{\pgfqpoint{1.096274in}{0.916176in}}%
\pgfpathlineto{\pgfqpoint{1.095331in}{0.917036in}}%
\pgfpathlineto{\pgfqpoint{1.080617in}{0.927093in}}%
\pgfpathlineto{\pgfqpoint{1.076452in}{0.930648in}}%
\pgfpathlineto{\pgfqpoint{1.064960in}{0.939432in}}%
\pgfpathlineto{\pgfqpoint{1.059338in}{0.944259in}}%
\pgfpathlineto{\pgfqpoint{1.049304in}{0.952981in}}%
\pgfpathlineto{\pgfqpoint{1.043752in}{0.957870in}}%
\pgfpathlineto{\pgfqpoint{1.033647in}{0.967860in}}%
\pgfpathlineto{\pgfqpoint{1.029559in}{0.971481in}}%
\pgfpathlineto{\pgfqpoint{1.017991in}{0.984272in}}%
\pgfpathlineto{\pgfqpoint{1.017001in}{0.985092in}}%
\pgfpathlineto{\pgfqpoint{1.005479in}{0.998703in}}%
\pgfpathlineto{\pgfqpoint{1.002334in}{1.006086in}}%
\pgfpathlineto{\pgfqpoint{0.997088in}{1.012314in}}%
\pgfpathlineto{\pgfqpoint{0.997088in}{1.025925in}}%
\pgfpathlineto{\pgfqpoint{1.002334in}{1.032153in}}%
\pgfpathlineto{\pgfqpoint{1.005479in}{1.039536in}}%
\pgfpathlineto{\pgfqpoint{1.017001in}{1.053148in}}%
\pgfpathlineto{\pgfqpoint{1.017991in}{1.053967in}}%
\pgfpathlineto{\pgfqpoint{1.029559in}{1.066759in}}%
\pgfpathlineto{\pgfqpoint{1.033647in}{1.070379in}}%
\pgfpathlineto{\pgfqpoint{1.043752in}{1.080370in}}%
\pgfpathlineto{\pgfqpoint{1.049304in}{1.085258in}}%
\pgfpathlineto{\pgfqpoint{1.059338in}{1.093981in}}%
\pgfpathlineto{\pgfqpoint{1.064960in}{1.098808in}}%
\pgfpathlineto{\pgfqpoint{1.076452in}{1.107592in}}%
\pgfpathlineto{\pgfqpoint{1.080617in}{1.111146in}}%
\pgfpathlineto{\pgfqpoint{1.095331in}{1.121203in}}%
\pgfpathlineto{\pgfqpoint{1.096274in}{1.122063in}}%
\pgfpathlineto{\pgfqpoint{1.111930in}{1.132080in}}%
\pgfpathlineto{\pgfqpoint{1.120423in}{1.134814in}}%
\pgfpathlineto{\pgfqpoint{1.127587in}{1.139375in}}%
\pgfpathlineto{\pgfqpoint{1.143243in}{1.139375in}}%
\pgfpathlineto{\pgfqpoint{1.150407in}{1.134814in}}%
\pgfpathlineto{\pgfqpoint{1.158900in}{1.132080in}}%
\pgfpathlineto{\pgfqpoint{1.174556in}{1.122063in}}%
\pgfpathlineto{\pgfqpoint{1.175499in}{1.121203in}}%
\pgfpathlineto{\pgfqpoint{1.190213in}{1.111146in}}%
\pgfpathlineto{\pgfqpoint{1.194378in}{1.107592in}}%
\pgfpathlineto{\pgfqpoint{1.205870in}{1.098808in}}%
\pgfpathlineto{\pgfqpoint{1.211492in}{1.093981in}}%
\pgfpathlineto{\pgfqpoint{1.221526in}{1.085258in}}%
\pgfpathlineto{\pgfqpoint{1.227078in}{1.080370in}}%
\pgfpathlineto{\pgfqpoint{1.237183in}{1.070379in}}%
\pgfpathlineto{\pgfqpoint{1.241271in}{1.066759in}}%
\pgfpathlineto{\pgfqpoint{1.252839in}{1.053967in}}%
\pgfpathlineto{\pgfqpoint{1.253829in}{1.053148in}}%
\pgfpathlineto{\pgfqpoint{1.265351in}{1.039536in}}%
\pgfpathlineto{\pgfqpoint{1.268496in}{1.032153in}}%
\pgfpathlineto{\pgfqpoint{1.273742in}{1.025925in}}%
\pgfpathlineto{\pgfqpoint{1.273742in}{1.012314in}}%
\pgfpathlineto{\pgfqpoint{1.268496in}{1.006086in}}%
\pgfpathlineto{\pgfqpoint{1.265351in}{0.998703in}}%
\pgfpathlineto{\pgfqpoint{1.253829in}{0.985092in}}%
\pgfpathlineto{\pgfqpoint{1.252839in}{0.984272in}}%
\pgfpathlineto{\pgfqpoint{1.241271in}{0.971481in}}%
\pgfpathlineto{\pgfqpoint{1.237183in}{0.967860in}}%
\pgfpathlineto{\pgfqpoint{1.227078in}{0.957870in}}%
\pgfpathlineto{\pgfqpoint{1.221526in}{0.952981in}}%
\pgfpathlineto{\pgfqpoint{1.211492in}{0.944259in}}%
\pgfpathlineto{\pgfqpoint{1.205870in}{0.939432in}}%
\pgfpathlineto{\pgfqpoint{1.194378in}{0.930648in}}%
\pgfpathlineto{\pgfqpoint{1.190213in}{0.927093in}}%
\pgfpathlineto{\pgfqpoint{1.175499in}{0.917036in}}%
\pgfpathlineto{\pgfqpoint{1.174556in}{0.916176in}}%
\pgfpathlineto{\pgfqpoint{1.158900in}{0.906159in}}%
\pgfpathlineto{\pgfqpoint{1.150407in}{0.903425in}}%
\pgfpathlineto{\pgfqpoint{1.143243in}{0.898864in}}%
\pgfpathlineto{\pgfqpoint{1.127587in}{0.898864in}}%
\pgfpathlineto{\pgfqpoint{1.120423in}{0.903425in}}%
\pgfpathclose%
\pgfpathmoveto{\pgfqpoint{1.430234in}{0.903425in}}%
\pgfpathlineto{\pgfqpoint{1.425061in}{0.904759in}}%
\pgfpathlineto{\pgfqpoint{1.409405in}{0.913771in}}%
\pgfpathlineto{\pgfqpoint{1.405619in}{0.917036in}}%
\pgfpathlineto{\pgfqpoint{1.393748in}{0.924682in}}%
\pgfpathlineto{\pgfqpoint{1.386524in}{0.930648in}}%
\pgfpathlineto{\pgfqpoint{1.378092in}{0.936871in}}%
\pgfpathlineto{\pgfqpoint{1.369363in}{0.944259in}}%
\pgfpathlineto{\pgfqpoint{1.362435in}{0.950191in}}%
\pgfpathlineto{\pgfqpoint{1.353762in}{0.957870in}}%
\pgfpathlineto{\pgfqpoint{1.346779in}{0.964816in}}%
\pgfpathlineto{\pgfqpoint{1.339487in}{0.971481in}}%
\pgfpathlineto{\pgfqpoint{1.331122in}{0.981046in}}%
\pgfpathlineto{\pgfqpoint{1.326588in}{0.985092in}}%
\pgfpathlineto{\pgfqpoint{1.316021in}{0.998703in}}%
\pgfpathlineto{\pgfqpoint{1.315466in}{1.000113in}}%
\pgfpathlineto{\pgfqpoint{1.306931in}{1.012314in}}%
\pgfpathlineto{\pgfqpoint{1.306931in}{1.025925in}}%
\pgfpathlineto{\pgfqpoint{1.315466in}{1.038126in}}%
\pgfpathlineto{\pgfqpoint{1.316021in}{1.039536in}}%
\pgfpathlineto{\pgfqpoint{1.326588in}{1.053148in}}%
\pgfpathlineto{\pgfqpoint{1.331122in}{1.057194in}}%
\pgfpathlineto{\pgfqpoint{1.339487in}{1.066759in}}%
\pgfpathlineto{\pgfqpoint{1.346779in}{1.073423in}}%
\pgfpathlineto{\pgfqpoint{1.353762in}{1.080370in}}%
\pgfpathlineto{\pgfqpoint{1.362435in}{1.088048in}}%
\pgfpathlineto{\pgfqpoint{1.369363in}{1.093981in}}%
\pgfpathlineto{\pgfqpoint{1.378092in}{1.101369in}}%
\pgfpathlineto{\pgfqpoint{1.386524in}{1.107592in}}%
\pgfpathlineto{\pgfqpoint{1.393748in}{1.113557in}}%
\pgfpathlineto{\pgfqpoint{1.405619in}{1.121203in}}%
\pgfpathlineto{\pgfqpoint{1.409405in}{1.124468in}}%
\pgfpathlineto{\pgfqpoint{1.425061in}{1.133481in}}%
\pgfpathlineto{\pgfqpoint{1.430234in}{1.134814in}}%
\pgfpathlineto{\pgfqpoint{1.440718in}{1.140180in}}%
\pgfpathlineto{\pgfqpoint{1.456375in}{1.138168in}}%
\pgfpathlineto{\pgfqpoint{1.460788in}{1.134814in}}%
\pgfpathlineto{\pgfqpoint{1.472031in}{1.130478in}}%
\pgfpathlineto{\pgfqpoint{1.485290in}{1.121203in}}%
\pgfpathlineto{\pgfqpoint{1.487688in}{1.119962in}}%
\pgfpathlineto{\pgfqpoint{1.503344in}{1.108601in}}%
\pgfpathlineto{\pgfqpoint{1.504492in}{1.107592in}}%
\pgfpathlineto{\pgfqpoint{1.519001in}{1.096153in}}%
\pgfpathlineto{\pgfqpoint{1.521505in}{1.093981in}}%
\pgfpathlineto{\pgfqpoint{1.534657in}{1.082419in}}%
\pgfpathlineto{\pgfqpoint{1.537008in}{1.080370in}}%
\pgfpathlineto{\pgfqpoint{1.550314in}{1.067356in}}%
\pgfpathlineto{\pgfqpoint{1.551014in}{1.066759in}}%
\pgfpathlineto{\pgfqpoint{1.563940in}{1.053148in}}%
\pgfpathlineto{\pgfqpoint{1.565971in}{1.050034in}}%
\pgfpathlineto{\pgfqpoint{1.575702in}{1.039536in}}%
\pgfpathlineto{\pgfqpoint{1.581627in}{1.026848in}}%
\pgfpathlineto{\pgfqpoint{1.582614in}{1.025925in}}%
\pgfpathlineto{\pgfqpoint{1.582614in}{1.012314in}}%
\pgfpathlineto{\pgfqpoint{1.581627in}{1.011392in}}%
\pgfpathlineto{\pgfqpoint{1.575702in}{0.998703in}}%
\pgfpathlineto{\pgfqpoint{1.565971in}{0.988205in}}%
\pgfpathlineto{\pgfqpoint{1.563940in}{0.985092in}}%
\pgfpathlineto{\pgfqpoint{1.551014in}{0.971481in}}%
\pgfpathlineto{\pgfqpoint{1.550314in}{0.970884in}}%
\pgfpathlineto{\pgfqpoint{1.537008in}{0.957870in}}%
\pgfpathlineto{\pgfqpoint{1.534657in}{0.955821in}}%
\pgfpathlineto{\pgfqpoint{1.521505in}{0.944259in}}%
\pgfpathlineto{\pgfqpoint{1.519001in}{0.942087in}}%
\pgfpathlineto{\pgfqpoint{1.504492in}{0.930648in}}%
\pgfpathlineto{\pgfqpoint{1.503344in}{0.929639in}}%
\pgfpathlineto{\pgfqpoint{1.487688in}{0.918278in}}%
\pgfpathlineto{\pgfqpoint{1.485290in}{0.917036in}}%
\pgfpathlineto{\pgfqpoint{1.472031in}{0.907761in}}%
\pgfpathlineto{\pgfqpoint{1.460788in}{0.903425in}}%
\pgfpathlineto{\pgfqpoint{1.456375in}{0.900071in}}%
\pgfpathlineto{\pgfqpoint{1.440718in}{0.898060in}}%
\pgfpathlineto{\pgfqpoint{1.430234in}{0.903425in}}%
\pgfpathclose%
\pgfpathmoveto{\pgfqpoint{1.738883in}{0.903425in}}%
\pgfpathlineto{\pgfqpoint{1.738193in}{0.903559in}}%
\pgfpathlineto{\pgfqpoint{1.722536in}{0.911567in}}%
\pgfpathlineto{\pgfqpoint{1.715784in}{0.917036in}}%
\pgfpathlineto{\pgfqpoint{1.706880in}{0.922408in}}%
\pgfpathlineto{\pgfqpoint{1.696524in}{0.930648in}}%
\pgfpathlineto{\pgfqpoint{1.691223in}{0.934409in}}%
\pgfpathlineto{\pgfqpoint{1.679377in}{0.944259in}}%
\pgfpathlineto{\pgfqpoint{1.675567in}{0.947460in}}%
\pgfpathlineto{\pgfqpoint{1.663825in}{0.957870in}}%
\pgfpathlineto{\pgfqpoint{1.659910in}{0.961770in}}%
\pgfpathlineto{\pgfqpoint{1.649551in}{0.971481in}}%
\pgfpathlineto{\pgfqpoint{1.644253in}{0.977706in}}%
\pgfpathlineto{\pgfqpoint{1.636486in}{0.985092in}}%
\pgfpathlineto{\pgfqpoint{1.628597in}{0.995978in}}%
\pgfpathlineto{\pgfqpoint{1.625425in}{0.998703in}}%
\pgfpathlineto{\pgfqpoint{1.617288in}{1.012314in}}%
\pgfpathlineto{\pgfqpoint{1.617288in}{1.025925in}}%
\pgfpathlineto{\pgfqpoint{1.625425in}{1.039536in}}%
\pgfpathlineto{\pgfqpoint{1.628597in}{1.042262in}}%
\pgfpathlineto{\pgfqpoint{1.636486in}{1.053148in}}%
\pgfpathlineto{\pgfqpoint{1.644253in}{1.060533in}}%
\pgfpathlineto{\pgfqpoint{1.649551in}{1.066759in}}%
\pgfpathlineto{\pgfqpoint{1.659910in}{1.076470in}}%
\pgfpathlineto{\pgfqpoint{1.663825in}{1.080370in}}%
\pgfpathlineto{\pgfqpoint{1.675567in}{1.090780in}}%
\pgfpathlineto{\pgfqpoint{1.679377in}{1.093981in}}%
\pgfpathlineto{\pgfqpoint{1.691223in}{1.103831in}}%
\pgfpathlineto{\pgfqpoint{1.696524in}{1.107592in}}%
\pgfpathlineto{\pgfqpoint{1.706880in}{1.115831in}}%
\pgfpathlineto{\pgfqpoint{1.715784in}{1.121203in}}%
\pgfpathlineto{\pgfqpoint{1.722536in}{1.126672in}}%
\pgfpathlineto{\pgfqpoint{1.738193in}{1.134681in}}%
\pgfpathlineto{\pgfqpoint{1.738883in}{1.134814in}}%
\pgfpathlineto{\pgfqpoint{1.753849in}{1.140582in}}%
\pgfpathlineto{\pgfqpoint{1.769506in}{1.136558in}}%
\pgfpathlineto{\pgfqpoint{1.771485in}{1.134814in}}%
\pgfpathlineto{\pgfqpoint{1.785162in}{1.128676in}}%
\pgfpathlineto{\pgfqpoint{1.795051in}{1.121203in}}%
\pgfpathlineto{\pgfqpoint{1.800819in}{1.117967in}}%
\pgfpathlineto{\pgfqpoint{1.814410in}{1.107592in}}%
\pgfpathlineto{\pgfqpoint{1.816476in}{1.106189in}}%
\pgfpathlineto{\pgfqpoint{1.831477in}{1.093981in}}%
\pgfpathlineto{\pgfqpoint{1.832132in}{1.093443in}}%
\pgfpathlineto{\pgfqpoint{1.846917in}{1.080370in}}%
\pgfpathlineto{\pgfqpoint{1.847789in}{1.079504in}}%
\pgfpathlineto{\pgfqpoint{1.861116in}{1.066759in}}%
\pgfpathlineto{\pgfqpoint{1.863445in}{1.063962in}}%
\pgfpathlineto{\pgfqpoint{1.874223in}{1.053148in}}%
\pgfpathlineto{\pgfqpoint{1.879102in}{1.046019in}}%
\pgfpathlineto{\pgfqpoint{1.885769in}{1.039536in}}%
\pgfpathlineto{\pgfqpoint{1.892887in}{1.025925in}}%
\pgfpathlineto{\pgfqpoint{1.892887in}{1.012314in}}%
\pgfpathlineto{\pgfqpoint{1.885769in}{0.998703in}}%
\pgfpathlineto{\pgfqpoint{1.879102in}{0.992220in}}%
\pgfpathlineto{\pgfqpoint{1.874223in}{0.985092in}}%
\pgfpathlineto{\pgfqpoint{1.863445in}{0.974277in}}%
\pgfpathlineto{\pgfqpoint{1.861116in}{0.971481in}}%
\pgfpathlineto{\pgfqpoint{1.847789in}{0.958736in}}%
\pgfpathlineto{\pgfqpoint{1.846917in}{0.957870in}}%
\pgfpathlineto{\pgfqpoint{1.832132in}{0.944797in}}%
\pgfpathlineto{\pgfqpoint{1.831477in}{0.944259in}}%
\pgfpathlineto{\pgfqpoint{1.816476in}{0.932050in}}%
\pgfpathlineto{\pgfqpoint{1.814410in}{0.930648in}}%
\pgfpathlineto{\pgfqpoint{1.800819in}{0.920273in}}%
\pgfpathlineto{\pgfqpoint{1.795051in}{0.917036in}}%
\pgfpathlineto{\pgfqpoint{1.785162in}{0.909564in}}%
\pgfpathlineto{\pgfqpoint{1.771485in}{0.903425in}}%
\pgfpathlineto{\pgfqpoint{1.769506in}{0.901681in}}%
\pgfpathlineto{\pgfqpoint{1.753849in}{0.897657in}}%
\pgfpathlineto{\pgfqpoint{1.738883in}{0.903425in}}%
\pgfpathclose%
\pgfpathmoveto{\pgfqpoint{0.663095in}{1.066759in}}%
\pgfpathlineto{\pgfqpoint{0.657890in}{1.067385in}}%
\pgfpathlineto{\pgfqpoint{0.642233in}{1.075042in}}%
\pgfpathlineto{\pgfqpoint{0.635281in}{1.080370in}}%
\pgfpathlineto{\pgfqpoint{0.626577in}{1.086403in}}%
\pgfpathlineto{\pgfqpoint{0.617554in}{1.093981in}}%
\pgfpathlineto{\pgfqpoint{0.610920in}{1.099495in}}%
\pgfpathlineto{\pgfqpoint{0.601676in}{1.107592in}}%
\pgfpathlineto{\pgfqpoint{0.595263in}{1.113807in}}%
\pgfpathlineto{\pgfqpoint{0.587260in}{1.121203in}}%
\pgfpathlineto{\pgfqpoint{0.579607in}{1.130482in}}%
\pgfpathlineto{\pgfqpoint{0.575516in}{1.134814in}}%
\pgfpathlineto{\pgfqpoint{0.568868in}{1.148425in}}%
\pgfpathlineto{\pgfqpoint{0.569633in}{1.162036in}}%
\pgfpathlineto{\pgfqpoint{0.577617in}{1.175647in}}%
\pgfpathlineto{\pgfqpoint{0.579607in}{1.177596in}}%
\pgfpathlineto{\pgfqpoint{0.589971in}{1.189259in}}%
\pgfpathlineto{\pgfqpoint{0.595263in}{1.193988in}}%
\pgfpathlineto{\pgfqpoint{0.604720in}{1.202870in}}%
\pgfpathlineto{\pgfqpoint{0.610920in}{1.208267in}}%
\pgfpathlineto{\pgfqpoint{0.620856in}{1.216481in}}%
\pgfpathlineto{\pgfqpoint{0.626577in}{1.221358in}}%
\pgfpathlineto{\pgfqpoint{0.639000in}{1.230092in}}%
\pgfpathlineto{\pgfqpoint{0.642233in}{1.232658in}}%
\pgfpathlineto{\pgfqpoint{0.657890in}{1.240482in}}%
\pgfpathlineto{\pgfqpoint{0.673546in}{1.242419in}}%
\pgfpathlineto{\pgfqpoint{0.689203in}{1.237978in}}%
\pgfpathlineto{\pgfqpoint{0.701802in}{1.230092in}}%
\pgfpathlineto{\pgfqpoint{0.704859in}{1.228377in}}%
\pgfpathlineto{\pgfqpoint{0.720300in}{1.216481in}}%
\pgfpathlineto{\pgfqpoint{0.720516in}{1.216318in}}%
\pgfpathlineto{\pgfqpoint{0.736173in}{1.202930in}}%
\pgfpathlineto{\pgfqpoint{0.736243in}{1.202870in}}%
\pgfpathlineto{\pgfqpoint{0.750804in}{1.189259in}}%
\pgfpathlineto{\pgfqpoint{0.751829in}{1.188055in}}%
\pgfpathlineto{\pgfqpoint{0.763236in}{1.175647in}}%
\pgfpathlineto{\pgfqpoint{0.767486in}{1.167813in}}%
\pgfpathlineto{\pgfqpoint{0.771174in}{1.162036in}}%
\pgfpathlineto{\pgfqpoint{0.771993in}{1.148425in}}%
\pgfpathlineto{\pgfqpoint{0.767486in}{1.139879in}}%
\pgfpathlineto{\pgfqpoint{0.765205in}{1.134814in}}%
\pgfpathlineto{\pgfqpoint{0.753550in}{1.121203in}}%
\pgfpathlineto{\pgfqpoint{0.751829in}{1.119690in}}%
\pgfpathlineto{\pgfqpoint{0.739230in}{1.107592in}}%
\pgfpathlineto{\pgfqpoint{0.736173in}{1.104940in}}%
\pgfpathlineto{\pgfqpoint{0.723438in}{1.093981in}}%
\pgfpathlineto{\pgfqpoint{0.720516in}{1.091433in}}%
\pgfpathlineto{\pgfqpoint{0.706064in}{1.080370in}}%
\pgfpathlineto{\pgfqpoint{0.704859in}{1.079344in}}%
\pgfpathlineto{\pgfqpoint{0.689203in}{1.069835in}}%
\pgfpathlineto{\pgfqpoint{0.678055in}{1.066759in}}%
\pgfpathlineto{\pgfqpoint{0.673546in}{1.065273in}}%
\pgfpathlineto{\pgfqpoint{0.663095in}{1.066759in}}%
\pgfpathclose%
\pgfpathmoveto{\pgfqpoint{0.970443in}{1.066759in}}%
\pgfpathlineto{\pgfqpoint{0.955364in}{1.073123in}}%
\pgfpathlineto{\pgfqpoint{0.945252in}{1.080370in}}%
\pgfpathlineto{\pgfqpoint{0.939708in}{1.083976in}}%
\pgfpathlineto{\pgfqpoint{0.927467in}{1.093981in}}%
\pgfpathlineto{\pgfqpoint{0.924051in}{1.096749in}}%
\pgfpathlineto{\pgfqpoint{0.911654in}{1.107592in}}%
\pgfpathlineto{\pgfqpoint{0.908395in}{1.110746in}}%
\pgfpathlineto{\pgfqpoint{0.897293in}{1.121203in}}%
\pgfpathlineto{\pgfqpoint{0.892738in}{1.126825in}}%
\pgfpathlineto{\pgfqpoint{0.885484in}{1.134814in}}%
\pgfpathlineto{\pgfqpoint{0.879009in}{1.148425in}}%
\pgfpathlineto{\pgfqpoint{0.879755in}{1.162036in}}%
\pgfpathlineto{\pgfqpoint{0.887531in}{1.175647in}}%
\pgfpathlineto{\pgfqpoint{0.892738in}{1.180950in}}%
\pgfpathlineto{\pgfqpoint{0.899991in}{1.189259in}}%
\pgfpathlineto{\pgfqpoint{0.908395in}{1.196912in}}%
\pgfpathlineto{\pgfqpoint{0.914747in}{1.202870in}}%
\pgfpathlineto{\pgfqpoint{0.924051in}{1.210959in}}%
\pgfpathlineto{\pgfqpoint{0.930904in}{1.216481in}}%
\pgfpathlineto{\pgfqpoint{0.939708in}{1.223786in}}%
\pgfpathlineto{\pgfqpoint{0.949265in}{1.230092in}}%
\pgfpathlineto{\pgfqpoint{0.955364in}{1.234619in}}%
\pgfpathlineto{\pgfqpoint{0.971021in}{1.241379in}}%
\pgfpathlineto{\pgfqpoint{0.986678in}{1.242027in}}%
\pgfpathlineto{\pgfqpoint{1.002334in}{1.236398in}}%
\pgfpathlineto{\pgfqpoint{1.011524in}{1.230092in}}%
\pgfpathlineto{\pgfqpoint{1.017991in}{1.226132in}}%
\pgfpathlineto{\pgfqpoint{1.030019in}{1.216481in}}%
\pgfpathlineto{\pgfqpoint{1.033647in}{1.213647in}}%
\pgfpathlineto{\pgfqpoint{1.046119in}{1.202870in}}%
\pgfpathlineto{\pgfqpoint{1.049304in}{1.199900in}}%
\pgfpathlineto{\pgfqpoint{1.060812in}{1.189259in}}%
\pgfpathlineto{\pgfqpoint{1.064960in}{1.184439in}}%
\pgfpathlineto{\pgfqpoint{1.073296in}{1.175647in}}%
\pgfpathlineto{\pgfqpoint{1.080617in}{1.162539in}}%
\pgfpathlineto{\pgfqpoint{1.080956in}{1.162036in}}%
\pgfpathlineto{\pgfqpoint{1.081810in}{1.148425in}}%
\pgfpathlineto{\pgfqpoint{1.080617in}{1.146282in}}%
\pgfpathlineto{\pgfqpoint{1.075299in}{1.134814in}}%
\pgfpathlineto{\pgfqpoint{1.064960in}{1.123021in}}%
\pgfpathlineto{\pgfqpoint{1.063508in}{1.121203in}}%
\pgfpathlineto{\pgfqpoint{1.049304in}{1.107620in}}%
\pgfpathlineto{\pgfqpoint{1.049275in}{1.107592in}}%
\pgfpathlineto{\pgfqpoint{1.033647in}{1.094006in}}%
\pgfpathlineto{\pgfqpoint{1.033615in}{1.093981in}}%
\pgfpathlineto{\pgfqpoint{1.017991in}{1.081632in}}%
\pgfpathlineto{\pgfqpoint{1.015900in}{1.080370in}}%
\pgfpathlineto{\pgfqpoint{1.002334in}{1.071382in}}%
\pgfpathlineto{\pgfqpoint{0.989143in}{1.066759in}}%
\pgfpathlineto{\pgfqpoint{0.986678in}{1.065722in}}%
\pgfpathlineto{\pgfqpoint{0.971021in}{1.066464in}}%
\pgfpathlineto{\pgfqpoint{0.970443in}{1.066759in}}%
\pgfpathclose%
\pgfpathmoveto{\pgfqpoint{1.281687in}{1.066759in}}%
\pgfpathlineto{\pgfqpoint{1.268496in}{1.071382in}}%
\pgfpathlineto{\pgfqpoint{1.254930in}{1.080370in}}%
\pgfpathlineto{\pgfqpoint{1.252839in}{1.081632in}}%
\pgfpathlineto{\pgfqpoint{1.237215in}{1.093981in}}%
\pgfpathlineto{\pgfqpoint{1.237183in}{1.094006in}}%
\pgfpathlineto{\pgfqpoint{1.221555in}{1.107592in}}%
\pgfpathlineto{\pgfqpoint{1.221526in}{1.107620in}}%
\pgfpathlineto{\pgfqpoint{1.207322in}{1.121203in}}%
\pgfpathlineto{\pgfqpoint{1.205870in}{1.123021in}}%
\pgfpathlineto{\pgfqpoint{1.195531in}{1.134814in}}%
\pgfpathlineto{\pgfqpoint{1.190213in}{1.146282in}}%
\pgfpathlineto{\pgfqpoint{1.189020in}{1.148425in}}%
\pgfpathlineto{\pgfqpoint{1.189874in}{1.162036in}}%
\pgfpathlineto{\pgfqpoint{1.190213in}{1.162539in}}%
\pgfpathlineto{\pgfqpoint{1.197534in}{1.175647in}}%
\pgfpathlineto{\pgfqpoint{1.205870in}{1.184439in}}%
\pgfpathlineto{\pgfqpoint{1.210018in}{1.189259in}}%
\pgfpathlineto{\pgfqpoint{1.221526in}{1.199900in}}%
\pgfpathlineto{\pgfqpoint{1.224711in}{1.202870in}}%
\pgfpathlineto{\pgfqpoint{1.237183in}{1.213647in}}%
\pgfpathlineto{\pgfqpoint{1.240811in}{1.216481in}}%
\pgfpathlineto{\pgfqpoint{1.252839in}{1.226132in}}%
\pgfpathlineto{\pgfqpoint{1.259306in}{1.230092in}}%
\pgfpathlineto{\pgfqpoint{1.268496in}{1.236398in}}%
\pgfpathlineto{\pgfqpoint{1.284152in}{1.242027in}}%
\pgfpathlineto{\pgfqpoint{1.299809in}{1.241379in}}%
\pgfpathlineto{\pgfqpoint{1.315466in}{1.234619in}}%
\pgfpathlineto{\pgfqpoint{1.321565in}{1.230092in}}%
\pgfpathlineto{\pgfqpoint{1.331122in}{1.223786in}}%
\pgfpathlineto{\pgfqpoint{1.339926in}{1.216481in}}%
\pgfpathlineto{\pgfqpoint{1.346779in}{1.210959in}}%
\pgfpathlineto{\pgfqpoint{1.356083in}{1.202870in}}%
\pgfpathlineto{\pgfqpoint{1.362435in}{1.196912in}}%
\pgfpathlineto{\pgfqpoint{1.370839in}{1.189259in}}%
\pgfpathlineto{\pgfqpoint{1.378092in}{1.180950in}}%
\pgfpathlineto{\pgfqpoint{1.383299in}{1.175647in}}%
\pgfpathlineto{\pgfqpoint{1.391075in}{1.162036in}}%
\pgfpathlineto{\pgfqpoint{1.391821in}{1.148425in}}%
\pgfpathlineto{\pgfqpoint{1.385346in}{1.134814in}}%
\pgfpathlineto{\pgfqpoint{1.378092in}{1.126825in}}%
\pgfpathlineto{\pgfqpoint{1.373537in}{1.121203in}}%
\pgfpathlineto{\pgfqpoint{1.362435in}{1.110746in}}%
\pgfpathlineto{\pgfqpoint{1.359176in}{1.107592in}}%
\pgfpathlineto{\pgfqpoint{1.346779in}{1.096749in}}%
\pgfpathlineto{\pgfqpoint{1.343363in}{1.093981in}}%
\pgfpathlineto{\pgfqpoint{1.331122in}{1.083976in}}%
\pgfpathlineto{\pgfqpoint{1.325578in}{1.080370in}}%
\pgfpathlineto{\pgfqpoint{1.315466in}{1.073123in}}%
\pgfpathlineto{\pgfqpoint{1.300387in}{1.066759in}}%
\pgfpathlineto{\pgfqpoint{1.299809in}{1.066464in}}%
\pgfpathlineto{\pgfqpoint{1.284152in}{1.065722in}}%
\pgfpathlineto{\pgfqpoint{1.281687in}{1.066759in}}%
\pgfpathclose%
\pgfpathmoveto{\pgfqpoint{1.592775in}{1.066759in}}%
\pgfpathlineto{\pgfqpoint{1.581627in}{1.069835in}}%
\pgfpathlineto{\pgfqpoint{1.565971in}{1.079344in}}%
\pgfpathlineto{\pgfqpoint{1.564766in}{1.080370in}}%
\pgfpathlineto{\pgfqpoint{1.550314in}{1.091433in}}%
\pgfpathlineto{\pgfqpoint{1.547392in}{1.093981in}}%
\pgfpathlineto{\pgfqpoint{1.534657in}{1.104940in}}%
\pgfpathlineto{\pgfqpoint{1.531600in}{1.107592in}}%
\pgfpathlineto{\pgfqpoint{1.519001in}{1.119690in}}%
\pgfpathlineto{\pgfqpoint{1.517280in}{1.121203in}}%
\pgfpathlineto{\pgfqpoint{1.505625in}{1.134814in}}%
\pgfpathlineto{\pgfqpoint{1.503344in}{1.139879in}}%
\pgfpathlineto{\pgfqpoint{1.498837in}{1.148425in}}%
\pgfpathlineto{\pgfqpoint{1.499656in}{1.162036in}}%
\pgfpathlineto{\pgfqpoint{1.503344in}{1.167813in}}%
\pgfpathlineto{\pgfqpoint{1.507594in}{1.175647in}}%
\pgfpathlineto{\pgfqpoint{1.519001in}{1.188055in}}%
\pgfpathlineto{\pgfqpoint{1.520026in}{1.189259in}}%
\pgfpathlineto{\pgfqpoint{1.534587in}{1.202870in}}%
\pgfpathlineto{\pgfqpoint{1.534657in}{1.202930in}}%
\pgfpathlineto{\pgfqpoint{1.550314in}{1.216318in}}%
\pgfpathlineto{\pgfqpoint{1.550530in}{1.216481in}}%
\pgfpathlineto{\pgfqpoint{1.565971in}{1.228377in}}%
\pgfpathlineto{\pgfqpoint{1.569028in}{1.230092in}}%
\pgfpathlineto{\pgfqpoint{1.581627in}{1.237978in}}%
\pgfpathlineto{\pgfqpoint{1.597284in}{1.242419in}}%
\pgfpathlineto{\pgfqpoint{1.612940in}{1.240482in}}%
\pgfpathlineto{\pgfqpoint{1.628597in}{1.232658in}}%
\pgfpathlineto{\pgfqpoint{1.631830in}{1.230092in}}%
\pgfpathlineto{\pgfqpoint{1.644253in}{1.221358in}}%
\pgfpathlineto{\pgfqpoint{1.649974in}{1.216481in}}%
\pgfpathlineto{\pgfqpoint{1.659910in}{1.208267in}}%
\pgfpathlineto{\pgfqpoint{1.666110in}{1.202870in}}%
\pgfpathlineto{\pgfqpoint{1.675567in}{1.193988in}}%
\pgfpathlineto{\pgfqpoint{1.680859in}{1.189259in}}%
\pgfpathlineto{\pgfqpoint{1.691223in}{1.177596in}}%
\pgfpathlineto{\pgfqpoint{1.693213in}{1.175647in}}%
\pgfpathlineto{\pgfqpoint{1.701197in}{1.162036in}}%
\pgfpathlineto{\pgfqpoint{1.701962in}{1.148425in}}%
\pgfpathlineto{\pgfqpoint{1.695314in}{1.134814in}}%
\pgfpathlineto{\pgfqpoint{1.691223in}{1.130482in}}%
\pgfpathlineto{\pgfqpoint{1.683570in}{1.121203in}}%
\pgfpathlineto{\pgfqpoint{1.675567in}{1.113807in}}%
\pgfpathlineto{\pgfqpoint{1.669154in}{1.107592in}}%
\pgfpathlineto{\pgfqpoint{1.659910in}{1.099495in}}%
\pgfpathlineto{\pgfqpoint{1.653276in}{1.093981in}}%
\pgfpathlineto{\pgfqpoint{1.644253in}{1.086403in}}%
\pgfpathlineto{\pgfqpoint{1.635549in}{1.080370in}}%
\pgfpathlineto{\pgfqpoint{1.628597in}{1.075042in}}%
\pgfpathlineto{\pgfqpoint{1.612940in}{1.067385in}}%
\pgfpathlineto{\pgfqpoint{1.607735in}{1.066759in}}%
\pgfpathlineto{\pgfqpoint{1.597284in}{1.065273in}}%
\pgfpathlineto{\pgfqpoint{1.592775in}{1.066759in}}%
\pgfpathclose%
\pgfpathmoveto{\pgfqpoint{0.494355in}{1.175647in}}%
\pgfpathlineto{\pgfqpoint{0.485668in}{1.179252in}}%
\pgfpathlineto{\pgfqpoint{0.471402in}{1.189259in}}%
\pgfpathlineto{\pgfqpoint{0.470011in}{1.190013in}}%
\pgfpathlineto{\pgfqpoint{0.454354in}{1.201521in}}%
\pgfpathlineto{\pgfqpoint{0.452809in}{1.202870in}}%
\pgfpathlineto{\pgfqpoint{0.438698in}{1.214231in}}%
\pgfpathlineto{\pgfqpoint{0.436107in}{1.216481in}}%
\pgfpathlineto{\pgfqpoint{0.423041in}{1.228224in}}%
\pgfpathlineto{\pgfqpoint{0.420897in}{1.230092in}}%
\pgfpathlineto{\pgfqpoint{0.407385in}{1.243564in}}%
\pgfpathlineto{\pgfqpoint{0.407219in}{1.243703in}}%
\pgfpathlineto{\pgfqpoint{0.394408in}{1.257314in}}%
\pgfpathlineto{\pgfqpoint{0.391728in}{1.261643in}}%
\pgfpathlineto{\pgfqpoint{0.383067in}{1.270925in}}%
\pgfpathlineto{\pgfqpoint{0.377374in}{1.284536in}}%
\pgfpathlineto{\pgfqpoint{0.378796in}{1.298148in}}%
\pgfpathlineto{\pgfqpoint{0.387342in}{1.311759in}}%
\pgfpathlineto{\pgfqpoint{0.391728in}{1.315671in}}%
\pgfpathlineto{\pgfqpoint{0.398989in}{1.325370in}}%
\pgfpathlineto{\pgfqpoint{0.407385in}{1.333375in}}%
\pgfpathlineto{\pgfqpoint{0.412313in}{1.338981in}}%
\pgfpathlineto{\pgfqpoint{0.423041in}{1.348948in}}%
\pgfpathlineto{\pgfqpoint{0.426825in}{1.352592in}}%
\pgfpathlineto{\pgfqpoint{0.438698in}{1.362981in}}%
\pgfpathlineto{\pgfqpoint{0.442666in}{1.366203in}}%
\pgfpathlineto{\pgfqpoint{0.454354in}{1.375812in}}%
\pgfpathlineto{\pgfqpoint{0.460185in}{1.379814in}}%
\pgfpathlineto{\pgfqpoint{0.470011in}{1.387607in}}%
\pgfpathlineto{\pgfqpoint{0.479970in}{1.393425in}}%
\pgfpathlineto{\pgfqpoint{0.485668in}{1.398126in}}%
\pgfpathlineto{\pgfqpoint{0.501324in}{1.405860in}}%
\pgfpathlineto{\pgfqpoint{0.509646in}{1.407036in}}%
\pgfpathlineto{\pgfqpoint{0.516981in}{1.409438in}}%
\pgfpathlineto{\pgfqpoint{0.521887in}{1.407036in}}%
\pgfpathlineto{\pgfqpoint{0.532637in}{1.404756in}}%
\pgfpathlineto{\pgfqpoint{0.548294in}{1.395914in}}%
\pgfpathlineto{\pgfqpoint{0.551108in}{1.393425in}}%
\pgfpathlineto{\pgfqpoint{0.563950in}{1.385358in}}%
\pgfpathlineto{\pgfqpoint{0.570657in}{1.379814in}}%
\pgfpathlineto{\pgfqpoint{0.579607in}{1.373395in}}%
\pgfpathlineto{\pgfqpoint{0.588169in}{1.366203in}}%
\pgfpathlineto{\pgfqpoint{0.595263in}{1.360309in}}%
\pgfpathlineto{\pgfqpoint{0.604059in}{1.352592in}}%
\pgfpathlineto{\pgfqpoint{0.610920in}{1.345965in}}%
\pgfpathlineto{\pgfqpoint{0.618590in}{1.338981in}}%
\pgfpathlineto{\pgfqpoint{0.626577in}{1.330087in}}%
\pgfpathlineto{\pgfqpoint{0.631798in}{1.325370in}}%
\pgfpathlineto{\pgfqpoint{0.642233in}{1.312204in}}%
\pgfpathlineto{\pgfqpoint{0.642799in}{1.311759in}}%
\pgfpathlineto{\pgfqpoint{0.652566in}{1.298147in}}%
\pgfpathlineto{\pgfqpoint{0.654192in}{1.284536in}}%
\pgfpathlineto{\pgfqpoint{0.647685in}{1.270925in}}%
\pgfpathlineto{\pgfqpoint{0.642233in}{1.265764in}}%
\pgfpathlineto{\pgfqpoint{0.636696in}{1.257314in}}%
\pgfpathlineto{\pgfqpoint{0.626577in}{1.247133in}}%
\pgfpathlineto{\pgfqpoint{0.623826in}{1.243703in}}%
\pgfpathlineto{\pgfqpoint{0.610920in}{1.231199in}}%
\pgfpathlineto{\pgfqpoint{0.609847in}{1.230092in}}%
\pgfpathlineto{\pgfqpoint{0.595263in}{1.216979in}}%
\pgfpathlineto{\pgfqpoint{0.594680in}{1.216481in}}%
\pgfpathlineto{\pgfqpoint{0.579607in}{1.204018in}}%
\pgfpathlineto{\pgfqpoint{0.577979in}{1.202870in}}%
\pgfpathlineto{\pgfqpoint{0.563950in}{1.192054in}}%
\pgfpathlineto{\pgfqpoint{0.559159in}{1.189259in}}%
\pgfpathlineto{\pgfqpoint{0.548294in}{1.181090in}}%
\pgfpathlineto{\pgfqpoint{0.536780in}{1.175647in}}%
\pgfpathlineto{\pgfqpoint{0.532637in}{1.172206in}}%
\pgfpathlineto{\pgfqpoint{0.516981in}{1.167234in}}%
\pgfpathlineto{\pgfqpoint{0.501324in}{1.170548in}}%
\pgfpathlineto{\pgfqpoint{0.494355in}{1.175647in}}%
\pgfpathclose%
\pgfpathmoveto{\pgfqpoint{0.804273in}{1.175647in}}%
\pgfpathlineto{\pgfqpoint{0.798799in}{1.177599in}}%
\pgfpathlineto{\pgfqpoint{0.783142in}{1.187706in}}%
\pgfpathlineto{\pgfqpoint{0.781400in}{1.189259in}}%
\pgfpathlineto{\pgfqpoint{0.767486in}{1.198963in}}%
\pgfpathlineto{\pgfqpoint{0.762896in}{1.202870in}}%
\pgfpathlineto{\pgfqpoint{0.751829in}{1.211543in}}%
\pgfpathlineto{\pgfqpoint{0.746105in}{1.216481in}}%
\pgfpathlineto{\pgfqpoint{0.736173in}{1.225345in}}%
\pgfpathlineto{\pgfqpoint{0.730802in}{1.230092in}}%
\pgfpathlineto{\pgfqpoint{0.720516in}{1.240512in}}%
\pgfpathlineto{\pgfqpoint{0.716895in}{1.243703in}}%
\pgfpathlineto{\pgfqpoint{0.704859in}{1.257263in}}%
\pgfpathlineto{\pgfqpoint{0.704796in}{1.257314in}}%
\pgfpathlineto{\pgfqpoint{0.693349in}{1.270925in}}%
\pgfpathlineto{\pgfqpoint{0.689203in}{1.282012in}}%
\pgfpathlineto{\pgfqpoint{0.687033in}{1.284536in}}%
\pgfpathlineto{\pgfqpoint{0.689203in}{1.294529in}}%
\pgfpathlineto{\pgfqpoint{0.689541in}{1.298147in}}%
\pgfpathlineto{\pgfqpoint{0.697162in}{1.311759in}}%
\pgfpathlineto{\pgfqpoint{0.704859in}{1.319375in}}%
\pgfpathlineto{\pgfqpoint{0.709136in}{1.325370in}}%
\pgfpathlineto{\pgfqpoint{0.720516in}{1.336727in}}%
\pgfpathlineto{\pgfqpoint{0.722466in}{1.338981in}}%
\pgfpathlineto{\pgfqpoint{0.736173in}{1.351904in}}%
\pgfpathlineto{\pgfqpoint{0.736892in}{1.352592in}}%
\pgfpathlineto{\pgfqpoint{0.751829in}{1.365578in}}%
\pgfpathlineto{\pgfqpoint{0.752621in}{1.366203in}}%
\pgfpathlineto{\pgfqpoint{0.767486in}{1.378119in}}%
\pgfpathlineto{\pgfqpoint{0.770079in}{1.379814in}}%
\pgfpathlineto{\pgfqpoint{0.783142in}{1.389708in}}%
\pgfpathlineto{\pgfqpoint{0.790038in}{1.393425in}}%
\pgfpathlineto{\pgfqpoint{0.798799in}{1.400117in}}%
\pgfpathlineto{\pgfqpoint{0.814455in}{1.406743in}}%
\pgfpathlineto{\pgfqpoint{0.818617in}{1.407036in}}%
\pgfpathlineto{\pgfqpoint{0.830112in}{1.408923in}}%
\pgfpathlineto{\pgfqpoint{0.833015in}{1.407036in}}%
\pgfpathlineto{\pgfqpoint{0.845769in}{1.403432in}}%
\pgfpathlineto{\pgfqpoint{0.861425in}{1.393481in}}%
\pgfpathlineto{\pgfqpoint{0.861484in}{1.393425in}}%
\pgfpathlineto{\pgfqpoint{0.877082in}{1.382962in}}%
\pgfpathlineto{\pgfqpoint{0.880752in}{1.379814in}}%
\pgfpathlineto{\pgfqpoint{0.892738in}{1.370872in}}%
\pgfpathlineto{\pgfqpoint{0.898199in}{1.366203in}}%
\pgfpathlineto{\pgfqpoint{0.908395in}{1.357569in}}%
\pgfpathlineto{\pgfqpoint{0.914075in}{1.352592in}}%
\pgfpathlineto{\pgfqpoint{0.924051in}{1.342971in}}%
\pgfpathlineto{\pgfqpoint{0.928545in}{1.338981in}}%
\pgfpathlineto{\pgfqpoint{0.939708in}{1.326885in}}%
\pgfpathlineto{\pgfqpoint{0.941494in}{1.325370in}}%
\pgfpathlineto{\pgfqpoint{0.953120in}{1.311759in}}%
\pgfpathlineto{\pgfqpoint{0.955364in}{1.307000in}}%
\pgfpathlineto{\pgfqpoint{0.962755in}{1.298147in}}%
\pgfpathlineto{\pgfqpoint{0.964661in}{1.284536in}}%
\pgfpathlineto{\pgfqpoint{0.957033in}{1.270925in}}%
\pgfpathlineto{\pgfqpoint{0.955364in}{1.269566in}}%
\pgfpathlineto{\pgfqpoint{0.946779in}{1.257314in}}%
\pgfpathlineto{\pgfqpoint{0.939708in}{1.250637in}}%
\pgfpathlineto{\pgfqpoint{0.933993in}{1.243703in}}%
\pgfpathlineto{\pgfqpoint{0.924051in}{1.234312in}}%
\pgfpathlineto{\pgfqpoint{0.919955in}{1.230092in}}%
\pgfpathlineto{\pgfqpoint{0.908395in}{1.219712in}}%
\pgfpathlineto{\pgfqpoint{0.904678in}{1.216481in}}%
\pgfpathlineto{\pgfqpoint{0.892738in}{1.206431in}}%
\pgfpathlineto{\pgfqpoint{0.887884in}{1.202870in}}%
\pgfpathlineto{\pgfqpoint{0.877082in}{1.194227in}}%
\pgfpathlineto{\pgfqpoint{0.869105in}{1.189259in}}%
\pgfpathlineto{\pgfqpoint{0.861425in}{1.183112in}}%
\pgfpathlineto{\pgfqpoint{0.847332in}{1.175647in}}%
\pgfpathlineto{\pgfqpoint{0.845769in}{1.174197in}}%
\pgfpathlineto{\pgfqpoint{0.830112in}{1.167565in}}%
\pgfpathlineto{\pgfqpoint{0.814455in}{1.169222in}}%
\pgfpathlineto{\pgfqpoint{0.804273in}{1.175647in}}%
\pgfpathclose%
\pgfpathmoveto{\pgfqpoint{1.113552in}{1.175647in}}%
\pgfpathlineto{\pgfqpoint{1.111930in}{1.176130in}}%
\pgfpathlineto{\pgfqpoint{1.096274in}{1.185317in}}%
\pgfpathlineto{\pgfqpoint{1.091619in}{1.189259in}}%
\pgfpathlineto{\pgfqpoint{1.080617in}{1.196530in}}%
\pgfpathlineto{\pgfqpoint{1.072951in}{1.202870in}}%
\pgfpathlineto{\pgfqpoint{1.064960in}{1.208941in}}%
\pgfpathlineto{\pgfqpoint{1.056128in}{1.216481in}}%
\pgfpathlineto{\pgfqpoint{1.049304in}{1.222504in}}%
\pgfpathlineto{\pgfqpoint{1.040806in}{1.230092in}}%
\pgfpathlineto{\pgfqpoint{1.033647in}{1.237422in}}%
\pgfpathlineto{\pgfqpoint{1.026786in}{1.243703in}}%
\pgfpathlineto{\pgfqpoint{1.017991in}{1.254023in}}%
\pgfpathlineto{\pgfqpoint{1.014235in}{1.257314in}}%
\pgfpathlineto{\pgfqpoint{1.003868in}{1.270925in}}%
\pgfpathlineto{\pgfqpoint{1.002334in}{1.275422in}}%
\pgfpathlineto{\pgfqpoint{0.996162in}{1.284536in}}%
\pgfpathlineto{\pgfqpoint{0.998476in}{1.298147in}}%
\pgfpathlineto{\pgfqpoint{1.002334in}{1.301984in}}%
\pgfpathlineto{\pgfqpoint{1.007321in}{1.311759in}}%
\pgfpathlineto{\pgfqpoint{1.017991in}{1.323285in}}%
\pgfpathlineto{\pgfqpoint{1.019419in}{1.325370in}}%
\pgfpathlineto{\pgfqpoint{1.032487in}{1.338981in}}%
\pgfpathlineto{\pgfqpoint{1.033647in}{1.339979in}}%
\pgfpathlineto{\pgfqpoint{1.046805in}{1.352592in}}%
\pgfpathlineto{\pgfqpoint{1.049304in}{1.354769in}}%
\pgfpathlineto{\pgfqpoint{1.062604in}{1.366203in}}%
\pgfpathlineto{\pgfqpoint{1.064960in}{1.368247in}}%
\pgfpathlineto{\pgfqpoint{1.079930in}{1.379814in}}%
\pgfpathlineto{\pgfqpoint{1.080617in}{1.380423in}}%
\pgfpathlineto{\pgfqpoint{1.096274in}{1.391660in}}%
\pgfpathlineto{\pgfqpoint{1.099855in}{1.393425in}}%
\pgfpathlineto{\pgfqpoint{1.111930in}{1.401885in}}%
\pgfpathlineto{\pgfqpoint{1.126526in}{1.407036in}}%
\pgfpathlineto{\pgfqpoint{1.127587in}{1.407894in}}%
\pgfpathlineto{\pgfqpoint{1.143243in}{1.407894in}}%
\pgfpathlineto{\pgfqpoint{1.144304in}{1.407036in}}%
\pgfpathlineto{\pgfqpoint{1.158900in}{1.401885in}}%
\pgfpathlineto{\pgfqpoint{1.170975in}{1.393425in}}%
\pgfpathlineto{\pgfqpoint{1.174556in}{1.391660in}}%
\pgfpathlineto{\pgfqpoint{1.190213in}{1.380423in}}%
\pgfpathlineto{\pgfqpoint{1.190900in}{1.379814in}}%
\pgfpathlineto{\pgfqpoint{1.205870in}{1.368247in}}%
\pgfpathlineto{\pgfqpoint{1.208226in}{1.366203in}}%
\pgfpathlineto{\pgfqpoint{1.221526in}{1.354769in}}%
\pgfpathlineto{\pgfqpoint{1.224025in}{1.352592in}}%
\pgfpathlineto{\pgfqpoint{1.237183in}{1.339979in}}%
\pgfpathlineto{\pgfqpoint{1.238343in}{1.338981in}}%
\pgfpathlineto{\pgfqpoint{1.251411in}{1.325370in}}%
\pgfpathlineto{\pgfqpoint{1.252839in}{1.323285in}}%
\pgfpathlineto{\pgfqpoint{1.263509in}{1.311759in}}%
\pgfpathlineto{\pgfqpoint{1.268496in}{1.301984in}}%
\pgfpathlineto{\pgfqpoint{1.272354in}{1.298147in}}%
\pgfpathlineto{\pgfqpoint{1.274668in}{1.284536in}}%
\pgfpathlineto{\pgfqpoint{1.268496in}{1.275422in}}%
\pgfpathlineto{\pgfqpoint{1.266962in}{1.270925in}}%
\pgfpathlineto{\pgfqpoint{1.256595in}{1.257314in}}%
\pgfpathlineto{\pgfqpoint{1.252839in}{1.254023in}}%
\pgfpathlineto{\pgfqpoint{1.244044in}{1.243703in}}%
\pgfpathlineto{\pgfqpoint{1.237183in}{1.237422in}}%
\pgfpathlineto{\pgfqpoint{1.230024in}{1.230092in}}%
\pgfpathlineto{\pgfqpoint{1.221526in}{1.222504in}}%
\pgfpathlineto{\pgfqpoint{1.214702in}{1.216481in}}%
\pgfpathlineto{\pgfqpoint{1.205870in}{1.208941in}}%
\pgfpathlineto{\pgfqpoint{1.197879in}{1.202870in}}%
\pgfpathlineto{\pgfqpoint{1.190213in}{1.196530in}}%
\pgfpathlineto{\pgfqpoint{1.179211in}{1.189259in}}%
\pgfpathlineto{\pgfqpoint{1.174556in}{1.185317in}}%
\pgfpathlineto{\pgfqpoint{1.158900in}{1.176130in}}%
\pgfpathlineto{\pgfqpoint{1.157278in}{1.175647in}}%
\pgfpathlineto{\pgfqpoint{1.143243in}{1.168228in}}%
\pgfpathlineto{\pgfqpoint{1.127587in}{1.168228in}}%
\pgfpathlineto{\pgfqpoint{1.113552in}{1.175647in}}%
\pgfpathclose%
\pgfpathmoveto{\pgfqpoint{1.423498in}{1.175647in}}%
\pgfpathlineto{\pgfqpoint{1.409405in}{1.183112in}}%
\pgfpathlineto{\pgfqpoint{1.401725in}{1.189259in}}%
\pgfpathlineto{\pgfqpoint{1.393748in}{1.194227in}}%
\pgfpathlineto{\pgfqpoint{1.382946in}{1.202870in}}%
\pgfpathlineto{\pgfqpoint{1.378092in}{1.206431in}}%
\pgfpathlineto{\pgfqpoint{1.366152in}{1.216481in}}%
\pgfpathlineto{\pgfqpoint{1.362435in}{1.219712in}}%
\pgfpathlineto{\pgfqpoint{1.350875in}{1.230092in}}%
\pgfpathlineto{\pgfqpoint{1.346779in}{1.234312in}}%
\pgfpathlineto{\pgfqpoint{1.336837in}{1.243703in}}%
\pgfpathlineto{\pgfqpoint{1.331122in}{1.250637in}}%
\pgfpathlineto{\pgfqpoint{1.324051in}{1.257314in}}%
\pgfpathlineto{\pgfqpoint{1.315466in}{1.269566in}}%
\pgfpathlineto{\pgfqpoint{1.313797in}{1.270925in}}%
\pgfpathlineto{\pgfqpoint{1.306169in}{1.284536in}}%
\pgfpathlineto{\pgfqpoint{1.308075in}{1.298147in}}%
\pgfpathlineto{\pgfqpoint{1.315466in}{1.307000in}}%
\pgfpathlineto{\pgfqpoint{1.317710in}{1.311759in}}%
\pgfpathlineto{\pgfqpoint{1.329336in}{1.325370in}}%
\pgfpathlineto{\pgfqpoint{1.331122in}{1.326885in}}%
\pgfpathlineto{\pgfqpoint{1.342285in}{1.338981in}}%
\pgfpathlineto{\pgfqpoint{1.346779in}{1.342971in}}%
\pgfpathlineto{\pgfqpoint{1.356755in}{1.352592in}}%
\pgfpathlineto{\pgfqpoint{1.362435in}{1.357569in}}%
\pgfpathlineto{\pgfqpoint{1.372631in}{1.366203in}}%
\pgfpathlineto{\pgfqpoint{1.378092in}{1.370872in}}%
\pgfpathlineto{\pgfqpoint{1.390078in}{1.379814in}}%
\pgfpathlineto{\pgfqpoint{1.393748in}{1.382962in}}%
\pgfpathlineto{\pgfqpoint{1.409346in}{1.393425in}}%
\pgfpathlineto{\pgfqpoint{1.409405in}{1.393481in}}%
\pgfpathlineto{\pgfqpoint{1.425061in}{1.403432in}}%
\pgfpathlineto{\pgfqpoint{1.437815in}{1.407036in}}%
\pgfpathlineto{\pgfqpoint{1.440718in}{1.408923in}}%
\pgfpathlineto{\pgfqpoint{1.452213in}{1.407036in}}%
\pgfpathlineto{\pgfqpoint{1.456375in}{1.406743in}}%
\pgfpathlineto{\pgfqpoint{1.472031in}{1.400117in}}%
\pgfpathlineto{\pgfqpoint{1.480792in}{1.393425in}}%
\pgfpathlineto{\pgfqpoint{1.487688in}{1.389708in}}%
\pgfpathlineto{\pgfqpoint{1.500751in}{1.379814in}}%
\pgfpathlineto{\pgfqpoint{1.503344in}{1.378119in}}%
\pgfpathlineto{\pgfqpoint{1.518209in}{1.366203in}}%
\pgfpathlineto{\pgfqpoint{1.519001in}{1.365578in}}%
\pgfpathlineto{\pgfqpoint{1.533938in}{1.352592in}}%
\pgfpathlineto{\pgfqpoint{1.534657in}{1.351904in}}%
\pgfpathlineto{\pgfqpoint{1.548364in}{1.338981in}}%
\pgfpathlineto{\pgfqpoint{1.550314in}{1.336727in}}%
\pgfpathlineto{\pgfqpoint{1.561694in}{1.325370in}}%
\pgfpathlineto{\pgfqpoint{1.565971in}{1.319375in}}%
\pgfpathlineto{\pgfqpoint{1.573668in}{1.311759in}}%
\pgfpathlineto{\pgfqpoint{1.581289in}{1.298147in}}%
\pgfpathlineto{\pgfqpoint{1.581627in}{1.294529in}}%
\pgfpathlineto{\pgfqpoint{1.583797in}{1.284536in}}%
\pgfpathlineto{\pgfqpoint{1.581627in}{1.282012in}}%
\pgfpathlineto{\pgfqpoint{1.577481in}{1.270925in}}%
\pgfpathlineto{\pgfqpoint{1.566034in}{1.257314in}}%
\pgfpathlineto{\pgfqpoint{1.565971in}{1.257263in}}%
\pgfpathlineto{\pgfqpoint{1.553935in}{1.243703in}}%
\pgfpathlineto{\pgfqpoint{1.550314in}{1.240512in}}%
\pgfpathlineto{\pgfqpoint{1.540028in}{1.230092in}}%
\pgfpathlineto{\pgfqpoint{1.534657in}{1.225345in}}%
\pgfpathlineto{\pgfqpoint{1.524725in}{1.216481in}}%
\pgfpathlineto{\pgfqpoint{1.519001in}{1.211543in}}%
\pgfpathlineto{\pgfqpoint{1.507934in}{1.202870in}}%
\pgfpathlineto{\pgfqpoint{1.503344in}{1.198963in}}%
\pgfpathlineto{\pgfqpoint{1.489430in}{1.189259in}}%
\pgfpathlineto{\pgfqpoint{1.487688in}{1.187706in}}%
\pgfpathlineto{\pgfqpoint{1.472031in}{1.177599in}}%
\pgfpathlineto{\pgfqpoint{1.466557in}{1.175647in}}%
\pgfpathlineto{\pgfqpoint{1.456375in}{1.169222in}}%
\pgfpathlineto{\pgfqpoint{1.440718in}{1.167565in}}%
\pgfpathlineto{\pgfqpoint{1.425061in}{1.174197in}}%
\pgfpathlineto{\pgfqpoint{1.423498in}{1.175647in}}%
\pgfpathclose%
\pgfpathmoveto{\pgfqpoint{1.734050in}{1.175647in}}%
\pgfpathlineto{\pgfqpoint{1.722536in}{1.181090in}}%
\pgfpathlineto{\pgfqpoint{1.711671in}{1.189259in}}%
\pgfpathlineto{\pgfqpoint{1.706880in}{1.192054in}}%
\pgfpathlineto{\pgfqpoint{1.692851in}{1.202870in}}%
\pgfpathlineto{\pgfqpoint{1.691223in}{1.204018in}}%
\pgfpathlineto{\pgfqpoint{1.676150in}{1.216481in}}%
\pgfpathlineto{\pgfqpoint{1.675567in}{1.216979in}}%
\pgfpathlineto{\pgfqpoint{1.660983in}{1.230092in}}%
\pgfpathlineto{\pgfqpoint{1.659910in}{1.231199in}}%
\pgfpathlineto{\pgfqpoint{1.647004in}{1.243703in}}%
\pgfpathlineto{\pgfqpoint{1.644253in}{1.247133in}}%
\pgfpathlineto{\pgfqpoint{1.634134in}{1.257314in}}%
\pgfpathlineto{\pgfqpoint{1.628597in}{1.265764in}}%
\pgfpathlineto{\pgfqpoint{1.623145in}{1.270925in}}%
\pgfpathlineto{\pgfqpoint{1.616638in}{1.284536in}}%
\pgfpathlineto{\pgfqpoint{1.618264in}{1.298147in}}%
\pgfpathlineto{\pgfqpoint{1.628031in}{1.311759in}}%
\pgfpathlineto{\pgfqpoint{1.628597in}{1.312204in}}%
\pgfpathlineto{\pgfqpoint{1.639032in}{1.325370in}}%
\pgfpathlineto{\pgfqpoint{1.644253in}{1.330087in}}%
\pgfpathlineto{\pgfqpoint{1.652240in}{1.338981in}}%
\pgfpathlineto{\pgfqpoint{1.659910in}{1.345965in}}%
\pgfpathlineto{\pgfqpoint{1.666771in}{1.352592in}}%
\pgfpathlineto{\pgfqpoint{1.675567in}{1.360309in}}%
\pgfpathlineto{\pgfqpoint{1.682661in}{1.366203in}}%
\pgfpathlineto{\pgfqpoint{1.691223in}{1.373395in}}%
\pgfpathlineto{\pgfqpoint{1.700173in}{1.379814in}}%
\pgfpathlineto{\pgfqpoint{1.706880in}{1.385358in}}%
\pgfpathlineto{\pgfqpoint{1.719722in}{1.393425in}}%
\pgfpathlineto{\pgfqpoint{1.722536in}{1.395914in}}%
\pgfpathlineto{\pgfqpoint{1.738193in}{1.404756in}}%
\pgfpathlineto{\pgfqpoint{1.748943in}{1.407036in}}%
\pgfpathlineto{\pgfqpoint{1.753849in}{1.409438in}}%
\pgfpathlineto{\pgfqpoint{1.761184in}{1.407036in}}%
\pgfpathlineto{\pgfqpoint{1.769506in}{1.405860in}}%
\pgfpathlineto{\pgfqpoint{1.785162in}{1.398126in}}%
\pgfpathlineto{\pgfqpoint{1.790860in}{1.393425in}}%
\pgfpathlineto{\pgfqpoint{1.800819in}{1.387607in}}%
\pgfpathlineto{\pgfqpoint{1.810645in}{1.379814in}}%
\pgfpathlineto{\pgfqpoint{1.816476in}{1.375812in}}%
\pgfpathlineto{\pgfqpoint{1.828164in}{1.366203in}}%
\pgfpathlineto{\pgfqpoint{1.832132in}{1.362981in}}%
\pgfpathlineto{\pgfqpoint{1.844005in}{1.352592in}}%
\pgfpathlineto{\pgfqpoint{1.847789in}{1.348948in}}%
\pgfpathlineto{\pgfqpoint{1.858517in}{1.338981in}}%
\pgfpathlineto{\pgfqpoint{1.863445in}{1.333375in}}%
\pgfpathlineto{\pgfqpoint{1.871841in}{1.325370in}}%
\pgfpathlineto{\pgfqpoint{1.879102in}{1.315671in}}%
\pgfpathlineto{\pgfqpoint{1.883488in}{1.311759in}}%
\pgfpathlineto{\pgfqpoint{1.892034in}{1.298147in}}%
\pgfpathlineto{\pgfqpoint{1.893456in}{1.284536in}}%
\pgfpathlineto{\pgfqpoint{1.887763in}{1.270925in}}%
\pgfpathlineto{\pgfqpoint{1.879102in}{1.261643in}}%
\pgfpathlineto{\pgfqpoint{1.876422in}{1.257314in}}%
\pgfpathlineto{\pgfqpoint{1.863611in}{1.243703in}}%
\pgfpathlineto{\pgfqpoint{1.863445in}{1.243564in}}%
\pgfpathlineto{\pgfqpoint{1.849933in}{1.230092in}}%
\pgfpathlineto{\pgfqpoint{1.847789in}{1.228224in}}%
\pgfpathlineto{\pgfqpoint{1.834723in}{1.216481in}}%
\pgfpathlineto{\pgfqpoint{1.832132in}{1.214231in}}%
\pgfpathlineto{\pgfqpoint{1.818021in}{1.202870in}}%
\pgfpathlineto{\pgfqpoint{1.816476in}{1.201521in}}%
\pgfpathlineto{\pgfqpoint{1.800819in}{1.190013in}}%
\pgfpathlineto{\pgfqpoint{1.799428in}{1.189259in}}%
\pgfpathlineto{\pgfqpoint{1.785162in}{1.179252in}}%
\pgfpathlineto{\pgfqpoint{1.776475in}{1.175647in}}%
\pgfpathlineto{\pgfqpoint{1.769506in}{1.170548in}}%
\pgfpathlineto{\pgfqpoint{1.753849in}{1.167234in}}%
\pgfpathlineto{\pgfqpoint{1.738193in}{1.172206in}}%
\pgfpathlineto{\pgfqpoint{1.734050in}{1.175647in}}%
\pgfpathclose%
\pgfpathmoveto{\pgfqpoint{0.653949in}{1.338981in}}%
\pgfpathlineto{\pgfqpoint{0.642233in}{1.344562in}}%
\pgfpathlineto{\pgfqpoint{0.631425in}{1.352592in}}%
\pgfpathlineto{\pgfqpoint{0.626577in}{1.355918in}}%
\pgfpathlineto{\pgfqpoint{0.614193in}{1.366203in}}%
\pgfpathlineto{\pgfqpoint{0.610920in}{1.368951in}}%
\pgfpathlineto{\pgfqpoint{0.598651in}{1.379814in}}%
\pgfpathlineto{\pgfqpoint{0.595263in}{1.383225in}}%
\pgfpathlineto{\pgfqpoint{0.584665in}{1.393425in}}%
\pgfpathlineto{\pgfqpoint{0.579607in}{1.400121in}}%
\pgfpathlineto{\pgfqpoint{0.573649in}{1.407036in}}%
\pgfpathlineto{\pgfqpoint{0.568405in}{1.420647in}}%
\pgfpathlineto{\pgfqpoint{0.570693in}{1.434259in}}%
\pgfpathlineto{\pgfqpoint{0.579607in}{1.447373in}}%
\pgfpathlineto{\pgfqpoint{0.579906in}{1.447870in}}%
\pgfpathlineto{\pgfqpoint{0.592777in}{1.461481in}}%
\pgfpathlineto{\pgfqpoint{0.595263in}{1.463643in}}%
\pgfpathlineto{\pgfqpoint{0.607768in}{1.475092in}}%
\pgfpathlineto{\pgfqpoint{0.610920in}{1.477832in}}%
\pgfpathlineto{\pgfqpoint{0.624089in}{1.488703in}}%
\pgfpathlineto{\pgfqpoint{0.626577in}{1.490865in}}%
\pgfpathlineto{\pgfqpoint{0.642233in}{1.502054in}}%
\pgfpathlineto{\pgfqpoint{0.642804in}{1.502314in}}%
\pgfpathlineto{\pgfqpoint{0.657890in}{1.510063in}}%
\pgfpathlineto{\pgfqpoint{0.673546in}{1.512052in}}%
\pgfpathlineto{\pgfqpoint{0.689203in}{1.507493in}}%
\pgfpathlineto{\pgfqpoint{0.697158in}{1.502314in}}%
\pgfpathlineto{\pgfqpoint{0.704859in}{1.497917in}}%
\pgfpathlineto{\pgfqpoint{0.716592in}{1.488703in}}%
\pgfpathlineto{\pgfqpoint{0.720516in}{1.485758in}}%
\pgfpathlineto{\pgfqpoint{0.733011in}{1.475092in}}%
\pgfpathlineto{\pgfqpoint{0.736173in}{1.472247in}}%
\pgfpathlineto{\pgfqpoint{0.748004in}{1.461481in}}%
\pgfpathlineto{\pgfqpoint{0.751829in}{1.457266in}}%
\pgfpathlineto{\pgfqpoint{0.761066in}{1.447870in}}%
\pgfpathlineto{\pgfqpoint{0.767486in}{1.437685in}}%
\pgfpathlineto{\pgfqpoint{0.770040in}{1.434259in}}%
\pgfpathlineto{\pgfqpoint{0.772488in}{1.420647in}}%
\pgfpathlineto{\pgfqpoint{0.767486in}{1.408538in}}%
\pgfpathlineto{\pgfqpoint{0.766953in}{1.407036in}}%
\pgfpathlineto{\pgfqpoint{0.756202in}{1.393425in}}%
\pgfpathlineto{\pgfqpoint{0.751829in}{1.389422in}}%
\pgfpathlineto{\pgfqpoint{0.742198in}{1.379814in}}%
\pgfpathlineto{\pgfqpoint{0.736173in}{1.374532in}}%
\pgfpathlineto{\pgfqpoint{0.726592in}{1.366203in}}%
\pgfpathlineto{\pgfqpoint{0.720516in}{1.360965in}}%
\pgfpathlineto{\pgfqpoint{0.709464in}{1.352592in}}%
\pgfpathlineto{\pgfqpoint{0.704859in}{1.348790in}}%
\pgfpathlineto{\pgfqpoint{0.689203in}{1.339444in}}%
\pgfpathlineto{\pgfqpoint{0.687476in}{1.338981in}}%
\pgfpathlineto{\pgfqpoint{0.673546in}{1.334632in}}%
\pgfpathlineto{\pgfqpoint{0.657890in}{1.336760in}}%
\pgfpathlineto{\pgfqpoint{0.653949in}{1.338981in}}%
\pgfpathclose%
\pgfpathmoveto{\pgfqpoint{0.964376in}{1.338981in}}%
\pgfpathlineto{\pgfqpoint{0.955364in}{1.342675in}}%
\pgfpathlineto{\pgfqpoint{0.941092in}{1.352592in}}%
\pgfpathlineto{\pgfqpoint{0.939708in}{1.353483in}}%
\pgfpathlineto{\pgfqpoint{0.924051in}{1.366142in}}%
\pgfpathlineto{\pgfqpoint{0.923982in}{1.366203in}}%
\pgfpathlineto{\pgfqpoint{0.908582in}{1.379814in}}%
\pgfpathlineto{\pgfqpoint{0.908395in}{1.380002in}}%
\pgfpathlineto{\pgfqpoint{0.894711in}{1.393425in}}%
\pgfpathlineto{\pgfqpoint{0.892738in}{1.396083in}}%
\pgfpathlineto{\pgfqpoint{0.883667in}{1.407036in}}%
\pgfpathlineto{\pgfqpoint{0.878559in}{1.420648in}}%
\pgfpathlineto{\pgfqpoint{0.880787in}{1.434259in}}%
\pgfpathlineto{\pgfqpoint{0.889787in}{1.447870in}}%
\pgfpathlineto{\pgfqpoint{0.892738in}{1.450680in}}%
\pgfpathlineto{\pgfqpoint{0.902784in}{1.461481in}}%
\pgfpathlineto{\pgfqpoint{0.908395in}{1.466454in}}%
\pgfpathlineto{\pgfqpoint{0.917843in}{1.475092in}}%
\pgfpathlineto{\pgfqpoint{0.924051in}{1.480482in}}%
\pgfpathlineto{\pgfqpoint{0.934268in}{1.488703in}}%
\pgfpathlineto{\pgfqpoint{0.939708in}{1.493304in}}%
\pgfpathlineto{\pgfqpoint{0.953123in}{1.502314in}}%
\pgfpathlineto{\pgfqpoint{0.955364in}{1.504044in}}%
\pgfpathlineto{\pgfqpoint{0.971021in}{1.510985in}}%
\pgfpathlineto{\pgfqpoint{0.986678in}{1.511650in}}%
\pgfpathlineto{\pgfqpoint{1.002334in}{1.505871in}}%
\pgfpathlineto{\pgfqpoint{1.007317in}{1.502314in}}%
\pgfpathlineto{\pgfqpoint{1.017991in}{1.495661in}}%
\pgfpathlineto{\pgfqpoint{1.026499in}{1.488703in}}%
\pgfpathlineto{\pgfqpoint{1.033647in}{1.483128in}}%
\pgfpathlineto{\pgfqpoint{1.042961in}{1.475092in}}%
\pgfpathlineto{\pgfqpoint{1.049304in}{1.469325in}}%
\pgfpathlineto{\pgfqpoint{1.058021in}{1.461481in}}%
\pgfpathlineto{\pgfqpoint{1.064960in}{1.453914in}}%
\pgfpathlineto{\pgfqpoint{1.071089in}{1.447870in}}%
\pgfpathlineto{\pgfqpoint{1.079896in}{1.434259in}}%
\pgfpathlineto{\pgfqpoint{1.080617in}{1.429733in}}%
\pgfpathlineto{\pgfqpoint{1.082326in}{1.420647in}}%
\pgfpathlineto{\pgfqpoint{1.080617in}{1.416728in}}%
\pgfpathlineto{\pgfqpoint{1.077078in}{1.407036in}}%
\pgfpathlineto{\pgfqpoint{1.066140in}{1.393425in}}%
\pgfpathlineto{\pgfqpoint{1.064960in}{1.392378in}}%
\pgfpathlineto{\pgfqpoint{1.052234in}{1.379814in}}%
\pgfpathlineto{\pgfqpoint{1.049304in}{1.377274in}}%
\pgfpathlineto{\pgfqpoint{1.036697in}{1.366203in}}%
\pgfpathlineto{\pgfqpoint{1.033647in}{1.363545in}}%
\pgfpathlineto{\pgfqpoint{1.019731in}{1.352592in}}%
\pgfpathlineto{\pgfqpoint{1.017991in}{1.351096in}}%
\pgfpathlineto{\pgfqpoint{1.002334in}{1.340964in}}%
\pgfpathlineto{\pgfqpoint{0.996509in}{1.338981in}}%
\pgfpathlineto{\pgfqpoint{0.986678in}{1.335063in}}%
\pgfpathlineto{\pgfqpoint{0.971021in}{1.335774in}}%
\pgfpathlineto{\pgfqpoint{0.964376in}{1.338981in}}%
\pgfpathclose%
\pgfpathmoveto{\pgfqpoint{1.274321in}{1.338981in}}%
\pgfpathlineto{\pgfqpoint{1.268496in}{1.340964in}}%
\pgfpathlineto{\pgfqpoint{1.252839in}{1.351096in}}%
\pgfpathlineto{\pgfqpoint{1.251099in}{1.352592in}}%
\pgfpathlineto{\pgfqpoint{1.237183in}{1.363545in}}%
\pgfpathlineto{\pgfqpoint{1.234133in}{1.366203in}}%
\pgfpathlineto{\pgfqpoint{1.221526in}{1.377274in}}%
\pgfpathlineto{\pgfqpoint{1.218596in}{1.379814in}}%
\pgfpathlineto{\pgfqpoint{1.205870in}{1.392378in}}%
\pgfpathlineto{\pgfqpoint{1.204690in}{1.393425in}}%
\pgfpathlineto{\pgfqpoint{1.193752in}{1.407036in}}%
\pgfpathlineto{\pgfqpoint{1.190213in}{1.416728in}}%
\pgfpathlineto{\pgfqpoint{1.188504in}{1.420648in}}%
\pgfpathlineto{\pgfqpoint{1.190213in}{1.429733in}}%
\pgfpathlineto{\pgfqpoint{1.190934in}{1.434259in}}%
\pgfpathlineto{\pgfqpoint{1.199741in}{1.447870in}}%
\pgfpathlineto{\pgfqpoint{1.205870in}{1.453914in}}%
\pgfpathlineto{\pgfqpoint{1.212809in}{1.461481in}}%
\pgfpathlineto{\pgfqpoint{1.221526in}{1.469325in}}%
\pgfpathlineto{\pgfqpoint{1.227869in}{1.475092in}}%
\pgfpathlineto{\pgfqpoint{1.237183in}{1.483128in}}%
\pgfpathlineto{\pgfqpoint{1.244331in}{1.488703in}}%
\pgfpathlineto{\pgfqpoint{1.252839in}{1.495661in}}%
\pgfpathlineto{\pgfqpoint{1.263513in}{1.502314in}}%
\pgfpathlineto{\pgfqpoint{1.268496in}{1.505871in}}%
\pgfpathlineto{\pgfqpoint{1.284152in}{1.511650in}}%
\pgfpathlineto{\pgfqpoint{1.299809in}{1.510985in}}%
\pgfpathlineto{\pgfqpoint{1.315466in}{1.504044in}}%
\pgfpathlineto{\pgfqpoint{1.317707in}{1.502314in}}%
\pgfpathlineto{\pgfqpoint{1.331122in}{1.493304in}}%
\pgfpathlineto{\pgfqpoint{1.336562in}{1.488703in}}%
\pgfpathlineto{\pgfqpoint{1.346779in}{1.480482in}}%
\pgfpathlineto{\pgfqpoint{1.352987in}{1.475092in}}%
\pgfpathlineto{\pgfqpoint{1.362435in}{1.466454in}}%
\pgfpathlineto{\pgfqpoint{1.368046in}{1.461481in}}%
\pgfpathlineto{\pgfqpoint{1.378092in}{1.450680in}}%
\pgfpathlineto{\pgfqpoint{1.381043in}{1.447870in}}%
\pgfpathlineto{\pgfqpoint{1.390043in}{1.434259in}}%
\pgfpathlineto{\pgfqpoint{1.392271in}{1.420648in}}%
\pgfpathlineto{\pgfqpoint{1.387163in}{1.407036in}}%
\pgfpathlineto{\pgfqpoint{1.378092in}{1.396083in}}%
\pgfpathlineto{\pgfqpoint{1.376119in}{1.393425in}}%
\pgfpathlineto{\pgfqpoint{1.362435in}{1.380002in}}%
\pgfpathlineto{\pgfqpoint{1.362248in}{1.379814in}}%
\pgfpathlineto{\pgfqpoint{1.346848in}{1.366203in}}%
\pgfpathlineto{\pgfqpoint{1.346779in}{1.366142in}}%
\pgfpathlineto{\pgfqpoint{1.331122in}{1.353483in}}%
\pgfpathlineto{\pgfqpoint{1.329738in}{1.352592in}}%
\pgfpathlineto{\pgfqpoint{1.315466in}{1.342675in}}%
\pgfpathlineto{\pgfqpoint{1.306454in}{1.338981in}}%
\pgfpathlineto{\pgfqpoint{1.299809in}{1.335774in}}%
\pgfpathlineto{\pgfqpoint{1.284152in}{1.335063in}}%
\pgfpathlineto{\pgfqpoint{1.274321in}{1.338981in}}%
\pgfpathclose%
\pgfpathmoveto{\pgfqpoint{1.583354in}{1.338981in}}%
\pgfpathlineto{\pgfqpoint{1.581627in}{1.339444in}}%
\pgfpathlineto{\pgfqpoint{1.565971in}{1.348790in}}%
\pgfpathlineto{\pgfqpoint{1.561366in}{1.352592in}}%
\pgfpathlineto{\pgfqpoint{1.550314in}{1.360965in}}%
\pgfpathlineto{\pgfqpoint{1.544238in}{1.366203in}}%
\pgfpathlineto{\pgfqpoint{1.534657in}{1.374532in}}%
\pgfpathlineto{\pgfqpoint{1.528632in}{1.379814in}}%
\pgfpathlineto{\pgfqpoint{1.519001in}{1.389422in}}%
\pgfpathlineto{\pgfqpoint{1.514628in}{1.393425in}}%
\pgfpathlineto{\pgfqpoint{1.503877in}{1.407036in}}%
\pgfpathlineto{\pgfqpoint{1.503344in}{1.408538in}}%
\pgfpathlineto{\pgfqpoint{1.498342in}{1.420647in}}%
\pgfpathlineto{\pgfqpoint{1.500790in}{1.434259in}}%
\pgfpathlineto{\pgfqpoint{1.503344in}{1.437685in}}%
\pgfpathlineto{\pgfqpoint{1.509764in}{1.447870in}}%
\pgfpathlineto{\pgfqpoint{1.519001in}{1.457266in}}%
\pgfpathlineto{\pgfqpoint{1.522826in}{1.461481in}}%
\pgfpathlineto{\pgfqpoint{1.534657in}{1.472247in}}%
\pgfpathlineto{\pgfqpoint{1.537819in}{1.475092in}}%
\pgfpathlineto{\pgfqpoint{1.550314in}{1.485758in}}%
\pgfpathlineto{\pgfqpoint{1.554238in}{1.488703in}}%
\pgfpathlineto{\pgfqpoint{1.565971in}{1.497917in}}%
\pgfpathlineto{\pgfqpoint{1.573672in}{1.502314in}}%
\pgfpathlineto{\pgfqpoint{1.581627in}{1.507493in}}%
\pgfpathlineto{\pgfqpoint{1.597284in}{1.512052in}}%
\pgfpathlineto{\pgfqpoint{1.612940in}{1.510063in}}%
\pgfpathlineto{\pgfqpoint{1.628026in}{1.502314in}}%
\pgfpathlineto{\pgfqpoint{1.628597in}{1.502054in}}%
\pgfpathlineto{\pgfqpoint{1.644253in}{1.490865in}}%
\pgfpathlineto{\pgfqpoint{1.646741in}{1.488703in}}%
\pgfpathlineto{\pgfqpoint{1.659910in}{1.477832in}}%
\pgfpathlineto{\pgfqpoint{1.663062in}{1.475092in}}%
\pgfpathlineto{\pgfqpoint{1.675567in}{1.463643in}}%
\pgfpathlineto{\pgfqpoint{1.678053in}{1.461481in}}%
\pgfpathlineto{\pgfqpoint{1.690924in}{1.447870in}}%
\pgfpathlineto{\pgfqpoint{1.691223in}{1.447373in}}%
\pgfpathlineto{\pgfqpoint{1.700137in}{1.434259in}}%
\pgfpathlineto{\pgfqpoint{1.702425in}{1.420648in}}%
\pgfpathlineto{\pgfqpoint{1.697181in}{1.407036in}}%
\pgfpathlineto{\pgfqpoint{1.691223in}{1.400121in}}%
\pgfpathlineto{\pgfqpoint{1.686165in}{1.393425in}}%
\pgfpathlineto{\pgfqpoint{1.675567in}{1.383225in}}%
\pgfpathlineto{\pgfqpoint{1.672179in}{1.379814in}}%
\pgfpathlineto{\pgfqpoint{1.659910in}{1.368951in}}%
\pgfpathlineto{\pgfqpoint{1.656637in}{1.366203in}}%
\pgfpathlineto{\pgfqpoint{1.644253in}{1.355918in}}%
\pgfpathlineto{\pgfqpoint{1.639405in}{1.352592in}}%
\pgfpathlineto{\pgfqpoint{1.628597in}{1.344562in}}%
\pgfpathlineto{\pgfqpoint{1.616881in}{1.338981in}}%
\pgfpathlineto{\pgfqpoint{1.612940in}{1.336760in}}%
\pgfpathlineto{\pgfqpoint{1.597284in}{1.334632in}}%
\pgfpathlineto{\pgfqpoint{1.583354in}{1.338981in}}%
\pgfpathclose%
\pgfpathmoveto{\pgfqpoint{0.488855in}{1.447870in}}%
\pgfpathlineto{\pgfqpoint{0.485668in}{1.449107in}}%
\pgfpathlineto{\pgfqpoint{0.470011in}{1.459325in}}%
\pgfpathlineto{\pgfqpoint{0.467549in}{1.461481in}}%
\pgfpathlineto{\pgfqpoint{0.454354in}{1.470883in}}%
\pgfpathlineto{\pgfqpoint{0.449409in}{1.475092in}}%
\pgfpathlineto{\pgfqpoint{0.438698in}{1.483703in}}%
\pgfpathlineto{\pgfqpoint{0.432931in}{1.488703in}}%
\pgfpathlineto{\pgfqpoint{0.423041in}{1.497763in}}%
\pgfpathlineto{\pgfqpoint{0.417908in}{1.502314in}}%
\pgfpathlineto{\pgfqpoint{0.407385in}{1.513227in}}%
\pgfpathlineto{\pgfqpoint{0.404297in}{1.515925in}}%
\pgfpathlineto{\pgfqpoint{0.392392in}{1.529536in}}%
\pgfpathlineto{\pgfqpoint{0.391728in}{1.530738in}}%
\pgfpathlineto{\pgfqpoint{0.381358in}{1.543148in}}%
\pgfpathlineto{\pgfqpoint{0.377090in}{1.556759in}}%
\pgfpathlineto{\pgfqpoint{0.379935in}{1.570370in}}%
\pgfpathlineto{\pgfqpoint{0.389908in}{1.583981in}}%
\pgfpathlineto{\pgfqpoint{0.391728in}{1.585483in}}%
\pgfpathlineto{\pgfqpoint{0.401553in}{1.597592in}}%
\pgfpathlineto{\pgfqpoint{0.407385in}{1.602903in}}%
\pgfpathlineto{\pgfqpoint{0.415045in}{1.611203in}}%
\pgfpathlineto{\pgfqpoint{0.423041in}{1.618446in}}%
\pgfpathlineto{\pgfqpoint{0.429834in}{1.624814in}}%
\pgfpathlineto{\pgfqpoint{0.438698in}{1.632520in}}%
\pgfpathlineto{\pgfqpoint{0.446023in}{1.638425in}}%
\pgfpathlineto{\pgfqpoint{0.454354in}{1.645377in}}%
\pgfpathlineto{\pgfqpoint{0.463902in}{1.652036in}}%
\pgfpathlineto{\pgfqpoint{0.470011in}{1.657106in}}%
\pgfpathlineto{\pgfqpoint{0.483939in}{1.665648in}}%
\pgfpathlineto{\pgfqpoint{0.485668in}{1.667229in}}%
\pgfpathlineto{\pgfqpoint{0.501324in}{1.675900in}}%
\pgfpathlineto{\pgfqpoint{0.516981in}{1.678373in}}%
\pgfpathlineto{\pgfqpoint{0.532637in}{1.674663in}}%
\pgfpathlineto{\pgfqpoint{0.546912in}{1.665648in}}%
\pgfpathlineto{\pgfqpoint{0.548294in}{1.665071in}}%
\pgfpathlineto{\pgfqpoint{0.563950in}{1.654721in}}%
\pgfpathlineto{\pgfqpoint{0.567054in}{1.652036in}}%
\pgfpathlineto{\pgfqpoint{0.579607in}{1.642888in}}%
\pgfpathlineto{\pgfqpoint{0.584842in}{1.638425in}}%
\pgfpathlineto{\pgfqpoint{0.595263in}{1.629828in}}%
\pgfpathlineto{\pgfqpoint{0.601015in}{1.624814in}}%
\pgfpathlineto{\pgfqpoint{0.610920in}{1.615502in}}%
\pgfpathlineto{\pgfqpoint{0.615762in}{1.611203in}}%
\pgfpathlineto{\pgfqpoint{0.626577in}{1.599732in}}%
\pgfpathlineto{\pgfqpoint{0.629057in}{1.597592in}}%
\pgfpathlineto{\pgfqpoint{0.640810in}{1.583981in}}%
\pgfpathlineto{\pgfqpoint{0.642233in}{1.581209in}}%
\pgfpathlineto{\pgfqpoint{0.651265in}{1.570370in}}%
\pgfpathlineto{\pgfqpoint{0.654517in}{1.556759in}}%
\pgfpathlineto{\pgfqpoint{0.649639in}{1.543148in}}%
\pgfpathlineto{\pgfqpoint{0.642233in}{1.535319in}}%
\pgfpathlineto{\pgfqpoint{0.638851in}{1.529536in}}%
\pgfpathlineto{\pgfqpoint{0.626577in}{1.516390in}}%
\pgfpathlineto{\pgfqpoint{0.626228in}{1.515925in}}%
\pgfpathlineto{\pgfqpoint{0.612799in}{1.502314in}}%
\pgfpathlineto{\pgfqpoint{0.610920in}{1.500681in}}%
\pgfpathlineto{\pgfqpoint{0.597882in}{1.488703in}}%
\pgfpathlineto{\pgfqpoint{0.595263in}{1.486426in}}%
\pgfpathlineto{\pgfqpoint{0.581485in}{1.475092in}}%
\pgfpathlineto{\pgfqpoint{0.579607in}{1.473459in}}%
\pgfpathlineto{\pgfqpoint{0.563950in}{1.461784in}}%
\pgfpathlineto{\pgfqpoint{0.563416in}{1.461481in}}%
\pgfpathlineto{\pgfqpoint{0.548294in}{1.450810in}}%
\pgfpathlineto{\pgfqpoint{0.541642in}{1.447870in}}%
\pgfpathlineto{\pgfqpoint{0.532637in}{1.441432in}}%
\pgfpathlineto{\pgfqpoint{0.516981in}{1.437191in}}%
\pgfpathlineto{\pgfqpoint{0.501324in}{1.440017in}}%
\pgfpathlineto{\pgfqpoint{0.488855in}{1.447870in}}%
\pgfpathclose%
\pgfpathmoveto{\pgfqpoint{0.798286in}{1.447870in}}%
\pgfpathlineto{\pgfqpoint{0.783142in}{1.456942in}}%
\pgfpathlineto{\pgfqpoint{0.777716in}{1.461481in}}%
\pgfpathlineto{\pgfqpoint{0.767486in}{1.468424in}}%
\pgfpathlineto{\pgfqpoint{0.759452in}{1.475092in}}%
\pgfpathlineto{\pgfqpoint{0.751829in}{1.481057in}}%
\pgfpathlineto{\pgfqpoint{0.742952in}{1.488703in}}%
\pgfpathlineto{\pgfqpoint{0.736173in}{1.494870in}}%
\pgfpathlineto{\pgfqpoint{0.727900in}{1.502314in}}%
\pgfpathlineto{\pgfqpoint{0.720516in}{1.510095in}}%
\pgfpathlineto{\pgfqpoint{0.714139in}{1.515925in}}%
\pgfpathlineto{\pgfqpoint{0.704859in}{1.527090in}}%
\pgfpathlineto{\pgfqpoint{0.701996in}{1.529536in}}%
\pgfpathlineto{\pgfqpoint{0.691825in}{1.543148in}}%
\pgfpathlineto{\pgfqpoint{0.689203in}{1.552493in}}%
\pgfpathlineto{\pgfqpoint{0.686441in}{1.556759in}}%
\pgfpathlineto{\pgfqpoint{0.689203in}{1.563135in}}%
\pgfpathlineto{\pgfqpoint{0.690556in}{1.570370in}}%
\pgfpathlineto{\pgfqpoint{0.699452in}{1.583981in}}%
\pgfpathlineto{\pgfqpoint{0.704859in}{1.588934in}}%
\pgfpathlineto{\pgfqpoint{0.711553in}{1.597592in}}%
\pgfpathlineto{\pgfqpoint{0.720516in}{1.606135in}}%
\pgfpathlineto{\pgfqpoint{0.725120in}{1.611203in}}%
\pgfpathlineto{\pgfqpoint{0.736173in}{1.621365in}}%
\pgfpathlineto{\pgfqpoint{0.739878in}{1.624814in}}%
\pgfpathlineto{\pgfqpoint{0.751829in}{1.635135in}}%
\pgfpathlineto{\pgfqpoint{0.756021in}{1.638425in}}%
\pgfpathlineto{\pgfqpoint{0.767486in}{1.647752in}}%
\pgfpathlineto{\pgfqpoint{0.773934in}{1.652036in}}%
\pgfpathlineto{\pgfqpoint{0.783142in}{1.659335in}}%
\pgfpathlineto{\pgfqpoint{0.794298in}{1.665648in}}%
\pgfpathlineto{\pgfqpoint{0.798799in}{1.669461in}}%
\pgfpathlineto{\pgfqpoint{0.814455in}{1.676890in}}%
\pgfpathlineto{\pgfqpoint{0.830112in}{1.678126in}}%
\pgfpathlineto{\pgfqpoint{0.845769in}{1.673177in}}%
\pgfpathlineto{\pgfqpoint{0.856446in}{1.665648in}}%
\pgfpathlineto{\pgfqpoint{0.861425in}{1.663318in}}%
\pgfpathlineto{\pgfqpoint{0.877082in}{1.652180in}}%
\pgfpathlineto{\pgfqpoint{0.877242in}{1.652036in}}%
\pgfpathlineto{\pgfqpoint{0.892738in}{1.640290in}}%
\pgfpathlineto{\pgfqpoint{0.894887in}{1.638425in}}%
\pgfpathlineto{\pgfqpoint{0.908395in}{1.627067in}}%
\pgfpathlineto{\pgfqpoint{0.910983in}{1.624814in}}%
\pgfpathlineto{\pgfqpoint{0.924051in}{1.612546in}}%
\pgfpathlineto{\pgfqpoint{0.925603in}{1.611203in}}%
\pgfpathlineto{\pgfqpoint{0.938840in}{1.597592in}}%
\pgfpathlineto{\pgfqpoint{0.939708in}{1.596383in}}%
\pgfpathlineto{\pgfqpoint{0.951218in}{1.583981in}}%
\pgfpathlineto{\pgfqpoint{0.955364in}{1.576428in}}%
\pgfpathlineto{\pgfqpoint{0.961230in}{1.570370in}}%
\pgfpathlineto{\pgfqpoint{0.965042in}{1.556759in}}%
\pgfpathlineto{\pgfqpoint{0.959323in}{1.543148in}}%
\pgfpathlineto{\pgfqpoint{0.955364in}{1.539546in}}%
\pgfpathlineto{\pgfqpoint{0.949104in}{1.529536in}}%
\pgfpathlineto{\pgfqpoint{0.939708in}{1.520091in}}%
\pgfpathlineto{\pgfqpoint{0.936493in}{1.515925in}}%
\pgfpathlineto{\pgfqpoint{0.924051in}{1.503729in}}%
\pgfpathlineto{\pgfqpoint{0.922731in}{1.502314in}}%
\pgfpathlineto{\pgfqpoint{0.908395in}{1.489210in}}%
\pgfpathlineto{\pgfqpoint{0.907822in}{1.488703in}}%
\pgfpathlineto{\pgfqpoint{0.892738in}{1.476025in}}%
\pgfpathlineto{\pgfqpoint{0.891465in}{1.475092in}}%
\pgfpathlineto{\pgfqpoint{0.877082in}{1.463872in}}%
\pgfpathlineto{\pgfqpoint{0.873136in}{1.461481in}}%
\pgfpathlineto{\pgfqpoint{0.861425in}{1.452684in}}%
\pgfpathlineto{\pgfqpoint{0.851706in}{1.447870in}}%
\pgfpathlineto{\pgfqpoint{0.845769in}{1.443130in}}%
\pgfpathlineto{\pgfqpoint{0.830112in}{1.437473in}}%
\pgfpathlineto{\pgfqpoint{0.814455in}{1.438886in}}%
\pgfpathlineto{\pgfqpoint{0.798799in}{1.447378in}}%
\pgfpathlineto{\pgfqpoint{0.798286in}{1.447870in}}%
\pgfpathclose%
\pgfpathmoveto{\pgfqpoint{1.108796in}{1.447870in}}%
\pgfpathlineto{\pgfqpoint{1.096274in}{1.454728in}}%
\pgfpathlineto{\pgfqpoint{1.087778in}{1.461481in}}%
\pgfpathlineto{\pgfqpoint{1.080617in}{1.466087in}}%
\pgfpathlineto{\pgfqpoint{1.069447in}{1.475092in}}%
\pgfpathlineto{\pgfqpoint{1.064960in}{1.478495in}}%
\pgfpathlineto{\pgfqpoint{1.052986in}{1.488703in}}%
\pgfpathlineto{\pgfqpoint{1.049304in}{1.492016in}}%
\pgfpathlineto{\pgfqpoint{1.037974in}{1.502314in}}%
\pgfpathlineto{\pgfqpoint{1.033647in}{1.506922in}}%
\pgfpathlineto{\pgfqpoint{1.024170in}{1.515925in}}%
\pgfpathlineto{\pgfqpoint{1.017991in}{1.523667in}}%
\pgfpathlineto{\pgfqpoint{1.011699in}{1.529536in}}%
\pgfpathlineto{\pgfqpoint{1.002488in}{1.543148in}}%
\pgfpathlineto{\pgfqpoint{1.002334in}{1.543748in}}%
\pgfpathlineto{\pgfqpoint{0.995700in}{1.556759in}}%
\pgfpathlineto{\pgfqpoint{1.000328in}{1.570370in}}%
\pgfpathlineto{\pgfqpoint{1.002334in}{1.572090in}}%
\pgfpathlineto{\pgfqpoint{1.009395in}{1.583981in}}%
\pgfpathlineto{\pgfqpoint{1.017991in}{1.592577in}}%
\pgfpathlineto{\pgfqpoint{1.021713in}{1.597592in}}%
\pgfpathlineto{\pgfqpoint{1.033647in}{1.609407in}}%
\pgfpathlineto{\pgfqpoint{1.035261in}{1.611203in}}%
\pgfpathlineto{\pgfqpoint{1.049304in}{1.624244in}}%
\pgfpathlineto{\pgfqpoint{1.049923in}{1.624814in}}%
\pgfpathlineto{\pgfqpoint{1.064960in}{1.637668in}}%
\pgfpathlineto{\pgfqpoint{1.065956in}{1.638425in}}%
\pgfpathlineto{\pgfqpoint{1.080617in}{1.650011in}}%
\pgfpathlineto{\pgfqpoint{1.083834in}{1.652036in}}%
\pgfpathlineto{\pgfqpoint{1.096274in}{1.661406in}}%
\pgfpathlineto{\pgfqpoint{1.104473in}{1.665648in}}%
\pgfpathlineto{\pgfqpoint{1.111930in}{1.671444in}}%
\pgfpathlineto{\pgfqpoint{1.127587in}{1.677632in}}%
\pgfpathlineto{\pgfqpoint{1.143243in}{1.677632in}}%
\pgfpathlineto{\pgfqpoint{1.158900in}{1.671444in}}%
\pgfpathlineto{\pgfqpoint{1.166357in}{1.665648in}}%
\pgfpathlineto{\pgfqpoint{1.174556in}{1.661406in}}%
\pgfpathlineto{\pgfqpoint{1.186996in}{1.652036in}}%
\pgfpathlineto{\pgfqpoint{1.190213in}{1.650011in}}%
\pgfpathlineto{\pgfqpoint{1.204874in}{1.638425in}}%
\pgfpathlineto{\pgfqpoint{1.205870in}{1.637668in}}%
\pgfpathlineto{\pgfqpoint{1.220907in}{1.624814in}}%
\pgfpathlineto{\pgfqpoint{1.221526in}{1.624244in}}%
\pgfpathlineto{\pgfqpoint{1.235569in}{1.611203in}}%
\pgfpathlineto{\pgfqpoint{1.237183in}{1.609407in}}%
\pgfpathlineto{\pgfqpoint{1.249117in}{1.597592in}}%
\pgfpathlineto{\pgfqpoint{1.252839in}{1.592577in}}%
\pgfpathlineto{\pgfqpoint{1.261435in}{1.583981in}}%
\pgfpathlineto{\pgfqpoint{1.268496in}{1.572090in}}%
\pgfpathlineto{\pgfqpoint{1.270502in}{1.570370in}}%
\pgfpathlineto{\pgfqpoint{1.275130in}{1.556759in}}%
\pgfpathlineto{\pgfqpoint{1.268496in}{1.543748in}}%
\pgfpathlineto{\pgfqpoint{1.268342in}{1.543148in}}%
\pgfpathlineto{\pgfqpoint{1.259131in}{1.529536in}}%
\pgfpathlineto{\pgfqpoint{1.252839in}{1.523667in}}%
\pgfpathlineto{\pgfqpoint{1.246660in}{1.515925in}}%
\pgfpathlineto{\pgfqpoint{1.237183in}{1.506922in}}%
\pgfpathlineto{\pgfqpoint{1.232856in}{1.502314in}}%
\pgfpathlineto{\pgfqpoint{1.221526in}{1.492016in}}%
\pgfpathlineto{\pgfqpoint{1.217844in}{1.488703in}}%
\pgfpathlineto{\pgfqpoint{1.205870in}{1.478495in}}%
\pgfpathlineto{\pgfqpoint{1.201383in}{1.475092in}}%
\pgfpathlineto{\pgfqpoint{1.190213in}{1.466087in}}%
\pgfpathlineto{\pgfqpoint{1.183052in}{1.461481in}}%
\pgfpathlineto{\pgfqpoint{1.174556in}{1.454728in}}%
\pgfpathlineto{\pgfqpoint{1.162034in}{1.447870in}}%
\pgfpathlineto{\pgfqpoint{1.158900in}{1.445112in}}%
\pgfpathlineto{\pgfqpoint{1.143243in}{1.438039in}}%
\pgfpathlineto{\pgfqpoint{1.127587in}{1.438039in}}%
\pgfpathlineto{\pgfqpoint{1.111930in}{1.445112in}}%
\pgfpathlineto{\pgfqpoint{1.108796in}{1.447870in}}%
\pgfpathclose%
\pgfpathmoveto{\pgfqpoint{1.419124in}{1.447870in}}%
\pgfpathlineto{\pgfqpoint{1.409405in}{1.452684in}}%
\pgfpathlineto{\pgfqpoint{1.397694in}{1.461481in}}%
\pgfpathlineto{\pgfqpoint{1.393748in}{1.463872in}}%
\pgfpathlineto{\pgfqpoint{1.379365in}{1.475092in}}%
\pgfpathlineto{\pgfqpoint{1.378092in}{1.476025in}}%
\pgfpathlineto{\pgfqpoint{1.363008in}{1.488703in}}%
\pgfpathlineto{\pgfqpoint{1.362435in}{1.489210in}}%
\pgfpathlineto{\pgfqpoint{1.348099in}{1.502314in}}%
\pgfpathlineto{\pgfqpoint{1.346779in}{1.503729in}}%
\pgfpathlineto{\pgfqpoint{1.334337in}{1.515925in}}%
\pgfpathlineto{\pgfqpoint{1.331122in}{1.520091in}}%
\pgfpathlineto{\pgfqpoint{1.321726in}{1.529536in}}%
\pgfpathlineto{\pgfqpoint{1.315466in}{1.539546in}}%
\pgfpathlineto{\pgfqpoint{1.311507in}{1.543148in}}%
\pgfpathlineto{\pgfqpoint{1.305788in}{1.556759in}}%
\pgfpathlineto{\pgfqpoint{1.309600in}{1.570370in}}%
\pgfpathlineto{\pgfqpoint{1.315466in}{1.576428in}}%
\pgfpathlineto{\pgfqpoint{1.319612in}{1.583981in}}%
\pgfpathlineto{\pgfqpoint{1.331122in}{1.596383in}}%
\pgfpathlineto{\pgfqpoint{1.331990in}{1.597592in}}%
\pgfpathlineto{\pgfqpoint{1.345227in}{1.611203in}}%
\pgfpathlineto{\pgfqpoint{1.346779in}{1.612546in}}%
\pgfpathlineto{\pgfqpoint{1.359847in}{1.624814in}}%
\pgfpathlineto{\pgfqpoint{1.362435in}{1.627067in}}%
\pgfpathlineto{\pgfqpoint{1.375943in}{1.638425in}}%
\pgfpathlineto{\pgfqpoint{1.378092in}{1.640290in}}%
\pgfpathlineto{\pgfqpoint{1.393588in}{1.652036in}}%
\pgfpathlineto{\pgfqpoint{1.393748in}{1.652180in}}%
\pgfpathlineto{\pgfqpoint{1.409405in}{1.663318in}}%
\pgfpathlineto{\pgfqpoint{1.414384in}{1.665648in}}%
\pgfpathlineto{\pgfqpoint{1.425061in}{1.673177in}}%
\pgfpathlineto{\pgfqpoint{1.440718in}{1.678126in}}%
\pgfpathlineto{\pgfqpoint{1.456375in}{1.676890in}}%
\pgfpathlineto{\pgfqpoint{1.472031in}{1.669461in}}%
\pgfpathlineto{\pgfqpoint{1.476532in}{1.665648in}}%
\pgfpathlineto{\pgfqpoint{1.487688in}{1.659335in}}%
\pgfpathlineto{\pgfqpoint{1.496896in}{1.652036in}}%
\pgfpathlineto{\pgfqpoint{1.503344in}{1.647752in}}%
\pgfpathlineto{\pgfqpoint{1.514809in}{1.638425in}}%
\pgfpathlineto{\pgfqpoint{1.519001in}{1.635135in}}%
\pgfpathlineto{\pgfqpoint{1.530952in}{1.624814in}}%
\pgfpathlineto{\pgfqpoint{1.534657in}{1.621365in}}%
\pgfpathlineto{\pgfqpoint{1.545710in}{1.611203in}}%
\pgfpathlineto{\pgfqpoint{1.550314in}{1.606135in}}%
\pgfpathlineto{\pgfqpoint{1.559277in}{1.597592in}}%
\pgfpathlineto{\pgfqpoint{1.565971in}{1.588934in}}%
\pgfpathlineto{\pgfqpoint{1.571378in}{1.583981in}}%
\pgfpathlineto{\pgfqpoint{1.580274in}{1.570370in}}%
\pgfpathlineto{\pgfqpoint{1.581627in}{1.563135in}}%
\pgfpathlineto{\pgfqpoint{1.584389in}{1.556759in}}%
\pgfpathlineto{\pgfqpoint{1.581627in}{1.552493in}}%
\pgfpathlineto{\pgfqpoint{1.579005in}{1.543148in}}%
\pgfpathlineto{\pgfqpoint{1.568834in}{1.529536in}}%
\pgfpathlineto{\pgfqpoint{1.565971in}{1.527090in}}%
\pgfpathlineto{\pgfqpoint{1.556691in}{1.515925in}}%
\pgfpathlineto{\pgfqpoint{1.550314in}{1.510095in}}%
\pgfpathlineto{\pgfqpoint{1.542930in}{1.502314in}}%
\pgfpathlineto{\pgfqpoint{1.534657in}{1.494870in}}%
\pgfpathlineto{\pgfqpoint{1.527878in}{1.488703in}}%
\pgfpathlineto{\pgfqpoint{1.519001in}{1.481057in}}%
\pgfpathlineto{\pgfqpoint{1.511378in}{1.475092in}}%
\pgfpathlineto{\pgfqpoint{1.503344in}{1.468424in}}%
\pgfpathlineto{\pgfqpoint{1.493114in}{1.461481in}}%
\pgfpathlineto{\pgfqpoint{1.487688in}{1.456942in}}%
\pgfpathlineto{\pgfqpoint{1.472544in}{1.447870in}}%
\pgfpathlineto{\pgfqpoint{1.472031in}{1.447378in}}%
\pgfpathlineto{\pgfqpoint{1.456375in}{1.438886in}}%
\pgfpathlineto{\pgfqpoint{1.440718in}{1.437473in}}%
\pgfpathlineto{\pgfqpoint{1.425061in}{1.443130in}}%
\pgfpathlineto{\pgfqpoint{1.419124in}{1.447870in}}%
\pgfpathclose%
\pgfpathmoveto{\pgfqpoint{1.729188in}{1.447870in}}%
\pgfpathlineto{\pgfqpoint{1.722536in}{1.450810in}}%
\pgfpathlineto{\pgfqpoint{1.707414in}{1.461481in}}%
\pgfpathlineto{\pgfqpoint{1.706880in}{1.461784in}}%
\pgfpathlineto{\pgfqpoint{1.691223in}{1.473459in}}%
\pgfpathlineto{\pgfqpoint{1.689345in}{1.475092in}}%
\pgfpathlineto{\pgfqpoint{1.675567in}{1.486426in}}%
\pgfpathlineto{\pgfqpoint{1.672948in}{1.488703in}}%
\pgfpathlineto{\pgfqpoint{1.659910in}{1.500681in}}%
\pgfpathlineto{\pgfqpoint{1.658031in}{1.502314in}}%
\pgfpathlineto{\pgfqpoint{1.644602in}{1.515925in}}%
\pgfpathlineto{\pgfqpoint{1.644253in}{1.516390in}}%
\pgfpathlineto{\pgfqpoint{1.631979in}{1.529536in}}%
\pgfpathlineto{\pgfqpoint{1.628597in}{1.535319in}}%
\pgfpathlineto{\pgfqpoint{1.621191in}{1.543148in}}%
\pgfpathlineto{\pgfqpoint{1.616313in}{1.556759in}}%
\pgfpathlineto{\pgfqpoint{1.619565in}{1.570370in}}%
\pgfpathlineto{\pgfqpoint{1.628597in}{1.581209in}}%
\pgfpathlineto{\pgfqpoint{1.630020in}{1.583981in}}%
\pgfpathlineto{\pgfqpoint{1.641773in}{1.597592in}}%
\pgfpathlineto{\pgfqpoint{1.644253in}{1.599732in}}%
\pgfpathlineto{\pgfqpoint{1.655068in}{1.611203in}}%
\pgfpathlineto{\pgfqpoint{1.659910in}{1.615502in}}%
\pgfpathlineto{\pgfqpoint{1.669815in}{1.624814in}}%
\pgfpathlineto{\pgfqpoint{1.675567in}{1.629828in}}%
\pgfpathlineto{\pgfqpoint{1.685988in}{1.638425in}}%
\pgfpathlineto{\pgfqpoint{1.691223in}{1.642888in}}%
\pgfpathlineto{\pgfqpoint{1.703776in}{1.652036in}}%
\pgfpathlineto{\pgfqpoint{1.706880in}{1.654721in}}%
\pgfpathlineto{\pgfqpoint{1.722536in}{1.665071in}}%
\pgfpathlineto{\pgfqpoint{1.723918in}{1.665648in}}%
\pgfpathlineto{\pgfqpoint{1.738193in}{1.674663in}}%
\pgfpathlineto{\pgfqpoint{1.753849in}{1.678373in}}%
\pgfpathlineto{\pgfqpoint{1.769506in}{1.675900in}}%
\pgfpathlineto{\pgfqpoint{1.785162in}{1.667229in}}%
\pgfpathlineto{\pgfqpoint{1.786891in}{1.665648in}}%
\pgfpathlineto{\pgfqpoint{1.800819in}{1.657106in}}%
\pgfpathlineto{\pgfqpoint{1.806928in}{1.652036in}}%
\pgfpathlineto{\pgfqpoint{1.816476in}{1.645377in}}%
\pgfpathlineto{\pgfqpoint{1.824807in}{1.638425in}}%
\pgfpathlineto{\pgfqpoint{1.832132in}{1.632520in}}%
\pgfpathlineto{\pgfqpoint{1.840996in}{1.624814in}}%
\pgfpathlineto{\pgfqpoint{1.847789in}{1.618446in}}%
\pgfpathlineto{\pgfqpoint{1.855785in}{1.611203in}}%
\pgfpathlineto{\pgfqpoint{1.863445in}{1.602903in}}%
\pgfpathlineto{\pgfqpoint{1.869277in}{1.597592in}}%
\pgfpathlineto{\pgfqpoint{1.879102in}{1.585483in}}%
\pgfpathlineto{\pgfqpoint{1.880922in}{1.583981in}}%
\pgfpathlineto{\pgfqpoint{1.890895in}{1.570370in}}%
\pgfpathlineto{\pgfqpoint{1.893740in}{1.556759in}}%
\pgfpathlineto{\pgfqpoint{1.889472in}{1.543148in}}%
\pgfpathlineto{\pgfqpoint{1.879102in}{1.530738in}}%
\pgfpathlineto{\pgfqpoint{1.878438in}{1.529536in}}%
\pgfpathlineto{\pgfqpoint{1.866533in}{1.515925in}}%
\pgfpathlineto{\pgfqpoint{1.863445in}{1.513227in}}%
\pgfpathlineto{\pgfqpoint{1.852922in}{1.502314in}}%
\pgfpathlineto{\pgfqpoint{1.847789in}{1.497763in}}%
\pgfpathlineto{\pgfqpoint{1.837899in}{1.488703in}}%
\pgfpathlineto{\pgfqpoint{1.832132in}{1.483703in}}%
\pgfpathlineto{\pgfqpoint{1.821421in}{1.475092in}}%
\pgfpathlineto{\pgfqpoint{1.816476in}{1.470883in}}%
\pgfpathlineto{\pgfqpoint{1.803281in}{1.461481in}}%
\pgfpathlineto{\pgfqpoint{1.800819in}{1.459325in}}%
\pgfpathlineto{\pgfqpoint{1.785162in}{1.449107in}}%
\pgfpathlineto{\pgfqpoint{1.781975in}{1.447870in}}%
\pgfpathlineto{\pgfqpoint{1.769506in}{1.440017in}}%
\pgfpathlineto{\pgfqpoint{1.753849in}{1.437191in}}%
\pgfpathlineto{\pgfqpoint{1.738193in}{1.441432in}}%
\pgfpathlineto{\pgfqpoint{1.729188in}{1.447870in}}%
\pgfpathclose%
\pgfusepath{fill}%
\end{pgfscope}%
\begin{pgfscope}%
\pgfpathrectangle{\pgfqpoint{0.360415in}{0.345370in}}{\pgfqpoint{1.550000in}{1.347500in}}%
\pgfusepath{clip}%
\pgfsetbuttcap%
\pgfsetroundjoin%
\definecolor{currentfill}{rgb}{0.534952,0.031217,0.650165}%
\pgfsetfillcolor{currentfill}%
\pgfsetlinewidth{0.000000pt}%
\definecolor{currentstroke}{rgb}{0.000000,0.000000,0.000000}%
\pgfsetstrokecolor{currentstroke}%
\pgfsetdash{}{0pt}%
\pgfpathmoveto{\pgfqpoint{0.438698in}{0.345370in}}%
\pgfpathlineto{\pgfqpoint{0.454354in}{0.345370in}}%
\pgfpathlineto{\pgfqpoint{0.462667in}{0.345370in}}%
\pgfpathlineto{\pgfqpoint{0.458762in}{0.358981in}}%
\pgfpathlineto{\pgfqpoint{0.454354in}{0.364438in}}%
\pgfpathlineto{\pgfqpoint{0.448681in}{0.372592in}}%
\pgfpathlineto{\pgfqpoint{0.438698in}{0.382447in}}%
\pgfpathlineto{\pgfqpoint{0.435123in}{0.386203in}}%
\pgfpathlineto{\pgfqpoint{0.423041in}{0.397041in}}%
\pgfpathlineto{\pgfqpoint{0.419852in}{0.399814in}}%
\pgfpathlineto{\pgfqpoint{0.407385in}{0.410318in}}%
\pgfpathlineto{\pgfqpoint{0.403064in}{0.413425in}}%
\pgfpathlineto{\pgfqpoint{0.391728in}{0.422104in}}%
\pgfpathlineto{\pgfqpoint{0.382348in}{0.427036in}}%
\pgfpathlineto{\pgfqpoint{0.376072in}{0.430868in}}%
\pgfpathlineto{\pgfqpoint{0.360415in}{0.434263in}}%
\pgfpathlineto{\pgfqpoint{0.360415in}{0.427036in}}%
\pgfpathlineto{\pgfqpoint{0.360415in}{0.413425in}}%
\pgfpathlineto{\pgfqpoint{0.360415in}{0.406467in}}%
\pgfpathlineto{\pgfqpoint{0.376072in}{0.403813in}}%
\pgfpathlineto{\pgfqpoint{0.384496in}{0.399814in}}%
\pgfpathlineto{\pgfqpoint{0.391728in}{0.396164in}}%
\pgfpathlineto{\pgfqpoint{0.405324in}{0.386203in}}%
\pgfpathlineto{\pgfqpoint{0.407385in}{0.384412in}}%
\pgfpathlineto{\pgfqpoint{0.418843in}{0.372592in}}%
\pgfpathlineto{\pgfqpoint{0.423041in}{0.366304in}}%
\pgfpathlineto{\pgfqpoint{0.427641in}{0.358981in}}%
\pgfpathlineto{\pgfqpoint{0.430694in}{0.345370in}}%
\pgfpathlineto{\pgfqpoint{0.438698in}{0.345370in}}%
\pgfpathclose%
\pgfpathmoveto{\pgfqpoint{0.579607in}{0.345370in}}%
\pgfpathlineto{\pgfqpoint{0.595263in}{0.345370in}}%
\pgfpathlineto{\pgfqpoint{0.600145in}{0.345370in}}%
\pgfpathlineto{\pgfqpoint{0.603234in}{0.358981in}}%
\pgfpathlineto{\pgfqpoint{0.610920in}{0.371254in}}%
\pgfpathlineto{\pgfqpoint{0.611831in}{0.372592in}}%
\pgfpathlineto{\pgfqpoint{0.625383in}{0.386203in}}%
\pgfpathlineto{\pgfqpoint{0.626577in}{0.387219in}}%
\pgfpathlineto{\pgfqpoint{0.642233in}{0.398112in}}%
\pgfpathlineto{\pgfqpoint{0.646052in}{0.399814in}}%
\pgfpathlineto{\pgfqpoint{0.657890in}{0.404755in}}%
\pgfpathlineto{\pgfqpoint{0.673546in}{0.406359in}}%
\pgfpathlineto{\pgfqpoint{0.689203in}{0.402682in}}%
\pgfpathlineto{\pgfqpoint{0.694624in}{0.399814in}}%
\pgfpathlineto{\pgfqpoint{0.704859in}{0.394084in}}%
\pgfpathlineto{\pgfqpoint{0.715108in}{0.386203in}}%
\pgfpathlineto{\pgfqpoint{0.720516in}{0.381283in}}%
\pgfpathlineto{\pgfqpoint{0.728808in}{0.372592in}}%
\pgfpathlineto{\pgfqpoint{0.736173in}{0.361398in}}%
\pgfpathlineto{\pgfqpoint{0.737701in}{0.358981in}}%
\pgfpathlineto{\pgfqpoint{0.740732in}{0.345370in}}%
\pgfpathlineto{\pgfqpoint{0.751829in}{0.345370in}}%
\pgfpathlineto{\pgfqpoint{0.767486in}{0.345370in}}%
\pgfpathlineto{\pgfqpoint{0.772653in}{0.345370in}}%
\pgfpathlineto{\pgfqpoint{0.768603in}{0.358981in}}%
\pgfpathlineto{\pgfqpoint{0.767486in}{0.360299in}}%
\pgfpathlineto{\pgfqpoint{0.758714in}{0.372592in}}%
\pgfpathlineto{\pgfqpoint{0.751829in}{0.379208in}}%
\pgfpathlineto{\pgfqpoint{0.745128in}{0.386203in}}%
\pgfpathlineto{\pgfqpoint{0.736173in}{0.394180in}}%
\pgfpathlineto{\pgfqpoint{0.729787in}{0.399814in}}%
\pgfpathlineto{\pgfqpoint{0.720516in}{0.407751in}}%
\pgfpathlineto{\pgfqpoint{0.712977in}{0.413425in}}%
\pgfpathlineto{\pgfqpoint{0.704859in}{0.419948in}}%
\pgfpathlineto{\pgfqpoint{0.692709in}{0.427036in}}%
\pgfpathlineto{\pgfqpoint{0.689203in}{0.429421in}}%
\pgfpathlineto{\pgfqpoint{0.673546in}{0.434124in}}%
\pgfpathlineto{\pgfqpoint{0.657890in}{0.432073in}}%
\pgfpathlineto{\pgfqpoint{0.648506in}{0.427036in}}%
\pgfpathlineto{\pgfqpoint{0.642233in}{0.424122in}}%
\pgfpathlineto{\pgfqpoint{0.627441in}{0.413425in}}%
\pgfpathlineto{\pgfqpoint{0.626577in}{0.412836in}}%
\pgfpathlineto{\pgfqpoint{0.610920in}{0.399916in}}%
\pgfpathlineto{\pgfqpoint{0.610803in}{0.399814in}}%
\pgfpathlineto{\pgfqpoint{0.595665in}{0.386203in}}%
\pgfpathlineto{\pgfqpoint{0.595263in}{0.385780in}}%
\pgfpathlineto{\pgfqpoint{0.582207in}{0.372592in}}%
\pgfpathlineto{\pgfqpoint{0.579607in}{0.368774in}}%
\pgfpathlineto{\pgfqpoint{0.572036in}{0.358981in}}%
\pgfpathlineto{\pgfqpoint{0.568251in}{0.345370in}}%
\pgfpathlineto{\pgfqpoint{0.579607in}{0.345370in}}%
\pgfpathclose%
\pgfpathmoveto{\pgfqpoint{0.892738in}{0.345370in}}%
\pgfpathlineto{\pgfqpoint{0.908395in}{0.345370in}}%
\pgfpathlineto{\pgfqpoint{0.910099in}{0.345370in}}%
\pgfpathlineto{\pgfqpoint{0.913237in}{0.358981in}}%
\pgfpathlineto{\pgfqpoint{0.921862in}{0.372592in}}%
\pgfpathlineto{\pgfqpoint{0.924051in}{0.374925in}}%
\pgfpathlineto{\pgfqpoint{0.935614in}{0.386203in}}%
\pgfpathlineto{\pgfqpoint{0.939708in}{0.389593in}}%
\pgfpathlineto{\pgfqpoint{0.955228in}{0.399814in}}%
\pgfpathlineto{\pgfqpoint{0.955364in}{0.399900in}}%
\pgfpathlineto{\pgfqpoint{0.971021in}{0.405498in}}%
\pgfpathlineto{\pgfqpoint{0.986678in}{0.406034in}}%
\pgfpathlineto{\pgfqpoint{1.002334in}{0.401373in}}%
\pgfpathlineto{\pgfqpoint{1.005022in}{0.399814in}}%
\pgfpathlineto{\pgfqpoint{1.017991in}{0.391888in}}%
\pgfpathlineto{\pgfqpoint{1.025089in}{0.386203in}}%
\pgfpathlineto{\pgfqpoint{1.033647in}{0.378114in}}%
\pgfpathlineto{\pgfqpoint{1.038860in}{0.372592in}}%
\pgfpathlineto{\pgfqpoint{1.047660in}{0.358981in}}%
\pgfpathlineto{\pgfqpoint{1.049304in}{0.351944in}}%
\pgfpathlineto{\pgfqpoint{1.050774in}{0.345370in}}%
\pgfpathlineto{\pgfqpoint{1.064960in}{0.345370in}}%
\pgfpathlineto{\pgfqpoint{1.080617in}{0.345370in}}%
\pgfpathlineto{\pgfqpoint{1.082498in}{0.345370in}}%
\pgfpathlineto{\pgfqpoint{1.080617in}{0.351384in}}%
\pgfpathlineto{\pgfqpoint{1.078616in}{0.358981in}}%
\pgfpathlineto{\pgfqpoint{1.068696in}{0.372592in}}%
\pgfpathlineto{\pgfqpoint{1.064960in}{0.376072in}}%
\pgfpathlineto{\pgfqpoint{1.055155in}{0.386203in}}%
\pgfpathlineto{\pgfqpoint{1.049304in}{0.391357in}}%
\pgfpathlineto{\pgfqpoint{1.039815in}{0.399814in}}%
\pgfpathlineto{\pgfqpoint{1.033647in}{0.405152in}}%
\pgfpathlineto{\pgfqpoint{1.023066in}{0.413425in}}%
\pgfpathlineto{\pgfqpoint{1.017991in}{0.417672in}}%
\pgfpathlineto{\pgfqpoint{1.003288in}{0.427036in}}%
\pgfpathlineto{\pgfqpoint{1.002334in}{0.427748in}}%
\pgfpathlineto{\pgfqpoint{0.986678in}{0.433710in}}%
\pgfpathlineto{\pgfqpoint{0.971021in}{0.433023in}}%
\pgfpathlineto{\pgfqpoint{0.957995in}{0.427036in}}%
\pgfpathlineto{\pgfqpoint{0.955364in}{0.425985in}}%
\pgfpathlineto{\pgfqpoint{0.939708in}{0.415295in}}%
\pgfpathlineto{\pgfqpoint{0.937548in}{0.413425in}}%
\pgfpathlineto{\pgfqpoint{0.924051in}{0.402535in}}%
\pgfpathlineto{\pgfqpoint{0.920926in}{0.399814in}}%
\pgfpathlineto{\pgfqpoint{0.908395in}{0.388583in}}%
\pgfpathlineto{\pgfqpoint{0.905652in}{0.386203in}}%
\pgfpathlineto{\pgfqpoint{0.892738in}{0.373047in}}%
\pgfpathlineto{\pgfqpoint{0.892232in}{0.372592in}}%
\pgfpathlineto{\pgfqpoint{0.882095in}{0.358981in}}%
\pgfpathlineto{\pgfqpoint{0.878408in}{0.345370in}}%
\pgfpathlineto{\pgfqpoint{0.892738in}{0.345370in}}%
\pgfpathclose%
\pgfpathmoveto{\pgfqpoint{1.190213in}{0.345370in}}%
\pgfpathlineto{\pgfqpoint{1.205870in}{0.345370in}}%
\pgfpathlineto{\pgfqpoint{1.220056in}{0.345370in}}%
\pgfpathlineto{\pgfqpoint{1.221526in}{0.351944in}}%
\pgfpathlineto{\pgfqpoint{1.223170in}{0.358981in}}%
\pgfpathlineto{\pgfqpoint{1.231970in}{0.372592in}}%
\pgfpathlineto{\pgfqpoint{1.237183in}{0.378114in}}%
\pgfpathlineto{\pgfqpoint{1.245741in}{0.386203in}}%
\pgfpathlineto{\pgfqpoint{1.252839in}{0.391888in}}%
\pgfpathlineto{\pgfqpoint{1.265808in}{0.399814in}}%
\pgfpathlineto{\pgfqpoint{1.268496in}{0.401373in}}%
\pgfpathlineto{\pgfqpoint{1.284152in}{0.406034in}}%
\pgfpathlineto{\pgfqpoint{1.299809in}{0.405498in}}%
\pgfpathlineto{\pgfqpoint{1.315466in}{0.399900in}}%
\pgfpathlineto{\pgfqpoint{1.315602in}{0.399814in}}%
\pgfpathlineto{\pgfqpoint{1.331122in}{0.389593in}}%
\pgfpathlineto{\pgfqpoint{1.335216in}{0.386203in}}%
\pgfpathlineto{\pgfqpoint{1.346779in}{0.374925in}}%
\pgfpathlineto{\pgfqpoint{1.348968in}{0.372592in}}%
\pgfpathlineto{\pgfqpoint{1.357593in}{0.358981in}}%
\pgfpathlineto{\pgfqpoint{1.360731in}{0.345370in}}%
\pgfpathlineto{\pgfqpoint{1.362435in}{0.345370in}}%
\pgfpathlineto{\pgfqpoint{1.378092in}{0.345370in}}%
\pgfpathlineto{\pgfqpoint{1.392422in}{0.345370in}}%
\pgfpathlineto{\pgfqpoint{1.388735in}{0.358981in}}%
\pgfpathlineto{\pgfqpoint{1.378598in}{0.372592in}}%
\pgfpathlineto{\pgfqpoint{1.378092in}{0.373047in}}%
\pgfpathlineto{\pgfqpoint{1.365178in}{0.386203in}}%
\pgfpathlineto{\pgfqpoint{1.362435in}{0.388583in}}%
\pgfpathlineto{\pgfqpoint{1.349904in}{0.399814in}}%
\pgfpathlineto{\pgfqpoint{1.346779in}{0.402535in}}%
\pgfpathlineto{\pgfqpoint{1.333282in}{0.413425in}}%
\pgfpathlineto{\pgfqpoint{1.331122in}{0.415295in}}%
\pgfpathlineto{\pgfqpoint{1.315466in}{0.425985in}}%
\pgfpathlineto{\pgfqpoint{1.312835in}{0.427036in}}%
\pgfpathlineto{\pgfqpoint{1.299809in}{0.433023in}}%
\pgfpathlineto{\pgfqpoint{1.284152in}{0.433710in}}%
\pgfpathlineto{\pgfqpoint{1.268496in}{0.427748in}}%
\pgfpathlineto{\pgfqpoint{1.267542in}{0.427036in}}%
\pgfpathlineto{\pgfqpoint{1.252839in}{0.417672in}}%
\pgfpathlineto{\pgfqpoint{1.247764in}{0.413425in}}%
\pgfpathlineto{\pgfqpoint{1.237183in}{0.405152in}}%
\pgfpathlineto{\pgfqpoint{1.231015in}{0.399814in}}%
\pgfpathlineto{\pgfqpoint{1.221526in}{0.391357in}}%
\pgfpathlineto{\pgfqpoint{1.215675in}{0.386203in}}%
\pgfpathlineto{\pgfqpoint{1.205870in}{0.376072in}}%
\pgfpathlineto{\pgfqpoint{1.202134in}{0.372592in}}%
\pgfpathlineto{\pgfqpoint{1.192214in}{0.358981in}}%
\pgfpathlineto{\pgfqpoint{1.190213in}{0.351384in}}%
\pgfpathlineto{\pgfqpoint{1.188332in}{0.345370in}}%
\pgfpathlineto{\pgfqpoint{1.190213in}{0.345370in}}%
\pgfpathclose%
\pgfpathmoveto{\pgfqpoint{1.503344in}{0.345370in}}%
\pgfpathlineto{\pgfqpoint{1.519001in}{0.345370in}}%
\pgfpathlineto{\pgfqpoint{1.530098in}{0.345370in}}%
\pgfpathlineto{\pgfqpoint{1.533129in}{0.358981in}}%
\pgfpathlineto{\pgfqpoint{1.534657in}{0.361398in}}%
\pgfpathlineto{\pgfqpoint{1.542022in}{0.372592in}}%
\pgfpathlineto{\pgfqpoint{1.550314in}{0.381283in}}%
\pgfpathlineto{\pgfqpoint{1.555722in}{0.386203in}}%
\pgfpathlineto{\pgfqpoint{1.565971in}{0.394084in}}%
\pgfpathlineto{\pgfqpoint{1.576206in}{0.399814in}}%
\pgfpathlineto{\pgfqpoint{1.581627in}{0.402682in}}%
\pgfpathlineto{\pgfqpoint{1.597284in}{0.406359in}}%
\pgfpathlineto{\pgfqpoint{1.612940in}{0.404755in}}%
\pgfpathlineto{\pgfqpoint{1.624778in}{0.399814in}}%
\pgfpathlineto{\pgfqpoint{1.628597in}{0.398112in}}%
\pgfpathlineto{\pgfqpoint{1.644253in}{0.387219in}}%
\pgfpathlineto{\pgfqpoint{1.645447in}{0.386203in}}%
\pgfpathlineto{\pgfqpoint{1.658999in}{0.372592in}}%
\pgfpathlineto{\pgfqpoint{1.659910in}{0.371254in}}%
\pgfpathlineto{\pgfqpoint{1.667596in}{0.358981in}}%
\pgfpathlineto{\pgfqpoint{1.670685in}{0.345370in}}%
\pgfpathlineto{\pgfqpoint{1.675567in}{0.345370in}}%
\pgfpathlineto{\pgfqpoint{1.691223in}{0.345370in}}%
\pgfpathlineto{\pgfqpoint{1.702579in}{0.345370in}}%
\pgfpathlineto{\pgfqpoint{1.698794in}{0.358981in}}%
\pgfpathlineto{\pgfqpoint{1.691223in}{0.368774in}}%
\pgfpathlineto{\pgfqpoint{1.688623in}{0.372592in}}%
\pgfpathlineto{\pgfqpoint{1.675567in}{0.385780in}}%
\pgfpathlineto{\pgfqpoint{1.675165in}{0.386203in}}%
\pgfpathlineto{\pgfqpoint{1.660027in}{0.399814in}}%
\pgfpathlineto{\pgfqpoint{1.659910in}{0.399916in}}%
\pgfpathlineto{\pgfqpoint{1.644253in}{0.412836in}}%
\pgfpathlineto{\pgfqpoint{1.643389in}{0.413425in}}%
\pgfpathlineto{\pgfqpoint{1.628597in}{0.424122in}}%
\pgfpathlineto{\pgfqpoint{1.622324in}{0.427036in}}%
\pgfpathlineto{\pgfqpoint{1.612940in}{0.432073in}}%
\pgfpathlineto{\pgfqpoint{1.597284in}{0.434124in}}%
\pgfpathlineto{\pgfqpoint{1.581627in}{0.429421in}}%
\pgfpathlineto{\pgfqpoint{1.578121in}{0.427036in}}%
\pgfpathlineto{\pgfqpoint{1.565971in}{0.419948in}}%
\pgfpathlineto{\pgfqpoint{1.557853in}{0.413425in}}%
\pgfpathlineto{\pgfqpoint{1.550314in}{0.407751in}}%
\pgfpathlineto{\pgfqpoint{1.541043in}{0.399814in}}%
\pgfpathlineto{\pgfqpoint{1.534657in}{0.394180in}}%
\pgfpathlineto{\pgfqpoint{1.525702in}{0.386203in}}%
\pgfpathlineto{\pgfqpoint{1.519001in}{0.379208in}}%
\pgfpathlineto{\pgfqpoint{1.512116in}{0.372592in}}%
\pgfpathlineto{\pgfqpoint{1.503344in}{0.360299in}}%
\pgfpathlineto{\pgfqpoint{1.502227in}{0.358981in}}%
\pgfpathlineto{\pgfqpoint{1.498177in}{0.345370in}}%
\pgfpathlineto{\pgfqpoint{1.503344in}{0.345370in}}%
\pgfpathclose%
\pgfpathmoveto{\pgfqpoint{1.816476in}{0.345370in}}%
\pgfpathlineto{\pgfqpoint{1.832132in}{0.345370in}}%
\pgfpathlineto{\pgfqpoint{1.840136in}{0.345370in}}%
\pgfpathlineto{\pgfqpoint{1.843189in}{0.358981in}}%
\pgfpathlineto{\pgfqpoint{1.847789in}{0.366304in}}%
\pgfpathlineto{\pgfqpoint{1.851987in}{0.372592in}}%
\pgfpathlineto{\pgfqpoint{1.863445in}{0.384412in}}%
\pgfpathlineto{\pgfqpoint{1.865506in}{0.386203in}}%
\pgfpathlineto{\pgfqpoint{1.879102in}{0.396164in}}%
\pgfpathlineto{\pgfqpoint{1.886334in}{0.399814in}}%
\pgfpathlineto{\pgfqpoint{1.894758in}{0.403813in}}%
\pgfpathlineto{\pgfqpoint{1.910415in}{0.406467in}}%
\pgfpathlineto{\pgfqpoint{1.910415in}{0.413425in}}%
\pgfpathlineto{\pgfqpoint{1.910415in}{0.427036in}}%
\pgfpathlineto{\pgfqpoint{1.910415in}{0.434263in}}%
\pgfpathlineto{\pgfqpoint{1.894758in}{0.430868in}}%
\pgfpathlineto{\pgfqpoint{1.888482in}{0.427036in}}%
\pgfpathlineto{\pgfqpoint{1.879102in}{0.422104in}}%
\pgfpathlineto{\pgfqpoint{1.867766in}{0.413425in}}%
\pgfpathlineto{\pgfqpoint{1.863445in}{0.410318in}}%
\pgfpathlineto{\pgfqpoint{1.850978in}{0.399814in}}%
\pgfpathlineto{\pgfqpoint{1.847789in}{0.397041in}}%
\pgfpathlineto{\pgfqpoint{1.835707in}{0.386203in}}%
\pgfpathlineto{\pgfqpoint{1.832132in}{0.382447in}}%
\pgfpathlineto{\pgfqpoint{1.822149in}{0.372592in}}%
\pgfpathlineto{\pgfqpoint{1.816476in}{0.364438in}}%
\pgfpathlineto{\pgfqpoint{1.812068in}{0.358981in}}%
\pgfpathlineto{\pgfqpoint{1.808163in}{0.345370in}}%
\pgfpathlineto{\pgfqpoint{1.816476in}{0.345370in}}%
\pgfpathclose%
\pgfpathmoveto{\pgfqpoint{0.376072in}{0.529344in}}%
\pgfpathlineto{\pgfqpoint{0.387336in}{0.535925in}}%
\pgfpathlineto{\pgfqpoint{0.391728in}{0.538185in}}%
\pgfpathlineto{\pgfqpoint{0.406899in}{0.549536in}}%
\pgfpathlineto{\pgfqpoint{0.407385in}{0.549885in}}%
\pgfpathlineto{\pgfqpoint{0.423041in}{0.563046in}}%
\pgfpathlineto{\pgfqpoint{0.423159in}{0.563148in}}%
\pgfpathlineto{\pgfqpoint{0.438020in}{0.576759in}}%
\pgfpathlineto{\pgfqpoint{0.438698in}{0.577510in}}%
\pgfpathlineto{\pgfqpoint{0.451002in}{0.590370in}}%
\pgfpathlineto{\pgfqpoint{0.454354in}{0.595823in}}%
\pgfpathlineto{\pgfqpoint{0.460148in}{0.603981in}}%
\pgfpathlineto{\pgfqpoint{0.462508in}{0.617592in}}%
\pgfpathlineto{\pgfqpoint{0.457098in}{0.631203in}}%
\pgfpathlineto{\pgfqpoint{0.454354in}{0.634251in}}%
\pgfpathlineto{\pgfqpoint{0.446201in}{0.644814in}}%
\pgfpathlineto{\pgfqpoint{0.438698in}{0.651871in}}%
\pgfpathlineto{\pgfqpoint{0.432171in}{0.658425in}}%
\pgfpathlineto{\pgfqpoint{0.423041in}{0.666485in}}%
\pgfpathlineto{\pgfqpoint{0.416561in}{0.672036in}}%
\pgfpathlineto{\pgfqpoint{0.407385in}{0.679822in}}%
\pgfpathlineto{\pgfqpoint{0.399338in}{0.685648in}}%
\pgfpathlineto{\pgfqpoint{0.391728in}{0.691633in}}%
\pgfpathlineto{\pgfqpoint{0.377587in}{0.699259in}}%
\pgfpathlineto{\pgfqpoint{0.376072in}{0.700230in}}%
\pgfpathlineto{\pgfqpoint{0.360415in}{0.703751in}}%
\pgfpathlineto{\pgfqpoint{0.360415in}{0.699259in}}%
\pgfpathlineto{\pgfqpoint{0.360415in}{0.685648in}}%
\pgfpathlineto{\pgfqpoint{0.360415in}{0.676000in}}%
\pgfpathlineto{\pgfqpoint{0.376072in}{0.673365in}}%
\pgfpathlineto{\pgfqpoint{0.378852in}{0.672036in}}%
\pgfpathlineto{\pgfqpoint{0.391728in}{0.665634in}}%
\pgfpathlineto{\pgfqpoint{0.401725in}{0.658425in}}%
\pgfpathlineto{\pgfqpoint{0.407385in}{0.653724in}}%
\pgfpathlineto{\pgfqpoint{0.416450in}{0.644814in}}%
\pgfpathlineto{\pgfqpoint{0.423041in}{0.635916in}}%
\pgfpathlineto{\pgfqpoint{0.426340in}{0.631203in}}%
\pgfpathlineto{\pgfqpoint{0.430569in}{0.617592in}}%
\pgfpathlineto{\pgfqpoint{0.428724in}{0.603981in}}%
\pgfpathlineto{\pgfqpoint{0.423041in}{0.593689in}}%
\pgfpathlineto{\pgfqpoint{0.421083in}{0.590370in}}%
\pgfpathlineto{\pgfqpoint{0.408553in}{0.576759in}}%
\pgfpathlineto{\pgfqpoint{0.407385in}{0.575721in}}%
\pgfpathlineto{\pgfqpoint{0.391728in}{0.563940in}}%
\pgfpathlineto{\pgfqpoint{0.390190in}{0.563148in}}%
\pgfpathlineto{\pgfqpoint{0.376072in}{0.556466in}}%
\pgfpathlineto{\pgfqpoint{0.360415in}{0.553780in}}%
\pgfpathlineto{\pgfqpoint{0.360415in}{0.549536in}}%
\pgfpathlineto{\pgfqpoint{0.360415in}{0.535925in}}%
\pgfpathlineto{\pgfqpoint{0.360415in}{0.526053in}}%
\pgfpathlineto{\pgfqpoint{0.376072in}{0.529344in}}%
\pgfpathclose%
\pgfpathmoveto{\pgfqpoint{0.657890in}{0.528176in}}%
\pgfpathlineto{\pgfqpoint{0.673546in}{0.526187in}}%
\pgfpathlineto{\pgfqpoint{0.689203in}{0.530746in}}%
\pgfpathlineto{\pgfqpoint{0.697158in}{0.535925in}}%
\pgfpathlineto{\pgfqpoint{0.704859in}{0.540323in}}%
\pgfpathlineto{\pgfqpoint{0.716592in}{0.549536in}}%
\pgfpathlineto{\pgfqpoint{0.720516in}{0.552482in}}%
\pgfpathlineto{\pgfqpoint{0.733011in}{0.563148in}}%
\pgfpathlineto{\pgfqpoint{0.736173in}{0.565993in}}%
\pgfpathlineto{\pgfqpoint{0.748004in}{0.576759in}}%
\pgfpathlineto{\pgfqpoint{0.751829in}{0.580973in}}%
\pgfpathlineto{\pgfqpoint{0.761066in}{0.590370in}}%
\pgfpathlineto{\pgfqpoint{0.767486in}{0.600555in}}%
\pgfpathlineto{\pgfqpoint{0.770040in}{0.603981in}}%
\pgfpathlineto{\pgfqpoint{0.772488in}{0.617592in}}%
\pgfpathlineto{\pgfqpoint{0.767486in}{0.629701in}}%
\pgfpathlineto{\pgfqpoint{0.766953in}{0.631203in}}%
\pgfpathlineto{\pgfqpoint{0.756202in}{0.644814in}}%
\pgfpathlineto{\pgfqpoint{0.751829in}{0.648818in}}%
\pgfpathlineto{\pgfqpoint{0.742198in}{0.658425in}}%
\pgfpathlineto{\pgfqpoint{0.736173in}{0.663707in}}%
\pgfpathlineto{\pgfqpoint{0.726592in}{0.672036in}}%
\pgfpathlineto{\pgfqpoint{0.720516in}{0.677274in}}%
\pgfpathlineto{\pgfqpoint{0.709464in}{0.685648in}}%
\pgfpathlineto{\pgfqpoint{0.704859in}{0.689449in}}%
\pgfpathlineto{\pgfqpoint{0.689203in}{0.698796in}}%
\pgfpathlineto{\pgfqpoint{0.687476in}{0.699259in}}%
\pgfpathlineto{\pgfqpoint{0.673546in}{0.703607in}}%
\pgfpathlineto{\pgfqpoint{0.657890in}{0.701479in}}%
\pgfpathlineto{\pgfqpoint{0.653949in}{0.699259in}}%
\pgfpathlineto{\pgfqpoint{0.642233in}{0.693677in}}%
\pgfpathlineto{\pgfqpoint{0.631425in}{0.685648in}}%
\pgfpathlineto{\pgfqpoint{0.626577in}{0.682322in}}%
\pgfpathlineto{\pgfqpoint{0.614193in}{0.672036in}}%
\pgfpathlineto{\pgfqpoint{0.610920in}{0.669288in}}%
\pgfpathlineto{\pgfqpoint{0.598651in}{0.658425in}}%
\pgfpathlineto{\pgfqpoint{0.595263in}{0.655014in}}%
\pgfpathlineto{\pgfqpoint{0.584665in}{0.644814in}}%
\pgfpathlineto{\pgfqpoint{0.579607in}{0.638119in}}%
\pgfpathlineto{\pgfqpoint{0.573649in}{0.631203in}}%
\pgfpathlineto{\pgfqpoint{0.568405in}{0.617592in}}%
\pgfpathlineto{\pgfqpoint{0.570693in}{0.603981in}}%
\pgfpathlineto{\pgfqpoint{0.579607in}{0.590866in}}%
\pgfpathlineto{\pgfqpoint{0.579906in}{0.590370in}}%
\pgfpathlineto{\pgfqpoint{0.592777in}{0.576759in}}%
\pgfpathlineto{\pgfqpoint{0.595263in}{0.574596in}}%
\pgfpathlineto{\pgfqpoint{0.607768in}{0.563148in}}%
\pgfpathlineto{\pgfqpoint{0.610920in}{0.560407in}}%
\pgfpathlineto{\pgfqpoint{0.624089in}{0.549536in}}%
\pgfpathlineto{\pgfqpoint{0.626577in}{0.547375in}}%
\pgfpathlineto{\pgfqpoint{0.642233in}{0.536185in}}%
\pgfpathlineto{\pgfqpoint{0.642804in}{0.535925in}}%
\pgfpathlineto{\pgfqpoint{0.657890in}{0.528176in}}%
\pgfpathclose%
\pgfpathmoveto{\pgfqpoint{0.640617in}{0.563148in}}%
\pgfpathlineto{\pgfqpoint{0.626577in}{0.573198in}}%
\pgfpathlineto{\pgfqpoint{0.622481in}{0.576759in}}%
\pgfpathlineto{\pgfqpoint{0.610920in}{0.588965in}}%
\pgfpathlineto{\pgfqpoint{0.609677in}{0.590370in}}%
\pgfpathlineto{\pgfqpoint{0.602138in}{0.603981in}}%
\pgfpathlineto{\pgfqpoint{0.600271in}{0.617592in}}%
\pgfpathlineto{\pgfqpoint{0.604550in}{0.631203in}}%
\pgfpathlineto{\pgfqpoint{0.610920in}{0.640331in}}%
\pgfpathlineto{\pgfqpoint{0.614308in}{0.644814in}}%
\pgfpathlineto{\pgfqpoint{0.626577in}{0.656618in}}%
\pgfpathlineto{\pgfqpoint{0.628873in}{0.658425in}}%
\pgfpathlineto{\pgfqpoint{0.642233in}{0.667525in}}%
\pgfpathlineto{\pgfqpoint{0.652503in}{0.672036in}}%
\pgfpathlineto{\pgfqpoint{0.657890in}{0.674300in}}%
\pgfpathlineto{\pgfqpoint{0.673546in}{0.675893in}}%
\pgfpathlineto{\pgfqpoint{0.689203in}{0.672243in}}%
\pgfpathlineto{\pgfqpoint{0.689590in}{0.672036in}}%
\pgfpathlineto{\pgfqpoint{0.704859in}{0.663614in}}%
\pgfpathlineto{\pgfqpoint{0.711715in}{0.658425in}}%
\pgfpathlineto{\pgfqpoint{0.720516in}{0.650774in}}%
\pgfpathlineto{\pgfqpoint{0.726484in}{0.644814in}}%
\pgfpathlineto{\pgfqpoint{0.736173in}{0.631540in}}%
\pgfpathlineto{\pgfqpoint{0.736410in}{0.631203in}}%
\pgfpathlineto{\pgfqpoint{0.740608in}{0.617592in}}%
\pgfpathlineto{\pgfqpoint{0.738777in}{0.603981in}}%
\pgfpathlineto{\pgfqpoint{0.736173in}{0.599298in}}%
\pgfpathlineto{\pgfqpoint{0.730983in}{0.590370in}}%
\pgfpathlineto{\pgfqpoint{0.720516in}{0.578755in}}%
\pgfpathlineto{\pgfqpoint{0.718437in}{0.576759in}}%
\pgfpathlineto{\pgfqpoint{0.704859in}{0.566093in}}%
\pgfpathlineto{\pgfqpoint{0.699702in}{0.563148in}}%
\pgfpathlineto{\pgfqpoint{0.689203in}{0.557610in}}%
\pgfpathlineto{\pgfqpoint{0.673546in}{0.553890in}}%
\pgfpathlineto{\pgfqpoint{0.657890in}{0.555513in}}%
\pgfpathlineto{\pgfqpoint{0.642233in}{0.562067in}}%
\pgfpathlineto{\pgfqpoint{0.640617in}{0.563148in}}%
\pgfpathclose%
\pgfpathmoveto{\pgfqpoint{0.955364in}{0.534195in}}%
\pgfpathlineto{\pgfqpoint{0.971021in}{0.527255in}}%
\pgfpathlineto{\pgfqpoint{0.986678in}{0.526589in}}%
\pgfpathlineto{\pgfqpoint{1.002334in}{0.532368in}}%
\pgfpathlineto{\pgfqpoint{1.007317in}{0.535925in}}%
\pgfpathlineto{\pgfqpoint{1.017991in}{0.542579in}}%
\pgfpathlineto{\pgfqpoint{1.026499in}{0.549536in}}%
\pgfpathlineto{\pgfqpoint{1.033647in}{0.555111in}}%
\pgfpathlineto{\pgfqpoint{1.042961in}{0.563148in}}%
\pgfpathlineto{\pgfqpoint{1.049304in}{0.568914in}}%
\pgfpathlineto{\pgfqpoint{1.058021in}{0.576759in}}%
\pgfpathlineto{\pgfqpoint{1.064960in}{0.584325in}}%
\pgfpathlineto{\pgfqpoint{1.071089in}{0.590370in}}%
\pgfpathlineto{\pgfqpoint{1.079896in}{0.603981in}}%
\pgfpathlineto{\pgfqpoint{1.080617in}{0.608506in}}%
\pgfpathlineto{\pgfqpoint{1.082326in}{0.617592in}}%
\pgfpathlineto{\pgfqpoint{1.080617in}{0.621512in}}%
\pgfpathlineto{\pgfqpoint{1.077078in}{0.631203in}}%
\pgfpathlineto{\pgfqpoint{1.066140in}{0.644814in}}%
\pgfpathlineto{\pgfqpoint{1.064960in}{0.645861in}}%
\pgfpathlineto{\pgfqpoint{1.052234in}{0.658425in}}%
\pgfpathlineto{\pgfqpoint{1.049304in}{0.660966in}}%
\pgfpathlineto{\pgfqpoint{1.036697in}{0.672036in}}%
\pgfpathlineto{\pgfqpoint{1.033647in}{0.674694in}}%
\pgfpathlineto{\pgfqpoint{1.019731in}{0.685648in}}%
\pgfpathlineto{\pgfqpoint{1.017991in}{0.687144in}}%
\pgfpathlineto{\pgfqpoint{1.002334in}{0.697275in}}%
\pgfpathlineto{\pgfqpoint{0.996509in}{0.699259in}}%
\pgfpathlineto{\pgfqpoint{0.986678in}{0.703177in}}%
\pgfpathlineto{\pgfqpoint{0.971021in}{0.702465in}}%
\pgfpathlineto{\pgfqpoint{0.964376in}{0.699259in}}%
\pgfpathlineto{\pgfqpoint{0.955364in}{0.695564in}}%
\pgfpathlineto{\pgfqpoint{0.941092in}{0.685648in}}%
\pgfpathlineto{\pgfqpoint{0.939708in}{0.684756in}}%
\pgfpathlineto{\pgfqpoint{0.924051in}{0.672097in}}%
\pgfpathlineto{\pgfqpoint{0.923982in}{0.672036in}}%
\pgfpathlineto{\pgfqpoint{0.908582in}{0.658425in}}%
\pgfpathlineto{\pgfqpoint{0.908395in}{0.658237in}}%
\pgfpathlineto{\pgfqpoint{0.894711in}{0.644814in}}%
\pgfpathlineto{\pgfqpoint{0.892738in}{0.642156in}}%
\pgfpathlineto{\pgfqpoint{0.883667in}{0.631203in}}%
\pgfpathlineto{\pgfqpoint{0.878559in}{0.617592in}}%
\pgfpathlineto{\pgfqpoint{0.880787in}{0.603981in}}%
\pgfpathlineto{\pgfqpoint{0.889787in}{0.590370in}}%
\pgfpathlineto{\pgfqpoint{0.892738in}{0.587559in}}%
\pgfpathlineto{\pgfqpoint{0.902784in}{0.576759in}}%
\pgfpathlineto{\pgfqpoint{0.908395in}{0.571786in}}%
\pgfpathlineto{\pgfqpoint{0.917843in}{0.563148in}}%
\pgfpathlineto{\pgfqpoint{0.924051in}{0.557758in}}%
\pgfpathlineto{\pgfqpoint{0.934268in}{0.549536in}}%
\pgfpathlineto{\pgfqpoint{0.939708in}{0.544935in}}%
\pgfpathlineto{\pgfqpoint{0.953123in}{0.535925in}}%
\pgfpathlineto{\pgfqpoint{0.955364in}{0.534195in}}%
\pgfpathclose%
\pgfpathmoveto{\pgfqpoint{0.951010in}{0.563148in}}%
\pgfpathlineto{\pgfqpoint{0.939708in}{0.570740in}}%
\pgfpathlineto{\pgfqpoint{0.932594in}{0.576759in}}%
\pgfpathlineto{\pgfqpoint{0.924051in}{0.585552in}}%
\pgfpathlineto{\pgfqpoint{0.919782in}{0.590370in}}%
\pgfpathlineto{\pgfqpoint{0.912123in}{0.603981in}}%
\pgfpathlineto{\pgfqpoint{0.910227in}{0.617592in}}%
\pgfpathlineto{\pgfqpoint{0.914574in}{0.631203in}}%
\pgfpathlineto{\pgfqpoint{0.924051in}{0.644764in}}%
\pgfpathlineto{\pgfqpoint{0.924090in}{0.644814in}}%
\pgfpathlineto{\pgfqpoint{0.938693in}{0.658425in}}%
\pgfpathlineto{\pgfqpoint{0.939708in}{0.659253in}}%
\pgfpathlineto{\pgfqpoint{0.955364in}{0.669270in}}%
\pgfpathlineto{\pgfqpoint{0.962681in}{0.672036in}}%
\pgfpathlineto{\pgfqpoint{0.971021in}{0.675038in}}%
\pgfpathlineto{\pgfqpoint{0.986678in}{0.675571in}}%
\pgfpathlineto{\pgfqpoint{0.998566in}{0.672036in}}%
\pgfpathlineto{\pgfqpoint{1.002334in}{0.670854in}}%
\pgfpathlineto{\pgfqpoint{1.017991in}{0.661481in}}%
\pgfpathlineto{\pgfqpoint{1.021868in}{0.658425in}}%
\pgfpathlineto{\pgfqpoint{1.033647in}{0.647787in}}%
\pgfpathlineto{\pgfqpoint{1.036592in}{0.644814in}}%
\pgfpathlineto{\pgfqpoint{1.046296in}{0.631203in}}%
\pgfpathlineto{\pgfqpoint{1.049304in}{0.621913in}}%
\pgfpathlineto{\pgfqpoint{1.050650in}{0.617592in}}%
\pgfpathlineto{\pgfqpoint{1.049304in}{0.607576in}}%
\pgfpathlineto{\pgfqpoint{1.048796in}{0.603981in}}%
\pgfpathlineto{\pgfqpoint{1.040982in}{0.590370in}}%
\pgfpathlineto{\pgfqpoint{1.033647in}{0.582142in}}%
\pgfpathlineto{\pgfqpoint{1.028250in}{0.576759in}}%
\pgfpathlineto{\pgfqpoint{1.017991in}{0.568366in}}%
\pgfpathlineto{\pgfqpoint{1.009622in}{0.563148in}}%
\pgfpathlineto{\pgfqpoint{1.002334in}{0.558934in}}%
\pgfpathlineto{\pgfqpoint{0.986678in}{0.554218in}}%
\pgfpathlineto{\pgfqpoint{0.971021in}{0.554761in}}%
\pgfpathlineto{\pgfqpoint{0.955364in}{0.560424in}}%
\pgfpathlineto{\pgfqpoint{0.951010in}{0.563148in}}%
\pgfpathclose%
\pgfpathmoveto{\pgfqpoint{1.268496in}{0.532368in}}%
\pgfpathlineto{\pgfqpoint{1.284152in}{0.526589in}}%
\pgfpathlineto{\pgfqpoint{1.299809in}{0.527255in}}%
\pgfpathlineto{\pgfqpoint{1.315466in}{0.534195in}}%
\pgfpathlineto{\pgfqpoint{1.317707in}{0.535925in}}%
\pgfpathlineto{\pgfqpoint{1.331122in}{0.544935in}}%
\pgfpathlineto{\pgfqpoint{1.336562in}{0.549536in}}%
\pgfpathlineto{\pgfqpoint{1.346779in}{0.557758in}}%
\pgfpathlineto{\pgfqpoint{1.352987in}{0.563148in}}%
\pgfpathlineto{\pgfqpoint{1.362435in}{0.571786in}}%
\pgfpathlineto{\pgfqpoint{1.368046in}{0.576759in}}%
\pgfpathlineto{\pgfqpoint{1.378092in}{0.587559in}}%
\pgfpathlineto{\pgfqpoint{1.381043in}{0.590370in}}%
\pgfpathlineto{\pgfqpoint{1.390043in}{0.603981in}}%
\pgfpathlineto{\pgfqpoint{1.392271in}{0.617592in}}%
\pgfpathlineto{\pgfqpoint{1.387163in}{0.631203in}}%
\pgfpathlineto{\pgfqpoint{1.378092in}{0.642156in}}%
\pgfpathlineto{\pgfqpoint{1.376119in}{0.644814in}}%
\pgfpathlineto{\pgfqpoint{1.362435in}{0.658237in}}%
\pgfpathlineto{\pgfqpoint{1.362248in}{0.658425in}}%
\pgfpathlineto{\pgfqpoint{1.346848in}{0.672036in}}%
\pgfpathlineto{\pgfqpoint{1.346779in}{0.672097in}}%
\pgfpathlineto{\pgfqpoint{1.331122in}{0.684756in}}%
\pgfpathlineto{\pgfqpoint{1.329738in}{0.685648in}}%
\pgfpathlineto{\pgfqpoint{1.315466in}{0.695564in}}%
\pgfpathlineto{\pgfqpoint{1.306454in}{0.699259in}}%
\pgfpathlineto{\pgfqpoint{1.299809in}{0.702465in}}%
\pgfpathlineto{\pgfqpoint{1.284152in}{0.703177in}}%
\pgfpathlineto{\pgfqpoint{1.274321in}{0.699259in}}%
\pgfpathlineto{\pgfqpoint{1.268496in}{0.697275in}}%
\pgfpathlineto{\pgfqpoint{1.252839in}{0.687144in}}%
\pgfpathlineto{\pgfqpoint{1.251099in}{0.685648in}}%
\pgfpathlineto{\pgfqpoint{1.237183in}{0.674694in}}%
\pgfpathlineto{\pgfqpoint{1.234133in}{0.672036in}}%
\pgfpathlineto{\pgfqpoint{1.221526in}{0.660966in}}%
\pgfpathlineto{\pgfqpoint{1.218596in}{0.658425in}}%
\pgfpathlineto{\pgfqpoint{1.205870in}{0.645861in}}%
\pgfpathlineto{\pgfqpoint{1.204690in}{0.644814in}}%
\pgfpathlineto{\pgfqpoint{1.193752in}{0.631203in}}%
\pgfpathlineto{\pgfqpoint{1.190213in}{0.621512in}}%
\pgfpathlineto{\pgfqpoint{1.188504in}{0.617592in}}%
\pgfpathlineto{\pgfqpoint{1.190213in}{0.608506in}}%
\pgfpathlineto{\pgfqpoint{1.190934in}{0.603981in}}%
\pgfpathlineto{\pgfqpoint{1.199741in}{0.590370in}}%
\pgfpathlineto{\pgfqpoint{1.205870in}{0.584325in}}%
\pgfpathlineto{\pgfqpoint{1.212809in}{0.576759in}}%
\pgfpathlineto{\pgfqpoint{1.221526in}{0.568914in}}%
\pgfpathlineto{\pgfqpoint{1.227869in}{0.563148in}}%
\pgfpathlineto{\pgfqpoint{1.237183in}{0.555111in}}%
\pgfpathlineto{\pgfqpoint{1.244331in}{0.549536in}}%
\pgfpathlineto{\pgfqpoint{1.252839in}{0.542579in}}%
\pgfpathlineto{\pgfqpoint{1.263513in}{0.535925in}}%
\pgfpathlineto{\pgfqpoint{1.268496in}{0.532368in}}%
\pgfpathclose%
\pgfpathmoveto{\pgfqpoint{1.261208in}{0.563148in}}%
\pgfpathlineto{\pgfqpoint{1.252839in}{0.568366in}}%
\pgfpathlineto{\pgfqpoint{1.242580in}{0.576759in}}%
\pgfpathlineto{\pgfqpoint{1.237183in}{0.582142in}}%
\pgfpathlineto{\pgfqpoint{1.229848in}{0.590370in}}%
\pgfpathlineto{\pgfqpoint{1.222034in}{0.603981in}}%
\pgfpathlineto{\pgfqpoint{1.221526in}{0.607576in}}%
\pgfpathlineto{\pgfqpoint{1.220180in}{0.617592in}}%
\pgfpathlineto{\pgfqpoint{1.221526in}{0.621913in}}%
\pgfpathlineto{\pgfqpoint{1.224534in}{0.631203in}}%
\pgfpathlineto{\pgfqpoint{1.234238in}{0.644814in}}%
\pgfpathlineto{\pgfqpoint{1.237183in}{0.647787in}}%
\pgfpathlineto{\pgfqpoint{1.248962in}{0.658425in}}%
\pgfpathlineto{\pgfqpoint{1.252839in}{0.661481in}}%
\pgfpathlineto{\pgfqpoint{1.268496in}{0.670854in}}%
\pgfpathlineto{\pgfqpoint{1.272264in}{0.672036in}}%
\pgfpathlineto{\pgfqpoint{1.284152in}{0.675571in}}%
\pgfpathlineto{\pgfqpoint{1.299809in}{0.675038in}}%
\pgfpathlineto{\pgfqpoint{1.308149in}{0.672036in}}%
\pgfpathlineto{\pgfqpoint{1.315466in}{0.669270in}}%
\pgfpathlineto{\pgfqpoint{1.331122in}{0.659253in}}%
\pgfpathlineto{\pgfqpoint{1.332137in}{0.658425in}}%
\pgfpathlineto{\pgfqpoint{1.346740in}{0.644814in}}%
\pgfpathlineto{\pgfqpoint{1.346779in}{0.644764in}}%
\pgfpathlineto{\pgfqpoint{1.356256in}{0.631203in}}%
\pgfpathlineto{\pgfqpoint{1.360603in}{0.617592in}}%
\pgfpathlineto{\pgfqpoint{1.358707in}{0.603981in}}%
\pgfpathlineto{\pgfqpoint{1.351048in}{0.590370in}}%
\pgfpathlineto{\pgfqpoint{1.346779in}{0.585552in}}%
\pgfpathlineto{\pgfqpoint{1.338236in}{0.576759in}}%
\pgfpathlineto{\pgfqpoint{1.331122in}{0.570740in}}%
\pgfpathlineto{\pgfqpoint{1.319820in}{0.563148in}}%
\pgfpathlineto{\pgfqpoint{1.315466in}{0.560424in}}%
\pgfpathlineto{\pgfqpoint{1.299809in}{0.554761in}}%
\pgfpathlineto{\pgfqpoint{1.284152in}{0.554218in}}%
\pgfpathlineto{\pgfqpoint{1.268496in}{0.558934in}}%
\pgfpathlineto{\pgfqpoint{1.261208in}{0.563148in}}%
\pgfpathclose%
\pgfpathmoveto{\pgfqpoint{1.581627in}{0.530746in}}%
\pgfpathlineto{\pgfqpoint{1.597284in}{0.526187in}}%
\pgfpathlineto{\pgfqpoint{1.612940in}{0.528176in}}%
\pgfpathlineto{\pgfqpoint{1.628026in}{0.535925in}}%
\pgfpathlineto{\pgfqpoint{1.628597in}{0.536185in}}%
\pgfpathlineto{\pgfqpoint{1.644253in}{0.547375in}}%
\pgfpathlineto{\pgfqpoint{1.646741in}{0.549536in}}%
\pgfpathlineto{\pgfqpoint{1.659910in}{0.560407in}}%
\pgfpathlineto{\pgfqpoint{1.663062in}{0.563148in}}%
\pgfpathlineto{\pgfqpoint{1.675567in}{0.574596in}}%
\pgfpathlineto{\pgfqpoint{1.678053in}{0.576759in}}%
\pgfpathlineto{\pgfqpoint{1.690924in}{0.590370in}}%
\pgfpathlineto{\pgfqpoint{1.691223in}{0.590866in}}%
\pgfpathlineto{\pgfqpoint{1.700137in}{0.603981in}}%
\pgfpathlineto{\pgfqpoint{1.702425in}{0.617592in}}%
\pgfpathlineto{\pgfqpoint{1.697181in}{0.631203in}}%
\pgfpathlineto{\pgfqpoint{1.691223in}{0.638119in}}%
\pgfpathlineto{\pgfqpoint{1.686165in}{0.644814in}}%
\pgfpathlineto{\pgfqpoint{1.675567in}{0.655014in}}%
\pgfpathlineto{\pgfqpoint{1.672179in}{0.658425in}}%
\pgfpathlineto{\pgfqpoint{1.659910in}{0.669288in}}%
\pgfpathlineto{\pgfqpoint{1.656637in}{0.672036in}}%
\pgfpathlineto{\pgfqpoint{1.644253in}{0.682322in}}%
\pgfpathlineto{\pgfqpoint{1.639405in}{0.685648in}}%
\pgfpathlineto{\pgfqpoint{1.628597in}{0.693677in}}%
\pgfpathlineto{\pgfqpoint{1.616881in}{0.699259in}}%
\pgfpathlineto{\pgfqpoint{1.612940in}{0.701479in}}%
\pgfpathlineto{\pgfqpoint{1.597284in}{0.703607in}}%
\pgfpathlineto{\pgfqpoint{1.583354in}{0.699259in}}%
\pgfpathlineto{\pgfqpoint{1.581627in}{0.698796in}}%
\pgfpathlineto{\pgfqpoint{1.565971in}{0.689449in}}%
\pgfpathlineto{\pgfqpoint{1.561366in}{0.685648in}}%
\pgfpathlineto{\pgfqpoint{1.550314in}{0.677274in}}%
\pgfpathlineto{\pgfqpoint{1.544238in}{0.672036in}}%
\pgfpathlineto{\pgfqpoint{1.534657in}{0.663707in}}%
\pgfpathlineto{\pgfqpoint{1.528632in}{0.658425in}}%
\pgfpathlineto{\pgfqpoint{1.519001in}{0.648818in}}%
\pgfpathlineto{\pgfqpoint{1.514628in}{0.644814in}}%
\pgfpathlineto{\pgfqpoint{1.503877in}{0.631203in}}%
\pgfpathlineto{\pgfqpoint{1.503344in}{0.629701in}}%
\pgfpathlineto{\pgfqpoint{1.498342in}{0.617592in}}%
\pgfpathlineto{\pgfqpoint{1.500790in}{0.603981in}}%
\pgfpathlineto{\pgfqpoint{1.503344in}{0.600555in}}%
\pgfpathlineto{\pgfqpoint{1.509764in}{0.590370in}}%
\pgfpathlineto{\pgfqpoint{1.519001in}{0.580973in}}%
\pgfpathlineto{\pgfqpoint{1.522826in}{0.576759in}}%
\pgfpathlineto{\pgfqpoint{1.534657in}{0.565993in}}%
\pgfpathlineto{\pgfqpoint{1.537819in}{0.563148in}}%
\pgfpathlineto{\pgfqpoint{1.550314in}{0.552482in}}%
\pgfpathlineto{\pgfqpoint{1.554238in}{0.549536in}}%
\pgfpathlineto{\pgfqpoint{1.565971in}{0.540323in}}%
\pgfpathlineto{\pgfqpoint{1.573672in}{0.535925in}}%
\pgfpathlineto{\pgfqpoint{1.581627in}{0.530746in}}%
\pgfpathclose%
\pgfpathmoveto{\pgfqpoint{1.571128in}{0.563148in}}%
\pgfpathlineto{\pgfqpoint{1.565971in}{0.566093in}}%
\pgfpathlineto{\pgfqpoint{1.552393in}{0.576759in}}%
\pgfpathlineto{\pgfqpoint{1.550314in}{0.578755in}}%
\pgfpathlineto{\pgfqpoint{1.539847in}{0.590370in}}%
\pgfpathlineto{\pgfqpoint{1.534657in}{0.599298in}}%
\pgfpathlineto{\pgfqpoint{1.532053in}{0.603981in}}%
\pgfpathlineto{\pgfqpoint{1.530222in}{0.617592in}}%
\pgfpathlineto{\pgfqpoint{1.534420in}{0.631203in}}%
\pgfpathlineto{\pgfqpoint{1.534657in}{0.631540in}}%
\pgfpathlineto{\pgfqpoint{1.544346in}{0.644814in}}%
\pgfpathlineto{\pgfqpoint{1.550314in}{0.650774in}}%
\pgfpathlineto{\pgfqpoint{1.559115in}{0.658425in}}%
\pgfpathlineto{\pgfqpoint{1.565971in}{0.663614in}}%
\pgfpathlineto{\pgfqpoint{1.581240in}{0.672036in}}%
\pgfpathlineto{\pgfqpoint{1.581627in}{0.672243in}}%
\pgfpathlineto{\pgfqpoint{1.597284in}{0.675893in}}%
\pgfpathlineto{\pgfqpoint{1.612940in}{0.674300in}}%
\pgfpathlineto{\pgfqpoint{1.618327in}{0.672036in}}%
\pgfpathlineto{\pgfqpoint{1.628597in}{0.667525in}}%
\pgfpathlineto{\pgfqpoint{1.641957in}{0.658425in}}%
\pgfpathlineto{\pgfqpoint{1.644253in}{0.656618in}}%
\pgfpathlineto{\pgfqpoint{1.656522in}{0.644814in}}%
\pgfpathlineto{\pgfqpoint{1.659910in}{0.640331in}}%
\pgfpathlineto{\pgfqpoint{1.666280in}{0.631203in}}%
\pgfpathlineto{\pgfqpoint{1.670559in}{0.617592in}}%
\pgfpathlineto{\pgfqpoint{1.668692in}{0.603981in}}%
\pgfpathlineto{\pgfqpoint{1.661153in}{0.590370in}}%
\pgfpathlineto{\pgfqpoint{1.659910in}{0.588965in}}%
\pgfpathlineto{\pgfqpoint{1.648349in}{0.576759in}}%
\pgfpathlineto{\pgfqpoint{1.644253in}{0.573198in}}%
\pgfpathlineto{\pgfqpoint{1.630213in}{0.563148in}}%
\pgfpathlineto{\pgfqpoint{1.628597in}{0.562067in}}%
\pgfpathlineto{\pgfqpoint{1.612940in}{0.555513in}}%
\pgfpathlineto{\pgfqpoint{1.597284in}{0.553890in}}%
\pgfpathlineto{\pgfqpoint{1.581627in}{0.557610in}}%
\pgfpathlineto{\pgfqpoint{1.571128in}{0.563148in}}%
\pgfpathclose%
\pgfpathmoveto{\pgfqpoint{1.894758in}{0.529344in}}%
\pgfpathlineto{\pgfqpoint{1.910415in}{0.526053in}}%
\pgfpathlineto{\pgfqpoint{1.910415in}{0.535925in}}%
\pgfpathlineto{\pgfqpoint{1.910415in}{0.549536in}}%
\pgfpathlineto{\pgfqpoint{1.910415in}{0.553780in}}%
\pgfpathlineto{\pgfqpoint{1.894758in}{0.556466in}}%
\pgfpathlineto{\pgfqpoint{1.880640in}{0.563148in}}%
\pgfpathlineto{\pgfqpoint{1.879102in}{0.563940in}}%
\pgfpathlineto{\pgfqpoint{1.863445in}{0.575721in}}%
\pgfpathlineto{\pgfqpoint{1.862277in}{0.576759in}}%
\pgfpathlineto{\pgfqpoint{1.849747in}{0.590370in}}%
\pgfpathlineto{\pgfqpoint{1.847789in}{0.593689in}}%
\pgfpathlineto{\pgfqpoint{1.842106in}{0.603981in}}%
\pgfpathlineto{\pgfqpoint{1.840261in}{0.617592in}}%
\pgfpathlineto{\pgfqpoint{1.844490in}{0.631203in}}%
\pgfpathlineto{\pgfqpoint{1.847789in}{0.635916in}}%
\pgfpathlineto{\pgfqpoint{1.854380in}{0.644814in}}%
\pgfpathlineto{\pgfqpoint{1.863445in}{0.653724in}}%
\pgfpathlineto{\pgfqpoint{1.869105in}{0.658425in}}%
\pgfpathlineto{\pgfqpoint{1.879102in}{0.665634in}}%
\pgfpathlineto{\pgfqpoint{1.891978in}{0.672036in}}%
\pgfpathlineto{\pgfqpoint{1.894758in}{0.673365in}}%
\pgfpathlineto{\pgfqpoint{1.910415in}{0.676000in}}%
\pgfpathlineto{\pgfqpoint{1.910415in}{0.685648in}}%
\pgfpathlineto{\pgfqpoint{1.910415in}{0.699259in}}%
\pgfpathlineto{\pgfqpoint{1.910415in}{0.703751in}}%
\pgfpathlineto{\pgfqpoint{1.894758in}{0.700230in}}%
\pgfpathlineto{\pgfqpoint{1.893243in}{0.699259in}}%
\pgfpathlineto{\pgfqpoint{1.879102in}{0.691633in}}%
\pgfpathlineto{\pgfqpoint{1.871492in}{0.685648in}}%
\pgfpathlineto{\pgfqpoint{1.863445in}{0.679822in}}%
\pgfpathlineto{\pgfqpoint{1.854269in}{0.672036in}}%
\pgfpathlineto{\pgfqpoint{1.847789in}{0.666485in}}%
\pgfpathlineto{\pgfqpoint{1.838659in}{0.658425in}}%
\pgfpathlineto{\pgfqpoint{1.832132in}{0.651871in}}%
\pgfpathlineto{\pgfqpoint{1.824629in}{0.644814in}}%
\pgfpathlineto{\pgfqpoint{1.816476in}{0.634251in}}%
\pgfpathlineto{\pgfqpoint{1.813732in}{0.631203in}}%
\pgfpathlineto{\pgfqpoint{1.808322in}{0.617592in}}%
\pgfpathlineto{\pgfqpoint{1.810682in}{0.603981in}}%
\pgfpathlineto{\pgfqpoint{1.816476in}{0.595823in}}%
\pgfpathlineto{\pgfqpoint{1.819828in}{0.590370in}}%
\pgfpathlineto{\pgfqpoint{1.832132in}{0.577510in}}%
\pgfpathlineto{\pgfqpoint{1.832810in}{0.576759in}}%
\pgfpathlineto{\pgfqpoint{1.847671in}{0.563148in}}%
\pgfpathlineto{\pgfqpoint{1.847789in}{0.563046in}}%
\pgfpathlineto{\pgfqpoint{1.863445in}{0.549885in}}%
\pgfpathlineto{\pgfqpoint{1.863931in}{0.549536in}}%
\pgfpathlineto{\pgfqpoint{1.879102in}{0.538185in}}%
\pgfpathlineto{\pgfqpoint{1.883494in}{0.535925in}}%
\pgfpathlineto{\pgfqpoint{1.894758in}{0.529344in}}%
\pgfpathclose%
\pgfpathmoveto{\pgfqpoint{0.376072in}{0.798895in}}%
\pgfpathlineto{\pgfqpoint{0.391728in}{0.807707in}}%
\pgfpathlineto{\pgfqpoint{0.392252in}{0.808148in}}%
\pgfpathlineto{\pgfqpoint{0.407385in}{0.819374in}}%
\pgfpathlineto{\pgfqpoint{0.410122in}{0.821759in}}%
\pgfpathlineto{\pgfqpoint{0.423041in}{0.832652in}}%
\pgfpathlineto{\pgfqpoint{0.426171in}{0.835370in}}%
\pgfpathlineto{\pgfqpoint{0.438698in}{0.847103in}}%
\pgfpathlineto{\pgfqpoint{0.440848in}{0.848981in}}%
\pgfpathlineto{\pgfqpoint{0.453145in}{0.862592in}}%
\pgfpathlineto{\pgfqpoint{0.454354in}{0.864879in}}%
\pgfpathlineto{\pgfqpoint{0.461241in}{0.876203in}}%
\pgfpathlineto{\pgfqpoint{0.462031in}{0.889814in}}%
\pgfpathlineto{\pgfqpoint{0.455173in}{0.903425in}}%
\pgfpathlineto{\pgfqpoint{0.454354in}{0.904254in}}%
\pgfpathlineto{\pgfqpoint{0.443583in}{0.917036in}}%
\pgfpathlineto{\pgfqpoint{0.438698in}{0.921448in}}%
\pgfpathlineto{\pgfqpoint{0.429181in}{0.930648in}}%
\pgfpathlineto{\pgfqpoint{0.423041in}{0.936010in}}%
\pgfpathlineto{\pgfqpoint{0.413313in}{0.944259in}}%
\pgfpathlineto{\pgfqpoint{0.407385in}{0.949345in}}%
\pgfpathlineto{\pgfqpoint{0.395731in}{0.957870in}}%
\pgfpathlineto{\pgfqpoint{0.391728in}{0.961118in}}%
\pgfpathlineto{\pgfqpoint{0.376072in}{0.969741in}}%
\pgfpathlineto{\pgfqpoint{0.367333in}{0.971481in}}%
\pgfpathlineto{\pgfqpoint{0.360415in}{0.973116in}}%
\pgfpathlineto{\pgfqpoint{0.360415in}{0.971481in}}%
\pgfpathlineto{\pgfqpoint{0.360415in}{0.957870in}}%
\pgfpathlineto{\pgfqpoint{0.360415in}{0.945536in}}%
\pgfpathlineto{\pgfqpoint{0.367977in}{0.944259in}}%
\pgfpathlineto{\pgfqpoint{0.376072in}{0.942830in}}%
\pgfpathlineto{\pgfqpoint{0.391728in}{0.935179in}}%
\pgfpathlineto{\pgfqpoint{0.398080in}{0.930648in}}%
\pgfpathlineto{\pgfqpoint{0.407385in}{0.923208in}}%
\pgfpathlineto{\pgfqpoint{0.413923in}{0.917036in}}%
\pgfpathlineto{\pgfqpoint{0.423041in}{0.905762in}}%
\pgfpathlineto{\pgfqpoint{0.424834in}{0.903425in}}%
\pgfpathlineto{\pgfqpoint{0.430196in}{0.889814in}}%
\pgfpathlineto{\pgfqpoint{0.429579in}{0.876203in}}%
\pgfpathlineto{\pgfqpoint{0.423140in}{0.862592in}}%
\pgfpathlineto{\pgfqpoint{0.423041in}{0.862473in}}%
\pgfpathlineto{\pgfqpoint{0.411285in}{0.848981in}}%
\pgfpathlineto{\pgfqpoint{0.407385in}{0.845422in}}%
\pgfpathlineto{\pgfqpoint{0.394412in}{0.835370in}}%
\pgfpathlineto{\pgfqpoint{0.391728in}{0.833467in}}%
\pgfpathlineto{\pgfqpoint{0.376072in}{0.825968in}}%
\pgfpathlineto{\pgfqpoint{0.360415in}{0.823240in}}%
\pgfpathlineto{\pgfqpoint{0.360415in}{0.821759in}}%
\pgfpathlineto{\pgfqpoint{0.360415in}{0.808148in}}%
\pgfpathlineto{\pgfqpoint{0.360415in}{0.795689in}}%
\pgfpathlineto{\pgfqpoint{0.376072in}{0.798895in}}%
\pgfpathclose%
\pgfpathmoveto{\pgfqpoint{0.642233in}{0.805582in}}%
\pgfpathlineto{\pgfqpoint{0.657890in}{0.797758in}}%
\pgfpathlineto{\pgfqpoint{0.673546in}{0.795820in}}%
\pgfpathlineto{\pgfqpoint{0.689203in}{0.800261in}}%
\pgfpathlineto{\pgfqpoint{0.701802in}{0.808148in}}%
\pgfpathlineto{\pgfqpoint{0.704859in}{0.809862in}}%
\pgfpathlineto{\pgfqpoint{0.720300in}{0.821759in}}%
\pgfpathlineto{\pgfqpoint{0.720516in}{0.821921in}}%
\pgfpathlineto{\pgfqpoint{0.736173in}{0.835309in}}%
\pgfpathlineto{\pgfqpoint{0.736243in}{0.835370in}}%
\pgfpathlineto{\pgfqpoint{0.750804in}{0.848981in}}%
\pgfpathlineto{\pgfqpoint{0.751829in}{0.850184in}}%
\pgfpathlineto{\pgfqpoint{0.763236in}{0.862592in}}%
\pgfpathlineto{\pgfqpoint{0.767486in}{0.870426in}}%
\pgfpathlineto{\pgfqpoint{0.771174in}{0.876203in}}%
\pgfpathlineto{\pgfqpoint{0.771993in}{0.889814in}}%
\pgfpathlineto{\pgfqpoint{0.767486in}{0.898361in}}%
\pgfpathlineto{\pgfqpoint{0.765205in}{0.903425in}}%
\pgfpathlineto{\pgfqpoint{0.753550in}{0.917036in}}%
\pgfpathlineto{\pgfqpoint{0.751829in}{0.918549in}}%
\pgfpathlineto{\pgfqpoint{0.739230in}{0.930648in}}%
\pgfpathlineto{\pgfqpoint{0.736173in}{0.933299in}}%
\pgfpathlineto{\pgfqpoint{0.723438in}{0.944259in}}%
\pgfpathlineto{\pgfqpoint{0.720516in}{0.946806in}}%
\pgfpathlineto{\pgfqpoint{0.706064in}{0.957870in}}%
\pgfpathlineto{\pgfqpoint{0.704859in}{0.958895in}}%
\pgfpathlineto{\pgfqpoint{0.689203in}{0.968404in}}%
\pgfpathlineto{\pgfqpoint{0.678055in}{0.971481in}}%
\pgfpathlineto{\pgfqpoint{0.673546in}{0.972966in}}%
\pgfpathlineto{\pgfqpoint{0.663095in}{0.971481in}}%
\pgfpathlineto{\pgfqpoint{0.657890in}{0.970854in}}%
\pgfpathlineto{\pgfqpoint{0.642233in}{0.963197in}}%
\pgfpathlineto{\pgfqpoint{0.635281in}{0.957870in}}%
\pgfpathlineto{\pgfqpoint{0.626577in}{0.951837in}}%
\pgfpathlineto{\pgfqpoint{0.617554in}{0.944259in}}%
\pgfpathlineto{\pgfqpoint{0.610920in}{0.938744in}}%
\pgfpathlineto{\pgfqpoint{0.601676in}{0.930648in}}%
\pgfpathlineto{\pgfqpoint{0.595263in}{0.924433in}}%
\pgfpathlineto{\pgfqpoint{0.587260in}{0.917036in}}%
\pgfpathlineto{\pgfqpoint{0.579607in}{0.907757in}}%
\pgfpathlineto{\pgfqpoint{0.575516in}{0.903425in}}%
\pgfpathlineto{\pgfqpoint{0.568868in}{0.889814in}}%
\pgfpathlineto{\pgfqpoint{0.569633in}{0.876203in}}%
\pgfpathlineto{\pgfqpoint{0.577617in}{0.862592in}}%
\pgfpathlineto{\pgfqpoint{0.579607in}{0.860644in}}%
\pgfpathlineto{\pgfqpoint{0.589971in}{0.848981in}}%
\pgfpathlineto{\pgfqpoint{0.595263in}{0.844251in}}%
\pgfpathlineto{\pgfqpoint{0.604720in}{0.835370in}}%
\pgfpathlineto{\pgfqpoint{0.610920in}{0.829972in}}%
\pgfpathlineto{\pgfqpoint{0.620856in}{0.821759in}}%
\pgfpathlineto{\pgfqpoint{0.626577in}{0.816881in}}%
\pgfpathlineto{\pgfqpoint{0.639000in}{0.808148in}}%
\pgfpathlineto{\pgfqpoint{0.642233in}{0.805582in}}%
\pgfpathclose%
\pgfpathmoveto{\pgfqpoint{0.636691in}{0.835370in}}%
\pgfpathlineto{\pgfqpoint{0.626577in}{0.842796in}}%
\pgfpathlineto{\pgfqpoint{0.619654in}{0.848981in}}%
\pgfpathlineto{\pgfqpoint{0.610920in}{0.858806in}}%
\pgfpathlineto{\pgfqpoint{0.607787in}{0.862592in}}%
\pgfpathlineto{\pgfqpoint{0.601273in}{0.876203in}}%
\pgfpathlineto{\pgfqpoint{0.600649in}{0.889814in}}%
\pgfpathlineto{\pgfqpoint{0.606073in}{0.903425in}}%
\pgfpathlineto{\pgfqpoint{0.610920in}{0.909761in}}%
\pgfpathlineto{\pgfqpoint{0.616922in}{0.917036in}}%
\pgfpathlineto{\pgfqpoint{0.626577in}{0.925955in}}%
\pgfpathlineto{\pgfqpoint{0.632769in}{0.930648in}}%
\pgfpathlineto{\pgfqpoint{0.642233in}{0.937024in}}%
\pgfpathlineto{\pgfqpoint{0.657890in}{0.943817in}}%
\pgfpathlineto{\pgfqpoint{0.662025in}{0.944259in}}%
\pgfpathlineto{\pgfqpoint{0.673546in}{0.945429in}}%
\pgfpathlineto{\pgfqpoint{0.678517in}{0.944259in}}%
\pgfpathlineto{\pgfqpoint{0.689203in}{0.941644in}}%
\pgfpathlineto{\pgfqpoint{0.704859in}{0.933207in}}%
\pgfpathlineto{\pgfqpoint{0.708279in}{0.930648in}}%
\pgfpathlineto{\pgfqpoint{0.720516in}{0.920407in}}%
\pgfpathlineto{\pgfqpoint{0.724031in}{0.917036in}}%
\pgfpathlineto{\pgfqpoint{0.734812in}{0.903425in}}%
\pgfpathlineto{\pgfqpoint{0.736173in}{0.900149in}}%
\pgfpathlineto{\pgfqpoint{0.740238in}{0.889814in}}%
\pgfpathlineto{\pgfqpoint{0.739625in}{0.876203in}}%
\pgfpathlineto{\pgfqpoint{0.736173in}{0.868953in}}%
\pgfpathlineto{\pgfqpoint{0.732991in}{0.862592in}}%
\pgfpathlineto{\pgfqpoint{0.721468in}{0.848981in}}%
\pgfpathlineto{\pgfqpoint{0.720516in}{0.848098in}}%
\pgfpathlineto{\pgfqpoint{0.704859in}{0.835404in}}%
\pgfpathlineto{\pgfqpoint{0.704802in}{0.835370in}}%
\pgfpathlineto{\pgfqpoint{0.689203in}{0.827130in}}%
\pgfpathlineto{\pgfqpoint{0.673546in}{0.823352in}}%
\pgfpathlineto{\pgfqpoint{0.657890in}{0.825000in}}%
\pgfpathlineto{\pgfqpoint{0.642233in}{0.831658in}}%
\pgfpathlineto{\pgfqpoint{0.636691in}{0.835370in}}%
\pgfpathclose%
\pgfpathmoveto{\pgfqpoint{0.955364in}{0.803621in}}%
\pgfpathlineto{\pgfqpoint{0.971021in}{0.796860in}}%
\pgfpathlineto{\pgfqpoint{0.986678in}{0.796212in}}%
\pgfpathlineto{\pgfqpoint{1.002334in}{0.801841in}}%
\pgfpathlineto{\pgfqpoint{1.011524in}{0.808148in}}%
\pgfpathlineto{\pgfqpoint{1.017991in}{0.812108in}}%
\pgfpathlineto{\pgfqpoint{1.030019in}{0.821759in}}%
\pgfpathlineto{\pgfqpoint{1.033647in}{0.824592in}}%
\pgfpathlineto{\pgfqpoint{1.046119in}{0.835370in}}%
\pgfpathlineto{\pgfqpoint{1.049304in}{0.838340in}}%
\pgfpathlineto{\pgfqpoint{1.060812in}{0.848981in}}%
\pgfpathlineto{\pgfqpoint{1.064960in}{0.853801in}}%
\pgfpathlineto{\pgfqpoint{1.073296in}{0.862592in}}%
\pgfpathlineto{\pgfqpoint{1.080617in}{0.875700in}}%
\pgfpathlineto{\pgfqpoint{1.080956in}{0.876203in}}%
\pgfpathlineto{\pgfqpoint{1.081810in}{0.889814in}}%
\pgfpathlineto{\pgfqpoint{1.080617in}{0.891957in}}%
\pgfpathlineto{\pgfqpoint{1.075299in}{0.903425in}}%
\pgfpathlineto{\pgfqpoint{1.064960in}{0.915218in}}%
\pgfpathlineto{\pgfqpoint{1.063508in}{0.917036in}}%
\pgfpathlineto{\pgfqpoint{1.049304in}{0.930620in}}%
\pgfpathlineto{\pgfqpoint{1.049275in}{0.930648in}}%
\pgfpathlineto{\pgfqpoint{1.033647in}{0.944233in}}%
\pgfpathlineto{\pgfqpoint{1.033615in}{0.944259in}}%
\pgfpathlineto{\pgfqpoint{1.017991in}{0.956607in}}%
\pgfpathlineto{\pgfqpoint{1.015900in}{0.957870in}}%
\pgfpathlineto{\pgfqpoint{1.002334in}{0.966858in}}%
\pgfpathlineto{\pgfqpoint{0.989143in}{0.971481in}}%
\pgfpathlineto{\pgfqpoint{0.986678in}{0.972518in}}%
\pgfpathlineto{\pgfqpoint{0.971021in}{0.971775in}}%
\pgfpathlineto{\pgfqpoint{0.970443in}{0.971481in}}%
\pgfpathlineto{\pgfqpoint{0.955364in}{0.965117in}}%
\pgfpathlineto{\pgfqpoint{0.945252in}{0.957870in}}%
\pgfpathlineto{\pgfqpoint{0.939708in}{0.954263in}}%
\pgfpathlineto{\pgfqpoint{0.927467in}{0.944259in}}%
\pgfpathlineto{\pgfqpoint{0.924051in}{0.941490in}}%
\pgfpathlineto{\pgfqpoint{0.911654in}{0.930648in}}%
\pgfpathlineto{\pgfqpoint{0.908395in}{0.927493in}}%
\pgfpathlineto{\pgfqpoint{0.897293in}{0.917036in}}%
\pgfpathlineto{\pgfqpoint{0.892738in}{0.911414in}}%
\pgfpathlineto{\pgfqpoint{0.885484in}{0.903425in}}%
\pgfpathlineto{\pgfqpoint{0.879009in}{0.889814in}}%
\pgfpathlineto{\pgfqpoint{0.879755in}{0.876203in}}%
\pgfpathlineto{\pgfqpoint{0.887531in}{0.862592in}}%
\pgfpathlineto{\pgfqpoint{0.892738in}{0.857290in}}%
\pgfpathlineto{\pgfqpoint{0.899991in}{0.848981in}}%
\pgfpathlineto{\pgfqpoint{0.908395in}{0.841327in}}%
\pgfpathlineto{\pgfqpoint{0.914747in}{0.835370in}}%
\pgfpathlineto{\pgfqpoint{0.924051in}{0.827281in}}%
\pgfpathlineto{\pgfqpoint{0.930904in}{0.821759in}}%
\pgfpathlineto{\pgfqpoint{0.939708in}{0.814453in}}%
\pgfpathlineto{\pgfqpoint{0.949265in}{0.808148in}}%
\pgfpathlineto{\pgfqpoint{0.955364in}{0.803621in}}%
\pgfpathclose%
\pgfpathmoveto{\pgfqpoint{0.946774in}{0.835370in}}%
\pgfpathlineto{\pgfqpoint{0.939708in}{0.840239in}}%
\pgfpathlineto{\pgfqpoint{0.929652in}{0.848981in}}%
\pgfpathlineto{\pgfqpoint{0.924051in}{0.855124in}}%
\pgfpathlineto{\pgfqpoint{0.917862in}{0.862592in}}%
\pgfpathlineto{\pgfqpoint{0.911245in}{0.876203in}}%
\pgfpathlineto{\pgfqpoint{0.910611in}{0.889814in}}%
\pgfpathlineto{\pgfqpoint{0.916121in}{0.903425in}}%
\pgfpathlineto{\pgfqpoint{0.924051in}{0.913776in}}%
\pgfpathlineto{\pgfqpoint{0.926811in}{0.917036in}}%
\pgfpathlineto{\pgfqpoint{0.939708in}{0.928632in}}%
\pgfpathlineto{\pgfqpoint{0.942543in}{0.930648in}}%
\pgfpathlineto{\pgfqpoint{0.955364in}{0.938727in}}%
\pgfpathlineto{\pgfqpoint{0.970145in}{0.944259in}}%
\pgfpathlineto{\pgfqpoint{0.971021in}{0.944577in}}%
\pgfpathlineto{\pgfqpoint{0.986678in}{0.945108in}}%
\pgfpathlineto{\pgfqpoint{0.989504in}{0.944259in}}%
\pgfpathlineto{\pgfqpoint{1.002334in}{0.940272in}}%
\pgfpathlineto{\pgfqpoint{1.017991in}{0.931126in}}%
\pgfpathlineto{\pgfqpoint{1.018605in}{0.930648in}}%
\pgfpathlineto{\pgfqpoint{1.033647in}{0.917570in}}%
\pgfpathlineto{\pgfqpoint{1.034198in}{0.917036in}}%
\pgfpathlineto{\pgfqpoint{1.044718in}{0.903425in}}%
\pgfpathlineto{\pgfqpoint{1.049304in}{0.892271in}}%
\pgfpathlineto{\pgfqpoint{1.050281in}{0.889814in}}%
\pgfpathlineto{\pgfqpoint{1.049670in}{0.876203in}}%
\pgfpathlineto{\pgfqpoint{1.049304in}{0.875442in}}%
\pgfpathlineto{\pgfqpoint{1.042941in}{0.862592in}}%
\pgfpathlineto{\pgfqpoint{1.033647in}{0.851445in}}%
\pgfpathlineto{\pgfqpoint{1.031329in}{0.848981in}}%
\pgfpathlineto{\pgfqpoint{1.017991in}{0.837768in}}%
\pgfpathlineto{\pgfqpoint{1.014240in}{0.835370in}}%
\pgfpathlineto{\pgfqpoint{1.002334in}{0.828475in}}%
\pgfpathlineto{\pgfqpoint{0.986678in}{0.823685in}}%
\pgfpathlineto{\pgfqpoint{0.971021in}{0.824236in}}%
\pgfpathlineto{\pgfqpoint{0.955364in}{0.829989in}}%
\pgfpathlineto{\pgfqpoint{0.946774in}{0.835370in}}%
\pgfpathclose%
\pgfpathmoveto{\pgfqpoint{1.268496in}{0.801841in}}%
\pgfpathlineto{\pgfqpoint{1.284152in}{0.796212in}}%
\pgfpathlineto{\pgfqpoint{1.299809in}{0.796860in}}%
\pgfpathlineto{\pgfqpoint{1.315466in}{0.803621in}}%
\pgfpathlineto{\pgfqpoint{1.321565in}{0.808148in}}%
\pgfpathlineto{\pgfqpoint{1.331122in}{0.814453in}}%
\pgfpathlineto{\pgfqpoint{1.339926in}{0.821759in}}%
\pgfpathlineto{\pgfqpoint{1.346779in}{0.827281in}}%
\pgfpathlineto{\pgfqpoint{1.356083in}{0.835370in}}%
\pgfpathlineto{\pgfqpoint{1.362435in}{0.841327in}}%
\pgfpathlineto{\pgfqpoint{1.370839in}{0.848981in}}%
\pgfpathlineto{\pgfqpoint{1.378092in}{0.857290in}}%
\pgfpathlineto{\pgfqpoint{1.383299in}{0.862592in}}%
\pgfpathlineto{\pgfqpoint{1.391075in}{0.876203in}}%
\pgfpathlineto{\pgfqpoint{1.391821in}{0.889814in}}%
\pgfpathlineto{\pgfqpoint{1.385346in}{0.903425in}}%
\pgfpathlineto{\pgfqpoint{1.378092in}{0.911414in}}%
\pgfpathlineto{\pgfqpoint{1.373537in}{0.917036in}}%
\pgfpathlineto{\pgfqpoint{1.362435in}{0.927493in}}%
\pgfpathlineto{\pgfqpoint{1.359176in}{0.930648in}}%
\pgfpathlineto{\pgfqpoint{1.346779in}{0.941490in}}%
\pgfpathlineto{\pgfqpoint{1.343363in}{0.944259in}}%
\pgfpathlineto{\pgfqpoint{1.331122in}{0.954263in}}%
\pgfpathlineto{\pgfqpoint{1.325578in}{0.957870in}}%
\pgfpathlineto{\pgfqpoint{1.315466in}{0.965117in}}%
\pgfpathlineto{\pgfqpoint{1.300387in}{0.971481in}}%
\pgfpathlineto{\pgfqpoint{1.299809in}{0.971775in}}%
\pgfpathlineto{\pgfqpoint{1.284152in}{0.972518in}}%
\pgfpathlineto{\pgfqpoint{1.281687in}{0.971481in}}%
\pgfpathlineto{\pgfqpoint{1.268496in}{0.966858in}}%
\pgfpathlineto{\pgfqpoint{1.254930in}{0.957870in}}%
\pgfpathlineto{\pgfqpoint{1.252839in}{0.956607in}}%
\pgfpathlineto{\pgfqpoint{1.237215in}{0.944259in}}%
\pgfpathlineto{\pgfqpoint{1.237183in}{0.944233in}}%
\pgfpathlineto{\pgfqpoint{1.221555in}{0.930648in}}%
\pgfpathlineto{\pgfqpoint{1.221526in}{0.930620in}}%
\pgfpathlineto{\pgfqpoint{1.207322in}{0.917036in}}%
\pgfpathlineto{\pgfqpoint{1.205870in}{0.915218in}}%
\pgfpathlineto{\pgfqpoint{1.195531in}{0.903425in}}%
\pgfpathlineto{\pgfqpoint{1.190213in}{0.891957in}}%
\pgfpathlineto{\pgfqpoint{1.189020in}{0.889814in}}%
\pgfpathlineto{\pgfqpoint{1.189874in}{0.876203in}}%
\pgfpathlineto{\pgfqpoint{1.190213in}{0.875700in}}%
\pgfpathlineto{\pgfqpoint{1.197534in}{0.862592in}}%
\pgfpathlineto{\pgfqpoint{1.205870in}{0.853801in}}%
\pgfpathlineto{\pgfqpoint{1.210018in}{0.848981in}}%
\pgfpathlineto{\pgfqpoint{1.221526in}{0.838340in}}%
\pgfpathlineto{\pgfqpoint{1.224711in}{0.835370in}}%
\pgfpathlineto{\pgfqpoint{1.237183in}{0.824592in}}%
\pgfpathlineto{\pgfqpoint{1.240811in}{0.821759in}}%
\pgfpathlineto{\pgfqpoint{1.252839in}{0.812108in}}%
\pgfpathlineto{\pgfqpoint{1.259306in}{0.808148in}}%
\pgfpathlineto{\pgfqpoint{1.268496in}{0.801841in}}%
\pgfpathclose%
\pgfpathmoveto{\pgfqpoint{1.256590in}{0.835370in}}%
\pgfpathlineto{\pgfqpoint{1.252839in}{0.837768in}}%
\pgfpathlineto{\pgfqpoint{1.239501in}{0.848981in}}%
\pgfpathlineto{\pgfqpoint{1.237183in}{0.851445in}}%
\pgfpathlineto{\pgfqpoint{1.227889in}{0.862592in}}%
\pgfpathlineto{\pgfqpoint{1.221526in}{0.875442in}}%
\pgfpathlineto{\pgfqpoint{1.221160in}{0.876203in}}%
\pgfpathlineto{\pgfqpoint{1.220549in}{0.889814in}}%
\pgfpathlineto{\pgfqpoint{1.221526in}{0.892271in}}%
\pgfpathlineto{\pgfqpoint{1.226112in}{0.903425in}}%
\pgfpathlineto{\pgfqpoint{1.236632in}{0.917036in}}%
\pgfpathlineto{\pgfqpoint{1.237183in}{0.917570in}}%
\pgfpathlineto{\pgfqpoint{1.252225in}{0.930648in}}%
\pgfpathlineto{\pgfqpoint{1.252839in}{0.931126in}}%
\pgfpathlineto{\pgfqpoint{1.268496in}{0.940272in}}%
\pgfpathlineto{\pgfqpoint{1.281326in}{0.944259in}}%
\pgfpathlineto{\pgfqpoint{1.284152in}{0.945108in}}%
\pgfpathlineto{\pgfqpoint{1.299809in}{0.944577in}}%
\pgfpathlineto{\pgfqpoint{1.300685in}{0.944259in}}%
\pgfpathlineto{\pgfqpoint{1.315466in}{0.938727in}}%
\pgfpathlineto{\pgfqpoint{1.328287in}{0.930648in}}%
\pgfpathlineto{\pgfqpoint{1.331122in}{0.928632in}}%
\pgfpathlineto{\pgfqpoint{1.344019in}{0.917036in}}%
\pgfpathlineto{\pgfqpoint{1.346779in}{0.913776in}}%
\pgfpathlineto{\pgfqpoint{1.354709in}{0.903425in}}%
\pgfpathlineto{\pgfqpoint{1.360219in}{0.889814in}}%
\pgfpathlineto{\pgfqpoint{1.359585in}{0.876203in}}%
\pgfpathlineto{\pgfqpoint{1.352968in}{0.862592in}}%
\pgfpathlineto{\pgfqpoint{1.346779in}{0.855124in}}%
\pgfpathlineto{\pgfqpoint{1.341178in}{0.848981in}}%
\pgfpathlineto{\pgfqpoint{1.331122in}{0.840239in}}%
\pgfpathlineto{\pgfqpoint{1.324056in}{0.835370in}}%
\pgfpathlineto{\pgfqpoint{1.315466in}{0.829989in}}%
\pgfpathlineto{\pgfqpoint{1.299809in}{0.824236in}}%
\pgfpathlineto{\pgfqpoint{1.284152in}{0.823685in}}%
\pgfpathlineto{\pgfqpoint{1.268496in}{0.828475in}}%
\pgfpathlineto{\pgfqpoint{1.256590in}{0.835370in}}%
\pgfpathclose%
\pgfpathmoveto{\pgfqpoint{1.581627in}{0.800261in}}%
\pgfpathlineto{\pgfqpoint{1.597284in}{0.795820in}}%
\pgfpathlineto{\pgfqpoint{1.612940in}{0.797758in}}%
\pgfpathlineto{\pgfqpoint{1.628597in}{0.805582in}}%
\pgfpathlineto{\pgfqpoint{1.631830in}{0.808148in}}%
\pgfpathlineto{\pgfqpoint{1.644253in}{0.816881in}}%
\pgfpathlineto{\pgfqpoint{1.649974in}{0.821759in}}%
\pgfpathlineto{\pgfqpoint{1.659910in}{0.829972in}}%
\pgfpathlineto{\pgfqpoint{1.666110in}{0.835370in}}%
\pgfpathlineto{\pgfqpoint{1.675567in}{0.844251in}}%
\pgfpathlineto{\pgfqpoint{1.680859in}{0.848981in}}%
\pgfpathlineto{\pgfqpoint{1.691223in}{0.860644in}}%
\pgfpathlineto{\pgfqpoint{1.693213in}{0.862592in}}%
\pgfpathlineto{\pgfqpoint{1.701197in}{0.876203in}}%
\pgfpathlineto{\pgfqpoint{1.701962in}{0.889814in}}%
\pgfpathlineto{\pgfqpoint{1.695314in}{0.903425in}}%
\pgfpathlineto{\pgfqpoint{1.691223in}{0.907757in}}%
\pgfpathlineto{\pgfqpoint{1.683570in}{0.917036in}}%
\pgfpathlineto{\pgfqpoint{1.675567in}{0.924433in}}%
\pgfpathlineto{\pgfqpoint{1.669154in}{0.930648in}}%
\pgfpathlineto{\pgfqpoint{1.659910in}{0.938744in}}%
\pgfpathlineto{\pgfqpoint{1.653276in}{0.944259in}}%
\pgfpathlineto{\pgfqpoint{1.644253in}{0.951837in}}%
\pgfpathlineto{\pgfqpoint{1.635549in}{0.957870in}}%
\pgfpathlineto{\pgfqpoint{1.628597in}{0.963197in}}%
\pgfpathlineto{\pgfqpoint{1.612940in}{0.970854in}}%
\pgfpathlineto{\pgfqpoint{1.607735in}{0.971481in}}%
\pgfpathlineto{\pgfqpoint{1.597284in}{0.972966in}}%
\pgfpathlineto{\pgfqpoint{1.592775in}{0.971481in}}%
\pgfpathlineto{\pgfqpoint{1.581627in}{0.968404in}}%
\pgfpathlineto{\pgfqpoint{1.565971in}{0.958895in}}%
\pgfpathlineto{\pgfqpoint{1.564766in}{0.957870in}}%
\pgfpathlineto{\pgfqpoint{1.550314in}{0.946806in}}%
\pgfpathlineto{\pgfqpoint{1.547392in}{0.944259in}}%
\pgfpathlineto{\pgfqpoint{1.534657in}{0.933299in}}%
\pgfpathlineto{\pgfqpoint{1.531600in}{0.930648in}}%
\pgfpathlineto{\pgfqpoint{1.519001in}{0.918549in}}%
\pgfpathlineto{\pgfqpoint{1.517280in}{0.917036in}}%
\pgfpathlineto{\pgfqpoint{1.505625in}{0.903425in}}%
\pgfpathlineto{\pgfqpoint{1.503344in}{0.898361in}}%
\pgfpathlineto{\pgfqpoint{1.498837in}{0.889814in}}%
\pgfpathlineto{\pgfqpoint{1.499656in}{0.876203in}}%
\pgfpathlineto{\pgfqpoint{1.503344in}{0.870426in}}%
\pgfpathlineto{\pgfqpoint{1.507594in}{0.862592in}}%
\pgfpathlineto{\pgfqpoint{1.519001in}{0.850184in}}%
\pgfpathlineto{\pgfqpoint{1.520026in}{0.848981in}}%
\pgfpathlineto{\pgfqpoint{1.534587in}{0.835370in}}%
\pgfpathlineto{\pgfqpoint{1.534657in}{0.835309in}}%
\pgfpathlineto{\pgfqpoint{1.550314in}{0.821921in}}%
\pgfpathlineto{\pgfqpoint{1.550530in}{0.821759in}}%
\pgfpathlineto{\pgfqpoint{1.565971in}{0.809862in}}%
\pgfpathlineto{\pgfqpoint{1.569028in}{0.808148in}}%
\pgfpathlineto{\pgfqpoint{1.581627in}{0.800261in}}%
\pgfpathclose%
\pgfpathmoveto{\pgfqpoint{1.566028in}{0.835370in}}%
\pgfpathlineto{\pgfqpoint{1.565971in}{0.835404in}}%
\pgfpathlineto{\pgfqpoint{1.550314in}{0.848098in}}%
\pgfpathlineto{\pgfqpoint{1.549362in}{0.848981in}}%
\pgfpathlineto{\pgfqpoint{1.537839in}{0.862592in}}%
\pgfpathlineto{\pgfqpoint{1.534657in}{0.868953in}}%
\pgfpathlineto{\pgfqpoint{1.531205in}{0.876203in}}%
\pgfpathlineto{\pgfqpoint{1.530592in}{0.889814in}}%
\pgfpathlineto{\pgfqpoint{1.534657in}{0.900149in}}%
\pgfpathlineto{\pgfqpoint{1.536018in}{0.903425in}}%
\pgfpathlineto{\pgfqpoint{1.546799in}{0.917036in}}%
\pgfpathlineto{\pgfqpoint{1.550314in}{0.920407in}}%
\pgfpathlineto{\pgfqpoint{1.562551in}{0.930648in}}%
\pgfpathlineto{\pgfqpoint{1.565971in}{0.933207in}}%
\pgfpathlineto{\pgfqpoint{1.581627in}{0.941644in}}%
\pgfpathlineto{\pgfqpoint{1.592313in}{0.944259in}}%
\pgfpathlineto{\pgfqpoint{1.597284in}{0.945429in}}%
\pgfpathlineto{\pgfqpoint{1.608805in}{0.944259in}}%
\pgfpathlineto{\pgfqpoint{1.612940in}{0.943817in}}%
\pgfpathlineto{\pgfqpoint{1.628597in}{0.937024in}}%
\pgfpathlineto{\pgfqpoint{1.638061in}{0.930648in}}%
\pgfpathlineto{\pgfqpoint{1.644253in}{0.925955in}}%
\pgfpathlineto{\pgfqpoint{1.653908in}{0.917036in}}%
\pgfpathlineto{\pgfqpoint{1.659910in}{0.909761in}}%
\pgfpathlineto{\pgfqpoint{1.664757in}{0.903425in}}%
\pgfpathlineto{\pgfqpoint{1.670181in}{0.889814in}}%
\pgfpathlineto{\pgfqpoint{1.669557in}{0.876203in}}%
\pgfpathlineto{\pgfqpoint{1.663043in}{0.862592in}}%
\pgfpathlineto{\pgfqpoint{1.659910in}{0.858806in}}%
\pgfpathlineto{\pgfqpoint{1.651176in}{0.848981in}}%
\pgfpathlineto{\pgfqpoint{1.644253in}{0.842796in}}%
\pgfpathlineto{\pgfqpoint{1.634139in}{0.835370in}}%
\pgfpathlineto{\pgfqpoint{1.628597in}{0.831658in}}%
\pgfpathlineto{\pgfqpoint{1.612940in}{0.825000in}}%
\pgfpathlineto{\pgfqpoint{1.597284in}{0.823352in}}%
\pgfpathlineto{\pgfqpoint{1.581627in}{0.827130in}}%
\pgfpathlineto{\pgfqpoint{1.566028in}{0.835370in}}%
\pgfpathclose%
\pgfpathmoveto{\pgfqpoint{1.879102in}{0.807707in}}%
\pgfpathlineto{\pgfqpoint{1.894758in}{0.798895in}}%
\pgfpathlineto{\pgfqpoint{1.910415in}{0.795689in}}%
\pgfpathlineto{\pgfqpoint{1.910415in}{0.808148in}}%
\pgfpathlineto{\pgfqpoint{1.910415in}{0.821759in}}%
\pgfpathlineto{\pgfqpoint{1.910415in}{0.823240in}}%
\pgfpathlineto{\pgfqpoint{1.894758in}{0.825968in}}%
\pgfpathlineto{\pgfqpoint{1.879102in}{0.833467in}}%
\pgfpathlineto{\pgfqpoint{1.876418in}{0.835370in}}%
\pgfpathlineto{\pgfqpoint{1.863445in}{0.845422in}}%
\pgfpathlineto{\pgfqpoint{1.859545in}{0.848981in}}%
\pgfpathlineto{\pgfqpoint{1.847789in}{0.862473in}}%
\pgfpathlineto{\pgfqpoint{1.847690in}{0.862592in}}%
\pgfpathlineto{\pgfqpoint{1.841251in}{0.876203in}}%
\pgfpathlineto{\pgfqpoint{1.840634in}{0.889814in}}%
\pgfpathlineto{\pgfqpoint{1.845996in}{0.903425in}}%
\pgfpathlineto{\pgfqpoint{1.847789in}{0.905762in}}%
\pgfpathlineto{\pgfqpoint{1.856907in}{0.917036in}}%
\pgfpathlineto{\pgfqpoint{1.863445in}{0.923208in}}%
\pgfpathlineto{\pgfqpoint{1.872750in}{0.930648in}}%
\pgfpathlineto{\pgfqpoint{1.879102in}{0.935179in}}%
\pgfpathlineto{\pgfqpoint{1.894758in}{0.942830in}}%
\pgfpathlineto{\pgfqpoint{1.902853in}{0.944259in}}%
\pgfpathlineto{\pgfqpoint{1.910415in}{0.945536in}}%
\pgfpathlineto{\pgfqpoint{1.910415in}{0.957870in}}%
\pgfpathlineto{\pgfqpoint{1.910415in}{0.971481in}}%
\pgfpathlineto{\pgfqpoint{1.910415in}{0.973116in}}%
\pgfpathlineto{\pgfqpoint{1.903497in}{0.971481in}}%
\pgfpathlineto{\pgfqpoint{1.894758in}{0.969741in}}%
\pgfpathlineto{\pgfqpoint{1.879102in}{0.961118in}}%
\pgfpathlineto{\pgfqpoint{1.875099in}{0.957870in}}%
\pgfpathlineto{\pgfqpoint{1.863445in}{0.949345in}}%
\pgfpathlineto{\pgfqpoint{1.857517in}{0.944259in}}%
\pgfpathlineto{\pgfqpoint{1.847789in}{0.936010in}}%
\pgfpathlineto{\pgfqpoint{1.841649in}{0.930648in}}%
\pgfpathlineto{\pgfqpoint{1.832132in}{0.921448in}}%
\pgfpathlineto{\pgfqpoint{1.827247in}{0.917036in}}%
\pgfpathlineto{\pgfqpoint{1.816476in}{0.904254in}}%
\pgfpathlineto{\pgfqpoint{1.815657in}{0.903425in}}%
\pgfpathlineto{\pgfqpoint{1.808799in}{0.889814in}}%
\pgfpathlineto{\pgfqpoint{1.809589in}{0.876203in}}%
\pgfpathlineto{\pgfqpoint{1.816476in}{0.864879in}}%
\pgfpathlineto{\pgfqpoint{1.817685in}{0.862592in}}%
\pgfpathlineto{\pgfqpoint{1.829982in}{0.848981in}}%
\pgfpathlineto{\pgfqpoint{1.832132in}{0.847103in}}%
\pgfpathlineto{\pgfqpoint{1.844659in}{0.835370in}}%
\pgfpathlineto{\pgfqpoint{1.847789in}{0.832652in}}%
\pgfpathlineto{\pgfqpoint{1.860708in}{0.821759in}}%
\pgfpathlineto{\pgfqpoint{1.863445in}{0.819374in}}%
\pgfpathlineto{\pgfqpoint{1.878578in}{0.808148in}}%
\pgfpathlineto{\pgfqpoint{1.879102in}{0.807707in}}%
\pgfpathclose%
\pgfpathmoveto{\pgfqpoint{0.367333in}{1.066759in}}%
\pgfpathlineto{\pgfqpoint{0.376072in}{1.068499in}}%
\pgfpathlineto{\pgfqpoint{0.391728in}{1.077122in}}%
\pgfpathlineto{\pgfqpoint{0.395731in}{1.080370in}}%
\pgfpathlineto{\pgfqpoint{0.407385in}{1.088894in}}%
\pgfpathlineto{\pgfqpoint{0.413313in}{1.093981in}}%
\pgfpathlineto{\pgfqpoint{0.423041in}{1.102230in}}%
\pgfpathlineto{\pgfqpoint{0.429181in}{1.107592in}}%
\pgfpathlineto{\pgfqpoint{0.438698in}{1.116791in}}%
\pgfpathlineto{\pgfqpoint{0.443583in}{1.121203in}}%
\pgfpathlineto{\pgfqpoint{0.454354in}{1.133985in}}%
\pgfpathlineto{\pgfqpoint{0.455173in}{1.134814in}}%
\pgfpathlineto{\pgfqpoint{0.462031in}{1.148425in}}%
\pgfpathlineto{\pgfqpoint{0.461241in}{1.162036in}}%
\pgfpathlineto{\pgfqpoint{0.454354in}{1.173360in}}%
\pgfpathlineto{\pgfqpoint{0.453145in}{1.175647in}}%
\pgfpathlineto{\pgfqpoint{0.440848in}{1.189259in}}%
\pgfpathlineto{\pgfqpoint{0.438698in}{1.191137in}}%
\pgfpathlineto{\pgfqpoint{0.426171in}{1.202870in}}%
\pgfpathlineto{\pgfqpoint{0.423041in}{1.205587in}}%
\pgfpathlineto{\pgfqpoint{0.410122in}{1.216481in}}%
\pgfpathlineto{\pgfqpoint{0.407385in}{1.218865in}}%
\pgfpathlineto{\pgfqpoint{0.392252in}{1.230092in}}%
\pgfpathlineto{\pgfqpoint{0.391728in}{1.230532in}}%
\pgfpathlineto{\pgfqpoint{0.376072in}{1.239344in}}%
\pgfpathlineto{\pgfqpoint{0.360415in}{1.242550in}}%
\pgfpathlineto{\pgfqpoint{0.360415in}{1.230092in}}%
\pgfpathlineto{\pgfqpoint{0.360415in}{1.216481in}}%
\pgfpathlineto{\pgfqpoint{0.360415in}{1.214999in}}%
\pgfpathlineto{\pgfqpoint{0.376072in}{1.212271in}}%
\pgfpathlineto{\pgfqpoint{0.391728in}{1.204773in}}%
\pgfpathlineto{\pgfqpoint{0.394412in}{1.202870in}}%
\pgfpathlineto{\pgfqpoint{0.407385in}{1.192818in}}%
\pgfpathlineto{\pgfqpoint{0.411285in}{1.189259in}}%
\pgfpathlineto{\pgfqpoint{0.423041in}{1.175766in}}%
\pgfpathlineto{\pgfqpoint{0.423140in}{1.175647in}}%
\pgfpathlineto{\pgfqpoint{0.429579in}{1.162036in}}%
\pgfpathlineto{\pgfqpoint{0.430196in}{1.148425in}}%
\pgfpathlineto{\pgfqpoint{0.424834in}{1.134814in}}%
\pgfpathlineto{\pgfqpoint{0.423041in}{1.132477in}}%
\pgfpathlineto{\pgfqpoint{0.413923in}{1.121203in}}%
\pgfpathlineto{\pgfqpoint{0.407385in}{1.115032in}}%
\pgfpathlineto{\pgfqpoint{0.398080in}{1.107592in}}%
\pgfpathlineto{\pgfqpoint{0.391728in}{1.103061in}}%
\pgfpathlineto{\pgfqpoint{0.376072in}{1.095410in}}%
\pgfpathlineto{\pgfqpoint{0.367977in}{1.093981in}}%
\pgfpathlineto{\pgfqpoint{0.360415in}{1.092703in}}%
\pgfpathlineto{\pgfqpoint{0.360415in}{1.080370in}}%
\pgfpathlineto{\pgfqpoint{0.360415in}{1.066759in}}%
\pgfpathlineto{\pgfqpoint{0.360415in}{1.065123in}}%
\pgfpathlineto{\pgfqpoint{0.367333in}{1.066759in}}%
\pgfpathclose%
\pgfpathmoveto{\pgfqpoint{0.673546in}{1.065273in}}%
\pgfpathlineto{\pgfqpoint{0.678055in}{1.066759in}}%
\pgfpathlineto{\pgfqpoint{0.689203in}{1.069835in}}%
\pgfpathlineto{\pgfqpoint{0.704859in}{1.079344in}}%
\pgfpathlineto{\pgfqpoint{0.706064in}{1.080370in}}%
\pgfpathlineto{\pgfqpoint{0.720516in}{1.091433in}}%
\pgfpathlineto{\pgfqpoint{0.723438in}{1.093981in}}%
\pgfpathlineto{\pgfqpoint{0.736173in}{1.104940in}}%
\pgfpathlineto{\pgfqpoint{0.739230in}{1.107592in}}%
\pgfpathlineto{\pgfqpoint{0.751829in}{1.119690in}}%
\pgfpathlineto{\pgfqpoint{0.753550in}{1.121203in}}%
\pgfpathlineto{\pgfqpoint{0.765205in}{1.134814in}}%
\pgfpathlineto{\pgfqpoint{0.767486in}{1.139879in}}%
\pgfpathlineto{\pgfqpoint{0.771993in}{1.148425in}}%
\pgfpathlineto{\pgfqpoint{0.771174in}{1.162036in}}%
\pgfpathlineto{\pgfqpoint{0.767486in}{1.167813in}}%
\pgfpathlineto{\pgfqpoint{0.763236in}{1.175647in}}%
\pgfpathlineto{\pgfqpoint{0.751829in}{1.188055in}}%
\pgfpathlineto{\pgfqpoint{0.750804in}{1.189259in}}%
\pgfpathlineto{\pgfqpoint{0.736243in}{1.202870in}}%
\pgfpathlineto{\pgfqpoint{0.736173in}{1.202930in}}%
\pgfpathlineto{\pgfqpoint{0.720516in}{1.216318in}}%
\pgfpathlineto{\pgfqpoint{0.720300in}{1.216481in}}%
\pgfpathlineto{\pgfqpoint{0.704859in}{1.228377in}}%
\pgfpathlineto{\pgfqpoint{0.701802in}{1.230092in}}%
\pgfpathlineto{\pgfqpoint{0.689203in}{1.237978in}}%
\pgfpathlineto{\pgfqpoint{0.673546in}{1.242419in}}%
\pgfpathlineto{\pgfqpoint{0.657890in}{1.240482in}}%
\pgfpathlineto{\pgfqpoint{0.642233in}{1.232658in}}%
\pgfpathlineto{\pgfqpoint{0.639000in}{1.230092in}}%
\pgfpathlineto{\pgfqpoint{0.626577in}{1.221358in}}%
\pgfpathlineto{\pgfqpoint{0.620856in}{1.216481in}}%
\pgfpathlineto{\pgfqpoint{0.610920in}{1.208267in}}%
\pgfpathlineto{\pgfqpoint{0.604720in}{1.202870in}}%
\pgfpathlineto{\pgfqpoint{0.595263in}{1.193988in}}%
\pgfpathlineto{\pgfqpoint{0.589971in}{1.189259in}}%
\pgfpathlineto{\pgfqpoint{0.579607in}{1.177596in}}%
\pgfpathlineto{\pgfqpoint{0.577617in}{1.175647in}}%
\pgfpathlineto{\pgfqpoint{0.569633in}{1.162036in}}%
\pgfpathlineto{\pgfqpoint{0.568868in}{1.148425in}}%
\pgfpathlineto{\pgfqpoint{0.575516in}{1.134814in}}%
\pgfpathlineto{\pgfqpoint{0.579607in}{1.130482in}}%
\pgfpathlineto{\pgfqpoint{0.587260in}{1.121203in}}%
\pgfpathlineto{\pgfqpoint{0.595263in}{1.113807in}}%
\pgfpathlineto{\pgfqpoint{0.601676in}{1.107592in}}%
\pgfpathlineto{\pgfqpoint{0.610920in}{1.099495in}}%
\pgfpathlineto{\pgfqpoint{0.617554in}{1.093981in}}%
\pgfpathlineto{\pgfqpoint{0.626577in}{1.086403in}}%
\pgfpathlineto{\pgfqpoint{0.635281in}{1.080370in}}%
\pgfpathlineto{\pgfqpoint{0.642233in}{1.075042in}}%
\pgfpathlineto{\pgfqpoint{0.657890in}{1.067385in}}%
\pgfpathlineto{\pgfqpoint{0.663095in}{1.066759in}}%
\pgfpathlineto{\pgfqpoint{0.673546in}{1.065273in}}%
\pgfpathclose%
\pgfpathmoveto{\pgfqpoint{0.662025in}{1.093981in}}%
\pgfpathlineto{\pgfqpoint{0.657890in}{1.094422in}}%
\pgfpathlineto{\pgfqpoint{0.642233in}{1.101215in}}%
\pgfpathlineto{\pgfqpoint{0.632769in}{1.107592in}}%
\pgfpathlineto{\pgfqpoint{0.626577in}{1.112284in}}%
\pgfpathlineto{\pgfqpoint{0.616922in}{1.121203in}}%
\pgfpathlineto{\pgfqpoint{0.610920in}{1.128479in}}%
\pgfpathlineto{\pgfqpoint{0.606073in}{1.134814in}}%
\pgfpathlineto{\pgfqpoint{0.600649in}{1.148425in}}%
\pgfpathlineto{\pgfqpoint{0.601273in}{1.162036in}}%
\pgfpathlineto{\pgfqpoint{0.607787in}{1.175647in}}%
\pgfpathlineto{\pgfqpoint{0.610920in}{1.179433in}}%
\pgfpathlineto{\pgfqpoint{0.619654in}{1.189259in}}%
\pgfpathlineto{\pgfqpoint{0.626577in}{1.195443in}}%
\pgfpathlineto{\pgfqpoint{0.636691in}{1.202870in}}%
\pgfpathlineto{\pgfqpoint{0.642233in}{1.206581in}}%
\pgfpathlineto{\pgfqpoint{0.657890in}{1.213239in}}%
\pgfpathlineto{\pgfqpoint{0.673546in}{1.214888in}}%
\pgfpathlineto{\pgfqpoint{0.689203in}{1.211109in}}%
\pgfpathlineto{\pgfqpoint{0.704802in}{1.202870in}}%
\pgfpathlineto{\pgfqpoint{0.704859in}{1.202836in}}%
\pgfpathlineto{\pgfqpoint{0.720516in}{1.190141in}}%
\pgfpathlineto{\pgfqpoint{0.721468in}{1.189259in}}%
\pgfpathlineto{\pgfqpoint{0.732991in}{1.175647in}}%
\pgfpathlineto{\pgfqpoint{0.736173in}{1.169287in}}%
\pgfpathlineto{\pgfqpoint{0.739625in}{1.162036in}}%
\pgfpathlineto{\pgfqpoint{0.740238in}{1.148425in}}%
\pgfpathlineto{\pgfqpoint{0.736173in}{1.138090in}}%
\pgfpathlineto{\pgfqpoint{0.734812in}{1.134814in}}%
\pgfpathlineto{\pgfqpoint{0.724031in}{1.121203in}}%
\pgfpathlineto{\pgfqpoint{0.720516in}{1.117833in}}%
\pgfpathlineto{\pgfqpoint{0.708279in}{1.107592in}}%
\pgfpathlineto{\pgfqpoint{0.704859in}{1.105032in}}%
\pgfpathlineto{\pgfqpoint{0.689203in}{1.096596in}}%
\pgfpathlineto{\pgfqpoint{0.678517in}{1.093981in}}%
\pgfpathlineto{\pgfqpoint{0.673546in}{1.092811in}}%
\pgfpathlineto{\pgfqpoint{0.662025in}{1.093981in}}%
\pgfpathclose%
\pgfpathmoveto{\pgfqpoint{0.971021in}{1.066464in}}%
\pgfpathlineto{\pgfqpoint{0.986678in}{1.065722in}}%
\pgfpathlineto{\pgfqpoint{0.989143in}{1.066759in}}%
\pgfpathlineto{\pgfqpoint{1.002334in}{1.071382in}}%
\pgfpathlineto{\pgfqpoint{1.015900in}{1.080370in}}%
\pgfpathlineto{\pgfqpoint{1.017991in}{1.081632in}}%
\pgfpathlineto{\pgfqpoint{1.033615in}{1.093981in}}%
\pgfpathlineto{\pgfqpoint{1.033647in}{1.094006in}}%
\pgfpathlineto{\pgfqpoint{1.049275in}{1.107592in}}%
\pgfpathlineto{\pgfqpoint{1.049304in}{1.107620in}}%
\pgfpathlineto{\pgfqpoint{1.063508in}{1.121203in}}%
\pgfpathlineto{\pgfqpoint{1.064960in}{1.123021in}}%
\pgfpathlineto{\pgfqpoint{1.075299in}{1.134814in}}%
\pgfpathlineto{\pgfqpoint{1.080617in}{1.146282in}}%
\pgfpathlineto{\pgfqpoint{1.081810in}{1.148425in}}%
\pgfpathlineto{\pgfqpoint{1.080956in}{1.162036in}}%
\pgfpathlineto{\pgfqpoint{1.080617in}{1.162539in}}%
\pgfpathlineto{\pgfqpoint{1.073296in}{1.175647in}}%
\pgfpathlineto{\pgfqpoint{1.064960in}{1.184439in}}%
\pgfpathlineto{\pgfqpoint{1.060812in}{1.189259in}}%
\pgfpathlineto{\pgfqpoint{1.049304in}{1.199900in}}%
\pgfpathlineto{\pgfqpoint{1.046119in}{1.202870in}}%
\pgfpathlineto{\pgfqpoint{1.033647in}{1.213647in}}%
\pgfpathlineto{\pgfqpoint{1.030019in}{1.216481in}}%
\pgfpathlineto{\pgfqpoint{1.017991in}{1.226132in}}%
\pgfpathlineto{\pgfqpoint{1.011524in}{1.230092in}}%
\pgfpathlineto{\pgfqpoint{1.002334in}{1.236398in}}%
\pgfpathlineto{\pgfqpoint{0.986678in}{1.242027in}}%
\pgfpathlineto{\pgfqpoint{0.971021in}{1.241379in}}%
\pgfpathlineto{\pgfqpoint{0.955364in}{1.234619in}}%
\pgfpathlineto{\pgfqpoint{0.949265in}{1.230092in}}%
\pgfpathlineto{\pgfqpoint{0.939708in}{1.223786in}}%
\pgfpathlineto{\pgfqpoint{0.930904in}{1.216481in}}%
\pgfpathlineto{\pgfqpoint{0.924051in}{1.210959in}}%
\pgfpathlineto{\pgfqpoint{0.914747in}{1.202870in}}%
\pgfpathlineto{\pgfqpoint{0.908395in}{1.196912in}}%
\pgfpathlineto{\pgfqpoint{0.899991in}{1.189259in}}%
\pgfpathlineto{\pgfqpoint{0.892738in}{1.180950in}}%
\pgfpathlineto{\pgfqpoint{0.887531in}{1.175647in}}%
\pgfpathlineto{\pgfqpoint{0.879755in}{1.162036in}}%
\pgfpathlineto{\pgfqpoint{0.879009in}{1.148425in}}%
\pgfpathlineto{\pgfqpoint{0.885484in}{1.134814in}}%
\pgfpathlineto{\pgfqpoint{0.892738in}{1.126825in}}%
\pgfpathlineto{\pgfqpoint{0.897293in}{1.121203in}}%
\pgfpathlineto{\pgfqpoint{0.908395in}{1.110746in}}%
\pgfpathlineto{\pgfqpoint{0.911654in}{1.107592in}}%
\pgfpathlineto{\pgfqpoint{0.924051in}{1.096749in}}%
\pgfpathlineto{\pgfqpoint{0.927467in}{1.093981in}}%
\pgfpathlineto{\pgfqpoint{0.939708in}{1.083976in}}%
\pgfpathlineto{\pgfqpoint{0.945252in}{1.080370in}}%
\pgfpathlineto{\pgfqpoint{0.955364in}{1.073123in}}%
\pgfpathlineto{\pgfqpoint{0.970443in}{1.066759in}}%
\pgfpathlineto{\pgfqpoint{0.971021in}{1.066464in}}%
\pgfpathclose%
\pgfpathmoveto{\pgfqpoint{0.970145in}{1.093981in}}%
\pgfpathlineto{\pgfqpoint{0.955364in}{1.099513in}}%
\pgfpathlineto{\pgfqpoint{0.942543in}{1.107592in}}%
\pgfpathlineto{\pgfqpoint{0.939708in}{1.109608in}}%
\pgfpathlineto{\pgfqpoint{0.926811in}{1.121203in}}%
\pgfpathlineto{\pgfqpoint{0.924051in}{1.124464in}}%
\pgfpathlineto{\pgfqpoint{0.916121in}{1.134814in}}%
\pgfpathlineto{\pgfqpoint{0.910611in}{1.148425in}}%
\pgfpathlineto{\pgfqpoint{0.911245in}{1.162036in}}%
\pgfpathlineto{\pgfqpoint{0.917862in}{1.175647in}}%
\pgfpathlineto{\pgfqpoint{0.924051in}{1.183116in}}%
\pgfpathlineto{\pgfqpoint{0.929652in}{1.189259in}}%
\pgfpathlineto{\pgfqpoint{0.939708in}{1.198000in}}%
\pgfpathlineto{\pgfqpoint{0.946774in}{1.202870in}}%
\pgfpathlineto{\pgfqpoint{0.955364in}{1.208250in}}%
\pgfpathlineto{\pgfqpoint{0.971021in}{1.214003in}}%
\pgfpathlineto{\pgfqpoint{0.986678in}{1.214555in}}%
\pgfpathlineto{\pgfqpoint{1.002334in}{1.209764in}}%
\pgfpathlineto{\pgfqpoint{1.014240in}{1.202870in}}%
\pgfpathlineto{\pgfqpoint{1.017991in}{1.200471in}}%
\pgfpathlineto{\pgfqpoint{1.031329in}{1.189259in}}%
\pgfpathlineto{\pgfqpoint{1.033647in}{1.186794in}}%
\pgfpathlineto{\pgfqpoint{1.042941in}{1.175647in}}%
\pgfpathlineto{\pgfqpoint{1.049304in}{1.162798in}}%
\pgfpathlineto{\pgfqpoint{1.049670in}{1.162036in}}%
\pgfpathlineto{\pgfqpoint{1.050281in}{1.148425in}}%
\pgfpathlineto{\pgfqpoint{1.049304in}{1.145969in}}%
\pgfpathlineto{\pgfqpoint{1.044718in}{1.134814in}}%
\pgfpathlineto{\pgfqpoint{1.034198in}{1.121203in}}%
\pgfpathlineto{\pgfqpoint{1.033647in}{1.120669in}}%
\pgfpathlineto{\pgfqpoint{1.018605in}{1.107592in}}%
\pgfpathlineto{\pgfqpoint{1.017991in}{1.107113in}}%
\pgfpathlineto{\pgfqpoint{1.002334in}{1.097968in}}%
\pgfpathlineto{\pgfqpoint{0.989504in}{1.093981in}}%
\pgfpathlineto{\pgfqpoint{0.986678in}{1.093131in}}%
\pgfpathlineto{\pgfqpoint{0.971021in}{1.093662in}}%
\pgfpathlineto{\pgfqpoint{0.970145in}{1.093981in}}%
\pgfpathclose%
\pgfpathmoveto{\pgfqpoint{1.284152in}{1.065722in}}%
\pgfpathlineto{\pgfqpoint{1.299809in}{1.066464in}}%
\pgfpathlineto{\pgfqpoint{1.300387in}{1.066759in}}%
\pgfpathlineto{\pgfqpoint{1.315466in}{1.073123in}}%
\pgfpathlineto{\pgfqpoint{1.325578in}{1.080370in}}%
\pgfpathlineto{\pgfqpoint{1.331122in}{1.083976in}}%
\pgfpathlineto{\pgfqpoint{1.343363in}{1.093981in}}%
\pgfpathlineto{\pgfqpoint{1.346779in}{1.096749in}}%
\pgfpathlineto{\pgfqpoint{1.359176in}{1.107592in}}%
\pgfpathlineto{\pgfqpoint{1.362435in}{1.110746in}}%
\pgfpathlineto{\pgfqpoint{1.373537in}{1.121203in}}%
\pgfpathlineto{\pgfqpoint{1.378092in}{1.126825in}}%
\pgfpathlineto{\pgfqpoint{1.385346in}{1.134814in}}%
\pgfpathlineto{\pgfqpoint{1.391821in}{1.148425in}}%
\pgfpathlineto{\pgfqpoint{1.391075in}{1.162036in}}%
\pgfpathlineto{\pgfqpoint{1.383299in}{1.175647in}}%
\pgfpathlineto{\pgfqpoint{1.378092in}{1.180950in}}%
\pgfpathlineto{\pgfqpoint{1.370839in}{1.189259in}}%
\pgfpathlineto{\pgfqpoint{1.362435in}{1.196912in}}%
\pgfpathlineto{\pgfqpoint{1.356083in}{1.202870in}}%
\pgfpathlineto{\pgfqpoint{1.346779in}{1.210959in}}%
\pgfpathlineto{\pgfqpoint{1.339926in}{1.216481in}}%
\pgfpathlineto{\pgfqpoint{1.331122in}{1.223786in}}%
\pgfpathlineto{\pgfqpoint{1.321565in}{1.230092in}}%
\pgfpathlineto{\pgfqpoint{1.315466in}{1.234619in}}%
\pgfpathlineto{\pgfqpoint{1.299809in}{1.241379in}}%
\pgfpathlineto{\pgfqpoint{1.284152in}{1.242027in}}%
\pgfpathlineto{\pgfqpoint{1.268496in}{1.236398in}}%
\pgfpathlineto{\pgfqpoint{1.259306in}{1.230092in}}%
\pgfpathlineto{\pgfqpoint{1.252839in}{1.226132in}}%
\pgfpathlineto{\pgfqpoint{1.240811in}{1.216481in}}%
\pgfpathlineto{\pgfqpoint{1.237183in}{1.213647in}}%
\pgfpathlineto{\pgfqpoint{1.224711in}{1.202870in}}%
\pgfpathlineto{\pgfqpoint{1.221526in}{1.199900in}}%
\pgfpathlineto{\pgfqpoint{1.210018in}{1.189259in}}%
\pgfpathlineto{\pgfqpoint{1.205870in}{1.184439in}}%
\pgfpathlineto{\pgfqpoint{1.197534in}{1.175647in}}%
\pgfpathlineto{\pgfqpoint{1.190213in}{1.162539in}}%
\pgfpathlineto{\pgfqpoint{1.189874in}{1.162036in}}%
\pgfpathlineto{\pgfqpoint{1.189020in}{1.148425in}}%
\pgfpathlineto{\pgfqpoint{1.190213in}{1.146282in}}%
\pgfpathlineto{\pgfqpoint{1.195531in}{1.134814in}}%
\pgfpathlineto{\pgfqpoint{1.205870in}{1.123021in}}%
\pgfpathlineto{\pgfqpoint{1.207322in}{1.121203in}}%
\pgfpathlineto{\pgfqpoint{1.221526in}{1.107620in}}%
\pgfpathlineto{\pgfqpoint{1.221555in}{1.107592in}}%
\pgfpathlineto{\pgfqpoint{1.237183in}{1.094006in}}%
\pgfpathlineto{\pgfqpoint{1.237215in}{1.093981in}}%
\pgfpathlineto{\pgfqpoint{1.252839in}{1.081632in}}%
\pgfpathlineto{\pgfqpoint{1.254930in}{1.080370in}}%
\pgfpathlineto{\pgfqpoint{1.268496in}{1.071382in}}%
\pgfpathlineto{\pgfqpoint{1.281687in}{1.066759in}}%
\pgfpathlineto{\pgfqpoint{1.284152in}{1.065722in}}%
\pgfpathclose%
\pgfpathmoveto{\pgfqpoint{1.281326in}{1.093981in}}%
\pgfpathlineto{\pgfqpoint{1.268496in}{1.097968in}}%
\pgfpathlineto{\pgfqpoint{1.252839in}{1.107113in}}%
\pgfpathlineto{\pgfqpoint{1.252225in}{1.107592in}}%
\pgfpathlineto{\pgfqpoint{1.237183in}{1.120669in}}%
\pgfpathlineto{\pgfqpoint{1.236632in}{1.121203in}}%
\pgfpathlineto{\pgfqpoint{1.226112in}{1.134814in}}%
\pgfpathlineto{\pgfqpoint{1.221526in}{1.145969in}}%
\pgfpathlineto{\pgfqpoint{1.220549in}{1.148425in}}%
\pgfpathlineto{\pgfqpoint{1.221160in}{1.162036in}}%
\pgfpathlineto{\pgfqpoint{1.221526in}{1.162798in}}%
\pgfpathlineto{\pgfqpoint{1.227889in}{1.175647in}}%
\pgfpathlineto{\pgfqpoint{1.237183in}{1.186794in}}%
\pgfpathlineto{\pgfqpoint{1.239501in}{1.189259in}}%
\pgfpathlineto{\pgfqpoint{1.252839in}{1.200471in}}%
\pgfpathlineto{\pgfqpoint{1.256590in}{1.202870in}}%
\pgfpathlineto{\pgfqpoint{1.268496in}{1.209764in}}%
\pgfpathlineto{\pgfqpoint{1.284152in}{1.214555in}}%
\pgfpathlineto{\pgfqpoint{1.299809in}{1.214003in}}%
\pgfpathlineto{\pgfqpoint{1.315466in}{1.208250in}}%
\pgfpathlineto{\pgfqpoint{1.324056in}{1.202870in}}%
\pgfpathlineto{\pgfqpoint{1.331122in}{1.198000in}}%
\pgfpathlineto{\pgfqpoint{1.341178in}{1.189259in}}%
\pgfpathlineto{\pgfqpoint{1.346779in}{1.183116in}}%
\pgfpathlineto{\pgfqpoint{1.352968in}{1.175647in}}%
\pgfpathlineto{\pgfqpoint{1.359585in}{1.162036in}}%
\pgfpathlineto{\pgfqpoint{1.360219in}{1.148425in}}%
\pgfpathlineto{\pgfqpoint{1.354709in}{1.134814in}}%
\pgfpathlineto{\pgfqpoint{1.346779in}{1.124464in}}%
\pgfpathlineto{\pgfqpoint{1.344019in}{1.121203in}}%
\pgfpathlineto{\pgfqpoint{1.331122in}{1.109608in}}%
\pgfpathlineto{\pgfqpoint{1.328287in}{1.107592in}}%
\pgfpathlineto{\pgfqpoint{1.315466in}{1.099513in}}%
\pgfpathlineto{\pgfqpoint{1.300685in}{1.093981in}}%
\pgfpathlineto{\pgfqpoint{1.299809in}{1.093662in}}%
\pgfpathlineto{\pgfqpoint{1.284152in}{1.093131in}}%
\pgfpathlineto{\pgfqpoint{1.281326in}{1.093981in}}%
\pgfpathclose%
\pgfpathmoveto{\pgfqpoint{1.597284in}{1.065273in}}%
\pgfpathlineto{\pgfqpoint{1.607735in}{1.066759in}}%
\pgfpathlineto{\pgfqpoint{1.612940in}{1.067385in}}%
\pgfpathlineto{\pgfqpoint{1.628597in}{1.075042in}}%
\pgfpathlineto{\pgfqpoint{1.635549in}{1.080370in}}%
\pgfpathlineto{\pgfqpoint{1.644253in}{1.086403in}}%
\pgfpathlineto{\pgfqpoint{1.653276in}{1.093981in}}%
\pgfpathlineto{\pgfqpoint{1.659910in}{1.099495in}}%
\pgfpathlineto{\pgfqpoint{1.669154in}{1.107592in}}%
\pgfpathlineto{\pgfqpoint{1.675567in}{1.113807in}}%
\pgfpathlineto{\pgfqpoint{1.683570in}{1.121203in}}%
\pgfpathlineto{\pgfqpoint{1.691223in}{1.130482in}}%
\pgfpathlineto{\pgfqpoint{1.695314in}{1.134814in}}%
\pgfpathlineto{\pgfqpoint{1.701962in}{1.148425in}}%
\pgfpathlineto{\pgfqpoint{1.701197in}{1.162036in}}%
\pgfpathlineto{\pgfqpoint{1.693213in}{1.175647in}}%
\pgfpathlineto{\pgfqpoint{1.691223in}{1.177596in}}%
\pgfpathlineto{\pgfqpoint{1.680859in}{1.189259in}}%
\pgfpathlineto{\pgfqpoint{1.675567in}{1.193988in}}%
\pgfpathlineto{\pgfqpoint{1.666110in}{1.202870in}}%
\pgfpathlineto{\pgfqpoint{1.659910in}{1.208267in}}%
\pgfpathlineto{\pgfqpoint{1.649974in}{1.216481in}}%
\pgfpathlineto{\pgfqpoint{1.644253in}{1.221358in}}%
\pgfpathlineto{\pgfqpoint{1.631830in}{1.230092in}}%
\pgfpathlineto{\pgfqpoint{1.628597in}{1.232658in}}%
\pgfpathlineto{\pgfqpoint{1.612940in}{1.240482in}}%
\pgfpathlineto{\pgfqpoint{1.597284in}{1.242419in}}%
\pgfpathlineto{\pgfqpoint{1.581627in}{1.237978in}}%
\pgfpathlineto{\pgfqpoint{1.569028in}{1.230092in}}%
\pgfpathlineto{\pgfqpoint{1.565971in}{1.228377in}}%
\pgfpathlineto{\pgfqpoint{1.550530in}{1.216481in}}%
\pgfpathlineto{\pgfqpoint{1.550314in}{1.216318in}}%
\pgfpathlineto{\pgfqpoint{1.534657in}{1.202930in}}%
\pgfpathlineto{\pgfqpoint{1.534587in}{1.202870in}}%
\pgfpathlineto{\pgfqpoint{1.520026in}{1.189259in}}%
\pgfpathlineto{\pgfqpoint{1.519001in}{1.188055in}}%
\pgfpathlineto{\pgfqpoint{1.507594in}{1.175647in}}%
\pgfpathlineto{\pgfqpoint{1.503344in}{1.167813in}}%
\pgfpathlineto{\pgfqpoint{1.499656in}{1.162036in}}%
\pgfpathlineto{\pgfqpoint{1.498837in}{1.148425in}}%
\pgfpathlineto{\pgfqpoint{1.503344in}{1.139879in}}%
\pgfpathlineto{\pgfqpoint{1.505625in}{1.134814in}}%
\pgfpathlineto{\pgfqpoint{1.517280in}{1.121203in}}%
\pgfpathlineto{\pgfqpoint{1.519001in}{1.119690in}}%
\pgfpathlineto{\pgfqpoint{1.531600in}{1.107592in}}%
\pgfpathlineto{\pgfqpoint{1.534657in}{1.104940in}}%
\pgfpathlineto{\pgfqpoint{1.547392in}{1.093981in}}%
\pgfpathlineto{\pgfqpoint{1.550314in}{1.091433in}}%
\pgfpathlineto{\pgfqpoint{1.564766in}{1.080370in}}%
\pgfpathlineto{\pgfqpoint{1.565971in}{1.079344in}}%
\pgfpathlineto{\pgfqpoint{1.581627in}{1.069835in}}%
\pgfpathlineto{\pgfqpoint{1.592775in}{1.066759in}}%
\pgfpathlineto{\pgfqpoint{1.597284in}{1.065273in}}%
\pgfpathclose%
\pgfpathmoveto{\pgfqpoint{1.592313in}{1.093981in}}%
\pgfpathlineto{\pgfqpoint{1.581627in}{1.096596in}}%
\pgfpathlineto{\pgfqpoint{1.565971in}{1.105032in}}%
\pgfpathlineto{\pgfqpoint{1.562551in}{1.107592in}}%
\pgfpathlineto{\pgfqpoint{1.550314in}{1.117833in}}%
\pgfpathlineto{\pgfqpoint{1.546799in}{1.121203in}}%
\pgfpathlineto{\pgfqpoint{1.536018in}{1.134814in}}%
\pgfpathlineto{\pgfqpoint{1.534657in}{1.138090in}}%
\pgfpathlineto{\pgfqpoint{1.530592in}{1.148425in}}%
\pgfpathlineto{\pgfqpoint{1.531205in}{1.162036in}}%
\pgfpathlineto{\pgfqpoint{1.534657in}{1.169287in}}%
\pgfpathlineto{\pgfqpoint{1.537839in}{1.175647in}}%
\pgfpathlineto{\pgfqpoint{1.549362in}{1.189259in}}%
\pgfpathlineto{\pgfqpoint{1.550314in}{1.190141in}}%
\pgfpathlineto{\pgfqpoint{1.565971in}{1.202836in}}%
\pgfpathlineto{\pgfqpoint{1.566028in}{1.202870in}}%
\pgfpathlineto{\pgfqpoint{1.581627in}{1.211109in}}%
\pgfpathlineto{\pgfqpoint{1.597284in}{1.214888in}}%
\pgfpathlineto{\pgfqpoint{1.612940in}{1.213239in}}%
\pgfpathlineto{\pgfqpoint{1.628597in}{1.206581in}}%
\pgfpathlineto{\pgfqpoint{1.634139in}{1.202870in}}%
\pgfpathlineto{\pgfqpoint{1.644253in}{1.195443in}}%
\pgfpathlineto{\pgfqpoint{1.651176in}{1.189259in}}%
\pgfpathlineto{\pgfqpoint{1.659910in}{1.179433in}}%
\pgfpathlineto{\pgfqpoint{1.663043in}{1.175647in}}%
\pgfpathlineto{\pgfqpoint{1.669557in}{1.162036in}}%
\pgfpathlineto{\pgfqpoint{1.670181in}{1.148425in}}%
\pgfpathlineto{\pgfqpoint{1.664757in}{1.134814in}}%
\pgfpathlineto{\pgfqpoint{1.659910in}{1.128479in}}%
\pgfpathlineto{\pgfqpoint{1.653908in}{1.121203in}}%
\pgfpathlineto{\pgfqpoint{1.644253in}{1.112284in}}%
\pgfpathlineto{\pgfqpoint{1.638061in}{1.107592in}}%
\pgfpathlineto{\pgfqpoint{1.628597in}{1.101215in}}%
\pgfpathlineto{\pgfqpoint{1.612940in}{1.094422in}}%
\pgfpathlineto{\pgfqpoint{1.608805in}{1.093981in}}%
\pgfpathlineto{\pgfqpoint{1.597284in}{1.092811in}}%
\pgfpathlineto{\pgfqpoint{1.592313in}{1.093981in}}%
\pgfpathclose%
\pgfpathmoveto{\pgfqpoint{1.910415in}{1.065123in}}%
\pgfpathlineto{\pgfqpoint{1.910415in}{1.066759in}}%
\pgfpathlineto{\pgfqpoint{1.910415in}{1.080370in}}%
\pgfpathlineto{\pgfqpoint{1.910415in}{1.092703in}}%
\pgfpathlineto{\pgfqpoint{1.902853in}{1.093981in}}%
\pgfpathlineto{\pgfqpoint{1.894758in}{1.095410in}}%
\pgfpathlineto{\pgfqpoint{1.879102in}{1.103061in}}%
\pgfpathlineto{\pgfqpoint{1.872750in}{1.107592in}}%
\pgfpathlineto{\pgfqpoint{1.863445in}{1.115032in}}%
\pgfpathlineto{\pgfqpoint{1.856907in}{1.121203in}}%
\pgfpathlineto{\pgfqpoint{1.847789in}{1.132477in}}%
\pgfpathlineto{\pgfqpoint{1.845996in}{1.134814in}}%
\pgfpathlineto{\pgfqpoint{1.840634in}{1.148425in}}%
\pgfpathlineto{\pgfqpoint{1.841251in}{1.162036in}}%
\pgfpathlineto{\pgfqpoint{1.847690in}{1.175647in}}%
\pgfpathlineto{\pgfqpoint{1.847789in}{1.175766in}}%
\pgfpathlineto{\pgfqpoint{1.859545in}{1.189259in}}%
\pgfpathlineto{\pgfqpoint{1.863445in}{1.192818in}}%
\pgfpathlineto{\pgfqpoint{1.876418in}{1.202870in}}%
\pgfpathlineto{\pgfqpoint{1.879102in}{1.204773in}}%
\pgfpathlineto{\pgfqpoint{1.894758in}{1.212271in}}%
\pgfpathlineto{\pgfqpoint{1.910415in}{1.214999in}}%
\pgfpathlineto{\pgfqpoint{1.910415in}{1.216481in}}%
\pgfpathlineto{\pgfqpoint{1.910415in}{1.230092in}}%
\pgfpathlineto{\pgfqpoint{1.910415in}{1.242550in}}%
\pgfpathlineto{\pgfqpoint{1.894758in}{1.239344in}}%
\pgfpathlineto{\pgfqpoint{1.879102in}{1.230532in}}%
\pgfpathlineto{\pgfqpoint{1.878578in}{1.230092in}}%
\pgfpathlineto{\pgfqpoint{1.863445in}{1.218865in}}%
\pgfpathlineto{\pgfqpoint{1.860708in}{1.216481in}}%
\pgfpathlineto{\pgfqpoint{1.847789in}{1.205587in}}%
\pgfpathlineto{\pgfqpoint{1.844659in}{1.202870in}}%
\pgfpathlineto{\pgfqpoint{1.832132in}{1.191137in}}%
\pgfpathlineto{\pgfqpoint{1.829982in}{1.189259in}}%
\pgfpathlineto{\pgfqpoint{1.817685in}{1.175647in}}%
\pgfpathlineto{\pgfqpoint{1.816476in}{1.173360in}}%
\pgfpathlineto{\pgfqpoint{1.809589in}{1.162036in}}%
\pgfpathlineto{\pgfqpoint{1.808799in}{1.148425in}}%
\pgfpathlineto{\pgfqpoint{1.815657in}{1.134814in}}%
\pgfpathlineto{\pgfqpoint{1.816476in}{1.133985in}}%
\pgfpathlineto{\pgfqpoint{1.827247in}{1.121203in}}%
\pgfpathlineto{\pgfqpoint{1.832132in}{1.116791in}}%
\pgfpathlineto{\pgfqpoint{1.841649in}{1.107592in}}%
\pgfpathlineto{\pgfqpoint{1.847789in}{1.102230in}}%
\pgfpathlineto{\pgfqpoint{1.857517in}{1.093981in}}%
\pgfpathlineto{\pgfqpoint{1.863445in}{1.088894in}}%
\pgfpathlineto{\pgfqpoint{1.875099in}{1.080370in}}%
\pgfpathlineto{\pgfqpoint{1.879102in}{1.077122in}}%
\pgfpathlineto{\pgfqpoint{1.894758in}{1.068499in}}%
\pgfpathlineto{\pgfqpoint{1.903497in}{1.066759in}}%
\pgfpathlineto{\pgfqpoint{1.910415in}{1.065123in}}%
\pgfpathclose%
\pgfpathmoveto{\pgfqpoint{0.376072in}{1.338009in}}%
\pgfpathlineto{\pgfqpoint{0.377587in}{1.338981in}}%
\pgfpathlineto{\pgfqpoint{0.391728in}{1.346606in}}%
\pgfpathlineto{\pgfqpoint{0.399338in}{1.352592in}}%
\pgfpathlineto{\pgfqpoint{0.407385in}{1.358417in}}%
\pgfpathlineto{\pgfqpoint{0.416561in}{1.366203in}}%
\pgfpathlineto{\pgfqpoint{0.423041in}{1.371754in}}%
\pgfpathlineto{\pgfqpoint{0.432171in}{1.379814in}}%
\pgfpathlineto{\pgfqpoint{0.438698in}{1.386368in}}%
\pgfpathlineto{\pgfqpoint{0.446201in}{1.393425in}}%
\pgfpathlineto{\pgfqpoint{0.454354in}{1.403988in}}%
\pgfpathlineto{\pgfqpoint{0.457098in}{1.407036in}}%
\pgfpathlineto{\pgfqpoint{0.462508in}{1.420648in}}%
\pgfpathlineto{\pgfqpoint{0.460148in}{1.434259in}}%
\pgfpathlineto{\pgfqpoint{0.454354in}{1.442416in}}%
\pgfpathlineto{\pgfqpoint{0.451002in}{1.447870in}}%
\pgfpathlineto{\pgfqpoint{0.438698in}{1.460729in}}%
\pgfpathlineto{\pgfqpoint{0.438020in}{1.461481in}}%
\pgfpathlineto{\pgfqpoint{0.423159in}{1.475092in}}%
\pgfpathlineto{\pgfqpoint{0.423041in}{1.475194in}}%
\pgfpathlineto{\pgfqpoint{0.407385in}{1.488354in}}%
\pgfpathlineto{\pgfqpoint{0.406899in}{1.488703in}}%
\pgfpathlineto{\pgfqpoint{0.391728in}{1.500054in}}%
\pgfpathlineto{\pgfqpoint{0.387336in}{1.502314in}}%
\pgfpathlineto{\pgfqpoint{0.376072in}{1.508896in}}%
\pgfpathlineto{\pgfqpoint{0.360415in}{1.512187in}}%
\pgfpathlineto{\pgfqpoint{0.360415in}{1.502314in}}%
\pgfpathlineto{\pgfqpoint{0.360415in}{1.488703in}}%
\pgfpathlineto{\pgfqpoint{0.360415in}{1.484459in}}%
\pgfpathlineto{\pgfqpoint{0.376072in}{1.481774in}}%
\pgfpathlineto{\pgfqpoint{0.390190in}{1.475092in}}%
\pgfpathlineto{\pgfqpoint{0.391728in}{1.474300in}}%
\pgfpathlineto{\pgfqpoint{0.407385in}{1.462518in}}%
\pgfpathlineto{\pgfqpoint{0.408553in}{1.461481in}}%
\pgfpathlineto{\pgfqpoint{0.421083in}{1.447870in}}%
\pgfpathlineto{\pgfqpoint{0.423041in}{1.444550in}}%
\pgfpathlineto{\pgfqpoint{0.428724in}{1.434259in}}%
\pgfpathlineto{\pgfqpoint{0.430569in}{1.420648in}}%
\pgfpathlineto{\pgfqpoint{0.426340in}{1.407036in}}%
\pgfpathlineto{\pgfqpoint{0.423041in}{1.402323in}}%
\pgfpathlineto{\pgfqpoint{0.416450in}{1.393425in}}%
\pgfpathlineto{\pgfqpoint{0.407385in}{1.384516in}}%
\pgfpathlineto{\pgfqpoint{0.401725in}{1.379814in}}%
\pgfpathlineto{\pgfqpoint{0.391728in}{1.372605in}}%
\pgfpathlineto{\pgfqpoint{0.378852in}{1.366203in}}%
\pgfpathlineto{\pgfqpoint{0.376072in}{1.364874in}}%
\pgfpathlineto{\pgfqpoint{0.360415in}{1.362239in}}%
\pgfpathlineto{\pgfqpoint{0.360415in}{1.352592in}}%
\pgfpathlineto{\pgfqpoint{0.360415in}{1.338981in}}%
\pgfpathlineto{\pgfqpoint{0.360415in}{1.334488in}}%
\pgfpathlineto{\pgfqpoint{0.376072in}{1.338009in}}%
\pgfpathclose%
\pgfpathmoveto{\pgfqpoint{0.657890in}{1.336760in}}%
\pgfpathlineto{\pgfqpoint{0.673546in}{1.334632in}}%
\pgfpathlineto{\pgfqpoint{0.687476in}{1.338981in}}%
\pgfpathlineto{\pgfqpoint{0.689203in}{1.339444in}}%
\pgfpathlineto{\pgfqpoint{0.704859in}{1.348790in}}%
\pgfpathlineto{\pgfqpoint{0.709464in}{1.352592in}}%
\pgfpathlineto{\pgfqpoint{0.720516in}{1.360965in}}%
\pgfpathlineto{\pgfqpoint{0.726592in}{1.366203in}}%
\pgfpathlineto{\pgfqpoint{0.736173in}{1.374532in}}%
\pgfpathlineto{\pgfqpoint{0.742198in}{1.379814in}}%
\pgfpathlineto{\pgfqpoint{0.751829in}{1.389422in}}%
\pgfpathlineto{\pgfqpoint{0.756202in}{1.393425in}}%
\pgfpathlineto{\pgfqpoint{0.766953in}{1.407036in}}%
\pgfpathlineto{\pgfqpoint{0.767486in}{1.408538in}}%
\pgfpathlineto{\pgfqpoint{0.772488in}{1.420647in}}%
\pgfpathlineto{\pgfqpoint{0.770040in}{1.434259in}}%
\pgfpathlineto{\pgfqpoint{0.767486in}{1.437685in}}%
\pgfpathlineto{\pgfqpoint{0.761066in}{1.447870in}}%
\pgfpathlineto{\pgfqpoint{0.751829in}{1.457266in}}%
\pgfpathlineto{\pgfqpoint{0.748004in}{1.461481in}}%
\pgfpathlineto{\pgfqpoint{0.736173in}{1.472247in}}%
\pgfpathlineto{\pgfqpoint{0.733011in}{1.475092in}}%
\pgfpathlineto{\pgfqpoint{0.720516in}{1.485758in}}%
\pgfpathlineto{\pgfqpoint{0.716592in}{1.488703in}}%
\pgfpathlineto{\pgfqpoint{0.704859in}{1.497917in}}%
\pgfpathlineto{\pgfqpoint{0.697158in}{1.502314in}}%
\pgfpathlineto{\pgfqpoint{0.689203in}{1.507493in}}%
\pgfpathlineto{\pgfqpoint{0.673546in}{1.512052in}}%
\pgfpathlineto{\pgfqpoint{0.657890in}{1.510063in}}%
\pgfpathlineto{\pgfqpoint{0.642804in}{1.502314in}}%
\pgfpathlineto{\pgfqpoint{0.642233in}{1.502054in}}%
\pgfpathlineto{\pgfqpoint{0.626577in}{1.490865in}}%
\pgfpathlineto{\pgfqpoint{0.624089in}{1.488703in}}%
\pgfpathlineto{\pgfqpoint{0.610920in}{1.477832in}}%
\pgfpathlineto{\pgfqpoint{0.607768in}{1.475092in}}%
\pgfpathlineto{\pgfqpoint{0.595263in}{1.463643in}}%
\pgfpathlineto{\pgfqpoint{0.592777in}{1.461481in}}%
\pgfpathlineto{\pgfqpoint{0.579906in}{1.447870in}}%
\pgfpathlineto{\pgfqpoint{0.579607in}{1.447373in}}%
\pgfpathlineto{\pgfqpoint{0.570693in}{1.434259in}}%
\pgfpathlineto{\pgfqpoint{0.568405in}{1.420647in}}%
\pgfpathlineto{\pgfqpoint{0.573649in}{1.407036in}}%
\pgfpathlineto{\pgfqpoint{0.579607in}{1.400121in}}%
\pgfpathlineto{\pgfqpoint{0.584665in}{1.393425in}}%
\pgfpathlineto{\pgfqpoint{0.595263in}{1.383225in}}%
\pgfpathlineto{\pgfqpoint{0.598651in}{1.379814in}}%
\pgfpathlineto{\pgfqpoint{0.610920in}{1.368951in}}%
\pgfpathlineto{\pgfqpoint{0.614193in}{1.366203in}}%
\pgfpathlineto{\pgfqpoint{0.626577in}{1.355918in}}%
\pgfpathlineto{\pgfqpoint{0.631425in}{1.352592in}}%
\pgfpathlineto{\pgfqpoint{0.642233in}{1.344562in}}%
\pgfpathlineto{\pgfqpoint{0.653949in}{1.338981in}}%
\pgfpathlineto{\pgfqpoint{0.657890in}{1.336760in}}%
\pgfpathclose%
\pgfpathmoveto{\pgfqpoint{0.652503in}{1.366203in}}%
\pgfpathlineto{\pgfqpoint{0.642233in}{1.370714in}}%
\pgfpathlineto{\pgfqpoint{0.628873in}{1.379814in}}%
\pgfpathlineto{\pgfqpoint{0.626577in}{1.381621in}}%
\pgfpathlineto{\pgfqpoint{0.614308in}{1.393425in}}%
\pgfpathlineto{\pgfqpoint{0.610920in}{1.397909in}}%
\pgfpathlineto{\pgfqpoint{0.604550in}{1.407036in}}%
\pgfpathlineto{\pgfqpoint{0.600271in}{1.420648in}}%
\pgfpathlineto{\pgfqpoint{0.602138in}{1.434259in}}%
\pgfpathlineto{\pgfqpoint{0.609677in}{1.447870in}}%
\pgfpathlineto{\pgfqpoint{0.610920in}{1.449275in}}%
\pgfpathlineto{\pgfqpoint{0.622481in}{1.461481in}}%
\pgfpathlineto{\pgfqpoint{0.626577in}{1.465042in}}%
\pgfpathlineto{\pgfqpoint{0.640617in}{1.475092in}}%
\pgfpathlineto{\pgfqpoint{0.642233in}{1.476173in}}%
\pgfpathlineto{\pgfqpoint{0.657890in}{1.482727in}}%
\pgfpathlineto{\pgfqpoint{0.673546in}{1.484349in}}%
\pgfpathlineto{\pgfqpoint{0.689203in}{1.480630in}}%
\pgfpathlineto{\pgfqpoint{0.699702in}{1.475092in}}%
\pgfpathlineto{\pgfqpoint{0.704859in}{1.472147in}}%
\pgfpathlineto{\pgfqpoint{0.718437in}{1.461481in}}%
\pgfpathlineto{\pgfqpoint{0.720516in}{1.459485in}}%
\pgfpathlineto{\pgfqpoint{0.730983in}{1.447870in}}%
\pgfpathlineto{\pgfqpoint{0.736173in}{1.438941in}}%
\pgfpathlineto{\pgfqpoint{0.738777in}{1.434259in}}%
\pgfpathlineto{\pgfqpoint{0.740608in}{1.420648in}}%
\pgfpathlineto{\pgfqpoint{0.736410in}{1.407036in}}%
\pgfpathlineto{\pgfqpoint{0.736173in}{1.406700in}}%
\pgfpathlineto{\pgfqpoint{0.726484in}{1.393425in}}%
\pgfpathlineto{\pgfqpoint{0.720516in}{1.387466in}}%
\pgfpathlineto{\pgfqpoint{0.711715in}{1.379814in}}%
\pgfpathlineto{\pgfqpoint{0.704859in}{1.374626in}}%
\pgfpathlineto{\pgfqpoint{0.689590in}{1.366203in}}%
\pgfpathlineto{\pgfqpoint{0.689203in}{1.365997in}}%
\pgfpathlineto{\pgfqpoint{0.673546in}{1.362347in}}%
\pgfpathlineto{\pgfqpoint{0.657890in}{1.363939in}}%
\pgfpathlineto{\pgfqpoint{0.652503in}{1.366203in}}%
\pgfpathclose%
\pgfpathmoveto{\pgfqpoint{0.971021in}{1.335774in}}%
\pgfpathlineto{\pgfqpoint{0.986678in}{1.335063in}}%
\pgfpathlineto{\pgfqpoint{0.996509in}{1.338981in}}%
\pgfpathlineto{\pgfqpoint{1.002334in}{1.340964in}}%
\pgfpathlineto{\pgfqpoint{1.017991in}{1.351096in}}%
\pgfpathlineto{\pgfqpoint{1.019731in}{1.352592in}}%
\pgfpathlineto{\pgfqpoint{1.033647in}{1.363545in}}%
\pgfpathlineto{\pgfqpoint{1.036697in}{1.366203in}}%
\pgfpathlineto{\pgfqpoint{1.049304in}{1.377274in}}%
\pgfpathlineto{\pgfqpoint{1.052234in}{1.379814in}}%
\pgfpathlineto{\pgfqpoint{1.064960in}{1.392378in}}%
\pgfpathlineto{\pgfqpoint{1.066140in}{1.393425in}}%
\pgfpathlineto{\pgfqpoint{1.077078in}{1.407036in}}%
\pgfpathlineto{\pgfqpoint{1.080617in}{1.416728in}}%
\pgfpathlineto{\pgfqpoint{1.082326in}{1.420648in}}%
\pgfpathlineto{\pgfqpoint{1.080617in}{1.429733in}}%
\pgfpathlineto{\pgfqpoint{1.079896in}{1.434259in}}%
\pgfpathlineto{\pgfqpoint{1.071089in}{1.447870in}}%
\pgfpathlineto{\pgfqpoint{1.064960in}{1.453914in}}%
\pgfpathlineto{\pgfqpoint{1.058021in}{1.461481in}}%
\pgfpathlineto{\pgfqpoint{1.049304in}{1.469325in}}%
\pgfpathlineto{\pgfqpoint{1.042961in}{1.475092in}}%
\pgfpathlineto{\pgfqpoint{1.033647in}{1.483128in}}%
\pgfpathlineto{\pgfqpoint{1.026499in}{1.488703in}}%
\pgfpathlineto{\pgfqpoint{1.017991in}{1.495661in}}%
\pgfpathlineto{\pgfqpoint{1.007317in}{1.502314in}}%
\pgfpathlineto{\pgfqpoint{1.002334in}{1.505871in}}%
\pgfpathlineto{\pgfqpoint{0.986678in}{1.511650in}}%
\pgfpathlineto{\pgfqpoint{0.971021in}{1.510985in}}%
\pgfpathlineto{\pgfqpoint{0.955364in}{1.504044in}}%
\pgfpathlineto{\pgfqpoint{0.953123in}{1.502314in}}%
\pgfpathlineto{\pgfqpoint{0.939708in}{1.493304in}}%
\pgfpathlineto{\pgfqpoint{0.934268in}{1.488703in}}%
\pgfpathlineto{\pgfqpoint{0.924051in}{1.480482in}}%
\pgfpathlineto{\pgfqpoint{0.917843in}{1.475092in}}%
\pgfpathlineto{\pgfqpoint{0.908395in}{1.466454in}}%
\pgfpathlineto{\pgfqpoint{0.902784in}{1.461481in}}%
\pgfpathlineto{\pgfqpoint{0.892738in}{1.450680in}}%
\pgfpathlineto{\pgfqpoint{0.889787in}{1.447870in}}%
\pgfpathlineto{\pgfqpoint{0.880787in}{1.434259in}}%
\pgfpathlineto{\pgfqpoint{0.878559in}{1.420648in}}%
\pgfpathlineto{\pgfqpoint{0.883667in}{1.407036in}}%
\pgfpathlineto{\pgfqpoint{0.892738in}{1.396083in}}%
\pgfpathlineto{\pgfqpoint{0.894711in}{1.393425in}}%
\pgfpathlineto{\pgfqpoint{0.908395in}{1.380002in}}%
\pgfpathlineto{\pgfqpoint{0.908582in}{1.379814in}}%
\pgfpathlineto{\pgfqpoint{0.923982in}{1.366203in}}%
\pgfpathlineto{\pgfqpoint{0.924051in}{1.366142in}}%
\pgfpathlineto{\pgfqpoint{0.939708in}{1.353483in}}%
\pgfpathlineto{\pgfqpoint{0.941092in}{1.352592in}}%
\pgfpathlineto{\pgfqpoint{0.955364in}{1.342675in}}%
\pgfpathlineto{\pgfqpoint{0.964376in}{1.338981in}}%
\pgfpathlineto{\pgfqpoint{0.971021in}{1.335774in}}%
\pgfpathclose%
\pgfpathmoveto{\pgfqpoint{0.962681in}{1.366203in}}%
\pgfpathlineto{\pgfqpoint{0.955364in}{1.368969in}}%
\pgfpathlineto{\pgfqpoint{0.939708in}{1.378987in}}%
\pgfpathlineto{\pgfqpoint{0.938693in}{1.379814in}}%
\pgfpathlineto{\pgfqpoint{0.924090in}{1.393425in}}%
\pgfpathlineto{\pgfqpoint{0.924051in}{1.393475in}}%
\pgfpathlineto{\pgfqpoint{0.914574in}{1.407036in}}%
\pgfpathlineto{\pgfqpoint{0.910227in}{1.420648in}}%
\pgfpathlineto{\pgfqpoint{0.912123in}{1.434259in}}%
\pgfpathlineto{\pgfqpoint{0.919782in}{1.447870in}}%
\pgfpathlineto{\pgfqpoint{0.924051in}{1.452688in}}%
\pgfpathlineto{\pgfqpoint{0.932594in}{1.461481in}}%
\pgfpathlineto{\pgfqpoint{0.939708in}{1.467499in}}%
\pgfpathlineto{\pgfqpoint{0.951010in}{1.475092in}}%
\pgfpathlineto{\pgfqpoint{0.955364in}{1.477816in}}%
\pgfpathlineto{\pgfqpoint{0.971021in}{1.483478in}}%
\pgfpathlineto{\pgfqpoint{0.986678in}{1.484021in}}%
\pgfpathlineto{\pgfqpoint{1.002334in}{1.479306in}}%
\pgfpathlineto{\pgfqpoint{1.009622in}{1.475092in}}%
\pgfpathlineto{\pgfqpoint{1.017991in}{1.469874in}}%
\pgfpathlineto{\pgfqpoint{1.028250in}{1.461481in}}%
\pgfpathlineto{\pgfqpoint{1.033647in}{1.456097in}}%
\pgfpathlineto{\pgfqpoint{1.040982in}{1.447870in}}%
\pgfpathlineto{\pgfqpoint{1.048796in}{1.434259in}}%
\pgfpathlineto{\pgfqpoint{1.049304in}{1.430663in}}%
\pgfpathlineto{\pgfqpoint{1.050650in}{1.420648in}}%
\pgfpathlineto{\pgfqpoint{1.049304in}{1.416326in}}%
\pgfpathlineto{\pgfqpoint{1.046296in}{1.407036in}}%
\pgfpathlineto{\pgfqpoint{1.036592in}{1.393425in}}%
\pgfpathlineto{\pgfqpoint{1.033647in}{1.390453in}}%
\pgfpathlineto{\pgfqpoint{1.021868in}{1.379814in}}%
\pgfpathlineto{\pgfqpoint{1.017991in}{1.376759in}}%
\pgfpathlineto{\pgfqpoint{1.002334in}{1.367386in}}%
\pgfpathlineto{\pgfqpoint{0.998566in}{1.366203in}}%
\pgfpathlineto{\pgfqpoint{0.986678in}{1.362669in}}%
\pgfpathlineto{\pgfqpoint{0.971021in}{1.363201in}}%
\pgfpathlineto{\pgfqpoint{0.962681in}{1.366203in}}%
\pgfpathclose%
\pgfpathmoveto{\pgfqpoint{1.284152in}{1.335063in}}%
\pgfpathlineto{\pgfqpoint{1.299809in}{1.335774in}}%
\pgfpathlineto{\pgfqpoint{1.306454in}{1.338981in}}%
\pgfpathlineto{\pgfqpoint{1.315466in}{1.342675in}}%
\pgfpathlineto{\pgfqpoint{1.329738in}{1.352592in}}%
\pgfpathlineto{\pgfqpoint{1.331122in}{1.353483in}}%
\pgfpathlineto{\pgfqpoint{1.346779in}{1.366142in}}%
\pgfpathlineto{\pgfqpoint{1.346848in}{1.366203in}}%
\pgfpathlineto{\pgfqpoint{1.362248in}{1.379814in}}%
\pgfpathlineto{\pgfqpoint{1.362435in}{1.380002in}}%
\pgfpathlineto{\pgfqpoint{1.376119in}{1.393425in}}%
\pgfpathlineto{\pgfqpoint{1.378092in}{1.396083in}}%
\pgfpathlineto{\pgfqpoint{1.387163in}{1.407036in}}%
\pgfpathlineto{\pgfqpoint{1.392271in}{1.420648in}}%
\pgfpathlineto{\pgfqpoint{1.390043in}{1.434259in}}%
\pgfpathlineto{\pgfqpoint{1.381043in}{1.447870in}}%
\pgfpathlineto{\pgfqpoint{1.378092in}{1.450680in}}%
\pgfpathlineto{\pgfqpoint{1.368046in}{1.461481in}}%
\pgfpathlineto{\pgfqpoint{1.362435in}{1.466454in}}%
\pgfpathlineto{\pgfqpoint{1.352987in}{1.475092in}}%
\pgfpathlineto{\pgfqpoint{1.346779in}{1.480482in}}%
\pgfpathlineto{\pgfqpoint{1.336562in}{1.488703in}}%
\pgfpathlineto{\pgfqpoint{1.331122in}{1.493304in}}%
\pgfpathlineto{\pgfqpoint{1.317707in}{1.502314in}}%
\pgfpathlineto{\pgfqpoint{1.315466in}{1.504044in}}%
\pgfpathlineto{\pgfqpoint{1.299809in}{1.510985in}}%
\pgfpathlineto{\pgfqpoint{1.284152in}{1.511650in}}%
\pgfpathlineto{\pgfqpoint{1.268496in}{1.505871in}}%
\pgfpathlineto{\pgfqpoint{1.263513in}{1.502314in}}%
\pgfpathlineto{\pgfqpoint{1.252839in}{1.495661in}}%
\pgfpathlineto{\pgfqpoint{1.244331in}{1.488703in}}%
\pgfpathlineto{\pgfqpoint{1.237183in}{1.483128in}}%
\pgfpathlineto{\pgfqpoint{1.227869in}{1.475092in}}%
\pgfpathlineto{\pgfqpoint{1.221526in}{1.469325in}}%
\pgfpathlineto{\pgfqpoint{1.212809in}{1.461481in}}%
\pgfpathlineto{\pgfqpoint{1.205870in}{1.453914in}}%
\pgfpathlineto{\pgfqpoint{1.199741in}{1.447870in}}%
\pgfpathlineto{\pgfqpoint{1.190934in}{1.434259in}}%
\pgfpathlineto{\pgfqpoint{1.190213in}{1.429733in}}%
\pgfpathlineto{\pgfqpoint{1.188504in}{1.420648in}}%
\pgfpathlineto{\pgfqpoint{1.190213in}{1.416728in}}%
\pgfpathlineto{\pgfqpoint{1.193752in}{1.407036in}}%
\pgfpathlineto{\pgfqpoint{1.204690in}{1.393425in}}%
\pgfpathlineto{\pgfqpoint{1.205870in}{1.392378in}}%
\pgfpathlineto{\pgfqpoint{1.218596in}{1.379814in}}%
\pgfpathlineto{\pgfqpoint{1.221526in}{1.377274in}}%
\pgfpathlineto{\pgfqpoint{1.234133in}{1.366203in}}%
\pgfpathlineto{\pgfqpoint{1.237183in}{1.363545in}}%
\pgfpathlineto{\pgfqpoint{1.251099in}{1.352592in}}%
\pgfpathlineto{\pgfqpoint{1.252839in}{1.351096in}}%
\pgfpathlineto{\pgfqpoint{1.268496in}{1.340964in}}%
\pgfpathlineto{\pgfqpoint{1.274321in}{1.338981in}}%
\pgfpathlineto{\pgfqpoint{1.284152in}{1.335063in}}%
\pgfpathclose%
\pgfpathmoveto{\pgfqpoint{1.272264in}{1.366203in}}%
\pgfpathlineto{\pgfqpoint{1.268496in}{1.367386in}}%
\pgfpathlineto{\pgfqpoint{1.252839in}{1.376759in}}%
\pgfpathlineto{\pgfqpoint{1.248962in}{1.379814in}}%
\pgfpathlineto{\pgfqpoint{1.237183in}{1.390453in}}%
\pgfpathlineto{\pgfqpoint{1.234238in}{1.393425in}}%
\pgfpathlineto{\pgfqpoint{1.224534in}{1.407036in}}%
\pgfpathlineto{\pgfqpoint{1.221526in}{1.416326in}}%
\pgfpathlineto{\pgfqpoint{1.220180in}{1.420648in}}%
\pgfpathlineto{\pgfqpoint{1.221526in}{1.430663in}}%
\pgfpathlineto{\pgfqpoint{1.222034in}{1.434259in}}%
\pgfpathlineto{\pgfqpoint{1.229848in}{1.447870in}}%
\pgfpathlineto{\pgfqpoint{1.237183in}{1.456097in}}%
\pgfpathlineto{\pgfqpoint{1.242580in}{1.461481in}}%
\pgfpathlineto{\pgfqpoint{1.252839in}{1.469874in}}%
\pgfpathlineto{\pgfqpoint{1.261208in}{1.475092in}}%
\pgfpathlineto{\pgfqpoint{1.268496in}{1.479306in}}%
\pgfpathlineto{\pgfqpoint{1.284152in}{1.484021in}}%
\pgfpathlineto{\pgfqpoint{1.299809in}{1.483478in}}%
\pgfpathlineto{\pgfqpoint{1.315466in}{1.477816in}}%
\pgfpathlineto{\pgfqpoint{1.319820in}{1.475092in}}%
\pgfpathlineto{\pgfqpoint{1.331122in}{1.467499in}}%
\pgfpathlineto{\pgfqpoint{1.338236in}{1.461481in}}%
\pgfpathlineto{\pgfqpoint{1.346779in}{1.452688in}}%
\pgfpathlineto{\pgfqpoint{1.351048in}{1.447870in}}%
\pgfpathlineto{\pgfqpoint{1.358707in}{1.434259in}}%
\pgfpathlineto{\pgfqpoint{1.360603in}{1.420648in}}%
\pgfpathlineto{\pgfqpoint{1.356256in}{1.407036in}}%
\pgfpathlineto{\pgfqpoint{1.346779in}{1.393475in}}%
\pgfpathlineto{\pgfqpoint{1.346740in}{1.393425in}}%
\pgfpathlineto{\pgfqpoint{1.332137in}{1.379814in}}%
\pgfpathlineto{\pgfqpoint{1.331122in}{1.378987in}}%
\pgfpathlineto{\pgfqpoint{1.315466in}{1.368969in}}%
\pgfpathlineto{\pgfqpoint{1.308149in}{1.366203in}}%
\pgfpathlineto{\pgfqpoint{1.299809in}{1.363201in}}%
\pgfpathlineto{\pgfqpoint{1.284152in}{1.362669in}}%
\pgfpathlineto{\pgfqpoint{1.272264in}{1.366203in}}%
\pgfpathclose%
\pgfpathmoveto{\pgfqpoint{1.597284in}{1.334632in}}%
\pgfpathlineto{\pgfqpoint{1.612940in}{1.336760in}}%
\pgfpathlineto{\pgfqpoint{1.616881in}{1.338981in}}%
\pgfpathlineto{\pgfqpoint{1.628597in}{1.344562in}}%
\pgfpathlineto{\pgfqpoint{1.639405in}{1.352592in}}%
\pgfpathlineto{\pgfqpoint{1.644253in}{1.355918in}}%
\pgfpathlineto{\pgfqpoint{1.656637in}{1.366203in}}%
\pgfpathlineto{\pgfqpoint{1.659910in}{1.368951in}}%
\pgfpathlineto{\pgfqpoint{1.672179in}{1.379814in}}%
\pgfpathlineto{\pgfqpoint{1.675567in}{1.383225in}}%
\pgfpathlineto{\pgfqpoint{1.686165in}{1.393425in}}%
\pgfpathlineto{\pgfqpoint{1.691223in}{1.400121in}}%
\pgfpathlineto{\pgfqpoint{1.697181in}{1.407036in}}%
\pgfpathlineto{\pgfqpoint{1.702425in}{1.420648in}}%
\pgfpathlineto{\pgfqpoint{1.700137in}{1.434259in}}%
\pgfpathlineto{\pgfqpoint{1.691223in}{1.447373in}}%
\pgfpathlineto{\pgfqpoint{1.690924in}{1.447870in}}%
\pgfpathlineto{\pgfqpoint{1.678053in}{1.461481in}}%
\pgfpathlineto{\pgfqpoint{1.675567in}{1.463643in}}%
\pgfpathlineto{\pgfqpoint{1.663062in}{1.475092in}}%
\pgfpathlineto{\pgfqpoint{1.659910in}{1.477832in}}%
\pgfpathlineto{\pgfqpoint{1.646741in}{1.488703in}}%
\pgfpathlineto{\pgfqpoint{1.644253in}{1.490865in}}%
\pgfpathlineto{\pgfqpoint{1.628597in}{1.502054in}}%
\pgfpathlineto{\pgfqpoint{1.628026in}{1.502314in}}%
\pgfpathlineto{\pgfqpoint{1.612940in}{1.510063in}}%
\pgfpathlineto{\pgfqpoint{1.597284in}{1.512052in}}%
\pgfpathlineto{\pgfqpoint{1.581627in}{1.507493in}}%
\pgfpathlineto{\pgfqpoint{1.573672in}{1.502314in}}%
\pgfpathlineto{\pgfqpoint{1.565971in}{1.497917in}}%
\pgfpathlineto{\pgfqpoint{1.554238in}{1.488703in}}%
\pgfpathlineto{\pgfqpoint{1.550314in}{1.485758in}}%
\pgfpathlineto{\pgfqpoint{1.537819in}{1.475092in}}%
\pgfpathlineto{\pgfqpoint{1.534657in}{1.472247in}}%
\pgfpathlineto{\pgfqpoint{1.522826in}{1.461481in}}%
\pgfpathlineto{\pgfqpoint{1.519001in}{1.457266in}}%
\pgfpathlineto{\pgfqpoint{1.509764in}{1.447870in}}%
\pgfpathlineto{\pgfqpoint{1.503344in}{1.437685in}}%
\pgfpathlineto{\pgfqpoint{1.500790in}{1.434259in}}%
\pgfpathlineto{\pgfqpoint{1.498342in}{1.420647in}}%
\pgfpathlineto{\pgfqpoint{1.503344in}{1.408538in}}%
\pgfpathlineto{\pgfqpoint{1.503877in}{1.407036in}}%
\pgfpathlineto{\pgfqpoint{1.514628in}{1.393425in}}%
\pgfpathlineto{\pgfqpoint{1.519001in}{1.389422in}}%
\pgfpathlineto{\pgfqpoint{1.528632in}{1.379814in}}%
\pgfpathlineto{\pgfqpoint{1.534657in}{1.374532in}}%
\pgfpathlineto{\pgfqpoint{1.544238in}{1.366203in}}%
\pgfpathlineto{\pgfqpoint{1.550314in}{1.360965in}}%
\pgfpathlineto{\pgfqpoint{1.561366in}{1.352592in}}%
\pgfpathlineto{\pgfqpoint{1.565971in}{1.348790in}}%
\pgfpathlineto{\pgfqpoint{1.581627in}{1.339444in}}%
\pgfpathlineto{\pgfqpoint{1.583354in}{1.338981in}}%
\pgfpathlineto{\pgfqpoint{1.597284in}{1.334632in}}%
\pgfpathclose%
\pgfpathmoveto{\pgfqpoint{1.581240in}{1.366203in}}%
\pgfpathlineto{\pgfqpoint{1.565971in}{1.374626in}}%
\pgfpathlineto{\pgfqpoint{1.559115in}{1.379814in}}%
\pgfpathlineto{\pgfqpoint{1.550314in}{1.387466in}}%
\pgfpathlineto{\pgfqpoint{1.544346in}{1.393425in}}%
\pgfpathlineto{\pgfqpoint{1.534657in}{1.406700in}}%
\pgfpathlineto{\pgfqpoint{1.534420in}{1.407036in}}%
\pgfpathlineto{\pgfqpoint{1.530222in}{1.420648in}}%
\pgfpathlineto{\pgfqpoint{1.532053in}{1.434259in}}%
\pgfpathlineto{\pgfqpoint{1.534657in}{1.438941in}}%
\pgfpathlineto{\pgfqpoint{1.539847in}{1.447870in}}%
\pgfpathlineto{\pgfqpoint{1.550314in}{1.459485in}}%
\pgfpathlineto{\pgfqpoint{1.552393in}{1.461481in}}%
\pgfpathlineto{\pgfqpoint{1.565971in}{1.472147in}}%
\pgfpathlineto{\pgfqpoint{1.571128in}{1.475092in}}%
\pgfpathlineto{\pgfqpoint{1.581627in}{1.480630in}}%
\pgfpathlineto{\pgfqpoint{1.597284in}{1.484349in}}%
\pgfpathlineto{\pgfqpoint{1.612940in}{1.482727in}}%
\pgfpathlineto{\pgfqpoint{1.628597in}{1.476173in}}%
\pgfpathlineto{\pgfqpoint{1.630213in}{1.475092in}}%
\pgfpathlineto{\pgfqpoint{1.644253in}{1.465042in}}%
\pgfpathlineto{\pgfqpoint{1.648349in}{1.461481in}}%
\pgfpathlineto{\pgfqpoint{1.659910in}{1.449275in}}%
\pgfpathlineto{\pgfqpoint{1.661153in}{1.447870in}}%
\pgfpathlineto{\pgfqpoint{1.668692in}{1.434259in}}%
\pgfpathlineto{\pgfqpoint{1.670559in}{1.420647in}}%
\pgfpathlineto{\pgfqpoint{1.666280in}{1.407036in}}%
\pgfpathlineto{\pgfqpoint{1.659910in}{1.397909in}}%
\pgfpathlineto{\pgfqpoint{1.656522in}{1.393425in}}%
\pgfpathlineto{\pgfqpoint{1.644253in}{1.381621in}}%
\pgfpathlineto{\pgfqpoint{1.641957in}{1.379814in}}%
\pgfpathlineto{\pgfqpoint{1.628597in}{1.370714in}}%
\pgfpathlineto{\pgfqpoint{1.618327in}{1.366203in}}%
\pgfpathlineto{\pgfqpoint{1.612940in}{1.363939in}}%
\pgfpathlineto{\pgfqpoint{1.597284in}{1.362347in}}%
\pgfpathlineto{\pgfqpoint{1.581627in}{1.365997in}}%
\pgfpathlineto{\pgfqpoint{1.581240in}{1.366203in}}%
\pgfpathclose%
\pgfpathmoveto{\pgfqpoint{1.894758in}{1.338009in}}%
\pgfpathlineto{\pgfqpoint{1.910415in}{1.334488in}}%
\pgfpathlineto{\pgfqpoint{1.910415in}{1.338981in}}%
\pgfpathlineto{\pgfqpoint{1.910415in}{1.352592in}}%
\pgfpathlineto{\pgfqpoint{1.910415in}{1.362239in}}%
\pgfpathlineto{\pgfqpoint{1.894758in}{1.364874in}}%
\pgfpathlineto{\pgfqpoint{1.891978in}{1.366203in}}%
\pgfpathlineto{\pgfqpoint{1.879102in}{1.372605in}}%
\pgfpathlineto{\pgfqpoint{1.869105in}{1.379814in}}%
\pgfpathlineto{\pgfqpoint{1.863445in}{1.384516in}}%
\pgfpathlineto{\pgfqpoint{1.854380in}{1.393425in}}%
\pgfpathlineto{\pgfqpoint{1.847789in}{1.402323in}}%
\pgfpathlineto{\pgfqpoint{1.844490in}{1.407036in}}%
\pgfpathlineto{\pgfqpoint{1.840261in}{1.420648in}}%
\pgfpathlineto{\pgfqpoint{1.842106in}{1.434259in}}%
\pgfpathlineto{\pgfqpoint{1.847789in}{1.444550in}}%
\pgfpathlineto{\pgfqpoint{1.849747in}{1.447870in}}%
\pgfpathlineto{\pgfqpoint{1.862277in}{1.461481in}}%
\pgfpathlineto{\pgfqpoint{1.863445in}{1.462518in}}%
\pgfpathlineto{\pgfqpoint{1.879102in}{1.474300in}}%
\pgfpathlineto{\pgfqpoint{1.880640in}{1.475092in}}%
\pgfpathlineto{\pgfqpoint{1.894758in}{1.481774in}}%
\pgfpathlineto{\pgfqpoint{1.910415in}{1.484459in}}%
\pgfpathlineto{\pgfqpoint{1.910415in}{1.488703in}}%
\pgfpathlineto{\pgfqpoint{1.910415in}{1.502314in}}%
\pgfpathlineto{\pgfqpoint{1.910415in}{1.512187in}}%
\pgfpathlineto{\pgfqpoint{1.894758in}{1.508896in}}%
\pgfpathlineto{\pgfqpoint{1.883494in}{1.502314in}}%
\pgfpathlineto{\pgfqpoint{1.879102in}{1.500054in}}%
\pgfpathlineto{\pgfqpoint{1.863931in}{1.488703in}}%
\pgfpathlineto{\pgfqpoint{1.863445in}{1.488354in}}%
\pgfpathlineto{\pgfqpoint{1.847789in}{1.475194in}}%
\pgfpathlineto{\pgfqpoint{1.847671in}{1.475092in}}%
\pgfpathlineto{\pgfqpoint{1.832810in}{1.461481in}}%
\pgfpathlineto{\pgfqpoint{1.832132in}{1.460729in}}%
\pgfpathlineto{\pgfqpoint{1.819828in}{1.447870in}}%
\pgfpathlineto{\pgfqpoint{1.816476in}{1.442416in}}%
\pgfpathlineto{\pgfqpoint{1.810682in}{1.434259in}}%
\pgfpathlineto{\pgfqpoint{1.808322in}{1.420648in}}%
\pgfpathlineto{\pgfqpoint{1.813732in}{1.407036in}}%
\pgfpathlineto{\pgfqpoint{1.816476in}{1.403988in}}%
\pgfpathlineto{\pgfqpoint{1.824629in}{1.393425in}}%
\pgfpathlineto{\pgfqpoint{1.832132in}{1.386368in}}%
\pgfpathlineto{\pgfqpoint{1.838659in}{1.379814in}}%
\pgfpathlineto{\pgfqpoint{1.847789in}{1.371754in}}%
\pgfpathlineto{\pgfqpoint{1.854269in}{1.366203in}}%
\pgfpathlineto{\pgfqpoint{1.863445in}{1.358417in}}%
\pgfpathlineto{\pgfqpoint{1.871492in}{1.352592in}}%
\pgfpathlineto{\pgfqpoint{1.879102in}{1.346606in}}%
\pgfpathlineto{\pgfqpoint{1.893243in}{1.338981in}}%
\pgfpathlineto{\pgfqpoint{1.894758in}{1.338009in}}%
\pgfpathclose%
\pgfpathmoveto{\pgfqpoint{0.376072in}{1.607372in}}%
\pgfpathlineto{\pgfqpoint{0.382348in}{1.611203in}}%
\pgfpathlineto{\pgfqpoint{0.391728in}{1.616135in}}%
\pgfpathlineto{\pgfqpoint{0.403064in}{1.624814in}}%
\pgfpathlineto{\pgfqpoint{0.407385in}{1.627922in}}%
\pgfpathlineto{\pgfqpoint{0.419852in}{1.638425in}}%
\pgfpathlineto{\pgfqpoint{0.423041in}{1.641198in}}%
\pgfpathlineto{\pgfqpoint{0.435123in}{1.652036in}}%
\pgfpathlineto{\pgfqpoint{0.438698in}{1.655793in}}%
\pgfpathlineto{\pgfqpoint{0.448681in}{1.665647in}}%
\pgfpathlineto{\pgfqpoint{0.454354in}{1.673802in}}%
\pgfpathlineto{\pgfqpoint{0.458762in}{1.679259in}}%
\pgfpathlineto{\pgfqpoint{0.462667in}{1.692870in}}%
\pgfpathlineto{\pgfqpoint{0.454354in}{1.692870in}}%
\pgfpathlineto{\pgfqpoint{0.438698in}{1.692870in}}%
\pgfpathlineto{\pgfqpoint{0.430694in}{1.692870in}}%
\pgfpathlineto{\pgfqpoint{0.427641in}{1.679259in}}%
\pgfpathlineto{\pgfqpoint{0.423041in}{1.671935in}}%
\pgfpathlineto{\pgfqpoint{0.418843in}{1.665648in}}%
\pgfpathlineto{\pgfqpoint{0.407385in}{1.653828in}}%
\pgfpathlineto{\pgfqpoint{0.405324in}{1.652036in}}%
\pgfpathlineto{\pgfqpoint{0.391728in}{1.642075in}}%
\pgfpathlineto{\pgfqpoint{0.384496in}{1.638425in}}%
\pgfpathlineto{\pgfqpoint{0.376072in}{1.634427in}}%
\pgfpathlineto{\pgfqpoint{0.360415in}{1.631772in}}%
\pgfpathlineto{\pgfqpoint{0.360415in}{1.624814in}}%
\pgfpathlineto{\pgfqpoint{0.360415in}{1.611203in}}%
\pgfpathlineto{\pgfqpoint{0.360415in}{1.603976in}}%
\pgfpathlineto{\pgfqpoint{0.376072in}{1.607372in}}%
\pgfpathclose%
\pgfpathmoveto{\pgfqpoint{0.657890in}{1.606167in}}%
\pgfpathlineto{\pgfqpoint{0.673546in}{1.604115in}}%
\pgfpathlineto{\pgfqpoint{0.689203in}{1.608818in}}%
\pgfpathlineto{\pgfqpoint{0.692709in}{1.611203in}}%
\pgfpathlineto{\pgfqpoint{0.704859in}{1.618291in}}%
\pgfpathlineto{\pgfqpoint{0.712977in}{1.624814in}}%
\pgfpathlineto{\pgfqpoint{0.720516in}{1.630489in}}%
\pgfpathlineto{\pgfqpoint{0.729787in}{1.638425in}}%
\pgfpathlineto{\pgfqpoint{0.736173in}{1.644059in}}%
\pgfpathlineto{\pgfqpoint{0.745128in}{1.652036in}}%
\pgfpathlineto{\pgfqpoint{0.751829in}{1.659032in}}%
\pgfpathlineto{\pgfqpoint{0.758714in}{1.665648in}}%
\pgfpathlineto{\pgfqpoint{0.767486in}{1.677941in}}%
\pgfpathlineto{\pgfqpoint{0.768603in}{1.679259in}}%
\pgfpathlineto{\pgfqpoint{0.772653in}{1.692870in}}%
\pgfpathlineto{\pgfqpoint{0.767486in}{1.692870in}}%
\pgfpathlineto{\pgfqpoint{0.751829in}{1.692870in}}%
\pgfpathlineto{\pgfqpoint{0.740732in}{1.692870in}}%
\pgfpathlineto{\pgfqpoint{0.737701in}{1.679259in}}%
\pgfpathlineto{\pgfqpoint{0.736173in}{1.676842in}}%
\pgfpathlineto{\pgfqpoint{0.728808in}{1.665647in}}%
\pgfpathlineto{\pgfqpoint{0.720516in}{1.656957in}}%
\pgfpathlineto{\pgfqpoint{0.715108in}{1.652036in}}%
\pgfpathlineto{\pgfqpoint{0.704859in}{1.644156in}}%
\pgfpathlineto{\pgfqpoint{0.694624in}{1.638425in}}%
\pgfpathlineto{\pgfqpoint{0.689203in}{1.635558in}}%
\pgfpathlineto{\pgfqpoint{0.673546in}{1.631881in}}%
\pgfpathlineto{\pgfqpoint{0.657890in}{1.633485in}}%
\pgfpathlineto{\pgfqpoint{0.646052in}{1.638425in}}%
\pgfpathlineto{\pgfqpoint{0.642233in}{1.640128in}}%
\pgfpathlineto{\pgfqpoint{0.626577in}{1.651021in}}%
\pgfpathlineto{\pgfqpoint{0.625383in}{1.652036in}}%
\pgfpathlineto{\pgfqpoint{0.611831in}{1.665648in}}%
\pgfpathlineto{\pgfqpoint{0.610920in}{1.666985in}}%
\pgfpathlineto{\pgfqpoint{0.603234in}{1.679259in}}%
\pgfpathlineto{\pgfqpoint{0.600145in}{1.692870in}}%
\pgfpathlineto{\pgfqpoint{0.595263in}{1.692870in}}%
\pgfpathlineto{\pgfqpoint{0.579607in}{1.692870in}}%
\pgfpathlineto{\pgfqpoint{0.568251in}{1.692870in}}%
\pgfpathlineto{\pgfqpoint{0.572036in}{1.679259in}}%
\pgfpathlineto{\pgfqpoint{0.579607in}{1.669465in}}%
\pgfpathlineto{\pgfqpoint{0.582207in}{1.665648in}}%
\pgfpathlineto{\pgfqpoint{0.595263in}{1.652459in}}%
\pgfpathlineto{\pgfqpoint{0.595665in}{1.652036in}}%
\pgfpathlineto{\pgfqpoint{0.610803in}{1.638425in}}%
\pgfpathlineto{\pgfqpoint{0.610920in}{1.638323in}}%
\pgfpathlineto{\pgfqpoint{0.626577in}{1.625403in}}%
\pgfpathlineto{\pgfqpoint{0.627441in}{1.624814in}}%
\pgfpathlineto{\pgfqpoint{0.642233in}{1.614117in}}%
\pgfpathlineto{\pgfqpoint{0.648506in}{1.611203in}}%
\pgfpathlineto{\pgfqpoint{0.657890in}{1.606167in}}%
\pgfpathclose%
\pgfpathmoveto{\pgfqpoint{0.971021in}{1.605216in}}%
\pgfpathlineto{\pgfqpoint{0.986678in}{1.604530in}}%
\pgfpathlineto{\pgfqpoint{1.002334in}{1.610492in}}%
\pgfpathlineto{\pgfqpoint{1.003288in}{1.611203in}}%
\pgfpathlineto{\pgfqpoint{1.017991in}{1.620567in}}%
\pgfpathlineto{\pgfqpoint{1.023066in}{1.624814in}}%
\pgfpathlineto{\pgfqpoint{1.033647in}{1.633088in}}%
\pgfpathlineto{\pgfqpoint{1.039815in}{1.638425in}}%
\pgfpathlineto{\pgfqpoint{1.049304in}{1.646882in}}%
\pgfpathlineto{\pgfqpoint{1.055155in}{1.652036in}}%
\pgfpathlineto{\pgfqpoint{1.064960in}{1.662167in}}%
\pgfpathlineto{\pgfqpoint{1.068696in}{1.665648in}}%
\pgfpathlineto{\pgfqpoint{1.078616in}{1.679259in}}%
\pgfpathlineto{\pgfqpoint{1.080617in}{1.686855in}}%
\pgfpathlineto{\pgfqpoint{1.082498in}{1.692870in}}%
\pgfpathlineto{\pgfqpoint{1.080617in}{1.692870in}}%
\pgfpathlineto{\pgfqpoint{1.064960in}{1.692870in}}%
\pgfpathlineto{\pgfqpoint{1.050774in}{1.692870in}}%
\pgfpathlineto{\pgfqpoint{1.049304in}{1.686296in}}%
\pgfpathlineto{\pgfqpoint{1.047660in}{1.679259in}}%
\pgfpathlineto{\pgfqpoint{1.038860in}{1.665647in}}%
\pgfpathlineto{\pgfqpoint{1.033647in}{1.660125in}}%
\pgfpathlineto{\pgfqpoint{1.025089in}{1.652036in}}%
\pgfpathlineto{\pgfqpoint{1.017991in}{1.646352in}}%
\pgfpathlineto{\pgfqpoint{1.005022in}{1.638425in}}%
\pgfpathlineto{\pgfqpoint{1.002334in}{1.636866in}}%
\pgfpathlineto{\pgfqpoint{0.986678in}{1.632205in}}%
\pgfpathlineto{\pgfqpoint{0.971021in}{1.632742in}}%
\pgfpathlineto{\pgfqpoint{0.955364in}{1.638340in}}%
\pgfpathlineto{\pgfqpoint{0.955228in}{1.638425in}}%
\pgfpathlineto{\pgfqpoint{0.939708in}{1.648646in}}%
\pgfpathlineto{\pgfqpoint{0.935614in}{1.652036in}}%
\pgfpathlineto{\pgfqpoint{0.924051in}{1.663314in}}%
\pgfpathlineto{\pgfqpoint{0.921862in}{1.665648in}}%
\pgfpathlineto{\pgfqpoint{0.913237in}{1.679259in}}%
\pgfpathlineto{\pgfqpoint{0.910099in}{1.692870in}}%
\pgfpathlineto{\pgfqpoint{0.908395in}{1.692870in}}%
\pgfpathlineto{\pgfqpoint{0.892738in}{1.692870in}}%
\pgfpathlineto{\pgfqpoint{0.878408in}{1.692870in}}%
\pgfpathlineto{\pgfqpoint{0.882095in}{1.679259in}}%
\pgfpathlineto{\pgfqpoint{0.892232in}{1.665647in}}%
\pgfpathlineto{\pgfqpoint{0.892738in}{1.665192in}}%
\pgfpathlineto{\pgfqpoint{0.905652in}{1.652036in}}%
\pgfpathlineto{\pgfqpoint{0.908395in}{1.649656in}}%
\pgfpathlineto{\pgfqpoint{0.920926in}{1.638425in}}%
\pgfpathlineto{\pgfqpoint{0.924051in}{1.635704in}}%
\pgfpathlineto{\pgfqpoint{0.937548in}{1.624814in}}%
\pgfpathlineto{\pgfqpoint{0.939708in}{1.622944in}}%
\pgfpathlineto{\pgfqpoint{0.955364in}{1.612255in}}%
\pgfpathlineto{\pgfqpoint{0.957995in}{1.611203in}}%
\pgfpathlineto{\pgfqpoint{0.971021in}{1.605216in}}%
\pgfpathclose%
\pgfpathmoveto{\pgfqpoint{1.268496in}{1.610492in}}%
\pgfpathlineto{\pgfqpoint{1.284152in}{1.604530in}}%
\pgfpathlineto{\pgfqpoint{1.299809in}{1.605216in}}%
\pgfpathlineto{\pgfqpoint{1.312835in}{1.611203in}}%
\pgfpathlineto{\pgfqpoint{1.315466in}{1.612255in}}%
\pgfpathlineto{\pgfqpoint{1.331122in}{1.622944in}}%
\pgfpathlineto{\pgfqpoint{1.333282in}{1.624814in}}%
\pgfpathlineto{\pgfqpoint{1.346779in}{1.635704in}}%
\pgfpathlineto{\pgfqpoint{1.349904in}{1.638425in}}%
\pgfpathlineto{\pgfqpoint{1.362435in}{1.649656in}}%
\pgfpathlineto{\pgfqpoint{1.365178in}{1.652036in}}%
\pgfpathlineto{\pgfqpoint{1.378092in}{1.665192in}}%
\pgfpathlineto{\pgfqpoint{1.378598in}{1.665648in}}%
\pgfpathlineto{\pgfqpoint{1.388735in}{1.679259in}}%
\pgfpathlineto{\pgfqpoint{1.392422in}{1.692870in}}%
\pgfpathlineto{\pgfqpoint{1.378092in}{1.692870in}}%
\pgfpathlineto{\pgfqpoint{1.362435in}{1.692870in}}%
\pgfpathlineto{\pgfqpoint{1.360731in}{1.692870in}}%
\pgfpathlineto{\pgfqpoint{1.357593in}{1.679259in}}%
\pgfpathlineto{\pgfqpoint{1.348968in}{1.665647in}}%
\pgfpathlineto{\pgfqpoint{1.346779in}{1.663314in}}%
\pgfpathlineto{\pgfqpoint{1.335216in}{1.652036in}}%
\pgfpathlineto{\pgfqpoint{1.331122in}{1.648646in}}%
\pgfpathlineto{\pgfqpoint{1.315602in}{1.638425in}}%
\pgfpathlineto{\pgfqpoint{1.315466in}{1.638340in}}%
\pgfpathlineto{\pgfqpoint{1.299809in}{1.632742in}}%
\pgfpathlineto{\pgfqpoint{1.284152in}{1.632205in}}%
\pgfpathlineto{\pgfqpoint{1.268496in}{1.636866in}}%
\pgfpathlineto{\pgfqpoint{1.265808in}{1.638425in}}%
\pgfpathlineto{\pgfqpoint{1.252839in}{1.646352in}}%
\pgfpathlineto{\pgfqpoint{1.245741in}{1.652036in}}%
\pgfpathlineto{\pgfqpoint{1.237183in}{1.660125in}}%
\pgfpathlineto{\pgfqpoint{1.231970in}{1.665648in}}%
\pgfpathlineto{\pgfqpoint{1.223170in}{1.679259in}}%
\pgfpathlineto{\pgfqpoint{1.221526in}{1.686296in}}%
\pgfpathlineto{\pgfqpoint{1.220056in}{1.692870in}}%
\pgfpathlineto{\pgfqpoint{1.205870in}{1.692870in}}%
\pgfpathlineto{\pgfqpoint{1.190213in}{1.692870in}}%
\pgfpathlineto{\pgfqpoint{1.188332in}{1.692870in}}%
\pgfpathlineto{\pgfqpoint{1.190213in}{1.686855in}}%
\pgfpathlineto{\pgfqpoint{1.192214in}{1.679259in}}%
\pgfpathlineto{\pgfqpoint{1.202134in}{1.665647in}}%
\pgfpathlineto{\pgfqpoint{1.205870in}{1.662167in}}%
\pgfpathlineto{\pgfqpoint{1.215675in}{1.652036in}}%
\pgfpathlineto{\pgfqpoint{1.221526in}{1.646882in}}%
\pgfpathlineto{\pgfqpoint{1.231015in}{1.638425in}}%
\pgfpathlineto{\pgfqpoint{1.237183in}{1.633088in}}%
\pgfpathlineto{\pgfqpoint{1.247764in}{1.624814in}}%
\pgfpathlineto{\pgfqpoint{1.252839in}{1.620567in}}%
\pgfpathlineto{\pgfqpoint{1.267542in}{1.611203in}}%
\pgfpathlineto{\pgfqpoint{1.268496in}{1.610492in}}%
\pgfpathclose%
\pgfpathmoveto{\pgfqpoint{1.581627in}{1.608818in}}%
\pgfpathlineto{\pgfqpoint{1.597284in}{1.604115in}}%
\pgfpathlineto{\pgfqpoint{1.612940in}{1.606167in}}%
\pgfpathlineto{\pgfqpoint{1.622324in}{1.611203in}}%
\pgfpathlineto{\pgfqpoint{1.628597in}{1.614117in}}%
\pgfpathlineto{\pgfqpoint{1.643389in}{1.624814in}}%
\pgfpathlineto{\pgfqpoint{1.644253in}{1.625403in}}%
\pgfpathlineto{\pgfqpoint{1.659910in}{1.638323in}}%
\pgfpathlineto{\pgfqpoint{1.660027in}{1.638425in}}%
\pgfpathlineto{\pgfqpoint{1.675165in}{1.652036in}}%
\pgfpathlineto{\pgfqpoint{1.675567in}{1.652459in}}%
\pgfpathlineto{\pgfqpoint{1.688623in}{1.665648in}}%
\pgfpathlineto{\pgfqpoint{1.691223in}{1.669465in}}%
\pgfpathlineto{\pgfqpoint{1.698794in}{1.679259in}}%
\pgfpathlineto{\pgfqpoint{1.702579in}{1.692870in}}%
\pgfpathlineto{\pgfqpoint{1.691223in}{1.692870in}}%
\pgfpathlineto{\pgfqpoint{1.675567in}{1.692870in}}%
\pgfpathlineto{\pgfqpoint{1.670685in}{1.692870in}}%
\pgfpathlineto{\pgfqpoint{1.667596in}{1.679259in}}%
\pgfpathlineto{\pgfqpoint{1.659910in}{1.666985in}}%
\pgfpathlineto{\pgfqpoint{1.658999in}{1.665647in}}%
\pgfpathlineto{\pgfqpoint{1.645447in}{1.652036in}}%
\pgfpathlineto{\pgfqpoint{1.644253in}{1.651021in}}%
\pgfpathlineto{\pgfqpoint{1.628597in}{1.640128in}}%
\pgfpathlineto{\pgfqpoint{1.624778in}{1.638425in}}%
\pgfpathlineto{\pgfqpoint{1.612940in}{1.633485in}}%
\pgfpathlineto{\pgfqpoint{1.597284in}{1.631881in}}%
\pgfpathlineto{\pgfqpoint{1.581627in}{1.635558in}}%
\pgfpathlineto{\pgfqpoint{1.576206in}{1.638425in}}%
\pgfpathlineto{\pgfqpoint{1.565971in}{1.644156in}}%
\pgfpathlineto{\pgfqpoint{1.555722in}{1.652036in}}%
\pgfpathlineto{\pgfqpoint{1.550314in}{1.656957in}}%
\pgfpathlineto{\pgfqpoint{1.542022in}{1.665648in}}%
\pgfpathlineto{\pgfqpoint{1.534657in}{1.676842in}}%
\pgfpathlineto{\pgfqpoint{1.533129in}{1.679259in}}%
\pgfpathlineto{\pgfqpoint{1.530098in}{1.692870in}}%
\pgfpathlineto{\pgfqpoint{1.519001in}{1.692870in}}%
\pgfpathlineto{\pgfqpoint{1.503344in}{1.692870in}}%
\pgfpathlineto{\pgfqpoint{1.498177in}{1.692870in}}%
\pgfpathlineto{\pgfqpoint{1.502227in}{1.679259in}}%
\pgfpathlineto{\pgfqpoint{1.503344in}{1.677941in}}%
\pgfpathlineto{\pgfqpoint{1.512116in}{1.665648in}}%
\pgfpathlineto{\pgfqpoint{1.519001in}{1.659032in}}%
\pgfpathlineto{\pgfqpoint{1.525702in}{1.652036in}}%
\pgfpathlineto{\pgfqpoint{1.534657in}{1.644059in}}%
\pgfpathlineto{\pgfqpoint{1.541043in}{1.638425in}}%
\pgfpathlineto{\pgfqpoint{1.550314in}{1.630489in}}%
\pgfpathlineto{\pgfqpoint{1.557853in}{1.624814in}}%
\pgfpathlineto{\pgfqpoint{1.565971in}{1.618291in}}%
\pgfpathlineto{\pgfqpoint{1.578121in}{1.611203in}}%
\pgfpathlineto{\pgfqpoint{1.581627in}{1.608818in}}%
\pgfpathclose%
\pgfpathmoveto{\pgfqpoint{1.894758in}{1.607372in}}%
\pgfpathlineto{\pgfqpoint{1.910415in}{1.603976in}}%
\pgfpathlineto{\pgfqpoint{1.910415in}{1.611203in}}%
\pgfpathlineto{\pgfqpoint{1.910415in}{1.624814in}}%
\pgfpathlineto{\pgfqpoint{1.910415in}{1.631772in}}%
\pgfpathlineto{\pgfqpoint{1.894758in}{1.634427in}}%
\pgfpathlineto{\pgfqpoint{1.886334in}{1.638425in}}%
\pgfpathlineto{\pgfqpoint{1.879102in}{1.642075in}}%
\pgfpathlineto{\pgfqpoint{1.865506in}{1.652036in}}%
\pgfpathlineto{\pgfqpoint{1.863445in}{1.653828in}}%
\pgfpathlineto{\pgfqpoint{1.851987in}{1.665647in}}%
\pgfpathlineto{\pgfqpoint{1.847789in}{1.671935in}}%
\pgfpathlineto{\pgfqpoint{1.843189in}{1.679259in}}%
\pgfpathlineto{\pgfqpoint{1.840136in}{1.692870in}}%
\pgfpathlineto{\pgfqpoint{1.832132in}{1.692870in}}%
\pgfpathlineto{\pgfqpoint{1.816476in}{1.692870in}}%
\pgfpathlineto{\pgfqpoint{1.808163in}{1.692870in}}%
\pgfpathlineto{\pgfqpoint{1.812068in}{1.679259in}}%
\pgfpathlineto{\pgfqpoint{1.816476in}{1.673802in}}%
\pgfpathlineto{\pgfqpoint{1.822149in}{1.665648in}}%
\pgfpathlineto{\pgfqpoint{1.832132in}{1.655793in}}%
\pgfpathlineto{\pgfqpoint{1.835707in}{1.652036in}}%
\pgfpathlineto{\pgfqpoint{1.847789in}{1.641198in}}%
\pgfpathlineto{\pgfqpoint{1.850978in}{1.638425in}}%
\pgfpathlineto{\pgfqpoint{1.863445in}{1.627922in}}%
\pgfpathlineto{\pgfqpoint{1.867766in}{1.624814in}}%
\pgfpathlineto{\pgfqpoint{1.879102in}{1.616135in}}%
\pgfpathlineto{\pgfqpoint{1.888482in}{1.611203in}}%
\pgfpathlineto{\pgfqpoint{1.894758in}{1.607372in}}%
\pgfpathclose%
\pgfusepath{fill}%
\end{pgfscope}%
\begin{pgfscope}%
\pgfpathrectangle{\pgfqpoint{0.360415in}{0.345370in}}{\pgfqpoint{1.550000in}{1.347500in}}%
\pgfusepath{clip}%
\pgfsetbuttcap%
\pgfsetroundjoin%
\definecolor{currentfill}{rgb}{0.362553,0.003243,0.649245}%
\pgfsetfillcolor{currentfill}%
\pgfsetlinewidth{0.000000pt}%
\definecolor{currentstroke}{rgb}{0.000000,0.000000,0.000000}%
\pgfsetstrokecolor{currentstroke}%
\pgfsetdash{}{0pt}%
\pgfpathmoveto{\pgfqpoint{0.391728in}{0.355760in}}%
\pgfpathlineto{\pgfqpoint{0.394521in}{0.345370in}}%
\pgfpathlineto{\pgfqpoint{0.407385in}{0.345370in}}%
\pgfpathlineto{\pgfqpoint{0.423041in}{0.345370in}}%
\pgfpathlineto{\pgfqpoint{0.430694in}{0.345370in}}%
\pgfpathlineto{\pgfqpoint{0.427641in}{0.358981in}}%
\pgfpathlineto{\pgfqpoint{0.423041in}{0.366304in}}%
\pgfpathlineto{\pgfqpoint{0.418843in}{0.372592in}}%
\pgfpathlineto{\pgfqpoint{0.407385in}{0.384412in}}%
\pgfpathlineto{\pgfqpoint{0.405324in}{0.386203in}}%
\pgfpathlineto{\pgfqpoint{0.391728in}{0.396164in}}%
\pgfpathlineto{\pgfqpoint{0.384496in}{0.399814in}}%
\pgfpathlineto{\pgfqpoint{0.376072in}{0.403813in}}%
\pgfpathlineto{\pgfqpoint{0.360415in}{0.406467in}}%
\pgfpathlineto{\pgfqpoint{0.360415in}{0.399814in}}%
\pgfpathlineto{\pgfqpoint{0.360415in}{0.386203in}}%
\pgfpathlineto{\pgfqpoint{0.360415in}{0.375020in}}%
\pgfpathlineto{\pgfqpoint{0.372366in}{0.372592in}}%
\pgfpathlineto{\pgfqpoint{0.376072in}{0.371437in}}%
\pgfpathlineto{\pgfqpoint{0.390399in}{0.358981in}}%
\pgfpathlineto{\pgfqpoint{0.391728in}{0.355760in}}%
\pgfpathclose%
\pgfpathmoveto{\pgfqpoint{0.610920in}{0.345370in}}%
\pgfpathlineto{\pgfqpoint{0.626577in}{0.345370in}}%
\pgfpathlineto{\pgfqpoint{0.636575in}{0.345370in}}%
\pgfpathlineto{\pgfqpoint{0.640473in}{0.358981in}}%
\pgfpathlineto{\pgfqpoint{0.642233in}{0.361143in}}%
\pgfpathlineto{\pgfqpoint{0.657141in}{0.372592in}}%
\pgfpathlineto{\pgfqpoint{0.657890in}{0.372975in}}%
\pgfpathlineto{\pgfqpoint{0.673546in}{0.374890in}}%
\pgfpathlineto{\pgfqpoint{0.681664in}{0.372592in}}%
\pgfpathlineto{\pgfqpoint{0.689203in}{0.369334in}}%
\pgfpathlineto{\pgfqpoint{0.699889in}{0.358981in}}%
\pgfpathlineto{\pgfqpoint{0.704859in}{0.345616in}}%
\pgfpathlineto{\pgfqpoint{0.704923in}{0.345370in}}%
\pgfpathlineto{\pgfqpoint{0.720516in}{0.345370in}}%
\pgfpathlineto{\pgfqpoint{0.736173in}{0.345370in}}%
\pgfpathlineto{\pgfqpoint{0.740732in}{0.345370in}}%
\pgfpathlineto{\pgfqpoint{0.737701in}{0.358981in}}%
\pgfpathlineto{\pgfqpoint{0.736173in}{0.361398in}}%
\pgfpathlineto{\pgfqpoint{0.728808in}{0.372592in}}%
\pgfpathlineto{\pgfqpoint{0.720516in}{0.381283in}}%
\pgfpathlineto{\pgfqpoint{0.715108in}{0.386203in}}%
\pgfpathlineto{\pgfqpoint{0.704859in}{0.394084in}}%
\pgfpathlineto{\pgfqpoint{0.694624in}{0.399814in}}%
\pgfpathlineto{\pgfqpoint{0.689203in}{0.402682in}}%
\pgfpathlineto{\pgfqpoint{0.673546in}{0.406359in}}%
\pgfpathlineto{\pgfqpoint{0.657890in}{0.404755in}}%
\pgfpathlineto{\pgfqpoint{0.646052in}{0.399814in}}%
\pgfpathlineto{\pgfqpoint{0.642233in}{0.398112in}}%
\pgfpathlineto{\pgfqpoint{0.626577in}{0.387219in}}%
\pgfpathlineto{\pgfqpoint{0.625383in}{0.386203in}}%
\pgfpathlineto{\pgfqpoint{0.611831in}{0.372592in}}%
\pgfpathlineto{\pgfqpoint{0.610920in}{0.371254in}}%
\pgfpathlineto{\pgfqpoint{0.603234in}{0.358981in}}%
\pgfpathlineto{\pgfqpoint{0.600145in}{0.345370in}}%
\pgfpathlineto{\pgfqpoint{0.610920in}{0.345370in}}%
\pgfpathclose%
\pgfpathmoveto{\pgfqpoint{0.924051in}{0.345370in}}%
\pgfpathlineto{\pgfqpoint{0.939708in}{0.345370in}}%
\pgfpathlineto{\pgfqpoint{0.946649in}{0.345370in}}%
\pgfpathlineto{\pgfqpoint{0.950854in}{0.358981in}}%
\pgfpathlineto{\pgfqpoint{0.955364in}{0.364162in}}%
\pgfpathlineto{\pgfqpoint{0.968118in}{0.372592in}}%
\pgfpathlineto{\pgfqpoint{0.971021in}{0.373862in}}%
\pgfpathlineto{\pgfqpoint{0.986678in}{0.374503in}}%
\pgfpathlineto{\pgfqpoint{0.991965in}{0.372592in}}%
\pgfpathlineto{\pgfqpoint{1.002334in}{0.366901in}}%
\pgfpathlineto{\pgfqpoint{1.009791in}{0.358981in}}%
\pgfpathlineto{\pgfqpoint{1.014377in}{0.345370in}}%
\pgfpathlineto{\pgfqpoint{1.017991in}{0.345370in}}%
\pgfpathlineto{\pgfqpoint{1.033647in}{0.345370in}}%
\pgfpathlineto{\pgfqpoint{1.049304in}{0.345370in}}%
\pgfpathlineto{\pgfqpoint{1.050774in}{0.345370in}}%
\pgfpathlineto{\pgfqpoint{1.049304in}{0.351944in}}%
\pgfpathlineto{\pgfqpoint{1.047660in}{0.358981in}}%
\pgfpathlineto{\pgfqpoint{1.038860in}{0.372592in}}%
\pgfpathlineto{\pgfqpoint{1.033647in}{0.378114in}}%
\pgfpathlineto{\pgfqpoint{1.025089in}{0.386203in}}%
\pgfpathlineto{\pgfqpoint{1.017991in}{0.391888in}}%
\pgfpathlineto{\pgfqpoint{1.005022in}{0.399814in}}%
\pgfpathlineto{\pgfqpoint{1.002334in}{0.401373in}}%
\pgfpathlineto{\pgfqpoint{0.986678in}{0.406034in}}%
\pgfpathlineto{\pgfqpoint{0.971021in}{0.405498in}}%
\pgfpathlineto{\pgfqpoint{0.955364in}{0.399900in}}%
\pgfpathlineto{\pgfqpoint{0.955228in}{0.399814in}}%
\pgfpathlineto{\pgfqpoint{0.939708in}{0.389593in}}%
\pgfpathlineto{\pgfqpoint{0.935614in}{0.386203in}}%
\pgfpathlineto{\pgfqpoint{0.924051in}{0.374925in}}%
\pgfpathlineto{\pgfqpoint{0.921862in}{0.372592in}}%
\pgfpathlineto{\pgfqpoint{0.913237in}{0.358981in}}%
\pgfpathlineto{\pgfqpoint{0.910099in}{0.345370in}}%
\pgfpathlineto{\pgfqpoint{0.924051in}{0.345370in}}%
\pgfpathclose%
\pgfpathmoveto{\pgfqpoint{1.221526in}{0.345370in}}%
\pgfpathlineto{\pgfqpoint{1.237183in}{0.345370in}}%
\pgfpathlineto{\pgfqpoint{1.252839in}{0.345370in}}%
\pgfpathlineto{\pgfqpoint{1.256453in}{0.345370in}}%
\pgfpathlineto{\pgfqpoint{1.261039in}{0.358981in}}%
\pgfpathlineto{\pgfqpoint{1.268496in}{0.366901in}}%
\pgfpathlineto{\pgfqpoint{1.278865in}{0.372592in}}%
\pgfpathlineto{\pgfqpoint{1.284152in}{0.374503in}}%
\pgfpathlineto{\pgfqpoint{1.299809in}{0.373862in}}%
\pgfpathlineto{\pgfqpoint{1.302712in}{0.372592in}}%
\pgfpathlineto{\pgfqpoint{1.315466in}{0.364162in}}%
\pgfpathlineto{\pgfqpoint{1.319976in}{0.358981in}}%
\pgfpathlineto{\pgfqpoint{1.324181in}{0.345370in}}%
\pgfpathlineto{\pgfqpoint{1.331122in}{0.345370in}}%
\pgfpathlineto{\pgfqpoint{1.346779in}{0.345370in}}%
\pgfpathlineto{\pgfqpoint{1.360731in}{0.345370in}}%
\pgfpathlineto{\pgfqpoint{1.357593in}{0.358981in}}%
\pgfpathlineto{\pgfqpoint{1.348968in}{0.372592in}}%
\pgfpathlineto{\pgfqpoint{1.346779in}{0.374925in}}%
\pgfpathlineto{\pgfqpoint{1.335216in}{0.386203in}}%
\pgfpathlineto{\pgfqpoint{1.331122in}{0.389593in}}%
\pgfpathlineto{\pgfqpoint{1.315602in}{0.399814in}}%
\pgfpathlineto{\pgfqpoint{1.315466in}{0.399900in}}%
\pgfpathlineto{\pgfqpoint{1.299809in}{0.405498in}}%
\pgfpathlineto{\pgfqpoint{1.284152in}{0.406034in}}%
\pgfpathlineto{\pgfqpoint{1.268496in}{0.401373in}}%
\pgfpathlineto{\pgfqpoint{1.265808in}{0.399814in}}%
\pgfpathlineto{\pgfqpoint{1.252839in}{0.391888in}}%
\pgfpathlineto{\pgfqpoint{1.245741in}{0.386203in}}%
\pgfpathlineto{\pgfqpoint{1.237183in}{0.378114in}}%
\pgfpathlineto{\pgfqpoint{1.231970in}{0.372592in}}%
\pgfpathlineto{\pgfqpoint{1.223170in}{0.358981in}}%
\pgfpathlineto{\pgfqpoint{1.221526in}{0.351944in}}%
\pgfpathlineto{\pgfqpoint{1.220056in}{0.345370in}}%
\pgfpathlineto{\pgfqpoint{1.221526in}{0.345370in}}%
\pgfpathclose%
\pgfpathmoveto{\pgfqpoint{1.534657in}{0.345370in}}%
\pgfpathlineto{\pgfqpoint{1.550314in}{0.345370in}}%
\pgfpathlineto{\pgfqpoint{1.565907in}{0.345370in}}%
\pgfpathlineto{\pgfqpoint{1.565971in}{0.345616in}}%
\pgfpathlineto{\pgfqpoint{1.570941in}{0.358981in}}%
\pgfpathlineto{\pgfqpoint{1.581627in}{0.369334in}}%
\pgfpathlineto{\pgfqpoint{1.589166in}{0.372592in}}%
\pgfpathlineto{\pgfqpoint{1.597284in}{0.374890in}}%
\pgfpathlineto{\pgfqpoint{1.612940in}{0.372975in}}%
\pgfpathlineto{\pgfqpoint{1.613689in}{0.372592in}}%
\pgfpathlineto{\pgfqpoint{1.628597in}{0.361143in}}%
\pgfpathlineto{\pgfqpoint{1.630357in}{0.358981in}}%
\pgfpathlineto{\pgfqpoint{1.634255in}{0.345370in}}%
\pgfpathlineto{\pgfqpoint{1.644253in}{0.345370in}}%
\pgfpathlineto{\pgfqpoint{1.659910in}{0.345370in}}%
\pgfpathlineto{\pgfqpoint{1.670685in}{0.345370in}}%
\pgfpathlineto{\pgfqpoint{1.667596in}{0.358981in}}%
\pgfpathlineto{\pgfqpoint{1.659910in}{0.371254in}}%
\pgfpathlineto{\pgfqpoint{1.658999in}{0.372592in}}%
\pgfpathlineto{\pgfqpoint{1.645447in}{0.386203in}}%
\pgfpathlineto{\pgfqpoint{1.644253in}{0.387219in}}%
\pgfpathlineto{\pgfqpoint{1.628597in}{0.398112in}}%
\pgfpathlineto{\pgfqpoint{1.624778in}{0.399814in}}%
\pgfpathlineto{\pgfqpoint{1.612940in}{0.404755in}}%
\pgfpathlineto{\pgfqpoint{1.597284in}{0.406359in}}%
\pgfpathlineto{\pgfqpoint{1.581627in}{0.402682in}}%
\pgfpathlineto{\pgfqpoint{1.576206in}{0.399814in}}%
\pgfpathlineto{\pgfqpoint{1.565971in}{0.394084in}}%
\pgfpathlineto{\pgfqpoint{1.555722in}{0.386203in}}%
\pgfpathlineto{\pgfqpoint{1.550314in}{0.381283in}}%
\pgfpathlineto{\pgfqpoint{1.542022in}{0.372592in}}%
\pgfpathlineto{\pgfqpoint{1.534657in}{0.361398in}}%
\pgfpathlineto{\pgfqpoint{1.533129in}{0.358981in}}%
\pgfpathlineto{\pgfqpoint{1.530098in}{0.345370in}}%
\pgfpathlineto{\pgfqpoint{1.534657in}{0.345370in}}%
\pgfpathclose%
\pgfpathmoveto{\pgfqpoint{1.847789in}{0.345370in}}%
\pgfpathlineto{\pgfqpoint{1.863445in}{0.345370in}}%
\pgfpathlineto{\pgfqpoint{1.876309in}{0.345370in}}%
\pgfpathlineto{\pgfqpoint{1.879102in}{0.355760in}}%
\pgfpathlineto{\pgfqpoint{1.880431in}{0.358981in}}%
\pgfpathlineto{\pgfqpoint{1.894758in}{0.371437in}}%
\pgfpathlineto{\pgfqpoint{1.898464in}{0.372592in}}%
\pgfpathlineto{\pgfqpoint{1.910415in}{0.375020in}}%
\pgfpathlineto{\pgfqpoint{1.910415in}{0.386203in}}%
\pgfpathlineto{\pgfqpoint{1.910415in}{0.399814in}}%
\pgfpathlineto{\pgfqpoint{1.910415in}{0.406467in}}%
\pgfpathlineto{\pgfqpoint{1.894758in}{0.403813in}}%
\pgfpathlineto{\pgfqpoint{1.886334in}{0.399814in}}%
\pgfpathlineto{\pgfqpoint{1.879102in}{0.396164in}}%
\pgfpathlineto{\pgfqpoint{1.865506in}{0.386203in}}%
\pgfpathlineto{\pgfqpoint{1.863445in}{0.384412in}}%
\pgfpathlineto{\pgfqpoint{1.851987in}{0.372592in}}%
\pgfpathlineto{\pgfqpoint{1.847789in}{0.366304in}}%
\pgfpathlineto{\pgfqpoint{1.843189in}{0.358981in}}%
\pgfpathlineto{\pgfqpoint{1.840136in}{0.345370in}}%
\pgfpathlineto{\pgfqpoint{1.847789in}{0.345370in}}%
\pgfpathclose%
\pgfpathmoveto{\pgfqpoint{0.376072in}{0.556466in}}%
\pgfpathlineto{\pgfqpoint{0.390190in}{0.563148in}}%
\pgfpathlineto{\pgfqpoint{0.391728in}{0.563940in}}%
\pgfpathlineto{\pgfqpoint{0.407385in}{0.575721in}}%
\pgfpathlineto{\pgfqpoint{0.408553in}{0.576759in}}%
\pgfpathlineto{\pgfqpoint{0.421083in}{0.590370in}}%
\pgfpathlineto{\pgfqpoint{0.423041in}{0.593689in}}%
\pgfpathlineto{\pgfqpoint{0.428724in}{0.603981in}}%
\pgfpathlineto{\pgfqpoint{0.430569in}{0.617592in}}%
\pgfpathlineto{\pgfqpoint{0.426340in}{0.631203in}}%
\pgfpathlineto{\pgfqpoint{0.423041in}{0.635916in}}%
\pgfpathlineto{\pgfqpoint{0.416450in}{0.644814in}}%
\pgfpathlineto{\pgfqpoint{0.407385in}{0.653724in}}%
\pgfpathlineto{\pgfqpoint{0.401725in}{0.658425in}}%
\pgfpathlineto{\pgfqpoint{0.391728in}{0.665634in}}%
\pgfpathlineto{\pgfqpoint{0.378852in}{0.672036in}}%
\pgfpathlineto{\pgfqpoint{0.376072in}{0.673365in}}%
\pgfpathlineto{\pgfqpoint{0.360415in}{0.676000in}}%
\pgfpathlineto{\pgfqpoint{0.360415in}{0.672036in}}%
\pgfpathlineto{\pgfqpoint{0.360415in}{0.658425in}}%
\pgfpathlineto{\pgfqpoint{0.360415in}{0.644869in}}%
\pgfpathlineto{\pgfqpoint{0.360699in}{0.644814in}}%
\pgfpathlineto{\pgfqpoint{0.376072in}{0.640493in}}%
\pgfpathlineto{\pgfqpoint{0.387980in}{0.631203in}}%
\pgfpathlineto{\pgfqpoint{0.391728in}{0.624649in}}%
\pgfpathlineto{\pgfqpoint{0.394372in}{0.617592in}}%
\pgfpathlineto{\pgfqpoint{0.392168in}{0.603981in}}%
\pgfpathlineto{\pgfqpoint{0.391728in}{0.603330in}}%
\pgfpathlineto{\pgfqpoint{0.378558in}{0.590370in}}%
\pgfpathlineto{\pgfqpoint{0.376072in}{0.588840in}}%
\pgfpathlineto{\pgfqpoint{0.360415in}{0.585451in}}%
\pgfpathlineto{\pgfqpoint{0.360415in}{0.576759in}}%
\pgfpathlineto{\pgfqpoint{0.360415in}{0.563148in}}%
\pgfpathlineto{\pgfqpoint{0.360415in}{0.553780in}}%
\pgfpathlineto{\pgfqpoint{0.376072in}{0.556466in}}%
\pgfpathclose%
\pgfpathmoveto{\pgfqpoint{0.642233in}{0.562067in}}%
\pgfpathlineto{\pgfqpoint{0.657890in}{0.555513in}}%
\pgfpathlineto{\pgfqpoint{0.673546in}{0.553890in}}%
\pgfpathlineto{\pgfqpoint{0.689203in}{0.557610in}}%
\pgfpathlineto{\pgfqpoint{0.699702in}{0.563148in}}%
\pgfpathlineto{\pgfqpoint{0.704859in}{0.566093in}}%
\pgfpathlineto{\pgfqpoint{0.718437in}{0.576759in}}%
\pgfpathlineto{\pgfqpoint{0.720516in}{0.578755in}}%
\pgfpathlineto{\pgfqpoint{0.730983in}{0.590370in}}%
\pgfpathlineto{\pgfqpoint{0.736173in}{0.599298in}}%
\pgfpathlineto{\pgfqpoint{0.738777in}{0.603981in}}%
\pgfpathlineto{\pgfqpoint{0.740608in}{0.617592in}}%
\pgfpathlineto{\pgfqpoint{0.736410in}{0.631203in}}%
\pgfpathlineto{\pgfqpoint{0.736173in}{0.631540in}}%
\pgfpathlineto{\pgfqpoint{0.726484in}{0.644814in}}%
\pgfpathlineto{\pgfqpoint{0.720516in}{0.650774in}}%
\pgfpathlineto{\pgfqpoint{0.711715in}{0.658425in}}%
\pgfpathlineto{\pgfqpoint{0.704859in}{0.663614in}}%
\pgfpathlineto{\pgfqpoint{0.689590in}{0.672036in}}%
\pgfpathlineto{\pgfqpoint{0.689203in}{0.672243in}}%
\pgfpathlineto{\pgfqpoint{0.673546in}{0.675893in}}%
\pgfpathlineto{\pgfqpoint{0.657890in}{0.674300in}}%
\pgfpathlineto{\pgfqpoint{0.652503in}{0.672036in}}%
\pgfpathlineto{\pgfqpoint{0.642233in}{0.667525in}}%
\pgfpathlineto{\pgfqpoint{0.628873in}{0.658425in}}%
\pgfpathlineto{\pgfqpoint{0.626577in}{0.656618in}}%
\pgfpathlineto{\pgfqpoint{0.614308in}{0.644814in}}%
\pgfpathlineto{\pgfqpoint{0.610920in}{0.640331in}}%
\pgfpathlineto{\pgfqpoint{0.604550in}{0.631203in}}%
\pgfpathlineto{\pgfqpoint{0.600271in}{0.617592in}}%
\pgfpathlineto{\pgfqpoint{0.602138in}{0.603981in}}%
\pgfpathlineto{\pgfqpoint{0.609677in}{0.590370in}}%
\pgfpathlineto{\pgfqpoint{0.610920in}{0.588965in}}%
\pgfpathlineto{\pgfqpoint{0.622481in}{0.576759in}}%
\pgfpathlineto{\pgfqpoint{0.626577in}{0.573198in}}%
\pgfpathlineto{\pgfqpoint{0.640617in}{0.563148in}}%
\pgfpathlineto{\pgfqpoint{0.642233in}{0.562067in}}%
\pgfpathclose%
\pgfpathmoveto{\pgfqpoint{0.652839in}{0.590370in}}%
\pgfpathlineto{\pgfqpoint{0.642233in}{0.599590in}}%
\pgfpathlineto{\pgfqpoint{0.639090in}{0.603981in}}%
\pgfpathlineto{\pgfqpoint{0.636734in}{0.617592in}}%
\pgfpathlineto{\pgfqpoint{0.642134in}{0.631203in}}%
\pgfpathlineto{\pgfqpoint{0.642233in}{0.631312in}}%
\pgfpathlineto{\pgfqpoint{0.657890in}{0.642055in}}%
\pgfpathlineto{\pgfqpoint{0.673546in}{0.644715in}}%
\pgfpathlineto{\pgfqpoint{0.689203in}{0.638618in}}%
\pgfpathlineto{\pgfqpoint{0.697732in}{0.631203in}}%
\pgfpathlineto{\pgfqpoint{0.704746in}{0.617592in}}%
\pgfpathlineto{\pgfqpoint{0.701686in}{0.603981in}}%
\pgfpathlineto{\pgfqpoint{0.689328in}{0.590370in}}%
\pgfpathlineto{\pgfqpoint{0.689203in}{0.590284in}}%
\pgfpathlineto{\pgfqpoint{0.673546in}{0.585589in}}%
\pgfpathlineto{\pgfqpoint{0.657890in}{0.587637in}}%
\pgfpathlineto{\pgfqpoint{0.652839in}{0.590370in}}%
\pgfpathclose%
\pgfpathmoveto{\pgfqpoint{0.955364in}{0.560424in}}%
\pgfpathlineto{\pgfqpoint{0.971021in}{0.554761in}}%
\pgfpathlineto{\pgfqpoint{0.986678in}{0.554218in}}%
\pgfpathlineto{\pgfqpoint{1.002334in}{0.558934in}}%
\pgfpathlineto{\pgfqpoint{1.009622in}{0.563148in}}%
\pgfpathlineto{\pgfqpoint{1.017991in}{0.568366in}}%
\pgfpathlineto{\pgfqpoint{1.028250in}{0.576759in}}%
\pgfpathlineto{\pgfqpoint{1.033647in}{0.582142in}}%
\pgfpathlineto{\pgfqpoint{1.040982in}{0.590370in}}%
\pgfpathlineto{\pgfqpoint{1.048796in}{0.603981in}}%
\pgfpathlineto{\pgfqpoint{1.049304in}{0.607576in}}%
\pgfpathlineto{\pgfqpoint{1.050650in}{0.617592in}}%
\pgfpathlineto{\pgfqpoint{1.049304in}{0.621913in}}%
\pgfpathlineto{\pgfqpoint{1.046296in}{0.631203in}}%
\pgfpathlineto{\pgfqpoint{1.036592in}{0.644814in}}%
\pgfpathlineto{\pgfqpoint{1.033647in}{0.647787in}}%
\pgfpathlineto{\pgfqpoint{1.021868in}{0.658425in}}%
\pgfpathlineto{\pgfqpoint{1.017991in}{0.661481in}}%
\pgfpathlineto{\pgfqpoint{1.002334in}{0.670854in}}%
\pgfpathlineto{\pgfqpoint{0.998566in}{0.672036in}}%
\pgfpathlineto{\pgfqpoint{0.986678in}{0.675571in}}%
\pgfpathlineto{\pgfqpoint{0.971021in}{0.675038in}}%
\pgfpathlineto{\pgfqpoint{0.962681in}{0.672036in}}%
\pgfpathlineto{\pgfqpoint{0.955364in}{0.669270in}}%
\pgfpathlineto{\pgfqpoint{0.939708in}{0.659253in}}%
\pgfpathlineto{\pgfqpoint{0.938693in}{0.658425in}}%
\pgfpathlineto{\pgfqpoint{0.924090in}{0.644814in}}%
\pgfpathlineto{\pgfqpoint{0.924051in}{0.644764in}}%
\pgfpathlineto{\pgfqpoint{0.914574in}{0.631203in}}%
\pgfpathlineto{\pgfqpoint{0.910227in}{0.617592in}}%
\pgfpathlineto{\pgfqpoint{0.912123in}{0.603981in}}%
\pgfpathlineto{\pgfqpoint{0.919782in}{0.590370in}}%
\pgfpathlineto{\pgfqpoint{0.924051in}{0.585552in}}%
\pgfpathlineto{\pgfqpoint{0.932594in}{0.576759in}}%
\pgfpathlineto{\pgfqpoint{0.939708in}{0.570740in}}%
\pgfpathlineto{\pgfqpoint{0.951010in}{0.563148in}}%
\pgfpathlineto{\pgfqpoint{0.955364in}{0.560424in}}%
\pgfpathclose%
\pgfpathmoveto{\pgfqpoint{0.963075in}{0.590370in}}%
\pgfpathlineto{\pgfqpoint{0.955364in}{0.596138in}}%
\pgfpathlineto{\pgfqpoint{0.949362in}{0.603981in}}%
\pgfpathlineto{\pgfqpoint{0.946821in}{0.617592in}}%
\pgfpathlineto{\pgfqpoint{0.952646in}{0.631203in}}%
\pgfpathlineto{\pgfqpoint{0.955364in}{0.634005in}}%
\pgfpathlineto{\pgfqpoint{0.971021in}{0.643288in}}%
\pgfpathlineto{\pgfqpoint{0.986678in}{0.644177in}}%
\pgfpathlineto{\pgfqpoint{1.002334in}{0.636448in}}%
\pgfpathlineto{\pgfqpoint{1.007837in}{0.631203in}}%
\pgfpathlineto{\pgfqpoint{1.014189in}{0.617592in}}%
\pgfpathlineto{\pgfqpoint{1.011418in}{0.603981in}}%
\pgfpathlineto{\pgfqpoint{1.002334in}{0.593007in}}%
\pgfpathlineto{\pgfqpoint{0.998088in}{0.590370in}}%
\pgfpathlineto{\pgfqpoint{0.986678in}{0.586003in}}%
\pgfpathlineto{\pgfqpoint{0.971021in}{0.586688in}}%
\pgfpathlineto{\pgfqpoint{0.963075in}{0.590370in}}%
\pgfpathclose%
\pgfpathmoveto{\pgfqpoint{1.268496in}{0.558934in}}%
\pgfpathlineto{\pgfqpoint{1.284152in}{0.554218in}}%
\pgfpathlineto{\pgfqpoint{1.299809in}{0.554761in}}%
\pgfpathlineto{\pgfqpoint{1.315466in}{0.560424in}}%
\pgfpathlineto{\pgfqpoint{1.319820in}{0.563148in}}%
\pgfpathlineto{\pgfqpoint{1.331122in}{0.570740in}}%
\pgfpathlineto{\pgfqpoint{1.338236in}{0.576759in}}%
\pgfpathlineto{\pgfqpoint{1.346779in}{0.585552in}}%
\pgfpathlineto{\pgfqpoint{1.351048in}{0.590370in}}%
\pgfpathlineto{\pgfqpoint{1.358707in}{0.603981in}}%
\pgfpathlineto{\pgfqpoint{1.360603in}{0.617592in}}%
\pgfpathlineto{\pgfqpoint{1.356256in}{0.631203in}}%
\pgfpathlineto{\pgfqpoint{1.346779in}{0.644764in}}%
\pgfpathlineto{\pgfqpoint{1.346740in}{0.644814in}}%
\pgfpathlineto{\pgfqpoint{1.332137in}{0.658425in}}%
\pgfpathlineto{\pgfqpoint{1.331122in}{0.659253in}}%
\pgfpathlineto{\pgfqpoint{1.315466in}{0.669270in}}%
\pgfpathlineto{\pgfqpoint{1.308149in}{0.672036in}}%
\pgfpathlineto{\pgfqpoint{1.299809in}{0.675038in}}%
\pgfpathlineto{\pgfqpoint{1.284152in}{0.675571in}}%
\pgfpathlineto{\pgfqpoint{1.272264in}{0.672036in}}%
\pgfpathlineto{\pgfqpoint{1.268496in}{0.670854in}}%
\pgfpathlineto{\pgfqpoint{1.252839in}{0.661481in}}%
\pgfpathlineto{\pgfqpoint{1.248962in}{0.658425in}}%
\pgfpathlineto{\pgfqpoint{1.237183in}{0.647787in}}%
\pgfpathlineto{\pgfqpoint{1.234238in}{0.644814in}}%
\pgfpathlineto{\pgfqpoint{1.224534in}{0.631203in}}%
\pgfpathlineto{\pgfqpoint{1.221526in}{0.621913in}}%
\pgfpathlineto{\pgfqpoint{1.220180in}{0.617592in}}%
\pgfpathlineto{\pgfqpoint{1.221526in}{0.607576in}}%
\pgfpathlineto{\pgfqpoint{1.222034in}{0.603981in}}%
\pgfpathlineto{\pgfqpoint{1.229848in}{0.590370in}}%
\pgfpathlineto{\pgfqpoint{1.237183in}{0.582142in}}%
\pgfpathlineto{\pgfqpoint{1.242580in}{0.576759in}}%
\pgfpathlineto{\pgfqpoint{1.252839in}{0.568366in}}%
\pgfpathlineto{\pgfqpoint{1.261208in}{0.563148in}}%
\pgfpathlineto{\pgfqpoint{1.268496in}{0.558934in}}%
\pgfpathclose%
\pgfpathmoveto{\pgfqpoint{1.272742in}{0.590370in}}%
\pgfpathlineto{\pgfqpoint{1.268496in}{0.593007in}}%
\pgfpathlineto{\pgfqpoint{1.259412in}{0.603981in}}%
\pgfpathlineto{\pgfqpoint{1.256641in}{0.617592in}}%
\pgfpathlineto{\pgfqpoint{1.262993in}{0.631203in}}%
\pgfpathlineto{\pgfqpoint{1.268496in}{0.636448in}}%
\pgfpathlineto{\pgfqpoint{1.284152in}{0.644177in}}%
\pgfpathlineto{\pgfqpoint{1.299809in}{0.643288in}}%
\pgfpathlineto{\pgfqpoint{1.315466in}{0.634005in}}%
\pgfpathlineto{\pgfqpoint{1.318184in}{0.631203in}}%
\pgfpathlineto{\pgfqpoint{1.324009in}{0.617592in}}%
\pgfpathlineto{\pgfqpoint{1.321468in}{0.603981in}}%
\pgfpathlineto{\pgfqpoint{1.315466in}{0.596138in}}%
\pgfpathlineto{\pgfqpoint{1.307755in}{0.590370in}}%
\pgfpathlineto{\pgfqpoint{1.299809in}{0.586688in}}%
\pgfpathlineto{\pgfqpoint{1.284152in}{0.586003in}}%
\pgfpathlineto{\pgfqpoint{1.272742in}{0.590370in}}%
\pgfpathclose%
\pgfpathmoveto{\pgfqpoint{1.581627in}{0.557610in}}%
\pgfpathlineto{\pgfqpoint{1.597284in}{0.553890in}}%
\pgfpathlineto{\pgfqpoint{1.612940in}{0.555513in}}%
\pgfpathlineto{\pgfqpoint{1.628597in}{0.562067in}}%
\pgfpathlineto{\pgfqpoint{1.630213in}{0.563148in}}%
\pgfpathlineto{\pgfqpoint{1.644253in}{0.573198in}}%
\pgfpathlineto{\pgfqpoint{1.648349in}{0.576759in}}%
\pgfpathlineto{\pgfqpoint{1.659910in}{0.588965in}}%
\pgfpathlineto{\pgfqpoint{1.661153in}{0.590370in}}%
\pgfpathlineto{\pgfqpoint{1.668692in}{0.603981in}}%
\pgfpathlineto{\pgfqpoint{1.670559in}{0.617592in}}%
\pgfpathlineto{\pgfqpoint{1.666280in}{0.631203in}}%
\pgfpathlineto{\pgfqpoint{1.659910in}{0.640331in}}%
\pgfpathlineto{\pgfqpoint{1.656522in}{0.644814in}}%
\pgfpathlineto{\pgfqpoint{1.644253in}{0.656618in}}%
\pgfpathlineto{\pgfqpoint{1.641957in}{0.658425in}}%
\pgfpathlineto{\pgfqpoint{1.628597in}{0.667525in}}%
\pgfpathlineto{\pgfqpoint{1.618327in}{0.672036in}}%
\pgfpathlineto{\pgfqpoint{1.612940in}{0.674300in}}%
\pgfpathlineto{\pgfqpoint{1.597284in}{0.675893in}}%
\pgfpathlineto{\pgfqpoint{1.581627in}{0.672243in}}%
\pgfpathlineto{\pgfqpoint{1.581240in}{0.672036in}}%
\pgfpathlineto{\pgfqpoint{1.565971in}{0.663614in}}%
\pgfpathlineto{\pgfqpoint{1.559115in}{0.658425in}}%
\pgfpathlineto{\pgfqpoint{1.550314in}{0.650774in}}%
\pgfpathlineto{\pgfqpoint{1.544346in}{0.644814in}}%
\pgfpathlineto{\pgfqpoint{1.534657in}{0.631540in}}%
\pgfpathlineto{\pgfqpoint{1.534420in}{0.631203in}}%
\pgfpathlineto{\pgfqpoint{1.530222in}{0.617592in}}%
\pgfpathlineto{\pgfqpoint{1.532053in}{0.603981in}}%
\pgfpathlineto{\pgfqpoint{1.534657in}{0.599298in}}%
\pgfpathlineto{\pgfqpoint{1.539847in}{0.590370in}}%
\pgfpathlineto{\pgfqpoint{1.550314in}{0.578755in}}%
\pgfpathlineto{\pgfqpoint{1.552393in}{0.576759in}}%
\pgfpathlineto{\pgfqpoint{1.565971in}{0.566093in}}%
\pgfpathlineto{\pgfqpoint{1.571128in}{0.563148in}}%
\pgfpathlineto{\pgfqpoint{1.581627in}{0.557610in}}%
\pgfpathclose%
\pgfpathmoveto{\pgfqpoint{1.581502in}{0.590370in}}%
\pgfpathlineto{\pgfqpoint{1.569144in}{0.603981in}}%
\pgfpathlineto{\pgfqpoint{1.566084in}{0.617592in}}%
\pgfpathlineto{\pgfqpoint{1.573098in}{0.631203in}}%
\pgfpathlineto{\pgfqpoint{1.581627in}{0.638618in}}%
\pgfpathlineto{\pgfqpoint{1.597284in}{0.644715in}}%
\pgfpathlineto{\pgfqpoint{1.612940in}{0.642055in}}%
\pgfpathlineto{\pgfqpoint{1.628597in}{0.631312in}}%
\pgfpathlineto{\pgfqpoint{1.628696in}{0.631203in}}%
\pgfpathlineto{\pgfqpoint{1.634096in}{0.617592in}}%
\pgfpathlineto{\pgfqpoint{1.631740in}{0.603981in}}%
\pgfpathlineto{\pgfqpoint{1.628597in}{0.599590in}}%
\pgfpathlineto{\pgfqpoint{1.617991in}{0.590370in}}%
\pgfpathlineto{\pgfqpoint{1.612940in}{0.587637in}}%
\pgfpathlineto{\pgfqpoint{1.597284in}{0.585589in}}%
\pgfpathlineto{\pgfqpoint{1.581627in}{0.590284in}}%
\pgfpathlineto{\pgfqpoint{1.581502in}{0.590370in}}%
\pgfpathclose%
\pgfpathmoveto{\pgfqpoint{1.894758in}{0.556466in}}%
\pgfpathlineto{\pgfqpoint{1.910415in}{0.553780in}}%
\pgfpathlineto{\pgfqpoint{1.910415in}{0.563148in}}%
\pgfpathlineto{\pgfqpoint{1.910415in}{0.576759in}}%
\pgfpathlineto{\pgfqpoint{1.910415in}{0.585451in}}%
\pgfpathlineto{\pgfqpoint{1.894758in}{0.588840in}}%
\pgfpathlineto{\pgfqpoint{1.892272in}{0.590370in}}%
\pgfpathlineto{\pgfqpoint{1.879102in}{0.603330in}}%
\pgfpathlineto{\pgfqpoint{1.878662in}{0.603981in}}%
\pgfpathlineto{\pgfqpoint{1.876458in}{0.617592in}}%
\pgfpathlineto{\pgfqpoint{1.879102in}{0.624649in}}%
\pgfpathlineto{\pgfqpoint{1.882850in}{0.631203in}}%
\pgfpathlineto{\pgfqpoint{1.894758in}{0.640493in}}%
\pgfpathlineto{\pgfqpoint{1.910131in}{0.644814in}}%
\pgfpathlineto{\pgfqpoint{1.910415in}{0.644869in}}%
\pgfpathlineto{\pgfqpoint{1.910415in}{0.658425in}}%
\pgfpathlineto{\pgfqpoint{1.910415in}{0.672036in}}%
\pgfpathlineto{\pgfqpoint{1.910415in}{0.676000in}}%
\pgfpathlineto{\pgfqpoint{1.894758in}{0.673365in}}%
\pgfpathlineto{\pgfqpoint{1.891978in}{0.672036in}}%
\pgfpathlineto{\pgfqpoint{1.879102in}{0.665634in}}%
\pgfpathlineto{\pgfqpoint{1.869105in}{0.658425in}}%
\pgfpathlineto{\pgfqpoint{1.863445in}{0.653724in}}%
\pgfpathlineto{\pgfqpoint{1.854380in}{0.644814in}}%
\pgfpathlineto{\pgfqpoint{1.847789in}{0.635916in}}%
\pgfpathlineto{\pgfqpoint{1.844490in}{0.631203in}}%
\pgfpathlineto{\pgfqpoint{1.840261in}{0.617592in}}%
\pgfpathlineto{\pgfqpoint{1.842106in}{0.603981in}}%
\pgfpathlineto{\pgfqpoint{1.847789in}{0.593689in}}%
\pgfpathlineto{\pgfqpoint{1.849747in}{0.590370in}}%
\pgfpathlineto{\pgfqpoint{1.862277in}{0.576759in}}%
\pgfpathlineto{\pgfqpoint{1.863445in}{0.575721in}}%
\pgfpathlineto{\pgfqpoint{1.879102in}{0.563940in}}%
\pgfpathlineto{\pgfqpoint{1.880640in}{0.563148in}}%
\pgfpathlineto{\pgfqpoint{1.894758in}{0.556466in}}%
\pgfpathclose%
\pgfpathmoveto{\pgfqpoint{0.376072in}{0.825968in}}%
\pgfpathlineto{\pgfqpoint{0.391728in}{0.833467in}}%
\pgfpathlineto{\pgfqpoint{0.394412in}{0.835370in}}%
\pgfpathlineto{\pgfqpoint{0.407385in}{0.845422in}}%
\pgfpathlineto{\pgfqpoint{0.411285in}{0.848981in}}%
\pgfpathlineto{\pgfqpoint{0.423041in}{0.862473in}}%
\pgfpathlineto{\pgfqpoint{0.423140in}{0.862592in}}%
\pgfpathlineto{\pgfqpoint{0.429579in}{0.876203in}}%
\pgfpathlineto{\pgfqpoint{0.430196in}{0.889814in}}%
\pgfpathlineto{\pgfqpoint{0.424834in}{0.903425in}}%
\pgfpathlineto{\pgfqpoint{0.423041in}{0.905762in}}%
\pgfpathlineto{\pgfqpoint{0.413923in}{0.917036in}}%
\pgfpathlineto{\pgfqpoint{0.407385in}{0.923208in}}%
\pgfpathlineto{\pgfqpoint{0.398080in}{0.930648in}}%
\pgfpathlineto{\pgfqpoint{0.391728in}{0.935179in}}%
\pgfpathlineto{\pgfqpoint{0.376072in}{0.942830in}}%
\pgfpathlineto{\pgfqpoint{0.367977in}{0.944259in}}%
\pgfpathlineto{\pgfqpoint{0.360415in}{0.945536in}}%
\pgfpathlineto{\pgfqpoint{0.360415in}{0.944259in}}%
\pgfpathlineto{\pgfqpoint{0.360415in}{0.930648in}}%
\pgfpathlineto{\pgfqpoint{0.360415in}{0.917036in}}%
\pgfpathlineto{\pgfqpoint{0.360415in}{0.913895in}}%
\pgfpathlineto{\pgfqpoint{0.376072in}{0.909908in}}%
\pgfpathlineto{\pgfqpoint{0.385182in}{0.903425in}}%
\pgfpathlineto{\pgfqpoint{0.391728in}{0.894410in}}%
\pgfpathlineto{\pgfqpoint{0.393926in}{0.889814in}}%
\pgfpathlineto{\pgfqpoint{0.393189in}{0.876203in}}%
\pgfpathlineto{\pgfqpoint{0.391728in}{0.873680in}}%
\pgfpathlineto{\pgfqpoint{0.382031in}{0.862592in}}%
\pgfpathlineto{\pgfqpoint{0.376072in}{0.858671in}}%
\pgfpathlineto{\pgfqpoint{0.360415in}{0.855015in}}%
\pgfpathlineto{\pgfqpoint{0.360415in}{0.848981in}}%
\pgfpathlineto{\pgfqpoint{0.360415in}{0.835370in}}%
\pgfpathlineto{\pgfqpoint{0.360415in}{0.823240in}}%
\pgfpathlineto{\pgfqpoint{0.376072in}{0.825968in}}%
\pgfpathclose%
\pgfpathmoveto{\pgfqpoint{0.642233in}{0.831658in}}%
\pgfpathlineto{\pgfqpoint{0.657890in}{0.825000in}}%
\pgfpathlineto{\pgfqpoint{0.673546in}{0.823352in}}%
\pgfpathlineto{\pgfqpoint{0.689203in}{0.827130in}}%
\pgfpathlineto{\pgfqpoint{0.704802in}{0.835370in}}%
\pgfpathlineto{\pgfqpoint{0.704859in}{0.835404in}}%
\pgfpathlineto{\pgfqpoint{0.720516in}{0.848098in}}%
\pgfpathlineto{\pgfqpoint{0.721468in}{0.848981in}}%
\pgfpathlineto{\pgfqpoint{0.732991in}{0.862592in}}%
\pgfpathlineto{\pgfqpoint{0.736173in}{0.868953in}}%
\pgfpathlineto{\pgfqpoint{0.739625in}{0.876203in}}%
\pgfpathlineto{\pgfqpoint{0.740238in}{0.889814in}}%
\pgfpathlineto{\pgfqpoint{0.736173in}{0.900149in}}%
\pgfpathlineto{\pgfqpoint{0.734812in}{0.903425in}}%
\pgfpathlineto{\pgfqpoint{0.724031in}{0.917036in}}%
\pgfpathlineto{\pgfqpoint{0.720516in}{0.920407in}}%
\pgfpathlineto{\pgfqpoint{0.708279in}{0.930648in}}%
\pgfpathlineto{\pgfqpoint{0.704859in}{0.933207in}}%
\pgfpathlineto{\pgfqpoint{0.689203in}{0.941644in}}%
\pgfpathlineto{\pgfqpoint{0.678517in}{0.944259in}}%
\pgfpathlineto{\pgfqpoint{0.673546in}{0.945429in}}%
\pgfpathlineto{\pgfqpoint{0.662025in}{0.944259in}}%
\pgfpathlineto{\pgfqpoint{0.657890in}{0.943817in}}%
\pgfpathlineto{\pgfqpoint{0.642233in}{0.937024in}}%
\pgfpathlineto{\pgfqpoint{0.632769in}{0.930648in}}%
\pgfpathlineto{\pgfqpoint{0.626577in}{0.925955in}}%
\pgfpathlineto{\pgfqpoint{0.616922in}{0.917036in}}%
\pgfpathlineto{\pgfqpoint{0.610920in}{0.909761in}}%
\pgfpathlineto{\pgfqpoint{0.606073in}{0.903425in}}%
\pgfpathlineto{\pgfqpoint{0.600649in}{0.889814in}}%
\pgfpathlineto{\pgfqpoint{0.601273in}{0.876203in}}%
\pgfpathlineto{\pgfqpoint{0.607787in}{0.862592in}}%
\pgfpathlineto{\pgfqpoint{0.610920in}{0.858806in}}%
\pgfpathlineto{\pgfqpoint{0.619654in}{0.848981in}}%
\pgfpathlineto{\pgfqpoint{0.626577in}{0.842796in}}%
\pgfpathlineto{\pgfqpoint{0.636691in}{0.835370in}}%
\pgfpathlineto{\pgfqpoint{0.642233in}{0.831658in}}%
\pgfpathclose%
\pgfpathmoveto{\pgfqpoint{0.648869in}{0.862592in}}%
\pgfpathlineto{\pgfqpoint{0.642233in}{0.869295in}}%
\pgfpathlineto{\pgfqpoint{0.637999in}{0.876203in}}%
\pgfpathlineto{\pgfqpoint{0.637211in}{0.889814in}}%
\pgfpathlineto{\pgfqpoint{0.642233in}{0.899734in}}%
\pgfpathlineto{\pgfqpoint{0.645267in}{0.903425in}}%
\pgfpathlineto{\pgfqpoint{0.657890in}{0.911323in}}%
\pgfpathlineto{\pgfqpoint{0.673546in}{0.913732in}}%
\pgfpathlineto{\pgfqpoint{0.689203in}{0.908210in}}%
\pgfpathlineto{\pgfqpoint{0.695236in}{0.903425in}}%
\pgfpathlineto{\pgfqpoint{0.704127in}{0.889814in}}%
\pgfpathlineto{\pgfqpoint{0.703103in}{0.876203in}}%
\pgfpathlineto{\pgfqpoint{0.692426in}{0.862592in}}%
\pgfpathlineto{\pgfqpoint{0.689203in}{0.860229in}}%
\pgfpathlineto{\pgfqpoint{0.673546in}{0.855164in}}%
\pgfpathlineto{\pgfqpoint{0.657890in}{0.857374in}}%
\pgfpathlineto{\pgfqpoint{0.648869in}{0.862592in}}%
\pgfpathclose%
\pgfpathmoveto{\pgfqpoint{0.955364in}{0.829989in}}%
\pgfpathlineto{\pgfqpoint{0.971021in}{0.824236in}}%
\pgfpathlineto{\pgfqpoint{0.986678in}{0.823685in}}%
\pgfpathlineto{\pgfqpoint{1.002334in}{0.828475in}}%
\pgfpathlineto{\pgfqpoint{1.014240in}{0.835370in}}%
\pgfpathlineto{\pgfqpoint{1.017991in}{0.837768in}}%
\pgfpathlineto{\pgfqpoint{1.031329in}{0.848981in}}%
\pgfpathlineto{\pgfqpoint{1.033647in}{0.851445in}}%
\pgfpathlineto{\pgfqpoint{1.042941in}{0.862592in}}%
\pgfpathlineto{\pgfqpoint{1.049304in}{0.875442in}}%
\pgfpathlineto{\pgfqpoint{1.049670in}{0.876203in}}%
\pgfpathlineto{\pgfqpoint{1.050281in}{0.889814in}}%
\pgfpathlineto{\pgfqpoint{1.049304in}{0.892271in}}%
\pgfpathlineto{\pgfqpoint{1.044718in}{0.903425in}}%
\pgfpathlineto{\pgfqpoint{1.034198in}{0.917036in}}%
\pgfpathlineto{\pgfqpoint{1.033647in}{0.917570in}}%
\pgfpathlineto{\pgfqpoint{1.018605in}{0.930648in}}%
\pgfpathlineto{\pgfqpoint{1.017991in}{0.931126in}}%
\pgfpathlineto{\pgfqpoint{1.002334in}{0.940272in}}%
\pgfpathlineto{\pgfqpoint{0.989504in}{0.944259in}}%
\pgfpathlineto{\pgfqpoint{0.986678in}{0.945108in}}%
\pgfpathlineto{\pgfqpoint{0.971021in}{0.944577in}}%
\pgfpathlineto{\pgfqpoint{0.970145in}{0.944259in}}%
\pgfpathlineto{\pgfqpoint{0.955364in}{0.938727in}}%
\pgfpathlineto{\pgfqpoint{0.942543in}{0.930648in}}%
\pgfpathlineto{\pgfqpoint{0.939708in}{0.928632in}}%
\pgfpathlineto{\pgfqpoint{0.926811in}{0.917036in}}%
\pgfpathlineto{\pgfqpoint{0.924051in}{0.913776in}}%
\pgfpathlineto{\pgfqpoint{0.916121in}{0.903425in}}%
\pgfpathlineto{\pgfqpoint{0.910611in}{0.889814in}}%
\pgfpathlineto{\pgfqpoint{0.911245in}{0.876203in}}%
\pgfpathlineto{\pgfqpoint{0.917862in}{0.862592in}}%
\pgfpathlineto{\pgfqpoint{0.924051in}{0.855124in}}%
\pgfpathlineto{\pgfqpoint{0.929652in}{0.848981in}}%
\pgfpathlineto{\pgfqpoint{0.939708in}{0.840239in}}%
\pgfpathlineto{\pgfqpoint{0.946774in}{0.835370in}}%
\pgfpathlineto{\pgfqpoint{0.955364in}{0.829989in}}%
\pgfpathclose%
\pgfpathmoveto{\pgfqpoint{0.958420in}{0.862592in}}%
\pgfpathlineto{\pgfqpoint{0.955364in}{0.865248in}}%
\pgfpathlineto{\pgfqpoint{0.948185in}{0.876203in}}%
\pgfpathlineto{\pgfqpoint{0.947334in}{0.889814in}}%
\pgfpathlineto{\pgfqpoint{0.954720in}{0.903425in}}%
\pgfpathlineto{\pgfqpoint{0.955364in}{0.904032in}}%
\pgfpathlineto{\pgfqpoint{0.971021in}{0.912439in}}%
\pgfpathlineto{\pgfqpoint{0.986678in}{0.913245in}}%
\pgfpathlineto{\pgfqpoint{1.002334in}{0.906244in}}%
\pgfpathlineto{\pgfqpoint{1.005577in}{0.903425in}}%
\pgfpathlineto{\pgfqpoint{1.013629in}{0.889814in}}%
\pgfpathlineto{\pgfqpoint{1.012702in}{0.876203in}}%
\pgfpathlineto{\pgfqpoint{1.003032in}{0.862592in}}%
\pgfpathlineto{\pgfqpoint{1.002334in}{0.862031in}}%
\pgfpathlineto{\pgfqpoint{0.986678in}{0.855611in}}%
\pgfpathlineto{\pgfqpoint{0.971021in}{0.856350in}}%
\pgfpathlineto{\pgfqpoint{0.958420in}{0.862592in}}%
\pgfpathclose%
\pgfpathmoveto{\pgfqpoint{1.268496in}{0.828475in}}%
\pgfpathlineto{\pgfqpoint{1.284152in}{0.823685in}}%
\pgfpathlineto{\pgfqpoint{1.299809in}{0.824236in}}%
\pgfpathlineto{\pgfqpoint{1.315466in}{0.829989in}}%
\pgfpathlineto{\pgfqpoint{1.324056in}{0.835370in}}%
\pgfpathlineto{\pgfqpoint{1.331122in}{0.840239in}}%
\pgfpathlineto{\pgfqpoint{1.341178in}{0.848981in}}%
\pgfpathlineto{\pgfqpoint{1.346779in}{0.855124in}}%
\pgfpathlineto{\pgfqpoint{1.352968in}{0.862592in}}%
\pgfpathlineto{\pgfqpoint{1.359585in}{0.876203in}}%
\pgfpathlineto{\pgfqpoint{1.360219in}{0.889814in}}%
\pgfpathlineto{\pgfqpoint{1.354709in}{0.903425in}}%
\pgfpathlineto{\pgfqpoint{1.346779in}{0.913776in}}%
\pgfpathlineto{\pgfqpoint{1.344019in}{0.917036in}}%
\pgfpathlineto{\pgfqpoint{1.331122in}{0.928632in}}%
\pgfpathlineto{\pgfqpoint{1.328287in}{0.930648in}}%
\pgfpathlineto{\pgfqpoint{1.315466in}{0.938727in}}%
\pgfpathlineto{\pgfqpoint{1.300685in}{0.944259in}}%
\pgfpathlineto{\pgfqpoint{1.299809in}{0.944577in}}%
\pgfpathlineto{\pgfqpoint{1.284152in}{0.945108in}}%
\pgfpathlineto{\pgfqpoint{1.281326in}{0.944259in}}%
\pgfpathlineto{\pgfqpoint{1.268496in}{0.940272in}}%
\pgfpathlineto{\pgfqpoint{1.252839in}{0.931126in}}%
\pgfpathlineto{\pgfqpoint{1.252225in}{0.930648in}}%
\pgfpathlineto{\pgfqpoint{1.237183in}{0.917570in}}%
\pgfpathlineto{\pgfqpoint{1.236632in}{0.917036in}}%
\pgfpathlineto{\pgfqpoint{1.226112in}{0.903425in}}%
\pgfpathlineto{\pgfqpoint{1.221526in}{0.892271in}}%
\pgfpathlineto{\pgfqpoint{1.220549in}{0.889814in}}%
\pgfpathlineto{\pgfqpoint{1.221160in}{0.876203in}}%
\pgfpathlineto{\pgfqpoint{1.221526in}{0.875442in}}%
\pgfpathlineto{\pgfqpoint{1.227889in}{0.862592in}}%
\pgfpathlineto{\pgfqpoint{1.237183in}{0.851445in}}%
\pgfpathlineto{\pgfqpoint{1.239501in}{0.848981in}}%
\pgfpathlineto{\pgfqpoint{1.252839in}{0.837768in}}%
\pgfpathlineto{\pgfqpoint{1.256590in}{0.835370in}}%
\pgfpathlineto{\pgfqpoint{1.268496in}{0.828475in}}%
\pgfpathclose%
\pgfpathmoveto{\pgfqpoint{1.267798in}{0.862592in}}%
\pgfpathlineto{\pgfqpoint{1.258128in}{0.876203in}}%
\pgfpathlineto{\pgfqpoint{1.257201in}{0.889814in}}%
\pgfpathlineto{\pgfqpoint{1.265253in}{0.903425in}}%
\pgfpathlineto{\pgfqpoint{1.268496in}{0.906244in}}%
\pgfpathlineto{\pgfqpoint{1.284152in}{0.913245in}}%
\pgfpathlineto{\pgfqpoint{1.299809in}{0.912439in}}%
\pgfpathlineto{\pgfqpoint{1.315466in}{0.904032in}}%
\pgfpathlineto{\pgfqpoint{1.316110in}{0.903425in}}%
\pgfpathlineto{\pgfqpoint{1.323496in}{0.889814in}}%
\pgfpathlineto{\pgfqpoint{1.322645in}{0.876203in}}%
\pgfpathlineto{\pgfqpoint{1.315466in}{0.865248in}}%
\pgfpathlineto{\pgfqpoint{1.312410in}{0.862592in}}%
\pgfpathlineto{\pgfqpoint{1.299809in}{0.856350in}}%
\pgfpathlineto{\pgfqpoint{1.284152in}{0.855611in}}%
\pgfpathlineto{\pgfqpoint{1.268496in}{0.862031in}}%
\pgfpathlineto{\pgfqpoint{1.267798in}{0.862592in}}%
\pgfpathclose%
\pgfpathmoveto{\pgfqpoint{1.581627in}{0.827130in}}%
\pgfpathlineto{\pgfqpoint{1.597284in}{0.823352in}}%
\pgfpathlineto{\pgfqpoint{1.612940in}{0.825000in}}%
\pgfpathlineto{\pgfqpoint{1.628597in}{0.831658in}}%
\pgfpathlineto{\pgfqpoint{1.634139in}{0.835370in}}%
\pgfpathlineto{\pgfqpoint{1.644253in}{0.842796in}}%
\pgfpathlineto{\pgfqpoint{1.651176in}{0.848981in}}%
\pgfpathlineto{\pgfqpoint{1.659910in}{0.858806in}}%
\pgfpathlineto{\pgfqpoint{1.663043in}{0.862592in}}%
\pgfpathlineto{\pgfqpoint{1.669557in}{0.876203in}}%
\pgfpathlineto{\pgfqpoint{1.670181in}{0.889814in}}%
\pgfpathlineto{\pgfqpoint{1.664757in}{0.903425in}}%
\pgfpathlineto{\pgfqpoint{1.659910in}{0.909761in}}%
\pgfpathlineto{\pgfqpoint{1.653908in}{0.917036in}}%
\pgfpathlineto{\pgfqpoint{1.644253in}{0.925955in}}%
\pgfpathlineto{\pgfqpoint{1.638061in}{0.930648in}}%
\pgfpathlineto{\pgfqpoint{1.628597in}{0.937024in}}%
\pgfpathlineto{\pgfqpoint{1.612940in}{0.943817in}}%
\pgfpathlineto{\pgfqpoint{1.608805in}{0.944259in}}%
\pgfpathlineto{\pgfqpoint{1.597284in}{0.945429in}}%
\pgfpathlineto{\pgfqpoint{1.592313in}{0.944259in}}%
\pgfpathlineto{\pgfqpoint{1.581627in}{0.941644in}}%
\pgfpathlineto{\pgfqpoint{1.565971in}{0.933207in}}%
\pgfpathlineto{\pgfqpoint{1.562551in}{0.930648in}}%
\pgfpathlineto{\pgfqpoint{1.550314in}{0.920407in}}%
\pgfpathlineto{\pgfqpoint{1.546799in}{0.917036in}}%
\pgfpathlineto{\pgfqpoint{1.536018in}{0.903425in}}%
\pgfpathlineto{\pgfqpoint{1.534657in}{0.900149in}}%
\pgfpathlineto{\pgfqpoint{1.530592in}{0.889814in}}%
\pgfpathlineto{\pgfqpoint{1.531205in}{0.876203in}}%
\pgfpathlineto{\pgfqpoint{1.534657in}{0.868953in}}%
\pgfpathlineto{\pgfqpoint{1.537839in}{0.862592in}}%
\pgfpathlineto{\pgfqpoint{1.549362in}{0.848981in}}%
\pgfpathlineto{\pgfqpoint{1.550314in}{0.848098in}}%
\pgfpathlineto{\pgfqpoint{1.565971in}{0.835404in}}%
\pgfpathlineto{\pgfqpoint{1.566028in}{0.835370in}}%
\pgfpathlineto{\pgfqpoint{1.581627in}{0.827130in}}%
\pgfpathclose%
\pgfpathmoveto{\pgfqpoint{1.578404in}{0.862592in}}%
\pgfpathlineto{\pgfqpoint{1.567727in}{0.876203in}}%
\pgfpathlineto{\pgfqpoint{1.566703in}{0.889814in}}%
\pgfpathlineto{\pgfqpoint{1.575594in}{0.903425in}}%
\pgfpathlineto{\pgfqpoint{1.581627in}{0.908210in}}%
\pgfpathlineto{\pgfqpoint{1.597284in}{0.913732in}}%
\pgfpathlineto{\pgfqpoint{1.612940in}{0.911323in}}%
\pgfpathlineto{\pgfqpoint{1.625563in}{0.903425in}}%
\pgfpathlineto{\pgfqpoint{1.628597in}{0.899734in}}%
\pgfpathlineto{\pgfqpoint{1.633619in}{0.889814in}}%
\pgfpathlineto{\pgfqpoint{1.632831in}{0.876203in}}%
\pgfpathlineto{\pgfqpoint{1.628597in}{0.869295in}}%
\pgfpathlineto{\pgfqpoint{1.621961in}{0.862592in}}%
\pgfpathlineto{\pgfqpoint{1.612940in}{0.857374in}}%
\pgfpathlineto{\pgfqpoint{1.597284in}{0.855164in}}%
\pgfpathlineto{\pgfqpoint{1.581627in}{0.860229in}}%
\pgfpathlineto{\pgfqpoint{1.578404in}{0.862592in}}%
\pgfpathclose%
\pgfpathmoveto{\pgfqpoint{1.879102in}{0.833467in}}%
\pgfpathlineto{\pgfqpoint{1.894758in}{0.825968in}}%
\pgfpathlineto{\pgfqpoint{1.910415in}{0.823240in}}%
\pgfpathlineto{\pgfqpoint{1.910415in}{0.835370in}}%
\pgfpathlineto{\pgfqpoint{1.910415in}{0.848981in}}%
\pgfpathlineto{\pgfqpoint{1.910415in}{0.855015in}}%
\pgfpathlineto{\pgfqpoint{1.894758in}{0.858671in}}%
\pgfpathlineto{\pgfqpoint{1.888799in}{0.862592in}}%
\pgfpathlineto{\pgfqpoint{1.879102in}{0.873680in}}%
\pgfpathlineto{\pgfqpoint{1.877641in}{0.876203in}}%
\pgfpathlineto{\pgfqpoint{1.876904in}{0.889814in}}%
\pgfpathlineto{\pgfqpoint{1.879102in}{0.894410in}}%
\pgfpathlineto{\pgfqpoint{1.885648in}{0.903425in}}%
\pgfpathlineto{\pgfqpoint{1.894758in}{0.909908in}}%
\pgfpathlineto{\pgfqpoint{1.910415in}{0.913895in}}%
\pgfpathlineto{\pgfqpoint{1.910415in}{0.917036in}}%
\pgfpathlineto{\pgfqpoint{1.910415in}{0.930648in}}%
\pgfpathlineto{\pgfqpoint{1.910415in}{0.944259in}}%
\pgfpathlineto{\pgfqpoint{1.910415in}{0.945536in}}%
\pgfpathlineto{\pgfqpoint{1.902853in}{0.944259in}}%
\pgfpathlineto{\pgfqpoint{1.894758in}{0.942830in}}%
\pgfpathlineto{\pgfqpoint{1.879102in}{0.935179in}}%
\pgfpathlineto{\pgfqpoint{1.872750in}{0.930648in}}%
\pgfpathlineto{\pgfqpoint{1.863445in}{0.923208in}}%
\pgfpathlineto{\pgfqpoint{1.856907in}{0.917036in}}%
\pgfpathlineto{\pgfqpoint{1.847789in}{0.905762in}}%
\pgfpathlineto{\pgfqpoint{1.845996in}{0.903425in}}%
\pgfpathlineto{\pgfqpoint{1.840634in}{0.889814in}}%
\pgfpathlineto{\pgfqpoint{1.841251in}{0.876203in}}%
\pgfpathlineto{\pgfqpoint{1.847690in}{0.862592in}}%
\pgfpathlineto{\pgfqpoint{1.847789in}{0.862473in}}%
\pgfpathlineto{\pgfqpoint{1.859545in}{0.848981in}}%
\pgfpathlineto{\pgfqpoint{1.863445in}{0.845422in}}%
\pgfpathlineto{\pgfqpoint{1.876418in}{0.835370in}}%
\pgfpathlineto{\pgfqpoint{1.879102in}{0.833467in}}%
\pgfpathclose%
\pgfpathmoveto{\pgfqpoint{0.367977in}{1.093981in}}%
\pgfpathlineto{\pgfqpoint{0.376072in}{1.095410in}}%
\pgfpathlineto{\pgfqpoint{0.391728in}{1.103061in}}%
\pgfpathlineto{\pgfqpoint{0.398080in}{1.107592in}}%
\pgfpathlineto{\pgfqpoint{0.407385in}{1.115032in}}%
\pgfpathlineto{\pgfqpoint{0.413923in}{1.121203in}}%
\pgfpathlineto{\pgfqpoint{0.423041in}{1.132477in}}%
\pgfpathlineto{\pgfqpoint{0.424834in}{1.134814in}}%
\pgfpathlineto{\pgfqpoint{0.430196in}{1.148425in}}%
\pgfpathlineto{\pgfqpoint{0.429579in}{1.162036in}}%
\pgfpathlineto{\pgfqpoint{0.423140in}{1.175647in}}%
\pgfpathlineto{\pgfqpoint{0.423041in}{1.175766in}}%
\pgfpathlineto{\pgfqpoint{0.411285in}{1.189259in}}%
\pgfpathlineto{\pgfqpoint{0.407385in}{1.192818in}}%
\pgfpathlineto{\pgfqpoint{0.394412in}{1.202870in}}%
\pgfpathlineto{\pgfqpoint{0.391728in}{1.204773in}}%
\pgfpathlineto{\pgfqpoint{0.376072in}{1.212271in}}%
\pgfpathlineto{\pgfqpoint{0.360415in}{1.214999in}}%
\pgfpathlineto{\pgfqpoint{0.360415in}{1.202870in}}%
\pgfpathlineto{\pgfqpoint{0.360415in}{1.189259in}}%
\pgfpathlineto{\pgfqpoint{0.360415in}{1.183225in}}%
\pgfpathlineto{\pgfqpoint{0.376072in}{1.179568in}}%
\pgfpathlineto{\pgfqpoint{0.382031in}{1.175647in}}%
\pgfpathlineto{\pgfqpoint{0.391728in}{1.164560in}}%
\pgfpathlineto{\pgfqpoint{0.393189in}{1.162036in}}%
\pgfpathlineto{\pgfqpoint{0.393926in}{1.148425in}}%
\pgfpathlineto{\pgfqpoint{0.391728in}{1.143829in}}%
\pgfpathlineto{\pgfqpoint{0.385182in}{1.134814in}}%
\pgfpathlineto{\pgfqpoint{0.376072in}{1.128331in}}%
\pgfpathlineto{\pgfqpoint{0.360415in}{1.124345in}}%
\pgfpathlineto{\pgfqpoint{0.360415in}{1.121203in}}%
\pgfpathlineto{\pgfqpoint{0.360415in}{1.107592in}}%
\pgfpathlineto{\pgfqpoint{0.360415in}{1.093981in}}%
\pgfpathlineto{\pgfqpoint{0.360415in}{1.092703in}}%
\pgfpathlineto{\pgfqpoint{0.367977in}{1.093981in}}%
\pgfpathclose%
\pgfpathmoveto{\pgfqpoint{0.673546in}{1.092811in}}%
\pgfpathlineto{\pgfqpoint{0.678517in}{1.093981in}}%
\pgfpathlineto{\pgfqpoint{0.689203in}{1.096596in}}%
\pgfpathlineto{\pgfqpoint{0.704859in}{1.105032in}}%
\pgfpathlineto{\pgfqpoint{0.708279in}{1.107592in}}%
\pgfpathlineto{\pgfqpoint{0.720516in}{1.117833in}}%
\pgfpathlineto{\pgfqpoint{0.724031in}{1.121203in}}%
\pgfpathlineto{\pgfqpoint{0.734812in}{1.134814in}}%
\pgfpathlineto{\pgfqpoint{0.736173in}{1.138090in}}%
\pgfpathlineto{\pgfqpoint{0.740238in}{1.148425in}}%
\pgfpathlineto{\pgfqpoint{0.739625in}{1.162036in}}%
\pgfpathlineto{\pgfqpoint{0.736173in}{1.169287in}}%
\pgfpathlineto{\pgfqpoint{0.732991in}{1.175647in}}%
\pgfpathlineto{\pgfqpoint{0.721468in}{1.189259in}}%
\pgfpathlineto{\pgfqpoint{0.720516in}{1.190141in}}%
\pgfpathlineto{\pgfqpoint{0.704859in}{1.202836in}}%
\pgfpathlineto{\pgfqpoint{0.704802in}{1.202870in}}%
\pgfpathlineto{\pgfqpoint{0.689203in}{1.211109in}}%
\pgfpathlineto{\pgfqpoint{0.673546in}{1.214888in}}%
\pgfpathlineto{\pgfqpoint{0.657890in}{1.213239in}}%
\pgfpathlineto{\pgfqpoint{0.642233in}{1.206581in}}%
\pgfpathlineto{\pgfqpoint{0.636691in}{1.202870in}}%
\pgfpathlineto{\pgfqpoint{0.626577in}{1.195443in}}%
\pgfpathlineto{\pgfqpoint{0.619654in}{1.189259in}}%
\pgfpathlineto{\pgfqpoint{0.610920in}{1.179433in}}%
\pgfpathlineto{\pgfqpoint{0.607787in}{1.175647in}}%
\pgfpathlineto{\pgfqpoint{0.601273in}{1.162036in}}%
\pgfpathlineto{\pgfqpoint{0.600649in}{1.148425in}}%
\pgfpathlineto{\pgfqpoint{0.606073in}{1.134814in}}%
\pgfpathlineto{\pgfqpoint{0.610920in}{1.128479in}}%
\pgfpathlineto{\pgfqpoint{0.616922in}{1.121203in}}%
\pgfpathlineto{\pgfqpoint{0.626577in}{1.112284in}}%
\pgfpathlineto{\pgfqpoint{0.632769in}{1.107592in}}%
\pgfpathlineto{\pgfqpoint{0.642233in}{1.101215in}}%
\pgfpathlineto{\pgfqpoint{0.657890in}{1.094422in}}%
\pgfpathlineto{\pgfqpoint{0.662025in}{1.093981in}}%
\pgfpathlineto{\pgfqpoint{0.673546in}{1.092811in}}%
\pgfpathclose%
\pgfpathmoveto{\pgfqpoint{0.645267in}{1.134814in}}%
\pgfpathlineto{\pgfqpoint{0.642233in}{1.138505in}}%
\pgfpathlineto{\pgfqpoint{0.637211in}{1.148425in}}%
\pgfpathlineto{\pgfqpoint{0.637999in}{1.162036in}}%
\pgfpathlineto{\pgfqpoint{0.642233in}{1.168945in}}%
\pgfpathlineto{\pgfqpoint{0.648869in}{1.175647in}}%
\pgfpathlineto{\pgfqpoint{0.657890in}{1.180866in}}%
\pgfpathlineto{\pgfqpoint{0.673546in}{1.183075in}}%
\pgfpathlineto{\pgfqpoint{0.689203in}{1.178011in}}%
\pgfpathlineto{\pgfqpoint{0.692426in}{1.175647in}}%
\pgfpathlineto{\pgfqpoint{0.703103in}{1.162036in}}%
\pgfpathlineto{\pgfqpoint{0.704127in}{1.148425in}}%
\pgfpathlineto{\pgfqpoint{0.695236in}{1.134814in}}%
\pgfpathlineto{\pgfqpoint{0.689203in}{1.130030in}}%
\pgfpathlineto{\pgfqpoint{0.673546in}{1.124508in}}%
\pgfpathlineto{\pgfqpoint{0.657890in}{1.126917in}}%
\pgfpathlineto{\pgfqpoint{0.645267in}{1.134814in}}%
\pgfpathclose%
\pgfpathmoveto{\pgfqpoint{0.971021in}{1.093662in}}%
\pgfpathlineto{\pgfqpoint{0.986678in}{1.093131in}}%
\pgfpathlineto{\pgfqpoint{0.989504in}{1.093981in}}%
\pgfpathlineto{\pgfqpoint{1.002334in}{1.097968in}}%
\pgfpathlineto{\pgfqpoint{1.017991in}{1.107113in}}%
\pgfpathlineto{\pgfqpoint{1.018605in}{1.107592in}}%
\pgfpathlineto{\pgfqpoint{1.033647in}{1.120669in}}%
\pgfpathlineto{\pgfqpoint{1.034198in}{1.121203in}}%
\pgfpathlineto{\pgfqpoint{1.044718in}{1.134814in}}%
\pgfpathlineto{\pgfqpoint{1.049304in}{1.145969in}}%
\pgfpathlineto{\pgfqpoint{1.050281in}{1.148425in}}%
\pgfpathlineto{\pgfqpoint{1.049670in}{1.162036in}}%
\pgfpathlineto{\pgfqpoint{1.049304in}{1.162798in}}%
\pgfpathlineto{\pgfqpoint{1.042941in}{1.175647in}}%
\pgfpathlineto{\pgfqpoint{1.033647in}{1.186794in}}%
\pgfpathlineto{\pgfqpoint{1.031329in}{1.189259in}}%
\pgfpathlineto{\pgfqpoint{1.017991in}{1.200471in}}%
\pgfpathlineto{\pgfqpoint{1.014240in}{1.202870in}}%
\pgfpathlineto{\pgfqpoint{1.002334in}{1.209764in}}%
\pgfpathlineto{\pgfqpoint{0.986678in}{1.214555in}}%
\pgfpathlineto{\pgfqpoint{0.971021in}{1.214003in}}%
\pgfpathlineto{\pgfqpoint{0.955364in}{1.208250in}}%
\pgfpathlineto{\pgfqpoint{0.946774in}{1.202870in}}%
\pgfpathlineto{\pgfqpoint{0.939708in}{1.198000in}}%
\pgfpathlineto{\pgfqpoint{0.929652in}{1.189259in}}%
\pgfpathlineto{\pgfqpoint{0.924051in}{1.183116in}}%
\pgfpathlineto{\pgfqpoint{0.917862in}{1.175647in}}%
\pgfpathlineto{\pgfqpoint{0.911245in}{1.162036in}}%
\pgfpathlineto{\pgfqpoint{0.910611in}{1.148425in}}%
\pgfpathlineto{\pgfqpoint{0.916121in}{1.134814in}}%
\pgfpathlineto{\pgfqpoint{0.924051in}{1.124464in}}%
\pgfpathlineto{\pgfqpoint{0.926811in}{1.121203in}}%
\pgfpathlineto{\pgfqpoint{0.939708in}{1.109608in}}%
\pgfpathlineto{\pgfqpoint{0.942543in}{1.107592in}}%
\pgfpathlineto{\pgfqpoint{0.955364in}{1.099513in}}%
\pgfpathlineto{\pgfqpoint{0.970145in}{1.093981in}}%
\pgfpathlineto{\pgfqpoint{0.971021in}{1.093662in}}%
\pgfpathclose%
\pgfpathmoveto{\pgfqpoint{0.954720in}{1.134814in}}%
\pgfpathlineto{\pgfqpoint{0.947334in}{1.148425in}}%
\pgfpathlineto{\pgfqpoint{0.948185in}{1.162036in}}%
\pgfpathlineto{\pgfqpoint{0.955364in}{1.172991in}}%
\pgfpathlineto{\pgfqpoint{0.958420in}{1.175647in}}%
\pgfpathlineto{\pgfqpoint{0.971021in}{1.181889in}}%
\pgfpathlineto{\pgfqpoint{0.986678in}{1.182628in}}%
\pgfpathlineto{\pgfqpoint{1.002334in}{1.176208in}}%
\pgfpathlineto{\pgfqpoint{1.003032in}{1.175647in}}%
\pgfpathlineto{\pgfqpoint{1.012702in}{1.162036in}}%
\pgfpathlineto{\pgfqpoint{1.013629in}{1.148425in}}%
\pgfpathlineto{\pgfqpoint{1.005577in}{1.134814in}}%
\pgfpathlineto{\pgfqpoint{1.002334in}{1.131995in}}%
\pgfpathlineto{\pgfqpoint{0.986678in}{1.124995in}}%
\pgfpathlineto{\pgfqpoint{0.971021in}{1.125801in}}%
\pgfpathlineto{\pgfqpoint{0.955364in}{1.134208in}}%
\pgfpathlineto{\pgfqpoint{0.954720in}{1.134814in}}%
\pgfpathclose%
\pgfpathmoveto{\pgfqpoint{1.284152in}{1.093131in}}%
\pgfpathlineto{\pgfqpoint{1.299809in}{1.093662in}}%
\pgfpathlineto{\pgfqpoint{1.300685in}{1.093981in}}%
\pgfpathlineto{\pgfqpoint{1.315466in}{1.099513in}}%
\pgfpathlineto{\pgfqpoint{1.328287in}{1.107592in}}%
\pgfpathlineto{\pgfqpoint{1.331122in}{1.109608in}}%
\pgfpathlineto{\pgfqpoint{1.344019in}{1.121203in}}%
\pgfpathlineto{\pgfqpoint{1.346779in}{1.124464in}}%
\pgfpathlineto{\pgfqpoint{1.354709in}{1.134814in}}%
\pgfpathlineto{\pgfqpoint{1.360219in}{1.148425in}}%
\pgfpathlineto{\pgfqpoint{1.359585in}{1.162036in}}%
\pgfpathlineto{\pgfqpoint{1.352968in}{1.175647in}}%
\pgfpathlineto{\pgfqpoint{1.346779in}{1.183116in}}%
\pgfpathlineto{\pgfqpoint{1.341178in}{1.189259in}}%
\pgfpathlineto{\pgfqpoint{1.331122in}{1.198000in}}%
\pgfpathlineto{\pgfqpoint{1.324056in}{1.202870in}}%
\pgfpathlineto{\pgfqpoint{1.315466in}{1.208250in}}%
\pgfpathlineto{\pgfqpoint{1.299809in}{1.214003in}}%
\pgfpathlineto{\pgfqpoint{1.284152in}{1.214555in}}%
\pgfpathlineto{\pgfqpoint{1.268496in}{1.209764in}}%
\pgfpathlineto{\pgfqpoint{1.256590in}{1.202870in}}%
\pgfpathlineto{\pgfqpoint{1.252839in}{1.200471in}}%
\pgfpathlineto{\pgfqpoint{1.239501in}{1.189259in}}%
\pgfpathlineto{\pgfqpoint{1.237183in}{1.186794in}}%
\pgfpathlineto{\pgfqpoint{1.227889in}{1.175647in}}%
\pgfpathlineto{\pgfqpoint{1.221526in}{1.162798in}}%
\pgfpathlineto{\pgfqpoint{1.221160in}{1.162036in}}%
\pgfpathlineto{\pgfqpoint{1.220549in}{1.148425in}}%
\pgfpathlineto{\pgfqpoint{1.221526in}{1.145969in}}%
\pgfpathlineto{\pgfqpoint{1.226112in}{1.134814in}}%
\pgfpathlineto{\pgfqpoint{1.236632in}{1.121203in}}%
\pgfpathlineto{\pgfqpoint{1.237183in}{1.120669in}}%
\pgfpathlineto{\pgfqpoint{1.252225in}{1.107592in}}%
\pgfpathlineto{\pgfqpoint{1.252839in}{1.107113in}}%
\pgfpathlineto{\pgfqpoint{1.268496in}{1.097968in}}%
\pgfpathlineto{\pgfqpoint{1.281326in}{1.093981in}}%
\pgfpathlineto{\pgfqpoint{1.284152in}{1.093131in}}%
\pgfpathclose%
\pgfpathmoveto{\pgfqpoint{1.265253in}{1.134814in}}%
\pgfpathlineto{\pgfqpoint{1.257201in}{1.148425in}}%
\pgfpathlineto{\pgfqpoint{1.258128in}{1.162036in}}%
\pgfpathlineto{\pgfqpoint{1.267798in}{1.175647in}}%
\pgfpathlineto{\pgfqpoint{1.268496in}{1.176208in}}%
\pgfpathlineto{\pgfqpoint{1.284152in}{1.182628in}}%
\pgfpathlineto{\pgfqpoint{1.299809in}{1.181889in}}%
\pgfpathlineto{\pgfqpoint{1.312410in}{1.175647in}}%
\pgfpathlineto{\pgfqpoint{1.315466in}{1.172991in}}%
\pgfpathlineto{\pgfqpoint{1.322645in}{1.162036in}}%
\pgfpathlineto{\pgfqpoint{1.323496in}{1.148425in}}%
\pgfpathlineto{\pgfqpoint{1.316110in}{1.134814in}}%
\pgfpathlineto{\pgfqpoint{1.315466in}{1.134208in}}%
\pgfpathlineto{\pgfqpoint{1.299809in}{1.125801in}}%
\pgfpathlineto{\pgfqpoint{1.284152in}{1.124995in}}%
\pgfpathlineto{\pgfqpoint{1.268496in}{1.131995in}}%
\pgfpathlineto{\pgfqpoint{1.265253in}{1.134814in}}%
\pgfpathclose%
\pgfpathmoveto{\pgfqpoint{1.597284in}{1.092811in}}%
\pgfpathlineto{\pgfqpoint{1.608805in}{1.093981in}}%
\pgfpathlineto{\pgfqpoint{1.612940in}{1.094422in}}%
\pgfpathlineto{\pgfqpoint{1.628597in}{1.101215in}}%
\pgfpathlineto{\pgfqpoint{1.638061in}{1.107592in}}%
\pgfpathlineto{\pgfqpoint{1.644253in}{1.112284in}}%
\pgfpathlineto{\pgfqpoint{1.653908in}{1.121203in}}%
\pgfpathlineto{\pgfqpoint{1.659910in}{1.128479in}}%
\pgfpathlineto{\pgfqpoint{1.664757in}{1.134814in}}%
\pgfpathlineto{\pgfqpoint{1.670181in}{1.148425in}}%
\pgfpathlineto{\pgfqpoint{1.669557in}{1.162036in}}%
\pgfpathlineto{\pgfqpoint{1.663043in}{1.175647in}}%
\pgfpathlineto{\pgfqpoint{1.659910in}{1.179433in}}%
\pgfpathlineto{\pgfqpoint{1.651176in}{1.189259in}}%
\pgfpathlineto{\pgfqpoint{1.644253in}{1.195443in}}%
\pgfpathlineto{\pgfqpoint{1.634139in}{1.202870in}}%
\pgfpathlineto{\pgfqpoint{1.628597in}{1.206581in}}%
\pgfpathlineto{\pgfqpoint{1.612940in}{1.213239in}}%
\pgfpathlineto{\pgfqpoint{1.597284in}{1.214888in}}%
\pgfpathlineto{\pgfqpoint{1.581627in}{1.211109in}}%
\pgfpathlineto{\pgfqpoint{1.566028in}{1.202870in}}%
\pgfpathlineto{\pgfqpoint{1.565971in}{1.202836in}}%
\pgfpathlineto{\pgfqpoint{1.550314in}{1.190141in}}%
\pgfpathlineto{\pgfqpoint{1.549362in}{1.189259in}}%
\pgfpathlineto{\pgfqpoint{1.537839in}{1.175647in}}%
\pgfpathlineto{\pgfqpoint{1.534657in}{1.169287in}}%
\pgfpathlineto{\pgfqpoint{1.531205in}{1.162036in}}%
\pgfpathlineto{\pgfqpoint{1.530592in}{1.148425in}}%
\pgfpathlineto{\pgfqpoint{1.534657in}{1.138090in}}%
\pgfpathlineto{\pgfqpoint{1.536018in}{1.134814in}}%
\pgfpathlineto{\pgfqpoint{1.546799in}{1.121203in}}%
\pgfpathlineto{\pgfqpoint{1.550314in}{1.117833in}}%
\pgfpathlineto{\pgfqpoint{1.562551in}{1.107592in}}%
\pgfpathlineto{\pgfqpoint{1.565971in}{1.105032in}}%
\pgfpathlineto{\pgfqpoint{1.581627in}{1.096596in}}%
\pgfpathlineto{\pgfqpoint{1.592313in}{1.093981in}}%
\pgfpathlineto{\pgfqpoint{1.597284in}{1.092811in}}%
\pgfpathclose%
\pgfpathmoveto{\pgfqpoint{1.575594in}{1.134814in}}%
\pgfpathlineto{\pgfqpoint{1.566703in}{1.148425in}}%
\pgfpathlineto{\pgfqpoint{1.567727in}{1.162036in}}%
\pgfpathlineto{\pgfqpoint{1.578404in}{1.175647in}}%
\pgfpathlineto{\pgfqpoint{1.581627in}{1.178011in}}%
\pgfpathlineto{\pgfqpoint{1.597284in}{1.183075in}}%
\pgfpathlineto{\pgfqpoint{1.612940in}{1.180866in}}%
\pgfpathlineto{\pgfqpoint{1.621961in}{1.175647in}}%
\pgfpathlineto{\pgfqpoint{1.628597in}{1.168945in}}%
\pgfpathlineto{\pgfqpoint{1.632831in}{1.162036in}}%
\pgfpathlineto{\pgfqpoint{1.633619in}{1.148425in}}%
\pgfpathlineto{\pgfqpoint{1.628597in}{1.138505in}}%
\pgfpathlineto{\pgfqpoint{1.625563in}{1.134814in}}%
\pgfpathlineto{\pgfqpoint{1.612940in}{1.126917in}}%
\pgfpathlineto{\pgfqpoint{1.597284in}{1.124508in}}%
\pgfpathlineto{\pgfqpoint{1.581627in}{1.130030in}}%
\pgfpathlineto{\pgfqpoint{1.575594in}{1.134814in}}%
\pgfpathclose%
\pgfpathmoveto{\pgfqpoint{1.910415in}{1.092703in}}%
\pgfpathlineto{\pgfqpoint{1.910415in}{1.093981in}}%
\pgfpathlineto{\pgfqpoint{1.910415in}{1.107592in}}%
\pgfpathlineto{\pgfqpoint{1.910415in}{1.121203in}}%
\pgfpathlineto{\pgfqpoint{1.910415in}{1.124345in}}%
\pgfpathlineto{\pgfqpoint{1.894758in}{1.128331in}}%
\pgfpathlineto{\pgfqpoint{1.885648in}{1.134814in}}%
\pgfpathlineto{\pgfqpoint{1.879102in}{1.143829in}}%
\pgfpathlineto{\pgfqpoint{1.876904in}{1.148425in}}%
\pgfpathlineto{\pgfqpoint{1.877641in}{1.162036in}}%
\pgfpathlineto{\pgfqpoint{1.879102in}{1.164560in}}%
\pgfpathlineto{\pgfqpoint{1.888799in}{1.175647in}}%
\pgfpathlineto{\pgfqpoint{1.894758in}{1.179568in}}%
\pgfpathlineto{\pgfqpoint{1.910415in}{1.183225in}}%
\pgfpathlineto{\pgfqpoint{1.910415in}{1.189259in}}%
\pgfpathlineto{\pgfqpoint{1.910415in}{1.202870in}}%
\pgfpathlineto{\pgfqpoint{1.910415in}{1.214999in}}%
\pgfpathlineto{\pgfqpoint{1.894758in}{1.212271in}}%
\pgfpathlineto{\pgfqpoint{1.879102in}{1.204773in}}%
\pgfpathlineto{\pgfqpoint{1.876418in}{1.202870in}}%
\pgfpathlineto{\pgfqpoint{1.863445in}{1.192818in}}%
\pgfpathlineto{\pgfqpoint{1.859545in}{1.189259in}}%
\pgfpathlineto{\pgfqpoint{1.847789in}{1.175766in}}%
\pgfpathlineto{\pgfqpoint{1.847690in}{1.175647in}}%
\pgfpathlineto{\pgfqpoint{1.841251in}{1.162036in}}%
\pgfpathlineto{\pgfqpoint{1.840634in}{1.148425in}}%
\pgfpathlineto{\pgfqpoint{1.845996in}{1.134814in}}%
\pgfpathlineto{\pgfqpoint{1.847789in}{1.132477in}}%
\pgfpathlineto{\pgfqpoint{1.856907in}{1.121203in}}%
\pgfpathlineto{\pgfqpoint{1.863445in}{1.115032in}}%
\pgfpathlineto{\pgfqpoint{1.872750in}{1.107592in}}%
\pgfpathlineto{\pgfqpoint{1.879102in}{1.103061in}}%
\pgfpathlineto{\pgfqpoint{1.894758in}{1.095410in}}%
\pgfpathlineto{\pgfqpoint{1.902853in}{1.093981in}}%
\pgfpathlineto{\pgfqpoint{1.910415in}{1.092703in}}%
\pgfpathclose%
\pgfpathmoveto{\pgfqpoint{0.376072in}{1.364874in}}%
\pgfpathlineto{\pgfqpoint{0.378852in}{1.366203in}}%
\pgfpathlineto{\pgfqpoint{0.391728in}{1.372605in}}%
\pgfpathlineto{\pgfqpoint{0.401725in}{1.379814in}}%
\pgfpathlineto{\pgfqpoint{0.407385in}{1.384516in}}%
\pgfpathlineto{\pgfqpoint{0.416450in}{1.393425in}}%
\pgfpathlineto{\pgfqpoint{0.423041in}{1.402323in}}%
\pgfpathlineto{\pgfqpoint{0.426340in}{1.407036in}}%
\pgfpathlineto{\pgfqpoint{0.430569in}{1.420648in}}%
\pgfpathlineto{\pgfqpoint{0.428724in}{1.434259in}}%
\pgfpathlineto{\pgfqpoint{0.423041in}{1.444550in}}%
\pgfpathlineto{\pgfqpoint{0.421083in}{1.447870in}}%
\pgfpathlineto{\pgfqpoint{0.408553in}{1.461481in}}%
\pgfpathlineto{\pgfqpoint{0.407385in}{1.462518in}}%
\pgfpathlineto{\pgfqpoint{0.391728in}{1.474300in}}%
\pgfpathlineto{\pgfqpoint{0.390190in}{1.475092in}}%
\pgfpathlineto{\pgfqpoint{0.376072in}{1.481774in}}%
\pgfpathlineto{\pgfqpoint{0.360415in}{1.484459in}}%
\pgfpathlineto{\pgfqpoint{0.360415in}{1.475092in}}%
\pgfpathlineto{\pgfqpoint{0.360415in}{1.461481in}}%
\pgfpathlineto{\pgfqpoint{0.360415in}{1.452789in}}%
\pgfpathlineto{\pgfqpoint{0.376072in}{1.449400in}}%
\pgfpathlineto{\pgfqpoint{0.378558in}{1.447870in}}%
\pgfpathlineto{\pgfqpoint{0.391728in}{1.434910in}}%
\pgfpathlineto{\pgfqpoint{0.392168in}{1.434259in}}%
\pgfpathlineto{\pgfqpoint{0.394372in}{1.420648in}}%
\pgfpathlineto{\pgfqpoint{0.391728in}{1.413590in}}%
\pgfpathlineto{\pgfqpoint{0.387980in}{1.407036in}}%
\pgfpathlineto{\pgfqpoint{0.376072in}{1.397746in}}%
\pgfpathlineto{\pgfqpoint{0.360699in}{1.393425in}}%
\pgfpathlineto{\pgfqpoint{0.360415in}{1.393370in}}%
\pgfpathlineto{\pgfqpoint{0.360415in}{1.379814in}}%
\pgfpathlineto{\pgfqpoint{0.360415in}{1.366203in}}%
\pgfpathlineto{\pgfqpoint{0.360415in}{1.362239in}}%
\pgfpathlineto{\pgfqpoint{0.376072in}{1.364874in}}%
\pgfpathclose%
\pgfpathmoveto{\pgfqpoint{0.657890in}{1.363939in}}%
\pgfpathlineto{\pgfqpoint{0.673546in}{1.362347in}}%
\pgfpathlineto{\pgfqpoint{0.689203in}{1.365997in}}%
\pgfpathlineto{\pgfqpoint{0.689590in}{1.366203in}}%
\pgfpathlineto{\pgfqpoint{0.704859in}{1.374626in}}%
\pgfpathlineto{\pgfqpoint{0.711715in}{1.379814in}}%
\pgfpathlineto{\pgfqpoint{0.720516in}{1.387466in}}%
\pgfpathlineto{\pgfqpoint{0.726484in}{1.393425in}}%
\pgfpathlineto{\pgfqpoint{0.736173in}{1.406700in}}%
\pgfpathlineto{\pgfqpoint{0.736410in}{1.407036in}}%
\pgfpathlineto{\pgfqpoint{0.740608in}{1.420648in}}%
\pgfpathlineto{\pgfqpoint{0.738777in}{1.434259in}}%
\pgfpathlineto{\pgfqpoint{0.736173in}{1.438941in}}%
\pgfpathlineto{\pgfqpoint{0.730983in}{1.447870in}}%
\pgfpathlineto{\pgfqpoint{0.720516in}{1.459485in}}%
\pgfpathlineto{\pgfqpoint{0.718437in}{1.461481in}}%
\pgfpathlineto{\pgfqpoint{0.704859in}{1.472147in}}%
\pgfpathlineto{\pgfqpoint{0.699702in}{1.475092in}}%
\pgfpathlineto{\pgfqpoint{0.689203in}{1.480630in}}%
\pgfpathlineto{\pgfqpoint{0.673546in}{1.484349in}}%
\pgfpathlineto{\pgfqpoint{0.657890in}{1.482727in}}%
\pgfpathlineto{\pgfqpoint{0.642233in}{1.476173in}}%
\pgfpathlineto{\pgfqpoint{0.640617in}{1.475092in}}%
\pgfpathlineto{\pgfqpoint{0.626577in}{1.465042in}}%
\pgfpathlineto{\pgfqpoint{0.622481in}{1.461481in}}%
\pgfpathlineto{\pgfqpoint{0.610920in}{1.449275in}}%
\pgfpathlineto{\pgfqpoint{0.609677in}{1.447870in}}%
\pgfpathlineto{\pgfqpoint{0.602138in}{1.434259in}}%
\pgfpathlineto{\pgfqpoint{0.600271in}{1.420648in}}%
\pgfpathlineto{\pgfqpoint{0.604550in}{1.407036in}}%
\pgfpathlineto{\pgfqpoint{0.610920in}{1.397909in}}%
\pgfpathlineto{\pgfqpoint{0.614308in}{1.393425in}}%
\pgfpathlineto{\pgfqpoint{0.626577in}{1.381621in}}%
\pgfpathlineto{\pgfqpoint{0.628873in}{1.379814in}}%
\pgfpathlineto{\pgfqpoint{0.642233in}{1.370714in}}%
\pgfpathlineto{\pgfqpoint{0.652503in}{1.366203in}}%
\pgfpathlineto{\pgfqpoint{0.657890in}{1.363939in}}%
\pgfpathclose%
\pgfpathmoveto{\pgfqpoint{0.642134in}{1.407036in}}%
\pgfpathlineto{\pgfqpoint{0.636734in}{1.420648in}}%
\pgfpathlineto{\pgfqpoint{0.639090in}{1.434259in}}%
\pgfpathlineto{\pgfqpoint{0.642233in}{1.438650in}}%
\pgfpathlineto{\pgfqpoint{0.652839in}{1.447870in}}%
\pgfpathlineto{\pgfqpoint{0.657890in}{1.450602in}}%
\pgfpathlineto{\pgfqpoint{0.673546in}{1.452650in}}%
\pgfpathlineto{\pgfqpoint{0.689203in}{1.447956in}}%
\pgfpathlineto{\pgfqpoint{0.689328in}{1.447870in}}%
\pgfpathlineto{\pgfqpoint{0.701686in}{1.434259in}}%
\pgfpathlineto{\pgfqpoint{0.704746in}{1.420648in}}%
\pgfpathlineto{\pgfqpoint{0.697732in}{1.407036in}}%
\pgfpathlineto{\pgfqpoint{0.689203in}{1.399621in}}%
\pgfpathlineto{\pgfqpoint{0.673546in}{1.393524in}}%
\pgfpathlineto{\pgfqpoint{0.657890in}{1.396184in}}%
\pgfpathlineto{\pgfqpoint{0.642233in}{1.406927in}}%
\pgfpathlineto{\pgfqpoint{0.642134in}{1.407036in}}%
\pgfpathclose%
\pgfpathmoveto{\pgfqpoint{0.971021in}{1.363201in}}%
\pgfpathlineto{\pgfqpoint{0.986678in}{1.362669in}}%
\pgfpathlineto{\pgfqpoint{0.998566in}{1.366203in}}%
\pgfpathlineto{\pgfqpoint{1.002334in}{1.367386in}}%
\pgfpathlineto{\pgfqpoint{1.017991in}{1.376759in}}%
\pgfpathlineto{\pgfqpoint{1.021868in}{1.379814in}}%
\pgfpathlineto{\pgfqpoint{1.033647in}{1.390453in}}%
\pgfpathlineto{\pgfqpoint{1.036592in}{1.393425in}}%
\pgfpathlineto{\pgfqpoint{1.046296in}{1.407036in}}%
\pgfpathlineto{\pgfqpoint{1.049304in}{1.416326in}}%
\pgfpathlineto{\pgfqpoint{1.050650in}{1.420648in}}%
\pgfpathlineto{\pgfqpoint{1.049304in}{1.430663in}}%
\pgfpathlineto{\pgfqpoint{1.048796in}{1.434259in}}%
\pgfpathlineto{\pgfqpoint{1.040982in}{1.447870in}}%
\pgfpathlineto{\pgfqpoint{1.033647in}{1.456097in}}%
\pgfpathlineto{\pgfqpoint{1.028250in}{1.461481in}}%
\pgfpathlineto{\pgfqpoint{1.017991in}{1.469874in}}%
\pgfpathlineto{\pgfqpoint{1.009622in}{1.475092in}}%
\pgfpathlineto{\pgfqpoint{1.002334in}{1.479306in}}%
\pgfpathlineto{\pgfqpoint{0.986678in}{1.484021in}}%
\pgfpathlineto{\pgfqpoint{0.971021in}{1.483478in}}%
\pgfpathlineto{\pgfqpoint{0.955364in}{1.477816in}}%
\pgfpathlineto{\pgfqpoint{0.951010in}{1.475092in}}%
\pgfpathlineto{\pgfqpoint{0.939708in}{1.467499in}}%
\pgfpathlineto{\pgfqpoint{0.932594in}{1.461481in}}%
\pgfpathlineto{\pgfqpoint{0.924051in}{1.452688in}}%
\pgfpathlineto{\pgfqpoint{0.919782in}{1.447870in}}%
\pgfpathlineto{\pgfqpoint{0.912123in}{1.434259in}}%
\pgfpathlineto{\pgfqpoint{0.910227in}{1.420648in}}%
\pgfpathlineto{\pgfqpoint{0.914574in}{1.407036in}}%
\pgfpathlineto{\pgfqpoint{0.924051in}{1.393475in}}%
\pgfpathlineto{\pgfqpoint{0.924090in}{1.393425in}}%
\pgfpathlineto{\pgfqpoint{0.938693in}{1.379814in}}%
\pgfpathlineto{\pgfqpoint{0.939708in}{1.378987in}}%
\pgfpathlineto{\pgfqpoint{0.955364in}{1.368969in}}%
\pgfpathlineto{\pgfqpoint{0.962681in}{1.366203in}}%
\pgfpathlineto{\pgfqpoint{0.971021in}{1.363201in}}%
\pgfpathclose%
\pgfpathmoveto{\pgfqpoint{0.952646in}{1.407036in}}%
\pgfpathlineto{\pgfqpoint{0.946821in}{1.420648in}}%
\pgfpathlineto{\pgfqpoint{0.949362in}{1.434259in}}%
\pgfpathlineto{\pgfqpoint{0.955364in}{1.442101in}}%
\pgfpathlineto{\pgfqpoint{0.963075in}{1.447870in}}%
\pgfpathlineto{\pgfqpoint{0.971021in}{1.451551in}}%
\pgfpathlineto{\pgfqpoint{0.986678in}{1.452236in}}%
\pgfpathlineto{\pgfqpoint{0.998088in}{1.447870in}}%
\pgfpathlineto{\pgfqpoint{1.002334in}{1.445232in}}%
\pgfpathlineto{\pgfqpoint{1.011418in}{1.434259in}}%
\pgfpathlineto{\pgfqpoint{1.014189in}{1.420648in}}%
\pgfpathlineto{\pgfqpoint{1.007837in}{1.407036in}}%
\pgfpathlineto{\pgfqpoint{1.002334in}{1.401791in}}%
\pgfpathlineto{\pgfqpoint{0.986678in}{1.394062in}}%
\pgfpathlineto{\pgfqpoint{0.971021in}{1.394952in}}%
\pgfpathlineto{\pgfqpoint{0.955364in}{1.404234in}}%
\pgfpathlineto{\pgfqpoint{0.952646in}{1.407036in}}%
\pgfpathclose%
\pgfpathmoveto{\pgfqpoint{1.284152in}{1.362669in}}%
\pgfpathlineto{\pgfqpoint{1.299809in}{1.363201in}}%
\pgfpathlineto{\pgfqpoint{1.308149in}{1.366203in}}%
\pgfpathlineto{\pgfqpoint{1.315466in}{1.368969in}}%
\pgfpathlineto{\pgfqpoint{1.331122in}{1.378987in}}%
\pgfpathlineto{\pgfqpoint{1.332137in}{1.379814in}}%
\pgfpathlineto{\pgfqpoint{1.346740in}{1.393425in}}%
\pgfpathlineto{\pgfqpoint{1.346779in}{1.393475in}}%
\pgfpathlineto{\pgfqpoint{1.356256in}{1.407036in}}%
\pgfpathlineto{\pgfqpoint{1.360603in}{1.420648in}}%
\pgfpathlineto{\pgfqpoint{1.358707in}{1.434259in}}%
\pgfpathlineto{\pgfqpoint{1.351048in}{1.447870in}}%
\pgfpathlineto{\pgfqpoint{1.346779in}{1.452688in}}%
\pgfpathlineto{\pgfqpoint{1.338236in}{1.461481in}}%
\pgfpathlineto{\pgfqpoint{1.331122in}{1.467499in}}%
\pgfpathlineto{\pgfqpoint{1.319820in}{1.475092in}}%
\pgfpathlineto{\pgfqpoint{1.315466in}{1.477816in}}%
\pgfpathlineto{\pgfqpoint{1.299809in}{1.483478in}}%
\pgfpathlineto{\pgfqpoint{1.284152in}{1.484021in}}%
\pgfpathlineto{\pgfqpoint{1.268496in}{1.479306in}}%
\pgfpathlineto{\pgfqpoint{1.261208in}{1.475092in}}%
\pgfpathlineto{\pgfqpoint{1.252839in}{1.469874in}}%
\pgfpathlineto{\pgfqpoint{1.242580in}{1.461481in}}%
\pgfpathlineto{\pgfqpoint{1.237183in}{1.456097in}}%
\pgfpathlineto{\pgfqpoint{1.229848in}{1.447870in}}%
\pgfpathlineto{\pgfqpoint{1.222034in}{1.434259in}}%
\pgfpathlineto{\pgfqpoint{1.221526in}{1.430663in}}%
\pgfpathlineto{\pgfqpoint{1.220180in}{1.420648in}}%
\pgfpathlineto{\pgfqpoint{1.221526in}{1.416326in}}%
\pgfpathlineto{\pgfqpoint{1.224534in}{1.407036in}}%
\pgfpathlineto{\pgfqpoint{1.234238in}{1.393425in}}%
\pgfpathlineto{\pgfqpoint{1.237183in}{1.390453in}}%
\pgfpathlineto{\pgfqpoint{1.248962in}{1.379814in}}%
\pgfpathlineto{\pgfqpoint{1.252839in}{1.376759in}}%
\pgfpathlineto{\pgfqpoint{1.268496in}{1.367386in}}%
\pgfpathlineto{\pgfqpoint{1.272264in}{1.366203in}}%
\pgfpathlineto{\pgfqpoint{1.284152in}{1.362669in}}%
\pgfpathclose%
\pgfpathmoveto{\pgfqpoint{1.262993in}{1.407036in}}%
\pgfpathlineto{\pgfqpoint{1.256641in}{1.420648in}}%
\pgfpathlineto{\pgfqpoint{1.259412in}{1.434259in}}%
\pgfpathlineto{\pgfqpoint{1.268496in}{1.445232in}}%
\pgfpathlineto{\pgfqpoint{1.272742in}{1.447870in}}%
\pgfpathlineto{\pgfqpoint{1.284152in}{1.452236in}}%
\pgfpathlineto{\pgfqpoint{1.299809in}{1.451551in}}%
\pgfpathlineto{\pgfqpoint{1.307755in}{1.447870in}}%
\pgfpathlineto{\pgfqpoint{1.315466in}{1.442101in}}%
\pgfpathlineto{\pgfqpoint{1.321468in}{1.434259in}}%
\pgfpathlineto{\pgfqpoint{1.324009in}{1.420647in}}%
\pgfpathlineto{\pgfqpoint{1.318184in}{1.407036in}}%
\pgfpathlineto{\pgfqpoint{1.315466in}{1.404234in}}%
\pgfpathlineto{\pgfqpoint{1.299809in}{1.394952in}}%
\pgfpathlineto{\pgfqpoint{1.284152in}{1.394062in}}%
\pgfpathlineto{\pgfqpoint{1.268496in}{1.401791in}}%
\pgfpathlineto{\pgfqpoint{1.262993in}{1.407036in}}%
\pgfpathclose%
\pgfpathmoveto{\pgfqpoint{1.581627in}{1.365997in}}%
\pgfpathlineto{\pgfqpoint{1.597284in}{1.362347in}}%
\pgfpathlineto{\pgfqpoint{1.612940in}{1.363939in}}%
\pgfpathlineto{\pgfqpoint{1.618327in}{1.366203in}}%
\pgfpathlineto{\pgfqpoint{1.628597in}{1.370714in}}%
\pgfpathlineto{\pgfqpoint{1.641957in}{1.379814in}}%
\pgfpathlineto{\pgfqpoint{1.644253in}{1.381621in}}%
\pgfpathlineto{\pgfqpoint{1.656522in}{1.393425in}}%
\pgfpathlineto{\pgfqpoint{1.659910in}{1.397909in}}%
\pgfpathlineto{\pgfqpoint{1.666280in}{1.407036in}}%
\pgfpathlineto{\pgfqpoint{1.670559in}{1.420648in}}%
\pgfpathlineto{\pgfqpoint{1.668692in}{1.434259in}}%
\pgfpathlineto{\pgfqpoint{1.661153in}{1.447870in}}%
\pgfpathlineto{\pgfqpoint{1.659910in}{1.449275in}}%
\pgfpathlineto{\pgfqpoint{1.648349in}{1.461481in}}%
\pgfpathlineto{\pgfqpoint{1.644253in}{1.465042in}}%
\pgfpathlineto{\pgfqpoint{1.630213in}{1.475092in}}%
\pgfpathlineto{\pgfqpoint{1.628597in}{1.476173in}}%
\pgfpathlineto{\pgfqpoint{1.612940in}{1.482727in}}%
\pgfpathlineto{\pgfqpoint{1.597284in}{1.484349in}}%
\pgfpathlineto{\pgfqpoint{1.581627in}{1.480630in}}%
\pgfpathlineto{\pgfqpoint{1.571128in}{1.475092in}}%
\pgfpathlineto{\pgfqpoint{1.565971in}{1.472147in}}%
\pgfpathlineto{\pgfqpoint{1.552393in}{1.461481in}}%
\pgfpathlineto{\pgfqpoint{1.550314in}{1.459485in}}%
\pgfpathlineto{\pgfqpoint{1.539847in}{1.447870in}}%
\pgfpathlineto{\pgfqpoint{1.534657in}{1.438941in}}%
\pgfpathlineto{\pgfqpoint{1.532053in}{1.434259in}}%
\pgfpathlineto{\pgfqpoint{1.530222in}{1.420648in}}%
\pgfpathlineto{\pgfqpoint{1.534420in}{1.407036in}}%
\pgfpathlineto{\pgfqpoint{1.534657in}{1.406700in}}%
\pgfpathlineto{\pgfqpoint{1.544346in}{1.393425in}}%
\pgfpathlineto{\pgfqpoint{1.550314in}{1.387466in}}%
\pgfpathlineto{\pgfqpoint{1.559115in}{1.379814in}}%
\pgfpathlineto{\pgfqpoint{1.565971in}{1.374626in}}%
\pgfpathlineto{\pgfqpoint{1.581240in}{1.366203in}}%
\pgfpathlineto{\pgfqpoint{1.581627in}{1.365997in}}%
\pgfpathclose%
\pgfpathmoveto{\pgfqpoint{1.573098in}{1.407036in}}%
\pgfpathlineto{\pgfqpoint{1.566084in}{1.420648in}}%
\pgfpathlineto{\pgfqpoint{1.569144in}{1.434259in}}%
\pgfpathlineto{\pgfqpoint{1.581502in}{1.447870in}}%
\pgfpathlineto{\pgfqpoint{1.581627in}{1.447956in}}%
\pgfpathlineto{\pgfqpoint{1.597284in}{1.452650in}}%
\pgfpathlineto{\pgfqpoint{1.612940in}{1.450602in}}%
\pgfpathlineto{\pgfqpoint{1.617991in}{1.447870in}}%
\pgfpathlineto{\pgfqpoint{1.628597in}{1.438650in}}%
\pgfpathlineto{\pgfqpoint{1.631740in}{1.434259in}}%
\pgfpathlineto{\pgfqpoint{1.634096in}{1.420648in}}%
\pgfpathlineto{\pgfqpoint{1.628696in}{1.407036in}}%
\pgfpathlineto{\pgfqpoint{1.628597in}{1.406927in}}%
\pgfpathlineto{\pgfqpoint{1.612940in}{1.396184in}}%
\pgfpathlineto{\pgfqpoint{1.597284in}{1.393524in}}%
\pgfpathlineto{\pgfqpoint{1.581627in}{1.399621in}}%
\pgfpathlineto{\pgfqpoint{1.573098in}{1.407036in}}%
\pgfpathclose%
\pgfpathmoveto{\pgfqpoint{1.894758in}{1.364874in}}%
\pgfpathlineto{\pgfqpoint{1.910415in}{1.362239in}}%
\pgfpathlineto{\pgfqpoint{1.910415in}{1.366203in}}%
\pgfpathlineto{\pgfqpoint{1.910415in}{1.379814in}}%
\pgfpathlineto{\pgfqpoint{1.910415in}{1.393370in}}%
\pgfpathlineto{\pgfqpoint{1.910131in}{1.393425in}}%
\pgfpathlineto{\pgfqpoint{1.894758in}{1.397746in}}%
\pgfpathlineto{\pgfqpoint{1.882850in}{1.407036in}}%
\pgfpathlineto{\pgfqpoint{1.879102in}{1.413590in}}%
\pgfpathlineto{\pgfqpoint{1.876458in}{1.420648in}}%
\pgfpathlineto{\pgfqpoint{1.878662in}{1.434259in}}%
\pgfpathlineto{\pgfqpoint{1.879102in}{1.434910in}}%
\pgfpathlineto{\pgfqpoint{1.892272in}{1.447870in}}%
\pgfpathlineto{\pgfqpoint{1.894758in}{1.449400in}}%
\pgfpathlineto{\pgfqpoint{1.910415in}{1.452789in}}%
\pgfpathlineto{\pgfqpoint{1.910415in}{1.461481in}}%
\pgfpathlineto{\pgfqpoint{1.910415in}{1.475092in}}%
\pgfpathlineto{\pgfqpoint{1.910415in}{1.484459in}}%
\pgfpathlineto{\pgfqpoint{1.894758in}{1.481774in}}%
\pgfpathlineto{\pgfqpoint{1.880640in}{1.475092in}}%
\pgfpathlineto{\pgfqpoint{1.879102in}{1.474300in}}%
\pgfpathlineto{\pgfqpoint{1.863445in}{1.462518in}}%
\pgfpathlineto{\pgfqpoint{1.862277in}{1.461481in}}%
\pgfpathlineto{\pgfqpoint{1.849747in}{1.447870in}}%
\pgfpathlineto{\pgfqpoint{1.847789in}{1.444550in}}%
\pgfpathlineto{\pgfqpoint{1.842106in}{1.434259in}}%
\pgfpathlineto{\pgfqpoint{1.840261in}{1.420648in}}%
\pgfpathlineto{\pgfqpoint{1.844490in}{1.407036in}}%
\pgfpathlineto{\pgfqpoint{1.847789in}{1.402323in}}%
\pgfpathlineto{\pgfqpoint{1.854380in}{1.393425in}}%
\pgfpathlineto{\pgfqpoint{1.863445in}{1.384516in}}%
\pgfpathlineto{\pgfqpoint{1.869105in}{1.379814in}}%
\pgfpathlineto{\pgfqpoint{1.879102in}{1.372605in}}%
\pgfpathlineto{\pgfqpoint{1.891978in}{1.366203in}}%
\pgfpathlineto{\pgfqpoint{1.894758in}{1.364874in}}%
\pgfpathclose%
\pgfpathmoveto{\pgfqpoint{0.376072in}{1.634427in}}%
\pgfpathlineto{\pgfqpoint{0.384496in}{1.638425in}}%
\pgfpathlineto{\pgfqpoint{0.391728in}{1.642075in}}%
\pgfpathlineto{\pgfqpoint{0.405324in}{1.652036in}}%
\pgfpathlineto{\pgfqpoint{0.407385in}{1.653828in}}%
\pgfpathlineto{\pgfqpoint{0.418843in}{1.665648in}}%
\pgfpathlineto{\pgfqpoint{0.423041in}{1.671935in}}%
\pgfpathlineto{\pgfqpoint{0.427641in}{1.679259in}}%
\pgfpathlineto{\pgfqpoint{0.430694in}{1.692870in}}%
\pgfpathlineto{\pgfqpoint{0.423041in}{1.692870in}}%
\pgfpathlineto{\pgfqpoint{0.407385in}{1.692870in}}%
\pgfpathlineto{\pgfqpoint{0.394521in}{1.692870in}}%
\pgfpathlineto{\pgfqpoint{0.391728in}{1.682480in}}%
\pgfpathlineto{\pgfqpoint{0.390399in}{1.679259in}}%
\pgfpathlineto{\pgfqpoint{0.376072in}{1.666803in}}%
\pgfpathlineto{\pgfqpoint{0.372366in}{1.665648in}}%
\pgfpathlineto{\pgfqpoint{0.360415in}{1.663220in}}%
\pgfpathlineto{\pgfqpoint{0.360415in}{1.652036in}}%
\pgfpathlineto{\pgfqpoint{0.360415in}{1.638425in}}%
\pgfpathlineto{\pgfqpoint{0.360415in}{1.631772in}}%
\pgfpathlineto{\pgfqpoint{0.376072in}{1.634427in}}%
\pgfpathclose%
\pgfpathmoveto{\pgfqpoint{0.657890in}{1.633485in}}%
\pgfpathlineto{\pgfqpoint{0.673546in}{1.631881in}}%
\pgfpathlineto{\pgfqpoint{0.689203in}{1.635558in}}%
\pgfpathlineto{\pgfqpoint{0.694624in}{1.638425in}}%
\pgfpathlineto{\pgfqpoint{0.704859in}{1.644156in}}%
\pgfpathlineto{\pgfqpoint{0.715108in}{1.652036in}}%
\pgfpathlineto{\pgfqpoint{0.720516in}{1.656957in}}%
\pgfpathlineto{\pgfqpoint{0.728808in}{1.665648in}}%
\pgfpathlineto{\pgfqpoint{0.736173in}{1.676842in}}%
\pgfpathlineto{\pgfqpoint{0.737701in}{1.679259in}}%
\pgfpathlineto{\pgfqpoint{0.740732in}{1.692870in}}%
\pgfpathlineto{\pgfqpoint{0.736173in}{1.692870in}}%
\pgfpathlineto{\pgfqpoint{0.720516in}{1.692870in}}%
\pgfpathlineto{\pgfqpoint{0.704923in}{1.692870in}}%
\pgfpathlineto{\pgfqpoint{0.704859in}{1.692623in}}%
\pgfpathlineto{\pgfqpoint{0.699889in}{1.679259in}}%
\pgfpathlineto{\pgfqpoint{0.689203in}{1.668906in}}%
\pgfpathlineto{\pgfqpoint{0.681664in}{1.665648in}}%
\pgfpathlineto{\pgfqpoint{0.673546in}{1.663349in}}%
\pgfpathlineto{\pgfqpoint{0.657890in}{1.665265in}}%
\pgfpathlineto{\pgfqpoint{0.657141in}{1.665648in}}%
\pgfpathlineto{\pgfqpoint{0.642233in}{1.677097in}}%
\pgfpathlineto{\pgfqpoint{0.640473in}{1.679259in}}%
\pgfpathlineto{\pgfqpoint{0.636575in}{1.692870in}}%
\pgfpathlineto{\pgfqpoint{0.626577in}{1.692870in}}%
\pgfpathlineto{\pgfqpoint{0.610920in}{1.692870in}}%
\pgfpathlineto{\pgfqpoint{0.600145in}{1.692870in}}%
\pgfpathlineto{\pgfqpoint{0.603234in}{1.679259in}}%
\pgfpathlineto{\pgfqpoint{0.610920in}{1.666985in}}%
\pgfpathlineto{\pgfqpoint{0.611831in}{1.665648in}}%
\pgfpathlineto{\pgfqpoint{0.625383in}{1.652036in}}%
\pgfpathlineto{\pgfqpoint{0.626577in}{1.651021in}}%
\pgfpathlineto{\pgfqpoint{0.642233in}{1.640128in}}%
\pgfpathlineto{\pgfqpoint{0.646052in}{1.638425in}}%
\pgfpathlineto{\pgfqpoint{0.657890in}{1.633485in}}%
\pgfpathclose%
\pgfpathmoveto{\pgfqpoint{0.955364in}{1.638340in}}%
\pgfpathlineto{\pgfqpoint{0.971021in}{1.632742in}}%
\pgfpathlineto{\pgfqpoint{0.986678in}{1.632205in}}%
\pgfpathlineto{\pgfqpoint{1.002334in}{1.636866in}}%
\pgfpathlineto{\pgfqpoint{1.005022in}{1.638425in}}%
\pgfpathlineto{\pgfqpoint{1.017991in}{1.646352in}}%
\pgfpathlineto{\pgfqpoint{1.025089in}{1.652036in}}%
\pgfpathlineto{\pgfqpoint{1.033647in}{1.660125in}}%
\pgfpathlineto{\pgfqpoint{1.038860in}{1.665647in}}%
\pgfpathlineto{\pgfqpoint{1.047660in}{1.679259in}}%
\pgfpathlineto{\pgfqpoint{1.049304in}{1.686296in}}%
\pgfpathlineto{\pgfqpoint{1.050774in}{1.692870in}}%
\pgfpathlineto{\pgfqpoint{1.049304in}{1.692870in}}%
\pgfpathlineto{\pgfqpoint{1.033647in}{1.692870in}}%
\pgfpathlineto{\pgfqpoint{1.017991in}{1.692870in}}%
\pgfpathlineto{\pgfqpoint{1.014377in}{1.692870in}}%
\pgfpathlineto{\pgfqpoint{1.009791in}{1.679259in}}%
\pgfpathlineto{\pgfqpoint{1.002334in}{1.671338in}}%
\pgfpathlineto{\pgfqpoint{0.991965in}{1.665647in}}%
\pgfpathlineto{\pgfqpoint{0.986678in}{1.663737in}}%
\pgfpathlineto{\pgfqpoint{0.971021in}{1.664377in}}%
\pgfpathlineto{\pgfqpoint{0.968118in}{1.665648in}}%
\pgfpathlineto{\pgfqpoint{0.955364in}{1.674078in}}%
\pgfpathlineto{\pgfqpoint{0.950854in}{1.679259in}}%
\pgfpathlineto{\pgfqpoint{0.946649in}{1.692870in}}%
\pgfpathlineto{\pgfqpoint{0.939708in}{1.692870in}}%
\pgfpathlineto{\pgfqpoint{0.924051in}{1.692870in}}%
\pgfpathlineto{\pgfqpoint{0.910099in}{1.692870in}}%
\pgfpathlineto{\pgfqpoint{0.913237in}{1.679259in}}%
\pgfpathlineto{\pgfqpoint{0.921862in}{1.665648in}}%
\pgfpathlineto{\pgfqpoint{0.924051in}{1.663314in}}%
\pgfpathlineto{\pgfqpoint{0.935614in}{1.652036in}}%
\pgfpathlineto{\pgfqpoint{0.939708in}{1.648646in}}%
\pgfpathlineto{\pgfqpoint{0.955228in}{1.638425in}}%
\pgfpathlineto{\pgfqpoint{0.955364in}{1.638340in}}%
\pgfpathclose%
\pgfpathmoveto{\pgfqpoint{1.268496in}{1.636866in}}%
\pgfpathlineto{\pgfqpoint{1.284152in}{1.632205in}}%
\pgfpathlineto{\pgfqpoint{1.299809in}{1.632742in}}%
\pgfpathlineto{\pgfqpoint{1.315466in}{1.638340in}}%
\pgfpathlineto{\pgfqpoint{1.315602in}{1.638425in}}%
\pgfpathlineto{\pgfqpoint{1.331122in}{1.648646in}}%
\pgfpathlineto{\pgfqpoint{1.335216in}{1.652036in}}%
\pgfpathlineto{\pgfqpoint{1.346779in}{1.663314in}}%
\pgfpathlineto{\pgfqpoint{1.348968in}{1.665647in}}%
\pgfpathlineto{\pgfqpoint{1.357593in}{1.679259in}}%
\pgfpathlineto{\pgfqpoint{1.360731in}{1.692870in}}%
\pgfpathlineto{\pgfqpoint{1.346779in}{1.692870in}}%
\pgfpathlineto{\pgfqpoint{1.331122in}{1.692870in}}%
\pgfpathlineto{\pgfqpoint{1.324181in}{1.692870in}}%
\pgfpathlineto{\pgfqpoint{1.319976in}{1.679259in}}%
\pgfpathlineto{\pgfqpoint{1.315466in}{1.674078in}}%
\pgfpathlineto{\pgfqpoint{1.302712in}{1.665648in}}%
\pgfpathlineto{\pgfqpoint{1.299809in}{1.664377in}}%
\pgfpathlineto{\pgfqpoint{1.284152in}{1.663737in}}%
\pgfpathlineto{\pgfqpoint{1.278865in}{1.665648in}}%
\pgfpathlineto{\pgfqpoint{1.268496in}{1.671338in}}%
\pgfpathlineto{\pgfqpoint{1.261039in}{1.679259in}}%
\pgfpathlineto{\pgfqpoint{1.256453in}{1.692870in}}%
\pgfpathlineto{\pgfqpoint{1.252839in}{1.692870in}}%
\pgfpathlineto{\pgfqpoint{1.237183in}{1.692870in}}%
\pgfpathlineto{\pgfqpoint{1.221526in}{1.692870in}}%
\pgfpathlineto{\pgfqpoint{1.220056in}{1.692870in}}%
\pgfpathlineto{\pgfqpoint{1.221526in}{1.686296in}}%
\pgfpathlineto{\pgfqpoint{1.223170in}{1.679259in}}%
\pgfpathlineto{\pgfqpoint{1.231970in}{1.665648in}}%
\pgfpathlineto{\pgfqpoint{1.237183in}{1.660125in}}%
\pgfpathlineto{\pgfqpoint{1.245741in}{1.652036in}}%
\pgfpathlineto{\pgfqpoint{1.252839in}{1.646352in}}%
\pgfpathlineto{\pgfqpoint{1.265808in}{1.638425in}}%
\pgfpathlineto{\pgfqpoint{1.268496in}{1.636866in}}%
\pgfpathclose%
\pgfpathmoveto{\pgfqpoint{1.581627in}{1.635558in}}%
\pgfpathlineto{\pgfqpoint{1.597284in}{1.631881in}}%
\pgfpathlineto{\pgfqpoint{1.612940in}{1.633485in}}%
\pgfpathlineto{\pgfqpoint{1.624778in}{1.638425in}}%
\pgfpathlineto{\pgfqpoint{1.628597in}{1.640128in}}%
\pgfpathlineto{\pgfqpoint{1.644253in}{1.651021in}}%
\pgfpathlineto{\pgfqpoint{1.645447in}{1.652036in}}%
\pgfpathlineto{\pgfqpoint{1.658999in}{1.665648in}}%
\pgfpathlineto{\pgfqpoint{1.659910in}{1.666985in}}%
\pgfpathlineto{\pgfqpoint{1.667596in}{1.679259in}}%
\pgfpathlineto{\pgfqpoint{1.670685in}{1.692870in}}%
\pgfpathlineto{\pgfqpoint{1.659910in}{1.692870in}}%
\pgfpathlineto{\pgfqpoint{1.644253in}{1.692870in}}%
\pgfpathlineto{\pgfqpoint{1.634255in}{1.692870in}}%
\pgfpathlineto{\pgfqpoint{1.630357in}{1.679259in}}%
\pgfpathlineto{\pgfqpoint{1.628597in}{1.677097in}}%
\pgfpathlineto{\pgfqpoint{1.613689in}{1.665648in}}%
\pgfpathlineto{\pgfqpoint{1.612940in}{1.665265in}}%
\pgfpathlineto{\pgfqpoint{1.597284in}{1.663349in}}%
\pgfpathlineto{\pgfqpoint{1.589166in}{1.665648in}}%
\pgfpathlineto{\pgfqpoint{1.581627in}{1.668906in}}%
\pgfpathlineto{\pgfqpoint{1.570941in}{1.679259in}}%
\pgfpathlineto{\pgfqpoint{1.565971in}{1.692623in}}%
\pgfpathlineto{\pgfqpoint{1.565907in}{1.692870in}}%
\pgfpathlineto{\pgfqpoint{1.550314in}{1.692870in}}%
\pgfpathlineto{\pgfqpoint{1.534657in}{1.692870in}}%
\pgfpathlineto{\pgfqpoint{1.530098in}{1.692870in}}%
\pgfpathlineto{\pgfqpoint{1.533129in}{1.679259in}}%
\pgfpathlineto{\pgfqpoint{1.534657in}{1.676842in}}%
\pgfpathlineto{\pgfqpoint{1.542022in}{1.665648in}}%
\pgfpathlineto{\pgfqpoint{1.550314in}{1.656957in}}%
\pgfpathlineto{\pgfqpoint{1.555722in}{1.652036in}}%
\pgfpathlineto{\pgfqpoint{1.565971in}{1.644156in}}%
\pgfpathlineto{\pgfqpoint{1.576206in}{1.638425in}}%
\pgfpathlineto{\pgfqpoint{1.581627in}{1.635558in}}%
\pgfpathclose%
\pgfpathmoveto{\pgfqpoint{1.894758in}{1.634427in}}%
\pgfpathlineto{\pgfqpoint{1.910415in}{1.631772in}}%
\pgfpathlineto{\pgfqpoint{1.910415in}{1.638425in}}%
\pgfpathlineto{\pgfqpoint{1.910415in}{1.652036in}}%
\pgfpathlineto{\pgfqpoint{1.910415in}{1.663220in}}%
\pgfpathlineto{\pgfqpoint{1.898464in}{1.665648in}}%
\pgfpathlineto{\pgfqpoint{1.894758in}{1.666803in}}%
\pgfpathlineto{\pgfqpoint{1.880431in}{1.679259in}}%
\pgfpathlineto{\pgfqpoint{1.879102in}{1.682480in}}%
\pgfpathlineto{\pgfqpoint{1.876309in}{1.692870in}}%
\pgfpathlineto{\pgfqpoint{1.863445in}{1.692870in}}%
\pgfpathlineto{\pgfqpoint{1.847789in}{1.692870in}}%
\pgfpathlineto{\pgfqpoint{1.840136in}{1.692870in}}%
\pgfpathlineto{\pgfqpoint{1.843189in}{1.679259in}}%
\pgfpathlineto{\pgfqpoint{1.847789in}{1.671935in}}%
\pgfpathlineto{\pgfqpoint{1.851987in}{1.665648in}}%
\pgfpathlineto{\pgfqpoint{1.863445in}{1.653828in}}%
\pgfpathlineto{\pgfqpoint{1.865506in}{1.652036in}}%
\pgfpathlineto{\pgfqpoint{1.879102in}{1.642075in}}%
\pgfpathlineto{\pgfqpoint{1.886334in}{1.638425in}}%
\pgfpathlineto{\pgfqpoint{1.894758in}{1.634427in}}%
\pgfpathclose%
\pgfusepath{fill}%
\end{pgfscope}%
\begin{pgfscope}%
\pgfpathrectangle{\pgfqpoint{0.360415in}{0.345370in}}{\pgfqpoint{1.550000in}{1.347500in}}%
\pgfusepath{clip}%
\pgfsetbuttcap%
\pgfsetroundjoin%
\definecolor{currentfill}{rgb}{0.178950,0.019252,0.584054}%
\pgfsetfillcolor{currentfill}%
\pgfsetlinewidth{0.000000pt}%
\definecolor{currentstroke}{rgb}{0.000000,0.000000,0.000000}%
\pgfsetstrokecolor{currentstroke}%
\pgfsetdash{}{0pt}%
\pgfpathmoveto{\pgfqpoint{0.376072in}{0.345370in}}%
\pgfpathlineto{\pgfqpoint{0.391728in}{0.345370in}}%
\pgfpathlineto{\pgfqpoint{0.394521in}{0.345370in}}%
\pgfpathlineto{\pgfqpoint{0.391728in}{0.355760in}}%
\pgfpathlineto{\pgfqpoint{0.390399in}{0.358981in}}%
\pgfpathlineto{\pgfqpoint{0.376072in}{0.371437in}}%
\pgfpathlineto{\pgfqpoint{0.372366in}{0.372592in}}%
\pgfpathlineto{\pgfqpoint{0.360415in}{0.375020in}}%
\pgfpathlineto{\pgfqpoint{0.360415in}{0.372592in}}%
\pgfpathlineto{\pgfqpoint{0.360415in}{0.358981in}}%
\pgfpathlineto{\pgfqpoint{0.360415in}{0.345370in}}%
\pgfpathlineto{\pgfqpoint{0.376072in}{0.345370in}}%
\pgfpathclose%
\pgfpathmoveto{\pgfqpoint{0.642233in}{0.345370in}}%
\pgfpathlineto{\pgfqpoint{0.657890in}{0.345370in}}%
\pgfpathlineto{\pgfqpoint{0.673546in}{0.345370in}}%
\pgfpathlineto{\pgfqpoint{0.689203in}{0.345370in}}%
\pgfpathlineto{\pgfqpoint{0.704859in}{0.345370in}}%
\pgfpathlineto{\pgfqpoint{0.704923in}{0.345370in}}%
\pgfpathlineto{\pgfqpoint{0.704859in}{0.345616in}}%
\pgfpathlineto{\pgfqpoint{0.699889in}{0.358981in}}%
\pgfpathlineto{\pgfqpoint{0.689203in}{0.369334in}}%
\pgfpathlineto{\pgfqpoint{0.681664in}{0.372592in}}%
\pgfpathlineto{\pgfqpoint{0.673546in}{0.374890in}}%
\pgfpathlineto{\pgfqpoint{0.657890in}{0.372975in}}%
\pgfpathlineto{\pgfqpoint{0.657141in}{0.372592in}}%
\pgfpathlineto{\pgfqpoint{0.642233in}{0.361143in}}%
\pgfpathlineto{\pgfqpoint{0.640473in}{0.358981in}}%
\pgfpathlineto{\pgfqpoint{0.636575in}{0.345370in}}%
\pgfpathlineto{\pgfqpoint{0.642233in}{0.345370in}}%
\pgfpathclose%
\pgfpathmoveto{\pgfqpoint{0.955364in}{0.345370in}}%
\pgfpathlineto{\pgfqpoint{0.971021in}{0.345370in}}%
\pgfpathlineto{\pgfqpoint{0.986678in}{0.345370in}}%
\pgfpathlineto{\pgfqpoint{1.002334in}{0.345370in}}%
\pgfpathlineto{\pgfqpoint{1.014377in}{0.345370in}}%
\pgfpathlineto{\pgfqpoint{1.009791in}{0.358981in}}%
\pgfpathlineto{\pgfqpoint{1.002334in}{0.366901in}}%
\pgfpathlineto{\pgfqpoint{0.991965in}{0.372592in}}%
\pgfpathlineto{\pgfqpoint{0.986678in}{0.374503in}}%
\pgfpathlineto{\pgfqpoint{0.971021in}{0.373862in}}%
\pgfpathlineto{\pgfqpoint{0.968118in}{0.372592in}}%
\pgfpathlineto{\pgfqpoint{0.955364in}{0.364162in}}%
\pgfpathlineto{\pgfqpoint{0.950854in}{0.358981in}}%
\pgfpathlineto{\pgfqpoint{0.946649in}{0.345370in}}%
\pgfpathlineto{\pgfqpoint{0.955364in}{0.345370in}}%
\pgfpathclose%
\pgfpathmoveto{\pgfqpoint{1.268496in}{0.345370in}}%
\pgfpathlineto{\pgfqpoint{1.284152in}{0.345370in}}%
\pgfpathlineto{\pgfqpoint{1.299809in}{0.345370in}}%
\pgfpathlineto{\pgfqpoint{1.315466in}{0.345370in}}%
\pgfpathlineto{\pgfqpoint{1.324181in}{0.345370in}}%
\pgfpathlineto{\pgfqpoint{1.319976in}{0.358981in}}%
\pgfpathlineto{\pgfqpoint{1.315466in}{0.364162in}}%
\pgfpathlineto{\pgfqpoint{1.302712in}{0.372592in}}%
\pgfpathlineto{\pgfqpoint{1.299809in}{0.373862in}}%
\pgfpathlineto{\pgfqpoint{1.284152in}{0.374503in}}%
\pgfpathlineto{\pgfqpoint{1.278865in}{0.372592in}}%
\pgfpathlineto{\pgfqpoint{1.268496in}{0.366901in}}%
\pgfpathlineto{\pgfqpoint{1.261039in}{0.358981in}}%
\pgfpathlineto{\pgfqpoint{1.256453in}{0.345370in}}%
\pgfpathlineto{\pgfqpoint{1.268496in}{0.345370in}}%
\pgfpathclose%
\pgfpathmoveto{\pgfqpoint{1.565971in}{0.345370in}}%
\pgfpathlineto{\pgfqpoint{1.581627in}{0.345370in}}%
\pgfpathlineto{\pgfqpoint{1.597284in}{0.345370in}}%
\pgfpathlineto{\pgfqpoint{1.612940in}{0.345370in}}%
\pgfpathlineto{\pgfqpoint{1.628597in}{0.345370in}}%
\pgfpathlineto{\pgfqpoint{1.634255in}{0.345370in}}%
\pgfpathlineto{\pgfqpoint{1.630357in}{0.358981in}}%
\pgfpathlineto{\pgfqpoint{1.628597in}{0.361143in}}%
\pgfpathlineto{\pgfqpoint{1.613689in}{0.372592in}}%
\pgfpathlineto{\pgfqpoint{1.612940in}{0.372975in}}%
\pgfpathlineto{\pgfqpoint{1.597284in}{0.374890in}}%
\pgfpathlineto{\pgfqpoint{1.589166in}{0.372592in}}%
\pgfpathlineto{\pgfqpoint{1.581627in}{0.369334in}}%
\pgfpathlineto{\pgfqpoint{1.570941in}{0.358981in}}%
\pgfpathlineto{\pgfqpoint{1.565971in}{0.345616in}}%
\pgfpathlineto{\pgfqpoint{1.565907in}{0.345370in}}%
\pgfpathlineto{\pgfqpoint{1.565971in}{0.345370in}}%
\pgfpathclose%
\pgfpathmoveto{\pgfqpoint{1.879102in}{0.345370in}}%
\pgfpathlineto{\pgfqpoint{1.894758in}{0.345370in}}%
\pgfpathlineto{\pgfqpoint{1.910415in}{0.345370in}}%
\pgfpathlineto{\pgfqpoint{1.910415in}{0.358981in}}%
\pgfpathlineto{\pgfqpoint{1.910415in}{0.372592in}}%
\pgfpathlineto{\pgfqpoint{1.910415in}{0.375020in}}%
\pgfpathlineto{\pgfqpoint{1.898464in}{0.372592in}}%
\pgfpathlineto{\pgfqpoint{1.894758in}{0.371437in}}%
\pgfpathlineto{\pgfqpoint{1.880431in}{0.358981in}}%
\pgfpathlineto{\pgfqpoint{1.879102in}{0.355760in}}%
\pgfpathlineto{\pgfqpoint{1.876309in}{0.345370in}}%
\pgfpathlineto{\pgfqpoint{1.879102in}{0.345370in}}%
\pgfpathclose%
\pgfpathmoveto{\pgfqpoint{0.376072in}{0.588840in}}%
\pgfpathlineto{\pgfqpoint{0.378558in}{0.590370in}}%
\pgfpathlineto{\pgfqpoint{0.391728in}{0.603330in}}%
\pgfpathlineto{\pgfqpoint{0.392168in}{0.603981in}}%
\pgfpathlineto{\pgfqpoint{0.394372in}{0.617592in}}%
\pgfpathlineto{\pgfqpoint{0.391728in}{0.624649in}}%
\pgfpathlineto{\pgfqpoint{0.387980in}{0.631203in}}%
\pgfpathlineto{\pgfqpoint{0.376072in}{0.640493in}}%
\pgfpathlineto{\pgfqpoint{0.360699in}{0.644814in}}%
\pgfpathlineto{\pgfqpoint{0.360415in}{0.644869in}}%
\pgfpathlineto{\pgfqpoint{0.360415in}{0.644814in}}%
\pgfpathlineto{\pgfqpoint{0.360415in}{0.631203in}}%
\pgfpathlineto{\pgfqpoint{0.360415in}{0.617592in}}%
\pgfpathlineto{\pgfqpoint{0.360415in}{0.603981in}}%
\pgfpathlineto{\pgfqpoint{0.360415in}{0.590370in}}%
\pgfpathlineto{\pgfqpoint{0.360415in}{0.585451in}}%
\pgfpathlineto{\pgfqpoint{0.376072in}{0.588840in}}%
\pgfpathclose%
\pgfpathmoveto{\pgfqpoint{0.657890in}{0.587637in}}%
\pgfpathlineto{\pgfqpoint{0.673546in}{0.585589in}}%
\pgfpathlineto{\pgfqpoint{0.689203in}{0.590284in}}%
\pgfpathlineto{\pgfqpoint{0.689328in}{0.590370in}}%
\pgfpathlineto{\pgfqpoint{0.701686in}{0.603981in}}%
\pgfpathlineto{\pgfqpoint{0.704746in}{0.617592in}}%
\pgfpathlineto{\pgfqpoint{0.697732in}{0.631203in}}%
\pgfpathlineto{\pgfqpoint{0.689203in}{0.638618in}}%
\pgfpathlineto{\pgfqpoint{0.673546in}{0.644715in}}%
\pgfpathlineto{\pgfqpoint{0.657890in}{0.642055in}}%
\pgfpathlineto{\pgfqpoint{0.642233in}{0.631312in}}%
\pgfpathlineto{\pgfqpoint{0.642134in}{0.631203in}}%
\pgfpathlineto{\pgfqpoint{0.636734in}{0.617592in}}%
\pgfpathlineto{\pgfqpoint{0.639090in}{0.603981in}}%
\pgfpathlineto{\pgfqpoint{0.642233in}{0.599590in}}%
\pgfpathlineto{\pgfqpoint{0.652839in}{0.590370in}}%
\pgfpathlineto{\pgfqpoint{0.657890in}{0.587637in}}%
\pgfpathclose%
\pgfpathmoveto{\pgfqpoint{0.971021in}{0.586688in}}%
\pgfpathlineto{\pgfqpoint{0.986678in}{0.586003in}}%
\pgfpathlineto{\pgfqpoint{0.998088in}{0.590370in}}%
\pgfpathlineto{\pgfqpoint{1.002334in}{0.593007in}}%
\pgfpathlineto{\pgfqpoint{1.011418in}{0.603981in}}%
\pgfpathlineto{\pgfqpoint{1.014189in}{0.617592in}}%
\pgfpathlineto{\pgfqpoint{1.007837in}{0.631203in}}%
\pgfpathlineto{\pgfqpoint{1.002334in}{0.636448in}}%
\pgfpathlineto{\pgfqpoint{0.986678in}{0.644177in}}%
\pgfpathlineto{\pgfqpoint{0.971021in}{0.643288in}}%
\pgfpathlineto{\pgfqpoint{0.955364in}{0.634005in}}%
\pgfpathlineto{\pgfqpoint{0.952646in}{0.631203in}}%
\pgfpathlineto{\pgfqpoint{0.946821in}{0.617592in}}%
\pgfpathlineto{\pgfqpoint{0.949362in}{0.603981in}}%
\pgfpathlineto{\pgfqpoint{0.955364in}{0.596138in}}%
\pgfpathlineto{\pgfqpoint{0.963075in}{0.590370in}}%
\pgfpathlineto{\pgfqpoint{0.971021in}{0.586688in}}%
\pgfpathclose%
\pgfpathmoveto{\pgfqpoint{1.284152in}{0.586003in}}%
\pgfpathlineto{\pgfqpoint{1.299809in}{0.586688in}}%
\pgfpathlineto{\pgfqpoint{1.307755in}{0.590370in}}%
\pgfpathlineto{\pgfqpoint{1.315466in}{0.596138in}}%
\pgfpathlineto{\pgfqpoint{1.321468in}{0.603981in}}%
\pgfpathlineto{\pgfqpoint{1.324009in}{0.617592in}}%
\pgfpathlineto{\pgfqpoint{1.318184in}{0.631203in}}%
\pgfpathlineto{\pgfqpoint{1.315466in}{0.634005in}}%
\pgfpathlineto{\pgfqpoint{1.299809in}{0.643288in}}%
\pgfpathlineto{\pgfqpoint{1.284152in}{0.644177in}}%
\pgfpathlineto{\pgfqpoint{1.268496in}{0.636448in}}%
\pgfpathlineto{\pgfqpoint{1.262993in}{0.631203in}}%
\pgfpathlineto{\pgfqpoint{1.256641in}{0.617592in}}%
\pgfpathlineto{\pgfqpoint{1.259412in}{0.603981in}}%
\pgfpathlineto{\pgfqpoint{1.268496in}{0.593007in}}%
\pgfpathlineto{\pgfqpoint{1.272742in}{0.590370in}}%
\pgfpathlineto{\pgfqpoint{1.284152in}{0.586003in}}%
\pgfpathclose%
\pgfpathmoveto{\pgfqpoint{1.581627in}{0.590284in}}%
\pgfpathlineto{\pgfqpoint{1.597284in}{0.585589in}}%
\pgfpathlineto{\pgfqpoint{1.612940in}{0.587637in}}%
\pgfpathlineto{\pgfqpoint{1.617991in}{0.590370in}}%
\pgfpathlineto{\pgfqpoint{1.628597in}{0.599590in}}%
\pgfpathlineto{\pgfqpoint{1.631740in}{0.603981in}}%
\pgfpathlineto{\pgfqpoint{1.634096in}{0.617592in}}%
\pgfpathlineto{\pgfqpoint{1.628696in}{0.631203in}}%
\pgfpathlineto{\pgfqpoint{1.628597in}{0.631312in}}%
\pgfpathlineto{\pgfqpoint{1.612940in}{0.642055in}}%
\pgfpathlineto{\pgfqpoint{1.597284in}{0.644715in}}%
\pgfpathlineto{\pgfqpoint{1.581627in}{0.638618in}}%
\pgfpathlineto{\pgfqpoint{1.573098in}{0.631203in}}%
\pgfpathlineto{\pgfqpoint{1.566084in}{0.617592in}}%
\pgfpathlineto{\pgfqpoint{1.569144in}{0.603981in}}%
\pgfpathlineto{\pgfqpoint{1.581502in}{0.590370in}}%
\pgfpathlineto{\pgfqpoint{1.581627in}{0.590284in}}%
\pgfpathclose%
\pgfpathmoveto{\pgfqpoint{1.894758in}{0.588840in}}%
\pgfpathlineto{\pgfqpoint{1.910415in}{0.585451in}}%
\pgfpathlineto{\pgfqpoint{1.910415in}{0.590370in}}%
\pgfpathlineto{\pgfqpoint{1.910415in}{0.603981in}}%
\pgfpathlineto{\pgfqpoint{1.910415in}{0.617592in}}%
\pgfpathlineto{\pgfqpoint{1.910415in}{0.631203in}}%
\pgfpathlineto{\pgfqpoint{1.910415in}{0.644814in}}%
\pgfpathlineto{\pgfqpoint{1.910415in}{0.644869in}}%
\pgfpathlineto{\pgfqpoint{1.910131in}{0.644814in}}%
\pgfpathlineto{\pgfqpoint{1.894758in}{0.640493in}}%
\pgfpathlineto{\pgfqpoint{1.882850in}{0.631203in}}%
\pgfpathlineto{\pgfqpoint{1.879102in}{0.624649in}}%
\pgfpathlineto{\pgfqpoint{1.876458in}{0.617592in}}%
\pgfpathlineto{\pgfqpoint{1.878662in}{0.603981in}}%
\pgfpathlineto{\pgfqpoint{1.879102in}{0.603330in}}%
\pgfpathlineto{\pgfqpoint{1.892272in}{0.590370in}}%
\pgfpathlineto{\pgfqpoint{1.894758in}{0.588840in}}%
\pgfpathclose%
\pgfpathmoveto{\pgfqpoint{0.376072in}{0.858671in}}%
\pgfpathlineto{\pgfqpoint{0.382031in}{0.862592in}}%
\pgfpathlineto{\pgfqpoint{0.391728in}{0.873680in}}%
\pgfpathlineto{\pgfqpoint{0.393189in}{0.876203in}}%
\pgfpathlineto{\pgfqpoint{0.393926in}{0.889814in}}%
\pgfpathlineto{\pgfqpoint{0.391728in}{0.894410in}}%
\pgfpathlineto{\pgfqpoint{0.385182in}{0.903425in}}%
\pgfpathlineto{\pgfqpoint{0.376072in}{0.909908in}}%
\pgfpathlineto{\pgfqpoint{0.360415in}{0.913895in}}%
\pgfpathlineto{\pgfqpoint{0.360415in}{0.903425in}}%
\pgfpathlineto{\pgfqpoint{0.360415in}{0.889814in}}%
\pgfpathlineto{\pgfqpoint{0.360415in}{0.876203in}}%
\pgfpathlineto{\pgfqpoint{0.360415in}{0.862592in}}%
\pgfpathlineto{\pgfqpoint{0.360415in}{0.855015in}}%
\pgfpathlineto{\pgfqpoint{0.376072in}{0.858671in}}%
\pgfpathclose%
\pgfpathmoveto{\pgfqpoint{0.657890in}{0.857374in}}%
\pgfpathlineto{\pgfqpoint{0.673546in}{0.855164in}}%
\pgfpathlineto{\pgfqpoint{0.689203in}{0.860229in}}%
\pgfpathlineto{\pgfqpoint{0.692426in}{0.862592in}}%
\pgfpathlineto{\pgfqpoint{0.703103in}{0.876203in}}%
\pgfpathlineto{\pgfqpoint{0.704127in}{0.889814in}}%
\pgfpathlineto{\pgfqpoint{0.695236in}{0.903425in}}%
\pgfpathlineto{\pgfqpoint{0.689203in}{0.908210in}}%
\pgfpathlineto{\pgfqpoint{0.673546in}{0.913732in}}%
\pgfpathlineto{\pgfqpoint{0.657890in}{0.911323in}}%
\pgfpathlineto{\pgfqpoint{0.645267in}{0.903425in}}%
\pgfpathlineto{\pgfqpoint{0.642233in}{0.899734in}}%
\pgfpathlineto{\pgfqpoint{0.637211in}{0.889814in}}%
\pgfpathlineto{\pgfqpoint{0.637999in}{0.876203in}}%
\pgfpathlineto{\pgfqpoint{0.642233in}{0.869295in}}%
\pgfpathlineto{\pgfqpoint{0.648869in}{0.862592in}}%
\pgfpathlineto{\pgfqpoint{0.657890in}{0.857374in}}%
\pgfpathclose%
\pgfpathmoveto{\pgfqpoint{0.971021in}{0.856350in}}%
\pgfpathlineto{\pgfqpoint{0.986678in}{0.855611in}}%
\pgfpathlineto{\pgfqpoint{1.002334in}{0.862031in}}%
\pgfpathlineto{\pgfqpoint{1.003032in}{0.862592in}}%
\pgfpathlineto{\pgfqpoint{1.012702in}{0.876203in}}%
\pgfpathlineto{\pgfqpoint{1.013629in}{0.889814in}}%
\pgfpathlineto{\pgfqpoint{1.005577in}{0.903425in}}%
\pgfpathlineto{\pgfqpoint{1.002334in}{0.906244in}}%
\pgfpathlineto{\pgfqpoint{0.986678in}{0.913245in}}%
\pgfpathlineto{\pgfqpoint{0.971021in}{0.912439in}}%
\pgfpathlineto{\pgfqpoint{0.955364in}{0.904032in}}%
\pgfpathlineto{\pgfqpoint{0.954720in}{0.903425in}}%
\pgfpathlineto{\pgfqpoint{0.947334in}{0.889814in}}%
\pgfpathlineto{\pgfqpoint{0.948185in}{0.876203in}}%
\pgfpathlineto{\pgfqpoint{0.955364in}{0.865248in}}%
\pgfpathlineto{\pgfqpoint{0.958420in}{0.862592in}}%
\pgfpathlineto{\pgfqpoint{0.971021in}{0.856350in}}%
\pgfpathclose%
\pgfpathmoveto{\pgfqpoint{1.268496in}{0.862031in}}%
\pgfpathlineto{\pgfqpoint{1.284152in}{0.855611in}}%
\pgfpathlineto{\pgfqpoint{1.299809in}{0.856350in}}%
\pgfpathlineto{\pgfqpoint{1.312410in}{0.862592in}}%
\pgfpathlineto{\pgfqpoint{1.315466in}{0.865248in}}%
\pgfpathlineto{\pgfqpoint{1.322645in}{0.876203in}}%
\pgfpathlineto{\pgfqpoint{1.323496in}{0.889814in}}%
\pgfpathlineto{\pgfqpoint{1.316110in}{0.903425in}}%
\pgfpathlineto{\pgfqpoint{1.315466in}{0.904032in}}%
\pgfpathlineto{\pgfqpoint{1.299809in}{0.912439in}}%
\pgfpathlineto{\pgfqpoint{1.284152in}{0.913245in}}%
\pgfpathlineto{\pgfqpoint{1.268496in}{0.906244in}}%
\pgfpathlineto{\pgfqpoint{1.265253in}{0.903425in}}%
\pgfpathlineto{\pgfqpoint{1.257201in}{0.889814in}}%
\pgfpathlineto{\pgfqpoint{1.258128in}{0.876203in}}%
\pgfpathlineto{\pgfqpoint{1.267798in}{0.862592in}}%
\pgfpathlineto{\pgfqpoint{1.268496in}{0.862031in}}%
\pgfpathclose%
\pgfpathmoveto{\pgfqpoint{1.581627in}{0.860229in}}%
\pgfpathlineto{\pgfqpoint{1.597284in}{0.855164in}}%
\pgfpathlineto{\pgfqpoint{1.612940in}{0.857374in}}%
\pgfpathlineto{\pgfqpoint{1.621961in}{0.862592in}}%
\pgfpathlineto{\pgfqpoint{1.628597in}{0.869295in}}%
\pgfpathlineto{\pgfqpoint{1.632831in}{0.876203in}}%
\pgfpathlineto{\pgfqpoint{1.633619in}{0.889814in}}%
\pgfpathlineto{\pgfqpoint{1.628597in}{0.899734in}}%
\pgfpathlineto{\pgfqpoint{1.625563in}{0.903425in}}%
\pgfpathlineto{\pgfqpoint{1.612940in}{0.911323in}}%
\pgfpathlineto{\pgfqpoint{1.597284in}{0.913732in}}%
\pgfpathlineto{\pgfqpoint{1.581627in}{0.908210in}}%
\pgfpathlineto{\pgfqpoint{1.575594in}{0.903425in}}%
\pgfpathlineto{\pgfqpoint{1.566703in}{0.889814in}}%
\pgfpathlineto{\pgfqpoint{1.567727in}{0.876203in}}%
\pgfpathlineto{\pgfqpoint{1.578404in}{0.862592in}}%
\pgfpathlineto{\pgfqpoint{1.581627in}{0.860229in}}%
\pgfpathclose%
\pgfpathmoveto{\pgfqpoint{1.894758in}{0.858671in}}%
\pgfpathlineto{\pgfqpoint{1.910415in}{0.855015in}}%
\pgfpathlineto{\pgfqpoint{1.910415in}{0.862592in}}%
\pgfpathlineto{\pgfqpoint{1.910415in}{0.876203in}}%
\pgfpathlineto{\pgfqpoint{1.910415in}{0.889814in}}%
\pgfpathlineto{\pgfqpoint{1.910415in}{0.903425in}}%
\pgfpathlineto{\pgfqpoint{1.910415in}{0.913895in}}%
\pgfpathlineto{\pgfqpoint{1.894758in}{0.909908in}}%
\pgfpathlineto{\pgfqpoint{1.885648in}{0.903425in}}%
\pgfpathlineto{\pgfqpoint{1.879102in}{0.894410in}}%
\pgfpathlineto{\pgfqpoint{1.876904in}{0.889814in}}%
\pgfpathlineto{\pgfqpoint{1.877641in}{0.876203in}}%
\pgfpathlineto{\pgfqpoint{1.879102in}{0.873680in}}%
\pgfpathlineto{\pgfqpoint{1.888799in}{0.862592in}}%
\pgfpathlineto{\pgfqpoint{1.894758in}{0.858671in}}%
\pgfpathclose%
\pgfpathmoveto{\pgfqpoint{0.376072in}{1.128331in}}%
\pgfpathlineto{\pgfqpoint{0.385182in}{1.134814in}}%
\pgfpathlineto{\pgfqpoint{0.391728in}{1.143829in}}%
\pgfpathlineto{\pgfqpoint{0.393926in}{1.148425in}}%
\pgfpathlineto{\pgfqpoint{0.393189in}{1.162036in}}%
\pgfpathlineto{\pgfqpoint{0.391728in}{1.164560in}}%
\pgfpathlineto{\pgfqpoint{0.382031in}{1.175647in}}%
\pgfpathlineto{\pgfqpoint{0.376072in}{1.179568in}}%
\pgfpathlineto{\pgfqpoint{0.360415in}{1.183225in}}%
\pgfpathlineto{\pgfqpoint{0.360415in}{1.175647in}}%
\pgfpathlineto{\pgfqpoint{0.360415in}{1.162036in}}%
\pgfpathlineto{\pgfqpoint{0.360415in}{1.148425in}}%
\pgfpathlineto{\pgfqpoint{0.360415in}{1.134814in}}%
\pgfpathlineto{\pgfqpoint{0.360415in}{1.124345in}}%
\pgfpathlineto{\pgfqpoint{0.376072in}{1.128331in}}%
\pgfpathclose%
\pgfpathmoveto{\pgfqpoint{0.657890in}{1.126917in}}%
\pgfpathlineto{\pgfqpoint{0.673546in}{1.124508in}}%
\pgfpathlineto{\pgfqpoint{0.689203in}{1.130030in}}%
\pgfpathlineto{\pgfqpoint{0.695236in}{1.134814in}}%
\pgfpathlineto{\pgfqpoint{0.704127in}{1.148425in}}%
\pgfpathlineto{\pgfqpoint{0.703103in}{1.162036in}}%
\pgfpathlineto{\pgfqpoint{0.692426in}{1.175647in}}%
\pgfpathlineto{\pgfqpoint{0.689203in}{1.178011in}}%
\pgfpathlineto{\pgfqpoint{0.673546in}{1.183075in}}%
\pgfpathlineto{\pgfqpoint{0.657890in}{1.180866in}}%
\pgfpathlineto{\pgfqpoint{0.648869in}{1.175647in}}%
\pgfpathlineto{\pgfqpoint{0.642233in}{1.168945in}}%
\pgfpathlineto{\pgfqpoint{0.637999in}{1.162036in}}%
\pgfpathlineto{\pgfqpoint{0.637211in}{1.148425in}}%
\pgfpathlineto{\pgfqpoint{0.642233in}{1.138505in}}%
\pgfpathlineto{\pgfqpoint{0.645267in}{1.134814in}}%
\pgfpathlineto{\pgfqpoint{0.657890in}{1.126917in}}%
\pgfpathclose%
\pgfpathmoveto{\pgfqpoint{0.955364in}{1.134208in}}%
\pgfpathlineto{\pgfqpoint{0.971021in}{1.125801in}}%
\pgfpathlineto{\pgfqpoint{0.986678in}{1.124995in}}%
\pgfpathlineto{\pgfqpoint{1.002334in}{1.131995in}}%
\pgfpathlineto{\pgfqpoint{1.005577in}{1.134814in}}%
\pgfpathlineto{\pgfqpoint{1.013629in}{1.148425in}}%
\pgfpathlineto{\pgfqpoint{1.012702in}{1.162036in}}%
\pgfpathlineto{\pgfqpoint{1.003032in}{1.175647in}}%
\pgfpathlineto{\pgfqpoint{1.002334in}{1.176208in}}%
\pgfpathlineto{\pgfqpoint{0.986678in}{1.182628in}}%
\pgfpathlineto{\pgfqpoint{0.971021in}{1.181889in}}%
\pgfpathlineto{\pgfqpoint{0.958420in}{1.175647in}}%
\pgfpathlineto{\pgfqpoint{0.955364in}{1.172991in}}%
\pgfpathlineto{\pgfqpoint{0.948185in}{1.162036in}}%
\pgfpathlineto{\pgfqpoint{0.947334in}{1.148425in}}%
\pgfpathlineto{\pgfqpoint{0.954720in}{1.134814in}}%
\pgfpathlineto{\pgfqpoint{0.955364in}{1.134208in}}%
\pgfpathclose%
\pgfpathmoveto{\pgfqpoint{1.268496in}{1.131995in}}%
\pgfpathlineto{\pgfqpoint{1.284152in}{1.124995in}}%
\pgfpathlineto{\pgfqpoint{1.299809in}{1.125801in}}%
\pgfpathlineto{\pgfqpoint{1.315466in}{1.134208in}}%
\pgfpathlineto{\pgfqpoint{1.316110in}{1.134814in}}%
\pgfpathlineto{\pgfqpoint{1.323496in}{1.148425in}}%
\pgfpathlineto{\pgfqpoint{1.322645in}{1.162036in}}%
\pgfpathlineto{\pgfqpoint{1.315466in}{1.172991in}}%
\pgfpathlineto{\pgfqpoint{1.312410in}{1.175647in}}%
\pgfpathlineto{\pgfqpoint{1.299809in}{1.181889in}}%
\pgfpathlineto{\pgfqpoint{1.284152in}{1.182628in}}%
\pgfpathlineto{\pgfqpoint{1.268496in}{1.176208in}}%
\pgfpathlineto{\pgfqpoint{1.267798in}{1.175647in}}%
\pgfpathlineto{\pgfqpoint{1.258128in}{1.162036in}}%
\pgfpathlineto{\pgfqpoint{1.257201in}{1.148425in}}%
\pgfpathlineto{\pgfqpoint{1.265253in}{1.134814in}}%
\pgfpathlineto{\pgfqpoint{1.268496in}{1.131995in}}%
\pgfpathclose%
\pgfpathmoveto{\pgfqpoint{1.581627in}{1.130030in}}%
\pgfpathlineto{\pgfqpoint{1.597284in}{1.124508in}}%
\pgfpathlineto{\pgfqpoint{1.612940in}{1.126917in}}%
\pgfpathlineto{\pgfqpoint{1.625563in}{1.134814in}}%
\pgfpathlineto{\pgfqpoint{1.628597in}{1.138505in}}%
\pgfpathlineto{\pgfqpoint{1.633619in}{1.148425in}}%
\pgfpathlineto{\pgfqpoint{1.632831in}{1.162036in}}%
\pgfpathlineto{\pgfqpoint{1.628597in}{1.168945in}}%
\pgfpathlineto{\pgfqpoint{1.621961in}{1.175647in}}%
\pgfpathlineto{\pgfqpoint{1.612940in}{1.180866in}}%
\pgfpathlineto{\pgfqpoint{1.597284in}{1.183075in}}%
\pgfpathlineto{\pgfqpoint{1.581627in}{1.178011in}}%
\pgfpathlineto{\pgfqpoint{1.578404in}{1.175647in}}%
\pgfpathlineto{\pgfqpoint{1.567727in}{1.162036in}}%
\pgfpathlineto{\pgfqpoint{1.566703in}{1.148425in}}%
\pgfpathlineto{\pgfqpoint{1.575594in}{1.134814in}}%
\pgfpathlineto{\pgfqpoint{1.581627in}{1.130030in}}%
\pgfpathclose%
\pgfpathmoveto{\pgfqpoint{1.894758in}{1.128331in}}%
\pgfpathlineto{\pgfqpoint{1.910415in}{1.124345in}}%
\pgfpathlineto{\pgfqpoint{1.910415in}{1.134814in}}%
\pgfpathlineto{\pgfqpoint{1.910415in}{1.148425in}}%
\pgfpathlineto{\pgfqpoint{1.910415in}{1.162036in}}%
\pgfpathlineto{\pgfqpoint{1.910415in}{1.175647in}}%
\pgfpathlineto{\pgfqpoint{1.910415in}{1.183225in}}%
\pgfpathlineto{\pgfqpoint{1.894758in}{1.179568in}}%
\pgfpathlineto{\pgfqpoint{1.888799in}{1.175647in}}%
\pgfpathlineto{\pgfqpoint{1.879102in}{1.164560in}}%
\pgfpathlineto{\pgfqpoint{1.877641in}{1.162036in}}%
\pgfpathlineto{\pgfqpoint{1.876904in}{1.148425in}}%
\pgfpathlineto{\pgfqpoint{1.879102in}{1.143829in}}%
\pgfpathlineto{\pgfqpoint{1.885648in}{1.134814in}}%
\pgfpathlineto{\pgfqpoint{1.894758in}{1.128331in}}%
\pgfpathclose%
\pgfpathmoveto{\pgfqpoint{0.360699in}{1.393425in}}%
\pgfpathlineto{\pgfqpoint{0.376072in}{1.397746in}}%
\pgfpathlineto{\pgfqpoint{0.387980in}{1.407036in}}%
\pgfpathlineto{\pgfqpoint{0.391728in}{1.413590in}}%
\pgfpathlineto{\pgfqpoint{0.394372in}{1.420648in}}%
\pgfpathlineto{\pgfqpoint{0.392168in}{1.434259in}}%
\pgfpathlineto{\pgfqpoint{0.391728in}{1.434910in}}%
\pgfpathlineto{\pgfqpoint{0.378558in}{1.447870in}}%
\pgfpathlineto{\pgfqpoint{0.376072in}{1.449400in}}%
\pgfpathlineto{\pgfqpoint{0.360415in}{1.452789in}}%
\pgfpathlineto{\pgfqpoint{0.360415in}{1.447870in}}%
\pgfpathlineto{\pgfqpoint{0.360415in}{1.434259in}}%
\pgfpathlineto{\pgfqpoint{0.360415in}{1.420648in}}%
\pgfpathlineto{\pgfqpoint{0.360415in}{1.407036in}}%
\pgfpathlineto{\pgfqpoint{0.360415in}{1.393425in}}%
\pgfpathlineto{\pgfqpoint{0.360415in}{1.393370in}}%
\pgfpathlineto{\pgfqpoint{0.360699in}{1.393425in}}%
\pgfpathclose%
\pgfpathmoveto{\pgfqpoint{1.910415in}{1.393370in}}%
\pgfpathlineto{\pgfqpoint{1.910415in}{1.393425in}}%
\pgfpathlineto{\pgfqpoint{1.910415in}{1.407036in}}%
\pgfpathlineto{\pgfqpoint{1.910415in}{1.420648in}}%
\pgfpathlineto{\pgfqpoint{1.910415in}{1.434259in}}%
\pgfpathlineto{\pgfqpoint{1.910415in}{1.447870in}}%
\pgfpathlineto{\pgfqpoint{1.910415in}{1.452789in}}%
\pgfpathlineto{\pgfqpoint{1.894758in}{1.449400in}}%
\pgfpathlineto{\pgfqpoint{1.892272in}{1.447870in}}%
\pgfpathlineto{\pgfqpoint{1.879102in}{1.434910in}}%
\pgfpathlineto{\pgfqpoint{1.878662in}{1.434259in}}%
\pgfpathlineto{\pgfqpoint{1.876458in}{1.420648in}}%
\pgfpathlineto{\pgfqpoint{1.879102in}{1.413590in}}%
\pgfpathlineto{\pgfqpoint{1.882850in}{1.407036in}}%
\pgfpathlineto{\pgfqpoint{1.894758in}{1.397746in}}%
\pgfpathlineto{\pgfqpoint{1.910131in}{1.393425in}}%
\pgfpathlineto{\pgfqpoint{1.910415in}{1.393370in}}%
\pgfpathclose%
\pgfpathmoveto{\pgfqpoint{0.642233in}{1.406927in}}%
\pgfpathlineto{\pgfqpoint{0.657890in}{1.396184in}}%
\pgfpathlineto{\pgfqpoint{0.673546in}{1.393524in}}%
\pgfpathlineto{\pgfqpoint{0.689203in}{1.399621in}}%
\pgfpathlineto{\pgfqpoint{0.697732in}{1.407036in}}%
\pgfpathlineto{\pgfqpoint{0.704746in}{1.420648in}}%
\pgfpathlineto{\pgfqpoint{0.701686in}{1.434259in}}%
\pgfpathlineto{\pgfqpoint{0.689328in}{1.447870in}}%
\pgfpathlineto{\pgfqpoint{0.689203in}{1.447956in}}%
\pgfpathlineto{\pgfqpoint{0.673546in}{1.452650in}}%
\pgfpathlineto{\pgfqpoint{0.657890in}{1.450602in}}%
\pgfpathlineto{\pgfqpoint{0.652839in}{1.447870in}}%
\pgfpathlineto{\pgfqpoint{0.642233in}{1.438650in}}%
\pgfpathlineto{\pgfqpoint{0.639090in}{1.434259in}}%
\pgfpathlineto{\pgfqpoint{0.636734in}{1.420648in}}%
\pgfpathlineto{\pgfqpoint{0.642134in}{1.407036in}}%
\pgfpathlineto{\pgfqpoint{0.642233in}{1.406927in}}%
\pgfpathclose%
\pgfpathmoveto{\pgfqpoint{0.955364in}{1.404234in}}%
\pgfpathlineto{\pgfqpoint{0.971021in}{1.394952in}}%
\pgfpathlineto{\pgfqpoint{0.986678in}{1.394062in}}%
\pgfpathlineto{\pgfqpoint{1.002334in}{1.401791in}}%
\pgfpathlineto{\pgfqpoint{1.007837in}{1.407036in}}%
\pgfpathlineto{\pgfqpoint{1.014189in}{1.420648in}}%
\pgfpathlineto{\pgfqpoint{1.011418in}{1.434259in}}%
\pgfpathlineto{\pgfqpoint{1.002334in}{1.445232in}}%
\pgfpathlineto{\pgfqpoint{0.998088in}{1.447870in}}%
\pgfpathlineto{\pgfqpoint{0.986678in}{1.452236in}}%
\pgfpathlineto{\pgfqpoint{0.971021in}{1.451551in}}%
\pgfpathlineto{\pgfqpoint{0.963075in}{1.447870in}}%
\pgfpathlineto{\pgfqpoint{0.955364in}{1.442101in}}%
\pgfpathlineto{\pgfqpoint{0.949362in}{1.434259in}}%
\pgfpathlineto{\pgfqpoint{0.946821in}{1.420648in}}%
\pgfpathlineto{\pgfqpoint{0.952646in}{1.407036in}}%
\pgfpathlineto{\pgfqpoint{0.955364in}{1.404234in}}%
\pgfpathclose%
\pgfpathmoveto{\pgfqpoint{1.268496in}{1.401791in}}%
\pgfpathlineto{\pgfqpoint{1.284152in}{1.394062in}}%
\pgfpathlineto{\pgfqpoint{1.299809in}{1.394952in}}%
\pgfpathlineto{\pgfqpoint{1.315466in}{1.404234in}}%
\pgfpathlineto{\pgfqpoint{1.318184in}{1.407036in}}%
\pgfpathlineto{\pgfqpoint{1.324009in}{1.420648in}}%
\pgfpathlineto{\pgfqpoint{1.321468in}{1.434259in}}%
\pgfpathlineto{\pgfqpoint{1.315466in}{1.442101in}}%
\pgfpathlineto{\pgfqpoint{1.307755in}{1.447870in}}%
\pgfpathlineto{\pgfqpoint{1.299809in}{1.451551in}}%
\pgfpathlineto{\pgfqpoint{1.284152in}{1.452236in}}%
\pgfpathlineto{\pgfqpoint{1.272742in}{1.447870in}}%
\pgfpathlineto{\pgfqpoint{1.268496in}{1.445232in}}%
\pgfpathlineto{\pgfqpoint{1.259412in}{1.434259in}}%
\pgfpathlineto{\pgfqpoint{1.256641in}{1.420648in}}%
\pgfpathlineto{\pgfqpoint{1.262993in}{1.407036in}}%
\pgfpathlineto{\pgfqpoint{1.268496in}{1.401791in}}%
\pgfpathclose%
\pgfpathmoveto{\pgfqpoint{1.581627in}{1.399621in}}%
\pgfpathlineto{\pgfqpoint{1.597284in}{1.393524in}}%
\pgfpathlineto{\pgfqpoint{1.612940in}{1.396184in}}%
\pgfpathlineto{\pgfqpoint{1.628597in}{1.406927in}}%
\pgfpathlineto{\pgfqpoint{1.628696in}{1.407036in}}%
\pgfpathlineto{\pgfqpoint{1.634096in}{1.420648in}}%
\pgfpathlineto{\pgfqpoint{1.631740in}{1.434259in}}%
\pgfpathlineto{\pgfqpoint{1.628597in}{1.438650in}}%
\pgfpathlineto{\pgfqpoint{1.617991in}{1.447870in}}%
\pgfpathlineto{\pgfqpoint{1.612940in}{1.450602in}}%
\pgfpathlineto{\pgfqpoint{1.597284in}{1.452650in}}%
\pgfpathlineto{\pgfqpoint{1.581627in}{1.447956in}}%
\pgfpathlineto{\pgfqpoint{1.581502in}{1.447870in}}%
\pgfpathlineto{\pgfqpoint{1.569144in}{1.434259in}}%
\pgfpathlineto{\pgfqpoint{1.566084in}{1.420648in}}%
\pgfpathlineto{\pgfqpoint{1.573098in}{1.407036in}}%
\pgfpathlineto{\pgfqpoint{1.581627in}{1.399621in}}%
\pgfpathclose%
\pgfpathmoveto{\pgfqpoint{0.372366in}{1.665648in}}%
\pgfpathlineto{\pgfqpoint{0.376072in}{1.666803in}}%
\pgfpathlineto{\pgfqpoint{0.390399in}{1.679259in}}%
\pgfpathlineto{\pgfqpoint{0.391728in}{1.682480in}}%
\pgfpathlineto{\pgfqpoint{0.394521in}{1.692870in}}%
\pgfpathlineto{\pgfqpoint{0.391728in}{1.692870in}}%
\pgfpathlineto{\pgfqpoint{0.376072in}{1.692870in}}%
\pgfpathlineto{\pgfqpoint{0.360415in}{1.692870in}}%
\pgfpathlineto{\pgfqpoint{0.360415in}{1.679259in}}%
\pgfpathlineto{\pgfqpoint{0.360415in}{1.665648in}}%
\pgfpathlineto{\pgfqpoint{0.360415in}{1.663220in}}%
\pgfpathlineto{\pgfqpoint{0.372366in}{1.665648in}}%
\pgfpathclose%
\pgfpathmoveto{\pgfqpoint{0.657890in}{1.665265in}}%
\pgfpathlineto{\pgfqpoint{0.673546in}{1.663349in}}%
\pgfpathlineto{\pgfqpoint{0.681664in}{1.665648in}}%
\pgfpathlineto{\pgfqpoint{0.689203in}{1.668906in}}%
\pgfpathlineto{\pgfqpoint{0.699889in}{1.679259in}}%
\pgfpathlineto{\pgfqpoint{0.704859in}{1.692623in}}%
\pgfpathlineto{\pgfqpoint{0.704923in}{1.692870in}}%
\pgfpathlineto{\pgfqpoint{0.704859in}{1.692870in}}%
\pgfpathlineto{\pgfqpoint{0.689203in}{1.692870in}}%
\pgfpathlineto{\pgfqpoint{0.673546in}{1.692870in}}%
\pgfpathlineto{\pgfqpoint{0.657890in}{1.692870in}}%
\pgfpathlineto{\pgfqpoint{0.642233in}{1.692870in}}%
\pgfpathlineto{\pgfqpoint{0.636575in}{1.692870in}}%
\pgfpathlineto{\pgfqpoint{0.640473in}{1.679259in}}%
\pgfpathlineto{\pgfqpoint{0.642233in}{1.677097in}}%
\pgfpathlineto{\pgfqpoint{0.657141in}{1.665648in}}%
\pgfpathlineto{\pgfqpoint{0.657890in}{1.665265in}}%
\pgfpathclose%
\pgfpathmoveto{\pgfqpoint{0.971021in}{1.664377in}}%
\pgfpathlineto{\pgfqpoint{0.986678in}{1.663737in}}%
\pgfpathlineto{\pgfqpoint{0.991965in}{1.665648in}}%
\pgfpathlineto{\pgfqpoint{1.002334in}{1.671338in}}%
\pgfpathlineto{\pgfqpoint{1.009791in}{1.679259in}}%
\pgfpathlineto{\pgfqpoint{1.014377in}{1.692870in}}%
\pgfpathlineto{\pgfqpoint{1.002334in}{1.692870in}}%
\pgfpathlineto{\pgfqpoint{0.986678in}{1.692870in}}%
\pgfpathlineto{\pgfqpoint{0.971021in}{1.692870in}}%
\pgfpathlineto{\pgfqpoint{0.955364in}{1.692870in}}%
\pgfpathlineto{\pgfqpoint{0.946649in}{1.692870in}}%
\pgfpathlineto{\pgfqpoint{0.950854in}{1.679259in}}%
\pgfpathlineto{\pgfqpoint{0.955364in}{1.674078in}}%
\pgfpathlineto{\pgfqpoint{0.968118in}{1.665648in}}%
\pgfpathlineto{\pgfqpoint{0.971021in}{1.664377in}}%
\pgfpathclose%
\pgfpathmoveto{\pgfqpoint{1.284152in}{1.663737in}}%
\pgfpathlineto{\pgfqpoint{1.299809in}{1.664377in}}%
\pgfpathlineto{\pgfqpoint{1.302712in}{1.665648in}}%
\pgfpathlineto{\pgfqpoint{1.315466in}{1.674078in}}%
\pgfpathlineto{\pgfqpoint{1.319976in}{1.679259in}}%
\pgfpathlineto{\pgfqpoint{1.324181in}{1.692870in}}%
\pgfpathlineto{\pgfqpoint{1.315466in}{1.692870in}}%
\pgfpathlineto{\pgfqpoint{1.299809in}{1.692870in}}%
\pgfpathlineto{\pgfqpoint{1.284152in}{1.692870in}}%
\pgfpathlineto{\pgfqpoint{1.268496in}{1.692870in}}%
\pgfpathlineto{\pgfqpoint{1.256453in}{1.692870in}}%
\pgfpathlineto{\pgfqpoint{1.261039in}{1.679259in}}%
\pgfpathlineto{\pgfqpoint{1.268496in}{1.671338in}}%
\pgfpathlineto{\pgfqpoint{1.278865in}{1.665648in}}%
\pgfpathlineto{\pgfqpoint{1.284152in}{1.663737in}}%
\pgfpathclose%
\pgfpathmoveto{\pgfqpoint{1.597284in}{1.663349in}}%
\pgfpathlineto{\pgfqpoint{1.612940in}{1.665265in}}%
\pgfpathlineto{\pgfqpoint{1.613689in}{1.665648in}}%
\pgfpathlineto{\pgfqpoint{1.628597in}{1.677097in}}%
\pgfpathlineto{\pgfqpoint{1.630357in}{1.679259in}}%
\pgfpathlineto{\pgfqpoint{1.634255in}{1.692870in}}%
\pgfpathlineto{\pgfqpoint{1.628597in}{1.692870in}}%
\pgfpathlineto{\pgfqpoint{1.612940in}{1.692870in}}%
\pgfpathlineto{\pgfqpoint{1.597284in}{1.692870in}}%
\pgfpathlineto{\pgfqpoint{1.581627in}{1.692870in}}%
\pgfpathlineto{\pgfqpoint{1.565971in}{1.692870in}}%
\pgfpathlineto{\pgfqpoint{1.565907in}{1.692870in}}%
\pgfpathlineto{\pgfqpoint{1.565971in}{1.692623in}}%
\pgfpathlineto{\pgfqpoint{1.570941in}{1.679259in}}%
\pgfpathlineto{\pgfqpoint{1.581627in}{1.668906in}}%
\pgfpathlineto{\pgfqpoint{1.589166in}{1.665648in}}%
\pgfpathlineto{\pgfqpoint{1.597284in}{1.663349in}}%
\pgfpathclose%
\pgfpathmoveto{\pgfqpoint{1.910415in}{1.663220in}}%
\pgfpathlineto{\pgfqpoint{1.910415in}{1.665648in}}%
\pgfpathlineto{\pgfqpoint{1.910415in}{1.679259in}}%
\pgfpathlineto{\pgfqpoint{1.910415in}{1.692870in}}%
\pgfpathlineto{\pgfqpoint{1.894758in}{1.692870in}}%
\pgfpathlineto{\pgfqpoint{1.879102in}{1.692870in}}%
\pgfpathlineto{\pgfqpoint{1.876309in}{1.692870in}}%
\pgfpathlineto{\pgfqpoint{1.879102in}{1.682480in}}%
\pgfpathlineto{\pgfqpoint{1.880431in}{1.679259in}}%
\pgfpathlineto{\pgfqpoint{1.894758in}{1.666803in}}%
\pgfpathlineto{\pgfqpoint{1.898464in}{1.665648in}}%
\pgfpathlineto{\pgfqpoint{1.910415in}{1.663220in}}%
\pgfpathclose%
\pgfusepath{fill}%
\end{pgfscope}%
\begin{pgfscope}%
\pgfsetbuttcap%
\pgfsetroundjoin%
\definecolor{currentfill}{rgb}{0.000000,0.000000,0.000000}%
\pgfsetfillcolor{currentfill}%
\pgfsetlinewidth{0.803000pt}%
\definecolor{currentstroke}{rgb}{0.000000,0.000000,0.000000}%
\pgfsetstrokecolor{currentstroke}%
\pgfsetdash{}{0pt}%
\pgfsys@defobject{currentmarker}{\pgfqpoint{0.000000in}{-0.048611in}}{\pgfqpoint{0.000000in}{0.000000in}}{%
\pgfpathmoveto{\pgfqpoint{0.000000in}{0.000000in}}%
\pgfpathlineto{\pgfqpoint{0.000000in}{-0.048611in}}%
\pgfusepath{stroke,fill}%
}%
\begin{pgfscope}%
\pgfsys@transformshift{0.360415in}{0.345370in}%
\pgfsys@useobject{currentmarker}{}%
\end{pgfscope}%
\end{pgfscope}%
\begin{pgfscope}%
\definecolor{textcolor}{rgb}{0.000000,0.000000,0.000000}%
\pgfsetstrokecolor{textcolor}%
\pgfsetfillcolor{textcolor}%
\pgftext[x=0.360415in,y=0.248148in,,top]{\color{textcolor}{\rmfamily\fontsize{12.000000}{14.400000}\selectfont\catcode`\^=\active\def^{\ifmmode\sp\else\^{}\fi}\catcode`\%=\active\def%{\%}$\mathdefault{0}$}}%
\end{pgfscope}%
\begin{pgfscope}%
\pgfsetbuttcap%
\pgfsetroundjoin%
\definecolor{currentfill}{rgb}{0.000000,0.000000,0.000000}%
\pgfsetfillcolor{currentfill}%
\pgfsetlinewidth{0.803000pt}%
\definecolor{currentstroke}{rgb}{0.000000,0.000000,0.000000}%
\pgfsetstrokecolor{currentstroke}%
\pgfsetdash{}{0pt}%
\pgfsys@defobject{currentmarker}{\pgfqpoint{0.000000in}{-0.048611in}}{\pgfqpoint{0.000000in}{0.000000in}}{%
\pgfpathmoveto{\pgfqpoint{0.000000in}{0.000000in}}%
\pgfpathlineto{\pgfqpoint{0.000000in}{-0.048611in}}%
\pgfusepath{stroke,fill}%
}%
\begin{pgfscope}%
\pgfsys@transformshift{0.877082in}{0.345370in}%
\pgfsys@useobject{currentmarker}{}%
\end{pgfscope}%
\end{pgfscope}%
\begin{pgfscope}%
\definecolor{textcolor}{rgb}{0.000000,0.000000,0.000000}%
\pgfsetstrokecolor{textcolor}%
\pgfsetfillcolor{textcolor}%
\pgftext[x=0.877082in,y=0.248148in,,top]{\color{textcolor}{\rmfamily\fontsize{12.000000}{14.400000}\selectfont\catcode`\^=\active\def^{\ifmmode\sp\else\^{}\fi}\catcode`\%=\active\def%{\%}$\mathdefault{10}$}}%
\end{pgfscope}%
\begin{pgfscope}%
\pgfsetbuttcap%
\pgfsetroundjoin%
\definecolor{currentfill}{rgb}{0.000000,0.000000,0.000000}%
\pgfsetfillcolor{currentfill}%
\pgfsetlinewidth{0.803000pt}%
\definecolor{currentstroke}{rgb}{0.000000,0.000000,0.000000}%
\pgfsetstrokecolor{currentstroke}%
\pgfsetdash{}{0pt}%
\pgfsys@defobject{currentmarker}{\pgfqpoint{0.000000in}{-0.048611in}}{\pgfqpoint{0.000000in}{0.000000in}}{%
\pgfpathmoveto{\pgfqpoint{0.000000in}{0.000000in}}%
\pgfpathlineto{\pgfqpoint{0.000000in}{-0.048611in}}%
\pgfusepath{stroke,fill}%
}%
\begin{pgfscope}%
\pgfsys@transformshift{1.393748in}{0.345370in}%
\pgfsys@useobject{currentmarker}{}%
\end{pgfscope}%
\end{pgfscope}%
\begin{pgfscope}%
\definecolor{textcolor}{rgb}{0.000000,0.000000,0.000000}%
\pgfsetstrokecolor{textcolor}%
\pgfsetfillcolor{textcolor}%
\pgftext[x=1.393748in,y=0.248148in,,top]{\color{textcolor}{\rmfamily\fontsize{12.000000}{14.400000}\selectfont\catcode`\^=\active\def^{\ifmmode\sp\else\^{}\fi}\catcode`\%=\active\def%{\%}$\mathdefault{20}$}}%
\end{pgfscope}%
\begin{pgfscope}%
\pgfsetbuttcap%
\pgfsetroundjoin%
\definecolor{currentfill}{rgb}{0.000000,0.000000,0.000000}%
\pgfsetfillcolor{currentfill}%
\pgfsetlinewidth{0.803000pt}%
\definecolor{currentstroke}{rgb}{0.000000,0.000000,0.000000}%
\pgfsetstrokecolor{currentstroke}%
\pgfsetdash{}{0pt}%
\pgfsys@defobject{currentmarker}{\pgfqpoint{0.000000in}{-0.048611in}}{\pgfqpoint{0.000000in}{0.000000in}}{%
\pgfpathmoveto{\pgfqpoint{0.000000in}{0.000000in}}%
\pgfpathlineto{\pgfqpoint{0.000000in}{-0.048611in}}%
\pgfusepath{stroke,fill}%
}%
\begin{pgfscope}%
\pgfsys@transformshift{1.910415in}{0.345370in}%
\pgfsys@useobject{currentmarker}{}%
\end{pgfscope}%
\end{pgfscope}%
\begin{pgfscope}%
\definecolor{textcolor}{rgb}{0.000000,0.000000,0.000000}%
\pgfsetstrokecolor{textcolor}%
\pgfsetfillcolor{textcolor}%
\pgftext[x=1.910415in,y=0.248148in,,top]{\color{textcolor}{\rmfamily\fontsize{12.000000}{14.400000}\selectfont\catcode`\^=\active\def^{\ifmmode\sp\else\^{}\fi}\catcode`\%=\active\def%{\%}$\mathdefault{30}$}}%
\end{pgfscope}%
\begin{pgfscope}%
\pgfsetbuttcap%
\pgfsetroundjoin%
\definecolor{currentfill}{rgb}{0.000000,0.000000,0.000000}%
\pgfsetfillcolor{currentfill}%
\pgfsetlinewidth{0.803000pt}%
\definecolor{currentstroke}{rgb}{0.000000,0.000000,0.000000}%
\pgfsetstrokecolor{currentstroke}%
\pgfsetdash{}{0pt}%
\pgfsys@defobject{currentmarker}{\pgfqpoint{-0.048611in}{0.000000in}}{\pgfqpoint{-0.000000in}{0.000000in}}{%
\pgfpathmoveto{\pgfqpoint{-0.000000in}{0.000000in}}%
\pgfpathlineto{\pgfqpoint{-0.048611in}{0.000000in}}%
\pgfusepath{stroke,fill}%
}%
\begin{pgfscope}%
\pgfsys@transformshift{0.360415in}{0.345370in}%
\pgfsys@useobject{currentmarker}{}%
\end{pgfscope}%
\end{pgfscope}%
\begin{pgfscope}%
\definecolor{textcolor}{rgb}{0.000000,0.000000,0.000000}%
\pgfsetstrokecolor{textcolor}%
\pgfsetfillcolor{textcolor}%
\pgftext[x=0.181596in, y=0.287500in, left, base]{\color{textcolor}{\rmfamily\fontsize{12.000000}{14.400000}\selectfont\catcode`\^=\active\def^{\ifmmode\sp\else\^{}\fi}\catcode`\%=\active\def%{\%}$\mathdefault{0}$}}%
\end{pgfscope}%
\begin{pgfscope}%
\pgfsetbuttcap%
\pgfsetroundjoin%
\definecolor{currentfill}{rgb}{0.000000,0.000000,0.000000}%
\pgfsetfillcolor{currentfill}%
\pgfsetlinewidth{0.803000pt}%
\definecolor{currentstroke}{rgb}{0.000000,0.000000,0.000000}%
\pgfsetstrokecolor{currentstroke}%
\pgfsetdash{}{0pt}%
\pgfsys@defobject{currentmarker}{\pgfqpoint{-0.048611in}{0.000000in}}{\pgfqpoint{-0.000000in}{0.000000in}}{%
\pgfpathmoveto{\pgfqpoint{-0.000000in}{0.000000in}}%
\pgfpathlineto{\pgfqpoint{-0.048611in}{0.000000in}}%
\pgfusepath{stroke,fill}%
}%
\begin{pgfscope}%
\pgfsys@transformshift{0.360415in}{0.794536in}%
\pgfsys@useobject{currentmarker}{}%
\end{pgfscope}%
\end{pgfscope}%
\begin{pgfscope}%
\definecolor{textcolor}{rgb}{0.000000,0.000000,0.000000}%
\pgfsetstrokecolor{textcolor}%
\pgfsetfillcolor{textcolor}%
\pgftext[x=0.100000in, y=0.736666in, left, base]{\color{textcolor}{\rmfamily\fontsize{12.000000}{14.400000}\selectfont\catcode`\^=\active\def^{\ifmmode\sp\else\^{}\fi}\catcode`\%=\active\def%{\%}$\mathdefault{10}$}}%
\end{pgfscope}%
\begin{pgfscope}%
\pgfsetbuttcap%
\pgfsetroundjoin%
\definecolor{currentfill}{rgb}{0.000000,0.000000,0.000000}%
\pgfsetfillcolor{currentfill}%
\pgfsetlinewidth{0.803000pt}%
\definecolor{currentstroke}{rgb}{0.000000,0.000000,0.000000}%
\pgfsetstrokecolor{currentstroke}%
\pgfsetdash{}{0pt}%
\pgfsys@defobject{currentmarker}{\pgfqpoint{-0.048611in}{0.000000in}}{\pgfqpoint{-0.000000in}{0.000000in}}{%
\pgfpathmoveto{\pgfqpoint{-0.000000in}{0.000000in}}%
\pgfpathlineto{\pgfqpoint{-0.048611in}{0.000000in}}%
\pgfusepath{stroke,fill}%
}%
\begin{pgfscope}%
\pgfsys@transformshift{0.360415in}{1.243703in}%
\pgfsys@useobject{currentmarker}{}%
\end{pgfscope}%
\end{pgfscope}%
\begin{pgfscope}%
\definecolor{textcolor}{rgb}{0.000000,0.000000,0.000000}%
\pgfsetstrokecolor{textcolor}%
\pgfsetfillcolor{textcolor}%
\pgftext[x=0.100000in, y=1.185833in, left, base]{\color{textcolor}{\rmfamily\fontsize{12.000000}{14.400000}\selectfont\catcode`\^=\active\def^{\ifmmode\sp\else\^{}\fi}\catcode`\%=\active\def%{\%}$\mathdefault{20}$}}%
\end{pgfscope}%
\begin{pgfscope}%
\pgfsetbuttcap%
\pgfsetroundjoin%
\definecolor{currentfill}{rgb}{0.000000,0.000000,0.000000}%
\pgfsetfillcolor{currentfill}%
\pgfsetlinewidth{0.803000pt}%
\definecolor{currentstroke}{rgb}{0.000000,0.000000,0.000000}%
\pgfsetstrokecolor{currentstroke}%
\pgfsetdash{}{0pt}%
\pgfsys@defobject{currentmarker}{\pgfqpoint{-0.048611in}{0.000000in}}{\pgfqpoint{-0.000000in}{0.000000in}}{%
\pgfpathmoveto{\pgfqpoint{-0.000000in}{0.000000in}}%
\pgfpathlineto{\pgfqpoint{-0.048611in}{0.000000in}}%
\pgfusepath{stroke,fill}%
}%
\begin{pgfscope}%
\pgfsys@transformshift{0.360415in}{1.692870in}%
\pgfsys@useobject{currentmarker}{}%
\end{pgfscope}%
\end{pgfscope}%
\begin{pgfscope}%
\definecolor{textcolor}{rgb}{0.000000,0.000000,0.000000}%
\pgfsetstrokecolor{textcolor}%
\pgfsetfillcolor{textcolor}%
\pgftext[x=0.100000in, y=1.635000in, left, base]{\color{textcolor}{\rmfamily\fontsize{12.000000}{14.400000}\selectfont\catcode`\^=\active\def^{\ifmmode\sp\else\^{}\fi}\catcode`\%=\active\def%{\%}$\mathdefault{30}$}}%
\end{pgfscope}%
\begin{pgfscope}%
\pgfsetrectcap%
\pgfsetmiterjoin%
\pgfsetlinewidth{0.803000pt}%
\definecolor{currentstroke}{rgb}{0.000000,0.000000,0.000000}%
\pgfsetstrokecolor{currentstroke}%
\pgfsetdash{}{0pt}%
\pgfpathmoveto{\pgfqpoint{0.360415in}{0.345370in}}%
\pgfpathlineto{\pgfqpoint{0.360415in}{1.692870in}}%
\pgfusepath{stroke}%
\end{pgfscope}%
\begin{pgfscope}%
\pgfsetrectcap%
\pgfsetmiterjoin%
\pgfsetlinewidth{0.803000pt}%
\definecolor{currentstroke}{rgb}{0.000000,0.000000,0.000000}%
\pgfsetstrokecolor{currentstroke}%
\pgfsetdash{}{0pt}%
\pgfpathmoveto{\pgfqpoint{1.910415in}{0.345370in}}%
\pgfpathlineto{\pgfqpoint{1.910415in}{1.692870in}}%
\pgfusepath{stroke}%
\end{pgfscope}%
\begin{pgfscope}%
\pgfsetrectcap%
\pgfsetmiterjoin%
\pgfsetlinewidth{0.803000pt}%
\definecolor{currentstroke}{rgb}{0.000000,0.000000,0.000000}%
\pgfsetstrokecolor{currentstroke}%
\pgfsetdash{}{0pt}%
\pgfpathmoveto{\pgfqpoint{0.360415in}{0.345370in}}%
\pgfpathlineto{\pgfqpoint{1.910415in}{0.345370in}}%
\pgfusepath{stroke}%
\end{pgfscope}%
\begin{pgfscope}%
\pgfsetrectcap%
\pgfsetmiterjoin%
\pgfsetlinewidth{0.803000pt}%
\definecolor{currentstroke}{rgb}{0.000000,0.000000,0.000000}%
\pgfsetstrokecolor{currentstroke}%
\pgfsetdash{}{0pt}%
\pgfpathmoveto{\pgfqpoint{0.360415in}{1.692870in}}%
\pgfpathlineto{\pgfqpoint{1.910415in}{1.692870in}}%
\pgfusepath{stroke}%
\end{pgfscope}%
\end{pgfpicture}%
\makeatother%
\endgroup%

        \caption{$n_c=5$}
        \label{fig:gaussian-well-5}
    \end{subfigure}
    \caption{Cross-sections of the periodic Gaussian well potential $V$ for different sizes $n_c$ of the supercell.}
    \label{fig:gaussian-well}
\end{figure}

For a fixed total number of random vectors $n_{\mtx{\Psi}} + n_{\mtx{\Omega}}$, we analyze the convergence behavior of the Chebyshev-Nyström++ method when run on the $1000 \times 1000$ matrix resulting from the finite difference discretization of the Hamiltonian for $n_c = 1$ described above (see \reffig{fig:convergence}). As a reference, we use the eigenvalues computed by NumPy's Hermitian eigenvalue solver applied to the matrix $\mtx{A}$.

\begin{figure}[ht]
    \centering
    %% Creator: Matplotlib, PGF backend
%%
%% To include the figure in your LaTeX document, write
%%   \input{<filename>.pgf}
%%
%% Make sure the required packages are loaded in your preamble
%%   \usepackage{pgf}
%%
%% Also ensure that all the required font packages are loaded; for instance,
%% the lmodern package is sometimes necessary when using math font.
%%   \usepackage{lmodern}
%%
%% Figures using additional raster images can only be included by \input if
%% they are in the same directory as the main LaTeX file. For loading figures
%% from other directories you can use the `import` package
%%   \usepackage{import}
%%
%% and then include the figures with
%%   \import{<path to file>}{<filename>.pgf}
%%
%% Matplotlib used the following preamble
%%   \def\mathdefault#1{#1}
%%   \everymath=\expandafter{\the\everymath\displaystyle}
%%   
%%   \ifdefined\pdftexversion\else  % non-pdftex case.
%%     \usepackage{fontspec}
%%     \setmainfont{DejaVuSerif.ttf}[Path=\detokenize{/opt/hostedtoolcache/Python/3.12.3/x64/lib/python3.12/site-packages/matplotlib/mpl-data/fonts/ttf/}]
%%     \setsansfont{DejaVuSans.ttf}[Path=\detokenize{/opt/hostedtoolcache/Python/3.12.3/x64/lib/python3.12/site-packages/matplotlib/mpl-data/fonts/ttf/}]
%%     \setmonofont{DejaVuSansMono.ttf}[Path=\detokenize{/opt/hostedtoolcache/Python/3.12.3/x64/lib/python3.12/site-packages/matplotlib/mpl-data/fonts/ttf/}]
%%   \fi
%%   \makeatletter\@ifpackageloaded{underscore}{}{\usepackage[strings]{underscore}}\makeatother
%%
\begingroup%
\makeatletter%
\begin{pgfpicture}%
\pgfpathrectangle{\pgfpointorigin}{\pgfqpoint{5.340163in}{2.959073in}}%
\pgfusepath{use as bounding box, clip}%
\begin{pgfscope}%
\pgfsetbuttcap%
\pgfsetmiterjoin%
\definecolor{currentfill}{rgb}{1.000000,1.000000,1.000000}%
\pgfsetfillcolor{currentfill}%
\pgfsetlinewidth{0.000000pt}%
\definecolor{currentstroke}{rgb}{1.000000,1.000000,1.000000}%
\pgfsetstrokecolor{currentstroke}%
\pgfsetdash{}{0pt}%
\pgfpathmoveto{\pgfqpoint{0.000000in}{-0.000000in}}%
\pgfpathlineto{\pgfqpoint{5.340163in}{-0.000000in}}%
\pgfpathlineto{\pgfqpoint{5.340163in}{2.959073in}}%
\pgfpathlineto{\pgfqpoint{0.000000in}{2.959073in}}%
\pgfpathlineto{\pgfqpoint{0.000000in}{-0.000000in}}%
\pgfpathclose%
\pgfusepath{fill}%
\end{pgfscope}%
\begin{pgfscope}%
\pgfsetbuttcap%
\pgfsetmiterjoin%
\definecolor{currentfill}{rgb}{1.000000,1.000000,1.000000}%
\pgfsetfillcolor{currentfill}%
\pgfsetlinewidth{0.000000pt}%
\definecolor{currentstroke}{rgb}{0.000000,0.000000,0.000000}%
\pgfsetstrokecolor{currentstroke}%
\pgfsetstrokeopacity{0.000000}%
\pgfsetdash{}{0pt}%
\pgfpathmoveto{\pgfqpoint{0.721913in}{0.549073in}}%
\pgfpathlineto{\pgfqpoint{5.240163in}{0.549073in}}%
\pgfpathlineto{\pgfqpoint{5.240163in}{2.859073in}}%
\pgfpathlineto{\pgfqpoint{0.721913in}{2.859073in}}%
\pgfpathlineto{\pgfqpoint{0.721913in}{0.549073in}}%
\pgfpathclose%
\pgfusepath{fill}%
\end{pgfscope}%
\begin{pgfscope}%
\pgfpathrectangle{\pgfqpoint{0.721913in}{0.549073in}}{\pgfqpoint{4.518250in}{2.310000in}}%
\pgfusepath{clip}%
\pgfsetrectcap%
\pgfsetroundjoin%
\pgfsetlinewidth{0.250937pt}%
\definecolor{currentstroke}{rgb}{0.000000,0.000000,0.000000}%
\pgfsetstrokecolor{currentstroke}%
\pgfsetstrokeopacity{0.200000}%
\pgfsetdash{}{0pt}%
\pgfpathmoveto{\pgfqpoint{1.301805in}{0.549073in}}%
\pgfpathlineto{\pgfqpoint{1.301805in}{2.859073in}}%
\pgfusepath{stroke}%
\end{pgfscope}%
\begin{pgfscope}%
\pgfsetbuttcap%
\pgfsetroundjoin%
\definecolor{currentfill}{rgb}{0.000000,0.000000,0.000000}%
\pgfsetfillcolor{currentfill}%
\pgfsetlinewidth{0.803000pt}%
\definecolor{currentstroke}{rgb}{0.000000,0.000000,0.000000}%
\pgfsetstrokecolor{currentstroke}%
\pgfsetdash{}{0pt}%
\pgfsys@defobject{currentmarker}{\pgfqpoint{0.000000in}{-0.048611in}}{\pgfqpoint{0.000000in}{0.000000in}}{%
\pgfpathmoveto{\pgfqpoint{0.000000in}{0.000000in}}%
\pgfpathlineto{\pgfqpoint{0.000000in}{-0.048611in}}%
\pgfusepath{stroke,fill}%
}%
\begin{pgfscope}%
\pgfsys@transformshift{1.301805in}{0.549073in}%
\pgfsys@useobject{currentmarker}{}%
\end{pgfscope}%
\end{pgfscope}%
\begin{pgfscope}%
\definecolor{textcolor}{rgb}{0.000000,0.000000,0.000000}%
\pgfsetstrokecolor{textcolor}%
\pgfsetfillcolor{textcolor}%
\pgftext[x=1.301805in,y=0.451851in,,top]{\color{textcolor}{\rmfamily\fontsize{12.000000}{14.400000}\selectfont\catcode`\^=\active\def^{\ifmmode\sp\else\^{}\fi}\catcode`\%=\active\def%{\%}$\mathdefault{10^{1}}$}}%
\end{pgfscope}%
\begin{pgfscope}%
\pgfpathrectangle{\pgfqpoint{0.721913in}{0.549073in}}{\pgfqpoint{4.518250in}{2.310000in}}%
\pgfusepath{clip}%
\pgfsetrectcap%
\pgfsetroundjoin%
\pgfsetlinewidth{0.250937pt}%
\definecolor{currentstroke}{rgb}{0.000000,0.000000,0.000000}%
\pgfsetstrokecolor{currentstroke}%
\pgfsetstrokeopacity{0.200000}%
\pgfsetdash{}{0pt}%
\pgfpathmoveto{\pgfqpoint{3.430797in}{0.549073in}}%
\pgfpathlineto{\pgfqpoint{3.430797in}{2.859073in}}%
\pgfusepath{stroke}%
\end{pgfscope}%
\begin{pgfscope}%
\pgfsetbuttcap%
\pgfsetroundjoin%
\definecolor{currentfill}{rgb}{0.000000,0.000000,0.000000}%
\pgfsetfillcolor{currentfill}%
\pgfsetlinewidth{0.803000pt}%
\definecolor{currentstroke}{rgb}{0.000000,0.000000,0.000000}%
\pgfsetstrokecolor{currentstroke}%
\pgfsetdash{}{0pt}%
\pgfsys@defobject{currentmarker}{\pgfqpoint{0.000000in}{-0.048611in}}{\pgfqpoint{0.000000in}{0.000000in}}{%
\pgfpathmoveto{\pgfqpoint{0.000000in}{0.000000in}}%
\pgfpathlineto{\pgfqpoint{0.000000in}{-0.048611in}}%
\pgfusepath{stroke,fill}%
}%
\begin{pgfscope}%
\pgfsys@transformshift{3.430797in}{0.549073in}%
\pgfsys@useobject{currentmarker}{}%
\end{pgfscope}%
\end{pgfscope}%
\begin{pgfscope}%
\definecolor{textcolor}{rgb}{0.000000,0.000000,0.000000}%
\pgfsetstrokecolor{textcolor}%
\pgfsetfillcolor{textcolor}%
\pgftext[x=3.430797in,y=0.451851in,,top]{\color{textcolor}{\rmfamily\fontsize{12.000000}{14.400000}\selectfont\catcode`\^=\active\def^{\ifmmode\sp\else\^{}\fi}\catcode`\%=\active\def%{\%}$\mathdefault{10^{2}}$}}%
\end{pgfscope}%
\begin{pgfscope}%
\pgfpathrectangle{\pgfqpoint{0.721913in}{0.549073in}}{\pgfqpoint{4.518250in}{2.310000in}}%
\pgfusepath{clip}%
\pgfsetrectcap%
\pgfsetroundjoin%
\pgfsetlinewidth{0.250937pt}%
\definecolor{currentstroke}{rgb}{0.000000,0.000000,0.000000}%
\pgfsetstrokecolor{currentstroke}%
\pgfsetstrokeopacity{0.200000}%
\pgfsetdash{}{0pt}%
\pgfpathmoveto{\pgfqpoint{0.829491in}{0.549073in}}%
\pgfpathlineto{\pgfqpoint{0.829491in}{2.859073in}}%
\pgfusepath{stroke}%
\end{pgfscope}%
\begin{pgfscope}%
\pgfsetbuttcap%
\pgfsetroundjoin%
\definecolor{currentfill}{rgb}{0.000000,0.000000,0.000000}%
\pgfsetfillcolor{currentfill}%
\pgfsetlinewidth{0.602250pt}%
\definecolor{currentstroke}{rgb}{0.000000,0.000000,0.000000}%
\pgfsetstrokecolor{currentstroke}%
\pgfsetdash{}{0pt}%
\pgfsys@defobject{currentmarker}{\pgfqpoint{0.000000in}{-0.027778in}}{\pgfqpoint{0.000000in}{0.000000in}}{%
\pgfpathmoveto{\pgfqpoint{0.000000in}{0.000000in}}%
\pgfpathlineto{\pgfqpoint{0.000000in}{-0.027778in}}%
\pgfusepath{stroke,fill}%
}%
\begin{pgfscope}%
\pgfsys@transformshift{0.829491in}{0.549073in}%
\pgfsys@useobject{currentmarker}{}%
\end{pgfscope}%
\end{pgfscope}%
\begin{pgfscope}%
\pgfpathrectangle{\pgfqpoint{0.721913in}{0.549073in}}{\pgfqpoint{4.518250in}{2.310000in}}%
\pgfusepath{clip}%
\pgfsetrectcap%
\pgfsetroundjoin%
\pgfsetlinewidth{0.250937pt}%
\definecolor{currentstroke}{rgb}{0.000000,0.000000,0.000000}%
\pgfsetstrokecolor{currentstroke}%
\pgfsetstrokeopacity{0.200000}%
\pgfsetdash{}{0pt}%
\pgfpathmoveto{\pgfqpoint{0.972020in}{0.549073in}}%
\pgfpathlineto{\pgfqpoint{0.972020in}{2.859073in}}%
\pgfusepath{stroke}%
\end{pgfscope}%
\begin{pgfscope}%
\pgfsetbuttcap%
\pgfsetroundjoin%
\definecolor{currentfill}{rgb}{0.000000,0.000000,0.000000}%
\pgfsetfillcolor{currentfill}%
\pgfsetlinewidth{0.602250pt}%
\definecolor{currentstroke}{rgb}{0.000000,0.000000,0.000000}%
\pgfsetstrokecolor{currentstroke}%
\pgfsetdash{}{0pt}%
\pgfsys@defobject{currentmarker}{\pgfqpoint{0.000000in}{-0.027778in}}{\pgfqpoint{0.000000in}{0.000000in}}{%
\pgfpathmoveto{\pgfqpoint{0.000000in}{0.000000in}}%
\pgfpathlineto{\pgfqpoint{0.000000in}{-0.027778in}}%
\pgfusepath{stroke,fill}%
}%
\begin{pgfscope}%
\pgfsys@transformshift{0.972020in}{0.549073in}%
\pgfsys@useobject{currentmarker}{}%
\end{pgfscope}%
\end{pgfscope}%
\begin{pgfscope}%
\pgfpathrectangle{\pgfqpoint{0.721913in}{0.549073in}}{\pgfqpoint{4.518250in}{2.310000in}}%
\pgfusepath{clip}%
\pgfsetrectcap%
\pgfsetroundjoin%
\pgfsetlinewidth{0.250937pt}%
\definecolor{currentstroke}{rgb}{0.000000,0.000000,0.000000}%
\pgfsetstrokecolor{currentstroke}%
\pgfsetstrokeopacity{0.200000}%
\pgfsetdash{}{0pt}%
\pgfpathmoveto{\pgfqpoint{1.095484in}{0.549073in}}%
\pgfpathlineto{\pgfqpoint{1.095484in}{2.859073in}}%
\pgfusepath{stroke}%
\end{pgfscope}%
\begin{pgfscope}%
\pgfsetbuttcap%
\pgfsetroundjoin%
\definecolor{currentfill}{rgb}{0.000000,0.000000,0.000000}%
\pgfsetfillcolor{currentfill}%
\pgfsetlinewidth{0.602250pt}%
\definecolor{currentstroke}{rgb}{0.000000,0.000000,0.000000}%
\pgfsetstrokecolor{currentstroke}%
\pgfsetdash{}{0pt}%
\pgfsys@defobject{currentmarker}{\pgfqpoint{0.000000in}{-0.027778in}}{\pgfqpoint{0.000000in}{0.000000in}}{%
\pgfpathmoveto{\pgfqpoint{0.000000in}{0.000000in}}%
\pgfpathlineto{\pgfqpoint{0.000000in}{-0.027778in}}%
\pgfusepath{stroke,fill}%
}%
\begin{pgfscope}%
\pgfsys@transformshift{1.095484in}{0.549073in}%
\pgfsys@useobject{currentmarker}{}%
\end{pgfscope}%
\end{pgfscope}%
\begin{pgfscope}%
\pgfpathrectangle{\pgfqpoint{0.721913in}{0.549073in}}{\pgfqpoint{4.518250in}{2.310000in}}%
\pgfusepath{clip}%
\pgfsetrectcap%
\pgfsetroundjoin%
\pgfsetlinewidth{0.250937pt}%
\definecolor{currentstroke}{rgb}{0.000000,0.000000,0.000000}%
\pgfsetstrokecolor{currentstroke}%
\pgfsetstrokeopacity{0.200000}%
\pgfsetdash{}{0pt}%
\pgfpathmoveto{\pgfqpoint{1.204388in}{0.549073in}}%
\pgfpathlineto{\pgfqpoint{1.204388in}{2.859073in}}%
\pgfusepath{stroke}%
\end{pgfscope}%
\begin{pgfscope}%
\pgfsetbuttcap%
\pgfsetroundjoin%
\definecolor{currentfill}{rgb}{0.000000,0.000000,0.000000}%
\pgfsetfillcolor{currentfill}%
\pgfsetlinewidth{0.602250pt}%
\definecolor{currentstroke}{rgb}{0.000000,0.000000,0.000000}%
\pgfsetstrokecolor{currentstroke}%
\pgfsetdash{}{0pt}%
\pgfsys@defobject{currentmarker}{\pgfqpoint{0.000000in}{-0.027778in}}{\pgfqpoint{0.000000in}{0.000000in}}{%
\pgfpathmoveto{\pgfqpoint{0.000000in}{0.000000in}}%
\pgfpathlineto{\pgfqpoint{0.000000in}{-0.027778in}}%
\pgfusepath{stroke,fill}%
}%
\begin{pgfscope}%
\pgfsys@transformshift{1.204388in}{0.549073in}%
\pgfsys@useobject{currentmarker}{}%
\end{pgfscope}%
\end{pgfscope}%
\begin{pgfscope}%
\pgfpathrectangle{\pgfqpoint{0.721913in}{0.549073in}}{\pgfqpoint{4.518250in}{2.310000in}}%
\pgfusepath{clip}%
\pgfsetrectcap%
\pgfsetroundjoin%
\pgfsetlinewidth{0.250937pt}%
\definecolor{currentstroke}{rgb}{0.000000,0.000000,0.000000}%
\pgfsetstrokecolor{currentstroke}%
\pgfsetstrokeopacity{0.200000}%
\pgfsetdash{}{0pt}%
\pgfpathmoveto{\pgfqpoint{1.942695in}{0.549073in}}%
\pgfpathlineto{\pgfqpoint{1.942695in}{2.859073in}}%
\pgfusepath{stroke}%
\end{pgfscope}%
\begin{pgfscope}%
\pgfsetbuttcap%
\pgfsetroundjoin%
\definecolor{currentfill}{rgb}{0.000000,0.000000,0.000000}%
\pgfsetfillcolor{currentfill}%
\pgfsetlinewidth{0.602250pt}%
\definecolor{currentstroke}{rgb}{0.000000,0.000000,0.000000}%
\pgfsetstrokecolor{currentstroke}%
\pgfsetdash{}{0pt}%
\pgfsys@defobject{currentmarker}{\pgfqpoint{0.000000in}{-0.027778in}}{\pgfqpoint{0.000000in}{0.000000in}}{%
\pgfpathmoveto{\pgfqpoint{0.000000in}{0.000000in}}%
\pgfpathlineto{\pgfqpoint{0.000000in}{-0.027778in}}%
\pgfusepath{stroke,fill}%
}%
\begin{pgfscope}%
\pgfsys@transformshift{1.942695in}{0.549073in}%
\pgfsys@useobject{currentmarker}{}%
\end{pgfscope}%
\end{pgfscope}%
\begin{pgfscope}%
\pgfpathrectangle{\pgfqpoint{0.721913in}{0.549073in}}{\pgfqpoint{4.518250in}{2.310000in}}%
\pgfusepath{clip}%
\pgfsetrectcap%
\pgfsetroundjoin%
\pgfsetlinewidth{0.250937pt}%
\definecolor{currentstroke}{rgb}{0.000000,0.000000,0.000000}%
\pgfsetstrokecolor{currentstroke}%
\pgfsetstrokeopacity{0.200000}%
\pgfsetdash{}{0pt}%
\pgfpathmoveto{\pgfqpoint{2.317592in}{0.549073in}}%
\pgfpathlineto{\pgfqpoint{2.317592in}{2.859073in}}%
\pgfusepath{stroke}%
\end{pgfscope}%
\begin{pgfscope}%
\pgfsetbuttcap%
\pgfsetroundjoin%
\definecolor{currentfill}{rgb}{0.000000,0.000000,0.000000}%
\pgfsetfillcolor{currentfill}%
\pgfsetlinewidth{0.602250pt}%
\definecolor{currentstroke}{rgb}{0.000000,0.000000,0.000000}%
\pgfsetstrokecolor{currentstroke}%
\pgfsetdash{}{0pt}%
\pgfsys@defobject{currentmarker}{\pgfqpoint{0.000000in}{-0.027778in}}{\pgfqpoint{0.000000in}{0.000000in}}{%
\pgfpathmoveto{\pgfqpoint{0.000000in}{0.000000in}}%
\pgfpathlineto{\pgfqpoint{0.000000in}{-0.027778in}}%
\pgfusepath{stroke,fill}%
}%
\begin{pgfscope}%
\pgfsys@transformshift{2.317592in}{0.549073in}%
\pgfsys@useobject{currentmarker}{}%
\end{pgfscope}%
\end{pgfscope}%
\begin{pgfscope}%
\pgfpathrectangle{\pgfqpoint{0.721913in}{0.549073in}}{\pgfqpoint{4.518250in}{2.310000in}}%
\pgfusepath{clip}%
\pgfsetrectcap%
\pgfsetroundjoin%
\pgfsetlinewidth{0.250937pt}%
\definecolor{currentstroke}{rgb}{0.000000,0.000000,0.000000}%
\pgfsetstrokecolor{currentstroke}%
\pgfsetstrokeopacity{0.200000}%
\pgfsetdash{}{0pt}%
\pgfpathmoveto{\pgfqpoint{2.583586in}{0.549073in}}%
\pgfpathlineto{\pgfqpoint{2.583586in}{2.859073in}}%
\pgfusepath{stroke}%
\end{pgfscope}%
\begin{pgfscope}%
\pgfsetbuttcap%
\pgfsetroundjoin%
\definecolor{currentfill}{rgb}{0.000000,0.000000,0.000000}%
\pgfsetfillcolor{currentfill}%
\pgfsetlinewidth{0.602250pt}%
\definecolor{currentstroke}{rgb}{0.000000,0.000000,0.000000}%
\pgfsetstrokecolor{currentstroke}%
\pgfsetdash{}{0pt}%
\pgfsys@defobject{currentmarker}{\pgfqpoint{0.000000in}{-0.027778in}}{\pgfqpoint{0.000000in}{0.000000in}}{%
\pgfpathmoveto{\pgfqpoint{0.000000in}{0.000000in}}%
\pgfpathlineto{\pgfqpoint{0.000000in}{-0.027778in}}%
\pgfusepath{stroke,fill}%
}%
\begin{pgfscope}%
\pgfsys@transformshift{2.583586in}{0.549073in}%
\pgfsys@useobject{currentmarker}{}%
\end{pgfscope}%
\end{pgfscope}%
\begin{pgfscope}%
\pgfpathrectangle{\pgfqpoint{0.721913in}{0.549073in}}{\pgfqpoint{4.518250in}{2.310000in}}%
\pgfusepath{clip}%
\pgfsetrectcap%
\pgfsetroundjoin%
\pgfsetlinewidth{0.250937pt}%
\definecolor{currentstroke}{rgb}{0.000000,0.000000,0.000000}%
\pgfsetstrokecolor{currentstroke}%
\pgfsetstrokeopacity{0.200000}%
\pgfsetdash{}{0pt}%
\pgfpathmoveto{\pgfqpoint{2.789906in}{0.549073in}}%
\pgfpathlineto{\pgfqpoint{2.789906in}{2.859073in}}%
\pgfusepath{stroke}%
\end{pgfscope}%
\begin{pgfscope}%
\pgfsetbuttcap%
\pgfsetroundjoin%
\definecolor{currentfill}{rgb}{0.000000,0.000000,0.000000}%
\pgfsetfillcolor{currentfill}%
\pgfsetlinewidth{0.602250pt}%
\definecolor{currentstroke}{rgb}{0.000000,0.000000,0.000000}%
\pgfsetstrokecolor{currentstroke}%
\pgfsetdash{}{0pt}%
\pgfsys@defobject{currentmarker}{\pgfqpoint{0.000000in}{-0.027778in}}{\pgfqpoint{0.000000in}{0.000000in}}{%
\pgfpathmoveto{\pgfqpoint{0.000000in}{0.000000in}}%
\pgfpathlineto{\pgfqpoint{0.000000in}{-0.027778in}}%
\pgfusepath{stroke,fill}%
}%
\begin{pgfscope}%
\pgfsys@transformshift{2.789906in}{0.549073in}%
\pgfsys@useobject{currentmarker}{}%
\end{pgfscope}%
\end{pgfscope}%
\begin{pgfscope}%
\pgfpathrectangle{\pgfqpoint{0.721913in}{0.549073in}}{\pgfqpoint{4.518250in}{2.310000in}}%
\pgfusepath{clip}%
\pgfsetrectcap%
\pgfsetroundjoin%
\pgfsetlinewidth{0.250937pt}%
\definecolor{currentstroke}{rgb}{0.000000,0.000000,0.000000}%
\pgfsetstrokecolor{currentstroke}%
\pgfsetstrokeopacity{0.200000}%
\pgfsetdash{}{0pt}%
\pgfpathmoveto{\pgfqpoint{2.958483in}{0.549073in}}%
\pgfpathlineto{\pgfqpoint{2.958483in}{2.859073in}}%
\pgfusepath{stroke}%
\end{pgfscope}%
\begin{pgfscope}%
\pgfsetbuttcap%
\pgfsetroundjoin%
\definecolor{currentfill}{rgb}{0.000000,0.000000,0.000000}%
\pgfsetfillcolor{currentfill}%
\pgfsetlinewidth{0.602250pt}%
\definecolor{currentstroke}{rgb}{0.000000,0.000000,0.000000}%
\pgfsetstrokecolor{currentstroke}%
\pgfsetdash{}{0pt}%
\pgfsys@defobject{currentmarker}{\pgfqpoint{0.000000in}{-0.027778in}}{\pgfqpoint{0.000000in}{0.000000in}}{%
\pgfpathmoveto{\pgfqpoint{0.000000in}{0.000000in}}%
\pgfpathlineto{\pgfqpoint{0.000000in}{-0.027778in}}%
\pgfusepath{stroke,fill}%
}%
\begin{pgfscope}%
\pgfsys@transformshift{2.958483in}{0.549073in}%
\pgfsys@useobject{currentmarker}{}%
\end{pgfscope}%
\end{pgfscope}%
\begin{pgfscope}%
\pgfpathrectangle{\pgfqpoint{0.721913in}{0.549073in}}{\pgfqpoint{4.518250in}{2.310000in}}%
\pgfusepath{clip}%
\pgfsetrectcap%
\pgfsetroundjoin%
\pgfsetlinewidth{0.250937pt}%
\definecolor{currentstroke}{rgb}{0.000000,0.000000,0.000000}%
\pgfsetstrokecolor{currentstroke}%
\pgfsetstrokeopacity{0.200000}%
\pgfsetdash{}{0pt}%
\pgfpathmoveto{\pgfqpoint{3.101012in}{0.549073in}}%
\pgfpathlineto{\pgfqpoint{3.101012in}{2.859073in}}%
\pgfusepath{stroke}%
\end{pgfscope}%
\begin{pgfscope}%
\pgfsetbuttcap%
\pgfsetroundjoin%
\definecolor{currentfill}{rgb}{0.000000,0.000000,0.000000}%
\pgfsetfillcolor{currentfill}%
\pgfsetlinewidth{0.602250pt}%
\definecolor{currentstroke}{rgb}{0.000000,0.000000,0.000000}%
\pgfsetstrokecolor{currentstroke}%
\pgfsetdash{}{0pt}%
\pgfsys@defobject{currentmarker}{\pgfqpoint{0.000000in}{-0.027778in}}{\pgfqpoint{0.000000in}{0.000000in}}{%
\pgfpathmoveto{\pgfqpoint{0.000000in}{0.000000in}}%
\pgfpathlineto{\pgfqpoint{0.000000in}{-0.027778in}}%
\pgfusepath{stroke,fill}%
}%
\begin{pgfscope}%
\pgfsys@transformshift{3.101012in}{0.549073in}%
\pgfsys@useobject{currentmarker}{}%
\end{pgfscope}%
\end{pgfscope}%
\begin{pgfscope}%
\pgfpathrectangle{\pgfqpoint{0.721913in}{0.549073in}}{\pgfqpoint{4.518250in}{2.310000in}}%
\pgfusepath{clip}%
\pgfsetrectcap%
\pgfsetroundjoin%
\pgfsetlinewidth{0.250937pt}%
\definecolor{currentstroke}{rgb}{0.000000,0.000000,0.000000}%
\pgfsetstrokecolor{currentstroke}%
\pgfsetstrokeopacity{0.200000}%
\pgfsetdash{}{0pt}%
\pgfpathmoveto{\pgfqpoint{3.224476in}{0.549073in}}%
\pgfpathlineto{\pgfqpoint{3.224476in}{2.859073in}}%
\pgfusepath{stroke}%
\end{pgfscope}%
\begin{pgfscope}%
\pgfsetbuttcap%
\pgfsetroundjoin%
\definecolor{currentfill}{rgb}{0.000000,0.000000,0.000000}%
\pgfsetfillcolor{currentfill}%
\pgfsetlinewidth{0.602250pt}%
\definecolor{currentstroke}{rgb}{0.000000,0.000000,0.000000}%
\pgfsetstrokecolor{currentstroke}%
\pgfsetdash{}{0pt}%
\pgfsys@defobject{currentmarker}{\pgfqpoint{0.000000in}{-0.027778in}}{\pgfqpoint{0.000000in}{0.000000in}}{%
\pgfpathmoveto{\pgfqpoint{0.000000in}{0.000000in}}%
\pgfpathlineto{\pgfqpoint{0.000000in}{-0.027778in}}%
\pgfusepath{stroke,fill}%
}%
\begin{pgfscope}%
\pgfsys@transformshift{3.224476in}{0.549073in}%
\pgfsys@useobject{currentmarker}{}%
\end{pgfscope}%
\end{pgfscope}%
\begin{pgfscope}%
\pgfpathrectangle{\pgfqpoint{0.721913in}{0.549073in}}{\pgfqpoint{4.518250in}{2.310000in}}%
\pgfusepath{clip}%
\pgfsetrectcap%
\pgfsetroundjoin%
\pgfsetlinewidth{0.250937pt}%
\definecolor{currentstroke}{rgb}{0.000000,0.000000,0.000000}%
\pgfsetstrokecolor{currentstroke}%
\pgfsetstrokeopacity{0.200000}%
\pgfsetdash{}{0pt}%
\pgfpathmoveto{\pgfqpoint{3.333379in}{0.549073in}}%
\pgfpathlineto{\pgfqpoint{3.333379in}{2.859073in}}%
\pgfusepath{stroke}%
\end{pgfscope}%
\begin{pgfscope}%
\pgfsetbuttcap%
\pgfsetroundjoin%
\definecolor{currentfill}{rgb}{0.000000,0.000000,0.000000}%
\pgfsetfillcolor{currentfill}%
\pgfsetlinewidth{0.602250pt}%
\definecolor{currentstroke}{rgb}{0.000000,0.000000,0.000000}%
\pgfsetstrokecolor{currentstroke}%
\pgfsetdash{}{0pt}%
\pgfsys@defobject{currentmarker}{\pgfqpoint{0.000000in}{-0.027778in}}{\pgfqpoint{0.000000in}{0.000000in}}{%
\pgfpathmoveto{\pgfqpoint{0.000000in}{0.000000in}}%
\pgfpathlineto{\pgfqpoint{0.000000in}{-0.027778in}}%
\pgfusepath{stroke,fill}%
}%
\begin{pgfscope}%
\pgfsys@transformshift{3.333379in}{0.549073in}%
\pgfsys@useobject{currentmarker}{}%
\end{pgfscope}%
\end{pgfscope}%
\begin{pgfscope}%
\pgfpathrectangle{\pgfqpoint{0.721913in}{0.549073in}}{\pgfqpoint{4.518250in}{2.310000in}}%
\pgfusepath{clip}%
\pgfsetrectcap%
\pgfsetroundjoin%
\pgfsetlinewidth{0.250937pt}%
\definecolor{currentstroke}{rgb}{0.000000,0.000000,0.000000}%
\pgfsetstrokecolor{currentstroke}%
\pgfsetstrokeopacity{0.200000}%
\pgfsetdash{}{0pt}%
\pgfpathmoveto{\pgfqpoint{4.071687in}{0.549073in}}%
\pgfpathlineto{\pgfqpoint{4.071687in}{2.859073in}}%
\pgfusepath{stroke}%
\end{pgfscope}%
\begin{pgfscope}%
\pgfsetbuttcap%
\pgfsetroundjoin%
\definecolor{currentfill}{rgb}{0.000000,0.000000,0.000000}%
\pgfsetfillcolor{currentfill}%
\pgfsetlinewidth{0.602250pt}%
\definecolor{currentstroke}{rgb}{0.000000,0.000000,0.000000}%
\pgfsetstrokecolor{currentstroke}%
\pgfsetdash{}{0pt}%
\pgfsys@defobject{currentmarker}{\pgfqpoint{0.000000in}{-0.027778in}}{\pgfqpoint{0.000000in}{0.000000in}}{%
\pgfpathmoveto{\pgfqpoint{0.000000in}{0.000000in}}%
\pgfpathlineto{\pgfqpoint{0.000000in}{-0.027778in}}%
\pgfusepath{stroke,fill}%
}%
\begin{pgfscope}%
\pgfsys@transformshift{4.071687in}{0.549073in}%
\pgfsys@useobject{currentmarker}{}%
\end{pgfscope}%
\end{pgfscope}%
\begin{pgfscope}%
\pgfpathrectangle{\pgfqpoint{0.721913in}{0.549073in}}{\pgfqpoint{4.518250in}{2.310000in}}%
\pgfusepath{clip}%
\pgfsetrectcap%
\pgfsetroundjoin%
\pgfsetlinewidth{0.250937pt}%
\definecolor{currentstroke}{rgb}{0.000000,0.000000,0.000000}%
\pgfsetstrokecolor{currentstroke}%
\pgfsetstrokeopacity{0.200000}%
\pgfsetdash{}{0pt}%
\pgfpathmoveto{\pgfqpoint{4.446584in}{0.549073in}}%
\pgfpathlineto{\pgfqpoint{4.446584in}{2.859073in}}%
\pgfusepath{stroke}%
\end{pgfscope}%
\begin{pgfscope}%
\pgfsetbuttcap%
\pgfsetroundjoin%
\definecolor{currentfill}{rgb}{0.000000,0.000000,0.000000}%
\pgfsetfillcolor{currentfill}%
\pgfsetlinewidth{0.602250pt}%
\definecolor{currentstroke}{rgb}{0.000000,0.000000,0.000000}%
\pgfsetstrokecolor{currentstroke}%
\pgfsetdash{}{0pt}%
\pgfsys@defobject{currentmarker}{\pgfqpoint{0.000000in}{-0.027778in}}{\pgfqpoint{0.000000in}{0.000000in}}{%
\pgfpathmoveto{\pgfqpoint{0.000000in}{0.000000in}}%
\pgfpathlineto{\pgfqpoint{0.000000in}{-0.027778in}}%
\pgfusepath{stroke,fill}%
}%
\begin{pgfscope}%
\pgfsys@transformshift{4.446584in}{0.549073in}%
\pgfsys@useobject{currentmarker}{}%
\end{pgfscope}%
\end{pgfscope}%
\begin{pgfscope}%
\pgfpathrectangle{\pgfqpoint{0.721913in}{0.549073in}}{\pgfqpoint{4.518250in}{2.310000in}}%
\pgfusepath{clip}%
\pgfsetrectcap%
\pgfsetroundjoin%
\pgfsetlinewidth{0.250937pt}%
\definecolor{currentstroke}{rgb}{0.000000,0.000000,0.000000}%
\pgfsetstrokecolor{currentstroke}%
\pgfsetstrokeopacity{0.200000}%
\pgfsetdash{}{0pt}%
\pgfpathmoveto{\pgfqpoint{4.712577in}{0.549073in}}%
\pgfpathlineto{\pgfqpoint{4.712577in}{2.859073in}}%
\pgfusepath{stroke}%
\end{pgfscope}%
\begin{pgfscope}%
\pgfsetbuttcap%
\pgfsetroundjoin%
\definecolor{currentfill}{rgb}{0.000000,0.000000,0.000000}%
\pgfsetfillcolor{currentfill}%
\pgfsetlinewidth{0.602250pt}%
\definecolor{currentstroke}{rgb}{0.000000,0.000000,0.000000}%
\pgfsetstrokecolor{currentstroke}%
\pgfsetdash{}{0pt}%
\pgfsys@defobject{currentmarker}{\pgfqpoint{0.000000in}{-0.027778in}}{\pgfqpoint{0.000000in}{0.000000in}}{%
\pgfpathmoveto{\pgfqpoint{0.000000in}{0.000000in}}%
\pgfpathlineto{\pgfqpoint{0.000000in}{-0.027778in}}%
\pgfusepath{stroke,fill}%
}%
\begin{pgfscope}%
\pgfsys@transformshift{4.712577in}{0.549073in}%
\pgfsys@useobject{currentmarker}{}%
\end{pgfscope}%
\end{pgfscope}%
\begin{pgfscope}%
\pgfpathrectangle{\pgfqpoint{0.721913in}{0.549073in}}{\pgfqpoint{4.518250in}{2.310000in}}%
\pgfusepath{clip}%
\pgfsetrectcap%
\pgfsetroundjoin%
\pgfsetlinewidth{0.250937pt}%
\definecolor{currentstroke}{rgb}{0.000000,0.000000,0.000000}%
\pgfsetstrokecolor{currentstroke}%
\pgfsetstrokeopacity{0.200000}%
\pgfsetdash{}{0pt}%
\pgfpathmoveto{\pgfqpoint{4.918898in}{0.549073in}}%
\pgfpathlineto{\pgfqpoint{4.918898in}{2.859073in}}%
\pgfusepath{stroke}%
\end{pgfscope}%
\begin{pgfscope}%
\pgfsetbuttcap%
\pgfsetroundjoin%
\definecolor{currentfill}{rgb}{0.000000,0.000000,0.000000}%
\pgfsetfillcolor{currentfill}%
\pgfsetlinewidth{0.602250pt}%
\definecolor{currentstroke}{rgb}{0.000000,0.000000,0.000000}%
\pgfsetstrokecolor{currentstroke}%
\pgfsetdash{}{0pt}%
\pgfsys@defobject{currentmarker}{\pgfqpoint{0.000000in}{-0.027778in}}{\pgfqpoint{0.000000in}{0.000000in}}{%
\pgfpathmoveto{\pgfqpoint{0.000000in}{0.000000in}}%
\pgfpathlineto{\pgfqpoint{0.000000in}{-0.027778in}}%
\pgfusepath{stroke,fill}%
}%
\begin{pgfscope}%
\pgfsys@transformshift{4.918898in}{0.549073in}%
\pgfsys@useobject{currentmarker}{}%
\end{pgfscope}%
\end{pgfscope}%
\begin{pgfscope}%
\pgfpathrectangle{\pgfqpoint{0.721913in}{0.549073in}}{\pgfqpoint{4.518250in}{2.310000in}}%
\pgfusepath{clip}%
\pgfsetrectcap%
\pgfsetroundjoin%
\pgfsetlinewidth{0.250937pt}%
\definecolor{currentstroke}{rgb}{0.000000,0.000000,0.000000}%
\pgfsetstrokecolor{currentstroke}%
\pgfsetstrokeopacity{0.200000}%
\pgfsetdash{}{0pt}%
\pgfpathmoveto{\pgfqpoint{5.087474in}{0.549073in}}%
\pgfpathlineto{\pgfqpoint{5.087474in}{2.859073in}}%
\pgfusepath{stroke}%
\end{pgfscope}%
\begin{pgfscope}%
\pgfsetbuttcap%
\pgfsetroundjoin%
\definecolor{currentfill}{rgb}{0.000000,0.000000,0.000000}%
\pgfsetfillcolor{currentfill}%
\pgfsetlinewidth{0.602250pt}%
\definecolor{currentstroke}{rgb}{0.000000,0.000000,0.000000}%
\pgfsetstrokecolor{currentstroke}%
\pgfsetdash{}{0pt}%
\pgfsys@defobject{currentmarker}{\pgfqpoint{0.000000in}{-0.027778in}}{\pgfqpoint{0.000000in}{0.000000in}}{%
\pgfpathmoveto{\pgfqpoint{0.000000in}{0.000000in}}%
\pgfpathlineto{\pgfqpoint{0.000000in}{-0.027778in}}%
\pgfusepath{stroke,fill}%
}%
\begin{pgfscope}%
\pgfsys@transformshift{5.087474in}{0.549073in}%
\pgfsys@useobject{currentmarker}{}%
\end{pgfscope}%
\end{pgfscope}%
\begin{pgfscope}%
\pgfpathrectangle{\pgfqpoint{0.721913in}{0.549073in}}{\pgfqpoint{4.518250in}{2.310000in}}%
\pgfusepath{clip}%
\pgfsetrectcap%
\pgfsetroundjoin%
\pgfsetlinewidth{0.250937pt}%
\definecolor{currentstroke}{rgb}{0.000000,0.000000,0.000000}%
\pgfsetstrokecolor{currentstroke}%
\pgfsetstrokeopacity{0.200000}%
\pgfsetdash{}{0pt}%
\pgfpathmoveto{\pgfqpoint{5.230003in}{0.549073in}}%
\pgfpathlineto{\pgfqpoint{5.230003in}{2.859073in}}%
\pgfusepath{stroke}%
\end{pgfscope}%
\begin{pgfscope}%
\pgfsetbuttcap%
\pgfsetroundjoin%
\definecolor{currentfill}{rgb}{0.000000,0.000000,0.000000}%
\pgfsetfillcolor{currentfill}%
\pgfsetlinewidth{0.602250pt}%
\definecolor{currentstroke}{rgb}{0.000000,0.000000,0.000000}%
\pgfsetstrokecolor{currentstroke}%
\pgfsetdash{}{0pt}%
\pgfsys@defobject{currentmarker}{\pgfqpoint{0.000000in}{-0.027778in}}{\pgfqpoint{0.000000in}{0.000000in}}{%
\pgfpathmoveto{\pgfqpoint{0.000000in}{0.000000in}}%
\pgfpathlineto{\pgfqpoint{0.000000in}{-0.027778in}}%
\pgfusepath{stroke,fill}%
}%
\begin{pgfscope}%
\pgfsys@transformshift{5.230003in}{0.549073in}%
\pgfsys@useobject{currentmarker}{}%
\end{pgfscope}%
\end{pgfscope}%
\begin{pgfscope}%
\definecolor{textcolor}{rgb}{0.000000,0.000000,0.000000}%
\pgfsetstrokecolor{textcolor}%
\pgfsetfillcolor{textcolor}%
\pgftext[x=2.981038in,y=0.248148in,,top]{\color{textcolor}{\rmfamily\fontsize{12.000000}{14.400000}\selectfont\catcode`\^=\active\def^{\ifmmode\sp\else\^{}\fi}\catcode`\%=\active\def%{\%}estimator size $n_{\mathbf{\Psi}} + n_{\mathbf{\Omega}}$}}%
\end{pgfscope}%
\begin{pgfscope}%
\pgfpathrectangle{\pgfqpoint{0.721913in}{0.549073in}}{\pgfqpoint{4.518250in}{2.310000in}}%
\pgfusepath{clip}%
\pgfsetrectcap%
\pgfsetroundjoin%
\pgfsetlinewidth{0.250937pt}%
\definecolor{currentstroke}{rgb}{0.000000,0.000000,0.000000}%
\pgfsetstrokecolor{currentstroke}%
\pgfsetstrokeopacity{0.200000}%
\pgfsetdash{}{0pt}%
\pgfpathmoveto{\pgfqpoint{0.721913in}{1.011125in}}%
\pgfpathlineto{\pgfqpoint{5.240163in}{1.011125in}}%
\pgfusepath{stroke}%
\end{pgfscope}%
\begin{pgfscope}%
\pgfsetbuttcap%
\pgfsetroundjoin%
\definecolor{currentfill}{rgb}{0.000000,0.000000,0.000000}%
\pgfsetfillcolor{currentfill}%
\pgfsetlinewidth{0.803000pt}%
\definecolor{currentstroke}{rgb}{0.000000,0.000000,0.000000}%
\pgfsetstrokecolor{currentstroke}%
\pgfsetdash{}{0pt}%
\pgfsys@defobject{currentmarker}{\pgfqpoint{-0.048611in}{0.000000in}}{\pgfqpoint{-0.000000in}{0.000000in}}{%
\pgfpathmoveto{\pgfqpoint{-0.000000in}{0.000000in}}%
\pgfpathlineto{\pgfqpoint{-0.048611in}{0.000000in}}%
\pgfusepath{stroke,fill}%
}%
\begin{pgfscope}%
\pgfsys@transformshift{0.721913in}{1.011125in}%
\pgfsys@useobject{currentmarker}{}%
\end{pgfscope}%
\end{pgfscope}%
\begin{pgfscope}%
\definecolor{textcolor}{rgb}{0.000000,0.000000,0.000000}%
\pgfsetstrokecolor{textcolor}%
\pgfsetfillcolor{textcolor}%
\pgftext[x=0.303703in, y=0.953254in, left, base]{\color{textcolor}{\rmfamily\fontsize{12.000000}{14.400000}\selectfont\catcode`\^=\active\def^{\ifmmode\sp\else\^{}\fi}\catcode`\%=\active\def%{\%}$\mathdefault{10^{-6}}$}}%
\end{pgfscope}%
\begin{pgfscope}%
\pgfpathrectangle{\pgfqpoint{0.721913in}{0.549073in}}{\pgfqpoint{4.518250in}{2.310000in}}%
\pgfusepath{clip}%
\pgfsetrectcap%
\pgfsetroundjoin%
\pgfsetlinewidth{0.250937pt}%
\definecolor{currentstroke}{rgb}{0.000000,0.000000,0.000000}%
\pgfsetstrokecolor{currentstroke}%
\pgfsetstrokeopacity{0.200000}%
\pgfsetdash{}{0pt}%
\pgfpathmoveto{\pgfqpoint{0.721913in}{1.577741in}}%
\pgfpathlineto{\pgfqpoint{5.240163in}{1.577741in}}%
\pgfusepath{stroke}%
\end{pgfscope}%
\begin{pgfscope}%
\pgfsetbuttcap%
\pgfsetroundjoin%
\definecolor{currentfill}{rgb}{0.000000,0.000000,0.000000}%
\pgfsetfillcolor{currentfill}%
\pgfsetlinewidth{0.803000pt}%
\definecolor{currentstroke}{rgb}{0.000000,0.000000,0.000000}%
\pgfsetstrokecolor{currentstroke}%
\pgfsetdash{}{0pt}%
\pgfsys@defobject{currentmarker}{\pgfqpoint{-0.048611in}{0.000000in}}{\pgfqpoint{-0.000000in}{0.000000in}}{%
\pgfpathmoveto{\pgfqpoint{-0.000000in}{0.000000in}}%
\pgfpathlineto{\pgfqpoint{-0.048611in}{0.000000in}}%
\pgfusepath{stroke,fill}%
}%
\begin{pgfscope}%
\pgfsys@transformshift{0.721913in}{1.577741in}%
\pgfsys@useobject{currentmarker}{}%
\end{pgfscope}%
\end{pgfscope}%
\begin{pgfscope}%
\definecolor{textcolor}{rgb}{0.000000,0.000000,0.000000}%
\pgfsetstrokecolor{textcolor}%
\pgfsetfillcolor{textcolor}%
\pgftext[x=0.303703in, y=1.519871in, left, base]{\color{textcolor}{\rmfamily\fontsize{12.000000}{14.400000}\selectfont\catcode`\^=\active\def^{\ifmmode\sp\else\^{}\fi}\catcode`\%=\active\def%{\%}$\mathdefault{10^{-4}}$}}%
\end{pgfscope}%
\begin{pgfscope}%
\pgfpathrectangle{\pgfqpoint{0.721913in}{0.549073in}}{\pgfqpoint{4.518250in}{2.310000in}}%
\pgfusepath{clip}%
\pgfsetrectcap%
\pgfsetroundjoin%
\pgfsetlinewidth{0.250937pt}%
\definecolor{currentstroke}{rgb}{0.000000,0.000000,0.000000}%
\pgfsetstrokecolor{currentstroke}%
\pgfsetstrokeopacity{0.200000}%
\pgfsetdash{}{0pt}%
\pgfpathmoveto{\pgfqpoint{0.721913in}{2.144358in}}%
\pgfpathlineto{\pgfqpoint{5.240163in}{2.144358in}}%
\pgfusepath{stroke}%
\end{pgfscope}%
\begin{pgfscope}%
\pgfsetbuttcap%
\pgfsetroundjoin%
\definecolor{currentfill}{rgb}{0.000000,0.000000,0.000000}%
\pgfsetfillcolor{currentfill}%
\pgfsetlinewidth{0.803000pt}%
\definecolor{currentstroke}{rgb}{0.000000,0.000000,0.000000}%
\pgfsetstrokecolor{currentstroke}%
\pgfsetdash{}{0pt}%
\pgfsys@defobject{currentmarker}{\pgfqpoint{-0.048611in}{0.000000in}}{\pgfqpoint{-0.000000in}{0.000000in}}{%
\pgfpathmoveto{\pgfqpoint{-0.000000in}{0.000000in}}%
\pgfpathlineto{\pgfqpoint{-0.048611in}{0.000000in}}%
\pgfusepath{stroke,fill}%
}%
\begin{pgfscope}%
\pgfsys@transformshift{0.721913in}{2.144358in}%
\pgfsys@useobject{currentmarker}{}%
\end{pgfscope}%
\end{pgfscope}%
\begin{pgfscope}%
\definecolor{textcolor}{rgb}{0.000000,0.000000,0.000000}%
\pgfsetstrokecolor{textcolor}%
\pgfsetfillcolor{textcolor}%
\pgftext[x=0.303703in, y=2.086488in, left, base]{\color{textcolor}{\rmfamily\fontsize{12.000000}{14.400000}\selectfont\catcode`\^=\active\def^{\ifmmode\sp\else\^{}\fi}\catcode`\%=\active\def%{\%}$\mathdefault{10^{-2}}$}}%
\end{pgfscope}%
\begin{pgfscope}%
\pgfpathrectangle{\pgfqpoint{0.721913in}{0.549073in}}{\pgfqpoint{4.518250in}{2.310000in}}%
\pgfusepath{clip}%
\pgfsetrectcap%
\pgfsetroundjoin%
\pgfsetlinewidth{0.250937pt}%
\definecolor{currentstroke}{rgb}{0.000000,0.000000,0.000000}%
\pgfsetstrokecolor{currentstroke}%
\pgfsetstrokeopacity{0.200000}%
\pgfsetdash{}{0pt}%
\pgfpathmoveto{\pgfqpoint{0.721913in}{2.710974in}}%
\pgfpathlineto{\pgfqpoint{5.240163in}{2.710974in}}%
\pgfusepath{stroke}%
\end{pgfscope}%
\begin{pgfscope}%
\pgfsetbuttcap%
\pgfsetroundjoin%
\definecolor{currentfill}{rgb}{0.000000,0.000000,0.000000}%
\pgfsetfillcolor{currentfill}%
\pgfsetlinewidth{0.803000pt}%
\definecolor{currentstroke}{rgb}{0.000000,0.000000,0.000000}%
\pgfsetstrokecolor{currentstroke}%
\pgfsetdash{}{0pt}%
\pgfsys@defobject{currentmarker}{\pgfqpoint{-0.048611in}{0.000000in}}{\pgfqpoint{-0.000000in}{0.000000in}}{%
\pgfpathmoveto{\pgfqpoint{-0.000000in}{0.000000in}}%
\pgfpathlineto{\pgfqpoint{-0.048611in}{0.000000in}}%
\pgfusepath{stroke,fill}%
}%
\begin{pgfscope}%
\pgfsys@transformshift{0.721913in}{2.710974in}%
\pgfsys@useobject{currentmarker}{}%
\end{pgfscope}%
\end{pgfscope}%
\begin{pgfscope}%
\definecolor{textcolor}{rgb}{0.000000,0.000000,0.000000}%
\pgfsetstrokecolor{textcolor}%
\pgfsetfillcolor{textcolor}%
\pgftext[x=0.395525in, y=2.653104in, left, base]{\color{textcolor}{\rmfamily\fontsize{12.000000}{14.400000}\selectfont\catcode`\^=\active\def^{\ifmmode\sp\else\^{}\fi}\catcode`\%=\active\def%{\%}$\mathdefault{10^{0}}$}}%
\end{pgfscope}%
\begin{pgfscope}%
\definecolor{textcolor}{rgb}{0.000000,0.000000,0.000000}%
\pgfsetstrokecolor{textcolor}%
\pgfsetfillcolor{textcolor}%
\pgftext[x=0.248148in,y=1.704073in,,bottom,rotate=90.000000]{\color{textcolor}{\rmfamily\fontsize{12.000000}{14.400000}\selectfont\catcode`\^=\active\def^{\ifmmode\sp\else\^{}\fi}\catcode`\%=\active\def%{\%}$L^1$-error}}%
\end{pgfscope}%
\begin{pgfscope}%
\pgfpathrectangle{\pgfqpoint{0.721913in}{0.549073in}}{\pgfqpoint{4.518250in}{2.310000in}}%
\pgfusepath{clip}%
\pgfsetrectcap%
\pgfsetroundjoin%
\pgfsetlinewidth{1.505625pt}%
\definecolor{currentstroke}{rgb}{0.392157,0.560784,1.000000}%
\pgfsetstrokecolor{currentstroke}%
\pgfsetdash{}{0pt}%
\pgfpathmoveto{\pgfqpoint{0.829491in}{2.445573in}}%
\pgfpathlineto{\pgfqpoint{1.544389in}{2.381657in}}%
\pgfpathlineto{\pgfqpoint{2.286246in}{2.337799in}}%
\pgfpathlineto{\pgfqpoint{3.003594in}{2.296985in}}%
\pgfpathlineto{\pgfqpoint{3.708276in}{2.259953in}}%
\pgfpathlineto{\pgfqpoint{4.421593in}{2.177816in}}%
\pgfpathlineto{\pgfqpoint{5.132586in}{2.165761in}}%
\pgfusepath{stroke}%
\end{pgfscope}%
\begin{pgfscope}%
\pgfpathrectangle{\pgfqpoint{0.721913in}{0.549073in}}{\pgfqpoint{4.518250in}{2.310000in}}%
\pgfusepath{clip}%
\pgfsetbuttcap%
\pgfsetroundjoin%
\definecolor{currentfill}{rgb}{0.392157,0.560784,1.000000}%
\pgfsetfillcolor{currentfill}%
\pgfsetlinewidth{1.003750pt}%
\definecolor{currentstroke}{rgb}{0.392157,0.560784,1.000000}%
\pgfsetstrokecolor{currentstroke}%
\pgfsetdash{}{0pt}%
\pgfsys@defobject{currentmarker}{\pgfqpoint{-0.041667in}{-0.041667in}}{\pgfqpoint{0.041667in}{0.041667in}}{%
\pgfpathmoveto{\pgfqpoint{0.000000in}{-0.041667in}}%
\pgfpathcurveto{\pgfqpoint{0.011050in}{-0.041667in}}{\pgfqpoint{0.021649in}{-0.037276in}}{\pgfqpoint{0.029463in}{-0.029463in}}%
\pgfpathcurveto{\pgfqpoint{0.037276in}{-0.021649in}}{\pgfqpoint{0.041667in}{-0.011050in}}{\pgfqpoint{0.041667in}{0.000000in}}%
\pgfpathcurveto{\pgfqpoint{0.041667in}{0.011050in}}{\pgfqpoint{0.037276in}{0.021649in}}{\pgfqpoint{0.029463in}{0.029463in}}%
\pgfpathcurveto{\pgfqpoint{0.021649in}{0.037276in}}{\pgfqpoint{0.011050in}{0.041667in}}{\pgfqpoint{0.000000in}{0.041667in}}%
\pgfpathcurveto{\pgfqpoint{-0.011050in}{0.041667in}}{\pgfqpoint{-0.021649in}{0.037276in}}{\pgfqpoint{-0.029463in}{0.029463in}}%
\pgfpathcurveto{\pgfqpoint{-0.037276in}{0.021649in}}{\pgfqpoint{-0.041667in}{0.011050in}}{\pgfqpoint{-0.041667in}{0.000000in}}%
\pgfpathcurveto{\pgfqpoint{-0.041667in}{-0.011050in}}{\pgfqpoint{-0.037276in}{-0.021649in}}{\pgfqpoint{-0.029463in}{-0.029463in}}%
\pgfpathcurveto{\pgfqpoint{-0.021649in}{-0.037276in}}{\pgfqpoint{-0.011050in}{-0.041667in}}{\pgfqpoint{0.000000in}{-0.041667in}}%
\pgfpathlineto{\pgfqpoint{0.000000in}{-0.041667in}}%
\pgfpathclose%
\pgfusepath{stroke,fill}%
}%
\begin{pgfscope}%
\pgfsys@transformshift{0.829491in}{2.445573in}%
\pgfsys@useobject{currentmarker}{}%
\end{pgfscope}%
\begin{pgfscope}%
\pgfsys@transformshift{1.544389in}{2.381657in}%
\pgfsys@useobject{currentmarker}{}%
\end{pgfscope}%
\begin{pgfscope}%
\pgfsys@transformshift{2.286246in}{2.337799in}%
\pgfsys@useobject{currentmarker}{}%
\end{pgfscope}%
\begin{pgfscope}%
\pgfsys@transformshift{3.003594in}{2.296985in}%
\pgfsys@useobject{currentmarker}{}%
\end{pgfscope}%
\begin{pgfscope}%
\pgfsys@transformshift{3.708276in}{2.259953in}%
\pgfsys@useobject{currentmarker}{}%
\end{pgfscope}%
\begin{pgfscope}%
\pgfsys@transformshift{4.421593in}{2.177816in}%
\pgfsys@useobject{currentmarker}{}%
\end{pgfscope}%
\begin{pgfscope}%
\pgfsys@transformshift{5.132586in}{2.165761in}%
\pgfsys@useobject{currentmarker}{}%
\end{pgfscope}%
\end{pgfscope}%
\begin{pgfscope}%
\pgfpathrectangle{\pgfqpoint{0.721913in}{0.549073in}}{\pgfqpoint{4.518250in}{2.310000in}}%
\pgfusepath{clip}%
\pgfsetrectcap%
\pgfsetroundjoin%
\pgfsetlinewidth{1.505625pt}%
\definecolor{currentstroke}{rgb}{0.862745,0.149020,0.498039}%
\pgfsetstrokecolor{currentstroke}%
\pgfsetdash{}{0pt}%
\pgfpathmoveto{\pgfqpoint{0.829491in}{2.430578in}}%
\pgfpathlineto{\pgfqpoint{1.544389in}{2.354200in}}%
\pgfpathlineto{\pgfqpoint{2.286246in}{2.281081in}}%
\pgfpathlineto{\pgfqpoint{3.003594in}{2.116231in}}%
\pgfpathlineto{\pgfqpoint{3.708276in}{1.828593in}}%
\pgfpathlineto{\pgfqpoint{4.421593in}{0.821322in}}%
\pgfpathlineto{\pgfqpoint{5.132586in}{0.741573in}}%
\pgfusepath{stroke}%
\end{pgfscope}%
\begin{pgfscope}%
\pgfpathrectangle{\pgfqpoint{0.721913in}{0.549073in}}{\pgfqpoint{4.518250in}{2.310000in}}%
\pgfusepath{clip}%
\pgfsetbuttcap%
\pgfsetmiterjoin%
\definecolor{currentfill}{rgb}{0.862745,0.149020,0.498039}%
\pgfsetfillcolor{currentfill}%
\pgfsetlinewidth{1.003750pt}%
\definecolor{currentstroke}{rgb}{0.862745,0.149020,0.498039}%
\pgfsetstrokecolor{currentstroke}%
\pgfsetdash{}{0pt}%
\pgfsys@defobject{currentmarker}{\pgfqpoint{-0.041667in}{-0.041667in}}{\pgfqpoint{0.041667in}{0.041667in}}{%
\pgfpathmoveto{\pgfqpoint{-0.041667in}{-0.041667in}}%
\pgfpathlineto{\pgfqpoint{0.041667in}{-0.041667in}}%
\pgfpathlineto{\pgfqpoint{0.041667in}{0.041667in}}%
\pgfpathlineto{\pgfqpoint{-0.041667in}{0.041667in}}%
\pgfpathlineto{\pgfqpoint{-0.041667in}{-0.041667in}}%
\pgfpathclose%
\pgfusepath{stroke,fill}%
}%
\begin{pgfscope}%
\pgfsys@transformshift{0.829491in}{2.430578in}%
\pgfsys@useobject{currentmarker}{}%
\end{pgfscope}%
\begin{pgfscope}%
\pgfsys@transformshift{1.544389in}{2.354200in}%
\pgfsys@useobject{currentmarker}{}%
\end{pgfscope}%
\begin{pgfscope}%
\pgfsys@transformshift{2.286246in}{2.281081in}%
\pgfsys@useobject{currentmarker}{}%
\end{pgfscope}%
\begin{pgfscope}%
\pgfsys@transformshift{3.003594in}{2.116231in}%
\pgfsys@useobject{currentmarker}{}%
\end{pgfscope}%
\begin{pgfscope}%
\pgfsys@transformshift{3.708276in}{1.828593in}%
\pgfsys@useobject{currentmarker}{}%
\end{pgfscope}%
\begin{pgfscope}%
\pgfsys@transformshift{4.421593in}{0.821322in}%
\pgfsys@useobject{currentmarker}{}%
\end{pgfscope}%
\begin{pgfscope}%
\pgfsys@transformshift{5.132586in}{0.741573in}%
\pgfsys@useobject{currentmarker}{}%
\end{pgfscope}%
\end{pgfscope}%
\begin{pgfscope}%
\pgfpathrectangle{\pgfqpoint{0.721913in}{0.549073in}}{\pgfqpoint{4.518250in}{2.310000in}}%
\pgfusepath{clip}%
\pgfsetrectcap%
\pgfsetroundjoin%
\pgfsetlinewidth{1.505625pt}%
\definecolor{currentstroke}{rgb}{1.000000,0.690196,0.000000}%
\pgfsetstrokecolor{currentstroke}%
\pgfsetdash{}{0pt}%
\pgfpathmoveto{\pgfqpoint{0.829491in}{2.666573in}}%
\pgfpathlineto{\pgfqpoint{1.544389in}{2.612107in}}%
\pgfpathlineto{\pgfqpoint{2.286246in}{2.507105in}}%
\pgfpathlineto{\pgfqpoint{3.003594in}{2.275009in}}%
\pgfpathlineto{\pgfqpoint{3.708276in}{1.153463in}}%
\pgfpathlineto{\pgfqpoint{4.421593in}{1.026854in}}%
\pgfpathlineto{\pgfqpoint{5.132586in}{0.992774in}}%
\pgfusepath{stroke}%
\end{pgfscope}%
\begin{pgfscope}%
\pgfpathrectangle{\pgfqpoint{0.721913in}{0.549073in}}{\pgfqpoint{4.518250in}{2.310000in}}%
\pgfusepath{clip}%
\pgfsetbuttcap%
\pgfsetmiterjoin%
\definecolor{currentfill}{rgb}{1.000000,0.690196,0.000000}%
\pgfsetfillcolor{currentfill}%
\pgfsetlinewidth{1.003750pt}%
\definecolor{currentstroke}{rgb}{1.000000,0.690196,0.000000}%
\pgfsetstrokecolor{currentstroke}%
\pgfsetdash{}{0pt}%
\pgfsys@defobject{currentmarker}{\pgfqpoint{-0.035355in}{-0.058926in}}{\pgfqpoint{0.035355in}{0.058926in}}{%
\pgfpathmoveto{\pgfqpoint{-0.000000in}{-0.058926in}}%
\pgfpathlineto{\pgfqpoint{0.035355in}{0.000000in}}%
\pgfpathlineto{\pgfqpoint{0.000000in}{0.058926in}}%
\pgfpathlineto{\pgfqpoint{-0.035355in}{0.000000in}}%
\pgfpathlineto{\pgfqpoint{-0.000000in}{-0.058926in}}%
\pgfpathclose%
\pgfusepath{stroke,fill}%
}%
\begin{pgfscope}%
\pgfsys@transformshift{0.829491in}{2.666573in}%
\pgfsys@useobject{currentmarker}{}%
\end{pgfscope}%
\begin{pgfscope}%
\pgfsys@transformshift{1.544389in}{2.612107in}%
\pgfsys@useobject{currentmarker}{}%
\end{pgfscope}%
\begin{pgfscope}%
\pgfsys@transformshift{2.286246in}{2.507105in}%
\pgfsys@useobject{currentmarker}{}%
\end{pgfscope}%
\begin{pgfscope}%
\pgfsys@transformshift{3.003594in}{2.275009in}%
\pgfsys@useobject{currentmarker}{}%
\end{pgfscope}%
\begin{pgfscope}%
\pgfsys@transformshift{3.708276in}{1.153463in}%
\pgfsys@useobject{currentmarker}{}%
\end{pgfscope}%
\begin{pgfscope}%
\pgfsys@transformshift{4.421593in}{1.026854in}%
\pgfsys@useobject{currentmarker}{}%
\end{pgfscope}%
\begin{pgfscope}%
\pgfsys@transformshift{5.132586in}{0.992774in}%
\pgfsys@useobject{currentmarker}{}%
\end{pgfscope}%
\end{pgfscope}%
\begin{pgfscope}%
\pgfpathrectangle{\pgfqpoint{0.721913in}{0.549073in}}{\pgfqpoint{4.518250in}{2.310000in}}%
\pgfusepath{clip}%
\pgfsetbuttcap%
\pgfsetroundjoin%
\pgfsetlinewidth{1.505625pt}%
\definecolor{currentstroke}{rgb}{0.478431,0.478431,0.478431}%
\pgfsetstrokecolor{currentstroke}%
\pgfsetstrokeopacity{0.500000}%
\pgfsetdash{{5.550000pt}{2.400000pt}}{0.000000pt}%
\pgfpathmoveto{\pgfqpoint{0.829491in}{2.361348in}}%
\pgfpathlineto{\pgfqpoint{1.544389in}{2.266216in}}%
\pgfpathlineto{\pgfqpoint{2.286246in}{2.167496in}}%
\pgfpathlineto{\pgfqpoint{3.003594in}{2.072037in}}%
\pgfpathlineto{\pgfqpoint{3.708276in}{1.978264in}}%
\pgfpathlineto{\pgfqpoint{4.421593in}{1.883342in}}%
\pgfpathlineto{\pgfqpoint{5.132586in}{1.788729in}}%
\pgfusepath{stroke}%
\end{pgfscope}%
\begin{pgfscope}%
\pgfpathrectangle{\pgfqpoint{0.721913in}{0.549073in}}{\pgfqpoint{4.518250in}{2.310000in}}%
\pgfusepath{clip}%
\pgfsetbuttcap%
\pgfsetroundjoin%
\pgfsetlinewidth{1.505625pt}%
\definecolor{currentstroke}{rgb}{0.478431,0.478431,0.478431}%
\pgfsetstrokecolor{currentstroke}%
\pgfsetstrokeopacity{0.500000}%
\pgfsetdash{{5.550000pt}{2.400000pt}}{0.000000pt}%
\pgfpathmoveto{\pgfqpoint{0.829491in}{2.502498in}}%
\pgfpathlineto{\pgfqpoint{1.544389in}{2.454932in}}%
\pgfpathlineto{\pgfqpoint{2.286246in}{2.405572in}}%
\pgfpathlineto{\pgfqpoint{3.003594in}{2.357843in}}%
\pgfpathlineto{\pgfqpoint{3.708276in}{2.310956in}}%
\pgfpathlineto{\pgfqpoint{4.421593in}{2.263495in}}%
\pgfpathlineto{\pgfqpoint{5.132586in}{2.216189in}}%
\pgfusepath{stroke}%
\end{pgfscope}%
\begin{pgfscope}%
\pgfsetrectcap%
\pgfsetmiterjoin%
\pgfsetlinewidth{0.803000pt}%
\definecolor{currentstroke}{rgb}{0.000000,0.000000,0.000000}%
\pgfsetstrokecolor{currentstroke}%
\pgfsetdash{}{0pt}%
\pgfpathmoveto{\pgfqpoint{0.721913in}{0.549073in}}%
\pgfpathlineto{\pgfqpoint{0.721913in}{2.859073in}}%
\pgfusepath{stroke}%
\end{pgfscope}%
\begin{pgfscope}%
\pgfsetrectcap%
\pgfsetmiterjoin%
\pgfsetlinewidth{0.803000pt}%
\definecolor{currentstroke}{rgb}{0.000000,0.000000,0.000000}%
\pgfsetstrokecolor{currentstroke}%
\pgfsetdash{}{0pt}%
\pgfpathmoveto{\pgfqpoint{5.240163in}{0.549073in}}%
\pgfpathlineto{\pgfqpoint{5.240163in}{2.859073in}}%
\pgfusepath{stroke}%
\end{pgfscope}%
\begin{pgfscope}%
\pgfsetrectcap%
\pgfsetmiterjoin%
\pgfsetlinewidth{0.803000pt}%
\definecolor{currentstroke}{rgb}{0.000000,0.000000,0.000000}%
\pgfsetstrokecolor{currentstroke}%
\pgfsetdash{}{0pt}%
\pgfpathmoveto{\pgfqpoint{0.721913in}{0.549073in}}%
\pgfpathlineto{\pgfqpoint{5.240163in}{0.549073in}}%
\pgfusepath{stroke}%
\end{pgfscope}%
\begin{pgfscope}%
\pgfsetrectcap%
\pgfsetmiterjoin%
\pgfsetlinewidth{0.803000pt}%
\definecolor{currentstroke}{rgb}{0.000000,0.000000,0.000000}%
\pgfsetstrokecolor{currentstroke}%
\pgfsetdash{}{0pt}%
\pgfpathmoveto{\pgfqpoint{0.721913in}{2.859073in}}%
\pgfpathlineto{\pgfqpoint{5.240163in}{2.859073in}}%
\pgfusepath{stroke}%
\end{pgfscope}%
\begin{pgfscope}%
\definecolor{textcolor}{rgb}{0.478431,0.478431,0.478431}%
\pgfsetstrokecolor{textcolor}%
\pgfsetfillcolor{textcolor}%
\pgftext[x=3.224476in,y=2.059074in,left,base]{\color{textcolor}{\rmfamily\fontsize{12.000000}{14.400000}\selectfont\catcode`\^=\active\def^{\ifmmode\sp\else\^{}\fi}\catcode`\%=\active\def%{\%}$\mathcal{O}(\varepsilon^{-1})$}}%
\end{pgfscope}%
\begin{pgfscope}%
\definecolor{textcolor}{rgb}{0.478431,0.478431,0.478431}%
\pgfsetstrokecolor{textcolor}%
\pgfsetfillcolor{textcolor}%
\pgftext[x=2.958483in,y=2.421355in,left,base]{\color{textcolor}{\rmfamily\fontsize{12.000000}{14.400000}\selectfont\catcode`\^=\active\def^{\ifmmode\sp\else\^{}\fi}\catcode`\%=\active\def%{\%}$\mathcal{O}(\varepsilon^{-2})$}}%
\end{pgfscope}%
\begin{pgfscope}%
\pgfsetbuttcap%
\pgfsetmiterjoin%
\definecolor{currentfill}{rgb}{1.000000,1.000000,1.000000}%
\pgfsetfillcolor{currentfill}%
\pgfsetfillopacity{0.800000}%
\pgfsetlinewidth{1.003750pt}%
\definecolor{currentstroke}{rgb}{0.800000,0.800000,0.800000}%
\pgfsetstrokecolor{currentstroke}%
\pgfsetstrokeopacity{0.800000}%
\pgfsetdash{}{0pt}%
\pgfpathmoveto{\pgfqpoint{0.805247in}{0.632406in}}%
\pgfpathlineto{\pgfqpoint{2.037651in}{0.632406in}}%
\pgfpathlineto{\pgfqpoint{2.037651in}{1.379627in}}%
\pgfpathlineto{\pgfqpoint{0.805247in}{1.379627in}}%
\pgfpathlineto{\pgfqpoint{0.805247in}{0.632406in}}%
\pgfpathclose%
\pgfusepath{stroke,fill}%
\end{pgfscope}%
\begin{pgfscope}%
\pgfsetrectcap%
\pgfsetroundjoin%
\pgfsetlinewidth{1.505625pt}%
\definecolor{currentstroke}{rgb}{0.392157,0.560784,1.000000}%
\pgfsetstrokecolor{currentstroke}%
\pgfsetdash{}{0pt}%
\pgfpathmoveto{\pgfqpoint{0.871913in}{1.254627in}}%
\pgfpathlineto{\pgfqpoint{1.038580in}{1.254627in}}%
\pgfpathlineto{\pgfqpoint{1.205247in}{1.254627in}}%
\pgfusepath{stroke}%
\end{pgfscope}%
\begin{pgfscope}%
\pgfsetbuttcap%
\pgfsetroundjoin%
\definecolor{currentfill}{rgb}{0.392157,0.560784,1.000000}%
\pgfsetfillcolor{currentfill}%
\pgfsetlinewidth{1.003750pt}%
\definecolor{currentstroke}{rgb}{0.392157,0.560784,1.000000}%
\pgfsetstrokecolor{currentstroke}%
\pgfsetdash{}{0pt}%
\pgfsys@defobject{currentmarker}{\pgfqpoint{-0.031250in}{-0.031250in}}{\pgfqpoint{0.031250in}{0.031250in}}{%
\pgfpathmoveto{\pgfqpoint{0.000000in}{-0.031250in}}%
\pgfpathcurveto{\pgfqpoint{0.008288in}{-0.031250in}}{\pgfqpoint{0.016237in}{-0.027957in}}{\pgfqpoint{0.022097in}{-0.022097in}}%
\pgfpathcurveto{\pgfqpoint{0.027957in}{-0.016237in}}{\pgfqpoint{0.031250in}{-0.008288in}}{\pgfqpoint{0.031250in}{0.000000in}}%
\pgfpathcurveto{\pgfqpoint{0.031250in}{0.008288in}}{\pgfqpoint{0.027957in}{0.016237in}}{\pgfqpoint{0.022097in}{0.022097in}}%
\pgfpathcurveto{\pgfqpoint{0.016237in}{0.027957in}}{\pgfqpoint{0.008288in}{0.031250in}}{\pgfqpoint{0.000000in}{0.031250in}}%
\pgfpathcurveto{\pgfqpoint{-0.008288in}{0.031250in}}{\pgfqpoint{-0.016237in}{0.027957in}}{\pgfqpoint{-0.022097in}{0.022097in}}%
\pgfpathcurveto{\pgfqpoint{-0.027957in}{0.016237in}}{\pgfqpoint{-0.031250in}{0.008288in}}{\pgfqpoint{-0.031250in}{0.000000in}}%
\pgfpathcurveto{\pgfqpoint{-0.031250in}{-0.008288in}}{\pgfqpoint{-0.027957in}{-0.016237in}}{\pgfqpoint{-0.022097in}{-0.022097in}}%
\pgfpathcurveto{\pgfqpoint{-0.016237in}{-0.027957in}}{\pgfqpoint{-0.008288in}{-0.031250in}}{\pgfqpoint{0.000000in}{-0.031250in}}%
\pgfpathlineto{\pgfqpoint{0.000000in}{-0.031250in}}%
\pgfpathclose%
\pgfusepath{stroke,fill}%
}%
\begin{pgfscope}%
\pgfsys@transformshift{1.038580in}{1.254627in}%
\pgfsys@useobject{currentmarker}{}%
\end{pgfscope}%
\end{pgfscope}%
\begin{pgfscope}%
\definecolor{textcolor}{rgb}{0.000000,0.000000,0.000000}%
\pgfsetstrokecolor{textcolor}%
\pgfsetfillcolor{textcolor}%
\pgftext[x=1.338580in,y=1.196294in,left,base]{\color{textcolor}{\rmfamily\fontsize{12.000000}{14.400000}\selectfont\catcode`\^=\active\def^{\ifmmode\sp\else\^{}\fi}\catcode`\%=\active\def%{\%}$n_{\mathbf{\Omega}} = 0$}}%
\end{pgfscope}%
\begin{pgfscope}%
\pgfsetrectcap%
\pgfsetroundjoin%
\pgfsetlinewidth{1.505625pt}%
\definecolor{currentstroke}{rgb}{0.862745,0.149020,0.498039}%
\pgfsetstrokecolor{currentstroke}%
\pgfsetdash{}{0pt}%
\pgfpathmoveto{\pgfqpoint{0.871913in}{1.022220in}}%
\pgfpathlineto{\pgfqpoint{1.038580in}{1.022220in}}%
\pgfpathlineto{\pgfqpoint{1.205247in}{1.022220in}}%
\pgfusepath{stroke}%
\end{pgfscope}%
\begin{pgfscope}%
\pgfsetbuttcap%
\pgfsetmiterjoin%
\definecolor{currentfill}{rgb}{0.862745,0.149020,0.498039}%
\pgfsetfillcolor{currentfill}%
\pgfsetlinewidth{1.003750pt}%
\definecolor{currentstroke}{rgb}{0.862745,0.149020,0.498039}%
\pgfsetstrokecolor{currentstroke}%
\pgfsetdash{}{0pt}%
\pgfsys@defobject{currentmarker}{\pgfqpoint{-0.031250in}{-0.031250in}}{\pgfqpoint{0.031250in}{0.031250in}}{%
\pgfpathmoveto{\pgfqpoint{-0.031250in}{-0.031250in}}%
\pgfpathlineto{\pgfqpoint{0.031250in}{-0.031250in}}%
\pgfpathlineto{\pgfqpoint{0.031250in}{0.031250in}}%
\pgfpathlineto{\pgfqpoint{-0.031250in}{0.031250in}}%
\pgfpathlineto{\pgfqpoint{-0.031250in}{-0.031250in}}%
\pgfpathclose%
\pgfusepath{stroke,fill}%
}%
\begin{pgfscope}%
\pgfsys@transformshift{1.038580in}{1.022220in}%
\pgfsys@useobject{currentmarker}{}%
\end{pgfscope}%
\end{pgfscope}%
\begin{pgfscope}%
\definecolor{textcolor}{rgb}{0.000000,0.000000,0.000000}%
\pgfsetstrokecolor{textcolor}%
\pgfsetfillcolor{textcolor}%
\pgftext[x=1.338580in,y=0.963887in,left,base]{\color{textcolor}{\rmfamily\fontsize{12.000000}{14.400000}\selectfont\catcode`\^=\active\def^{\ifmmode\sp\else\^{}\fi}\catcode`\%=\active\def%{\%}$n_{\mathbf{\Psi}} = n_{\mathbf{\Omega}}$}}%
\end{pgfscope}%
\begin{pgfscope}%
\pgfsetrectcap%
\pgfsetroundjoin%
\pgfsetlinewidth{1.505625pt}%
\definecolor{currentstroke}{rgb}{1.000000,0.690196,0.000000}%
\pgfsetstrokecolor{currentstroke}%
\pgfsetdash{}{0pt}%
\pgfpathmoveto{\pgfqpoint{0.871913in}{0.789813in}}%
\pgfpathlineto{\pgfqpoint{1.038580in}{0.789813in}}%
\pgfpathlineto{\pgfqpoint{1.205247in}{0.789813in}}%
\pgfusepath{stroke}%
\end{pgfscope}%
\begin{pgfscope}%
\pgfsetbuttcap%
\pgfsetmiterjoin%
\definecolor{currentfill}{rgb}{1.000000,0.690196,0.000000}%
\pgfsetfillcolor{currentfill}%
\pgfsetlinewidth{1.003750pt}%
\definecolor{currentstroke}{rgb}{1.000000,0.690196,0.000000}%
\pgfsetstrokecolor{currentstroke}%
\pgfsetdash{}{0pt}%
\pgfsys@defobject{currentmarker}{\pgfqpoint{-0.026517in}{-0.044194in}}{\pgfqpoint{0.026517in}{0.044194in}}{%
\pgfpathmoveto{\pgfqpoint{-0.000000in}{-0.044194in}}%
\pgfpathlineto{\pgfqpoint{0.026517in}{0.000000in}}%
\pgfpathlineto{\pgfqpoint{0.000000in}{0.044194in}}%
\pgfpathlineto{\pgfqpoint{-0.026517in}{0.000000in}}%
\pgfpathlineto{\pgfqpoint{-0.000000in}{-0.044194in}}%
\pgfpathclose%
\pgfusepath{stroke,fill}%
}%
\begin{pgfscope}%
\pgfsys@transformshift{1.038580in}{0.789813in}%
\pgfsys@useobject{currentmarker}{}%
\end{pgfscope}%
\end{pgfscope}%
\begin{pgfscope}%
\definecolor{textcolor}{rgb}{0.000000,0.000000,0.000000}%
\pgfsetstrokecolor{textcolor}%
\pgfsetfillcolor{textcolor}%
\pgftext[x=1.338580in,y=0.731480in,left,base]{\color{textcolor}{\rmfamily\fontsize{12.000000}{14.400000}\selectfont\catcode`\^=\active\def^{\ifmmode\sp\else\^{}\fi}\catcode`\%=\active\def%{\%}$n_{\mathbf{\Psi}} = 0$}}%
\end{pgfscope}%
\end{pgfpicture}%
\makeatother%
\endgroup%

    \caption{For increasing values of $n_{\mtx{\Psi}} + n_{\mtx{\Omega}}$ but fixed $m$, we apply the Chebyshev-Nyström++ method to the Hamiltonian matrix described in \refsec{subsec:hamiltonian}. We plot the $L^1$-approximation error for $\sigma=0.005$ on the model problem with $n_c = 1$.}
    \label{fig:convergence}
\end{figure}

Indeed, we observe that for $n_{\mtx{\Omega}} = 0$, we require $n_{\mtx{\Psi}} = \mathcal{O}(\varepsilon^{-2})$ to achieve an error of order $\varepsilon$, as shown in \refthm{thm:hutchinson}. Because the eigenvalues of the matrix are quite uniformly distributed, we observe the stronger convergence of the Nyström approximation discussed in the end of \refsec{sec:application} once $n_{\mtx{\Omega}}$ is large enough.

For a fixed budget of random vectors $n_{\mtx{\Omega}} + n_{\mtx{\Psi}}$, \reffig{fig:distribution} visualizes how for different values of the smoothing parameter $\sigma$ the random vectors should either be invested into the Nyström approximation $n_{\mtx{\Omega}}$ or the estimation of the residual trace $n_{\mtx{\Psi}}$. Again, since for small $\sigma$ the matrix function $g_{\sigma}^{(m)}(t \mtx{I}_n - \mtx{A})$ has a small numerical rank, a Nyström approximation alone is already enough to achieve a good approximation. On the other hand, when $\sigma$ is large, the Nyström approximation by itself is not effective, and the correction with the residual trace becomes indispensable.

\begin{figure}[ht]
    \centering
    %% Creator: Matplotlib, PGF backend
%%
%% To include the figure in your LaTeX document, write
%%   \input{<filename>.pgf}
%%
%% Make sure the required packages are loaded in your preamble
%%   \usepackage{pgf}
%%
%% Also ensure that all the required font packages are loaded; for instance,
%% the lmodern package is sometimes necessary when using math font.
%%   \usepackage{lmodern}
%%
%% Figures using additional raster images can only be included by \input if
%% they are in the same directory as the main LaTeX file. For loading figures
%% from other directories you can use the `import` package
%%   \usepackage{import}
%%
%% and then include the figures with
%%   \import{<path to file>}{<filename>.pgf}
%%
%% Matplotlib used the following preamble
%%   \def\mathdefault#1{#1}
%%   \everymath=\expandafter{\the\everymath\displaystyle}
%%   
%%   \ifdefined\pdftexversion\else  % non-pdftex case.
%%     \usepackage{fontspec}
%%     \setmainfont{DejaVuSerif.ttf}[Path=\detokenize{/opt/hostedtoolcache/Python/3.12.7/x64/lib/python3.12/site-packages/matplotlib/mpl-data/fonts/ttf/}]
%%     \setsansfont{DejaVuSans.ttf}[Path=\detokenize{/opt/hostedtoolcache/Python/3.12.7/x64/lib/python3.12/site-packages/matplotlib/mpl-data/fonts/ttf/}]
%%     \setmonofont{DejaVuSansMono.ttf}[Path=\detokenize{/opt/hostedtoolcache/Python/3.12.7/x64/lib/python3.12/site-packages/matplotlib/mpl-data/fonts/ttf/}]
%%   \fi
%%   \makeatletter\@ifpackageloaded{underscore}{}{\usepackage[strings]{underscore}}\makeatother
%%
\begingroup%
\makeatletter%
\begin{pgfpicture}%
\pgfpathrectangle{\pgfpointorigin}{\pgfqpoint{5.393080in}{2.959073in}}%
\pgfusepath{use as bounding box, clip}%
\begin{pgfscope}%
\pgfsetbuttcap%
\pgfsetmiterjoin%
\definecolor{currentfill}{rgb}{1.000000,1.000000,1.000000}%
\pgfsetfillcolor{currentfill}%
\pgfsetlinewidth{0.000000pt}%
\definecolor{currentstroke}{rgb}{1.000000,1.000000,1.000000}%
\pgfsetstrokecolor{currentstroke}%
\pgfsetdash{}{0pt}%
\pgfpathmoveto{\pgfqpoint{0.000000in}{-0.000000in}}%
\pgfpathlineto{\pgfqpoint{5.393080in}{-0.000000in}}%
\pgfpathlineto{\pgfqpoint{5.393080in}{2.959073in}}%
\pgfpathlineto{\pgfqpoint{0.000000in}{2.959073in}}%
\pgfpathlineto{\pgfqpoint{0.000000in}{-0.000000in}}%
\pgfpathclose%
\pgfusepath{fill}%
\end{pgfscope}%
\begin{pgfscope}%
\pgfsetbuttcap%
\pgfsetmiterjoin%
\definecolor{currentfill}{rgb}{1.000000,1.000000,1.000000}%
\pgfsetfillcolor{currentfill}%
\pgfsetlinewidth{0.000000pt}%
\definecolor{currentstroke}{rgb}{0.000000,0.000000,0.000000}%
\pgfsetstrokecolor{currentstroke}%
\pgfsetstrokeopacity{0.000000}%
\pgfsetdash{}{0pt}%
\pgfpathmoveto{\pgfqpoint{0.721913in}{0.549073in}}%
\pgfpathlineto{\pgfqpoint{5.240163in}{0.549073in}}%
\pgfpathlineto{\pgfqpoint{5.240163in}{2.859073in}}%
\pgfpathlineto{\pgfqpoint{0.721913in}{2.859073in}}%
\pgfpathlineto{\pgfqpoint{0.721913in}{0.549073in}}%
\pgfpathclose%
\pgfusepath{fill}%
\end{pgfscope}%
\begin{pgfscope}%
\pgfpathrectangle{\pgfqpoint{0.721913in}{0.549073in}}{\pgfqpoint{4.518250in}{2.310000in}}%
\pgfusepath{clip}%
\pgfsetrectcap%
\pgfsetroundjoin%
\pgfsetlinewidth{0.250937pt}%
\definecolor{currentstroke}{rgb}{0.000000,0.000000,0.000000}%
\pgfsetstrokecolor{currentstroke}%
\pgfsetstrokeopacity{0.200000}%
\pgfsetdash{}{0pt}%
\pgfpathmoveto{\pgfqpoint{1.690110in}{0.549073in}}%
\pgfpathlineto{\pgfqpoint{1.690110in}{2.859073in}}%
\pgfusepath{stroke}%
\end{pgfscope}%
\begin{pgfscope}%
\pgfsetbuttcap%
\pgfsetroundjoin%
\definecolor{currentfill}{rgb}{0.000000,0.000000,0.000000}%
\pgfsetfillcolor{currentfill}%
\pgfsetlinewidth{0.803000pt}%
\definecolor{currentstroke}{rgb}{0.000000,0.000000,0.000000}%
\pgfsetstrokecolor{currentstroke}%
\pgfsetdash{}{0pt}%
\pgfsys@defobject{currentmarker}{\pgfqpoint{0.000000in}{-0.048611in}}{\pgfqpoint{0.000000in}{0.000000in}}{%
\pgfpathmoveto{\pgfqpoint{0.000000in}{0.000000in}}%
\pgfpathlineto{\pgfqpoint{0.000000in}{-0.048611in}}%
\pgfusepath{stroke,fill}%
}%
\begin{pgfscope}%
\pgfsys@transformshift{1.690110in}{0.549073in}%
\pgfsys@useobject{currentmarker}{}%
\end{pgfscope}%
\end{pgfscope}%
\begin{pgfscope}%
\definecolor{textcolor}{rgb}{0.000000,0.000000,0.000000}%
\pgfsetstrokecolor{textcolor}%
\pgfsetfillcolor{textcolor}%
\pgftext[x=1.690110in,y=0.451851in,,top]{\color{textcolor}{\rmfamily\fontsize{12.000000}{14.400000}\selectfont\catcode`\^=\active\def^{\ifmmode\sp\else\^{}\fi}\catcode`\%=\active\def%{\%}$\mathdefault{10^{-3}}$}}%
\end{pgfscope}%
\begin{pgfscope}%
\pgfpathrectangle{\pgfqpoint{0.721913in}{0.549073in}}{\pgfqpoint{4.518250in}{2.310000in}}%
\pgfusepath{clip}%
\pgfsetrectcap%
\pgfsetroundjoin%
\pgfsetlinewidth{0.250937pt}%
\definecolor{currentstroke}{rgb}{0.000000,0.000000,0.000000}%
\pgfsetstrokecolor{currentstroke}%
\pgfsetstrokeopacity{0.200000}%
\pgfsetdash{}{0pt}%
\pgfpathmoveto{\pgfqpoint{3.411348in}{0.549073in}}%
\pgfpathlineto{\pgfqpoint{3.411348in}{2.859073in}}%
\pgfusepath{stroke}%
\end{pgfscope}%
\begin{pgfscope}%
\pgfsetbuttcap%
\pgfsetroundjoin%
\definecolor{currentfill}{rgb}{0.000000,0.000000,0.000000}%
\pgfsetfillcolor{currentfill}%
\pgfsetlinewidth{0.803000pt}%
\definecolor{currentstroke}{rgb}{0.000000,0.000000,0.000000}%
\pgfsetstrokecolor{currentstroke}%
\pgfsetdash{}{0pt}%
\pgfsys@defobject{currentmarker}{\pgfqpoint{0.000000in}{-0.048611in}}{\pgfqpoint{0.000000in}{0.000000in}}{%
\pgfpathmoveto{\pgfqpoint{0.000000in}{0.000000in}}%
\pgfpathlineto{\pgfqpoint{0.000000in}{-0.048611in}}%
\pgfusepath{stroke,fill}%
}%
\begin{pgfscope}%
\pgfsys@transformshift{3.411348in}{0.549073in}%
\pgfsys@useobject{currentmarker}{}%
\end{pgfscope}%
\end{pgfscope}%
\begin{pgfscope}%
\definecolor{textcolor}{rgb}{0.000000,0.000000,0.000000}%
\pgfsetstrokecolor{textcolor}%
\pgfsetfillcolor{textcolor}%
\pgftext[x=3.411348in,y=0.451851in,,top]{\color{textcolor}{\rmfamily\fontsize{12.000000}{14.400000}\selectfont\catcode`\^=\active\def^{\ifmmode\sp\else\^{}\fi}\catcode`\%=\active\def%{\%}$\mathdefault{10^{-2}}$}}%
\end{pgfscope}%
\begin{pgfscope}%
\pgfpathrectangle{\pgfqpoint{0.721913in}{0.549073in}}{\pgfqpoint{4.518250in}{2.310000in}}%
\pgfusepath{clip}%
\pgfsetrectcap%
\pgfsetroundjoin%
\pgfsetlinewidth{0.250937pt}%
\definecolor{currentstroke}{rgb}{0.000000,0.000000,0.000000}%
\pgfsetstrokecolor{currentstroke}%
\pgfsetstrokeopacity{0.200000}%
\pgfsetdash{}{0pt}%
\pgfpathmoveto{\pgfqpoint{5.132586in}{0.549073in}}%
\pgfpathlineto{\pgfqpoint{5.132586in}{2.859073in}}%
\pgfusepath{stroke}%
\end{pgfscope}%
\begin{pgfscope}%
\pgfsetbuttcap%
\pgfsetroundjoin%
\definecolor{currentfill}{rgb}{0.000000,0.000000,0.000000}%
\pgfsetfillcolor{currentfill}%
\pgfsetlinewidth{0.803000pt}%
\definecolor{currentstroke}{rgb}{0.000000,0.000000,0.000000}%
\pgfsetstrokecolor{currentstroke}%
\pgfsetdash{}{0pt}%
\pgfsys@defobject{currentmarker}{\pgfqpoint{0.000000in}{-0.048611in}}{\pgfqpoint{0.000000in}{0.000000in}}{%
\pgfpathmoveto{\pgfqpoint{0.000000in}{0.000000in}}%
\pgfpathlineto{\pgfqpoint{0.000000in}{-0.048611in}}%
\pgfusepath{stroke,fill}%
}%
\begin{pgfscope}%
\pgfsys@transformshift{5.132586in}{0.549073in}%
\pgfsys@useobject{currentmarker}{}%
\end{pgfscope}%
\end{pgfscope}%
\begin{pgfscope}%
\definecolor{textcolor}{rgb}{0.000000,0.000000,0.000000}%
\pgfsetstrokecolor{textcolor}%
\pgfsetfillcolor{textcolor}%
\pgftext[x=5.132586in,y=0.451851in,,top]{\color{textcolor}{\rmfamily\fontsize{12.000000}{14.400000}\selectfont\catcode`\^=\active\def^{\ifmmode\sp\else\^{}\fi}\catcode`\%=\active\def%{\%}$\mathdefault{10^{-1}}$}}%
\end{pgfscope}%
\begin{pgfscope}%
\pgfpathrectangle{\pgfqpoint{0.721913in}{0.549073in}}{\pgfqpoint{4.518250in}{2.310000in}}%
\pgfusepath{clip}%
\pgfsetrectcap%
\pgfsetroundjoin%
\pgfsetlinewidth{0.250937pt}%
\definecolor{currentstroke}{rgb}{0.000000,0.000000,0.000000}%
\pgfsetstrokecolor{currentstroke}%
\pgfsetstrokeopacity{0.200000}%
\pgfsetdash{}{0pt}%
\pgfpathmoveto{\pgfqpoint{0.790111in}{0.549073in}}%
\pgfpathlineto{\pgfqpoint{0.790111in}{2.859073in}}%
\pgfusepath{stroke}%
\end{pgfscope}%
\begin{pgfscope}%
\pgfsetbuttcap%
\pgfsetroundjoin%
\definecolor{currentfill}{rgb}{0.000000,0.000000,0.000000}%
\pgfsetfillcolor{currentfill}%
\pgfsetlinewidth{0.602250pt}%
\definecolor{currentstroke}{rgb}{0.000000,0.000000,0.000000}%
\pgfsetstrokecolor{currentstroke}%
\pgfsetdash{}{0pt}%
\pgfsys@defobject{currentmarker}{\pgfqpoint{0.000000in}{-0.027778in}}{\pgfqpoint{0.000000in}{0.000000in}}{%
\pgfpathmoveto{\pgfqpoint{0.000000in}{0.000000in}}%
\pgfpathlineto{\pgfqpoint{0.000000in}{-0.027778in}}%
\pgfusepath{stroke,fill}%
}%
\begin{pgfscope}%
\pgfsys@transformshift{0.790111in}{0.549073in}%
\pgfsys@useobject{currentmarker}{}%
\end{pgfscope}%
\end{pgfscope}%
\begin{pgfscope}%
\pgfpathrectangle{\pgfqpoint{0.721913in}{0.549073in}}{\pgfqpoint{4.518250in}{2.310000in}}%
\pgfusepath{clip}%
\pgfsetrectcap%
\pgfsetroundjoin%
\pgfsetlinewidth{0.250937pt}%
\definecolor{currentstroke}{rgb}{0.000000,0.000000,0.000000}%
\pgfsetstrokecolor{currentstroke}%
\pgfsetstrokeopacity{0.200000}%
\pgfsetdash{}{0pt}%
\pgfpathmoveto{\pgfqpoint{1.005160in}{0.549073in}}%
\pgfpathlineto{\pgfqpoint{1.005160in}{2.859073in}}%
\pgfusepath{stroke}%
\end{pgfscope}%
\begin{pgfscope}%
\pgfsetbuttcap%
\pgfsetroundjoin%
\definecolor{currentfill}{rgb}{0.000000,0.000000,0.000000}%
\pgfsetfillcolor{currentfill}%
\pgfsetlinewidth{0.602250pt}%
\definecolor{currentstroke}{rgb}{0.000000,0.000000,0.000000}%
\pgfsetstrokecolor{currentstroke}%
\pgfsetdash{}{0pt}%
\pgfsys@defobject{currentmarker}{\pgfqpoint{0.000000in}{-0.027778in}}{\pgfqpoint{0.000000in}{0.000000in}}{%
\pgfpathmoveto{\pgfqpoint{0.000000in}{0.000000in}}%
\pgfpathlineto{\pgfqpoint{0.000000in}{-0.027778in}}%
\pgfusepath{stroke,fill}%
}%
\begin{pgfscope}%
\pgfsys@transformshift{1.005160in}{0.549073in}%
\pgfsys@useobject{currentmarker}{}%
\end{pgfscope}%
\end{pgfscope}%
\begin{pgfscope}%
\pgfpathrectangle{\pgfqpoint{0.721913in}{0.549073in}}{\pgfqpoint{4.518250in}{2.310000in}}%
\pgfusepath{clip}%
\pgfsetrectcap%
\pgfsetroundjoin%
\pgfsetlinewidth{0.250937pt}%
\definecolor{currentstroke}{rgb}{0.000000,0.000000,0.000000}%
\pgfsetstrokecolor{currentstroke}%
\pgfsetstrokeopacity{0.200000}%
\pgfsetdash{}{0pt}%
\pgfpathmoveto{\pgfqpoint{1.171966in}{0.549073in}}%
\pgfpathlineto{\pgfqpoint{1.171966in}{2.859073in}}%
\pgfusepath{stroke}%
\end{pgfscope}%
\begin{pgfscope}%
\pgfsetbuttcap%
\pgfsetroundjoin%
\definecolor{currentfill}{rgb}{0.000000,0.000000,0.000000}%
\pgfsetfillcolor{currentfill}%
\pgfsetlinewidth{0.602250pt}%
\definecolor{currentstroke}{rgb}{0.000000,0.000000,0.000000}%
\pgfsetstrokecolor{currentstroke}%
\pgfsetdash{}{0pt}%
\pgfsys@defobject{currentmarker}{\pgfqpoint{0.000000in}{-0.027778in}}{\pgfqpoint{0.000000in}{0.000000in}}{%
\pgfpathmoveto{\pgfqpoint{0.000000in}{0.000000in}}%
\pgfpathlineto{\pgfqpoint{0.000000in}{-0.027778in}}%
\pgfusepath{stroke,fill}%
}%
\begin{pgfscope}%
\pgfsys@transformshift{1.171966in}{0.549073in}%
\pgfsys@useobject{currentmarker}{}%
\end{pgfscope}%
\end{pgfscope}%
\begin{pgfscope}%
\pgfpathrectangle{\pgfqpoint{0.721913in}{0.549073in}}{\pgfqpoint{4.518250in}{2.310000in}}%
\pgfusepath{clip}%
\pgfsetrectcap%
\pgfsetroundjoin%
\pgfsetlinewidth{0.250937pt}%
\definecolor{currentstroke}{rgb}{0.000000,0.000000,0.000000}%
\pgfsetstrokecolor{currentstroke}%
\pgfsetstrokeopacity{0.200000}%
\pgfsetdash{}{0pt}%
\pgfpathmoveto{\pgfqpoint{1.308255in}{0.549073in}}%
\pgfpathlineto{\pgfqpoint{1.308255in}{2.859073in}}%
\pgfusepath{stroke}%
\end{pgfscope}%
\begin{pgfscope}%
\pgfsetbuttcap%
\pgfsetroundjoin%
\definecolor{currentfill}{rgb}{0.000000,0.000000,0.000000}%
\pgfsetfillcolor{currentfill}%
\pgfsetlinewidth{0.602250pt}%
\definecolor{currentstroke}{rgb}{0.000000,0.000000,0.000000}%
\pgfsetstrokecolor{currentstroke}%
\pgfsetdash{}{0pt}%
\pgfsys@defobject{currentmarker}{\pgfqpoint{0.000000in}{-0.027778in}}{\pgfqpoint{0.000000in}{0.000000in}}{%
\pgfpathmoveto{\pgfqpoint{0.000000in}{0.000000in}}%
\pgfpathlineto{\pgfqpoint{0.000000in}{-0.027778in}}%
\pgfusepath{stroke,fill}%
}%
\begin{pgfscope}%
\pgfsys@transformshift{1.308255in}{0.549073in}%
\pgfsys@useobject{currentmarker}{}%
\end{pgfscope}%
\end{pgfscope}%
\begin{pgfscope}%
\pgfpathrectangle{\pgfqpoint{0.721913in}{0.549073in}}{\pgfqpoint{4.518250in}{2.310000in}}%
\pgfusepath{clip}%
\pgfsetrectcap%
\pgfsetroundjoin%
\pgfsetlinewidth{0.250937pt}%
\definecolor{currentstroke}{rgb}{0.000000,0.000000,0.000000}%
\pgfsetstrokecolor{currentstroke}%
\pgfsetstrokeopacity{0.200000}%
\pgfsetdash{}{0pt}%
\pgfpathmoveto{\pgfqpoint{1.423487in}{0.549073in}}%
\pgfpathlineto{\pgfqpoint{1.423487in}{2.859073in}}%
\pgfusepath{stroke}%
\end{pgfscope}%
\begin{pgfscope}%
\pgfsetbuttcap%
\pgfsetroundjoin%
\definecolor{currentfill}{rgb}{0.000000,0.000000,0.000000}%
\pgfsetfillcolor{currentfill}%
\pgfsetlinewidth{0.602250pt}%
\definecolor{currentstroke}{rgb}{0.000000,0.000000,0.000000}%
\pgfsetstrokecolor{currentstroke}%
\pgfsetdash{}{0pt}%
\pgfsys@defobject{currentmarker}{\pgfqpoint{0.000000in}{-0.027778in}}{\pgfqpoint{0.000000in}{0.000000in}}{%
\pgfpathmoveto{\pgfqpoint{0.000000in}{0.000000in}}%
\pgfpathlineto{\pgfqpoint{0.000000in}{-0.027778in}}%
\pgfusepath{stroke,fill}%
}%
\begin{pgfscope}%
\pgfsys@transformshift{1.423487in}{0.549073in}%
\pgfsys@useobject{currentmarker}{}%
\end{pgfscope}%
\end{pgfscope}%
\begin{pgfscope}%
\pgfpathrectangle{\pgfqpoint{0.721913in}{0.549073in}}{\pgfqpoint{4.518250in}{2.310000in}}%
\pgfusepath{clip}%
\pgfsetrectcap%
\pgfsetroundjoin%
\pgfsetlinewidth{0.250937pt}%
\definecolor{currentstroke}{rgb}{0.000000,0.000000,0.000000}%
\pgfsetstrokecolor{currentstroke}%
\pgfsetstrokeopacity{0.200000}%
\pgfsetdash{}{0pt}%
\pgfpathmoveto{\pgfqpoint{1.523305in}{0.549073in}}%
\pgfpathlineto{\pgfqpoint{1.523305in}{2.859073in}}%
\pgfusepath{stroke}%
\end{pgfscope}%
\begin{pgfscope}%
\pgfsetbuttcap%
\pgfsetroundjoin%
\definecolor{currentfill}{rgb}{0.000000,0.000000,0.000000}%
\pgfsetfillcolor{currentfill}%
\pgfsetlinewidth{0.602250pt}%
\definecolor{currentstroke}{rgb}{0.000000,0.000000,0.000000}%
\pgfsetstrokecolor{currentstroke}%
\pgfsetdash{}{0pt}%
\pgfsys@defobject{currentmarker}{\pgfqpoint{0.000000in}{-0.027778in}}{\pgfqpoint{0.000000in}{0.000000in}}{%
\pgfpathmoveto{\pgfqpoint{0.000000in}{0.000000in}}%
\pgfpathlineto{\pgfqpoint{0.000000in}{-0.027778in}}%
\pgfusepath{stroke,fill}%
}%
\begin{pgfscope}%
\pgfsys@transformshift{1.523305in}{0.549073in}%
\pgfsys@useobject{currentmarker}{}%
\end{pgfscope}%
\end{pgfscope}%
\begin{pgfscope}%
\pgfpathrectangle{\pgfqpoint{0.721913in}{0.549073in}}{\pgfqpoint{4.518250in}{2.310000in}}%
\pgfusepath{clip}%
\pgfsetrectcap%
\pgfsetroundjoin%
\pgfsetlinewidth{0.250937pt}%
\definecolor{currentstroke}{rgb}{0.000000,0.000000,0.000000}%
\pgfsetstrokecolor{currentstroke}%
\pgfsetstrokeopacity{0.200000}%
\pgfsetdash{}{0pt}%
\pgfpathmoveto{\pgfqpoint{1.611350in}{0.549073in}}%
\pgfpathlineto{\pgfqpoint{1.611350in}{2.859073in}}%
\pgfusepath{stroke}%
\end{pgfscope}%
\begin{pgfscope}%
\pgfsetbuttcap%
\pgfsetroundjoin%
\definecolor{currentfill}{rgb}{0.000000,0.000000,0.000000}%
\pgfsetfillcolor{currentfill}%
\pgfsetlinewidth{0.602250pt}%
\definecolor{currentstroke}{rgb}{0.000000,0.000000,0.000000}%
\pgfsetstrokecolor{currentstroke}%
\pgfsetdash{}{0pt}%
\pgfsys@defobject{currentmarker}{\pgfqpoint{0.000000in}{-0.027778in}}{\pgfqpoint{0.000000in}{0.000000in}}{%
\pgfpathmoveto{\pgfqpoint{0.000000in}{0.000000in}}%
\pgfpathlineto{\pgfqpoint{0.000000in}{-0.027778in}}%
\pgfusepath{stroke,fill}%
}%
\begin{pgfscope}%
\pgfsys@transformshift{1.611350in}{0.549073in}%
\pgfsys@useobject{currentmarker}{}%
\end{pgfscope}%
\end{pgfscope}%
\begin{pgfscope}%
\pgfpathrectangle{\pgfqpoint{0.721913in}{0.549073in}}{\pgfqpoint{4.518250in}{2.310000in}}%
\pgfusepath{clip}%
\pgfsetrectcap%
\pgfsetroundjoin%
\pgfsetlinewidth{0.250937pt}%
\definecolor{currentstroke}{rgb}{0.000000,0.000000,0.000000}%
\pgfsetstrokecolor{currentstroke}%
\pgfsetstrokeopacity{0.200000}%
\pgfsetdash{}{0pt}%
\pgfpathmoveto{\pgfqpoint{2.208254in}{0.549073in}}%
\pgfpathlineto{\pgfqpoint{2.208254in}{2.859073in}}%
\pgfusepath{stroke}%
\end{pgfscope}%
\begin{pgfscope}%
\pgfsetbuttcap%
\pgfsetroundjoin%
\definecolor{currentfill}{rgb}{0.000000,0.000000,0.000000}%
\pgfsetfillcolor{currentfill}%
\pgfsetlinewidth{0.602250pt}%
\definecolor{currentstroke}{rgb}{0.000000,0.000000,0.000000}%
\pgfsetstrokecolor{currentstroke}%
\pgfsetdash{}{0pt}%
\pgfsys@defobject{currentmarker}{\pgfqpoint{0.000000in}{-0.027778in}}{\pgfqpoint{0.000000in}{0.000000in}}{%
\pgfpathmoveto{\pgfqpoint{0.000000in}{0.000000in}}%
\pgfpathlineto{\pgfqpoint{0.000000in}{-0.027778in}}%
\pgfusepath{stroke,fill}%
}%
\begin{pgfscope}%
\pgfsys@transformshift{2.208254in}{0.549073in}%
\pgfsys@useobject{currentmarker}{}%
\end{pgfscope}%
\end{pgfscope}%
\begin{pgfscope}%
\pgfpathrectangle{\pgfqpoint{0.721913in}{0.549073in}}{\pgfqpoint{4.518250in}{2.310000in}}%
\pgfusepath{clip}%
\pgfsetrectcap%
\pgfsetroundjoin%
\pgfsetlinewidth{0.250937pt}%
\definecolor{currentstroke}{rgb}{0.000000,0.000000,0.000000}%
\pgfsetstrokecolor{currentstroke}%
\pgfsetstrokeopacity{0.200000}%
\pgfsetdash{}{0pt}%
\pgfpathmoveto{\pgfqpoint{2.511349in}{0.549073in}}%
\pgfpathlineto{\pgfqpoint{2.511349in}{2.859073in}}%
\pgfusepath{stroke}%
\end{pgfscope}%
\begin{pgfscope}%
\pgfsetbuttcap%
\pgfsetroundjoin%
\definecolor{currentfill}{rgb}{0.000000,0.000000,0.000000}%
\pgfsetfillcolor{currentfill}%
\pgfsetlinewidth{0.602250pt}%
\definecolor{currentstroke}{rgb}{0.000000,0.000000,0.000000}%
\pgfsetstrokecolor{currentstroke}%
\pgfsetdash{}{0pt}%
\pgfsys@defobject{currentmarker}{\pgfqpoint{0.000000in}{-0.027778in}}{\pgfqpoint{0.000000in}{0.000000in}}{%
\pgfpathmoveto{\pgfqpoint{0.000000in}{0.000000in}}%
\pgfpathlineto{\pgfqpoint{0.000000in}{-0.027778in}}%
\pgfusepath{stroke,fill}%
}%
\begin{pgfscope}%
\pgfsys@transformshift{2.511349in}{0.549073in}%
\pgfsys@useobject{currentmarker}{}%
\end{pgfscope}%
\end{pgfscope}%
\begin{pgfscope}%
\pgfpathrectangle{\pgfqpoint{0.721913in}{0.549073in}}{\pgfqpoint{4.518250in}{2.310000in}}%
\pgfusepath{clip}%
\pgfsetrectcap%
\pgfsetroundjoin%
\pgfsetlinewidth{0.250937pt}%
\definecolor{currentstroke}{rgb}{0.000000,0.000000,0.000000}%
\pgfsetstrokecolor{currentstroke}%
\pgfsetstrokeopacity{0.200000}%
\pgfsetdash{}{0pt}%
\pgfpathmoveto{\pgfqpoint{2.726398in}{0.549073in}}%
\pgfpathlineto{\pgfqpoint{2.726398in}{2.859073in}}%
\pgfusepath{stroke}%
\end{pgfscope}%
\begin{pgfscope}%
\pgfsetbuttcap%
\pgfsetroundjoin%
\definecolor{currentfill}{rgb}{0.000000,0.000000,0.000000}%
\pgfsetfillcolor{currentfill}%
\pgfsetlinewidth{0.602250pt}%
\definecolor{currentstroke}{rgb}{0.000000,0.000000,0.000000}%
\pgfsetstrokecolor{currentstroke}%
\pgfsetdash{}{0pt}%
\pgfsys@defobject{currentmarker}{\pgfqpoint{0.000000in}{-0.027778in}}{\pgfqpoint{0.000000in}{0.000000in}}{%
\pgfpathmoveto{\pgfqpoint{0.000000in}{0.000000in}}%
\pgfpathlineto{\pgfqpoint{0.000000in}{-0.027778in}}%
\pgfusepath{stroke,fill}%
}%
\begin{pgfscope}%
\pgfsys@transformshift{2.726398in}{0.549073in}%
\pgfsys@useobject{currentmarker}{}%
\end{pgfscope}%
\end{pgfscope}%
\begin{pgfscope}%
\pgfpathrectangle{\pgfqpoint{0.721913in}{0.549073in}}{\pgfqpoint{4.518250in}{2.310000in}}%
\pgfusepath{clip}%
\pgfsetrectcap%
\pgfsetroundjoin%
\pgfsetlinewidth{0.250937pt}%
\definecolor{currentstroke}{rgb}{0.000000,0.000000,0.000000}%
\pgfsetstrokecolor{currentstroke}%
\pgfsetstrokeopacity{0.200000}%
\pgfsetdash{}{0pt}%
\pgfpathmoveto{\pgfqpoint{2.893204in}{0.549073in}}%
\pgfpathlineto{\pgfqpoint{2.893204in}{2.859073in}}%
\pgfusepath{stroke}%
\end{pgfscope}%
\begin{pgfscope}%
\pgfsetbuttcap%
\pgfsetroundjoin%
\definecolor{currentfill}{rgb}{0.000000,0.000000,0.000000}%
\pgfsetfillcolor{currentfill}%
\pgfsetlinewidth{0.602250pt}%
\definecolor{currentstroke}{rgb}{0.000000,0.000000,0.000000}%
\pgfsetstrokecolor{currentstroke}%
\pgfsetdash{}{0pt}%
\pgfsys@defobject{currentmarker}{\pgfqpoint{0.000000in}{-0.027778in}}{\pgfqpoint{0.000000in}{0.000000in}}{%
\pgfpathmoveto{\pgfqpoint{0.000000in}{0.000000in}}%
\pgfpathlineto{\pgfqpoint{0.000000in}{-0.027778in}}%
\pgfusepath{stroke,fill}%
}%
\begin{pgfscope}%
\pgfsys@transformshift{2.893204in}{0.549073in}%
\pgfsys@useobject{currentmarker}{}%
\end{pgfscope}%
\end{pgfscope}%
\begin{pgfscope}%
\pgfpathrectangle{\pgfqpoint{0.721913in}{0.549073in}}{\pgfqpoint{4.518250in}{2.310000in}}%
\pgfusepath{clip}%
\pgfsetrectcap%
\pgfsetroundjoin%
\pgfsetlinewidth{0.250937pt}%
\definecolor{currentstroke}{rgb}{0.000000,0.000000,0.000000}%
\pgfsetstrokecolor{currentstroke}%
\pgfsetstrokeopacity{0.200000}%
\pgfsetdash{}{0pt}%
\pgfpathmoveto{\pgfqpoint{3.029493in}{0.549073in}}%
\pgfpathlineto{\pgfqpoint{3.029493in}{2.859073in}}%
\pgfusepath{stroke}%
\end{pgfscope}%
\begin{pgfscope}%
\pgfsetbuttcap%
\pgfsetroundjoin%
\definecolor{currentfill}{rgb}{0.000000,0.000000,0.000000}%
\pgfsetfillcolor{currentfill}%
\pgfsetlinewidth{0.602250pt}%
\definecolor{currentstroke}{rgb}{0.000000,0.000000,0.000000}%
\pgfsetstrokecolor{currentstroke}%
\pgfsetdash{}{0pt}%
\pgfsys@defobject{currentmarker}{\pgfqpoint{0.000000in}{-0.027778in}}{\pgfqpoint{0.000000in}{0.000000in}}{%
\pgfpathmoveto{\pgfqpoint{0.000000in}{0.000000in}}%
\pgfpathlineto{\pgfqpoint{0.000000in}{-0.027778in}}%
\pgfusepath{stroke,fill}%
}%
\begin{pgfscope}%
\pgfsys@transformshift{3.029493in}{0.549073in}%
\pgfsys@useobject{currentmarker}{}%
\end{pgfscope}%
\end{pgfscope}%
\begin{pgfscope}%
\pgfpathrectangle{\pgfqpoint{0.721913in}{0.549073in}}{\pgfqpoint{4.518250in}{2.310000in}}%
\pgfusepath{clip}%
\pgfsetrectcap%
\pgfsetroundjoin%
\pgfsetlinewidth{0.250937pt}%
\definecolor{currentstroke}{rgb}{0.000000,0.000000,0.000000}%
\pgfsetstrokecolor{currentstroke}%
\pgfsetstrokeopacity{0.200000}%
\pgfsetdash{}{0pt}%
\pgfpathmoveto{\pgfqpoint{3.144725in}{0.549073in}}%
\pgfpathlineto{\pgfqpoint{3.144725in}{2.859073in}}%
\pgfusepath{stroke}%
\end{pgfscope}%
\begin{pgfscope}%
\pgfsetbuttcap%
\pgfsetroundjoin%
\definecolor{currentfill}{rgb}{0.000000,0.000000,0.000000}%
\pgfsetfillcolor{currentfill}%
\pgfsetlinewidth{0.602250pt}%
\definecolor{currentstroke}{rgb}{0.000000,0.000000,0.000000}%
\pgfsetstrokecolor{currentstroke}%
\pgfsetdash{}{0pt}%
\pgfsys@defobject{currentmarker}{\pgfqpoint{0.000000in}{-0.027778in}}{\pgfqpoint{0.000000in}{0.000000in}}{%
\pgfpathmoveto{\pgfqpoint{0.000000in}{0.000000in}}%
\pgfpathlineto{\pgfqpoint{0.000000in}{-0.027778in}}%
\pgfusepath{stroke,fill}%
}%
\begin{pgfscope}%
\pgfsys@transformshift{3.144725in}{0.549073in}%
\pgfsys@useobject{currentmarker}{}%
\end{pgfscope}%
\end{pgfscope}%
\begin{pgfscope}%
\pgfpathrectangle{\pgfqpoint{0.721913in}{0.549073in}}{\pgfqpoint{4.518250in}{2.310000in}}%
\pgfusepath{clip}%
\pgfsetrectcap%
\pgfsetroundjoin%
\pgfsetlinewidth{0.250937pt}%
\definecolor{currentstroke}{rgb}{0.000000,0.000000,0.000000}%
\pgfsetstrokecolor{currentstroke}%
\pgfsetstrokeopacity{0.200000}%
\pgfsetdash{}{0pt}%
\pgfpathmoveto{\pgfqpoint{3.244543in}{0.549073in}}%
\pgfpathlineto{\pgfqpoint{3.244543in}{2.859073in}}%
\pgfusepath{stroke}%
\end{pgfscope}%
\begin{pgfscope}%
\pgfsetbuttcap%
\pgfsetroundjoin%
\definecolor{currentfill}{rgb}{0.000000,0.000000,0.000000}%
\pgfsetfillcolor{currentfill}%
\pgfsetlinewidth{0.602250pt}%
\definecolor{currentstroke}{rgb}{0.000000,0.000000,0.000000}%
\pgfsetstrokecolor{currentstroke}%
\pgfsetdash{}{0pt}%
\pgfsys@defobject{currentmarker}{\pgfqpoint{0.000000in}{-0.027778in}}{\pgfqpoint{0.000000in}{0.000000in}}{%
\pgfpathmoveto{\pgfqpoint{0.000000in}{0.000000in}}%
\pgfpathlineto{\pgfqpoint{0.000000in}{-0.027778in}}%
\pgfusepath{stroke,fill}%
}%
\begin{pgfscope}%
\pgfsys@transformshift{3.244543in}{0.549073in}%
\pgfsys@useobject{currentmarker}{}%
\end{pgfscope}%
\end{pgfscope}%
\begin{pgfscope}%
\pgfpathrectangle{\pgfqpoint{0.721913in}{0.549073in}}{\pgfqpoint{4.518250in}{2.310000in}}%
\pgfusepath{clip}%
\pgfsetrectcap%
\pgfsetroundjoin%
\pgfsetlinewidth{0.250937pt}%
\definecolor{currentstroke}{rgb}{0.000000,0.000000,0.000000}%
\pgfsetstrokecolor{currentstroke}%
\pgfsetstrokeopacity{0.200000}%
\pgfsetdash{}{0pt}%
\pgfpathmoveto{\pgfqpoint{3.332588in}{0.549073in}}%
\pgfpathlineto{\pgfqpoint{3.332588in}{2.859073in}}%
\pgfusepath{stroke}%
\end{pgfscope}%
\begin{pgfscope}%
\pgfsetbuttcap%
\pgfsetroundjoin%
\definecolor{currentfill}{rgb}{0.000000,0.000000,0.000000}%
\pgfsetfillcolor{currentfill}%
\pgfsetlinewidth{0.602250pt}%
\definecolor{currentstroke}{rgb}{0.000000,0.000000,0.000000}%
\pgfsetstrokecolor{currentstroke}%
\pgfsetdash{}{0pt}%
\pgfsys@defobject{currentmarker}{\pgfqpoint{0.000000in}{-0.027778in}}{\pgfqpoint{0.000000in}{0.000000in}}{%
\pgfpathmoveto{\pgfqpoint{0.000000in}{0.000000in}}%
\pgfpathlineto{\pgfqpoint{0.000000in}{-0.027778in}}%
\pgfusepath{stroke,fill}%
}%
\begin{pgfscope}%
\pgfsys@transformshift{3.332588in}{0.549073in}%
\pgfsys@useobject{currentmarker}{}%
\end{pgfscope}%
\end{pgfscope}%
\begin{pgfscope}%
\pgfpathrectangle{\pgfqpoint{0.721913in}{0.549073in}}{\pgfqpoint{4.518250in}{2.310000in}}%
\pgfusepath{clip}%
\pgfsetrectcap%
\pgfsetroundjoin%
\pgfsetlinewidth{0.250937pt}%
\definecolor{currentstroke}{rgb}{0.000000,0.000000,0.000000}%
\pgfsetstrokecolor{currentstroke}%
\pgfsetstrokeopacity{0.200000}%
\pgfsetdash{}{0pt}%
\pgfpathmoveto{\pgfqpoint{3.929492in}{0.549073in}}%
\pgfpathlineto{\pgfqpoint{3.929492in}{2.859073in}}%
\pgfusepath{stroke}%
\end{pgfscope}%
\begin{pgfscope}%
\pgfsetbuttcap%
\pgfsetroundjoin%
\definecolor{currentfill}{rgb}{0.000000,0.000000,0.000000}%
\pgfsetfillcolor{currentfill}%
\pgfsetlinewidth{0.602250pt}%
\definecolor{currentstroke}{rgb}{0.000000,0.000000,0.000000}%
\pgfsetstrokecolor{currentstroke}%
\pgfsetdash{}{0pt}%
\pgfsys@defobject{currentmarker}{\pgfqpoint{0.000000in}{-0.027778in}}{\pgfqpoint{0.000000in}{0.000000in}}{%
\pgfpathmoveto{\pgfqpoint{0.000000in}{0.000000in}}%
\pgfpathlineto{\pgfqpoint{0.000000in}{-0.027778in}}%
\pgfusepath{stroke,fill}%
}%
\begin{pgfscope}%
\pgfsys@transformshift{3.929492in}{0.549073in}%
\pgfsys@useobject{currentmarker}{}%
\end{pgfscope}%
\end{pgfscope}%
\begin{pgfscope}%
\pgfpathrectangle{\pgfqpoint{0.721913in}{0.549073in}}{\pgfqpoint{4.518250in}{2.310000in}}%
\pgfusepath{clip}%
\pgfsetrectcap%
\pgfsetroundjoin%
\pgfsetlinewidth{0.250937pt}%
\definecolor{currentstroke}{rgb}{0.000000,0.000000,0.000000}%
\pgfsetstrokecolor{currentstroke}%
\pgfsetstrokeopacity{0.200000}%
\pgfsetdash{}{0pt}%
\pgfpathmoveto{\pgfqpoint{4.232587in}{0.549073in}}%
\pgfpathlineto{\pgfqpoint{4.232587in}{2.859073in}}%
\pgfusepath{stroke}%
\end{pgfscope}%
\begin{pgfscope}%
\pgfsetbuttcap%
\pgfsetroundjoin%
\definecolor{currentfill}{rgb}{0.000000,0.000000,0.000000}%
\pgfsetfillcolor{currentfill}%
\pgfsetlinewidth{0.602250pt}%
\definecolor{currentstroke}{rgb}{0.000000,0.000000,0.000000}%
\pgfsetstrokecolor{currentstroke}%
\pgfsetdash{}{0pt}%
\pgfsys@defobject{currentmarker}{\pgfqpoint{0.000000in}{-0.027778in}}{\pgfqpoint{0.000000in}{0.000000in}}{%
\pgfpathmoveto{\pgfqpoint{0.000000in}{0.000000in}}%
\pgfpathlineto{\pgfqpoint{0.000000in}{-0.027778in}}%
\pgfusepath{stroke,fill}%
}%
\begin{pgfscope}%
\pgfsys@transformshift{4.232587in}{0.549073in}%
\pgfsys@useobject{currentmarker}{}%
\end{pgfscope}%
\end{pgfscope}%
\begin{pgfscope}%
\pgfpathrectangle{\pgfqpoint{0.721913in}{0.549073in}}{\pgfqpoint{4.518250in}{2.310000in}}%
\pgfusepath{clip}%
\pgfsetrectcap%
\pgfsetroundjoin%
\pgfsetlinewidth{0.250937pt}%
\definecolor{currentstroke}{rgb}{0.000000,0.000000,0.000000}%
\pgfsetstrokecolor{currentstroke}%
\pgfsetstrokeopacity{0.200000}%
\pgfsetdash{}{0pt}%
\pgfpathmoveto{\pgfqpoint{4.447637in}{0.549073in}}%
\pgfpathlineto{\pgfqpoint{4.447637in}{2.859073in}}%
\pgfusepath{stroke}%
\end{pgfscope}%
\begin{pgfscope}%
\pgfsetbuttcap%
\pgfsetroundjoin%
\definecolor{currentfill}{rgb}{0.000000,0.000000,0.000000}%
\pgfsetfillcolor{currentfill}%
\pgfsetlinewidth{0.602250pt}%
\definecolor{currentstroke}{rgb}{0.000000,0.000000,0.000000}%
\pgfsetstrokecolor{currentstroke}%
\pgfsetdash{}{0pt}%
\pgfsys@defobject{currentmarker}{\pgfqpoint{0.000000in}{-0.027778in}}{\pgfqpoint{0.000000in}{0.000000in}}{%
\pgfpathmoveto{\pgfqpoint{0.000000in}{0.000000in}}%
\pgfpathlineto{\pgfqpoint{0.000000in}{-0.027778in}}%
\pgfusepath{stroke,fill}%
}%
\begin{pgfscope}%
\pgfsys@transformshift{4.447637in}{0.549073in}%
\pgfsys@useobject{currentmarker}{}%
\end{pgfscope}%
\end{pgfscope}%
\begin{pgfscope}%
\pgfpathrectangle{\pgfqpoint{0.721913in}{0.549073in}}{\pgfqpoint{4.518250in}{2.310000in}}%
\pgfusepath{clip}%
\pgfsetrectcap%
\pgfsetroundjoin%
\pgfsetlinewidth{0.250937pt}%
\definecolor{currentstroke}{rgb}{0.000000,0.000000,0.000000}%
\pgfsetstrokecolor{currentstroke}%
\pgfsetstrokeopacity{0.200000}%
\pgfsetdash{}{0pt}%
\pgfpathmoveto{\pgfqpoint{4.614442in}{0.549073in}}%
\pgfpathlineto{\pgfqpoint{4.614442in}{2.859073in}}%
\pgfusepath{stroke}%
\end{pgfscope}%
\begin{pgfscope}%
\pgfsetbuttcap%
\pgfsetroundjoin%
\definecolor{currentfill}{rgb}{0.000000,0.000000,0.000000}%
\pgfsetfillcolor{currentfill}%
\pgfsetlinewidth{0.602250pt}%
\definecolor{currentstroke}{rgb}{0.000000,0.000000,0.000000}%
\pgfsetstrokecolor{currentstroke}%
\pgfsetdash{}{0pt}%
\pgfsys@defobject{currentmarker}{\pgfqpoint{0.000000in}{-0.027778in}}{\pgfqpoint{0.000000in}{0.000000in}}{%
\pgfpathmoveto{\pgfqpoint{0.000000in}{0.000000in}}%
\pgfpathlineto{\pgfqpoint{0.000000in}{-0.027778in}}%
\pgfusepath{stroke,fill}%
}%
\begin{pgfscope}%
\pgfsys@transformshift{4.614442in}{0.549073in}%
\pgfsys@useobject{currentmarker}{}%
\end{pgfscope}%
\end{pgfscope}%
\begin{pgfscope}%
\pgfpathrectangle{\pgfqpoint{0.721913in}{0.549073in}}{\pgfqpoint{4.518250in}{2.310000in}}%
\pgfusepath{clip}%
\pgfsetrectcap%
\pgfsetroundjoin%
\pgfsetlinewidth{0.250937pt}%
\definecolor{currentstroke}{rgb}{0.000000,0.000000,0.000000}%
\pgfsetstrokecolor{currentstroke}%
\pgfsetstrokeopacity{0.200000}%
\pgfsetdash{}{0pt}%
\pgfpathmoveto{\pgfqpoint{4.750732in}{0.549073in}}%
\pgfpathlineto{\pgfqpoint{4.750732in}{2.859073in}}%
\pgfusepath{stroke}%
\end{pgfscope}%
\begin{pgfscope}%
\pgfsetbuttcap%
\pgfsetroundjoin%
\definecolor{currentfill}{rgb}{0.000000,0.000000,0.000000}%
\pgfsetfillcolor{currentfill}%
\pgfsetlinewidth{0.602250pt}%
\definecolor{currentstroke}{rgb}{0.000000,0.000000,0.000000}%
\pgfsetstrokecolor{currentstroke}%
\pgfsetdash{}{0pt}%
\pgfsys@defobject{currentmarker}{\pgfqpoint{0.000000in}{-0.027778in}}{\pgfqpoint{0.000000in}{0.000000in}}{%
\pgfpathmoveto{\pgfqpoint{0.000000in}{0.000000in}}%
\pgfpathlineto{\pgfqpoint{0.000000in}{-0.027778in}}%
\pgfusepath{stroke,fill}%
}%
\begin{pgfscope}%
\pgfsys@transformshift{4.750732in}{0.549073in}%
\pgfsys@useobject{currentmarker}{}%
\end{pgfscope}%
\end{pgfscope}%
\begin{pgfscope}%
\pgfpathrectangle{\pgfqpoint{0.721913in}{0.549073in}}{\pgfqpoint{4.518250in}{2.310000in}}%
\pgfusepath{clip}%
\pgfsetrectcap%
\pgfsetroundjoin%
\pgfsetlinewidth{0.250937pt}%
\definecolor{currentstroke}{rgb}{0.000000,0.000000,0.000000}%
\pgfsetstrokecolor{currentstroke}%
\pgfsetstrokeopacity{0.200000}%
\pgfsetdash{}{0pt}%
\pgfpathmoveto{\pgfqpoint{4.865963in}{0.549073in}}%
\pgfpathlineto{\pgfqpoint{4.865963in}{2.859073in}}%
\pgfusepath{stroke}%
\end{pgfscope}%
\begin{pgfscope}%
\pgfsetbuttcap%
\pgfsetroundjoin%
\definecolor{currentfill}{rgb}{0.000000,0.000000,0.000000}%
\pgfsetfillcolor{currentfill}%
\pgfsetlinewidth{0.602250pt}%
\definecolor{currentstroke}{rgb}{0.000000,0.000000,0.000000}%
\pgfsetstrokecolor{currentstroke}%
\pgfsetdash{}{0pt}%
\pgfsys@defobject{currentmarker}{\pgfqpoint{0.000000in}{-0.027778in}}{\pgfqpoint{0.000000in}{0.000000in}}{%
\pgfpathmoveto{\pgfqpoint{0.000000in}{0.000000in}}%
\pgfpathlineto{\pgfqpoint{0.000000in}{-0.027778in}}%
\pgfusepath{stroke,fill}%
}%
\begin{pgfscope}%
\pgfsys@transformshift{4.865963in}{0.549073in}%
\pgfsys@useobject{currentmarker}{}%
\end{pgfscope}%
\end{pgfscope}%
\begin{pgfscope}%
\pgfpathrectangle{\pgfqpoint{0.721913in}{0.549073in}}{\pgfqpoint{4.518250in}{2.310000in}}%
\pgfusepath{clip}%
\pgfsetrectcap%
\pgfsetroundjoin%
\pgfsetlinewidth{0.250937pt}%
\definecolor{currentstroke}{rgb}{0.000000,0.000000,0.000000}%
\pgfsetstrokecolor{currentstroke}%
\pgfsetstrokeopacity{0.200000}%
\pgfsetdash{}{0pt}%
\pgfpathmoveto{\pgfqpoint{4.965781in}{0.549073in}}%
\pgfpathlineto{\pgfqpoint{4.965781in}{2.859073in}}%
\pgfusepath{stroke}%
\end{pgfscope}%
\begin{pgfscope}%
\pgfsetbuttcap%
\pgfsetroundjoin%
\definecolor{currentfill}{rgb}{0.000000,0.000000,0.000000}%
\pgfsetfillcolor{currentfill}%
\pgfsetlinewidth{0.602250pt}%
\definecolor{currentstroke}{rgb}{0.000000,0.000000,0.000000}%
\pgfsetstrokecolor{currentstroke}%
\pgfsetdash{}{0pt}%
\pgfsys@defobject{currentmarker}{\pgfqpoint{0.000000in}{-0.027778in}}{\pgfqpoint{0.000000in}{0.000000in}}{%
\pgfpathmoveto{\pgfqpoint{0.000000in}{0.000000in}}%
\pgfpathlineto{\pgfqpoint{0.000000in}{-0.027778in}}%
\pgfusepath{stroke,fill}%
}%
\begin{pgfscope}%
\pgfsys@transformshift{4.965781in}{0.549073in}%
\pgfsys@useobject{currentmarker}{}%
\end{pgfscope}%
\end{pgfscope}%
\begin{pgfscope}%
\pgfpathrectangle{\pgfqpoint{0.721913in}{0.549073in}}{\pgfqpoint{4.518250in}{2.310000in}}%
\pgfusepath{clip}%
\pgfsetrectcap%
\pgfsetroundjoin%
\pgfsetlinewidth{0.250937pt}%
\definecolor{currentstroke}{rgb}{0.000000,0.000000,0.000000}%
\pgfsetstrokecolor{currentstroke}%
\pgfsetstrokeopacity{0.200000}%
\pgfsetdash{}{0pt}%
\pgfpathmoveto{\pgfqpoint{5.053827in}{0.549073in}}%
\pgfpathlineto{\pgfqpoint{5.053827in}{2.859073in}}%
\pgfusepath{stroke}%
\end{pgfscope}%
\begin{pgfscope}%
\pgfsetbuttcap%
\pgfsetroundjoin%
\definecolor{currentfill}{rgb}{0.000000,0.000000,0.000000}%
\pgfsetfillcolor{currentfill}%
\pgfsetlinewidth{0.602250pt}%
\definecolor{currentstroke}{rgb}{0.000000,0.000000,0.000000}%
\pgfsetstrokecolor{currentstroke}%
\pgfsetdash{}{0pt}%
\pgfsys@defobject{currentmarker}{\pgfqpoint{0.000000in}{-0.027778in}}{\pgfqpoint{0.000000in}{0.000000in}}{%
\pgfpathmoveto{\pgfqpoint{0.000000in}{0.000000in}}%
\pgfpathlineto{\pgfqpoint{0.000000in}{-0.027778in}}%
\pgfusepath{stroke,fill}%
}%
\begin{pgfscope}%
\pgfsys@transformshift{5.053827in}{0.549073in}%
\pgfsys@useobject{currentmarker}{}%
\end{pgfscope}%
\end{pgfscope}%
\begin{pgfscope}%
\definecolor{textcolor}{rgb}{0.000000,0.000000,0.000000}%
\pgfsetstrokecolor{textcolor}%
\pgfsetfillcolor{textcolor}%
\pgftext[x=2.981038in,y=0.248148in,,top]{\color{textcolor}{\rmfamily\fontsize{12.000000}{14.400000}\selectfont\catcode`\^=\active\def^{\ifmmode\sp\else\^{}\fi}\catcode`\%=\active\def%{\%}smoothing parameter $\sigma$}}%
\end{pgfscope}%
\begin{pgfscope}%
\pgfpathrectangle{\pgfqpoint{0.721913in}{0.549073in}}{\pgfqpoint{4.518250in}{2.310000in}}%
\pgfusepath{clip}%
\pgfsetrectcap%
\pgfsetroundjoin%
\pgfsetlinewidth{0.250937pt}%
\definecolor{currentstroke}{rgb}{0.000000,0.000000,0.000000}%
\pgfsetstrokecolor{currentstroke}%
\pgfsetstrokeopacity{0.200000}%
\pgfsetdash{}{0pt}%
\pgfpathmoveto{\pgfqpoint{0.721913in}{1.001646in}}%
\pgfpathlineto{\pgfqpoint{5.240163in}{1.001646in}}%
\pgfusepath{stroke}%
\end{pgfscope}%
\begin{pgfscope}%
\pgfsetbuttcap%
\pgfsetroundjoin%
\definecolor{currentfill}{rgb}{0.000000,0.000000,0.000000}%
\pgfsetfillcolor{currentfill}%
\pgfsetlinewidth{0.803000pt}%
\definecolor{currentstroke}{rgb}{0.000000,0.000000,0.000000}%
\pgfsetstrokecolor{currentstroke}%
\pgfsetdash{}{0pt}%
\pgfsys@defobject{currentmarker}{\pgfqpoint{-0.048611in}{0.000000in}}{\pgfqpoint{-0.000000in}{0.000000in}}{%
\pgfpathmoveto{\pgfqpoint{-0.000000in}{0.000000in}}%
\pgfpathlineto{\pgfqpoint{-0.048611in}{0.000000in}}%
\pgfusepath{stroke,fill}%
}%
\begin{pgfscope}%
\pgfsys@transformshift{0.721913in}{1.001646in}%
\pgfsys@useobject{currentmarker}{}%
\end{pgfscope}%
\end{pgfscope}%
\begin{pgfscope}%
\definecolor{textcolor}{rgb}{0.000000,0.000000,0.000000}%
\pgfsetstrokecolor{textcolor}%
\pgfsetfillcolor{textcolor}%
\pgftext[x=0.303703in, y=0.943776in, left, base]{\color{textcolor}{\rmfamily\fontsize{12.000000}{14.400000}\selectfont\catcode`\^=\active\def^{\ifmmode\sp\else\^{}\fi}\catcode`\%=\active\def%{\%}$\mathdefault{10^{-6}}$}}%
\end{pgfscope}%
\begin{pgfscope}%
\pgfpathrectangle{\pgfqpoint{0.721913in}{0.549073in}}{\pgfqpoint{4.518250in}{2.310000in}}%
\pgfusepath{clip}%
\pgfsetrectcap%
\pgfsetroundjoin%
\pgfsetlinewidth{0.250937pt}%
\definecolor{currentstroke}{rgb}{0.000000,0.000000,0.000000}%
\pgfsetstrokecolor{currentstroke}%
\pgfsetstrokeopacity{0.200000}%
\pgfsetdash{}{0pt}%
\pgfpathmoveto{\pgfqpoint{0.721913in}{1.576677in}}%
\pgfpathlineto{\pgfqpoint{5.240163in}{1.576677in}}%
\pgfusepath{stroke}%
\end{pgfscope}%
\begin{pgfscope}%
\pgfsetbuttcap%
\pgfsetroundjoin%
\definecolor{currentfill}{rgb}{0.000000,0.000000,0.000000}%
\pgfsetfillcolor{currentfill}%
\pgfsetlinewidth{0.803000pt}%
\definecolor{currentstroke}{rgb}{0.000000,0.000000,0.000000}%
\pgfsetstrokecolor{currentstroke}%
\pgfsetdash{}{0pt}%
\pgfsys@defobject{currentmarker}{\pgfqpoint{-0.048611in}{0.000000in}}{\pgfqpoint{-0.000000in}{0.000000in}}{%
\pgfpathmoveto{\pgfqpoint{-0.000000in}{0.000000in}}%
\pgfpathlineto{\pgfqpoint{-0.048611in}{0.000000in}}%
\pgfusepath{stroke,fill}%
}%
\begin{pgfscope}%
\pgfsys@transformshift{0.721913in}{1.576677in}%
\pgfsys@useobject{currentmarker}{}%
\end{pgfscope}%
\end{pgfscope}%
\begin{pgfscope}%
\definecolor{textcolor}{rgb}{0.000000,0.000000,0.000000}%
\pgfsetstrokecolor{textcolor}%
\pgfsetfillcolor{textcolor}%
\pgftext[x=0.303703in, y=1.518807in, left, base]{\color{textcolor}{\rmfamily\fontsize{12.000000}{14.400000}\selectfont\catcode`\^=\active\def^{\ifmmode\sp\else\^{}\fi}\catcode`\%=\active\def%{\%}$\mathdefault{10^{-4}}$}}%
\end{pgfscope}%
\begin{pgfscope}%
\pgfpathrectangle{\pgfqpoint{0.721913in}{0.549073in}}{\pgfqpoint{4.518250in}{2.310000in}}%
\pgfusepath{clip}%
\pgfsetrectcap%
\pgfsetroundjoin%
\pgfsetlinewidth{0.250937pt}%
\definecolor{currentstroke}{rgb}{0.000000,0.000000,0.000000}%
\pgfsetstrokecolor{currentstroke}%
\pgfsetstrokeopacity{0.200000}%
\pgfsetdash{}{0pt}%
\pgfpathmoveto{\pgfqpoint{0.721913in}{2.151707in}}%
\pgfpathlineto{\pgfqpoint{5.240163in}{2.151707in}}%
\pgfusepath{stroke}%
\end{pgfscope}%
\begin{pgfscope}%
\pgfsetbuttcap%
\pgfsetroundjoin%
\definecolor{currentfill}{rgb}{0.000000,0.000000,0.000000}%
\pgfsetfillcolor{currentfill}%
\pgfsetlinewidth{0.803000pt}%
\definecolor{currentstroke}{rgb}{0.000000,0.000000,0.000000}%
\pgfsetstrokecolor{currentstroke}%
\pgfsetdash{}{0pt}%
\pgfsys@defobject{currentmarker}{\pgfqpoint{-0.048611in}{0.000000in}}{\pgfqpoint{-0.000000in}{0.000000in}}{%
\pgfpathmoveto{\pgfqpoint{-0.000000in}{0.000000in}}%
\pgfpathlineto{\pgfqpoint{-0.048611in}{0.000000in}}%
\pgfusepath{stroke,fill}%
}%
\begin{pgfscope}%
\pgfsys@transformshift{0.721913in}{2.151707in}%
\pgfsys@useobject{currentmarker}{}%
\end{pgfscope}%
\end{pgfscope}%
\begin{pgfscope}%
\definecolor{textcolor}{rgb}{0.000000,0.000000,0.000000}%
\pgfsetstrokecolor{textcolor}%
\pgfsetfillcolor{textcolor}%
\pgftext[x=0.303703in, y=2.093837in, left, base]{\color{textcolor}{\rmfamily\fontsize{12.000000}{14.400000}\selectfont\catcode`\^=\active\def^{\ifmmode\sp\else\^{}\fi}\catcode`\%=\active\def%{\%}$\mathdefault{10^{-2}}$}}%
\end{pgfscope}%
\begin{pgfscope}%
\pgfpathrectangle{\pgfqpoint{0.721913in}{0.549073in}}{\pgfqpoint{4.518250in}{2.310000in}}%
\pgfusepath{clip}%
\pgfsetrectcap%
\pgfsetroundjoin%
\pgfsetlinewidth{0.250937pt}%
\definecolor{currentstroke}{rgb}{0.000000,0.000000,0.000000}%
\pgfsetstrokecolor{currentstroke}%
\pgfsetstrokeopacity{0.200000}%
\pgfsetdash{}{0pt}%
\pgfpathmoveto{\pgfqpoint{0.721913in}{2.726738in}}%
\pgfpathlineto{\pgfqpoint{5.240163in}{2.726738in}}%
\pgfusepath{stroke}%
\end{pgfscope}%
\begin{pgfscope}%
\pgfsetbuttcap%
\pgfsetroundjoin%
\definecolor{currentfill}{rgb}{0.000000,0.000000,0.000000}%
\pgfsetfillcolor{currentfill}%
\pgfsetlinewidth{0.803000pt}%
\definecolor{currentstroke}{rgb}{0.000000,0.000000,0.000000}%
\pgfsetstrokecolor{currentstroke}%
\pgfsetdash{}{0pt}%
\pgfsys@defobject{currentmarker}{\pgfqpoint{-0.048611in}{0.000000in}}{\pgfqpoint{-0.000000in}{0.000000in}}{%
\pgfpathmoveto{\pgfqpoint{-0.000000in}{0.000000in}}%
\pgfpathlineto{\pgfqpoint{-0.048611in}{0.000000in}}%
\pgfusepath{stroke,fill}%
}%
\begin{pgfscope}%
\pgfsys@transformshift{0.721913in}{2.726738in}%
\pgfsys@useobject{currentmarker}{}%
\end{pgfscope}%
\end{pgfscope}%
\begin{pgfscope}%
\definecolor{textcolor}{rgb}{0.000000,0.000000,0.000000}%
\pgfsetstrokecolor{textcolor}%
\pgfsetfillcolor{textcolor}%
\pgftext[x=0.395525in, y=2.668868in, left, base]{\color{textcolor}{\rmfamily\fontsize{12.000000}{14.400000}\selectfont\catcode`\^=\active\def^{\ifmmode\sp\else\^{}\fi}\catcode`\%=\active\def%{\%}$\mathdefault{10^{0}}$}}%
\end{pgfscope}%
\begin{pgfscope}%
\definecolor{textcolor}{rgb}{0.000000,0.000000,0.000000}%
\pgfsetstrokecolor{textcolor}%
\pgfsetfillcolor{textcolor}%
\pgftext[x=0.248148in,y=1.704073in,,bottom,rotate=90.000000]{\color{textcolor}{\rmfamily\fontsize{12.000000}{14.400000}\selectfont\catcode`\^=\active\def^{\ifmmode\sp\else\^{}\fi}\catcode`\%=\active\def%{\%}$L^1$-error}}%
\end{pgfscope}%
\begin{pgfscope}%
\pgfpathrectangle{\pgfqpoint{0.721913in}{0.549073in}}{\pgfqpoint{4.518250in}{2.310000in}}%
\pgfusepath{clip}%
\pgfsetrectcap%
\pgfsetroundjoin%
\pgfsetlinewidth{1.505625pt}%
\definecolor{currentstroke}{rgb}{0.392157,0.560784,1.000000}%
\pgfsetstrokecolor{currentstroke}%
\pgfsetdash{}{0pt}%
\pgfpathmoveto{\pgfqpoint{0.829491in}{2.418629in}}%
\pgfpathlineto{\pgfqpoint{1.546673in}{2.384336in}}%
\pgfpathlineto{\pgfqpoint{2.263856in}{2.330194in}}%
\pgfpathlineto{\pgfqpoint{2.981038in}{2.278675in}}%
\pgfpathlineto{\pgfqpoint{3.698221in}{2.257279in}}%
\pgfpathlineto{\pgfqpoint{4.415404in}{2.217185in}}%
\pgfpathlineto{\pgfqpoint{5.132586in}{2.147019in}}%
\pgfusepath{stroke}%
\end{pgfscope}%
\begin{pgfscope}%
\pgfpathrectangle{\pgfqpoint{0.721913in}{0.549073in}}{\pgfqpoint{4.518250in}{2.310000in}}%
\pgfusepath{clip}%
\pgfsetbuttcap%
\pgfsetroundjoin%
\definecolor{currentfill}{rgb}{0.392157,0.560784,1.000000}%
\pgfsetfillcolor{currentfill}%
\pgfsetlinewidth{1.003750pt}%
\definecolor{currentstroke}{rgb}{0.392157,0.560784,1.000000}%
\pgfsetstrokecolor{currentstroke}%
\pgfsetdash{}{0pt}%
\pgfsys@defobject{currentmarker}{\pgfqpoint{-0.041667in}{-0.041667in}}{\pgfqpoint{0.041667in}{0.041667in}}{%
\pgfpathmoveto{\pgfqpoint{0.000000in}{-0.041667in}}%
\pgfpathcurveto{\pgfqpoint{0.011050in}{-0.041667in}}{\pgfqpoint{0.021649in}{-0.037276in}}{\pgfqpoint{0.029463in}{-0.029463in}}%
\pgfpathcurveto{\pgfqpoint{0.037276in}{-0.021649in}}{\pgfqpoint{0.041667in}{-0.011050in}}{\pgfqpoint{0.041667in}{0.000000in}}%
\pgfpathcurveto{\pgfqpoint{0.041667in}{0.011050in}}{\pgfqpoint{0.037276in}{0.021649in}}{\pgfqpoint{0.029463in}{0.029463in}}%
\pgfpathcurveto{\pgfqpoint{0.021649in}{0.037276in}}{\pgfqpoint{0.011050in}{0.041667in}}{\pgfqpoint{0.000000in}{0.041667in}}%
\pgfpathcurveto{\pgfqpoint{-0.011050in}{0.041667in}}{\pgfqpoint{-0.021649in}{0.037276in}}{\pgfqpoint{-0.029463in}{0.029463in}}%
\pgfpathcurveto{\pgfqpoint{-0.037276in}{0.021649in}}{\pgfqpoint{-0.041667in}{0.011050in}}{\pgfqpoint{-0.041667in}{0.000000in}}%
\pgfpathcurveto{\pgfqpoint{-0.041667in}{-0.011050in}}{\pgfqpoint{-0.037276in}{-0.021649in}}{\pgfqpoint{-0.029463in}{-0.029463in}}%
\pgfpathcurveto{\pgfqpoint{-0.021649in}{-0.037276in}}{\pgfqpoint{-0.011050in}{-0.041667in}}{\pgfqpoint{0.000000in}{-0.041667in}}%
\pgfpathlineto{\pgfqpoint{0.000000in}{-0.041667in}}%
\pgfpathclose%
\pgfusepath{stroke,fill}%
}%
\begin{pgfscope}%
\pgfsys@transformshift{0.829491in}{2.418629in}%
\pgfsys@useobject{currentmarker}{}%
\end{pgfscope}%
\begin{pgfscope}%
\pgfsys@transformshift{1.546673in}{2.384336in}%
\pgfsys@useobject{currentmarker}{}%
\end{pgfscope}%
\begin{pgfscope}%
\pgfsys@transformshift{2.263856in}{2.330194in}%
\pgfsys@useobject{currentmarker}{}%
\end{pgfscope}%
\begin{pgfscope}%
\pgfsys@transformshift{2.981038in}{2.278675in}%
\pgfsys@useobject{currentmarker}{}%
\end{pgfscope}%
\begin{pgfscope}%
\pgfsys@transformshift{3.698221in}{2.257279in}%
\pgfsys@useobject{currentmarker}{}%
\end{pgfscope}%
\begin{pgfscope}%
\pgfsys@transformshift{4.415404in}{2.217185in}%
\pgfsys@useobject{currentmarker}{}%
\end{pgfscope}%
\begin{pgfscope}%
\pgfsys@transformshift{5.132586in}{2.147019in}%
\pgfsys@useobject{currentmarker}{}%
\end{pgfscope}%
\end{pgfscope}%
\begin{pgfscope}%
\pgfpathrectangle{\pgfqpoint{0.721913in}{0.549073in}}{\pgfqpoint{4.518250in}{2.310000in}}%
\pgfusepath{clip}%
\pgfsetrectcap%
\pgfsetroundjoin%
\pgfsetlinewidth{1.505625pt}%
\definecolor{currentstroke}{rgb}{0.470588,0.368627,0.941176}%
\pgfsetstrokecolor{currentstroke}%
\pgfsetdash{}{0pt}%
\pgfpathmoveto{\pgfqpoint{0.829491in}{0.985164in}}%
\pgfpathlineto{\pgfqpoint{1.546673in}{1.910072in}}%
\pgfpathlineto{\pgfqpoint{2.263856in}{2.014059in}}%
\pgfpathlineto{\pgfqpoint{2.981038in}{2.151866in}}%
\pgfpathlineto{\pgfqpoint{3.698221in}{2.206828in}}%
\pgfpathlineto{\pgfqpoint{4.415404in}{2.196190in}}%
\pgfpathlineto{\pgfqpoint{5.132586in}{2.145483in}}%
\pgfusepath{stroke}%
\end{pgfscope}%
\begin{pgfscope}%
\pgfpathrectangle{\pgfqpoint{0.721913in}{0.549073in}}{\pgfqpoint{4.518250in}{2.310000in}}%
\pgfusepath{clip}%
\pgfsetbuttcap%
\pgfsetmiterjoin%
\definecolor{currentfill}{rgb}{0.470588,0.368627,0.941176}%
\pgfsetfillcolor{currentfill}%
\pgfsetlinewidth{1.003750pt}%
\definecolor{currentstroke}{rgb}{0.470588,0.368627,0.941176}%
\pgfsetstrokecolor{currentstroke}%
\pgfsetdash{}{0pt}%
\pgfsys@defobject{currentmarker}{\pgfqpoint{-0.041667in}{-0.041667in}}{\pgfqpoint{0.041667in}{0.041667in}}{%
\pgfpathmoveto{\pgfqpoint{0.000000in}{0.041667in}}%
\pgfpathlineto{\pgfqpoint{-0.041667in}{-0.041667in}}%
\pgfpathlineto{\pgfqpoint{0.041667in}{-0.041667in}}%
\pgfpathlineto{\pgfqpoint{0.000000in}{0.041667in}}%
\pgfpathclose%
\pgfusepath{stroke,fill}%
}%
\begin{pgfscope}%
\pgfsys@transformshift{0.829491in}{0.985164in}%
\pgfsys@useobject{currentmarker}{}%
\end{pgfscope}%
\begin{pgfscope}%
\pgfsys@transformshift{1.546673in}{1.910072in}%
\pgfsys@useobject{currentmarker}{}%
\end{pgfscope}%
\begin{pgfscope}%
\pgfsys@transformshift{2.263856in}{2.014059in}%
\pgfsys@useobject{currentmarker}{}%
\end{pgfscope}%
\begin{pgfscope}%
\pgfsys@transformshift{2.981038in}{2.151866in}%
\pgfsys@useobject{currentmarker}{}%
\end{pgfscope}%
\begin{pgfscope}%
\pgfsys@transformshift{3.698221in}{2.206828in}%
\pgfsys@useobject{currentmarker}{}%
\end{pgfscope}%
\begin{pgfscope}%
\pgfsys@transformshift{4.415404in}{2.196190in}%
\pgfsys@useobject{currentmarker}{}%
\end{pgfscope}%
\begin{pgfscope}%
\pgfsys@transformshift{5.132586in}{2.145483in}%
\pgfsys@useobject{currentmarker}{}%
\end{pgfscope}%
\end{pgfscope}%
\begin{pgfscope}%
\pgfpathrectangle{\pgfqpoint{0.721913in}{0.549073in}}{\pgfqpoint{4.518250in}{2.310000in}}%
\pgfusepath{clip}%
\pgfsetrectcap%
\pgfsetroundjoin%
\pgfsetlinewidth{1.505625pt}%
\definecolor{currentstroke}{rgb}{0.862745,0.149020,0.498039}%
\pgfsetstrokecolor{currentstroke}%
\pgfsetdash{}{0pt}%
\pgfpathmoveto{\pgfqpoint{0.829491in}{0.907071in}}%
\pgfpathlineto{\pgfqpoint{1.546673in}{1.011484in}}%
\pgfpathlineto{\pgfqpoint{2.263856in}{1.905587in}}%
\pgfpathlineto{\pgfqpoint{2.981038in}{2.105821in}}%
\pgfpathlineto{\pgfqpoint{3.698221in}{2.113765in}}%
\pgfpathlineto{\pgfqpoint{4.415404in}{2.153874in}}%
\pgfpathlineto{\pgfqpoint{5.132586in}{2.186193in}}%
\pgfusepath{stroke}%
\end{pgfscope}%
\begin{pgfscope}%
\pgfpathrectangle{\pgfqpoint{0.721913in}{0.549073in}}{\pgfqpoint{4.518250in}{2.310000in}}%
\pgfusepath{clip}%
\pgfsetbuttcap%
\pgfsetmiterjoin%
\definecolor{currentfill}{rgb}{0.862745,0.149020,0.498039}%
\pgfsetfillcolor{currentfill}%
\pgfsetlinewidth{1.003750pt}%
\definecolor{currentstroke}{rgb}{0.862745,0.149020,0.498039}%
\pgfsetstrokecolor{currentstroke}%
\pgfsetdash{}{0pt}%
\pgfsys@defobject{currentmarker}{\pgfqpoint{-0.041667in}{-0.041667in}}{\pgfqpoint{0.041667in}{0.041667in}}{%
\pgfpathmoveto{\pgfqpoint{-0.041667in}{-0.041667in}}%
\pgfpathlineto{\pgfqpoint{0.041667in}{-0.041667in}}%
\pgfpathlineto{\pgfqpoint{0.041667in}{0.041667in}}%
\pgfpathlineto{\pgfqpoint{-0.041667in}{0.041667in}}%
\pgfpathlineto{\pgfqpoint{-0.041667in}{-0.041667in}}%
\pgfpathclose%
\pgfusepath{stroke,fill}%
}%
\begin{pgfscope}%
\pgfsys@transformshift{0.829491in}{0.907071in}%
\pgfsys@useobject{currentmarker}{}%
\end{pgfscope}%
\begin{pgfscope}%
\pgfsys@transformshift{1.546673in}{1.011484in}%
\pgfsys@useobject{currentmarker}{}%
\end{pgfscope}%
\begin{pgfscope}%
\pgfsys@transformshift{2.263856in}{1.905587in}%
\pgfsys@useobject{currentmarker}{}%
\end{pgfscope}%
\begin{pgfscope}%
\pgfsys@transformshift{2.981038in}{2.105821in}%
\pgfsys@useobject{currentmarker}{}%
\end{pgfscope}%
\begin{pgfscope}%
\pgfsys@transformshift{3.698221in}{2.113765in}%
\pgfsys@useobject{currentmarker}{}%
\end{pgfscope}%
\begin{pgfscope}%
\pgfsys@transformshift{4.415404in}{2.153874in}%
\pgfsys@useobject{currentmarker}{}%
\end{pgfscope}%
\begin{pgfscope}%
\pgfsys@transformshift{5.132586in}{2.186193in}%
\pgfsys@useobject{currentmarker}{}%
\end{pgfscope}%
\end{pgfscope}%
\begin{pgfscope}%
\pgfpathrectangle{\pgfqpoint{0.721913in}{0.549073in}}{\pgfqpoint{4.518250in}{2.310000in}}%
\pgfusepath{clip}%
\pgfsetrectcap%
\pgfsetroundjoin%
\pgfsetlinewidth{1.505625pt}%
\definecolor{currentstroke}{rgb}{0.996078,0.380392,0.000000}%
\pgfsetstrokecolor{currentstroke}%
\pgfsetdash{}{0pt}%
\pgfpathmoveto{\pgfqpoint{0.829491in}{0.741573in}}%
\pgfpathlineto{\pgfqpoint{1.546673in}{0.952998in}}%
\pgfpathlineto{\pgfqpoint{2.263856in}{1.460182in}}%
\pgfpathlineto{\pgfqpoint{2.981038in}{1.951734in}}%
\pgfpathlineto{\pgfqpoint{3.698221in}{2.019961in}}%
\pgfpathlineto{\pgfqpoint{4.415404in}{2.149334in}}%
\pgfpathlineto{\pgfqpoint{5.132586in}{2.199676in}}%
\pgfusepath{stroke}%
\end{pgfscope}%
\begin{pgfscope}%
\pgfpathrectangle{\pgfqpoint{0.721913in}{0.549073in}}{\pgfqpoint{4.518250in}{2.310000in}}%
\pgfusepath{clip}%
\pgfsetbuttcap%
\pgfsetmiterjoin%
\definecolor{currentfill}{rgb}{0.996078,0.380392,0.000000}%
\pgfsetfillcolor{currentfill}%
\pgfsetlinewidth{1.003750pt}%
\definecolor{currentstroke}{rgb}{0.996078,0.380392,0.000000}%
\pgfsetstrokecolor{currentstroke}%
\pgfsetdash{}{0pt}%
\pgfsys@defobject{currentmarker}{\pgfqpoint{-0.039627in}{-0.033709in}}{\pgfqpoint{0.039627in}{0.041667in}}{%
\pgfpathmoveto{\pgfqpoint{0.000000in}{0.041667in}}%
\pgfpathlineto{\pgfqpoint{-0.039627in}{0.012876in}}%
\pgfpathlineto{\pgfqpoint{-0.024491in}{-0.033709in}}%
\pgfpathlineto{\pgfqpoint{0.024491in}{-0.033709in}}%
\pgfpathlineto{\pgfqpoint{0.039627in}{0.012876in}}%
\pgfpathlineto{\pgfqpoint{0.000000in}{0.041667in}}%
\pgfpathclose%
\pgfusepath{stroke,fill}%
}%
\begin{pgfscope}%
\pgfsys@transformshift{0.829491in}{0.741573in}%
\pgfsys@useobject{currentmarker}{}%
\end{pgfscope}%
\begin{pgfscope}%
\pgfsys@transformshift{1.546673in}{0.952998in}%
\pgfsys@useobject{currentmarker}{}%
\end{pgfscope}%
\begin{pgfscope}%
\pgfsys@transformshift{2.263856in}{1.460182in}%
\pgfsys@useobject{currentmarker}{}%
\end{pgfscope}%
\begin{pgfscope}%
\pgfsys@transformshift{2.981038in}{1.951734in}%
\pgfsys@useobject{currentmarker}{}%
\end{pgfscope}%
\begin{pgfscope}%
\pgfsys@transformshift{3.698221in}{2.019961in}%
\pgfsys@useobject{currentmarker}{}%
\end{pgfscope}%
\begin{pgfscope}%
\pgfsys@transformshift{4.415404in}{2.149334in}%
\pgfsys@useobject{currentmarker}{}%
\end{pgfscope}%
\begin{pgfscope}%
\pgfsys@transformshift{5.132586in}{2.199676in}%
\pgfsys@useobject{currentmarker}{}%
\end{pgfscope}%
\end{pgfscope}%
\begin{pgfscope}%
\pgfpathrectangle{\pgfqpoint{0.721913in}{0.549073in}}{\pgfqpoint{4.518250in}{2.310000in}}%
\pgfusepath{clip}%
\pgfsetrectcap%
\pgfsetroundjoin%
\pgfsetlinewidth{1.505625pt}%
\definecolor{currentstroke}{rgb}{1.000000,0.690196,0.000000}%
\pgfsetstrokecolor{currentstroke}%
\pgfsetdash{}{0pt}%
\pgfpathmoveto{\pgfqpoint{0.829491in}{0.795061in}}%
\pgfpathlineto{\pgfqpoint{1.546673in}{1.093176in}}%
\pgfpathlineto{\pgfqpoint{2.263856in}{1.334298in}}%
\pgfpathlineto{\pgfqpoint{2.981038in}{2.046622in}}%
\pgfpathlineto{\pgfqpoint{3.698221in}{2.364267in}}%
\pgfpathlineto{\pgfqpoint{4.415404in}{2.573213in}}%
\pgfpathlineto{\pgfqpoint{5.132586in}{2.666573in}}%
\pgfusepath{stroke}%
\end{pgfscope}%
\begin{pgfscope}%
\pgfpathrectangle{\pgfqpoint{0.721913in}{0.549073in}}{\pgfqpoint{4.518250in}{2.310000in}}%
\pgfusepath{clip}%
\pgfsetbuttcap%
\pgfsetmiterjoin%
\definecolor{currentfill}{rgb}{1.000000,0.690196,0.000000}%
\pgfsetfillcolor{currentfill}%
\pgfsetlinewidth{1.003750pt}%
\definecolor{currentstroke}{rgb}{1.000000,0.690196,0.000000}%
\pgfsetstrokecolor{currentstroke}%
\pgfsetdash{}{0pt}%
\pgfsys@defobject{currentmarker}{\pgfqpoint{-0.035355in}{-0.058926in}}{\pgfqpoint{0.035355in}{0.058926in}}{%
\pgfpathmoveto{\pgfqpoint{-0.000000in}{-0.058926in}}%
\pgfpathlineto{\pgfqpoint{0.035355in}{0.000000in}}%
\pgfpathlineto{\pgfqpoint{0.000000in}{0.058926in}}%
\pgfpathlineto{\pgfqpoint{-0.035355in}{0.000000in}}%
\pgfpathlineto{\pgfqpoint{-0.000000in}{-0.058926in}}%
\pgfpathclose%
\pgfusepath{stroke,fill}%
}%
\begin{pgfscope}%
\pgfsys@transformshift{0.829491in}{0.795061in}%
\pgfsys@useobject{currentmarker}{}%
\end{pgfscope}%
\begin{pgfscope}%
\pgfsys@transformshift{1.546673in}{1.093176in}%
\pgfsys@useobject{currentmarker}{}%
\end{pgfscope}%
\begin{pgfscope}%
\pgfsys@transformshift{2.263856in}{1.334298in}%
\pgfsys@useobject{currentmarker}{}%
\end{pgfscope}%
\begin{pgfscope}%
\pgfsys@transformshift{2.981038in}{2.046622in}%
\pgfsys@useobject{currentmarker}{}%
\end{pgfscope}%
\begin{pgfscope}%
\pgfsys@transformshift{3.698221in}{2.364267in}%
\pgfsys@useobject{currentmarker}{}%
\end{pgfscope}%
\begin{pgfscope}%
\pgfsys@transformshift{4.415404in}{2.573213in}%
\pgfsys@useobject{currentmarker}{}%
\end{pgfscope}%
\begin{pgfscope}%
\pgfsys@transformshift{5.132586in}{2.666573in}%
\pgfsys@useobject{currentmarker}{}%
\end{pgfscope}%
\end{pgfscope}%
\begin{pgfscope}%
\pgfsetrectcap%
\pgfsetmiterjoin%
\pgfsetlinewidth{0.803000pt}%
\definecolor{currentstroke}{rgb}{0.000000,0.000000,0.000000}%
\pgfsetstrokecolor{currentstroke}%
\pgfsetdash{}{0pt}%
\pgfpathmoveto{\pgfqpoint{0.721913in}{0.549073in}}%
\pgfpathlineto{\pgfqpoint{0.721913in}{2.859073in}}%
\pgfusepath{stroke}%
\end{pgfscope}%
\begin{pgfscope}%
\pgfsetrectcap%
\pgfsetmiterjoin%
\pgfsetlinewidth{0.803000pt}%
\definecolor{currentstroke}{rgb}{0.000000,0.000000,0.000000}%
\pgfsetstrokecolor{currentstroke}%
\pgfsetdash{}{0pt}%
\pgfpathmoveto{\pgfqpoint{5.240163in}{0.549073in}}%
\pgfpathlineto{\pgfqpoint{5.240163in}{2.859073in}}%
\pgfusepath{stroke}%
\end{pgfscope}%
\begin{pgfscope}%
\pgfsetrectcap%
\pgfsetmiterjoin%
\pgfsetlinewidth{0.803000pt}%
\definecolor{currentstroke}{rgb}{0.000000,0.000000,0.000000}%
\pgfsetstrokecolor{currentstroke}%
\pgfsetdash{}{0pt}%
\pgfpathmoveto{\pgfqpoint{0.721913in}{0.549073in}}%
\pgfpathlineto{\pgfqpoint{5.240163in}{0.549073in}}%
\pgfusepath{stroke}%
\end{pgfscope}%
\begin{pgfscope}%
\pgfsetrectcap%
\pgfsetmiterjoin%
\pgfsetlinewidth{0.803000pt}%
\definecolor{currentstroke}{rgb}{0.000000,0.000000,0.000000}%
\pgfsetstrokecolor{currentstroke}%
\pgfsetdash{}{0pt}%
\pgfpathmoveto{\pgfqpoint{0.721913in}{2.859073in}}%
\pgfpathlineto{\pgfqpoint{5.240163in}{2.859073in}}%
\pgfusepath{stroke}%
\end{pgfscope}%
\begin{pgfscope}%
\pgfsetbuttcap%
\pgfsetmiterjoin%
\definecolor{currentfill}{rgb}{1.000000,1.000000,1.000000}%
\pgfsetfillcolor{currentfill}%
\pgfsetfillopacity{0.800000}%
\pgfsetlinewidth{1.003750pt}%
\definecolor{currentstroke}{rgb}{0.800000,0.800000,0.800000}%
\pgfsetstrokecolor{currentstroke}%
\pgfsetstrokeopacity{0.800000}%
\pgfsetdash{}{0pt}%
\pgfpathmoveto{\pgfqpoint{3.278793in}{0.632406in}}%
\pgfpathlineto{\pgfqpoint{5.156830in}{0.632406in}}%
\pgfpathlineto{\pgfqpoint{5.156830in}{1.844442in}}%
\pgfpathlineto{\pgfqpoint{3.278793in}{1.844442in}}%
\pgfpathlineto{\pgfqpoint{3.278793in}{0.632406in}}%
\pgfpathclose%
\pgfusepath{stroke,fill}%
\end{pgfscope}%
\begin{pgfscope}%
\pgfsetrectcap%
\pgfsetroundjoin%
\pgfsetlinewidth{1.505625pt}%
\definecolor{currentstroke}{rgb}{0.392157,0.560784,1.000000}%
\pgfsetstrokecolor{currentstroke}%
\pgfsetdash{}{0pt}%
\pgfpathmoveto{\pgfqpoint{3.345459in}{1.719442in}}%
\pgfpathlineto{\pgfqpoint{3.512126in}{1.719442in}}%
\pgfpathlineto{\pgfqpoint{3.678793in}{1.719442in}}%
\pgfusepath{stroke}%
\end{pgfscope}%
\begin{pgfscope}%
\pgfsetbuttcap%
\pgfsetroundjoin%
\definecolor{currentfill}{rgb}{0.392157,0.560784,1.000000}%
\pgfsetfillcolor{currentfill}%
\pgfsetlinewidth{1.003750pt}%
\definecolor{currentstroke}{rgb}{0.392157,0.560784,1.000000}%
\pgfsetstrokecolor{currentstroke}%
\pgfsetdash{}{0pt}%
\pgfsys@defobject{currentmarker}{\pgfqpoint{-0.031250in}{-0.031250in}}{\pgfqpoint{0.031250in}{0.031250in}}{%
\pgfpathmoveto{\pgfqpoint{0.000000in}{-0.031250in}}%
\pgfpathcurveto{\pgfqpoint{0.008288in}{-0.031250in}}{\pgfqpoint{0.016237in}{-0.027957in}}{\pgfqpoint{0.022097in}{-0.022097in}}%
\pgfpathcurveto{\pgfqpoint{0.027957in}{-0.016237in}}{\pgfqpoint{0.031250in}{-0.008288in}}{\pgfqpoint{0.031250in}{0.000000in}}%
\pgfpathcurveto{\pgfqpoint{0.031250in}{0.008288in}}{\pgfqpoint{0.027957in}{0.016237in}}{\pgfqpoint{0.022097in}{0.022097in}}%
\pgfpathcurveto{\pgfqpoint{0.016237in}{0.027957in}}{\pgfqpoint{0.008288in}{0.031250in}}{\pgfqpoint{0.000000in}{0.031250in}}%
\pgfpathcurveto{\pgfqpoint{-0.008288in}{0.031250in}}{\pgfqpoint{-0.016237in}{0.027957in}}{\pgfqpoint{-0.022097in}{0.022097in}}%
\pgfpathcurveto{\pgfqpoint{-0.027957in}{0.016237in}}{\pgfqpoint{-0.031250in}{0.008288in}}{\pgfqpoint{-0.031250in}{0.000000in}}%
\pgfpathcurveto{\pgfqpoint{-0.031250in}{-0.008288in}}{\pgfqpoint{-0.027957in}{-0.016237in}}{\pgfqpoint{-0.022097in}{-0.022097in}}%
\pgfpathcurveto{\pgfqpoint{-0.016237in}{-0.027957in}}{\pgfqpoint{-0.008288in}{-0.031250in}}{\pgfqpoint{0.000000in}{-0.031250in}}%
\pgfpathlineto{\pgfqpoint{0.000000in}{-0.031250in}}%
\pgfpathclose%
\pgfusepath{stroke,fill}%
}%
\begin{pgfscope}%
\pgfsys@transformshift{3.512126in}{1.719442in}%
\pgfsys@useobject{currentmarker}{}%
\end{pgfscope}%
\end{pgfscope}%
\begin{pgfscope}%
\definecolor{textcolor}{rgb}{0.000000,0.000000,0.000000}%
\pgfsetstrokecolor{textcolor}%
\pgfsetfillcolor{textcolor}%
\pgftext[x=3.812126in,y=1.661108in,left,base]{\color{textcolor}{\rmfamily\fontsize{12.000000}{14.400000}\selectfont\catcode`\^=\active\def^{\ifmmode\sp\else\^{}\fi}\catcode`\%=\active\def%{\%}$n_{\mathbf{\Psi}} = 80$, $n_{\mathbf{\Omega}} = 0$}}%
\end{pgfscope}%
\begin{pgfscope}%
\pgfsetrectcap%
\pgfsetroundjoin%
\pgfsetlinewidth{1.505625pt}%
\definecolor{currentstroke}{rgb}{0.470588,0.368627,0.941176}%
\pgfsetstrokecolor{currentstroke}%
\pgfsetdash{}{0pt}%
\pgfpathmoveto{\pgfqpoint{3.345459in}{1.487034in}}%
\pgfpathlineto{\pgfqpoint{3.512126in}{1.487034in}}%
\pgfpathlineto{\pgfqpoint{3.678793in}{1.487034in}}%
\pgfusepath{stroke}%
\end{pgfscope}%
\begin{pgfscope}%
\pgfsetbuttcap%
\pgfsetmiterjoin%
\definecolor{currentfill}{rgb}{0.470588,0.368627,0.941176}%
\pgfsetfillcolor{currentfill}%
\pgfsetlinewidth{1.003750pt}%
\definecolor{currentstroke}{rgb}{0.470588,0.368627,0.941176}%
\pgfsetstrokecolor{currentstroke}%
\pgfsetdash{}{0pt}%
\pgfsys@defobject{currentmarker}{\pgfqpoint{-0.031250in}{-0.031250in}}{\pgfqpoint{0.031250in}{0.031250in}}{%
\pgfpathmoveto{\pgfqpoint{0.000000in}{0.031250in}}%
\pgfpathlineto{\pgfqpoint{-0.031250in}{-0.031250in}}%
\pgfpathlineto{\pgfqpoint{0.031250in}{-0.031250in}}%
\pgfpathlineto{\pgfqpoint{0.000000in}{0.031250in}}%
\pgfpathclose%
\pgfusepath{stroke,fill}%
}%
\begin{pgfscope}%
\pgfsys@transformshift{3.512126in}{1.487034in}%
\pgfsys@useobject{currentmarker}{}%
\end{pgfscope}%
\end{pgfscope}%
\begin{pgfscope}%
\definecolor{textcolor}{rgb}{0.000000,0.000000,0.000000}%
\pgfsetstrokecolor{textcolor}%
\pgfsetfillcolor{textcolor}%
\pgftext[x=3.812126in,y=1.428701in,left,base]{\color{textcolor}{\rmfamily\fontsize{12.000000}{14.400000}\selectfont\catcode`\^=\active\def^{\ifmmode\sp\else\^{}\fi}\catcode`\%=\active\def%{\%}$n_{\mathbf{\Psi}} = 60$, $n_{\mathbf{\Omega}} = 20$}}%
\end{pgfscope}%
\begin{pgfscope}%
\pgfsetrectcap%
\pgfsetroundjoin%
\pgfsetlinewidth{1.505625pt}%
\definecolor{currentstroke}{rgb}{0.862745,0.149020,0.498039}%
\pgfsetstrokecolor{currentstroke}%
\pgfsetdash{}{0pt}%
\pgfpathmoveto{\pgfqpoint{3.345459in}{1.254627in}}%
\pgfpathlineto{\pgfqpoint{3.512126in}{1.254627in}}%
\pgfpathlineto{\pgfqpoint{3.678793in}{1.254627in}}%
\pgfusepath{stroke}%
\end{pgfscope}%
\begin{pgfscope}%
\pgfsetbuttcap%
\pgfsetmiterjoin%
\definecolor{currentfill}{rgb}{0.862745,0.149020,0.498039}%
\pgfsetfillcolor{currentfill}%
\pgfsetlinewidth{1.003750pt}%
\definecolor{currentstroke}{rgb}{0.862745,0.149020,0.498039}%
\pgfsetstrokecolor{currentstroke}%
\pgfsetdash{}{0pt}%
\pgfsys@defobject{currentmarker}{\pgfqpoint{-0.031250in}{-0.031250in}}{\pgfqpoint{0.031250in}{0.031250in}}{%
\pgfpathmoveto{\pgfqpoint{-0.031250in}{-0.031250in}}%
\pgfpathlineto{\pgfqpoint{0.031250in}{-0.031250in}}%
\pgfpathlineto{\pgfqpoint{0.031250in}{0.031250in}}%
\pgfpathlineto{\pgfqpoint{-0.031250in}{0.031250in}}%
\pgfpathlineto{\pgfqpoint{-0.031250in}{-0.031250in}}%
\pgfpathclose%
\pgfusepath{stroke,fill}%
}%
\begin{pgfscope}%
\pgfsys@transformshift{3.512126in}{1.254627in}%
\pgfsys@useobject{currentmarker}{}%
\end{pgfscope}%
\end{pgfscope}%
\begin{pgfscope}%
\definecolor{textcolor}{rgb}{0.000000,0.000000,0.000000}%
\pgfsetstrokecolor{textcolor}%
\pgfsetfillcolor{textcolor}%
\pgftext[x=3.812126in,y=1.196294in,left,base]{\color{textcolor}{\rmfamily\fontsize{12.000000}{14.400000}\selectfont\catcode`\^=\active\def^{\ifmmode\sp\else\^{}\fi}\catcode`\%=\active\def%{\%}$n_{\mathbf{\Psi}} = 40$, $n_{\mathbf{\Omega}} = 40$}}%
\end{pgfscope}%
\begin{pgfscope}%
\pgfsetrectcap%
\pgfsetroundjoin%
\pgfsetlinewidth{1.505625pt}%
\definecolor{currentstroke}{rgb}{0.996078,0.380392,0.000000}%
\pgfsetstrokecolor{currentstroke}%
\pgfsetdash{}{0pt}%
\pgfpathmoveto{\pgfqpoint{3.345459in}{1.022220in}}%
\pgfpathlineto{\pgfqpoint{3.512126in}{1.022220in}}%
\pgfpathlineto{\pgfqpoint{3.678793in}{1.022220in}}%
\pgfusepath{stroke}%
\end{pgfscope}%
\begin{pgfscope}%
\pgfsetbuttcap%
\pgfsetmiterjoin%
\definecolor{currentfill}{rgb}{0.996078,0.380392,0.000000}%
\pgfsetfillcolor{currentfill}%
\pgfsetlinewidth{1.003750pt}%
\definecolor{currentstroke}{rgb}{0.996078,0.380392,0.000000}%
\pgfsetstrokecolor{currentstroke}%
\pgfsetdash{}{0pt}%
\pgfsys@defobject{currentmarker}{\pgfqpoint{-0.029721in}{-0.025282in}}{\pgfqpoint{0.029721in}{0.031250in}}{%
\pgfpathmoveto{\pgfqpoint{0.000000in}{0.031250in}}%
\pgfpathlineto{\pgfqpoint{-0.029721in}{0.009657in}}%
\pgfpathlineto{\pgfqpoint{-0.018368in}{-0.025282in}}%
\pgfpathlineto{\pgfqpoint{0.018368in}{-0.025282in}}%
\pgfpathlineto{\pgfqpoint{0.029721in}{0.009657in}}%
\pgfpathlineto{\pgfqpoint{0.000000in}{0.031250in}}%
\pgfpathclose%
\pgfusepath{stroke,fill}%
}%
\begin{pgfscope}%
\pgfsys@transformshift{3.512126in}{1.022220in}%
\pgfsys@useobject{currentmarker}{}%
\end{pgfscope}%
\end{pgfscope}%
\begin{pgfscope}%
\definecolor{textcolor}{rgb}{0.000000,0.000000,0.000000}%
\pgfsetstrokecolor{textcolor}%
\pgfsetfillcolor{textcolor}%
\pgftext[x=3.812126in,y=0.963887in,left,base]{\color{textcolor}{\rmfamily\fontsize{12.000000}{14.400000}\selectfont\catcode`\^=\active\def^{\ifmmode\sp\else\^{}\fi}\catcode`\%=\active\def%{\%}$n_{\mathbf{\Psi}} = 20$, $n_{\mathbf{\Omega}} = 60$}}%
\end{pgfscope}%
\begin{pgfscope}%
\pgfsetrectcap%
\pgfsetroundjoin%
\pgfsetlinewidth{1.505625pt}%
\definecolor{currentstroke}{rgb}{1.000000,0.690196,0.000000}%
\pgfsetstrokecolor{currentstroke}%
\pgfsetdash{}{0pt}%
\pgfpathmoveto{\pgfqpoint{3.345459in}{0.789813in}}%
\pgfpathlineto{\pgfqpoint{3.512126in}{0.789813in}}%
\pgfpathlineto{\pgfqpoint{3.678793in}{0.789813in}}%
\pgfusepath{stroke}%
\end{pgfscope}%
\begin{pgfscope}%
\pgfsetbuttcap%
\pgfsetmiterjoin%
\definecolor{currentfill}{rgb}{1.000000,0.690196,0.000000}%
\pgfsetfillcolor{currentfill}%
\pgfsetlinewidth{1.003750pt}%
\definecolor{currentstroke}{rgb}{1.000000,0.690196,0.000000}%
\pgfsetstrokecolor{currentstroke}%
\pgfsetdash{}{0pt}%
\pgfsys@defobject{currentmarker}{\pgfqpoint{-0.026517in}{-0.044194in}}{\pgfqpoint{0.026517in}{0.044194in}}{%
\pgfpathmoveto{\pgfqpoint{-0.000000in}{-0.044194in}}%
\pgfpathlineto{\pgfqpoint{0.026517in}{0.000000in}}%
\pgfpathlineto{\pgfqpoint{0.000000in}{0.044194in}}%
\pgfpathlineto{\pgfqpoint{-0.026517in}{0.000000in}}%
\pgfpathlineto{\pgfqpoint{-0.000000in}{-0.044194in}}%
\pgfpathclose%
\pgfusepath{stroke,fill}%
}%
\begin{pgfscope}%
\pgfsys@transformshift{3.512126in}{0.789813in}%
\pgfsys@useobject{currentmarker}{}%
\end{pgfscope}%
\end{pgfscope}%
\begin{pgfscope}%
\definecolor{textcolor}{rgb}{0.000000,0.000000,0.000000}%
\pgfsetstrokecolor{textcolor}%
\pgfsetfillcolor{textcolor}%
\pgftext[x=3.812126in,y=0.731480in,left,base]{\color{textcolor}{\rmfamily\fontsize{12.000000}{14.400000}\selectfont\catcode`\^=\active\def^{\ifmmode\sp\else\^{}\fi}\catcode`\%=\active\def%{\%}$n_{\mathbf{\Psi}} = 0$, $n_{\mathbf{\Omega}} = 80$}}%
\end{pgfscope}%
\end{pgfpicture}%
\makeatother%
\endgroup%

    \caption{For a range of values of $\sigma$, the Chebyshev-Nyström++ method is applied to the Hamiltonian matrix described in \refsec{subsec:hamiltonian} with $n_c = 1$ for different ways of allocating a total of $n_{\mtx{\Psi}} + n_{\mtx{\Omega}}=80$ random vectors to either the Nystr\"om low-rank approximation or the Girard-Hutchinson trace estimation. We make the approximation error committed by the Chebyshev approximation negligible by rescaling $m=16 / \sigma$ (based on \reflem{lem:non-negative-chebyshev-error}).}
    \label{fig:distribution}
\end{figure}

\subsection{Comparison to Krylov-aware stochastic trace estimation}
\label{subsec:krylov-aware}

The Krylov-aware stochastic trace estimator \cite[Algorithm 3.1]{chen-2023-krylovaware-stochastic} is also suitable in the context of spectral density estimation. It also samples two Gaussian random matrices $\mtx{\Omega} \in \mathbb{R}^{n \times n_{\mtx{\Omega}}}$ and $\mtx{\Psi} \in \mathbb{R}^{n \times n_{\mtx{\Psi}}}$. $\mtx{\Omega}$ is used to extract a low-rank approximation to $g_{\sigma}(t \mtx{I}_n - \mtx{A})$ by running the block-Lanczos algorithm on $\mtx{A}$ with starting block $\mtx{\Omega}$ for $r$ iterations with reorthogonalization and subsequently $q$ without. The columns of $\mtx{\Psi}$ are then used to estimate the trace of the approximation residual with $r$ iterations of the Lanczos algorithm on $\mtx{A}$.

We run our own, faithfully optimized implementation of the Krylov-aware stochastic trace estimator \cite[Algorithm 3.1]{chen-2023-krylovaware-stochastic} in multiple configurations on the example from \refsec{subsec:hamiltonian} and plot the approximation errors for logarithmically spaced values of the smoothing parameter $\sigma$ in \reffig{fig:krylov-aware-density}. For reference, we add the error of the Chebyshev-Nyström++ method on parameters which lead to a comparable run-time.

\begin{figure}[ht]
    \begin{minipage}[c]{.475\linewidth}
        \centering
        %% Creator: Matplotlib, PGF backend
%%
%% To include the figure in your LaTeX document, write
%%   \input{<filename>.pgf}
%%
%% Make sure the required packages are loaded in your preamble
%%   \usepackage{pgf}
%%
%% Also ensure that all the required font packages are loaded; for instance,
%% the lmodern package is sometimes necessary when using math font.
%%   \usepackage{lmodern}
%%
%% Figures using additional raster images can only be included by \input if
%% they are in the same directory as the main LaTeX file. For loading figures
%% from other directories you can use the `import` package
%%   \usepackage{import}
%%
%% and then include the figures with
%%   \import{<path to file>}{<filename>.pgf}
%%
%% Matplotlib used the following preamble
%%   \def\mathdefault#1{#1}
%%   \everymath=\expandafter{\the\everymath\displaystyle}
%%   
%%   \ifdefined\pdftexversion\else  % non-pdftex case.
%%     \usepackage{fontspec}
%%     \setmainfont{DejaVuSerif.ttf}[Path=\detokenize{/opt/hostedtoolcache/Python/3.12.3/x64/lib/python3.12/site-packages/matplotlib/mpl-data/fonts/ttf/}]
%%     \setsansfont{DejaVuSans.ttf}[Path=\detokenize{/opt/hostedtoolcache/Python/3.12.3/x64/lib/python3.12/site-packages/matplotlib/mpl-data/fonts/ttf/}]
%%     \setmonofont{DejaVuSansMono.ttf}[Path=\detokenize{/opt/hostedtoolcache/Python/3.12.3/x64/lib/python3.12/site-packages/matplotlib/mpl-data/fonts/ttf/}]
%%   \fi
%%   \makeatletter\@ifpackageloaded{underscore}{}{\usepackage[strings]{underscore}}\makeatother
%%
\begingroup%
\makeatletter%
\begin{pgfpicture}%
\pgfpathrectangle{\pgfpointorigin}{\pgfqpoint{3.252050in}{2.959073in}}%
\pgfusepath{use as bounding box, clip}%
\begin{pgfscope}%
\pgfsetbuttcap%
\pgfsetmiterjoin%
\definecolor{currentfill}{rgb}{1.000000,1.000000,1.000000}%
\pgfsetfillcolor{currentfill}%
\pgfsetlinewidth{0.000000pt}%
\definecolor{currentstroke}{rgb}{1.000000,1.000000,1.000000}%
\pgfsetstrokecolor{currentstroke}%
\pgfsetdash{}{0pt}%
\pgfpathmoveto{\pgfqpoint{0.000000in}{-0.000000in}}%
\pgfpathlineto{\pgfqpoint{3.252050in}{-0.000000in}}%
\pgfpathlineto{\pgfqpoint{3.252050in}{2.959073in}}%
\pgfpathlineto{\pgfqpoint{0.000000in}{2.959073in}}%
\pgfpathlineto{\pgfqpoint{0.000000in}{-0.000000in}}%
\pgfpathclose%
\pgfusepath{fill}%
\end{pgfscope}%
\begin{pgfscope}%
\pgfsetbuttcap%
\pgfsetmiterjoin%
\definecolor{currentfill}{rgb}{1.000000,1.000000,1.000000}%
\pgfsetfillcolor{currentfill}%
\pgfsetlinewidth{0.000000pt}%
\definecolor{currentstroke}{rgb}{0.000000,0.000000,0.000000}%
\pgfsetstrokecolor{currentstroke}%
\pgfsetstrokeopacity{0.000000}%
\pgfsetdash{}{0pt}%
\pgfpathmoveto{\pgfqpoint{0.721913in}{0.549073in}}%
\pgfpathlineto{\pgfqpoint{3.046913in}{0.549073in}}%
\pgfpathlineto{\pgfqpoint{3.046913in}{2.859073in}}%
\pgfpathlineto{\pgfqpoint{0.721913in}{2.859073in}}%
\pgfpathlineto{\pgfqpoint{0.721913in}{0.549073in}}%
\pgfpathclose%
\pgfusepath{fill}%
\end{pgfscope}%
\begin{pgfscope}%
\pgfpathrectangle{\pgfqpoint{0.721913in}{0.549073in}}{\pgfqpoint{2.325000in}{2.310000in}}%
\pgfusepath{clip}%
\pgfsetrectcap%
\pgfsetroundjoin%
\pgfsetlinewidth{0.250937pt}%
\definecolor{currentstroke}{rgb}{0.000000,0.000000,0.000000}%
\pgfsetstrokecolor{currentstroke}%
\pgfsetstrokeopacity{0.200000}%
\pgfsetdash{}{0pt}%
\pgfpathmoveto{\pgfqpoint{1.220128in}{0.549073in}}%
\pgfpathlineto{\pgfqpoint{1.220128in}{2.859073in}}%
\pgfusepath{stroke}%
\end{pgfscope}%
\begin{pgfscope}%
\pgfsetbuttcap%
\pgfsetroundjoin%
\definecolor{currentfill}{rgb}{0.000000,0.000000,0.000000}%
\pgfsetfillcolor{currentfill}%
\pgfsetlinewidth{0.803000pt}%
\definecolor{currentstroke}{rgb}{0.000000,0.000000,0.000000}%
\pgfsetstrokecolor{currentstroke}%
\pgfsetdash{}{0pt}%
\pgfsys@defobject{currentmarker}{\pgfqpoint{0.000000in}{-0.048611in}}{\pgfqpoint{0.000000in}{0.000000in}}{%
\pgfpathmoveto{\pgfqpoint{0.000000in}{0.000000in}}%
\pgfpathlineto{\pgfqpoint{0.000000in}{-0.048611in}}%
\pgfusepath{stroke,fill}%
}%
\begin{pgfscope}%
\pgfsys@transformshift{1.220128in}{0.549073in}%
\pgfsys@useobject{currentmarker}{}%
\end{pgfscope}%
\end{pgfscope}%
\begin{pgfscope}%
\definecolor{textcolor}{rgb}{0.000000,0.000000,0.000000}%
\pgfsetstrokecolor{textcolor}%
\pgfsetfillcolor{textcolor}%
\pgftext[x=1.220128in,y=0.451851in,,top]{\color{textcolor}{\rmfamily\fontsize{12.000000}{14.400000}\selectfont\catcode`\^=\active\def^{\ifmmode\sp\else\^{}\fi}\catcode`\%=\active\def%{\%}$\mathdefault{10^{-3}}$}}%
\end{pgfscope}%
\begin{pgfscope}%
\pgfpathrectangle{\pgfqpoint{0.721913in}{0.549073in}}{\pgfqpoint{2.325000in}{2.310000in}}%
\pgfusepath{clip}%
\pgfsetrectcap%
\pgfsetroundjoin%
\pgfsetlinewidth{0.250937pt}%
\definecolor{currentstroke}{rgb}{0.000000,0.000000,0.000000}%
\pgfsetstrokecolor{currentstroke}%
\pgfsetstrokeopacity{0.200000}%
\pgfsetdash{}{0pt}%
\pgfpathmoveto{\pgfqpoint{2.105842in}{0.549073in}}%
\pgfpathlineto{\pgfqpoint{2.105842in}{2.859073in}}%
\pgfusepath{stroke}%
\end{pgfscope}%
\begin{pgfscope}%
\pgfsetbuttcap%
\pgfsetroundjoin%
\definecolor{currentfill}{rgb}{0.000000,0.000000,0.000000}%
\pgfsetfillcolor{currentfill}%
\pgfsetlinewidth{0.803000pt}%
\definecolor{currentstroke}{rgb}{0.000000,0.000000,0.000000}%
\pgfsetstrokecolor{currentstroke}%
\pgfsetdash{}{0pt}%
\pgfsys@defobject{currentmarker}{\pgfqpoint{0.000000in}{-0.048611in}}{\pgfqpoint{0.000000in}{0.000000in}}{%
\pgfpathmoveto{\pgfqpoint{0.000000in}{0.000000in}}%
\pgfpathlineto{\pgfqpoint{0.000000in}{-0.048611in}}%
\pgfusepath{stroke,fill}%
}%
\begin{pgfscope}%
\pgfsys@transformshift{2.105842in}{0.549073in}%
\pgfsys@useobject{currentmarker}{}%
\end{pgfscope}%
\end{pgfscope}%
\begin{pgfscope}%
\definecolor{textcolor}{rgb}{0.000000,0.000000,0.000000}%
\pgfsetstrokecolor{textcolor}%
\pgfsetfillcolor{textcolor}%
\pgftext[x=2.105842in,y=0.451851in,,top]{\color{textcolor}{\rmfamily\fontsize{12.000000}{14.400000}\selectfont\catcode`\^=\active\def^{\ifmmode\sp\else\^{}\fi}\catcode`\%=\active\def%{\%}$\mathdefault{10^{-2}}$}}%
\end{pgfscope}%
\begin{pgfscope}%
\pgfpathrectangle{\pgfqpoint{0.721913in}{0.549073in}}{\pgfqpoint{2.325000in}{2.310000in}}%
\pgfusepath{clip}%
\pgfsetrectcap%
\pgfsetroundjoin%
\pgfsetlinewidth{0.250937pt}%
\definecolor{currentstroke}{rgb}{0.000000,0.000000,0.000000}%
\pgfsetstrokecolor{currentstroke}%
\pgfsetstrokeopacity{0.200000}%
\pgfsetdash{}{0pt}%
\pgfpathmoveto{\pgfqpoint{2.991556in}{0.549073in}}%
\pgfpathlineto{\pgfqpoint{2.991556in}{2.859073in}}%
\pgfusepath{stroke}%
\end{pgfscope}%
\begin{pgfscope}%
\pgfsetbuttcap%
\pgfsetroundjoin%
\definecolor{currentfill}{rgb}{0.000000,0.000000,0.000000}%
\pgfsetfillcolor{currentfill}%
\pgfsetlinewidth{0.803000pt}%
\definecolor{currentstroke}{rgb}{0.000000,0.000000,0.000000}%
\pgfsetstrokecolor{currentstroke}%
\pgfsetdash{}{0pt}%
\pgfsys@defobject{currentmarker}{\pgfqpoint{0.000000in}{-0.048611in}}{\pgfqpoint{0.000000in}{0.000000in}}{%
\pgfpathmoveto{\pgfqpoint{0.000000in}{0.000000in}}%
\pgfpathlineto{\pgfqpoint{0.000000in}{-0.048611in}}%
\pgfusepath{stroke,fill}%
}%
\begin{pgfscope}%
\pgfsys@transformshift{2.991556in}{0.549073in}%
\pgfsys@useobject{currentmarker}{}%
\end{pgfscope}%
\end{pgfscope}%
\begin{pgfscope}%
\definecolor{textcolor}{rgb}{0.000000,0.000000,0.000000}%
\pgfsetstrokecolor{textcolor}%
\pgfsetfillcolor{textcolor}%
\pgftext[x=2.991556in,y=0.451851in,,top]{\color{textcolor}{\rmfamily\fontsize{12.000000}{14.400000}\selectfont\catcode`\^=\active\def^{\ifmmode\sp\else\^{}\fi}\catcode`\%=\active\def%{\%}$\mathdefault{10^{-1}}$}}%
\end{pgfscope}%
\begin{pgfscope}%
\pgfpathrectangle{\pgfqpoint{0.721913in}{0.549073in}}{\pgfqpoint{2.325000in}{2.310000in}}%
\pgfusepath{clip}%
\pgfsetrectcap%
\pgfsetroundjoin%
\pgfsetlinewidth{0.250937pt}%
\definecolor{currentstroke}{rgb}{0.000000,0.000000,0.000000}%
\pgfsetstrokecolor{currentstroke}%
\pgfsetstrokeopacity{0.200000}%
\pgfsetdash{}{0pt}%
\pgfpathmoveto{\pgfqpoint{0.757007in}{0.549073in}}%
\pgfpathlineto{\pgfqpoint{0.757007in}{2.859073in}}%
\pgfusepath{stroke}%
\end{pgfscope}%
\begin{pgfscope}%
\pgfsetbuttcap%
\pgfsetroundjoin%
\definecolor{currentfill}{rgb}{0.000000,0.000000,0.000000}%
\pgfsetfillcolor{currentfill}%
\pgfsetlinewidth{0.602250pt}%
\definecolor{currentstroke}{rgb}{0.000000,0.000000,0.000000}%
\pgfsetstrokecolor{currentstroke}%
\pgfsetdash{}{0pt}%
\pgfsys@defobject{currentmarker}{\pgfqpoint{0.000000in}{-0.027778in}}{\pgfqpoint{0.000000in}{0.000000in}}{%
\pgfpathmoveto{\pgfqpoint{0.000000in}{0.000000in}}%
\pgfpathlineto{\pgfqpoint{0.000000in}{-0.027778in}}%
\pgfusepath{stroke,fill}%
}%
\begin{pgfscope}%
\pgfsys@transformshift{0.757007in}{0.549073in}%
\pgfsys@useobject{currentmarker}{}%
\end{pgfscope}%
\end{pgfscope}%
\begin{pgfscope}%
\pgfpathrectangle{\pgfqpoint{0.721913in}{0.549073in}}{\pgfqpoint{2.325000in}{2.310000in}}%
\pgfusepath{clip}%
\pgfsetrectcap%
\pgfsetroundjoin%
\pgfsetlinewidth{0.250937pt}%
\definecolor{currentstroke}{rgb}{0.000000,0.000000,0.000000}%
\pgfsetstrokecolor{currentstroke}%
\pgfsetstrokeopacity{0.200000}%
\pgfsetdash{}{0pt}%
\pgfpathmoveto{\pgfqpoint{0.867667in}{0.549073in}}%
\pgfpathlineto{\pgfqpoint{0.867667in}{2.859073in}}%
\pgfusepath{stroke}%
\end{pgfscope}%
\begin{pgfscope}%
\pgfsetbuttcap%
\pgfsetroundjoin%
\definecolor{currentfill}{rgb}{0.000000,0.000000,0.000000}%
\pgfsetfillcolor{currentfill}%
\pgfsetlinewidth{0.602250pt}%
\definecolor{currentstroke}{rgb}{0.000000,0.000000,0.000000}%
\pgfsetstrokecolor{currentstroke}%
\pgfsetdash{}{0pt}%
\pgfsys@defobject{currentmarker}{\pgfqpoint{0.000000in}{-0.027778in}}{\pgfqpoint{0.000000in}{0.000000in}}{%
\pgfpathmoveto{\pgfqpoint{0.000000in}{0.000000in}}%
\pgfpathlineto{\pgfqpoint{0.000000in}{-0.027778in}}%
\pgfusepath{stroke,fill}%
}%
\begin{pgfscope}%
\pgfsys@transformshift{0.867667in}{0.549073in}%
\pgfsys@useobject{currentmarker}{}%
\end{pgfscope}%
\end{pgfscope}%
\begin{pgfscope}%
\pgfpathrectangle{\pgfqpoint{0.721913in}{0.549073in}}{\pgfqpoint{2.325000in}{2.310000in}}%
\pgfusepath{clip}%
\pgfsetrectcap%
\pgfsetroundjoin%
\pgfsetlinewidth{0.250937pt}%
\definecolor{currentstroke}{rgb}{0.000000,0.000000,0.000000}%
\pgfsetstrokecolor{currentstroke}%
\pgfsetstrokeopacity{0.200000}%
\pgfsetdash{}{0pt}%
\pgfpathmoveto{\pgfqpoint{0.953501in}{0.549073in}}%
\pgfpathlineto{\pgfqpoint{0.953501in}{2.859073in}}%
\pgfusepath{stroke}%
\end{pgfscope}%
\begin{pgfscope}%
\pgfsetbuttcap%
\pgfsetroundjoin%
\definecolor{currentfill}{rgb}{0.000000,0.000000,0.000000}%
\pgfsetfillcolor{currentfill}%
\pgfsetlinewidth{0.602250pt}%
\definecolor{currentstroke}{rgb}{0.000000,0.000000,0.000000}%
\pgfsetstrokecolor{currentstroke}%
\pgfsetdash{}{0pt}%
\pgfsys@defobject{currentmarker}{\pgfqpoint{0.000000in}{-0.027778in}}{\pgfqpoint{0.000000in}{0.000000in}}{%
\pgfpathmoveto{\pgfqpoint{0.000000in}{0.000000in}}%
\pgfpathlineto{\pgfqpoint{0.000000in}{-0.027778in}}%
\pgfusepath{stroke,fill}%
}%
\begin{pgfscope}%
\pgfsys@transformshift{0.953501in}{0.549073in}%
\pgfsys@useobject{currentmarker}{}%
\end{pgfscope}%
\end{pgfscope}%
\begin{pgfscope}%
\pgfpathrectangle{\pgfqpoint{0.721913in}{0.549073in}}{\pgfqpoint{2.325000in}{2.310000in}}%
\pgfusepath{clip}%
\pgfsetrectcap%
\pgfsetroundjoin%
\pgfsetlinewidth{0.250937pt}%
\definecolor{currentstroke}{rgb}{0.000000,0.000000,0.000000}%
\pgfsetstrokecolor{currentstroke}%
\pgfsetstrokeopacity{0.200000}%
\pgfsetdash{}{0pt}%
\pgfpathmoveto{\pgfqpoint{1.023633in}{0.549073in}}%
\pgfpathlineto{\pgfqpoint{1.023633in}{2.859073in}}%
\pgfusepath{stroke}%
\end{pgfscope}%
\begin{pgfscope}%
\pgfsetbuttcap%
\pgfsetroundjoin%
\definecolor{currentfill}{rgb}{0.000000,0.000000,0.000000}%
\pgfsetfillcolor{currentfill}%
\pgfsetlinewidth{0.602250pt}%
\definecolor{currentstroke}{rgb}{0.000000,0.000000,0.000000}%
\pgfsetstrokecolor{currentstroke}%
\pgfsetdash{}{0pt}%
\pgfsys@defobject{currentmarker}{\pgfqpoint{0.000000in}{-0.027778in}}{\pgfqpoint{0.000000in}{0.000000in}}{%
\pgfpathmoveto{\pgfqpoint{0.000000in}{0.000000in}}%
\pgfpathlineto{\pgfqpoint{0.000000in}{-0.027778in}}%
\pgfusepath{stroke,fill}%
}%
\begin{pgfscope}%
\pgfsys@transformshift{1.023633in}{0.549073in}%
\pgfsys@useobject{currentmarker}{}%
\end{pgfscope}%
\end{pgfscope}%
\begin{pgfscope}%
\pgfpathrectangle{\pgfqpoint{0.721913in}{0.549073in}}{\pgfqpoint{2.325000in}{2.310000in}}%
\pgfusepath{clip}%
\pgfsetrectcap%
\pgfsetroundjoin%
\pgfsetlinewidth{0.250937pt}%
\definecolor{currentstroke}{rgb}{0.000000,0.000000,0.000000}%
\pgfsetstrokecolor{currentstroke}%
\pgfsetstrokeopacity{0.200000}%
\pgfsetdash{}{0pt}%
\pgfpathmoveto{\pgfqpoint{1.082929in}{0.549073in}}%
\pgfpathlineto{\pgfqpoint{1.082929in}{2.859073in}}%
\pgfusepath{stroke}%
\end{pgfscope}%
\begin{pgfscope}%
\pgfsetbuttcap%
\pgfsetroundjoin%
\definecolor{currentfill}{rgb}{0.000000,0.000000,0.000000}%
\pgfsetfillcolor{currentfill}%
\pgfsetlinewidth{0.602250pt}%
\definecolor{currentstroke}{rgb}{0.000000,0.000000,0.000000}%
\pgfsetstrokecolor{currentstroke}%
\pgfsetdash{}{0pt}%
\pgfsys@defobject{currentmarker}{\pgfqpoint{0.000000in}{-0.027778in}}{\pgfqpoint{0.000000in}{0.000000in}}{%
\pgfpathmoveto{\pgfqpoint{0.000000in}{0.000000in}}%
\pgfpathlineto{\pgfqpoint{0.000000in}{-0.027778in}}%
\pgfusepath{stroke,fill}%
}%
\begin{pgfscope}%
\pgfsys@transformshift{1.082929in}{0.549073in}%
\pgfsys@useobject{currentmarker}{}%
\end{pgfscope}%
\end{pgfscope}%
\begin{pgfscope}%
\pgfpathrectangle{\pgfqpoint{0.721913in}{0.549073in}}{\pgfqpoint{2.325000in}{2.310000in}}%
\pgfusepath{clip}%
\pgfsetrectcap%
\pgfsetroundjoin%
\pgfsetlinewidth{0.250937pt}%
\definecolor{currentstroke}{rgb}{0.000000,0.000000,0.000000}%
\pgfsetstrokecolor{currentstroke}%
\pgfsetstrokeopacity{0.200000}%
\pgfsetdash{}{0pt}%
\pgfpathmoveto{\pgfqpoint{1.134293in}{0.549073in}}%
\pgfpathlineto{\pgfqpoint{1.134293in}{2.859073in}}%
\pgfusepath{stroke}%
\end{pgfscope}%
\begin{pgfscope}%
\pgfsetbuttcap%
\pgfsetroundjoin%
\definecolor{currentfill}{rgb}{0.000000,0.000000,0.000000}%
\pgfsetfillcolor{currentfill}%
\pgfsetlinewidth{0.602250pt}%
\definecolor{currentstroke}{rgb}{0.000000,0.000000,0.000000}%
\pgfsetstrokecolor{currentstroke}%
\pgfsetdash{}{0pt}%
\pgfsys@defobject{currentmarker}{\pgfqpoint{0.000000in}{-0.027778in}}{\pgfqpoint{0.000000in}{0.000000in}}{%
\pgfpathmoveto{\pgfqpoint{0.000000in}{0.000000in}}%
\pgfpathlineto{\pgfqpoint{0.000000in}{-0.027778in}}%
\pgfusepath{stroke,fill}%
}%
\begin{pgfscope}%
\pgfsys@transformshift{1.134293in}{0.549073in}%
\pgfsys@useobject{currentmarker}{}%
\end{pgfscope}%
\end{pgfscope}%
\begin{pgfscope}%
\pgfpathrectangle{\pgfqpoint{0.721913in}{0.549073in}}{\pgfqpoint{2.325000in}{2.310000in}}%
\pgfusepath{clip}%
\pgfsetrectcap%
\pgfsetroundjoin%
\pgfsetlinewidth{0.250937pt}%
\definecolor{currentstroke}{rgb}{0.000000,0.000000,0.000000}%
\pgfsetstrokecolor{currentstroke}%
\pgfsetstrokeopacity{0.200000}%
\pgfsetdash{}{0pt}%
\pgfpathmoveto{\pgfqpoint{1.179600in}{0.549073in}}%
\pgfpathlineto{\pgfqpoint{1.179600in}{2.859073in}}%
\pgfusepath{stroke}%
\end{pgfscope}%
\begin{pgfscope}%
\pgfsetbuttcap%
\pgfsetroundjoin%
\definecolor{currentfill}{rgb}{0.000000,0.000000,0.000000}%
\pgfsetfillcolor{currentfill}%
\pgfsetlinewidth{0.602250pt}%
\definecolor{currentstroke}{rgb}{0.000000,0.000000,0.000000}%
\pgfsetstrokecolor{currentstroke}%
\pgfsetdash{}{0pt}%
\pgfsys@defobject{currentmarker}{\pgfqpoint{0.000000in}{-0.027778in}}{\pgfqpoint{0.000000in}{0.000000in}}{%
\pgfpathmoveto{\pgfqpoint{0.000000in}{0.000000in}}%
\pgfpathlineto{\pgfqpoint{0.000000in}{-0.027778in}}%
\pgfusepath{stroke,fill}%
}%
\begin{pgfscope}%
\pgfsys@transformshift{1.179600in}{0.549073in}%
\pgfsys@useobject{currentmarker}{}%
\end{pgfscope}%
\end{pgfscope}%
\begin{pgfscope}%
\pgfpathrectangle{\pgfqpoint{0.721913in}{0.549073in}}{\pgfqpoint{2.325000in}{2.310000in}}%
\pgfusepath{clip}%
\pgfsetrectcap%
\pgfsetroundjoin%
\pgfsetlinewidth{0.250937pt}%
\definecolor{currentstroke}{rgb}{0.000000,0.000000,0.000000}%
\pgfsetstrokecolor{currentstroke}%
\pgfsetstrokeopacity{0.200000}%
\pgfsetdash{}{0pt}%
\pgfpathmoveto{\pgfqpoint{1.486754in}{0.549073in}}%
\pgfpathlineto{\pgfqpoint{1.486754in}{2.859073in}}%
\pgfusepath{stroke}%
\end{pgfscope}%
\begin{pgfscope}%
\pgfsetbuttcap%
\pgfsetroundjoin%
\definecolor{currentfill}{rgb}{0.000000,0.000000,0.000000}%
\pgfsetfillcolor{currentfill}%
\pgfsetlinewidth{0.602250pt}%
\definecolor{currentstroke}{rgb}{0.000000,0.000000,0.000000}%
\pgfsetstrokecolor{currentstroke}%
\pgfsetdash{}{0pt}%
\pgfsys@defobject{currentmarker}{\pgfqpoint{0.000000in}{-0.027778in}}{\pgfqpoint{0.000000in}{0.000000in}}{%
\pgfpathmoveto{\pgfqpoint{0.000000in}{0.000000in}}%
\pgfpathlineto{\pgfqpoint{0.000000in}{-0.027778in}}%
\pgfusepath{stroke,fill}%
}%
\begin{pgfscope}%
\pgfsys@transformshift{1.486754in}{0.549073in}%
\pgfsys@useobject{currentmarker}{}%
\end{pgfscope}%
\end{pgfscope}%
\begin{pgfscope}%
\pgfpathrectangle{\pgfqpoint{0.721913in}{0.549073in}}{\pgfqpoint{2.325000in}{2.310000in}}%
\pgfusepath{clip}%
\pgfsetrectcap%
\pgfsetroundjoin%
\pgfsetlinewidth{0.250937pt}%
\definecolor{currentstroke}{rgb}{0.000000,0.000000,0.000000}%
\pgfsetstrokecolor{currentstroke}%
\pgfsetstrokeopacity{0.200000}%
\pgfsetdash{}{0pt}%
\pgfpathmoveto{\pgfqpoint{1.642721in}{0.549073in}}%
\pgfpathlineto{\pgfqpoint{1.642721in}{2.859073in}}%
\pgfusepath{stroke}%
\end{pgfscope}%
\begin{pgfscope}%
\pgfsetbuttcap%
\pgfsetroundjoin%
\definecolor{currentfill}{rgb}{0.000000,0.000000,0.000000}%
\pgfsetfillcolor{currentfill}%
\pgfsetlinewidth{0.602250pt}%
\definecolor{currentstroke}{rgb}{0.000000,0.000000,0.000000}%
\pgfsetstrokecolor{currentstroke}%
\pgfsetdash{}{0pt}%
\pgfsys@defobject{currentmarker}{\pgfqpoint{0.000000in}{-0.027778in}}{\pgfqpoint{0.000000in}{0.000000in}}{%
\pgfpathmoveto{\pgfqpoint{0.000000in}{0.000000in}}%
\pgfpathlineto{\pgfqpoint{0.000000in}{-0.027778in}}%
\pgfusepath{stroke,fill}%
}%
\begin{pgfscope}%
\pgfsys@transformshift{1.642721in}{0.549073in}%
\pgfsys@useobject{currentmarker}{}%
\end{pgfscope}%
\end{pgfscope}%
\begin{pgfscope}%
\pgfpathrectangle{\pgfqpoint{0.721913in}{0.549073in}}{\pgfqpoint{2.325000in}{2.310000in}}%
\pgfusepath{clip}%
\pgfsetrectcap%
\pgfsetroundjoin%
\pgfsetlinewidth{0.250937pt}%
\definecolor{currentstroke}{rgb}{0.000000,0.000000,0.000000}%
\pgfsetstrokecolor{currentstroke}%
\pgfsetstrokeopacity{0.200000}%
\pgfsetdash{}{0pt}%
\pgfpathmoveto{\pgfqpoint{1.753381in}{0.549073in}}%
\pgfpathlineto{\pgfqpoint{1.753381in}{2.859073in}}%
\pgfusepath{stroke}%
\end{pgfscope}%
\begin{pgfscope}%
\pgfsetbuttcap%
\pgfsetroundjoin%
\definecolor{currentfill}{rgb}{0.000000,0.000000,0.000000}%
\pgfsetfillcolor{currentfill}%
\pgfsetlinewidth{0.602250pt}%
\definecolor{currentstroke}{rgb}{0.000000,0.000000,0.000000}%
\pgfsetstrokecolor{currentstroke}%
\pgfsetdash{}{0pt}%
\pgfsys@defobject{currentmarker}{\pgfqpoint{0.000000in}{-0.027778in}}{\pgfqpoint{0.000000in}{0.000000in}}{%
\pgfpathmoveto{\pgfqpoint{0.000000in}{0.000000in}}%
\pgfpathlineto{\pgfqpoint{0.000000in}{-0.027778in}}%
\pgfusepath{stroke,fill}%
}%
\begin{pgfscope}%
\pgfsys@transformshift{1.753381in}{0.549073in}%
\pgfsys@useobject{currentmarker}{}%
\end{pgfscope}%
\end{pgfscope}%
\begin{pgfscope}%
\pgfpathrectangle{\pgfqpoint{0.721913in}{0.549073in}}{\pgfqpoint{2.325000in}{2.310000in}}%
\pgfusepath{clip}%
\pgfsetrectcap%
\pgfsetroundjoin%
\pgfsetlinewidth{0.250937pt}%
\definecolor{currentstroke}{rgb}{0.000000,0.000000,0.000000}%
\pgfsetstrokecolor{currentstroke}%
\pgfsetstrokeopacity{0.200000}%
\pgfsetdash{}{0pt}%
\pgfpathmoveto{\pgfqpoint{1.839215in}{0.549073in}}%
\pgfpathlineto{\pgfqpoint{1.839215in}{2.859073in}}%
\pgfusepath{stroke}%
\end{pgfscope}%
\begin{pgfscope}%
\pgfsetbuttcap%
\pgfsetroundjoin%
\definecolor{currentfill}{rgb}{0.000000,0.000000,0.000000}%
\pgfsetfillcolor{currentfill}%
\pgfsetlinewidth{0.602250pt}%
\definecolor{currentstroke}{rgb}{0.000000,0.000000,0.000000}%
\pgfsetstrokecolor{currentstroke}%
\pgfsetdash{}{0pt}%
\pgfsys@defobject{currentmarker}{\pgfqpoint{0.000000in}{-0.027778in}}{\pgfqpoint{0.000000in}{0.000000in}}{%
\pgfpathmoveto{\pgfqpoint{0.000000in}{0.000000in}}%
\pgfpathlineto{\pgfqpoint{0.000000in}{-0.027778in}}%
\pgfusepath{stroke,fill}%
}%
\begin{pgfscope}%
\pgfsys@transformshift{1.839215in}{0.549073in}%
\pgfsys@useobject{currentmarker}{}%
\end{pgfscope}%
\end{pgfscope}%
\begin{pgfscope}%
\pgfpathrectangle{\pgfqpoint{0.721913in}{0.549073in}}{\pgfqpoint{2.325000in}{2.310000in}}%
\pgfusepath{clip}%
\pgfsetrectcap%
\pgfsetroundjoin%
\pgfsetlinewidth{0.250937pt}%
\definecolor{currentstroke}{rgb}{0.000000,0.000000,0.000000}%
\pgfsetstrokecolor{currentstroke}%
\pgfsetstrokeopacity{0.200000}%
\pgfsetdash{}{0pt}%
\pgfpathmoveto{\pgfqpoint{1.909347in}{0.549073in}}%
\pgfpathlineto{\pgfqpoint{1.909347in}{2.859073in}}%
\pgfusepath{stroke}%
\end{pgfscope}%
\begin{pgfscope}%
\pgfsetbuttcap%
\pgfsetroundjoin%
\definecolor{currentfill}{rgb}{0.000000,0.000000,0.000000}%
\pgfsetfillcolor{currentfill}%
\pgfsetlinewidth{0.602250pt}%
\definecolor{currentstroke}{rgb}{0.000000,0.000000,0.000000}%
\pgfsetstrokecolor{currentstroke}%
\pgfsetdash{}{0pt}%
\pgfsys@defobject{currentmarker}{\pgfqpoint{0.000000in}{-0.027778in}}{\pgfqpoint{0.000000in}{0.000000in}}{%
\pgfpathmoveto{\pgfqpoint{0.000000in}{0.000000in}}%
\pgfpathlineto{\pgfqpoint{0.000000in}{-0.027778in}}%
\pgfusepath{stroke,fill}%
}%
\begin{pgfscope}%
\pgfsys@transformshift{1.909347in}{0.549073in}%
\pgfsys@useobject{currentmarker}{}%
\end{pgfscope}%
\end{pgfscope}%
\begin{pgfscope}%
\pgfpathrectangle{\pgfqpoint{0.721913in}{0.549073in}}{\pgfqpoint{2.325000in}{2.310000in}}%
\pgfusepath{clip}%
\pgfsetrectcap%
\pgfsetroundjoin%
\pgfsetlinewidth{0.250937pt}%
\definecolor{currentstroke}{rgb}{0.000000,0.000000,0.000000}%
\pgfsetstrokecolor{currentstroke}%
\pgfsetstrokeopacity{0.200000}%
\pgfsetdash{}{0pt}%
\pgfpathmoveto{\pgfqpoint{1.968643in}{0.549073in}}%
\pgfpathlineto{\pgfqpoint{1.968643in}{2.859073in}}%
\pgfusepath{stroke}%
\end{pgfscope}%
\begin{pgfscope}%
\pgfsetbuttcap%
\pgfsetroundjoin%
\definecolor{currentfill}{rgb}{0.000000,0.000000,0.000000}%
\pgfsetfillcolor{currentfill}%
\pgfsetlinewidth{0.602250pt}%
\definecolor{currentstroke}{rgb}{0.000000,0.000000,0.000000}%
\pgfsetstrokecolor{currentstroke}%
\pgfsetdash{}{0pt}%
\pgfsys@defobject{currentmarker}{\pgfqpoint{0.000000in}{-0.027778in}}{\pgfqpoint{0.000000in}{0.000000in}}{%
\pgfpathmoveto{\pgfqpoint{0.000000in}{0.000000in}}%
\pgfpathlineto{\pgfqpoint{0.000000in}{-0.027778in}}%
\pgfusepath{stroke,fill}%
}%
\begin{pgfscope}%
\pgfsys@transformshift{1.968643in}{0.549073in}%
\pgfsys@useobject{currentmarker}{}%
\end{pgfscope}%
\end{pgfscope}%
\begin{pgfscope}%
\pgfpathrectangle{\pgfqpoint{0.721913in}{0.549073in}}{\pgfqpoint{2.325000in}{2.310000in}}%
\pgfusepath{clip}%
\pgfsetrectcap%
\pgfsetroundjoin%
\pgfsetlinewidth{0.250937pt}%
\definecolor{currentstroke}{rgb}{0.000000,0.000000,0.000000}%
\pgfsetstrokecolor{currentstroke}%
\pgfsetstrokeopacity{0.200000}%
\pgfsetdash{}{0pt}%
\pgfpathmoveto{\pgfqpoint{2.020007in}{0.549073in}}%
\pgfpathlineto{\pgfqpoint{2.020007in}{2.859073in}}%
\pgfusepath{stroke}%
\end{pgfscope}%
\begin{pgfscope}%
\pgfsetbuttcap%
\pgfsetroundjoin%
\definecolor{currentfill}{rgb}{0.000000,0.000000,0.000000}%
\pgfsetfillcolor{currentfill}%
\pgfsetlinewidth{0.602250pt}%
\definecolor{currentstroke}{rgb}{0.000000,0.000000,0.000000}%
\pgfsetstrokecolor{currentstroke}%
\pgfsetdash{}{0pt}%
\pgfsys@defobject{currentmarker}{\pgfqpoint{0.000000in}{-0.027778in}}{\pgfqpoint{0.000000in}{0.000000in}}{%
\pgfpathmoveto{\pgfqpoint{0.000000in}{0.000000in}}%
\pgfpathlineto{\pgfqpoint{0.000000in}{-0.027778in}}%
\pgfusepath{stroke,fill}%
}%
\begin{pgfscope}%
\pgfsys@transformshift{2.020007in}{0.549073in}%
\pgfsys@useobject{currentmarker}{}%
\end{pgfscope}%
\end{pgfscope}%
\begin{pgfscope}%
\pgfpathrectangle{\pgfqpoint{0.721913in}{0.549073in}}{\pgfqpoint{2.325000in}{2.310000in}}%
\pgfusepath{clip}%
\pgfsetrectcap%
\pgfsetroundjoin%
\pgfsetlinewidth{0.250937pt}%
\definecolor{currentstroke}{rgb}{0.000000,0.000000,0.000000}%
\pgfsetstrokecolor{currentstroke}%
\pgfsetstrokeopacity{0.200000}%
\pgfsetdash{}{0pt}%
\pgfpathmoveto{\pgfqpoint{2.065314in}{0.549073in}}%
\pgfpathlineto{\pgfqpoint{2.065314in}{2.859073in}}%
\pgfusepath{stroke}%
\end{pgfscope}%
\begin{pgfscope}%
\pgfsetbuttcap%
\pgfsetroundjoin%
\definecolor{currentfill}{rgb}{0.000000,0.000000,0.000000}%
\pgfsetfillcolor{currentfill}%
\pgfsetlinewidth{0.602250pt}%
\definecolor{currentstroke}{rgb}{0.000000,0.000000,0.000000}%
\pgfsetstrokecolor{currentstroke}%
\pgfsetdash{}{0pt}%
\pgfsys@defobject{currentmarker}{\pgfqpoint{0.000000in}{-0.027778in}}{\pgfqpoint{0.000000in}{0.000000in}}{%
\pgfpathmoveto{\pgfqpoint{0.000000in}{0.000000in}}%
\pgfpathlineto{\pgfqpoint{0.000000in}{-0.027778in}}%
\pgfusepath{stroke,fill}%
}%
\begin{pgfscope}%
\pgfsys@transformshift{2.065314in}{0.549073in}%
\pgfsys@useobject{currentmarker}{}%
\end{pgfscope}%
\end{pgfscope}%
\begin{pgfscope}%
\pgfpathrectangle{\pgfqpoint{0.721913in}{0.549073in}}{\pgfqpoint{2.325000in}{2.310000in}}%
\pgfusepath{clip}%
\pgfsetrectcap%
\pgfsetroundjoin%
\pgfsetlinewidth{0.250937pt}%
\definecolor{currentstroke}{rgb}{0.000000,0.000000,0.000000}%
\pgfsetstrokecolor{currentstroke}%
\pgfsetstrokeopacity{0.200000}%
\pgfsetdash{}{0pt}%
\pgfpathmoveto{\pgfqpoint{2.372469in}{0.549073in}}%
\pgfpathlineto{\pgfqpoint{2.372469in}{2.859073in}}%
\pgfusepath{stroke}%
\end{pgfscope}%
\begin{pgfscope}%
\pgfsetbuttcap%
\pgfsetroundjoin%
\definecolor{currentfill}{rgb}{0.000000,0.000000,0.000000}%
\pgfsetfillcolor{currentfill}%
\pgfsetlinewidth{0.602250pt}%
\definecolor{currentstroke}{rgb}{0.000000,0.000000,0.000000}%
\pgfsetstrokecolor{currentstroke}%
\pgfsetdash{}{0pt}%
\pgfsys@defobject{currentmarker}{\pgfqpoint{0.000000in}{-0.027778in}}{\pgfqpoint{0.000000in}{0.000000in}}{%
\pgfpathmoveto{\pgfqpoint{0.000000in}{0.000000in}}%
\pgfpathlineto{\pgfqpoint{0.000000in}{-0.027778in}}%
\pgfusepath{stroke,fill}%
}%
\begin{pgfscope}%
\pgfsys@transformshift{2.372469in}{0.549073in}%
\pgfsys@useobject{currentmarker}{}%
\end{pgfscope}%
\end{pgfscope}%
\begin{pgfscope}%
\pgfpathrectangle{\pgfqpoint{0.721913in}{0.549073in}}{\pgfqpoint{2.325000in}{2.310000in}}%
\pgfusepath{clip}%
\pgfsetrectcap%
\pgfsetroundjoin%
\pgfsetlinewidth{0.250937pt}%
\definecolor{currentstroke}{rgb}{0.000000,0.000000,0.000000}%
\pgfsetstrokecolor{currentstroke}%
\pgfsetstrokeopacity{0.200000}%
\pgfsetdash{}{0pt}%
\pgfpathmoveto{\pgfqpoint{2.528435in}{0.549073in}}%
\pgfpathlineto{\pgfqpoint{2.528435in}{2.859073in}}%
\pgfusepath{stroke}%
\end{pgfscope}%
\begin{pgfscope}%
\pgfsetbuttcap%
\pgfsetroundjoin%
\definecolor{currentfill}{rgb}{0.000000,0.000000,0.000000}%
\pgfsetfillcolor{currentfill}%
\pgfsetlinewidth{0.602250pt}%
\definecolor{currentstroke}{rgb}{0.000000,0.000000,0.000000}%
\pgfsetstrokecolor{currentstroke}%
\pgfsetdash{}{0pt}%
\pgfsys@defobject{currentmarker}{\pgfqpoint{0.000000in}{-0.027778in}}{\pgfqpoint{0.000000in}{0.000000in}}{%
\pgfpathmoveto{\pgfqpoint{0.000000in}{0.000000in}}%
\pgfpathlineto{\pgfqpoint{0.000000in}{-0.027778in}}%
\pgfusepath{stroke,fill}%
}%
\begin{pgfscope}%
\pgfsys@transformshift{2.528435in}{0.549073in}%
\pgfsys@useobject{currentmarker}{}%
\end{pgfscope}%
\end{pgfscope}%
\begin{pgfscope}%
\pgfpathrectangle{\pgfqpoint{0.721913in}{0.549073in}}{\pgfqpoint{2.325000in}{2.310000in}}%
\pgfusepath{clip}%
\pgfsetrectcap%
\pgfsetroundjoin%
\pgfsetlinewidth{0.250937pt}%
\definecolor{currentstroke}{rgb}{0.000000,0.000000,0.000000}%
\pgfsetstrokecolor{currentstroke}%
\pgfsetstrokeopacity{0.200000}%
\pgfsetdash{}{0pt}%
\pgfpathmoveto{\pgfqpoint{2.639095in}{0.549073in}}%
\pgfpathlineto{\pgfqpoint{2.639095in}{2.859073in}}%
\pgfusepath{stroke}%
\end{pgfscope}%
\begin{pgfscope}%
\pgfsetbuttcap%
\pgfsetroundjoin%
\definecolor{currentfill}{rgb}{0.000000,0.000000,0.000000}%
\pgfsetfillcolor{currentfill}%
\pgfsetlinewidth{0.602250pt}%
\definecolor{currentstroke}{rgb}{0.000000,0.000000,0.000000}%
\pgfsetstrokecolor{currentstroke}%
\pgfsetdash{}{0pt}%
\pgfsys@defobject{currentmarker}{\pgfqpoint{0.000000in}{-0.027778in}}{\pgfqpoint{0.000000in}{0.000000in}}{%
\pgfpathmoveto{\pgfqpoint{0.000000in}{0.000000in}}%
\pgfpathlineto{\pgfqpoint{0.000000in}{-0.027778in}}%
\pgfusepath{stroke,fill}%
}%
\begin{pgfscope}%
\pgfsys@transformshift{2.639095in}{0.549073in}%
\pgfsys@useobject{currentmarker}{}%
\end{pgfscope}%
\end{pgfscope}%
\begin{pgfscope}%
\pgfpathrectangle{\pgfqpoint{0.721913in}{0.549073in}}{\pgfqpoint{2.325000in}{2.310000in}}%
\pgfusepath{clip}%
\pgfsetrectcap%
\pgfsetroundjoin%
\pgfsetlinewidth{0.250937pt}%
\definecolor{currentstroke}{rgb}{0.000000,0.000000,0.000000}%
\pgfsetstrokecolor{currentstroke}%
\pgfsetstrokeopacity{0.200000}%
\pgfsetdash{}{0pt}%
\pgfpathmoveto{\pgfqpoint{2.724930in}{0.549073in}}%
\pgfpathlineto{\pgfqpoint{2.724930in}{2.859073in}}%
\pgfusepath{stroke}%
\end{pgfscope}%
\begin{pgfscope}%
\pgfsetbuttcap%
\pgfsetroundjoin%
\definecolor{currentfill}{rgb}{0.000000,0.000000,0.000000}%
\pgfsetfillcolor{currentfill}%
\pgfsetlinewidth{0.602250pt}%
\definecolor{currentstroke}{rgb}{0.000000,0.000000,0.000000}%
\pgfsetstrokecolor{currentstroke}%
\pgfsetdash{}{0pt}%
\pgfsys@defobject{currentmarker}{\pgfqpoint{0.000000in}{-0.027778in}}{\pgfqpoint{0.000000in}{0.000000in}}{%
\pgfpathmoveto{\pgfqpoint{0.000000in}{0.000000in}}%
\pgfpathlineto{\pgfqpoint{0.000000in}{-0.027778in}}%
\pgfusepath{stroke,fill}%
}%
\begin{pgfscope}%
\pgfsys@transformshift{2.724930in}{0.549073in}%
\pgfsys@useobject{currentmarker}{}%
\end{pgfscope}%
\end{pgfscope}%
\begin{pgfscope}%
\pgfpathrectangle{\pgfqpoint{0.721913in}{0.549073in}}{\pgfqpoint{2.325000in}{2.310000in}}%
\pgfusepath{clip}%
\pgfsetrectcap%
\pgfsetroundjoin%
\pgfsetlinewidth{0.250937pt}%
\definecolor{currentstroke}{rgb}{0.000000,0.000000,0.000000}%
\pgfsetstrokecolor{currentstroke}%
\pgfsetstrokeopacity{0.200000}%
\pgfsetdash{}{0pt}%
\pgfpathmoveto{\pgfqpoint{2.795062in}{0.549073in}}%
\pgfpathlineto{\pgfqpoint{2.795062in}{2.859073in}}%
\pgfusepath{stroke}%
\end{pgfscope}%
\begin{pgfscope}%
\pgfsetbuttcap%
\pgfsetroundjoin%
\definecolor{currentfill}{rgb}{0.000000,0.000000,0.000000}%
\pgfsetfillcolor{currentfill}%
\pgfsetlinewidth{0.602250pt}%
\definecolor{currentstroke}{rgb}{0.000000,0.000000,0.000000}%
\pgfsetstrokecolor{currentstroke}%
\pgfsetdash{}{0pt}%
\pgfsys@defobject{currentmarker}{\pgfqpoint{0.000000in}{-0.027778in}}{\pgfqpoint{0.000000in}{0.000000in}}{%
\pgfpathmoveto{\pgfqpoint{0.000000in}{0.000000in}}%
\pgfpathlineto{\pgfqpoint{0.000000in}{-0.027778in}}%
\pgfusepath{stroke,fill}%
}%
\begin{pgfscope}%
\pgfsys@transformshift{2.795062in}{0.549073in}%
\pgfsys@useobject{currentmarker}{}%
\end{pgfscope}%
\end{pgfscope}%
\begin{pgfscope}%
\pgfpathrectangle{\pgfqpoint{0.721913in}{0.549073in}}{\pgfqpoint{2.325000in}{2.310000in}}%
\pgfusepath{clip}%
\pgfsetrectcap%
\pgfsetroundjoin%
\pgfsetlinewidth{0.250937pt}%
\definecolor{currentstroke}{rgb}{0.000000,0.000000,0.000000}%
\pgfsetstrokecolor{currentstroke}%
\pgfsetstrokeopacity{0.200000}%
\pgfsetdash{}{0pt}%
\pgfpathmoveto{\pgfqpoint{2.854357in}{0.549073in}}%
\pgfpathlineto{\pgfqpoint{2.854357in}{2.859073in}}%
\pgfusepath{stroke}%
\end{pgfscope}%
\begin{pgfscope}%
\pgfsetbuttcap%
\pgfsetroundjoin%
\definecolor{currentfill}{rgb}{0.000000,0.000000,0.000000}%
\pgfsetfillcolor{currentfill}%
\pgfsetlinewidth{0.602250pt}%
\definecolor{currentstroke}{rgb}{0.000000,0.000000,0.000000}%
\pgfsetstrokecolor{currentstroke}%
\pgfsetdash{}{0pt}%
\pgfsys@defobject{currentmarker}{\pgfqpoint{0.000000in}{-0.027778in}}{\pgfqpoint{0.000000in}{0.000000in}}{%
\pgfpathmoveto{\pgfqpoint{0.000000in}{0.000000in}}%
\pgfpathlineto{\pgfqpoint{0.000000in}{-0.027778in}}%
\pgfusepath{stroke,fill}%
}%
\begin{pgfscope}%
\pgfsys@transformshift{2.854357in}{0.549073in}%
\pgfsys@useobject{currentmarker}{}%
\end{pgfscope}%
\end{pgfscope}%
\begin{pgfscope}%
\pgfpathrectangle{\pgfqpoint{0.721913in}{0.549073in}}{\pgfqpoint{2.325000in}{2.310000in}}%
\pgfusepath{clip}%
\pgfsetrectcap%
\pgfsetroundjoin%
\pgfsetlinewidth{0.250937pt}%
\definecolor{currentstroke}{rgb}{0.000000,0.000000,0.000000}%
\pgfsetstrokecolor{currentstroke}%
\pgfsetstrokeopacity{0.200000}%
\pgfsetdash{}{0pt}%
\pgfpathmoveto{\pgfqpoint{2.905722in}{0.549073in}}%
\pgfpathlineto{\pgfqpoint{2.905722in}{2.859073in}}%
\pgfusepath{stroke}%
\end{pgfscope}%
\begin{pgfscope}%
\pgfsetbuttcap%
\pgfsetroundjoin%
\definecolor{currentfill}{rgb}{0.000000,0.000000,0.000000}%
\pgfsetfillcolor{currentfill}%
\pgfsetlinewidth{0.602250pt}%
\definecolor{currentstroke}{rgb}{0.000000,0.000000,0.000000}%
\pgfsetstrokecolor{currentstroke}%
\pgfsetdash{}{0pt}%
\pgfsys@defobject{currentmarker}{\pgfqpoint{0.000000in}{-0.027778in}}{\pgfqpoint{0.000000in}{0.000000in}}{%
\pgfpathmoveto{\pgfqpoint{0.000000in}{0.000000in}}%
\pgfpathlineto{\pgfqpoint{0.000000in}{-0.027778in}}%
\pgfusepath{stroke,fill}%
}%
\begin{pgfscope}%
\pgfsys@transformshift{2.905722in}{0.549073in}%
\pgfsys@useobject{currentmarker}{}%
\end{pgfscope}%
\end{pgfscope}%
\begin{pgfscope}%
\pgfpathrectangle{\pgfqpoint{0.721913in}{0.549073in}}{\pgfqpoint{2.325000in}{2.310000in}}%
\pgfusepath{clip}%
\pgfsetrectcap%
\pgfsetroundjoin%
\pgfsetlinewidth{0.250937pt}%
\definecolor{currentstroke}{rgb}{0.000000,0.000000,0.000000}%
\pgfsetstrokecolor{currentstroke}%
\pgfsetstrokeopacity{0.200000}%
\pgfsetdash{}{0pt}%
\pgfpathmoveto{\pgfqpoint{2.951028in}{0.549073in}}%
\pgfpathlineto{\pgfqpoint{2.951028in}{2.859073in}}%
\pgfusepath{stroke}%
\end{pgfscope}%
\begin{pgfscope}%
\pgfsetbuttcap%
\pgfsetroundjoin%
\definecolor{currentfill}{rgb}{0.000000,0.000000,0.000000}%
\pgfsetfillcolor{currentfill}%
\pgfsetlinewidth{0.602250pt}%
\definecolor{currentstroke}{rgb}{0.000000,0.000000,0.000000}%
\pgfsetstrokecolor{currentstroke}%
\pgfsetdash{}{0pt}%
\pgfsys@defobject{currentmarker}{\pgfqpoint{0.000000in}{-0.027778in}}{\pgfqpoint{0.000000in}{0.000000in}}{%
\pgfpathmoveto{\pgfqpoint{0.000000in}{0.000000in}}%
\pgfpathlineto{\pgfqpoint{0.000000in}{-0.027778in}}%
\pgfusepath{stroke,fill}%
}%
\begin{pgfscope}%
\pgfsys@transformshift{2.951028in}{0.549073in}%
\pgfsys@useobject{currentmarker}{}%
\end{pgfscope}%
\end{pgfscope}%
\begin{pgfscope}%
\definecolor{textcolor}{rgb}{0.000000,0.000000,0.000000}%
\pgfsetstrokecolor{textcolor}%
\pgfsetfillcolor{textcolor}%
\pgftext[x=1.884413in,y=0.248148in,,top]{\color{textcolor}{\rmfamily\fontsize{12.000000}{14.400000}\selectfont\catcode`\^=\active\def^{\ifmmode\sp\else\^{}\fi}\catcode`\%=\active\def%{\%}smoothing parameter $\sigma$}}%
\end{pgfscope}%
\begin{pgfscope}%
\pgfpathrectangle{\pgfqpoint{0.721913in}{0.549073in}}{\pgfqpoint{2.325000in}{2.310000in}}%
\pgfusepath{clip}%
\pgfsetrectcap%
\pgfsetroundjoin%
\pgfsetlinewidth{0.250937pt}%
\definecolor{currentstroke}{rgb}{0.000000,0.000000,0.000000}%
\pgfsetstrokecolor{currentstroke}%
\pgfsetstrokeopacity{0.200000}%
\pgfsetdash{}{0pt}%
\pgfpathmoveto{\pgfqpoint{0.721913in}{0.809159in}}%
\pgfpathlineto{\pgfqpoint{3.046913in}{0.809159in}}%
\pgfusepath{stroke}%
\end{pgfscope}%
\begin{pgfscope}%
\pgfsetbuttcap%
\pgfsetroundjoin%
\definecolor{currentfill}{rgb}{0.000000,0.000000,0.000000}%
\pgfsetfillcolor{currentfill}%
\pgfsetlinewidth{0.803000pt}%
\definecolor{currentstroke}{rgb}{0.000000,0.000000,0.000000}%
\pgfsetstrokecolor{currentstroke}%
\pgfsetdash{}{0pt}%
\pgfsys@defobject{currentmarker}{\pgfqpoint{-0.048611in}{0.000000in}}{\pgfqpoint{-0.000000in}{0.000000in}}{%
\pgfpathmoveto{\pgfqpoint{-0.000000in}{0.000000in}}%
\pgfpathlineto{\pgfqpoint{-0.048611in}{0.000000in}}%
\pgfusepath{stroke,fill}%
}%
\begin{pgfscope}%
\pgfsys@transformshift{0.721913in}{0.809159in}%
\pgfsys@useobject{currentmarker}{}%
\end{pgfscope}%
\end{pgfscope}%
\begin{pgfscope}%
\definecolor{textcolor}{rgb}{0.000000,0.000000,0.000000}%
\pgfsetstrokecolor{textcolor}%
\pgfsetfillcolor{textcolor}%
\pgftext[x=0.303703in, y=0.751289in, left, base]{\color{textcolor}{\rmfamily\fontsize{12.000000}{14.400000}\selectfont\catcode`\^=\active\def^{\ifmmode\sp\else\^{}\fi}\catcode`\%=\active\def%{\%}$\mathdefault{10^{-2}}$}}%
\end{pgfscope}%
\begin{pgfscope}%
\pgfpathrectangle{\pgfqpoint{0.721913in}{0.549073in}}{\pgfqpoint{2.325000in}{2.310000in}}%
\pgfusepath{clip}%
\pgfsetrectcap%
\pgfsetroundjoin%
\pgfsetlinewidth{0.250937pt}%
\definecolor{currentstroke}{rgb}{0.000000,0.000000,0.000000}%
\pgfsetstrokecolor{currentstroke}%
\pgfsetstrokeopacity{0.200000}%
\pgfsetdash{}{0pt}%
\pgfpathmoveto{\pgfqpoint{0.721913in}{1.679582in}}%
\pgfpathlineto{\pgfqpoint{3.046913in}{1.679582in}}%
\pgfusepath{stroke}%
\end{pgfscope}%
\begin{pgfscope}%
\pgfsetbuttcap%
\pgfsetroundjoin%
\definecolor{currentfill}{rgb}{0.000000,0.000000,0.000000}%
\pgfsetfillcolor{currentfill}%
\pgfsetlinewidth{0.803000pt}%
\definecolor{currentstroke}{rgb}{0.000000,0.000000,0.000000}%
\pgfsetstrokecolor{currentstroke}%
\pgfsetdash{}{0pt}%
\pgfsys@defobject{currentmarker}{\pgfqpoint{-0.048611in}{0.000000in}}{\pgfqpoint{-0.000000in}{0.000000in}}{%
\pgfpathmoveto{\pgfqpoint{-0.000000in}{0.000000in}}%
\pgfpathlineto{\pgfqpoint{-0.048611in}{0.000000in}}%
\pgfusepath{stroke,fill}%
}%
\begin{pgfscope}%
\pgfsys@transformshift{0.721913in}{1.679582in}%
\pgfsys@useobject{currentmarker}{}%
\end{pgfscope}%
\end{pgfscope}%
\begin{pgfscope}%
\definecolor{textcolor}{rgb}{0.000000,0.000000,0.000000}%
\pgfsetstrokecolor{textcolor}%
\pgfsetfillcolor{textcolor}%
\pgftext[x=0.303703in, y=1.621712in, left, base]{\color{textcolor}{\rmfamily\fontsize{12.000000}{14.400000}\selectfont\catcode`\^=\active\def^{\ifmmode\sp\else\^{}\fi}\catcode`\%=\active\def%{\%}$\mathdefault{10^{-1}}$}}%
\end{pgfscope}%
\begin{pgfscope}%
\pgfpathrectangle{\pgfqpoint{0.721913in}{0.549073in}}{\pgfqpoint{2.325000in}{2.310000in}}%
\pgfusepath{clip}%
\pgfsetrectcap%
\pgfsetroundjoin%
\pgfsetlinewidth{0.250937pt}%
\definecolor{currentstroke}{rgb}{0.000000,0.000000,0.000000}%
\pgfsetstrokecolor{currentstroke}%
\pgfsetstrokeopacity{0.200000}%
\pgfsetdash{}{0pt}%
\pgfpathmoveto{\pgfqpoint{0.721913in}{2.550006in}}%
\pgfpathlineto{\pgfqpoint{3.046913in}{2.550006in}}%
\pgfusepath{stroke}%
\end{pgfscope}%
\begin{pgfscope}%
\pgfsetbuttcap%
\pgfsetroundjoin%
\definecolor{currentfill}{rgb}{0.000000,0.000000,0.000000}%
\pgfsetfillcolor{currentfill}%
\pgfsetlinewidth{0.803000pt}%
\definecolor{currentstroke}{rgb}{0.000000,0.000000,0.000000}%
\pgfsetstrokecolor{currentstroke}%
\pgfsetdash{}{0pt}%
\pgfsys@defobject{currentmarker}{\pgfqpoint{-0.048611in}{0.000000in}}{\pgfqpoint{-0.000000in}{0.000000in}}{%
\pgfpathmoveto{\pgfqpoint{-0.000000in}{0.000000in}}%
\pgfpathlineto{\pgfqpoint{-0.048611in}{0.000000in}}%
\pgfusepath{stroke,fill}%
}%
\begin{pgfscope}%
\pgfsys@transformshift{0.721913in}{2.550006in}%
\pgfsys@useobject{currentmarker}{}%
\end{pgfscope}%
\end{pgfscope}%
\begin{pgfscope}%
\definecolor{textcolor}{rgb}{0.000000,0.000000,0.000000}%
\pgfsetstrokecolor{textcolor}%
\pgfsetfillcolor{textcolor}%
\pgftext[x=0.395525in, y=2.492135in, left, base]{\color{textcolor}{\rmfamily\fontsize{12.000000}{14.400000}\selectfont\catcode`\^=\active\def^{\ifmmode\sp\else\^{}\fi}\catcode`\%=\active\def%{\%}$\mathdefault{10^{0}}$}}%
\end{pgfscope}%
\begin{pgfscope}%
\pgfpathrectangle{\pgfqpoint{0.721913in}{0.549073in}}{\pgfqpoint{2.325000in}{2.310000in}}%
\pgfusepath{clip}%
\pgfsetrectcap%
\pgfsetroundjoin%
\pgfsetlinewidth{0.250937pt}%
\definecolor{currentstroke}{rgb}{0.000000,0.000000,0.000000}%
\pgfsetstrokecolor{currentstroke}%
\pgfsetstrokeopacity{0.200000}%
\pgfsetdash{}{0pt}%
\pgfpathmoveto{\pgfqpoint{0.721913in}{0.616056in}}%
\pgfpathlineto{\pgfqpoint{3.046913in}{0.616056in}}%
\pgfusepath{stroke}%
\end{pgfscope}%
\begin{pgfscope}%
\pgfsetbuttcap%
\pgfsetroundjoin%
\definecolor{currentfill}{rgb}{0.000000,0.000000,0.000000}%
\pgfsetfillcolor{currentfill}%
\pgfsetlinewidth{0.602250pt}%
\definecolor{currentstroke}{rgb}{0.000000,0.000000,0.000000}%
\pgfsetstrokecolor{currentstroke}%
\pgfsetdash{}{0pt}%
\pgfsys@defobject{currentmarker}{\pgfqpoint{-0.027778in}{0.000000in}}{\pgfqpoint{-0.000000in}{0.000000in}}{%
\pgfpathmoveto{\pgfqpoint{-0.000000in}{0.000000in}}%
\pgfpathlineto{\pgfqpoint{-0.027778in}{0.000000in}}%
\pgfusepath{stroke,fill}%
}%
\begin{pgfscope}%
\pgfsys@transformshift{0.721913in}{0.616056in}%
\pgfsys@useobject{currentmarker}{}%
\end{pgfscope}%
\end{pgfscope}%
\begin{pgfscope}%
\pgfpathrectangle{\pgfqpoint{0.721913in}{0.549073in}}{\pgfqpoint{2.325000in}{2.310000in}}%
\pgfusepath{clip}%
\pgfsetrectcap%
\pgfsetroundjoin%
\pgfsetlinewidth{0.250937pt}%
\definecolor{currentstroke}{rgb}{0.000000,0.000000,0.000000}%
\pgfsetstrokecolor{currentstroke}%
\pgfsetstrokeopacity{0.200000}%
\pgfsetdash{}{0pt}%
\pgfpathmoveto{\pgfqpoint{0.721913in}{0.674328in}}%
\pgfpathlineto{\pgfqpoint{3.046913in}{0.674328in}}%
\pgfusepath{stroke}%
\end{pgfscope}%
\begin{pgfscope}%
\pgfsetbuttcap%
\pgfsetroundjoin%
\definecolor{currentfill}{rgb}{0.000000,0.000000,0.000000}%
\pgfsetfillcolor{currentfill}%
\pgfsetlinewidth{0.602250pt}%
\definecolor{currentstroke}{rgb}{0.000000,0.000000,0.000000}%
\pgfsetstrokecolor{currentstroke}%
\pgfsetdash{}{0pt}%
\pgfsys@defobject{currentmarker}{\pgfqpoint{-0.027778in}{0.000000in}}{\pgfqpoint{-0.000000in}{0.000000in}}{%
\pgfpathmoveto{\pgfqpoint{-0.000000in}{0.000000in}}%
\pgfpathlineto{\pgfqpoint{-0.027778in}{0.000000in}}%
\pgfusepath{stroke,fill}%
}%
\begin{pgfscope}%
\pgfsys@transformshift{0.721913in}{0.674328in}%
\pgfsys@useobject{currentmarker}{}%
\end{pgfscope}%
\end{pgfscope}%
\begin{pgfscope}%
\pgfpathrectangle{\pgfqpoint{0.721913in}{0.549073in}}{\pgfqpoint{2.325000in}{2.310000in}}%
\pgfusepath{clip}%
\pgfsetrectcap%
\pgfsetroundjoin%
\pgfsetlinewidth{0.250937pt}%
\definecolor{currentstroke}{rgb}{0.000000,0.000000,0.000000}%
\pgfsetstrokecolor{currentstroke}%
\pgfsetstrokeopacity{0.200000}%
\pgfsetdash{}{0pt}%
\pgfpathmoveto{\pgfqpoint{0.721913in}{0.724806in}}%
\pgfpathlineto{\pgfqpoint{3.046913in}{0.724806in}}%
\pgfusepath{stroke}%
\end{pgfscope}%
\begin{pgfscope}%
\pgfsetbuttcap%
\pgfsetroundjoin%
\definecolor{currentfill}{rgb}{0.000000,0.000000,0.000000}%
\pgfsetfillcolor{currentfill}%
\pgfsetlinewidth{0.602250pt}%
\definecolor{currentstroke}{rgb}{0.000000,0.000000,0.000000}%
\pgfsetstrokecolor{currentstroke}%
\pgfsetdash{}{0pt}%
\pgfsys@defobject{currentmarker}{\pgfqpoint{-0.027778in}{0.000000in}}{\pgfqpoint{-0.000000in}{0.000000in}}{%
\pgfpathmoveto{\pgfqpoint{-0.000000in}{0.000000in}}%
\pgfpathlineto{\pgfqpoint{-0.027778in}{0.000000in}}%
\pgfusepath{stroke,fill}%
}%
\begin{pgfscope}%
\pgfsys@transformshift{0.721913in}{0.724806in}%
\pgfsys@useobject{currentmarker}{}%
\end{pgfscope}%
\end{pgfscope}%
\begin{pgfscope}%
\pgfpathrectangle{\pgfqpoint{0.721913in}{0.549073in}}{\pgfqpoint{2.325000in}{2.310000in}}%
\pgfusepath{clip}%
\pgfsetrectcap%
\pgfsetroundjoin%
\pgfsetlinewidth{0.250937pt}%
\definecolor{currentstroke}{rgb}{0.000000,0.000000,0.000000}%
\pgfsetstrokecolor{currentstroke}%
\pgfsetstrokeopacity{0.200000}%
\pgfsetdash{}{0pt}%
\pgfpathmoveto{\pgfqpoint{0.721913in}{0.769330in}}%
\pgfpathlineto{\pgfqpoint{3.046913in}{0.769330in}}%
\pgfusepath{stroke}%
\end{pgfscope}%
\begin{pgfscope}%
\pgfsetbuttcap%
\pgfsetroundjoin%
\definecolor{currentfill}{rgb}{0.000000,0.000000,0.000000}%
\pgfsetfillcolor{currentfill}%
\pgfsetlinewidth{0.602250pt}%
\definecolor{currentstroke}{rgb}{0.000000,0.000000,0.000000}%
\pgfsetstrokecolor{currentstroke}%
\pgfsetdash{}{0pt}%
\pgfsys@defobject{currentmarker}{\pgfqpoint{-0.027778in}{0.000000in}}{\pgfqpoint{-0.000000in}{0.000000in}}{%
\pgfpathmoveto{\pgfqpoint{-0.000000in}{0.000000in}}%
\pgfpathlineto{\pgfqpoint{-0.027778in}{0.000000in}}%
\pgfusepath{stroke,fill}%
}%
\begin{pgfscope}%
\pgfsys@transformshift{0.721913in}{0.769330in}%
\pgfsys@useobject{currentmarker}{}%
\end{pgfscope}%
\end{pgfscope}%
\begin{pgfscope}%
\pgfpathrectangle{\pgfqpoint{0.721913in}{0.549073in}}{\pgfqpoint{2.325000in}{2.310000in}}%
\pgfusepath{clip}%
\pgfsetrectcap%
\pgfsetroundjoin%
\pgfsetlinewidth{0.250937pt}%
\definecolor{currentstroke}{rgb}{0.000000,0.000000,0.000000}%
\pgfsetstrokecolor{currentstroke}%
\pgfsetstrokeopacity{0.200000}%
\pgfsetdash{}{0pt}%
\pgfpathmoveto{\pgfqpoint{0.721913in}{1.071182in}}%
\pgfpathlineto{\pgfqpoint{3.046913in}{1.071182in}}%
\pgfusepath{stroke}%
\end{pgfscope}%
\begin{pgfscope}%
\pgfsetbuttcap%
\pgfsetroundjoin%
\definecolor{currentfill}{rgb}{0.000000,0.000000,0.000000}%
\pgfsetfillcolor{currentfill}%
\pgfsetlinewidth{0.602250pt}%
\definecolor{currentstroke}{rgb}{0.000000,0.000000,0.000000}%
\pgfsetstrokecolor{currentstroke}%
\pgfsetdash{}{0pt}%
\pgfsys@defobject{currentmarker}{\pgfqpoint{-0.027778in}{0.000000in}}{\pgfqpoint{-0.000000in}{0.000000in}}{%
\pgfpathmoveto{\pgfqpoint{-0.000000in}{0.000000in}}%
\pgfpathlineto{\pgfqpoint{-0.027778in}{0.000000in}}%
\pgfusepath{stroke,fill}%
}%
\begin{pgfscope}%
\pgfsys@transformshift{0.721913in}{1.071182in}%
\pgfsys@useobject{currentmarker}{}%
\end{pgfscope}%
\end{pgfscope}%
\begin{pgfscope}%
\pgfpathrectangle{\pgfqpoint{0.721913in}{0.549073in}}{\pgfqpoint{2.325000in}{2.310000in}}%
\pgfusepath{clip}%
\pgfsetrectcap%
\pgfsetroundjoin%
\pgfsetlinewidth{0.250937pt}%
\definecolor{currentstroke}{rgb}{0.000000,0.000000,0.000000}%
\pgfsetstrokecolor{currentstroke}%
\pgfsetstrokeopacity{0.200000}%
\pgfsetdash{}{0pt}%
\pgfpathmoveto{\pgfqpoint{0.721913in}{1.224456in}}%
\pgfpathlineto{\pgfqpoint{3.046913in}{1.224456in}}%
\pgfusepath{stroke}%
\end{pgfscope}%
\begin{pgfscope}%
\pgfsetbuttcap%
\pgfsetroundjoin%
\definecolor{currentfill}{rgb}{0.000000,0.000000,0.000000}%
\pgfsetfillcolor{currentfill}%
\pgfsetlinewidth{0.602250pt}%
\definecolor{currentstroke}{rgb}{0.000000,0.000000,0.000000}%
\pgfsetstrokecolor{currentstroke}%
\pgfsetdash{}{0pt}%
\pgfsys@defobject{currentmarker}{\pgfqpoint{-0.027778in}{0.000000in}}{\pgfqpoint{-0.000000in}{0.000000in}}{%
\pgfpathmoveto{\pgfqpoint{-0.000000in}{0.000000in}}%
\pgfpathlineto{\pgfqpoint{-0.027778in}{0.000000in}}%
\pgfusepath{stroke,fill}%
}%
\begin{pgfscope}%
\pgfsys@transformshift{0.721913in}{1.224456in}%
\pgfsys@useobject{currentmarker}{}%
\end{pgfscope}%
\end{pgfscope}%
\begin{pgfscope}%
\pgfpathrectangle{\pgfqpoint{0.721913in}{0.549073in}}{\pgfqpoint{2.325000in}{2.310000in}}%
\pgfusepath{clip}%
\pgfsetrectcap%
\pgfsetroundjoin%
\pgfsetlinewidth{0.250937pt}%
\definecolor{currentstroke}{rgb}{0.000000,0.000000,0.000000}%
\pgfsetstrokecolor{currentstroke}%
\pgfsetstrokeopacity{0.200000}%
\pgfsetdash{}{0pt}%
\pgfpathmoveto{\pgfqpoint{0.721913in}{1.333206in}}%
\pgfpathlineto{\pgfqpoint{3.046913in}{1.333206in}}%
\pgfusepath{stroke}%
\end{pgfscope}%
\begin{pgfscope}%
\pgfsetbuttcap%
\pgfsetroundjoin%
\definecolor{currentfill}{rgb}{0.000000,0.000000,0.000000}%
\pgfsetfillcolor{currentfill}%
\pgfsetlinewidth{0.602250pt}%
\definecolor{currentstroke}{rgb}{0.000000,0.000000,0.000000}%
\pgfsetstrokecolor{currentstroke}%
\pgfsetdash{}{0pt}%
\pgfsys@defobject{currentmarker}{\pgfqpoint{-0.027778in}{0.000000in}}{\pgfqpoint{-0.000000in}{0.000000in}}{%
\pgfpathmoveto{\pgfqpoint{-0.000000in}{0.000000in}}%
\pgfpathlineto{\pgfqpoint{-0.027778in}{0.000000in}}%
\pgfusepath{stroke,fill}%
}%
\begin{pgfscope}%
\pgfsys@transformshift{0.721913in}{1.333206in}%
\pgfsys@useobject{currentmarker}{}%
\end{pgfscope}%
\end{pgfscope}%
\begin{pgfscope}%
\pgfpathrectangle{\pgfqpoint{0.721913in}{0.549073in}}{\pgfqpoint{2.325000in}{2.310000in}}%
\pgfusepath{clip}%
\pgfsetrectcap%
\pgfsetroundjoin%
\pgfsetlinewidth{0.250937pt}%
\definecolor{currentstroke}{rgb}{0.000000,0.000000,0.000000}%
\pgfsetstrokecolor{currentstroke}%
\pgfsetstrokeopacity{0.200000}%
\pgfsetdash{}{0pt}%
\pgfpathmoveto{\pgfqpoint{0.721913in}{1.417559in}}%
\pgfpathlineto{\pgfqpoint{3.046913in}{1.417559in}}%
\pgfusepath{stroke}%
\end{pgfscope}%
\begin{pgfscope}%
\pgfsetbuttcap%
\pgfsetroundjoin%
\definecolor{currentfill}{rgb}{0.000000,0.000000,0.000000}%
\pgfsetfillcolor{currentfill}%
\pgfsetlinewidth{0.602250pt}%
\definecolor{currentstroke}{rgb}{0.000000,0.000000,0.000000}%
\pgfsetstrokecolor{currentstroke}%
\pgfsetdash{}{0pt}%
\pgfsys@defobject{currentmarker}{\pgfqpoint{-0.027778in}{0.000000in}}{\pgfqpoint{-0.000000in}{0.000000in}}{%
\pgfpathmoveto{\pgfqpoint{-0.000000in}{0.000000in}}%
\pgfpathlineto{\pgfqpoint{-0.027778in}{0.000000in}}%
\pgfusepath{stroke,fill}%
}%
\begin{pgfscope}%
\pgfsys@transformshift{0.721913in}{1.417559in}%
\pgfsys@useobject{currentmarker}{}%
\end{pgfscope}%
\end{pgfscope}%
\begin{pgfscope}%
\pgfpathrectangle{\pgfqpoint{0.721913in}{0.549073in}}{\pgfqpoint{2.325000in}{2.310000in}}%
\pgfusepath{clip}%
\pgfsetrectcap%
\pgfsetroundjoin%
\pgfsetlinewidth{0.250937pt}%
\definecolor{currentstroke}{rgb}{0.000000,0.000000,0.000000}%
\pgfsetstrokecolor{currentstroke}%
\pgfsetstrokeopacity{0.200000}%
\pgfsetdash{}{0pt}%
\pgfpathmoveto{\pgfqpoint{0.721913in}{1.486480in}}%
\pgfpathlineto{\pgfqpoint{3.046913in}{1.486480in}}%
\pgfusepath{stroke}%
\end{pgfscope}%
\begin{pgfscope}%
\pgfsetbuttcap%
\pgfsetroundjoin%
\definecolor{currentfill}{rgb}{0.000000,0.000000,0.000000}%
\pgfsetfillcolor{currentfill}%
\pgfsetlinewidth{0.602250pt}%
\definecolor{currentstroke}{rgb}{0.000000,0.000000,0.000000}%
\pgfsetstrokecolor{currentstroke}%
\pgfsetdash{}{0pt}%
\pgfsys@defobject{currentmarker}{\pgfqpoint{-0.027778in}{0.000000in}}{\pgfqpoint{-0.000000in}{0.000000in}}{%
\pgfpathmoveto{\pgfqpoint{-0.000000in}{0.000000in}}%
\pgfpathlineto{\pgfqpoint{-0.027778in}{0.000000in}}%
\pgfusepath{stroke,fill}%
}%
\begin{pgfscope}%
\pgfsys@transformshift{0.721913in}{1.486480in}%
\pgfsys@useobject{currentmarker}{}%
\end{pgfscope}%
\end{pgfscope}%
\begin{pgfscope}%
\pgfpathrectangle{\pgfqpoint{0.721913in}{0.549073in}}{\pgfqpoint{2.325000in}{2.310000in}}%
\pgfusepath{clip}%
\pgfsetrectcap%
\pgfsetroundjoin%
\pgfsetlinewidth{0.250937pt}%
\definecolor{currentstroke}{rgb}{0.000000,0.000000,0.000000}%
\pgfsetstrokecolor{currentstroke}%
\pgfsetstrokeopacity{0.200000}%
\pgfsetdash{}{0pt}%
\pgfpathmoveto{\pgfqpoint{0.721913in}{1.544752in}}%
\pgfpathlineto{\pgfqpoint{3.046913in}{1.544752in}}%
\pgfusepath{stroke}%
\end{pgfscope}%
\begin{pgfscope}%
\pgfsetbuttcap%
\pgfsetroundjoin%
\definecolor{currentfill}{rgb}{0.000000,0.000000,0.000000}%
\pgfsetfillcolor{currentfill}%
\pgfsetlinewidth{0.602250pt}%
\definecolor{currentstroke}{rgb}{0.000000,0.000000,0.000000}%
\pgfsetstrokecolor{currentstroke}%
\pgfsetdash{}{0pt}%
\pgfsys@defobject{currentmarker}{\pgfqpoint{-0.027778in}{0.000000in}}{\pgfqpoint{-0.000000in}{0.000000in}}{%
\pgfpathmoveto{\pgfqpoint{-0.000000in}{0.000000in}}%
\pgfpathlineto{\pgfqpoint{-0.027778in}{0.000000in}}%
\pgfusepath{stroke,fill}%
}%
\begin{pgfscope}%
\pgfsys@transformshift{0.721913in}{1.544752in}%
\pgfsys@useobject{currentmarker}{}%
\end{pgfscope}%
\end{pgfscope}%
\begin{pgfscope}%
\pgfpathrectangle{\pgfqpoint{0.721913in}{0.549073in}}{\pgfqpoint{2.325000in}{2.310000in}}%
\pgfusepath{clip}%
\pgfsetrectcap%
\pgfsetroundjoin%
\pgfsetlinewidth{0.250937pt}%
\definecolor{currentstroke}{rgb}{0.000000,0.000000,0.000000}%
\pgfsetstrokecolor{currentstroke}%
\pgfsetstrokeopacity{0.200000}%
\pgfsetdash{}{0pt}%
\pgfpathmoveto{\pgfqpoint{0.721913in}{1.595229in}}%
\pgfpathlineto{\pgfqpoint{3.046913in}{1.595229in}}%
\pgfusepath{stroke}%
\end{pgfscope}%
\begin{pgfscope}%
\pgfsetbuttcap%
\pgfsetroundjoin%
\definecolor{currentfill}{rgb}{0.000000,0.000000,0.000000}%
\pgfsetfillcolor{currentfill}%
\pgfsetlinewidth{0.602250pt}%
\definecolor{currentstroke}{rgb}{0.000000,0.000000,0.000000}%
\pgfsetstrokecolor{currentstroke}%
\pgfsetdash{}{0pt}%
\pgfsys@defobject{currentmarker}{\pgfqpoint{-0.027778in}{0.000000in}}{\pgfqpoint{-0.000000in}{0.000000in}}{%
\pgfpathmoveto{\pgfqpoint{-0.000000in}{0.000000in}}%
\pgfpathlineto{\pgfqpoint{-0.027778in}{0.000000in}}%
\pgfusepath{stroke,fill}%
}%
\begin{pgfscope}%
\pgfsys@transformshift{0.721913in}{1.595229in}%
\pgfsys@useobject{currentmarker}{}%
\end{pgfscope}%
\end{pgfscope}%
\begin{pgfscope}%
\pgfpathrectangle{\pgfqpoint{0.721913in}{0.549073in}}{\pgfqpoint{2.325000in}{2.310000in}}%
\pgfusepath{clip}%
\pgfsetrectcap%
\pgfsetroundjoin%
\pgfsetlinewidth{0.250937pt}%
\definecolor{currentstroke}{rgb}{0.000000,0.000000,0.000000}%
\pgfsetstrokecolor{currentstroke}%
\pgfsetstrokeopacity{0.200000}%
\pgfsetdash{}{0pt}%
\pgfpathmoveto{\pgfqpoint{0.721913in}{1.639754in}}%
\pgfpathlineto{\pgfqpoint{3.046913in}{1.639754in}}%
\pgfusepath{stroke}%
\end{pgfscope}%
\begin{pgfscope}%
\pgfsetbuttcap%
\pgfsetroundjoin%
\definecolor{currentfill}{rgb}{0.000000,0.000000,0.000000}%
\pgfsetfillcolor{currentfill}%
\pgfsetlinewidth{0.602250pt}%
\definecolor{currentstroke}{rgb}{0.000000,0.000000,0.000000}%
\pgfsetstrokecolor{currentstroke}%
\pgfsetdash{}{0pt}%
\pgfsys@defobject{currentmarker}{\pgfqpoint{-0.027778in}{0.000000in}}{\pgfqpoint{-0.000000in}{0.000000in}}{%
\pgfpathmoveto{\pgfqpoint{-0.000000in}{0.000000in}}%
\pgfpathlineto{\pgfqpoint{-0.027778in}{0.000000in}}%
\pgfusepath{stroke,fill}%
}%
\begin{pgfscope}%
\pgfsys@transformshift{0.721913in}{1.639754in}%
\pgfsys@useobject{currentmarker}{}%
\end{pgfscope}%
\end{pgfscope}%
\begin{pgfscope}%
\pgfpathrectangle{\pgfqpoint{0.721913in}{0.549073in}}{\pgfqpoint{2.325000in}{2.310000in}}%
\pgfusepath{clip}%
\pgfsetrectcap%
\pgfsetroundjoin%
\pgfsetlinewidth{0.250937pt}%
\definecolor{currentstroke}{rgb}{0.000000,0.000000,0.000000}%
\pgfsetstrokecolor{currentstroke}%
\pgfsetstrokeopacity{0.200000}%
\pgfsetdash{}{0pt}%
\pgfpathmoveto{\pgfqpoint{0.721913in}{1.941606in}}%
\pgfpathlineto{\pgfqpoint{3.046913in}{1.941606in}}%
\pgfusepath{stroke}%
\end{pgfscope}%
\begin{pgfscope}%
\pgfsetbuttcap%
\pgfsetroundjoin%
\definecolor{currentfill}{rgb}{0.000000,0.000000,0.000000}%
\pgfsetfillcolor{currentfill}%
\pgfsetlinewidth{0.602250pt}%
\definecolor{currentstroke}{rgb}{0.000000,0.000000,0.000000}%
\pgfsetstrokecolor{currentstroke}%
\pgfsetdash{}{0pt}%
\pgfsys@defobject{currentmarker}{\pgfqpoint{-0.027778in}{0.000000in}}{\pgfqpoint{-0.000000in}{0.000000in}}{%
\pgfpathmoveto{\pgfqpoint{-0.000000in}{0.000000in}}%
\pgfpathlineto{\pgfqpoint{-0.027778in}{0.000000in}}%
\pgfusepath{stroke,fill}%
}%
\begin{pgfscope}%
\pgfsys@transformshift{0.721913in}{1.941606in}%
\pgfsys@useobject{currentmarker}{}%
\end{pgfscope}%
\end{pgfscope}%
\begin{pgfscope}%
\pgfpathrectangle{\pgfqpoint{0.721913in}{0.549073in}}{\pgfqpoint{2.325000in}{2.310000in}}%
\pgfusepath{clip}%
\pgfsetrectcap%
\pgfsetroundjoin%
\pgfsetlinewidth{0.250937pt}%
\definecolor{currentstroke}{rgb}{0.000000,0.000000,0.000000}%
\pgfsetstrokecolor{currentstroke}%
\pgfsetstrokeopacity{0.200000}%
\pgfsetdash{}{0pt}%
\pgfpathmoveto{\pgfqpoint{0.721913in}{2.094880in}}%
\pgfpathlineto{\pgfqpoint{3.046913in}{2.094880in}}%
\pgfusepath{stroke}%
\end{pgfscope}%
\begin{pgfscope}%
\pgfsetbuttcap%
\pgfsetroundjoin%
\definecolor{currentfill}{rgb}{0.000000,0.000000,0.000000}%
\pgfsetfillcolor{currentfill}%
\pgfsetlinewidth{0.602250pt}%
\definecolor{currentstroke}{rgb}{0.000000,0.000000,0.000000}%
\pgfsetstrokecolor{currentstroke}%
\pgfsetdash{}{0pt}%
\pgfsys@defobject{currentmarker}{\pgfqpoint{-0.027778in}{0.000000in}}{\pgfqpoint{-0.000000in}{0.000000in}}{%
\pgfpathmoveto{\pgfqpoint{-0.000000in}{0.000000in}}%
\pgfpathlineto{\pgfqpoint{-0.027778in}{0.000000in}}%
\pgfusepath{stroke,fill}%
}%
\begin{pgfscope}%
\pgfsys@transformshift{0.721913in}{2.094880in}%
\pgfsys@useobject{currentmarker}{}%
\end{pgfscope}%
\end{pgfscope}%
\begin{pgfscope}%
\pgfpathrectangle{\pgfqpoint{0.721913in}{0.549073in}}{\pgfqpoint{2.325000in}{2.310000in}}%
\pgfusepath{clip}%
\pgfsetrectcap%
\pgfsetroundjoin%
\pgfsetlinewidth{0.250937pt}%
\definecolor{currentstroke}{rgb}{0.000000,0.000000,0.000000}%
\pgfsetstrokecolor{currentstroke}%
\pgfsetstrokeopacity{0.200000}%
\pgfsetdash{}{0pt}%
\pgfpathmoveto{\pgfqpoint{0.721913in}{2.203629in}}%
\pgfpathlineto{\pgfqpoint{3.046913in}{2.203629in}}%
\pgfusepath{stroke}%
\end{pgfscope}%
\begin{pgfscope}%
\pgfsetbuttcap%
\pgfsetroundjoin%
\definecolor{currentfill}{rgb}{0.000000,0.000000,0.000000}%
\pgfsetfillcolor{currentfill}%
\pgfsetlinewidth{0.602250pt}%
\definecolor{currentstroke}{rgb}{0.000000,0.000000,0.000000}%
\pgfsetstrokecolor{currentstroke}%
\pgfsetdash{}{0pt}%
\pgfsys@defobject{currentmarker}{\pgfqpoint{-0.027778in}{0.000000in}}{\pgfqpoint{-0.000000in}{0.000000in}}{%
\pgfpathmoveto{\pgfqpoint{-0.000000in}{0.000000in}}%
\pgfpathlineto{\pgfqpoint{-0.027778in}{0.000000in}}%
\pgfusepath{stroke,fill}%
}%
\begin{pgfscope}%
\pgfsys@transformshift{0.721913in}{2.203629in}%
\pgfsys@useobject{currentmarker}{}%
\end{pgfscope}%
\end{pgfscope}%
\begin{pgfscope}%
\pgfpathrectangle{\pgfqpoint{0.721913in}{0.549073in}}{\pgfqpoint{2.325000in}{2.310000in}}%
\pgfusepath{clip}%
\pgfsetrectcap%
\pgfsetroundjoin%
\pgfsetlinewidth{0.250937pt}%
\definecolor{currentstroke}{rgb}{0.000000,0.000000,0.000000}%
\pgfsetstrokecolor{currentstroke}%
\pgfsetstrokeopacity{0.200000}%
\pgfsetdash{}{0pt}%
\pgfpathmoveto{\pgfqpoint{0.721913in}{2.287982in}}%
\pgfpathlineto{\pgfqpoint{3.046913in}{2.287982in}}%
\pgfusepath{stroke}%
\end{pgfscope}%
\begin{pgfscope}%
\pgfsetbuttcap%
\pgfsetroundjoin%
\definecolor{currentfill}{rgb}{0.000000,0.000000,0.000000}%
\pgfsetfillcolor{currentfill}%
\pgfsetlinewidth{0.602250pt}%
\definecolor{currentstroke}{rgb}{0.000000,0.000000,0.000000}%
\pgfsetstrokecolor{currentstroke}%
\pgfsetdash{}{0pt}%
\pgfsys@defobject{currentmarker}{\pgfqpoint{-0.027778in}{0.000000in}}{\pgfqpoint{-0.000000in}{0.000000in}}{%
\pgfpathmoveto{\pgfqpoint{-0.000000in}{0.000000in}}%
\pgfpathlineto{\pgfqpoint{-0.027778in}{0.000000in}}%
\pgfusepath{stroke,fill}%
}%
\begin{pgfscope}%
\pgfsys@transformshift{0.721913in}{2.287982in}%
\pgfsys@useobject{currentmarker}{}%
\end{pgfscope}%
\end{pgfscope}%
\begin{pgfscope}%
\pgfpathrectangle{\pgfqpoint{0.721913in}{0.549073in}}{\pgfqpoint{2.325000in}{2.310000in}}%
\pgfusepath{clip}%
\pgfsetrectcap%
\pgfsetroundjoin%
\pgfsetlinewidth{0.250937pt}%
\definecolor{currentstroke}{rgb}{0.000000,0.000000,0.000000}%
\pgfsetstrokecolor{currentstroke}%
\pgfsetstrokeopacity{0.200000}%
\pgfsetdash{}{0pt}%
\pgfpathmoveto{\pgfqpoint{0.721913in}{2.356903in}}%
\pgfpathlineto{\pgfqpoint{3.046913in}{2.356903in}}%
\pgfusepath{stroke}%
\end{pgfscope}%
\begin{pgfscope}%
\pgfsetbuttcap%
\pgfsetroundjoin%
\definecolor{currentfill}{rgb}{0.000000,0.000000,0.000000}%
\pgfsetfillcolor{currentfill}%
\pgfsetlinewidth{0.602250pt}%
\definecolor{currentstroke}{rgb}{0.000000,0.000000,0.000000}%
\pgfsetstrokecolor{currentstroke}%
\pgfsetdash{}{0pt}%
\pgfsys@defobject{currentmarker}{\pgfqpoint{-0.027778in}{0.000000in}}{\pgfqpoint{-0.000000in}{0.000000in}}{%
\pgfpathmoveto{\pgfqpoint{-0.000000in}{0.000000in}}%
\pgfpathlineto{\pgfqpoint{-0.027778in}{0.000000in}}%
\pgfusepath{stroke,fill}%
}%
\begin{pgfscope}%
\pgfsys@transformshift{0.721913in}{2.356903in}%
\pgfsys@useobject{currentmarker}{}%
\end{pgfscope}%
\end{pgfscope}%
\begin{pgfscope}%
\pgfpathrectangle{\pgfqpoint{0.721913in}{0.549073in}}{\pgfqpoint{2.325000in}{2.310000in}}%
\pgfusepath{clip}%
\pgfsetrectcap%
\pgfsetroundjoin%
\pgfsetlinewidth{0.250937pt}%
\definecolor{currentstroke}{rgb}{0.000000,0.000000,0.000000}%
\pgfsetstrokecolor{currentstroke}%
\pgfsetstrokeopacity{0.200000}%
\pgfsetdash{}{0pt}%
\pgfpathmoveto{\pgfqpoint{0.721913in}{2.415175in}}%
\pgfpathlineto{\pgfqpoint{3.046913in}{2.415175in}}%
\pgfusepath{stroke}%
\end{pgfscope}%
\begin{pgfscope}%
\pgfsetbuttcap%
\pgfsetroundjoin%
\definecolor{currentfill}{rgb}{0.000000,0.000000,0.000000}%
\pgfsetfillcolor{currentfill}%
\pgfsetlinewidth{0.602250pt}%
\definecolor{currentstroke}{rgb}{0.000000,0.000000,0.000000}%
\pgfsetstrokecolor{currentstroke}%
\pgfsetdash{}{0pt}%
\pgfsys@defobject{currentmarker}{\pgfqpoint{-0.027778in}{0.000000in}}{\pgfqpoint{-0.000000in}{0.000000in}}{%
\pgfpathmoveto{\pgfqpoint{-0.000000in}{0.000000in}}%
\pgfpathlineto{\pgfqpoint{-0.027778in}{0.000000in}}%
\pgfusepath{stroke,fill}%
}%
\begin{pgfscope}%
\pgfsys@transformshift{0.721913in}{2.415175in}%
\pgfsys@useobject{currentmarker}{}%
\end{pgfscope}%
\end{pgfscope}%
\begin{pgfscope}%
\pgfpathrectangle{\pgfqpoint{0.721913in}{0.549073in}}{\pgfqpoint{2.325000in}{2.310000in}}%
\pgfusepath{clip}%
\pgfsetrectcap%
\pgfsetroundjoin%
\pgfsetlinewidth{0.250937pt}%
\definecolor{currentstroke}{rgb}{0.000000,0.000000,0.000000}%
\pgfsetstrokecolor{currentstroke}%
\pgfsetstrokeopacity{0.200000}%
\pgfsetdash{}{0pt}%
\pgfpathmoveto{\pgfqpoint{0.721913in}{2.465653in}}%
\pgfpathlineto{\pgfqpoint{3.046913in}{2.465653in}}%
\pgfusepath{stroke}%
\end{pgfscope}%
\begin{pgfscope}%
\pgfsetbuttcap%
\pgfsetroundjoin%
\definecolor{currentfill}{rgb}{0.000000,0.000000,0.000000}%
\pgfsetfillcolor{currentfill}%
\pgfsetlinewidth{0.602250pt}%
\definecolor{currentstroke}{rgb}{0.000000,0.000000,0.000000}%
\pgfsetstrokecolor{currentstroke}%
\pgfsetdash{}{0pt}%
\pgfsys@defobject{currentmarker}{\pgfqpoint{-0.027778in}{0.000000in}}{\pgfqpoint{-0.000000in}{0.000000in}}{%
\pgfpathmoveto{\pgfqpoint{-0.000000in}{0.000000in}}%
\pgfpathlineto{\pgfqpoint{-0.027778in}{0.000000in}}%
\pgfusepath{stroke,fill}%
}%
\begin{pgfscope}%
\pgfsys@transformshift{0.721913in}{2.465653in}%
\pgfsys@useobject{currentmarker}{}%
\end{pgfscope}%
\end{pgfscope}%
\begin{pgfscope}%
\pgfpathrectangle{\pgfqpoint{0.721913in}{0.549073in}}{\pgfqpoint{2.325000in}{2.310000in}}%
\pgfusepath{clip}%
\pgfsetrectcap%
\pgfsetroundjoin%
\pgfsetlinewidth{0.250937pt}%
\definecolor{currentstroke}{rgb}{0.000000,0.000000,0.000000}%
\pgfsetstrokecolor{currentstroke}%
\pgfsetstrokeopacity{0.200000}%
\pgfsetdash{}{0pt}%
\pgfpathmoveto{\pgfqpoint{0.721913in}{2.510177in}}%
\pgfpathlineto{\pgfqpoint{3.046913in}{2.510177in}}%
\pgfusepath{stroke}%
\end{pgfscope}%
\begin{pgfscope}%
\pgfsetbuttcap%
\pgfsetroundjoin%
\definecolor{currentfill}{rgb}{0.000000,0.000000,0.000000}%
\pgfsetfillcolor{currentfill}%
\pgfsetlinewidth{0.602250pt}%
\definecolor{currentstroke}{rgb}{0.000000,0.000000,0.000000}%
\pgfsetstrokecolor{currentstroke}%
\pgfsetdash{}{0pt}%
\pgfsys@defobject{currentmarker}{\pgfqpoint{-0.027778in}{0.000000in}}{\pgfqpoint{-0.000000in}{0.000000in}}{%
\pgfpathmoveto{\pgfqpoint{-0.000000in}{0.000000in}}%
\pgfpathlineto{\pgfqpoint{-0.027778in}{0.000000in}}%
\pgfusepath{stroke,fill}%
}%
\begin{pgfscope}%
\pgfsys@transformshift{0.721913in}{2.510177in}%
\pgfsys@useobject{currentmarker}{}%
\end{pgfscope}%
\end{pgfscope}%
\begin{pgfscope}%
\pgfpathrectangle{\pgfqpoint{0.721913in}{0.549073in}}{\pgfqpoint{2.325000in}{2.310000in}}%
\pgfusepath{clip}%
\pgfsetrectcap%
\pgfsetroundjoin%
\pgfsetlinewidth{0.250937pt}%
\definecolor{currentstroke}{rgb}{0.000000,0.000000,0.000000}%
\pgfsetstrokecolor{currentstroke}%
\pgfsetstrokeopacity{0.200000}%
\pgfsetdash{}{0pt}%
\pgfpathmoveto{\pgfqpoint{0.721913in}{2.812029in}}%
\pgfpathlineto{\pgfqpoint{3.046913in}{2.812029in}}%
\pgfusepath{stroke}%
\end{pgfscope}%
\begin{pgfscope}%
\pgfsetbuttcap%
\pgfsetroundjoin%
\definecolor{currentfill}{rgb}{0.000000,0.000000,0.000000}%
\pgfsetfillcolor{currentfill}%
\pgfsetlinewidth{0.602250pt}%
\definecolor{currentstroke}{rgb}{0.000000,0.000000,0.000000}%
\pgfsetstrokecolor{currentstroke}%
\pgfsetdash{}{0pt}%
\pgfsys@defobject{currentmarker}{\pgfqpoint{-0.027778in}{0.000000in}}{\pgfqpoint{-0.000000in}{0.000000in}}{%
\pgfpathmoveto{\pgfqpoint{-0.000000in}{0.000000in}}%
\pgfpathlineto{\pgfqpoint{-0.027778in}{0.000000in}}%
\pgfusepath{stroke,fill}%
}%
\begin{pgfscope}%
\pgfsys@transformshift{0.721913in}{2.812029in}%
\pgfsys@useobject{currentmarker}{}%
\end{pgfscope}%
\end{pgfscope}%
\begin{pgfscope}%
\definecolor{textcolor}{rgb}{0.000000,0.000000,0.000000}%
\pgfsetstrokecolor{textcolor}%
\pgfsetfillcolor{textcolor}%
\pgftext[x=0.248147in,y=1.704073in,,bottom,rotate=90.000000]{\color{textcolor}{\rmfamily\fontsize{12.000000}{14.400000}\selectfont\catcode`\^=\active\def^{\ifmmode\sp\else\^{}\fi}\catcode`\%=\active\def%{\%}$L^1$-error}}%
\end{pgfscope}%
\begin{pgfscope}%
\pgfpathrectangle{\pgfqpoint{0.721913in}{0.549073in}}{\pgfqpoint{2.325000in}{2.310000in}}%
\pgfusepath{clip}%
\pgfsetrectcap%
\pgfsetroundjoin%
\pgfsetlinewidth{1.505625pt}%
\definecolor{currentstroke}{rgb}{1.000000,0.690196,0.000000}%
\pgfsetstrokecolor{currentstroke}%
\pgfsetdash{}{0pt}%
\pgfpathmoveto{\pgfqpoint{0.777271in}{2.131690in}}%
\pgfpathlineto{\pgfqpoint{1.146318in}{2.213960in}}%
\pgfpathlineto{\pgfqpoint{1.515366in}{2.155714in}}%
\pgfpathlineto{\pgfqpoint{1.884413in}{2.107198in}}%
\pgfpathlineto{\pgfqpoint{2.253461in}{1.981863in}}%
\pgfpathlineto{\pgfqpoint{2.622509in}{1.836737in}}%
\pgfpathlineto{\pgfqpoint{2.991556in}{1.756044in}}%
\pgfusepath{stroke}%
\end{pgfscope}%
\begin{pgfscope}%
\pgfpathrectangle{\pgfqpoint{0.721913in}{0.549073in}}{\pgfqpoint{2.325000in}{2.310000in}}%
\pgfusepath{clip}%
\pgfsetbuttcap%
\pgfsetmiterjoin%
\definecolor{currentfill}{rgb}{1.000000,0.690196,0.000000}%
\pgfsetfillcolor{currentfill}%
\pgfsetlinewidth{1.003750pt}%
\definecolor{currentstroke}{rgb}{1.000000,0.690196,0.000000}%
\pgfsetstrokecolor{currentstroke}%
\pgfsetdash{}{0pt}%
\pgfsys@defobject{currentmarker}{\pgfqpoint{-0.035355in}{-0.058926in}}{\pgfqpoint{0.035355in}{0.058926in}}{%
\pgfpathmoveto{\pgfqpoint{-0.000000in}{-0.058926in}}%
\pgfpathlineto{\pgfqpoint{0.035355in}{0.000000in}}%
\pgfpathlineto{\pgfqpoint{0.000000in}{0.058926in}}%
\pgfpathlineto{\pgfqpoint{-0.035355in}{0.000000in}}%
\pgfpathlineto{\pgfqpoint{-0.000000in}{-0.058926in}}%
\pgfpathclose%
\pgfusepath{stroke,fill}%
}%
\begin{pgfscope}%
\pgfsys@transformshift{0.777271in}{2.131690in}%
\pgfsys@useobject{currentmarker}{}%
\end{pgfscope}%
\begin{pgfscope}%
\pgfsys@transformshift{1.146318in}{2.213960in}%
\pgfsys@useobject{currentmarker}{}%
\end{pgfscope}%
\begin{pgfscope}%
\pgfsys@transformshift{1.515366in}{2.155714in}%
\pgfsys@useobject{currentmarker}{}%
\end{pgfscope}%
\begin{pgfscope}%
\pgfsys@transformshift{1.884413in}{2.107198in}%
\pgfsys@useobject{currentmarker}{}%
\end{pgfscope}%
\begin{pgfscope}%
\pgfsys@transformshift{2.253461in}{1.981863in}%
\pgfsys@useobject{currentmarker}{}%
\end{pgfscope}%
\begin{pgfscope}%
\pgfsys@transformshift{2.622509in}{1.836737in}%
\pgfsys@useobject{currentmarker}{}%
\end{pgfscope}%
\begin{pgfscope}%
\pgfsys@transformshift{2.991556in}{1.756044in}%
\pgfsys@useobject{currentmarker}{}%
\end{pgfscope}%
\end{pgfscope}%
\begin{pgfscope}%
\pgfpathrectangle{\pgfqpoint{0.721913in}{0.549073in}}{\pgfqpoint{2.325000in}{2.310000in}}%
\pgfusepath{clip}%
\pgfsetrectcap%
\pgfsetroundjoin%
\pgfsetlinewidth{1.505625pt}%
\definecolor{currentstroke}{rgb}{0.996078,0.380392,0.000000}%
\pgfsetstrokecolor{currentstroke}%
\pgfsetdash{}{0pt}%
\pgfpathmoveto{\pgfqpoint{0.777271in}{2.525966in}}%
\pgfpathlineto{\pgfqpoint{1.146318in}{2.360040in}}%
\pgfpathlineto{\pgfqpoint{1.515366in}{2.211526in}}%
\pgfpathlineto{\pgfqpoint{1.884413in}{1.949743in}}%
\pgfpathlineto{\pgfqpoint{2.253461in}{1.447050in}}%
\pgfpathlineto{\pgfqpoint{2.622509in}{0.922175in}}%
\pgfpathlineto{\pgfqpoint{2.991556in}{0.741573in}}%
\pgfusepath{stroke}%
\end{pgfscope}%
\begin{pgfscope}%
\pgfpathrectangle{\pgfqpoint{0.721913in}{0.549073in}}{\pgfqpoint{2.325000in}{2.310000in}}%
\pgfusepath{clip}%
\pgfsetbuttcap%
\pgfsetmiterjoin%
\definecolor{currentfill}{rgb}{0.996078,0.380392,0.000000}%
\pgfsetfillcolor{currentfill}%
\pgfsetlinewidth{1.003750pt}%
\definecolor{currentstroke}{rgb}{0.996078,0.380392,0.000000}%
\pgfsetstrokecolor{currentstroke}%
\pgfsetdash{}{0pt}%
\pgfsys@defobject{currentmarker}{\pgfqpoint{-0.039627in}{-0.033709in}}{\pgfqpoint{0.039627in}{0.041667in}}{%
\pgfpathmoveto{\pgfqpoint{0.000000in}{0.041667in}}%
\pgfpathlineto{\pgfqpoint{-0.039627in}{0.012876in}}%
\pgfpathlineto{\pgfqpoint{-0.024491in}{-0.033709in}}%
\pgfpathlineto{\pgfqpoint{0.024491in}{-0.033709in}}%
\pgfpathlineto{\pgfqpoint{0.039627in}{0.012876in}}%
\pgfpathlineto{\pgfqpoint{0.000000in}{0.041667in}}%
\pgfpathclose%
\pgfusepath{stroke,fill}%
}%
\begin{pgfscope}%
\pgfsys@transformshift{0.777271in}{2.525966in}%
\pgfsys@useobject{currentmarker}{}%
\end{pgfscope}%
\begin{pgfscope}%
\pgfsys@transformshift{1.146318in}{2.360040in}%
\pgfsys@useobject{currentmarker}{}%
\end{pgfscope}%
\begin{pgfscope}%
\pgfsys@transformshift{1.515366in}{2.211526in}%
\pgfsys@useobject{currentmarker}{}%
\end{pgfscope}%
\begin{pgfscope}%
\pgfsys@transformshift{1.884413in}{1.949743in}%
\pgfsys@useobject{currentmarker}{}%
\end{pgfscope}%
\begin{pgfscope}%
\pgfsys@transformshift{2.253461in}{1.447050in}%
\pgfsys@useobject{currentmarker}{}%
\end{pgfscope}%
\begin{pgfscope}%
\pgfsys@transformshift{2.622509in}{0.922175in}%
\pgfsys@useobject{currentmarker}{}%
\end{pgfscope}%
\begin{pgfscope}%
\pgfsys@transformshift{2.991556in}{0.741573in}%
\pgfsys@useobject{currentmarker}{}%
\end{pgfscope}%
\end{pgfscope}%
\begin{pgfscope}%
\pgfpathrectangle{\pgfqpoint{0.721913in}{0.549073in}}{\pgfqpoint{2.325000in}{2.310000in}}%
\pgfusepath{clip}%
\pgfsetrectcap%
\pgfsetroundjoin%
\pgfsetlinewidth{1.505625pt}%
\definecolor{currentstroke}{rgb}{0.470588,0.368627,0.941176}%
\pgfsetstrokecolor{currentstroke}%
\pgfsetdash{}{0pt}%
\pgfpathmoveto{\pgfqpoint{0.777271in}{2.147961in}}%
\pgfpathlineto{\pgfqpoint{1.146318in}{2.171009in}}%
\pgfpathlineto{\pgfqpoint{1.515366in}{2.135456in}}%
\pgfpathlineto{\pgfqpoint{1.884413in}{2.027217in}}%
\pgfpathlineto{\pgfqpoint{2.253461in}{1.895790in}}%
\pgfpathlineto{\pgfqpoint{2.622509in}{1.746123in}}%
\pgfpathlineto{\pgfqpoint{2.991556in}{1.583787in}}%
\pgfusepath{stroke}%
\end{pgfscope}%
\begin{pgfscope}%
\pgfpathrectangle{\pgfqpoint{0.721913in}{0.549073in}}{\pgfqpoint{2.325000in}{2.310000in}}%
\pgfusepath{clip}%
\pgfsetbuttcap%
\pgfsetmiterjoin%
\definecolor{currentfill}{rgb}{0.470588,0.368627,0.941176}%
\pgfsetfillcolor{currentfill}%
\pgfsetlinewidth{1.003750pt}%
\definecolor{currentstroke}{rgb}{0.470588,0.368627,0.941176}%
\pgfsetstrokecolor{currentstroke}%
\pgfsetdash{}{0pt}%
\pgfsys@defobject{currentmarker}{\pgfqpoint{-0.041667in}{-0.041667in}}{\pgfqpoint{0.041667in}{0.041667in}}{%
\pgfpathmoveto{\pgfqpoint{0.000000in}{0.041667in}}%
\pgfpathlineto{\pgfqpoint{-0.041667in}{-0.041667in}}%
\pgfpathlineto{\pgfqpoint{0.041667in}{-0.041667in}}%
\pgfpathlineto{\pgfqpoint{0.000000in}{0.041667in}}%
\pgfpathclose%
\pgfusepath{stroke,fill}%
}%
\begin{pgfscope}%
\pgfsys@transformshift{0.777271in}{2.147961in}%
\pgfsys@useobject{currentmarker}{}%
\end{pgfscope}%
\begin{pgfscope}%
\pgfsys@transformshift{1.146318in}{2.171009in}%
\pgfsys@useobject{currentmarker}{}%
\end{pgfscope}%
\begin{pgfscope}%
\pgfsys@transformshift{1.515366in}{2.135456in}%
\pgfsys@useobject{currentmarker}{}%
\end{pgfscope}%
\begin{pgfscope}%
\pgfsys@transformshift{1.884413in}{2.027217in}%
\pgfsys@useobject{currentmarker}{}%
\end{pgfscope}%
\begin{pgfscope}%
\pgfsys@transformshift{2.253461in}{1.895790in}%
\pgfsys@useobject{currentmarker}{}%
\end{pgfscope}%
\begin{pgfscope}%
\pgfsys@transformshift{2.622509in}{1.746123in}%
\pgfsys@useobject{currentmarker}{}%
\end{pgfscope}%
\begin{pgfscope}%
\pgfsys@transformshift{2.991556in}{1.583787in}%
\pgfsys@useobject{currentmarker}{}%
\end{pgfscope}%
\end{pgfscope}%
\begin{pgfscope}%
\pgfpathrectangle{\pgfqpoint{0.721913in}{0.549073in}}{\pgfqpoint{2.325000in}{2.310000in}}%
\pgfusepath{clip}%
\pgfsetrectcap%
\pgfsetroundjoin%
\pgfsetlinewidth{1.505625pt}%
\definecolor{currentstroke}{rgb}{0.392157,0.560784,1.000000}%
\pgfsetstrokecolor{currentstroke}%
\pgfsetdash{}{0pt}%
\pgfpathmoveto{\pgfqpoint{0.777271in}{2.588212in}}%
\pgfpathlineto{\pgfqpoint{1.146318in}{2.512825in}}%
\pgfpathlineto{\pgfqpoint{1.515366in}{2.340056in}}%
\pgfpathlineto{\pgfqpoint{1.884413in}{2.057287in}}%
\pgfpathlineto{\pgfqpoint{2.253461in}{1.536574in}}%
\pgfpathlineto{\pgfqpoint{2.622509in}{1.056296in}}%
\pgfpathlineto{\pgfqpoint{2.991556in}{0.941805in}}%
\pgfusepath{stroke}%
\end{pgfscope}%
\begin{pgfscope}%
\pgfpathrectangle{\pgfqpoint{0.721913in}{0.549073in}}{\pgfqpoint{2.325000in}{2.310000in}}%
\pgfusepath{clip}%
\pgfsetbuttcap%
\pgfsetroundjoin%
\definecolor{currentfill}{rgb}{0.392157,0.560784,1.000000}%
\pgfsetfillcolor{currentfill}%
\pgfsetlinewidth{1.003750pt}%
\definecolor{currentstroke}{rgb}{0.392157,0.560784,1.000000}%
\pgfsetstrokecolor{currentstroke}%
\pgfsetdash{}{0pt}%
\pgfsys@defobject{currentmarker}{\pgfqpoint{-0.041667in}{-0.041667in}}{\pgfqpoint{0.041667in}{0.041667in}}{%
\pgfpathmoveto{\pgfqpoint{0.000000in}{-0.041667in}}%
\pgfpathcurveto{\pgfqpoint{0.011050in}{-0.041667in}}{\pgfqpoint{0.021649in}{-0.037276in}}{\pgfqpoint{0.029463in}{-0.029463in}}%
\pgfpathcurveto{\pgfqpoint{0.037276in}{-0.021649in}}{\pgfqpoint{0.041667in}{-0.011050in}}{\pgfqpoint{0.041667in}{0.000000in}}%
\pgfpathcurveto{\pgfqpoint{0.041667in}{0.011050in}}{\pgfqpoint{0.037276in}{0.021649in}}{\pgfqpoint{0.029463in}{0.029463in}}%
\pgfpathcurveto{\pgfqpoint{0.021649in}{0.037276in}}{\pgfqpoint{0.011050in}{0.041667in}}{\pgfqpoint{0.000000in}{0.041667in}}%
\pgfpathcurveto{\pgfqpoint{-0.011050in}{0.041667in}}{\pgfqpoint{-0.021649in}{0.037276in}}{\pgfqpoint{-0.029463in}{0.029463in}}%
\pgfpathcurveto{\pgfqpoint{-0.037276in}{0.021649in}}{\pgfqpoint{-0.041667in}{0.011050in}}{\pgfqpoint{-0.041667in}{0.000000in}}%
\pgfpathcurveto{\pgfqpoint{-0.041667in}{-0.011050in}}{\pgfqpoint{-0.037276in}{-0.021649in}}{\pgfqpoint{-0.029463in}{-0.029463in}}%
\pgfpathcurveto{\pgfqpoint{-0.021649in}{-0.037276in}}{\pgfqpoint{-0.011050in}{-0.041667in}}{\pgfqpoint{0.000000in}{-0.041667in}}%
\pgfpathlineto{\pgfqpoint{0.000000in}{-0.041667in}}%
\pgfpathclose%
\pgfusepath{stroke,fill}%
}%
\begin{pgfscope}%
\pgfsys@transformshift{0.777271in}{2.588212in}%
\pgfsys@useobject{currentmarker}{}%
\end{pgfscope}%
\begin{pgfscope}%
\pgfsys@transformshift{1.146318in}{2.512825in}%
\pgfsys@useobject{currentmarker}{}%
\end{pgfscope}%
\begin{pgfscope}%
\pgfsys@transformshift{1.515366in}{2.340056in}%
\pgfsys@useobject{currentmarker}{}%
\end{pgfscope}%
\begin{pgfscope}%
\pgfsys@transformshift{1.884413in}{2.057287in}%
\pgfsys@useobject{currentmarker}{}%
\end{pgfscope}%
\begin{pgfscope}%
\pgfsys@transformshift{2.253461in}{1.536574in}%
\pgfsys@useobject{currentmarker}{}%
\end{pgfscope}%
\begin{pgfscope}%
\pgfsys@transformshift{2.622509in}{1.056296in}%
\pgfsys@useobject{currentmarker}{}%
\end{pgfscope}%
\begin{pgfscope}%
\pgfsys@transformshift{2.991556in}{0.941805in}%
\pgfsys@useobject{currentmarker}{}%
\end{pgfscope}%
\end{pgfscope}%
\begin{pgfscope}%
\pgfpathrectangle{\pgfqpoint{0.721913in}{0.549073in}}{\pgfqpoint{2.325000in}{2.310000in}}%
\pgfusepath{clip}%
\pgfsetrectcap%
\pgfsetroundjoin%
\pgfsetlinewidth{1.505625pt}%
\definecolor{currentstroke}{rgb}{0.862745,0.149020,0.498039}%
\pgfsetstrokecolor{currentstroke}%
\pgfsetdash{}{0pt}%
\pgfpathmoveto{\pgfqpoint{0.777271in}{2.666573in}}%
\pgfpathlineto{\pgfqpoint{1.146318in}{1.825465in}}%
\pgfpathlineto{\pgfqpoint{1.515366in}{0.747722in}}%
\pgfpathlineto{\pgfqpoint{1.884413in}{1.010381in}}%
\pgfpathlineto{\pgfqpoint{2.253461in}{1.113530in}}%
\pgfpathlineto{\pgfqpoint{2.622509in}{1.065040in}}%
\pgfpathlineto{\pgfqpoint{2.991556in}{0.929672in}}%
\pgfusepath{stroke}%
\end{pgfscope}%
\begin{pgfscope}%
\pgfpathrectangle{\pgfqpoint{0.721913in}{0.549073in}}{\pgfqpoint{2.325000in}{2.310000in}}%
\pgfusepath{clip}%
\pgfsetbuttcap%
\pgfsetmiterjoin%
\definecolor{currentfill}{rgb}{0.862745,0.149020,0.498039}%
\pgfsetfillcolor{currentfill}%
\pgfsetlinewidth{1.003750pt}%
\definecolor{currentstroke}{rgb}{0.862745,0.149020,0.498039}%
\pgfsetstrokecolor{currentstroke}%
\pgfsetdash{}{0pt}%
\pgfsys@defobject{currentmarker}{\pgfqpoint{-0.041667in}{-0.041667in}}{\pgfqpoint{0.041667in}{0.041667in}}{%
\pgfpathmoveto{\pgfqpoint{-0.041667in}{-0.041667in}}%
\pgfpathlineto{\pgfqpoint{0.041667in}{-0.041667in}}%
\pgfpathlineto{\pgfqpoint{0.041667in}{0.041667in}}%
\pgfpathlineto{\pgfqpoint{-0.041667in}{0.041667in}}%
\pgfpathlineto{\pgfqpoint{-0.041667in}{-0.041667in}}%
\pgfpathclose%
\pgfusepath{stroke,fill}%
}%
\begin{pgfscope}%
\pgfsys@transformshift{0.777271in}{2.666573in}%
\pgfsys@useobject{currentmarker}{}%
\end{pgfscope}%
\begin{pgfscope}%
\pgfsys@transformshift{1.146318in}{1.825465in}%
\pgfsys@useobject{currentmarker}{}%
\end{pgfscope}%
\begin{pgfscope}%
\pgfsys@transformshift{1.515366in}{0.747722in}%
\pgfsys@useobject{currentmarker}{}%
\end{pgfscope}%
\begin{pgfscope}%
\pgfsys@transformshift{1.884413in}{1.010381in}%
\pgfsys@useobject{currentmarker}{}%
\end{pgfscope}%
\begin{pgfscope}%
\pgfsys@transformshift{2.253461in}{1.113530in}%
\pgfsys@useobject{currentmarker}{}%
\end{pgfscope}%
\begin{pgfscope}%
\pgfsys@transformshift{2.622509in}{1.065040in}%
\pgfsys@useobject{currentmarker}{}%
\end{pgfscope}%
\begin{pgfscope}%
\pgfsys@transformshift{2.991556in}{0.929672in}%
\pgfsys@useobject{currentmarker}{}%
\end{pgfscope}%
\end{pgfscope}%
\begin{pgfscope}%
\pgfsetrectcap%
\pgfsetmiterjoin%
\pgfsetlinewidth{0.803000pt}%
\definecolor{currentstroke}{rgb}{0.000000,0.000000,0.000000}%
\pgfsetstrokecolor{currentstroke}%
\pgfsetdash{}{0pt}%
\pgfpathmoveto{\pgfqpoint{0.721913in}{0.549073in}}%
\pgfpathlineto{\pgfqpoint{0.721913in}{2.859073in}}%
\pgfusepath{stroke}%
\end{pgfscope}%
\begin{pgfscope}%
\pgfsetrectcap%
\pgfsetmiterjoin%
\pgfsetlinewidth{0.803000pt}%
\definecolor{currentstroke}{rgb}{0.000000,0.000000,0.000000}%
\pgfsetstrokecolor{currentstroke}%
\pgfsetdash{}{0pt}%
\pgfpathmoveto{\pgfqpoint{3.046913in}{0.549073in}}%
\pgfpathlineto{\pgfqpoint{3.046913in}{2.859073in}}%
\pgfusepath{stroke}%
\end{pgfscope}%
\begin{pgfscope}%
\pgfsetrectcap%
\pgfsetmiterjoin%
\pgfsetlinewidth{0.803000pt}%
\definecolor{currentstroke}{rgb}{0.000000,0.000000,0.000000}%
\pgfsetstrokecolor{currentstroke}%
\pgfsetdash{}{0pt}%
\pgfpathmoveto{\pgfqpoint{0.721913in}{0.549073in}}%
\pgfpathlineto{\pgfqpoint{3.046913in}{0.549073in}}%
\pgfusepath{stroke}%
\end{pgfscope}%
\begin{pgfscope}%
\pgfsetrectcap%
\pgfsetmiterjoin%
\pgfsetlinewidth{0.803000pt}%
\definecolor{currentstroke}{rgb}{0.000000,0.000000,0.000000}%
\pgfsetstrokecolor{currentstroke}%
\pgfsetdash{}{0pt}%
\pgfpathmoveto{\pgfqpoint{0.721913in}{2.859073in}}%
\pgfpathlineto{\pgfqpoint{3.046913in}{2.859073in}}%
\pgfusepath{stroke}%
\end{pgfscope}%
\begin{pgfscope}%
\pgfsetbuttcap%
\pgfsetmiterjoin%
\definecolor{currentfill}{rgb}{1.000000,1.000000,1.000000}%
\pgfsetfillcolor{currentfill}%
\pgfsetfillopacity{0.800000}%
\pgfsetlinewidth{1.003750pt}%
\definecolor{currentstroke}{rgb}{0.800000,0.800000,0.800000}%
\pgfsetstrokecolor{currentstroke}%
\pgfsetstrokeopacity{0.800000}%
\pgfsetdash{}{0pt}%
\pgfpathmoveto{\pgfqpoint{0.805247in}{0.632406in}}%
\pgfpathlineto{\pgfqpoint{2.021463in}{0.632406in}}%
\pgfpathlineto{\pgfqpoint{2.021463in}{1.914813in}}%
\pgfpathlineto{\pgfqpoint{0.805247in}{1.914813in}}%
\pgfpathlineto{\pgfqpoint{0.805247in}{0.632406in}}%
\pgfpathclose%
\pgfusepath{stroke,fill}%
\end{pgfscope}%
\begin{pgfscope}%
\pgfsetrectcap%
\pgfsetroundjoin%
\pgfsetlinewidth{1.505625pt}%
\definecolor{currentstroke}{rgb}{1.000000,0.690196,0.000000}%
\pgfsetstrokecolor{currentstroke}%
\pgfsetdash{}{0pt}%
\pgfpathmoveto{\pgfqpoint{0.871913in}{1.781480in}}%
\pgfpathlineto{\pgfqpoint{1.038580in}{1.781480in}}%
\pgfpathlineto{\pgfqpoint{1.205247in}{1.781480in}}%
\pgfusepath{stroke}%
\end{pgfscope}%
\begin{pgfscope}%
\pgfsetbuttcap%
\pgfsetmiterjoin%
\definecolor{currentfill}{rgb}{1.000000,0.690196,0.000000}%
\pgfsetfillcolor{currentfill}%
\pgfsetlinewidth{1.003750pt}%
\definecolor{currentstroke}{rgb}{1.000000,0.690196,0.000000}%
\pgfsetstrokecolor{currentstroke}%
\pgfsetdash{}{0pt}%
\pgfsys@defobject{currentmarker}{\pgfqpoint{-0.026517in}{-0.044194in}}{\pgfqpoint{0.026517in}{0.044194in}}{%
\pgfpathmoveto{\pgfqpoint{-0.000000in}{-0.044194in}}%
\pgfpathlineto{\pgfqpoint{0.026517in}{0.000000in}}%
\pgfpathlineto{\pgfqpoint{0.000000in}{0.044194in}}%
\pgfpathlineto{\pgfqpoint{-0.026517in}{0.000000in}}%
\pgfpathlineto{\pgfqpoint{-0.000000in}{-0.044194in}}%
\pgfpathclose%
\pgfusepath{stroke,fill}%
}%
\begin{pgfscope}%
\pgfsys@transformshift{1.038580in}{1.781480in}%
\pgfsys@useobject{currentmarker}{}%
\end{pgfscope}%
\end{pgfscope}%
\begin{pgfscope}%
\definecolor{textcolor}{rgb}{0.000000,0.000000,0.000000}%
\pgfsetstrokecolor{textcolor}%
\pgfsetfillcolor{textcolor}%
\pgftext[x=1.338580in,y=1.723147in,left,base]{\color{textcolor}{\rmfamily\fontsize{12.000000}{14.400000}\selectfont\catcode`\^=\active\def^{\ifmmode\sp\else\^{}\fi}\catcode`\%=\active\def%{\%}KA (I)}}%
\end{pgfscope}%
\begin{pgfscope}%
\pgfsetrectcap%
\pgfsetroundjoin%
\pgfsetlinewidth{1.505625pt}%
\definecolor{currentstroke}{rgb}{0.996078,0.380392,0.000000}%
\pgfsetstrokecolor{currentstroke}%
\pgfsetdash{}{0pt}%
\pgfpathmoveto{\pgfqpoint{0.871913in}{1.531480in}}%
\pgfpathlineto{\pgfqpoint{1.038580in}{1.531480in}}%
\pgfpathlineto{\pgfqpoint{1.205247in}{1.531480in}}%
\pgfusepath{stroke}%
\end{pgfscope}%
\begin{pgfscope}%
\pgfsetbuttcap%
\pgfsetmiterjoin%
\definecolor{currentfill}{rgb}{0.996078,0.380392,0.000000}%
\pgfsetfillcolor{currentfill}%
\pgfsetlinewidth{1.003750pt}%
\definecolor{currentstroke}{rgb}{0.996078,0.380392,0.000000}%
\pgfsetstrokecolor{currentstroke}%
\pgfsetdash{}{0pt}%
\pgfsys@defobject{currentmarker}{\pgfqpoint{-0.029721in}{-0.025282in}}{\pgfqpoint{0.029721in}{0.031250in}}{%
\pgfpathmoveto{\pgfqpoint{0.000000in}{0.031250in}}%
\pgfpathlineto{\pgfqpoint{-0.029721in}{0.009657in}}%
\pgfpathlineto{\pgfqpoint{-0.018368in}{-0.025282in}}%
\pgfpathlineto{\pgfqpoint{0.018368in}{-0.025282in}}%
\pgfpathlineto{\pgfqpoint{0.029721in}{0.009657in}}%
\pgfpathlineto{\pgfqpoint{0.000000in}{0.031250in}}%
\pgfpathclose%
\pgfusepath{stroke,fill}%
}%
\begin{pgfscope}%
\pgfsys@transformshift{1.038580in}{1.531480in}%
\pgfsys@useobject{currentmarker}{}%
\end{pgfscope}%
\end{pgfscope}%
\begin{pgfscope}%
\definecolor{textcolor}{rgb}{0.000000,0.000000,0.000000}%
\pgfsetstrokecolor{textcolor}%
\pgfsetfillcolor{textcolor}%
\pgftext[x=1.338580in,y=1.473147in,left,base]{\color{textcolor}{\rmfamily\fontsize{12.000000}{14.400000}\selectfont\catcode`\^=\active\def^{\ifmmode\sp\else\^{}\fi}\catcode`\%=\active\def%{\%}KA (II)}}%
\end{pgfscope}%
\begin{pgfscope}%
\pgfsetrectcap%
\pgfsetroundjoin%
\pgfsetlinewidth{1.505625pt}%
\definecolor{currentstroke}{rgb}{0.470588,0.368627,0.941176}%
\pgfsetstrokecolor{currentstroke}%
\pgfsetdash{}{0pt}%
\pgfpathmoveto{\pgfqpoint{0.871913in}{1.281480in}}%
\pgfpathlineto{\pgfqpoint{1.038580in}{1.281480in}}%
\pgfpathlineto{\pgfqpoint{1.205247in}{1.281480in}}%
\pgfusepath{stroke}%
\end{pgfscope}%
\begin{pgfscope}%
\pgfsetbuttcap%
\pgfsetmiterjoin%
\definecolor{currentfill}{rgb}{0.470588,0.368627,0.941176}%
\pgfsetfillcolor{currentfill}%
\pgfsetlinewidth{1.003750pt}%
\definecolor{currentstroke}{rgb}{0.470588,0.368627,0.941176}%
\pgfsetstrokecolor{currentstroke}%
\pgfsetdash{}{0pt}%
\pgfsys@defobject{currentmarker}{\pgfqpoint{-0.031250in}{-0.031250in}}{\pgfqpoint{0.031250in}{0.031250in}}{%
\pgfpathmoveto{\pgfqpoint{0.000000in}{0.031250in}}%
\pgfpathlineto{\pgfqpoint{-0.031250in}{-0.031250in}}%
\pgfpathlineto{\pgfqpoint{0.031250in}{-0.031250in}}%
\pgfpathlineto{\pgfqpoint{0.000000in}{0.031250in}}%
\pgfpathclose%
\pgfusepath{stroke,fill}%
}%
\begin{pgfscope}%
\pgfsys@transformshift{1.038580in}{1.281480in}%
\pgfsys@useobject{currentmarker}{}%
\end{pgfscope}%
\end{pgfscope}%
\begin{pgfscope}%
\definecolor{textcolor}{rgb}{0.000000,0.000000,0.000000}%
\pgfsetstrokecolor{textcolor}%
\pgfsetfillcolor{textcolor}%
\pgftext[x=1.338580in,y=1.223147in,left,base]{\color{textcolor}{\rmfamily\fontsize{12.000000}{14.400000}\selectfont\catcode`\^=\active\def^{\ifmmode\sp\else\^{}\fi}\catcode`\%=\active\def%{\%}KA (III)}}%
\end{pgfscope}%
\begin{pgfscope}%
\pgfsetrectcap%
\pgfsetroundjoin%
\pgfsetlinewidth{1.505625pt}%
\definecolor{currentstroke}{rgb}{0.392157,0.560784,1.000000}%
\pgfsetstrokecolor{currentstroke}%
\pgfsetdash{}{0pt}%
\pgfpathmoveto{\pgfqpoint{0.871913in}{1.031480in}}%
\pgfpathlineto{\pgfqpoint{1.038580in}{1.031480in}}%
\pgfpathlineto{\pgfqpoint{1.205247in}{1.031480in}}%
\pgfusepath{stroke}%
\end{pgfscope}%
\begin{pgfscope}%
\pgfsetbuttcap%
\pgfsetroundjoin%
\definecolor{currentfill}{rgb}{0.392157,0.560784,1.000000}%
\pgfsetfillcolor{currentfill}%
\pgfsetlinewidth{1.003750pt}%
\definecolor{currentstroke}{rgb}{0.392157,0.560784,1.000000}%
\pgfsetstrokecolor{currentstroke}%
\pgfsetdash{}{0pt}%
\pgfsys@defobject{currentmarker}{\pgfqpoint{-0.031250in}{-0.031250in}}{\pgfqpoint{0.031250in}{0.031250in}}{%
\pgfpathmoveto{\pgfqpoint{0.000000in}{-0.031250in}}%
\pgfpathcurveto{\pgfqpoint{0.008288in}{-0.031250in}}{\pgfqpoint{0.016237in}{-0.027957in}}{\pgfqpoint{0.022097in}{-0.022097in}}%
\pgfpathcurveto{\pgfqpoint{0.027957in}{-0.016237in}}{\pgfqpoint{0.031250in}{-0.008288in}}{\pgfqpoint{0.031250in}{0.000000in}}%
\pgfpathcurveto{\pgfqpoint{0.031250in}{0.008288in}}{\pgfqpoint{0.027957in}{0.016237in}}{\pgfqpoint{0.022097in}{0.022097in}}%
\pgfpathcurveto{\pgfqpoint{0.016237in}{0.027957in}}{\pgfqpoint{0.008288in}{0.031250in}}{\pgfqpoint{0.000000in}{0.031250in}}%
\pgfpathcurveto{\pgfqpoint{-0.008288in}{0.031250in}}{\pgfqpoint{-0.016237in}{0.027957in}}{\pgfqpoint{-0.022097in}{0.022097in}}%
\pgfpathcurveto{\pgfqpoint{-0.027957in}{0.016237in}}{\pgfqpoint{-0.031250in}{0.008288in}}{\pgfqpoint{-0.031250in}{0.000000in}}%
\pgfpathcurveto{\pgfqpoint{-0.031250in}{-0.008288in}}{\pgfqpoint{-0.027957in}{-0.016237in}}{\pgfqpoint{-0.022097in}{-0.022097in}}%
\pgfpathcurveto{\pgfqpoint{-0.016237in}{-0.027957in}}{\pgfqpoint{-0.008288in}{-0.031250in}}{\pgfqpoint{0.000000in}{-0.031250in}}%
\pgfpathlineto{\pgfqpoint{0.000000in}{-0.031250in}}%
\pgfpathclose%
\pgfusepath{stroke,fill}%
}%
\begin{pgfscope}%
\pgfsys@transformshift{1.038580in}{1.031480in}%
\pgfsys@useobject{currentmarker}{}%
\end{pgfscope}%
\end{pgfscope}%
\begin{pgfscope}%
\definecolor{textcolor}{rgb}{0.000000,0.000000,0.000000}%
\pgfsetstrokecolor{textcolor}%
\pgfsetfillcolor{textcolor}%
\pgftext[x=1.338580in,y=0.973147in,left,base]{\color{textcolor}{\rmfamily\fontsize{12.000000}{14.400000}\selectfont\catcode`\^=\active\def^{\ifmmode\sp\else\^{}\fi}\catcode`\%=\active\def%{\%}KA (IV)}}%
\end{pgfscope}%
\begin{pgfscope}%
\pgfsetrectcap%
\pgfsetroundjoin%
\pgfsetlinewidth{1.505625pt}%
\definecolor{currentstroke}{rgb}{0.862745,0.149020,0.498039}%
\pgfsetstrokecolor{currentstroke}%
\pgfsetdash{}{0pt}%
\pgfpathmoveto{\pgfqpoint{0.871913in}{0.789813in}}%
\pgfpathlineto{\pgfqpoint{1.038580in}{0.789813in}}%
\pgfpathlineto{\pgfqpoint{1.205247in}{0.789813in}}%
\pgfusepath{stroke}%
\end{pgfscope}%
\begin{pgfscope}%
\pgfsetbuttcap%
\pgfsetmiterjoin%
\definecolor{currentfill}{rgb}{0.862745,0.149020,0.498039}%
\pgfsetfillcolor{currentfill}%
\pgfsetlinewidth{1.003750pt}%
\definecolor{currentstroke}{rgb}{0.862745,0.149020,0.498039}%
\pgfsetstrokecolor{currentstroke}%
\pgfsetdash{}{0pt}%
\pgfsys@defobject{currentmarker}{\pgfqpoint{-0.031250in}{-0.031250in}}{\pgfqpoint{0.031250in}{0.031250in}}{%
\pgfpathmoveto{\pgfqpoint{-0.031250in}{-0.031250in}}%
\pgfpathlineto{\pgfqpoint{0.031250in}{-0.031250in}}%
\pgfpathlineto{\pgfqpoint{0.031250in}{0.031250in}}%
\pgfpathlineto{\pgfqpoint{-0.031250in}{0.031250in}}%
\pgfpathlineto{\pgfqpoint{-0.031250in}{-0.031250in}}%
\pgfpathclose%
\pgfusepath{stroke,fill}%
}%
\begin{pgfscope}%
\pgfsys@transformshift{1.038580in}{0.789813in}%
\pgfsys@useobject{currentmarker}{}%
\end{pgfscope}%
\end{pgfscope}%
\begin{pgfscope}%
\definecolor{textcolor}{rgb}{0.000000,0.000000,0.000000}%
\pgfsetstrokecolor{textcolor}%
\pgfsetfillcolor{textcolor}%
\pgftext[x=1.338580in,y=0.731480in,left,base]{\color{textcolor}{\rmfamily\fontsize{12.000000}{14.400000}\selectfont\catcode`\^=\active\def^{\ifmmode\sp\else\^{}\fi}\catcode`\%=\active\def%{\%}CN++}}%
\end{pgfscope}%
\end{pgfpicture}%
\makeatother%
\endgroup%

    \end{minipage}\hfill%
    \begin{minipage}[c]{.475\linewidth}
        \vspace{-35pt}
        \centering
\renewcommand{\arraystretch}{1.2}
\begin{tabular}{@{}lccccc@{}}
\toprule
 & $n_{\mtx{\Omega}}$ & $n_{\mtx{\Psi}}$ & $q$ & $r$ & time (s)\\
\midrule
KA (I) & $10$ & $60$ & $20$ & $10$ & $16.80$ \\
KA (II) & $20$ & $30$ & $20$ & $10$ & $6.01$ \\
KA (III) & $10$ & $30$ & $40$ & $10$ & $13.61$ \\
KA (IV) & $10$ & $30$ & $20$ & $20$ & $4.91$ \\
\bottomrule
\end{tabular}

        \newline
        \vspace{15pt}
        \newline
        \centering
\renewcommand{\arraystretch}{1.2}
\begin{tabular}{@{}lcccc@{}}
\toprule
 & $n_{\mtx{\Omega}}$ & $n_{\mtx{\Psi}}$ & $m$ & time (s)\\
\midrule
CN++ & $20$ & $20$ & $2000$ & $3.76$ \\
\bottomrule
\end{tabular}

    \end{minipage}
    \caption{For the example from \refsec{subsec:hamiltonian} we compare the Chebyshev-Nyström++ (CN++) method with the Krylov-Aware (KA) stochastic trace estimator in multiple configurations. To this extent, we compute the $L^1$-error of the methods for various levels of smoothing $\sigma$ (left). The parameters of the algorithms and run-time averaged over all values of $\sigma$ can be found in the tables (right).}
    \label{fig:krylov-aware-density}
\end{figure}

\subsection{Spectral density of Hessian matrix of neural network}
\label{subsec:hessian}

When optimizing a neural network over some data, it is of great interest to determine whether a minimum has been found and if this minimum is robust, i.e. a small perturbation of the parameters of the neural network will not lead to a significant decrease in the fit of the neural network. Both properties are reflected in the eigenvalues of the Hessian matrix $\mtx{A}$ with respect to some loss function: if all eigenvalues are non-negative, then the loss attains a local minimum, and if additionally all of them are small, the minimum is robust.

Since neural networks are usually parametrized by thousands of parameters, assembling the Hessian matrix explicitly and then computing its eigenvalues is out of question. Luckily, there exists a method which can exactly compute matrix-vector products $\mtx{A} \vct{x}$ for any vector $\vct{x}$ which scales proportionally with the number of parameters \cite{pearlmutter-1994-fast-exact}.

To demonstrate that our algorithm can effectively be applied in the setting of neural network optimization, we approximate the spectral density of a small fully connected convolutional neural network with 6782 parameters. We train this network on the handwritten digit classification task given by the MNIST dataset\footnote{Handwritten digit classification; taken from \url{http://yann.lecun.com/exdb/mnist}.} in PyTorch 2.4.1 in a standard way. We use the bound from \cite[Theorem 1]{zhou-2011-bounding-spectrum} to estimate a bound on the spectrum.

We plot the approximation of the spectral density of the Hessian matrix of the untrained neural network, and at different stages of training in \reffig{fig:hessian-density}, as well as the corresponding mean squared error loss. It can well be observed that the eigenvalues creep towards zero as the training proceeds. Furthermore, despite the loss steadily decreasing, the presence of relatively large eigenvalues in some epochs indicates a sharp minimum of the network, hence, unfavorable generalization properties.

\begin{figure}
    \begin{minipage}[c]{.49\linewidth}
        \centering
        %% Creator: Matplotlib, PGF backend
%%
%% To include the figure in your LaTeX document, write
%%   \input{<filename>.pgf}
%%
%% Make sure the required packages are loaded in your preamble
%%   \usepackage{pgf}
%%
%% Also ensure that all the required font packages are loaded; for instance,
%% the lmodern package is sometimes necessary when using math font.
%%   \usepackage{lmodern}
%%
%% Figures using additional raster images can only be included by \input if
%% they are in the same directory as the main LaTeX file. For loading figures
%% from other directories you can use the `import` package
%%   \usepackage{import}
%%
%% and then include the figures with
%%   \import{<path to file>}{<filename>.pgf}
%%
%% Matplotlib used the following preamble
%%   \def\mathdefault#1{#1}
%%   \everymath=\expandafter{\the\everymath\displaystyle}
%%   
%%   \ifdefined\pdftexversion\else  % non-pdftex case.
%%     \usepackage{fontspec}
%%     \setmainfont{DejaVuSerif.ttf}[Path=\detokenize{/home/matti/Documents/.venv/lib/python3.12/site-packages/matplotlib/mpl-data/fonts/ttf/}]
%%     \setsansfont{DejaVuSans.ttf}[Path=\detokenize{/home/matti/Documents/.venv/lib/python3.12/site-packages/matplotlib/mpl-data/fonts/ttf/}]
%%     \setmonofont{DejaVuSansMono.ttf}[Path=\detokenize{/home/matti/Documents/.venv/lib/python3.12/site-packages/matplotlib/mpl-data/fonts/ttf/}]
%%   \fi
%%   \makeatletter\@ifpackageloaded{underscore}{}{\usepackage[strings]{underscore}}\makeatother
%%
\begingroup%
\makeatletter%
\begin{pgfpicture}%
\pgfpathrectangle{\pgfpointorigin}{\pgfqpoint{5.327815in}{2.959073in}}%
\pgfusepath{use as bounding box, clip}%
\begin{pgfscope}%
\pgfsetbuttcap%
\pgfsetmiterjoin%
\definecolor{currentfill}{rgb}{1.000000,1.000000,1.000000}%
\pgfsetfillcolor{currentfill}%
\pgfsetlinewidth{0.000000pt}%
\definecolor{currentstroke}{rgb}{1.000000,1.000000,1.000000}%
\pgfsetstrokecolor{currentstroke}%
\pgfsetdash{}{0pt}%
\pgfpathmoveto{\pgfqpoint{0.000000in}{-0.000000in}}%
\pgfpathlineto{\pgfqpoint{5.327815in}{-0.000000in}}%
\pgfpathlineto{\pgfqpoint{5.327815in}{2.959073in}}%
\pgfpathlineto{\pgfqpoint{0.000000in}{2.959073in}}%
\pgfpathlineto{\pgfqpoint{0.000000in}{-0.000000in}}%
\pgfpathclose%
\pgfusepath{fill}%
\end{pgfscope}%
\begin{pgfscope}%
\pgfsetbuttcap%
\pgfsetmiterjoin%
\definecolor{currentfill}{rgb}{1.000000,1.000000,1.000000}%
\pgfsetfillcolor{currentfill}%
\pgfsetlinewidth{0.000000pt}%
\definecolor{currentstroke}{rgb}{0.000000,0.000000,0.000000}%
\pgfsetstrokecolor{currentstroke}%
\pgfsetstrokeopacity{0.000000}%
\pgfsetdash{}{0pt}%
\pgfpathmoveto{\pgfqpoint{0.709565in}{0.549073in}}%
\pgfpathlineto{\pgfqpoint{5.227815in}{0.549073in}}%
\pgfpathlineto{\pgfqpoint{5.227815in}{2.859073in}}%
\pgfpathlineto{\pgfqpoint{0.709565in}{2.859073in}}%
\pgfpathlineto{\pgfqpoint{0.709565in}{0.549073in}}%
\pgfpathclose%
\pgfusepath{fill}%
\end{pgfscope}%
\begin{pgfscope}%
\pgfpathrectangle{\pgfqpoint{0.709565in}{0.549073in}}{\pgfqpoint{4.518250in}{2.310000in}}%
\pgfusepath{clip}%
\pgfsetrectcap%
\pgfsetroundjoin%
\pgfsetlinewidth{0.250937pt}%
\definecolor{currentstroke}{rgb}{0.000000,0.000000,0.000000}%
\pgfsetstrokecolor{currentstroke}%
\pgfsetstrokeopacity{0.200000}%
\pgfsetdash{}{0pt}%
\pgfpathmoveto{\pgfqpoint{1.151921in}{0.549073in}}%
\pgfpathlineto{\pgfqpoint{1.151921in}{2.859073in}}%
\pgfusepath{stroke}%
\end{pgfscope}%
\begin{pgfscope}%
\pgfsetbuttcap%
\pgfsetroundjoin%
\definecolor{currentfill}{rgb}{0.000000,0.000000,0.000000}%
\pgfsetfillcolor{currentfill}%
\pgfsetlinewidth{0.803000pt}%
\definecolor{currentstroke}{rgb}{0.000000,0.000000,0.000000}%
\pgfsetstrokecolor{currentstroke}%
\pgfsetdash{}{0pt}%
\pgfsys@defobject{currentmarker}{\pgfqpoint{0.000000in}{-0.048611in}}{\pgfqpoint{0.000000in}{0.000000in}}{%
\pgfpathmoveto{\pgfqpoint{0.000000in}{0.000000in}}%
\pgfpathlineto{\pgfqpoint{0.000000in}{-0.048611in}}%
\pgfusepath{stroke,fill}%
}%
\begin{pgfscope}%
\pgfsys@transformshift{1.151921in}{0.549073in}%
\pgfsys@useobject{currentmarker}{}%
\end{pgfscope}%
\end{pgfscope}%
\begin{pgfscope}%
\definecolor{textcolor}{rgb}{0.000000,0.000000,0.000000}%
\pgfsetstrokecolor{textcolor}%
\pgfsetfillcolor{textcolor}%
\pgftext[x=1.151921in,y=0.451851in,,top]{\color{textcolor}{\rmfamily\fontsize{12.000000}{14.400000}\selectfont\catcode`\^=\active\def^{\ifmmode\sp\else\^{}\fi}\catcode`\%=\active\def%{\%}$\mathdefault{0.5}$}}%
\end{pgfscope}%
\begin{pgfscope}%
\pgfpathrectangle{\pgfqpoint{0.709565in}{0.549073in}}{\pgfqpoint{4.518250in}{2.310000in}}%
\pgfusepath{clip}%
\pgfsetrectcap%
\pgfsetroundjoin%
\pgfsetlinewidth{0.250937pt}%
\definecolor{currentstroke}{rgb}{0.000000,0.000000,0.000000}%
\pgfsetstrokecolor{currentstroke}%
\pgfsetstrokeopacity{0.200000}%
\pgfsetdash{}{0pt}%
\pgfpathmoveto{\pgfqpoint{1.713178in}{0.549073in}}%
\pgfpathlineto{\pgfqpoint{1.713178in}{2.859073in}}%
\pgfusepath{stroke}%
\end{pgfscope}%
\begin{pgfscope}%
\pgfsetbuttcap%
\pgfsetroundjoin%
\definecolor{currentfill}{rgb}{0.000000,0.000000,0.000000}%
\pgfsetfillcolor{currentfill}%
\pgfsetlinewidth{0.803000pt}%
\definecolor{currentstroke}{rgb}{0.000000,0.000000,0.000000}%
\pgfsetstrokecolor{currentstroke}%
\pgfsetdash{}{0pt}%
\pgfsys@defobject{currentmarker}{\pgfqpoint{0.000000in}{-0.048611in}}{\pgfqpoint{0.000000in}{0.000000in}}{%
\pgfpathmoveto{\pgfqpoint{0.000000in}{0.000000in}}%
\pgfpathlineto{\pgfqpoint{0.000000in}{-0.048611in}}%
\pgfusepath{stroke,fill}%
}%
\begin{pgfscope}%
\pgfsys@transformshift{1.713178in}{0.549073in}%
\pgfsys@useobject{currentmarker}{}%
\end{pgfscope}%
\end{pgfscope}%
\begin{pgfscope}%
\definecolor{textcolor}{rgb}{0.000000,0.000000,0.000000}%
\pgfsetstrokecolor{textcolor}%
\pgfsetfillcolor{textcolor}%
\pgftext[x=1.713178in,y=0.451851in,,top]{\color{textcolor}{\rmfamily\fontsize{12.000000}{14.400000}\selectfont\catcode`\^=\active\def^{\ifmmode\sp\else\^{}\fi}\catcode`\%=\active\def%{\%}$\mathdefault{1.0}$}}%
\end{pgfscope}%
\begin{pgfscope}%
\pgfpathrectangle{\pgfqpoint{0.709565in}{0.549073in}}{\pgfqpoint{4.518250in}{2.310000in}}%
\pgfusepath{clip}%
\pgfsetrectcap%
\pgfsetroundjoin%
\pgfsetlinewidth{0.250937pt}%
\definecolor{currentstroke}{rgb}{0.000000,0.000000,0.000000}%
\pgfsetstrokecolor{currentstroke}%
\pgfsetstrokeopacity{0.200000}%
\pgfsetdash{}{0pt}%
\pgfpathmoveto{\pgfqpoint{2.274436in}{0.549073in}}%
\pgfpathlineto{\pgfqpoint{2.274436in}{2.859073in}}%
\pgfusepath{stroke}%
\end{pgfscope}%
\begin{pgfscope}%
\pgfsetbuttcap%
\pgfsetroundjoin%
\definecolor{currentfill}{rgb}{0.000000,0.000000,0.000000}%
\pgfsetfillcolor{currentfill}%
\pgfsetlinewidth{0.803000pt}%
\definecolor{currentstroke}{rgb}{0.000000,0.000000,0.000000}%
\pgfsetstrokecolor{currentstroke}%
\pgfsetdash{}{0pt}%
\pgfsys@defobject{currentmarker}{\pgfqpoint{0.000000in}{-0.048611in}}{\pgfqpoint{0.000000in}{0.000000in}}{%
\pgfpathmoveto{\pgfqpoint{0.000000in}{0.000000in}}%
\pgfpathlineto{\pgfqpoint{0.000000in}{-0.048611in}}%
\pgfusepath{stroke,fill}%
}%
\begin{pgfscope}%
\pgfsys@transformshift{2.274436in}{0.549073in}%
\pgfsys@useobject{currentmarker}{}%
\end{pgfscope}%
\end{pgfscope}%
\begin{pgfscope}%
\definecolor{textcolor}{rgb}{0.000000,0.000000,0.000000}%
\pgfsetstrokecolor{textcolor}%
\pgfsetfillcolor{textcolor}%
\pgftext[x=2.274436in,y=0.451851in,,top]{\color{textcolor}{\rmfamily\fontsize{12.000000}{14.400000}\selectfont\catcode`\^=\active\def^{\ifmmode\sp\else\^{}\fi}\catcode`\%=\active\def%{\%}$\mathdefault{1.5}$}}%
\end{pgfscope}%
\begin{pgfscope}%
\pgfpathrectangle{\pgfqpoint{0.709565in}{0.549073in}}{\pgfqpoint{4.518250in}{2.310000in}}%
\pgfusepath{clip}%
\pgfsetrectcap%
\pgfsetroundjoin%
\pgfsetlinewidth{0.250937pt}%
\definecolor{currentstroke}{rgb}{0.000000,0.000000,0.000000}%
\pgfsetstrokecolor{currentstroke}%
\pgfsetstrokeopacity{0.200000}%
\pgfsetdash{}{0pt}%
\pgfpathmoveto{\pgfqpoint{2.835693in}{0.549073in}}%
\pgfpathlineto{\pgfqpoint{2.835693in}{2.859073in}}%
\pgfusepath{stroke}%
\end{pgfscope}%
\begin{pgfscope}%
\pgfsetbuttcap%
\pgfsetroundjoin%
\definecolor{currentfill}{rgb}{0.000000,0.000000,0.000000}%
\pgfsetfillcolor{currentfill}%
\pgfsetlinewidth{0.803000pt}%
\definecolor{currentstroke}{rgb}{0.000000,0.000000,0.000000}%
\pgfsetstrokecolor{currentstroke}%
\pgfsetdash{}{0pt}%
\pgfsys@defobject{currentmarker}{\pgfqpoint{0.000000in}{-0.048611in}}{\pgfqpoint{0.000000in}{0.000000in}}{%
\pgfpathmoveto{\pgfqpoint{0.000000in}{0.000000in}}%
\pgfpathlineto{\pgfqpoint{0.000000in}{-0.048611in}}%
\pgfusepath{stroke,fill}%
}%
\begin{pgfscope}%
\pgfsys@transformshift{2.835693in}{0.549073in}%
\pgfsys@useobject{currentmarker}{}%
\end{pgfscope}%
\end{pgfscope}%
\begin{pgfscope}%
\definecolor{textcolor}{rgb}{0.000000,0.000000,0.000000}%
\pgfsetstrokecolor{textcolor}%
\pgfsetfillcolor{textcolor}%
\pgftext[x=2.835693in,y=0.451851in,,top]{\color{textcolor}{\rmfamily\fontsize{12.000000}{14.400000}\selectfont\catcode`\^=\active\def^{\ifmmode\sp\else\^{}\fi}\catcode`\%=\active\def%{\%}$\mathdefault{2.0}$}}%
\end{pgfscope}%
\begin{pgfscope}%
\pgfpathrectangle{\pgfqpoint{0.709565in}{0.549073in}}{\pgfqpoint{4.518250in}{2.310000in}}%
\pgfusepath{clip}%
\pgfsetrectcap%
\pgfsetroundjoin%
\pgfsetlinewidth{0.250937pt}%
\definecolor{currentstroke}{rgb}{0.000000,0.000000,0.000000}%
\pgfsetstrokecolor{currentstroke}%
\pgfsetstrokeopacity{0.200000}%
\pgfsetdash{}{0pt}%
\pgfpathmoveto{\pgfqpoint{3.396951in}{0.549073in}}%
\pgfpathlineto{\pgfqpoint{3.396951in}{2.859073in}}%
\pgfusepath{stroke}%
\end{pgfscope}%
\begin{pgfscope}%
\pgfsetbuttcap%
\pgfsetroundjoin%
\definecolor{currentfill}{rgb}{0.000000,0.000000,0.000000}%
\pgfsetfillcolor{currentfill}%
\pgfsetlinewidth{0.803000pt}%
\definecolor{currentstroke}{rgb}{0.000000,0.000000,0.000000}%
\pgfsetstrokecolor{currentstroke}%
\pgfsetdash{}{0pt}%
\pgfsys@defobject{currentmarker}{\pgfqpoint{0.000000in}{-0.048611in}}{\pgfqpoint{0.000000in}{0.000000in}}{%
\pgfpathmoveto{\pgfqpoint{0.000000in}{0.000000in}}%
\pgfpathlineto{\pgfqpoint{0.000000in}{-0.048611in}}%
\pgfusepath{stroke,fill}%
}%
\begin{pgfscope}%
\pgfsys@transformshift{3.396951in}{0.549073in}%
\pgfsys@useobject{currentmarker}{}%
\end{pgfscope}%
\end{pgfscope}%
\begin{pgfscope}%
\definecolor{textcolor}{rgb}{0.000000,0.000000,0.000000}%
\pgfsetstrokecolor{textcolor}%
\pgfsetfillcolor{textcolor}%
\pgftext[x=3.396951in,y=0.451851in,,top]{\color{textcolor}{\rmfamily\fontsize{12.000000}{14.400000}\selectfont\catcode`\^=\active\def^{\ifmmode\sp\else\^{}\fi}\catcode`\%=\active\def%{\%}$\mathdefault{2.5}$}}%
\end{pgfscope}%
\begin{pgfscope}%
\pgfpathrectangle{\pgfqpoint{0.709565in}{0.549073in}}{\pgfqpoint{4.518250in}{2.310000in}}%
\pgfusepath{clip}%
\pgfsetrectcap%
\pgfsetroundjoin%
\pgfsetlinewidth{0.250937pt}%
\definecolor{currentstroke}{rgb}{0.000000,0.000000,0.000000}%
\pgfsetstrokecolor{currentstroke}%
\pgfsetstrokeopacity{0.200000}%
\pgfsetdash{}{0pt}%
\pgfpathmoveto{\pgfqpoint{3.958208in}{0.549073in}}%
\pgfpathlineto{\pgfqpoint{3.958208in}{2.859073in}}%
\pgfusepath{stroke}%
\end{pgfscope}%
\begin{pgfscope}%
\pgfsetbuttcap%
\pgfsetroundjoin%
\definecolor{currentfill}{rgb}{0.000000,0.000000,0.000000}%
\pgfsetfillcolor{currentfill}%
\pgfsetlinewidth{0.803000pt}%
\definecolor{currentstroke}{rgb}{0.000000,0.000000,0.000000}%
\pgfsetstrokecolor{currentstroke}%
\pgfsetdash{}{0pt}%
\pgfsys@defobject{currentmarker}{\pgfqpoint{0.000000in}{-0.048611in}}{\pgfqpoint{0.000000in}{0.000000in}}{%
\pgfpathmoveto{\pgfqpoint{0.000000in}{0.000000in}}%
\pgfpathlineto{\pgfqpoint{0.000000in}{-0.048611in}}%
\pgfusepath{stroke,fill}%
}%
\begin{pgfscope}%
\pgfsys@transformshift{3.958208in}{0.549073in}%
\pgfsys@useobject{currentmarker}{}%
\end{pgfscope}%
\end{pgfscope}%
\begin{pgfscope}%
\definecolor{textcolor}{rgb}{0.000000,0.000000,0.000000}%
\pgfsetstrokecolor{textcolor}%
\pgfsetfillcolor{textcolor}%
\pgftext[x=3.958208in,y=0.451851in,,top]{\color{textcolor}{\rmfamily\fontsize{12.000000}{14.400000}\selectfont\catcode`\^=\active\def^{\ifmmode\sp\else\^{}\fi}\catcode`\%=\active\def%{\%}$\mathdefault{3.0}$}}%
\end{pgfscope}%
\begin{pgfscope}%
\pgfpathrectangle{\pgfqpoint{0.709565in}{0.549073in}}{\pgfqpoint{4.518250in}{2.310000in}}%
\pgfusepath{clip}%
\pgfsetrectcap%
\pgfsetroundjoin%
\pgfsetlinewidth{0.250937pt}%
\definecolor{currentstroke}{rgb}{0.000000,0.000000,0.000000}%
\pgfsetstrokecolor{currentstroke}%
\pgfsetstrokeopacity{0.200000}%
\pgfsetdash{}{0pt}%
\pgfpathmoveto{\pgfqpoint{4.519466in}{0.549073in}}%
\pgfpathlineto{\pgfqpoint{4.519466in}{2.859073in}}%
\pgfusepath{stroke}%
\end{pgfscope}%
\begin{pgfscope}%
\pgfsetbuttcap%
\pgfsetroundjoin%
\definecolor{currentfill}{rgb}{0.000000,0.000000,0.000000}%
\pgfsetfillcolor{currentfill}%
\pgfsetlinewidth{0.803000pt}%
\definecolor{currentstroke}{rgb}{0.000000,0.000000,0.000000}%
\pgfsetstrokecolor{currentstroke}%
\pgfsetdash{}{0pt}%
\pgfsys@defobject{currentmarker}{\pgfqpoint{0.000000in}{-0.048611in}}{\pgfqpoint{0.000000in}{0.000000in}}{%
\pgfpathmoveto{\pgfqpoint{0.000000in}{0.000000in}}%
\pgfpathlineto{\pgfqpoint{0.000000in}{-0.048611in}}%
\pgfusepath{stroke,fill}%
}%
\begin{pgfscope}%
\pgfsys@transformshift{4.519466in}{0.549073in}%
\pgfsys@useobject{currentmarker}{}%
\end{pgfscope}%
\end{pgfscope}%
\begin{pgfscope}%
\definecolor{textcolor}{rgb}{0.000000,0.000000,0.000000}%
\pgfsetstrokecolor{textcolor}%
\pgfsetfillcolor{textcolor}%
\pgftext[x=4.519466in,y=0.451851in,,top]{\color{textcolor}{\rmfamily\fontsize{12.000000}{14.400000}\selectfont\catcode`\^=\active\def^{\ifmmode\sp\else\^{}\fi}\catcode`\%=\active\def%{\%}$\mathdefault{3.5}$}}%
\end{pgfscope}%
\begin{pgfscope}%
\pgfpathrectangle{\pgfqpoint{0.709565in}{0.549073in}}{\pgfqpoint{4.518250in}{2.310000in}}%
\pgfusepath{clip}%
\pgfsetrectcap%
\pgfsetroundjoin%
\pgfsetlinewidth{0.250937pt}%
\definecolor{currentstroke}{rgb}{0.000000,0.000000,0.000000}%
\pgfsetstrokecolor{currentstroke}%
\pgfsetstrokeopacity{0.200000}%
\pgfsetdash{}{0pt}%
\pgfpathmoveto{\pgfqpoint{5.080723in}{0.549073in}}%
\pgfpathlineto{\pgfqpoint{5.080723in}{2.859073in}}%
\pgfusepath{stroke}%
\end{pgfscope}%
\begin{pgfscope}%
\pgfsetbuttcap%
\pgfsetroundjoin%
\definecolor{currentfill}{rgb}{0.000000,0.000000,0.000000}%
\pgfsetfillcolor{currentfill}%
\pgfsetlinewidth{0.803000pt}%
\definecolor{currentstroke}{rgb}{0.000000,0.000000,0.000000}%
\pgfsetstrokecolor{currentstroke}%
\pgfsetdash{}{0pt}%
\pgfsys@defobject{currentmarker}{\pgfqpoint{0.000000in}{-0.048611in}}{\pgfqpoint{0.000000in}{0.000000in}}{%
\pgfpathmoveto{\pgfqpoint{0.000000in}{0.000000in}}%
\pgfpathlineto{\pgfqpoint{0.000000in}{-0.048611in}}%
\pgfusepath{stroke,fill}%
}%
\begin{pgfscope}%
\pgfsys@transformshift{5.080723in}{0.549073in}%
\pgfsys@useobject{currentmarker}{}%
\end{pgfscope}%
\end{pgfscope}%
\begin{pgfscope}%
\definecolor{textcolor}{rgb}{0.000000,0.000000,0.000000}%
\pgfsetstrokecolor{textcolor}%
\pgfsetfillcolor{textcolor}%
\pgftext[x=5.080723in,y=0.451851in,,top]{\color{textcolor}{\rmfamily\fontsize{12.000000}{14.400000}\selectfont\catcode`\^=\active\def^{\ifmmode\sp\else\^{}\fi}\catcode`\%=\active\def%{\%}$\mathdefault{4.0}$}}%
\end{pgfscope}%
\begin{pgfscope}%
\definecolor{textcolor}{rgb}{0.000000,0.000000,0.000000}%
\pgfsetstrokecolor{textcolor}%
\pgfsetfillcolor{textcolor}%
\pgftext[x=2.968690in,y=0.248148in,,top]{\color{textcolor}{\rmfamily\fontsize{12.000000}{14.400000}\selectfont\catcode`\^=\active\def^{\ifmmode\sp\else\^{}\fi}\catcode`\%=\active\def%{\%}spectral parameter $t$}}%
\end{pgfscope}%
\begin{pgfscope}%
\pgfpathrectangle{\pgfqpoint{0.709565in}{0.549073in}}{\pgfqpoint{4.518250in}{2.310000in}}%
\pgfusepath{clip}%
\pgfsetrectcap%
\pgfsetroundjoin%
\pgfsetlinewidth{0.250937pt}%
\definecolor{currentstroke}{rgb}{0.000000,0.000000,0.000000}%
\pgfsetstrokecolor{currentstroke}%
\pgfsetstrokeopacity{0.200000}%
\pgfsetdash{}{0pt}%
\pgfpathmoveto{\pgfqpoint{0.709565in}{0.741573in}}%
\pgfpathlineto{\pgfqpoint{5.227815in}{0.741573in}}%
\pgfusepath{stroke}%
\end{pgfscope}%
\begin{pgfscope}%
\pgfsetbuttcap%
\pgfsetroundjoin%
\definecolor{currentfill}{rgb}{0.000000,0.000000,0.000000}%
\pgfsetfillcolor{currentfill}%
\pgfsetlinewidth{0.803000pt}%
\definecolor{currentstroke}{rgb}{0.000000,0.000000,0.000000}%
\pgfsetstrokecolor{currentstroke}%
\pgfsetdash{}{0pt}%
\pgfsys@defobject{currentmarker}{\pgfqpoint{-0.048611in}{0.000000in}}{\pgfqpoint{-0.000000in}{0.000000in}}{%
\pgfpathmoveto{\pgfqpoint{-0.000000in}{0.000000in}}%
\pgfpathlineto{\pgfqpoint{-0.048611in}{0.000000in}}%
\pgfusepath{stroke,fill}%
}%
\begin{pgfscope}%
\pgfsys@transformshift{0.709565in}{0.741573in}%
\pgfsys@useobject{currentmarker}{}%
\end{pgfscope}%
\end{pgfscope}%
\begin{pgfscope}%
\definecolor{textcolor}{rgb}{0.000000,0.000000,0.000000}%
\pgfsetstrokecolor{textcolor}%
\pgfsetfillcolor{textcolor}%
\pgftext[x=0.322222in, y=0.683703in, left, base]{\color{textcolor}{\rmfamily\fontsize{12.000000}{14.400000}\selectfont\catcode`\^=\active\def^{\ifmmode\sp\else\^{}\fi}\catcode`\%=\active\def%{\%}$\mathdefault{0.00}$}}%
\end{pgfscope}%
\begin{pgfscope}%
\pgfpathrectangle{\pgfqpoint{0.709565in}{0.549073in}}{\pgfqpoint{4.518250in}{2.310000in}}%
\pgfusepath{clip}%
\pgfsetrectcap%
\pgfsetroundjoin%
\pgfsetlinewidth{0.250937pt}%
\definecolor{currentstroke}{rgb}{0.000000,0.000000,0.000000}%
\pgfsetstrokecolor{currentstroke}%
\pgfsetstrokeopacity{0.200000}%
\pgfsetdash{}{0pt}%
\pgfpathmoveto{\pgfqpoint{0.709565in}{1.241047in}}%
\pgfpathlineto{\pgfqpoint{5.227815in}{1.241047in}}%
\pgfusepath{stroke}%
\end{pgfscope}%
\begin{pgfscope}%
\pgfsetbuttcap%
\pgfsetroundjoin%
\definecolor{currentfill}{rgb}{0.000000,0.000000,0.000000}%
\pgfsetfillcolor{currentfill}%
\pgfsetlinewidth{0.803000pt}%
\definecolor{currentstroke}{rgb}{0.000000,0.000000,0.000000}%
\pgfsetstrokecolor{currentstroke}%
\pgfsetdash{}{0pt}%
\pgfsys@defobject{currentmarker}{\pgfqpoint{-0.048611in}{0.000000in}}{\pgfqpoint{-0.000000in}{0.000000in}}{%
\pgfpathmoveto{\pgfqpoint{-0.000000in}{0.000000in}}%
\pgfpathlineto{\pgfqpoint{-0.048611in}{0.000000in}}%
\pgfusepath{stroke,fill}%
}%
\begin{pgfscope}%
\pgfsys@transformshift{0.709565in}{1.241047in}%
\pgfsys@useobject{currentmarker}{}%
\end{pgfscope}%
\end{pgfscope}%
\begin{pgfscope}%
\definecolor{textcolor}{rgb}{0.000000,0.000000,0.000000}%
\pgfsetstrokecolor{textcolor}%
\pgfsetfillcolor{textcolor}%
\pgftext[x=0.322222in, y=1.183177in, left, base]{\color{textcolor}{\rmfamily\fontsize{12.000000}{14.400000}\selectfont\catcode`\^=\active\def^{\ifmmode\sp\else\^{}\fi}\catcode`\%=\active\def%{\%}$\mathdefault{0.05}$}}%
\end{pgfscope}%
\begin{pgfscope}%
\pgfpathrectangle{\pgfqpoint{0.709565in}{0.549073in}}{\pgfqpoint{4.518250in}{2.310000in}}%
\pgfusepath{clip}%
\pgfsetrectcap%
\pgfsetroundjoin%
\pgfsetlinewidth{0.250937pt}%
\definecolor{currentstroke}{rgb}{0.000000,0.000000,0.000000}%
\pgfsetstrokecolor{currentstroke}%
\pgfsetstrokeopacity{0.200000}%
\pgfsetdash{}{0pt}%
\pgfpathmoveto{\pgfqpoint{0.709565in}{1.740522in}}%
\pgfpathlineto{\pgfqpoint{5.227815in}{1.740522in}}%
\pgfusepath{stroke}%
\end{pgfscope}%
\begin{pgfscope}%
\pgfsetbuttcap%
\pgfsetroundjoin%
\definecolor{currentfill}{rgb}{0.000000,0.000000,0.000000}%
\pgfsetfillcolor{currentfill}%
\pgfsetlinewidth{0.803000pt}%
\definecolor{currentstroke}{rgb}{0.000000,0.000000,0.000000}%
\pgfsetstrokecolor{currentstroke}%
\pgfsetdash{}{0pt}%
\pgfsys@defobject{currentmarker}{\pgfqpoint{-0.048611in}{0.000000in}}{\pgfqpoint{-0.000000in}{0.000000in}}{%
\pgfpathmoveto{\pgfqpoint{-0.000000in}{0.000000in}}%
\pgfpathlineto{\pgfqpoint{-0.048611in}{0.000000in}}%
\pgfusepath{stroke,fill}%
}%
\begin{pgfscope}%
\pgfsys@transformshift{0.709565in}{1.740522in}%
\pgfsys@useobject{currentmarker}{}%
\end{pgfscope}%
\end{pgfscope}%
\begin{pgfscope}%
\definecolor{textcolor}{rgb}{0.000000,0.000000,0.000000}%
\pgfsetstrokecolor{textcolor}%
\pgfsetfillcolor{textcolor}%
\pgftext[x=0.322222in, y=1.682651in, left, base]{\color{textcolor}{\rmfamily\fontsize{12.000000}{14.400000}\selectfont\catcode`\^=\active\def^{\ifmmode\sp\else\^{}\fi}\catcode`\%=\active\def%{\%}$\mathdefault{0.10}$}}%
\end{pgfscope}%
\begin{pgfscope}%
\pgfpathrectangle{\pgfqpoint{0.709565in}{0.549073in}}{\pgfqpoint{4.518250in}{2.310000in}}%
\pgfusepath{clip}%
\pgfsetrectcap%
\pgfsetroundjoin%
\pgfsetlinewidth{0.250937pt}%
\definecolor{currentstroke}{rgb}{0.000000,0.000000,0.000000}%
\pgfsetstrokecolor{currentstroke}%
\pgfsetstrokeopacity{0.200000}%
\pgfsetdash{}{0pt}%
\pgfpathmoveto{\pgfqpoint{0.709565in}{2.239996in}}%
\pgfpathlineto{\pgfqpoint{5.227815in}{2.239996in}}%
\pgfusepath{stroke}%
\end{pgfscope}%
\begin{pgfscope}%
\pgfsetbuttcap%
\pgfsetroundjoin%
\definecolor{currentfill}{rgb}{0.000000,0.000000,0.000000}%
\pgfsetfillcolor{currentfill}%
\pgfsetlinewidth{0.803000pt}%
\definecolor{currentstroke}{rgb}{0.000000,0.000000,0.000000}%
\pgfsetstrokecolor{currentstroke}%
\pgfsetdash{}{0pt}%
\pgfsys@defobject{currentmarker}{\pgfqpoint{-0.048611in}{0.000000in}}{\pgfqpoint{-0.000000in}{0.000000in}}{%
\pgfpathmoveto{\pgfqpoint{-0.000000in}{0.000000in}}%
\pgfpathlineto{\pgfqpoint{-0.048611in}{0.000000in}}%
\pgfusepath{stroke,fill}%
}%
\begin{pgfscope}%
\pgfsys@transformshift{0.709565in}{2.239996in}%
\pgfsys@useobject{currentmarker}{}%
\end{pgfscope}%
\end{pgfscope}%
\begin{pgfscope}%
\definecolor{textcolor}{rgb}{0.000000,0.000000,0.000000}%
\pgfsetstrokecolor{textcolor}%
\pgfsetfillcolor{textcolor}%
\pgftext[x=0.322222in, y=2.182126in, left, base]{\color{textcolor}{\rmfamily\fontsize{12.000000}{14.400000}\selectfont\catcode`\^=\active\def^{\ifmmode\sp\else\^{}\fi}\catcode`\%=\active\def%{\%}$\mathdefault{0.15}$}}%
\end{pgfscope}%
\begin{pgfscope}%
\pgfpathrectangle{\pgfqpoint{0.709565in}{0.549073in}}{\pgfqpoint{4.518250in}{2.310000in}}%
\pgfusepath{clip}%
\pgfsetrectcap%
\pgfsetroundjoin%
\pgfsetlinewidth{0.250937pt}%
\definecolor{currentstroke}{rgb}{0.000000,0.000000,0.000000}%
\pgfsetstrokecolor{currentstroke}%
\pgfsetstrokeopacity{0.200000}%
\pgfsetdash{}{0pt}%
\pgfpathmoveto{\pgfqpoint{0.709565in}{2.739471in}}%
\pgfpathlineto{\pgfqpoint{5.227815in}{2.739471in}}%
\pgfusepath{stroke}%
\end{pgfscope}%
\begin{pgfscope}%
\pgfsetbuttcap%
\pgfsetroundjoin%
\definecolor{currentfill}{rgb}{0.000000,0.000000,0.000000}%
\pgfsetfillcolor{currentfill}%
\pgfsetlinewidth{0.803000pt}%
\definecolor{currentstroke}{rgb}{0.000000,0.000000,0.000000}%
\pgfsetstrokecolor{currentstroke}%
\pgfsetdash{}{0pt}%
\pgfsys@defobject{currentmarker}{\pgfqpoint{-0.048611in}{0.000000in}}{\pgfqpoint{-0.000000in}{0.000000in}}{%
\pgfpathmoveto{\pgfqpoint{-0.000000in}{0.000000in}}%
\pgfpathlineto{\pgfqpoint{-0.048611in}{0.000000in}}%
\pgfusepath{stroke,fill}%
}%
\begin{pgfscope}%
\pgfsys@transformshift{0.709565in}{2.739471in}%
\pgfsys@useobject{currentmarker}{}%
\end{pgfscope}%
\end{pgfscope}%
\begin{pgfscope}%
\definecolor{textcolor}{rgb}{0.000000,0.000000,0.000000}%
\pgfsetstrokecolor{textcolor}%
\pgfsetfillcolor{textcolor}%
\pgftext[x=0.322222in, y=2.681600in, left, base]{\color{textcolor}{\rmfamily\fontsize{12.000000}{14.400000}\selectfont\catcode`\^=\active\def^{\ifmmode\sp\else\^{}\fi}\catcode`\%=\active\def%{\%}$\mathdefault{0.20}$}}%
\end{pgfscope}%
\begin{pgfscope}%
\definecolor{textcolor}{rgb}{0.000000,0.000000,0.000000}%
\pgfsetstrokecolor{textcolor}%
\pgfsetfillcolor{textcolor}%
\pgftext[x=0.266667in,y=1.704073in,,bottom,rotate=90.000000]{\color{textcolor}{\rmfamily\fontsize{12.000000}{14.400000}\selectfont\catcode`\^=\active\def^{\ifmmode\sp\else\^{}\fi}\catcode`\%=\active\def%{\%}smoothed spectral density $\phi_{\sigma}(t)$}}%
\end{pgfscope}%
\begin{pgfscope}%
\pgfpathrectangle{\pgfqpoint{0.709565in}{0.549073in}}{\pgfqpoint{4.518250in}{2.310000in}}%
\pgfusepath{clip}%
\pgfsetrectcap%
\pgfsetroundjoin%
\pgfsetlinewidth{1.505625pt}%
\definecolor{currentstroke}{rgb}{0.392157,0.560784,1.000000}%
\pgfsetstrokecolor{currentstroke}%
\pgfsetdash{}{0pt}%
\pgfpathmoveto{\pgfqpoint{0.817142in}{2.666573in}}%
\pgfpathlineto{\pgfqpoint{0.846022in}{2.436137in}}%
\pgfpathlineto{\pgfqpoint{0.874902in}{1.943469in}}%
\pgfpathlineto{\pgfqpoint{0.903782in}{1.811023in}}%
\pgfpathlineto{\pgfqpoint{0.932662in}{1.710201in}}%
\pgfpathlineto{\pgfqpoint{0.961541in}{1.529869in}}%
\pgfpathlineto{\pgfqpoint{0.990421in}{1.287596in}}%
\pgfpathlineto{\pgfqpoint{1.019301in}{1.171492in}}%
\pgfpathlineto{\pgfqpoint{1.048181in}{1.152872in}}%
\pgfpathlineto{\pgfqpoint{1.077061in}{1.173679in}}%
\pgfpathlineto{\pgfqpoint{1.105941in}{1.166737in}}%
\pgfpathlineto{\pgfqpoint{1.134820in}{1.217309in}}%
\pgfpathlineto{\pgfqpoint{1.163700in}{1.284917in}}%
\pgfpathlineto{\pgfqpoint{1.192580in}{1.151310in}}%
\pgfpathlineto{\pgfqpoint{1.221460in}{1.008515in}}%
\pgfpathlineto{\pgfqpoint{1.250340in}{0.926490in}}%
\pgfpathlineto{\pgfqpoint{1.279220in}{0.855826in}}%
\pgfpathlineto{\pgfqpoint{1.308099in}{0.837333in}}%
\pgfpathlineto{\pgfqpoint{1.336979in}{0.892864in}}%
\pgfpathlineto{\pgfqpoint{1.365859in}{0.899306in}}%
\pgfpathlineto{\pgfqpoint{1.394739in}{0.889501in}}%
\pgfpathlineto{\pgfqpoint{1.423619in}{0.916959in}}%
\pgfpathlineto{\pgfqpoint{1.452499in}{0.956783in}}%
\pgfpathlineto{\pgfqpoint{1.481378in}{0.969445in}}%
\pgfpathlineto{\pgfqpoint{1.510258in}{0.918678in}}%
\pgfpathlineto{\pgfqpoint{1.539138in}{0.870720in}}%
\pgfpathlineto{\pgfqpoint{1.568018in}{0.866978in}}%
\pgfpathlineto{\pgfqpoint{1.596898in}{0.849511in}}%
\pgfpathlineto{\pgfqpoint{1.625778in}{0.854341in}}%
\pgfpathlineto{\pgfqpoint{1.654657in}{0.773568in}}%
\pgfpathlineto{\pgfqpoint{1.683537in}{0.751130in}}%
\pgfpathlineto{\pgfqpoint{1.712417in}{0.757719in}}%
\pgfpathlineto{\pgfqpoint{1.741297in}{0.832466in}}%
\pgfpathlineto{\pgfqpoint{1.770177in}{0.842136in}}%
\pgfpathlineto{\pgfqpoint{1.799057in}{0.763458in}}%
\pgfpathlineto{\pgfqpoint{1.827936in}{0.805592in}}%
\pgfpathlineto{\pgfqpoint{1.856816in}{0.741573in}}%
\pgfpathlineto{\pgfqpoint{2.058975in}{0.741573in}}%
\pgfpathlineto{\pgfqpoint{2.087855in}{0.745201in}}%
\pgfpathlineto{\pgfqpoint{2.116735in}{0.788046in}}%
\pgfpathlineto{\pgfqpoint{2.145615in}{0.858652in}}%
\pgfpathlineto{\pgfqpoint{2.174494in}{0.799592in}}%
\pgfpathlineto{\pgfqpoint{2.203374in}{0.747229in}}%
\pgfpathlineto{\pgfqpoint{2.232254in}{1.218852in}}%
\pgfpathlineto{\pgfqpoint{2.261134in}{0.741573in}}%
\pgfpathlineto{\pgfqpoint{2.405533in}{0.741573in}}%
\pgfpathlineto{\pgfqpoint{2.434413in}{0.744967in}}%
\pgfpathlineto{\pgfqpoint{2.463293in}{0.786466in}}%
\pgfpathlineto{\pgfqpoint{2.492173in}{0.858360in}}%
\pgfpathlineto{\pgfqpoint{2.521052in}{0.801334in}}%
\pgfpathlineto{\pgfqpoint{2.549932in}{0.747474in}}%
\pgfpathlineto{\pgfqpoint{2.578812in}{1.112486in}}%
\pgfpathlineto{\pgfqpoint{2.607692in}{0.756699in}}%
\pgfpathlineto{\pgfqpoint{2.636572in}{0.832110in}}%
\pgfpathlineto{\pgfqpoint{2.665452in}{0.842442in}}%
\pgfpathlineto{\pgfqpoint{2.694331in}{0.763679in}}%
\pgfpathlineto{\pgfqpoint{2.723211in}{1.123777in}}%
\pgfpathlineto{\pgfqpoint{2.752091in}{0.741573in}}%
\pgfpathlineto{\pgfqpoint{2.780971in}{0.741573in}}%
\pgfpathlineto{\pgfqpoint{2.809851in}{1.310193in}}%
\pgfpathlineto{\pgfqpoint{2.838731in}{0.760451in}}%
\pgfpathlineto{\pgfqpoint{2.867610in}{0.837478in}}%
\pgfpathlineto{\pgfqpoint{2.896490in}{0.839067in}}%
\pgfpathlineto{\pgfqpoint{2.925370in}{0.790318in}}%
\pgfpathlineto{\pgfqpoint{2.954250in}{0.851659in}}%
\pgfpathlineto{\pgfqpoint{2.983130in}{0.820423in}}%
\pgfpathlineto{\pgfqpoint{3.012010in}{0.752712in}}%
\pgfpathlineto{\pgfqpoint{3.040889in}{1.448151in}}%
\pgfpathlineto{\pgfqpoint{3.069769in}{0.741573in}}%
\pgfpathlineto{\pgfqpoint{3.329688in}{0.741573in}}%
\pgfpathlineto{\pgfqpoint{3.358568in}{1.531431in}}%
\pgfpathlineto{\pgfqpoint{3.387447in}{0.748409in}}%
\pgfpathlineto{\pgfqpoint{3.416327in}{0.805202in}}%
\pgfpathlineto{\pgfqpoint{3.445207in}{0.857842in}}%
\pgfpathlineto{\pgfqpoint{3.474087in}{0.795104in}}%
\pgfpathlineto{\pgfqpoint{3.502967in}{0.825315in}}%
\pgfpathlineto{\pgfqpoint{3.531847in}{0.849665in}}%
\pgfpathlineto{\pgfqpoint{3.560726in}{0.770006in}}%
\pgfpathlineto{\pgfqpoint{3.589606in}{0.932539in}}%
\pgfpathlineto{\pgfqpoint{3.618486in}{0.741573in}}%
\pgfpathlineto{\pgfqpoint{3.647366in}{0.741573in}}%
\pgfpathlineto{\pgfqpoint{3.676246in}{1.625218in}}%
\pgfpathlineto{\pgfqpoint{3.705126in}{0.749831in}}%
\pgfpathlineto{\pgfqpoint{3.734005in}{0.810558in}}%
\pgfpathlineto{\pgfqpoint{3.762885in}{0.856032in}}%
\pgfpathlineto{\pgfqpoint{3.791765in}{0.792315in}}%
\pgfpathlineto{\pgfqpoint{3.820645in}{0.829580in}}%
\pgfpathlineto{\pgfqpoint{3.849525in}{0.846362in}}%
\pgfpathlineto{\pgfqpoint{3.878405in}{0.766772in}}%
\pgfpathlineto{\pgfqpoint{3.907284in}{0.839242in}}%
\pgfpathlineto{\pgfqpoint{3.936164in}{0.741573in}}%
\pgfpathlineto{\pgfqpoint{5.120237in}{0.741573in}}%
\pgfpathlineto{\pgfqpoint{5.120237in}{0.741573in}}%
\pgfusepath{stroke}%
\end{pgfscope}%
\begin{pgfscope}%
\pgfpathrectangle{\pgfqpoint{0.709565in}{0.549073in}}{\pgfqpoint{4.518250in}{2.310000in}}%
\pgfusepath{clip}%
\pgfsetrectcap%
\pgfsetroundjoin%
\pgfsetlinewidth{1.505625pt}%
\definecolor{currentstroke}{rgb}{0.470588,0.368627,0.941176}%
\pgfsetstrokecolor{currentstroke}%
\pgfsetdash{}{0pt}%
\pgfpathmoveto{\pgfqpoint{0.817142in}{2.343692in}}%
\pgfpathlineto{\pgfqpoint{0.846022in}{2.037890in}}%
\pgfpathlineto{\pgfqpoint{0.874902in}{1.526650in}}%
\pgfpathlineto{\pgfqpoint{0.903782in}{1.326429in}}%
\pgfpathlineto{\pgfqpoint{0.932662in}{1.274265in}}%
\pgfpathlineto{\pgfqpoint{0.961541in}{1.039231in}}%
\pgfpathlineto{\pgfqpoint{0.990421in}{1.114127in}}%
\pgfpathlineto{\pgfqpoint{1.019301in}{1.148970in}}%
\pgfpathlineto{\pgfqpoint{1.048181in}{0.957373in}}%
\pgfpathlineto{\pgfqpoint{1.077061in}{0.876247in}}%
\pgfpathlineto{\pgfqpoint{1.105941in}{0.867600in}}%
\pgfpathlineto{\pgfqpoint{1.134820in}{0.794690in}}%
\pgfpathlineto{\pgfqpoint{1.163700in}{0.807213in}}%
\pgfpathlineto{\pgfqpoint{1.192580in}{0.870825in}}%
\pgfpathlineto{\pgfqpoint{1.221460in}{0.868560in}}%
\pgfpathlineto{\pgfqpoint{1.250340in}{0.861752in}}%
\pgfpathlineto{\pgfqpoint{1.279220in}{0.846058in}}%
\pgfpathlineto{\pgfqpoint{1.308099in}{0.852659in}}%
\pgfpathlineto{\pgfqpoint{1.336979in}{0.772004in}}%
\pgfpathlineto{\pgfqpoint{1.365859in}{0.753334in}}%
\pgfpathlineto{\pgfqpoint{1.394739in}{0.752112in}}%
\pgfpathlineto{\pgfqpoint{1.423619in}{0.773833in}}%
\pgfpathlineto{\pgfqpoint{1.452499in}{0.882378in}}%
\pgfpathlineto{\pgfqpoint{1.481378in}{0.929507in}}%
\pgfpathlineto{\pgfqpoint{1.510258in}{0.829660in}}%
\pgfpathlineto{\pgfqpoint{1.539138in}{0.753566in}}%
\pgfpathlineto{\pgfqpoint{1.568018in}{0.776725in}}%
\pgfpathlineto{\pgfqpoint{1.596898in}{0.741573in}}%
\pgfpathlineto{\pgfqpoint{1.914576in}{0.741573in}}%
\pgfpathlineto{\pgfqpoint{1.943456in}{0.982944in}}%
\pgfpathlineto{\pgfqpoint{1.972336in}{0.756124in}}%
\pgfpathlineto{\pgfqpoint{2.001215in}{0.835467in}}%
\pgfpathlineto{\pgfqpoint{2.030095in}{0.908432in}}%
\pgfpathlineto{\pgfqpoint{2.058975in}{0.881828in}}%
\pgfpathlineto{\pgfqpoint{2.087855in}{0.784563in}}%
\pgfpathlineto{\pgfqpoint{2.116735in}{0.746273in}}%
\pgfpathlineto{\pgfqpoint{2.145615in}{0.773227in}}%
\pgfpathlineto{\pgfqpoint{2.174494in}{0.851849in}}%
\pgfpathlineto{\pgfqpoint{2.203374in}{0.825960in}}%
\pgfpathlineto{\pgfqpoint{2.232254in}{0.816579in}}%
\pgfpathlineto{\pgfqpoint{2.261134in}{0.859213in}}%
\pgfpathlineto{\pgfqpoint{2.290014in}{0.812893in}}%
\pgfpathlineto{\pgfqpoint{2.318894in}{0.854148in}}%
\pgfpathlineto{\pgfqpoint{2.347773in}{0.819777in}}%
\pgfpathlineto{\pgfqpoint{2.376653in}{0.757191in}}%
\pgfpathlineto{\pgfqpoint{2.405533in}{0.796889in}}%
\pgfpathlineto{\pgfqpoint{2.434413in}{0.859295in}}%
\pgfpathlineto{\pgfqpoint{2.463293in}{0.790866in}}%
\pgfpathlineto{\pgfqpoint{2.492173in}{0.745644in}}%
\pgfpathlineto{\pgfqpoint{2.521052in}{0.741573in}}%
\pgfpathlineto{\pgfqpoint{2.549932in}{0.741573in}}%
\pgfpathlineto{\pgfqpoint{2.578812in}{0.840528in}}%
\pgfpathlineto{\pgfqpoint{2.607692in}{0.778876in}}%
\pgfpathlineto{\pgfqpoint{2.636572in}{0.855766in}}%
\pgfpathlineto{\pgfqpoint{2.665452in}{0.810335in}}%
\pgfpathlineto{\pgfqpoint{2.694331in}{0.749717in}}%
\pgfpathlineto{\pgfqpoint{2.723211in}{1.522866in}}%
\pgfpathlineto{\pgfqpoint{2.752091in}{0.741573in}}%
\pgfpathlineto{\pgfqpoint{3.127529in}{0.741573in}}%
\pgfpathlineto{\pgfqpoint{3.156409in}{0.935520in}}%
\pgfpathlineto{\pgfqpoint{3.185289in}{0.756767in}}%
\pgfpathlineto{\pgfqpoint{3.214168in}{0.830446in}}%
\pgfpathlineto{\pgfqpoint{3.243048in}{0.843827in}}%
\pgfpathlineto{\pgfqpoint{3.271928in}{0.764715in}}%
\pgfpathlineto{\pgfqpoint{3.300808in}{0.742603in}}%
\pgfpathlineto{\pgfqpoint{3.358568in}{0.741573in}}%
\pgfpathlineto{\pgfqpoint{5.120237in}{0.741573in}}%
\pgfpathlineto{\pgfqpoint{5.120237in}{0.741573in}}%
\pgfusepath{stroke}%
\end{pgfscope}%
\begin{pgfscope}%
\pgfpathrectangle{\pgfqpoint{0.709565in}{0.549073in}}{\pgfqpoint{4.518250in}{2.310000in}}%
\pgfusepath{clip}%
\pgfsetrectcap%
\pgfsetroundjoin%
\pgfsetlinewidth{1.505625pt}%
\definecolor{currentstroke}{rgb}{0.862745,0.149020,0.498039}%
\pgfsetstrokecolor{currentstroke}%
\pgfsetdash{}{0pt}%
\pgfpathmoveto{\pgfqpoint{0.817142in}{1.969796in}}%
\pgfpathlineto{\pgfqpoint{0.846022in}{1.787832in}}%
\pgfpathlineto{\pgfqpoint{0.874902in}{1.539811in}}%
\pgfpathlineto{\pgfqpoint{0.903782in}{1.468055in}}%
\pgfpathlineto{\pgfqpoint{0.932662in}{1.363254in}}%
\pgfpathlineto{\pgfqpoint{0.961541in}{1.127468in}}%
\pgfpathlineto{\pgfqpoint{0.990421in}{1.057509in}}%
\pgfpathlineto{\pgfqpoint{1.019301in}{1.006715in}}%
\pgfpathlineto{\pgfqpoint{1.048181in}{0.945749in}}%
\pgfpathlineto{\pgfqpoint{1.077061in}{0.926551in}}%
\pgfpathlineto{\pgfqpoint{1.105941in}{0.888146in}}%
\pgfpathlineto{\pgfqpoint{1.134820in}{0.817073in}}%
\pgfpathlineto{\pgfqpoint{1.163700in}{0.853122in}}%
\pgfpathlineto{\pgfqpoint{1.192580in}{0.822244in}}%
\pgfpathlineto{\pgfqpoint{1.221460in}{0.763857in}}%
\pgfpathlineto{\pgfqpoint{1.250340in}{0.818910in}}%
\pgfpathlineto{\pgfqpoint{1.279220in}{0.852736in}}%
\pgfpathlineto{\pgfqpoint{1.308099in}{0.795092in}}%
\pgfpathlineto{\pgfqpoint{1.336979in}{0.845227in}}%
\pgfpathlineto{\pgfqpoint{1.365859in}{0.841394in}}%
\pgfpathlineto{\pgfqpoint{1.394739in}{0.832926in}}%
\pgfpathlineto{\pgfqpoint{1.423619in}{0.853066in}}%
\pgfpathlineto{\pgfqpoint{1.452499in}{0.773483in}}%
\pgfpathlineto{\pgfqpoint{1.481378in}{0.751949in}}%
\pgfpathlineto{\pgfqpoint{1.510258in}{0.765318in}}%
\pgfpathlineto{\pgfqpoint{1.539138in}{0.844560in}}%
\pgfpathlineto{\pgfqpoint{1.568018in}{0.829533in}}%
\pgfpathlineto{\pgfqpoint{1.596898in}{0.756351in}}%
\pgfpathlineto{\pgfqpoint{1.625778in}{0.822355in}}%
\pgfpathlineto{\pgfqpoint{1.654657in}{0.741573in}}%
\pgfpathlineto{\pgfqpoint{1.856816in}{0.741573in}}%
\pgfpathlineto{\pgfqpoint{1.885696in}{0.744262in}}%
\pgfpathlineto{\pgfqpoint{1.914576in}{0.781268in}}%
\pgfpathlineto{\pgfqpoint{1.943456in}{0.856811in}}%
\pgfpathlineto{\pgfqpoint{1.972336in}{0.807379in}}%
\pgfpathlineto{\pgfqpoint{2.001215in}{0.749033in}}%
\pgfpathlineto{\pgfqpoint{2.030095in}{0.743378in}}%
\pgfpathlineto{\pgfqpoint{2.058975in}{0.783496in}}%
\pgfpathlineto{\pgfqpoint{2.087855in}{0.863050in}}%
\pgfpathlineto{\pgfqpoint{2.116735in}{0.861013in}}%
\pgfpathlineto{\pgfqpoint{2.145615in}{0.866762in}}%
\pgfpathlineto{\pgfqpoint{2.174494in}{0.815188in}}%
\pgfpathlineto{\pgfqpoint{2.203374in}{0.850725in}}%
\pgfpathlineto{\pgfqpoint{2.232254in}{0.826511in}}%
\pgfpathlineto{\pgfqpoint{2.261134in}{0.758799in}}%
\pgfpathlineto{\pgfqpoint{2.290014in}{0.746973in}}%
\pgfpathlineto{\pgfqpoint{2.318894in}{0.797175in}}%
\pgfpathlineto{\pgfqpoint{2.347773in}{0.859225in}}%
\pgfpathlineto{\pgfqpoint{2.376653in}{0.799846in}}%
\pgfpathlineto{\pgfqpoint{2.405533in}{0.815926in}}%
\pgfpathlineto{\pgfqpoint{2.434413in}{0.855255in}}%
\pgfpathlineto{\pgfqpoint{2.463293in}{0.777738in}}%
\pgfpathlineto{\pgfqpoint{2.492173in}{0.836242in}}%
\pgfpathlineto{\pgfqpoint{2.521052in}{0.741573in}}%
\pgfpathlineto{\pgfqpoint{2.636572in}{0.741573in}}%
\pgfpathlineto{\pgfqpoint{2.665452in}{1.369728in}}%
\pgfpathlineto{\pgfqpoint{2.694331in}{0.754241in}}%
\pgfpathlineto{\pgfqpoint{2.723211in}{0.824467in}}%
\pgfpathlineto{\pgfqpoint{2.752091in}{0.848272in}}%
\pgfpathlineto{\pgfqpoint{2.780971in}{0.768588in}}%
\pgfpathlineto{\pgfqpoint{2.809851in}{0.840375in}}%
\pgfpathlineto{\pgfqpoint{2.838731in}{0.741573in}}%
\pgfpathlineto{\pgfqpoint{3.098649in}{0.741573in}}%
\pgfpathlineto{\pgfqpoint{3.127529in}{1.604347in}}%
\pgfpathlineto{\pgfqpoint{3.156409in}{0.747737in}}%
\pgfpathlineto{\pgfqpoint{3.214168in}{0.858222in}}%
\pgfpathlineto{\pgfqpoint{3.243048in}{0.785844in}}%
\pgfpathlineto{\pgfqpoint{3.271928in}{0.744878in}}%
\pgfpathlineto{\pgfqpoint{3.300808in}{0.741573in}}%
\pgfpathlineto{\pgfqpoint{5.120237in}{0.741573in}}%
\pgfpathlineto{\pgfqpoint{5.120237in}{0.741573in}}%
\pgfusepath{stroke}%
\end{pgfscope}%
\begin{pgfscope}%
\pgfpathrectangle{\pgfqpoint{0.709565in}{0.549073in}}{\pgfqpoint{4.518250in}{2.310000in}}%
\pgfusepath{clip}%
\pgfsetrectcap%
\pgfsetroundjoin%
\pgfsetlinewidth{1.505625pt}%
\definecolor{currentstroke}{rgb}{0.996078,0.380392,0.000000}%
\pgfsetstrokecolor{currentstroke}%
\pgfsetdash{}{0pt}%
\pgfpathmoveto{\pgfqpoint{0.817142in}{1.656369in}}%
\pgfpathlineto{\pgfqpoint{0.846022in}{1.740196in}}%
\pgfpathlineto{\pgfqpoint{0.874902in}{1.652190in}}%
\pgfpathlineto{\pgfqpoint{0.903782in}{1.468997in}}%
\pgfpathlineto{\pgfqpoint{0.932662in}{1.307136in}}%
\pgfpathlineto{\pgfqpoint{0.961541in}{1.160019in}}%
\pgfpathlineto{\pgfqpoint{0.990421in}{1.050556in}}%
\pgfpathlineto{\pgfqpoint{1.019301in}{1.045478in}}%
\pgfpathlineto{\pgfqpoint{1.048181in}{0.967052in}}%
\pgfpathlineto{\pgfqpoint{1.077061in}{0.859350in}}%
\pgfpathlineto{\pgfqpoint{1.105941in}{0.880620in}}%
\pgfpathlineto{\pgfqpoint{1.134820in}{0.888290in}}%
\pgfpathlineto{\pgfqpoint{1.163700in}{0.903939in}}%
\pgfpathlineto{\pgfqpoint{1.221460in}{0.870810in}}%
\pgfpathlineto{\pgfqpoint{1.250340in}{0.842105in}}%
\pgfpathlineto{\pgfqpoint{1.279220in}{0.764331in}}%
\pgfpathlineto{\pgfqpoint{1.308099in}{0.778972in}}%
\pgfpathlineto{\pgfqpoint{1.336979in}{0.855710in}}%
\pgfpathlineto{\pgfqpoint{1.365859in}{0.817506in}}%
\pgfpathlineto{\pgfqpoint{1.394739in}{0.812527in}}%
\pgfpathlineto{\pgfqpoint{1.423619in}{0.857914in}}%
\pgfpathlineto{\pgfqpoint{1.452499in}{0.783858in}}%
\pgfpathlineto{\pgfqpoint{1.481378in}{0.746165in}}%
\pgfpathlineto{\pgfqpoint{1.510258in}{0.758658in}}%
\pgfpathlineto{\pgfqpoint{1.539138in}{0.834201in}}%
\pgfpathlineto{\pgfqpoint{1.568018in}{0.840605in}}%
\pgfpathlineto{\pgfqpoint{1.596898in}{0.762400in}}%
\pgfpathlineto{\pgfqpoint{1.625778in}{0.852048in}}%
\pgfpathlineto{\pgfqpoint{1.654657in}{0.741573in}}%
\pgfpathlineto{\pgfqpoint{1.885696in}{0.741573in}}%
\pgfpathlineto{\pgfqpoint{1.914576in}{1.092946in}}%
\pgfpathlineto{\pgfqpoint{1.943456in}{0.749998in}}%
\pgfpathlineto{\pgfqpoint{1.972336in}{0.811706in}}%
\pgfpathlineto{\pgfqpoint{2.001215in}{0.855764in}}%
\pgfpathlineto{\pgfqpoint{2.030095in}{0.791335in}}%
\pgfpathlineto{\pgfqpoint{2.058975in}{0.828316in}}%
\pgfpathlineto{\pgfqpoint{2.087855in}{0.850009in}}%
\pgfpathlineto{\pgfqpoint{2.116735in}{0.823023in}}%
\pgfpathlineto{\pgfqpoint{2.145615in}{0.956713in}}%
\pgfpathlineto{\pgfqpoint{2.174494in}{0.914082in}}%
\pgfpathlineto{\pgfqpoint{2.203374in}{0.834831in}}%
\pgfpathlineto{\pgfqpoint{2.232254in}{0.856765in}}%
\pgfpathlineto{\pgfqpoint{2.261134in}{0.779008in}}%
\pgfpathlineto{\pgfqpoint{2.290014in}{0.990795in}}%
\pgfpathlineto{\pgfqpoint{2.318894in}{0.741573in}}%
\pgfpathlineto{\pgfqpoint{2.347773in}{0.741573in}}%
\pgfpathlineto{\pgfqpoint{2.376653in}{0.745683in}}%
\pgfpathlineto{\pgfqpoint{2.405533in}{0.791106in}}%
\pgfpathlineto{\pgfqpoint{2.434413in}{0.859006in}}%
\pgfpathlineto{\pgfqpoint{2.463293in}{0.796337in}}%
\pgfpathlineto{\pgfqpoint{2.492173in}{0.746596in}}%
\pgfpathlineto{\pgfqpoint{2.521052in}{1.436452in}}%
\pgfpathlineto{\pgfqpoint{2.549932in}{0.741573in}}%
\pgfpathlineto{\pgfqpoint{2.578812in}{0.741573in}}%
\pgfpathlineto{\pgfqpoint{2.607692in}{0.745633in}}%
\pgfpathlineto{\pgfqpoint{2.636572in}{0.790807in}}%
\pgfpathlineto{\pgfqpoint{2.665452in}{0.858983in}}%
\pgfpathlineto{\pgfqpoint{2.694331in}{0.796648in}}%
\pgfpathlineto{\pgfqpoint{2.723211in}{0.746655in}}%
\pgfpathlineto{\pgfqpoint{2.752091in}{0.741573in}}%
\pgfpathlineto{\pgfqpoint{3.185289in}{0.741573in}}%
\pgfpathlineto{\pgfqpoint{3.214168in}{1.622538in}}%
\pgfpathlineto{\pgfqpoint{3.243048in}{0.747661in}}%
\pgfpathlineto{\pgfqpoint{3.271928in}{0.801676in}}%
\pgfpathlineto{\pgfqpoint{3.300808in}{0.858294in}}%
\pgfpathlineto{\pgfqpoint{3.329688in}{0.786161in}}%
\pgfpathlineto{\pgfqpoint{3.358568in}{0.744924in}}%
\pgfpathlineto{\pgfqpoint{3.387447in}{0.741573in}}%
\pgfpathlineto{\pgfqpoint{5.120237in}{0.741573in}}%
\pgfpathlineto{\pgfqpoint{5.120237in}{0.741573in}}%
\pgfusepath{stroke}%
\end{pgfscope}%
\begin{pgfscope}%
\pgfpathrectangle{\pgfqpoint{0.709565in}{0.549073in}}{\pgfqpoint{4.518250in}{2.310000in}}%
\pgfusepath{clip}%
\pgfsetrectcap%
\pgfsetroundjoin%
\pgfsetlinewidth{1.505625pt}%
\definecolor{currentstroke}{rgb}{1.000000,0.690196,0.000000}%
\pgfsetstrokecolor{currentstroke}%
\pgfsetdash{}{0pt}%
\pgfpathmoveto{\pgfqpoint{0.817142in}{1.646862in}}%
\pgfpathlineto{\pgfqpoint{0.846022in}{1.626645in}}%
\pgfpathlineto{\pgfqpoint{0.874902in}{1.464190in}}%
\pgfpathlineto{\pgfqpoint{0.903782in}{1.366369in}}%
\pgfpathlineto{\pgfqpoint{0.932662in}{1.330422in}}%
\pgfpathlineto{\pgfqpoint{0.961541in}{1.194357in}}%
\pgfpathlineto{\pgfqpoint{0.990421in}{1.096185in}}%
\pgfpathlineto{\pgfqpoint{1.019301in}{1.033924in}}%
\pgfpathlineto{\pgfqpoint{1.048181in}{1.019597in}}%
\pgfpathlineto{\pgfqpoint{1.077061in}{1.010170in}}%
\pgfpathlineto{\pgfqpoint{1.105941in}{0.889392in}}%
\pgfpathlineto{\pgfqpoint{1.134820in}{0.794858in}}%
\pgfpathlineto{\pgfqpoint{1.163700in}{0.846561in}}%
\pgfpathlineto{\pgfqpoint{1.192580in}{0.879043in}}%
\pgfpathlineto{\pgfqpoint{1.221460in}{0.906794in}}%
\pgfpathlineto{\pgfqpoint{1.250340in}{0.913907in}}%
\pgfpathlineto{\pgfqpoint{1.279220in}{0.823406in}}%
\pgfpathlineto{\pgfqpoint{1.308099in}{0.751954in}}%
\pgfpathlineto{\pgfqpoint{1.336979in}{0.749029in}}%
\pgfpathlineto{\pgfqpoint{1.365859in}{0.805689in}}%
\pgfpathlineto{\pgfqpoint{1.394739in}{0.858031in}}%
\pgfpathlineto{\pgfqpoint{1.423619in}{0.800575in}}%
\pgfpathlineto{\pgfqpoint{1.452499in}{0.838462in}}%
\pgfpathlineto{\pgfqpoint{1.481378in}{0.842453in}}%
\pgfpathlineto{\pgfqpoint{1.510258in}{0.804039in}}%
\pgfpathlineto{\pgfqpoint{1.539138in}{0.858553in}}%
\pgfpathlineto{\pgfqpoint{1.596898in}{0.748198in}}%
\pgfpathlineto{\pgfqpoint{1.625778in}{0.914594in}}%
\pgfpathlineto{\pgfqpoint{1.654657in}{0.741573in}}%
\pgfpathlineto{\pgfqpoint{1.770177in}{0.741573in}}%
\pgfpathlineto{\pgfqpoint{1.799057in}{0.820710in}}%
\pgfpathlineto{\pgfqpoint{1.827936in}{0.770354in}}%
\pgfpathlineto{\pgfqpoint{1.856816in}{0.849962in}}%
\pgfpathlineto{\pgfqpoint{1.885696in}{0.821864in}}%
\pgfpathlineto{\pgfqpoint{1.914576in}{0.753334in}}%
\pgfpathlineto{\pgfqpoint{1.943456in}{0.745308in}}%
\pgfpathlineto{\pgfqpoint{1.972336in}{0.786720in}}%
\pgfpathlineto{\pgfqpoint{2.001215in}{0.859197in}}%
\pgfpathlineto{\pgfqpoint{2.030095in}{0.821657in}}%
\pgfpathlineto{\pgfqpoint{2.058975in}{0.845949in}}%
\pgfpathlineto{\pgfqpoint{2.087855in}{0.836969in}}%
\pgfpathlineto{\pgfqpoint{2.116735in}{0.791925in}}%
\pgfpathlineto{\pgfqpoint{2.145615in}{0.896650in}}%
\pgfpathlineto{\pgfqpoint{2.174494in}{0.932736in}}%
\pgfpathlineto{\pgfqpoint{2.203374in}{0.813360in}}%
\pgfpathlineto{\pgfqpoint{2.232254in}{0.748292in}}%
\pgfpathlineto{\pgfqpoint{2.261134in}{1.330236in}}%
\pgfpathlineto{\pgfqpoint{2.290014in}{0.741573in}}%
\pgfpathlineto{\pgfqpoint{2.376653in}{0.741573in}}%
\pgfpathlineto{\pgfqpoint{2.405533in}{1.133987in}}%
\pgfpathlineto{\pgfqpoint{2.434413in}{0.757350in}}%
\pgfpathlineto{\pgfqpoint{2.463293in}{0.831681in}}%
\pgfpathlineto{\pgfqpoint{2.492173in}{0.842806in}}%
\pgfpathlineto{\pgfqpoint{2.521052in}{0.764093in}}%
\pgfpathlineto{\pgfqpoint{2.549932in}{0.749619in}}%
\pgfpathlineto{\pgfqpoint{2.578812in}{0.806013in}}%
\pgfpathlineto{\pgfqpoint{2.607692in}{0.857218in}}%
\pgfpathlineto{\pgfqpoint{2.636572in}{0.782367in}}%
\pgfpathlineto{\pgfqpoint{2.665452in}{0.837148in}}%
\pgfpathlineto{\pgfqpoint{2.694331in}{0.741573in}}%
\pgfpathlineto{\pgfqpoint{3.243048in}{0.741573in}}%
\pgfpathlineto{\pgfqpoint{3.271928in}{0.743215in}}%
\pgfpathlineto{\pgfqpoint{3.300808in}{0.771824in}}%
\pgfpathlineto{\pgfqpoint{3.329688in}{0.851227in}}%
\pgfpathlineto{\pgfqpoint{3.358568in}{0.819757in}}%
\pgfpathlineto{\pgfqpoint{3.387447in}{0.752539in}}%
\pgfpathlineto{\pgfqpoint{3.416327in}{1.030527in}}%
\pgfpathlineto{\pgfqpoint{3.445207in}{0.741573in}}%
\pgfpathlineto{\pgfqpoint{5.120237in}{0.741573in}}%
\pgfpathlineto{\pgfqpoint{5.120237in}{0.741573in}}%
\pgfusepath{stroke}%
\end{pgfscope}%
\begin{pgfscope}%
\pgfsetrectcap%
\pgfsetmiterjoin%
\pgfsetlinewidth{0.803000pt}%
\definecolor{currentstroke}{rgb}{0.000000,0.000000,0.000000}%
\pgfsetstrokecolor{currentstroke}%
\pgfsetdash{}{0pt}%
\pgfpathmoveto{\pgfqpoint{0.709565in}{0.549073in}}%
\pgfpathlineto{\pgfqpoint{0.709565in}{2.859073in}}%
\pgfusepath{stroke}%
\end{pgfscope}%
\begin{pgfscope}%
\pgfsetrectcap%
\pgfsetmiterjoin%
\pgfsetlinewidth{0.803000pt}%
\definecolor{currentstroke}{rgb}{0.000000,0.000000,0.000000}%
\pgfsetstrokecolor{currentstroke}%
\pgfsetdash{}{0pt}%
\pgfpathmoveto{\pgfqpoint{5.227815in}{0.549073in}}%
\pgfpathlineto{\pgfqpoint{5.227815in}{2.859073in}}%
\pgfusepath{stroke}%
\end{pgfscope}%
\begin{pgfscope}%
\pgfsetrectcap%
\pgfsetmiterjoin%
\pgfsetlinewidth{0.803000pt}%
\definecolor{currentstroke}{rgb}{0.000000,0.000000,0.000000}%
\pgfsetstrokecolor{currentstroke}%
\pgfsetdash{}{0pt}%
\pgfpathmoveto{\pgfqpoint{0.709565in}{0.549073in}}%
\pgfpathlineto{\pgfqpoint{5.227815in}{0.549073in}}%
\pgfusepath{stroke}%
\end{pgfscope}%
\begin{pgfscope}%
\pgfsetrectcap%
\pgfsetmiterjoin%
\pgfsetlinewidth{0.803000pt}%
\definecolor{currentstroke}{rgb}{0.000000,0.000000,0.000000}%
\pgfsetstrokecolor{currentstroke}%
\pgfsetdash{}{0pt}%
\pgfpathmoveto{\pgfqpoint{0.709565in}{2.859073in}}%
\pgfpathlineto{\pgfqpoint{5.227815in}{2.859073in}}%
\pgfusepath{stroke}%
\end{pgfscope}%
\begin{pgfscope}%
\pgfsetbuttcap%
\pgfsetmiterjoin%
\definecolor{currentfill}{rgb}{1.000000,1.000000,1.000000}%
\pgfsetfillcolor{currentfill}%
\pgfsetfillopacity{0.800000}%
\pgfsetlinewidth{1.003750pt}%
\definecolor{currentstroke}{rgb}{0.800000,0.800000,0.800000}%
\pgfsetstrokecolor{currentstroke}%
\pgfsetstrokeopacity{0.800000}%
\pgfsetdash{}{0pt}%
\pgfpathmoveto{\pgfqpoint{3.859979in}{1.563704in}}%
\pgfpathlineto{\pgfqpoint{5.144482in}{1.563704in}}%
\pgfpathlineto{\pgfqpoint{5.144482in}{2.775739in}}%
\pgfpathlineto{\pgfqpoint{3.859979in}{2.775739in}}%
\pgfpathlineto{\pgfqpoint{3.859979in}{1.563704in}}%
\pgfpathclose%
\pgfusepath{stroke,fill}%
\end{pgfscope}%
\begin{pgfscope}%
\pgfsetrectcap%
\pgfsetroundjoin%
\pgfsetlinewidth{1.505625pt}%
\definecolor{currentstroke}{rgb}{0.392157,0.560784,1.000000}%
\pgfsetstrokecolor{currentstroke}%
\pgfsetdash{}{0pt}%
\pgfpathmoveto{\pgfqpoint{3.926645in}{2.650739in}}%
\pgfpathlineto{\pgfqpoint{4.093312in}{2.650739in}}%
\pgfpathlineto{\pgfqpoint{4.259979in}{2.650739in}}%
\pgfusepath{stroke}%
\end{pgfscope}%
\begin{pgfscope}%
\definecolor{textcolor}{rgb}{0.000000,0.000000,0.000000}%
\pgfsetstrokecolor{textcolor}%
\pgfsetfillcolor{textcolor}%
\pgftext[x=4.393312in,y=2.592406in,left,base]{\color{textcolor}{\rmfamily\fontsize{12.000000}{14.400000}\selectfont\catcode`\^=\active\def^{\ifmmode\sp\else\^{}\fi}\catcode`\%=\active\def%{\%}untrained}}%
\end{pgfscope}%
\begin{pgfscope}%
\pgfsetrectcap%
\pgfsetroundjoin%
\pgfsetlinewidth{1.505625pt}%
\definecolor{currentstroke}{rgb}{0.470588,0.368627,0.941176}%
\pgfsetstrokecolor{currentstroke}%
\pgfsetdash{}{0pt}%
\pgfpathmoveto{\pgfqpoint{3.926645in}{2.418332in}}%
\pgfpathlineto{\pgfqpoint{4.093312in}{2.418332in}}%
\pgfpathlineto{\pgfqpoint{4.259979in}{2.418332in}}%
\pgfusepath{stroke}%
\end{pgfscope}%
\begin{pgfscope}%
\definecolor{textcolor}{rgb}{0.000000,0.000000,0.000000}%
\pgfsetstrokecolor{textcolor}%
\pgfsetfillcolor{textcolor}%
\pgftext[x=4.393312in,y=2.359999in,left,base]{\color{textcolor}{\rmfamily\fontsize{12.000000}{14.400000}\selectfont\catcode`\^=\active\def^{\ifmmode\sp\else\^{}\fi}\catcode`\%=\active\def%{\%}epoch $2$}}%
\end{pgfscope}%
\begin{pgfscope}%
\pgfsetrectcap%
\pgfsetroundjoin%
\pgfsetlinewidth{1.505625pt}%
\definecolor{currentstroke}{rgb}{0.862745,0.149020,0.498039}%
\pgfsetstrokecolor{currentstroke}%
\pgfsetdash{}{0pt}%
\pgfpathmoveto{\pgfqpoint{3.926645in}{2.185925in}}%
\pgfpathlineto{\pgfqpoint{4.093312in}{2.185925in}}%
\pgfpathlineto{\pgfqpoint{4.259979in}{2.185925in}}%
\pgfusepath{stroke}%
\end{pgfscope}%
\begin{pgfscope}%
\definecolor{textcolor}{rgb}{0.000000,0.000000,0.000000}%
\pgfsetstrokecolor{textcolor}%
\pgfsetfillcolor{textcolor}%
\pgftext[x=4.393312in,y=2.127592in,left,base]{\color{textcolor}{\rmfamily\fontsize{12.000000}{14.400000}\selectfont\catcode`\^=\active\def^{\ifmmode\sp\else\^{}\fi}\catcode`\%=\active\def%{\%}epoch $4$}}%
\end{pgfscope}%
\begin{pgfscope}%
\pgfsetrectcap%
\pgfsetroundjoin%
\pgfsetlinewidth{1.505625pt}%
\definecolor{currentstroke}{rgb}{0.996078,0.380392,0.000000}%
\pgfsetstrokecolor{currentstroke}%
\pgfsetdash{}{0pt}%
\pgfpathmoveto{\pgfqpoint{3.926645in}{1.953518in}}%
\pgfpathlineto{\pgfqpoint{4.093312in}{1.953518in}}%
\pgfpathlineto{\pgfqpoint{4.259979in}{1.953518in}}%
\pgfusepath{stroke}%
\end{pgfscope}%
\begin{pgfscope}%
\definecolor{textcolor}{rgb}{0.000000,0.000000,0.000000}%
\pgfsetstrokecolor{textcolor}%
\pgfsetfillcolor{textcolor}%
\pgftext[x=4.393312in,y=1.895185in,left,base]{\color{textcolor}{\rmfamily\fontsize{12.000000}{14.400000}\selectfont\catcode`\^=\active\def^{\ifmmode\sp\else\^{}\fi}\catcode`\%=\active\def%{\%}epoch $6$}}%
\end{pgfscope}%
\begin{pgfscope}%
\pgfsetrectcap%
\pgfsetroundjoin%
\pgfsetlinewidth{1.505625pt}%
\definecolor{currentstroke}{rgb}{1.000000,0.690196,0.000000}%
\pgfsetstrokecolor{currentstroke}%
\pgfsetdash{}{0pt}%
\pgfpathmoveto{\pgfqpoint{3.926645in}{1.721111in}}%
\pgfpathlineto{\pgfqpoint{4.093312in}{1.721111in}}%
\pgfpathlineto{\pgfqpoint{4.259979in}{1.721111in}}%
\pgfusepath{stroke}%
\end{pgfscope}%
\begin{pgfscope}%
\definecolor{textcolor}{rgb}{0.000000,0.000000,0.000000}%
\pgfsetstrokecolor{textcolor}%
\pgfsetfillcolor{textcolor}%
\pgftext[x=4.393312in,y=1.662778in,left,base]{\color{textcolor}{\rmfamily\fontsize{12.000000}{14.400000}\selectfont\catcode`\^=\active\def^{\ifmmode\sp\else\^{}\fi}\catcode`\%=\active\def%{\%}epoch $8$}}%
\end{pgfscope}%
\end{pgfpicture}%
\makeatother%
\endgroup%

    \end{minipage}\hfill%
    \begin{minipage}[c]{.49\linewidth}
        \centering
        %% Creator: Matplotlib, PGF backend
%%
%% To include the figure in your LaTeX document, write
%%   \input{<filename>.pgf}
%%
%% Make sure the required packages are loaded in your preamble
%%   \usepackage{pgf}
%%
%% Also ensure that all the required font packages are loaded; for instance,
%% the lmodern package is sometimes necessary when using math font.
%%   \usepackage{lmodern}
%%
%% Figures using additional raster images can only be included by \input if
%% they are in the same directory as the main LaTeX file. For loading figures
%% from other directories you can use the `import` package
%%   \usepackage{import}
%%
%% and then include the figures with
%%   \import{<path to file>}{<filename>.pgf}
%%
%% Matplotlib used the following preamble
%%   \def\mathdefault#1{#1}
%%   \everymath=\expandafter{\the\everymath\displaystyle}
%%   \IfFileExists{scrextend.sty}{
%%     \usepackage[fontsize=12.000000pt]{scrextend}
%%   }{
%%     \renewcommand{\normalsize}{\fontsize{12.000000}{14.400000}\selectfont}
%%     \normalsize
%%   }
%%   
%%   \ifdefined\pdftexversion\else  % non-pdftex case.
%%     \usepackage{fontspec}
%%     \setmainfont{DejaVuSerif.ttf}[Path=\detokenize{/opt/hostedtoolcache/Python/3.12.9/x64/lib/python3.12/site-packages/matplotlib/mpl-data/fonts/ttf/}]
%%     \setsansfont{DejaVuSans.ttf}[Path=\detokenize{/opt/hostedtoolcache/Python/3.12.9/x64/lib/python3.12/site-packages/matplotlib/mpl-data/fonts/ttf/}]
%%     \setmonofont{DejaVuSansMono.ttf}[Path=\detokenize{/opt/hostedtoolcache/Python/3.12.9/x64/lib/python3.12/site-packages/matplotlib/mpl-data/fonts/ttf/}]
%%   \fi
%%   \makeatletter\@ifpackageloaded{underscore}{}{\usepackage[strings]{underscore}}\makeatother
%%
\begingroup%
\makeatletter%
\begin{pgfpicture}%
\pgfpathrectangle{\pgfpointorigin}{\pgfqpoint{3.116046in}{2.959073in}}%
\pgfusepath{use as bounding box, clip}%
\begin{pgfscope}%
\pgfsetbuttcap%
\pgfsetmiterjoin%
\definecolor{currentfill}{rgb}{1.000000,1.000000,1.000000}%
\pgfsetfillcolor{currentfill}%
\pgfsetlinewidth{0.000000pt}%
\definecolor{currentstroke}{rgb}{1.000000,1.000000,1.000000}%
\pgfsetstrokecolor{currentstroke}%
\pgfsetdash{}{0pt}%
\pgfpathmoveto{\pgfqpoint{0.000000in}{-0.000000in}}%
\pgfpathlineto{\pgfqpoint{3.116046in}{-0.000000in}}%
\pgfpathlineto{\pgfqpoint{3.116046in}{2.959073in}}%
\pgfpathlineto{\pgfqpoint{0.000000in}{2.959073in}}%
\pgfpathlineto{\pgfqpoint{0.000000in}{-0.000000in}}%
\pgfpathclose%
\pgfusepath{fill}%
\end{pgfscope}%
\begin{pgfscope}%
\pgfsetbuttcap%
\pgfsetmiterjoin%
\definecolor{currentfill}{rgb}{1.000000,1.000000,1.000000}%
\pgfsetfillcolor{currentfill}%
\pgfsetlinewidth{0.000000pt}%
\definecolor{currentstroke}{rgb}{0.000000,0.000000,0.000000}%
\pgfsetstrokecolor{currentstroke}%
\pgfsetstrokeopacity{0.000000}%
\pgfsetdash{}{0pt}%
\pgfpathmoveto{\pgfqpoint{0.691046in}{0.549073in}}%
\pgfpathlineto{\pgfqpoint{3.016046in}{0.549073in}}%
\pgfpathlineto{\pgfqpoint{3.016046in}{2.859073in}}%
\pgfpathlineto{\pgfqpoint{0.691046in}{2.859073in}}%
\pgfpathlineto{\pgfqpoint{0.691046in}{0.549073in}}%
\pgfpathclose%
\pgfusepath{fill}%
\end{pgfscope}%
\begin{pgfscope}%
\pgfpathrectangle{\pgfqpoint{0.691046in}{0.549073in}}{\pgfqpoint{2.325000in}{2.310000in}}%
\pgfusepath{clip}%
\pgfsetrectcap%
\pgfsetroundjoin%
\pgfsetlinewidth{0.250937pt}%
\definecolor{currentstroke}{rgb}{0.000000,0.000000,0.000000}%
\pgfsetstrokecolor{currentstroke}%
\pgfsetstrokeopacity{0.200000}%
\pgfsetdash{}{0pt}%
\pgfpathmoveto{\pgfqpoint{0.746403in}{0.549073in}}%
\pgfpathlineto{\pgfqpoint{0.746403in}{2.859073in}}%
\pgfusepath{stroke}%
\end{pgfscope}%
\begin{pgfscope}%
\pgfsetbuttcap%
\pgfsetroundjoin%
\definecolor{currentfill}{rgb}{0.000000,0.000000,0.000000}%
\pgfsetfillcolor{currentfill}%
\pgfsetlinewidth{0.803000pt}%
\definecolor{currentstroke}{rgb}{0.000000,0.000000,0.000000}%
\pgfsetstrokecolor{currentstroke}%
\pgfsetdash{}{0pt}%
\pgfsys@defobject{currentmarker}{\pgfqpoint{0.000000in}{-0.048611in}}{\pgfqpoint{0.000000in}{0.000000in}}{%
\pgfpathmoveto{\pgfqpoint{0.000000in}{0.000000in}}%
\pgfpathlineto{\pgfqpoint{0.000000in}{-0.048611in}}%
\pgfusepath{stroke,fill}%
}%
\begin{pgfscope}%
\pgfsys@transformshift{0.746403in}{0.549073in}%
\pgfsys@useobject{currentmarker}{}%
\end{pgfscope}%
\end{pgfscope}%
\begin{pgfscope}%
\definecolor{textcolor}{rgb}{0.000000,0.000000,0.000000}%
\pgfsetstrokecolor{textcolor}%
\pgfsetfillcolor{textcolor}%
\pgftext[x=0.746403in,y=0.451851in,,top]{\color{textcolor}{\rmfamily\fontsize{12.000000}{14.400000}\selectfont\catcode`\^=\active\def^{\ifmmode\sp\else\^{}\fi}\catcode`\%=\active\def%{\%}0}}%
\end{pgfscope}%
\begin{pgfscope}%
\pgfpathrectangle{\pgfqpoint{0.691046in}{0.549073in}}{\pgfqpoint{2.325000in}{2.310000in}}%
\pgfusepath{clip}%
\pgfsetrectcap%
\pgfsetroundjoin%
\pgfsetlinewidth{0.250937pt}%
\definecolor{currentstroke}{rgb}{0.000000,0.000000,0.000000}%
\pgfsetstrokecolor{currentstroke}%
\pgfsetstrokeopacity{0.200000}%
\pgfsetdash{}{0pt}%
\pgfpathmoveto{\pgfqpoint{1.299974in}{0.549073in}}%
\pgfpathlineto{\pgfqpoint{1.299974in}{2.859073in}}%
\pgfusepath{stroke}%
\end{pgfscope}%
\begin{pgfscope}%
\pgfsetbuttcap%
\pgfsetroundjoin%
\definecolor{currentfill}{rgb}{0.000000,0.000000,0.000000}%
\pgfsetfillcolor{currentfill}%
\pgfsetlinewidth{0.803000pt}%
\definecolor{currentstroke}{rgb}{0.000000,0.000000,0.000000}%
\pgfsetstrokecolor{currentstroke}%
\pgfsetdash{}{0pt}%
\pgfsys@defobject{currentmarker}{\pgfqpoint{0.000000in}{-0.048611in}}{\pgfqpoint{0.000000in}{0.000000in}}{%
\pgfpathmoveto{\pgfqpoint{0.000000in}{0.000000in}}%
\pgfpathlineto{\pgfqpoint{0.000000in}{-0.048611in}}%
\pgfusepath{stroke,fill}%
}%
\begin{pgfscope}%
\pgfsys@transformshift{1.299974in}{0.549073in}%
\pgfsys@useobject{currentmarker}{}%
\end{pgfscope}%
\end{pgfscope}%
\begin{pgfscope}%
\definecolor{textcolor}{rgb}{0.000000,0.000000,0.000000}%
\pgfsetstrokecolor{textcolor}%
\pgfsetfillcolor{textcolor}%
\pgftext[x=1.299974in,y=0.451851in,,top]{\color{textcolor}{\rmfamily\fontsize{12.000000}{14.400000}\selectfont\catcode`\^=\active\def^{\ifmmode\sp\else\^{}\fi}\catcode`\%=\active\def%{\%}2}}%
\end{pgfscope}%
\begin{pgfscope}%
\pgfpathrectangle{\pgfqpoint{0.691046in}{0.549073in}}{\pgfqpoint{2.325000in}{2.310000in}}%
\pgfusepath{clip}%
\pgfsetrectcap%
\pgfsetroundjoin%
\pgfsetlinewidth{0.250937pt}%
\definecolor{currentstroke}{rgb}{0.000000,0.000000,0.000000}%
\pgfsetstrokecolor{currentstroke}%
\pgfsetstrokeopacity{0.200000}%
\pgfsetdash{}{0pt}%
\pgfpathmoveto{\pgfqpoint{1.853546in}{0.549073in}}%
\pgfpathlineto{\pgfqpoint{1.853546in}{2.859073in}}%
\pgfusepath{stroke}%
\end{pgfscope}%
\begin{pgfscope}%
\pgfsetbuttcap%
\pgfsetroundjoin%
\definecolor{currentfill}{rgb}{0.000000,0.000000,0.000000}%
\pgfsetfillcolor{currentfill}%
\pgfsetlinewidth{0.803000pt}%
\definecolor{currentstroke}{rgb}{0.000000,0.000000,0.000000}%
\pgfsetstrokecolor{currentstroke}%
\pgfsetdash{}{0pt}%
\pgfsys@defobject{currentmarker}{\pgfqpoint{0.000000in}{-0.048611in}}{\pgfqpoint{0.000000in}{0.000000in}}{%
\pgfpathmoveto{\pgfqpoint{0.000000in}{0.000000in}}%
\pgfpathlineto{\pgfqpoint{0.000000in}{-0.048611in}}%
\pgfusepath{stroke,fill}%
}%
\begin{pgfscope}%
\pgfsys@transformshift{1.853546in}{0.549073in}%
\pgfsys@useobject{currentmarker}{}%
\end{pgfscope}%
\end{pgfscope}%
\begin{pgfscope}%
\definecolor{textcolor}{rgb}{0.000000,0.000000,0.000000}%
\pgfsetstrokecolor{textcolor}%
\pgfsetfillcolor{textcolor}%
\pgftext[x=1.853546in,y=0.451851in,,top]{\color{textcolor}{\rmfamily\fontsize{12.000000}{14.400000}\selectfont\catcode`\^=\active\def^{\ifmmode\sp\else\^{}\fi}\catcode`\%=\active\def%{\%}4}}%
\end{pgfscope}%
\begin{pgfscope}%
\pgfpathrectangle{\pgfqpoint{0.691046in}{0.549073in}}{\pgfqpoint{2.325000in}{2.310000in}}%
\pgfusepath{clip}%
\pgfsetrectcap%
\pgfsetroundjoin%
\pgfsetlinewidth{0.250937pt}%
\definecolor{currentstroke}{rgb}{0.000000,0.000000,0.000000}%
\pgfsetstrokecolor{currentstroke}%
\pgfsetstrokeopacity{0.200000}%
\pgfsetdash{}{0pt}%
\pgfpathmoveto{\pgfqpoint{2.407117in}{0.549073in}}%
\pgfpathlineto{\pgfqpoint{2.407117in}{2.859073in}}%
\pgfusepath{stroke}%
\end{pgfscope}%
\begin{pgfscope}%
\pgfsetbuttcap%
\pgfsetroundjoin%
\definecolor{currentfill}{rgb}{0.000000,0.000000,0.000000}%
\pgfsetfillcolor{currentfill}%
\pgfsetlinewidth{0.803000pt}%
\definecolor{currentstroke}{rgb}{0.000000,0.000000,0.000000}%
\pgfsetstrokecolor{currentstroke}%
\pgfsetdash{}{0pt}%
\pgfsys@defobject{currentmarker}{\pgfqpoint{0.000000in}{-0.048611in}}{\pgfqpoint{0.000000in}{0.000000in}}{%
\pgfpathmoveto{\pgfqpoint{0.000000in}{0.000000in}}%
\pgfpathlineto{\pgfqpoint{0.000000in}{-0.048611in}}%
\pgfusepath{stroke,fill}%
}%
\begin{pgfscope}%
\pgfsys@transformshift{2.407117in}{0.549073in}%
\pgfsys@useobject{currentmarker}{}%
\end{pgfscope}%
\end{pgfscope}%
\begin{pgfscope}%
\definecolor{textcolor}{rgb}{0.000000,0.000000,0.000000}%
\pgfsetstrokecolor{textcolor}%
\pgfsetfillcolor{textcolor}%
\pgftext[x=2.407117in,y=0.451851in,,top]{\color{textcolor}{\rmfamily\fontsize{12.000000}{14.400000}\selectfont\catcode`\^=\active\def^{\ifmmode\sp\else\^{}\fi}\catcode`\%=\active\def%{\%}6}}%
\end{pgfscope}%
\begin{pgfscope}%
\pgfpathrectangle{\pgfqpoint{0.691046in}{0.549073in}}{\pgfqpoint{2.325000in}{2.310000in}}%
\pgfusepath{clip}%
\pgfsetrectcap%
\pgfsetroundjoin%
\pgfsetlinewidth{0.250937pt}%
\definecolor{currentstroke}{rgb}{0.000000,0.000000,0.000000}%
\pgfsetstrokecolor{currentstroke}%
\pgfsetstrokeopacity{0.200000}%
\pgfsetdash{}{0pt}%
\pgfpathmoveto{\pgfqpoint{2.960689in}{0.549073in}}%
\pgfpathlineto{\pgfqpoint{2.960689in}{2.859073in}}%
\pgfusepath{stroke}%
\end{pgfscope}%
\begin{pgfscope}%
\pgfsetbuttcap%
\pgfsetroundjoin%
\definecolor{currentfill}{rgb}{0.000000,0.000000,0.000000}%
\pgfsetfillcolor{currentfill}%
\pgfsetlinewidth{0.803000pt}%
\definecolor{currentstroke}{rgb}{0.000000,0.000000,0.000000}%
\pgfsetstrokecolor{currentstroke}%
\pgfsetdash{}{0pt}%
\pgfsys@defobject{currentmarker}{\pgfqpoint{0.000000in}{-0.048611in}}{\pgfqpoint{0.000000in}{0.000000in}}{%
\pgfpathmoveto{\pgfqpoint{0.000000in}{0.000000in}}%
\pgfpathlineto{\pgfqpoint{0.000000in}{-0.048611in}}%
\pgfusepath{stroke,fill}%
}%
\begin{pgfscope}%
\pgfsys@transformshift{2.960689in}{0.549073in}%
\pgfsys@useobject{currentmarker}{}%
\end{pgfscope}%
\end{pgfscope}%
\begin{pgfscope}%
\definecolor{textcolor}{rgb}{0.000000,0.000000,0.000000}%
\pgfsetstrokecolor{textcolor}%
\pgfsetfillcolor{textcolor}%
\pgftext[x=2.960689in,y=0.451851in,,top]{\color{textcolor}{\rmfamily\fontsize{12.000000}{14.400000}\selectfont\catcode`\^=\active\def^{\ifmmode\sp\else\^{}\fi}\catcode`\%=\active\def%{\%}8}}%
\end{pgfscope}%
\begin{pgfscope}%
\definecolor{textcolor}{rgb}{0.000000,0.000000,0.000000}%
\pgfsetstrokecolor{textcolor}%
\pgfsetfillcolor{textcolor}%
\pgftext[x=1.853546in,y=0.248148in,,top]{\color{textcolor}{\rmfamily\fontsize{12.000000}{14.400000}\selectfont\catcode`\^=\active\def^{\ifmmode\sp\else\^{}\fi}\catcode`\%=\active\def%{\%}epoch}}%
\end{pgfscope}%
\begin{pgfscope}%
\pgfpathrectangle{\pgfqpoint{0.691046in}{0.549073in}}{\pgfqpoint{2.325000in}{2.310000in}}%
\pgfusepath{clip}%
\pgfsetrectcap%
\pgfsetroundjoin%
\pgfsetlinewidth{0.250937pt}%
\definecolor{currentstroke}{rgb}{0.000000,0.000000,0.000000}%
\pgfsetstrokecolor{currentstroke}%
\pgfsetstrokeopacity{0.200000}%
\pgfsetdash{}{0pt}%
\pgfpathmoveto{\pgfqpoint{0.691046in}{0.553662in}}%
\pgfpathlineto{\pgfqpoint{3.016046in}{0.553662in}}%
\pgfusepath{stroke}%
\end{pgfscope}%
\begin{pgfscope}%
\pgfsetbuttcap%
\pgfsetroundjoin%
\definecolor{currentfill}{rgb}{0.000000,0.000000,0.000000}%
\pgfsetfillcolor{currentfill}%
\pgfsetlinewidth{0.803000pt}%
\definecolor{currentstroke}{rgb}{0.000000,0.000000,0.000000}%
\pgfsetstrokecolor{currentstroke}%
\pgfsetdash{}{0pt}%
\pgfsys@defobject{currentmarker}{\pgfqpoint{-0.048611in}{0.000000in}}{\pgfqpoint{-0.000000in}{0.000000in}}{%
\pgfpathmoveto{\pgfqpoint{-0.000000in}{0.000000in}}%
\pgfpathlineto{\pgfqpoint{-0.048611in}{0.000000in}}%
\pgfusepath{stroke,fill}%
}%
\begin{pgfscope}%
\pgfsys@transformshift{0.691046in}{0.553662in}%
\pgfsys@useobject{currentmarker}{}%
\end{pgfscope}%
\end{pgfscope}%
\begin{pgfscope}%
\definecolor{textcolor}{rgb}{0.000000,0.000000,0.000000}%
\pgfsetstrokecolor{textcolor}%
\pgfsetfillcolor{textcolor}%
\pgftext[x=0.303703in, y=0.495792in, left, base]{\color{textcolor}{\rmfamily\fontsize{12.000000}{14.400000}\selectfont\catcode`\^=\active\def^{\ifmmode\sp\else\^{}\fi}\catcode`\%=\active\def%{\%}$\mathdefault{0.00}$}}%
\end{pgfscope}%
\begin{pgfscope}%
\pgfpathrectangle{\pgfqpoint{0.691046in}{0.549073in}}{\pgfqpoint{2.325000in}{2.310000in}}%
\pgfusepath{clip}%
\pgfsetrectcap%
\pgfsetroundjoin%
\pgfsetlinewidth{0.250937pt}%
\definecolor{currentstroke}{rgb}{0.000000,0.000000,0.000000}%
\pgfsetstrokecolor{currentstroke}%
\pgfsetstrokeopacity{0.200000}%
\pgfsetdash{}{0pt}%
\pgfpathmoveto{\pgfqpoint{0.691046in}{0.949834in}}%
\pgfpathlineto{\pgfqpoint{3.016046in}{0.949834in}}%
\pgfusepath{stroke}%
\end{pgfscope}%
\begin{pgfscope}%
\pgfsetbuttcap%
\pgfsetroundjoin%
\definecolor{currentfill}{rgb}{0.000000,0.000000,0.000000}%
\pgfsetfillcolor{currentfill}%
\pgfsetlinewidth{0.803000pt}%
\definecolor{currentstroke}{rgb}{0.000000,0.000000,0.000000}%
\pgfsetstrokecolor{currentstroke}%
\pgfsetdash{}{0pt}%
\pgfsys@defobject{currentmarker}{\pgfqpoint{-0.048611in}{0.000000in}}{\pgfqpoint{-0.000000in}{0.000000in}}{%
\pgfpathmoveto{\pgfqpoint{-0.000000in}{0.000000in}}%
\pgfpathlineto{\pgfqpoint{-0.048611in}{0.000000in}}%
\pgfusepath{stroke,fill}%
}%
\begin{pgfscope}%
\pgfsys@transformshift{0.691046in}{0.949834in}%
\pgfsys@useobject{currentmarker}{}%
\end{pgfscope}%
\end{pgfscope}%
\begin{pgfscope}%
\definecolor{textcolor}{rgb}{0.000000,0.000000,0.000000}%
\pgfsetstrokecolor{textcolor}%
\pgfsetfillcolor{textcolor}%
\pgftext[x=0.303703in, y=0.891964in, left, base]{\color{textcolor}{\rmfamily\fontsize{12.000000}{14.400000}\selectfont\catcode`\^=\active\def^{\ifmmode\sp\else\^{}\fi}\catcode`\%=\active\def%{\%}$\mathdefault{0.05}$}}%
\end{pgfscope}%
\begin{pgfscope}%
\pgfpathrectangle{\pgfqpoint{0.691046in}{0.549073in}}{\pgfqpoint{2.325000in}{2.310000in}}%
\pgfusepath{clip}%
\pgfsetrectcap%
\pgfsetroundjoin%
\pgfsetlinewidth{0.250937pt}%
\definecolor{currentstroke}{rgb}{0.000000,0.000000,0.000000}%
\pgfsetstrokecolor{currentstroke}%
\pgfsetstrokeopacity{0.200000}%
\pgfsetdash{}{0pt}%
\pgfpathmoveto{\pgfqpoint{0.691046in}{1.346006in}}%
\pgfpathlineto{\pgfqpoint{3.016046in}{1.346006in}}%
\pgfusepath{stroke}%
\end{pgfscope}%
\begin{pgfscope}%
\pgfsetbuttcap%
\pgfsetroundjoin%
\definecolor{currentfill}{rgb}{0.000000,0.000000,0.000000}%
\pgfsetfillcolor{currentfill}%
\pgfsetlinewidth{0.803000pt}%
\definecolor{currentstroke}{rgb}{0.000000,0.000000,0.000000}%
\pgfsetstrokecolor{currentstroke}%
\pgfsetdash{}{0pt}%
\pgfsys@defobject{currentmarker}{\pgfqpoint{-0.048611in}{0.000000in}}{\pgfqpoint{-0.000000in}{0.000000in}}{%
\pgfpathmoveto{\pgfqpoint{-0.000000in}{0.000000in}}%
\pgfpathlineto{\pgfqpoint{-0.048611in}{0.000000in}}%
\pgfusepath{stroke,fill}%
}%
\begin{pgfscope}%
\pgfsys@transformshift{0.691046in}{1.346006in}%
\pgfsys@useobject{currentmarker}{}%
\end{pgfscope}%
\end{pgfscope}%
\begin{pgfscope}%
\definecolor{textcolor}{rgb}{0.000000,0.000000,0.000000}%
\pgfsetstrokecolor{textcolor}%
\pgfsetfillcolor{textcolor}%
\pgftext[x=0.303703in, y=1.288136in, left, base]{\color{textcolor}{\rmfamily\fontsize{12.000000}{14.400000}\selectfont\catcode`\^=\active\def^{\ifmmode\sp\else\^{}\fi}\catcode`\%=\active\def%{\%}$\mathdefault{0.10}$}}%
\end{pgfscope}%
\begin{pgfscope}%
\pgfpathrectangle{\pgfqpoint{0.691046in}{0.549073in}}{\pgfqpoint{2.325000in}{2.310000in}}%
\pgfusepath{clip}%
\pgfsetrectcap%
\pgfsetroundjoin%
\pgfsetlinewidth{0.250937pt}%
\definecolor{currentstroke}{rgb}{0.000000,0.000000,0.000000}%
\pgfsetstrokecolor{currentstroke}%
\pgfsetstrokeopacity{0.200000}%
\pgfsetdash{}{0pt}%
\pgfpathmoveto{\pgfqpoint{0.691046in}{1.742179in}}%
\pgfpathlineto{\pgfqpoint{3.016046in}{1.742179in}}%
\pgfusepath{stroke}%
\end{pgfscope}%
\begin{pgfscope}%
\pgfsetbuttcap%
\pgfsetroundjoin%
\definecolor{currentfill}{rgb}{0.000000,0.000000,0.000000}%
\pgfsetfillcolor{currentfill}%
\pgfsetlinewidth{0.803000pt}%
\definecolor{currentstroke}{rgb}{0.000000,0.000000,0.000000}%
\pgfsetstrokecolor{currentstroke}%
\pgfsetdash{}{0pt}%
\pgfsys@defobject{currentmarker}{\pgfqpoint{-0.048611in}{0.000000in}}{\pgfqpoint{-0.000000in}{0.000000in}}{%
\pgfpathmoveto{\pgfqpoint{-0.000000in}{0.000000in}}%
\pgfpathlineto{\pgfqpoint{-0.048611in}{0.000000in}}%
\pgfusepath{stroke,fill}%
}%
\begin{pgfscope}%
\pgfsys@transformshift{0.691046in}{1.742179in}%
\pgfsys@useobject{currentmarker}{}%
\end{pgfscope}%
\end{pgfscope}%
\begin{pgfscope}%
\definecolor{textcolor}{rgb}{0.000000,0.000000,0.000000}%
\pgfsetstrokecolor{textcolor}%
\pgfsetfillcolor{textcolor}%
\pgftext[x=0.303703in, y=1.684308in, left, base]{\color{textcolor}{\rmfamily\fontsize{12.000000}{14.400000}\selectfont\catcode`\^=\active\def^{\ifmmode\sp\else\^{}\fi}\catcode`\%=\active\def%{\%}$\mathdefault{0.15}$}}%
\end{pgfscope}%
\begin{pgfscope}%
\pgfpathrectangle{\pgfqpoint{0.691046in}{0.549073in}}{\pgfqpoint{2.325000in}{2.310000in}}%
\pgfusepath{clip}%
\pgfsetrectcap%
\pgfsetroundjoin%
\pgfsetlinewidth{0.250937pt}%
\definecolor{currentstroke}{rgb}{0.000000,0.000000,0.000000}%
\pgfsetstrokecolor{currentstroke}%
\pgfsetstrokeopacity{0.200000}%
\pgfsetdash{}{0pt}%
\pgfpathmoveto{\pgfqpoint{0.691046in}{2.138351in}}%
\pgfpathlineto{\pgfqpoint{3.016046in}{2.138351in}}%
\pgfusepath{stroke}%
\end{pgfscope}%
\begin{pgfscope}%
\pgfsetbuttcap%
\pgfsetroundjoin%
\definecolor{currentfill}{rgb}{0.000000,0.000000,0.000000}%
\pgfsetfillcolor{currentfill}%
\pgfsetlinewidth{0.803000pt}%
\definecolor{currentstroke}{rgb}{0.000000,0.000000,0.000000}%
\pgfsetstrokecolor{currentstroke}%
\pgfsetdash{}{0pt}%
\pgfsys@defobject{currentmarker}{\pgfqpoint{-0.048611in}{0.000000in}}{\pgfqpoint{-0.000000in}{0.000000in}}{%
\pgfpathmoveto{\pgfqpoint{-0.000000in}{0.000000in}}%
\pgfpathlineto{\pgfqpoint{-0.048611in}{0.000000in}}%
\pgfusepath{stroke,fill}%
}%
\begin{pgfscope}%
\pgfsys@transformshift{0.691046in}{2.138351in}%
\pgfsys@useobject{currentmarker}{}%
\end{pgfscope}%
\end{pgfscope}%
\begin{pgfscope}%
\definecolor{textcolor}{rgb}{0.000000,0.000000,0.000000}%
\pgfsetstrokecolor{textcolor}%
\pgfsetfillcolor{textcolor}%
\pgftext[x=0.303703in, y=2.080480in, left, base]{\color{textcolor}{\rmfamily\fontsize{12.000000}{14.400000}\selectfont\catcode`\^=\active\def^{\ifmmode\sp\else\^{}\fi}\catcode`\%=\active\def%{\%}$\mathdefault{0.20}$}}%
\end{pgfscope}%
\begin{pgfscope}%
\pgfpathrectangle{\pgfqpoint{0.691046in}{0.549073in}}{\pgfqpoint{2.325000in}{2.310000in}}%
\pgfusepath{clip}%
\pgfsetrectcap%
\pgfsetroundjoin%
\pgfsetlinewidth{0.250937pt}%
\definecolor{currentstroke}{rgb}{0.000000,0.000000,0.000000}%
\pgfsetstrokecolor{currentstroke}%
\pgfsetstrokeopacity{0.200000}%
\pgfsetdash{}{0pt}%
\pgfpathmoveto{\pgfqpoint{0.691046in}{2.534523in}}%
\pgfpathlineto{\pgfqpoint{3.016046in}{2.534523in}}%
\pgfusepath{stroke}%
\end{pgfscope}%
\begin{pgfscope}%
\pgfsetbuttcap%
\pgfsetroundjoin%
\definecolor{currentfill}{rgb}{0.000000,0.000000,0.000000}%
\pgfsetfillcolor{currentfill}%
\pgfsetlinewidth{0.803000pt}%
\definecolor{currentstroke}{rgb}{0.000000,0.000000,0.000000}%
\pgfsetstrokecolor{currentstroke}%
\pgfsetdash{}{0pt}%
\pgfsys@defobject{currentmarker}{\pgfqpoint{-0.048611in}{0.000000in}}{\pgfqpoint{-0.000000in}{0.000000in}}{%
\pgfpathmoveto{\pgfqpoint{-0.000000in}{0.000000in}}%
\pgfpathlineto{\pgfqpoint{-0.048611in}{0.000000in}}%
\pgfusepath{stroke,fill}%
}%
\begin{pgfscope}%
\pgfsys@transformshift{0.691046in}{2.534523in}%
\pgfsys@useobject{currentmarker}{}%
\end{pgfscope}%
\end{pgfscope}%
\begin{pgfscope}%
\definecolor{textcolor}{rgb}{0.000000,0.000000,0.000000}%
\pgfsetstrokecolor{textcolor}%
\pgfsetfillcolor{textcolor}%
\pgftext[x=0.303703in, y=2.476653in, left, base]{\color{textcolor}{\rmfamily\fontsize{12.000000}{14.400000}\selectfont\catcode`\^=\active\def^{\ifmmode\sp\else\^{}\fi}\catcode`\%=\active\def%{\%}$\mathdefault{0.25}$}}%
\end{pgfscope}%
\begin{pgfscope}%
\definecolor{textcolor}{rgb}{0.000000,0.000000,0.000000}%
\pgfsetstrokecolor{textcolor}%
\pgfsetfillcolor{textcolor}%
\pgftext[x=0.248148in,y=1.704073in,,bottom,rotate=90.000000]{\color{textcolor}{\rmfamily\fontsize{12.000000}{14.400000}\selectfont\catcode`\^=\active\def^{\ifmmode\sp\else\^{}\fi}\catcode`\%=\active\def%{\%}training loss}}%
\end{pgfscope}%
\begin{pgfscope}%
\pgfpathrectangle{\pgfqpoint{0.691046in}{0.549073in}}{\pgfqpoint{2.325000in}{2.310000in}}%
\pgfusepath{clip}%
\pgfsetrectcap%
\pgfsetroundjoin%
\pgfsetlinewidth{1.505625pt}%
\definecolor{currentstroke}{rgb}{0.000000,0.000000,0.000000}%
\pgfsetstrokecolor{currentstroke}%
\pgfsetdash{}{0pt}%
\pgfpathmoveto{\pgfqpoint{0.746403in}{2.666573in}}%
\pgfpathlineto{\pgfqpoint{1.023189in}{1.179967in}}%
\pgfpathlineto{\pgfqpoint{1.299974in}{0.889775in}}%
\pgfpathlineto{\pgfqpoint{1.576760in}{0.818136in}}%
\pgfpathlineto{\pgfqpoint{1.853546in}{0.786107in}}%
\pgfpathlineto{\pgfqpoint{2.130331in}{0.768042in}}%
\pgfpathlineto{\pgfqpoint{2.407117in}{0.756283in}}%
\pgfpathlineto{\pgfqpoint{2.683903in}{0.747908in}}%
\pgfpathlineto{\pgfqpoint{2.960689in}{0.741573in}}%
\pgfusepath{stroke}%
\end{pgfscope}%
\begin{pgfscope}%
\pgfpathrectangle{\pgfqpoint{0.691046in}{0.549073in}}{\pgfqpoint{2.325000in}{2.310000in}}%
\pgfusepath{clip}%
\pgfsetbuttcap%
\pgfsetmiterjoin%
\definecolor{currentfill}{rgb}{0.000000,0.000000,0.000000}%
\pgfsetfillcolor{currentfill}%
\pgfsetlinewidth{1.003750pt}%
\definecolor{currentstroke}{rgb}{0.000000,0.000000,0.000000}%
\pgfsetstrokecolor{currentstroke}%
\pgfsetdash{}{0pt}%
\pgfsys@defobject{currentmarker}{\pgfqpoint{-0.035355in}{-0.058926in}}{\pgfqpoint{0.035355in}{0.058926in}}{%
\pgfpathmoveto{\pgfqpoint{-0.000000in}{-0.058926in}}%
\pgfpathlineto{\pgfqpoint{0.035355in}{0.000000in}}%
\pgfpathlineto{\pgfqpoint{0.000000in}{0.058926in}}%
\pgfpathlineto{\pgfqpoint{-0.035355in}{0.000000in}}%
\pgfpathlineto{\pgfqpoint{-0.000000in}{-0.058926in}}%
\pgfpathclose%
\pgfusepath{stroke,fill}%
}%
\begin{pgfscope}%
\pgfsys@transformshift{0.746403in}{2.666573in}%
\pgfsys@useobject{currentmarker}{}%
\end{pgfscope}%
\begin{pgfscope}%
\pgfsys@transformshift{1.023189in}{1.179967in}%
\pgfsys@useobject{currentmarker}{}%
\end{pgfscope}%
\begin{pgfscope}%
\pgfsys@transformshift{1.299974in}{0.889775in}%
\pgfsys@useobject{currentmarker}{}%
\end{pgfscope}%
\begin{pgfscope}%
\pgfsys@transformshift{1.576760in}{0.818136in}%
\pgfsys@useobject{currentmarker}{}%
\end{pgfscope}%
\begin{pgfscope}%
\pgfsys@transformshift{1.853546in}{0.786107in}%
\pgfsys@useobject{currentmarker}{}%
\end{pgfscope}%
\begin{pgfscope}%
\pgfsys@transformshift{2.130331in}{0.768042in}%
\pgfsys@useobject{currentmarker}{}%
\end{pgfscope}%
\begin{pgfscope}%
\pgfsys@transformshift{2.407117in}{0.756283in}%
\pgfsys@useobject{currentmarker}{}%
\end{pgfscope}%
\begin{pgfscope}%
\pgfsys@transformshift{2.683903in}{0.747908in}%
\pgfsys@useobject{currentmarker}{}%
\end{pgfscope}%
\begin{pgfscope}%
\pgfsys@transformshift{2.960689in}{0.741573in}%
\pgfsys@useobject{currentmarker}{}%
\end{pgfscope}%
\end{pgfscope}%
\begin{pgfscope}%
\pgfsetrectcap%
\pgfsetmiterjoin%
\pgfsetlinewidth{0.803000pt}%
\definecolor{currentstroke}{rgb}{0.000000,0.000000,0.000000}%
\pgfsetstrokecolor{currentstroke}%
\pgfsetdash{}{0pt}%
\pgfpathmoveto{\pgfqpoint{0.691046in}{0.549073in}}%
\pgfpathlineto{\pgfqpoint{0.691046in}{2.859073in}}%
\pgfusepath{stroke}%
\end{pgfscope}%
\begin{pgfscope}%
\pgfsetrectcap%
\pgfsetmiterjoin%
\pgfsetlinewidth{0.803000pt}%
\definecolor{currentstroke}{rgb}{0.000000,0.000000,0.000000}%
\pgfsetstrokecolor{currentstroke}%
\pgfsetdash{}{0pt}%
\pgfpathmoveto{\pgfqpoint{3.016046in}{0.549073in}}%
\pgfpathlineto{\pgfqpoint{3.016046in}{2.859073in}}%
\pgfusepath{stroke}%
\end{pgfscope}%
\begin{pgfscope}%
\pgfsetrectcap%
\pgfsetmiterjoin%
\pgfsetlinewidth{0.803000pt}%
\definecolor{currentstroke}{rgb}{0.000000,0.000000,0.000000}%
\pgfsetstrokecolor{currentstroke}%
\pgfsetdash{}{0pt}%
\pgfpathmoveto{\pgfqpoint{0.691046in}{0.549073in}}%
\pgfpathlineto{\pgfqpoint{3.016046in}{0.549073in}}%
\pgfusepath{stroke}%
\end{pgfscope}%
\begin{pgfscope}%
\pgfsetrectcap%
\pgfsetmiterjoin%
\pgfsetlinewidth{0.803000pt}%
\definecolor{currentstroke}{rgb}{0.000000,0.000000,0.000000}%
\pgfsetstrokecolor{currentstroke}%
\pgfsetdash{}{0pt}%
\pgfpathmoveto{\pgfqpoint{0.691046in}{2.859073in}}%
\pgfpathlineto{\pgfqpoint{3.016046in}{2.859073in}}%
\pgfusepath{stroke}%
\end{pgfscope}%
\end{pgfpicture}%
\makeatother%
\endgroup%

    \end{minipage}
    \caption{The mean squared error training loss (right) and the corresponding approximate spectral density of the Hessian matrix of a fully connected convolutional neural network in different epochs of training on the MNIST dataset (left). The spectral density is approximated in $n_t=150$ uniformly spaced points using the Chebyshev-Nyström++ method (\refalg{alg:nystrom-chebyshev-pp}) with parameters $n_{\mtx{\Omega}} = 10$, $n_{\mtx{\Psi}} = 10$, $m = 1000$, and $\sigma = 0.005$.}
    \label{fig:hessian-density}
\end{figure}

\subsection{Approximation of the partition function of a quantum system}
\label{subsec:spin-chain}

To demonstrate that the Chebyshev-Nyström++ estimator is also effective on problems unrelated to spectral density estimation, we consider the problem of approximating the partition function $\Trace(\exp(-\beta \mtx{A}))$ for the Hamiltonian $\mtx{A}$ of a planar Heisenberg spin chain \cite[Section 5.1]{chen-2023-krylovaware-stochastic}. We compute the pointwise error from the analytic solution as a function of the temperature $\beta^{-1}$ and compare it to the error achieved by the Krylov-aware trace estimator in \reffig{fig:krylov-aware-spin}.

\begin{figure}[ht]
    \begin{minipage}[c]{.475\linewidth}
        \centering
        %% Creator: Matplotlib, PGF backend
%%
%% To include the figure in your LaTeX document, write
%%   \input{<filename>.pgf}
%%
%% Make sure the required packages are loaded in your preamble
%%   \usepackage{pgf}
%%
%% Also ensure that all the required font packages are loaded; for instance,
%% the lmodern package is sometimes necessary when using math font.
%%   \usepackage{lmodern}
%%
%% Figures using additional raster images can only be included by \input if
%% they are in the same directory as the main LaTeX file. For loading figures
%% from other directories you can use the `import` package
%%   \usepackage{import}
%%
%% and then include the figures with
%%   \import{<path to file>}{<filename>.pgf}
%%
%% Matplotlib used the following preamble
%%   \def\mathdefault#1{#1}
%%   \everymath=\expandafter{\the\everymath\displaystyle}
%%   
%%   \ifdefined\pdftexversion\else  % non-pdftex case.
%%     \usepackage{fontspec}
%%     \setmainfont{DejaVuSerif.ttf}[Path=\detokenize{/opt/hostedtoolcache/Python/3.12.3/x64/lib/python3.12/site-packages/matplotlib/mpl-data/fonts/ttf/}]
%%     \setsansfont{DejaVuSans.ttf}[Path=\detokenize{/opt/hostedtoolcache/Python/3.12.3/x64/lib/python3.12/site-packages/matplotlib/mpl-data/fonts/ttf/}]
%%     \setmonofont{DejaVuSansMono.ttf}[Path=\detokenize{/opt/hostedtoolcache/Python/3.12.3/x64/lib/python3.12/site-packages/matplotlib/mpl-data/fonts/ttf/}]
%%   \fi
%%   \makeatletter\@ifpackageloaded{underscore}{}{\usepackage[strings]{underscore}}\makeatother
%%
\begingroup%
\makeatletter%
\begin{pgfpicture}%
\pgfpathrectangle{\pgfpointorigin}{\pgfqpoint{3.206139in}{2.983753in}}%
\pgfusepath{use as bounding box, clip}%
\begin{pgfscope}%
\pgfsetbuttcap%
\pgfsetmiterjoin%
\definecolor{currentfill}{rgb}{1.000000,1.000000,1.000000}%
\pgfsetfillcolor{currentfill}%
\pgfsetlinewidth{0.000000pt}%
\definecolor{currentstroke}{rgb}{1.000000,1.000000,1.000000}%
\pgfsetstrokecolor{currentstroke}%
\pgfsetdash{}{0pt}%
\pgfpathmoveto{\pgfqpoint{0.000000in}{0.000000in}}%
\pgfpathlineto{\pgfqpoint{3.206139in}{0.000000in}}%
\pgfpathlineto{\pgfqpoint{3.206139in}{2.983753in}}%
\pgfpathlineto{\pgfqpoint{0.000000in}{2.983753in}}%
\pgfpathlineto{\pgfqpoint{0.000000in}{0.000000in}}%
\pgfpathclose%
\pgfusepath{fill}%
\end{pgfscope}%
\begin{pgfscope}%
\pgfsetbuttcap%
\pgfsetmiterjoin%
\definecolor{currentfill}{rgb}{1.000000,1.000000,1.000000}%
\pgfsetfillcolor{currentfill}%
\pgfsetlinewidth{0.000000pt}%
\definecolor{currentstroke}{rgb}{0.000000,0.000000,0.000000}%
\pgfsetstrokecolor{currentstroke}%
\pgfsetstrokeopacity{0.000000}%
\pgfsetdash{}{0pt}%
\pgfpathmoveto{\pgfqpoint{0.721913in}{0.573753in}}%
\pgfpathlineto{\pgfqpoint{3.046913in}{0.573753in}}%
\pgfpathlineto{\pgfqpoint{3.046913in}{2.883753in}}%
\pgfpathlineto{\pgfqpoint{0.721913in}{2.883753in}}%
\pgfpathlineto{\pgfqpoint{0.721913in}{0.573753in}}%
\pgfpathclose%
\pgfusepath{fill}%
\end{pgfscope}%
\begin{pgfscope}%
\pgfpathrectangle{\pgfqpoint{0.721913in}{0.573753in}}{\pgfqpoint{2.325000in}{2.310000in}}%
\pgfusepath{clip}%
\pgfsetrectcap%
\pgfsetroundjoin%
\pgfsetlinewidth{0.250937pt}%
\definecolor{currentstroke}{rgb}{0.000000,0.000000,0.000000}%
\pgfsetstrokecolor{currentstroke}%
\pgfsetstrokeopacity{0.200000}%
\pgfsetdash{}{0pt}%
\pgfpathmoveto{\pgfqpoint{1.381167in}{0.573753in}}%
\pgfpathlineto{\pgfqpoint{1.381167in}{2.883753in}}%
\pgfusepath{stroke}%
\end{pgfscope}%
\begin{pgfscope}%
\pgfsetbuttcap%
\pgfsetroundjoin%
\definecolor{currentfill}{rgb}{0.000000,0.000000,0.000000}%
\pgfsetfillcolor{currentfill}%
\pgfsetlinewidth{0.803000pt}%
\definecolor{currentstroke}{rgb}{0.000000,0.000000,0.000000}%
\pgfsetstrokecolor{currentstroke}%
\pgfsetdash{}{0pt}%
\pgfsys@defobject{currentmarker}{\pgfqpoint{0.000000in}{-0.048611in}}{\pgfqpoint{0.000000in}{0.000000in}}{%
\pgfpathmoveto{\pgfqpoint{0.000000in}{0.000000in}}%
\pgfpathlineto{\pgfqpoint{0.000000in}{-0.048611in}}%
\pgfusepath{stroke,fill}%
}%
\begin{pgfscope}%
\pgfsys@transformshift{1.381167in}{0.573753in}%
\pgfsys@useobject{currentmarker}{}%
\end{pgfscope}%
\end{pgfscope}%
\begin{pgfscope}%
\definecolor{textcolor}{rgb}{0.000000,0.000000,0.000000}%
\pgfsetstrokecolor{textcolor}%
\pgfsetfillcolor{textcolor}%
\pgftext[x=1.381167in,y=0.476530in,,top]{\color{textcolor}{\rmfamily\fontsize{12.000000}{14.400000}\selectfont\catcode`\^=\active\def^{\ifmmode\sp\else\^{}\fi}\catcode`\%=\active\def%{\%}$\mathdefault{10^{-1}}$}}%
\end{pgfscope}%
\begin{pgfscope}%
\pgfpathrectangle{\pgfqpoint{0.721913in}{0.573753in}}{\pgfqpoint{2.325000in}{2.310000in}}%
\pgfusepath{clip}%
\pgfsetrectcap%
\pgfsetroundjoin%
\pgfsetlinewidth{0.250937pt}%
\definecolor{currentstroke}{rgb}{0.000000,0.000000,0.000000}%
\pgfsetstrokecolor{currentstroke}%
\pgfsetstrokeopacity{0.200000}%
\pgfsetdash{}{0pt}%
\pgfpathmoveto{\pgfqpoint{2.186362in}{0.573753in}}%
\pgfpathlineto{\pgfqpoint{2.186362in}{2.883753in}}%
\pgfusepath{stroke}%
\end{pgfscope}%
\begin{pgfscope}%
\pgfsetbuttcap%
\pgfsetroundjoin%
\definecolor{currentfill}{rgb}{0.000000,0.000000,0.000000}%
\pgfsetfillcolor{currentfill}%
\pgfsetlinewidth{0.803000pt}%
\definecolor{currentstroke}{rgb}{0.000000,0.000000,0.000000}%
\pgfsetstrokecolor{currentstroke}%
\pgfsetdash{}{0pt}%
\pgfsys@defobject{currentmarker}{\pgfqpoint{0.000000in}{-0.048611in}}{\pgfqpoint{0.000000in}{0.000000in}}{%
\pgfpathmoveto{\pgfqpoint{0.000000in}{0.000000in}}%
\pgfpathlineto{\pgfqpoint{0.000000in}{-0.048611in}}%
\pgfusepath{stroke,fill}%
}%
\begin{pgfscope}%
\pgfsys@transformshift{2.186362in}{0.573753in}%
\pgfsys@useobject{currentmarker}{}%
\end{pgfscope}%
\end{pgfscope}%
\begin{pgfscope}%
\definecolor{textcolor}{rgb}{0.000000,0.000000,0.000000}%
\pgfsetstrokecolor{textcolor}%
\pgfsetfillcolor{textcolor}%
\pgftext[x=2.186362in,y=0.476530in,,top]{\color{textcolor}{\rmfamily\fontsize{12.000000}{14.400000}\selectfont\catcode`\^=\active\def^{\ifmmode\sp\else\^{}\fi}\catcode`\%=\active\def%{\%}$\mathdefault{10^{1}}$}}%
\end{pgfscope}%
\begin{pgfscope}%
\pgfpathrectangle{\pgfqpoint{0.721913in}{0.573753in}}{\pgfqpoint{2.325000in}{2.310000in}}%
\pgfusepath{clip}%
\pgfsetrectcap%
\pgfsetroundjoin%
\pgfsetlinewidth{0.250937pt}%
\definecolor{currentstroke}{rgb}{0.000000,0.000000,0.000000}%
\pgfsetstrokecolor{currentstroke}%
\pgfsetstrokeopacity{0.200000}%
\pgfsetdash{}{0pt}%
\pgfpathmoveto{\pgfqpoint{2.991556in}{0.573753in}}%
\pgfpathlineto{\pgfqpoint{2.991556in}{2.883753in}}%
\pgfusepath{stroke}%
\end{pgfscope}%
\begin{pgfscope}%
\pgfsetbuttcap%
\pgfsetroundjoin%
\definecolor{currentfill}{rgb}{0.000000,0.000000,0.000000}%
\pgfsetfillcolor{currentfill}%
\pgfsetlinewidth{0.803000pt}%
\definecolor{currentstroke}{rgb}{0.000000,0.000000,0.000000}%
\pgfsetstrokecolor{currentstroke}%
\pgfsetdash{}{0pt}%
\pgfsys@defobject{currentmarker}{\pgfqpoint{0.000000in}{-0.048611in}}{\pgfqpoint{0.000000in}{0.000000in}}{%
\pgfpathmoveto{\pgfqpoint{0.000000in}{0.000000in}}%
\pgfpathlineto{\pgfqpoint{0.000000in}{-0.048611in}}%
\pgfusepath{stroke,fill}%
}%
\begin{pgfscope}%
\pgfsys@transformshift{2.991556in}{0.573753in}%
\pgfsys@useobject{currentmarker}{}%
\end{pgfscope}%
\end{pgfscope}%
\begin{pgfscope}%
\definecolor{textcolor}{rgb}{0.000000,0.000000,0.000000}%
\pgfsetstrokecolor{textcolor}%
\pgfsetfillcolor{textcolor}%
\pgftext[x=2.991556in,y=0.476530in,,top]{\color{textcolor}{\rmfamily\fontsize{12.000000}{14.400000}\selectfont\catcode`\^=\active\def^{\ifmmode\sp\else\^{}\fi}\catcode`\%=\active\def%{\%}$\mathdefault{10^{3}}$}}%
\end{pgfscope}%
\begin{pgfscope}%
\definecolor{textcolor}{rgb}{0.000000,0.000000,0.000000}%
\pgfsetstrokecolor{textcolor}%
\pgfsetfillcolor{textcolor}%
\pgftext[x=1.884413in,y=0.272827in,,top]{\color{textcolor}{\rmfamily\fontsize{12.000000}{14.400000}\selectfont\catcode`\^=\active\def^{\ifmmode\sp\else\^{}\fi}\catcode`\%=\active\def%{\%}temperature parameter $\beta^{-1}$}}%
\end{pgfscope}%
\begin{pgfscope}%
\pgfpathrectangle{\pgfqpoint{0.721913in}{0.573753in}}{\pgfqpoint{2.325000in}{2.310000in}}%
\pgfusepath{clip}%
\pgfsetrectcap%
\pgfsetroundjoin%
\pgfsetlinewidth{0.250937pt}%
\definecolor{currentstroke}{rgb}{0.000000,0.000000,0.000000}%
\pgfsetstrokecolor{currentstroke}%
\pgfsetstrokeopacity{0.200000}%
\pgfsetdash{}{0pt}%
\pgfpathmoveto{\pgfqpoint{0.721913in}{1.003648in}}%
\pgfpathlineto{\pgfqpoint{3.046913in}{1.003648in}}%
\pgfusepath{stroke}%
\end{pgfscope}%
\begin{pgfscope}%
\pgfsetbuttcap%
\pgfsetroundjoin%
\definecolor{currentfill}{rgb}{0.000000,0.000000,0.000000}%
\pgfsetfillcolor{currentfill}%
\pgfsetlinewidth{0.803000pt}%
\definecolor{currentstroke}{rgb}{0.000000,0.000000,0.000000}%
\pgfsetstrokecolor{currentstroke}%
\pgfsetdash{}{0pt}%
\pgfsys@defobject{currentmarker}{\pgfqpoint{-0.048611in}{0.000000in}}{\pgfqpoint{-0.000000in}{0.000000in}}{%
\pgfpathmoveto{\pgfqpoint{-0.000000in}{0.000000in}}%
\pgfpathlineto{\pgfqpoint{-0.048611in}{0.000000in}}%
\pgfusepath{stroke,fill}%
}%
\begin{pgfscope}%
\pgfsys@transformshift{0.721913in}{1.003648in}%
\pgfsys@useobject{currentmarker}{}%
\end{pgfscope}%
\end{pgfscope}%
\begin{pgfscope}%
\definecolor{textcolor}{rgb}{0.000000,0.000000,0.000000}%
\pgfsetstrokecolor{textcolor}%
\pgfsetfillcolor{textcolor}%
\pgftext[x=0.303703in, y=0.945778in, left, base]{\color{textcolor}{\rmfamily\fontsize{12.000000}{14.400000}\selectfont\catcode`\^=\active\def^{\ifmmode\sp\else\^{}\fi}\catcode`\%=\active\def%{\%}$\mathdefault{10^{-6}}$}}%
\end{pgfscope}%
\begin{pgfscope}%
\pgfpathrectangle{\pgfqpoint{0.721913in}{0.573753in}}{\pgfqpoint{2.325000in}{2.310000in}}%
\pgfusepath{clip}%
\pgfsetrectcap%
\pgfsetroundjoin%
\pgfsetlinewidth{0.250937pt}%
\definecolor{currentstroke}{rgb}{0.000000,0.000000,0.000000}%
\pgfsetstrokecolor{currentstroke}%
\pgfsetstrokeopacity{0.200000}%
\pgfsetdash{}{0pt}%
\pgfpathmoveto{\pgfqpoint{0.721913in}{1.506557in}}%
\pgfpathlineto{\pgfqpoint{3.046913in}{1.506557in}}%
\pgfusepath{stroke}%
\end{pgfscope}%
\begin{pgfscope}%
\pgfsetbuttcap%
\pgfsetroundjoin%
\definecolor{currentfill}{rgb}{0.000000,0.000000,0.000000}%
\pgfsetfillcolor{currentfill}%
\pgfsetlinewidth{0.803000pt}%
\definecolor{currentstroke}{rgb}{0.000000,0.000000,0.000000}%
\pgfsetstrokecolor{currentstroke}%
\pgfsetdash{}{0pt}%
\pgfsys@defobject{currentmarker}{\pgfqpoint{-0.048611in}{0.000000in}}{\pgfqpoint{-0.000000in}{0.000000in}}{%
\pgfpathmoveto{\pgfqpoint{-0.000000in}{0.000000in}}%
\pgfpathlineto{\pgfqpoint{-0.048611in}{0.000000in}}%
\pgfusepath{stroke,fill}%
}%
\begin{pgfscope}%
\pgfsys@transformshift{0.721913in}{1.506557in}%
\pgfsys@useobject{currentmarker}{}%
\end{pgfscope}%
\end{pgfscope}%
\begin{pgfscope}%
\definecolor{textcolor}{rgb}{0.000000,0.000000,0.000000}%
\pgfsetstrokecolor{textcolor}%
\pgfsetfillcolor{textcolor}%
\pgftext[x=0.303703in, y=1.448687in, left, base]{\color{textcolor}{\rmfamily\fontsize{12.000000}{14.400000}\selectfont\catcode`\^=\active\def^{\ifmmode\sp\else\^{}\fi}\catcode`\%=\active\def%{\%}$\mathdefault{10^{-4}}$}}%
\end{pgfscope}%
\begin{pgfscope}%
\pgfpathrectangle{\pgfqpoint{0.721913in}{0.573753in}}{\pgfqpoint{2.325000in}{2.310000in}}%
\pgfusepath{clip}%
\pgfsetrectcap%
\pgfsetroundjoin%
\pgfsetlinewidth{0.250937pt}%
\definecolor{currentstroke}{rgb}{0.000000,0.000000,0.000000}%
\pgfsetstrokecolor{currentstroke}%
\pgfsetstrokeopacity{0.200000}%
\pgfsetdash{}{0pt}%
\pgfpathmoveto{\pgfqpoint{0.721913in}{2.009466in}}%
\pgfpathlineto{\pgfqpoint{3.046913in}{2.009466in}}%
\pgfusepath{stroke}%
\end{pgfscope}%
\begin{pgfscope}%
\pgfsetbuttcap%
\pgfsetroundjoin%
\definecolor{currentfill}{rgb}{0.000000,0.000000,0.000000}%
\pgfsetfillcolor{currentfill}%
\pgfsetlinewidth{0.803000pt}%
\definecolor{currentstroke}{rgb}{0.000000,0.000000,0.000000}%
\pgfsetstrokecolor{currentstroke}%
\pgfsetdash{}{0pt}%
\pgfsys@defobject{currentmarker}{\pgfqpoint{-0.048611in}{0.000000in}}{\pgfqpoint{-0.000000in}{0.000000in}}{%
\pgfpathmoveto{\pgfqpoint{-0.000000in}{0.000000in}}%
\pgfpathlineto{\pgfqpoint{-0.048611in}{0.000000in}}%
\pgfusepath{stroke,fill}%
}%
\begin{pgfscope}%
\pgfsys@transformshift{0.721913in}{2.009466in}%
\pgfsys@useobject{currentmarker}{}%
\end{pgfscope}%
\end{pgfscope}%
\begin{pgfscope}%
\definecolor{textcolor}{rgb}{0.000000,0.000000,0.000000}%
\pgfsetstrokecolor{textcolor}%
\pgfsetfillcolor{textcolor}%
\pgftext[x=0.303703in, y=1.951596in, left, base]{\color{textcolor}{\rmfamily\fontsize{12.000000}{14.400000}\selectfont\catcode`\^=\active\def^{\ifmmode\sp\else\^{}\fi}\catcode`\%=\active\def%{\%}$\mathdefault{10^{-2}}$}}%
\end{pgfscope}%
\begin{pgfscope}%
\pgfpathrectangle{\pgfqpoint{0.721913in}{0.573753in}}{\pgfqpoint{2.325000in}{2.310000in}}%
\pgfusepath{clip}%
\pgfsetrectcap%
\pgfsetroundjoin%
\pgfsetlinewidth{0.250937pt}%
\definecolor{currentstroke}{rgb}{0.000000,0.000000,0.000000}%
\pgfsetstrokecolor{currentstroke}%
\pgfsetstrokeopacity{0.200000}%
\pgfsetdash{}{0pt}%
\pgfpathmoveto{\pgfqpoint{0.721913in}{2.512375in}}%
\pgfpathlineto{\pgfqpoint{3.046913in}{2.512375in}}%
\pgfusepath{stroke}%
\end{pgfscope}%
\begin{pgfscope}%
\pgfsetbuttcap%
\pgfsetroundjoin%
\definecolor{currentfill}{rgb}{0.000000,0.000000,0.000000}%
\pgfsetfillcolor{currentfill}%
\pgfsetlinewidth{0.803000pt}%
\definecolor{currentstroke}{rgb}{0.000000,0.000000,0.000000}%
\pgfsetstrokecolor{currentstroke}%
\pgfsetdash{}{0pt}%
\pgfsys@defobject{currentmarker}{\pgfqpoint{-0.048611in}{0.000000in}}{\pgfqpoint{-0.000000in}{0.000000in}}{%
\pgfpathmoveto{\pgfqpoint{-0.000000in}{0.000000in}}%
\pgfpathlineto{\pgfqpoint{-0.048611in}{0.000000in}}%
\pgfusepath{stroke,fill}%
}%
\begin{pgfscope}%
\pgfsys@transformshift{0.721913in}{2.512375in}%
\pgfsys@useobject{currentmarker}{}%
\end{pgfscope}%
\end{pgfscope}%
\begin{pgfscope}%
\definecolor{textcolor}{rgb}{0.000000,0.000000,0.000000}%
\pgfsetstrokecolor{textcolor}%
\pgfsetfillcolor{textcolor}%
\pgftext[x=0.395525in, y=2.454505in, left, base]{\color{textcolor}{\rmfamily\fontsize{12.000000}{14.400000}\selectfont\catcode`\^=\active\def^{\ifmmode\sp\else\^{}\fi}\catcode`\%=\active\def%{\%}$\mathdefault{10^{0}}$}}%
\end{pgfscope}%
\begin{pgfscope}%
\definecolor{textcolor}{rgb}{0.000000,0.000000,0.000000}%
\pgfsetstrokecolor{textcolor}%
\pgfsetfillcolor{textcolor}%
\pgftext[x=0.248147in,y=1.728753in,,bottom,rotate=90.000000]{\color{textcolor}{\rmfamily\fontsize{12.000000}{14.400000}\selectfont\catcode`\^=\active\def^{\ifmmode\sp\else\^{}\fi}\catcode`\%=\active\def%{\%}relative error}}%
\end{pgfscope}%
\begin{pgfscope}%
\pgfpathrectangle{\pgfqpoint{0.721913in}{0.573753in}}{\pgfqpoint{2.325000in}{2.310000in}}%
\pgfusepath{clip}%
\pgfsetrectcap%
\pgfsetroundjoin%
\pgfsetlinewidth{1.505625pt}%
\definecolor{currentstroke}{rgb}{1.000000,0.690196,0.000000}%
\pgfsetstrokecolor{currentstroke}%
\pgfsetdash{}{0pt}%
\pgfpathmoveto{\pgfqpoint{0.777271in}{1.667830in}}%
\pgfpathlineto{\pgfqpoint{0.924890in}{1.576050in}}%
\pgfpathlineto{\pgfqpoint{1.072509in}{1.485022in}}%
\pgfpathlineto{\pgfqpoint{1.220128in}{1.423393in}}%
\pgfpathlineto{\pgfqpoint{1.367747in}{1.772479in}}%
\pgfpathlineto{\pgfqpoint{1.515366in}{2.217517in}}%
\pgfpathlineto{\pgfqpoint{1.662985in}{2.457680in}}%
\pgfpathlineto{\pgfqpoint{1.810604in}{2.509197in}}%
\pgfpathlineto{\pgfqpoint{1.958223in}{2.512173in}}%
\pgfpathlineto{\pgfqpoint{2.105842in}{2.512328in}}%
\pgfpathlineto{\pgfqpoint{2.253461in}{2.512346in}}%
\pgfpathlineto{\pgfqpoint{2.401080in}{2.512349in}}%
\pgfpathlineto{\pgfqpoint{2.548699in}{2.512349in}}%
\pgfpathlineto{\pgfqpoint{2.696318in}{2.512349in}}%
\pgfpathlineto{\pgfqpoint{2.843937in}{2.512349in}}%
\pgfpathlineto{\pgfqpoint{2.991556in}{2.512349in}}%
\pgfusepath{stroke}%
\end{pgfscope}%
\begin{pgfscope}%
\pgfpathrectangle{\pgfqpoint{0.721913in}{0.573753in}}{\pgfqpoint{2.325000in}{2.310000in}}%
\pgfusepath{clip}%
\pgfsetbuttcap%
\pgfsetmiterjoin%
\definecolor{currentfill}{rgb}{1.000000,0.690196,0.000000}%
\pgfsetfillcolor{currentfill}%
\pgfsetlinewidth{1.003750pt}%
\definecolor{currentstroke}{rgb}{1.000000,0.690196,0.000000}%
\pgfsetstrokecolor{currentstroke}%
\pgfsetdash{}{0pt}%
\pgfsys@defobject{currentmarker}{\pgfqpoint{-0.035355in}{-0.058926in}}{\pgfqpoint{0.035355in}{0.058926in}}{%
\pgfpathmoveto{\pgfqpoint{-0.000000in}{-0.058926in}}%
\pgfpathlineto{\pgfqpoint{0.035355in}{0.000000in}}%
\pgfpathlineto{\pgfqpoint{0.000000in}{0.058926in}}%
\pgfpathlineto{\pgfqpoint{-0.035355in}{0.000000in}}%
\pgfpathlineto{\pgfqpoint{-0.000000in}{-0.058926in}}%
\pgfpathclose%
\pgfusepath{stroke,fill}%
}%
\begin{pgfscope}%
\pgfsys@transformshift{0.777271in}{1.667830in}%
\pgfsys@useobject{currentmarker}{}%
\end{pgfscope}%
\begin{pgfscope}%
\pgfsys@transformshift{0.924890in}{1.576050in}%
\pgfsys@useobject{currentmarker}{}%
\end{pgfscope}%
\begin{pgfscope}%
\pgfsys@transformshift{1.072509in}{1.485022in}%
\pgfsys@useobject{currentmarker}{}%
\end{pgfscope}%
\begin{pgfscope}%
\pgfsys@transformshift{1.220128in}{1.423393in}%
\pgfsys@useobject{currentmarker}{}%
\end{pgfscope}%
\begin{pgfscope}%
\pgfsys@transformshift{1.367747in}{1.772479in}%
\pgfsys@useobject{currentmarker}{}%
\end{pgfscope}%
\begin{pgfscope}%
\pgfsys@transformshift{1.515366in}{2.217517in}%
\pgfsys@useobject{currentmarker}{}%
\end{pgfscope}%
\begin{pgfscope}%
\pgfsys@transformshift{1.662985in}{2.457680in}%
\pgfsys@useobject{currentmarker}{}%
\end{pgfscope}%
\begin{pgfscope}%
\pgfsys@transformshift{1.810604in}{2.509197in}%
\pgfsys@useobject{currentmarker}{}%
\end{pgfscope}%
\begin{pgfscope}%
\pgfsys@transformshift{1.958223in}{2.512173in}%
\pgfsys@useobject{currentmarker}{}%
\end{pgfscope}%
\begin{pgfscope}%
\pgfsys@transformshift{2.105842in}{2.512328in}%
\pgfsys@useobject{currentmarker}{}%
\end{pgfscope}%
\begin{pgfscope}%
\pgfsys@transformshift{2.253461in}{2.512346in}%
\pgfsys@useobject{currentmarker}{}%
\end{pgfscope}%
\begin{pgfscope}%
\pgfsys@transformshift{2.401080in}{2.512349in}%
\pgfsys@useobject{currentmarker}{}%
\end{pgfscope}%
\begin{pgfscope}%
\pgfsys@transformshift{2.548699in}{2.512349in}%
\pgfsys@useobject{currentmarker}{}%
\end{pgfscope}%
\begin{pgfscope}%
\pgfsys@transformshift{2.696318in}{2.512349in}%
\pgfsys@useobject{currentmarker}{}%
\end{pgfscope}%
\begin{pgfscope}%
\pgfsys@transformshift{2.843937in}{2.512349in}%
\pgfsys@useobject{currentmarker}{}%
\end{pgfscope}%
\begin{pgfscope}%
\pgfsys@transformshift{2.991556in}{2.512349in}%
\pgfsys@useobject{currentmarker}{}%
\end{pgfscope}%
\end{pgfscope}%
\begin{pgfscope}%
\pgfpathrectangle{\pgfqpoint{0.721913in}{0.573753in}}{\pgfqpoint{2.325000in}{2.310000in}}%
\pgfusepath{clip}%
\pgfsetrectcap%
\pgfsetroundjoin%
\pgfsetlinewidth{1.505625pt}%
\definecolor{currentstroke}{rgb}{0.996078,0.380392,0.000000}%
\pgfsetstrokecolor{currentstroke}%
\pgfsetdash{}{0pt}%
\pgfpathmoveto{\pgfqpoint{0.777271in}{2.459661in}}%
\pgfpathlineto{\pgfqpoint{0.924890in}{2.459854in}}%
\pgfpathlineto{\pgfqpoint{1.072509in}{2.459888in}}%
\pgfpathlineto{\pgfqpoint{1.220128in}{2.452916in}}%
\pgfpathlineto{\pgfqpoint{1.367747in}{2.402622in}}%
\pgfpathlineto{\pgfqpoint{1.515366in}{2.310006in}}%
\pgfpathlineto{\pgfqpoint{1.662985in}{2.117727in}}%
\pgfpathlineto{\pgfqpoint{1.810604in}{1.835726in}}%
\pgfpathlineto{\pgfqpoint{1.958223in}{1.730106in}}%
\pgfpathlineto{\pgfqpoint{2.105842in}{1.618308in}}%
\pgfpathlineto{\pgfqpoint{2.253461in}{1.512245in}}%
\pgfpathlineto{\pgfqpoint{2.401080in}{1.412318in}}%
\pgfpathlineto{\pgfqpoint{2.548699in}{1.316378in}}%
\pgfpathlineto{\pgfqpoint{2.696318in}{1.222484in}}%
\pgfpathlineto{\pgfqpoint{2.843937in}{1.129539in}}%
\pgfpathlineto{\pgfqpoint{2.991556in}{1.037016in}}%
\pgfusepath{stroke}%
\end{pgfscope}%
\begin{pgfscope}%
\pgfpathrectangle{\pgfqpoint{0.721913in}{0.573753in}}{\pgfqpoint{2.325000in}{2.310000in}}%
\pgfusepath{clip}%
\pgfsetbuttcap%
\pgfsetmiterjoin%
\definecolor{currentfill}{rgb}{0.996078,0.380392,0.000000}%
\pgfsetfillcolor{currentfill}%
\pgfsetlinewidth{1.003750pt}%
\definecolor{currentstroke}{rgb}{0.996078,0.380392,0.000000}%
\pgfsetstrokecolor{currentstroke}%
\pgfsetdash{}{0pt}%
\pgfsys@defobject{currentmarker}{\pgfqpoint{-0.039627in}{-0.033709in}}{\pgfqpoint{0.039627in}{0.041667in}}{%
\pgfpathmoveto{\pgfqpoint{0.000000in}{0.041667in}}%
\pgfpathlineto{\pgfqpoint{-0.039627in}{0.012876in}}%
\pgfpathlineto{\pgfqpoint{-0.024491in}{-0.033709in}}%
\pgfpathlineto{\pgfqpoint{0.024491in}{-0.033709in}}%
\pgfpathlineto{\pgfqpoint{0.039627in}{0.012876in}}%
\pgfpathlineto{\pgfqpoint{0.000000in}{0.041667in}}%
\pgfpathclose%
\pgfusepath{stroke,fill}%
}%
\begin{pgfscope}%
\pgfsys@transformshift{0.777271in}{2.459661in}%
\pgfsys@useobject{currentmarker}{}%
\end{pgfscope}%
\begin{pgfscope}%
\pgfsys@transformshift{0.924890in}{2.459854in}%
\pgfsys@useobject{currentmarker}{}%
\end{pgfscope}%
\begin{pgfscope}%
\pgfsys@transformshift{1.072509in}{2.459888in}%
\pgfsys@useobject{currentmarker}{}%
\end{pgfscope}%
\begin{pgfscope}%
\pgfsys@transformshift{1.220128in}{2.452916in}%
\pgfsys@useobject{currentmarker}{}%
\end{pgfscope}%
\begin{pgfscope}%
\pgfsys@transformshift{1.367747in}{2.402622in}%
\pgfsys@useobject{currentmarker}{}%
\end{pgfscope}%
\begin{pgfscope}%
\pgfsys@transformshift{1.515366in}{2.310006in}%
\pgfsys@useobject{currentmarker}{}%
\end{pgfscope}%
\begin{pgfscope}%
\pgfsys@transformshift{1.662985in}{2.117727in}%
\pgfsys@useobject{currentmarker}{}%
\end{pgfscope}%
\begin{pgfscope}%
\pgfsys@transformshift{1.810604in}{1.835726in}%
\pgfsys@useobject{currentmarker}{}%
\end{pgfscope}%
\begin{pgfscope}%
\pgfsys@transformshift{1.958223in}{1.730106in}%
\pgfsys@useobject{currentmarker}{}%
\end{pgfscope}%
\begin{pgfscope}%
\pgfsys@transformshift{2.105842in}{1.618308in}%
\pgfsys@useobject{currentmarker}{}%
\end{pgfscope}%
\begin{pgfscope}%
\pgfsys@transformshift{2.253461in}{1.512245in}%
\pgfsys@useobject{currentmarker}{}%
\end{pgfscope}%
\begin{pgfscope}%
\pgfsys@transformshift{2.401080in}{1.412318in}%
\pgfsys@useobject{currentmarker}{}%
\end{pgfscope}%
\begin{pgfscope}%
\pgfsys@transformshift{2.548699in}{1.316378in}%
\pgfsys@useobject{currentmarker}{}%
\end{pgfscope}%
\begin{pgfscope}%
\pgfsys@transformshift{2.696318in}{1.222484in}%
\pgfsys@useobject{currentmarker}{}%
\end{pgfscope}%
\begin{pgfscope}%
\pgfsys@transformshift{2.843937in}{1.129539in}%
\pgfsys@useobject{currentmarker}{}%
\end{pgfscope}%
\begin{pgfscope}%
\pgfsys@transformshift{2.991556in}{1.037016in}%
\pgfsys@useobject{currentmarker}{}%
\end{pgfscope}%
\end{pgfscope}%
\begin{pgfscope}%
\pgfpathrectangle{\pgfqpoint{0.721913in}{0.573753in}}{\pgfqpoint{2.325000in}{2.310000in}}%
\pgfusepath{clip}%
\pgfsetrectcap%
\pgfsetroundjoin%
\pgfsetlinewidth{1.505625pt}%
\definecolor{currentstroke}{rgb}{0.470588,0.368627,0.941176}%
\pgfsetstrokecolor{currentstroke}%
\pgfsetdash{}{0pt}%
\pgfpathmoveto{\pgfqpoint{0.777271in}{1.614730in}}%
\pgfpathlineto{\pgfqpoint{0.924890in}{1.522281in}}%
\pgfpathlineto{\pgfqpoint{1.072509in}{1.429528in}}%
\pgfpathlineto{\pgfqpoint{1.220128in}{1.344533in}}%
\pgfpathlineto{\pgfqpoint{1.367747in}{1.673204in}}%
\pgfpathlineto{\pgfqpoint{1.515366in}{1.966949in}}%
\pgfpathlineto{\pgfqpoint{1.662985in}{2.004168in}}%
\pgfpathlineto{\pgfqpoint{1.810604in}{1.566883in}}%
\pgfpathlineto{\pgfqpoint{1.958223in}{1.662899in}}%
\pgfpathlineto{\pgfqpoint{2.105842in}{1.471941in}}%
\pgfpathlineto{\pgfqpoint{2.253461in}{1.139079in}}%
\pgfpathlineto{\pgfqpoint{2.401080in}{1.153129in}}%
\pgfpathlineto{\pgfqpoint{2.548699in}{1.112051in}}%
\pgfpathlineto{\pgfqpoint{2.696318in}{1.036137in}}%
\pgfpathlineto{\pgfqpoint{2.843937in}{0.950254in}}%
\pgfpathlineto{\pgfqpoint{2.991556in}{0.860661in}}%
\pgfusepath{stroke}%
\end{pgfscope}%
\begin{pgfscope}%
\pgfpathrectangle{\pgfqpoint{0.721913in}{0.573753in}}{\pgfqpoint{2.325000in}{2.310000in}}%
\pgfusepath{clip}%
\pgfsetbuttcap%
\pgfsetmiterjoin%
\definecolor{currentfill}{rgb}{0.470588,0.368627,0.941176}%
\pgfsetfillcolor{currentfill}%
\pgfsetlinewidth{1.003750pt}%
\definecolor{currentstroke}{rgb}{0.470588,0.368627,0.941176}%
\pgfsetstrokecolor{currentstroke}%
\pgfsetdash{}{0pt}%
\pgfsys@defobject{currentmarker}{\pgfqpoint{-0.041667in}{-0.041667in}}{\pgfqpoint{0.041667in}{0.041667in}}{%
\pgfpathmoveto{\pgfqpoint{0.000000in}{0.041667in}}%
\pgfpathlineto{\pgfqpoint{-0.041667in}{-0.041667in}}%
\pgfpathlineto{\pgfqpoint{0.041667in}{-0.041667in}}%
\pgfpathlineto{\pgfqpoint{0.000000in}{0.041667in}}%
\pgfpathclose%
\pgfusepath{stroke,fill}%
}%
\begin{pgfscope}%
\pgfsys@transformshift{0.777271in}{1.614730in}%
\pgfsys@useobject{currentmarker}{}%
\end{pgfscope}%
\begin{pgfscope}%
\pgfsys@transformshift{0.924890in}{1.522281in}%
\pgfsys@useobject{currentmarker}{}%
\end{pgfscope}%
\begin{pgfscope}%
\pgfsys@transformshift{1.072509in}{1.429528in}%
\pgfsys@useobject{currentmarker}{}%
\end{pgfscope}%
\begin{pgfscope}%
\pgfsys@transformshift{1.220128in}{1.344533in}%
\pgfsys@useobject{currentmarker}{}%
\end{pgfscope}%
\begin{pgfscope}%
\pgfsys@transformshift{1.367747in}{1.673204in}%
\pgfsys@useobject{currentmarker}{}%
\end{pgfscope}%
\begin{pgfscope}%
\pgfsys@transformshift{1.515366in}{1.966949in}%
\pgfsys@useobject{currentmarker}{}%
\end{pgfscope}%
\begin{pgfscope}%
\pgfsys@transformshift{1.662985in}{2.004168in}%
\pgfsys@useobject{currentmarker}{}%
\end{pgfscope}%
\begin{pgfscope}%
\pgfsys@transformshift{1.810604in}{1.566883in}%
\pgfsys@useobject{currentmarker}{}%
\end{pgfscope}%
\begin{pgfscope}%
\pgfsys@transformshift{1.958223in}{1.662899in}%
\pgfsys@useobject{currentmarker}{}%
\end{pgfscope}%
\begin{pgfscope}%
\pgfsys@transformshift{2.105842in}{1.471941in}%
\pgfsys@useobject{currentmarker}{}%
\end{pgfscope}%
\begin{pgfscope}%
\pgfsys@transformshift{2.253461in}{1.139079in}%
\pgfsys@useobject{currentmarker}{}%
\end{pgfscope}%
\begin{pgfscope}%
\pgfsys@transformshift{2.401080in}{1.153129in}%
\pgfsys@useobject{currentmarker}{}%
\end{pgfscope}%
\begin{pgfscope}%
\pgfsys@transformshift{2.548699in}{1.112051in}%
\pgfsys@useobject{currentmarker}{}%
\end{pgfscope}%
\begin{pgfscope}%
\pgfsys@transformshift{2.696318in}{1.036137in}%
\pgfsys@useobject{currentmarker}{}%
\end{pgfscope}%
\begin{pgfscope}%
\pgfsys@transformshift{2.843937in}{0.950254in}%
\pgfsys@useobject{currentmarker}{}%
\end{pgfscope}%
\begin{pgfscope}%
\pgfsys@transformshift{2.991556in}{0.860661in}%
\pgfsys@useobject{currentmarker}{}%
\end{pgfscope}%
\end{pgfscope}%
\begin{pgfscope}%
\pgfpathrectangle{\pgfqpoint{0.721913in}{0.573753in}}{\pgfqpoint{2.325000in}{2.310000in}}%
\pgfusepath{clip}%
\pgfsetrectcap%
\pgfsetroundjoin%
\pgfsetlinewidth{1.505625pt}%
\definecolor{currentstroke}{rgb}{0.392157,0.560784,1.000000}%
\pgfsetstrokecolor{currentstroke}%
\pgfsetdash{}{0pt}%
\pgfpathmoveto{\pgfqpoint{0.777271in}{1.965597in}}%
\pgfpathlineto{\pgfqpoint{0.924890in}{1.872956in}}%
\pgfpathlineto{\pgfqpoint{1.072509in}{1.779500in}}%
\pgfpathlineto{\pgfqpoint{1.220128in}{1.706020in}}%
\pgfpathlineto{\pgfqpoint{1.367747in}{1.865534in}}%
\pgfpathlineto{\pgfqpoint{1.515366in}{2.092970in}}%
\pgfpathlineto{\pgfqpoint{1.662985in}{2.114378in}}%
\pgfpathlineto{\pgfqpoint{1.810604in}{1.896954in}}%
\pgfpathlineto{\pgfqpoint{1.958223in}{1.617772in}}%
\pgfpathlineto{\pgfqpoint{2.105842in}{1.422540in}}%
\pgfpathlineto{\pgfqpoint{2.253461in}{1.205171in}}%
\pgfpathlineto{\pgfqpoint{2.401080in}{0.936230in}}%
\pgfpathlineto{\pgfqpoint{2.548699in}{0.994537in}}%
\pgfpathlineto{\pgfqpoint{2.696318in}{0.934371in}}%
\pgfpathlineto{\pgfqpoint{2.843937in}{0.853772in}}%
\pgfpathlineto{\pgfqpoint{2.991556in}{0.766253in}}%
\pgfusepath{stroke}%
\end{pgfscope}%
\begin{pgfscope}%
\pgfpathrectangle{\pgfqpoint{0.721913in}{0.573753in}}{\pgfqpoint{2.325000in}{2.310000in}}%
\pgfusepath{clip}%
\pgfsetbuttcap%
\pgfsetroundjoin%
\definecolor{currentfill}{rgb}{0.392157,0.560784,1.000000}%
\pgfsetfillcolor{currentfill}%
\pgfsetlinewidth{1.003750pt}%
\definecolor{currentstroke}{rgb}{0.392157,0.560784,1.000000}%
\pgfsetstrokecolor{currentstroke}%
\pgfsetdash{}{0pt}%
\pgfsys@defobject{currentmarker}{\pgfqpoint{-0.041667in}{-0.041667in}}{\pgfqpoint{0.041667in}{0.041667in}}{%
\pgfpathmoveto{\pgfqpoint{0.000000in}{-0.041667in}}%
\pgfpathcurveto{\pgfqpoint{0.011050in}{-0.041667in}}{\pgfqpoint{0.021649in}{-0.037276in}}{\pgfqpoint{0.029463in}{-0.029463in}}%
\pgfpathcurveto{\pgfqpoint{0.037276in}{-0.021649in}}{\pgfqpoint{0.041667in}{-0.011050in}}{\pgfqpoint{0.041667in}{0.000000in}}%
\pgfpathcurveto{\pgfqpoint{0.041667in}{0.011050in}}{\pgfqpoint{0.037276in}{0.021649in}}{\pgfqpoint{0.029463in}{0.029463in}}%
\pgfpathcurveto{\pgfqpoint{0.021649in}{0.037276in}}{\pgfqpoint{0.011050in}{0.041667in}}{\pgfqpoint{0.000000in}{0.041667in}}%
\pgfpathcurveto{\pgfqpoint{-0.011050in}{0.041667in}}{\pgfqpoint{-0.021649in}{0.037276in}}{\pgfqpoint{-0.029463in}{0.029463in}}%
\pgfpathcurveto{\pgfqpoint{-0.037276in}{0.021649in}}{\pgfqpoint{-0.041667in}{0.011050in}}{\pgfqpoint{-0.041667in}{0.000000in}}%
\pgfpathcurveto{\pgfqpoint{-0.041667in}{-0.011050in}}{\pgfqpoint{-0.037276in}{-0.021649in}}{\pgfqpoint{-0.029463in}{-0.029463in}}%
\pgfpathcurveto{\pgfqpoint{-0.021649in}{-0.037276in}}{\pgfqpoint{-0.011050in}{-0.041667in}}{\pgfqpoint{0.000000in}{-0.041667in}}%
\pgfpathlineto{\pgfqpoint{0.000000in}{-0.041667in}}%
\pgfpathclose%
\pgfusepath{stroke,fill}%
}%
\begin{pgfscope}%
\pgfsys@transformshift{0.777271in}{1.965597in}%
\pgfsys@useobject{currentmarker}{}%
\end{pgfscope}%
\begin{pgfscope}%
\pgfsys@transformshift{0.924890in}{1.872956in}%
\pgfsys@useobject{currentmarker}{}%
\end{pgfscope}%
\begin{pgfscope}%
\pgfsys@transformshift{1.072509in}{1.779500in}%
\pgfsys@useobject{currentmarker}{}%
\end{pgfscope}%
\begin{pgfscope}%
\pgfsys@transformshift{1.220128in}{1.706020in}%
\pgfsys@useobject{currentmarker}{}%
\end{pgfscope}%
\begin{pgfscope}%
\pgfsys@transformshift{1.367747in}{1.865534in}%
\pgfsys@useobject{currentmarker}{}%
\end{pgfscope}%
\begin{pgfscope}%
\pgfsys@transformshift{1.515366in}{2.092970in}%
\pgfsys@useobject{currentmarker}{}%
\end{pgfscope}%
\begin{pgfscope}%
\pgfsys@transformshift{1.662985in}{2.114378in}%
\pgfsys@useobject{currentmarker}{}%
\end{pgfscope}%
\begin{pgfscope}%
\pgfsys@transformshift{1.810604in}{1.896954in}%
\pgfsys@useobject{currentmarker}{}%
\end{pgfscope}%
\begin{pgfscope}%
\pgfsys@transformshift{1.958223in}{1.617772in}%
\pgfsys@useobject{currentmarker}{}%
\end{pgfscope}%
\begin{pgfscope}%
\pgfsys@transformshift{2.105842in}{1.422540in}%
\pgfsys@useobject{currentmarker}{}%
\end{pgfscope}%
\begin{pgfscope}%
\pgfsys@transformshift{2.253461in}{1.205171in}%
\pgfsys@useobject{currentmarker}{}%
\end{pgfscope}%
\begin{pgfscope}%
\pgfsys@transformshift{2.401080in}{0.936230in}%
\pgfsys@useobject{currentmarker}{}%
\end{pgfscope}%
\begin{pgfscope}%
\pgfsys@transformshift{2.548699in}{0.994537in}%
\pgfsys@useobject{currentmarker}{}%
\end{pgfscope}%
\begin{pgfscope}%
\pgfsys@transformshift{2.696318in}{0.934371in}%
\pgfsys@useobject{currentmarker}{}%
\end{pgfscope}%
\begin{pgfscope}%
\pgfsys@transformshift{2.843937in}{0.853772in}%
\pgfsys@useobject{currentmarker}{}%
\end{pgfscope}%
\begin{pgfscope}%
\pgfsys@transformshift{2.991556in}{0.766253in}%
\pgfsys@useobject{currentmarker}{}%
\end{pgfscope}%
\end{pgfscope}%
\begin{pgfscope}%
\pgfpathrectangle{\pgfqpoint{0.721913in}{0.573753in}}{\pgfqpoint{2.325000in}{2.310000in}}%
\pgfusepath{clip}%
\pgfsetrectcap%
\pgfsetroundjoin%
\pgfsetlinewidth{1.505625pt}%
\definecolor{currentstroke}{rgb}{0.862745,0.149020,0.498039}%
\pgfsetstrokecolor{currentstroke}%
\pgfsetdash{}{0pt}%
\pgfpathmoveto{\pgfqpoint{0.777271in}{2.691253in}}%
\pgfpathlineto{\pgfqpoint{0.924890in}{2.681865in}}%
\pgfpathlineto{\pgfqpoint{1.072509in}{2.625729in}}%
\pgfpathlineto{\pgfqpoint{1.220128in}{2.180289in}}%
\pgfpathlineto{\pgfqpoint{1.367747in}{1.744313in}}%
\pgfpathlineto{\pgfqpoint{1.515366in}{1.840604in}}%
\pgfpathlineto{\pgfqpoint{1.662985in}{1.754304in}}%
\pgfpathlineto{\pgfqpoint{1.810604in}{1.806303in}}%
\pgfpathlineto{\pgfqpoint{1.958223in}{1.699558in}}%
\pgfpathlineto{\pgfqpoint{2.105842in}{1.653955in}}%
\pgfpathlineto{\pgfqpoint{2.253461in}{1.653457in}}%
\pgfpathlineto{\pgfqpoint{2.401080in}{1.655636in}}%
\pgfpathlineto{\pgfqpoint{2.548699in}{1.656801in}}%
\pgfpathlineto{\pgfqpoint{2.696318in}{1.657327in}}%
\pgfpathlineto{\pgfqpoint{2.843937in}{1.657557in}}%
\pgfpathlineto{\pgfqpoint{2.991556in}{1.657656in}}%
\pgfusepath{stroke}%
\end{pgfscope}%
\begin{pgfscope}%
\pgfpathrectangle{\pgfqpoint{0.721913in}{0.573753in}}{\pgfqpoint{2.325000in}{2.310000in}}%
\pgfusepath{clip}%
\pgfsetbuttcap%
\pgfsetmiterjoin%
\definecolor{currentfill}{rgb}{0.862745,0.149020,0.498039}%
\pgfsetfillcolor{currentfill}%
\pgfsetlinewidth{1.003750pt}%
\definecolor{currentstroke}{rgb}{0.862745,0.149020,0.498039}%
\pgfsetstrokecolor{currentstroke}%
\pgfsetdash{}{0pt}%
\pgfsys@defobject{currentmarker}{\pgfqpoint{-0.041667in}{-0.041667in}}{\pgfqpoint{0.041667in}{0.041667in}}{%
\pgfpathmoveto{\pgfqpoint{-0.041667in}{-0.041667in}}%
\pgfpathlineto{\pgfqpoint{0.041667in}{-0.041667in}}%
\pgfpathlineto{\pgfqpoint{0.041667in}{0.041667in}}%
\pgfpathlineto{\pgfqpoint{-0.041667in}{0.041667in}}%
\pgfpathlineto{\pgfqpoint{-0.041667in}{-0.041667in}}%
\pgfpathclose%
\pgfusepath{stroke,fill}%
}%
\begin{pgfscope}%
\pgfsys@transformshift{0.777271in}{2.691253in}%
\pgfsys@useobject{currentmarker}{}%
\end{pgfscope}%
\begin{pgfscope}%
\pgfsys@transformshift{0.924890in}{2.681865in}%
\pgfsys@useobject{currentmarker}{}%
\end{pgfscope}%
\begin{pgfscope}%
\pgfsys@transformshift{1.072509in}{2.625729in}%
\pgfsys@useobject{currentmarker}{}%
\end{pgfscope}%
\begin{pgfscope}%
\pgfsys@transformshift{1.220128in}{2.180289in}%
\pgfsys@useobject{currentmarker}{}%
\end{pgfscope}%
\begin{pgfscope}%
\pgfsys@transformshift{1.367747in}{1.744313in}%
\pgfsys@useobject{currentmarker}{}%
\end{pgfscope}%
\begin{pgfscope}%
\pgfsys@transformshift{1.515366in}{1.840604in}%
\pgfsys@useobject{currentmarker}{}%
\end{pgfscope}%
\begin{pgfscope}%
\pgfsys@transformshift{1.662985in}{1.754304in}%
\pgfsys@useobject{currentmarker}{}%
\end{pgfscope}%
\begin{pgfscope}%
\pgfsys@transformshift{1.810604in}{1.806303in}%
\pgfsys@useobject{currentmarker}{}%
\end{pgfscope}%
\begin{pgfscope}%
\pgfsys@transformshift{1.958223in}{1.699558in}%
\pgfsys@useobject{currentmarker}{}%
\end{pgfscope}%
\begin{pgfscope}%
\pgfsys@transformshift{2.105842in}{1.653955in}%
\pgfsys@useobject{currentmarker}{}%
\end{pgfscope}%
\begin{pgfscope}%
\pgfsys@transformshift{2.253461in}{1.653457in}%
\pgfsys@useobject{currentmarker}{}%
\end{pgfscope}%
\begin{pgfscope}%
\pgfsys@transformshift{2.401080in}{1.655636in}%
\pgfsys@useobject{currentmarker}{}%
\end{pgfscope}%
\begin{pgfscope}%
\pgfsys@transformshift{2.548699in}{1.656801in}%
\pgfsys@useobject{currentmarker}{}%
\end{pgfscope}%
\begin{pgfscope}%
\pgfsys@transformshift{2.696318in}{1.657327in}%
\pgfsys@useobject{currentmarker}{}%
\end{pgfscope}%
\begin{pgfscope}%
\pgfsys@transformshift{2.843937in}{1.657557in}%
\pgfsys@useobject{currentmarker}{}%
\end{pgfscope}%
\begin{pgfscope}%
\pgfsys@transformshift{2.991556in}{1.657656in}%
\pgfsys@useobject{currentmarker}{}%
\end{pgfscope}%
\end{pgfscope}%
\begin{pgfscope}%
\pgfsetrectcap%
\pgfsetmiterjoin%
\pgfsetlinewidth{0.803000pt}%
\definecolor{currentstroke}{rgb}{0.000000,0.000000,0.000000}%
\pgfsetstrokecolor{currentstroke}%
\pgfsetdash{}{0pt}%
\pgfpathmoveto{\pgfqpoint{0.721913in}{0.573753in}}%
\pgfpathlineto{\pgfqpoint{0.721913in}{2.883753in}}%
\pgfusepath{stroke}%
\end{pgfscope}%
\begin{pgfscope}%
\pgfsetrectcap%
\pgfsetmiterjoin%
\pgfsetlinewidth{0.803000pt}%
\definecolor{currentstroke}{rgb}{0.000000,0.000000,0.000000}%
\pgfsetstrokecolor{currentstroke}%
\pgfsetdash{}{0pt}%
\pgfpathmoveto{\pgfqpoint{3.046913in}{0.573753in}}%
\pgfpathlineto{\pgfqpoint{3.046913in}{2.883753in}}%
\pgfusepath{stroke}%
\end{pgfscope}%
\begin{pgfscope}%
\pgfsetrectcap%
\pgfsetmiterjoin%
\pgfsetlinewidth{0.803000pt}%
\definecolor{currentstroke}{rgb}{0.000000,0.000000,0.000000}%
\pgfsetstrokecolor{currentstroke}%
\pgfsetdash{}{0pt}%
\pgfpathmoveto{\pgfqpoint{0.721913in}{0.573753in}}%
\pgfpathlineto{\pgfqpoint{3.046913in}{0.573753in}}%
\pgfusepath{stroke}%
\end{pgfscope}%
\begin{pgfscope}%
\pgfsetrectcap%
\pgfsetmiterjoin%
\pgfsetlinewidth{0.803000pt}%
\definecolor{currentstroke}{rgb}{0.000000,0.000000,0.000000}%
\pgfsetstrokecolor{currentstroke}%
\pgfsetdash{}{0pt}%
\pgfpathmoveto{\pgfqpoint{0.721913in}{2.883753in}}%
\pgfpathlineto{\pgfqpoint{3.046913in}{2.883753in}}%
\pgfusepath{stroke}%
\end{pgfscope}%
\begin{pgfscope}%
\pgfsetbuttcap%
\pgfsetmiterjoin%
\definecolor{currentfill}{rgb}{1.000000,1.000000,1.000000}%
\pgfsetfillcolor{currentfill}%
\pgfsetfillopacity{0.800000}%
\pgfsetlinewidth{1.003750pt}%
\definecolor{currentstroke}{rgb}{0.800000,0.800000,0.800000}%
\pgfsetstrokecolor{currentstroke}%
\pgfsetstrokeopacity{0.800000}%
\pgfsetdash{}{0pt}%
\pgfpathmoveto{\pgfqpoint{1.797054in}{1.518012in}}%
\pgfpathlineto{\pgfqpoint{2.963580in}{1.518012in}}%
\pgfpathlineto{\pgfqpoint{2.963580in}{2.800419in}}%
\pgfpathlineto{\pgfqpoint{1.797054in}{2.800419in}}%
\pgfpathlineto{\pgfqpoint{1.797054in}{1.518012in}}%
\pgfpathclose%
\pgfusepath{stroke,fill}%
\end{pgfscope}%
\begin{pgfscope}%
\pgfsetrectcap%
\pgfsetroundjoin%
\pgfsetlinewidth{1.505625pt}%
\definecolor{currentstroke}{rgb}{1.000000,0.690196,0.000000}%
\pgfsetstrokecolor{currentstroke}%
\pgfsetdash{}{0pt}%
\pgfpathmoveto{\pgfqpoint{1.863721in}{2.667086in}}%
\pgfpathlineto{\pgfqpoint{2.030388in}{2.667086in}}%
\pgfpathlineto{\pgfqpoint{2.197054in}{2.667086in}}%
\pgfusepath{stroke}%
\end{pgfscope}%
\begin{pgfscope}%
\pgfsetbuttcap%
\pgfsetmiterjoin%
\definecolor{currentfill}{rgb}{1.000000,0.690196,0.000000}%
\pgfsetfillcolor{currentfill}%
\pgfsetlinewidth{1.003750pt}%
\definecolor{currentstroke}{rgb}{1.000000,0.690196,0.000000}%
\pgfsetstrokecolor{currentstroke}%
\pgfsetdash{}{0pt}%
\pgfsys@defobject{currentmarker}{\pgfqpoint{-0.026517in}{-0.044194in}}{\pgfqpoint{0.026517in}{0.044194in}}{%
\pgfpathmoveto{\pgfqpoint{-0.000000in}{-0.044194in}}%
\pgfpathlineto{\pgfqpoint{0.026517in}{0.000000in}}%
\pgfpathlineto{\pgfqpoint{0.000000in}{0.044194in}}%
\pgfpathlineto{\pgfqpoint{-0.026517in}{0.000000in}}%
\pgfpathlineto{\pgfqpoint{-0.000000in}{-0.044194in}}%
\pgfpathclose%
\pgfusepath{stroke,fill}%
}%
\begin{pgfscope}%
\pgfsys@transformshift{2.030388in}{2.667086in}%
\pgfsys@useobject{currentmarker}{}%
\end{pgfscope}%
\end{pgfscope}%
\begin{pgfscope}%
\definecolor{textcolor}{rgb}{0.000000,0.000000,0.000000}%
\pgfsetstrokecolor{textcolor}%
\pgfsetfillcolor{textcolor}%
\pgftext[x=2.330388in,y=2.608753in,left,base]{\color{textcolor}{\rmfamily\fontsize{12.000000}{14.400000}\selectfont\catcode`\^=\active\def^{\ifmmode\sp\else\^{}\fi}\catcode`\%=\active\def%{\%}KA (i)}}%
\end{pgfscope}%
\begin{pgfscope}%
\pgfsetrectcap%
\pgfsetroundjoin%
\pgfsetlinewidth{1.505625pt}%
\definecolor{currentstroke}{rgb}{0.996078,0.380392,0.000000}%
\pgfsetstrokecolor{currentstroke}%
\pgfsetdash{}{0pt}%
\pgfpathmoveto{\pgfqpoint{1.863721in}{2.417086in}}%
\pgfpathlineto{\pgfqpoint{2.030388in}{2.417086in}}%
\pgfpathlineto{\pgfqpoint{2.197054in}{2.417086in}}%
\pgfusepath{stroke}%
\end{pgfscope}%
\begin{pgfscope}%
\pgfsetbuttcap%
\pgfsetmiterjoin%
\definecolor{currentfill}{rgb}{0.996078,0.380392,0.000000}%
\pgfsetfillcolor{currentfill}%
\pgfsetlinewidth{1.003750pt}%
\definecolor{currentstroke}{rgb}{0.996078,0.380392,0.000000}%
\pgfsetstrokecolor{currentstroke}%
\pgfsetdash{}{0pt}%
\pgfsys@defobject{currentmarker}{\pgfqpoint{-0.029721in}{-0.025282in}}{\pgfqpoint{0.029721in}{0.031250in}}{%
\pgfpathmoveto{\pgfqpoint{0.000000in}{0.031250in}}%
\pgfpathlineto{\pgfqpoint{-0.029721in}{0.009657in}}%
\pgfpathlineto{\pgfqpoint{-0.018368in}{-0.025282in}}%
\pgfpathlineto{\pgfqpoint{0.018368in}{-0.025282in}}%
\pgfpathlineto{\pgfqpoint{0.029721in}{0.009657in}}%
\pgfpathlineto{\pgfqpoint{0.000000in}{0.031250in}}%
\pgfpathclose%
\pgfusepath{stroke,fill}%
}%
\begin{pgfscope}%
\pgfsys@transformshift{2.030388in}{2.417086in}%
\pgfsys@useobject{currentmarker}{}%
\end{pgfscope}%
\end{pgfscope}%
\begin{pgfscope}%
\definecolor{textcolor}{rgb}{0.000000,0.000000,0.000000}%
\pgfsetstrokecolor{textcolor}%
\pgfsetfillcolor{textcolor}%
\pgftext[x=2.330388in,y=2.358753in,left,base]{\color{textcolor}{\rmfamily\fontsize{12.000000}{14.400000}\selectfont\catcode`\^=\active\def^{\ifmmode\sp\else\^{}\fi}\catcode`\%=\active\def%{\%}KA (ii)}}%
\end{pgfscope}%
\begin{pgfscope}%
\pgfsetrectcap%
\pgfsetroundjoin%
\pgfsetlinewidth{1.505625pt}%
\definecolor{currentstroke}{rgb}{0.470588,0.368627,0.941176}%
\pgfsetstrokecolor{currentstroke}%
\pgfsetdash{}{0pt}%
\pgfpathmoveto{\pgfqpoint{1.863721in}{2.167086in}}%
\pgfpathlineto{\pgfqpoint{2.030388in}{2.167086in}}%
\pgfpathlineto{\pgfqpoint{2.197054in}{2.167086in}}%
\pgfusepath{stroke}%
\end{pgfscope}%
\begin{pgfscope}%
\pgfsetbuttcap%
\pgfsetmiterjoin%
\definecolor{currentfill}{rgb}{0.470588,0.368627,0.941176}%
\pgfsetfillcolor{currentfill}%
\pgfsetlinewidth{1.003750pt}%
\definecolor{currentstroke}{rgb}{0.470588,0.368627,0.941176}%
\pgfsetstrokecolor{currentstroke}%
\pgfsetdash{}{0pt}%
\pgfsys@defobject{currentmarker}{\pgfqpoint{-0.031250in}{-0.031250in}}{\pgfqpoint{0.031250in}{0.031250in}}{%
\pgfpathmoveto{\pgfqpoint{0.000000in}{0.031250in}}%
\pgfpathlineto{\pgfqpoint{-0.031250in}{-0.031250in}}%
\pgfpathlineto{\pgfqpoint{0.031250in}{-0.031250in}}%
\pgfpathlineto{\pgfqpoint{0.000000in}{0.031250in}}%
\pgfpathclose%
\pgfusepath{stroke,fill}%
}%
\begin{pgfscope}%
\pgfsys@transformshift{2.030388in}{2.167086in}%
\pgfsys@useobject{currentmarker}{}%
\end{pgfscope}%
\end{pgfscope}%
\begin{pgfscope}%
\definecolor{textcolor}{rgb}{0.000000,0.000000,0.000000}%
\pgfsetstrokecolor{textcolor}%
\pgfsetfillcolor{textcolor}%
\pgftext[x=2.330388in,y=2.108753in,left,base]{\color{textcolor}{\rmfamily\fontsize{12.000000}{14.400000}\selectfont\catcode`\^=\active\def^{\ifmmode\sp\else\^{}\fi}\catcode`\%=\active\def%{\%}KA (iii)}}%
\end{pgfscope}%
\begin{pgfscope}%
\pgfsetrectcap%
\pgfsetroundjoin%
\pgfsetlinewidth{1.505625pt}%
\definecolor{currentstroke}{rgb}{0.392157,0.560784,1.000000}%
\pgfsetstrokecolor{currentstroke}%
\pgfsetdash{}{0pt}%
\pgfpathmoveto{\pgfqpoint{1.863721in}{1.917086in}}%
\pgfpathlineto{\pgfqpoint{2.030388in}{1.917086in}}%
\pgfpathlineto{\pgfqpoint{2.197054in}{1.917086in}}%
\pgfusepath{stroke}%
\end{pgfscope}%
\begin{pgfscope}%
\pgfsetbuttcap%
\pgfsetroundjoin%
\definecolor{currentfill}{rgb}{0.392157,0.560784,1.000000}%
\pgfsetfillcolor{currentfill}%
\pgfsetlinewidth{1.003750pt}%
\definecolor{currentstroke}{rgb}{0.392157,0.560784,1.000000}%
\pgfsetstrokecolor{currentstroke}%
\pgfsetdash{}{0pt}%
\pgfsys@defobject{currentmarker}{\pgfqpoint{-0.031250in}{-0.031250in}}{\pgfqpoint{0.031250in}{0.031250in}}{%
\pgfpathmoveto{\pgfqpoint{0.000000in}{-0.031250in}}%
\pgfpathcurveto{\pgfqpoint{0.008288in}{-0.031250in}}{\pgfqpoint{0.016237in}{-0.027957in}}{\pgfqpoint{0.022097in}{-0.022097in}}%
\pgfpathcurveto{\pgfqpoint{0.027957in}{-0.016237in}}{\pgfqpoint{0.031250in}{-0.008288in}}{\pgfqpoint{0.031250in}{0.000000in}}%
\pgfpathcurveto{\pgfqpoint{0.031250in}{0.008288in}}{\pgfqpoint{0.027957in}{0.016237in}}{\pgfqpoint{0.022097in}{0.022097in}}%
\pgfpathcurveto{\pgfqpoint{0.016237in}{0.027957in}}{\pgfqpoint{0.008288in}{0.031250in}}{\pgfqpoint{0.000000in}{0.031250in}}%
\pgfpathcurveto{\pgfqpoint{-0.008288in}{0.031250in}}{\pgfqpoint{-0.016237in}{0.027957in}}{\pgfqpoint{-0.022097in}{0.022097in}}%
\pgfpathcurveto{\pgfqpoint{-0.027957in}{0.016237in}}{\pgfqpoint{-0.031250in}{0.008288in}}{\pgfqpoint{-0.031250in}{0.000000in}}%
\pgfpathcurveto{\pgfqpoint{-0.031250in}{-0.008288in}}{\pgfqpoint{-0.027957in}{-0.016237in}}{\pgfqpoint{-0.022097in}{-0.022097in}}%
\pgfpathcurveto{\pgfqpoint{-0.016237in}{-0.027957in}}{\pgfqpoint{-0.008288in}{-0.031250in}}{\pgfqpoint{0.000000in}{-0.031250in}}%
\pgfpathlineto{\pgfqpoint{0.000000in}{-0.031250in}}%
\pgfpathclose%
\pgfusepath{stroke,fill}%
}%
\begin{pgfscope}%
\pgfsys@transformshift{2.030388in}{1.917086in}%
\pgfsys@useobject{currentmarker}{}%
\end{pgfscope}%
\end{pgfscope}%
\begin{pgfscope}%
\definecolor{textcolor}{rgb}{0.000000,0.000000,0.000000}%
\pgfsetstrokecolor{textcolor}%
\pgfsetfillcolor{textcolor}%
\pgftext[x=2.330388in,y=1.858753in,left,base]{\color{textcolor}{\rmfamily\fontsize{12.000000}{14.400000}\selectfont\catcode`\^=\active\def^{\ifmmode\sp\else\^{}\fi}\catcode`\%=\active\def%{\%}KA (iv)}}%
\end{pgfscope}%
\begin{pgfscope}%
\pgfsetrectcap%
\pgfsetroundjoin%
\pgfsetlinewidth{1.505625pt}%
\definecolor{currentstroke}{rgb}{0.862745,0.149020,0.498039}%
\pgfsetstrokecolor{currentstroke}%
\pgfsetdash{}{0pt}%
\pgfpathmoveto{\pgfqpoint{1.863721in}{1.675419in}}%
\pgfpathlineto{\pgfqpoint{2.030388in}{1.675419in}}%
\pgfpathlineto{\pgfqpoint{2.197054in}{1.675419in}}%
\pgfusepath{stroke}%
\end{pgfscope}%
\begin{pgfscope}%
\pgfsetbuttcap%
\pgfsetmiterjoin%
\definecolor{currentfill}{rgb}{0.862745,0.149020,0.498039}%
\pgfsetfillcolor{currentfill}%
\pgfsetlinewidth{1.003750pt}%
\definecolor{currentstroke}{rgb}{0.862745,0.149020,0.498039}%
\pgfsetstrokecolor{currentstroke}%
\pgfsetdash{}{0pt}%
\pgfsys@defobject{currentmarker}{\pgfqpoint{-0.031250in}{-0.031250in}}{\pgfqpoint{0.031250in}{0.031250in}}{%
\pgfpathmoveto{\pgfqpoint{-0.031250in}{-0.031250in}}%
\pgfpathlineto{\pgfqpoint{0.031250in}{-0.031250in}}%
\pgfpathlineto{\pgfqpoint{0.031250in}{0.031250in}}%
\pgfpathlineto{\pgfqpoint{-0.031250in}{0.031250in}}%
\pgfpathlineto{\pgfqpoint{-0.031250in}{-0.031250in}}%
\pgfpathclose%
\pgfusepath{stroke,fill}%
}%
\begin{pgfscope}%
\pgfsys@transformshift{2.030388in}{1.675419in}%
\pgfsys@useobject{currentmarker}{}%
\end{pgfscope}%
\end{pgfscope}%
\begin{pgfscope}%
\definecolor{textcolor}{rgb}{0.000000,0.000000,0.000000}%
\pgfsetstrokecolor{textcolor}%
\pgfsetfillcolor{textcolor}%
\pgftext[x=2.330388in,y=1.617086in,left,base]{\color{textcolor}{\rmfamily\fontsize{12.000000}{14.400000}\selectfont\catcode`\^=\active\def^{\ifmmode\sp\else\^{}\fi}\catcode`\%=\active\def%{\%}CN++}}%
\end{pgfscope}%
\end{pgfpicture}%
\makeatother%
\endgroup%

    \end{minipage}\hfill%
    \begin{minipage}[c]{.475\linewidth}
        \vspace{-35pt}
        \centering
\renewcommand{\arraystretch}{1.2}
\begin{tabular}{@{}lccccc@{}}
\toprule
 & $n_{\mtx{\Omega}}$ & $n_{\mtx{\Psi}}$ & $q$ & $r$ & time (s)\\
\midrule
KA (i) & $8$ & $0$ & $30$ & $50$ & $10.45$ \\
KA (ii) & $0$ & $13$ & $0$ & $50$ & $4.29$ \\
KA (iii) & $8$ & $13$ & $30$ & $50$ & $18.40$ \\
KA (iv) & $4$ & $6$ & $30$ & $50$ & $9.36$ \\
\bottomrule
\end{tabular}

        \newline
        \vspace{15pt}
        \newline
        \centering
\renewcommand{\arraystretch}{1.2}
\begin{tabular}{@{}lcccc@{}}
\toprule
 & $n_{\mtx{\Omega}}$ & $n_{\mtx{\Psi}}$ & $m$ & time (s)\\
\midrule
CN++ & $40$ & $40$ & $50$ & $23.04$ \\
\bottomrule
\end{tabular}

    \end{minipage}
    \caption{For the Heisenberg spin chain example, we compare the Chebyshev-Nyström++ (CN++) method with the Krylov-Aware stochastic trace estimator in the same configurations as in \cite[Table 5.1]{chen-2023-krylovaware-stochastic} (right). To this extent, we compute the $L^1$-error of the methods for various choices of the temperature parameter $\beta^{-1}$ (left).}
    \label{fig:krylov-aware-spin}
\end{figure}

Unlike in the spectral density example from \refsec{subsec:hamiltonian}, where the numerical rank of the matrix stays low for all $t$, it changes wildly with the parameter $\beta$. The Chebyshev-Nyström++ method requires a fixed allocation of random vectors to the low-rank approximation for all values of the parameter $\beta$. Hence, both for large $\beta$, where the matrix $\exp(-\beta \mtx{A})$ is low-rank, and for small $\beta$, where the singular values only decay slowly, we cannot expect a good approximation.
