\appendix

\color{blue}
\section{Moment bounds for non-standard Gaussian random vectors and matrices}

The goal of this section is to establish moment bounds for $\lVert \mtx{A} \mtx{\Omega} \rVert _2^2$ with a fixed matrix $\mtx{A}$ and a Gaussian random matrix $\mtx{\Omega}$. These bounds are needed in the proof of \reflem{lem:nystrom}, but may also be of independent interest.
%For reasons which will become clear in the proof of the final and most interesting of these lemmas (\reflem{lem:nystrom}), we will need to bound $\mathbb{E}^{\sfrac{p}{2}}\left[\lVert \mtx{A} \mtx{\Omega} \rVert _2^2 \right]$ for a constant matrix $\mtx{A} \in \mathbb{R}^{m \times m}$ and a standard Gaussian random matrix $\mtx{\Omega} \in \mathbb{R}^{m \times k}$.
The corresponding result for the Frobenius norm can be found in~\cite[Lemma 3]{kressner-2024-randomized-lowrank}. For the spectral norm, moment bounds for $\lVert \mtx{\Omega} \rVert _2^2$ (that is, $\mtx{A} = \mtx{I}$) are well established\cite{chen-2005-condition-numbers, edelman-1988-eigenvalues-condition, james-1964-distributions-matrix}. In \cite[Lemma B.1]{tropp-2023-randomized-algorithms} bounds on the first- and second-order moments for general $\mtx{A}$ are derived. In the following, we will generalize this result to moments of arbitrarily high order. To do so, we first establish two preliminary results.

\begin{lemma}{Moment bound of chi-squared random variable}{gamma}
    Let $X \sim \chi_k^2$, where $\chi_k^2$ denotes the chi-squared distribution with $k \ge 2$ degrees of freedom. Then 
    $
        \mathbb{E}^{{p}}[X] \leq k + p-1
        %\frac{\Gamma(\frac{k + p}{2})}{\Gamma(\frac{k}{2})} \leq \left( \frac{k + p}{2} \right)^{\sfrac{p}{2}}.
    $ holds for every $p \ge 1$, $p\in \mathbb R$.
\end{lemma}%
\begin{proof}
It is well known that
%    From \cite[Theorem 3.3.2]{hogg-2013-introduction-mathematical} we know
    $
        \mathbb{E}^{p}[X] = 2 \Big( \frac{\Gamma(k/2+p)}{\Gamma(k/2)} \Big)^{\sfrac{1}{p}}.
    $ For every $\alpha \ge 0$, $\beta \ge 2$ the bound $\frac{\Gamma(\alpha+\beta)}{\Gamma(\alpha + 1)} \le (\alpha+\beta/2)^{\beta-1}$ holds~\cite[Equation 2.2]{laforgia-1984-further-inequalities}. The result of the lemma follows by using this bound with $\alpha = k/2-1$ and $\beta = p+1$.
%     We first treat the case $k=p=1$, for which indeed
%     \begin{equation}
%         2 \left( \frac{\Gamma(\frac{1+1}{2})}{\Gamma(\frac{1}{2})} \right)^2 = \frac{2}{\pi} \leq 2 = 1 + 1
%     \end{equation}
%     So we now assume $k + p \geq 3$. We can use a telescoping product and the property $\Gamma(z+1)/\Gamma(z) = z$ to bound
%     \begin{align}
%         \frac{\Gamma(\frac{k + p}{2})}{\Gamma(\frac{k}{2})} 
%         &= \frac{\Gamma(\frac{k}{2} + 1)}{\Gamma(\frac{k}{2})} \frac{\Gamma(\frac{k}{2} + 2)}{\Gamma(\frac{k}{2} + 1)} \cdots \frac{\Gamma(\frac{k}{2} + \lfloor \frac{p}{2} \rfloor)}{\Gamma(\frac{k}{2} + \lfloor \frac{p}{2} \rfloor - 1)} \frac{\Gamma( \frac{k + p}{2})}{\Gamma(\frac{k}{2} + \lfloor \frac{p}{2} \rfloor)} \notag \\
%         &= \left(\frac{k}{2}\right) \left(\frac{k}{2} + 1\right) \cdots \left( \frac{k}{2} + \bigg\lfloor \frac{p}{2} \bigg\rfloor - 1 \right) \frac{\Gamma( \frac{k + p}{2})}{\Gamma(\frac{k}{2} + \lfloor \frac{p}{2} \rfloor)} \notag \\
%         &\leq \left(\frac{k}{2} + \bigg\lfloor \frac{p}{2} \bigg\rfloor - 1\right)^{\lfloor \sfrac{p}{2} \rfloor} \frac{\Gamma( \frac{k + p}{2})}{\Gamma(\frac{k}{2} + \lfloor \frac{p}{2} \rfloor)}
%     \end{align}
%     If $p$ is even, then $\lfloor p/2 \rfloor = p/2$ and hence
%     \begin{equation}
%         \frac{\Gamma(\frac{k + p}{2})}{\Gamma(\frac{k}{2})} 
%         \leq \left( \frac{k + p}{2} - 1 \right)^{\sfrac{p}{2}}
%         \leq \left( \frac{k + p}{2}\right)^{\sfrac{p}{2}}.
%     \end{equation}
%     If $p$ is odd, then $\lfloor p/2 \rfloor = (p - 1)/2$, from which follows with Gautschi's inequality \cite{kershaw-1983-extensions-gautschi}
%     \begin{equation}
%         \frac{\Gamma(\frac{k + p}{2})}{\Gamma(\frac{k}{2})}
%         \leq \left(\frac{k + p}{2} - \frac{3}{2} \right)^{ \frac{p - 1}{2}} \frac{\Gamma( \frac{k + p}{2})}{\Gamma(\frac{k + p}{2} - \frac{1}{2})}
%         \leq \left(\frac{k + p}{2}\right)^{ \frac{p - 1}{2}} \left(\frac{k + p}{2}\right)^{\sfrac{1}{2}}
%         = \left(\frac{k + p}{2}\right)^{ \sfrac{p}{2}}
%     \end{equation}
%     Hence,
%     \begin{equation}
%         \mathbb{E}^{\sfrac{p}{2}}[X] \leq 2 \left( \frac{k + p}{2} \right) \leq k + p.
%     \end{equation}
\end{proof}



\begin{lemma}{Spectral norm moments of non-standard Gaussian random vector}{spectral-norm-moment-vector}
    Given $\mtx{A} \in \mathbb{R}^{m \times m}$ and a Gaussian random vector $\vct{\omega} \in \mathbb{R}^{m}$, the bound
    \[
        \mathbb{E}^{p}\big[ \lVert \mtx{A} \vct{\omega} \rVert _2^2 \big]
        \leq  (k + p - 1) \Big( \lVert \mtx{A} \rVert _2^2 + \frac{1}{k} \lVert \mtx{A} \rVert _F^2 \Big).
    \]
    %\begin{equation}
    %    \mathbb{E}^{p}\left[ \lVert \mtx{\Sigma} \mtx{\Omega}_2\rVert _2 \right]
    %    &\leq \sqrt{\frac{k + p}{2}} \cdot \left( 2 + \frac{1}{\sqrt{k}} \right) \cdot \lVert \mtx{\Sigma} \rVert _2 + \sqrt{\frac{k + p}{2k}} \cdot \lVert \mtx{\Sigma} \rVert _F. \notag \\
    %    \mathbb{E}^{p}\left[ \lVert \mtx{\Omega}_1^{\dagger} \rVert _2 \right]
    %    &\leq \frac{1}{2} \left( 1 + \frac{p}{k - r + 1 - p} \right)^{\sfrac{1}{p}} \left( \frac{k + r}{k - r + 2} \right).
    %\end{equation}
    %If we choose 
    holds for every $k \ge 2$, $k\in \mathbb{N}$, and $p \ge 1$, $p\in \mathbb R$.
\end{lemma}%
% \begin{remark}
%     In particular, this result can also be used to bound
%     \begin{equation}
%         \mathbb{E}^{p}\left[ \lVert \mtx{A} \vct{\omega} \rVert _2 \right] = \sqrt{\mathbb{E}^{\sfrac{p}{2}}\left[ \lVert \mtx{A} \vct{\omega} \rVert _2^2 \right]} \leq \sqrt{k + p} \cdot \left(\lVert \mtx{A} \rVert _2 + \frac{1}{\sqrt{k}}\lVert \mtx{A} \rVert _F\right)
%     \end{equation}
%     for any $k,p \in \mathbb{N}$.
% \end{remark}
\begin{proof}
By the unitary invariance of Gaussian random vectors, we may assume w.l.o.g. that $\mtx{A} = \mtx{\Sigma} = \operatorname{diag}(\sigma_1, \dots, \sigma_m)$ with $\sigma_1 \geq \dots \geq \sigma_m \geq 0$.
Following the proof of \cite[Theorem 1]{cohen-2016-optimal-approximate}, we split the singular values into $\ell = \lceil m/k \rceil$ groups of size $k$:
    \begin{equation}
        \overbrace{\underbrace{\sigma_1, \dots, \sigma_k}_{\leq \sigma_1}}^{\geq \sigma_{k+1}}, \overbrace{\underbrace{\sigma_{k+1}, \dots, \sigma_{2k}}_{\leq \sigma_{k+1}}}^{\geq \sigma_{2k+1}}, \dots, \overbrace{\underbrace{\sigma_{(\ell - 1)k + 1}, \dots, \sigma_{\ell k}}_{\leq \sigma_{(\ell - 1)k + 1}}}^{\geq 0}.
    \end{equation}
    If $m$ is not a multiple of $k$, we set $\sigma_i = 0$ for $i > m$. Using that
    %Since $\sigma_1 \geq \dots \geq \sigma_{\ell k} \geq 0$,
    %\begin{equation}
     %   \sigma_1^2 = \lVert \mtx{\Sigma} \rVert _2^2
      %  \quad \text{and} \quad
        $\sigma_{ik + 1}^2 \leq ( \sigma_{(i-1)k + 1} + \cdots + \sigma_{ik - 1} + \sigma_{ik} ) / k$ for $i = 1,\ldots, \ell-1$, we get
    \begin{equation}
        \sum_{i=0}^{\ell-1} \sigma_{ik + 1}^2 = \lVert \mtx{\Sigma} \rVert _2^2 + \sum_{i=1}^{\ell-1} \sigma_{ik + 1}^2  \leq \lVert \mtx{\Sigma} \rVert _2^2 + \frac{1}{k} \sum_{j=1}^{(\ell - 1)k} \sigma_j^2 \leq \lVert \mtx{\Sigma} \rVert _2^2 + \frac{1}{k} \lVert \mtx{\Sigma} \rVert _F^2.
        \label{equ:singular-value-group-bound}
    \end{equation}
    This allows us to bound
    \begin{align*}
    \mathbb{E}^{p}\big[ \lVert \mtx{\Sigma} \vct{\omega} \rVert _2^2 \big] & = 
        \mathbb{E}^{p}\Big[ \sum_{i=1}^{m} \sigma_i^2 \omega_i^2 \Big]
        = \mathbb{E}^{p}\Big[ \sum_{i=0}^{\ell - 1} \sum_{j=1}^{k} \sigma_{ik + j}^2 \omega_{ik + j}^2 \Big] \\
        &\leq \mathbb{E}^{p}\Big[ \sum_{i=0}^{\ell - 1} \sigma_{ik + 1}^2 \sum_{j=1}^{k} \omega_{ik + j}^2 \Big] 
        \leq \sum_{i=0}^{\ell - 1} \sigma_{ik + 1}^2 ~ \mathbb{E}^{p}\Big[ \sum_{j=1}^{k} \omega_{ik + j}^2 \Big] \\
        &\leq (k + p - 1) \sum_{i=0}^{\ell - 1} \sigma_{ik + 1}^2 \leq (k + p - 1) \Big( \lVert \mtx{\Sigma} \rVert _2^2 + \frac{1}{k} \lVert \mtx{\Sigma} \rVert _F^2 \Big), 
    \end{align*}
    where we used the triangle inequality, \reflem{lem:gamma}, and \refequ{equ:singular-value-group-bound} for the last three inequalities.
    %The moments of $\lVert \mtx{\Omega}_1^{\dagger} \rVert _2$ can be bound with help of the proof of \cite[Lemma B.3]{tropp-2023-randomized-algorithms}
    %\begin{align}
    %    \mathbb{E}^{p}\left[ \lVert \mtx{\Omega}_1^{\dagger} \rVert _2 \right]
    %    &= \mathbb{E}\left[ \lVert ( \mtx{\Omega}_1 \mtx{\Omega}_1^{\top} )^{-1} \rVert _2^{p} \right]^{\sfrac{1}{p}} \notag \\
    %    &\leq \left( 1 + \frac{p}{k - r + 1 - p} \right)^{\sfrac{1}{p}} \left( \frac{1}{\Gamma(k - r + 2)} \right)^{\frac{1}{k - r + 1}} \left( \frac{k + r}{2} \right) \notag \\
    %    &\leq \frac{1}{2} \left( 1 + \frac{p}{k - r + 1 - p} \right)^{\sfrac{1}{p}} \left( \frac{k + r}{k - r + 2} \right)
    %\end{align}
    %With the Taylor series expansion of the exponential function it can be shown that $e^n \geq 1 + n$ and $e^n \geq \frac{n^n}{n!}$ for all $n \in \mathbb{N}$, from which $(n + 1)^{\sfrac{1}{n}} \leq e$ and $\left( \frac{1}{n!} \right)^{\sfrac{1}{n}} \leq \frac{e}{n}$ follow respectively. Hence,
    %\begin{equation}
    %    \mathbb{E}^{k}\left[ \lVert \mtx{\Omega}_1^{\dagger} \rVert _2 \right]
    %    \leq \frac{e^2}{k + 1}\sqrt{ \frac{3k}{2} }
    %    \leq e^2 \sqrt{\frac{3}{2k}}.
    %    \label{equ:OSE-moment-bound-second}
    %\end{equation}
\end{proof}

\begin{lemma}{Spectral norm moments of non-standard Gaussian random matrix}{spectral-norm-moment}
    Given $\mtx{A} \in \mathbb{R}^{m \times m}$ and a Gaussian random matrix $\mtx{\Omega} \in \mathbb{R}^{m \times k}$, the bound
    \begin{equation}
        \mathbb{E}^{p}\left[ \lVert \mtx{A} \mtx{\Omega} \rVert _2^2 \right]
        \leq  (k + p-1) \Big( 2 \lVert \mtx{A} \rVert _2^2 + \frac{1}{k} \lVert \mtx{A} \rVert _F^2 \Big).
    \end{equation}
    holds for every $k \ge 2$, $k\in \mathbb{N}$, and $p \ge 1$, $p\in \mathbb R$.
    %\begin{equation}
    %    \mathbb{E}^{p}\left[ \lVert \mtx{\Sigma} \mtx{\Omega}_2\rVert _2 \right]
    %    &\leq \sqrt{\frac{k + p}{2}} \cdot \left( 2 + \frac{1}{\sqrt{k}} \right) \cdot \lVert \mtx{\Sigma} \rVert _2 + \sqrt{\frac{k + p}{2k}} \cdot \lVert \mtx{\Sigma} \rVert _F. \notag \\
    %    \mathbb{E}^{p}\left[ \lVert \mtx{\Omega}_1^{\dagger} \rVert _2 \right]
    %    &\leq \frac{1}{2} \left( 1 + \frac{p}{k - r + 1 - p} \right)^{\sfrac{1}{p}} \left( \frac{k + r}{k - r + 2} \right).
    %\end{equation}
    %If we choose 
\end{lemma}
\begin{proof}
    By the proof of \cite[Lemma B.1]{tropp-2023-randomized-algorithms}, we have that
    %it is shown that Slepian's inequality \cite[Theorem 7.2.1]{vershynin-2018-highdimensional-probability} applied to two appropriately constructed Gaussian random fields yields
    \begin{equation}
        \mathbb{E}^{p}\big[ \lVert \mtx{A} \mtx{\Omega} \rVert _2^2 \big]
        \leq \lVert \mtx{A} \rVert _2^2 \mathbb{E}^{p}\big[ \lVert \vct{\omega}_1 \rVert _2^2 \big] + \mathbb{E}^{p}\big[ \lVert \mtx{A} \vct{\omega}_2 \rVert _2^2 \big]
    \end{equation}
    for Gaussian random vectors $\vct{\omega}_1 \in \mathbb{R}^{k}$ and $\vct{\omega}_2 \in \mathbb{R}^{m}$. The claimed result follows from applying \reflem{lem:gamma} and \reflem{lem:spectral-norm-moment-vector}:
    \[
        \mathbb{E}^{p}\big[ \lVert \mtx{A} \mtx{\Omega} \rVert _2^2 \big]
        \leq (k + p - 1) \lVert \mtx{A} \rVert _2^2  + (k + p - 1) \Big( \lVert \mtx{A} \rVert _2^2 + \frac{1}{k} \lVert \mtx{A} \rVert _F^2 \Big).
    \]
    %The moments of $\lVert \mtx{\Omega}_1^{\dagger} \rVert _2$ can be bound with help of the proof of \cite[Lemma B.3]{tropp-2023-randomized-algorithms}
    %\begin{align}
    %    \mathbb{E}^{p}\left[ \lVert \mtx{\Omega}_1^{\dagger} \rVert _2 \right]
    %    &= \mathbb{E}\left[ \lVert ( \mtx{\Omega}_1 \mtx{\Omega}_1^{\top} )^{-1} \rVert _2^{p} \right]^{\sfrac{1}{p}} \notag \\
    %    &\leq \left( 1 + \frac{p}{k - r + 1 - p} \right)^{\sfrac{1}{p}} \left( \frac{1}{\Gamma(k - r + 2)} \right)^{\frac{1}{k - r + 1}} \left( \frac{k + r}{2} \right) \notag \\
    %    &\leq \frac{1}{2} \left( 1 + \frac{p}{k - r + 1 - p} \right)^{\sfrac{1}{p}} \left( \frac{k + r}{k - r + 2} \right)
    %\end{align}
    %With the Taylor series expansion of the exponential function it can be shown that $e^n \geq 1 + n$ and $e^n \geq \frac{n^n}{n!}$ for all $n \in \mathbb{N}$, from which $(n + 1)^{\sfrac{1}{n}} \leq e$ and $\left( \frac{1}{n!} \right)^{\sfrac{1}{n}} \leq \frac{e}{n}$ follow respectively. Hence,
    %\begin{equation}
    %    \mathbb{E}^{k}\left[ \lVert \mtx{\Omega}_1^{\dagger} \rVert _2 \right]
    %    \leq \frac{e^2}{k + 1}\sqrt{ \frac{3k}{2} }
    %    \leq e^2 \sqrt{\frac{3}{2k}}.
    %    \label{equ:OSE-moment-bound-second}
    %\end{equation}
\end{proof}


In passing, we note that \reflem{lem:spectral-norm-moment}  yields the bound
    \begin{equation}
        \mathbb{E}^{p}\big[ \lVert \mtx{A} \mtx{\Omega} \rVert _2 \big] = \sqrt{\mathbb{E}^{\sfrac{p}{2}}\big[ \lVert \mtx{A} \mtx{\Omega} \rVert _2^2 \big]} \leq \sqrt{k + p/2-1} \cdot \Big(\sqrt{2} \lVert \mtx{A} \rVert _2 + \frac{1}{\sqrt{k}}\lVert \mtx{A} \rVert_F\Big)
    \end{equation}
    with $k\ge 2, p\ge 2$. For $p = 2$, this nearly matches the corresponding bound from~\cite[Lemma B.1]{tropp-2023-randomized-algorithms}, except for the additional factor $\sqrt{2}$.

\color{black}
%\section{Moment bound for simple trace estimator}
%
%In this section we will bound the high order moments of the residual of the single-query Girard-Hutchinson estimator \refequ{equ:hutchinson-trace-estimator}, i.e. when $n_{\mtx{\Psi}} = 1$. This bound is needed in the proof of \refthm{thm:hutchinson}.
%
%\begin{lemma}{Moment bound for single-query Girard-Hutchinson estimator}{hanson-wright-moments}
%    Given a symmetric $\mtx{A} \in \mathbb{R}^{m \times m}$ and a standard Gaussian random vector $\vct{\psi} \in \mathbb{R}^{m}$, then
%    \begin{equation}
%        \mathbb{E}^{p}\left[\Trace(\mtx{A}) - \vct{\psi}^{\top} \mtx{A} \vct{\psi}\right] \leq c \max\{ \sqrt{p} \lVert \mtx{A} \rVert _F, p \lVert \mtx{A} \rVert _2 \}
%    \end{equation}
%    holds for any $p \geq 1, p \in \mathbb{R}$ and a universal constant $c > 0$.
%\end{lemma}
%
%From \cite[Theorem 1.1]{rudelson-2013-hansonwright-inequality} it holds
%\begin{equation}
%    \mathbb{P}(|\Trace(\mtx{A}) - \vct{\psi}^{\top} \mtx{A} \vct{\psi}| > t) \leq 2 \exp \left( - c \min \left\{ \frac{t^2}{\lVert \mtx{A} \rVert _F^2},  \frac{t}{\lVert \mtx{A} \rVert _2} \right\} \right)
%\end{equation}
%for some universal constant $c > 0$. Therefore we can directly compute the high-order moments with an integral
%\begin{align}
%    &\mathbb{E}\left[ |\Trace(\mtx{A}) - \vct{\psi}^{\top} \mtx{A} \vct{\psi}|^p \right] \notag \\
%    &= p \int_{0}^{\infty} t^{p-1} \mathbb{P}(|\Trace(\mtx{A}) - \vct{\psi}^{\top} \mtx{A} \vct{\psi}| > t) \mathrm{d}t \notag \\
%    &\leq 2 p \int_{0}^{\infty} t^{p-1} \exp \left( - c \min \left\{ \frac{t^2}{\lVert \mtx{A} \rVert _F^2},  \frac{t}{\lVert \mtx{A} \rVert _2} \right\} \right) \mathrm{d}t \notag \\
%    &\leq 4 p \max\left\{ \int_{0}^{\infty} t^{p-1} \exp \left( - c \frac{t^2}{\lVert \mtx{A} \rVert _F^2} \right) \mathrm{d}t, \int_{0}^{\infty} t^{p-1} \exp \left( - c \frac{t}{\lVert \mtx{A} \rVert _2} \right) \mathrm{d}t \right\} \notag \\
%    %&\leq 4 p \max\left\{ \int_{0}^{\infty} t^{p-1} \exp \left( - c \frac{t^2}{\lVert \mtx{A} \rVert _F^2} \right) \mathrm{d}t, \int_{0}^{\infty} t^{p-1} \exp \left( - c \frac{t}{\lVert \mtx{A} \rVert _2} \right) \mathrm{d}t \right\} \notag \\
%    &=  4 p \max\left\{ \frac{1}{2} \left( \frac{\lVert \mtx{A} \rVert _F}{\sqrt{c}} \right)^p \Gamma\left(\frac{p}{2}\right), \left(\frac{\lVert \mtx{A} \rVert _2}{c} \right)^p \Gamma(p) \right\}
%\end{align}
%Since $\Gamma(p)^{\sfrac{1}{p}} \leq p$ and $\Gamma(p/2)^{\sfrac{1}{p}} \leq \sqrt{p \pi}$ for $p \geq 1$, we have 
%\begin{equation}
%    \mathbb{E}^p\left[ \Trace(\mtx{A}) - \vct{\psi}^{\top} \mtx{A} \vct{\psi} \right]
%    \leq \widetilde{c} \max\{ \sqrt{p} \lVert \mtx{A} \rVert _F, p \lVert \mtx{A} \rVert _2 \}
%\end{equation}
%for some other universal constant $\widetilde{c} > 0$.

%\section{Moment bounds for Girard-Hutchinson estimator}

%In this section we first recall that the (centered) single-query Girard-Hutchinson estimator~\refequ{equ:hutchinson-trace-estimator} is sub-exponential. Combined with a bound on moments of sub-exponential random variables, this will be used in the proof of \refthm{thm:hutchinson}.

%\textcolor{red}{Connection between sub-gamma and sub-exponential?}
%\begin{lemma}{Sub-exponential property of $1$-query Girard-Hutchinson estimator}{hutchinson-sub-exponential}
%    Let $\mtx{A} \in \mathbb{R}^{m \times m}$ be symmetric and $\vct{\psi} \in \mathbb{R}^m$ a Gaussian random vector. Then the real random variable $\Trace(\mtx{A}) - \vct{\psi}^{\top} \mtx{A} \vct{\psi}$ is centered and $(2 \lVert \mtx{A} \rVert _F^2 (1 - 1/\beta)^{-1}, 2 \lVert \mtx{A} \rVert _2 \beta )$-sub-exponential for any $\beta > 1, \beta \in \mathbb{R}$.
%\end{lemma}

%\begin{proof}
%    The proof starts similarly as the proof of \cite[Lemma 3]{cortinovis-2022-randomized-trace}. The random variable $\Trace(\mtx{A}) - \vct{\psi}^{\top} \mtx{A} \vct{\psi}$ is centered because $\mathbb{E}[\vct{\psi}^{\top} \mtx{A} \vct{\psi}] = \operatorname{Tr}(\mtx{A})$. With the eigenvalue decomposition of $\mtx{A} = \mtx{U} \mtx{\Lambda} \mtx{U}^{\top}$ and the orthogonality of $\mtx{U}$, we can rewrite
%    \begin{equation}
%        \Trace(\mtx{A}) - \vct{\psi}^{\top} \mtx{U}\mtx{\Lambda} \mtx{U}^{\top} \vct{\psi}  \sim \Trace(\mtx{A}) - \vct{\psi}^{\top} \mtx{\Lambda} \vct{\psi}  = \sum_{i=1}^{m} \lambda_i (1 - \psi_i^2) = \sum_{i=1}^{m} X_i.
%    \end{equation}
%    The random variables $X_i = \lambda_i(1 - \psi_i^2)$ are independent and defined by the standard Gaussian random variables $\psi_i$ for $i=1, \dots, m$. A simple integration shows that
%    \begin{equation}
%        \mathbb{E}[\exp(t X_i)] = \frac{\exp\left(t \lambda_i\right) }{\sqrt{1 + 2t \lambda_i}} = \exp\left(t \lambda_i - \frac{1}{2} \log(1 + 2t \lambda_i) \right)
%    \end{equation}
%    if $t \lambda_i > - 1/2$. To respect this restriction, we take one sub-exponential parameter as $\alpha_i := 2 |\lambda_i| \beta$ for an arbitrary $\beta > 1, \beta \in \mathbb{R}$. We now want to find a good value for the other sub-exponential parameter $\nu_i^2$. For all $|t| < 1 / \alpha_i$ (where $1/0$ should be understood as $+ \infty$), some standard calculus manipulations show that
%    \begin{equation}
%        t \lambda_i - \frac{1}{2} \log(1 + 2t \lambda_i) \leq \frac{t^2 \nu_i^2}{2},
%    \end{equation}
%    as long as we choose
%    \begin{equation}
%        \nu_i^2 = \sup_{|t| < 1 / \alpha_i} \frac{2 \lambda_i^2}{1 + 2 t \lambda_i} = \frac{2 \lambda_i^2}{1 - 2 |\lambda_i| / \alpha_i} = \frac{2 \lambda_i^2}{1 - 1 / \beta}.
%    \end{equation}
%    This shows that the random variables $X_i$ are $(\nu_i^2, \alpha_i)$-sub-exponential.

%    The sum of independent $(\nu_i^2, \alpha_i)$-sub-exponential random variables is $(\sum_i \nu_i^2, \max_i \alpha_i)$-sub-exponential. Therefore, $\Trace(\mtx{A}) - \vct{\psi}^{\top} \mtx{A} \vct{\psi}$ is $(2 \lVert \mtx{A} \rVert _F^2 (1 - 1/\beta)^{-1}, 2 \lVert \mtx{A} \rVert _2 \beta)$-sub-exponential for any $\beta > 1, \beta \in \mathbb{R}$.
%\end{proof}


%It is well-known that $\mathbb{E}^p[X] \lesssim p$ for a sub-exponential random variable~\cite[Proposition 2.7.1]{vershynin-2018-highdimensional-probability}. The following result provides a bound that pays attention to  the constants involved..
%\begin{lemma}{Moment bound for sub-exponential random variable}{sub-exponential-moments}
%    Let $X$ be a centered, $(\nu^2, \alpha)$-sub-exponential random variable. Then
%    \begin{equation}
%        \mathbb{E}^p[X] \leq \sqrt{2 \pi p} \nu + 4 p \alpha
%    \end{equation}
%    holds for any $p \geq 1, p \in \mathbb{R}$.
%\end{lemma}
%
%\begin{proof}
%    Let $X$ be a centered, $(\nu^2, \alpha)$-sub-exponential random variable and $p \geq 1, p \in \mathbb{R}$. With an integration by parts and \cite[Proposition 2.9]{wainwright-2019-highdimensional-statistics}, we can bound
%    \begin{align}
%        \mathbb{E}[|X|^p] 
%        &= p \int_{0}^{\infty} t^{p-1} \mathbb{P}(|X| > t)~\mathrm{d}t \notag \\
%        &= p \int_{0}^{\sfrac{\nu^2}{ \alpha}} t^{p-1} \mathbb{P}(|X| > t)~\mathrm{d}t + p \int_{\sfrac{\nu^2}{ \alpha}}^{\infty} t^{p-1} \mathbb{P}(|X| > t)~\mathrm{d}t \notag \\
%        &\leq 2 p \int_{0}^{\sfrac{\nu^2}{ \alpha}} t^{p-1} \exp\left(-\frac{1}{2} \frac{t^2}{\nu^2} \right)~\mathrm{d}t + 2 p \int_{\sfrac{\nu^2}{ \alpha}}^{\infty} t^{p-1} \exp\left(-\frac{1}{2} \frac{t}{\alpha} \right)~\mathrm{d}t \notag \\
%        &\leq 2 p \int_{0}^{\infty} t^{p-1} \exp\left(-\frac{1}{2} \frac{t^2}{\nu^2} \right)~\mathrm{d}t + 2 p \int_{0}^{\infty} t^{p-1} \exp\left(-\frac{1}{2} \frac{t}{\alpha} \right)~\mathrm{d}t \notag \\
%        &= p 2^{\sfrac{p}{2}} \nu^p \Gamma\left(\frac{p}{2}\right) + 2 p 2^{p} \alpha^p \Gamma\left(p\right) \notag \\
%        &= 2^{\sfrac{p}{2} + 1} \Gamma\left(\frac{p}{2} + 1\right) \nu^p  + 2^{p + 1}  \Gamma\left(p + 1\right) \alpha^p.
%        \label{equ:sub-exponential-moment-bound}
%    \end{align}
%    Consequently, using $(a + b)^{\sfrac{1}{p}} \leq a^{\sfrac{1}{p}} + b^{\sfrac{1}{p}}$ for all $a,b \geq 0$ and when $p \geq 1$, we get
%    %\begin{equation}
%    %    \mathbb{E}^p[X] 
%    %    \leq p^{\sfrac{1}{p}} \sqrt{2} \nu \Gamma\left(\frac{p}{2}\right)^{\sfrac{1}{p}} + p^{\sfrac{1}{p}} 2^{\frac{p+1}{p}} \alpha \Gamma\left(p\right)^{\sfrac{1}{p}}
%    %    \leq e^{\sfrac{1}{e}} (\sqrt{2 \pi p} \nu + 4 p \alpha),
%    %\end{equation}
%    \begin{equation}
%        \mathbb{E}^p[X] 
%        \leq \left(2^{\sfrac{p}{2} + 1} \Gamma\left(\frac{p}{2} + 1\right)\right)^{\sfrac{1}{p}} \nu  + \left(2^{p + 1}  \Gamma\left(p + 1\right)\right)^{\sfrac{1}{p}} \alpha
%        \leq \sqrt{2 \pi p} \nu  + 4 p \alpha,
%    \end{equation}
%    where we made use of basic bounds for the Gamma function in the last step.
%\end{proof}

%    We will need
%    \begin{equation}
%        \leq \frac{\mu_i^2 t^2}{2} \implies \mathbb{E}[\exp(t Y_i)] \leq \exp\left(\frac{\mu_i^2 t^2}{2}\right)
%    \end{equation}
%    \todo{cite \cite[Lemma 3]{cortinovis-2022-randomized-trace}}
%    where we needed to choose $\mu_i^2 \geq 2 (1 - 2t)^{-1}$ (this restriction can be shown by evaluating the above expression at $t=0$ and analyzing its derivatives).
%
%    Consequently, from the independence of $Y_i$, and by restricting $|t| < (2 \lVert \boldsymbol{A} \rVert _2)^{-1}$ (such that we can ensure $|t \lambda_i| < 1/2$ for all $i$)
%    \begin{align*}
%        \mathbb{E}[\exp(tX)]
%        &= \mathbb{E}[\exp(t \sum_{i=1}^{n} \lambda_i Y_i)] \\
%        &= \prod_{i=1}^{n} \mathbb{E}[\exp(t \lambda_i Y_i)] \\
%        &\leq \prod_{i=1}^{n} \exp\left(\frac{\mu^2 t^2 \lambda_i^2}{2}\right) \\
%        &= \exp\left(\frac{\mu^2 t^2 \lVert \boldsymbol{A} \rVert _F^2}{2}\right)
%    \end{align*}
%
%    for (a different) $\mu^2 \geq \max_i\{ 2 (1 - 2 \lambda_i t)^{-1}\} =  2 (1 - 2 \lVert \boldsymbol{A} \rVert _2 t)^{-1}$.
%
%
%    Now, since the "constant" $\mu^2$ still depends on $t$, and technically needs to be $+\infty$ at the maximum admissible value for $t$, we choose a (possibly smaller) sub-exponential parameter $\alpha := d \cdot (2 \lVert \boldsymbol{A} \rVert _2)$ for any constant $d > 1$, such that we can then restrict our view to $|t| \leq \alpha^{-1} < (2 \lVert \boldsymbol{A} \rVert _2)^{-1}$. The other parameter is fixed as $\nu^2 := (1 - 1/d)^{-1} \cdot (2\lVert \boldsymbol{A} \rVert _F^2)$ such that
%    \begin{equation*}
%    \nu^2 / \lVert \boldsymbol{A} \rVert _F^2 = \frac{2}{1 - 1/d} = \frac{2}{1 - 2 \lVert \boldsymbol{A} \rVert _2 / \alpha} \geq \frac{2}{1 - 2 \lVert \boldsymbol{A} \rVert _2 |t| } \geq \frac{2}{1 - 2 \lVert \boldsymbol{A} \rVert _2 t} 
%    \end{equation*}
%
%    as is required.

%\textcolor{red}{NEW!!!}

\section{Moment bounds for sub-gamma random variables}

A real centered random variable $X$ is called sub-gamma with parameters $(\nu,c)$ if its moment generating function satisfies
\[
 \mathbb E[ \exp( tX ) ] \le \exp\left( \frac{ t^2 \nu}{2(1-c|t|)} \right) \ \text{for every} \ 0 < |t| < 1/c.
\]
Any such $X$ satisfies the following tail bound~\cite[P. 29]{boucheron-2013-concentration-inequalities}:
\begin{equation}
  \label{equ:subgammatail}
  \mathbb P\big(|X| > \sqrt{2\nu t} + ct\big) \le 2 e^{-t} \ \text{for every} \ t>0.
\end{equation}

\begin{lemma}{Moment bounds for sub-gamma random variable}{sub-gamma-moments}
    Let $X$ be a centered, $(\nu,c)$-sub-gamma random variable for $\nu,c>0$. Then
    \begin{equation}
        \mathbb{E}^p[X] \leq 2 \sqrt{2 \nu p} + 4 c p
    \end{equation}
    holds for any $p \geq 1$, $p \in \mathbb{R}$.
\end{lemma}
\begin{proof}
The statement follows from a straightforward extension of the proof of~\cite[Theorem 2.3]{boucheron-2013-concentration-inequalities} from  even integers $p$ to general $p$. We include the detailed argument for completeness. Using integration by parts, the reparametrization $x = \sqrt{2\nu t} + ct$, and~\refequ{equ:subgammatail} one gets
\begin{align*}
 \mathbb{E}\big[|X|^p\big] &= p \int_0^\infty  |x|^{p-1} \mathbb P\big(|X| > x\big) \,\mathrm{d}x \\
 &=  p \int_0^\infty  (\sqrt{2\nu t} + ct)^{p-1} \mathbb P\big(|X| > \sqrt{2\nu t} + ct \big) \frac{ \sqrt{2\nu t} + 2ct}{2t} \,\mathrm{d}t \\
 &\le p  \int_0^\infty  (\sqrt{2\nu t} + 2ct)^{p}  \frac{e^{-t}}{t} \,\mathrm{d}t.
\end{align*}
Using the inequality $( (a+b)/2 )^p \le (a^p + b^p)/2$ implied by the convexity of $x^p$ on $[0,\infty)$,
we have that
\begin{align*}
 \mathbb{E}\big[|X|^p\big] & \le 
 p 2^{p-1}   \int_0^\infty  \big( (2\nu t)^{p/2} + (2ct)^{p}\big)  \frac{e^{-t}}{t} \,\mathrm{d}t \\
 &= p 2^{p-1} \big( (2\nu)^{p/2} \Gamma(p/2) + (2c)^p  \Gamma(p) \big) \\
 &= 2^{p} (2\nu)^{p/2} \Gamma(p/2+1) + 2^{p-1} (2c)^p  \Gamma(p+1) \big.
\end{align*}
Using $(a+b)^{1/p} \le a^{1/p} + b^{1/p}$, we obtain
\[
\mathbb{E}^p[X] \leq 2 \sqrt{2\nu} \Gamma(p/2+1)^{1/p}  + 4 c \Gamma(p+1)^{1/p}.
\]
Using $\Gamma(p/2+1)^{1/p} \le \sqrt{p}$ and $\Gamma(p+1)^{1/p} \le p$ completes the proof.

\end{proof}





