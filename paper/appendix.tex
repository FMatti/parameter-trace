\appendix

\color{blue}
\section{Moment bounds for non-standard Gaussian random vectors and matrices}

The goal of this section is to establish moment bounds for $\lVert \mtx{A} \mtx{\Omega} \rVert _2^2$ with a fixed matrix $\mtx{A}$ and a Gaussian random matrix $\mtx{\Omega}$. These bounds are needed in the proof of \reflem{lem:nystrom}, but may also be of independent interest.
%For reasons which will become clear in the proof of the final and most interesting of these lemmas (\reflem{lem:nystrom}), we will need to bound $\mathbb{E}^{\sfrac{p}{2}}\left[\lVert \mtx{A} \mtx{\Omega} \rVert _2^2 \right]$ for a constant matrix $\mtx{A} \in \mathbb{R}^{m \times m}$ and a standard Gaussian random matrix $\mtx{\Omega} \in \mathbb{R}^{m \times k}$.
The corresponding result for the Frobenius norm can be found in~\cite[Lemma 3]{kressner-2024-randomized-lowrank}. For the spectral norm, moment bounds for $\lVert \mtx{\Omega} \rVert _2^2$ (that is, $\mtx{A} = \mtx{I}$) are well established\cite{chen-2005-condition-numbers, edelman-1988-eigenvalues-condition, james-1964-distributions-matrix}. In \cite[Lemma B.1]{tropp-2023-randomized-algorithms} bounds on the first- and second-order moments for general $\mtx{A}$ are derived. In the following, we will generalize this result to moments of arbitrary high order. To do so, we first establish two preliminary results.

\begin{lemma}{Moment bound of $\chi^2$-random variable}{gamma}
    Let $X$ follow a $\chi^2$ distribution with $k \ge 2$, $k\in \mathbb{N}$, degrees of freedom. Then 
    $
        \mathbb{E}^{{p}}[X] \leq k + p-1
        %\frac{\Gamma(\frac{k + p}{2})}{\Gamma(\frac{k}{2})} \leq \left( \frac{k + p}{2} \right)^{\sfrac{p}{2}}.
    $ holds for every $p \ge 1$, $p\in \mathbb R$.
\end{lemma}

\begin{proof}
It is well known that
%    From \cite[Theorem 3.3.2]{hogg-2013-introduction-mathematical} we know
    $
        \mathbb{E}^{p}[X] = 2 \Big( \frac{\Gamma(k/2+p)}{\Gamma(k/2)} \Big)^{\sfrac{1}{p}}.
    $ For every $\alpha \ge 0$, $\beta \ge 2$ the bound $\frac{\Gamma(\alpha+\beta)}{\Gamma(\alpha + 1)} \le (\alpha+\beta/2)^{\beta-1}$ holds~\cite[Equation 2.2]{laforgia-1984-further-inequalities}. The result of the lemma follows by using this bound with $\alpha = k/2-1$ and $\beta = p+1$.
%     We first treat the case $k=p=1$, for which indeed
%     \begin{equation}
%         2 \left( \frac{\Gamma(\frac{1+1}{2})}{\Gamma(\frac{1}{2})} \right)^2 = \frac{2}{\pi} \leq 2 = 1 + 1
%     \end{equation}
%     So we now assume $k + p \geq 3$. We can use a telescoping product and the property $\Gamma(z+1)/\Gamma(z) = z$ to bound
%     \begin{align}
%         \frac{\Gamma(\frac{k + p}{2})}{\Gamma(\frac{k}{2})} 
%         &= \frac{\Gamma(\frac{k}{2} + 1)}{\Gamma(\frac{k}{2})} \frac{\Gamma(\frac{k}{2} + 2)}{\Gamma(\frac{k}{2} + 1)} \cdots \frac{\Gamma(\frac{k}{2} + \lfloor \frac{p}{2} \rfloor)}{\Gamma(\frac{k}{2} + \lfloor \frac{p}{2} \rfloor - 1)} \frac{\Gamma( \frac{k + p}{2})}{\Gamma(\frac{k}{2} + \lfloor \frac{p}{2} \rfloor)} \notag \\
%         &= \left(\frac{k}{2}\right) \left(\frac{k}{2} + 1\right) \cdots \left( \frac{k}{2} + \bigg\lfloor \frac{p}{2} \bigg\rfloor - 1 \right) \frac{\Gamma( \frac{k + p}{2})}{\Gamma(\frac{k}{2} + \lfloor \frac{p}{2} \rfloor)} \notag \\
%         &\leq \left(\frac{k}{2} + \bigg\lfloor \frac{p}{2} \bigg\rfloor - 1\right)^{\lfloor \sfrac{p}{2} \rfloor} \frac{\Gamma( \frac{k + p}{2})}{\Gamma(\frac{k}{2} + \lfloor \frac{p}{2} \rfloor)}
%     \end{align}
%     If $p$ is even, then $\lfloor p/2 \rfloor = p/2$ and hence
%     \begin{equation}
%         \frac{\Gamma(\frac{k + p}{2})}{\Gamma(\frac{k}{2})} 
%         \leq \left( \frac{k + p}{2} - 1 \right)^{\sfrac{p}{2}}
%         \leq \left( \frac{k + p}{2}\right)^{\sfrac{p}{2}}.
%     \end{equation}
%     If $p$ is odd, then $\lfloor p/2 \rfloor = (p - 1)/2$, from which follows with Gautschi's inequality \cite{kershaw-1983-extensions-gautschi}
%     \begin{equation}
%         \frac{\Gamma(\frac{k + p}{2})}{\Gamma(\frac{k}{2})}
%         \leq \left(\frac{k + p}{2} - \frac{3}{2} \right)^{ \frac{p - 1}{2}} \frac{\Gamma( \frac{k + p}{2})}{\Gamma(\frac{k + p}{2} - \frac{1}{2})}
%         \leq \left(\frac{k + p}{2}\right)^{ \frac{p - 1}{2}} \left(\frac{k + p}{2}\right)^{\sfrac{1}{2}}
%         = \left(\frac{k + p}{2}\right)^{ \sfrac{p}{2}}
%     \end{equation}
%     Hence,
%     \begin{equation}
%         \mathbb{E}^{\sfrac{p}{2}}[X] \leq 2 \left( \frac{k + p}{2} \right) \leq k + p.
%     \end{equation}
\end{proof}



\begin{lemma}{Spectral norm moments of non-standard Gaussian random vector}{spectral-norm-moment-vector}
    Given $\mtx{A} \in \mathbb{R}^{m \times m}$ and a standard Gaussian random vector $\vct{\omega} \in \mathbb{R}^{m}$, the bound
    \[
        \mathbb{E}^{p}\big[ \lVert \mtx{A} \vct{\omega} \rVert _2^2 \big]
        \leq  (k + p - 1) \Big( \lVert \mtx{A} \rVert _2^2 + \frac{1}{k} \lVert \mtx{A} \rVert _F^2 \Big).
    \]
    %\begin{equation}
    %    \mathbb{E}^{p}\left[ \lVert \mtx{\Sigma} \mtx{\Omega}_2\rVert _2 \right]
    %    &\leq \sqrt{\frac{k + p}{2}} \cdot \left( 2 + \frac{1}{\sqrt{k}} \right) \cdot \lVert \mtx{\Sigma} \rVert _2 + \sqrt{\frac{k + p}{2k}} \cdot \lVert \mtx{\Sigma} \rVert _F. \notag \\
    %    \mathbb{E}^{p}\left[ \lVert \mtx{\Omega}_1^{\dagger} \rVert _2 \right]
    %    &\leq \frac{1}{2} \left( 1 + \frac{p}{k - r + 1 - p} \right)^{\sfrac{1}{p}} \left( \frac{k + r}{k - r + 2} \right).
    %\end{equation}
    %If we choose 
    holds for every $k \ge 2$, $k\in \mathbb{N}$, and $p \ge 1$, $p\in \mathbb R$.
\end{lemma}
% \begin{remark}
%     In particular, this result can also be used to bound
%     \begin{equation}
%         \mathbb{E}^{p}\left[ \lVert \mtx{A} \vct{\omega} \rVert _2 \right] = \sqrt{\mathbb{E}^{\sfrac{p}{2}}\left[ \lVert \mtx{A} \vct{\omega} \rVert _2^2 \right]} \leq \sqrt{k + p} \cdot \left(\lVert \mtx{A} \rVert _2 + \frac{1}{\sqrt{k}}\lVert \mtx{A} \rVert _F\right)
%     \end{equation}
%     for any $k,p \in \mathbb{N}$.
% \end{remark}
\begin{proof}
By the unitary invariance of Gaussian random vectors, we may assume w.l.o.g. that $\mtx{A} = \mtx{\Sigma} = \operatorname{diag}(\sigma_1, \dots, \sigma_m)$ with $\sigma_1 \geq \dots \geq \sigma_m \geq 0$.
Following the proof of \cite[Theorem 1]{cohen-2016-optimal-approximate}, we split the singular values into $\ell = \lceil m/k \rceil$ groups of size $k$:
    \begin{equation}
        \overbrace{\underbrace{\sigma_1, \dots, \sigma_k}_{\leq \sigma_1}}^{\geq \sigma_{k+1}}, \overbrace{\underbrace{\sigma_{k+1}, \dots, \sigma_{2k}}_{\leq \sigma_{k+1}}}^{\geq \sigma_{2k+1}}, \dots, \overbrace{\underbrace{\sigma_{(\ell - 1)k + 1}, \dots, \sigma_{\ell k}}_{\leq \sigma_{(\ell - 1)k + 1}}}^{\geq 0}.
    \end{equation}
    If $m$ is not a multiple of $k$, we set $\sigma_i = 0$ for $i > m$. Using that
    %Since $\sigma_1 \geq \dots \geq \sigma_{\ell k} \geq 0$,
    %\begin{equation}
     %   \sigma_1^2 = \lVert \mtx{\Sigma} \rVert _2^2
      %  \quad \text{and} \quad
        $\sigma_{ik + 1}^2 \leq ( \sigma_{(i-1)k + 1} + \cdots + \sigma_{ik - 1} + \sigma_{ik} ) / k$ for $i = 1,\ldots, \ell-1$, we get
    \begin{equation}
        \sum_{i=0}^{\ell-1} \sigma_{ik + 1}^2 = \lVert \mtx{\Sigma} \rVert _2^2 + \sum_{i=1}^{\ell-1} \sigma_{ik + 1}^2  \leq \lVert \mtx{\Sigma} \rVert _2^2 + \frac{1}{k} \sum_{j=1}^{(\ell - 1)k} \sigma_j^2 \leq \lVert \mtx{\Sigma} \rVert _2^2 + \frac{1}{k} \lVert \mtx{\Sigma} \rVert _F^2.
        \label{equ:singular-value-group-bound}
    \end{equation}
    This allows us to bound
    \begin{align*}
    \mathbb{E}^{p}\big[ \lVert \mtx{\Sigma} \vct{\omega} \rVert _2^2 \big] & = 
        \mathbb{E}^{p}\Big[ \sum_{i=1}^{m} \sigma_i^2 \omega_i^2 \Big]
        = \mathbb{E}^{p}\Big[ \sum_{i=0}^{\ell - 1} \sum_{j=1}^{k} \sigma_{ik + j}^2 \omega_{ik + j}^2 \Big] \\
        &\leq \mathbb{E}^{p}\Big[ \sum_{i=0}^{\ell - 1} \sigma_{ik + 1}^2 \sum_{j=1}^{k} \omega_{ik + j}^2 \Big] 
        \leq \sum_{i=0}^{\ell - 1} \sigma_{ik + 1}^2 ~ \mathbb{E}^{p}\Big[ \sum_{j=1}^{k} \omega_{ik + j}^2 \Big] \\
        &\leq (k + p - 1) \sum_{i=0}^{\ell - 1} \sigma_{ik + 1}^2 \leq (k + p - 1) \left( \lVert \mtx{\Sigma} \rVert _2^2 + \frac{1}{k} \lVert \mtx{\Sigma} \rVert _F^2 \right), 
    \end{align*}
    where we used the triangle inequality, \reflem{lem:gamma}, and \refequ{equ:singular-value-group-bound} for the last three inequalities.
    %The moments of $\lVert \mtx{\Omega}_1^{\dagger} \rVert _2$ can be bound with help of the proof of \cite[Lemma B.3]{tropp-2023-randomized-algorithms}
    %\begin{align}
    %    \mathbb{E}^{p}\left[ \lVert \mtx{\Omega}_1^{\dagger} \rVert _2 \right]
    %    &= \mathbb{E}\left[ \lVert ( \mtx{\Omega}_1 \mtx{\Omega}_1^{\top} )^{-1} \rVert _2^{p} \right]^{\sfrac{1}{p}} \notag \\
    %    &\leq \left( 1 + \frac{p}{k - r + 1 - p} \right)^{\sfrac{1}{p}} \left( \frac{1}{\Gamma(k - r + 2)} \right)^{\frac{1}{k - r + 1}} \left( \frac{k + r}{2} \right) \notag \\
    %    &\leq \frac{1}{2} \left( 1 + \frac{p}{k - r + 1 - p} \right)^{\sfrac{1}{p}} \left( \frac{k + r}{k - r + 2} \right)
    %\end{align}
    %With the Taylor series expansion of the exponential function it can be shown that $e^n \geq 1 + n$ and $e^n \geq \frac{n^n}{n!}$ for all $n \in \mathbb{N}$, from which $(n + 1)^{\sfrac{1}{n}} \leq e$ and $\left( \frac{1}{n!} \right)^{\sfrac{1}{n}} \leq \frac{e}{n}$ follow respectively. Hence,
    %\begin{equation}
    %    \mathbb{E}^{k}\left[ \lVert \mtx{\Omega}_1^{\dagger} \rVert _2 \right]
    %    \leq \frac{e^2}{k + 1}\sqrt{ \frac{3k}{2} }
    %    \leq e^2 \sqrt{\frac{3}{2k}}.
    %    \label{equ:OSE-moment-bound-second}
    %\end{equation}
\end{proof}

\begin{lemma}{Spectral norm moments of non-standard Gaussian random matrix}{spectral-norm-moment}
    Given $\mtx{A} \in \mathbb{R}^{m \times m}$ and a standard Gaussian random matrix $\mtx{\Omega} \in \mathbb{R}^{m \times k}$, the bound
    \begin{equation}
        \mathbb{E}^{p}\left[ \lVert \mtx{A} \mtx{\Omega} \rVert _2^2 \right]
        \leq  (k + p-1) \left( 2 \lVert \mtx{A} \rVert _2^2 + \frac{1}{k} \lVert \mtx{A} \rVert _F^2 \right).
    \end{equation}
    holds for every $k \ge 2$, $k\in \mathbb{N}$, and $p \ge 1$, $p\in \mathbb R$.
    %\begin{equation}
    %    \mathbb{E}^{p}\left[ \lVert \mtx{\Sigma} \mtx{\Omega}_2\rVert _2 \right]
    %    &\leq \sqrt{\frac{k + p}{2}} \cdot \left( 2 + \frac{1}{\sqrt{k}} \right) \cdot \lVert \mtx{\Sigma} \rVert _2 + \sqrt{\frac{k + p}{2k}} \cdot \lVert \mtx{\Sigma} \rVert _F. \notag \\
    %    \mathbb{E}^{p}\left[ \lVert \mtx{\Omega}_1^{\dagger} \rVert _2 \right]
    %    &\leq \frac{1}{2} \left( 1 + \frac{p}{k - r + 1 - p} \right)^{\sfrac{1}{p}} \left( \frac{k + r}{k - r + 2} \right).
    %\end{equation}
    %If we choose 
\end{lemma}
\begin{proof}
    By the proof of \cite[Lemma B.1]{tropp-2023-randomized-algorithms}, we have that
    %it is shown that Slepian's inequality \cite[Theorem 7.2.1]{vershynin-2018-highdimensional-probability} applied to two appropriately constructed Gaussian random fields yields
    \begin{equation}
        \mathbb{E}^{p}\big[ \lVert \mtx{A} \mtx{\Omega} \rVert _2^2 \big]
        \leq \lVert \mtx{A} \rVert _2^2 \mathbb{E}^{p}\big[ \lVert \vct{\omega}_1 \rVert _2^2 \big] + \mathbb{E}^{p}\big[ \lVert \mtx{A} \vct{\omega}_2 \rVert _2^2 \big]
    \end{equation}
    for independent standard Gaussian random vectors $\vct{\omega}_1 \in \mathbb{R}^{k}$ and $\vct{\omega}_2 \in \mathbb{R}^{m}$. The claimed result follows from applying \reflem{lem:gamma} and \reflem{lem:spectral-norm-moment-vector}:
    \[
        \mathbb{E}^{p}\big[ \lVert \mtx{A} \mtx{\Omega} \rVert _2^2 \big]
        \leq (k + p - 1) \lVert \mtx{A} \rVert _2^2  + (k + p - 1) \left( \lVert \mtx{A} \rVert _2^2 + \frac{1}{k} \lVert \mtx{A} \rVert _F^2 \right).
    \]
    %The moments of $\lVert \mtx{\Omega}_1^{\dagger} \rVert _2$ can be bound with help of the proof of \cite[Lemma B.3]{tropp-2023-randomized-algorithms}
    %\begin{align}
    %    \mathbb{E}^{p}\left[ \lVert \mtx{\Omega}_1^{\dagger} \rVert _2 \right]
    %    &= \mathbb{E}\left[ \lVert ( \mtx{\Omega}_1 \mtx{\Omega}_1^{\top} )^{-1} \rVert _2^{p} \right]^{\sfrac{1}{p}} \notag \\
    %    &\leq \left( 1 + \frac{p}{k - r + 1 - p} \right)^{\sfrac{1}{p}} \left( \frac{1}{\Gamma(k - r + 2)} \right)^{\frac{1}{k - r + 1}} \left( \frac{k + r}{2} \right) \notag \\
    %    &\leq \frac{1}{2} \left( 1 + \frac{p}{k - r + 1 - p} \right)^{\sfrac{1}{p}} \left( \frac{k + r}{k - r + 2} \right)
    %\end{align}
    %With the Taylor series expansion of the exponential function it can be shown that $e^n \geq 1 + n$ and $e^n \geq \frac{n^n}{n!}$ for all $n \in \mathbb{N}$, from which $(n + 1)^{\sfrac{1}{n}} \leq e$ and $\left( \frac{1}{n!} \right)^{\sfrac{1}{n}} \leq \frac{e}{n}$ follow respectively. Hence,
    %\begin{equation}
    %    \mathbb{E}^{k}\left[ \lVert \mtx{\Omega}_1^{\dagger} \rVert _2 \right]
    %    \leq \frac{e^2}{k + 1}\sqrt{ \frac{3k}{2} }
    %    \leq e^2 \sqrt{\frac{3}{2k}}.
    %    \label{equ:OSE-moment-bound-second}
    %\end{equation}
\end{proof}


In passing, we note that \reflem{lem:spectral-norm-moment}  yields the bound
    \begin{equation}
        \mathbb{E}^{p}\big[ \lVert \mtx{A} \mtx{\Omega} \rVert _2 \big] = \sqrt{\mathbb{E}^{p/2}\big[ \lVert \mtx{A} \mtx{\Omega} \rVert _2^2 \big]} \leq \sqrt{k + p/2-1} \cdot \Big(\sqrt{2} \lVert \mtx{A} \rVert _2 + \frac{1}{\sqrt{k}}\lVert \mtx{A} \rVert_F\Big)
    \end{equation}
    with $k\ge 2, p\ge 2$. For $p = 2$, this nearly matches the corresponding bound from~\cite[Lemma B.1]{tropp-2023-randomized-algorithms}, except for the additional factor $\sqrt{2}$.
