
\section{Application to spectral density approximation}
\label{sec:application}

We now get back to the task of computing the spectral density $1/n \sum_{i=1}^{n} \delta(t - \lambda_i)$ of a real symmetric matrix $\mtx{A} \in \mathbb{R}^{n \times n}$ with eigenvalues $\lambda_1, \dots, \lambda_n$, where $\delta(\cdot)$ denotes the Dirac delta distribution. As explained in the introduction, finding accurate approximations to this distribution is challenging and we will therefore work with
a smoothed version. %This still yields enough information
%. Furthermore, in most applications the exact location of each individual eigenvalue is not important, but more so their approximate locations relative to each other, such as eigenvalue clusters, spectral gaps, or outliers. In the following we will work with a smoothed version of the spectral density, which is significantly easier to approximate at the cost of losing some of the finer characteristics of the spectrum.

\subsection{Smoothed spectral density}
\label{subsec:spectral-density}

We recall that the smoothed spectral density $\phi_{\sigma}$ is defined as
\begin{equation}
    \phi_{\sigma}(t) = \frac{1}{n} \sum_{i=1}^{n} g_{\sigma}(t - \lambda_i) = \frac{1}{n} \Trace(g_{\sigma}(t\mtx{I}_n - \mtx{A}))
    \label{equ:smooth-spectral-density}
\end{equation}
for a smoothing kernel $g_{\sigma}$ parametrized by a smoothing parameter $\sigma > 0$. In this work, we choose a Gaussian smoothing kernel of width $\sigma > 0$:
\begin{equation}
    g_{\sigma}(s) = \frac{1}{\sigma \sqrt{2\pi}} e^{-\sfrac{s^2}{2\sigma^2}}.
    \label{equ:smoothing-kernel}
\end{equation}
The expression~\refequ{equ:smooth-spectral-density} gives rise to the task of estimating the trace of the parameter-dependent SPSD matrix $g_{\sigma}(t\mtx{I}_n - \mtx{A})$, which has been analyzed in \refsec{sec:analysis}.

Choosing the smoothing parameter $\sigma$ results in the following trade off: A larger $\sigma$ usually makes it  easier to approximate the matrix function $g_{\sigma}(t\mtx{I}_n - \mtx{A})$, whereas a smaller $\sigma$ allow one to stay closer to the original spectral density $\phi$. To quantify the latter, one can measure the error between $\phi$ and its smoothened version $\phi_{\sigma}$ by
\begin{equation}
    \sup_{f \in \mathcal{S}} \int_{a}^{b} f(t) (\phi(t) - \phi_{\sigma}(t))~\mathrm{d}t,
    \label{equ:error-metric}
\end{equation}
where $\mathcal{S}$ represents an appropriately chosen space of test functions. Among others, the reference~\cite{lin-2016-approximating-spectral} uses $\mathcal{S} = \{ f: f(t) \equiv g_{\sigma}(t - s), s \in [a, b]\}$, \cite{chen-2021-analysis-stochastic} uses $\mathcal{S} = \{f : f(t) = \Theta(s - t), s \in [a, b] \}$ with the Heaviside step function $\Theta$, and \cite{braverman-2022-sublinear-time,bhattacharjee-2024-improved-spectral} use $\mathcal{S} = \{f : |f(t) - f(s)| \leq |t - s| \}$ for which the error \refequ{equ:error-metric} is equivalent to the Wasserstein-1 distance between $\phi$ and $\phi_{\sigma}$.
When choosing 
the last metric,
assuming that the eigenvalues are somewhat uniformly distributed and $\sigma$ is rather small, a simple calculation allows one to conclude that the smoothing error \refequ{equ:error-metric} is of the order $\sigma$. Consequently, to ensure that the approximation deviates by at most a factor of $\varepsilon > 0$, one needs to choose $\sigma \sim \varepsilon$.

\subsection{Chebyshev expansion of smoothing kernel}
\label{subsec:chebyshev-expansion}

To estimate the trace in \refequ{equ:smooth-spectral-density} efficiently, we approximate the function 
$s\mapsto g_{\sigma}(t- s)$ by a truncated
Chebyshev expansion $g_{\sigma}^{(m)}(t-s) = 
\sum_{l=0}^{m} \mu_l(t) T_l(s)$
with the expansion coefficients $\mu_l(t) \in \mathbb R$ and the Chebyshev polynomials
$T_l(s) = \cos(l \cdot \arccos(s))$ for $l = 0,1,\ldots,m$.
Assuming that the spectrum of $\mtx{A}$ is contained in $[-1,1]$, this gives rise to
%need to evaluate the involved matrix function $g_{\sigma}(t\mtx{I}_n - \mtx{A})$. However, doing this exactly would require us to first diagonalize $\mtx{A}$, which is often a prohibitively expensive operation. For algorithmic reasons (see \refsec{subsubsec:chebyshev-nystrom-implementation}), we seek for approximations of this matrix function in terms of affine linear expansions, of which the was determined to be most suitable due to its approximation error guarantees and ease of manipulation. In every $t \in \mathbb{R}$ we expand 
\begin{equation}
    g_{\sigma}(t\mtx{I}_n - \mtx{A}) \approx g_{\sigma}^{(m)}(t\mtx{I}_n - \mtx{A}) = \sum_{l=0}^{m} \mu_l(t) T_l(\mtx{A}),
    \label{equ:matrix-expansion}
\end{equation}
Inserting~\refequ{equ:matrix-expansion} into~\refequ{equ:smooth-spectral-density} yields the approximate smoothed spectral density
\begin{equation}
    \phi_{\sigma}(t) \approx \phi_{\sigma}^{(m)}(t) =  \frac{1}{n} \Trace(g_{\sigma}^{(m)}(t \mtx{I}_n - \mtx{A})).
    \label{equ:expanded-spectral-density}
\end{equation}
\begin{remark}
For a general symmetric matrix $\mtx{A}$, one first estimates an interval $[a,b]$ containing the spectrum of $A$~\cite{zhou-2011-bounding-spectrum}. The usual affine transform
$\tau(t) = \frac{2t - a - b}{b - a}$
then lets us define a matrix $\bar{\mtx{A}} = \tau(\mtx{A})$ for which the spectrum is contained in $[-1, 1]$ and, hence, the expansion \refequ{equ:matrix-expansion} is valid.\footnote{Note that one has to adjust the smoothing parameter $\bar{\sigma} = 2 \sigma (b - a)^{-1}$ to obtain an undistorted approximation of the original spectrum when inverting the transformation.} This allows to assume, from now on, without loss of generality that the spectrum of $\mtx{A}$ is contained in $[-1, 1]$.
\end{remark}


\subsubsection{Expansion error}
\label{subsubsec:expansion-error}

We now analyze what error we make when using the expanded spectral density $\phi_{\sigma}^{(m)}$ to approximate the spectral density $\phi_{\sigma}$. To this extent, we start with the error of the Chebyshev expansion for a Gaussian smoothing kernel $g_{\sigma}$.

\begin{lemma}{Chebyshev expansion error}{chebyshev-error}
    The expansion $g_{\sigma}^{(m)}$ of the Gaussian smoothing kernel $g_{{\sigma}}$ \refequ{equ:smoothing-kernel} satisfies
    \begin{equation}
        \sup_{t \in [-1, 1]} \left| g_{\sigma}(t - s) - g_{\sigma}^{(m)}(t - s) \right| \leq \sqrt{\frac{2e}{\pi}} \frac{1}{\sigma^2} (1 + \sigma)^{-m} \equiv E_{\sigma, m}
        \label{equ:chebyshev-interpolation-sup-error-kernel}
        %\lVert  \phi_{\sigma} - \phi_{\sigma}^{(m)} \rVert _1 &\leq \frac{2\sqrt{2}}{\sigma^2} (1 + \sigma)^{-m}.
        %\label{equ:2-chebyshev-interpolation-error}
    \end{equation}
    for all degrees $m \in \mathbb{N}$, smoothing parameters $\sigma > 0$, and $s \in [-1, 1]$.
\end{lemma}

The proof specializes Bernstein's theorem for Chebyshev interpolation to the case of a Gaussian smoothing kernel \refequ{equ:smoothing-kernel}. It loosely follows the proof given in the arXiv preprint of \cite[Theorem 2]{lin-2017-randomized-estimation}, but attains a slightly different dependence on $\sigma$. \\ \vspace{\parskip}

\begin{proof}
    From Bernstein's theorem \cite[Theorem 4.3]{trefethen-2008-gauss-quadrature} it follows that if $f$ is analytic within the Bernstein ellipse $\mathcal{E}_{\chi} \subset \mathbb{C}$ with foci $\{-1, +1\}$ and sum of semi-axes $\chi > 1$, then for any $m \in \mathbb{N}$
    \begin{equation}
        \sup_{s \in [-1, 1]} \left| f(s) - f^{(m)}(s) \right| \leq \frac{2}{\chi^m (\chi - 1)} \max_{z \in \mathcal{E}_{\chi}} |f(z)|.
        \label{equ:bernstein-bound}
    \end{equation}
    In particular, when $f = g_{\sigma}(t - \cdot)$ we observe for $z = x + iy \in \mathcal{E}_{\chi}$
    \begin{equation}
    | g_{\sigma}(t - z) | 
    = \frac{1}{\sigma \sqrt{2 \pi}} \left| e^{-\sfrac{(t - z)^2}{2\sigma^2}} \right|
    = \frac{1}{\sigma \sqrt{2 \pi}} e^{-\sfrac{(t - x)^2}{2\sigma^2}}e^{\sfrac{y^2}{2\sigma^2}}.
    %\leq \frac{1}{n \sigma \sqrt{2 \pi}} \max_{x + iy \in \mathcal{E}_{\chi}} e^{\sfrac{y^2}{2\sigma^2}} 
    \end{equation}
    Because $e^{-\sfrac{(t - x)^2}{2\sigma^2}} \leq 1$ for all $x, t \in \mathbb{R}$, and the maximum absolute value of $y$ is limited by the length of the semi-axis of the Bernstein ellipse in the direction of the imaginary axis, $(\chi - \chi^{-1}) / 2$, we upper bound
    \begin{equation}
        \max_{z \in \mathcal{E}_{\chi}} | g_{\sigma}(t - z) | 
        \leq \frac{1}{\sigma \sqrt{2 \pi}} e^{\sfrac{(\chi - \chi^{-1})^2}{8 \sigma^2}}.
    \end{equation}

    Choosing $\chi = 1 + \sigma$ gives a simple and good bound, since in this case $\chi - \chi^{-1} \leq 2\sigma$, which implies
    \begin{equation}
        \sup_{t \in [-1, 1]} \left| g_{\sigma}(t - s) - g_{\sigma}^{(m)}(t - s) \right| \leq \sqrt{\frac{2e}{\pi}} \frac{1}{\sigma^2} (1 + \sigma)^{-m}.
    \end{equation}
\end{proof}

\reflem{lem:chebyshev-error} shows that when $0 < \sigma \ll 1$ and if we choose $m = \mathcal{O}(\sigma^{-1}(\log(\varepsilon^{-1}) + \log(\sigma^{-1})))$, the error we induce with the Chebyshev expansion is at most $\mathcal{O}(\varepsilon)$. It now also allows us to bound the $L^1$-error we make when expanding the spectral density.
\begin{align}
    \int_{-1}^{1} \left| \phi_{\sigma}(t) - \phi_{\sigma}^{(m)}(t) \right|~\mathrm{d}t 
    &= \int_{-1}^{1} \left| \frac{1}{n} \sum_{i=1}^n \left(g_{\sigma}(t - \lambda_i) - g_{\sigma}^{(m)}(t - \lambda_i)\right) \right|~\mathrm{d}t \notag \\
    &\leq \int_{-1}^{1} \max_{i = 1, \dots, n} \left| g_{\sigma}(t - \lambda_i) - g_{\sigma}^{(m)}(t - \lambda_i) \right|~\mathrm{d}t \notag \\
    &\leq 2 E_{\sigma, m}
    \label{equ:chebyshev-interpolation-spectral-density}
\end{align}

%\begin{theorem}{$L^1$-error of Chebyshev expansion of spectral density}{chebyshev-error}
%    Let $\phi_{\sigma}$ be the smoothed spectral density of a symmetric matrix $\mtx{A} \in \mathbb{R}^{n \times n}$ whose spectrum is contained in $[-1, 1]$. Then the spectal density with Chebyshev expansion $\phi_{\sigma}^{(m)}$ satisfies
%    \begin{equation}
%        \int_{-1}^{1} \left| \phi_{\sigma}(t) - \phi_{\sigma}^{(m)}(t) \right|~\mathrm{d}t \leq 2 E_{\sigma, m}
%        \label{equ:chebyshev-interpolation-spectral-density}
%        %\lVert  \phi_{\sigma} - \phi_{\sigma}^{(m)} \rVert _1 &\leq \frac{2\sqrt{2}}{\sigma^2} (1 + \sigma)^{-m}.
%        %\label{equ:2-chebyshev-interpolation-error}
%    \end{equation}
%    for all degrees $m \in \mathbb{N}$, smoothing parameters $\sigma > 0$, and the error $E_{\sigma, m}$ from \reflem{lem:chebyshev-error}.%, and $t \in \mathbb{R}$. %The same result holds for the expansion of a non-negative Chebyshev expansion of \emph{even} degree $m \in \mathbb{N}$ but with $\underline{E}_{\sigma, m}$.
%\end{theorem}
%
%\begin{proof}
%    We observe that for all $t \in [-1, 1]$
%    \begin{align}
%        &\left| \phi_{\sigma}(t) - \phi_{\sigma}^{(m)}(t) \right| \notag \\
%        &= \left| \frac{1}{n} \Trace(g_{\sigma}(t\mtx{I}_n - \mtx{A})) - \frac{1}{n} \Trace(g_{\sigma}^{(m)}(t\mtx{I}_n - \mtx{A})) \right|
%        && \text{(definitions \refequ{equ:smooth-spectral-density} and \refequ{equ:expanded-spectral-density})} \notag \\
%        &= \left| \frac{1}{n} \sum_{i=1}^n \left(g_{\sigma}(t - \lambda_i) - g_{\sigma}^{(m)}(t - \lambda_i)\right) \right|
%        && \text{($\lambda_1, \dots, \lambda_n$ eigenvalues of $\mtx{A}$)} \notag \\
%        &\leq \max_{i = 1, \dots, n} \left| g_{\sigma}(t - \lambda_i) - g_{\sigma}^{(m)}(t - \lambda_i) \right|
%        && \text{(conservative upper bound)} \notag \\
%        &\leq E_{\sigma, m}.% ~ \text{(or $\underline{E}_{\sigma, m}$)}.
%        && \text{(using \reflem{lem:chebyshev-error})} \notag \\
%    \end{align}
%    Then, Hölder's inequality allows us to conclude
%    \begin{equation}
%        \int_{-1}^{1} | \phi_{\sigma}(t) - \phi_{\sigma}^{(m)}(t) | ~\mathrm{d}t
%        \leq 2 \sup_{t \in [-1, 1]} \left| \phi_{\sigma}(t) - \phi_{\sigma}^{(m)}(t) \right|
%        \leq 2 E_{\sigma, m}.% ~ \text{(or $2 \underline{E}_{\sigma, m}$)}.
%    \end{equation}
%\end{proof}

\subsubsection{Preservation of non-negativity}

Our goal is to apply the Nyström++ trace estimator which we have analyzed in \refsec{subsec:nystrom-pp} to the Chebyshev expansion of the smoothing kernel $g_{\sigma}$ to approximate the smoothed spectral density $\phi_{\sigma}$. However, due to the oscillatory nature of the Chebyshev expansion, we can no longer guarantee that $g_{\sigma}^{(m)} \geq 0$, and consequently, the matrix function $g_{\sigma}^{(m)}(t \mtx{I}_n - \mtx{A})$ may be (slightly) indefinite. While in our experiments this never seemed to be an issue, it disallows us to directly use \refthm{thm:nystrom-pp}. While variants of the Nyström approximation designed for indefinite matrices have been analyzed, no practical algorithms for computing them exist \cite{nakatsukasa-2023-randomized-lowrank}.

To still apply the bound from \refthm{thm:nystrom-pp} in this scenario, we will have to ensure that the Chebyshev expansion is non-negative. We propose the following method to produce a non-negative Chebyshev expansion \textcolor{red}{[REFERENCE NEEDED]}: Fix an even $m$ and expand the square root of the smoothing kernel in a basis of Chebyshev polynomials up to degree $m/2$
\begin{equation}
    \sqrt{g_{\sigma}}^{(\sfrac{m}{2})}(t - s) = \sum_{l=0}^{m/2} \xi_l(t) T_l(s).
    \label{equ:square-root-chebyshev-expansion}
\end{equation}
Then, square this expansion with \refalg{alg:chebyshev-squaring} to obtain the non-negative Chebyshev expansion
\begin{equation}
    \underline{g}_{\sigma}^{(m)}(t - s) = \sum_{l=0}^{m} \underline{\mu}_l(t) T_l(s).
    \label{equ:non-negative-chebyshev-expansion}
\end{equation}

??? UNTIL HERE ???

%With a complexity of $\mathcal{O}(m \log(m))$ this procedure is computationally inexpensive and allows for very similar convergence bounds to \reflem{lem:chebyshev-error}, which we will derive \textcolor{red}{in the following?} in \refsec{subsubsec:chebyshev-nystrom-analysis}.

\color{black}

%To this extent, we will first analyze the non-negative Chebyshev expansion.

The resulting non-negative version of the approximate smoothed spectral density \refequ{equ:expanded-spectral-density}
\begin{equation}
    \phi_{\sigma}(t) \approx \underline{\phi}_{\sigma}^{(m)}(t) = \frac{1}{n} \Trace(\underline{g}_{\sigma}^{(m)}(t\mtx{I}_n - \mtx{A}))
    \label{equ:non-negative-expanded-density}
\end{equation}
then satisfies the following bound:

\begin{lemma}{$L^1$-error of non-negative approximate smooth spectral density}{non-negative-chebyshev-error}
Let $\phi_{\sigma}$ be the smoothed spectral density of a symmetric matrix $\mtx{A} \in \mathbb{R}^{n \times n}$ whose spectrum is contained in $[-1, 1]$. Then the non-negative approximate smoothed spectral density $\underline{\phi}_{\sigma}^{(m)}$ satisfies
    \begin{equation}
        \int_{-1}^{1} \left| \phi_{\sigma}(t) - \underline{\phi}_{\sigma}^{(m)}(t) \right|~\mathrm{d}t \leq 4 \sqrt{2} \left(1 + \sigma \sqrt{\pi} \cdot E_{\sqrt{2}\sigma, \sfrac{m}{2}}\right) E_{\sqrt{2}\sigma, \sfrac{m}{2}} \equiv \underline{E}_{\sigma, m}
        \label{equ:chebyshev-interpolation-sup-error-kernel-nonneg}
        %\lVert  \phi_{\sigma} - \phi_{\sigma}^{(m)} \rVert _1 &\leq \frac{2\sqrt{2}}{\sigma^2} (1 + \sigma)^{-m}.
        %\label{equ:2-chebyshev-interpolation-error}
    \end{equation}
    for all \emph{even} degrees $m \in \mathbb{N}$, smoothing parameters $\sigma > 0$, and with $E_{\sigma, m}$ as defined in \reflem{lem:chebyshev-error}.
\end{lemma}

\begin{proof}
    For any numbers $a, b \in \mathbb{R}$ it holds $| a^2 - b^2 | \leq (2 | a | + | a - b |)  | a - b |$. Therefore,
    \begin{equation}
        \left| g_{\sigma} - \underline{g}_{\sigma}^{(m)} \right| = \left| (\sqrt{g_{\sigma}})^2 - (\sqrt{g_{\sigma}}^{(\sfrac{m}{2})})^2 \right| \leq \left( 2 \left| \sqrt{g_{\sigma}} \right| + \left| \sqrt{g_{\sigma}} - \sqrt{g_{\sigma}}^{(\sfrac{m}{2})} \right| \right) \left| \sqrt{g_{\sigma}} - \sqrt{g_{\sigma}}^{(\sfrac{m}{2})} \right|.
    \end{equation}
    Since $\sqrt{g_{\sigma}} = \sqrt{2 \sigma \sqrt{2 \pi}} \cdot g_{\sqrt{2}\sigma}$, we can apply \reflem{lem:chebyshev-error} with $\sigma \to \sqrt{2}\sigma$ and $m \to m/2$ along with the bound $|\sqrt{g_{\sigma}}| \leq 1/\sqrt{\sigma \sqrt{2 \pi}}$ to get
    \begin{align}
        &\sup_{t \in [-1, 1]} \left| g_{\sigma}(t - s) - \underline{g}_{\sigma}^{(m)}(t - s) \right| \notag \\
        &\leq \left( 2 \frac{1}{\sqrt{ \sigma \sqrt{2\pi}}} + \sqrt{2 \sigma \sqrt{2 \pi}} \cdot E_{\sqrt{2}\sigma, \sfrac{m}{2}}\right)\sqrt{2 \sigma \sqrt{2 \pi}} \cdot E_{\sqrt{2}\sigma, \sfrac{m}{2}} \notag \\
        &= 2\sqrt{2} \left(1 + \sigma \sqrt{\pi} \cdot E_{\sqrt{2}\sigma, \sfrac{m}{2}}\right) E_{\sqrt{2}\sigma, \sfrac{m}{2}}.
    \end{align}
    An identical treatment to \refequ{equ:chebyshev-interpolation-spectral-density} then leads to the statement of the lemma.
\end{proof}

On one hand, in the non-negative expansion of degree $m$ \refequ{equ:non-negative-chebyshev-expansion}, the underlying Chebyshev expansion -- whose error is shown in \reflem{lem:chebyshev-error} to exponentially decay with $m$ -- is only computed to half the degree (i.e. $m/2$) of the standard Chebyshev expansion, hence, usually not as accurate. On the other hand, in the non-negative expansion we approximate $\sqrt{g_{\sigma}} \propto g_{\sqrt{2}\sigma}$ which is less \enquote{pointy} than $g_{\sigma}$ and therefore easier to approximate with Chebyshev polynomials. Overall, we will need to choose the degree $m$ of the non-negative expansion roughly a factor of $\sqrt{2}$ larger to match the approximation error of the standard expansion (cf. \reffig{fig:interpolation-issue}).

%In the following, we will show how the coefficients $\underline{\mu}_0(t), \dots, \underline{\mu}_m(t)$ of the non-negative Chebyshev expansion \refequ{equ:non-negative-chebyshev-expansion} can efficiently be computed and that this expansion exhibits similar convergence properties to the standard Chebyshev expansion.
%\todo{
%Maybe mention indefinite Nyström [Nakatsukasa], which won't give us desirable guarantees, and refinement of Chebyshev expansion [Francis Bach], which only slightly improves, but is much harder to implement. Side-note: it is possible to refine the coefficients of the Chebyshev expansion to get a better approximation in the supremum-norm over the set of all non-negative polynomials \cite%{fejer-1916-uber-trigonometrische}. The non-convex nature of this problem and the relatively small improvement we get with this technique made us decide to not pursue this further.
%}

%\subsection{Computation of expansion coefficients}

\subsubsection{Computation of Chebyshev coefficients}
\label{subsubsec:dct}

It is well known that the coefficients of a Chebyshev expansion of a function can --- in exact arithmetic --- be computed by inverting the discrete cosine transform (DCT)
\begin{equation}
    g_{\sigma}(t - \cos(s_i)) = \sum_{l=0}^{m} \mu_l(t) \cos\left(\frac{\pi i l}{m} \right)
    \label{equ:discrete-cosine-transform}
\end{equation}
between the expansion coefficients $\mu_l(t),~i=0,\dots,m$ and the function evaluations at the Chebyshev nodes of the second kind $s_i = \cos(\pi i / m),~i=0,\dots,m$ \cite{vanloan-1992-computational-frameworks, baszenski-1997-fast-polynomial}. This can be accomplished in $\mathcal{O}(m \log(m))$ operations.

A desirable property of the Chebyshev expansion is the fact that such an expansion can be squared in $\mathcal{O}(m \log(m))$ operations by using the DCT and its inverse \cite{baszenski-1997-fast-polynomial}. This allows us to efficiently compute the non-negative Chebyshev expansion \refequ{equ:non-negative-chebyshev-expansion}, and will also be useful in \refsec{subsec:chebyshev-nystrom}. The below algorithm implements this observation.

\begin{algo}{Fast squaring of Chebyshev expansion}{chebyshev-squaring}
    \begin{algorithmic}[1]
        \Statex \textbf{Input:} Coefficients $\{ \mu_l \}_{l=0}^{m}$ of the expansion $\sum_{l=0}^{m} \mu_l T_l$
        \Statex \textbf{Output:} Coefficients $\{ \nu_l \}_{l=0}^{2m}$ such that $\sum_{l=0}^{2m} \nu_l T_l = (\sum_{l=0}^{m} \mu_l T_l)^2$
        \State Define $\vct{\mu} \in \mathbb{R}^{2m + 1}$ whose first $m + 1$ entries are the coefficients $\mu_0, \dots, \mu_m$ and the remaining $m$ are $0$
    \State Compute $\vct{f} = \operatorname{DCT}(\vct{\mu})$
    \State Compute $\vct{\nu} = \operatorname{DCT}^{-1}(\vct{f} \odot \vct{f})$ whose entries are the coefficients $\nu_0, \dots, \nu_{2m}$
    \label{lin:inverse-DCT}
    \end{algorithmic}
\end{algo}
On \reflin{lin:inverse-DCT}, $\odot$ denotes the entrywise product. 

In \cite{lin-2017-randomized-estimation}, a slightly different approach to computing the expansion coefficients in \refequ{equ:discrete-cosine-transform} is taken. They express the coefficients $\mu_l(t)$ in terms of an integral, which they subsequently approximate with a quadrature supported in at least $2m + 1$ nodes. They evaluate the resulting approximation with the fast Fourier transform (FFT). Besides not yielding the exact expansion coefficients $\mu_l(t)$, this procedure is also significantly slower due to the larger number of quadrature nodes and the implementational benefits of the DCT over the FFT. In \reftab{tab:chebyshev-timing-interpolation}, we compare a benevolent implementation of \cite[Algorithm 1]{lin-2017-randomized-estimation} with our approach, both for the standard and the non-negative Chebyshev expansion.

\begin{table}[ht]
    \caption{We compare the runtime in milliseconds for computing the coefficients of the Chebyshev expansion of the smoothing kernel $g_{\sigma}$ \refequ{equ:smoothing-kernel} in three different ways: as proposed in \cite[Algorithm 1]{lin-2017-randomized-estimation} (\emph{FFT}), using the discrete cosine transform (\emph{DCT}) \refequ{equ:discrete-cosine-transform}, and using the discrete cosine transform to compute a non-negative Chebyshev expansion with \refalg{alg:chebyshev-squaring} (\emph{non-negative DCT}). We use $\sigma=0.005$, $n_t=1000$ parameter values, and various degrees $m$. For stability, we average over 7 runs of the algorithms and repeat these runs 1000 times to form the mean and standard deviation which are given in the below table.}
    \label{tab:chebyshev-timing-interpolation}
   \centering
\renewcommand{\arraystretch}{1.2}
\begin{tabular}{@{}lcccc@{}}
\toprule
 & $m=800$ & $m=1600$ & $m=2400$ & $m=3200$\\
\midrule
oversampled FFT & $196.9$ $\pm$ $3.0$ & $1235.4$ $\pm$ $23.4$ & $734.6$ $\pm$ $3.8$ & $1167.9$ $\pm$ $2.5$ \\
DCT & $79.6$ $\pm$ $53.5$ & $84.8$ $\pm$ $1.0$ & $112.6$ $\pm$ $0.6$ & $139.5$ $\pm$ $0.9$ \\
non-negative DCT & $145.1$ $\pm$ $1.0$ & $176.5$ $\pm$ $0.7$ & $209.0$ $\pm$ $1.3$ & $236.5$ $\pm$ $1.3$ \\
\bottomrule
\end{tabular}

\end{table}

\subsection{Chebyshev-Nyström++ method for spectral density approximation}
\label{subsec:chebyshev-nystrom}

Combining the results from \refsec{sec:analysis} for $\mtx{B}(t) = \underline{g}_{\sigma}^{(m)}(t \mtx{I}_n - \mtx{A})$ with the Chebyshev expansion framework from \refsec{subsec:chebyshev-expansion}, we now approximate the smoothed spectral density with
\begin{equation}
    \phi_{\sigma}(t) \approx \widetilde{\underline{\phi}}_{\sigma}^{(m)}(t) = \frac{1}{n} \Nystrpp{\mtx{\Omega}}{\mtx{\Psi}}(\underline{g}_{\sigma}^{(m)}(t \mtx{I}_n - \mtx{A})),
    \label{equ:chebyshev-nystrom-formula}
\end{equation}
which is the Nyström++ estimator $\Nystrpp{\mtx{\Omega}}{\mtx{\Psi}}$ \refequ{equ:nystrompp-trace-estimator} applied to the non-negative Chebyshev expansion $\underline{g}_{\sigma}^{(m)}$ \refequ{equ:non-negative-chebyshev-expansion}.


\subsubsection{Implementation}
\label{subsubsec:chebyshev-nystrom-implementation}

An efficient implementation of the approximation \refequ{equ:chebyshev-nystrom-formula} can be achieved thanks to the invariance of the trace under cyclic permutation of its arguments and the three-term recurrence relation
\begin{equation}
    T_0(s) = 1,\quad  T_1(s) = s,\quad 
    T_l(s) = 2 s T_{l-1}(s) - T_{l-2}(s) \text{ for $l \geq 2$,}
    \label{equ:chebyshev-recurrence}
\end{equation}
which the Chebyshev polynomials satisfy. To illustrate this, consider the approximation \refequ{equ:chebyshev-nystrom-formula} when $n_{\mtx{\Psi}}$ is zero, i.e. only a low-rank approximation is computed. Using the definition of the Nyström approximation \refequ{equ:nystrom-approximation}, the cyclic invariance of the trace, and the symmetry of $\mtx{A}$, we can rewrite it as
\begin{equation}
    \frac{1}{n} \Trace \big((\mtx{B}(t) \mtx{\Omega}) (\mtx{\Omega}^{\top} \mtx{B}(t) \mtx{\Omega})^{\dagger} (\mtx{B}(t) \mtx{\Omega})^{\top} \big) =
    \frac{1}{n} \Trace \big((\underbrace{\mtx{\Omega}^{\top} \mtx{B}(t) \mtx{\Omega}}_{= \mtx{K}_1(t)})^{\dagger} (\underbrace{\mtx{\Omega}^{\top} \mtx{B}(t)^2 \mtx{\Omega}}_{= \mtx{K}_2(t)}) \big).
    %\frac{1}{n} \Trace((\mtx{\Omega}^{\top} \underline{g}_{\sigma}^{(m)}(t \mtx{I}_n - \mtx{A}) \mtx{\Omega})^{\dagger} (\mtx{\Omega}^{\top} \underline{g}_{\sigma}^{(m)}(t \mtx{I}_n - \mtx{A})^2 \mtx{\Omega})).
    \label{equ:cyclic-property}
\end{equation}
Thanks to the affine linear form of the Chebyshev expansion, we have
\begin{equation}
    \mtx{K}_1(t) = \mtx{\Omega}^{\top} \underline{g}_{\sigma}^{(m)}(t \mtx{I}_n - \mtx{A}) \mtx{\Omega} = \sum_{l=0}^{m} \underline{\mu}_l(t) (\mtx{\Omega}^{\top} T_l(\mtx{A}) \mtx{\Omega}),
\end{equation}
meaning we can first precompute the small $n_{\mtx{\Omega}} \times n_{\mtx{\Omega}}$ matrices $\mtx{\Omega}^{\top} T_l(\mtx{A}) \mtx{\Omega}$ for $l=0, \dots, m$ using the Chebyshev recurrence \refequ{equ:chebyshev-recurrence} and then inexpensively sum them up with the corresponding coefficients $\underline{\mu}_l(t)$ for all $t$ in which we want to evaluate the approximation. The fast and exact squaring of Chebyshev expansions (cf. \refalg{alg:chebyshev-squaring}), allows us to also express $\underline{g}_{\sigma}^{(m)}(t \mtx{I}_n - \mtx{A})^2 = \sum_{l=0}^{2m} \underline{\nu}_l(t) T_l(\mtx{A})$ in terms of an affine linear sum of the small matrices $\mtx{\Omega}^{\top} T_l(\mtx{A}) \mtx{\Omega}$ for $l=0, \dots, 2m$, such that the matrix $\mtx{K}_2(t)$ can be built in an analogous way to $\mtx{K}_1(t)$. Alternatively, \cite{lin-2017-randomized-estimation} proposes to expand the squared smoothing kernel $g_{\sigma}^2$ in a separate Chebyshev expansion of degree $2m$. While this approximation of $g_{\sigma}^2$ might be more accurate, it is \emph{inconsistent} in the sense that computing a Chebyshev expansion of $g_{\sigma}^2$ is in general not the same as squaring an existing Chebyshev expansion of $g_{\sigma}$. We have observed the consistency between the approximation of $g_{\sigma}$ and $g_{\sigma}^2$ to make a crucial difference in terms of accuracy (cf. \reffig{fig:interpolation-issue}). The consistency also allows bounds to be derived for the estimator \refequ{equ:chebyshev-nystrom-formula}.

\begin{figure}[ht]
    \centering
    %% Creator: Matplotlib, PGF backend
%%
%% To include the figure in your LaTeX document, write
%%   \input{<filename>.pgf}
%%
%% Make sure the required packages are loaded in your preamble
%%   \usepackage{pgf}
%%
%% Also ensure that all the required font packages are loaded; for instance,
%% the lmodern package is sometimes necessary when using math font.
%%   \usepackage{lmodern}
%%
%% Figures using additional raster images can only be included by \input if
%% they are in the same directory as the main LaTeX file. For loading figures
%% from other directories you can use the `import` package
%%   \usepackage{import}
%%
%% and then include the figures with
%%   \import{<path to file>}{<filename>.pgf}
%%
%% Matplotlib used the following preamble
%%   \def\mathdefault#1{#1}
%%   \everymath=\expandafter{\the\everymath\displaystyle}
%%   
%%   \makeatletter\@ifpackageloaded{underscore}{}{\usepackage[strings]{underscore}}\makeatother
%%
\begingroup%
\makeatletter%
\begin{pgfpicture}%
\pgfpathrectangle{\pgfpointorigin}{\pgfqpoint{3.921913in}{2.959073in}}%
\pgfusepath{use as bounding box, clip}%
\begin{pgfscope}%
\pgfsetbuttcap%
\pgfsetmiterjoin%
\definecolor{currentfill}{rgb}{1.000000,1.000000,1.000000}%
\pgfsetfillcolor{currentfill}%
\pgfsetlinewidth{0.000000pt}%
\definecolor{currentstroke}{rgb}{1.000000,1.000000,1.000000}%
\pgfsetstrokecolor{currentstroke}%
\pgfsetdash{}{0pt}%
\pgfpathmoveto{\pgfqpoint{0.000000in}{-0.000000in}}%
\pgfpathlineto{\pgfqpoint{3.921913in}{-0.000000in}}%
\pgfpathlineto{\pgfqpoint{3.921913in}{2.959073in}}%
\pgfpathlineto{\pgfqpoint{0.000000in}{2.959073in}}%
\pgfpathlineto{\pgfqpoint{0.000000in}{-0.000000in}}%
\pgfpathclose%
\pgfusepath{fill}%
\end{pgfscope}%
\begin{pgfscope}%
\pgfsetbuttcap%
\pgfsetmiterjoin%
\definecolor{currentfill}{rgb}{1.000000,1.000000,1.000000}%
\pgfsetfillcolor{currentfill}%
\pgfsetlinewidth{0.000000pt}%
\definecolor{currentstroke}{rgb}{0.000000,0.000000,0.000000}%
\pgfsetstrokecolor{currentstroke}%
\pgfsetstrokeopacity{0.000000}%
\pgfsetdash{}{0pt}%
\pgfpathmoveto{\pgfqpoint{0.721913in}{0.549073in}}%
\pgfpathlineto{\pgfqpoint{3.821913in}{0.549073in}}%
\pgfpathlineto{\pgfqpoint{3.821913in}{2.859073in}}%
\pgfpathlineto{\pgfqpoint{0.721913in}{2.859073in}}%
\pgfpathlineto{\pgfqpoint{0.721913in}{0.549073in}}%
\pgfpathclose%
\pgfusepath{fill}%
\end{pgfscope}%
\begin{pgfscope}%
\pgfsetbuttcap%
\pgfsetroundjoin%
\definecolor{currentfill}{rgb}{0.000000,0.000000,0.000000}%
\pgfsetfillcolor{currentfill}%
\pgfsetlinewidth{0.803000pt}%
\definecolor{currentstroke}{rgb}{0.000000,0.000000,0.000000}%
\pgfsetstrokecolor{currentstroke}%
\pgfsetdash{}{0pt}%
\pgfsys@defobject{currentmarker}{\pgfqpoint{0.000000in}{-0.048611in}}{\pgfqpoint{0.000000in}{0.000000in}}{%
\pgfpathmoveto{\pgfqpoint{0.000000in}{0.000000in}}%
\pgfpathlineto{\pgfqpoint{0.000000in}{-0.048611in}}%
\pgfusepath{stroke,fill}%
}%
\begin{pgfscope}%
\pgfsys@transformshift{2.510068in}{0.549073in}%
\pgfsys@useobject{currentmarker}{}%
\end{pgfscope}%
\end{pgfscope}%
\begin{pgfscope}%
\definecolor{textcolor}{rgb}{0.000000,0.000000,0.000000}%
\pgfsetstrokecolor{textcolor}%
\pgfsetfillcolor{textcolor}%
\pgftext[x=2.510068in,y=0.451851in,,top]{\color{textcolor}{\rmfamily\fontsize{12.000000}{14.400000}\selectfont\catcode`\^=\active\def^{\ifmmode\sp\else\^{}\fi}\catcode`\%=\active\def%{\%}$\mathdefault{10^{3}}$}}%
\end{pgfscope}%
\begin{pgfscope}%
\pgfsetbuttcap%
\pgfsetroundjoin%
\definecolor{currentfill}{rgb}{0.000000,0.000000,0.000000}%
\pgfsetfillcolor{currentfill}%
\pgfsetlinewidth{0.602250pt}%
\definecolor{currentstroke}{rgb}{0.000000,0.000000,0.000000}%
\pgfsetstrokecolor{currentstroke}%
\pgfsetdash{}{0pt}%
\pgfsys@defobject{currentmarker}{\pgfqpoint{0.000000in}{-0.027778in}}{\pgfqpoint{0.000000in}{0.000000in}}{%
\pgfpathmoveto{\pgfqpoint{0.000000in}{0.000000in}}%
\pgfpathlineto{\pgfqpoint{0.000000in}{-0.027778in}}%
\pgfusepath{stroke,fill}%
}%
\begin{pgfscope}%
\pgfsys@transformshift{0.873045in}{0.549073in}%
\pgfsys@useobject{currentmarker}{}%
\end{pgfscope}%
\end{pgfscope}%
\begin{pgfscope}%
\pgfsetbuttcap%
\pgfsetroundjoin%
\definecolor{currentfill}{rgb}{0.000000,0.000000,0.000000}%
\pgfsetfillcolor{currentfill}%
\pgfsetlinewidth{0.602250pt}%
\definecolor{currentstroke}{rgb}{0.000000,0.000000,0.000000}%
\pgfsetstrokecolor{currentstroke}%
\pgfsetdash{}{0pt}%
\pgfsys@defobject{currentmarker}{\pgfqpoint{0.000000in}{-0.027778in}}{\pgfqpoint{0.000000in}{0.000000in}}{%
\pgfpathmoveto{\pgfqpoint{0.000000in}{0.000000in}}%
\pgfpathlineto{\pgfqpoint{0.000000in}{-0.027778in}}%
\pgfusepath{stroke,fill}%
}%
\begin{pgfscope}%
\pgfsys@transformshift{1.285460in}{0.549073in}%
\pgfsys@useobject{currentmarker}{}%
\end{pgfscope}%
\end{pgfscope}%
\begin{pgfscope}%
\pgfsetbuttcap%
\pgfsetroundjoin%
\definecolor{currentfill}{rgb}{0.000000,0.000000,0.000000}%
\pgfsetfillcolor{currentfill}%
\pgfsetlinewidth{0.602250pt}%
\definecolor{currentstroke}{rgb}{0.000000,0.000000,0.000000}%
\pgfsetstrokecolor{currentstroke}%
\pgfsetdash{}{0pt}%
\pgfsys@defobject{currentmarker}{\pgfqpoint{0.000000in}{-0.027778in}}{\pgfqpoint{0.000000in}{0.000000in}}{%
\pgfpathmoveto{\pgfqpoint{0.000000in}{0.000000in}}%
\pgfpathlineto{\pgfqpoint{0.000000in}{-0.027778in}}%
\pgfusepath{stroke,fill}%
}%
\begin{pgfscope}%
\pgfsys@transformshift{1.578073in}{0.549073in}%
\pgfsys@useobject{currentmarker}{}%
\end{pgfscope}%
\end{pgfscope}%
\begin{pgfscope}%
\pgfsetbuttcap%
\pgfsetroundjoin%
\definecolor{currentfill}{rgb}{0.000000,0.000000,0.000000}%
\pgfsetfillcolor{currentfill}%
\pgfsetlinewidth{0.602250pt}%
\definecolor{currentstroke}{rgb}{0.000000,0.000000,0.000000}%
\pgfsetstrokecolor{currentstroke}%
\pgfsetdash{}{0pt}%
\pgfsys@defobject{currentmarker}{\pgfqpoint{0.000000in}{-0.027778in}}{\pgfqpoint{0.000000in}{0.000000in}}{%
\pgfpathmoveto{\pgfqpoint{0.000000in}{0.000000in}}%
\pgfpathlineto{\pgfqpoint{0.000000in}{-0.027778in}}%
\pgfusepath{stroke,fill}%
}%
\begin{pgfscope}%
\pgfsys@transformshift{1.805041in}{0.549073in}%
\pgfsys@useobject{currentmarker}{}%
\end{pgfscope}%
\end{pgfscope}%
\begin{pgfscope}%
\pgfsetbuttcap%
\pgfsetroundjoin%
\definecolor{currentfill}{rgb}{0.000000,0.000000,0.000000}%
\pgfsetfillcolor{currentfill}%
\pgfsetlinewidth{0.602250pt}%
\definecolor{currentstroke}{rgb}{0.000000,0.000000,0.000000}%
\pgfsetstrokecolor{currentstroke}%
\pgfsetdash{}{0pt}%
\pgfsys@defobject{currentmarker}{\pgfqpoint{0.000000in}{-0.027778in}}{\pgfqpoint{0.000000in}{0.000000in}}{%
\pgfpathmoveto{\pgfqpoint{0.000000in}{0.000000in}}%
\pgfpathlineto{\pgfqpoint{0.000000in}{-0.027778in}}%
\pgfusepath{stroke,fill}%
}%
\begin{pgfscope}%
\pgfsys@transformshift{1.990487in}{0.549073in}%
\pgfsys@useobject{currentmarker}{}%
\end{pgfscope}%
\end{pgfscope}%
\begin{pgfscope}%
\pgfsetbuttcap%
\pgfsetroundjoin%
\definecolor{currentfill}{rgb}{0.000000,0.000000,0.000000}%
\pgfsetfillcolor{currentfill}%
\pgfsetlinewidth{0.602250pt}%
\definecolor{currentstroke}{rgb}{0.000000,0.000000,0.000000}%
\pgfsetstrokecolor{currentstroke}%
\pgfsetdash{}{0pt}%
\pgfsys@defobject{currentmarker}{\pgfqpoint{0.000000in}{-0.027778in}}{\pgfqpoint{0.000000in}{0.000000in}}{%
\pgfpathmoveto{\pgfqpoint{0.000000in}{0.000000in}}%
\pgfpathlineto{\pgfqpoint{0.000000in}{-0.027778in}}%
\pgfusepath{stroke,fill}%
}%
\begin{pgfscope}%
\pgfsys@transformshift{2.147280in}{0.549073in}%
\pgfsys@useobject{currentmarker}{}%
\end{pgfscope}%
\end{pgfscope}%
\begin{pgfscope}%
\pgfsetbuttcap%
\pgfsetroundjoin%
\definecolor{currentfill}{rgb}{0.000000,0.000000,0.000000}%
\pgfsetfillcolor{currentfill}%
\pgfsetlinewidth{0.602250pt}%
\definecolor{currentstroke}{rgb}{0.000000,0.000000,0.000000}%
\pgfsetstrokecolor{currentstroke}%
\pgfsetdash{}{0pt}%
\pgfsys@defobject{currentmarker}{\pgfqpoint{0.000000in}{-0.027778in}}{\pgfqpoint{0.000000in}{0.000000in}}{%
\pgfpathmoveto{\pgfqpoint{0.000000in}{0.000000in}}%
\pgfpathlineto{\pgfqpoint{0.000000in}{-0.027778in}}%
\pgfusepath{stroke,fill}%
}%
\begin{pgfscope}%
\pgfsys@transformshift{2.283100in}{0.549073in}%
\pgfsys@useobject{currentmarker}{}%
\end{pgfscope}%
\end{pgfscope}%
\begin{pgfscope}%
\pgfsetbuttcap%
\pgfsetroundjoin%
\definecolor{currentfill}{rgb}{0.000000,0.000000,0.000000}%
\pgfsetfillcolor{currentfill}%
\pgfsetlinewidth{0.602250pt}%
\definecolor{currentstroke}{rgb}{0.000000,0.000000,0.000000}%
\pgfsetstrokecolor{currentstroke}%
\pgfsetdash{}{0pt}%
\pgfsys@defobject{currentmarker}{\pgfqpoint{0.000000in}{-0.027778in}}{\pgfqpoint{0.000000in}{0.000000in}}{%
\pgfpathmoveto{\pgfqpoint{0.000000in}{0.000000in}}%
\pgfpathlineto{\pgfqpoint{0.000000in}{-0.027778in}}%
\pgfusepath{stroke,fill}%
}%
\begin{pgfscope}%
\pgfsys@transformshift{2.402902in}{0.549073in}%
\pgfsys@useobject{currentmarker}{}%
\end{pgfscope}%
\end{pgfscope}%
\begin{pgfscope}%
\pgfsetbuttcap%
\pgfsetroundjoin%
\definecolor{currentfill}{rgb}{0.000000,0.000000,0.000000}%
\pgfsetfillcolor{currentfill}%
\pgfsetlinewidth{0.602250pt}%
\definecolor{currentstroke}{rgb}{0.000000,0.000000,0.000000}%
\pgfsetstrokecolor{currentstroke}%
\pgfsetdash{}{0pt}%
\pgfsys@defobject{currentmarker}{\pgfqpoint{0.000000in}{-0.027778in}}{\pgfqpoint{0.000000in}{0.000000in}}{%
\pgfpathmoveto{\pgfqpoint{0.000000in}{0.000000in}}%
\pgfpathlineto{\pgfqpoint{0.000000in}{-0.027778in}}%
\pgfusepath{stroke,fill}%
}%
\begin{pgfscope}%
\pgfsys@transformshift{3.215096in}{0.549073in}%
\pgfsys@useobject{currentmarker}{}%
\end{pgfscope}%
\end{pgfscope}%
\begin{pgfscope}%
\pgfsetbuttcap%
\pgfsetroundjoin%
\definecolor{currentfill}{rgb}{0.000000,0.000000,0.000000}%
\pgfsetfillcolor{currentfill}%
\pgfsetlinewidth{0.602250pt}%
\definecolor{currentstroke}{rgb}{0.000000,0.000000,0.000000}%
\pgfsetstrokecolor{currentstroke}%
\pgfsetdash{}{0pt}%
\pgfsys@defobject{currentmarker}{\pgfqpoint{0.000000in}{-0.027778in}}{\pgfqpoint{0.000000in}{0.000000in}}{%
\pgfpathmoveto{\pgfqpoint{0.000000in}{0.000000in}}%
\pgfpathlineto{\pgfqpoint{0.000000in}{-0.027778in}}%
\pgfusepath{stroke,fill}%
}%
\begin{pgfscope}%
\pgfsys@transformshift{3.627511in}{0.549073in}%
\pgfsys@useobject{currentmarker}{}%
\end{pgfscope}%
\end{pgfscope}%
\begin{pgfscope}%
\definecolor{textcolor}{rgb}{0.000000,0.000000,0.000000}%
\pgfsetstrokecolor{textcolor}%
\pgfsetfillcolor{textcolor}%
\pgftext[x=2.271913in,y=0.248148in,,top]{\color{textcolor}{\rmfamily\fontsize{12.000000}{14.400000}\selectfont\catcode`\^=\active\def^{\ifmmode\sp\else\^{}\fi}\catcode`\%=\active\def%{\%}expansion degree $m$}}%
\end{pgfscope}%
\begin{pgfscope}%
\pgfsetbuttcap%
\pgfsetroundjoin%
\definecolor{currentfill}{rgb}{0.000000,0.000000,0.000000}%
\pgfsetfillcolor{currentfill}%
\pgfsetlinewidth{0.803000pt}%
\definecolor{currentstroke}{rgb}{0.000000,0.000000,0.000000}%
\pgfsetstrokecolor{currentstroke}%
\pgfsetdash{}{0pt}%
\pgfsys@defobject{currentmarker}{\pgfqpoint{-0.048611in}{0.000000in}}{\pgfqpoint{-0.000000in}{0.000000in}}{%
\pgfpathmoveto{\pgfqpoint{-0.000000in}{0.000000in}}%
\pgfpathlineto{\pgfqpoint{-0.048611in}{0.000000in}}%
\pgfusepath{stroke,fill}%
}%
\begin{pgfscope}%
\pgfsys@transformshift{0.721913in}{0.868622in}%
\pgfsys@useobject{currentmarker}{}%
\end{pgfscope}%
\end{pgfscope}%
\begin{pgfscope}%
\definecolor{textcolor}{rgb}{0.000000,0.000000,0.000000}%
\pgfsetstrokecolor{textcolor}%
\pgfsetfillcolor{textcolor}%
\pgftext[x=0.303703in, y=0.810752in, left, base]{\color{textcolor}{\rmfamily\fontsize{12.000000}{14.400000}\selectfont\catcode`\^=\active\def^{\ifmmode\sp\else\^{}\fi}\catcode`\%=\active\def%{\%}$\mathdefault{10^{-6}}$}}%
\end{pgfscope}%
\begin{pgfscope}%
\pgfsetbuttcap%
\pgfsetroundjoin%
\definecolor{currentfill}{rgb}{0.000000,0.000000,0.000000}%
\pgfsetfillcolor{currentfill}%
\pgfsetlinewidth{0.803000pt}%
\definecolor{currentstroke}{rgb}{0.000000,0.000000,0.000000}%
\pgfsetstrokecolor{currentstroke}%
\pgfsetdash{}{0pt}%
\pgfsys@defobject{currentmarker}{\pgfqpoint{-0.048611in}{0.000000in}}{\pgfqpoint{-0.000000in}{0.000000in}}{%
\pgfpathmoveto{\pgfqpoint{-0.000000in}{0.000000in}}%
\pgfpathlineto{\pgfqpoint{-0.048611in}{0.000000in}}%
\pgfusepath{stroke,fill}%
}%
\begin{pgfscope}%
\pgfsys@transformshift{0.721913in}{1.427227in}%
\pgfsys@useobject{currentmarker}{}%
\end{pgfscope}%
\end{pgfscope}%
\begin{pgfscope}%
\definecolor{textcolor}{rgb}{0.000000,0.000000,0.000000}%
\pgfsetstrokecolor{textcolor}%
\pgfsetfillcolor{textcolor}%
\pgftext[x=0.303703in, y=1.369357in, left, base]{\color{textcolor}{\rmfamily\fontsize{12.000000}{14.400000}\selectfont\catcode`\^=\active\def^{\ifmmode\sp\else\^{}\fi}\catcode`\%=\active\def%{\%}$\mathdefault{10^{-4}}$}}%
\end{pgfscope}%
\begin{pgfscope}%
\pgfsetbuttcap%
\pgfsetroundjoin%
\definecolor{currentfill}{rgb}{0.000000,0.000000,0.000000}%
\pgfsetfillcolor{currentfill}%
\pgfsetlinewidth{0.803000pt}%
\definecolor{currentstroke}{rgb}{0.000000,0.000000,0.000000}%
\pgfsetstrokecolor{currentstroke}%
\pgfsetdash{}{0pt}%
\pgfsys@defobject{currentmarker}{\pgfqpoint{-0.048611in}{0.000000in}}{\pgfqpoint{-0.000000in}{0.000000in}}{%
\pgfpathmoveto{\pgfqpoint{-0.000000in}{0.000000in}}%
\pgfpathlineto{\pgfqpoint{-0.048611in}{0.000000in}}%
\pgfusepath{stroke,fill}%
}%
\begin{pgfscope}%
\pgfsys@transformshift{0.721913in}{1.985832in}%
\pgfsys@useobject{currentmarker}{}%
\end{pgfscope}%
\end{pgfscope}%
\begin{pgfscope}%
\definecolor{textcolor}{rgb}{0.000000,0.000000,0.000000}%
\pgfsetstrokecolor{textcolor}%
\pgfsetfillcolor{textcolor}%
\pgftext[x=0.303703in, y=1.927962in, left, base]{\color{textcolor}{\rmfamily\fontsize{12.000000}{14.400000}\selectfont\catcode`\^=\active\def^{\ifmmode\sp\else\^{}\fi}\catcode`\%=\active\def%{\%}$\mathdefault{10^{-2}}$}}%
\end{pgfscope}%
\begin{pgfscope}%
\pgfsetbuttcap%
\pgfsetroundjoin%
\definecolor{currentfill}{rgb}{0.000000,0.000000,0.000000}%
\pgfsetfillcolor{currentfill}%
\pgfsetlinewidth{0.803000pt}%
\definecolor{currentstroke}{rgb}{0.000000,0.000000,0.000000}%
\pgfsetstrokecolor{currentstroke}%
\pgfsetdash{}{0pt}%
\pgfsys@defobject{currentmarker}{\pgfqpoint{-0.048611in}{0.000000in}}{\pgfqpoint{-0.000000in}{0.000000in}}{%
\pgfpathmoveto{\pgfqpoint{-0.000000in}{0.000000in}}%
\pgfpathlineto{\pgfqpoint{-0.048611in}{0.000000in}}%
\pgfusepath{stroke,fill}%
}%
\begin{pgfscope}%
\pgfsys@transformshift{0.721913in}{2.544437in}%
\pgfsys@useobject{currentmarker}{}%
\end{pgfscope}%
\end{pgfscope}%
\begin{pgfscope}%
\definecolor{textcolor}{rgb}{0.000000,0.000000,0.000000}%
\pgfsetstrokecolor{textcolor}%
\pgfsetfillcolor{textcolor}%
\pgftext[x=0.395525in, y=2.486567in, left, base]{\color{textcolor}{\rmfamily\fontsize{12.000000}{14.400000}\selectfont\catcode`\^=\active\def^{\ifmmode\sp\else\^{}\fi}\catcode`\%=\active\def%{\%}$\mathdefault{10^{0}}$}}%
\end{pgfscope}%
\begin{pgfscope}%
\definecolor{textcolor}{rgb}{0.000000,0.000000,0.000000}%
\pgfsetstrokecolor{textcolor}%
\pgfsetfillcolor{textcolor}%
\pgftext[x=0.248148in,y=1.704073in,,bottom,rotate=90.000000]{\color{textcolor}{\rmfamily\fontsize{12.000000}{14.400000}\selectfont\catcode`\^=\active\def^{\ifmmode\sp\else\^{}\fi}\catcode`\%=\active\def%{\%}$L^1$ error}}%
\end{pgfscope}%
\begin{pgfscope}%
\pgfpathrectangle{\pgfqpoint{0.721913in}{0.549073in}}{\pgfqpoint{3.100000in}{2.310000in}}%
\pgfusepath{clip}%
\pgfsetrectcap%
\pgfsetroundjoin%
\pgfsetlinewidth{1.003750pt}%
\definecolor{currentstroke}{rgb}{1.000000,0.690196,0.000000}%
\pgfsetstrokecolor{currentstroke}%
\pgfsetdash{}{0pt}%
\pgfpathmoveto{\pgfqpoint{0.862823in}{2.754073in}}%
\pgfpathlineto{\pgfqpoint{1.338310in}{2.717608in}}%
\pgfpathlineto{\pgfqpoint{1.805041in}{2.589209in}}%
\pgfpathlineto{\pgfqpoint{2.275443in}{2.218266in}}%
\pgfpathlineto{\pgfqpoint{2.743525in}{0.929687in}}%
\pgfpathlineto{\pgfqpoint{3.212040in}{0.654073in}}%
\pgfpathlineto{\pgfqpoint{3.681004in}{0.795725in}}%
\pgfusepath{stroke}%
\end{pgfscope}%
\begin{pgfscope}%
\pgfpathrectangle{\pgfqpoint{0.721913in}{0.549073in}}{\pgfqpoint{3.100000in}{2.310000in}}%
\pgfusepath{clip}%
\pgfsetbuttcap%
\pgfsetroundjoin%
\definecolor{currentfill}{rgb}{1.000000,0.690196,0.000000}%
\pgfsetfillcolor{currentfill}%
\pgfsetlinewidth{1.003750pt}%
\definecolor{currentstroke}{rgb}{1.000000,0.690196,0.000000}%
\pgfsetstrokecolor{currentstroke}%
\pgfsetdash{}{0pt}%
\pgfsys@defobject{currentmarker}{\pgfqpoint{-0.041667in}{-0.041667in}}{\pgfqpoint{0.041667in}{0.041667in}}{%
\pgfpathmoveto{\pgfqpoint{0.000000in}{-0.041667in}}%
\pgfpathcurveto{\pgfqpoint{0.011050in}{-0.041667in}}{\pgfqpoint{0.021649in}{-0.037276in}}{\pgfqpoint{0.029463in}{-0.029463in}}%
\pgfpathcurveto{\pgfqpoint{0.037276in}{-0.021649in}}{\pgfqpoint{0.041667in}{-0.011050in}}{\pgfqpoint{0.041667in}{0.000000in}}%
\pgfpathcurveto{\pgfqpoint{0.041667in}{0.011050in}}{\pgfqpoint{0.037276in}{0.021649in}}{\pgfqpoint{0.029463in}{0.029463in}}%
\pgfpathcurveto{\pgfqpoint{0.021649in}{0.037276in}}{\pgfqpoint{0.011050in}{0.041667in}}{\pgfqpoint{0.000000in}{0.041667in}}%
\pgfpathcurveto{\pgfqpoint{-0.011050in}{0.041667in}}{\pgfqpoint{-0.021649in}{0.037276in}}{\pgfqpoint{-0.029463in}{0.029463in}}%
\pgfpathcurveto{\pgfqpoint{-0.037276in}{0.021649in}}{\pgfqpoint{-0.041667in}{0.011050in}}{\pgfqpoint{-0.041667in}{0.000000in}}%
\pgfpathcurveto{\pgfqpoint{-0.041667in}{-0.011050in}}{\pgfqpoint{-0.037276in}{-0.021649in}}{\pgfqpoint{-0.029463in}{-0.029463in}}%
\pgfpathcurveto{\pgfqpoint{-0.021649in}{-0.037276in}}{\pgfqpoint{-0.011050in}{-0.041667in}}{\pgfqpoint{0.000000in}{-0.041667in}}%
\pgfpathlineto{\pgfqpoint{0.000000in}{-0.041667in}}%
\pgfpathclose%
\pgfusepath{stroke,fill}%
}%
\begin{pgfscope}%
\pgfsys@transformshift{0.862823in}{2.754073in}%
\pgfsys@useobject{currentmarker}{}%
\end{pgfscope}%
\begin{pgfscope}%
\pgfsys@transformshift{1.338310in}{2.717608in}%
\pgfsys@useobject{currentmarker}{}%
\end{pgfscope}%
\begin{pgfscope}%
\pgfsys@transformshift{1.805041in}{2.589209in}%
\pgfsys@useobject{currentmarker}{}%
\end{pgfscope}%
\begin{pgfscope}%
\pgfsys@transformshift{2.275443in}{2.218266in}%
\pgfsys@useobject{currentmarker}{}%
\end{pgfscope}%
\begin{pgfscope}%
\pgfsys@transformshift{2.743525in}{0.929687in}%
\pgfsys@useobject{currentmarker}{}%
\end{pgfscope}%
\begin{pgfscope}%
\pgfsys@transformshift{3.212040in}{0.654073in}%
\pgfsys@useobject{currentmarker}{}%
\end{pgfscope}%
\begin{pgfscope}%
\pgfsys@transformshift{3.681004in}{0.795725in}%
\pgfsys@useobject{currentmarker}{}%
\end{pgfscope}%
\end{pgfscope}%
\begin{pgfscope}%
\pgfpathrectangle{\pgfqpoint{0.721913in}{0.549073in}}{\pgfqpoint{3.100000in}{2.310000in}}%
\pgfusepath{clip}%
\pgfsetrectcap%
\pgfsetroundjoin%
\pgfsetlinewidth{1.003750pt}%
\definecolor{currentstroke}{rgb}{0.392157,0.560784,1.000000}%
\pgfsetstrokecolor{currentstroke}%
\pgfsetdash{}{0pt}%
\pgfpathmoveto{\pgfqpoint{0.862823in}{2.585949in}}%
\pgfpathlineto{\pgfqpoint{1.338310in}{2.463912in}}%
\pgfpathlineto{\pgfqpoint{1.805041in}{2.278055in}}%
\pgfpathlineto{\pgfqpoint{2.275443in}{1.755127in}}%
\pgfpathlineto{\pgfqpoint{2.743525in}{0.772068in}}%
\pgfpathlineto{\pgfqpoint{3.212040in}{0.702181in}}%
\pgfpathlineto{\pgfqpoint{3.681004in}{0.658138in}}%
\pgfusepath{stroke}%
\end{pgfscope}%
\begin{pgfscope}%
\pgfpathrectangle{\pgfqpoint{0.721913in}{0.549073in}}{\pgfqpoint{3.100000in}{2.310000in}}%
\pgfusepath{clip}%
\pgfsetbuttcap%
\pgfsetmiterjoin%
\definecolor{currentfill}{rgb}{0.392157,0.560784,1.000000}%
\pgfsetfillcolor{currentfill}%
\pgfsetlinewidth{1.003750pt}%
\definecolor{currentstroke}{rgb}{0.392157,0.560784,1.000000}%
\pgfsetstrokecolor{currentstroke}%
\pgfsetdash{}{0pt}%
\pgfsys@defobject{currentmarker}{\pgfqpoint{-0.041667in}{-0.041667in}}{\pgfqpoint{0.041667in}{0.041667in}}{%
\pgfpathmoveto{\pgfqpoint{0.000000in}{0.041667in}}%
\pgfpathlineto{\pgfqpoint{-0.041667in}{-0.041667in}}%
\pgfpathlineto{\pgfqpoint{0.041667in}{-0.041667in}}%
\pgfpathlineto{\pgfqpoint{0.000000in}{0.041667in}}%
\pgfpathclose%
\pgfusepath{stroke,fill}%
}%
\begin{pgfscope}%
\pgfsys@transformshift{0.862823in}{2.585949in}%
\pgfsys@useobject{currentmarker}{}%
\end{pgfscope}%
\begin{pgfscope}%
\pgfsys@transformshift{1.338310in}{2.463912in}%
\pgfsys@useobject{currentmarker}{}%
\end{pgfscope}%
\begin{pgfscope}%
\pgfsys@transformshift{1.805041in}{2.278055in}%
\pgfsys@useobject{currentmarker}{}%
\end{pgfscope}%
\begin{pgfscope}%
\pgfsys@transformshift{2.275443in}{1.755127in}%
\pgfsys@useobject{currentmarker}{}%
\end{pgfscope}%
\begin{pgfscope}%
\pgfsys@transformshift{2.743525in}{0.772068in}%
\pgfsys@useobject{currentmarker}{}%
\end{pgfscope}%
\begin{pgfscope}%
\pgfsys@transformshift{3.212040in}{0.702181in}%
\pgfsys@useobject{currentmarker}{}%
\end{pgfscope}%
\begin{pgfscope}%
\pgfsys@transformshift{3.681004in}{0.658138in}%
\pgfsys@useobject{currentmarker}{}%
\end{pgfscope}%
\end{pgfscope}%
\begin{pgfscope}%
\pgfpathrectangle{\pgfqpoint{0.721913in}{0.549073in}}{\pgfqpoint{3.100000in}{2.310000in}}%
\pgfusepath{clip}%
\pgfsetrectcap%
\pgfsetroundjoin%
\pgfsetlinewidth{1.003750pt}%
\definecolor{currentstroke}{rgb}{0.862745,0.149020,0.498039}%
\pgfsetstrokecolor{currentstroke}%
\pgfsetdash{}{0pt}%
\pgfpathmoveto{\pgfqpoint{0.862823in}{2.585949in}}%
\pgfpathlineto{\pgfqpoint{1.338310in}{2.463912in}}%
\pgfpathlineto{\pgfqpoint{1.805041in}{2.278055in}}%
\pgfpathlineto{\pgfqpoint{2.275443in}{1.755127in}}%
\pgfpathlineto{\pgfqpoint{2.743525in}{0.772068in}}%
\pgfpathlineto{\pgfqpoint{3.212040in}{0.702181in}}%
\pgfpathlineto{\pgfqpoint{3.681004in}{0.658138in}}%
\pgfusepath{stroke}%
\end{pgfscope}%
\begin{pgfscope}%
\pgfpathrectangle{\pgfqpoint{0.721913in}{0.549073in}}{\pgfqpoint{3.100000in}{2.310000in}}%
\pgfusepath{clip}%
\pgfsetbuttcap%
\pgfsetmiterjoin%
\definecolor{currentfill}{rgb}{0.862745,0.149020,0.498039}%
\pgfsetfillcolor{currentfill}%
\pgfsetlinewidth{1.003750pt}%
\definecolor{currentstroke}{rgb}{0.862745,0.149020,0.498039}%
\pgfsetstrokecolor{currentstroke}%
\pgfsetdash{}{0pt}%
\pgfsys@defobject{currentmarker}{\pgfqpoint{-0.041667in}{-0.041667in}}{\pgfqpoint{0.041667in}{0.041667in}}{%
\pgfpathmoveto{\pgfqpoint{-0.041667in}{-0.041667in}}%
\pgfpathlineto{\pgfqpoint{0.041667in}{-0.041667in}}%
\pgfpathlineto{\pgfqpoint{0.041667in}{0.041667in}}%
\pgfpathlineto{\pgfqpoint{-0.041667in}{0.041667in}}%
\pgfpathlineto{\pgfqpoint{-0.041667in}{-0.041667in}}%
\pgfpathclose%
\pgfusepath{stroke,fill}%
}%
\begin{pgfscope}%
\pgfsys@transformshift{0.862823in}{2.585949in}%
\pgfsys@useobject{currentmarker}{}%
\end{pgfscope}%
\begin{pgfscope}%
\pgfsys@transformshift{1.338310in}{2.463912in}%
\pgfsys@useobject{currentmarker}{}%
\end{pgfscope}%
\begin{pgfscope}%
\pgfsys@transformshift{1.805041in}{2.278055in}%
\pgfsys@useobject{currentmarker}{}%
\end{pgfscope}%
\begin{pgfscope}%
\pgfsys@transformshift{2.275443in}{1.755127in}%
\pgfsys@useobject{currentmarker}{}%
\end{pgfscope}%
\begin{pgfscope}%
\pgfsys@transformshift{2.743525in}{0.772068in}%
\pgfsys@useobject{currentmarker}{}%
\end{pgfscope}%
\begin{pgfscope}%
\pgfsys@transformshift{3.212040in}{0.702181in}%
\pgfsys@useobject{currentmarker}{}%
\end{pgfscope}%
\begin{pgfscope}%
\pgfsys@transformshift{3.681004in}{0.658138in}%
\pgfsys@useobject{currentmarker}{}%
\end{pgfscope}%
\end{pgfscope}%
\begin{pgfscope}%
\pgfsetrectcap%
\pgfsetmiterjoin%
\pgfsetlinewidth{0.803000pt}%
\definecolor{currentstroke}{rgb}{0.000000,0.000000,0.000000}%
\pgfsetstrokecolor{currentstroke}%
\pgfsetdash{}{0pt}%
\pgfpathmoveto{\pgfqpoint{0.721913in}{0.549073in}}%
\pgfpathlineto{\pgfqpoint{0.721913in}{2.859073in}}%
\pgfusepath{stroke}%
\end{pgfscope}%
\begin{pgfscope}%
\pgfsetrectcap%
\pgfsetmiterjoin%
\pgfsetlinewidth{0.803000pt}%
\definecolor{currentstroke}{rgb}{0.000000,0.000000,0.000000}%
\pgfsetstrokecolor{currentstroke}%
\pgfsetdash{}{0pt}%
\pgfpathmoveto{\pgfqpoint{3.821913in}{0.549073in}}%
\pgfpathlineto{\pgfqpoint{3.821913in}{2.859073in}}%
\pgfusepath{stroke}%
\end{pgfscope}%
\begin{pgfscope}%
\pgfsetrectcap%
\pgfsetmiterjoin%
\pgfsetlinewidth{0.803000pt}%
\definecolor{currentstroke}{rgb}{0.000000,0.000000,0.000000}%
\pgfsetstrokecolor{currentstroke}%
\pgfsetdash{}{0pt}%
\pgfpathmoveto{\pgfqpoint{0.721913in}{0.549073in}}%
\pgfpathlineto{\pgfqpoint{3.821913in}{0.549073in}}%
\pgfusepath{stroke}%
\end{pgfscope}%
\begin{pgfscope}%
\pgfsetrectcap%
\pgfsetmiterjoin%
\pgfsetlinewidth{0.803000pt}%
\definecolor{currentstroke}{rgb}{0.000000,0.000000,0.000000}%
\pgfsetstrokecolor{currentstroke}%
\pgfsetdash{}{0pt}%
\pgfpathmoveto{\pgfqpoint{0.721913in}{2.859073in}}%
\pgfpathlineto{\pgfqpoint{3.821913in}{2.859073in}}%
\pgfusepath{stroke}%
\end{pgfscope}%
\begin{pgfscope}%
\pgfsetbuttcap%
\pgfsetmiterjoin%
\definecolor{currentfill}{rgb}{1.000000,1.000000,1.000000}%
\pgfsetfillcolor{currentfill}%
\pgfsetfillopacity{0.800000}%
\pgfsetlinewidth{1.003750pt}%
\definecolor{currentstroke}{rgb}{0.800000,0.800000,0.800000}%
\pgfsetstrokecolor{currentstroke}%
\pgfsetstrokeopacity{0.800000}%
\pgfsetdash{}{0pt}%
\pgfpathmoveto{\pgfqpoint{0.838580in}{0.632406in}}%
\pgfpathlineto{\pgfqpoint{2.287606in}{0.632406in}}%
\pgfpathquadraticcurveto{\pgfqpoint{2.320939in}{0.632406in}}{\pgfqpoint{2.320939in}{0.665739in}}%
\pgfpathlineto{\pgfqpoint{2.320939in}{1.346294in}}%
\pgfpathquadraticcurveto{\pgfqpoint{2.320939in}{1.379627in}}{\pgfqpoint{2.287606in}{1.379627in}}%
\pgfpathlineto{\pgfqpoint{0.838580in}{1.379627in}}%
\pgfpathquadraticcurveto{\pgfqpoint{0.805247in}{1.379627in}}{\pgfqpoint{0.805247in}{1.346294in}}%
\pgfpathlineto{\pgfqpoint{0.805247in}{0.665739in}}%
\pgfpathquadraticcurveto{\pgfqpoint{0.805247in}{0.632406in}}{\pgfqpoint{0.838580in}{0.632406in}}%
\pgfpathlineto{\pgfqpoint{0.838580in}{0.632406in}}%
\pgfpathclose%
\pgfusepath{stroke,fill}%
\end{pgfscope}%
\begin{pgfscope}%
\pgfsetrectcap%
\pgfsetroundjoin%
\pgfsetlinewidth{1.003750pt}%
\definecolor{currentstroke}{rgb}{1.000000,0.690196,0.000000}%
\pgfsetstrokecolor{currentstroke}%
\pgfsetdash{}{0pt}%
\pgfpathmoveto{\pgfqpoint{0.871913in}{1.254627in}}%
\pgfpathlineto{\pgfqpoint{1.038580in}{1.254627in}}%
\pgfpathlineto{\pgfqpoint{1.205247in}{1.254627in}}%
\pgfusepath{stroke}%
\end{pgfscope}%
\begin{pgfscope}%
\pgfsetbuttcap%
\pgfsetroundjoin%
\definecolor{currentfill}{rgb}{1.000000,0.690196,0.000000}%
\pgfsetfillcolor{currentfill}%
\pgfsetlinewidth{1.003750pt}%
\definecolor{currentstroke}{rgb}{1.000000,0.690196,0.000000}%
\pgfsetstrokecolor{currentstroke}%
\pgfsetdash{}{0pt}%
\pgfsys@defobject{currentmarker}{\pgfqpoint{-0.041667in}{-0.041667in}}{\pgfqpoint{0.041667in}{0.041667in}}{%
\pgfpathmoveto{\pgfqpoint{0.000000in}{-0.041667in}}%
\pgfpathcurveto{\pgfqpoint{0.011050in}{-0.041667in}}{\pgfqpoint{0.021649in}{-0.037276in}}{\pgfqpoint{0.029463in}{-0.029463in}}%
\pgfpathcurveto{\pgfqpoint{0.037276in}{-0.021649in}}{\pgfqpoint{0.041667in}{-0.011050in}}{\pgfqpoint{0.041667in}{0.000000in}}%
\pgfpathcurveto{\pgfqpoint{0.041667in}{0.011050in}}{\pgfqpoint{0.037276in}{0.021649in}}{\pgfqpoint{0.029463in}{0.029463in}}%
\pgfpathcurveto{\pgfqpoint{0.021649in}{0.037276in}}{\pgfqpoint{0.011050in}{0.041667in}}{\pgfqpoint{0.000000in}{0.041667in}}%
\pgfpathcurveto{\pgfqpoint{-0.011050in}{0.041667in}}{\pgfqpoint{-0.021649in}{0.037276in}}{\pgfqpoint{-0.029463in}{0.029463in}}%
\pgfpathcurveto{\pgfqpoint{-0.037276in}{0.021649in}}{\pgfqpoint{-0.041667in}{0.011050in}}{\pgfqpoint{-0.041667in}{0.000000in}}%
\pgfpathcurveto{\pgfqpoint{-0.041667in}{-0.011050in}}{\pgfqpoint{-0.037276in}{-0.021649in}}{\pgfqpoint{-0.029463in}{-0.029463in}}%
\pgfpathcurveto{\pgfqpoint{-0.021649in}{-0.037276in}}{\pgfqpoint{-0.011050in}{-0.041667in}}{\pgfqpoint{0.000000in}{-0.041667in}}%
\pgfpathlineto{\pgfqpoint{0.000000in}{-0.041667in}}%
\pgfpathclose%
\pgfusepath{stroke,fill}%
}%
\begin{pgfscope}%
\pgfsys@transformshift{1.038580in}{1.254627in}%
\pgfsys@useobject{currentmarker}{}%
\end{pgfscope}%
\end{pgfscope}%
\begin{pgfscope}%
\definecolor{textcolor}{rgb}{0.000000,0.000000,0.000000}%
\pgfsetstrokecolor{textcolor}%
\pgfsetfillcolor{textcolor}%
\pgftext[x=1.338580in,y=1.196294in,left,base]{\color{textcolor}{\rmfamily\fontsize{12.000000}{14.400000}\selectfont\catcode`\^=\active\def^{\ifmmode\sp\else\^{}\fi}\catcode`\%=\active\def%{\%}interpolation}}%
\end{pgfscope}%
\begin{pgfscope}%
\pgfsetrectcap%
\pgfsetroundjoin%
\pgfsetlinewidth{1.003750pt}%
\definecolor{currentstroke}{rgb}{0.392157,0.560784,1.000000}%
\pgfsetstrokecolor{currentstroke}%
\pgfsetdash{}{0pt}%
\pgfpathmoveto{\pgfqpoint{0.871913in}{1.022220in}}%
\pgfpathlineto{\pgfqpoint{1.038580in}{1.022220in}}%
\pgfpathlineto{\pgfqpoint{1.205247in}{1.022220in}}%
\pgfusepath{stroke}%
\end{pgfscope}%
\begin{pgfscope}%
\pgfsetbuttcap%
\pgfsetmiterjoin%
\definecolor{currentfill}{rgb}{0.392157,0.560784,1.000000}%
\pgfsetfillcolor{currentfill}%
\pgfsetlinewidth{1.003750pt}%
\definecolor{currentstroke}{rgb}{0.392157,0.560784,1.000000}%
\pgfsetstrokecolor{currentstroke}%
\pgfsetdash{}{0pt}%
\pgfsys@defobject{currentmarker}{\pgfqpoint{-0.041667in}{-0.041667in}}{\pgfqpoint{0.041667in}{0.041667in}}{%
\pgfpathmoveto{\pgfqpoint{0.000000in}{0.041667in}}%
\pgfpathlineto{\pgfqpoint{-0.041667in}{-0.041667in}}%
\pgfpathlineto{\pgfqpoint{0.041667in}{-0.041667in}}%
\pgfpathlineto{\pgfqpoint{0.000000in}{0.041667in}}%
\pgfpathclose%
\pgfusepath{stroke,fill}%
}%
\begin{pgfscope}%
\pgfsys@transformshift{1.038580in}{1.022220in}%
\pgfsys@useobject{currentmarker}{}%
\end{pgfscope}%
\end{pgfscope}%
\begin{pgfscope}%
\definecolor{textcolor}{rgb}{0.000000,0.000000,0.000000}%
\pgfsetstrokecolor{textcolor}%
\pgfsetfillcolor{textcolor}%
\pgftext[x=1.338580in,y=0.963887in,left,base]{\color{textcolor}{\rmfamily\fontsize{12.000000}{14.400000}\selectfont\catcode`\^=\active\def^{\ifmmode\sp\else\^{}\fi}\catcode`\%=\active\def%{\%}squaring}}%
\end{pgfscope}%
\begin{pgfscope}%
\pgfsetrectcap%
\pgfsetroundjoin%
\pgfsetlinewidth{1.003750pt}%
\definecolor{currentstroke}{rgb}{0.862745,0.149020,0.498039}%
\pgfsetstrokecolor{currentstroke}%
\pgfsetdash{}{0pt}%
\pgfpathmoveto{\pgfqpoint{0.871913in}{0.789813in}}%
\pgfpathlineto{\pgfqpoint{1.038580in}{0.789813in}}%
\pgfpathlineto{\pgfqpoint{1.205247in}{0.789813in}}%
\pgfusepath{stroke}%
\end{pgfscope}%
\begin{pgfscope}%
\pgfsetbuttcap%
\pgfsetmiterjoin%
\definecolor{currentfill}{rgb}{0.862745,0.149020,0.498039}%
\pgfsetfillcolor{currentfill}%
\pgfsetlinewidth{1.003750pt}%
\definecolor{currentstroke}{rgb}{0.862745,0.149020,0.498039}%
\pgfsetstrokecolor{currentstroke}%
\pgfsetdash{}{0pt}%
\pgfsys@defobject{currentmarker}{\pgfqpoint{-0.041667in}{-0.041667in}}{\pgfqpoint{0.041667in}{0.041667in}}{%
\pgfpathmoveto{\pgfqpoint{-0.041667in}{-0.041667in}}%
\pgfpathlineto{\pgfqpoint{0.041667in}{-0.041667in}}%
\pgfpathlineto{\pgfqpoint{0.041667in}{0.041667in}}%
\pgfpathlineto{\pgfqpoint{-0.041667in}{0.041667in}}%
\pgfpathlineto{\pgfqpoint{-0.041667in}{-0.041667in}}%
\pgfpathclose%
\pgfusepath{stroke,fill}%
}%
\begin{pgfscope}%
\pgfsys@transformshift{1.038580in}{0.789813in}%
\pgfsys@useobject{currentmarker}{}%
\end{pgfscope}%
\end{pgfscope}%
\begin{pgfscope}%
\definecolor{textcolor}{rgb}{0.000000,0.000000,0.000000}%
\pgfsetstrokecolor{textcolor}%
\pgfsetfillcolor{textcolor}%
\pgftext[x=1.338580in,y=0.731480in,left,base]{\color{textcolor}{\rmfamily\fontsize{12.000000}{14.400000}\selectfont\catcode`\^=\active\def^{\ifmmode\sp\else\^{}\fi}\catcode`\%=\active\def%{\%}non-negative}}%
\end{pgfscope}%
\end{pgfpicture}%
\makeatother%
\endgroup%

    \caption{Difference in the accuracy when computing the estimate \refequ{equ:cyclic-property} once by separately expanding the squared smoothing kernel $g_{\sigma}^2$ like \cite{lin-2017-randomized-estimation} (\emph{inconsistent}), once by squaring the Chebyshev expansion $g_{\sigma}^{(m)}$ using \refalg{alg:chebyshev-squaring} (\emph{consistent}), and once with a non-negative Chebyshev expansion $\underline{g}_{\sigma}^{(m)}$ \refequ{equ:non-negative-chebyshev-expansion} which is also squared with \refalg{alg:chebyshev-squaring} (\emph{non-negative}). We use a small example matrix from \refsec{sec:results}, fix the size of the Nyström approximation to $n_{\mtx{\Omega}} = 80$, let $\sigma = 0.005$, and compute the $L^1$ error from the approximation of the spectral density at $n_t = 100$ uniformly spaced values of $t$.}
    \label{fig:interpolation-issue}
\end{figure}

Analogous observations for the \enquote{Girard-Hutchinson} term which appears in $\Nystrpp{\mtx{\Omega}}{\mtx{\Psi}}$ when $n_{\mtx{\Psi}}$ is nonzero show that also then, the estimator can be expressed in terms of small quadratic forms involving the matrix $\mtx{B}(t) = \underline{g}_{\sigma}^{(m)}(t \mtx{I}_n - \mtx{A})$:
\begin{align}
    &\frac{1}{n n_{\mtx{\Psi}}} \Trace \big(\mtx{\Psi}^{\top} (\mtx{B}(t) - (\mtx{B}(t) \mtx{\Omega}) (\mtx{\Omega}^{\top} \mtx{B}(t) \mtx{\Omega})^{\dagger} (\mtx{B}(t) \mtx{\Omega})^{\top}) \mtx{\Psi}^{\top} \big) \notag \\
    &= \frac{1}{n n_{\mtx{\Psi}}} \big( \underbrace{\Trace(\mtx{\Psi}^{\top} \mtx{B}(t)\mtx{\Psi})}_{= \ell(t)} - \Trace((\underbrace{\mtx{\Psi}^{\top} \mtx{B}(t) \mtx{\Omega}}_{= \mtx{L}_1(t)^{\top}}) (\underbrace{\mtx{\Omega}^{\top} \mtx{B}(t) \mtx{\Omega}}_{= \mtx{K}_1(t)^{\top}})^{\dagger} (\underbrace{\mtx{\Omega}^{\top} \mtx{B}(t) \mtx{\Psi}^{\top}}_{= \mtx{L}_1(t)})) \big).
    \label{equ:cyclic-property-hutchinson}
\end{align}

An efficient implementation of the whole procedure described in this section can be found in \refalg{alg:nystrom-chebyshev-pp}.

\begin{algo}{Chebyshev-Nyström++ method}{nystrom-chebyshev-pp}
\begin{algorithmic}[1]
    \Statex \textbf{Input:} Symmetric $\mtx{A} \in \mathbb{R}^{n \times n}$ with spectrum in $[-1, 1]$, points $\{t_i\}_{i=1}^{n_t} \subset \mathbb{R}$
    \Statex \textbf{Parameters:} Even degree of expansion $m \in \mathbb{N}$, number of Girard-Hutchinson queries $n_{\mtx{\Psi}} \in \mathbb{N}_0$, Nyström sketch size $n_{\mtx{\Omega}} \in \mathbb{N}_0$,  smoothing parameter $\sigma > 0$
    \Statex \textbf{Output:} Approximate evaluations of the spectral density $\{\widetilde{\underline{\phi}}_{\sigma}^{(m)}(t_i)\}_{i=1}^{n_t}$
    \For {$i = 1, \dots, n_t$}
        \State Compute the coefficients $\{\xi_l(t_i)\}_{l=0}^{m/2}$ of the Chebyshev expansion of $\sqrt{g_{\sigma}}$ \refequ{equ:square-root-chebyshev-expansion}
        \State Compute the coefficients $\{\underline{\mu}_l(t_i)\}_{l=0}^{m}$ of the non-negative Chebyshev expansion by applying \refalg{alg:chebyshev-squaring} to $\{\xi_l(t_i)\}_{l=0}^{m/2}$
        \State Compute the coefficients $\{\underline{\nu}_l(t_i)\}_{l=0}^{2m}$ of the square of the non-negative Chebyshev expansion by applying \refalg{alg:chebyshev-squaring} to $\{\underline{\mu}_l(t_i)\}_{l=0}^{m}$
    \EndFor
    %\State Compute the coefficients $\{\underline{\mu}_l(t_i)\}_{l=0}^{m}$ of the non-negative Chebyshev expansion \refequ{equ:non-negative-chebyshev-expansion} for all points $t_i$ \textcolor{red}{maybe more details needed (maybe compute $\xi$ for m/2, then twice Alg 3.3)}
    %\State Compute the coefficients $\{\nu_l(t_i)\}_{l=0}^{2m}$ of the square of the non-negative Chebyshev expansion with coefficients $\{\underline{\mu}_l(t_i)\}_{l=0}^{m}$ using \refalg{alg:chebyshev-squaring}%\refalg{alg:2-chebyshev-chebyshev-expansion}
    %\State Compute  for all $t_i$ using %\refalg{alg:3-nystrom-chebyshev-exponentiation}
    \State Generate Gaussian random matrices $\mtx{\Omega} \in \mathbb{R}^{n \times n_{\mtx{\Omega}}}$ and $\mtx{\Psi} \in \mathbb{R}^{n \times n_{\mtx{\Psi}}}$%\glsfirst{sketching-matrix} $\in \mathbb{R}^{n \times n_{\mtx{\Omega}}}$
    %\State Generate standard Gaussian %\glsfirst{random-matrix} $\in \mathbb{R}^{n \times n_{\mtx{\Psi}}}$
    \State Initialize $[\mtx{V}_1, \mtx{V}_2, \mtx{V}_3] \gets [\mtx{0}_{n \times n_{\mtx{\Omega}}}, \mtx{\mtx{\Omega}}, \mtx{0}_{n \times n_{\mtx{\Omega}}}]$
    \State Initialize $[\mtx{W}_1, \mtx{W}_2, \mtx{W}_3] \gets [\mtx{0}_{n \times n_{\mtx{\Psi}}}, \mtx{\Psi}, \mtx{0}_{n \times n_{\mtx{\Psi}}}]$
    \State Initialize $[\mtx{K}_1(t_i), \mtx{K}_2(t_i)] \gets [\mtx{0}_{n_{\mtx{\Omega}} \times n_{\mtx{\Omega}}}, \mtx{0}_{n_{\mtx{\Omega}} \times n_{\mtx{\Omega}}}]$ for all $t_i$
    \State Initialize $[\mtx{L}_1(t_i), \ell(t_i)] \gets [\mtx{0}_{n_{\mtx{\Omega}} \times n_{\mtx{\Psi}}}, 0]$ for all $t_i$
    %\State Set $\breve{\phi}_{\sigma}^{(m)}(t_i) \gets 0$ for all $t_i$
    \For {$l = 0, \dots, 2m$}
    \State $[\mtx{X}, \mtx{Y}] \gets \mtx{\mtx{\Omega}}^{\top} [\mtx{V}_2, \mtx{W}_2]$  
    %\State $\mtx{X} \gets \mtx{\mtx{\Omega}}^{\top} \mtx{V}_2$
      %\State $\mtx{Y} \gets \mtx{\mtx{\Omega}}^{\top} \mtx{W}_2$
      \State $z \gets \Trace(\mtx{\Psi}^{\top} \mtx{W}_2)$
      \For {$i = 1, \dots, n_t$}
        \If {$l \leq m$}
            \State $\mtx{K}_1(t_i) \gets \mtx{K}_1(t_i) + \underline{\mu}_l(t_i) \mtx{X}$ \Comment{assembles $\mtx{\Omega}^{\top} \underline{g}_{\sigma}^{(m)}(t\mtx{I}_n - \mtx{A}) \mtx{\Omega}$}
            \State $\mtx{L}_1(t_i) \gets \mtx{L}_1(t_i) + \underline{\mu}_l(t_i) \mtx{Y}$ \Comment{assembles $\mtx{\Omega}^{\top} \underline{g}_{\sigma}^{(m)}(t\mtx{I}_n - \mtx{A}) \mtx{\Psi}$}
            \State $\ell(t_i) \gets \ell(t_i) + \underline{\mu}_l(t_i) z$ \Comment{assembles $\Trace(\mtx{\Psi}^{\top} \underline{g}_{\sigma}^{(m)}(t\mtx{I}_n - \mtx{A}) \mtx{\Psi})$}
        \EndIf
        \State $\mtx{K}_2(t_i) \gets \mtx{K}_2(t_i) + \underline{\nu}_l(t_i) \mtx{X}$ \Comment{assembles $\mtx{\Omega}^{\top} \underline{g}_{\sigma}^{(m)}(t\mtx{I}_n - \mtx{A})^2 \mtx{\Omega}$}
      \EndFor
      \State $[\mtx{V}_3, \mtx{W}_3] \gets (2 - \delta_{l0}) \mtx{A} [\mtx{V}_2, \mtx{W}_2] - [\mtx{V}_1, \mtx{W}_1]$ \Comment{Chebyshev recurrence}
      \State $[\mtx{V}_1, \mtx{W}_1] \gets [\mtx{V}_2, \mtx{W}_2]$
      \State $[\mtx{V}_2, \mtx{W}_2] \gets [\mtx{V}_3, \mtx{W}_3]$
      %\State $\mtx{V}_3 \gets (2 - \delta_{l0}) \mtx{A} \mtx{V}_2 - \mtx{V}_1$ %\Comment{Chebyshev recurrence \refequ{equ:2-chebyshev-chebyshev-recursion}}
      %\State $\mtx{V}_1 \gets \mtx{V}_2, \mtx{V}_2 \gets \mtx{V}_3$
      %\State $\mtx{W}_3 \gets (2 - \delta_{l0}) \mtx{A} \mtx{W}_2 - \mtx{W}_1$ %\Comment{Chebyshev recurrence \refequ{equ:2-chebyshev-chebyshev-recursion}}
      %\State $\mtx{W}_1 \gets \mtx{W}_2, \mtx{W}_2 \gets \mtx{W}_3$
    \EndFor
    \For {$i = 1, \dots, n_t$}
      \State $\widetilde{\underline{\phi}}_{\sigma}(t_i) \gets \frac{1}{n} \Trace\left( \mtx{K}_1(t_i)^{\dagger}\mtx{K}_2(t_i) \right) + \frac{1}{n n_{\mtx{\Psi}}} \left( \ell(t_i) + \Trace\left( \mtx{L}_1(t_i)^{\top} \mtx{K}_1(t_i)^{\dagger} \mtx{L}_1(t_i) \right)  \right) $ \label{lin:4-nystromchebyshev-nystrom-pp}
    \EndFor
\end{algorithmic}
\end{algo}

The dominating cost in \refalg{alg:nystrom-chebyshev-pp} stems from the $(2m + 1)(n_{\mtx{\Psi}} + n_{\mtx{\Omega}})$ matrix-vector multiplications with $\mtx{A}$ and the cost associated with computing the estimate in each evaluation point $t_1, \dots, t_{n_t}$ given by $\mathcal{O}(m \log(m) + m (n_{\mtx{\Omega}}^2 + n_{\mtx{\Omega}} n_{\mtx{\Psi}} + 1) + (n_{\mtx{\Omega}}^3 + n_{\mtx{\Omega}} n_{\mtx{\Psi}}^2 + 1))$.

The algorithm also accomodates the two other estimators introduced in \refsec{sec:analysis}: For $n_{\mtx{\Omega}} = 0$ we recover the Girard-Hutchinson estimator \refequ{equ:hutchinson-trace-estimator}, and the complexity for each evaluation of the estimate decreases significantly. For $n_{\mtx{\Psi}} = 0$ we end up with the Nyström estimator \refequ{equ:nystrom-trace-estimator}, provided the implementation evaluates $0/0$ to $0$ in the second term on \reflin{lin:4-nystromchebyshev-nystrom-pp}. Notice that in this case, \refalg{alg:nystrom-chebyshev-pp} is equivalent to the \enquote{spectrum sweeping} method from \cite[Algorithm 5]{lin-2017-randomized-estimation}, but has a significantly lower computational complexity.

%Chebyshev $\mathcal{O}(n_t (m \log(m) + m (n_{\mtx{\Omega}}^2 + n_{\mtx{\Omega}} n_{\mtx{\Psi}} + 1) + n_{\mtx{\Omega}}^3) + mn^2(n_{\mtx{\Psi}} + n_{\mtx{\Omega}}))$.

%\todo{Complexity in terms of both $n_{\mtx{\Omega}}$ and $n_{\mtx{\Psi}}$.}
%With the cost of a matrix-vector product denoted by $c(n)$, and supposing $n_{\mtx{\Omega}} \approx n_{\mtx{\Psi}}$, we determine the computational complexity of the Chebyshev-Nyström++ method to be $\mathcal{O}(m \log(m) n_t + m n_{\mtx{\Omega}}^2 n + m n_t n_{\mtx{\Omega}}^2 +  m c(n) n_{\mtx{\Omega}} + n_t n_{\mtx{\Omega}}^3)$, with $\mathcal{O}(m n_t + n n_{\mtx{\Omega}} + n_{\mtx{\Omega}}^2 n_t)$ required additional storage.

%\todo{Explain modifications for other estimators, and improved complexity for quadratic trace estimator}

The pseudoinverses on \reflin{lin:4-nystromchebyshev-nystrom-pp} need to be computed with care. It is possible to reformulate the problem as a generalized eigenvalue problem and apply thresholding to the smallest eigenvalues of a certain matrix to enforce stability \cite{lin-2017-randomized-estimation, epperly-2022-theory-quantum}. Since this renders the algorithm significantly more complicated but only improved the accuracy slightly during our experiments, we will instead evaluate these expressions in \reflin{lin:4-nystromchebyshev-nystrom-pp} using a least-squares solver which likewise uses a truncation of the smallest eigenvalues to enforce better conditioning.

What we have found to be crucial for a successful approximation of the full spectral density was the detection of regions wherein lie no eigenvalues, i.e. regions where the spectral density nearly vanishes. For small smoothing parameters $\sigma$, the rapid decay of the Gaussian $g_{\sigma}$ will cause $\mtx{K}_1(t) \approx \mtx{\Omega}^{\top} g_{\sigma}(t\mtx{I}_n - \mtx{A}) \mtx{\Omega}$ to be close to the zero matrix for parameter values $t$ located in such regions. Hence, on \reflin{lin:4-nystromchebyshev-nystrom-pp} we would compute the pseudoinverse of an almost zero matrix, which can be extremely unstable (see \reffig{fig:zerocheck}). Notice that with $\mtx{K}_1(t)$ -- a matrix we need to assemble anyway -- we can cheaply compute $\Trace(\mtx{K}_1(t)) / n_{\mtx{\Omega}}$, which is the $n_{\mtx{\Omega}}$-query Girard-Hutchinson estimator. We may use it as a rough indicator whether $t$ lies within a region where the spectral density nearly vanishes, and subsequently set $\widetilde{\underline{\phi}}_{\sigma}^{(m)}(t) = 0$ instead of computing the full expression on \reflin{lin:4-nystromchebyshev-nystrom-pp}.

\begin{figure}[ht]
    \centering
    %% Creator: Matplotlib, PGF backend
%%
%% To include the figure in your LaTeX document, write
%%   \input{<filename>.pgf}
%%
%% Make sure the required packages are loaded in your preamble
%%   \usepackage{pgf}
%%
%% Also ensure that all the required font packages are loaded; for instance,
%% the lmodern package is sometimes necessary when using math font.
%%   \usepackage{lmodern}
%%
%% Figures using additional raster images can only be included by \input if
%% they are in the same directory as the main LaTeX file. For loading figures
%% from other directories you can use the `import` package
%%   \usepackage{import}
%%
%% and then include the figures with
%%   \import{<path to file>}{<filename>.pgf}
%%
%% Matplotlib used the following preamble
%%   \def\mathdefault#1{#1}
%%   \everymath=\expandafter{\the\everymath\displaystyle}
%%   
%%   \ifdefined\pdftexversion\else  % non-pdftex case.
%%     \usepackage{fontspec}
%%     \setmainfont{DejaVuSerif.ttf}[Path=\detokenize{/opt/hostedtoolcache/Python/3.12.4/x64/lib/python3.12/site-packages/matplotlib/mpl-data/fonts/ttf/}]
%%     \setsansfont{DejaVuSans.ttf}[Path=\detokenize{/opt/hostedtoolcache/Python/3.12.4/x64/lib/python3.12/site-packages/matplotlib/mpl-data/fonts/ttf/}]
%%     \setmonofont{DejaVuSansMono.ttf}[Path=\detokenize{/opt/hostedtoolcache/Python/3.12.4/x64/lib/python3.12/site-packages/matplotlib/mpl-data/fonts/ttf/}]
%%   \fi
%%   \makeatletter\@ifpackageloaded{underscore}{}{\usepackage[strings]{underscore}}\makeatother
%%
\begingroup%
\makeatletter%
\begin{pgfpicture}%
\pgfpathrectangle{\pgfpointorigin}{\pgfqpoint{5.238370in}{2.959073in}}%
\pgfusepath{use as bounding box, clip}%
\begin{pgfscope}%
\pgfsetbuttcap%
\pgfsetmiterjoin%
\definecolor{currentfill}{rgb}{1.000000,1.000000,1.000000}%
\pgfsetfillcolor{currentfill}%
\pgfsetlinewidth{0.000000pt}%
\definecolor{currentstroke}{rgb}{1.000000,1.000000,1.000000}%
\pgfsetstrokecolor{currentstroke}%
\pgfsetdash{}{0pt}%
\pgfpathmoveto{\pgfqpoint{0.000000in}{-0.000000in}}%
\pgfpathlineto{\pgfqpoint{5.238370in}{-0.000000in}}%
\pgfpathlineto{\pgfqpoint{5.238370in}{2.959073in}}%
\pgfpathlineto{\pgfqpoint{0.000000in}{2.959073in}}%
\pgfpathlineto{\pgfqpoint{0.000000in}{-0.000000in}}%
\pgfpathclose%
\pgfusepath{fill}%
\end{pgfscope}%
\begin{pgfscope}%
\pgfsetbuttcap%
\pgfsetmiterjoin%
\definecolor{currentfill}{rgb}{1.000000,1.000000,1.000000}%
\pgfsetfillcolor{currentfill}%
\pgfsetlinewidth{0.000000pt}%
\definecolor{currentstroke}{rgb}{0.000000,0.000000,0.000000}%
\pgfsetstrokecolor{currentstroke}%
\pgfsetstrokeopacity{0.000000}%
\pgfsetdash{}{0pt}%
\pgfpathmoveto{\pgfqpoint{0.582637in}{0.549073in}}%
\pgfpathlineto{\pgfqpoint{5.100887in}{0.549073in}}%
\pgfpathlineto{\pgfqpoint{5.100887in}{2.859073in}}%
\pgfpathlineto{\pgfqpoint{0.582637in}{2.859073in}}%
\pgfpathlineto{\pgfqpoint{0.582637in}{0.549073in}}%
\pgfpathclose%
\pgfusepath{fill}%
\end{pgfscope}%
\begin{pgfscope}%
\pgfpathrectangle{\pgfqpoint{0.582637in}{0.549073in}}{\pgfqpoint{4.518250in}{2.310000in}}%
\pgfusepath{clip}%
\pgfsetrectcap%
\pgfsetroundjoin%
\pgfsetlinewidth{0.250937pt}%
\definecolor{currentstroke}{rgb}{0.000000,0.000000,0.000000}%
\pgfsetstrokecolor{currentstroke}%
\pgfsetstrokeopacity{0.200000}%
\pgfsetdash{}{0pt}%
\pgfpathmoveto{\pgfqpoint{0.690215in}{0.549073in}}%
\pgfpathlineto{\pgfqpoint{0.690215in}{2.859073in}}%
\pgfusepath{stroke}%
\end{pgfscope}%
\begin{pgfscope}%
\pgfsetbuttcap%
\pgfsetroundjoin%
\definecolor{currentfill}{rgb}{0.000000,0.000000,0.000000}%
\pgfsetfillcolor{currentfill}%
\pgfsetlinewidth{0.803000pt}%
\definecolor{currentstroke}{rgb}{0.000000,0.000000,0.000000}%
\pgfsetstrokecolor{currentstroke}%
\pgfsetdash{}{0pt}%
\pgfsys@defobject{currentmarker}{\pgfqpoint{0.000000in}{-0.048611in}}{\pgfqpoint{0.000000in}{0.000000in}}{%
\pgfpathmoveto{\pgfqpoint{0.000000in}{0.000000in}}%
\pgfpathlineto{\pgfqpoint{0.000000in}{-0.048611in}}%
\pgfusepath{stroke,fill}%
}%
\begin{pgfscope}%
\pgfsys@transformshift{0.690215in}{0.549073in}%
\pgfsys@useobject{currentmarker}{}%
\end{pgfscope}%
\end{pgfscope}%
\begin{pgfscope}%
\definecolor{textcolor}{rgb}{0.000000,0.000000,0.000000}%
\pgfsetstrokecolor{textcolor}%
\pgfsetfillcolor{textcolor}%
\pgftext[x=0.690215in,y=0.451851in,,top]{\color{textcolor}{\rmfamily\fontsize{12.000000}{14.400000}\selectfont\catcode`\^=\active\def^{\ifmmode\sp\else\^{}\fi}\catcode`\%=\active\def%{\%}$\mathdefault{\ensuremath{-}1.00}$}}%
\end{pgfscope}%
\begin{pgfscope}%
\pgfpathrectangle{\pgfqpoint{0.582637in}{0.549073in}}{\pgfqpoint{4.518250in}{2.310000in}}%
\pgfusepath{clip}%
\pgfsetrectcap%
\pgfsetroundjoin%
\pgfsetlinewidth{0.250937pt}%
\definecolor{currentstroke}{rgb}{0.000000,0.000000,0.000000}%
\pgfsetstrokecolor{currentstroke}%
\pgfsetstrokeopacity{0.200000}%
\pgfsetdash{}{0pt}%
\pgfpathmoveto{\pgfqpoint{1.228102in}{0.549073in}}%
\pgfpathlineto{\pgfqpoint{1.228102in}{2.859073in}}%
\pgfusepath{stroke}%
\end{pgfscope}%
\begin{pgfscope}%
\pgfsetbuttcap%
\pgfsetroundjoin%
\definecolor{currentfill}{rgb}{0.000000,0.000000,0.000000}%
\pgfsetfillcolor{currentfill}%
\pgfsetlinewidth{0.803000pt}%
\definecolor{currentstroke}{rgb}{0.000000,0.000000,0.000000}%
\pgfsetstrokecolor{currentstroke}%
\pgfsetdash{}{0pt}%
\pgfsys@defobject{currentmarker}{\pgfqpoint{0.000000in}{-0.048611in}}{\pgfqpoint{0.000000in}{0.000000in}}{%
\pgfpathmoveto{\pgfqpoint{0.000000in}{0.000000in}}%
\pgfpathlineto{\pgfqpoint{0.000000in}{-0.048611in}}%
\pgfusepath{stroke,fill}%
}%
\begin{pgfscope}%
\pgfsys@transformshift{1.228102in}{0.549073in}%
\pgfsys@useobject{currentmarker}{}%
\end{pgfscope}%
\end{pgfscope}%
\begin{pgfscope}%
\definecolor{textcolor}{rgb}{0.000000,0.000000,0.000000}%
\pgfsetstrokecolor{textcolor}%
\pgfsetfillcolor{textcolor}%
\pgftext[x=1.228102in,y=0.451851in,,top]{\color{textcolor}{\rmfamily\fontsize{12.000000}{14.400000}\selectfont\catcode`\^=\active\def^{\ifmmode\sp\else\^{}\fi}\catcode`\%=\active\def%{\%}$\mathdefault{\ensuremath{-}0.75}$}}%
\end{pgfscope}%
\begin{pgfscope}%
\pgfpathrectangle{\pgfqpoint{0.582637in}{0.549073in}}{\pgfqpoint{4.518250in}{2.310000in}}%
\pgfusepath{clip}%
\pgfsetrectcap%
\pgfsetroundjoin%
\pgfsetlinewidth{0.250937pt}%
\definecolor{currentstroke}{rgb}{0.000000,0.000000,0.000000}%
\pgfsetstrokecolor{currentstroke}%
\pgfsetstrokeopacity{0.200000}%
\pgfsetdash{}{0pt}%
\pgfpathmoveto{\pgfqpoint{1.765988in}{0.549073in}}%
\pgfpathlineto{\pgfqpoint{1.765988in}{2.859073in}}%
\pgfusepath{stroke}%
\end{pgfscope}%
\begin{pgfscope}%
\pgfsetbuttcap%
\pgfsetroundjoin%
\definecolor{currentfill}{rgb}{0.000000,0.000000,0.000000}%
\pgfsetfillcolor{currentfill}%
\pgfsetlinewidth{0.803000pt}%
\definecolor{currentstroke}{rgb}{0.000000,0.000000,0.000000}%
\pgfsetstrokecolor{currentstroke}%
\pgfsetdash{}{0pt}%
\pgfsys@defobject{currentmarker}{\pgfqpoint{0.000000in}{-0.048611in}}{\pgfqpoint{0.000000in}{0.000000in}}{%
\pgfpathmoveto{\pgfqpoint{0.000000in}{0.000000in}}%
\pgfpathlineto{\pgfqpoint{0.000000in}{-0.048611in}}%
\pgfusepath{stroke,fill}%
}%
\begin{pgfscope}%
\pgfsys@transformshift{1.765988in}{0.549073in}%
\pgfsys@useobject{currentmarker}{}%
\end{pgfscope}%
\end{pgfscope}%
\begin{pgfscope}%
\definecolor{textcolor}{rgb}{0.000000,0.000000,0.000000}%
\pgfsetstrokecolor{textcolor}%
\pgfsetfillcolor{textcolor}%
\pgftext[x=1.765988in,y=0.451851in,,top]{\color{textcolor}{\rmfamily\fontsize{12.000000}{14.400000}\selectfont\catcode`\^=\active\def^{\ifmmode\sp\else\^{}\fi}\catcode`\%=\active\def%{\%}$\mathdefault{\ensuremath{-}0.50}$}}%
\end{pgfscope}%
\begin{pgfscope}%
\pgfpathrectangle{\pgfqpoint{0.582637in}{0.549073in}}{\pgfqpoint{4.518250in}{2.310000in}}%
\pgfusepath{clip}%
\pgfsetrectcap%
\pgfsetroundjoin%
\pgfsetlinewidth{0.250937pt}%
\definecolor{currentstroke}{rgb}{0.000000,0.000000,0.000000}%
\pgfsetstrokecolor{currentstroke}%
\pgfsetstrokeopacity{0.200000}%
\pgfsetdash{}{0pt}%
\pgfpathmoveto{\pgfqpoint{2.303875in}{0.549073in}}%
\pgfpathlineto{\pgfqpoint{2.303875in}{2.859073in}}%
\pgfusepath{stroke}%
\end{pgfscope}%
\begin{pgfscope}%
\pgfsetbuttcap%
\pgfsetroundjoin%
\definecolor{currentfill}{rgb}{0.000000,0.000000,0.000000}%
\pgfsetfillcolor{currentfill}%
\pgfsetlinewidth{0.803000pt}%
\definecolor{currentstroke}{rgb}{0.000000,0.000000,0.000000}%
\pgfsetstrokecolor{currentstroke}%
\pgfsetdash{}{0pt}%
\pgfsys@defobject{currentmarker}{\pgfqpoint{0.000000in}{-0.048611in}}{\pgfqpoint{0.000000in}{0.000000in}}{%
\pgfpathmoveto{\pgfqpoint{0.000000in}{0.000000in}}%
\pgfpathlineto{\pgfqpoint{0.000000in}{-0.048611in}}%
\pgfusepath{stroke,fill}%
}%
\begin{pgfscope}%
\pgfsys@transformshift{2.303875in}{0.549073in}%
\pgfsys@useobject{currentmarker}{}%
\end{pgfscope}%
\end{pgfscope}%
\begin{pgfscope}%
\definecolor{textcolor}{rgb}{0.000000,0.000000,0.000000}%
\pgfsetstrokecolor{textcolor}%
\pgfsetfillcolor{textcolor}%
\pgftext[x=2.303875in,y=0.451851in,,top]{\color{textcolor}{\rmfamily\fontsize{12.000000}{14.400000}\selectfont\catcode`\^=\active\def^{\ifmmode\sp\else\^{}\fi}\catcode`\%=\active\def%{\%}$\mathdefault{\ensuremath{-}0.25}$}}%
\end{pgfscope}%
\begin{pgfscope}%
\pgfpathrectangle{\pgfqpoint{0.582637in}{0.549073in}}{\pgfqpoint{4.518250in}{2.310000in}}%
\pgfusepath{clip}%
\pgfsetrectcap%
\pgfsetroundjoin%
\pgfsetlinewidth{0.250937pt}%
\definecolor{currentstroke}{rgb}{0.000000,0.000000,0.000000}%
\pgfsetstrokecolor{currentstroke}%
\pgfsetstrokeopacity{0.200000}%
\pgfsetdash{}{0pt}%
\pgfpathmoveto{\pgfqpoint{2.841762in}{0.549073in}}%
\pgfpathlineto{\pgfqpoint{2.841762in}{2.859073in}}%
\pgfusepath{stroke}%
\end{pgfscope}%
\begin{pgfscope}%
\pgfsetbuttcap%
\pgfsetroundjoin%
\definecolor{currentfill}{rgb}{0.000000,0.000000,0.000000}%
\pgfsetfillcolor{currentfill}%
\pgfsetlinewidth{0.803000pt}%
\definecolor{currentstroke}{rgb}{0.000000,0.000000,0.000000}%
\pgfsetstrokecolor{currentstroke}%
\pgfsetdash{}{0pt}%
\pgfsys@defobject{currentmarker}{\pgfqpoint{0.000000in}{-0.048611in}}{\pgfqpoint{0.000000in}{0.000000in}}{%
\pgfpathmoveto{\pgfqpoint{0.000000in}{0.000000in}}%
\pgfpathlineto{\pgfqpoint{0.000000in}{-0.048611in}}%
\pgfusepath{stroke,fill}%
}%
\begin{pgfscope}%
\pgfsys@transformshift{2.841762in}{0.549073in}%
\pgfsys@useobject{currentmarker}{}%
\end{pgfscope}%
\end{pgfscope}%
\begin{pgfscope}%
\definecolor{textcolor}{rgb}{0.000000,0.000000,0.000000}%
\pgfsetstrokecolor{textcolor}%
\pgfsetfillcolor{textcolor}%
\pgftext[x=2.841762in,y=0.451851in,,top]{\color{textcolor}{\rmfamily\fontsize{12.000000}{14.400000}\selectfont\catcode`\^=\active\def^{\ifmmode\sp\else\^{}\fi}\catcode`\%=\active\def%{\%}$\mathdefault{0.00}$}}%
\end{pgfscope}%
\begin{pgfscope}%
\pgfpathrectangle{\pgfqpoint{0.582637in}{0.549073in}}{\pgfqpoint{4.518250in}{2.310000in}}%
\pgfusepath{clip}%
\pgfsetrectcap%
\pgfsetroundjoin%
\pgfsetlinewidth{0.250937pt}%
\definecolor{currentstroke}{rgb}{0.000000,0.000000,0.000000}%
\pgfsetstrokecolor{currentstroke}%
\pgfsetstrokeopacity{0.200000}%
\pgfsetdash{}{0pt}%
\pgfpathmoveto{\pgfqpoint{3.379649in}{0.549073in}}%
\pgfpathlineto{\pgfqpoint{3.379649in}{2.859073in}}%
\pgfusepath{stroke}%
\end{pgfscope}%
\begin{pgfscope}%
\pgfsetbuttcap%
\pgfsetroundjoin%
\definecolor{currentfill}{rgb}{0.000000,0.000000,0.000000}%
\pgfsetfillcolor{currentfill}%
\pgfsetlinewidth{0.803000pt}%
\definecolor{currentstroke}{rgb}{0.000000,0.000000,0.000000}%
\pgfsetstrokecolor{currentstroke}%
\pgfsetdash{}{0pt}%
\pgfsys@defobject{currentmarker}{\pgfqpoint{0.000000in}{-0.048611in}}{\pgfqpoint{0.000000in}{0.000000in}}{%
\pgfpathmoveto{\pgfqpoint{0.000000in}{0.000000in}}%
\pgfpathlineto{\pgfqpoint{0.000000in}{-0.048611in}}%
\pgfusepath{stroke,fill}%
}%
\begin{pgfscope}%
\pgfsys@transformshift{3.379649in}{0.549073in}%
\pgfsys@useobject{currentmarker}{}%
\end{pgfscope}%
\end{pgfscope}%
\begin{pgfscope}%
\definecolor{textcolor}{rgb}{0.000000,0.000000,0.000000}%
\pgfsetstrokecolor{textcolor}%
\pgfsetfillcolor{textcolor}%
\pgftext[x=3.379649in,y=0.451851in,,top]{\color{textcolor}{\rmfamily\fontsize{12.000000}{14.400000}\selectfont\catcode`\^=\active\def^{\ifmmode\sp\else\^{}\fi}\catcode`\%=\active\def%{\%}$\mathdefault{0.25}$}}%
\end{pgfscope}%
\begin{pgfscope}%
\pgfpathrectangle{\pgfqpoint{0.582637in}{0.549073in}}{\pgfqpoint{4.518250in}{2.310000in}}%
\pgfusepath{clip}%
\pgfsetrectcap%
\pgfsetroundjoin%
\pgfsetlinewidth{0.250937pt}%
\definecolor{currentstroke}{rgb}{0.000000,0.000000,0.000000}%
\pgfsetstrokecolor{currentstroke}%
\pgfsetstrokeopacity{0.200000}%
\pgfsetdash{}{0pt}%
\pgfpathmoveto{\pgfqpoint{3.917536in}{0.549073in}}%
\pgfpathlineto{\pgfqpoint{3.917536in}{2.859073in}}%
\pgfusepath{stroke}%
\end{pgfscope}%
\begin{pgfscope}%
\pgfsetbuttcap%
\pgfsetroundjoin%
\definecolor{currentfill}{rgb}{0.000000,0.000000,0.000000}%
\pgfsetfillcolor{currentfill}%
\pgfsetlinewidth{0.803000pt}%
\definecolor{currentstroke}{rgb}{0.000000,0.000000,0.000000}%
\pgfsetstrokecolor{currentstroke}%
\pgfsetdash{}{0pt}%
\pgfsys@defobject{currentmarker}{\pgfqpoint{0.000000in}{-0.048611in}}{\pgfqpoint{0.000000in}{0.000000in}}{%
\pgfpathmoveto{\pgfqpoint{0.000000in}{0.000000in}}%
\pgfpathlineto{\pgfqpoint{0.000000in}{-0.048611in}}%
\pgfusepath{stroke,fill}%
}%
\begin{pgfscope}%
\pgfsys@transformshift{3.917536in}{0.549073in}%
\pgfsys@useobject{currentmarker}{}%
\end{pgfscope}%
\end{pgfscope}%
\begin{pgfscope}%
\definecolor{textcolor}{rgb}{0.000000,0.000000,0.000000}%
\pgfsetstrokecolor{textcolor}%
\pgfsetfillcolor{textcolor}%
\pgftext[x=3.917536in,y=0.451851in,,top]{\color{textcolor}{\rmfamily\fontsize{12.000000}{14.400000}\selectfont\catcode`\^=\active\def^{\ifmmode\sp\else\^{}\fi}\catcode`\%=\active\def%{\%}$\mathdefault{0.50}$}}%
\end{pgfscope}%
\begin{pgfscope}%
\pgfpathrectangle{\pgfqpoint{0.582637in}{0.549073in}}{\pgfqpoint{4.518250in}{2.310000in}}%
\pgfusepath{clip}%
\pgfsetrectcap%
\pgfsetroundjoin%
\pgfsetlinewidth{0.250937pt}%
\definecolor{currentstroke}{rgb}{0.000000,0.000000,0.000000}%
\pgfsetstrokecolor{currentstroke}%
\pgfsetstrokeopacity{0.200000}%
\pgfsetdash{}{0pt}%
\pgfpathmoveto{\pgfqpoint{4.455423in}{0.549073in}}%
\pgfpathlineto{\pgfqpoint{4.455423in}{2.859073in}}%
\pgfusepath{stroke}%
\end{pgfscope}%
\begin{pgfscope}%
\pgfsetbuttcap%
\pgfsetroundjoin%
\definecolor{currentfill}{rgb}{0.000000,0.000000,0.000000}%
\pgfsetfillcolor{currentfill}%
\pgfsetlinewidth{0.803000pt}%
\definecolor{currentstroke}{rgb}{0.000000,0.000000,0.000000}%
\pgfsetstrokecolor{currentstroke}%
\pgfsetdash{}{0pt}%
\pgfsys@defobject{currentmarker}{\pgfqpoint{0.000000in}{-0.048611in}}{\pgfqpoint{0.000000in}{0.000000in}}{%
\pgfpathmoveto{\pgfqpoint{0.000000in}{0.000000in}}%
\pgfpathlineto{\pgfqpoint{0.000000in}{-0.048611in}}%
\pgfusepath{stroke,fill}%
}%
\begin{pgfscope}%
\pgfsys@transformshift{4.455423in}{0.549073in}%
\pgfsys@useobject{currentmarker}{}%
\end{pgfscope}%
\end{pgfscope}%
\begin{pgfscope}%
\definecolor{textcolor}{rgb}{0.000000,0.000000,0.000000}%
\pgfsetstrokecolor{textcolor}%
\pgfsetfillcolor{textcolor}%
\pgftext[x=4.455423in,y=0.451851in,,top]{\color{textcolor}{\rmfamily\fontsize{12.000000}{14.400000}\selectfont\catcode`\^=\active\def^{\ifmmode\sp\else\^{}\fi}\catcode`\%=\active\def%{\%}$\mathdefault{0.75}$}}%
\end{pgfscope}%
\begin{pgfscope}%
\pgfpathrectangle{\pgfqpoint{0.582637in}{0.549073in}}{\pgfqpoint{4.518250in}{2.310000in}}%
\pgfusepath{clip}%
\pgfsetrectcap%
\pgfsetroundjoin%
\pgfsetlinewidth{0.250937pt}%
\definecolor{currentstroke}{rgb}{0.000000,0.000000,0.000000}%
\pgfsetstrokecolor{currentstroke}%
\pgfsetstrokeopacity{0.200000}%
\pgfsetdash{}{0pt}%
\pgfpathmoveto{\pgfqpoint{4.993310in}{0.549073in}}%
\pgfpathlineto{\pgfqpoint{4.993310in}{2.859073in}}%
\pgfusepath{stroke}%
\end{pgfscope}%
\begin{pgfscope}%
\pgfsetbuttcap%
\pgfsetroundjoin%
\definecolor{currentfill}{rgb}{0.000000,0.000000,0.000000}%
\pgfsetfillcolor{currentfill}%
\pgfsetlinewidth{0.803000pt}%
\definecolor{currentstroke}{rgb}{0.000000,0.000000,0.000000}%
\pgfsetstrokecolor{currentstroke}%
\pgfsetdash{}{0pt}%
\pgfsys@defobject{currentmarker}{\pgfqpoint{0.000000in}{-0.048611in}}{\pgfqpoint{0.000000in}{0.000000in}}{%
\pgfpathmoveto{\pgfqpoint{0.000000in}{0.000000in}}%
\pgfpathlineto{\pgfqpoint{0.000000in}{-0.048611in}}%
\pgfusepath{stroke,fill}%
}%
\begin{pgfscope}%
\pgfsys@transformshift{4.993310in}{0.549073in}%
\pgfsys@useobject{currentmarker}{}%
\end{pgfscope}%
\end{pgfscope}%
\begin{pgfscope}%
\definecolor{textcolor}{rgb}{0.000000,0.000000,0.000000}%
\pgfsetstrokecolor{textcolor}%
\pgfsetfillcolor{textcolor}%
\pgftext[x=4.993310in,y=0.451851in,,top]{\color{textcolor}{\rmfamily\fontsize{12.000000}{14.400000}\selectfont\catcode`\^=\active\def^{\ifmmode\sp\else\^{}\fi}\catcode`\%=\active\def%{\%}$\mathdefault{1.00}$}}%
\end{pgfscope}%
\begin{pgfscope}%
\definecolor{textcolor}{rgb}{0.000000,0.000000,0.000000}%
\pgfsetstrokecolor{textcolor}%
\pgfsetfillcolor{textcolor}%
\pgftext[x=2.841762in,y=0.248148in,,top]{\color{textcolor}{\rmfamily\fontsize{12.000000}{14.400000}\selectfont\catcode`\^=\active\def^{\ifmmode\sp\else\^{}\fi}\catcode`\%=\active\def%{\%}spectral parameter $t$}}%
\end{pgfscope}%
\begin{pgfscope}%
\pgfpathrectangle{\pgfqpoint{0.582637in}{0.549073in}}{\pgfqpoint{4.518250in}{2.310000in}}%
\pgfusepath{clip}%
\pgfsetrectcap%
\pgfsetroundjoin%
\pgfsetlinewidth{0.250937pt}%
\definecolor{currentstroke}{rgb}{0.000000,0.000000,0.000000}%
\pgfsetstrokecolor{currentstroke}%
\pgfsetstrokeopacity{0.200000}%
\pgfsetdash{}{0pt}%
\pgfpathmoveto{\pgfqpoint{0.582637in}{0.741636in}}%
\pgfpathlineto{\pgfqpoint{5.100887in}{0.741636in}}%
\pgfusepath{stroke}%
\end{pgfscope}%
\begin{pgfscope}%
\pgfsetbuttcap%
\pgfsetroundjoin%
\definecolor{currentfill}{rgb}{0.000000,0.000000,0.000000}%
\pgfsetfillcolor{currentfill}%
\pgfsetlinewidth{0.803000pt}%
\definecolor{currentstroke}{rgb}{0.000000,0.000000,0.000000}%
\pgfsetstrokecolor{currentstroke}%
\pgfsetdash{}{0pt}%
\pgfsys@defobject{currentmarker}{\pgfqpoint{-0.048611in}{0.000000in}}{\pgfqpoint{-0.000000in}{0.000000in}}{%
\pgfpathmoveto{\pgfqpoint{-0.000000in}{0.000000in}}%
\pgfpathlineto{\pgfqpoint{-0.048611in}{0.000000in}}%
\pgfusepath{stroke,fill}%
}%
\begin{pgfscope}%
\pgfsys@transformshift{0.582637in}{0.741636in}%
\pgfsys@useobject{currentmarker}{}%
\end{pgfscope}%
\end{pgfscope}%
\begin{pgfscope}%
\definecolor{textcolor}{rgb}{0.000000,0.000000,0.000000}%
\pgfsetstrokecolor{textcolor}%
\pgfsetfillcolor{textcolor}%
\pgftext[x=0.403819in, y=0.683766in, left, base]{\color{textcolor}{\rmfamily\fontsize{12.000000}{14.400000}\selectfont\catcode`\^=\active\def^{\ifmmode\sp\else\^{}\fi}\catcode`\%=\active\def%{\%}$\mathdefault{0}$}}%
\end{pgfscope}%
\begin{pgfscope}%
\pgfpathrectangle{\pgfqpoint{0.582637in}{0.549073in}}{\pgfqpoint{4.518250in}{2.310000in}}%
\pgfusepath{clip}%
\pgfsetrectcap%
\pgfsetroundjoin%
\pgfsetlinewidth{0.250937pt}%
\definecolor{currentstroke}{rgb}{0.000000,0.000000,0.000000}%
\pgfsetstrokecolor{currentstroke}%
\pgfsetstrokeopacity{0.200000}%
\pgfsetdash{}{0pt}%
\pgfpathmoveto{\pgfqpoint{0.582637in}{1.274814in}}%
\pgfpathlineto{\pgfqpoint{5.100887in}{1.274814in}}%
\pgfusepath{stroke}%
\end{pgfscope}%
\begin{pgfscope}%
\pgfsetbuttcap%
\pgfsetroundjoin%
\definecolor{currentfill}{rgb}{0.000000,0.000000,0.000000}%
\pgfsetfillcolor{currentfill}%
\pgfsetlinewidth{0.803000pt}%
\definecolor{currentstroke}{rgb}{0.000000,0.000000,0.000000}%
\pgfsetstrokecolor{currentstroke}%
\pgfsetdash{}{0pt}%
\pgfsys@defobject{currentmarker}{\pgfqpoint{-0.048611in}{0.000000in}}{\pgfqpoint{-0.000000in}{0.000000in}}{%
\pgfpathmoveto{\pgfqpoint{-0.000000in}{0.000000in}}%
\pgfpathlineto{\pgfqpoint{-0.048611in}{0.000000in}}%
\pgfusepath{stroke,fill}%
}%
\begin{pgfscope}%
\pgfsys@transformshift{0.582637in}{1.274814in}%
\pgfsys@useobject{currentmarker}{}%
\end{pgfscope}%
\end{pgfscope}%
\begin{pgfscope}%
\definecolor{textcolor}{rgb}{0.000000,0.000000,0.000000}%
\pgfsetstrokecolor{textcolor}%
\pgfsetfillcolor{textcolor}%
\pgftext[x=0.322222in, y=1.216944in, left, base]{\color{textcolor}{\rmfamily\fontsize{12.000000}{14.400000}\selectfont\catcode`\^=\active\def^{\ifmmode\sp\else\^{}\fi}\catcode`\%=\active\def%{\%}$\mathdefault{20}$}}%
\end{pgfscope}%
\begin{pgfscope}%
\pgfpathrectangle{\pgfqpoint{0.582637in}{0.549073in}}{\pgfqpoint{4.518250in}{2.310000in}}%
\pgfusepath{clip}%
\pgfsetrectcap%
\pgfsetroundjoin%
\pgfsetlinewidth{0.250937pt}%
\definecolor{currentstroke}{rgb}{0.000000,0.000000,0.000000}%
\pgfsetstrokecolor{currentstroke}%
\pgfsetstrokeopacity{0.200000}%
\pgfsetdash{}{0pt}%
\pgfpathmoveto{\pgfqpoint{0.582637in}{1.807991in}}%
\pgfpathlineto{\pgfqpoint{5.100887in}{1.807991in}}%
\pgfusepath{stroke}%
\end{pgfscope}%
\begin{pgfscope}%
\pgfsetbuttcap%
\pgfsetroundjoin%
\definecolor{currentfill}{rgb}{0.000000,0.000000,0.000000}%
\pgfsetfillcolor{currentfill}%
\pgfsetlinewidth{0.803000pt}%
\definecolor{currentstroke}{rgb}{0.000000,0.000000,0.000000}%
\pgfsetstrokecolor{currentstroke}%
\pgfsetdash{}{0pt}%
\pgfsys@defobject{currentmarker}{\pgfqpoint{-0.048611in}{0.000000in}}{\pgfqpoint{-0.000000in}{0.000000in}}{%
\pgfpathmoveto{\pgfqpoint{-0.000000in}{0.000000in}}%
\pgfpathlineto{\pgfqpoint{-0.048611in}{0.000000in}}%
\pgfusepath{stroke,fill}%
}%
\begin{pgfscope}%
\pgfsys@transformshift{0.582637in}{1.807991in}%
\pgfsys@useobject{currentmarker}{}%
\end{pgfscope}%
\end{pgfscope}%
\begin{pgfscope}%
\definecolor{textcolor}{rgb}{0.000000,0.000000,0.000000}%
\pgfsetstrokecolor{textcolor}%
\pgfsetfillcolor{textcolor}%
\pgftext[x=0.322222in, y=1.750121in, left, base]{\color{textcolor}{\rmfamily\fontsize{12.000000}{14.400000}\selectfont\catcode`\^=\active\def^{\ifmmode\sp\else\^{}\fi}\catcode`\%=\active\def%{\%}$\mathdefault{40}$}}%
\end{pgfscope}%
\begin{pgfscope}%
\pgfpathrectangle{\pgfqpoint{0.582637in}{0.549073in}}{\pgfqpoint{4.518250in}{2.310000in}}%
\pgfusepath{clip}%
\pgfsetrectcap%
\pgfsetroundjoin%
\pgfsetlinewidth{0.250937pt}%
\definecolor{currentstroke}{rgb}{0.000000,0.000000,0.000000}%
\pgfsetstrokecolor{currentstroke}%
\pgfsetstrokeopacity{0.200000}%
\pgfsetdash{}{0pt}%
\pgfpathmoveto{\pgfqpoint{0.582637in}{2.341168in}}%
\pgfpathlineto{\pgfqpoint{5.100887in}{2.341168in}}%
\pgfusepath{stroke}%
\end{pgfscope}%
\begin{pgfscope}%
\pgfsetbuttcap%
\pgfsetroundjoin%
\definecolor{currentfill}{rgb}{0.000000,0.000000,0.000000}%
\pgfsetfillcolor{currentfill}%
\pgfsetlinewidth{0.803000pt}%
\definecolor{currentstroke}{rgb}{0.000000,0.000000,0.000000}%
\pgfsetstrokecolor{currentstroke}%
\pgfsetdash{}{0pt}%
\pgfsys@defobject{currentmarker}{\pgfqpoint{-0.048611in}{0.000000in}}{\pgfqpoint{-0.000000in}{0.000000in}}{%
\pgfpathmoveto{\pgfqpoint{-0.000000in}{0.000000in}}%
\pgfpathlineto{\pgfqpoint{-0.048611in}{0.000000in}}%
\pgfusepath{stroke,fill}%
}%
\begin{pgfscope}%
\pgfsys@transformshift{0.582637in}{2.341168in}%
\pgfsys@useobject{currentmarker}{}%
\end{pgfscope}%
\end{pgfscope}%
\begin{pgfscope}%
\definecolor{textcolor}{rgb}{0.000000,0.000000,0.000000}%
\pgfsetstrokecolor{textcolor}%
\pgfsetfillcolor{textcolor}%
\pgftext[x=0.322222in, y=2.283298in, left, base]{\color{textcolor}{\rmfamily\fontsize{12.000000}{14.400000}\selectfont\catcode`\^=\active\def^{\ifmmode\sp\else\^{}\fi}\catcode`\%=\active\def%{\%}$\mathdefault{60}$}}%
\end{pgfscope}%
\begin{pgfscope}%
\definecolor{textcolor}{rgb}{0.000000,0.000000,0.000000}%
\pgfsetstrokecolor{textcolor}%
\pgfsetfillcolor{textcolor}%
\pgftext[x=0.266667in,y=1.704073in,,bottom,rotate=90.000000]{\color{textcolor}{\rmfamily\fontsize{12.000000}{14.400000}\selectfont\catcode`\^=\active\def^{\ifmmode\sp\else\^{}\fi}\catcode`\%=\active\def%{\%}smoothed spectral density $\phi_{\sigma}(t)$}}%
\end{pgfscope}%
\begin{pgfscope}%
\pgfpathrectangle{\pgfqpoint{0.582637in}{0.549073in}}{\pgfqpoint{4.518250in}{2.310000in}}%
\pgfusepath{clip}%
\pgfsetrectcap%
\pgfsetroundjoin%
\pgfsetlinewidth{1.505625pt}%
\definecolor{currentstroke}{rgb}{0.392157,0.560784,1.000000}%
\pgfsetstrokecolor{currentstroke}%
\pgfsetdash{}{0pt}%
\pgfpathmoveto{\pgfqpoint{0.690215in}{0.744295in}}%
\pgfpathlineto{\pgfqpoint{0.716085in}{0.741665in}}%
\pgfpathlineto{\pgfqpoint{0.819566in}{0.742003in}}%
\pgfpathlineto{\pgfqpoint{0.828190in}{0.744382in}}%
\pgfpathlineto{\pgfqpoint{0.836813in}{0.749506in}}%
\pgfpathlineto{\pgfqpoint{0.845436in}{0.751083in}}%
\pgfpathlineto{\pgfqpoint{0.854060in}{0.748321in}}%
\pgfpathlineto{\pgfqpoint{0.871307in}{0.746462in}}%
\pgfpathlineto{\pgfqpoint{0.879930in}{0.743339in}}%
\pgfpathlineto{\pgfqpoint{0.888554in}{0.741862in}}%
\pgfpathlineto{\pgfqpoint{0.940294in}{0.741636in}}%
\pgfpathlineto{\pgfqpoint{0.966165in}{0.741837in}}%
\pgfpathlineto{\pgfqpoint{0.974788in}{0.743528in}}%
\pgfpathlineto{\pgfqpoint{0.992035in}{0.754052in}}%
\pgfpathlineto{\pgfqpoint{1.000658in}{0.757161in}}%
\pgfpathlineto{\pgfqpoint{1.009282in}{0.758247in}}%
\pgfpathlineto{\pgfqpoint{1.017905in}{0.754800in}}%
\pgfpathlineto{\pgfqpoint{1.035152in}{0.745438in}}%
\pgfpathlineto{\pgfqpoint{1.043776in}{0.742478in}}%
\pgfpathlineto{\pgfqpoint{1.061022in}{0.741639in}}%
\pgfpathlineto{\pgfqpoint{1.121386in}{0.742239in}}%
\pgfpathlineto{\pgfqpoint{1.138633in}{0.746709in}}%
\pgfpathlineto{\pgfqpoint{1.155880in}{0.753722in}}%
\pgfpathlineto{\pgfqpoint{1.164504in}{0.753408in}}%
\pgfpathlineto{\pgfqpoint{1.173127in}{0.749907in}}%
\pgfpathlineto{\pgfqpoint{1.181751in}{0.748017in}}%
\pgfpathlineto{\pgfqpoint{1.190374in}{0.753773in}}%
\pgfpathlineto{\pgfqpoint{1.198997in}{0.757424in}}%
\pgfpathlineto{\pgfqpoint{1.216244in}{0.743116in}}%
\pgfpathlineto{\pgfqpoint{1.224868in}{0.741737in}}%
\pgfpathlineto{\pgfqpoint{1.311102in}{0.741664in}}%
\pgfpathlineto{\pgfqpoint{1.319726in}{0.742286in}}%
\pgfpathlineto{\pgfqpoint{1.328349in}{0.747642in}}%
\pgfpathlineto{\pgfqpoint{1.345596in}{0.779055in}}%
\pgfpathlineto{\pgfqpoint{1.354219in}{0.778279in}}%
\pgfpathlineto{\pgfqpoint{1.362843in}{0.771420in}}%
\pgfpathlineto{\pgfqpoint{1.380090in}{0.748274in}}%
\pgfpathlineto{\pgfqpoint{1.388713in}{0.742643in}}%
\pgfpathlineto{\pgfqpoint{1.405960in}{0.741638in}}%
\pgfpathlineto{\pgfqpoint{1.466324in}{0.742227in}}%
\pgfpathlineto{\pgfqpoint{1.474947in}{0.745674in}}%
\pgfpathlineto{\pgfqpoint{1.483571in}{0.754501in}}%
\pgfpathlineto{\pgfqpoint{1.500818in}{0.775939in}}%
\pgfpathlineto{\pgfqpoint{1.509441in}{0.779241in}}%
\pgfpathlineto{\pgfqpoint{1.518065in}{0.769883in}}%
\pgfpathlineto{\pgfqpoint{1.526688in}{0.755614in}}%
\pgfpathlineto{\pgfqpoint{1.535311in}{0.745308in}}%
\pgfpathlineto{\pgfqpoint{1.543935in}{0.742048in}}%
\pgfpathlineto{\pgfqpoint{1.569805in}{0.741636in}}%
\pgfpathlineto{\pgfqpoint{1.604299in}{0.741860in}}%
\pgfpathlineto{\pgfqpoint{1.612922in}{0.744162in}}%
\pgfpathlineto{\pgfqpoint{1.621546in}{0.752105in}}%
\pgfpathlineto{\pgfqpoint{1.630169in}{0.757533in}}%
\pgfpathlineto{\pgfqpoint{1.647416in}{0.743440in}}%
\pgfpathlineto{\pgfqpoint{1.656040in}{0.741771in}}%
\pgfpathlineto{\pgfqpoint{1.673286in}{0.741880in}}%
\pgfpathlineto{\pgfqpoint{1.681910in}{0.745008in}}%
\pgfpathlineto{\pgfqpoint{1.699157in}{0.773327in}}%
\pgfpathlineto{\pgfqpoint{1.707780in}{0.763189in}}%
\pgfpathlineto{\pgfqpoint{1.716404in}{0.747007in}}%
\pgfpathlineto{\pgfqpoint{1.725027in}{0.742127in}}%
\pgfpathlineto{\pgfqpoint{1.742274in}{0.741644in}}%
\pgfpathlineto{\pgfqpoint{1.750897in}{0.741893in}}%
\pgfpathlineto{\pgfqpoint{1.759521in}{0.745160in}}%
\pgfpathlineto{\pgfqpoint{1.768144in}{0.759759in}}%
\pgfpathlineto{\pgfqpoint{1.776768in}{0.779108in}}%
\pgfpathlineto{\pgfqpoint{1.785391in}{0.780814in}}%
\pgfpathlineto{\pgfqpoint{1.794015in}{0.773194in}}%
\pgfpathlineto{\pgfqpoint{1.802638in}{0.762368in}}%
\pgfpathlineto{\pgfqpoint{1.811261in}{0.749238in}}%
\pgfpathlineto{\pgfqpoint{1.819885in}{0.743762in}}%
\pgfpathlineto{\pgfqpoint{1.828508in}{0.748674in}}%
\pgfpathlineto{\pgfqpoint{1.845755in}{0.777510in}}%
\pgfpathlineto{\pgfqpoint{1.854379in}{0.781807in}}%
\pgfpathlineto{\pgfqpoint{1.863002in}{0.775690in}}%
\pgfpathlineto{\pgfqpoint{1.871626in}{0.757638in}}%
\pgfpathlineto{\pgfqpoint{1.880249in}{0.744872in}}%
\pgfpathlineto{\pgfqpoint{1.888872in}{0.741897in}}%
\pgfpathlineto{\pgfqpoint{1.897496in}{0.741779in}}%
\pgfpathlineto{\pgfqpoint{1.906119in}{0.743454in}}%
\pgfpathlineto{\pgfqpoint{1.914743in}{0.751036in}}%
\pgfpathlineto{\pgfqpoint{1.931990in}{0.775753in}}%
\pgfpathlineto{\pgfqpoint{1.940613in}{0.782532in}}%
\pgfpathlineto{\pgfqpoint{1.949236in}{0.774682in}}%
\pgfpathlineto{\pgfqpoint{1.957860in}{0.762939in}}%
\pgfpathlineto{\pgfqpoint{1.966483in}{0.754104in}}%
\pgfpathlineto{\pgfqpoint{1.975107in}{0.749637in}}%
\pgfpathlineto{\pgfqpoint{1.983730in}{0.750105in}}%
\pgfpathlineto{\pgfqpoint{1.992354in}{0.747731in}}%
\pgfpathlineto{\pgfqpoint{2.000977in}{0.743372in}}%
\pgfpathlineto{\pgfqpoint{2.009601in}{0.741819in}}%
\pgfpathlineto{\pgfqpoint{2.069965in}{0.741636in}}%
\pgfpathlineto{\pgfqpoint{2.087211in}{0.741766in}}%
\pgfpathlineto{\pgfqpoint{2.095835in}{0.743467in}}%
\pgfpathlineto{\pgfqpoint{2.104458in}{0.751977in}}%
\pgfpathlineto{\pgfqpoint{2.113082in}{0.770521in}}%
\pgfpathlineto{\pgfqpoint{2.130329in}{0.827025in}}%
\pgfpathlineto{\pgfqpoint{2.138952in}{0.823422in}}%
\pgfpathlineto{\pgfqpoint{2.147576in}{0.789475in}}%
\pgfpathlineto{\pgfqpoint{2.156199in}{0.760871in}}%
\pgfpathlineto{\pgfqpoint{2.164822in}{0.747352in}}%
\pgfpathlineto{\pgfqpoint{2.173446in}{0.742791in}}%
\pgfpathlineto{\pgfqpoint{2.182069in}{0.744416in}}%
\pgfpathlineto{\pgfqpoint{2.199316in}{0.762899in}}%
\pgfpathlineto{\pgfqpoint{2.207940in}{0.754838in}}%
\pgfpathlineto{\pgfqpoint{2.216563in}{0.744640in}}%
\pgfpathlineto{\pgfqpoint{2.225186in}{0.741888in}}%
\pgfpathlineto{\pgfqpoint{2.242433in}{0.742565in}}%
\pgfpathlineto{\pgfqpoint{2.251057in}{0.747800in}}%
\pgfpathlineto{\pgfqpoint{2.259680in}{0.765082in}}%
\pgfpathlineto{\pgfqpoint{2.268304in}{0.802922in}}%
\pgfpathlineto{\pgfqpoint{2.276927in}{0.849022in}}%
\pgfpathlineto{\pgfqpoint{2.285551in}{0.872715in}}%
\pgfpathlineto{\pgfqpoint{2.294174in}{0.860731in}}%
\pgfpathlineto{\pgfqpoint{2.311421in}{0.769667in}}%
\pgfpathlineto{\pgfqpoint{2.320044in}{0.747071in}}%
\pgfpathlineto{\pgfqpoint{2.328668in}{0.742141in}}%
\pgfpathlineto{\pgfqpoint{2.354538in}{0.741636in}}%
\pgfpathlineto{\pgfqpoint{2.406279in}{0.742380in}}%
\pgfpathlineto{\pgfqpoint{2.414902in}{0.746042in}}%
\pgfpathlineto{\pgfqpoint{2.432149in}{0.761489in}}%
\pgfpathlineto{\pgfqpoint{2.440772in}{0.763479in}}%
\pgfpathlineto{\pgfqpoint{2.449396in}{0.757562in}}%
\pgfpathlineto{\pgfqpoint{2.458019in}{0.757545in}}%
\pgfpathlineto{\pgfqpoint{2.466643in}{0.772236in}}%
\pgfpathlineto{\pgfqpoint{2.475266in}{0.780258in}}%
\pgfpathlineto{\pgfqpoint{2.483890in}{0.780166in}}%
\pgfpathlineto{\pgfqpoint{2.492513in}{0.770987in}}%
\pgfpathlineto{\pgfqpoint{2.501136in}{0.751847in}}%
\pgfpathlineto{\pgfqpoint{2.509760in}{0.742985in}}%
\pgfpathlineto{\pgfqpoint{2.518383in}{0.741702in}}%
\pgfpathlineto{\pgfqpoint{2.535630in}{0.741711in}}%
\pgfpathlineto{\pgfqpoint{2.544254in}{0.743117in}}%
\pgfpathlineto{\pgfqpoint{2.552877in}{0.752368in}}%
\pgfpathlineto{\pgfqpoint{2.561501in}{0.770148in}}%
\pgfpathlineto{\pgfqpoint{2.570124in}{0.769394in}}%
\pgfpathlineto{\pgfqpoint{2.578747in}{0.751543in}}%
\pgfpathlineto{\pgfqpoint{2.587371in}{0.743115in}}%
\pgfpathlineto{\pgfqpoint{2.595994in}{0.744329in}}%
\pgfpathlineto{\pgfqpoint{2.604618in}{0.757576in}}%
\pgfpathlineto{\pgfqpoint{2.613241in}{0.789785in}}%
\pgfpathlineto{\pgfqpoint{2.630488in}{0.875431in}}%
\pgfpathlineto{\pgfqpoint{2.639111in}{0.877378in}}%
\pgfpathlineto{\pgfqpoint{2.647735in}{0.829994in}}%
\pgfpathlineto{\pgfqpoint{2.656358in}{0.775256in}}%
\pgfpathlineto{\pgfqpoint{2.664982in}{0.748171in}}%
\pgfpathlineto{\pgfqpoint{2.673605in}{0.742201in}}%
\pgfpathlineto{\pgfqpoint{2.682229in}{0.741746in}}%
\pgfpathlineto{\pgfqpoint{2.690852in}{0.743323in}}%
\pgfpathlineto{\pgfqpoint{2.699476in}{0.753443in}}%
\pgfpathlineto{\pgfqpoint{2.708099in}{0.774176in}}%
\pgfpathlineto{\pgfqpoint{2.716722in}{0.785670in}}%
\pgfpathlineto{\pgfqpoint{2.725346in}{0.782463in}}%
\pgfpathlineto{\pgfqpoint{2.742593in}{0.747077in}}%
\pgfpathlineto{\pgfqpoint{2.751216in}{0.742379in}}%
\pgfpathlineto{\pgfqpoint{2.759840in}{0.744900in}}%
\pgfpathlineto{\pgfqpoint{2.768463in}{0.758384in}}%
\pgfpathlineto{\pgfqpoint{2.777086in}{0.779565in}}%
\pgfpathlineto{\pgfqpoint{2.785710in}{0.788429in}}%
\pgfpathlineto{\pgfqpoint{2.794333in}{0.778104in}}%
\pgfpathlineto{\pgfqpoint{2.802957in}{0.756862in}}%
\pgfpathlineto{\pgfqpoint{2.811580in}{0.744400in}}%
\pgfpathlineto{\pgfqpoint{2.820204in}{0.741835in}}%
\pgfpathlineto{\pgfqpoint{2.871944in}{0.741724in}}%
\pgfpathlineto{\pgfqpoint{2.880568in}{0.743134in}}%
\pgfpathlineto{\pgfqpoint{2.889191in}{0.751671in}}%
\pgfpathlineto{\pgfqpoint{2.897815in}{0.768993in}}%
\pgfpathlineto{\pgfqpoint{2.906438in}{0.777502in}}%
\pgfpathlineto{\pgfqpoint{2.915061in}{0.777734in}}%
\pgfpathlineto{\pgfqpoint{2.923685in}{0.772982in}}%
\pgfpathlineto{\pgfqpoint{2.932308in}{0.756027in}}%
\pgfpathlineto{\pgfqpoint{2.940932in}{0.744417in}}%
\pgfpathlineto{\pgfqpoint{2.949555in}{0.742699in}}%
\pgfpathlineto{\pgfqpoint{2.958179in}{0.749417in}}%
\pgfpathlineto{\pgfqpoint{2.966802in}{0.768782in}}%
\pgfpathlineto{\pgfqpoint{2.975426in}{0.784617in}}%
\pgfpathlineto{\pgfqpoint{2.984049in}{0.784943in}}%
\pgfpathlineto{\pgfqpoint{2.992672in}{0.769767in}}%
\pgfpathlineto{\pgfqpoint{3.001296in}{0.750030in}}%
\pgfpathlineto{\pgfqpoint{3.009919in}{0.742606in}}%
\pgfpathlineto{\pgfqpoint{3.018543in}{0.741768in}}%
\pgfpathlineto{\pgfqpoint{3.027166in}{0.743215in}}%
\pgfpathlineto{\pgfqpoint{3.035790in}{0.753032in}}%
\pgfpathlineto{\pgfqpoint{3.044413in}{0.780978in}}%
\pgfpathlineto{\pgfqpoint{3.061660in}{0.870652in}}%
\pgfpathlineto{\pgfqpoint{3.070283in}{0.882698in}}%
\pgfpathlineto{\pgfqpoint{3.078907in}{0.841249in}}%
\pgfpathlineto{\pgfqpoint{3.087530in}{0.784161in}}%
\pgfpathlineto{\pgfqpoint{3.096154in}{0.751361in}}%
\pgfpathlineto{\pgfqpoint{3.104777in}{0.742799in}}%
\pgfpathlineto{\pgfqpoint{3.113400in}{0.744065in}}%
\pgfpathlineto{\pgfqpoint{3.122024in}{0.755780in}}%
\pgfpathlineto{\pgfqpoint{3.130647in}{0.772370in}}%
\pgfpathlineto{\pgfqpoint{3.139271in}{0.766108in}}%
\pgfpathlineto{\pgfqpoint{3.147894in}{0.748776in}}%
\pgfpathlineto{\pgfqpoint{3.156518in}{0.742400in}}%
\pgfpathlineto{\pgfqpoint{3.173765in}{0.741638in}}%
\pgfpathlineto{\pgfqpoint{3.182388in}{0.741717in}}%
\pgfpathlineto{\pgfqpoint{3.191011in}{0.743216in}}%
\pgfpathlineto{\pgfqpoint{3.199635in}{0.753210in}}%
\pgfpathlineto{\pgfqpoint{3.208258in}{0.775841in}}%
\pgfpathlineto{\pgfqpoint{3.216882in}{0.790930in}}%
\pgfpathlineto{\pgfqpoint{3.225505in}{0.785060in}}%
\pgfpathlineto{\pgfqpoint{3.234129in}{0.766711in}}%
\pgfpathlineto{\pgfqpoint{3.242752in}{0.754918in}}%
\pgfpathlineto{\pgfqpoint{3.251375in}{0.755399in}}%
\pgfpathlineto{\pgfqpoint{3.259999in}{0.759439in}}%
\pgfpathlineto{\pgfqpoint{3.268622in}{0.756170in}}%
\pgfpathlineto{\pgfqpoint{3.294493in}{0.742426in}}%
\pgfpathlineto{\pgfqpoint{3.311740in}{0.741638in}}%
\pgfpathlineto{\pgfqpoint{3.363480in}{0.741875in}}%
\pgfpathlineto{\pgfqpoint{3.372104in}{0.744029in}}%
\pgfpathlineto{\pgfqpoint{3.380727in}{0.753043in}}%
\pgfpathlineto{\pgfqpoint{3.389350in}{0.775731in}}%
\pgfpathlineto{\pgfqpoint{3.406597in}{0.853933in}}%
\pgfpathlineto{\pgfqpoint{3.415221in}{0.872528in}}%
\pgfpathlineto{\pgfqpoint{3.423844in}{0.851423in}}%
\pgfpathlineto{\pgfqpoint{3.432468in}{0.802492in}}%
\pgfpathlineto{\pgfqpoint{3.441091in}{0.761769in}}%
\pgfpathlineto{\pgfqpoint{3.449715in}{0.745262in}}%
\pgfpathlineto{\pgfqpoint{3.458338in}{0.741955in}}%
\pgfpathlineto{\pgfqpoint{3.475585in}{0.742122in}}%
\pgfpathlineto{\pgfqpoint{3.484208in}{0.746255in}}%
\pgfpathlineto{\pgfqpoint{3.492832in}{0.757743in}}%
\pgfpathlineto{\pgfqpoint{3.501455in}{0.762219in}}%
\pgfpathlineto{\pgfqpoint{3.510079in}{0.751276in}}%
\pgfpathlineto{\pgfqpoint{3.518702in}{0.743322in}}%
\pgfpathlineto{\pgfqpoint{3.527325in}{0.742316in}}%
\pgfpathlineto{\pgfqpoint{3.535949in}{0.746095in}}%
\pgfpathlineto{\pgfqpoint{3.544572in}{0.760022in}}%
\pgfpathlineto{\pgfqpoint{3.553196in}{0.792346in}}%
\pgfpathlineto{\pgfqpoint{3.561819in}{0.831974in}}%
\pgfpathlineto{\pgfqpoint{3.570443in}{0.836560in}}%
\pgfpathlineto{\pgfqpoint{3.587690in}{0.760661in}}%
\pgfpathlineto{\pgfqpoint{3.596313in}{0.744706in}}%
\pgfpathlineto{\pgfqpoint{3.604936in}{0.741851in}}%
\pgfpathlineto{\pgfqpoint{3.656677in}{0.741636in}}%
\pgfpathlineto{\pgfqpoint{3.691171in}{0.742308in}}%
\pgfpathlineto{\pgfqpoint{3.699794in}{0.745730in}}%
\pgfpathlineto{\pgfqpoint{3.708418in}{0.752112in}}%
\pgfpathlineto{\pgfqpoint{3.717041in}{0.754837in}}%
\pgfpathlineto{\pgfqpoint{3.725665in}{0.751786in}}%
\pgfpathlineto{\pgfqpoint{3.734288in}{0.752744in}}%
\pgfpathlineto{\pgfqpoint{3.760158in}{0.780183in}}%
\pgfpathlineto{\pgfqpoint{3.768782in}{0.775099in}}%
\pgfpathlineto{\pgfqpoint{3.786029in}{0.748746in}}%
\pgfpathlineto{\pgfqpoint{3.794652in}{0.742731in}}%
\pgfpathlineto{\pgfqpoint{3.803275in}{0.741732in}}%
\pgfpathlineto{\pgfqpoint{3.811899in}{0.742295in}}%
\pgfpathlineto{\pgfqpoint{3.820522in}{0.747168in}}%
\pgfpathlineto{\pgfqpoint{3.837769in}{0.776513in}}%
\pgfpathlineto{\pgfqpoint{3.846393in}{0.781495in}}%
\pgfpathlineto{\pgfqpoint{3.855016in}{0.777237in}}%
\pgfpathlineto{\pgfqpoint{3.863640in}{0.760206in}}%
\pgfpathlineto{\pgfqpoint{3.872263in}{0.745906in}}%
\pgfpathlineto{\pgfqpoint{3.880886in}{0.742689in}}%
\pgfpathlineto{\pgfqpoint{3.889510in}{0.747071in}}%
\pgfpathlineto{\pgfqpoint{3.898133in}{0.760306in}}%
\pgfpathlineto{\pgfqpoint{3.906757in}{0.777675in}}%
\pgfpathlineto{\pgfqpoint{3.915380in}{0.789903in}}%
\pgfpathlineto{\pgfqpoint{3.924004in}{0.778244in}}%
\pgfpathlineto{\pgfqpoint{3.932627in}{0.753866in}}%
\pgfpathlineto{\pgfqpoint{3.941250in}{0.743250in}}%
\pgfpathlineto{\pgfqpoint{3.949874in}{0.741717in}}%
\pgfpathlineto{\pgfqpoint{3.975744in}{0.742438in}}%
\pgfpathlineto{\pgfqpoint{3.984368in}{0.748993in}}%
\pgfpathlineto{\pgfqpoint{3.992991in}{0.766403in}}%
\pgfpathlineto{\pgfqpoint{4.001615in}{0.772212in}}%
\pgfpathlineto{\pgfqpoint{4.010238in}{0.755479in}}%
\pgfpathlineto{\pgfqpoint{4.018861in}{0.743935in}}%
\pgfpathlineto{\pgfqpoint{4.027485in}{0.741778in}}%
\pgfpathlineto{\pgfqpoint{4.036108in}{0.741722in}}%
\pgfpathlineto{\pgfqpoint{4.044732in}{0.742928in}}%
\pgfpathlineto{\pgfqpoint{4.053355in}{0.749031in}}%
\pgfpathlineto{\pgfqpoint{4.061979in}{0.757144in}}%
\pgfpathlineto{\pgfqpoint{4.070602in}{0.753552in}}%
\pgfpathlineto{\pgfqpoint{4.079225in}{0.744991in}}%
\pgfpathlineto{\pgfqpoint{4.087849in}{0.741983in}}%
\pgfpathlineto{\pgfqpoint{4.122343in}{0.741636in}}%
\pgfpathlineto{\pgfqpoint{4.148213in}{0.741853in}}%
\pgfpathlineto{\pgfqpoint{4.156836in}{0.743652in}}%
\pgfpathlineto{\pgfqpoint{4.165460in}{0.749966in}}%
\pgfpathlineto{\pgfqpoint{4.191330in}{0.776362in}}%
\pgfpathlineto{\pgfqpoint{4.199954in}{0.775220in}}%
\pgfpathlineto{\pgfqpoint{4.208577in}{0.764843in}}%
\pgfpathlineto{\pgfqpoint{4.217200in}{0.752529in}}%
\pgfpathlineto{\pgfqpoint{4.225824in}{0.744194in}}%
\pgfpathlineto{\pgfqpoint{4.234447in}{0.741881in}}%
\pgfpathlineto{\pgfqpoint{4.277565in}{0.741636in}}%
\pgfpathlineto{\pgfqpoint{4.312058in}{0.742204in}}%
\pgfpathlineto{\pgfqpoint{4.320682in}{0.746294in}}%
\pgfpathlineto{\pgfqpoint{4.329305in}{0.758179in}}%
\pgfpathlineto{\pgfqpoint{4.337929in}{0.773755in}}%
\pgfpathlineto{\pgfqpoint{4.346552in}{0.786056in}}%
\pgfpathlineto{\pgfqpoint{4.355175in}{0.781437in}}%
\pgfpathlineto{\pgfqpoint{4.363799in}{0.759447in}}%
\pgfpathlineto{\pgfqpoint{4.372422in}{0.745055in}}%
\pgfpathlineto{\pgfqpoint{4.381046in}{0.741898in}}%
\pgfpathlineto{\pgfqpoint{4.424163in}{0.741636in}}%
\pgfpathlineto{\pgfqpoint{4.475904in}{0.742445in}}%
\pgfpathlineto{\pgfqpoint{4.484527in}{0.747295in}}%
\pgfpathlineto{\pgfqpoint{4.493150in}{0.756273in}}%
\pgfpathlineto{\pgfqpoint{4.501774in}{0.756282in}}%
\pgfpathlineto{\pgfqpoint{4.510397in}{0.749078in}}%
\pgfpathlineto{\pgfqpoint{4.519021in}{0.746700in}}%
\pgfpathlineto{\pgfqpoint{4.536268in}{0.751920in}}%
\pgfpathlineto{\pgfqpoint{4.544891in}{0.752187in}}%
\pgfpathlineto{\pgfqpoint{4.553515in}{0.750462in}}%
\pgfpathlineto{\pgfqpoint{4.570761in}{0.743533in}}%
\pgfpathlineto{\pgfqpoint{4.579385in}{0.742005in}}%
\pgfpathlineto{\pgfqpoint{4.613879in}{0.741636in}}%
\pgfpathlineto{\pgfqpoint{4.656996in}{0.742401in}}%
\pgfpathlineto{\pgfqpoint{4.665619in}{0.744265in}}%
\pgfpathlineto{\pgfqpoint{4.674243in}{0.747453in}}%
\pgfpathlineto{\pgfqpoint{4.691490in}{0.760519in}}%
\pgfpathlineto{\pgfqpoint{4.700113in}{0.761059in}}%
\pgfpathlineto{\pgfqpoint{4.708736in}{0.755375in}}%
\pgfpathlineto{\pgfqpoint{4.717360in}{0.746710in}}%
\pgfpathlineto{\pgfqpoint{4.725983in}{0.742440in}}%
\pgfpathlineto{\pgfqpoint{4.743230in}{0.741638in}}%
\pgfpathlineto{\pgfqpoint{4.812218in}{0.741930in}}%
\pgfpathlineto{\pgfqpoint{4.820841in}{0.743202in}}%
\pgfpathlineto{\pgfqpoint{4.829465in}{0.745680in}}%
\pgfpathlineto{\pgfqpoint{4.846711in}{0.754465in}}%
\pgfpathlineto{\pgfqpoint{4.855335in}{0.751199in}}%
\pgfpathlineto{\pgfqpoint{4.863958in}{0.744553in}}%
\pgfpathlineto{\pgfqpoint{4.872582in}{0.741983in}}%
\pgfpathlineto{\pgfqpoint{4.907075in}{0.741636in}}%
\pgfpathlineto{\pgfqpoint{4.976063in}{0.741993in}}%
\pgfpathlineto{\pgfqpoint{4.993310in}{0.744295in}}%
\pgfpathlineto{\pgfqpoint{4.993310in}{0.744295in}}%
\pgfusepath{stroke}%
\end{pgfscope}%
\begin{pgfscope}%
\pgfpathrectangle{\pgfqpoint{0.582637in}{0.549073in}}{\pgfqpoint{4.518250in}{2.310000in}}%
\pgfusepath{clip}%
\pgfsetrectcap%
\pgfsetroundjoin%
\pgfsetlinewidth{1.505625pt}%
\definecolor{currentstroke}{rgb}{0.862745,0.149020,0.498039}%
\pgfsetstrokecolor{currentstroke}%
\pgfsetdash{}{0pt}%
\pgfpathmoveto{\pgfqpoint{0.690215in}{0.744295in}}%
\pgfpathlineto{\pgfqpoint{0.716085in}{0.741665in}}%
\pgfpathlineto{\pgfqpoint{0.741955in}{0.741636in}}%
\pgfpathlineto{\pgfqpoint{0.750579in}{0.823983in}}%
\pgfpathlineto{\pgfqpoint{0.759202in}{0.887541in}}%
\pgfpathlineto{\pgfqpoint{0.767826in}{2.666573in}}%
\pgfpathlineto{\pgfqpoint{0.776449in}{0.882025in}}%
\pgfpathlineto{\pgfqpoint{0.785072in}{0.741636in}}%
\pgfpathlineto{\pgfqpoint{0.819566in}{0.742003in}}%
\pgfpathlineto{\pgfqpoint{0.828190in}{0.744382in}}%
\pgfpathlineto{\pgfqpoint{0.836813in}{0.749506in}}%
\pgfpathlineto{\pgfqpoint{0.845436in}{0.751083in}}%
\pgfpathlineto{\pgfqpoint{0.854060in}{0.748321in}}%
\pgfpathlineto{\pgfqpoint{0.871307in}{0.746462in}}%
\pgfpathlineto{\pgfqpoint{0.879930in}{0.743339in}}%
\pgfpathlineto{\pgfqpoint{0.888554in}{0.741862in}}%
\pgfpathlineto{\pgfqpoint{0.940294in}{0.741636in}}%
\pgfpathlineto{\pgfqpoint{0.966165in}{0.741837in}}%
\pgfpathlineto{\pgfqpoint{0.974788in}{0.743528in}}%
\pgfpathlineto{\pgfqpoint{0.992035in}{0.754052in}}%
\pgfpathlineto{\pgfqpoint{1.000658in}{0.757161in}}%
\pgfpathlineto{\pgfqpoint{1.009282in}{0.758247in}}%
\pgfpathlineto{\pgfqpoint{1.017905in}{0.754800in}}%
\pgfpathlineto{\pgfqpoint{1.035152in}{0.745438in}}%
\pgfpathlineto{\pgfqpoint{1.043776in}{0.742478in}}%
\pgfpathlineto{\pgfqpoint{1.061022in}{0.741639in}}%
\pgfpathlineto{\pgfqpoint{1.121386in}{0.742239in}}%
\pgfpathlineto{\pgfqpoint{1.138633in}{0.746709in}}%
\pgfpathlineto{\pgfqpoint{1.155880in}{0.753722in}}%
\pgfpathlineto{\pgfqpoint{1.164504in}{0.753408in}}%
\pgfpathlineto{\pgfqpoint{1.173127in}{0.749907in}}%
\pgfpathlineto{\pgfqpoint{1.181751in}{0.748017in}}%
\pgfpathlineto{\pgfqpoint{1.190374in}{0.753773in}}%
\pgfpathlineto{\pgfqpoint{1.198997in}{0.757424in}}%
\pgfpathlineto{\pgfqpoint{1.216244in}{0.743116in}}%
\pgfpathlineto{\pgfqpoint{1.224868in}{0.741737in}}%
\pgfpathlineto{\pgfqpoint{1.250738in}{0.741636in}}%
\pgfpathlineto{\pgfqpoint{1.259361in}{0.763903in}}%
\pgfpathlineto{\pgfqpoint{1.267985in}{0.758419in}}%
\pgfpathlineto{\pgfqpoint{1.276608in}{0.833778in}}%
\pgfpathlineto{\pgfqpoint{1.285232in}{0.741633in}}%
\pgfpathlineto{\pgfqpoint{1.319726in}{0.742286in}}%
\pgfpathlineto{\pgfqpoint{1.328349in}{0.747642in}}%
\pgfpathlineto{\pgfqpoint{1.345596in}{0.779055in}}%
\pgfpathlineto{\pgfqpoint{1.354219in}{0.778279in}}%
\pgfpathlineto{\pgfqpoint{1.362843in}{0.771420in}}%
\pgfpathlineto{\pgfqpoint{1.380090in}{0.748274in}}%
\pgfpathlineto{\pgfqpoint{1.388713in}{0.742643in}}%
\pgfpathlineto{\pgfqpoint{1.405960in}{0.741638in}}%
\pgfpathlineto{\pgfqpoint{1.466324in}{0.742227in}}%
\pgfpathlineto{\pgfqpoint{1.474947in}{0.745674in}}%
\pgfpathlineto{\pgfqpoint{1.483571in}{0.754501in}}%
\pgfpathlineto{\pgfqpoint{1.500818in}{0.775939in}}%
\pgfpathlineto{\pgfqpoint{1.509441in}{0.779241in}}%
\pgfpathlineto{\pgfqpoint{1.518065in}{0.769883in}}%
\pgfpathlineto{\pgfqpoint{1.526688in}{0.755614in}}%
\pgfpathlineto{\pgfqpoint{1.535311in}{0.745308in}}%
\pgfpathlineto{\pgfqpoint{1.543935in}{0.742048in}}%
\pgfpathlineto{\pgfqpoint{1.569805in}{0.741636in}}%
\pgfpathlineto{\pgfqpoint{1.604299in}{0.741860in}}%
\pgfpathlineto{\pgfqpoint{1.612922in}{0.744162in}}%
\pgfpathlineto{\pgfqpoint{1.621546in}{0.752105in}}%
\pgfpathlineto{\pgfqpoint{1.630169in}{0.757533in}}%
\pgfpathlineto{\pgfqpoint{1.647416in}{0.743440in}}%
\pgfpathlineto{\pgfqpoint{1.656040in}{0.741771in}}%
\pgfpathlineto{\pgfqpoint{1.673286in}{0.741880in}}%
\pgfpathlineto{\pgfqpoint{1.681910in}{0.745008in}}%
\pgfpathlineto{\pgfqpoint{1.699157in}{0.773327in}}%
\pgfpathlineto{\pgfqpoint{1.707780in}{0.763189in}}%
\pgfpathlineto{\pgfqpoint{1.716404in}{0.747007in}}%
\pgfpathlineto{\pgfqpoint{1.725027in}{0.742127in}}%
\pgfpathlineto{\pgfqpoint{1.742274in}{0.741644in}}%
\pgfpathlineto{\pgfqpoint{1.750897in}{0.741893in}}%
\pgfpathlineto{\pgfqpoint{1.759521in}{0.745160in}}%
\pgfpathlineto{\pgfqpoint{1.768144in}{0.759759in}}%
\pgfpathlineto{\pgfqpoint{1.776768in}{0.779108in}}%
\pgfpathlineto{\pgfqpoint{1.785391in}{0.780814in}}%
\pgfpathlineto{\pgfqpoint{1.794015in}{0.773194in}}%
\pgfpathlineto{\pgfqpoint{1.802638in}{0.762368in}}%
\pgfpathlineto{\pgfqpoint{1.811261in}{0.749238in}}%
\pgfpathlineto{\pgfqpoint{1.819885in}{0.743762in}}%
\pgfpathlineto{\pgfqpoint{1.828508in}{0.748674in}}%
\pgfpathlineto{\pgfqpoint{1.845755in}{0.777510in}}%
\pgfpathlineto{\pgfqpoint{1.854379in}{0.781807in}}%
\pgfpathlineto{\pgfqpoint{1.863002in}{0.775690in}}%
\pgfpathlineto{\pgfqpoint{1.871626in}{0.757638in}}%
\pgfpathlineto{\pgfqpoint{1.880249in}{0.744872in}}%
\pgfpathlineto{\pgfqpoint{1.888872in}{0.741897in}}%
\pgfpathlineto{\pgfqpoint{1.897496in}{0.741779in}}%
\pgfpathlineto{\pgfqpoint{1.906119in}{0.743454in}}%
\pgfpathlineto{\pgfqpoint{1.914743in}{0.751036in}}%
\pgfpathlineto{\pgfqpoint{1.931990in}{0.775753in}}%
\pgfpathlineto{\pgfqpoint{1.940613in}{0.782532in}}%
\pgfpathlineto{\pgfqpoint{1.949236in}{0.774682in}}%
\pgfpathlineto{\pgfqpoint{1.957860in}{0.762939in}}%
\pgfpathlineto{\pgfqpoint{1.966483in}{0.754103in}}%
\pgfpathlineto{\pgfqpoint{1.975107in}{0.749637in}}%
\pgfpathlineto{\pgfqpoint{1.983730in}{0.750105in}}%
\pgfpathlineto{\pgfqpoint{1.992354in}{0.747731in}}%
\pgfpathlineto{\pgfqpoint{2.000977in}{0.743372in}}%
\pgfpathlineto{\pgfqpoint{2.009601in}{0.741819in}}%
\pgfpathlineto{\pgfqpoint{2.044094in}{0.741717in}}%
\pgfpathlineto{\pgfqpoint{2.052718in}{0.757513in}}%
\pgfpathlineto{\pgfqpoint{2.061341in}{0.741636in}}%
\pgfpathlineto{\pgfqpoint{2.087211in}{0.741766in}}%
\pgfpathlineto{\pgfqpoint{2.095835in}{0.743467in}}%
\pgfpathlineto{\pgfqpoint{2.104458in}{0.751977in}}%
\pgfpathlineto{\pgfqpoint{2.113082in}{0.770521in}}%
\pgfpathlineto{\pgfqpoint{2.130329in}{0.827025in}}%
\pgfpathlineto{\pgfqpoint{2.138952in}{0.823422in}}%
\pgfpathlineto{\pgfqpoint{2.147576in}{0.789475in}}%
\pgfpathlineto{\pgfqpoint{2.156199in}{0.760871in}}%
\pgfpathlineto{\pgfqpoint{2.164822in}{0.747352in}}%
\pgfpathlineto{\pgfqpoint{2.173446in}{0.742791in}}%
\pgfpathlineto{\pgfqpoint{2.182069in}{0.744416in}}%
\pgfpathlineto{\pgfqpoint{2.199316in}{0.762899in}}%
\pgfpathlineto{\pgfqpoint{2.207940in}{0.754838in}}%
\pgfpathlineto{\pgfqpoint{2.216563in}{0.744640in}}%
\pgfpathlineto{\pgfqpoint{2.225186in}{0.741888in}}%
\pgfpathlineto{\pgfqpoint{2.242433in}{0.742565in}}%
\pgfpathlineto{\pgfqpoint{2.251057in}{0.747799in}}%
\pgfpathlineto{\pgfqpoint{2.259680in}{0.765075in}}%
\pgfpathlineto{\pgfqpoint{2.268304in}{0.802848in}}%
\pgfpathlineto{\pgfqpoint{2.276927in}{0.848123in}}%
\pgfpathlineto{\pgfqpoint{2.285551in}{0.869438in}}%
\pgfpathlineto{\pgfqpoint{2.294174in}{0.860116in}}%
\pgfpathlineto{\pgfqpoint{2.311421in}{0.769657in}}%
\pgfpathlineto{\pgfqpoint{2.320044in}{0.747070in}}%
\pgfpathlineto{\pgfqpoint{2.328668in}{0.742141in}}%
\pgfpathlineto{\pgfqpoint{2.354538in}{0.741636in}}%
\pgfpathlineto{\pgfqpoint{2.406279in}{0.742380in}}%
\pgfpathlineto{\pgfqpoint{2.414902in}{0.746042in}}%
\pgfpathlineto{\pgfqpoint{2.432149in}{0.761489in}}%
\pgfpathlineto{\pgfqpoint{2.440772in}{0.763479in}}%
\pgfpathlineto{\pgfqpoint{2.449396in}{0.757562in}}%
\pgfpathlineto{\pgfqpoint{2.458019in}{0.757545in}}%
\pgfpathlineto{\pgfqpoint{2.466643in}{0.772236in}}%
\pgfpathlineto{\pgfqpoint{2.475266in}{0.780258in}}%
\pgfpathlineto{\pgfqpoint{2.483890in}{0.780166in}}%
\pgfpathlineto{\pgfqpoint{2.492513in}{0.770987in}}%
\pgfpathlineto{\pgfqpoint{2.501136in}{0.751847in}}%
\pgfpathlineto{\pgfqpoint{2.509760in}{0.742985in}}%
\pgfpathlineto{\pgfqpoint{2.518383in}{0.741702in}}%
\pgfpathlineto{\pgfqpoint{2.535630in}{0.741711in}}%
\pgfpathlineto{\pgfqpoint{2.544254in}{0.743117in}}%
\pgfpathlineto{\pgfqpoint{2.552877in}{0.752368in}}%
\pgfpathlineto{\pgfqpoint{2.561501in}{0.770148in}}%
\pgfpathlineto{\pgfqpoint{2.570124in}{0.769394in}}%
\pgfpathlineto{\pgfqpoint{2.578747in}{0.751543in}}%
\pgfpathlineto{\pgfqpoint{2.587371in}{0.743114in}}%
\pgfpathlineto{\pgfqpoint{2.595994in}{0.744329in}}%
\pgfpathlineto{\pgfqpoint{2.604618in}{0.757569in}}%
\pgfpathlineto{\pgfqpoint{2.613241in}{0.789741in}}%
\pgfpathlineto{\pgfqpoint{2.621865in}{0.834176in}}%
\pgfpathlineto{\pgfqpoint{2.630488in}{0.871810in}}%
\pgfpathlineto{\pgfqpoint{2.639111in}{0.875776in}}%
\pgfpathlineto{\pgfqpoint{2.647735in}{0.829871in}}%
\pgfpathlineto{\pgfqpoint{2.656358in}{0.775248in}}%
\pgfpathlineto{\pgfqpoint{2.664982in}{0.748170in}}%
\pgfpathlineto{\pgfqpoint{2.673605in}{0.742201in}}%
\pgfpathlineto{\pgfqpoint{2.682229in}{0.741746in}}%
\pgfpathlineto{\pgfqpoint{2.690852in}{0.743323in}}%
\pgfpathlineto{\pgfqpoint{2.699476in}{0.753443in}}%
\pgfpathlineto{\pgfqpoint{2.708099in}{0.774176in}}%
\pgfpathlineto{\pgfqpoint{2.716722in}{0.785670in}}%
\pgfpathlineto{\pgfqpoint{2.725346in}{0.782463in}}%
\pgfpathlineto{\pgfqpoint{2.742593in}{0.747077in}}%
\pgfpathlineto{\pgfqpoint{2.751216in}{0.742379in}}%
\pgfpathlineto{\pgfqpoint{2.759840in}{0.744900in}}%
\pgfpathlineto{\pgfqpoint{2.768463in}{0.758384in}}%
\pgfpathlineto{\pgfqpoint{2.777086in}{0.779565in}}%
\pgfpathlineto{\pgfqpoint{2.785710in}{0.788429in}}%
\pgfpathlineto{\pgfqpoint{2.794333in}{0.778104in}}%
\pgfpathlineto{\pgfqpoint{2.802957in}{0.756862in}}%
\pgfpathlineto{\pgfqpoint{2.811580in}{0.744400in}}%
\pgfpathlineto{\pgfqpoint{2.820204in}{0.741835in}}%
\pgfpathlineto{\pgfqpoint{2.871944in}{0.741724in}}%
\pgfpathlineto{\pgfqpoint{2.880568in}{0.743134in}}%
\pgfpathlineto{\pgfqpoint{2.889191in}{0.751671in}}%
\pgfpathlineto{\pgfqpoint{2.897815in}{0.768993in}}%
\pgfpathlineto{\pgfqpoint{2.906438in}{0.777502in}}%
\pgfpathlineto{\pgfqpoint{2.915061in}{0.777734in}}%
\pgfpathlineto{\pgfqpoint{2.923685in}{0.772982in}}%
\pgfpathlineto{\pgfqpoint{2.932308in}{0.756026in}}%
\pgfpathlineto{\pgfqpoint{2.940932in}{0.744417in}}%
\pgfpathlineto{\pgfqpoint{2.949555in}{0.742699in}}%
\pgfpathlineto{\pgfqpoint{2.958179in}{0.749417in}}%
\pgfpathlineto{\pgfqpoint{2.966802in}{0.768782in}}%
\pgfpathlineto{\pgfqpoint{2.975426in}{0.784617in}}%
\pgfpathlineto{\pgfqpoint{2.984049in}{0.784943in}}%
\pgfpathlineto{\pgfqpoint{2.992672in}{0.769767in}}%
\pgfpathlineto{\pgfqpoint{3.001296in}{0.750030in}}%
\pgfpathlineto{\pgfqpoint{3.009919in}{0.742606in}}%
\pgfpathlineto{\pgfqpoint{3.018543in}{0.741768in}}%
\pgfpathlineto{\pgfqpoint{3.027166in}{0.743215in}}%
\pgfpathlineto{\pgfqpoint{3.035790in}{0.753028in}}%
\pgfpathlineto{\pgfqpoint{3.044413in}{0.780948in}}%
\pgfpathlineto{\pgfqpoint{3.061660in}{0.867283in}}%
\pgfpathlineto{\pgfqpoint{3.070283in}{0.880132in}}%
\pgfpathlineto{\pgfqpoint{3.078907in}{0.840959in}}%
\pgfpathlineto{\pgfqpoint{3.087530in}{0.784150in}}%
\pgfpathlineto{\pgfqpoint{3.096154in}{0.751354in}}%
\pgfpathlineto{\pgfqpoint{3.104777in}{0.742799in}}%
\pgfpathlineto{\pgfqpoint{3.113400in}{0.744064in}}%
\pgfpathlineto{\pgfqpoint{3.122024in}{0.755780in}}%
\pgfpathlineto{\pgfqpoint{3.130647in}{0.772370in}}%
\pgfpathlineto{\pgfqpoint{3.139271in}{0.766108in}}%
\pgfpathlineto{\pgfqpoint{3.147894in}{0.748776in}}%
\pgfpathlineto{\pgfqpoint{3.156518in}{0.742400in}}%
\pgfpathlineto{\pgfqpoint{3.173765in}{0.741638in}}%
\pgfpathlineto{\pgfqpoint{3.182388in}{0.741717in}}%
\pgfpathlineto{\pgfqpoint{3.191011in}{0.743216in}}%
\pgfpathlineto{\pgfqpoint{3.199635in}{0.753210in}}%
\pgfpathlineto{\pgfqpoint{3.208258in}{0.775841in}}%
\pgfpathlineto{\pgfqpoint{3.216882in}{0.790930in}}%
\pgfpathlineto{\pgfqpoint{3.225505in}{0.785060in}}%
\pgfpathlineto{\pgfqpoint{3.234129in}{0.766711in}}%
\pgfpathlineto{\pgfqpoint{3.242752in}{0.754918in}}%
\pgfpathlineto{\pgfqpoint{3.251375in}{0.755398in}}%
\pgfpathlineto{\pgfqpoint{3.259999in}{0.759439in}}%
\pgfpathlineto{\pgfqpoint{3.268622in}{0.756170in}}%
\pgfpathlineto{\pgfqpoint{3.294493in}{0.742426in}}%
\pgfpathlineto{\pgfqpoint{3.311740in}{0.741638in}}%
\pgfpathlineto{\pgfqpoint{3.363480in}{0.741875in}}%
\pgfpathlineto{\pgfqpoint{3.372104in}{0.744028in}}%
\pgfpathlineto{\pgfqpoint{3.380727in}{0.753042in}}%
\pgfpathlineto{\pgfqpoint{3.389350in}{0.775722in}}%
\pgfpathlineto{\pgfqpoint{3.406597in}{0.852806in}}%
\pgfpathlineto{\pgfqpoint{3.415221in}{0.871157in}}%
\pgfpathlineto{\pgfqpoint{3.423844in}{0.851304in}}%
\pgfpathlineto{\pgfqpoint{3.432468in}{0.802462in}}%
\pgfpathlineto{\pgfqpoint{3.441091in}{0.761765in}}%
\pgfpathlineto{\pgfqpoint{3.449715in}{0.745262in}}%
\pgfpathlineto{\pgfqpoint{3.458338in}{0.741955in}}%
\pgfpathlineto{\pgfqpoint{3.475585in}{0.742122in}}%
\pgfpathlineto{\pgfqpoint{3.484208in}{0.746255in}}%
\pgfpathlineto{\pgfqpoint{3.492832in}{0.757743in}}%
\pgfpathlineto{\pgfqpoint{3.501455in}{0.762219in}}%
\pgfpathlineto{\pgfqpoint{3.510079in}{0.751276in}}%
\pgfpathlineto{\pgfqpoint{3.518702in}{0.743322in}}%
\pgfpathlineto{\pgfqpoint{3.527325in}{0.742316in}}%
\pgfpathlineto{\pgfqpoint{3.535949in}{0.746095in}}%
\pgfpathlineto{\pgfqpoint{3.544572in}{0.760022in}}%
\pgfpathlineto{\pgfqpoint{3.553196in}{0.792346in}}%
\pgfpathlineto{\pgfqpoint{3.561819in}{0.831974in}}%
\pgfpathlineto{\pgfqpoint{3.570443in}{0.836560in}}%
\pgfpathlineto{\pgfqpoint{3.587690in}{0.760661in}}%
\pgfpathlineto{\pgfqpoint{3.596313in}{0.744706in}}%
\pgfpathlineto{\pgfqpoint{3.604936in}{0.741851in}}%
\pgfpathlineto{\pgfqpoint{3.630807in}{0.741636in}}%
\pgfpathlineto{\pgfqpoint{3.639430in}{0.745003in}}%
\pgfpathlineto{\pgfqpoint{3.648054in}{0.758370in}}%
\pgfpathlineto{\pgfqpoint{3.656677in}{0.741636in}}%
\pgfpathlineto{\pgfqpoint{3.691171in}{0.742308in}}%
\pgfpathlineto{\pgfqpoint{3.699794in}{0.745730in}}%
\pgfpathlineto{\pgfqpoint{3.708418in}{0.752112in}}%
\pgfpathlineto{\pgfqpoint{3.717041in}{0.754837in}}%
\pgfpathlineto{\pgfqpoint{3.725665in}{0.751786in}}%
\pgfpathlineto{\pgfqpoint{3.734288in}{0.752744in}}%
\pgfpathlineto{\pgfqpoint{3.760158in}{0.780183in}}%
\pgfpathlineto{\pgfqpoint{3.768782in}{0.775099in}}%
\pgfpathlineto{\pgfqpoint{3.786029in}{0.748746in}}%
\pgfpathlineto{\pgfqpoint{3.794652in}{0.742731in}}%
\pgfpathlineto{\pgfqpoint{3.803275in}{0.741732in}}%
\pgfpathlineto{\pgfqpoint{3.811899in}{0.742295in}}%
\pgfpathlineto{\pgfqpoint{3.820522in}{0.747168in}}%
\pgfpathlineto{\pgfqpoint{3.837769in}{0.776513in}}%
\pgfpathlineto{\pgfqpoint{3.846393in}{0.781495in}}%
\pgfpathlineto{\pgfqpoint{3.855016in}{0.777237in}}%
\pgfpathlineto{\pgfqpoint{3.863640in}{0.760206in}}%
\pgfpathlineto{\pgfqpoint{3.872263in}{0.745906in}}%
\pgfpathlineto{\pgfqpoint{3.880886in}{0.742689in}}%
\pgfpathlineto{\pgfqpoint{3.889510in}{0.747071in}}%
\pgfpathlineto{\pgfqpoint{3.898133in}{0.760306in}}%
\pgfpathlineto{\pgfqpoint{3.906757in}{0.777675in}}%
\pgfpathlineto{\pgfqpoint{3.915380in}{0.789903in}}%
\pgfpathlineto{\pgfqpoint{3.924004in}{0.778244in}}%
\pgfpathlineto{\pgfqpoint{3.932627in}{0.753866in}}%
\pgfpathlineto{\pgfqpoint{3.941250in}{0.743250in}}%
\pgfpathlineto{\pgfqpoint{3.949874in}{0.741717in}}%
\pgfpathlineto{\pgfqpoint{3.975744in}{0.742438in}}%
\pgfpathlineto{\pgfqpoint{3.984368in}{0.748993in}}%
\pgfpathlineto{\pgfqpoint{3.992991in}{0.766403in}}%
\pgfpathlineto{\pgfqpoint{4.001615in}{0.772212in}}%
\pgfpathlineto{\pgfqpoint{4.010238in}{0.755479in}}%
\pgfpathlineto{\pgfqpoint{4.018861in}{0.743935in}}%
\pgfpathlineto{\pgfqpoint{4.027485in}{0.741778in}}%
\pgfpathlineto{\pgfqpoint{4.036108in}{0.741722in}}%
\pgfpathlineto{\pgfqpoint{4.044732in}{0.742928in}}%
\pgfpathlineto{\pgfqpoint{4.053355in}{0.749031in}}%
\pgfpathlineto{\pgfqpoint{4.061979in}{0.757144in}}%
\pgfpathlineto{\pgfqpoint{4.070602in}{0.753552in}}%
\pgfpathlineto{\pgfqpoint{4.079225in}{0.744991in}}%
\pgfpathlineto{\pgfqpoint{4.087849in}{0.741983in}}%
\pgfpathlineto{\pgfqpoint{4.122343in}{0.741636in}}%
\pgfpathlineto{\pgfqpoint{4.148213in}{0.741853in}}%
\pgfpathlineto{\pgfqpoint{4.156836in}{0.743652in}}%
\pgfpathlineto{\pgfqpoint{4.165460in}{0.749966in}}%
\pgfpathlineto{\pgfqpoint{4.191330in}{0.776362in}}%
\pgfpathlineto{\pgfqpoint{4.199954in}{0.775220in}}%
\pgfpathlineto{\pgfqpoint{4.208577in}{0.764843in}}%
\pgfpathlineto{\pgfqpoint{4.217200in}{0.752529in}}%
\pgfpathlineto{\pgfqpoint{4.225824in}{0.744194in}}%
\pgfpathlineto{\pgfqpoint{4.234447in}{0.741881in}}%
\pgfpathlineto{\pgfqpoint{4.277565in}{0.741630in}}%
\pgfpathlineto{\pgfqpoint{4.312058in}{0.742204in}}%
\pgfpathlineto{\pgfqpoint{4.320682in}{0.746294in}}%
\pgfpathlineto{\pgfqpoint{4.329305in}{0.758179in}}%
\pgfpathlineto{\pgfqpoint{4.337929in}{0.773755in}}%
\pgfpathlineto{\pgfqpoint{4.346552in}{0.786056in}}%
\pgfpathlineto{\pgfqpoint{4.355175in}{0.781437in}}%
\pgfpathlineto{\pgfqpoint{4.363799in}{0.759447in}}%
\pgfpathlineto{\pgfqpoint{4.372422in}{0.745055in}}%
\pgfpathlineto{\pgfqpoint{4.381046in}{0.741898in}}%
\pgfpathlineto{\pgfqpoint{4.415540in}{0.741573in}}%
\pgfpathlineto{\pgfqpoint{4.424163in}{0.871688in}}%
\pgfpathlineto{\pgfqpoint{4.432786in}{0.812509in}}%
\pgfpathlineto{\pgfqpoint{4.441410in}{0.741636in}}%
\pgfpathlineto{\pgfqpoint{4.475904in}{0.742445in}}%
\pgfpathlineto{\pgfqpoint{4.484527in}{0.747295in}}%
\pgfpathlineto{\pgfqpoint{4.493150in}{0.756273in}}%
\pgfpathlineto{\pgfqpoint{4.501774in}{0.756282in}}%
\pgfpathlineto{\pgfqpoint{4.510397in}{0.749078in}}%
\pgfpathlineto{\pgfqpoint{4.519021in}{0.746700in}}%
\pgfpathlineto{\pgfqpoint{4.536268in}{0.751920in}}%
\pgfpathlineto{\pgfqpoint{4.544891in}{0.752187in}}%
\pgfpathlineto{\pgfqpoint{4.553515in}{0.750462in}}%
\pgfpathlineto{\pgfqpoint{4.570761in}{0.743533in}}%
\pgfpathlineto{\pgfqpoint{4.579385in}{0.742005in}}%
\pgfpathlineto{\pgfqpoint{4.613879in}{0.741637in}}%
\pgfpathlineto{\pgfqpoint{4.656996in}{0.742401in}}%
\pgfpathlineto{\pgfqpoint{4.665619in}{0.744265in}}%
\pgfpathlineto{\pgfqpoint{4.674243in}{0.747453in}}%
\pgfpathlineto{\pgfqpoint{4.691490in}{0.760519in}}%
\pgfpathlineto{\pgfqpoint{4.700113in}{0.761059in}}%
\pgfpathlineto{\pgfqpoint{4.708736in}{0.755375in}}%
\pgfpathlineto{\pgfqpoint{4.717360in}{0.746710in}}%
\pgfpathlineto{\pgfqpoint{4.725983in}{0.742440in}}%
\pgfpathlineto{\pgfqpoint{4.743230in}{0.741638in}}%
\pgfpathlineto{\pgfqpoint{4.760477in}{0.741636in}}%
\pgfpathlineto{\pgfqpoint{4.769100in}{0.822794in}}%
\pgfpathlineto{\pgfqpoint{4.777724in}{0.741636in}}%
\pgfpathlineto{\pgfqpoint{4.812218in}{0.741930in}}%
\pgfpathlineto{\pgfqpoint{4.820841in}{0.743202in}}%
\pgfpathlineto{\pgfqpoint{4.829465in}{0.745680in}}%
\pgfpathlineto{\pgfqpoint{4.846711in}{0.754465in}}%
\pgfpathlineto{\pgfqpoint{4.855335in}{0.751199in}}%
\pgfpathlineto{\pgfqpoint{4.863958in}{0.744553in}}%
\pgfpathlineto{\pgfqpoint{4.872582in}{0.741983in}}%
\pgfpathlineto{\pgfqpoint{4.907075in}{0.741636in}}%
\pgfpathlineto{\pgfqpoint{4.915699in}{0.888250in}}%
\pgfpathlineto{\pgfqpoint{4.924322in}{0.865372in}}%
\pgfpathlineto{\pgfqpoint{4.932946in}{0.824122in}}%
\pgfpathlineto{\pgfqpoint{4.941569in}{0.741636in}}%
\pgfpathlineto{\pgfqpoint{4.976063in}{0.741993in}}%
\pgfpathlineto{\pgfqpoint{4.993310in}{0.744295in}}%
\pgfpathlineto{\pgfqpoint{4.993310in}{0.744295in}}%
\pgfusepath{stroke}%
\end{pgfscope}%
\begin{pgfscope}%
\pgfpathrectangle{\pgfqpoint{0.582637in}{0.549073in}}{\pgfqpoint{4.518250in}{2.310000in}}%
\pgfusepath{clip}%
\pgfsetrectcap%
\pgfsetroundjoin%
\pgfsetlinewidth{1.505625pt}%
\definecolor{currentstroke}{rgb}{1.000000,0.690196,0.000000}%
\pgfsetstrokecolor{currentstroke}%
\pgfsetdash{}{0pt}%
\pgfpathmoveto{\pgfqpoint{0.690215in}{0.744295in}}%
\pgfpathlineto{\pgfqpoint{0.716085in}{0.741665in}}%
\pgfpathlineto{\pgfqpoint{0.819566in}{0.742003in}}%
\pgfpathlineto{\pgfqpoint{0.828190in}{0.744382in}}%
\pgfpathlineto{\pgfqpoint{0.836813in}{0.749506in}}%
\pgfpathlineto{\pgfqpoint{0.845436in}{0.751083in}}%
\pgfpathlineto{\pgfqpoint{0.854060in}{0.748321in}}%
\pgfpathlineto{\pgfqpoint{0.871307in}{0.746462in}}%
\pgfpathlineto{\pgfqpoint{0.879930in}{0.743339in}}%
\pgfpathlineto{\pgfqpoint{0.888554in}{0.741862in}}%
\pgfpathlineto{\pgfqpoint{0.940294in}{0.741636in}}%
\pgfpathlineto{\pgfqpoint{0.966165in}{0.741837in}}%
\pgfpathlineto{\pgfqpoint{0.974788in}{0.743528in}}%
\pgfpathlineto{\pgfqpoint{0.992035in}{0.754052in}}%
\pgfpathlineto{\pgfqpoint{1.000658in}{0.757161in}}%
\pgfpathlineto{\pgfqpoint{1.009282in}{0.758247in}}%
\pgfpathlineto{\pgfqpoint{1.017905in}{0.754800in}}%
\pgfpathlineto{\pgfqpoint{1.035152in}{0.745438in}}%
\pgfpathlineto{\pgfqpoint{1.043776in}{0.742478in}}%
\pgfpathlineto{\pgfqpoint{1.061022in}{0.741639in}}%
\pgfpathlineto{\pgfqpoint{1.121386in}{0.742239in}}%
\pgfpathlineto{\pgfqpoint{1.138633in}{0.746709in}}%
\pgfpathlineto{\pgfqpoint{1.155880in}{0.753722in}}%
\pgfpathlineto{\pgfqpoint{1.164504in}{0.753408in}}%
\pgfpathlineto{\pgfqpoint{1.173127in}{0.749907in}}%
\pgfpathlineto{\pgfqpoint{1.181751in}{0.748017in}}%
\pgfpathlineto{\pgfqpoint{1.190374in}{0.753773in}}%
\pgfpathlineto{\pgfqpoint{1.198997in}{0.757424in}}%
\pgfpathlineto{\pgfqpoint{1.216244in}{0.743116in}}%
\pgfpathlineto{\pgfqpoint{1.224868in}{0.741737in}}%
\pgfpathlineto{\pgfqpoint{1.311102in}{0.741664in}}%
\pgfpathlineto{\pgfqpoint{1.319726in}{0.742286in}}%
\pgfpathlineto{\pgfqpoint{1.328349in}{0.747642in}}%
\pgfpathlineto{\pgfqpoint{1.345596in}{0.779055in}}%
\pgfpathlineto{\pgfqpoint{1.354219in}{0.778279in}}%
\pgfpathlineto{\pgfqpoint{1.362843in}{0.771420in}}%
\pgfpathlineto{\pgfqpoint{1.380090in}{0.748274in}}%
\pgfpathlineto{\pgfqpoint{1.388713in}{0.742643in}}%
\pgfpathlineto{\pgfqpoint{1.405960in}{0.741638in}}%
\pgfpathlineto{\pgfqpoint{1.466324in}{0.742227in}}%
\pgfpathlineto{\pgfqpoint{1.474947in}{0.745674in}}%
\pgfpathlineto{\pgfqpoint{1.483571in}{0.754501in}}%
\pgfpathlineto{\pgfqpoint{1.500818in}{0.775939in}}%
\pgfpathlineto{\pgfqpoint{1.509441in}{0.779241in}}%
\pgfpathlineto{\pgfqpoint{1.518065in}{0.769883in}}%
\pgfpathlineto{\pgfqpoint{1.526688in}{0.755614in}}%
\pgfpathlineto{\pgfqpoint{1.535311in}{0.745308in}}%
\pgfpathlineto{\pgfqpoint{1.543935in}{0.742048in}}%
\pgfpathlineto{\pgfqpoint{1.569805in}{0.741636in}}%
\pgfpathlineto{\pgfqpoint{1.604299in}{0.741860in}}%
\pgfpathlineto{\pgfqpoint{1.612922in}{0.744162in}}%
\pgfpathlineto{\pgfqpoint{1.621546in}{0.752105in}}%
\pgfpathlineto{\pgfqpoint{1.630169in}{0.757533in}}%
\pgfpathlineto{\pgfqpoint{1.647416in}{0.743440in}}%
\pgfpathlineto{\pgfqpoint{1.656040in}{0.741771in}}%
\pgfpathlineto{\pgfqpoint{1.673286in}{0.741880in}}%
\pgfpathlineto{\pgfqpoint{1.681910in}{0.745008in}}%
\pgfpathlineto{\pgfqpoint{1.699157in}{0.773327in}}%
\pgfpathlineto{\pgfqpoint{1.707780in}{0.763189in}}%
\pgfpathlineto{\pgfqpoint{1.716404in}{0.747007in}}%
\pgfpathlineto{\pgfqpoint{1.725027in}{0.742127in}}%
\pgfpathlineto{\pgfqpoint{1.742274in}{0.741644in}}%
\pgfpathlineto{\pgfqpoint{1.750897in}{0.741893in}}%
\pgfpathlineto{\pgfqpoint{1.759521in}{0.745160in}}%
\pgfpathlineto{\pgfqpoint{1.768144in}{0.759759in}}%
\pgfpathlineto{\pgfqpoint{1.776768in}{0.779108in}}%
\pgfpathlineto{\pgfqpoint{1.785391in}{0.780814in}}%
\pgfpathlineto{\pgfqpoint{1.794015in}{0.773194in}}%
\pgfpathlineto{\pgfqpoint{1.802638in}{0.762368in}}%
\pgfpathlineto{\pgfqpoint{1.811261in}{0.749238in}}%
\pgfpathlineto{\pgfqpoint{1.819885in}{0.743762in}}%
\pgfpathlineto{\pgfqpoint{1.828508in}{0.748674in}}%
\pgfpathlineto{\pgfqpoint{1.845755in}{0.777510in}}%
\pgfpathlineto{\pgfqpoint{1.854379in}{0.781807in}}%
\pgfpathlineto{\pgfqpoint{1.863002in}{0.775690in}}%
\pgfpathlineto{\pgfqpoint{1.871626in}{0.757638in}}%
\pgfpathlineto{\pgfqpoint{1.880249in}{0.744872in}}%
\pgfpathlineto{\pgfqpoint{1.888872in}{0.741897in}}%
\pgfpathlineto{\pgfqpoint{1.897496in}{0.741779in}}%
\pgfpathlineto{\pgfqpoint{1.906119in}{0.743454in}}%
\pgfpathlineto{\pgfqpoint{1.914743in}{0.751036in}}%
\pgfpathlineto{\pgfqpoint{1.931990in}{0.775753in}}%
\pgfpathlineto{\pgfqpoint{1.940613in}{0.782532in}}%
\pgfpathlineto{\pgfqpoint{1.949236in}{0.774682in}}%
\pgfpathlineto{\pgfqpoint{1.957860in}{0.762939in}}%
\pgfpathlineto{\pgfqpoint{1.966483in}{0.754103in}}%
\pgfpathlineto{\pgfqpoint{1.975107in}{0.749637in}}%
\pgfpathlineto{\pgfqpoint{1.983730in}{0.750105in}}%
\pgfpathlineto{\pgfqpoint{1.992354in}{0.747731in}}%
\pgfpathlineto{\pgfqpoint{2.000977in}{0.743372in}}%
\pgfpathlineto{\pgfqpoint{2.009601in}{0.741819in}}%
\pgfpathlineto{\pgfqpoint{2.069965in}{0.741636in}}%
\pgfpathlineto{\pgfqpoint{2.087211in}{0.741766in}}%
\pgfpathlineto{\pgfqpoint{2.095835in}{0.743467in}}%
\pgfpathlineto{\pgfqpoint{2.104458in}{0.751977in}}%
\pgfpathlineto{\pgfqpoint{2.113082in}{0.770521in}}%
\pgfpathlineto{\pgfqpoint{2.130329in}{0.827025in}}%
\pgfpathlineto{\pgfqpoint{2.138952in}{0.823422in}}%
\pgfpathlineto{\pgfqpoint{2.147576in}{0.789475in}}%
\pgfpathlineto{\pgfqpoint{2.156199in}{0.760871in}}%
\pgfpathlineto{\pgfqpoint{2.164822in}{0.747352in}}%
\pgfpathlineto{\pgfqpoint{2.173446in}{0.742791in}}%
\pgfpathlineto{\pgfqpoint{2.182069in}{0.744416in}}%
\pgfpathlineto{\pgfqpoint{2.199316in}{0.762899in}}%
\pgfpathlineto{\pgfqpoint{2.207940in}{0.754838in}}%
\pgfpathlineto{\pgfqpoint{2.216563in}{0.744640in}}%
\pgfpathlineto{\pgfqpoint{2.225186in}{0.741888in}}%
\pgfpathlineto{\pgfqpoint{2.242433in}{0.742565in}}%
\pgfpathlineto{\pgfqpoint{2.251057in}{0.747799in}}%
\pgfpathlineto{\pgfqpoint{2.259680in}{0.765075in}}%
\pgfpathlineto{\pgfqpoint{2.268304in}{0.802848in}}%
\pgfpathlineto{\pgfqpoint{2.276927in}{0.848123in}}%
\pgfpathlineto{\pgfqpoint{2.285551in}{0.869438in}}%
\pgfpathlineto{\pgfqpoint{2.294174in}{0.860116in}}%
\pgfpathlineto{\pgfqpoint{2.311421in}{0.769657in}}%
\pgfpathlineto{\pgfqpoint{2.320044in}{0.747070in}}%
\pgfpathlineto{\pgfqpoint{2.328668in}{0.742141in}}%
\pgfpathlineto{\pgfqpoint{2.354538in}{0.741636in}}%
\pgfpathlineto{\pgfqpoint{2.406279in}{0.742380in}}%
\pgfpathlineto{\pgfqpoint{2.414902in}{0.746042in}}%
\pgfpathlineto{\pgfqpoint{2.432149in}{0.761489in}}%
\pgfpathlineto{\pgfqpoint{2.440772in}{0.763479in}}%
\pgfpathlineto{\pgfqpoint{2.449396in}{0.757562in}}%
\pgfpathlineto{\pgfqpoint{2.458019in}{0.757545in}}%
\pgfpathlineto{\pgfqpoint{2.466643in}{0.772236in}}%
\pgfpathlineto{\pgfqpoint{2.475266in}{0.780258in}}%
\pgfpathlineto{\pgfqpoint{2.483890in}{0.780166in}}%
\pgfpathlineto{\pgfqpoint{2.492513in}{0.770987in}}%
\pgfpathlineto{\pgfqpoint{2.501136in}{0.751847in}}%
\pgfpathlineto{\pgfqpoint{2.509760in}{0.742985in}}%
\pgfpathlineto{\pgfqpoint{2.518383in}{0.741702in}}%
\pgfpathlineto{\pgfqpoint{2.535630in}{0.741711in}}%
\pgfpathlineto{\pgfqpoint{2.544254in}{0.743117in}}%
\pgfpathlineto{\pgfqpoint{2.552877in}{0.752368in}}%
\pgfpathlineto{\pgfqpoint{2.561501in}{0.770148in}}%
\pgfpathlineto{\pgfqpoint{2.570124in}{0.769394in}}%
\pgfpathlineto{\pgfqpoint{2.578747in}{0.751543in}}%
\pgfpathlineto{\pgfqpoint{2.587371in}{0.743114in}}%
\pgfpathlineto{\pgfqpoint{2.595994in}{0.744329in}}%
\pgfpathlineto{\pgfqpoint{2.604618in}{0.757569in}}%
\pgfpathlineto{\pgfqpoint{2.613241in}{0.789741in}}%
\pgfpathlineto{\pgfqpoint{2.621865in}{0.834176in}}%
\pgfpathlineto{\pgfqpoint{2.630488in}{0.871810in}}%
\pgfpathlineto{\pgfqpoint{2.639111in}{0.875776in}}%
\pgfpathlineto{\pgfqpoint{2.647735in}{0.829871in}}%
\pgfpathlineto{\pgfqpoint{2.656358in}{0.775248in}}%
\pgfpathlineto{\pgfqpoint{2.664982in}{0.748170in}}%
\pgfpathlineto{\pgfqpoint{2.673605in}{0.742201in}}%
\pgfpathlineto{\pgfqpoint{2.682229in}{0.741746in}}%
\pgfpathlineto{\pgfqpoint{2.690852in}{0.743323in}}%
\pgfpathlineto{\pgfqpoint{2.699476in}{0.753443in}}%
\pgfpathlineto{\pgfqpoint{2.708099in}{0.774176in}}%
\pgfpathlineto{\pgfqpoint{2.716722in}{0.785670in}}%
\pgfpathlineto{\pgfqpoint{2.725346in}{0.782463in}}%
\pgfpathlineto{\pgfqpoint{2.742593in}{0.747077in}}%
\pgfpathlineto{\pgfqpoint{2.751216in}{0.742379in}}%
\pgfpathlineto{\pgfqpoint{2.759840in}{0.744900in}}%
\pgfpathlineto{\pgfqpoint{2.768463in}{0.758384in}}%
\pgfpathlineto{\pgfqpoint{2.777086in}{0.779565in}}%
\pgfpathlineto{\pgfqpoint{2.785710in}{0.788429in}}%
\pgfpathlineto{\pgfqpoint{2.794333in}{0.778104in}}%
\pgfpathlineto{\pgfqpoint{2.802957in}{0.756862in}}%
\pgfpathlineto{\pgfqpoint{2.811580in}{0.744400in}}%
\pgfpathlineto{\pgfqpoint{2.820204in}{0.741835in}}%
\pgfpathlineto{\pgfqpoint{2.871944in}{0.741724in}}%
\pgfpathlineto{\pgfqpoint{2.880568in}{0.743134in}}%
\pgfpathlineto{\pgfqpoint{2.889191in}{0.751671in}}%
\pgfpathlineto{\pgfqpoint{2.897815in}{0.768993in}}%
\pgfpathlineto{\pgfqpoint{2.906438in}{0.777502in}}%
\pgfpathlineto{\pgfqpoint{2.915061in}{0.777734in}}%
\pgfpathlineto{\pgfqpoint{2.923685in}{0.772982in}}%
\pgfpathlineto{\pgfqpoint{2.932308in}{0.756026in}}%
\pgfpathlineto{\pgfqpoint{2.940932in}{0.744417in}}%
\pgfpathlineto{\pgfqpoint{2.949555in}{0.742699in}}%
\pgfpathlineto{\pgfqpoint{2.958179in}{0.749417in}}%
\pgfpathlineto{\pgfqpoint{2.966802in}{0.768782in}}%
\pgfpathlineto{\pgfqpoint{2.975426in}{0.784617in}}%
\pgfpathlineto{\pgfqpoint{2.984049in}{0.784943in}}%
\pgfpathlineto{\pgfqpoint{2.992672in}{0.769767in}}%
\pgfpathlineto{\pgfqpoint{3.001296in}{0.750030in}}%
\pgfpathlineto{\pgfqpoint{3.009919in}{0.742606in}}%
\pgfpathlineto{\pgfqpoint{3.018543in}{0.741768in}}%
\pgfpathlineto{\pgfqpoint{3.027166in}{0.743215in}}%
\pgfpathlineto{\pgfqpoint{3.035790in}{0.753028in}}%
\pgfpathlineto{\pgfqpoint{3.044413in}{0.780948in}}%
\pgfpathlineto{\pgfqpoint{3.061660in}{0.867283in}}%
\pgfpathlineto{\pgfqpoint{3.070283in}{0.880132in}}%
\pgfpathlineto{\pgfqpoint{3.078907in}{0.840959in}}%
\pgfpathlineto{\pgfqpoint{3.087530in}{0.784150in}}%
\pgfpathlineto{\pgfqpoint{3.096154in}{0.751354in}}%
\pgfpathlineto{\pgfqpoint{3.104777in}{0.742799in}}%
\pgfpathlineto{\pgfqpoint{3.113400in}{0.744064in}}%
\pgfpathlineto{\pgfqpoint{3.122024in}{0.755780in}}%
\pgfpathlineto{\pgfqpoint{3.130647in}{0.772370in}}%
\pgfpathlineto{\pgfqpoint{3.139271in}{0.766108in}}%
\pgfpathlineto{\pgfqpoint{3.147894in}{0.748776in}}%
\pgfpathlineto{\pgfqpoint{3.156518in}{0.742400in}}%
\pgfpathlineto{\pgfqpoint{3.173765in}{0.741638in}}%
\pgfpathlineto{\pgfqpoint{3.182388in}{0.741717in}}%
\pgfpathlineto{\pgfqpoint{3.191011in}{0.743216in}}%
\pgfpathlineto{\pgfqpoint{3.199635in}{0.753210in}}%
\pgfpathlineto{\pgfqpoint{3.208258in}{0.775841in}}%
\pgfpathlineto{\pgfqpoint{3.216882in}{0.790930in}}%
\pgfpathlineto{\pgfqpoint{3.225505in}{0.785060in}}%
\pgfpathlineto{\pgfqpoint{3.234129in}{0.766711in}}%
\pgfpathlineto{\pgfqpoint{3.242752in}{0.754918in}}%
\pgfpathlineto{\pgfqpoint{3.251375in}{0.755398in}}%
\pgfpathlineto{\pgfqpoint{3.259999in}{0.759439in}}%
\pgfpathlineto{\pgfqpoint{3.268622in}{0.756170in}}%
\pgfpathlineto{\pgfqpoint{3.294493in}{0.742426in}}%
\pgfpathlineto{\pgfqpoint{3.311740in}{0.741638in}}%
\pgfpathlineto{\pgfqpoint{3.363480in}{0.741875in}}%
\pgfpathlineto{\pgfqpoint{3.372104in}{0.744028in}}%
\pgfpathlineto{\pgfqpoint{3.380727in}{0.753042in}}%
\pgfpathlineto{\pgfqpoint{3.389350in}{0.775722in}}%
\pgfpathlineto{\pgfqpoint{3.406597in}{0.852806in}}%
\pgfpathlineto{\pgfqpoint{3.415221in}{0.871157in}}%
\pgfpathlineto{\pgfqpoint{3.423844in}{0.851304in}}%
\pgfpathlineto{\pgfqpoint{3.432468in}{0.802462in}}%
\pgfpathlineto{\pgfqpoint{3.441091in}{0.761765in}}%
\pgfpathlineto{\pgfqpoint{3.449715in}{0.745262in}}%
\pgfpathlineto{\pgfqpoint{3.458338in}{0.741955in}}%
\pgfpathlineto{\pgfqpoint{3.475585in}{0.742122in}}%
\pgfpathlineto{\pgfqpoint{3.484208in}{0.746255in}}%
\pgfpathlineto{\pgfqpoint{3.492832in}{0.757743in}}%
\pgfpathlineto{\pgfqpoint{3.501455in}{0.762219in}}%
\pgfpathlineto{\pgfqpoint{3.510079in}{0.751276in}}%
\pgfpathlineto{\pgfqpoint{3.518702in}{0.743322in}}%
\pgfpathlineto{\pgfqpoint{3.527325in}{0.742316in}}%
\pgfpathlineto{\pgfqpoint{3.535949in}{0.746095in}}%
\pgfpathlineto{\pgfqpoint{3.544572in}{0.760022in}}%
\pgfpathlineto{\pgfqpoint{3.553196in}{0.792346in}}%
\pgfpathlineto{\pgfqpoint{3.561819in}{0.831974in}}%
\pgfpathlineto{\pgfqpoint{3.570443in}{0.836560in}}%
\pgfpathlineto{\pgfqpoint{3.587690in}{0.760661in}}%
\pgfpathlineto{\pgfqpoint{3.596313in}{0.744706in}}%
\pgfpathlineto{\pgfqpoint{3.604936in}{0.741851in}}%
\pgfpathlineto{\pgfqpoint{3.656677in}{0.741636in}}%
\pgfpathlineto{\pgfqpoint{3.691171in}{0.742308in}}%
\pgfpathlineto{\pgfqpoint{3.699794in}{0.745730in}}%
\pgfpathlineto{\pgfqpoint{3.708418in}{0.752112in}}%
\pgfpathlineto{\pgfqpoint{3.717041in}{0.754837in}}%
\pgfpathlineto{\pgfqpoint{3.725665in}{0.751786in}}%
\pgfpathlineto{\pgfqpoint{3.734288in}{0.752744in}}%
\pgfpathlineto{\pgfqpoint{3.760158in}{0.780183in}}%
\pgfpathlineto{\pgfqpoint{3.768782in}{0.775099in}}%
\pgfpathlineto{\pgfqpoint{3.786029in}{0.748746in}}%
\pgfpathlineto{\pgfqpoint{3.794652in}{0.742731in}}%
\pgfpathlineto{\pgfqpoint{3.803275in}{0.741732in}}%
\pgfpathlineto{\pgfqpoint{3.811899in}{0.742295in}}%
\pgfpathlineto{\pgfqpoint{3.820522in}{0.747168in}}%
\pgfpathlineto{\pgfqpoint{3.837769in}{0.776513in}}%
\pgfpathlineto{\pgfqpoint{3.846393in}{0.781495in}}%
\pgfpathlineto{\pgfqpoint{3.855016in}{0.777237in}}%
\pgfpathlineto{\pgfqpoint{3.863640in}{0.760206in}}%
\pgfpathlineto{\pgfqpoint{3.872263in}{0.745906in}}%
\pgfpathlineto{\pgfqpoint{3.880886in}{0.742689in}}%
\pgfpathlineto{\pgfqpoint{3.889510in}{0.747071in}}%
\pgfpathlineto{\pgfqpoint{3.898133in}{0.760306in}}%
\pgfpathlineto{\pgfqpoint{3.906757in}{0.777675in}}%
\pgfpathlineto{\pgfqpoint{3.915380in}{0.789903in}}%
\pgfpathlineto{\pgfqpoint{3.924004in}{0.778244in}}%
\pgfpathlineto{\pgfqpoint{3.932627in}{0.753866in}}%
\pgfpathlineto{\pgfqpoint{3.941250in}{0.743250in}}%
\pgfpathlineto{\pgfqpoint{3.949874in}{0.741717in}}%
\pgfpathlineto{\pgfqpoint{3.975744in}{0.742438in}}%
\pgfpathlineto{\pgfqpoint{3.984368in}{0.748993in}}%
\pgfpathlineto{\pgfqpoint{3.992991in}{0.766403in}}%
\pgfpathlineto{\pgfqpoint{4.001615in}{0.772212in}}%
\pgfpathlineto{\pgfqpoint{4.010238in}{0.755479in}}%
\pgfpathlineto{\pgfqpoint{4.018861in}{0.743935in}}%
\pgfpathlineto{\pgfqpoint{4.027485in}{0.741778in}}%
\pgfpathlineto{\pgfqpoint{4.036108in}{0.741722in}}%
\pgfpathlineto{\pgfqpoint{4.044732in}{0.742928in}}%
\pgfpathlineto{\pgfqpoint{4.053355in}{0.749031in}}%
\pgfpathlineto{\pgfqpoint{4.061979in}{0.757144in}}%
\pgfpathlineto{\pgfqpoint{4.070602in}{0.753552in}}%
\pgfpathlineto{\pgfqpoint{4.079225in}{0.744991in}}%
\pgfpathlineto{\pgfqpoint{4.087849in}{0.741983in}}%
\pgfpathlineto{\pgfqpoint{4.122343in}{0.741636in}}%
\pgfpathlineto{\pgfqpoint{4.148213in}{0.741853in}}%
\pgfpathlineto{\pgfqpoint{4.156836in}{0.743652in}}%
\pgfpathlineto{\pgfqpoint{4.165460in}{0.749966in}}%
\pgfpathlineto{\pgfqpoint{4.191330in}{0.776362in}}%
\pgfpathlineto{\pgfqpoint{4.199954in}{0.775220in}}%
\pgfpathlineto{\pgfqpoint{4.208577in}{0.764843in}}%
\pgfpathlineto{\pgfqpoint{4.217200in}{0.752529in}}%
\pgfpathlineto{\pgfqpoint{4.225824in}{0.744194in}}%
\pgfpathlineto{\pgfqpoint{4.234447in}{0.741881in}}%
\pgfpathlineto{\pgfqpoint{4.277565in}{0.741636in}}%
\pgfpathlineto{\pgfqpoint{4.312058in}{0.742204in}}%
\pgfpathlineto{\pgfqpoint{4.320682in}{0.746294in}}%
\pgfpathlineto{\pgfqpoint{4.329305in}{0.758179in}}%
\pgfpathlineto{\pgfqpoint{4.337929in}{0.773755in}}%
\pgfpathlineto{\pgfqpoint{4.346552in}{0.786056in}}%
\pgfpathlineto{\pgfqpoint{4.355175in}{0.781437in}}%
\pgfpathlineto{\pgfqpoint{4.363799in}{0.759447in}}%
\pgfpathlineto{\pgfqpoint{4.372422in}{0.745055in}}%
\pgfpathlineto{\pgfqpoint{4.381046in}{0.741898in}}%
\pgfpathlineto{\pgfqpoint{4.424163in}{0.741636in}}%
\pgfpathlineto{\pgfqpoint{4.475904in}{0.742445in}}%
\pgfpathlineto{\pgfqpoint{4.484527in}{0.747295in}}%
\pgfpathlineto{\pgfqpoint{4.493150in}{0.756273in}}%
\pgfpathlineto{\pgfqpoint{4.501774in}{0.756282in}}%
\pgfpathlineto{\pgfqpoint{4.510397in}{0.749078in}}%
\pgfpathlineto{\pgfqpoint{4.519021in}{0.746700in}}%
\pgfpathlineto{\pgfqpoint{4.536268in}{0.751920in}}%
\pgfpathlineto{\pgfqpoint{4.544891in}{0.752187in}}%
\pgfpathlineto{\pgfqpoint{4.553515in}{0.750462in}}%
\pgfpathlineto{\pgfqpoint{4.570761in}{0.743533in}}%
\pgfpathlineto{\pgfqpoint{4.579385in}{0.742005in}}%
\pgfpathlineto{\pgfqpoint{4.613879in}{0.741636in}}%
\pgfpathlineto{\pgfqpoint{4.656996in}{0.742401in}}%
\pgfpathlineto{\pgfqpoint{4.665619in}{0.744265in}}%
\pgfpathlineto{\pgfqpoint{4.674243in}{0.747453in}}%
\pgfpathlineto{\pgfqpoint{4.691490in}{0.760519in}}%
\pgfpathlineto{\pgfqpoint{4.700113in}{0.761059in}}%
\pgfpathlineto{\pgfqpoint{4.708736in}{0.755375in}}%
\pgfpathlineto{\pgfqpoint{4.717360in}{0.746710in}}%
\pgfpathlineto{\pgfqpoint{4.725983in}{0.742440in}}%
\pgfpathlineto{\pgfqpoint{4.743230in}{0.741638in}}%
\pgfpathlineto{\pgfqpoint{4.812218in}{0.741930in}}%
\pgfpathlineto{\pgfqpoint{4.820841in}{0.743202in}}%
\pgfpathlineto{\pgfqpoint{4.829465in}{0.745680in}}%
\pgfpathlineto{\pgfqpoint{4.846711in}{0.754465in}}%
\pgfpathlineto{\pgfqpoint{4.855335in}{0.751199in}}%
\pgfpathlineto{\pgfqpoint{4.863958in}{0.744553in}}%
\pgfpathlineto{\pgfqpoint{4.872582in}{0.741983in}}%
\pgfpathlineto{\pgfqpoint{4.907075in}{0.741636in}}%
\pgfpathlineto{\pgfqpoint{4.976063in}{0.741993in}}%
\pgfpathlineto{\pgfqpoint{4.993310in}{0.744295in}}%
\pgfpathlineto{\pgfqpoint{4.993310in}{0.744295in}}%
\pgfusepath{stroke}%
\end{pgfscope}%
\begin{pgfscope}%
\pgfsetrectcap%
\pgfsetmiterjoin%
\pgfsetlinewidth{0.803000pt}%
\definecolor{currentstroke}{rgb}{0.000000,0.000000,0.000000}%
\pgfsetstrokecolor{currentstroke}%
\pgfsetdash{}{0pt}%
\pgfpathmoveto{\pgfqpoint{0.582637in}{0.549073in}}%
\pgfpathlineto{\pgfqpoint{0.582637in}{2.859073in}}%
\pgfusepath{stroke}%
\end{pgfscope}%
\begin{pgfscope}%
\pgfsetrectcap%
\pgfsetmiterjoin%
\pgfsetlinewidth{0.803000pt}%
\definecolor{currentstroke}{rgb}{0.000000,0.000000,0.000000}%
\pgfsetstrokecolor{currentstroke}%
\pgfsetdash{}{0pt}%
\pgfpathmoveto{\pgfqpoint{5.100887in}{0.549073in}}%
\pgfpathlineto{\pgfqpoint{5.100887in}{2.859073in}}%
\pgfusepath{stroke}%
\end{pgfscope}%
\begin{pgfscope}%
\pgfsetrectcap%
\pgfsetmiterjoin%
\pgfsetlinewidth{0.803000pt}%
\definecolor{currentstroke}{rgb}{0.000000,0.000000,0.000000}%
\pgfsetstrokecolor{currentstroke}%
\pgfsetdash{}{0pt}%
\pgfpathmoveto{\pgfqpoint{0.582637in}{0.549073in}}%
\pgfpathlineto{\pgfqpoint{5.100887in}{0.549073in}}%
\pgfusepath{stroke}%
\end{pgfscope}%
\begin{pgfscope}%
\pgfsetrectcap%
\pgfsetmiterjoin%
\pgfsetlinewidth{0.803000pt}%
\definecolor{currentstroke}{rgb}{0.000000,0.000000,0.000000}%
\pgfsetstrokecolor{currentstroke}%
\pgfsetdash{}{0pt}%
\pgfpathmoveto{\pgfqpoint{0.582637in}{2.859073in}}%
\pgfpathlineto{\pgfqpoint{5.100887in}{2.859073in}}%
\pgfusepath{stroke}%
\end{pgfscope}%
\begin{pgfscope}%
\pgfsetbuttcap%
\pgfsetmiterjoin%
\definecolor{currentfill}{rgb}{1.000000,1.000000,1.000000}%
\pgfsetfillcolor{currentfill}%
\pgfsetfillopacity{0.800000}%
\pgfsetlinewidth{1.003750pt}%
\definecolor{currentstroke}{rgb}{0.800000,0.800000,0.800000}%
\pgfsetstrokecolor{currentstroke}%
\pgfsetstrokeopacity{0.800000}%
\pgfsetdash{}{0pt}%
\pgfpathmoveto{\pgfqpoint{3.080280in}{2.028518in}}%
\pgfpathlineto{\pgfqpoint{5.017554in}{2.028518in}}%
\pgfpathlineto{\pgfqpoint{5.017554in}{2.775739in}}%
\pgfpathlineto{\pgfqpoint{3.080280in}{2.775739in}}%
\pgfpathlineto{\pgfqpoint{3.080280in}{2.028518in}}%
\pgfpathclose%
\pgfusepath{stroke,fill}%
\end{pgfscope}%
\begin{pgfscope}%
\pgfsetrectcap%
\pgfsetroundjoin%
\pgfsetlinewidth{1.505625pt}%
\definecolor{currentstroke}{rgb}{0.392157,0.560784,1.000000}%
\pgfsetstrokecolor{currentstroke}%
\pgfsetdash{}{0pt}%
\pgfpathmoveto{\pgfqpoint{3.146947in}{2.650739in}}%
\pgfpathlineto{\pgfqpoint{3.313613in}{2.650739in}}%
\pgfpathlineto{\pgfqpoint{3.480280in}{2.650739in}}%
\pgfusepath{stroke}%
\end{pgfscope}%
\begin{pgfscope}%
\definecolor{textcolor}{rgb}{0.000000,0.000000,0.000000}%
\pgfsetstrokecolor{textcolor}%
\pgfsetfillcolor{textcolor}%
\pgftext[x=3.613613in,y=2.592406in,left,base]{\color{textcolor}{\rmfamily\fontsize{12.000000}{14.400000}\selectfont\catcode`\^=\active\def^{\ifmmode\sp\else\^{}\fi}\catcode`\%=\active\def%{\%}baseline}}%
\end{pgfscope}%
\begin{pgfscope}%
\pgfsetrectcap%
\pgfsetroundjoin%
\pgfsetlinewidth{1.505625pt}%
\definecolor{currentstroke}{rgb}{0.862745,0.149020,0.498039}%
\pgfsetstrokecolor{currentstroke}%
\pgfsetdash{}{0pt}%
\pgfpathmoveto{\pgfqpoint{3.146947in}{2.418332in}}%
\pgfpathlineto{\pgfqpoint{3.313613in}{2.418332in}}%
\pgfpathlineto{\pgfqpoint{3.480280in}{2.418332in}}%
\pgfusepath{stroke}%
\end{pgfscope}%
\begin{pgfscope}%
\definecolor{textcolor}{rgb}{0.000000,0.000000,0.000000}%
\pgfsetstrokecolor{textcolor}%
\pgfsetfillcolor{textcolor}%
\pgftext[x=3.613613in,y=2.359999in,left,base]{\color{textcolor}{\rmfamily\fontsize{12.000000}{14.400000}\selectfont\catcode`\^=\active\def^{\ifmmode\sp\else\^{}\fi}\catcode`\%=\active\def%{\%}without zero-check}}%
\end{pgfscope}%
\begin{pgfscope}%
\pgfsetrectcap%
\pgfsetroundjoin%
\pgfsetlinewidth{1.505625pt}%
\definecolor{currentstroke}{rgb}{1.000000,0.690196,0.000000}%
\pgfsetstrokecolor{currentstroke}%
\pgfsetdash{}{0pt}%
\pgfpathmoveto{\pgfqpoint{3.146947in}{2.185925in}}%
\pgfpathlineto{\pgfqpoint{3.313613in}{2.185925in}}%
\pgfpathlineto{\pgfqpoint{3.480280in}{2.185925in}}%
\pgfusepath{stroke}%
\end{pgfscope}%
\begin{pgfscope}%
\definecolor{textcolor}{rgb}{0.000000,0.000000,0.000000}%
\pgfsetstrokecolor{textcolor}%
\pgfsetfillcolor{textcolor}%
\pgftext[x=3.613613in,y=2.127592in,left,base]{\color{textcolor}{\rmfamily\fontsize{12.000000}{14.400000}\selectfont\catcode`\^=\active\def^{\ifmmode\sp\else\^{}\fi}\catcode`\%=\active\def%{\%}with zero-check}}%
\end{pgfscope}%
\end{pgfpicture}%
\makeatother%
\endgroup%

    \caption{The difference the non-zero check can make when approximating a spectral density using \refalg{alg:nystrom-chebyshev-pp}. Here, we used a matrix described in \refsec{sec:results} and ran the Chebyshev-Nyström++ method with and without estimating if the matrix function is zero before taking its pseudo-inverse. We let $m=2000$, $n_{\mtx{\Omega}}=80$, $n_{\mtx{\Psi}}=0$, and $\sigma = 0.004$.}
    \label{fig:zerocheck}
\end{figure}

\subsubsection{Analysis}
\label{subsubsec:chebyshev-nystrom-analysis}

In this section, we will upper bound the output of the Chebyshev-Nyström++ method (see. \refalg{alg:nystrom-chebyshev-pp}). 
\begin{theorem}{Chebyshev-Nyström++ method}{chebyshev-nystrom}
    Let $\phi_{\sigma}$ be the smoothed spectral density of a symmetric matrix $\mtx{A} \in \mathbb{R}^{n \times n}$ whose spectrum is contained in $[-1, 1]$. Then the Chebyshev-Nyström++ estimate $\widetilde{\underline{\phi}}_{\sigma}^{(m)}$ with \emph{even} degree $m = \mathcal{O}(\sigma^{-1}(\log(\varepsilon^{-1}) + \log(\sigma^{-1})))$, \emph{even} $n_{\mtx{\Omega}} = n_{\mtx{\Psi}} = \mathcal{O}(\varepsilon^{-1}\log(\delta^{-1})^2)$, and smoothing parameter $0 < \sigma \ll 1$ satisfies with probability at least $1 - \delta$
    \begin{equation}
        \int_{-1}^{1} \left| \phi_{\sigma}(t) - \widetilde{\underline{\phi}}_{\sigma}^{(m)}(t) \right|~\mathrm{d}t \leq \varepsilon
        \label{equ:chebyshev-nystrom-error}
        %\lVert  \phi_{\sigma} - \phi_{\sigma}^{(m)} \rVert _1 &\leq \frac{2\sqrt{2}}{\sigma^2} (1 + \sigma)^{-m}.
        %\label{equ:2-chebyshev-interpolation-error}
    \end{equation}
    for all $\varepsilon \in (0, 1)$ and $\delta \in (0, 1/2)$.
\end{theorem}

\begin{proof}
    The previously derived results can be applied to
    \begin{align}
        &\int_{-1}^{1} \left| \phi_{\sigma}(t) - \widetilde{\underline{\phi}}_{\sigma}^{(m)}(t) \right|~\mathrm{d}t \notag \\
        &\leq \int_{-1}^{1} \left| \phi_{\sigma}(t) - \underline{\phi}_{\sigma}^{(m)}(t) \right|~\mathrm{d}t + \int_{-1}^{1} \left| \underline{\phi}_{\sigma}^{(m)}(t) - \widetilde{\underline{\phi}}_{\sigma}^{(m)}(t) \right|~\mathrm{d}t && \text{(triangle inequality)} \notag \\
        &\leq \int_{-1}^{1} \left| \phi_{\sigma}(t) - \underline{\phi}_{\sigma}^{(m)}(t) \right|~\mathrm{d}t + \widetilde{\varepsilon} \int_{-1}^{1} \left| \underline{\phi}_{\sigma}^{(m)}(t) \right|~\mathrm{d}t && \text{(\refthm{thm:nystrom-pp} w.p. $\geq 1 - \delta$)} \notag \\
        &\leq (1 + \widetilde{\varepsilon}) \int_{-1}^{1} \left| \phi_{\sigma}(t) - \underline{\phi}_{\sigma}^{(m)}(t) \right|~\mathrm{d}t + \widetilde{\varepsilon} \int_{-1}^{1} \left| \phi_{\sigma}(t) \right|~\mathrm{d}t && \text{(triangle inequality)} \notag \\
        &\leq (1 + \widetilde{\varepsilon}) \underline{E}_{\sigma, m} + \widetilde{\varepsilon}. && \text{(\reflem{lem:non-negative-chebyshev-error}, normalization)} \notag \\
    \end{align}
    From \reflem{lem:non-negative-chebyshev-error} we see that if $0 < \sigma \ll 1$, choosing $m = \mathcal{O}(\sigma^{-1}(\log(\varepsilon^{-1}) + \log(\sigma^{-1})))$ ensures $\underline{E}_{\sigma, m} \leq \widetilde{\varepsilon}$, and thus, since $\widetilde{\varepsilon} \leq 1$,
    \begin{equation}
        \int_{-1}^{1} \left| \phi_{\sigma}(t) - \underline{\phi}_{\sigma}^{(m)}(t) \right|~\mathrm{d}t \leq (1+\widetilde{\varepsilon}) \widetilde{\varepsilon} + \widetilde{\varepsilon} \leq 3 \widetilde{\varepsilon}
    \end{equation}
    from which the result follows by identifying $\varepsilon = 3 \widetilde{\varepsilon}$.
\end{proof}

%\todo{Maybe specialize \refthm{thm:nystrom} to spectral densities?}

%We now discuss what value to choose for $n_{\mtx{\Omega}}$ in the Nyström trace estimator introduced in \refsec{subsec:nystrom}. In order to guarantee an even better approximation error based on \refthm{thm:nystrom}, we analyze the numerical rank of the matrix function $g_{\sigma}(t\mtx{I}_n - \mtx{A})$.

In practice, the approximation error is often significantly better than guaranteed in \refthm{thm:chebyshev-nystrom}, particularly when $n_{\mtx{\Omega}}$ is chosen rather large. The reason is that for small $\sigma$ the singular values of the matrix function $g_{\sigma}(t\mtx{I}_n - \mtx{A})$ decay so quickly that already the Nyström approximation alone is highly accurate (cf. \refthm{thm:nystrom}). To establish a relation between the singular value decay of $g_{\sigma}(t\mtx{I}_n - \mtx{A})$ and what choice of $n_{\mtx{\Omega}}$ is needed for a significantly better approximation than ensured by \refthm{thm:chebyshev-nystrom}, we study the numerical rank of $g_{\sigma}(t\mtx{I}_n - \mtx{A}) / n$\footnote{The factor $1/n$ comes from the definition of the spectral density \refequ{equ:smooth-spectral-density} when it is written as the trace of a matrix function.}.

We can express the numerical rank of a matrix with respect to the nuclear norm in terms of its singular values $\sigma_1 \geq \dots \geq \sigma_n \geq 0$ as
\begin{equation}
    r_{\varepsilon} = \min \{1 \leq r \leq n: \sum_{j=r+1}^n \sigma_{j} \leq \varepsilon \},
    \label{equ:numerical-rank}
\end{equation}
where $\varepsilon$ is usually taken to be the double machine precision, i.e. $10^{-16}$. 

%\begin{theorem}{Nyström estimator for parameter-dependent matrices}%{chebyshev-nystrom-raw}
%    Let $\phi_{\sigma}$ be the smooth spectral density of a symmetric %matrix $\mtx{A} \in \mathbb{R}^{n \times n}$ whose spectrum is %contained in $[-1, 1]$. Then the Chebyshev-Nyström++ estimate %$\widetilde{\phi}_{\sigma}^{(m)}$ with \emph{even} degree $m \in %\mathbb{N}$, $n_{\mtx{\Omega}} \geq \max(6, \max_{t \in [-1, 1]}(r_%{\varepsilon}(g_{\sigma}(t\mtx{I}_n - \mtx{A}))) + \log(\delta^%{-1}))$, $n_{\mtx{\Psi}} = 0$, and smoothing parameter $\sigma > 0$ %satisfies with for all $\gamma \geq 1$ with probability at least $1 %- \delta$
%    \begin{equation}
%        \int_{a}^{b} \left| \phi_{\sigma}(t) - \widetilde{\phi}_{\sigma}%(t) \right| ~\mathrm{d}t
%        \leq 2 e^2 (1 + r_{\varepsilon}) (\varepsilon + (n - r_%{\varepsilon}) \underline{E}_{\sigma, m}) + \underline{E}_%{\sigma, m}.
%    \end{equation}
%\end{theorem}
%
%\begin{proof}
%    \todo{TODO}
%\end{proof}

When using a Gaussian smoothing-kernel, we can explicitly write the singular values of $g_{\sigma}(t\mtx{I}_n - \mtx{A}) / n$ as
\begin{equation}
    \sigma_i(t) = g_{\sigma}(t - \lambda_{(i)}) / n = \frac{1}{n \sqrt{2 \pi \sigma^2}} e^{-\sfrac{(t - \lambda_{(i)})^2}{2 \sigma^2}},
    \label{equ:gaussian-kernel-eigenvalues}
\end{equation}
where $\lambda_{(1)}, \dots, \lambda_{(n)}$ denote the eigenvalues of $\mtx{A}$ sorted by increasing distance from spectral-parameter $t$, such that $\sigma_1(t) \geq \dots \geq \sigma_n(t)$. Consequently, by using the closed-form expression of the eigenvalues \refequ{equ:gaussian-kernel-eigenvalues}, we may generously upper bound the numerical rank by
\begin{equation}
    r_{\varepsilon}(g_{\sigma}(t\mtx{I}_n - \mtx{A}) / n) \leq \#\left\{1\leq i\leq n: |t - \lambda_i| < d_{\varepsilon, \sigma} \right\},
    \label{equ:gaussian-kernel-numerical-rank}
\end{equation}
with $d_{\varepsilon, \sigma} = \sigma \sqrt{-2 \log(\sqrt{2 \pi} \sigma \varepsilon)}$ and the eigenvalues $\lambda_1, \dots, \lambda_n$ of $\mtx{A}$ and the constants. The expression \refequ{equ:gaussian-kernel-numerical-rank} has a nice visual interpretation: The numerical rank of a matrix is at most equal to the number of eigenvalues which are closer to the parameter $t$ than $d_{\varepsilon, \sigma}$. This is illustrated in \reffig{fig:numerical-rank}.
\begin{figure}[ht]
    \centering
    \begin{tikzpicture}
    \fill[lightblue] (-2, -0.15) rectangle (1.5, 0.15);
    \draw[thick, ->] (-5, 0) to (5, 0);
    \fill[darkblue] (-4, 0) circle (0.075) node[above] {$\lambda_{(7)}$};
    \fill[darkblue] (-3, 0) circle (0.075) node[above] {$\lambda_{(5)}$};
    \fill[darkblue] (-2.3, 0) circle (0.075) node[above] {$\lambda_{(4)}$};
    \fill[darkblue] (-1, 0) circle (0.075) node[above] {$\lambda_{(2)}$};
    \fill[darkblue] (0.3, 0) circle (0.075) node[above] {$\lambda_{(1)}$};
    \fill[darkblue] (1, 0) circle (0.075) node[above] {$\lambda_{(3)}$};
    \fill[darkblue] (2.8, 0) circle (0.075) node[above] {$\lambda_{(6)}$};
    \fill[darkblue] (3.6, 0) circle (0.075) node[above] {$\lambda_{(8)}$};
    \fill[darkblue] (4.4, 0) circle (0.075) node[above] {$\lambda_{(9)}$};
    \draw[darkblue, ultra thick] (-0.155, 0.15) to (-0.155, -0.15) node[below] {$t$};
    \draw[darkblue, thick] (-2, 0.15) to (-2, -0.15) node[below] {$t - d_{\varepsilon, \sigma}$};
    \draw[darkblue, thick] (1.5, 0.15) to (1.5, -0.15) node[below] {$t + d_{\varepsilon, \sigma}$};
\end{tikzpicture}
    \caption{The numerical rank of $g_{\sigma}(t\mtx{I}_n - \mtx{A}) / n$ can be approximately computed by counting the number of eigenvalues $\lambda_{(1)}, \dots, \lambda_{(n)}$ of the matrix $\mtx{A}$ which lie less than a constant $d_{\varepsilon, \sigma}$ away from $t$.}
    \label{fig:numerical-rank}
\end{figure}

If we additionally assume the eigenvalues of the matrix $\mtx{A}$ to be evenly distributed within $[-1, 1]$, that is, in any subinterval of fixed length in $[a, b]$, $-1 \leq a < b \leq 1$, we can expect to find roughly the same number of eigenvalues, then we can estimate the numerical rank of $g_{\sigma}(t\mtx{I}_n - \mtx{A}) / n$ to be
\begin{equation}
    r_{\varepsilon}(g_{\sigma}(t\mtx{I}_n - \mtx{A}) / n) \lessapprox n d_{\varepsilon, \sigma} \approx 6 n \sigma,
    \label{equ:gaussian-kernel-numerical-rank-uniform}
\end{equation}
provided we set $\varepsilon = 10^{-16}$ and choose $\sigma$ roughly between $10^{-8}$ and $1$. Hence, choosing $n_{\mtx{\Omega}} = \mathcal{O}(n \sigma)$ will yield approximations which can be significantly better than \refthm{thm:chebyshev-nystrom}.
